\tdoc{《剩余价值理论》}

\tpartnonum{《剩余价值理论》第一册}

\tchapternonum{[《剩余价值理论》手稿目录]}

\indentpara{0em}[\endnote{《剩余价值理论》是马克思的主要著作《资本论》的第四卷。马克思把《资本论》的前三卷称为理论部分,把第四卷称为历史部分、历史批判部分或历史文献部分。在这一卷中,马克思围绕着剩余价值理论这个政治经济学的核心问题,对各派资产阶级经济学家的理论进行了系统的、历史的分析批判,同时以论战的形式阐述了自己的政治经济学理论的许多重要方面。马克思从十九世纪四十年代起就开始研究政治经济学,并计划写一部批判现存制度和资产阶级政治经济学的巨著。经过长期系统研究,于 1857—1858 年写了一部经济学手稿,在这个手稿的基础上于 1859 年出版了《政治经济学批判》(第一分册)。从 1861 年 8 月到 1863 年 7 月,又写了一部篇幅很大的手稿。这部手稿的大部分,也是整理得最细致的部分,构成《剩余价值理论》。手稿的其余部分,即理论部分,后来经马克思重新修改和补充,形成了《资本论》前三卷的内容,而《剩余价值理论》这一历史部分没有重新加工,仍保持着原来的样子。马克思生前出版了《资本论》第一卷。他逝世后,恩格斯整理出版了《资本论》第二卷和第三卷,但是没有来得及整理出版《资本论》第四卷即《剩余价值理论》。1905—1910 年卡尔·考茨基编辑出版了《剩余价值理论》,他对马克思的手稿做了许多删改和变动。1954—1961 年按马克思的手稿次序编辑出版了《剩余价值理论》俄文新版本;1956—1962 年出版了该书德文新版本;1962—1964 年则作为《马克思恩格斯全集》俄文第二版第二十六卷(共三册)出版。《剩余价值理论》的章节标题大部分是由俄文版编者拟定的。编者加的标题和文字,用方括号[]标出。马克思手稿中使用的方括号则改用花括号\fontbox{~\{} \fontbox{\}~}。马克思手稿的稿本编号和页码,一律用方括号标出,括号中的罗马数字表示稿本编号,阿拉伯数字表示页码。——第 1 页。}VI—219b]第 VI 本目录:\endnote{《剩余价值理论》手稿的这一目录是马克思写在 1861—1863 年手稿第 VI—XV 各稿本的封面上的。其中有几本的目录比正文先写,这从后来写完相应稿本的正文时对目录所作的修改和删节中可以看出。第 XIV 本封面上的目录并不符合稿本的实际内容:这一目录是第 XIV、XV 和 XVIII 本中所完成的计划。——第 3 页。}

\indentpara{2em}(5)剩余价值理论\endnote{马克思在《剩余价值理论》这一标题之前写了阿拉伯数字“5”。这表示关于资本的研究的第一章第五节,即最后一节;马克思打算把这一研究作为论述商品和货币的《政治经济学批判》第一分册的直接继续来发表。在这第五节之前,在手稿第 I—V 本中有三节概述:(1)货币转化为资本,(2)绝对剩余价值,(3)相对剩余价值。在第 V 本第 184 页,马克思指出:“在相对剩余价值之后,应当考察绝对剩余价值和相对剩余价值两者的结合。”这一考察本应构成第四节,但当时并未写成。马克思还没有写完第三节就马上转写第五节,即《剩余价值理论》。——第 3 页。}

\indentpara{4em}(a)詹姆斯·斯图亚特爵士

\indentpara{4em}(b)重农学派

\indentpara{4em}(c)亚·斯密[VI—219b]

\indentpara{0em}[VII—272b][第 VII 本目录]

\indentpara{2em}(5)剩余价值理论

\indentpara{4em}(c)亚·斯密(续篇)

\indentpara{6em}(研究年利润和年工资怎样才能购买一年内生产的、除利润和工资外还包括不变资本的商品)[VII—272b]

\indentpara{0em}[VIII—331b][第 VIII 本目录]

\indentpara{2em}(5)剩余价值理论

\indentpara{4em}(c)亚·斯密(结尾)\endnote{实际上这并不是论斯密这一节的“结尾”,而只是“续篇”。这一节的结尾部分在下一本即手稿第 IX 本中。——第 3 页。}[VIII—331b]

\indentpara{0em}[IX—376b][第 IX 本目录]

\indentpara{2em}(5)剩余价值理论

\indentpara{4em}(c)亚·斯密。结尾

\indentpara{4em}(d)奈克尔[IX—376b]

\indentpara{0em}[X—421c][第 X 本目录]

\indentpara{2em}(5)剩余价值理论

\indentpara{6em}插入部分。魁奈的经济表

\indentpara{4em}(e)兰盖

\indentpara{4em}(f)布雷

\indentpara{4em}(g)洛贝尔图斯先生。插入部分。新的地租理论[X—421c]

\indentpara{0em}[XI—490a][第 XI 本目录]

\indentpara{2em}(5)剩余价值理论

\indentpara{4em}(g)洛贝尔图斯

\indentpara{6em}插入部分。评所谓李嘉图规律的发现史

\indentpara{4em}(h)李嘉图

\indentpara{6em}李嘉图和亚·斯密的费用价格理论(批驳部分)

\indentpara{6em}李嘉图的地租理论

\indentpara{6em}级差地租表及其说明[XI—490a]

\indentpara{0em}[XII—580b][第 XII 本目录]

\indentpara{2em}(5)剩余价值理论

\indentpara{4em}(h)李嘉图

\indentpara{6em}级差地租表及其说明

\indentpara{6em}(考察生活资料和原料的价值——以及机器的价值——的变动对资本有机构成的影响)

\indentpara{6em}李嘉图的地租理论

\indentpara{6em}亚·斯密的地租理论

\indentpara{6em}李嘉图的剩余价值理论

\indentpara{6em}李嘉图的利润理论[XII—580b]

\indentpara{0em}[XIII—670a][第 XIII 本目录]

\indentpara{2em}(5)剩余价值理论及其他

\indentpara{4em}(h)李嘉图

\indentpara{6em}李嘉图的利润理论

\indentpara{6em}李嘉图的积累理论。对这个理论的批判。(从资本的基本形式得出危机)

\indentpara{6em}李嘉图的其他方面。论李嘉图这一节的结尾(约翰·巴顿)

\indentpara{4em}(i)马尔萨斯[XIII—670a]

\indentpara{0em}[XIV—771a][第 XIV 本目录和《剩余价值理论》最后几章的计划]

\indentpara{2em}(5)剩余价值理论

\indentpara{4em}(i)马尔萨斯

\indentpara{4em}(k)李嘉图学派的解体(托伦斯、詹姆斯·穆勒、普雷沃、几部论战著作、麦克库洛赫、威克菲尔德、斯特林、约·斯·穆勒)

\indentpara{4em}(l)政治经济学家的反对派\endnote{关于政治经济学家的反对派一章在手稿第 XIV 本中仅仅是开始。这一章的续篇在第 XV 本前半部分。——第 5 页。}

\indentpara{6em}(政治经济学家的反对派布雷)\endnote{布雷《劳动中的不公正现象及其消除办法》一书的摘录和马克思所加的为数不多的评语包含在手稿第 X 本中。——第 5 页。}

\indentpara{4em}(m)拉姆赛

\indentpara{4em}(n)舍尔比利埃

\indentpara{4em}(o)理查·琼斯。\endnote{论拉姆赛、舍尔比利埃和理·琼斯的几章包含在手稿第 XVIII 本中。——第 5 页。}(这第五部分结束)

\indentpara{6em}补充部分:收入及其源泉\endnote{马克思在手稿第 XV 本后半部分论述了收入及其源泉,在这方面揭示了庸俗政治经济学的阶级根源和认识论根源。这个“补充部分”马克思后来决定放在《资本论》第三部分,这从他在 1863 年 1 月拟定的这一部分的计划可以看出;按照这一计划,第九章的标题应该是《收入及其源泉》(见本册第 447 页)。——第 5 页。}[XIV—771a]

\indentpara{0em}[XV—862a][第 XV 本目录]

\indentpara{2em}(5)剩余价值理论

\indentpara{4em}(1)以李嘉图理论为依据的无产阶级反对派

\indentpara{4em}(2)莱文斯顿。结尾\endnote{论莱文斯顿一节是在前一本(手稿第 XIV 本)第 861 页开始的。第 XIV 本中在这一节之前有标以数码“1”的一节,即论匿名小册子《根据政治经济学基本原理得出的国民困难的原因及其解决办法》。——第 6 页。}

\indentpara{4em}(3)和(4)霍吉斯金\endnote{论霍吉斯金一节的结尾包含在手稿第 XVIII 本中(第 1084—1086 页)。——第 6 页。}

\indentpara{4em}(现存财富同生产运动的关系)

\indentpara{4em}所谓积累不过是流通现象(储备等等是流通的蓄水池)

\indentpara{4em}(复利;根据复利说明利润率的下降)

\indentpara{6em}庸俗政治经济学\endnote{马克思在手稿第 XV 本中研究收入及其源泉的问题时,对庸俗政治经济学进行了分析。他在这一本第 935 页注明参看“论庸俗经济学家一节”,即他的著作中尚未完成的一章,说他在这一章里“将回过头来谈”顺便涉及的蒲鲁东和巴师夏之间的论战。这一提示表明马克思打算专门写一章来批判庸俗政治经济学,但是没有写成。手稿第 XVIII 本中,马克思在结束对霍吉斯金观点的分析并提到后者对资产阶级辩护士的理论的反驳时,注明:“要在论庸俗经济学家一章中谈这一点”(第 1086 页)。这句话也证明马克思打算以后专门写一章来论述庸俗政治经济学。在 1863 年 1 月拟定的《资本论》第三部分的计划中,倒数第二章即第十一章的标题是《庸俗政治经济学》(见本册第 447 页)。——第 6 页。}

\indentpara{4em}(生息资本在资本主义生产基础上的发展)

\indentpara{4em}(生息资本和商业资本同产业资本的关系。更为古老的形式。派生的形式)

\indentpara{4em}(高利贷。路德等等)\endnote{马克思在手稿第 XV 本封面上写下了这一本的目录,目录中的某些标题是写在旁边或上面的。在本卷发表的目录中,这些标题是按照符合稿本实际内容的次序排列的。——第 6 页。}[XV—862a]

\tchapternonum{[总的评论]}

[VI—220]所有经济学家都犯了一个错误:他们不是就剩余价值的纯粹形式,不是就剩余价值本身,而是就利润和地租这些特殊形式来考察剩余价值。由此必然会产生哪些理论谬误,这将在第三章中得到更充分的揭示,那里要分析以利润形式出现的剩余价值所采取的完全转化了的形式。\endnote{马克思这里说的“第三章”是指关于“资本一般”的研究的第三部分。这一章的标题应为:《资本的生产过程和流通过程的统一,或资本和利润》。以后(例如,见第 IX 本第 398 页和第 XI 本第 526 页)马克思不用“第三章”而用“第三篇”(《dritterAbschnitt》)。后来他就把这第三章称作“第三册”(例如,在 1865 年 7 月 31 日给恩格斯的信中)。关于“资本一般”的研究的“第三章”马克思是在第 XVI 本开始的。从这“第三章”或“第三篇”的计划草稿(见本册第 447 页)中可以看出,马克思打算在那里写两篇专门关于利润理论的历史补充部分。但是马克思在写作《剩余价值理论》的过程中,就已在自己的这一历史批判研究的范围内,详细地批判分析了各种资产阶级经济学家对利润的看法。因此,马克思在《剩余价值理论》中,特别是在这一著作的第二册和第三册中,就已进一步更充分地揭示了由于把剩余价值和利润混淆起来而产生的理论谬误。——第 7、87、272 页。}

\tchapternonum{[第一章]詹姆斯·斯图亚特爵士}

\vicetitle{[区分“让渡利润”和财富的绝对增加]}

在重农学派以前,剩余价值——即利润,利润形式的剩余价值——完全是用\textbf{交换},用商品高于它的价值出卖来解释的。詹姆斯·斯图亚特爵士,总的说来,并没有超出这种狭隘看法;甚至可以更确切地说,正是斯图亚特科学地复制了这种看法。我说:“科学地”复制。因为斯图亚特不同意这种幻想:单个资本家由于商品高于它的价值出卖而获得的剩余价值,就是新财富的创造。因此,他把\textbf{绝对}利润和\textbf{相对}利润区分开来:

\begin{quote}“\textbf{绝对利润}对谁都不意味着亏损;它是劳动、勤勉或技能的\textbf{增进}的结果,它能引起\textbf{社会财富}的扩大或增加……\textbf{相对利润}对有的人意味着亏损;它表示财富的天平在有关双方之间的摆动,但并不意味着\textbf{总基金的任何增加}……\textbf{混合}利润很容易理解:这种利润……一部分是\textbf{相对的},一部分是\textbf{绝对的}……二者能够不可分割地存在于同一交易中。”(《政治经济学原理研究》,由其子詹姆斯·斯图亚特爵士将军汇编的《詹姆斯·斯图亚特爵士著作集》(六卷集)第 1 卷,1805 年伦敦版第 275—276 页)\end{quote}

\textbf{绝对}利润是由“劳动、勤勉和技能的增进”产生的。究竟它怎样由这种增进产生,斯图亚特并没有想弄清楚。他所加的关于这个利润能引起“\textbf{社会财富}”的扩大和增加的这句话,看来,可以使人得出这样的结论:斯图亚特所指的,仅仅是由劳动生产力的发展造成的使用价值量的增加,他完全离开总是以交换价值的增加为前提的资本家的利润来考察这个绝对利润。这样的解释完全被他进一步的叙述证实了。

他是这样说的:

\begin{quote}“在商品的\textbf{价格}中,我认为有两个东西是实际存在而又彼此\textbf{完全不同}的:商品的\textbf{实际价值}和\textbf{让渡利润}。”(第 244 页)\end{quote}

可见,商品的价格包含着两个彼此完全不同的要素:第一,商品的\textbf{实际价值};第二,“\textbf{让渡利润}”,即让出或卖出商品时实现的利润。

[221]因此,这个“\textbf{让渡利润}”是由于商品的价格高于商品的实际价值而产生的,换句话说,是由于商品\textbf{高于}它的价值出卖而产生的。这里,一方的赢利总是意味着另一方的亏损。不会造成“总基金的增加”。利润——应该说是剩余价值——是相对的,并且归结为“财富的天平在有关双方之间的摆动”。斯图亚特自己舍弃可以用这种办法来说明剩余价值的看法。他的关于“财富的天平在有关双方之间的摆动”的理论,虽然丝毫没有触及剩余价值本身的性质和起源问题,但是对于考察剩余价值在不同阶级之间按利润、利息、地租这些不同项目进行的分配,有重要的意义。

从下面的引文中可以看出,斯图亚特认为,单个资本家的全部利润只限于这种“相对利润”,“让渡利润”。

\begin{quote}他说:“实际价值”决定于“该国一个劳动者平常……在一天、一周、一月……平均能够完成的”劳动“量”。第二,决定于“劳动者用以满足他个人的需要和……购置适合于他的职业的工具的生存资料和必要费用的价值;这些同样也必须平均计算”……第三,决定于“材料的价值”。(第 244—245 页)“如果这三项是已知的,产品的价格就确定了。它不能低于这三项的总和,即不能低于\textbf{实际价值。凡是超过实际价值的,就是厂主的利润}。这个利润将始终同\textbf{需求}成比例,因此它将随情况而变动。”(同上,第 245 页)“由此看来,为了促进制造业的繁荣,必须有大规模的需求……工业家是按照他们有把握取得的利润,来安排自己的开支和自己的生活方式的。”(同上,第 246 页)\end{quote}

从这里可以清楚地看出,“厂主”即单个资本家的利润,总是“相对利润”,总是“让渡利润”,总是由于商品的价格高于商品的实际价值,由于\textbf{商品高于它的价值出卖}而产生的。因此,如果一切商品都按它的\textbf{价值}出卖,那就不会有任何利润了。

关于这个问题,斯图亚特写了专门的一章,他详细地研究

\begin{quote}“利润怎样同生产费用结成一体”。(同上,第 3 卷第 11 页及以下各页)\end{quote}

一方面,斯图亚特抛弃了货币主义和重商主义体系的这样一种看法,即认为商品高于它的价值出卖以及由此产生的利润,形成剩余价值,造成财富的绝对增加;\authornote{其实,连货币主义也认为,这个利润不是在国内产生,而只是在同其他国家的交换中产生。重商主义体系只看到,这个价值表现为货币(金和银),因此剩余价值表现为用货币结算的贸易差额。}另一方面,他仍然维护它们的这样一种观点,即单个资本家的利润无非是价格超过[222]价值的这个余额——“让渡利润”,不过按照他的意见,这种利润只是\textbf{相对的},一方的赢利相当于另一方的亏损,因此,利润的运动归结为“财富的天平在有关双方之间的摆动”。

可见,在这方面,斯图亚特是货币主义和重商主义体系的\textbf{合理的}表达者。

在对资本的理解方面,他的功绩在于:他指出了生产条件作为一定阶级的财产同劳动能力\endnote{原文是:《Arbeitsvermögen》(“劳动能力”)。马克思在 1861—1863 年手稿中在绝大多数场合都用《Arbeitsvermögen》(“劳动能力”)这个术语,而没有用《Arbeitskraft》(“劳动力”)这个术语。在《资本论》第一卷中,马克思把这两个术语当作同一概念使用:“我们把劳动力或劳动能力,理解为人的身体即活的人体中存在的、每当人生产某种使用价值时就运用的体力和智力的总和。”(见马克思《资本论》第 1 卷第 4 章第 3 节)——第 13 页。}分离的过程是怎样发生的。斯图亚特十分注意资本的这个\textbf{产生过程};诚然,他还没有把这个过程直接理解为资本的产生过程,但是,他仍然把这个过程看成是大工业存在的条件。斯图亚特特别在农业中考察了这个过程,并且正确地认为,只是因为农业中发生了这个分离过程,真正的制造业才产生出来。在亚·斯密的著作里,是以这个分离过程已经完成为前提的。

(斯图亚特的书于 1767 年在伦敦[出版],\textbf{杜尔哥}的书[写于]1766 年,亚当·斯密的书——1775 年。)

\tchapternonum{[第二章]重农学派}

\tsectionnonum{[(1)把剩余价值的起源问题从流通领域转到生产领域。把地租看成剩余价值的唯一形式]}

重农学派的重大功绩在于,他们在资产阶级视野以内对\textbf{资本}进行了分析。正是这个功绩,使他们成为现代政治经济学的真正鼻祖。首先,他们分析了资本在劳动过程中借以存在并分解成的各种\textbf{物质组成部分}。决不能责备重农学派,说他们和他们所有的后继者一样,把资本存在的这些物质形式——工具、原料等等,当作跟它们在资本主义生产中出现时的社会条件脱离的资本来理解,简言之,不管劳动过程的社会形式如何,只从它们是一般劳动过程的要素这个形式来理解;从而,把生产的资本主义形式变成生产的一种永恒的自然形式。对于他们来说,生产的资产阶级形式必然以生产的自然形式出现。重农学派的巨大功绩是,他们把这些形式看成社会的生理形式,即从生产本身的自然必然性产生的,不以意志、政策等等为转移的形式。这是物质规律;错误只在于,他们把社会的一个特定历史阶段的物质规律看成同样支配着一切社会形式的抽象规律。

除了对资本在劳动过程中借以组成的物质要素进行这种分析以外,重农学派还研究了资本在流通中所采取的形式(固定资本、流动资本,不过重农学派用的是别的术语),并且一般地确定了资本的流通过程和再生产过程之间的联系。这一点在论流通那一章\endnote{指关于“资本一般”的研究的第二章,这一章最后发展成为《资本论》第二卷。《资本论》第二卷第十章(《关于固定资本和流动资本的理论。重农学派和亚当·斯密》)包含对重农学派关于固定资本和流动资本的观点的分析。在《社会总资本的再生产和流通》一篇的第十九章《前人对这个问题的阐述》中有专门论重农学派的一节。——第 16 页。}再谈。

在这两个要点上,亚·斯密继承了重农学派的遗产。他的功绩,在这方面,不过是把抽象范畴固定下来,对重农学派所分析的差别采用了更稳定的名称。

[223]我们已经看到\endnote{马克思指他的 1861—1863 年手稿第 II 本第 58—60 页(《货币转化为资本》一节,《转化过程的两个组成部分》一小节)。——第 16 页。},资本主义生产发展的基础,一般说来,是\textbf{劳动能力}这种属于工人的\textbf{商品}同劳动条件这种固着于资本形式并脱离工人而独立存在的商品相对立。劳动能力作为商品,它的\textbf{价值}规定具有极重要的意义。这个价值等于把再生产劳动能力所必需的生活资料创造出来的劳动时间,或者说,等于工人作为工人生存所必需的生活资料的价格。只有在这个基础上,才出现劳动能力的\textbf{价值}和这个劳动能力\textbf{所创造的价值}之间的差额,——任何别的商品都没有这个差额,因为任何别的商品的使用价值,从而它的使用,都不能提高它的\textbf{交换价值}或提高从它得到的交换价值。

因此,从事分析资本主义生产的现代政治经济学的基础,就是把\textbf{劳动能力的价值}看作某种固定的东西,已知的量,而实际上它在每一个特定的场合,也就是一个已知量。所以,\textbf{最低限度的工资}理所当然地构成重农学派的学说的轴心。虽然他们还不了解价值本身的性质,他们却能够确定最低限度的工资的概念,这是因为这个\textbf{劳动能力的价值}表现为必要生活资料的价格,因而表现为一定使用价值的总和。他们尽管没有弄清一般价值的性质,但仍然能够在他们的研究所必需的范围内,把劳动能力的价值理解为一定的量。其次,如果说,他们错误地把这个\textbf{最低限度}看作不变的量,在他们看来,这个量完全决定于自然,而不决定于本身就是一个变量的历史发展阶段,那末,这丝毫也不影响他们的结论的抽象正确性,因为劳动能力的价值和这个劳动能力所创造的价值之间的差额,同我们假定劳动能力的价值是大是小毫无关系。

重农学派把关于剩余价值起源的研究从流通领域转到直接生产领域,这样就为分析资本主义生产奠定了基础。

他们完全正确地提出了这样一个基本论点:只有创造\textbf{剩余价值}的劳动,即只有劳动产品中包含的价值超过生产该产品时消费的价值总和的那种劳动,才是\textbf{生产的}。既然原料和材料的价值是已知的,劳动能力的价值又等于最低限度的工资,那末很明显,这个剩余价值只能由工人向资本家提供的劳动超过工人以工资形式得到的劳动量的余额构成。当然,在重农学派那里,剩余价值还不是以这种形式出现的,因为他们还没有把一般价值归结为它的简单实体:劳动量,或劳动时间。

[224]自然,重农学派的表述方式必然决定于他们对价值性质的一般看法,按照他们的理解,价值不是人的活动(劳动)的一定的社会存在方式,而是由土地即自然所提供的物质以及这个物质的各种变态构成的。

劳动能力的\textbf{价值}和这个劳动能力\textbf{所创造的价值}之间的差额,也就是劳动能力使用者由于购买劳动能力而取得的剩余价值,无论在哪个\textbf{生产部门}都不如在\textbf{农业}这个最初的生产部门表现得这样显而易见,这样无可争辩。劳动者逐年消费的生活资料总量,或者说,他消费的物质总量,小于他所生产的生活资料总量。在工业中,一般既不能直接看到工人生产自己的生活资料,也不能直接看到他还生产超过这个生活资料的余额。在这里,过程以买卖为中介,以各种流通行为为中介,而要理解这个过程,就必须分析价值。在农业中,过程在生产出的使用价值超过劳动者消费的使用价值的余额上直接表现出来,因此,不分析价值,不弄清价值的性质,也能够理解这个过程。因此,在把价值归结为使用价值,又把使用价值归结为一般物质的情况下,也能够理解这个过程。所以在重农学派看来,农业劳动是唯一的\textbf{生产劳动},因为按照他们的意见,这是唯一\textbf{创造剩余价值}的劳动,而\textbf{地租}是他们所知道的\textbf{剩余价值的唯一形式}。他们认为,在工业中,工人并不增加物质的量:他只改变物质的形式。材料——物质总量——是农业供给他的。他诚然把价值加到物质上,但这不是靠他的劳动,而是靠他的劳动的生产费用,也就是靠他在劳动期间所消费的、等于他从农业得到的最低限度工资的生活资料总额。既然农业劳动被看成唯一的生产劳动,那末,把农业劳动同工业劳动区别开来的剩余价值形式,即\textbf{地租},就被看成剩余价值的唯一形式。

因此,在重农学派那里不存在资本的\textbf{利润}——真正的利润,而地租本身只不过是这种利润的一个分枝。重农学派认为利润只是一种较高的工资,这种工资由土地所有者支付,并且由资本家作为收入来消费(因此,它完全象普通工人所得的最低限度的工资一样,加入生产费用),它增大原料的价值,因为它加入资本家即工业家在生产产品、变原料为新产品时的消费费用。

因此,某些重农主义者,例如老米拉波,把\textbf{货币利息}形式的剩余价值——利润的另一分枝——称为违反自然的高利贷。相反,杜尔哥认为货币利息是正当的,因为货币资本家本来可以购买土地,即购买地租,所以他的货币资本应当使他得到他把这笔资本变成地产时所能得到的那样多的剩余价值。由此可见,根据这种看法,连货币利息也不是新创造的价值,不是剩余价值;这里只是说明土地所有者得到的剩余价值的一部分为什么会以利息形式流到货币资本家手里,正如用别的理由[225]说明这个剩余价值的一部分为什么会以利润形式流到工业资本家手里一样。按照重农学派的意见,既然\textbf{农业劳动}是唯一的生产劳动,是唯一创造剩余价值的劳动,那末,把农业劳动同其他一切劳动部门区别开来的\textbf{剩余价值形式},即\textbf{地租},就是\textbf{剩余价值的一般形式}。工业利润和货币利息只是地租依以进行分配的各个不同项目,地租按照这些项目以一定的份额从土地所有者手里转到其他阶级手里。这同从亚当·斯密开始的后来的政治经济学家所持的观点完全相反,因为这些政治经济学家正确地把\textbf{工业利润}看成剩余价值\textbf{最初}为资本占有的\textbf{形式},从而看成剩余价值的最初的一般形式,而把利息和地租仅仅解释为由工业资本家分配给剩余价值共同占有者各阶级的工业利润的分枝。

除了上面所说的理由,即农业劳动是一种使剩余价值的创造在物质上显而易见,并且可以不经过流通过程就表现出来的劳动,重农学派还有一些别的理由说明他们的观点。

\textbf{第一},在农业中,地租表现为第三要素,表现为一种在工业中或者根本不存在,或者只是转瞬即逝的剩余价值形式。这是超过剩余价值(超过利润)的剩余价值,因此是最显而易见和最引人注目的剩余价值形式,是二次方的剩余价值。

\begin{quote}粗俗的政治经济学家\textbf{卡尔·阿伦德}(《合乎自然的国民经济学》1845 年哈瑙版第 461—462 页)说:“农业以地租形式创造一种在工业和商业中遇不到的价值:一种在补偿全部支付了的工资和全部消耗了的资本利润之后剩下来的价值。”\end{quote}

\textbf{第二},如果撇开对外贸易(重农学派为了抽象地考察资产阶级社会,完全正确地这样做了,而且应当这样做),那末很明显,从事加工工业等等而完全脱离农业的工人(斯图亚特称之为“自由人手”)的数目,取决于农业劳动者所生产的超过自己消费的农产品的数量。

\begin{quote}“显然,不从事农业劳动而能生活的人的相对数,完全取决于土地耕种者的劳动生产率。”(\textbf{理查·琼斯}《论财富的分配》1831 年伦敦版第 159—160 页)\end{quote}

可见,农业劳动不仅对于农业领域本身的剩余劳动来说是自然基础(关于这一点见前面的一个稿本)\endnote{马克思指他的 1861—1863 年手稿第 III 本第 105—106 页,在那里他也顺便提到了重农学派(《绝对剩余价值》一节,《剩余劳动的性质》一小节)。——第 22 页。},而且对于其他一切劳动部门之变为独立劳动部门,从而对于这些部门中创造的剩余价值来说,也是自然基础;因此很明显,只要价值实体被认为是一定的具体劳动,而不是抽象劳动及其尺度即劳动时间,农业劳动就必定被看作是剩余价值的创造者。

[226]\textbf{第三},一切剩余价值,不仅相对剩余价值,而且绝对剩余价值,都是以一定的劳动生产率为基础的。如果劳动生产率只达到这样的发展程度:一个人的劳动时间只够维持他本人的生活,只够生产和再生产他本人的生活资料,那就没有任何剩余劳动和任何剩余价值,就根本没有劳动能力的价值和这个劳动能力所创造的价值之间的差额了。因此,剩余劳动和剩余价值的可能性要以一定的劳动生产率为条件,这个生产率使劳动能力能够创造出超过本身价值的新价值,能够生产比维持生活过程所必需的更多的东西。而且,正象我们在\textbf{第二}点已经看到的,这个生产率,这个作为出发前提的生产率阶段,必定首先存在于农业劳动中,因而表现为\textbf{自然的赐予,自然的生产力}。在这里,在农业中,自然力的协助——通过运用和开发自动发生作用的自然力来提高人的劳动力,从一开始就具有广大的规模。在工业中,自然力的这种大规模的利用是随着大工业的发展才出现的。农业的一定发展阶段,不管是本国的还是外国的,是资本发展的基础。就这点来说,绝对剩余价值同相对剩余价值是一致的。(连重农学派的大敌\textbf{布坎南}都用这一点来反对亚·斯密,力图证明,甚至在现代城市工业产生之前,已先有农业的发展。)

\textbf{第四},因为重农学派的功绩和特征在于,它不是从流通中而是从生产中引出价值和剩余价值,所以它同货币主义和重商主义体系相反,必然从这样的生产部门开始,这个生产部门一般可以同流通、交换脱离开来单独考察,并且是不以人和人之间的交换为前提,而只以人和自然之间的交换为前提的。

\tsectionnonum{[(2)重农学派体系的矛盾:这个体系的封建主义外貌和它的资产阶级实质;对剩余价值的解释中的二重性]}

从上述情况也就产生了重农学派体系的矛盾。

实际上这是第一个对资本主义生产进行分析,并把资本在其中被生产出来又在其中进行生产的那些条件当作生产的永恒自然规律来表述的体系。但是另一方面,这个体系宁可说是封建制度即土地所有权统治的资产阶级式的再现;而资本最先得到独立发展的工业部门,在它看来却是“非生产的”劳动部门,只不过是农业的附庸而已。资本发展的第一个条件,是土地所有权同劳动分离,是土地——这个劳动的最初条件——作为独立的力量,作为掌握在特殊阶级手中的力量,开始同自由劳动者相对立。因此,在重农学派的解释中,土地所有者表现为真正的资本家,即剩余劳动的占有者。可见,在这里,封建主义是从资产阶级生产的角度来加以表述和说明的,而农业则被解释成唯一进行资本主义生产即剩余价值生产的生产部门。这样,封建主义就具有了资产阶级的性质,资产阶级社会获得了封建主义的外观。

这个外观曾迷惑了魁奈医生的贵族出身的门徒们,例如守旧的怪人老\textbf{米拉波}。在那些眼光比较远大的重农主义体系[227]代表者那里,特别是在\textbf{杜尔哥}那里,这个外观完全消失了,重农主义体系就成为在封建社会的框子里为自己开辟道路的新的资本主义社会的表现了。因而,这个体系是同刚从封建主义中孵化出来的资产阶级社会相适应的。所以出发点是在法国这个以农业为主的国家,而不是在英国这个以工业、商业和航海业为主的国家。在英国,目光自然集中到流通过程,看到的是产品只有作为一般社会劳动的表现,作为货币,才取得价值,变成商品。因此,只要问题涉及的不是价值形式,而是价值量和价值增殖,那末在这里首先看到的就是“\textbf{让渡利润}”,即斯图亚特所描述的相对利润。但是,如果要证明剩余价值是在生产领域本身创造的,那末,首先必须从剩余价值不依赖流通过程就能表现出来的劳动部门即农业着手。因而这方面的首创精神,是在一个以农业为主的国家中表现出来的。在重农学派的前辈老作家中,已经可以零星地看到近似重农学派的思想,例如在法国的布阿吉尔贝尔那里就可以部分地看到。但是这些思想只有在重农学派那里,才成为标志着科学新阶段的体系。

农业劳动者只能得到最低限度的工资,即“最必需品”,而他们再生产出来的东西却多于这个“最必需品”,这个余额就是地租,就是由劳动的基本条件——自然——的所有者占有的\textbf{剩余价值}。因此,重农学派不是说:劳动者是超过再生产他的劳动能力所必需的劳动时间进行劳动的,所以他创造的价值高于他的劳动能力的价值,换句话说,他付出的劳动大于他以工资形式得到的劳动量。但是他们说:劳动者在生产时消费的使用价值的总和小于他所生产的使用价值的总和,因而剩下一个使用价值的余额。——如果他只用再生产自己的劳动能力所必需的时间来进行劳动,那就没有什么余额了。但是重农学派只抓住这样一点:土地的生产力使劳动者能够在一个工作日(假定为已知量)生产出多于他维持生活所必需消费的东西。这样一来,这个剩余价值就表现为\textbf{自然的赐予},在自然的协助下,一定量的有机物(种子、畜群)使劳动能够把更多的无机物变为有机物。

另一方面,不言而喻,这里是假定土地所有者作为资本家同劳动者相对立的。土地所有者向劳动者支付劳动能力的代价,——这种劳动能力是劳动者当作商品提供给他的,——而作为补偿,他不但得到一个等价物,而且占有这种劳动能力所创造的价值增殖额。在这个交换中,必须以劳动的物质条件和劳动能力本身彼此脱离为前提。出发点是封建土地所有者,但他表现为一个资本家,表现为一个纯粹的商品所有者,他使他用来同劳动交换的商品的价值增殖,并且不仅收回这些商品的等价物,还收回超过这个等价物的余额,因为他把劳动能力只当作商品来支付代价。他作为商品所有者而同自由工人相对立。换句话说,这个土地所有者实质上是资本家。在这方面重农主义体系也是对的,因为劳动者同土地和土地所有权的分离[228]是资本主义生产和资本的生产的基本条件。

因此,在这一体系中就产生了以下矛盾:它最先试图用对于别人劳动的占有来解释\textbf{剩余价值},并且根据商品交换来解释这种占有,但是在它看来,价值不是社会劳动的形式,剩余价值不是剩余劳动;价值只是使用价值,只是物质,而剩余价值只是自然的赐予,——自然还给劳动的不是既定量的有机物,而是较大量的有机物。一方面,地租,即土地所有权的实际经济形式,脱去了土地所有权的封建外壳,归结为超出工资之上的纯粹的剩余价值。另一方面,这个剩余价值——又按封建主义的精神——是从自然而不是从社会,是从对土地的关系而不是从社会关系引伸出来的。价值本身只不过归结为使用价值,从而归结为物质。而在这个物质中,重农学派所关心的只是量的方面,即生产出来的使用价值超过消费掉的使用价值的余额,因而只是使用价值相互之间的量的关系,只是它们的最终要归结为劳动时间的交换价值。

这一切都是资本主义生产初期的矛盾,那时资本主义生产正从封建社会内部挣脱出来,暂时还只能给这个封建社会本身以资产阶级的解释,还没有找到它本身的形式;这正象哲学一样,哲学最初在意识的宗教形式中形成,从而一方面它消灭宗教本身,另一方面从它的积极内容说来,它自己还只在这个理想化的、化为思想的宗教领域内活动。

因此,在重农学派本身得出的结论中,对土地所有权的表面上的推崇,也就变成了对土地所有权的经济上的否定和对资本主义生产的肯定。一方面,全部赋税都转到地租上,换句话说,土地所有权部分地被没收了——而这正是法国革命制定的法律打算实施的办法,也是李嘉图学派的充分发展的现代政治经济学\endnote{马克思指激进的李嘉图学派。这个学派从李嘉图的理论中得出了反对土地私有制存在的实际结论,建议把这一制度(全部或部分)转变为资产阶级国家所有制。属于这个激进的李嘉图学派的有詹姆斯·穆勒、约翰·斯图亚特·穆勒、希尔迪奇,在一定程度上也有舍尔比利埃。关于这一点见本卷第二册,马克思手稿第 458 页;第三册,马克思手稿第 791、1120 和 1139 页;并见《哲学的贫困》(《马克思恩格斯全集》中文版第 4 卷第 187 页)和马克思 1881 年 6 月 20 日给左尔格的信(见《马克思恩格斯全集》中文版第 35 卷第 190—194 页)。——第 26、42 页。}的最终结论。因为地租被认为是唯一的剩余价值,并且根据这一点,一切赋税都落到地租身上,所以对其他形式的收入课税,只不过是对土地所有权采取间接的、因而在经济上有害的、妨碍生产的课税办法。结果,赋税的负担,从而国家的各种干涉,都落不到工业身上,工业也就摆脱了国家的任何干涉。这样做,表面上是有利于土地所有权,不是为了工业的利益而是为了土地所有权的利益。

与此有关的是:自由放任\endnote{自由放任(原文是:laissezfaire,laissezaller,亦译听之任之)是重农学派的口号。重农学派认为,经济生活是受自然规律调节的,国家不得对经济事务进行干涉和监督;国家用各种规章进行干涉,不仅无益,而且有害;他们要求实行自由主义的经济政策。——第 27、42、162 页。},无拘无束的自由竞争,工业摆脱国家的任何干涉,取消垄断等等。按照重农学派的意见,既然工业什么也不创造,只是把农业提供给它的价值变成另一种形式;既然它没有在这个价值上增加任何新价值,只是把提供给它的价值以另一种形式作为等价物归还,那末,很自然,最好是这个转变过程不受干扰地、最便宜地进行,而要达到这一点,只有通过自由竞争,听任资本主义生产自行其是。这样一来,把资产阶级社会从建立在封建社会废墟上的君主专制下解放出来,就只是为了[229]已经变成资本家并一心一意想发财致富的封建土地所有者的利益。资本家成为仅仅为了土地所有者的利益的资本家,正象进一步发展了的政治经济学让资本家成为仅仅为了工人阶级的利益的资本家一样。

从上述一切可以看到,现代的经济学家,如出版了重农学派的著作和自己论述重农学派的得奖论文的欧仁·德尔先生,认为重农学派关于只有农业劳动才具有生产性、关于地租是剩余价值的唯一形式、关于土地所有者在生产体系中占杰出地位这些独特的论点,同重农学派的自由竞争的宣传、大工业和资本主义生产的原则毫无联系,只是偶然地凑合在一起,——他们这种看法是多么不了解重农学派。同时也就可以理解,这个体系的封建主义外观——完全象启蒙时代的贵族腔调——必然会使不少的封建老爷成为这个实质上是宣告在封建废墟上建立资产阶级生产制度的体系的狂热的拥护者和传播者。

\tsectionnonum{[(3)魁奈论社会的三个阶级。杜尔哥对重农主义理论的进一步发展:对资本主义关系作更深入分析的因素]}

现在我们来考察几段引文,一方面为了阐明上述论点,一方面为了给以证明。

在\textbf{魁奈}的《经济表的分析》一书中,国民由三个市民阶级组成:

\begin{quote}“\textbf{生产阶级}〈农业劳动者〉、\textbf{土地所有者阶级}”和“\textbf{不生产阶级}〈“所有从事农业以外的其他职务和其他工作的市民”〉”。\authornote{本卷引文中凡是尖括号〈〉和花括号\fontbox{~\{}\fontbox{\}~}内的话都是马克思加的。——译者注}(《重农学派》,欧仁·德尔出版,1846 年巴黎版第 1 部第 58 页)\end{quote}

只有农业劳动者才是生产阶级,创造剩余价值的阶级,土地所有者就不是。土地所有者阶级不是“不生产的”,因为它代表“剩余价值”,这个阶级的重要并不是由于它创造这个剩余价值,而完全是由于它占有这个剩余价值。

在\textbf{杜尔哥}那里,重农主义体系发展到最高峰。他的著作中某些地方甚至把“纯粹的自然赐予”看作\textbf{剩余劳动},另一方面,他用劳动者脱离劳动条件、劳动条件作为拿这些条件做买卖的那个阶级的财产同劳动者相对立这种情况,来说明工人提供的东西必须超过维持生活的工资。

说明只有农业劳动是生产劳动的第一个理由是:农业劳动是其他一切劳动得以独立存在的自然基础和前提。

\begin{quote}“他的〈土地耕种者的〉劳动,在社会不同成员所分担的各种劳动中占着首要地位……正象在社会分工以前,人为取得食物而必须进行的劳动,在他为满足自己的各种需要而不得不进行的各种劳动中占着首要地位一样。这不是在荣誉或尊严的意义上的首要地位;这是由\textbf{生理的必然性}决定的首要地位……土地耕种者的劳动使土地生产出超过他本人需要的东西,这些东西是社会其他一切成员用自己的劳动换来的工资的唯一基金。当后者利用从这种交换中得来的报酬再来购买土地耕种者的产品时,他们归还土地耕种者的〈在物质形式上〉,恰好只是他们原来得到的。这就是这两种劳动之间的本质[230]差别。”(《关于财富的形成和分配的考察》(1766 年),载于德尔出版的《杜尔哥全集》1844 年巴黎版第 1 卷第 9—10 页)\end{quote}

剩余价值究竟是怎样产生的呢\fontbox{?}它不是从流通中产生的,但是它在流通中实现。产品是按自己的价值出卖的,不是\textbf{高于}自己的价值出卖的。这里没有价格超过价值的余额。但是,因为产品按它的价值出卖,卖者就实现了剩余价值。这种情况所以可能,只是因为卖者本人对他所卖的价值没有全部支付过代价,换句话说,因为产品中包含着卖者没有支付过代价的、没有用等价物补偿的价值组成部分。农业劳动的情况正是这样。卖者出卖他没有买过的东西。杜尔哥最初把这个没有买过的东西描绘成“\textbf{纯粹的自然赐予}”。但是我们将会看到,这个“纯粹的自然赐予”在他那里,不知不觉地变成土地所有者没有买过而以农产品形式出卖的土地耕种者的剩余劳动。

\begin{quote}“土地耕种者的劳动一旦\textbf{生产出超过}他的需要\textbf{的东西},他就可以用这个余额——\textbf{自然给他的}超过他的劳动报酬的\textbf{纯粹的赐予}——去购买社会其他成员的劳动。后者向他出卖自己的劳动时所得到的只能维持生活;而土地耕种者除了自己的生存资料以外,还得到一种独立的和可以自由支配的财富,这是\textbf{他没有买却拿去卖的}财富。因此,他是财富(财富通过自己的流通使社会上一切劳动活跃起来)的唯一源泉,\textbf{因为他的劳动是唯一生产出超过劳动报酬的东西的劳动}。”(同上,第 11 页)\end{quote}

在这第一个解释中,第一,掌握了剩余价值的本质,就是说,剩余价值是卖者没有支付过等价物,即没有买过而拿去出卖时实现的价值。它是\textbf{没有支付过代价的价值}。但是第二,这个超过“\textbf{劳动报酬}”的余额被看成是“纯粹的自然赐予”,因为劳动者在他的工作日中所能生产的东西,比再生产他的劳动能力所必需的东西多,比他的工资多,这种情况一般地说就是自然的赐予,是取决于自然的生产率的。按照这第一个解释,全部产品还是归劳动者本人占有。但它分成两部分。第一部分形成劳动者的工资——他被看作是自己的雇佣劳动者,他把再生产他的劳动能力,维持他的生活所必需的那部分产品支付给自己。除此以外的第二部分是\textbf{自然的赐予},形成剩余价值。但是,只要抛开“土地耕种者-土地所有者”这个前提,而产品的两部分即工资和剩余价值分别属于不同的阶级,一部分属于雇佣劳动者,另一部分属于土地所有者,那末,这个剩余价值的性质,这个“纯粹的自然赐予”的性质,就更清楚地表现出来了。

不论工业还是农业本身,要形成雇佣劳动者阶级(最初,一切从事工业的人只表现为“土地耕种者-土地所有者”的雇工,即雇佣劳动者),劳动条件必须同劳动能力分离,而这种分离的基础是,土地本身表现为社会上一部分人的私有财产,以致社会上另一部分人失去了借以运用自己劳动的这个物质条件。

\begin{quote}“在最初的时代,土地所有者同土地耕种者还没有区别……在那个最初的时代,每一个勤劳的人要多少土地,就可以找到多少土地,[231]谁也不会想到\textbf{为别人劳动}……但是,到了最后,每一块土地都有了主人,那些没有得到土地的人最初没有别的出路,只好从事\textbf{雇佣}阶级的职业〈即手工业者阶级,一句话,一切非农业劳动者阶级〉,\textbf{用自己双手的劳动去换取}土地耕种者-土地所有者的产品余额。”(第 12 页)“土地耕种者-土地所有者”可以用土地对其劳动所赐予的“相当多的余额支付别人,要别人为他耕种土地。对于靠工资过活的人来说,无论从事哪种劳动来挣工资,都是一样。\textbf{因此,土地所有权必定要同农业劳动分离,而且不久也真的分离了}……土地所有者开始把耕种土地的劳动交给雇佣的土地耕种者去担负”。(同上,第 13 页)\end{quote}

这样,资本和雇佣劳动的关系就在农业中出现了。只有当一定数量的人丧失对劳动条件——首先是土地——的所有权,并且除了自己的劳动之外再也没有什么可以出卖的时候,这种关系才会出现。

现在,对于已经不能生产任何商品而不得不出卖自己的劳动的雇佣工人来说,\textbf{最低限度}的工资,即必要生活资料的等价物,必然成为他同劳动条件的所有者交换时的规律。

\begin{quote}“只凭双手和勤劳的普通工人,除了能够把他的劳动出卖给别人以外,就一无所有……在一切劳动部门,工人的工资都必定是,而实际上也是限于维持他的生活所必需的东西。”(同上,第 10 页)\end{quote}

而且,雇佣劳动一出现,

\begin{quote}“土地产品就分成两部分:一部分包括土地耕种者的生存资料和利润,这是他的劳动的报酬,也是他耕种土地所有者的土地的条件;余下的就是那个独立的可以自由支配的部分,这是\textbf{土地作为纯粹的赐予交给耕种土地的人的}、超过他的预付和他的劳动报酬的部分;这是归土地所有者的份额,或者说,是土地所有者赖以不劳动而生活并且可以任意花费的收入”。(同上,第 14 页)\end{quote}

但是,这个“纯粹的土地赐予”现在已经明确地表现为土地给“耕种土地的人”的礼物,即土地给劳动的礼物,表现为用在土地上的劳动的生产力,这种生产力是劳动由于利用自然的生产力所具有的,从而是劳动从土地中吸取的,是劳动只作为劳动从土地中吸取的。因此,在土地所有者手中,余额已经不再表现为“自然的赐予”,而表现为对于别人劳动的——不给等价物的——占有,后者的劳动由于自然的生产率,能够生产出超过本身需要的生存资料,但是它由于是雇佣劳动,在全部劳动产品中只能占有“维持他的生活所必需的东西”。

\begin{quote}“\textbf{土地耕种者}生产\textbf{他自己的工资},此外还生产用来支付整个手工业者和其他雇佣人员阶级的收入。\textbf{土地所有者没有土地耕种者的劳动,就一无所有}〈可见不是靠“纯粹的自然赐予”〉;他从土地耕种者那里[232]得到他的生存资料和用来支付其他雇佣人员劳动的东西……土地耕种者需要土地所有者,却仅仅由于习俗和法律。”(同上,第 15 页)\end{quote}

可见,在这里,剩余价值直接被描绘成土地所有者不给等价物而占有的土地耕种者劳动的一部分,因而这部分劳动的产品是他没有买过而拿去出卖的。但是,杜尔哥所指的不是交换价值本身,不是劳动时间本身,而是土地耕种者的劳动超出他自己的工资之上提供给土地所有者的产品余额;但这个产品余额,只不过是土地耕种者在他为再生产自己的工资而劳动的时间以外,白白地为土地所有者劳动的那一定量时间的体现。

因此,我们看到,重农学派在\textbf{农业劳动范围内}是正确地理解剩余价值的,他们把剩余价值看成雇佣劳动者的劳动产品,虽然对于这种劳动本身,他们又是从它表现为使用价值的具体形式来考察的。

顺便指出,杜尔哥认为,农业的资本主义经营方式——“土地出租”是

\begin{quote}“一切方式中最有利的方式,但是采用这种方式应以已经富庶的地区为前提”。(同上,第 21 页)\end{quote}

\fontbox{~\{}在考察剩余价值时,必须从流通领域转到生产领域,就是说,不是简单地从商品同商品的交换中,而是从劳动条件的所有者和工人之间在生产范围内进行的交换中,引出剩余价值。而劳动条件的所有者和工人又是作为商品所有者彼此对立的,因此,这里决不是以脱离交换的生产为前提。\fontbox{\}~}

\fontbox{~\{}在重农主义体系中,土地所有者是“雇主”,而其他一切生产部门的工人和企业主是“工资所得者”,或者说,“雇佣人员”。由此也就有了“管理者”和“被管理者”。\fontbox{\}~}

杜尔哥这样分析劳动条件:

\begin{quote}“在任何劳动部门,劳动者事先都要有劳动工具,都要有足够数量的材料作为他的劳动对象;而且都要在他的成品出卖之前有可能维持生活。”(同上,第 34 页)\end{quote}

所有这些“预付”,这些使劳动有可能进行,因而成为劳动过程的\textbf{前提}的条件,最初是由土地无偿提供的:

\begin{quote}“在土地完全没有耕种以前,土地就提供了最初的预付基金”,如果实、鱼、兽类等等,还有工具——例如树枝、石块、家畜,后者的数量由于繁殖而增加起来,它们每年还提供“乳、毛、皮和其他材料,这些产品连同从森林里采伐来的木材一起,成了工业生产的最初基金”。(同上,第 34 页)\end{quote}

这些劳动条件,这些“预付”,一旦必须由第三者预付给工人,就变成\textbf{资本},而这种情况,从工人除了本身的劳动能力外一无所有的时候起,就出现了。

\begin{quote}“\textbf{当}社会上大部分成员\textbf{只靠自己的双手谋生的时候},这些靠工资生活的人,无论是为了取得加工的原料,还是为了在发工资之前维持生活,都必须\textbf{事先}得到\textbf{一些东西}。”(同上,第 37—38 页)\end{quote}

[233]杜尔哥给“\textbf{资本}”下的定义是

\begin{quote}“积累起来的流动的价值”。(同上,第 38 页)最初,土地所有者或土地耕种者每天直接把工资和材料付给,比如说,纺麻女工。随着工业的发展,必须使用较大量的“预付”,并保证这个生产过程的不断进行。于是这件事就由“资本所有者”担当起来。这些“资本所有者”必须在自己产品的价格中收回他的全部“\textbf{预付}”,取得等于“假定他用货币购买一块〈土地〉而给他带来的东西”的一笔利润,还要取得他的“工资”,“因为,毫无疑问,如果利润一样多,他就宁可毫不费力地靠那笔资本能够买到的土地的收入来生活了”。(第 38—39 页)\end{quote}

“工业雇佣阶级”又划分为

\begin{quote}“企业资本家和普通工人”等等。(第 39 页)\end{quote}

“租地农场企业主”的情形也和这些企业资本家的情形一样。他们也象上述情况一样,必须收回全部“预付”,同时取得利润。

\begin{quote}“所有这一切都必须从土地产品的价格中事先扣除;\textbf{余下的部分}由土地耕种者交给土地所有者,作为允许他利用后者的土地来建立企业的报酬。这就是租金,就是土地所有者的收入,就是\textbf{纯产品},因为土地生产出来补偿各种预付和这些预付的提供者的利润的全部东西,不能看成收入,而只能看成\textbf{土地耕作费用的补偿};要知道,如果土地耕种者收不回这些费用,他就不会花费自己的资金和劳动去耕种别人的土地。”(同上,第 40 页)\end{quote}

最后:

\begin{quote}“虽然资本有一部分是由劳动者阶级的利润积蓄而成,但是,既然这些利润总是来自土地(因为所有这些利润不是由收入来支付,便是由生产这种收入的费用来支付),那末很明显,资本也完全象收入一样,来自土地;或者更确切地说,资本不外是土地所生产的一部分价值的积累,这一部分价值是收入的所有者或分享者可以每年储存起来,而不用来满足自己的需要的。”(第 66 页)\end{quote}

不言而喻,既然地租成为剩余价值的唯一形式,那末唯有地租才是资本积累的源泉。资本家在此以外积累的东西,是从他们的“工资”中(从供他们消费的收入中,因为利润正是被看成这种收入)积攒下来的。

因为利润和工资一样,算在土地耕作费用中,只有余下的部分才成为土地所有者的收入,所以土地所有者尽管被摆在可敬的地位,事实上,丝毫不分摊土地耕作费用,因而他不再是生产当事人——这一点同李嘉图学派的看法完全一样。

重农主义的产生,既同反对柯尔培尔主义\endnote{指法国路易十四时期柯尔培尔的重商主义经济政策。柯尔培尔当时任财政总稽核,他采取的财政经济政策是为了巩固专制国家的。例如,改革税收制度,建立垄断性的对外贸易公司,通过统一关税率来促进国内贸易,建立国家工场手工业,以及修建道路和港口。柯尔培尔主义客观上促进了新兴的资本主义经济方式的发展。它是法国资本原始积累的一个工具。但是随着资本主义生产方式日益强大,国家的这些强制性措施就由于日益阻碍经济发展而失去作用。——第 35、42 页。}有关系,又特别是同罗氏制度的破产\endnote{英国银行家和经济学家约翰·罗于 1716 年在巴黎创办一家私人银行,该银行于 1718 年改组为国家银行。罗力图依靠这家银行来实现他的荒唐主张,即国家通过发行不可兑的银行券来增加国内财富。罗氏银行无限制地发行纸币,同时回收金属货币。结果,交易所买空卖空和投机倒把活动空前盛行。到 1720 年,国家银行倒闭,罗氏“制度”也就彻底破产。罗逃往国外。——第 35、40 页。}有关系。

\tsectionnonum{[(4)把价值同自然物质混淆起来(帕奥累蒂)]}

[234]把价值同自然物质混淆起来,或者确切些说,把两者等同起来的看法,以及这种看法同重农学派的整套见解的联系,在后面这段引文中表现得很清楚。这段引文摘自\textbf{斐迪南多·帕奥累蒂}的著作《谋求幸福社会的真正手段》(这部著作一部分是针对维里的,维里在他的《政治经济学研究》(1771 年)中曾经反对重农学派)。(托斯卡纳的帕奥累蒂所写的这部著作,见库斯托第出版的《意大利政治经济学名家文集》(现代部分)第 20 卷。)

\begin{quote}象“土地产品”这样的“\textbf{物质数量倍增的情况}”,“在工业中无疑是没有的,而且永远也不可能发生,因为工业只给物质以形式,仅仅使物质发生形态变化;所以工业什么也不创造。但是,有人反驳我说,工业既给物质以形式,那它就是生产的;因为它即使不是物质的生产,也还是形式的生产。好吧,我不否认这一点;可是,\textbf{这不是财富的创造,相反,这无非是一种支出}……作为政治经济学的前提和研究对象的,是物质的和实在的生产,而这种生产只能在农业中发生,因为只有农业才使构成财富的物质和产品的数量倍增……工业从农业购买原料,以便把它加工。工业劳动,前面已经说过,只给这个原料以形式,但什么也不给它添加,不能使它倍增”。(第 196—197 页)“给厨师一定数量的豌豆,要他用来准备午餐;他好好烹调之后,将烧好的豌豆端到你桌上,但是数量同他拿去的一样;相反,把同量的豌豆交给种菜人,让他把豌豆拜托给土地,到时候,他归还给你的至少比他领去的多 3 倍。这才是真正的唯一的生产。”(第 197 页)“物由于人的需要才有价值。因此,商品的价值,或这个价值的增加,不是工业劳动的结果,而是劳动者支出的结果。”(第 198 页)“一种最新的工业品刚一出现,它很快就在国内外风行起来;可是,其他工业家、商人的竞争会\textbf{极快地}把它的价格压低到它应有的水平,这个水平……决定于原料和工人生存资料的价值。”(第 204—205 页)\end{quote}

\tsectionnonum{[(5)亚当•斯密著作中重农主义理论的因素]}

把自然力大规模地使用于生产过程,在农业中要比在其他一切生产部门中早。自然力在工业中的使用,只是在工业发展到比较高的阶段才明显。从后面的引文可以看出,亚·斯密在这里还反映大工业的史前时期,因此他表达的是重农主义的观点,而李嘉图则从现代工业的观点来回答他。

[235]亚·斯密在他的著作《国民财富的性质和原因的研究》第二篇第五章中,谈到地租时说道:

\begin{quote}“地租是扣除或补偿一切可以看作人工产物的东西之后所留下的自然的产物。它很少少于总产品的四分之一,而常常多于总产品的三分之一。制造业中使用的等量生产劳动,决不可能引起这样大的再生产。\textbf{在制造业中,自然什么也没有做,一切都是人做的};并且再生产必须始终和实行再生产的当事人的力量成比例。”\end{quote}

对于这一点,李嘉图在他的《政治经济学和赋税原理》(1819 年第 2 版第 61—62 页上的注)中作了回答:

\begin{quote}“在工业中,自然没有替人做什么吗\fontbox{?}那些推动我们的机器和船只的风力和水力,不算数吗\fontbox{?}那些使我们能开动最惊人的机器的大气压力和蒸汽压力,不是自然的赐予吗\fontbox{?}至于在软化和溶化金属时热的作用以及在染色和发酵过程中大气的作用,就更不用提了。在人们所能举出的任何一个工业部门中,自然都给人以帮助,并且是慷慨而无代价的帮助。”\end{quote}

至于重农学派把利润只看成是地租的扣除部分:

\begin{quote}“重农学派说,例如一幅花边的价格,它的一部分只补偿工人的消费,而另一部分则由一个人\fontbox{~\{}也就是土地所有者\fontbox{\}~}的口袋转入另一个人的口袋。”(《论马尔萨斯先生近来提倡的关于需求的性质和消费的必要性的原理》1821 年伦敦版第 96 页)\end{quote}

重农学派认为利润(包括利息)只是用于资本家消费的收入,从这种见解也产生了亚·斯密和追随他的经济学家的以下观点:资本的积累应归功于资本家个人的节俭、节约和节欲。重农学派所以作出这个论断,是因为他们认为,只有地租才是真正的、经济的、可以说是合法的积累源泉。

\begin{quote}杜尔哥说:“它〈即土地耕种者的劳动〉是唯一生产出\textbf{超过劳动报酬}的东西的劳动。”(\textbf{杜尔哥},同上第 11 页)\end{quote}

可见,利润在这里完全包括在“劳动报酬”之中。

\begin{quote}[236]“土地耕种者除了这个补偿〈补偿他自己的工资〉以外,还生产出土地所有者的收入,而手工业者既不为自己也不为别人生产任何收入。”(同上,第 16 页)“土地生产出来补偿各种预付和这些预付的提供者的利润的全部东西,\textbf{不能看成收入},而只能看成\textbf{土地耕作费用的补偿}。”(同上,第 40 页)\end{quote}

阿·布朗基在《欧洲政治经济学从古代到现代的历史》(1839 年布鲁塞尔版第 139 页)中,谈到重农学派时说道:

\begin{quote}“他们认为,用于耕种土地的劳动,不仅生产出劳动者在整个劳动期间为维持本人生活所必需的东西,而且还生产出一个可以加到已有财富量上的\textbf{价值余额}〈剩余价值〉,他们把这个余额称为\textbf{纯产品}。”\end{quote}

(因而他们是从剩余价值借以表现的使用价值的形式来看剩余价值的。)

\begin{quote}“从他们的观点来看,纯产品必定属于土地所有者,并且成为他手中完全可以由他支配的收入。那末什么是其他劳动部门的纯产品呢\fontbox{?}……工业家、商人、工人——他们都被看成是农业的伙计、\textbf{雇佣劳动者},而农业是一切财富的至高无上的创造者和分配者。根据经济学家\endnote{“经济学家”是十八世纪下半叶和十九世纪上半叶在法国对重农学派的称呼。——第 38、139、223、411 页。}的体系,所有这些人的劳动产品只代表他们在劳动期间消费掉的东西的等价物,因此,在劳动完成之后,\textbf{除非工人或业主把他们有权消费的东西储存下来},也就是说\CJKunderdot{\textbf{节约下来}},财富的总量同以前是完全一样的。因此,只有用在土地上的劳动,才能生产财富,而其他生产部门的劳动是\CJKunderdot{\textbf{不生产的}},因为\textbf{它不能使社会资本有任何增加}。”\end{quote}

\fontbox{~\{}总之,重农学派认为,资本主义生产的实质在于剩余价值的生产。他们应当解释的正是这种现象。在他们驳倒了重商主义的“让渡利润”之后,问题也就在这里。

\begin{quote}\textbf{里维埃尔的迈尔西埃}说:“要有货币,人就必须购买货币,在这种购买之后,他并不比以前更富;他不过是把他以商品形式付出去的同一个价值,以货币形式取回来。”(\textbf{里维埃尔的迈尔西埃}《政治社会天然固有的秩序》第 2 卷第 338 页)\end{quote}

这一点适用于[237]买,也适用于卖,同样适用于商品的整个形态变化的结果,即买卖的结果;适用于各种商品按其价值进行的交换,即等价物的交换。但在这种情况下,剩余价值是从哪里来的呢\fontbox{?}也就是说,资本是从哪里来的呢\fontbox{?}摆在重农学派面前的正是这个问题。他们的错误在于,他们把那种由于植物自然生长和动物自然繁殖而使农业和畜牧业有别于工业的\textbf{物质增加},同\textbf{交换价值的增殖}混淆起来了。在他们看来,使用价值是基础。而一切商品的使用价值(如果用烦琐哲学家的术语来说,则归结为一般实质),在他们看来,就是自然物质本身,而自然物质在其既定形式上的增加,只有在农业中才会发生。\fontbox{\}~}

亚·斯密著作的翻译者热·加尔涅本人是一个重农主义者,他正确地叙述了重农主义的\textbf{节约论}等。首先他告诉我们,工业——而重商学派认为是一切生产部门——\textbf{只有}靠“让渡利润”,靠商品高于其价值出卖,才能创造剩余价值,因此,发生的只是已创造的价值的新分配,而不是已创造的价值的新增加。

\begin{quote}“手工业者和工业家的劳动不开辟财富的任何新源泉,它只有靠\textbf{有利的交换才能获得利润},并且只具有纯粹相对的价值,这种价值,如果\textbf{靠交换获利}的机会不再出现,也就不会再有了。”(亚·斯密《国民财富的性质和原因的研究》,加尔涅的译本,1802 年巴黎版第 5 卷第 266 页)\endnote{热尔门·加尔涅翻译的亚当·斯密的《国富论》法译本(1802 年版)第五卷中包含有《译者的注释》,即热尔门·加尔涅的注释。——第 39 页。}\end{quote}

或者说,他们的节约——除去开支以外给自己保留下来的价值——必须依靠缩减自己的消费来实现。

\begin{quote}“虽然除了雇佣工人和资本家的节约以外,手工业者和工业家的劳动不可能把其他任何东西加到社会财富的总量上去,但是它依靠这种节约,能促使社会富裕。”(同上,第 266 页)\end{quote}

下面一段话说得更详细:

\begin{quote}“农业劳动者正是以自己的劳动产品使国家富裕;相反,工商业劳动者\textbf{只有节约自己的消费}才能够使国家富裕。经济学家的这个论断是从他们对农业劳动和工业劳动所作的区分得出的,并且同这种区分本身一样是无可争议的。事实上,手工业者和工业家的劳动可以加到物质价值上去的,仅仅是他们自己劳动的\textbf{价值},也就是这种劳动根据国内当时通行的工资率[238]和利润率必定带来的工资和利润的价值。这种工资,无论是高是低,都是劳动的报酬;这是工人有权消费并且假定正在消费的东西;因为只有通过消费,他才能享受自己的劳动果实,而这种享受实际上也就是他的全部报酬。同样,利润无论是高是低,也被看成资本家天天消费的东西,当然,假定资本家也是按照资本带给他的收入的多少来安排自己的享受的。总之,如果工人不放弃他按照适合于\textbf{他的劳动}的通行的工资率有权享受的一部分福利,如果资本家不把资本带给他的一部分收入储蓄起来,那末,工人和资本家在完成劳动时,也就消费了这个劳动带来的全部价值。因此,在他们的劳动完成之后,社会财富的总量依然和以前一样,\textbf{除非他们}把他们有权消费并且能够消费而不致被指责为浪费的一部分东西\textbf{节约下来};在后一场合,社会财富总量就增加了\textbf{这种节约的全部价值}。因此,完全可以说,从事工商业的人\textbf{只有通过个人的节俭}才能\textbf{增加社会现有的财富总量}。”(同上,第 263—264 页)\end{quote}

加尔涅也完全正确地觉察到,亚·斯密关于通过节约进行积累的理论,是建立在这个重农主义基础上的(亚·斯密深受重农主义的影响,这种影响在他对重农主义的批判上表现得最明显)。加尔涅说:

\begin{quote}“最后,如果经济学家曾经断言,工业和商业只有通过节俭才能增加国民财富,那末,斯密同样说过,如果经济不通过本身的节约来增加资本,工业就会白白经营,一国的资本也就永远不会增加(第 2 篇第 3 章)。由此可见,斯密完全同意经济学家的意见”等等。(同上,第 270 页)\end{quote}

\tsectionnonum{[(6)重农学派是资本主义大农业的拥护者]}

[239]阿·布朗基在前面引用过的著作中指出,促使重农主义传播、甚至促使它产生的一个直接历史情况是:

\begin{quote}“在\textbf{制度}〈罗氏制度\endnote{英国银行家和经济学家约翰·罗于 1716 年在巴黎创办一家私人银行,该银行于 1718 年改组为国家银行。罗力图依靠这家银行来实现他的荒唐主张,即国家通过发行不可兑的银行券来增加国内财富。罗氏银行无限制地发行纸币,同时回收金属货币。结果,交易所买空卖空和投机倒把活动空前盛行。到 1720 年,国家银行倒闭,罗氏“制度”也就彻底破产。罗逃往国外。——第 35、40 页。}〉的狂热气氛中猛长起来的一切价值,除了毁灭、荒芜、破产之外,毫无所留。\textbf{唯独土地所有权}在这次风暴中未受损伤。”\end{quote}

\fontbox{~\{}因此,蒲鲁东先生在《贫困的哲学》中,也让土地所有权跟在信贷后面出现。\fontbox{\}~}

\begin{quote}“它的地位甚至改善了,因为它——也许是从封建时代以来\textbf{第一次}——转了手,并且\textbf{被大规模地分割了}。”(同上,第 138 页)\end{quote}

这就是说:

\begin{quote}“在该制度的影响下发生的无数次转手,开始了土地所有权的分割……土地所有权第一次摆脱了封建制度长期来使它所处的僵化状态。对农业来说,这是土地所有权的真正的苏醒……它〈土地〉从死手制度转入了流通制度。”(第 137—138 页)\end{quote}

正象\textbf{魁奈}和他的其他门徒一样,杜尔哥也主张农业中的\textbf{资本主义}生产。例如,杜尔哥说:

\begin{quote}“土地的出租……最后这种方式〈以现代租佃制为基础的大农业〉是一切方式中最有利的方式,但是采用这种方式应以已经富庶的地区为前提。”(见\textbf{杜尔哥},同上第 21 页)\end{quote}

魁奈在他的《农业国经济管理的一般原则》中说:

\begin{quote}“用于种植谷物的土地应当尽可能地联合成由富裕的土地耕种者〈即资本家〉经营的大农场,因为大农业企业与小农业企业相比,建筑物的维修费较低,生产费用也相应地少得多,而纯产品多得多。”[《重农学派》,德尔出版,第 1 部第 96—97 页]\end{quote}

同时,魁奈在上述地方承认:农业劳动生产率提高的成果,应当归“纯收入”,因而首先落到土地所有者手里,即剩余价值占有者手里;剩余价值的相对增加不是由土地产生的,而是由提高劳动生产率的社会措施和其他措施产生的。[240]他在上述地方说:

\begin{quote}“可以利用动物、机器、水力等等进行的劳动,它的任何有利的\fontbox{~\{}对“纯产品”有利的\fontbox{\}~}节约,都造福于居民[和国家,因为较大量的纯产品能保证从事其他职业和工作的人有较多的工资]。”\end{quote}

同时,里维埃尔的迈尔西埃(同上,第 2 卷第 407 页)模糊地猜到:剩余价值至少在工业中(前面已经指出,杜尔哥把这一点推广到一切生产部门)同工业工人本身有某种关系。他在这个地方大声疾呼:

\begin{quote}“盲目崇拜工业的虚假产品的人们,请把你们的狂喜劲儿收敛一下吧!在你们赞赏工业奇迹之前,睁开眼睛看看,那些有手艺把 20 苏变为 1000 埃巨价值的工人是多么贫穷,至少是多么拮据!\textbf{价值的这个巨大的增殖额落到谁手里去呢\fontbox{?}请看:亲手创造价值增殖额的人却过不了宽裕日子!请注意这个对照吧}!”\end{quote}

\tsectionnonum{[(7)重农学派政治观点中的矛盾。重农学派和法国革命]}

经济学家的整个体系的矛盾。魁奈是君主专制的拥护者之一。

\begin{quote}“政权应当是统一的……在政体上,保持各种相互对抗的力量的制度是有害的,它只证明上层不和睦和下层受压迫。”(见前面引证的《农业国经济管理的一般原则》[《重农学派》,德尔出版,第 1 部第 81 页])\end{quote}

里维埃尔的迈尔西埃写道:

\begin{quote}“人注定要在社会内生活,单单这一点就注定他要在专制制度的统治下生活。”(《政治社会天然固有的秩序》第 1 卷第 281 页)\end{quote}

这里还有“人民之友”\endnote{老米拉波活着的时候,人们根据他的一本著作的标题称他为《L’Amideshommes》(“人民之友”、“人类之友”)。——第 42 页。}米拉波侯爵,老米拉波!正是这个学派以自己的自由放任\endnote{自由放任(原文是:laissezfaire,laissezaller,亦译听之任之)是重农学派的口号。重农学派认为,经济生活是受自然规律调节的,国家不得对经济事务进行干涉和监督;国家用各种规章进行干涉,不仅无益,而且有害;他们要求实行自由主义的经济政策。——第 27、42、162 页。}口号推翻了柯尔培尔主义\endnote{指法国路易十四时期柯尔培尔的重商主义经济政策。柯尔培尔当时任财政总稽核,他采取的财政经济政策是为了巩固专制国家的。例如,改革税收制度,建立垄断性的对外贸易公司,通过统一关税率来促进国内贸易,建立国家工场手工业,以及修建道路和港口。柯尔培尔主义客观上促进了新兴的资本主义经济方式的发展。它是法国资本原始积累的一个工具。但是随着资本主义生产方式日益强大,国家的这些强制性措施就由于日益阻碍经济发展而失去作用。——第 35、42 页。},并根本否定政府对市民社会活动的任何干涉。它只让国家在这个社会的缝隙中生活,就象伊壁鸠鲁让神在世界的缝隙中存在\endnote{古希腊哲学家伊壁鸠鲁认为,神存在于世界与世界之间的空隙、间隙中,它对宇宙的发展和人类的生活没有任何影响。——第 42 页。}一样!对土地所有权的颂扬,在实践中变成了把赋税全都转到地租上的要求,这就包含着国家没收地产的可能性,——这一点完全同李嘉图学派的激进分子\endnote{马克思指激进的李嘉图学派。这个学派从李嘉图的理论中得出了反对土地私有制存在的实际结论,建议把这一制度(全部或部分)转变为资产阶级国家所有制。属于这个激进的李嘉图学派的有詹姆斯·穆勒、约翰·斯图亚特·穆勒、希尔迪奇,在一定程度上也有舍尔比利埃。关于这一点见本卷第二册,马克思手稿第 458 页;第三册,马克思手稿第 791、1120 和 1139 页;并见《哲学的贫困》(《马克思恩格斯全集》中文版第 4 卷第 187 页)和马克思 1881 年 6 月 20 日给左尔格的信(见《马克思恩格斯全集》中文版第 35 卷第 190—194 页)。——第 26、42 页。}一样。法国革命不顾勒代雷和其他人的反对,采纳了这种赋税理论。

杜尔哥本人是给法国革命引路的激进资产阶级大臣。重农学派虽然有它的假封建主义外貌,但他们同百科全书派\endnote{指法国《百科全书或科学、艺术和工艺详解辞典》(1751 至 1772 年出版,共 28 卷)的作者。百科全书是十八世纪最著名的法国启蒙运动者的著作。主编是狄德罗。参加编篡工作的还有:达兰贝尔、霍尔巴赫、爱尔维修和拉美特利等。孟德斯鸠、伏尔泰和毕丰参与撰写自然科学的条目,孔狄亚克参与撰写哲学的条目。魁奈和杜尔哥在他们撰写的政治经济学条目中阐述了重农主义体系。百科全书派是由具有不同政治观点的人组成的。这部著作为法国革命的思想准备作出了贡献。——第 42 页。}齐心协力地工作。[240]

[241]杜尔哥试图预先采取法国革命的措施。他以\textbf{1776 年二月敕令}废除了行会。(这个敕令在颁布三个月后就撤销了。)同样,杜尔哥还使农民摆脱了筑路义务,并试图对地租实行单一税。\endnote{手稿中这一段是在下面三段之后(仍在第 241 页)。它被用横线同上下文隔开,同前后两段都没有直接联系。因此本版把这一段移至第 240 页末,就其内容来说,它直接同这一页有关。——第 42 页。}

[241]后面,我们还要回过头来谈重农学派在分析资本方面的巨大功绩。\endnote{参看前面第 15—16 页和那里的注 15。在《剩余价值理论》中,马克思在手稿第 X 本中又回过头来谈重农学派,那里有题为《魁奈的经济表》的长篇“插入部分”(见本册第 323—366 页)。——第 43 页。}

这里暂时先指出一点。按照重农学派的意见,剩余价值的产生有赖于特种劳动的生产率即农业的生产率。而这种特殊的生产率的存在,总的说来,有赖于自然本身。

根据重商主义体系,剩余价值只是相对的:一人赢利就是他人亏损。“让渡利润”,或者说“财富的天平在有关双方之间的摆动”。\authornote{见本册第 11—13 页。——编者注}因此,从一国总资本来看,在这个国家内部,实际上并没有形成剩余价值。剩余价值只有在一个国家同另一些国家的关系中才能形成。一国所实现的超过另一国的余额表现在货币上(贸易差额),因为货币正是交换价值的直接的和独立的形式。与此相反,——因为重商主义体系事实上否定绝对剩余价值的形成,——重农主义愿意把绝对剩余价值解释为“\textbf{纯产品}”。因为重农学派把注意力集中在使用价值上,所以他们认为农业是\textbf{这种“纯产品”的唯一创造者}。

\tsectionnonum{[(8)普鲁士反动分子施马尔茨把重农主义学说庸俗化]}

我们看到,追查蛊惑者\endnote{蛊惑者是十九世纪二十年代在德国对本国知识分子中间反政府运动的参加者的称呼。这个词是在 1819 年 8 月举行的德意志各邦大臣卡尔斯巴德联席会议通过了一项对付“蛊惑者的阴谋”的专门决议之后流行开来的。——第 43 页。}的老手,普鲁士王国枢密顾问施马尔茨是重农主义的最幼稚的代表之一——他同杜尔哥相差不知多远!例如,施马尔茨说:

\begin{quote}“既然自然付给他〈土地所有者〉\textbf{比合法货币利息多一倍的利息},那末,根据什么明显的理由可以剥夺他的这种收入呢\fontbox{?}”(《政治经济学》,昂利·茹弗鲁瓦译自德文,1826 年巴黎版第 1 卷第 90 页)\endnote{施马尔茨的著作德文原本于 1818 年在柏林出版,题为《政治经济学。致德意志某王储书柬》第一册和第二册。——第 44 页。}\end{quote}

关于最低限度的工资,重农学派是这样表述的:工人的消费(或开支)等于\textbf{他们所得的}工资。或者象施马尔茨先生那样把这一点一般地表述为:

\begin{quote}“某一职业的平均工资,等于从事这一职业的人在他劳动期间的平均消费额。”(同上,第 120 页)\end{quote}

[接着,我们在施马尔茨的书里读到:]

\begin{quote}“\textbf{地租}是国民收入的唯一要素;[242]投资的利息和各种劳动的工资,都不过是把这个地租的产品从一个人的手里转到另一个人的手里。”(同上,第 309—310 页)“国民财富仅仅在于土地每年生产地租的能力。”(同上,第 310 页)“一切东西,不管它们是什么样的,如果要追究它们的\textbf{价值}的基础或原始要素,那就必须承认,这个价值无非是纯粹的自然产品的价值。这就是说,虽然劳动使这些东西具有新价值,因而增加了它们的价格,但这个新价值,或者说,这个增加了的价格,仍然不过是为了使这些东西具有新形式,而由工人以各种方式毁坏、消费或用掉的一切自然产品的价值的总和。”(同上,第 313 页)“这种劳动〈真正的农业〉是唯一有助于生产\textbf{新\CJKunderdot{物体}}的劳动,因而是唯一在某种程度上可以称为生产劳动的劳动。至于加工工业的劳动……它只赋予自然所生产的物体以新的形式。”(同上,第 15—16 页)\end{quote}

\tsectionnonum{[(9)对重农学派在农业问题上的偏见的最初批判(维里)]}

反对重农学派的偏见。

\textbf{维里(彼得罗)}《政治经济学研究》(1771 年第一次刊印),见库斯托第出版的《意大利政治经济学名家文集》现代部分,第十五卷。

\begin{quote}“宇宙的一切现象,不论是由人手创造的,还是由物理学的一般规律引起的,都不是真正的\textbf{新创造},而只是物质的\textbf{形态变化}。\textbf{结合}和\textbf{分离}是人的智慧在分析\textbf{再生产}的观念时一再发现的唯一要素;\textbf{价值和财富的再生产},如土地、空气和水在田地上变成谷物,或者昆虫的分泌物经过人的手变成丝绸,或者一些金属片被装配成钟表,也是这样。”(第 21—22 页)\end{quote}

接着他写道:

\begin{quote}重农学派把“工业劳动者阶级称为\textbf{不生产}阶级,因为按照他们的意见,\textbf{工业品的价值等于原料加上工业劳动者在加工这个原料}时所消费的\textbf{食品}”。(第 25 页)\end{quote}

[243]相反,维里却注意到土地耕种者经常贫穷,而工业劳动者日益富裕,然后他继续写道:

\begin{quote}“这证明,工业家从他卖得的价格中不仅获得\textbf{消费的补偿,而且在这个补偿之外多得一部分,而这一部分就是}一年生产中\textbf{所创造的新的价值量}。”(第 26 页)“新创造的价值,就是农产品或工业品的价格中\textbf{超过}物质和物质加工时所必要的消费费用的\textbf{原有价值的余额}。在农业中必须扣除种子和土地耕种者的消费;在工业中同样要扣除原料和劳动者的消费,而每年所创造的\textbf{新价值和扣除后的余额}一样多。”(第 26—27 页)\end{quote}

\tchapternonum{[第三章]亚当·斯密}

\tsectionnonum{[(1)斯密著作中两种不同的价值规定:价值决定于商品中包含的已耗费的劳动量;价值决定于用这个商品可以买到的活劳动量]}

亚·斯密和一切值得一谈的经济学家一样,从重农学派那里接受了平均工资的概念,他把平均工资叫做“工资的自然价格”:

\begin{quote}“一个人总要靠自己的劳动来生活,他的工资至少要够维持他的生存。在大多数情况下,他的工资甚至应略高于这个水平,否则,工人就不可能养活一家人,这些工人就不能传宗接代。”([加尔涅的译本]第 1 卷第 1 篇第 8 章第 136 页)\end{quote}

亚·斯密十分明确地断定,劳动生产力的发展,对工人本身并没有好处。例如我们在他的著作中读到(麦克库洛赫版,第 1 篇第 8 章,1828 年伦敦版):

\begin{quote}“劳动的产品构成劳动的自然报酬或工资。在\textbf{土地私有制产生和资本积累}之前的社会原始状态中,全部劳动产品都属于劳动者。既没有土地所有者,也没有老板来同他分享。假如社会的这种状态保持下去,那末工资\textbf{就会随着分工}引起的\textbf{劳动生产力的增长而增长}。一切东西就会逐渐便宜起来。”\end{quote}

\fontbox{~\{}无论如何,在再生产时需要劳动量较少的一切东西,都是如此。但是,它们不仅“会”便宜起来,实际上已经便宜了。\fontbox{\}~}

\begin{quote}“它们将会由较少量的劳动生产出来;而因为在这种状态下同量劳动生产的商品自然会相互交换,所以它们也就可以用劳动量较少的产品[244]来购买……但是,这种由劳动者享有自己的全部劳动产品的社会原始状态,在\textbf{土地私有制产生和资本积累之后,不可能保持下去}。因此,这种状态在劳动生产力取得最重大发展之前早就不存在了,所以,进一步研究这种状态对劳动报酬或工资可能发生什么影响,就没有用处了。”(第 1 卷第 107—109 页)\end{quote}

亚·斯密在这里非常确切地指出,劳动生产力真正大规模的发展,只是从劳动变为雇佣劳动,而劳动条件作为土地所有权和作为资本同劳动相对立的时刻才开始的。因而劳动生产力的发展只是在劳动者自己再也不能占有这一发展成果的条件下才开始的。因此,研究生产力的这种增长在假定劳动产品(或这个产品的价值)属于劳动者本人的情况下对“工资”——在这里等于劳动产品——会有(或应当有)什么影响,就完全没有用处了。

亚·斯密深受重农主义观点的影响,在他的著作中,往往夹杂着许多属于重农学派而同他自己提出的观点完全矛盾的东西。例如地租学说等等,就是如此。斯密著作的这些部分并不表现他的特点,他在这些地方纯粹是一个重农主义者,\endnote{马克思在《剩余价值理论》第二册(手稿第 628—632 页,《亚·斯密的地租理论》一章)中对斯密的地租观点中的重农主义因素作了批判的分析。参看前面《重农学派》一章,第 36—40 页。——第 47 页。}从我们的研究目的来说,这些部分可以完全不去注意。

我在这部著作的第一部分分析商品时已经指出,\endnote{马克思指《政治经济学批判》第一分册。见《马克思恩格斯全集》中文版第 13 卷第 49—50 页。——第 47 页。}亚·斯密在两种不同的交换价值规定之间摇摆不定:一方面认为\textbf{商品}的价值决定于生产商品所必要的劳动量,另一方面又认为商品的价值决定于可以买到商品的活劳动量,或者同样可以说,决定于可以买到一定量活劳动的商品量;他时而把第一种规定同第二种规定混淆起来,时而以后者顶替前者。在第二种规定中,斯密把劳动的\textbf{交换价值},实际上就是把\textbf{工资}当作商品的价值尺度,因为工资等于用一定量活劳动可以购得的商品量,或者说,等于用一定量商品可以买到的劳动量。但是,劳动的价值,或者确切些说,劳动能力的价值,也和其他任何商品的价值一样,是变化的,它和其他商品的价值没有什么特殊的区别。这里把价值本身当作价值标准和说明价值存在的理由,因此成了循环论证。

但是,从后面的叙述中可以看到,斯密的这种摇摆不定以及把完全不同的规定混为一谈,并不妨碍他对剩余价值的性质和来源的探讨,因为斯密凡是在发挥他的论点的地方,实际上甚至不自觉地坚持了商品交换价值的正确规定,即商品的交换价值决定于商品中包含的已耗费的劳动量或劳动时间。[244]

[VII—283a]\fontbox{~\{}可以举出许多例子证明,亚·斯密在他的整部著作中,凡是说明真正事实的地方,往往把产品中包含的劳动量理解为价值和决定价值的因素。这方面的材料,在李嘉图的著作中引用了一部分。\endnote{指李嘉图的《政治经济学和赋税原理》第一篇第一章。——第 48 页。}斯密关于分工和机器改良对商品价格的影响的全部学说,就是建立在这个基础上的。这里只引一个地方就够了。亚·斯密在第一篇第十一章谈到,他那个时代同前几个世纪比较,有许多工业品便宜了,关于前几个世纪,他指出:

\begin{quote}“那时,为了制造这些商品供应市场,要花费多得多的[283B]劳动量,因此商品上市以后,在交换中必定买回或得到一个多得多的劳动量的价格。”([加尔涅的译本]第 2 卷第 156 页)\fontbox{\}~}[VII—283b]\end{quote}

[VI—245]其次,亚·斯密著作中的上述矛盾以及他从一种解释方法到另一种解释方法的转变,是有更深刻的基础的。(李嘉图发现了斯密的矛盾,但没有觉察到这个更深刻的基础,没有对他所发现的矛盾做出正确的评价,因此也没有解决这个矛盾。)假定所有劳动者都是商品生产者,他们不仅生产自己的商品,而且出卖这些商品。这些商品的价值决定于商品中包含的必要劳动时间。因此,如果商品按它们的价值出卖,那末劳动者用一个作为 12 小时劳动时间的产品的商品,仍然可以买到以另一个商品为形式的 12 小时劳动时间,即物化在另一个使用价值中的 12 小时劳动时间。由此看来,他的劳动的价值等于他的商品的价值,即等于 12 小时劳动时间的产品。卖和随之而来的买,总之,整个交换过程——商品的形态变化——在这里没有引起任何改变。它所改变的只是表现这 12 小时劳动时间的使用价值的形态。因此,劳动的价值等于劳动产品的价值。第一,以商品形式相交换的——只要商品按它们的价值进行交换——是等量物化劳动。而第二,一定量活劳动同等量物化劳动相交换,因为一方面,活劳动物化在属于劳动者的产品即商品中,另一方面,这个商品又同包含等量劳动的另一个商品相交换。因而实际上是一定量活劳动同等量物化劳动相交换。由此可见,不仅是商品同商品按照它们所代表的等量物化劳动时间的比例相交换,而且是一定量活劳动与代表同量物化劳动的商品相交换。

在这种前提下,劳动的价值(用一定量劳动可以买到的商品量,或者说,用一定量商品可以买到的劳动量),就和商品中包含的劳动量完全一样,可以看作商品的价值尺度。这是因为,劳动的价值所表现的物化劳动量总是等于生产这个商品所必要的活劳动量,换句话说,一定量的活劳动时间总是支配着代表同样多的物化劳动时间的商品量。但是,在劳动的物质条件属于一个阶级(或几个阶级),而只有劳动能力属于另一个阶级即工人阶级的一切生产方式下——特别是在资本主义生产方式下——情况正好相反。劳动产品或劳动产品的价值不属于工人。一定量活劳动支配的不是同它等量的物化劳动;换句话说,一定量物化在商品中的劳动所支配的活劳动量,大于该商品本身包含的活劳动量。

但是,因为亚·斯密完全正确地从商品以及商品交换出发,从而生产者最初只是作为商品所有者——商品的卖者和买者——相互对立,所以,他发现(他以为),在资本和雇佣劳动的交换、[246]物化劳动和活劳动的交换中,一般规律立即失效了,商品(因为劳动既然被买卖,那它也是商品)已经不按照它们所代表的劳动量来交换了。\textbf{由此}他得出结论:一旦劳动条件以土地所有权和资本的形式同雇佣工人相对立,劳动时间就不再是调节商品交换价值的内在尺度了。正如李嘉图正确地评论他的那样,斯密倒是应当做出相反的结论:“劳动的量”和“劳动的价值”这两个用语不再是等同的了,因而,商品的相对价值,虽然由商品中包含的劳动时间调节,但已经不再由劳动的价值调节了,因为后一个用语只有在同前一个用语等同的时候,才是正确的。以后谈到马尔萨斯的时候,\endnote{在《剩余价值理论》第三册《托·罗·马尔萨斯》一章(手稿第 753—781 页)中,马克思对马尔萨斯的价值观点和剩余价值观点作了详细的批判(手稿第 753—767 页)。——第 50、67 页。}将会证明,即使在劳动者占有自己的产品即自己产品的价值的情况下,把这个价值或劳动的价值当作象劳动时间或劳动本身作为价值尺度和创造价值的要素那种意义的价值尺度,这本身就是错误的和荒谬的。即使在这种情况下,一种商品可以买到的劳动,也不能当作与商品中所包含的劳动有同样意义的尺度,其中的一个只不过是另一个的指数。

无论如何,亚·斯密感到,从决定商品交换的规律中很难引伸出资本和劳动之间的交换,后者显然是建立在同这一规律完全对立和矛盾的原则上的。只要资本直接同劳动相对立,而不是同劳动能力相对立,这种矛盾就无法解释。亚·斯密知道得很清楚,再生产和维持劳动能力所耗费的劳动时间,与劳动能力本身所能提供的劳动是大不相同的。关于这个问题,他甚至引证康替龙的著作《试论一般商业的性质》。

\begin{quote}斯密谈到康替龙时写道:“这位作者补充说,强壮奴隶的劳动据估计有两倍于他的生活费用的价值,而一个最弱工人的劳动所具有的价值,在他看来,也不会比强壮奴隶的劳动少。”(第 1 篇第 8 章第 137 页,\textbf{加尔涅}的译本,第 1 卷)\end{quote}

另一方面,奇怪的是,亚·斯密竟不了解,他的疑问同调节商品交换的规律没有什么关系。商品 A 和商品 B 按它们所包含的劳动时间的比例进行交换,这丝毫不会由于产品 A 或产品 B 的生产者相互之间分配产品 A 和产品 B(或者确切些说,分配它们的价值)的比例而受到破坏。如果产品 A 的一部分归土地所有者,第二部分归资本家,第三部分归工人,那末,无论他们所得的份额是多少,丝毫也不会改变 A 本身是按其价值同 B 相交换的情况。A 和 B 这两种商品所包含的劳动时间的比例,完全不因 A 或 B 所包含的劳动时间如何由不同的人占有而受到影响。

\begin{quote}“当呢绒和麻布进行交换的时候,呢绒的生产者就会在麻布上恰恰占有他们以前在呢绒上所占有的那一份。”(《哲学的贫困》第 29 页)\endnote{马克思引用的是他的著作《哲学的贫困》法文第一版(1847 年巴黎—布鲁塞尔版)。见《马克思恩格斯全集》中文版第 4 卷第 95—96 页。——第 51 页。}\end{quote}

这也就是李嘉图学派后来完全正当地提出来反对[247]亚·斯密的论据。马尔萨斯主义者约翰·卡泽诺夫同样写道:

\begin{quote}“商品的交换和商品的分配应当分开来考察……对其中一个有影响的情况并不总是对另一个也有影响。例如,某一种商品的生产费用的减少,会改变它对其他一切商品的比例;但不一定会改变这种商品本身的分配,或者根本不会影响其他商品的分配。另一方面,\textbf{对一切}商品\textbf{同样发生影响的}价值普遍下降,不会改变商品之间的比例。它可能影响——但也可能不影响——它们的分配”等等。(\textbf{约翰·卡泽诺夫}《为\textbf{马尔萨斯}的〈政治经济学定义〉所写的序言》1853 年伦敦版)\end{quote}

但是,因为在资本家和工人之间进行的产品价值的“分配”本身,是以商品交换——商品和劳动能力之间的交换——为基础的,所以这就自然引起亚·斯密的混乱。亚·斯密还把劳动的价值或某一商品(或货币)对劳动的购买力当作价值尺度,这就使他在阐述价格理论、研究竞争对利润率的影响等等地方乱了思路,使他的著作在总的方面失去了任何统一性,甚至使他把许多重大问题排除在研究范围之外。然而,我们在后面马上就会看到,这并没有影响他关于\textbf{剩余价值的一般}思路,因为斯密在这里始终坚持了价值决定于各种商品中包含的已耗费的劳动时间这一正确规定。

现在我们就来谈谈他对问题的阐述。

不过,还要先提到一个情况。亚·斯密把不同的东西混淆起来了。首先,他在第一篇第五章中说:

\begin{quote}“一个人是富是贫,要看他能取得必需品、舒适品和娱乐品的程度如何。但是,自从各个部门的分工确立之后,一个人依靠自己的劳动能够取得的只是这些物品中的极小部分,极大部分必须\textbf{仰给于他人的劳动};所以他是富是贫,\textbf{就要看他能够支配或买到的劳动量有多大}。因此,任何一种商品,对于占有这种商品而不打算自己使用或消费,却打算\textbf{用它交换其他商品的人来说},\textbf{它的价值等于}这个\textbf{商品能够买到或支配的劳动量}。由此看来,劳动是一切商品的\textbf{交换价值的真实的}尺度。”(第 1 卷第 59—60 页)\end{quote}

接着,他说:

\begin{quote}“\textbf{它们〈商品〉}包含着\textbf{一定量劳动的价值,我们就用这一定量的劳动去同假定}[248]\textbf{在当时包含着同量劳动的价值的东西相交换}……世界上的一切财富原先都不是用金或银,而只是用劳动购买的;这些财富的价值,对于占有它们并想用它们交换什么新产品的人来说,恰好等于他能够买到或支配的劳动量。”(第 1 篇第 5 章第 60—61 页)\end{quote}

最后,他说:

\begin{quote}“霍布斯先生说,\textbf{财富}就是\textbf{权力};但是,获得或继承了大宗财产的人,不一定因此就得到民政的或军事的政治权力……财富直接提供给他的权力,无非是购买的权力;这是一种支配\textbf{当时市场上有的一切他人劳动\CJKunderdot{或者说}他人劳动的一切产品}的权力。”(同上,第 61 页)\end{quote}

我们看到,在所有这些地方,斯密都把“\textbf{他人劳动}”同“\textbf{他人劳动的产品}”混淆起来了。自从分工确立之后,属于某一个人的商品的交换价值,就表现为这个人所能买到的别人的商品,也就是表现为这些商品中包含的别人劳动的量,即物化了的别人劳动的量。而别人劳动的这个量等于他自己的商品中包含的劳动量。斯密十分明确地说:

\begin{quote}“商品包含着一定量劳动的价值,我们就用这一定量的劳动去同假定在当时包含着\textbf{同量劳动的价值}的东西相交换。”\end{quote}

这里的重点在于\textbf{分工}所引起的变化,它表现在:财富已经不再由本人劳动的产品构成,而由这个产品支配的别人劳动的量构成,也就是由它能够买到的并由它本身包含的劳动量决定的那个社会劳动的量构成。其实,这里只包含着交换价值的概念——我的劳动只有作为社会劳动才决定我的财富,因而我的财富是由使我能够支配等量社会劳动的我的劳动产品决定的。我的商品包含着一定量必要劳动时间,它使我能够支配任何其他具有相等价值的商品,因而支配物化在其他使用价值中的等量的别人劳动。这里的重点在于分工和交换价值引起的对\textbf{我的}劳动和\textbf{别人}劳动的同等看待,换句话说,对社会劳动的同等看待(亚当忽略了一点:连\textbf{我的}劳动,或者我的商品中包含的劳动,也已经被\textbf{社会地}规定,它已经根本改变了自己的性质),而决不在于\textbf{物化}劳动同\textbf{活}劳动之间的差别和两者交换的特殊规律。事实上,亚·斯密在这里谈的仅仅是:商品的价值决定于它们所包含的劳动时间,商品所有者的财富由他所支配的社会劳动量构成。

然而,把\textbf{劳动}和\textbf{劳动的产品}等同起来[249],的确在这里已经为混淆两种不同的价值规定——商品价值决定于它们所包含的劳动量;商品价值决定于用这些商品可以买到的活劳动量,即商品价值决定于劳动的价值——提供了最初的根据。既然亚·斯密说:

\begin{quote}“一个人财富的多少同这个权力的大小恰成比例,也就是说,同他能够支配的他人劳动量成比例,或者同样可以说〈这里就错误地等同起来!〉,同他能够买到的他人劳动的产品成比例。”(同上,第 61 页)\end{quote}

那末,他同样可以说:一个人的财富同他自己的商品即他的“财富”所包含的社会劳动量成比例。斯密也指出了这一点:

\begin{quote}“它们〈商品〉包含着一定量劳动的价值,我们就用这一定量的劳动去同假定在当时包含着\textbf{同量劳动的}价值的东西相交换。”(“\textbf{价值}”一词在这里是多余的,没有意义的。)\end{quote}

错误的结论已经在这第五章中表现出来,例如他说:

\begin{quote}“这样看来,劳动\textbf{本身的价值}决不改变,因而劳动是在任何时候和任何地方都可以用来衡量和比较一切商品的价值的唯一真实的和最终的尺度。”(同上,第 66 页)\end{quote}

在这里,把适用于劳动本身,因而也适用于劳动尺度即劳动时间的话——无论\textbf{劳动价值}如何变化,商品价值总是同物化在商品中的劳动时间成比例——硬用于这个变化不定的劳动价值本身。

在这里,亚·斯密只是考察一般商品交换:交换价值、分工以及货币的性质。交换者还只是作为商品所有者相对立。他们是以购买商品的形式购买别人的劳动,就象他们本人的劳动也是以商品的形式出现一样。因此,他们所支配的社会劳动量,等于他们自己用来购买东西的那个商品所包含的劳动量。但是,亚·斯密在以后几章谈到物化劳动和活劳动之间的交换、资本家和工人之间的交换,而且\textbf{强调指出},现在商品的价值已经不再决定于商品本身所包含的劳动量,而决定于这个商品可以支配即可以买到的、和商品本身包含的劳动量不同的别人活劳动的量,实际上他这种说法决不意味着,商品本身现在已经不按照商品所包含的劳动时间来进行交换。这只是意味着,\textbf{发财致富},商品所包含的价值的增殖以及这种增殖的程度,取决于物化劳动所推动的活劳动量的大小。只有这样理解,这才是正确的,但斯密在这里仍然没有弄清楚。

\tsectionnonum{[(2)斯密对剩余价值的一般见解。把利润、地租和利息看成工人劳动产品的扣除部分]}

[250]在第一篇第六章,亚·斯密从假定生产者只作为商品出卖者和商品所有者互相对立的关系,转到劳动条件所有者和单纯的劳动能力所有者之间进行交换的关系。

\begin{quote}“在\textbf{资本积累和土地私有制产生之前的}社会原始不发达状态中,\textbf{为获得各种交换对象所必要的劳动量},看来是能够提供交换准则的唯一根据……通常需要两天或两小时劳动制造的产品,自然比通常需要一天或一小时劳动制造的产品,有加倍的价值。”(加尔涅的译本,第 1 卷第 1 篇第 6 章第 94—95 页)\end{quote}

因此,生产各种商品所必要的劳动时间,决定着商品相互交换的比例,换句话说,决定着它们的\textbf{交换价值}。

\begin{quote}“在这种情况下,全部劳动产品属于劳动者,通常为获得或生产某一商品所耗费的劳动量,是决定用这个商品通常可以买到、支配或换得的那个劳动量的唯一条件。”(同上,第 96 页)\end{quote}

由此可见,在这种前提下,劳动者是单纯的商品出卖者,一个人只有在他用自己的商品购买别人的商品的时候,才能支配别人的劳动。因此,他用自己的商品所支配的别人劳动的量,只有他自己的商品中包含的那样多,因为他们两个人彼此交换的只是商品,而商品的交换价值是由商品中包含的劳动时间或劳动量决定的。

但是,亚当继续说道:

\begin{quote}“一旦\textbf{资本在个别人手中积累起来},其中某些人自然就利用它使勤劳者去劳动,向他们提供材料和生活资料,\textbf{以便从他们的劳动产品的出售中,\CJKunderdot{或者说},从这些工人的劳动加到那些加工材料价值上的东西中,取得利润}。”(同上,第 96 页)\end{quote}

在继续往下读以前,我们先在这里停一下。首先,既无生存资料,又无劳动材料的“勤劳者”——失去了立足之地的人,究竟是从哪里来的呢\fontbox{?}如果把斯密说法中的天真形式去掉,它的含义就是:资本主义生产是在劳动条件归一个阶级所有,而另一个阶级仅仅支配劳动能力的时刻开始的。劳动和劳动条件的这种分离成为资本主义生产的前提。

但是,第二,亚·斯密说,“雇主”使用工人,“以便\textbf{从他们的劳动产品的出售中,\CJKunderdot{或者说}},从这些工人的劳动[251]加到那些加工材料价值上的东西中,\textbf{取得利润}”,这句话是什么意思呢\fontbox{?}他是不是说:利润从\textbf{出售}中产生,商品\textbf{高于}它的价值出售,因此利润是斯图亚特所说的“让渡利润”,它无非是“财富的天平在有关双方之间的摆动”\authornote{见本册第 11—13 页。——编者注}\fontbox{?}下面就是他自己的回答:

\begin{quote}“在用\textbf{成品}同货币或\textbf{劳动}〈这里是新的错误的根源〉或其他商品交换时,除了偿付材料价格和工人工资以外,\textbf{还必须有一些东西},作为在这个事业上冒风险投资的企业主的利润。”(同上)\end{quote}

至于这个“风险”,在后面谈到对利润的辩护论的解释那一章(见第 VII 本札记本第 173 页)\endnote{马克思引用的是他摘录所读过的著作的“札记本”之一。在第 VII 本札记本第 173 页(根据第 VII 本这一部分的报纸摘录来判断,第 173 页写于 1860 年 1 月),马克思引用了斯密的《国富论》第一篇第六章中的话,并加了批语,指出企图从“企业主的风险”中得出利润是荒谬的。至于“对利润的辩护论的解释那一章”,马克思原来是打算为他的关于“资本一般”的研究的第三部分写的。在《剩余价值理论》第三册(手稿第 777 页)中,马克思在同一意义上提到要写的《对资本和雇佣劳动的关系的辩护论的解释》一节。马克思在 1861—1863 年手稿第 X 本中分析魁奈的《经济表》时,对于把利润看成“风险费”的资产阶级观点也进行了批判(见本册第 332—340 页)。——第 57 页。}再讲。“在用成品交换时作为企业主的\textbf{利润}的一些东西”,是不是由于商品高于它的价值出售而产生的呢\fontbox{?}它是不是斯图亚特的“让渡利润”呢\fontbox{?}

\begin{quote}亚当紧接着说:“因此,\textbf{工人加到材料上的价值,这时}〈从资本主义生产发生的时候起〉分成\textbf{两部分,一部分支付工人的工资,另一部分支付企业主的利润,作为他预付工资和加工材料的资本总额的报酬}。”(同上,第 96—97 页)\end{quote}

可见,斯密在这里说得十分明确:出售“成品”时所得的利润,不是从\textbf{出售本身}产生的,不是由于商品\textbf{高于}它的价值出售而产生的,它不是“让渡利润”。情况恰恰相反。工人加到材料上的价值即劳动量分成两部分。一部分支付工人的工资,或者说,已经用工人得到的工资支付。工人交还的这一部分劳动量,只等于他们以工资形式得到的劳动量。另一部分构成资本家的利润,它是资本家没有支付过代价而拿去出售的一定量劳动。因此,如果资本家按商品价值即按商品中包含的劳动时间来出售商品,换句话说,如果这一商品按价值规律同别的商品相交换,那末,资本家的利润就由于资本家对商品中包含的一部分劳动没有\textbf{支付过代价却拿去出售}而产生。这样一来,亚·斯密自己就驳倒了自己的这种想法,即认为当工人的全部劳动产品已不再属于工人自己,他不得不同资本所有者分享这种产品或产品价值的时候,商品相互交换的比例即商品的交换价值决定于物化在商品中的劳动时间量这一规律就会失效。何况他自己就认为,正因为资本家对加到商品上的一部分劳动没有支付过代价,所以产生了他在出售商品时得到的利润。后面我们将会看到,斯密后来更直接地从工人超出他用来\textbf{支付}(即用等价物来补偿)自己工资的那个劳动量之上所完成的劳动,引伸出利润。从而斯密认识到了剩余价值的真正起源。同时他还十分明确地指出,剩余价值不是从[252]预付基金中产生的,无论预付基金在现实的劳动过程中如何有用,它的价值不过是在产品中再现而已。剩余价值仅仅是在新的生产过程中从“工人\textbf{加到材料上的}”新劳动中产生的,在这个新的生产过程中,预付基金表现为劳动资料或劳动工具。

相反,“在用成品同货币\textbf{或劳动}或其他商品交换时”这句话,是不对的(并且是由于前面提到的混淆产生的)。

在资本家用商品同货币或商品交换的时候,他的利润所以产生,是因为他拿去出售的劳动量多于他支付过代价的劳动量,就是说,资本家没有用等量的物化劳动去交换等量的活劳动。因此,亚·斯密不该把成品“同货币或其他商品交换”和“成品同劳动交换”相提并论。因为在前一类交换中,剩余价值所以产生,是由于商品按它们的价值交换,按它们包含的劳动时间交换,但是这个劳动时间中有一部分是\textbf{没有支付过代价}的。这里的前提是:资本家不是用等量的过去劳动交换等量的活劳动;他占有的活劳动量大于他支付过代价的活劳动量。否则工人的工资就会等于他的产品的价值了。因此,在用“成品”同货币或商品交换时,即在它们按它们的价值交换时,利润所以产生,是因为“成品”同活劳动的交换服从于另外的规律,这里不是等价物相交换。因而这两种情况不能混为一谈。

可见,利润不是别的,正是工人加到劳动材料上的价值中的扣除部分。但工人加到材料上的无非是新的劳动量。所以,工人的劳动时间分为两部分:其中一部分,工人用来向资本家换得一个等价物,即自己的工资;另一部分,由工人无偿地交给资本家,从而构成\textbf{利润}。亚·斯密正确地强调指出,只有工人新加到材料上的那部分劳动(价值)才分解为工资和利润;所以,新创造的剩余价值本身,同花费在材料和工具上的那部分资本,是毫不相干的。

亚·斯密这样把利润归结为对无酬的别人劳动的占有之后,接着说:

\begin{quote}“可是,也许有人会说,资本的利润不过是特种劳动即监督和管理的劳动的工资的别名。”(第 97 页)\end{quote}

他也反驳了这种关于“监督和管理的劳动”的错误观点。我们在后面另一章还要谈到这个问题。\endnote{马克思在《剩余价值理论》第三册(论拉姆赛的一章和补充部分《收入及其源泉。庸俗政治经济学》)中,对于把企业主的收入看成资本家因进行“监督和管理的劳动”而取得的工资这种辩护论观点进行了批判。并见马克思《资本论》第 1 卷第 11 章和第 3 卷第 23 章。——第 59 页。}这里重要的只是指出,亚·斯密清楚地看到并且坚决地强调,他的关于利润起源的观点是同这种辩护论观点对立的。在强调这种对立之后,他继续说道:

\begin{quote}[253]“在这种情况下,劳动者的全部劳动产品并不总是属于劳动者。在大多数场合,他必须同雇用他的\textbf{资本所有者}一起分享劳动产品。在这种情况下,通常为获得或生产某一商品所耗费的劳动量,不再是决定用这个商品通常可以买到、支配或换得的那个劳动量的唯一条件。显然,这里还应当有一个劳动的\textbf{追加量},价为预付工资和给工人提供材料的资本的利润。”(同上,第 99 页)\end{quote}

这完全正确。如果我们谈的是资本主义生产,那末表现为货币或商品的物化劳动所买到的,除了它本身包含的劳动量之外,总还有一个活劳动的“追加量”,“作为资本的利润”,但是,换句话说,这不过意味着,物化劳动无偿地占有,不付代价地占有一部分活劳动。斯密胜过李嘉图的地方是,他有力地强调指出,这个变化是随着资本主义生产而出现的。相反,斯密不如李嘉图的地方是,他总不能摆脱掉被他自己在研究过程中驳倒了的那种观点,即认为由于物化劳动和活劳动相交换时发生的这种新关系,商品(它们彼此不过代表物化劳动,代表已知量的实现了的劳动)的相对价值规定也就发生变化。

斯密把剩余价值的一种形式,利润形式,表述为工人超出他补偿自己工资的那部分劳动之上所完成的劳动部分以后,对于剩余价值的另一种形式——\textbf{地租},也作了同样的表述。从劳动那里夺走因而作为别人的财产同劳动相对立的劳动的一个物质条件是\textbf{资本};另一个物质条件是\textbf{土地}本身,是作为\textbf{地产}的土地。所以,亚·斯密谈完了“\textbf{资本所有者}”之后接着说:

\begin{quote}“一旦一个国家的土地全部变成了私有财产,土地所有者也象\textbf{所有其他人}一样,喜欢在他们未曾播种的地方得到收获,甚至对土地的自然成果也索取\textbf{地租}……他〈劳动者〉必须\textbf{把用自己的劳动收集或生产的东西}让给土地所有者一部分,这\textbf{一部分},或者说,这一部分的价格,就构成\textbf{地租}……”(同上,第 99—100 页)\end{quote}

因此,象真正的工业利润一样,地租只不过是工人加到材料上的一部分劳动,也就是“\textbf{他让给}”土地所有者、无偿地给予土地所有者的一部分劳动;因此它只不过是超出工人补偿自己工资(或为工资中包含的劳动时间提供等价物)的那部分劳动时间之上所完成的剩余劳动部分。

可见,亚·斯密把\textbf{剩余价值},即剩余劳动——已经完成并物化在商品中的劳动超过有酬劳动即超过以工资形式取得自己等价物的劳动的余额——理解为\textbf{一般范畴},[254]而本来意义上的利润和地租只是这一般范畴的分枝。然而,他并没有把剩余价值本身作为一个专门范畴同它在利润和地租中所具有的特殊形式区别开来。斯密尤其是李嘉图在研究中的许多错误和缺点,都是由此而产生的。

剩余价值的另一种表现形式是\textbf{资本利息},借贷利息(货币的利息)。但这种

\begin{quote}“\textbf{货币的利息总是}〈斯密在同一章说道〉\textbf{一种派生的收入},如果它不从使用这些货币所取得的\textbf{利润}中支付,那也一定是从他种收入源泉中支付\end{quote}

(因此,不是从地租中支付,就是从工资中支付;在后一种情况下,假定平均工资是已知的,利息就不是从剩余价值中取得,而是工资中的扣除部分,或者说,不过是利润的另一种形式——在以后的研究过程中我们将会看到,在资本主义生产不发达的条件下,利息就是以这种形式出现的)\endnote{马克思在补充部分《收入及其源泉。庸俗政治经济学》中考察了资本的“洪水期前的形式”这一问题(手稿第 899—901 页)。并见《资本论》第 3 卷第 36 章《资本主义以前的状态》。——第 61 页。},

\begin{quote}除非借债人是靠借新债来还旧债利息的浪费者。”(同上,第 105—106 页)\end{quote}

可见,利息或者是用借来的资本赚得的\textbf{利润}的一部分;在这种情况下,利息就是利润本身的派生形式,是它的一个分枝,因而只是以利润形式占有的剩余价值在不同的人之间的进一步分配。利息或者是从地租中支付,那末情况也是一样。最后,利息或者是由借债人从自己的资本或别人的资本中支付;在这种情况下,利息就根本不是剩余价值,而只是已有财富的另一次分配,是“财富的天平在有关双方之间的摆动”,就象“让渡利润”一样。除了利息根本不是剩余价值的形式这最后一种情况之外(并且除了利息是工资中的扣除部分,或者说,它本身是利润的一种形式的情况之外;最后这种情况,亚当根本没有谈到),利息只是剩余价值的派生形式,只是利润或地租的一部分(这只同利润和地租的分配有关);因而利息也无非表现了无酬的剩余劳动的一部分。

\begin{quote}“放债人总是把借出生息的货币资金\textbf{看成}资本。他希望货币资金能按期归还,而借债人在这个期间将为使用这笔货币资金,付给放债人一定的年金。借债人可以把这笔资金当作\textbf{资本}来使用,也可以当作\textbf{直接消费基金}来使用。如果他把这笔资金当作资本来使用,他就用它们来维持生产工人的生活,\textbf{而工人则再生产出它们的价值,并提供利润}。在这种情况下,他不转让和动用任何其他的收入源泉,就可以归还资本,支付利息。如果他把这笔资金用于直接消费,他就成了一个浪费者,把那些原来应该维持勤劳者生活的东西挥霍在有闲者身上。在这种情况下,他不转让或动用别的收入源泉,如地产或地租,就不能归还资本,也不能支付利息。”(麦克库洛赫版,第 2 卷第 2 篇第 4 章第 127 页)\end{quote}

[255]因此,借进货币的人——从这里来看,是指借进资本的人——或者他自己把货币用作资本,从中取得利润。在这种情况下,他付给放债人的利息,不过是利润的一个具有\textbf{特殊名称}的部分。或者他把借来的货币浪费掉,那末,他就会使放债人的财富增加,而使自己的财富减少。这里发生的仅仅是财富的另一次分配,财富从浪费者手里转到高利贷者手里,但在这里没有创造剩余价值的过程。由此看来,只要利息一般代表剩余价值,它就不过是利润的一部分,而利润本身又无非是剩余价值即无酬劳动的一定形式。

最后,亚·斯密指出,连靠税收生活的人的一切收入,也是或者由工资支付,即成为工资本身的扣除部分,或者来源于利润和地租,因而只是意味着各个不同社会阶层分享利润和地租的权利,而利润和地租本身只是剩余价值的不同形式。

\begin{quote}“一切税收和以税收为基础的一切收入——薪俸、津贴、各种年金——归根到底都是从收入的这三个原始源泉中得来的,都是直接或间接地从工资、资本的利润或者地租中支付的。”(加尔涅的译本,第 1 篇第 6 章第 106 页)\end{quote}

因此,货币利息以及税收和由税收而来的收入,只要不是工资本身的扣除部分,那就只是利润和地租的分成而已,而利润和地租又归结为剩余价值,即无酬的劳动时间。

这就是亚·斯密的一般剩余价值理论。

亚·斯密又一次把自己的整个见解加以总括。这里看得特别清楚,他并不打算哪怕是稍微证明一下:工人加到产品上的价值(在扣除生产费用即原料和劳动工具的价值之后),由于工人不是全部占有这个价值,而是被迫同资本家和土地所有者分享这个价值或产品,似乎已不再由包含在产品中的劳动时间决定了。商品的价值以什么方式在商品生产者之间分配,这当然丝毫不会改变这个价值的性质,以及商品与商品之间的价值比例。

\begin{quote}“一旦土地成为私有财产,对劳动者在这块土地上所能生产和收集的几乎一切产品,土地所有者都要求得到一份。\textbf{他的地租是对耕种土地的劳动所生产的产品的第一个扣除}。但是,种地人在收获以前很少有维持自己生活的资金。他的生活费通常是从他的雇主即租地农场主的资本中预付的,如果租地农场主不能从劳动者的劳动的产品中得到一份,或者说,如果他的资本不能得到补偿并带来利润,他就没有兴趣雇人了。\textbf{这种利润是}[256]\textbf{对耕种土地的劳动所生产的产品的第二个扣除}。几乎所有其他劳动的产品都要作\textbf{这样的扣除,来支付利润}。在所有手工业和制造业中,大多数劳动者都需要雇主预付给他们劳动材料以及工资和生活费,直到劳动完成的时候为止。\textbf{这个雇主从他们劳动的产品中得到一份,或者说,从他们的劳动加到加工材料上的价值中得到一份,这一份也就是雇主的利润}。”(麦克库洛赫版,第 1 卷第 1 篇第 8 章第 109—110 页)\end{quote}

总之,亚·斯密在这里直截了当地把地租和资本的利润称为纯粹是工人产品中的\textbf{扣除部分},或者说,是与工人加到原料上的劳动量相等的产品价值中的\textbf{扣除部分}。但是,正如亚·斯密自己在前面证明过的,这个扣除部分只能由工人加到原料上的、超过只支付他的工资或只提供他的工资等价物的劳动量的那部分劳动构成;因而这个扣除部分是由工人的剩余劳动,即工人劳动的无酬部分构成。(因此,顺便指出,利润和地租,或者说,资本和地产,决不可能是“\textbf{价值的源泉}”。)

\tsectionnonum{[(3)斯密把剩余价值的概念推广到社会劳动的一切领域]}

我们看到,在对剩余价值的分析上,因而在对资本的分析上,亚·斯密比重农学派前进了一大步。在重农学派的著作中,创造剩余价值的,仅仅是一个特定种类的实在劳动——农业劳动。因此,他们考察的是劳动的使用价值,而不是劳动时间,不是作为价值的唯一源泉的一般社会劳动。而在这特定种类的劳动中,实际上创造剩余价值的又是\textbf{自然},是土地,剩余价值被理解为物质(有机物质)的量的增加,理解为生产出来的物质超过消费了的物质的余额。他们还只是在十分狭隘的形式中考察问题,因而夹杂着空想的观念。相反,在亚·斯密的著作中,创造价值的,是一般社会劳动(不管它表现为哪一种使用价值),仅仅是必要劳动的量。剩余价值,无论它表现为利润、地租的形式,还是表现为派生的利息形式,都不过是劳动的物质条件的所有者在同活劳动交换过程中占有的这种必要劳动的一部分。因此,在重农学派看来,剩余价值只表现为地租形式,而在亚·斯密看来,地租、利润和利息都不过是剩余价值的不同形式。

我把与预付资本总额相联系的剩余价值,称为\textbf{资本的利润},我所以这样称谓,是因为直接参与生产的资本家\textbf{直接}占有剩余劳动,不管他以后还要把这个剩余价值分成哪些项目,也不管是同土地所有者分享,还是同资本的出借人分享。例如租地农场主直接向土地所有者支付;例如工厂主从他占有的剩余价值中向工厂地基所有者支付地租,向出借资本给他的资本家支付利息。

[257]\fontbox{~\{}现在还有以下几点要考察:(1)亚·斯密把剩余价值和利润混淆起来;(2)他关于生产劳动的观点;(3)他如何把地租和利润变为\textbf{价值的源泉},他对商品的“自然价格”的分析如何错误,他认为,在商品的“自然价格”中,原料和劳动工具的价值不应离开三个“收入源泉”的“价格”而存在,因而不应单独加以考察。\fontbox{\}~}

\tsectionnonum{[(4)斯密不懂得价值规律在资本同雇佣劳动的交换中的特殊作用]}

工资,或者说,资本家用来购买对劳动能力的暂时支配权的等价物,不是直接形式的商品,而是经过了形态变化的商品,是货币,即作为交换价值,作为社会劳动、一般劳动时间的直接化身的独立形式的商品。当然,工人和任何其他货币所有者一样,按照同样的价格用这些货币购买商品\fontbox{~\{}关于那些细节,例如工人是在对他比较不利的条件和情况下购买,等等,这里撇开不谈\fontbox{\}~}。工人象任何其他货币所有者一样,作为买者同商品的卖者相对立。在商品流通过程本身,工人不是作为工人,而是作为同商品极相对立的货币极,作为随时可以交换的一般形式的商品的所有者出现。他的货币又转化为给他充当使用价值的商品,他在这个交换过程中,按市场上出卖商品的价格,一般说来,按商品的价值购买商品。他在这里只完成 G—W 的行为,从其一般形式来看,这个行为表示的是形式的改变,而决不是价值量的改变。但是,因为工人通过他自己的物化在产品中的劳动,不仅加进了包含在他获得的货币中的那个劳动时间量,不仅支付了等价物,而且还无偿地提供了恰恰成为利润源泉的剩余劳动,所以,\textbf{实际上}(从结果来看,包含在劳动能力出卖中的中介运动不见了)工人提供的价值,高于作为他的工资的那个货币额的价值。他用更多的劳动时间购得了作为工资流到他手里的货币所体现的那个劳动量。因此可以说,工人购买由他挣得的货币(这只是一定量社会劳动时间的独立表现)转化成的那一切商品,间接地用了比这些商品包含的劳动时间更多的劳动时间,虽然他和任何其他买者一样,或者说,和完成了第一转化的商品的所有者一样,按照同样的价格购买商品。相反,资本家用来购买劳动的货币包含的劳动量或劳动时间,比工人生产的商品包含的劳动量或劳动时间要少。除了作为工资的那个货币额所包含的劳动量之外,资本家还买到一个他没有支付过代价的追加劳动量,即超出他支付的货币所包含的劳动量的余额。这个追加劳动量也就构成资本所创造的剩余价值。

但是,因为[258]资本家用来购买劳动(从结果来看,实际上是购买劳动,虽然这里也是通过同劳动能力的交换作为中介,而不是直接同劳动交换)的货币,无非是\textbf{其他一切商品}的转化形式,是其他一切商品作为交换价值的独立存在,所以也可以说,一切商品在同活劳动相交换时买到的劳动多于这些商品本身所包含的劳动。这个追加量也就构成剩余价值。

亚·斯密的巨大功绩在于:他正是在第一篇的几章(第六、七、八章)中,即在从简单商品交换及其固有的价值规律转到物化劳动同活劳动之间的交换,转到资本和雇佣劳动之间的交换,转到从一般形式来考察利润和地租,总之,转到剩余价值的起源问题的那几章中,就感觉到已经出现了缺口;他感觉到,——不管他所不理解的中介环节是怎样的,——从结果来看,规律实际上是失效了:较大量的劳动同较小量的劳动相交换(从工人方面说),较小量的劳动同较大量的劳动相交换(从资本家方面说)。斯密的功绩在于,他强调指出了下面这一点(而这一点也把他弄糊涂了):\textbf{随着资本积累和土地所有权的产生},因而随着同劳动本身相对立的劳动条件的独立化,发生了一个转变,价值规律似乎变成了(从结果来看,也确实变成了)它的对立面。如果说,亚·斯密的理论的长处在于,他感觉到并强调了这个矛盾,那末,他的理论的短处在于,这个矛盾甚至在他考察一般规律如何运用于简单商品交换的时候也把他弄糊涂了;他不懂得,这个矛盾之所以产生,是由于劳动能力本身成了商品,作为这种特殊的商品,它的使用价值本身(因而同它的交换价值毫无关系)是一种创造交换价值的能力。李嘉图胜过亚·斯密的地方是,这个似乎存在而从结果来看也确实存在的矛盾,并没有把他弄糊涂。但是,他不如亚·斯密的地方是,他竟从来没有料到这里有问题,因此价值规律随着资本的出现而发生的\textbf{特殊}发展,丝毫没有引起他的不安,更没有促使他去研究。后面我们将会看到,亚·斯密著作中的天才的东西,到马尔萨斯著作中怎样变成了攻击李嘉图观点的反动的东西。\endnote{在《剩余价值理论》第三册《托·罗·马尔萨斯》一章(手稿第 753—781 页)中,马克思对马尔萨斯的价值观点和剩余价值观点作了详细的批判(手稿第 753—767 页)。——第 50、67 页。}

当然,正是这个观点,使亚·斯密摇摆不定、没有把握,它抽掉了他脚下的坚实基础,使他和李嘉图相反,不能做到对资产阶级制度的抽象的一般基础有一个连贯的理论见解。

[259]前面提到,亚·斯密说,商品买到的劳动多于商品本身所包含的劳动,或者说,工人为商品支付的价值大于商品所包含的价值,这个论点在霍吉斯金的《通俗政治经济学》一书中是这样表达的:

\begin{quote}“\textbf{自然价格}(或\textbf{必要}价格)意味着自然为生产某一商品而要求于人的总\textbf{劳动量}……在我们同自然的相互关系中,劳动是最初的并且永远是唯一的购买手段。不管生产某一商品所必需的劳动量有多大,在现代社会状态下,工人为了获得并占有这个商品所必须付出的劳动,总是比向自然直接购买时所必需的劳动多得多。这样增加了(对工人来说)的自然价格,就是\textbf{社会价格}。必须随时把这两种价格区别开来。”(\textbf{托马斯·霍吉斯金}《通俗政治经济学》1827 年伦敦版第 219—220 页)\end{quote}

霍吉斯金的这种看法既反映了亚·斯密的见解中正确的东西,又反映了使斯密本人糊涂也使别人糊涂的东西。

\tsectionnonum{[(5)斯密把剩余价值同利润混淆起来。斯密理论中的庸俗成分]}

我们已经看到,亚·斯密如何考察一般剩余价值,而地租和利润只不过是\textbf{剩余价值}的不同形式和组成部分。按照他的解释,由原料和生产资料构成的那部分资本,同剩余价值的创造没有任何直接的关系。剩余价值完全是从工人提供的\textbf{超出}仅仅构成他的工资等价物的那部分劳动\textbf{之上}的追加劳动量产生的。因而,剩余价值只是直接由花费在工资上的那部分资本产生的,因为这是资本中唯一不仅再生产自己,而且生产一个“余额”的部分。相反,在利润的形式上,剩余价值是按照预付资本总额计算的,而且,除了这一个形态变化之外,由于资本主义生产各个不同领域中利润的平均化,还有一些新的形态变化。

亚当虽然实质上是考察剩余价值,但是他没有清楚地用一个不同于剩余价值特殊形式的特定范畴来阐明剩余价值,因此,后来他不通过任何中介环节,直接就把剩余价值同更发展的形式即利润混淆起来了。这个错误,在李嘉图和以后的所有经济学家的著作中,仍然存在。由此就产生了一系列不一贯的说法、没有解决的矛盾和荒谬的东西(在李嘉图的著作中,这种情况更加突出,因为他更加系统而一致地、始终如一地贯彻了价值的基本规律,所以不一贯的说法和矛盾表现得更为突出),对于这一切,李嘉图学派企图用烦琐论证的办法,靠玩弄词句来加以解决(我们在后面关于利润那一篇中将会看到这一点)\endnote{见注 13。马克思在继续写作《剩余价值理论》的过程中,也对李嘉图学派的利润观点进行了批判。在《剩余价值理论》第三册《李嘉图学派的解体》一章中,马克思专门谈到李嘉图主义者詹姆斯·穆勒想用烦琐论证的方法来解决李嘉图的利润理论的矛盾,谈到约翰·斯图亚特·穆勒徒劳无益地试图直接从价值理论中得出李嘉图关于利润率和工资额成反比的论点。——第 69 页。}。粗俗的经验主义变成了虚伪的形而上学,变成了烦琐哲学,它绞尽脑汁,想用简单的形式抽象,直接从一般规律中得出不可否认的经验现象,或者巧妙地使经验现象去迁就一般规律。在这里,在分析斯密的剩余价值观点的时候,我们就举一个这方面的例子,因为斯密的混乱不是发生在他专门谈论利润和地租这些剩余价值的特殊形式的地方,而是发生在他把利润和地租仅仅当作剩余价值的一般形式,当作“工人加到材料上的劳动中的\textbf{扣除部分}”的地方。

[260]亚·斯密在第一篇第六章中说:

\begin{quote}“因此,工人\textbf{加到}材料上的价值,这时分成两部分,一部分支付工人的工资,另一部分支付企业主的利润,作为他预付工资和加工材料的资本总额的报酬。”[加尔涅的译本,第 1 卷第 96—97 页]\end{quote}

然后他接着说:

\begin{quote}“如果他〈企业主〉从出卖工人生产的产品中,除了用于补偿他的资本所必需的数额以外,不指望再多得一个余额,他就不会有兴趣雇用这些工人了;同样,如果他的利润不同所使用的资本的量成一定的比例,他就不会有兴趣使用较大的资本来代替较小的资本。”[同上,第 97 页]\end{quote}

首先我们要注意下面一点:亚·斯密起初把剩余价值,即“企业主”除了“用于补偿他的资本”所必需的价值量以外得到的那个“余额”,归结为工人加到材料上的劳动中超出补偿他们工资的劳动之上的部分;因而,斯密完全是从花费在工资上的那部分资本中得出这个余额的。但是,随后他马上就从利润的形式来考察这个余额,也就是说,不把这个余额同它所由产生的那部分资本联系起来看,而认为它是超出预付资本总价值,即超出“预付工资\textbf{和}加工材料〈由于疏忽,这里遗漏了生产工具〉的资本总额”之上的余额。因此,他是直接从利润的形式来考察剩余价值的。从而立刻就产生了困难。

亚·斯密说:如果资本家“从出卖工人生产的产品中,除了用于补偿他的资本所必需的数额以外,不指望\textbf{再多得一个余额},他就不会有兴趣雇用这些工人了”。只要以资本主义关系为前提,那末这句话是完全正确的。资本家不是为了用产品满足自己的需要而生产,一般说来,消费并不是他生产的直接目的。他是为生产剩余价值而生产。但是,亚·斯密不象他后来的许多无能的门徒,他并不是用这个仅仅表明在资本主义生产条件下资本家是为了剩余价值而生产的前提来\textbf{说明剩余价值}的。这就是说,他不是用资本家的兴趣,资本家追求剩余价值的愿望,来说明剩余价值的存在。相反,他已经从工人“加到材料上的、超出他为补偿所得工资而加的价值之上”的那个价值,得出了剩余价值。但是,紧接着他又说:如果资本家的利润不同预付资本的量成一定的比例,他就不会有兴趣使用较大的资本来代替较小的资本。这里,已经不是用剩余价值的本质,而是用资本家的“兴趣”来说明利润了。这是庸俗的和荒谬的。

亚·斯密没有感觉到,他这样直接地把剩余价值同利润,又把利润同剩余价值混淆起来,也就推翻了刚才由他提出的剩余价值起源的规律。[261]如果剩余价值只是“工人\textbf{所加}的、\textbf{超出}他为补偿自己工资而加到材料上的那个数额之上”的“那部分价值〈或劳动量〉”,那末,为什么这部分价值会直接因为预付资本的价值在一种场合下比在另一种场合下大,就一定增加呢\fontbox{?}亚·斯密紧接着自己举了一个例子,来驳斥那种说利润是所谓“监督的劳动”的工资的意见,在这个例子中,上述的矛盾就更加明显了。

他是这样说的:

\begin{quote}“但是,它〈资本利润〉与工资根本不同;它受完全不同的原则支配,并且同这种所谓监督和管理的劳动的数量和性质不成任何比例。\textbf{它完全取决于所使用的资本的价值},它的大小同这一资本的多少成比例。例如,假定某地工业中的\textbf{资本利润每年通常为 10\%},有两个不同的制造厂,各雇 20 个工人,每个工人每年工资 15 镑,这样,每个制造厂每年支出的工资为 300 镑。再假定,其中一个制造厂加工低等材料,每年只花费 700 镑,另一个制造厂加工比较高等的材料,值 7000 镑。在这种情况下,前一个制造厂每年所使用的资本总共只有 1000 镑,而后一个制造厂所使用的资本达 7300 镑。按 10\%的比率,前一个厂主只能指望得到年利润约 100 镑,后一个厂主则可指望得到年利润约 730 镑。尽管他们的利润额相差这样大,但他们的监督和管理的劳动却可能是一样的,或者几乎一样的。”[加尔涅的法译本,第 1 卷第 97—98 页]\end{quote}

在这里,我们一下子就从一般形式的剩余价值转到同问题没有直接关系的一般利润率上来了。我们且往前看!两个工厂都使用 20 个工人的劳动,两处工人的工资都是 300 镑。可见,并不是其中的一个厂比另一个厂用了更高级的劳动,以致一个厂的一小时劳动,也就是说一小时剩余劳动,等于另一个厂的几小时剩余劳动。相反,两处都是假定用同样的平均劳动,这已由两个厂的工资相等这一点表明了。可是,为什么一个厂的工人在他们的工资价格之外加上的剩余劳动会比另一个厂高 6 倍呢\fontbox{?}或者说,虽然两处工人获得同样的工资,因而两处工人劳动了同样多的时间来[262]再生产这笔工资,可是,为什么一个厂仅仅因为加工的材料比另一个厂贵 6 倍,它的工人提供的剩余劳动就一定会多 6 倍呢\fontbox{?}

可见,一个厂比另一个厂所得到的利润多 6 倍这一情况,或者一般地说,利润同预付资本的量成比例的规律,乍一看来同剩余价值仅仅表明工人的无酬剩余劳动这个剩余价值规律,或者说利润规律(因为亚·斯密直接把剩余价值和利润等同起来),是矛盾的。亚·斯密极其天真地、不加思索地说出了这一点,丝毫没有想到这里产生的矛盾。所有后来的经济学家——由于他们当中没有一个人离开剩余价值的特定形式而从一般形式来考察剩余价值——在这方面都信守斯密的思想。前面已经指出,这一点在李嘉图的著作中,只不过表现得更突出罢了。

因为亚·斯密不仅把剩余价值归结为利润,而且归结为地租,——这是剩余价值的两个特殊形式,它们的运动取决于完全不同的规律,——所以仅仅这一点本来就应当使他意识到,决不能不通过任何中介环节,而把一般的抽象形式同它的任何一个特殊形式混淆起来。不论是斯密,还是后来所有的资产阶级经济学家,照例都缺乏对于阐明经济关系的形式差别所必要的理论认识,——他们都是粗略地抓住现成的经验材料,只对这些材料感兴趣。正由于这个原因,在问题纯粹涉及在价值量不变的条件下交换价值形式的各种变化的地方,他们也就不能正确地理解货币的本质。

\tsectionnonum{[(6)斯密把利润、地租和工资看成价值源泉的错误观点]}

\textbf{罗德戴尔}在《论公共财富的性质和起源》(拉让蒂·德·拉瓦伊斯译,1808 年巴黎版)一书中,反对斯密对剩余价值的理解,——他说斯密的理解同洛克早已提出的观点一致,——他指责说:按照这种理解,资本就不象斯密所说的那样是财富的原始源泉,而只是派生源泉。关于这个问题,罗德戴尔说道:

\begin{quote}“一百多年以前,洛克已经提出了〈同亚·斯密〉几乎一样的观点……洛克说,货币是不结果实的,它不会生产任何东西;从货币中得到的全部好处,就是它通过相互协议,把作为一个人的劳动报酬的利润转入另一个人的口袋。”(\textbf{罗德戴尔}的著作第 116 页)“如果对资本利润的这种理解真正正确的话,那就会得出结论说:利润不是收入的原始源泉,而只是派生源泉,并且,决不能把资本看作财富的源泉之一,因为资本带来的利润不过是收入从工人的口袋转到资本家的口袋而已。”(同上,第 116—117 页)\end{quote}

从资本的价值再现在产品中这一点来说,不能把资本称为“财富的源泉”。在这里,资本仅仅作为积累的劳动,作为一定量的物化劳动,把自己的价值加到产品上。

资本只有作为一种\textbf{关系},——从资本作为对雇佣劳动的强制力量,迫使雇佣劳动提供剩余劳动,或者促使劳动生产力去创造相对剩余价值这一点来说,——才生产价值。在这两种情况下,资本都只是[263]作为劳动本身的物质条件所具有的从劳动异化的而又支配劳动的力量,总之,只是作为雇佣劳动本身的一种形式,作为雇佣劳动的条件,才生产价值。按照经济学家通常的理解,资本是以货币和商品形式存在的积累的劳动,它象一切劳动条件(包括不花钱的自然力在内)一样,在劳动过程中,在创造使用价值时,发挥生产性的作用,但它永远不会成为价值的源泉。资本不创造任何新价值,一般地说,它把交换价值加到产品上,只是由于它本身具有交换价值,也就是说,由于它本身可归结为物化劳动时间,因而由于劳动是它的价值的源泉。

罗德戴尔说得对:亚·斯密在研究了剩余价值和价值的本质之后,错误地把资本和土地说成是交换价值的独立源泉。资本和土地,只有成为占有工人超过为补偿他的工资所必需的劳动时间而被迫完成的那一定量剩余劳动的根据,才是它们的所有者的收入的源泉。例如,亚·斯密说:

\begin{quote}“\textbf{工资、利润和地租},是一切收入的\textbf{三个原始源泉,也是一切交换价值的三个原始源泉}。”(第 1 篇第 6 章)\end{quote}

说它们是“一切收入的三个原始源泉”,这是对的;说它们“也是\textbf{一切交换价值}的三个原始源泉”,就不对了,因为商品的价值是完全由商品中包含的劳动时间决定的。亚·斯密刚说了地租和利润纯粹是工人加到原料上的价值或劳动中的“扣除部分”,怎么可以随后又把它们称为“交换价值的原始源泉”呢\fontbox{?}(只有在推动“原始源泉”,即强迫工人完成剩余劳动这个意义上说,它们才能起这种作用。)它们只有成为占有一部分价值即一部分物化在商品中的劳动的根据(条件),才是它们的所有者的收入的源泉。但是价值的分配,或者说,价值的占有,决不是被占有的价值的源泉。如果没有这种占有,工人以工资形式得到自己劳动的全部产品,那末,虽然土地所有者和资本家没有来分享,生产出来的商品的价值仍然同从前一样。

土地所有权和资本,对于它们的所有者来说,是收入的源泉,也就是说,使它们的所有者有权占有劳动创造的价值的一部分,可是它们并不因此就成为它们的所有者占有的价值的源泉。但是,说工资构成交换价值的原始源泉,同样是不正确的,虽然工资,或者确切些说,劳动能力的不断出卖,也构成工人的收入的源泉。创造价值的是工人的劳动,而不是工人的工资。工资只不过是已经存在的价值,或者从整个生产来看,只不过是工人创造的价值中由工人自己占有的那一部分。但是这种占有并不创造价值。因此,工人的工资可以增减,但并不影响他们生产的商品的价值。[263]

[265]\fontbox{~\{}对于上述内容,还要补充以下一段引文,以证明亚·斯密把商品价值被占有时分成的不同项目说成这个价值的源泉。他在驳斥了那种认为利润只是资本家的工资或“监督劳动的工资”的别名的看法之后,得出结论说:

\begin{quote}“因此,在商品\textbf{价格}中,基金即资本的利润是与工资根本\textbf{不同的价值源泉},它受完全不同的原则支配。”(第 1 篇第 6 章)\end{quote}

然而,斯密刚才还证明,工人加到材料上的价值在工人和资本家之间以工资和利润的形式分配;因此,劳动是唯一的\textbf{价值源泉},“工资价格”和“利润价格”都是从这个价值源泉产生出来的。但是这些“价格”本身——无论工资还是利润——都不是\textbf{价值源泉}。\fontbox{\}~}[265]

\tsectionnonum{[(7)斯密对价值和收入的关系的看法的二重性。斯密关于“自然价格”是工资、利润和地租的总和这一见解中的循环论证]}

[263]在这里,我们想完全不谈亚·斯密在多大程度上把地租看作商品价格的构成要素。这个问题在这里对我们的研究更是无关紧要,因为斯密把地租看成和利润完全一样,纯粹是剩余价值的一部分,即“工人加到原料上的劳动中的扣除部分”。从而,斯密[264]实质上也把地租理解为“利润中的扣除部分”,因为全部无酬剩余劳动是由同劳动对立的资本家\textbf{直接}占有,不管他以后还要同生产条件所有者(无论土地所有者还是资本出借人)按哪些项目分享这个剩余价值。所以,为简单起见,我们将只谈工资和利润,作为新创造的价值分成的两个项目。

假定某一商品体现 12 小时的劳动时间(消费在这个商品上的原料和劳动工具的价值\textbf{撇开不谈})。这个商品的价值本身,我们只能用货币来表现。再假定 5 先令也体现 12 小时的劳动时间。在这种情况下,商品价值就等于 5 先令。亚·斯密所理解的“商品的自然价格”不是别的,正是以货币表现的商品价值。(商品的市场价格当然高于或低于商品的价值。甚至商品的平均价格也\textbf{总是不同于}商品的价值,这一点我将在后面说明。\endnote{“平均价格”(《Durchschnittspreis》)这一术语,马克思这里是指“生产价格”,就是指生产费用(C+v)加平均利润。马克思在《剩余价值理论》第二册——论洛贝尔图斯一章和《李嘉图和亚当·斯密的费用价格理论》一章中,考察了商品价值和商品的“平均价格”之间的相互关系问题。“平均价格”这一术语本身说明,这里所指的,正如马克思在手稿第 605 页(《李嘉图的地租理论[结尾]》一章)所解释的那样,是“一个相当长的时期内的平均市场价格,或者市场价格所趋向的中心”。——第 76 页。}但是亚·斯密在考察“自然价格”时,根本没有提到这个问题。况且,如果没有对价值本性的正确看法作基础,那就不能理解商品的市场价格,更不能理解商品的平均价格的波动。)

如果商品中包含的剩余价值,占商品总价值的 20\%,或者同样可以说,占商品中包含的必要劳动的 25\%,那末,这 5 先令价值,即商品的“自然价格”,就可以分为 4 先令工资和 1 先令剩余价值(在这里我们仿效亚·斯密,也把剩余价值叫做利润)。说不依赖于工资和利润而决定的商品价值量,或商品的“自然价格”,可以分为 4 先令工资(“劳动价格”)和 1 先令利润(“利润价格”),这是对的。但是,说商品价值由不受商品价值调节的工资价格和利润价格相加或合计而成,那就错了。在后一情况下,就找不到任何理由说明:为什么商品总价值不会随着人们假定工资等于 5 先令、利润等于 3 先令等等情况而成为 8 先令、10 先令等等。

亚·斯密在研究工资的“自然率”或工资的“自然价格”时所遵循的指导线索是什么呢\fontbox{?}是再生产劳动能力所必需的生活资料的自然价格。但是,他又用什么来决定这些生活资料的自然价格呢\fontbox{?}当他一般地决定这个价格时,他又回到正确的价值规定上来,也就是说,回到价值决定于生产这些生活资料所必要的劳动时间这个规定上来。但是,只要斯密离开这条正确的道路,他就陷入循环论证。决定工资自然价格的这些生活资料的自然价格,他是用什么来决定的呢\fontbox{?}用“工资”、“利润”、“地租”的自然价格;这三者构成这些生活资料的自然价格,也构成一切商品的自然价格。如此反复,以至无穷。关于供求规律的空谈,当然无助于摆脱这种循环论证。因为“自然价格”,或者说,与商品价值相适应的价格,恰好发生在供求彼此相符的时候,也就是在商品价格不因供求的波动而高于或低于商品价值的时候,换句话说,在商品的费用价格\endnote{“费用价格”(《Kostenpreis》或《Kostpreis》,《costprice》)这一术语,马克思用在三种不同的意义上:(1)资本家的生产费用(C+v),(2)同商品的价值一致的商品的“内在的生产费用”(C+v+m),(3)生产价格(C+v+平均利润)。这里,这一术语是用在第二种意义上,也就是指内在的生产费用。在《剩余价值理论》第二册中,“费用价格”这一术语马克思是用在第三种意义上,即生产价格,或“平均价格”。在那里马克思直接把这些术语等同了起来。例如,在手稿第 529 页,马克思写道:“……不同于价值本身的平均价格,即我们后面所说的费用价格,这个费用价格不直接决定于商品价值,而决定于预付在这些商品上的资本加平均利润。”在第 624 页,马克思指出:“价格是提供商品的必要条件,是使商品生产出来并作为商品出现在市场上的必要条件,它当然是商品的生产价格或费用价格。”在《剩余价值理论》第三册中,《Kostenpreis》这一术语马克思有时用在生产价格的意义上,有时用在资本家的生产费用的意义上,也就是指 C+v。《Kostenpreis》这一术语所以有三种用法,是由于《Kosten》(“费用”、“生产费用”)这个词在经济科学中被用在三种意思上,正如马克思在《剩余价值理论》第三册(1861—1863 年手稿第 788—790 页和第 928 页)特别指出的,这三种意思是:(1)资本家预付的东西,(2)预付资本的价格加平均利润,(3)商品本身的实在的(或内在的)生产费用。除了资产阶级政治经济学古典作家使用的这三种意思以外,“生产费用”这一术语还有第四种庸俗的意思,即让·巴·萨伊给“生产费用”下的定义:“生产费用是为劳动、资本和土地的生产性服务支付的东西。”(让·巴·萨伊《论政治经济学》1814 年巴黎第 2 版第 2 卷第 453 页)马克思坚决否定了对“生产费用”的这种庸俗的理解(例如见《剩余价值理论》第 2 册手稿第 506 页和第 693—694 页)。——第 77 页。}(或卖者供应的商品的价值)同时就是需求所支付的价格的时候。

[265]但是,前面已经说过,亚·斯密在研究工资的自然价格时,他事实上——至少在一些地方——又回到商品的正确的价值规定上来了。相反,在关于利润的自然率或利润的自然价格的那一章,就本应解决的题目来说,他却在毫无意义的老生常谈和同义反复之中迷失了方向。事实上,他原来是用商品价值来调节工资、利润和地租的。但是后来,他反过来了(这更接近于经验的外观和平常的印象),企图用工资、利润和地租的自然价格的相加数来决定商品的自然价格。李嘉图的主要功绩之一,就是消除了这种混乱。以后我们讲到李嘉图的时候,还要简单地谈谈这一点。\endnote{在《剩余价值理论》第二册中,篇幅巨大的论李嘉图的那一节是在马克思手稿第 XI、XII 和 XIII 本,其中有一章《李嘉图和亚·斯密的费用价格理论(批驳部分)》,马克思在那里又回过头来分析斯密的“自然价格”观点(手稿第 549—560 页)。——第 78 页。}

在这里我们还要指出的只是下面一点:作为支付工资和利润的基金的商品价值的\textbf{已知量},在工业家面前,从经验上看,表现为这样的形式:不管工资有什么波动,商品的一定的市场价格在一个时期内保持不变。

总之,应当注意亚·斯密书中的奇怪的思路:起先他研究商品的价值,在一些地方正确地规定价值,而且正确到这样的程度,大体上说,他找到了剩余价值及其特殊形式的源泉——他从商品价值推出工资和利润。但是后来,他走上了相反的道路,又想倒过来从工资、利润和地租的自然价格的相加数来推出商品价值(他已经从商品价值推出了工资和利润)。正由于后面这种情况,斯密对于工资、利润等等的波动给予商品价格的影响,没有一个地方做出了正确的分析,因为他没有基础。[VI—265]

\centerbox{※     ※     ※}

[VIII—364]\fontbox{~\{}\textbf{亚·斯密。价值及其组成部分}。斯密违反他原来的正确观点而发挥的错误看法(见前面),也表现在下面这段话里:

\begin{quote}“地租成为……商品价格的组成部分,但与利润和工资完全不同。利润和工资的高低是\textbf{谷物价格高低的原因,而地租的多少是这一价格的结果}。”(《国富论》第 1 篇第 11 章)\endnote{马克思在《剩余价值理论》第二册《亚·斯密的地租理论》一章(手稿第 620—625 页)中,对斯密关于地租以不同于利润和工资的方式加入产品价格的论点作了批判的分析。斯密《国富论》的这段引文,马克思引自加尼耳的《论政治经济学的各种体系》一书(1821 年巴黎版第 2 卷第 3 页)。——第 78 页。}\fontbox{\}~}[VIII—364]\end{quote}

\tsectionnonum{[(8)斯密的错误——把社会产品的全部价值归结为收入。斯密关于总收入和纯收入的看法的矛盾]}

[VI—265]现在我们来谈谈同商品价格或商品价值(这里还假定它们两者是同一个东西)的分解有关的另一个问题。假定亚·斯密正确地作了计算,也就是说,他以商品价值为出发点,把商品价值分解成这个价值在不同的生产当事人之间进行分配的各个组成部分,而不想倒过来从这些组成部分的价格推出价值,——这一点这里撇开不谈。我们也不谈他的片面看法,即把工资和利润只当作分配的形式,因而在同等意义上把这两者描写成由它们的所有者消费的收入。撇开这一切不谈,应当指出,亚·斯密自己[对于把产品的全部价值归结为收入]曾提出某种疑问;这里他胜过李嘉图的地方,仍然不是他正确地解决了他所提出的疑问,而是他一般地提出了这种疑问。

[266]亚·斯密是这样说的:

\begin{quote}“这三部分〈工资、利润、土地所有者的地租〉看来直接地或最终地构成谷物的\textbf{全部}价格〈指一般商品的全部价格,亚·斯密在这里说谷物,是因为在他看来,有许多商品的价格并不包括地租这一组成部分〉。也许有人以为必须有\textbf{第四个部分},用来补偿租地农场主的资本,或者说,补偿他的役畜和其他农具的损耗。但是必须考虑到,任何一种农具的价格,例如一匹役马的价格,本身又是由上述三个部分构成:养马用的土地的地租,养马的\textbf{劳动},预付这块土地的地租和这种劳动的工资的租地农场主的利润。\fontbox{~\{}这里利润表现为原始形式,也把地租包括在内。\fontbox{\}~}因此,谷物的价格虽然要补偿马的价格和给养费用,但\textbf{全部}价格仍然直接地或最终地分解为这三个部分:地租、劳动和利润。”(第 1 篇第 6 章)(在这里,斯密突然不说“工资”,而说“劳动”,可是他又说“地租”和“利润”,而不说“土地所有权”和“资本”,这是完全荒谬的。)\end{quote}

但同样明显的是,这里要注意到下面的情况:正象租地农场主把马和犁的价格包括在小麦的价格中一样,那些把马和犁卖给租地农场主的养马人和制犁人,也会把生产工具的价格(例如养马人可能把另一匹马的价格)和原料(饲料和铁)的价格包括在马和犁的价格中,而养马人和制犁人用来\textbf{支付}工资和利润(和地租)的基金,仅仅由他们在自己的生产领域中加到现有不变资本价值额上的新劳动组成。因此,如果亚·斯密在谈到租地农场主的时候,承认在他的谷物价格中,除了他支付给自己和别人的工资、利润和地租以外,还包括一个\textbf{不同于这些部分的第四个组成部分},即租地农场主使用的不变资本的价值,例如马、农具等等的价值,那末,对于养马人和农具制造人来说,这也是适用的,斯密把我们从本丢推给彼拉多\authornote{此语出自福音书路加福音第 23 章。本丢和彼拉多是罗马的一个犹太总督的名和姓。据福音书记载,耶稣被解送到本丢那里受审,本丢知道耶稣是加利利人,属希律所管,就把他送交给希律,希律拒绝审讯,又把他送回彼拉多。人们沿用此语时省去希律,而说“从本丢推给彼拉多”,意思是推来推去,不解决问题。——译者注}完全是徒劳无益的。而且选用租地农场主的例子,把我们推来推去,尤其不恰当,因为在这个例子中,不变资本项目中包括了完全不必向别人购买的东西,即种子,难道价值的这一组成部分会分解成谁的工资、利润和地租吗\fontbox{?}

但是,我们且往前走,先看看斯密是否始终贯彻了自己的观点:一切商品的价值都可以归结为某一收入源泉或全部收入源泉——工资、利润、地租,也就是说,一切商品都可以作为供消费用的产品来消费掉,或者说,无论如何都可以这样或那样地用于个人需要(而不是用于生产消费)。不过[267]还要先说明一点。例如在采集浆果等等的时候,浆果等等的价值可以只归结为工资,虽然在大多数情况下也要有篮筐等等用具作为劳动资料。可是这里谈的是资本主义生产,这一类的例子是根本不相干的。

最初他又重复第一篇第六章说过的观点。

在\textbf{第二篇第二章}(\textbf{加尔涅}的译本,第 2 卷第 212—213 页)中说:

\begin{quote}“我们说过……\textbf{大部分商品的价格}都分解为三部分,其中一部分支付工资,第二部分支付资本利润,第三部分支付地租。”\end{quote}

按照这种说法,一切商品的全部价值都可分解为各种收入,并且作为消费基金而归于依靠这种收入过活的这个或那个阶级。但是,既然一国的总产量,例如年产量,只由已生产出来的商品的价值总额构成,而这些商品中的单个商品的价值又分解为各种收入,那末,商品的总额——劳动的年产品,即总收入,也就能够在一年内以这种形式消费掉。可是斯密马上起来反驳自己:

\begin{quote}“既然就每一个特殊商品分别来说是如此,那末,就形成每一个国家的土地和劳动的全部年产品的一切商品\textbf{整体}来说也必然是如此。这个年产品的\textbf{全部价格\CJKunderdot{或}交换价值},必须分解为同样三个部分,在国内不同居民之间进行分配,或是作为他们的劳动的工资,或是作为他们的资本的利润,或是作为他们占有的土地的地租。”(同上,第 213 页)\end{quote}

这确实是必然的结论。适用于单个商品的必定适用于商品总额。但是亚当说并非如此。他接着说:

\begin{quote}“虽然一国土地和劳动的年产品的总价值这样在国内不同居民之间分配,构成他们的收入,但是,就象我们把私人地产的收入区分为\textbf{总收入}和\textbf{纯收入}一样,我们也可以对一个大国\textbf{全体居民}的收入作这样的区分。”(同上,第 213 页)\end{quote}

(等一等!他在前面对我们说的恰好相反:在单个租地农场主的产品中,例如在小麦中,我们还可以在这一产品价值分解成的各部分中,分出第四个部分,就是只补偿已使用的不变资本的部分;这对于单个租地农场主\textbf{直接地说}是正确的,但如果我们再往前走,就会看到,作为租地农场主的不变资本的那一部分,在更早的阶段——在别人手里的时候,在成为租地农场主的资本以前,就分解为工资、利润等等,一句话,分解为各种收入。因此,如果说商品从它们在单个生产者手中来考察,包含一部分不构成收入的价值,是正确的,那末从“一个大国全体居民”来说,在他看来就是不正确的了,因为在一个人手中成为不变资本的东西之所以具有价值,是由于它作为工资、利润和地租的总价格来自别人手中。现在他说的恰好相反。)

亚·斯密接着说:

\begin{quote}[268]“私人地产的\textbf{总}收入包括租地农场主所支付的一切;\textbf{纯收入}则是扣除管理、修理的开支以及其他一切\textbf{必要费用}之后,留归\textbf{土地所有者}的东西,换句话说,是他不损及自己的财产而可以归入用于直接消费即吃喝等等的基金的东西。土地所有者的实际财富不同他的\textbf{总}收入成比例,而同他的\textbf{纯}收入成比例。”\end{quote}

(第一,斯密在这里谈的是不相干的东西。租地农场主作为地租支付给土地所有者的,和他作为工资支付给工人的毫无差别,都同他自己的利润一样,是商品价值或价格中分解为各种收入的那一部分。问题在于,商品是否还包括价值的另一个组成部分。他在这里是承认这一点的,正象他在谈到租地农场主时曾经不得不承认这一点一样,不过这种承认并没有妨碍他宣称,租地农场主生产出来的谷物(即他的谷物的价格\textbf{或}交换价值)只分解为各种收入。第二,我们要顺便指出下面一点。单个租地农场主作为\textbf{租地农场主}能够支配的实际财富,取决于他的利润。但另一方面,他作为商品所有者,可以把他的农场卖掉,或者说,如果土地不属于他,可以把土地上的全部不变资本如役畜、农具等卖掉。他由此所能实现的价值,从而,他所能支配的财富,就取决于他的不变资本的价值,也就是取决于这个不变资本的大小。但是他只能把这些东西再卖给另一个租地农场主,而在后者手中,这些东西并不是可以自由支配的财富,而是不变资本。因此,我们仍然没有前进一步。)

\begin{quote}“一个大国全体居民的\textbf{总}收入,包括他们的土地和劳动的\textbf{全部}年产品\end{quote}

(前面我们听到,这全部产品——它的价值——都分解为工资、利润和地租,也就是说,仅仅分解为各种形式的纯收入);

\begin{quote}\textbf{纯}收入是在先扣除\textbf{固定资本}的维持费用,再扣除\textbf{流动资本}的维持费用之后,余下供他们使用的部分\end{quote}

(可见,斯密现在把劳动工具和原料扣除了),

\begin{quote}或者说,是他们可以列入\textbf{直接消费基金}……而不侵占资本的部分。”\end{quote}

(因此,我们现在知道,商品总量的价格或交换价值,无论就单个资本家来说,还是就全国来说,都还包含第四个部分,这部分对任何人都不构成收入,既不能归结为工资、利润,也不能归结为地租。)

\begin{quote}“维持\textbf{固定资本}的全部费用,显然要从社会纯收入中排除掉。无论是为维持有用机器、生产工具、经营用的建筑物等等\textbf{所必需的材料},还是为使这些材料转化为适当的形式\textbf{所必需的劳动的产品},从来都不可能成为社会\textbf{纯}收入的一部分。\textbf{这种劳动的价格},当然可以是社会纯收入的一部分,因为从事这种劳动的工人,可以把[269]他们\textbf{工资的全部价值}用在他们的\textbf{直接消费基金}上。但是,在其他各种劳动中,\textbf{劳动的价格和劳动的产品二者都加入这个消费基金};劳动的价格加入工人的消费基金,劳动的产品则加入另一些人的消费基金,这些人靠这种工人的劳动来增加自己的生活必需品、舒适品和享乐品。”(同上,第 214—215 页)\authornote{马克思在这里用铅笔加了一句话:“这毕竟是比其他经济学家的看法更接近正确的观点”。——编者注}\end{quote}

这里,亚·斯密又避开了他应该回答的问题——关于商品全部价格的第四个部分,即不归结为工资、利润、地租的那一部分的问题。首先我们指出一个大错误。要知道,在机器厂主那里,也象在其他所有工业资本家那里一样,把机器等等的原料变成适当形式的劳动分解为必要劳动和剩余劳动,因而不仅分解为工人的工资,而且分解为资本家的利润。但原料的价值和工人把这些原料变成适当形式时使用的工具的价值,既不归结为工资,也不归结为利润。那些从性质来说不用于个人消费而用于生产消费的产品并不加入直接消费基金,这一点,是与问题毫无关系的。例如种子(播种用的那部分小麦),从性质来说也可以加入消费基金,但是从经济上说必须加入生产基金。其次,说用于个人消费的产品的全部价格同产品一起都加入\textbf{消费基金},是完全错误的。例如麻布,如果不是用来作帆或用于别的生产目的,它就作为产品全部加入消费;但是对于麻布的价格却不能这样说,因为这个价格的一部分补偿麻纱,另一部分补偿织机等等,麻布的价格只有一部分归结为这种或那种收入。

亚当刚对我们说过,机器、经营用的建筑物等等所必需的材料,也同由这些材料制造的机器等等一样,“从来都不可能成为\textbf{纯}收入的一部分”;这就是说,它们加入总收入。但是就在这些话后面不远,就在第二篇第二章第 220 页上,他却说出相反的话:

\begin{quote}“机器和工具等等构成个人或社会的\textbf{固定资本},它们既不构成\textbf{个人或社会的总收入}的一部分,也不构成\textbf{个人或社会的纯收入}的一部分,\textbf{货币}也是一样”等等。\end{quote}

亚当的混乱、矛盾、离题,证明他既然把工资、利润、地租当作产品的交换价值或全部价格的组成部分,在这里就必然寸步难行、陷入困境。

\tsectionnonum{[(9)萨伊是斯密理论的庸俗化者。萨伊把社会总产品和社会收入等同起来。施托尔希和拉姆赛试图把这两者区别开来]}

萨伊把亚·斯密的不一贯的说法和错误的意见化为十分一般的词句,来掩饰他自己的陈腐的浅薄见解。我们在他的著作中读到:

\begin{quote}“从整个国家来看,根本没有纯产品。因为\textbf{产品}的价值等于产品的生产\textbf{费用},所以,如果我们把这些\textbf{费用}扣除,也就把全部\textbf{产品价值}扣除……\textbf{年收入}就是\textbf{总收入}。”(《论政治经济学》1817 年巴黎第 3 版第 2 卷第 469 页)\end{quote}

年产品总额的价值等于物化在这些产品中的[270]劳动时间量。如果从年产品中把这个总价值扣除,那末实际上——就价值来说——就没有任何价值留下来了,因而无论纯收入还是总收入统统都没有了。但是,萨伊认为,每年生产的价值,当年会消费掉。所以,对整个国家来说,根本不存在纯产品,只存在总产品。第一,说每年生产的价值,当年会消费掉,这是错误的。固定资本的大部分就不是这种情况。一年内生产的价值大部分进入劳动过程,而不进入价值形成过程;这就是说,并不是这些东西的总价值全部在一年内消费掉。第二,每年消费的价值中有一部分是由不加入消费基金而作为生产资料来消费的那种价值构成的,这些生产资料从生产过程出来,又以自身的实物形式或以等价物的形式,重新回到生产过程中去。另一部分则由扣除上述第一部分之后能够加入个人消费的那种价值构成;这部分价值就构成“纯产品”。

关于萨伊的这种胡言乱语,施托尔希说:

\begin{quote}“很明显,年产品的价值分成资本和利润两部分。\textbf{年产品价值的这两部分}中,每一部分\textbf{都要有规则地用来购买国民所需要的产品},以便维持该国的资本和更新它的消费基金。”(\textbf{施托尔希}《政治经济学教程》第 5 卷:《论国民收入的性质》1824 年巴黎版第 134—135 页)“我们要问,靠自己的劳动来满足自己的全部需要的家庭(我们在俄国可以看到许多这样的例子)……其\textbf{收入}是否等于这个家庭的土地、资本和劳动的总产品\fontbox{?}难道一家人能够住自己的粮仓和畜棚,吃自己的谷种和饲料,穿自己役畜的毛皮,用自己的农具当娱乐品吗\fontbox{?}按照萨伊先生的论点,对所有这些问题必须作肯定的回答。”(同上,第 135—136 页)“萨伊把总产品看成社会的收入,并由此得出结论说,社会可以把等于这个产品的价值消费掉。”(同上,第 145 页)“一国的纯收入,不是由已生产出来的价值\textbf{超过消费了的价值总额的}余额构成,就象萨伊所描写的那样,而只是由已生产出来的价值\textbf{超过为生产目的而消费了的价值的}余额构成。因此,如果一个国家在一年内消费这全部余额,那末它就是消费自己的全部纯收入。”(同上,第 146 页)“如果承认一个国家的收入等于该国的总产品,就是说不必扣除任何\textbf{资本},那末也必须承认,这个国家可以把年产品的全部价值非生产地消费掉,而丝毫无损于该国的未来收入。”(同上,第 147 页)“\textbf{构成一个国家的〈不变〉资本的产品,是不能消费的}。”(同上,第 150 页)\end{quote}

\textbf{拉姆赛}(\textbf{乔治})在《论财富的分配》(1836 年爱丁堡版)中,对于亚·斯密称为“全部价格的第四个组成部分”的东西,也就是我为了同花在工资上的资本相区别而称为不变资本的东西,提出如下意见:

\begin{quote}[271]他说:“李嘉图忘记了,全部产品不仅分为工资和利润,而且还必须有一部分补偿固定资本。”(第 174 页注)\end{quote}

拉姆赛理解的“固定资本”,不仅包括生产工具等等,而且包括原料,总之,就是我在各生产领域内称为不变资本的东西。当李嘉图谈到产品分为利润和工资的时候,他总是假定,预付在生产上并在生产中消费了的资本已经扣除。然而拉姆赛基本上是对的。李嘉图对资本的不变部分没有作任何进一步的分析,忽视了它,犯了重大的错误,特别是把利润和剩余价值混淆起来,其次在研究利润率的波动等等问题上也犯了错误。

现在我们听听拉姆赛本人是怎样说的:

\begin{quote}“怎样才能把产品和花费在产品上的资本加以比较呢\fontbox{?}……如果指整个国家而言……那末很清楚,花费了的资本的各个不同要素应当在这个或那个经济部门再生产出来,否则国家的生产就不能继续以原有的规模进行。工业的原料,工业和农业中使用的工具,工业中无数复杂的机器,生产和贮存产品所必需的建筑物,这一切都应当是一个国家总产品的组成部分,同时也应当是一个国家资本主义企业主的全部预付的组成部分。因此,总产品的量可以同全部预付的量相比较,因为每一项物品都可以看成是与同类的其他物品并列的。”(同上,第 137—139 页)“至于单个资本家,由于他不是以实物来补偿自己的支出,他的支出的大部分必须通过交换来取得,而交换就需要一定份额的产品,由于这种情况,单个资本主义企业主不得不把更大的注意力放在自己产品的交换价值上,而不是放在产品的量上。”(同上,第 145—146 页)“他的\textbf{产品的价值}愈高于预付\textbf{资本的价值},他的利润就愈大。因此,资本家计算利润时,是拿价值同价值相比,而不是拿量同量相比……利润的上升或下降,同总产品或它的\textbf{价值}中用来\textbf{补偿必要预付}的那个份额的下降或上升成比例……因此,利润率决定于以下两个因素:第一,全部产品中归工人所得的那个份额;第二,为了以实物形式或通过交换来补偿固定资本而必须储存的那个份额。”(同上,第 146—148 页)\end{quote}

\fontbox{~\{}拉姆赛在这里谈的关于利润率的意见,要放到第三章(关于利润)去考察。\endnote{马克思这里说的“第三章”是指关于“资本一般”的研究的第三部分。这一章的标题应为:《资本的生产过程和流通过程的统一,或资本和利润》。以后(例如,见第 IX 本第 398 页和第 XI 本第 526 页)马克思不用“第三章”而用“第三篇”(《dritterAbschnitt》)。后来他就把这第三章称作“第三册”(例如,在 1865 年 7 月 31 日给恩格斯的信中)。关于“资本一般”的研究的“第三章”马克思是在第 XVI 本开始的。从这“第三章”或“第三篇”的计划草稿(见本册第 447 页)中可以看出,马克思打算在那里写两篇专门关于利润理论的历史补充部分。但是马克思在写作《剩余价值理论》的过程中,就已在自己的这一历史批判研究的范围内,详细地批判分析了各种资产阶级经济学家对利润的看法。因此,马克思在《剩余价值理论》中,特别是在这一著作的第二册和第三册中,就已进一步更充分地揭示了由于把剩余价值和利润混淆起来而产生的理论谬误。——第 7、87、272 页。}重要的是,他正确地强调指出了这个要素。从一方面来看,李嘉图说,构成不变资本(拉姆赛说的“固定资本”,就是指这个)的那些商品的落价,总会使现有资本的一部分价值下降,这是对的。这种说法特别适用于真正的固定资本——机器等等。同全部资本相比剩余价值的增加,如果是由资本家的不变资本总价值下降引起的(资本家在总价值下降之前就占有了这些不变资本),这对单个资本家来说毫无利益可言。不过这种说法仅仅在极小的程度上适用于由原料或成品(不加入固定资本的成品)构成的那部分资本。原料或成品的现有量可能发生价值下降,但这个现有量同总产量相比,始终只是一个微不足道的量。在单个资本家那里,这种说法仅仅在很小的程度上适用于他投入流动资本的那部分资本。从另一方面来看,很清楚,因为利润等于剩余价值和总预付资本之比,因为可以被吸收的劳动量不取决于价值,而取决于原料的量和生产资料的效率,不取决于它们的交换价值,而取决于它们的使用价值,所以,其产品[272]构成不变资本要素的那些部门中的劳动的生产能力愈高,生产一定量剩余价值所必需的不变资本的支出愈少,这个剩余价值和全部预付资本之比就愈大,从而,在剩余价值量已知的情况下,利润率就愈高。\fontbox{\}~}

(被拉姆赛当作两个独立现象来考察的东西——在再生产过程中,就全国而言,是以产品补偿产品,就单个资本家而言,是以价值补偿价值——可归结为两个观点,这两个观点即使对于单个资本来说,在分析\textbf{资本的流通过程,同时也就是再生产过程}时,都是应当加以考虑的。)

拉姆赛没有解决亚·斯密所研究的并使他陷入重重矛盾的实际困难。为了直截了当地讲清楚这个困难,我们把它表述如下:整个资本(作为价值)都归结为劳动,它无非是一定量的物化劳动。但是,有酬劳动等于工人的工资,无酬劳动等于资本家的利润。因此,整个资本都可以直接地或间接地归结为工资和利润。也许,什么地方在完成这样一种劳动,它既不归结为工资,也不归结为利润,它的目的只是为了补偿在生产过程中消费了的、同时又是作为再生产条件的那种价值\fontbox{?}但是谁来完成这种劳动呢\fontbox{?}要知道,工人的一切劳动都分为两部分,一部分用来恢复他自身的生产能力,另一部分构成资本利润。

\tsectionnonum{[(10)]研究年利润和年工资怎样才能购买一年内生产的、除利润和工资外还包含不变资本的商品}

\tsubsectionnonum{[(a)靠消费品生产者之间的交换不可能补偿消费品生产者的不变资本]}

为了把各种虚假的搀杂的东西从问题中清除出去,首先还要指出下面一点。当资本家把自己的利润即自己的收入的一部分转化为资本,转化为劳动资料和劳动材料的时候,他是用工人为他无偿地完成的那部分劳动来支付这两者的。这里有一个新的劳动量,它构成一个新商品量的等价物,而这些商品按其使用价值来说就是劳动资料和劳动材料。所以,这种情况属于资本积累的问题,它本身不包含任何困难;这里我们碰到的,是不变资本超过它原有界限的增长,或者说,超出已经存在的和待补偿的不变资本量之上的新不变资本的形成。困难在于\textbf{已经存在的}不变资本的再生产,而不在于超出有待再生产的不变资本量之上的新不变资本的形成。新的不变资本显然来源于利润;它以收入的形式存在极短时间,随后即转化为资本。这部分利润归结为\textbf{剩余劳动时间,即使没有资本存在,社会也必须不断地完成这个剩余劳动时间,以便能支配一个所谓发展基金——仅仅人口的增长,就已使这个发展基金成为必要的了}。\endnote{马克思在《资本论》第三卷第四十九章对这里所提出的问题作了如下的表述:“怎样才能使工人用他的工资、资本家用他的利润、土地所有者用他的地租买到各自不是只包含这三个组成部分之一,而是包含所有这三个组成部分的商品\fontbox{?}怎样才能使工资、利润、地租,即收入的三个源泉加在一起的价值总额买到构成这些收入所得者的全部消费的商品(这些商品除了价值的这三个组成部分之外,还包含价值的另一个组成部分,也就是不变资本)\fontbox{?}他们怎样才能用三部分的价值买到四部分的价值\fontbox{?}”接着马克思写道:“我们已经在第二卷第三篇作了分析。”这是指《社会总资本的再生产和流通》一篇(见马克思《资本论》第 2 卷第 3 篇)。——第 89 页。}

\fontbox{~\{}在拉姆赛的著作第 166 页上,我们看到他对不变资本作了很好的说明,不过他仅仅是从使用价值方面来谈的。他说:

\begin{quote}“无论总产品〈例如租地农场主的总产品〉的数额是多少,其中用来补偿在生产过程中以不同形式消费了的全部东西的那个量,不应当有任何变动。只要生产以原有的规模进行,这个量就必须认为是\textbf{不变的}。”\fontbox{\}~}\end{quote}

所以,首先必须从以下事实出发:和已经存在的不变资本的再生产不同,不变资本的新形成是以利润为源泉;这里假定:一方面,工资只够用来再生产劳动能力,另一方面,全部剩余价值统统归入“利润”范畴,因为\textbf{直接占有}全部剩余价值的不是别人,正是产业资本家,不管他以后在什么地方还得把其中的一部分分给谁。

\begin{quote}\fontbox{~\{}“资本主义企业主是财富的总分配者:他付给工人工资,付给资本家(货币资本家)利息,付给土地所有者地租。”(\textbf{拉姆赛}的著作第 218—219 页)\end{quote}

我们把全部剩余价值称为利润,是把资本家看成这样一种人:(1)他直接占有生产出来的全部剩余价值,(2)他拿这种剩余价值在他自己、货币资本家和土地所有者之间进行分配。\fontbox{\}~}

[VII—273]然而,这个新的不变资本由利润产生,无非是说,新的不变资本来源于工人的一部分剩余劳动。这好比野蛮人除了打猎的时间以外,还必须花费一定的时间来制造弓箭,又好比农民在宗法式农业条件下,除了耕地的时间以外,还要花费一定量的劳动时间来制造他的大部分工具。

但是,这里问题在于:究竟由谁劳动,以补偿生产中已经耗费的不变资本的等价\fontbox{?}工人为自己完成的那部分劳动,补偿他的工资,或者从整个生产来看,就是创造他的工资。相反,他的剩余劳动,即构成利润的劳动,一部分成为资本家的消费基金,一部分转化为追加资本。但是,资本家并不是用这个剩余劳动或利润,来补偿他自己生产中已耗费的资本。\fontbox{~\{}如果情况是这样的话,那末剩余价值就不会成为形成新资本的基金,而变成保存旧资本的基金了。\fontbox{\}~}然而,形成工资的必要劳动和形成利润的剩余劳动,已经构成了整个工作日,再没有任何其他的劳动存在的余地了。(就算资本家担任的“监督劳动”也归入工资之内。从这方面看,资本家是雇佣工人,不过不是别的资本家的雇佣工人,而是他自己的资本的雇佣工人。)那末,补偿不变资本的那个源泉,那个劳动,究竟从何而来呢\fontbox{?}

花在工资上的那部分资本,由新的生产来补偿(这里不谈剩余劳动)。工人消费工资,但他耗费掉多少旧劳动,他就加进多少新劳动。如果我们考察整个工人阶级,而不受分工的干扰,那末就会看到,工人不仅会再生产出同一个价值,而且会再生产出同样的使用价值。这样,根据工人的劳动生产率的不同,同一个价值,同一个劳动量,会以较大量或较小量的同样的使用价值形式再生产出来。

拿社会来说,无论什么时候我们都会看到,有一定的不变资本作为生产条件,以极不相同的比例,同时存在于一切生产领域,它永远属于生产,并且必须归还给生产,就象种子要归还给土地一样。这个不变部分的\textbf{价值}固然可能降低或提高,这取决于构成这个不变部分的那些商品的再生产是更便宜,还是更贵。但是,这种\textbf{价值变动}决不会影响下面这一点:作为生产条件进入生产过程的资本不变部分,在生产过程中是一个事先已知的价值,它必须再现在产品的价值中。因此,不变资本本身的价值变动,在这里可以不加考虑。在一切场合,不变资本在这里都表现为一定量的\textbf{过去的、物化的}劳动,这一定量的劳动要作为决定产品价值的因素之一转移到产品价值中去。因此,为了更明确地说明问题,我们假定资本不变部分的生产费用\endnote{“生产费用”(《Produktionskosten》)这一术语这里是用在“内在的生产费用”的意义上,即指 C+v+m。参看注 41。——第 91 页。}或价值也保持不变,始终一样。又如,不变资本的价值在一年内不是全部都转移到产品中,而是(就固定资本的价值而言)在许多年内才转移到这个时期所生产的产品总量中,这种情况也不会使问题发生任何变化。因为这里谈的仅仅是一年内实际消费的,因而必须在当年得到补偿的那部分不变资本。

十分明显,不变资本的再生产问题,要在关于资本再生产过程或流通过程的那一篇里谈,但这并不妨碍在这里就把基本问题弄清楚。

[274]先谈工人的工资。工人为资本家一天劳动 12 小时,得到一定量的货币,假定这笔货币体现 10 劳动小时。这笔工资转化为生活资料。这些生活资料全都是商品。假定这些商品的价格同它们的价值相等。但是,在这些商品的价值中,有一个组成部分,是抵补商品中包含的原料和损耗了的生产资料的价值的。但是,这些商品的价值的所有组成部分合在一起,也象工人支出的工资一样,只包含 10 劳动小时。假定这些商品的价值的 2/3 由它们包含的不变资本的价值构成,1/3 由完成生产过程并把产品变为消费品的劳动构成。这样,工人是用自己的 10 小时活劳动补偿 2/3 的不变资本和 1/3(当年加到对象上去的)活劳动。如果在生活资料中,即在工人购买的商品中,根本不包含不变资本;如果这些商品的原料不花费什么,并且生产这些商品时不需要任何劳动工具,那就可能出现以下两种情况之一。或者,商品象原先一样,还是包含 10 小时劳动。在这种情况下,工人就是用 10 小时活劳动补偿 10 小时活劳动。或者,由工人的工资转化成的、工人再生产其劳动能力所必需的那个使用价值量,只值 3+(1/3)小时劳动(假定没有劳动工具和那种本身就是劳动产品的原料)。在这种情况下,工人的必要劳动就只是 3+(1/3)小时,他的工资事实上就会降低到 3+(1/3)小时物化劳动时间。

假定商品是麻布,12 码麻布(在这里,我们当然根本不必注意实际价格如何)=36 先令,或 1 镑 16 先令。其中 1/3 为新加劳动,2/3 用于原料(纱)和机器的损耗。假定必要劳动时间等于 10 小时;因而剩余劳动就等于 2 小时。1 劳动小时用货币来表现,等于 1 先令。在这种情况下,12 劳动小时=12 先令,工资=10 先令,利润=2 先令。假定工人和资本家购买麻布作为消费品时,花掉全部工资和全部利润(共 12 先令),换句话说,花掉加到原料和机器上的全部价值,即在纱变为麻布的过程中物化的全部新的劳动时间量。(可能以后购买自己生产的产品要花费一个工作日以上的时间。)1 码麻布值 3 先令。把工资和利润加在一起,工人和资本家用 12 先令只能买到 4 码麻布。这 4 码麻布包含 12 劳动小时,然而其中只有 4 小时代表新加劳动,8 小时则代表物化在不变资本中的劳动。工资和利润加在一起,用 12 劳动小时只能买到自己总产品的 1/3,因为这个总产品的 2/3 由不变资本构成。12 劳动小时分为 4+8,其中 4 小时自己补偿自己,8 小时补偿那个同织布过程中加进的劳动无关的、以物化了的形式即纱和机器的形式加入织布过程的劳动。

因此,谈到用工资和利润交换或购买来作为消费品(或者是为了再生产本身的某种目的,因为购买商品的目的丝毫不会改变这里的问题)的那部分产品或商品,那末很清楚,这种产品的价值中相当于不变资本的部分,是由分解为工资和利润的新加劳动基金支付的。有多少不变资本以及有多少在最后生产过程中加进的劳动是用工资和利润加在一起来购买的;在生产的最后阶段加进的劳动按什么比例来支付,物化在不变资本中的劳动按什么比例来支付;——这一切都取决于它们作为价值组成部分加入成品的最初比例。为了简单起见,我们假定这个比例是:物化在不变资本中的劳动为 2/3,新加劳动为 1/3。

[275]因此,有两点是清楚的:

\textbf{第一},我们为麻布所假设的比例,也就是为工人和资本家把工资和利润实现在自己生产的商品上,即他们买回自己产品的一部分这种情况所假设的比例,——这个比例,即使工人和资本家把同一个价值额花在其他产品上,也保持不变。根据上面的假设,每个商品都包含 2/3 不变资本和 1/3 新加劳动,工资和利润加在一起,始终只能购买产品的 1/3。12 小时劳动=4 码麻布。如果这 4 码麻布转化为货币,它就以 12 先令的形式存在。如果这 12 先令又转化为麻布以外的其他商品,这笔货币购买的也是价值为 12 劳动小时的商品,其中 4 小时是新加劳动,8 小时是物化在不变资本中的劳动。因此,假设在其他商品中也象在麻布中一样,在生产的最后阶段加进的劳动和物化在不变资本中的劳动之间保持同样的最初比例,这个比例就是普遍的。

\textbf{第二},如果一天的新加劳动等于 12 小时,那末在这 12 小时中,只有 4 小时自己补偿自己,即补偿活的、新加的劳动,而 8 小时支付物化在不变资本中的劳动。但是,不由活劳动本身来补偿的这 8 小时活劳动究竟由谁来支付呢\fontbox{?}正是由包含在不变资本中并同 8 小时活劳动相交换的那 8 小时物化劳动来支付。

因此,毫无疑问,用工资和利润总额(两者加在一起,只不过代表新加到不变资本上的劳动总量)购买的那部分成品,都以它的各个要素的形式得到补偿。这部分成品所包含的新加劳动得到补偿,不变资本所包含的劳动量也得到补偿。其次,毫无疑问,不变资本所包含的劳动在这里从新加到不变资本上的活劳动基金中得到了自己的等价。

但是,在这里就发生困难了。12 小时织布劳动的总产品(这个总产品与织布劳动本身所生产的大不相同),等于 12 码麻布,价值为 36 劳动小时或 36 先令。但是,工资和利润合起来,或者说 12 小时总劳动时间,只能从这 36 劳动小时中买回 12 小时;换句话说,只能从总产品中买回 4 码,多 1 码也不行。那末,其余 8 码又将怎样呢\fontbox{?}(\textbf{福尔卡德、蒲鲁东}。)\endnote{“(福尔卡德、蒲鲁东)”这两个名字是马克思在手稿上用铅笔加上的。马克思这里指的是他在第 XVI 本札记本中从法国资产阶级政论家、庸俗经济学家福尔卡德发表在 1848 年《两大陆评论》杂志(第 24 卷第 998—999 页)上的文章《社会主义的战争》(第二篇)摘录的一段话。福尔卡德在这段话中批判了蒲鲁东的“工人不能买回自己的产品,因为其中包含加入产品成本的利息”这一论点(见蒲鲁东的《什么是财产》1840 年巴黎版第 4 章第 5 节)。福尔卡德概括了蒲鲁东以极其狭隘的形式提出的困难,并指出商品价格包含着一个不仅超过工资而且超过利润的余额,因为它还包含原料等等的价值。福尔卡德企图以这种概括的形式来解决问题,他以“国民资本的不断增长”为理由,似乎就能解释上述“买回”。马克思在《资本论》第三卷第四十九章注 53 指出了福尔卡德这样以资本增长为理由是荒谬的,并痛斥这种说法是“资产阶级无知的乐观主义”。《两大陆评论》是资产阶级文学、艺术和政论双周刊,从 1829 年起在巴黎出版。——第 95 页。}

首先我们要指出,这 8 码代表的无非是已耗费的不变资本。但这个不变资本的使用价值的形式已经改变了。它是以新产品的形式,不再以纱、织机等等的形式,而以麻布的形式存在了。这 8 码麻布,也象用工资和利润购买的其余 4 码麻布一样,按价值来说,1/3 由织布过程中加进的劳动构成,2/3 由过去的、物化在不变资本中的劳动构成。而在前面那 4 码麻布的情况下,新加劳动的 1/3 抵补了 4 码麻布中包含的织布劳动,也就是自己抵补了自己,织布劳动其余的 2/3 抵补了 4 码麻布中包含的不变资本;而现在正相反,8 码麻布包含的不变资本由不变资本的 2/3 来抵补,它包含的新加劳动由不变资本的 1/3 来抵补。

这 8 码麻布本身包含了、吸收了整个不变资本的价值,——这个价值在 12 小时的织布劳动期间,转移到产品中,加入到产品的生产过程中,而现在以供直接的个人消费(不是生产消费)的产品形式存在,——这 8 码麻布本身又将怎样呢\fontbox{?}

这 8 码属于资本家。如果资本家想自己把这 8 码消费掉,就象他把代表他的利润的 2/3 码消费掉一样,[276]那他就不能把加入 12 小时织布过程的不变资本再生产出来了;他也就根本不能——就这里所谈的加入这 12 小时过程的资本来说——继续执行资本家的职能了。所以,他要卖掉 8 码麻布,把它们变成 24 先令货币或 24 劳动小时。但在这里我们又遇到了困难。他把这 8 码麻布卖给谁呢\fontbox{?}他把这些麻布转化为谁的货币呢\fontbox{?}我们很快将回过头来谈这个问题,现在让我们先看看下一段过程。

资本家一旦把 8 码麻布,即他的产品中和他预付的不变资本相等的那部分价值,转化为货币,把它们卖掉,使它们转为交换价值形式,他就用这些货币重新购买和原先构成他的不变资本的那些商品同类的(按使用价值来说是同类的)商品,他购买纱、织机等等。他按照生产新麻布所必需的比例,把 24 先令分别用来购买原料和生产工具。

由此可见,按使用价值来说,他的不变资本原先由哪种劳动的产品构成,现在就用那种劳动的新产品来补偿。资本家再生产了不变资本。但这些新的纱、新的织机等等,同样(按照假定)也是 2/3 由不变资本构成,1/3 由新加劳动构成。因而,如果说前 4 码麻布(新加劳动和不变资本)完全由新加劳动来支付,那末这 8 码麻布就由生产本身所必需的新生产出来的各个要素来补偿,而这些新生产出来的要素也是一部分由新加劳动,一部分由不变资本构成的。这样,看来至少有一部分不变资本要同另一种形式的不变资本相交换。产品的补偿是实在的事情,因为在纱被加工为麻布的同时,亚麻正被加工为纱,亚麻的种子正种成亚麻;同样,在织机受到磨损的同时,新的织机正在制造出来,在制造新织机的时候,新的木材和铁正在开采出来。某些要素在某一生产领域内被生产出来的时候,在另一生产领域内正在对它们进行加工。在所有这些\textbf{同时进行的}生产过程中,虽然每个过程都代表制造产品的一个更高的阶段,但是不变资本却以各种不同的比例同时被消费。

\textbf{总之,麻布这一成品的价值分为两部分},一部分用来重新购买这个时期生产出来的不变资本各个要素,另一部分则用在消费品上。为了简单起见,这里我们完全撇开一部分利润再转化为资本的问题,也就是说,正象在这整个研究中一样,我们假定,工资加利润,即加到不变资本上的全部劳动量,都作为收入被消费掉。

尚待回答的问题只是:有一部分总产品的价值被用来重新购买这个时期新生产的不变资本各个要素,究竟由谁购买这部分总产品\fontbox{?}谁购买 8 码麻布\fontbox{?}为了切断各种遁词的后路,我们假定,这是一种专供个人消费的麻布,而不是供生产消费(如制帆)的麻布。这里还必须把纯粹中间性的商业活动撇开,因为这些活动只起中介作用。例如,8 码麻布被卖给商人,并且不是经过一个商人的手,而是经过整整二十个商人的手,经过二十次买而再卖;那末,在第二十次,麻布终究还是要被商人卖给实际消费者,因此,实际消费者事实上或者支付给生产者,或者支付给\textbf{最后一个}即第二十个商人,而这个商人对消费者来说,是代表\textbf{第一个商人}即实际生产者。这些中间交易只会把最终交易推迟,或者也可以说,只会为最终交易起中介作用,但是不能说明最终交易。无论我们是问,谁从麻织厂主手里购买这 8 码麻布,还是问,[277]谁从第二十个商人手里(麻布经过一系列交换行为才落到他手里)购买这 8 码麻布——问题仍然是一样的。

这 8 码麻布和前 4 码麻布完全一样,必定要转入消费基金。这就是说,它们只能由工资和利润来支付,因为工资和利润是生产者的收入的唯一源泉,而在这里只有这些生产者才以消费者的身分出现。8 码麻布包含 24 劳动小时。我们假定(以 12 劳动小时为通行的正常工作日),其他两个部门的工人和资本家把自己的全部工资和利润花在麻布上,就象织布业中的工人和资本家把自己的整个工作日(工人把自己的 10 小时,资本家把他靠工人赚得的,就是靠 10 小时赚得的 2 小时剩余价值)都花在麻布上一样。在这种情况下,麻织厂主就会卖掉自己的 8 码麻布;这样,用于织造这 12 码的不变资本的\textbf{价值}就会得到补偿,这个价值可以重新花在构成不变资本的那些商品上,\textbf{因为}所有这些商品,如纱、织机等等,在市场上都有,它们在纱和织机被加工为麻布的时候已经生产出来了。纱和织机作为产品\textbf{生产出来}的过程,同它们作为产品加入(而不是作为产品从中出来)的生产过程\textbf{同时进行},这种情况说明,为什么麻布的\textbf{价值}中和被加工的材料、织机等等的价值相等的那部分,能够重新转化为纱、织机等等。如果麻布的各个要素的生产和麻布本身的生产不是同时进行,8 码麻布即使卖掉了,即使转化为货币,也不能从货币再转化为麻布的各个不变要素。\authornote{例如,目前由于美国内战,棉纺织厂主的棉纱和棉布就发生了这样的情况。即使他们的产品卖出去了,也不能保证实现前面说的那种再转化,因为市场上没有棉花。}

但是,另一方面,尽管市场上有新的纱、新的织机等等,因而新的纱、新的织机等等的生产是和已有的纱、已有的织机转化为麻布同时进行的;尽管纱和织机是和麻布同时生产出来,但是,在这 8 码麻布没有卖掉,没有转化为货币以前,它们也不能再转化为织布生产的不变资本的这些物质要素。因此,在我们还没有弄清楚,购买 8 码麻布,使它们重新具有货币形式即独立的交换价值形式所必需的基金从哪里来以前,麻布各个要素的不断的现实的生产始终和麻布本身的生产同时并进这个事实,还是不能向我们说明不变资本的再生产。

为了解决这个最后的困难,我们假定有 B 和 C(比方说,一个是制鞋业者,一个是屠宰业者)把自己的工资和利润总额,即他们所支配的 24 小时劳动时间,全都花在麻布上。这样,关于麻布织造业者 A,我们就摆脱了困难了。他的全部产品 12 码麻布(其中物化了 36 小时劳动),完全由工资和利润来补偿了,也就是说,由 A、B 和 C 这三个生产领域中新加到不变资本上的全部劳动时间来补偿了。麻布中包含的全部劳动时间,无论是原先就已体现在它的不变资本中的,还是在织布过程中新加的,现在都同这样一个劳动时间相交换了,这个劳动时间不是以不变资本的形式原先就存在于某一个生产领域中的,而是在上述 A、B、C 三个生产领域中\textbf{在生产的最后阶段}同时加到不变资本上去的。

因此,如果说麻布的原有价值只分解为工资和利润,还是错误的,——因为它实际上分解为两部分,一部分是同工资和利润总额相等的价值,即 12 小时织布劳动,一部分是同织布过程无关的,包含在纱、织机中的,总之,包含在不变资本中的 24 劳动小时,——那末,相反,说 12 码麻布的等价物,即 12 码麻布卖得的 36 先令,只分解为工资和利润,也就是说,不仅织布劳动,而且包含在纱和织机中的劳动,都完全由新加劳动来补偿,就是由 A 的 12 小时劳动、B 的 12 小时劳动和 C 的 12 小时劳动来补偿,却是正确的。

卖出的商品本身具有的价值,分解为[278]新加劳动(工资和利润)和过去劳动(不变资本的价值);这就是卖者的商品的价值(这也是商品的实际价值)。相反,购买这个商品的价值,即买者给予卖者的等价物,从我们的例子来看,只归结为新加劳动,归结为工资和利润。但是,如果由于任何商品在卖出以前都是待卖的商品,而它只有通过单纯的形式变化才转化为货币,就认为任何商品作为出卖的商品时的价值组成部分不同于它作为购买的商品(作为货币)时的价值组成部分,那是荒谬的。其次,认为社会例如在一年内完成的劳动不仅可以自己抵补自己,——这样,如果把全部商品量分为两个相等的部分,年劳动的一半就成为另一半的等价,——而且年产品所包含的总劳动中由当年劳动构成的 1/3 的劳动,可以抵补 3/3 的劳动,也就是说,它的大小等于自己的 3 倍,那就更加荒谬了。

在上述例子中,我们把困难推移了,从 A 移到了 B 和 C。但是困难并没有减少,反而增加了。\textbf{第一},我们在谈 A 的时候,有一个解决办法,就是 4 码所包含的劳动时间恰好等于加在纱上的劳动时间,也就是 A 领域的利润和工资总额,这 4 码是以麻布的形式,以自己劳动产品的形式消费的。B 和 C 的情况却不是这样,因为这两个领域是以 A 领域的产品麻布的形式,而不是以 B 和 C 的产品的形式,来消费它们加进的劳动时间总量,即工资和利润总额的。因此,它们不仅要卖掉代表不变资本包含的 24 劳动小时的那部分产品,而且要卖掉代表新加到不变资本上的 12 小时劳动时间的那部分产品。B 领域要卖掉 36 劳动小时,而不是象 A 领域那样只卖掉 24 劳动小时。C 的情况也是这样。\textbf{第二},为了把 A 领域的不变资本卖掉,推销出去,转化为货币,不仅需要 B 领域的全部新加劳动,而且需要 C 领域的全部新加劳动。\textbf{第三},B 和 C 不能把自己产品中的任何部分卖给 A 领域,因为 A 产品中归结为收入的部分,已经由 A 产品的生产者全部花在 A 领域本身了。B 和 C 也不可能用自己产品中的任何部分来补偿 A 的不变部分,因为根据假定,他们的产品不是 A 的生产要素,而是加入个人消费的商品。每前进一步,困难都在增加。

为了使 A 产品包含的 36 小时(就是说,不变资本包含的 2/3 即 24 小时,新加劳动包含的 1/3 即 12 小时)可以完全同加在不变资本上的劳动相交换,就必须使 A 的工资和利润,即 A 领域中加进的 12 小时劳动,自己消费掉本领域的产品的 1/3。总产品的其余 2/3,即 24 小时,代表不变资本包含的价值。这个价值已同 B 和 C 领域的工资和利润总额,也就是同 B 和 C 领域的新加劳动相交换。但是,为了使 B 和 C 能够用他们的产品包含的、归结为工资[和利润]的 24 小时来购买麻布,他们就要以他们自己的产品形式卖掉这 24 小时。此外,他们还要以他们自己生产的产品形式卖掉 48 小时来补偿不变资本。因此,他们就要卖掉共 72 小时的 B 和 C 的产品,来同其他生产领域 D、E 等等的利润和工资总额相交换;同时(假定正常工作日等于 12 小时),为了购买 B 和 C 的产品,就必须花 12×6=72 小时,即其他 6 个生产领域中的新加劳动;[279]因而,必须花费 D、E、F、G、H、I 这些领域的利润和工资,即这些领域中加到各自不变资本上的全部劳动量。

在这种情况下,B+C 总产品的价值,就会完全由 D、E、F、G、H、I 这 6 个生产领域的新加劳动,即工资和利润总额来支付。但这 6 个领域的全部总产品也要卖掉(因为它们的产品的任何部分都不是由它们的生产者自己消费,这些人已经把自己的全部收入投在 B 和 C 的产品上了),并且这个总产品的任何部分都不能在这些领域内部实现。因此,这里就牵涉到 6×36 劳动小时=216 劳动小时的产品,其中 144 劳动小时为不变资本,72(即 6×12)劳动小时为新加劳动。现在为了使 D 等等的产品也按同样的方式转化为工资和利润,即转化为新加劳动,就必须使$K^{1}$—$K^{18}$这 18 个生产领域的全部新加劳动,即这 18 个领域的工资和利润总额,统统都花在 D、E、F、G、H、I 这些领域的产品上。$K^{1}$—$K^{18}$这 18 个领域并不消费自己产品的任何部分,相反,它们已经把自己的全部收入花在 D—I 这 6 个领域中,所以它们又要卖掉 18×36 劳动小时=648 劳动小时,其中 18×12(216 小时)代表新加劳动,432 小时代表不变资本包含的劳动。因此,为了把$K^{1}$—$K^{18}$的这个总产品归结为其他领域的新加劳动,换句话说,归结为工资和利润总额,就需要有$L^{1}$—$L^{54}$领域的新加劳动,也就是 12×54=648 劳动小时。$L^{1}$—$L^{54}$领域的总产品等于 1944 小时(其中 648=12×54 为新加劳动,1296 劳动小时等于不变资本所包含的劳动),为了使这个总产品同新加劳动相交换,这些领域就要吸收$M^{1}$—$M^{162}$领域的新加劳动,因为 162×12=1944,而$M^{1}$—$M^{162}$领域又要吸收$N^{1}$—$N^{486}$领域的新加劳动,依此类推。

这是一个美妙的无止境的演进,如果我们认为,一切产品的价值都归结为工资和利润,即归结为新加劳动,同时,不仅新加到商品上的劳动,而且这个商品所包含的不变资本,都必须由其他某个生产领域的新加劳动来支付,那末,我们就会陷入这种境况。

为了把 A 产品包含的劳动时间,即 36 小时(1/3 为新加劳动,2/3 为不变资本),归结为新加劳动,也就是说,为了假定这个劳动时间由工资和利润来支付,我们首先就假定,产品的 1/3(这 1/3 的价值等于工资和利润总额)由 A 领域的生产者自己消费,或者同样可以说,由他们自己购买。以后的进程如下\endnote{后面,马克思在保留前面引用的数字材料的同时,改换了生产领域的字母符号(A 除外)。马克思用 B1—B2(或 B1-2)代替 B 和 C;用 C1—C6(或 C1-6)代替 D、E、F、G、H、I;用 D1—D18(或 D1-18)代替 K1—K18;用 E1—E54(或 E1-54)代替 L1—L54;用 F1—F162(或 F1-162)代替 M1—M162;用 G1—G486(或 G1-486)代替 N1—N486。——第 102 页。}:

(1)\textbf{生产领域}A。产品=36 劳动小时。24 劳动小时为不变资本。12 小时为新加劳动。产品的 1/3 由参加这 12 小时分配的双方——工资和利润,即工人和资本家来消费。A 产品的 2/3,等于不变资本包含的 24 劳动小时,则有待卖出。

(2)\textbf{生产领域}$B^{1}$—$B^{2}$。产品=72 劳动小时;其中 24 为新加劳动,48 为不变资本。这些领域用新加劳动购买 A 产品的 2/3 即补偿 A 的不变资本价值的那部分产品。但是,$B^{1}$—$B^{2}$领域共计要卖掉构成它们总产品价值的 72 劳动小时。

(3)\textbf{生产领域}$C^{1}$—$C^{6}$。产品=216 劳动小时,其中 72 小时为新加劳动(工资和利润)。它们用新加劳动购买$B^{1}$—$B^{2}$的全部产品。但是,它们要卖掉 216,其中 144 为不变资本。

[280](4)\textbf{生产领域}$D^{1}$—$D^{18}$。产品=648 劳动小时;216 为新加劳动,432 为不变资本。它们用新加劳动购买生产领域$C^{1}$—$C^{6}$的总产品=216。但是,它们要卖掉 648。(5)\textbf{生产领域}$E^{1}$—$E^{54}$。\textbf{产品}=1944 劳动小时;648 为新加劳动,1296 为不变资本。它们购买生产领域$D^{1}$-$D^{18}$的总产品,但是要卖掉 1944。

(6)\textbf{生产领域}$F^{1}$—$F^{162}$。\textbf{产品}=5832,其中 1944 为新加劳动,3888 为不变资本。它们用 1944 购买$E^{1}$—$E^{54}$的产品。而它们要卖掉 5832。

(7)\textbf{生产领域}$G^{1}$—$G^{486}$。

为了简单起见,假定每一个生产领域每次都只有一个工作日——12 小时——为资本家和工人所分享。增加这种工作日的数目并不能解决问题,反而会毫无必要地使问题复杂化。

这样,为了使这个序列的规律能够看得更清楚,可以写成:

(1)A。\textbf{产品}=36 小时;不变资本=24 小时。\textbf{工资和利润总额},或者说,\textbf{新加劳动}=12 小时。后者以 A 领域本身的产品形式由资本和劳动消费掉。

A 的待卖产品=它的不变资本=24 小时。

(2)$B^{1}$—$B^{2}$。这里我们需要 2 工作日,因而需要 2 个生产领域,以便支付 A 领域的 24 小时。

产品=2×36=72 小时,其中 24 小时为新加劳动,48 小时为不变资本。

$B^{1}$—$B^{2}$的待卖产品=72 劳动小时;这一产品的任何部分都不会在这些领域本身消费掉。

(6)$C^{1}$—$C^{6}$。这里我们需要 6 工作日,因为 72=12×6,$B^{1}$—$B^{2}$的全部产品必须由$C^{1}$—$C^{6}$加进的劳动消费掉。产品=6×36=216 劳动小时,其中 72 为新加劳动,144 为不变资本。

(18)$D^{1}$—$D^{18}$。这里我们需要 18 工作日,因为 216=12×18。既然每个工作日应有 2/3 不变资本,所以总产品=18×36=648(432 为不变资本)。

依此类推。

每段开头的 1、2 等数字,是指工作日的数目或不同生产领域的不同劳动种类的数目,因为我们假定每一个领域只有一个工作日。

可见,(1)A。产品为 36 小时。新加劳动为 12 小时。\textbf{待卖产品}(不变资本)=24 小时。

或者说:

(1)A。\textbf{待卖产品}即\textbf{不变资本}=24 小时。总产品为 36 小时。新加劳动为\textbf{12 小时}。后者\textbf{在 A 领域本身被消费掉}。

(2)$B^{1}$—$B^{2}$。这些领域用新加劳动购买 A 的 24 小时。\textbf{不变资本}为 48 小时。\textbf{总产品}为 72 小时。

(6)$C^{1}$—$C^{6}$。它们用新加劳动购买$B^{1}$—$B^{2}$的\textbf{72}小时(=12×6)。\textbf{不变资本}为 144,总产品为 216。依此类推。

[281]因此:

(1)A。产品=3 工作日(36 小时)。12 小时为新加劳动。\textbf{24 小时}为不变资本。

(2)$B^{1-2}$。\textbf{产品}=2×3=6 工作日(72 小时)。新加劳动=\textbf{12×2=24 小时}。\textbf{不变资本}=48=2×24 小时。

(6)$C^{1-6}$。\textbf{产品}=3×6 工作日=3×72 小时=216 劳动小时。\textbf{新加劳动}=6×12 劳动小时(=72 劳动小时)。\textbf{不变资本}=2×72=144。

(18)$D^{1-18}$。\textbf{产品}=3×3×6 工作日=3×18 工作日=54 工作日=648 劳动小时。新加劳动=12×18=\textbf{216}。不变资本=432 劳动小时。

(54)$E^{1-54}$。\textbf{产品}=3×54 工作日=162 工作日=1944 劳动小时。新加劳动=54 工作日=648 劳动小时;不变资本=1296 劳动小时。

(162)$F^{1}$—$F^{162}$。\textbf{产品}=3×162 工作日=486 工作日=5832 劳动小时,其中 162 工作日即 1944 劳动小时为新加劳动,3888 劳动小时为不变资本。

(486)$G^{1-486}$。\textbf{产品=3×486 工作日},其中 486 工作日即 5832 劳动小时为新加劳动,11664 劳动小时为不变资本。

依此类推。

这里我们已经有了一个相当可观的数目,它由 729 个不同生产领域的不同工作日 1+2+6+18+54+162+486 合计而成。这已经是一个分工相当精细的社会了。

在 A 领域中,加到不变资本 2 工作日上的,只有 12 小时劳动即 1 工作日,而工资和利润是消费它自己的产品;为了从 A 领域的总产品中仅仅卖掉不变资本 24 小时,并且又只是同新加劳动即工资和利润相交换,我们就需要:

$B^{1}$和$B^{2}$的 2 工作日。但是这 2 工作日又应有不变资本 4 工作日,这样,$B^{1-2}$的总产品等于 6 工作日。这 6 工作日必须\textbf{全部}卖掉,因为\textbf{从这里起},就假定后一个领域都不消费自己的产品,而只是把自己的利润和工资花在前一个领域的产品上。为了补偿$B^{1-2}$产品所包含的这 6 工作日,就必需有 6 工作日,而这 6 工作日又要有不变资本 12 工作日。因此$C^{1-6}$的总产品等于 18 工作日。为了用新加劳动来补偿它们,就必需有 18 工作日($D^{1-18}$),而这 18 工作日又要有不变资本 36 工作日。因而产品等于 54 工作日。为了补偿这些工作日又需要$E^{1-54}$的 54 工作日,而这 54 工作日又要有不变资本 108 工作日。产品就会等于 162 工作日。最后,为了补偿这些工作日,就需有 162 工作日,而这 162 工作日又要有不变资本 324 工作日;因而总产品就是 486 工作日。这也就是$F^{1}$—$F^{162}$的产品。最后,为了补偿这$F^{1-162}$的产品,就需有 486 工作日($G^{1-486}$),而这 486 工作日又要有不变资本 972 工作日。因此,$G^{1-486}$的总产品=972+486=1458 工作日。

但现在假定,到 G 领域,我们已到了尽头,再也不能推移下去了。[282]在任何一个社会,上述这种从一个领域到另一领域的推移,也会很快达到尽头的。这时的情况怎样呢\fontbox{?}我们有这样一种产品,它包含 1458 工作日,其中 486 日为新加劳动,972 日为物化在不变资本中的劳动。新加劳动 486 日可以在前一个领域$F^{1-162}$交换。但是不变资本包含的 972 工作日用什么东西来购买呢\fontbox{?}在$G^{486}$领域以外再也没有任何新的生产领域,因而也就没有任何新的交换领域了。在它前面的各个领域,除了$F^{1-162}$以外,什么也交换不到。而且$G^{1-486}$领域本身又把它包含的工资和利润都花在$F^{1-162}$领域了。由此看来,物化在$G^{1-486}$产品中、等于它包含的不变资本价值的 972 工作日,是不可能卖出去了。可见,我们把我们碰到的困难——A 领域中的 8 码麻布,或者说,这个领域的产品中代表不变资本价值的 24 劳动小时,即 2 工作日——推移到了将近 800 个生产部门,还是无济于事。

有人认为,如果 A 领域不是把自己的全部利润和工资花在麻布上,而是把其中一部分花在 B 和 C 的产品上,那计算就会不同了。这种想法也是无济于事的。A、B 和 C 包含的新加劳动时间量是支出的界限,因此,这些领域在任何情况下都只能支配同它们的新加劳动相等的劳动时间量。它们多买这一种产品,便会少买那一种产品。这只会把计算搅乱,丝毫也不会改变结果。

那末,我们怎么办呢\fontbox{?}

在上述计算中,我们看到:

总之,在我们的计算中,相当于新加劳动的 243 工作日事实上是可以消费掉的。最后一种产品的价值等于 486 工作日,而 A—F 所包含的全部不变资本的价值也等于 486 工作日,两者正好相等。为了说明这些工作日,我们假定 G 有 486 日的新劳动。这样一来,我们就不必再去分析 486 日的不变资本的问题,[283]可是现在,我们还得说明 G 产品中包含的 972 工作日的不变资本,因为 G 产品等于 1458 工作日(972 不变资本+486 劳动)。如果我们想要摆脱困难,假定 G 领域在劳动过程中不使用不变资本,因此产品仅仅等于 486 日的新加劳动,那末,我们的计算当然也就算清了。但是,产品中构成不变资本的价值组成部分究竟由谁支付的问题所以能解决,只是因为我们假定不变资本等于零,因而它不构成产品的价值组成部分。

为了使 A 的全部产品都能卖掉,同新加劳动相交换;为了有可能把它归结为利润和工资;A、B、C 的\textbf{全部新加劳动}\endnote{符号 B 和 C,马克思这里是在第 102 页以前使用的意义上使用的(见注 47)。马克思这里是指两个生产领域,其中每一个领域的新加劳动都是一个工作日。A、B 和 C 三个领域的新加劳动总额等于三个工作日,即等于物化在 A 领域的产品中的劳动。——第 108 页。}就必须以 A 领域的劳动产品的形式花掉。同样,为了使 B+C 的全部产品能够卖掉,就必须拿出$D^{1}$—$D^{18}$的全部新加劳动来同它相交换。\endnote{马克思在这里使用的字母符号 B 和 C 已经不是指两个生产领域,因为两个生产领域的产品总共只有 6 工作日,而马克思这里说的是 18 工作日。但是马克思用这些符号也不是指 B1—B2 和 C1—C6(马克思用 B1—B2 表示由两个生产领域组成的一组,用 C1—C6 表示由 6 个生产领域组成的一组;这 8 个领域的总产品是 24 工作日)。马克思在这里是指由 6 个生产领域组成的一组。它们的总产品为 18 工作日,因而可以同 D1—D18 的也等于 18 工作日的新加劳动相交换。——第 108 页。}同样,为了购买$D^{1}$—$D^{18}$的全部产品,必须有$E^{1-54}$的全部新加劳动。为了购买$E^{1-54}$的全部产品,必须有$F^{1-162}$的全部新加劳动。最后,为了购买$F^{1-162}$的全部产品,就需要有$G^{1-486}$的全部新加劳动时间。归根到底,由$G^{1-486}$领域代表的这 486 个生产领域中,全部新加劳动时间等于 162 个 F 领域的全部产品,而 F 领域由劳动来补偿的全部产品又等于 A、$B^{1-2}$、$C^{1-6}$、$D^{1-18}$、$E^{1-54}$、$F^{1-162}$领域的不变资本。但是,G 领域的不变资本(它比 A—$F^{162}$领域使用的不变资本多 1 倍)仍然没有得到补偿,而且也不可能得到补偿。

事实上我们发现,因为根据我们的假定,在每一个生产领域,新加劳动和过去劳动之比等于 1∶2,所以,为了购买前面那些领域的产品,每次都需要[比前面所有领域加在一起的数目]\endnote{方括号中的话是根据马克思的整个思想进程加的。按照马克思的计算,在每下一组生产领域中,生产领域的数目都比前面所有领域的总数大倍。例如,在有 18 个生产领域的 D1-18 这一组中,生产领域的数目比前面所有各组的领域的数目加在一起大 1 倍(A——1 个领域,B1-2——2 个领域,C1-6——6 个领域;共计 9 个领域)。因此,马克思在 D1-18 符号后面,在括号内写着 2×9。——第 108 页。}多 1 倍的新领域拿出全部新加劳动来。为了购买 A 的总产品,需要 A 和$B^{1-2}$的新加劳动,为了购买$C^{1-6}$的产品,需要 18 个 D 即$D^{1-18}$(即 2×9)的新加劳动,依此类推。简单说,总是需要比产品本身包含的新加劳动多 1 倍的新加劳动量,因此,对最后一个生产领域 G 来说,情况就是:为了购买这个领域的全部产品,就需要比已有的多 1 倍的新加劳动量。总之,在终点 G 上我们碰到的,恰恰是在起点 A 上已经存在的那种情况:新加劳动无论如何不可能从自己的产品中购买一个比它本身大的量,它\textbf{不可能}购买不变资本中包含的过去劳动。

因此,要用收入的价值抵补整个产品的价值是不可能的。因为除了收入以外,\textbf{没有任何基金可以用来支付生产者卖给(个人)消费者的产品},所以,整个产品的价值减去收入的价值之后,根本不可能被卖掉、被支付或被(个人)消费。但是,另一方面,任何产品都必须卖掉,并按其价格得到支付(按照假定,这里价格等于价值)。

此外,从一开始就可以预见到,把中间性交换行为即各种商品或各个生产领域的产品的卖和买加进来,并不能使我们前进一步。在考察 A 领域即第一种商品麻布时,我们有 1/3 或[283a]12 小时新加劳动和 2×12(或 24)小时包含在[不变]资本中的过去劳动。工资和利润只能从 A 商品中,因而也只能从作为 A 商品的等价物的其他某种产品中买回等于 12 劳动小时的商品。它们不可能买回自己的 24 小时的不变资本,因而也不可能买回其他某种商品形式的这个不变资本的等价物。

在 B 商品中,新加劳动和不变资本之间的比例可能不同。但是,在各个生产领域中,不变资本和新加劳动之间的比例无论怎样不同,我们总能算出这个比例的平均数。我们可以说,在整个社会或整个资本家阶级的产品中,在资本的总产品中,新加劳动等于 a,作为不变资本存在的过去劳动等于 B;换句话说,我们对 A 即麻布所假定的比例 1∶2,只不过是 a∶B 的一种象征,只不过说明在这两个要素之间,在本年内或任何一段时间内加进的活劳动和作为不变资本存在的过去劳动之间,存在着某种无论如何是确定的并且是可以确定的比例。如果加在纱上的 12 小时不是全都用来购买麻布,如果购买麻布例如只用 4 小时,那末其余 8 小时就可以用来购买其他任何产品;但是购买的总数决不能超过 12 小时。如果 8 小时用来购买其他产品,那末 A 就必须卖掉 32 小时的麻布。因此,A 的例子,对整个社会的总资本也是适用的,把不同商品的中间性交换行为加进来,只会把问题搅乱,但丝毫不会改变问题的实质。

我们假定 A 是社会的总产品;在这种情况下,这个总产品的 1/3 可以由生产者买来自己消费,生产者会用他们的等于新加劳动量即总收入额的工资和利润总额来支付它。但是要支付、购买和消费其余的 2/3,他们就没有必要的基金了。因此,新加劳动即完全分解为利润和工资的 1/3 总劳动,用它自己的产品抵补自己,或者说,只是把包含 1/3 总劳动即新加劳动或它的等价物的那部分产品价值留下来,同样,属于过去劳动的那 2/3 也必须用这种过去劳动本身的产品来抵补。换句话说,不变资本始终等于它自己,它由总产品中代表它自己的那部分价值来补偿。不同商品之间的交换,不同生产领域之间的一系列买和卖的行为,只是从这样一种意义来说会引起某种形式上的区别,即不同生产领域的不变资本都按照它们原来在这些生产领域中保持的比例相互抵补。

现在我们必须更详细地考察这一点。[283a]

\tsubsectionnonum{[(b)靠消费品生产者和生产资料生产者之间的交换不可能补偿全部社会不变资本]}

[283b]亚当·斯密在第二篇第二章考察货币流通和信用制度时(后面将同\textbf{图克}的说法相对比)提出了同样的看法,即认为一国的年产品分解为工资和利润(地租、利息等等包括在利润之中)。在那里,他说:

\begin{quote}“每一个国家的流通都可以认为是分成两个不同的领域:实业家(dealers)〈这里注明:dealers 是指“全体商人、制造业者、手工业者等等,一句话,是指一国工商业的全体当事人”〉之间的流通,实业家和消费者之间的流通。即使同一些货币,纸币或者金属货币,可以时而用于这个流通领域,时而用于那个流通领域,但这两个流通过程是不断同时进行的,因此,要使流通进行下去,各自需要有一定量的这种或那种货币。\textbf{各种实业家之间流通的商品的价值,绝不能超过实业家和消费者之间流通的商品的价值,因为无论实业家购买什么,最终必然会卖给消费者}。”(第 2 卷第 2 篇第 2 章第 292—293 页)\endnote{马克思这里引的是斯密著作的加尔涅译本。马克思在尖括号内提到的关于《dealers》这一术语的说明是加尔涅加的。——第 111 页。}\end{quote}

这个问题,还有图克,以后再谈。\endnote{对斯密和图克的这一错误论点,马克思在后面第 130—131 和 256—257 页分别作了评论。在《资本论》第二卷第二十章中,马克思指出,斯密和图克的“年收入流通所需要的货币,也足以使全部年产品流通”的观点,同斯密把社会产品的全部价值归结为收入的教条是密切相关的(马克思《资本论》第 2 卷第 20 章第 12 节)。并见马克思《资本论》第 3 卷第 49 章。——第 111 页。}

现在回过来谈我们的例子。把麻纱变为麻布的 A 领域,它的日产品等于 12 码,或 36 先令,或 36 劳动小时。在这 36 小时中,12 小时新加劳动分解为工资和利润,24 小时或 2 日等于不变资本的价值。不过后者现在已经不是以原来的纱和织机的形式存在,而是以麻布的形式存在,并且是以等于 24 小时或 24 先令的麻布量存在。这个麻布量所包含的劳动量,同麻布现在所代替的纱和织机包含的劳动量一样多。因此,用这个麻布量去交换,可以重新买回同样数量的纱和织机(假定纱和织机的价值照旧,这些工业部门的劳动生产率不变)。纺纱业者和织机厂主必须把他们的全部产品——年产品或日产品(在这里,对于我们的目的来说,是无关紧要的)——卖给织布业者,因为他们的商品只对织布业者才有使用价值,织布业者是这种商品的唯一消费者。

如果织布业者的不变资本(他每天消费的不变资本)等于 2 工作日,那末织布业者的 1 工作日就需要有纺纱业者和机器厂主的 2 工作日,这 2 工作日又按极不相同的比例分解为新加劳动和不变资本。但纺纱业者和机器厂主两人加在一起的总的日产品(假定机器厂主只生产织机),也就是说,不变资本和新加劳动合在一起,不可能多于 2 工作日,而织布业者的日产品,由于他新加进 12 小时劳动,则是 3 工作日。纺纱业者和机器厂主消费的活劳动时间,可能同织布业者消费的一样多;在这种情况下,他们的不变资本包含的劳动时间必定少些。无论如何,纺纱业者和机器厂主使用的劳动量,即物化劳动和活劳动的量(总合起来),决不可能同织布业者使用的一样多。织布业者使用的活劳动时间,可能比纺纱业者少(例如,后者一定会比亚麻种植业者少);在这种情况下,他的不变资本就会超过资本的可变部分更多。

[284]于是,织布业者的不变资本就补偿纺纱业者和织机厂主的全部资本,不仅补偿后两者自己的不变资本,而且补偿纺纱过程和机器生产过程中新加的劳动。这样一来,在这里新的不变资本就完全补偿其他的不变资本,此外还补偿全部新加到不变资本上的劳动。纺纱业者和织机厂主由于把自己的商品卖给织布业者,就不仅补偿了自己的不变资本,而且得到了自己新加劳动的报酬。织布业者的不变资本补偿纺纱业者和织机厂主自己的不变资本,并实现他们的收入(工资和利润的总额)。既然织布业者的不变资本只是补偿纺纱业者和织机厂主自己的、以纱和织机形式交给织布业者的不变资本,那末这只是一种形式的不变资本同另一种形式的不变资本相交换。实际上不变资本的价值没有任何变化。

但是,我们再往回走。纺纱业者的产品也分解为两部分:一部分是亚麻、纱锭、煤炭等等,一句话,纺纱业者的不变资本;另一部分是新加劳动。机器厂主的总产品也是这样。当纺纱业者补偿自己的不变资本时,他不仅支付纱锭制造厂主等等的全部资本,而且支付亚麻种植业者的全部资本。纺纱业者的不变资本支付这些人的资本的不变部分加上新加劳动。至于亚麻种植业者,他的不变资本,扣除农具等等之后,就归结为种子、肥料等等。我们假定,租地农场主的这部分不变资本,构成他自己的产品中每年的扣除部分(这种情况在农业中总是或多或少间接地发生),这一扣除部分每年从租地农场主自己的产品中归还给土地,即归还给生产本身。在这里,我们发现有一部分不变资本是自己补偿自己,从来不拿去出卖,因而从来不被支付,也从来不被消费,不加入个人消费。种子等东西等于一定量的劳动时间。种子等等的价值加入总产品的价值;但同一价值——因为这里指的是同一产品量(假定劳动生产率不变)——又从总产品中留出,归还给生产,而不进入流通。

这里我们看到,至少有一部分不变资本,即可以看成农业原料的东西,是自己补偿自己的。因此,这里我们有一个年生产的巨大部门(按规模和投入的资本量来说是最巨大的部门),其中很大一部分由原料(人造肥料等等除外)构成的不变资本,是自己补偿自己,不加入流通的,也就是说,不是由任何形式的收入来补偿的。因此,纺纱业者不必为了补偿(由亚麻种植业者自行补偿、自行支付的)这一部分不变资本,而把它偿还给亚麻种植业者,同样,织布业者也不必把它偿还给纺纱业者,麻布购买者也不必把它偿还给织布业者。

我们假定,所有直接或间接参加生产 12 码麻布(=36 先令=3 工作日,即 36 劳动小时)的人,都以麻布本身的形式得到补偿。首先很清楚,麻布各要素的生产者,即麻布的不变资本的生产者,\textbf{不可能消费他们自己的产品},因为这种产品是为了进一步生产而生产出来的,不加入直接[285]消费。因此,这些生产者必须用自己的工资和利润来购买麻布,购买最终加入个人消费的产品。凡是他们不以麻布的形式来消费的东西,他们必定要以麻布换来的其他可消费的产品的形式来消费。因此,其他产品的生产者消费的麻布,同麻布各要素的生产者不消费麻布而消费的其他消费品,(按价值来说)正好一样多。结果就好比麻布各要素的生产者自己以麻布的形式来消费,因为他们消费多少其他产品,其他产品的生产者就消费多少麻布。因此,完全不问交换,只考察这 12 码麻布如何在所有生产者——参加麻布本身的生产或麻布各要素的生产的人——之间分配,这整个谜就能解开。

在纱和织机中,纺纱业者和织机厂主(假定他同时又是纺机厂主)的新加劳动占 1/3,他们的不变资本占 2/3。因此,在补偿他们全部产品的 8 码麻布(即 24 小时)或 24 先令中,他们能够消费 8/3 码麻布(2+(2/3)),即 8 小时劳动或 8 先令。这样,剩下来要说明的是 5+(1/3)码或 16 劳动小时。

5+(1/3)码麻布或 16 劳动小时代表纺纱业者和织机厂主的不变资本。如果我们假定,在纺纱业者的不变资本中,原料(在这里是亚麻)占 2/3,那末亚麻种植业者就能够以麻布的形式完全消费这 2/3,因为他根本没有把自己的不变资本\fontbox{~\{}我们在这里假定,他的劳动工具等等的损耗=0\fontbox{\}~}投入流通,相反,他已经把自己的不变资本从产品中扣除,并且为了再生产而保存起来。因此,他能够从 5+(1/3)码麻布\endnote{根据前面的计算,5+(1/3)码麻布代表纺纱业者和织机厂主的全部不变资本。因此,为了确定亚麻种植业者的份额,就不应当以 5+(1/3)码,而应当以更少的麻布量为初量。后来马克思纠正了这个不确切的地方,假定只有 4 码麻布代表纺纱业者的不变资本。——第 115 页。},或者说,从 16 劳动小时中,购买 2/3;这部分等于 3+(5/9)码或 10+(2/3)劳动小时。这样,剩下来要说明的就只是 5+(1/3)码减 3+(5/9)码,或 16 劳动小时减 10+(2/3)小时,即 1+(7/9)码或 5+(1/3)劳动小时。这 1+(7/9)码或 5+(1/3)劳动小时又分解为两部分:织机厂主的不变资本和纺机厂主的总产品,这里同时假定,织机厂主和纺机厂主是一个人。[286]

我们再说一遍:

\todo{}

由此可见,在补偿织布业者不变资本的 8 码中,2 码(=6 先令=6 小时)由纺纱业者消费,2/3 码(2 先令,2 劳动小时)由制造织机和其他生产工具的厂主消费。

这样,剩下来要说明的是 8 减 2+(2/3),即 5+(1/3)码(=16 先令=16 劳动小时)。这剩下来的 5+(1/3)码(=16 先令=16 劳动小时)分配如下。假定,在代表纺纱业者的不变资本,也就是代表他的纱的各要素的 4 码中,3/4 等于亚麻,1/4 等于纺机。纺机的各要素,[287]我们在以后考察织机厂主的不变资本时再计算。纺机厂主和织机厂主是一个人。

这样,在补偿纺纱业者不变资本的 4 码中,3/4 即 3 码是\textbf{亚麻}。但是生产亚麻时使用的很大一部分不变资本无需补偿,因为它已由亚麻种植业者本人以\textbf{种子、肥料、饲料、牲畜}等等形式归还给农业生产。因此,在亚麻种植业者出卖的那部分产品中,我们在计算时应当仅仅把他的劳动工具等等的损耗算作不变资本。这里我们必须把新加劳动估计为至少等于 2/3,待补偿的不变资本至多等于 1/3。

于是:

\todo{}

因此,剩下要我们计算的是:

1 码(3 先令,3 劳动小时)=亚麻种植业者的不变资本;

1+(1/3)码(4 先令,4 劳动小时)=织机的不变资本;

最后,1 码(3 先令,3 劳动小时)应付给机器制造业者,以支付他的包含在纺机中的\textbf{总产品}。

因此,首先应当扣除机器制造业者用他制造的纺机来交换的可消费部分:

\todo{}

其次,我们把\textbf{农业机器}的价值,亚麻种植业者的不变资本,分为可消费部分和其他部分。

\todo{}

我们把织布业者的总产品中属于机器的部分合在一起,就得出:织机 2 码,纺机 1 码,农业机器 1 码,共 4 码(12 先令,12 劳动小时,或全部产品 12 码麻布的 1/3)。在这 4 码中,织机厂主可以消费 2/3 码,纺机厂主可以消费 1/3 码,农业机器厂主也可以消费 1/3 码,共 1+(1/3)码。剩下 2+(2/3)码,即:织机的不变资本 4/3 码,纺机的不变资本 2/3 码,农业机器的不变资本 2/3 码,共 8/3=2+(2/3)码(=8 先令=8 劳动小时)。剩下的这个数字就是机器制造业者的待补偿的不变资本。这个不变资本又分解为哪些部分呢\fontbox{?}一方面,分解为原料——铁、木材、皮带等等。另一方面,分解为他生产机器时所必需的工作机的损耗部分(假定这种工作机由机器制造业者自己制造)。我们假定,原料占不变资本的 2/3,机器制造机的损耗占 1/3(我们在后面再考察这 1/3)。用于木材和铁的 2/3,[288]是 2+(2/3)码(2+(2/3)=8/3=24/9)的 2/3。2+(2/3)的 1/3 是 8/9。因此,2/3 是 16/9 码。

假定在这里[在木材和铁的生产中],机器占 1/3,新加劳动占 2/3(因为这里原料不占份额)。在这种情况下,16/9 码的 2/3 补偿新加劳动,1/3 补偿机器。因此,用来补偿机器的是 16/27 码。木材生产者和铁生产者(一句话,采掘工业)的不变资本不是由原料构成,而仅仅是由我们在这里统称为机器的生产工具构成。

由此可见,8/9 码用来补偿机器制造机,16/27 码用来补偿铁生产者和木材生产者使用的机器。这样,24/27+16/27=40/27=1+(13/27)码。这个麻布量还应付给机器制造业者。

\textbf{机器}。24/27 码是对机器制造机的补偿。但机器制造机又分解为原料(铁、木材等等)、生产机器制造机时机器设备的损耗部分以及新加劳动。因而,假定这些要素各等于 1/3,那末 8/27 码就属于新加劳动,而剩下的 16/27 码,用来补偿机器制造机的\textbf{不变资本},也就是说,8/27 码用于原料,8/27 码补偿加工这种原料时机器损耗的那个价值组成部分(共 16/27 码)。

另一方面,用来补偿铁生产者和木材生产者的机器的那 16/27 码,也分解为原料、机器和新加劳动。如果新加劳动等于 1/3,那末它就等于 16/27×3=16/81 码,而这部分机器的不变资本则表现为 32/81 码,其中 16/81 码用于原料,16/81 码补偿机器的损耗。

由此可见,在机器制造业者手中,——为了补偿他的机器的损耗,——仍剩下作为不变资本的 8/27 码(他用这部分补偿他的机器制造机的损耗)以及 16/81 码(用于那些必须由铁生产者和木材生产者补偿的机器的损耗)。

另一方面,机器制造业者必须从自己的不变资本中拿出 8/27 码来补偿机器制造机中包含的原料,拿出 16/81 码来补偿铁生产者和木材生产者的机器中包含的原料。在这个麻布量中,2/3 又归结为新加劳动,1/3 归结为机器损耗。这样一来,在 24/81+16/81(=40/81)中,有 2/3 即[26+(2/3)/81]支付劳动。在这原料中[289]又剩下[13+(1/3)/81]码以补偿机器。因此,这[13+(1/3)/81]码麻布是归还给机器制造业者的。

现在,机器制造业者手中又有:8/27 码,用来补偿机器制造机的损耗,16/81 码,用来补偿铁生产者和木材生产者使用的机器的损耗;[13+(1/3)/81]属于机器制造业者的原料(铁等等)中用来补偿机器的那个价值组成部分。

我们可以这样无止境地计算下去,分数愈来愈小,但是我们这 12 码麻布永远也分不尽。

我们把以上的研究进程简略地概括一下。

起初我们说过,在不同的生产领域中,新加劳动(它一部分补偿花在工资上的可变资本,一部分构成利润即无酬剩余劳动)同有新加劳动加于其上的不变资本之间,存在着不同的比例。但是,我们可以假定 a(新加劳动)和 B(不变资本)之间的平均比例;例如,可以假定后者和前者之比平均为 2∶1=2/3∶1/3。接着,我们还说过,如果资本的每个生产领域中都是这样的比例,那末在某一个生产领域中,新加劳动(工资和利润一起)始终只能购买自己产品的 1/3,因为工资和利润加在一起,只构成物化在产品中的总劳动时间的 1/3。补偿资本家不变资本的那 2/3 产品,当然也属于资本家。但是,资本家要想继续进行生产,就必须补偿自己的不变资本,因而,必须把自己 2/3 的产品再转化为不变资本。为此,他就必须卖掉这 2/3。

但是卖给谁呢\fontbox{?}可以用利润和工资总额来购买的那 1/3 产品,已经被我们扣除。如果这个总额代表 1 工作日或 12 小时,那末价值等于不变资本的那部分产品,就代表 2 工作日或 24 小时。因此我们假定,[第二个]1/3 产品由另一生产部门的利润和工资总额来购买,最后的 1/3 又由第三个生产部门的利润和工资总额来购买。但是,在这种情况下,我们就把产品 I 的不变资本完全同工资和利润相交换,即同新加劳动相交换,办法是让产品 II 和 III 中包含的全部新加劳动去消费产品 I。而产品 II 和 III 包含的 6 工作日(不仅是新加劳动,还有过去劳动)中,任何一个工作日都既没有用产品 I 包含的劳动来补偿或购买,也没有用产品 II 和 III 包含的劳动来补偿或购买。因此,我们必须再假设,其他产品的生产者花费自己的全部新加劳动来购买产品 II 和 III,依此类推。最后,我们必须停在某种产品 X 上,它的新加劳动等于以前一切产品的不变资本的总和;可是占这个产品 2/3 的不变资本仍然卖不出去。可见,对于解决问题来说,仍然没有前进一步。在产品 X 上,就象在产品 I 上一样,问题仍然是那一个:补偿不变资本的那部分产品卖给谁\fontbox{?}难道占产品 1/3 的新加劳动能补偿产品中包含的 1/3 新劳动和 2/3 过去劳动吗\fontbox{?}难道 1/3=3/3 吗\fontbox{?}

由此可见,把困难从产品 I 推到产品 II,并依此类推下去,一句话,把仅仅作为商品交换的中介环节引进来,是不能说明任何问题的。

[290]我们不得不换个方式提出问题。

我们假定,12 码麻布(=36 先令=36 劳动小时)这个产品包含织布业者的 12 劳动小时,或者说,他的 1 工作日(必要劳动和剩余劳动合在一起,即利润和工资总额),而 2/3 代表麻布中包含的不变资本(纱和机器等)的价值。其次,为了切断各种遁词的后路并避免把中间交易引进来,我们又假定,我们的这种麻布只用于个人消费,因此不能再用作某种新产品的原料。从而我们也就假定,这种产品必须由工资和利润来支付,必须同收入相交换。最后,为了简单起见,我们假定,利润中没有一个部分再转化为资本,全部利润都作为收入来花费。

至于前 4 码,即产品的第一个 1/3(等于织布业者加进的 12 劳动小时),我们很快就把它的问题解决了。它分解为工资和利润;它们的价值等于织布业者的利润和工资总额的价值。因而它由织布企业主和他的工人自己消费。对于这 4 码来说,问题是彻底解决了。事实上,如果利润和工资不是以麻布的形式,而是以其他某种产品的形式消费掉,那末,这只是因为其他产品的生产者以麻布的形式,而不是以自己产品的形式消费本来用于他自己消费的那部分产品。例如,在 4 码麻布中,如果织布业者自己只消费 1 码,而其余 3 码,他以肉、面包、呢绒的形式来消费,那末,4 码麻布的价值仍旧是被织布业者自己消费掉,只有一点不同,就是织布业者是以其他商品的形式消费这一价值的 3/4,而其他商品的生产者则以麻布的形式来消费那些可以作为工资和利润被他们消费的肉、面包、呢绒。\fontbox{~\{}当然,在这里也象在全部研究中一样,我们始终假定,商品能够卖出去,而且是按它的价值卖出去的。\fontbox{\}~}

现在我们接触到问题本身。织布业者的不变资本现在以 8 码麻布(=24 劳动小时=24 先令)的形式存在。如果织布业者想要继续进行生产,他就必须把这 8 码麻布转化为货币,转化为 24 先令,并用这 24 先令购买市场上现有的、新生产出来的、构成他的不变资本的那些商品。为使问题简单起见,我们假定,织布业者不是过若干年后一下子补偿自己的机器,而是每天用他出卖自己产品得到的货币,以实物形式补偿机器的一部分,即这些机器每天损耗的那部分价值。等于生产过程中消费的不变资本价值的那部分产品,织布业者必须用该不变资本的各个要素来补偿,也就是用织布劳动所必需的物质生产条件来补偿。另一方面,织布业者的产品麻布,并不作为生产条件加入其他任何生产领域,它只加入个人消费。因此,织布业者为了能够补偿他的产品中代表他的不变资本的部分,只有一种办法,即把这一部分产品同收入相交换,也就是同其他生产者的产品中归结为工资和利润,因而归结为新加劳动的那部分价值相交换。这样问题就正确地提出来了。现在只是要问:在什么条件下问题才能够解决。

我们第一次提出问题时发生的一个困难,现在已经部分地消除了。虽然在每个生产领域中新加劳动都等于 1/3,而不变资本根据假定等于 2/3,但是这个由新加劳动构成的 1/3,或者说,收入(工资和利润;前面已经说明,我们在这里不谈再转化为资本的那部分利润)的价值总额,只能以直接为个人消费而进行生产的那些部门的产品的形式来消费。其余一切生产部门的产品只能作为资本来消费,只能加入生产消费。

[291]8 码(=24 小时=24 先令)所代表的不变资本,由纱(原料)和机器构成,比方说,其中 3/4 为原料,1/4 为机器。(在这里,也可以把全部辅助材料如机油、煤炭等等,都算在原料中,但为简单起见,最好把它们完全撇开不谈。)在这种情况下,纱值 18 先令或 18 劳动小时=6 码,而机器值 6 先令=6 劳动小时=2 码。

因此,如果织布业者用他的 8 码购买值 6 码的纱和值 2 码的机器,他用自己的 8 码不变资本就不仅抵补了纺纱业者和织机厂主的不变资本,而且抵补了他们的新加劳动。由此可见,在织布业者那里表现为不变资本的价值的一部分,在纺纱业者和织机厂主方面就表现为新加劳动,因此,这部分对他们来说,并不归结为资本,而归结为收入。

在 6 码麻布中,纺纱业者可以自己消费 1/3,即 2 码(=新加劳动,即利润和工资),而 4 码只是补偿他的亚麻和机器。比方说,3 码用于亚麻,1 码用于机器。当他重新购买时,他又必须用这 4 码来支付。在[得自织布业者的]2 码中,机器制造业者可以自己消费 2/3 码,其余 4/3 码只是补偿他的铁和木材即原料,以及生产机器时使用的机器设备。比方说,在 4/3 码中,1 码用于原料,1/3 码用于机器。

到现在为止,在 12 码中:(1)4 码被织布业者消费;(2)2 码被纺纱业者消费;(3)2/3 码被机器制造业者消费。总共 6+(2/3)码。因而,还剩下 5+(1/3)码。它们分配如下:

纺纱业者必须从 4 码的价值中拿出 3 码补偿亚麻,拿出 1 码补偿机器。

机器制造业者必须从 4/3 码的价值中拿出 1 码补偿铁等等,拿出 1/3 码补偿机器(他自己在制造机器时使用的那些机器)。

由此可见,纺纱业者购买亚麻,把 3 码支付给亚麻种植业者。但是亚麻种植业者有一个特点,就是他的不变资本的一部分(即种子、肥料等等,一句话,所有由亚麻种植业者再归还给土地的土地产品)完全不加入流通,因而无需从他出卖的产品中扣除;他出卖的产品(除了补偿机器、人造肥料等等的那部分以外)只代表新加劳动,所以,这个产品只分解为工资和利润。如果我们在这里也象前面那样,假定新加劳动占总产品的 1/3,那末 3 码中就有 1 码属于新加劳动这一范畴。至于其余的 2 码,我们象以前一样假定,其中 1/4 用于机器;它等于 2/4 码。相反,其余的 6/4 码仍然不得不归入新加劳动,因为亚麻种植业者的这一部分产品不包含不变资本——这种不变资本已经被他事先扣除了。因此,在亚麻种植业者那里,属于工资和利润的是 2+(2/4)码,用来补偿机器的是 2/4 码。\fontbox{~\{}这样,按照我们的计算,在有待消费的 5+(1/3)码中,已经用掉 2+(2/4)(5+(4/12)-2+(6/12)=2+(10/12)=2+(5/6)码)。\fontbox{\}~}因此,最后的这 2/4 码,亚麻种植业者必定会用来购买机器。

机器制造业者的账目现在是这样:从用于织机的不变资本中,他把 1 码花在铁等等上,把 1/3 码花在生产织机过程中机器制造机的损耗上。

但是,其次,纺纱业者会用 1 码向机器制造业者购买纺机,亚麻种植业者会用 2/4 码向机器制造业者购买农具。在这 6/4 码中,机器制造业者要消费 1/3 来补偿他的新加劳动,2/3 则用来补偿投入纺机和农具的不变资本。而 6/4=18/12。因此,机器制造业者[292]又要消费 6/12 码,而把 12/12 即 1 码转化为不变资本。(这样,在尚未消费的 2+(5/6)码麻布中,减去 1/2 码;剩下 14/6 码,即 2+(2/6)或 2+(1/3)码。)

在机器制造业者手中剩下来补偿他的不变资本的 1 码麻布中,机器制造业者必须把 3/4 花在原料即铁、木材等等上,把 1/4 支付给自己,以补偿机器制造机。

全部计算现在是这样:

\todo{}

这样,1+(3/4)码被用来向制铁业者和木材业者购买价值相等的铁和木材。7/4=21/12。但是这里产生了新问题。在亚麻种植业者那里,一部分不变资本(原料)并没有加入他所出卖的产品,因为事先已经扣除了。而在我们现在所考察的这种场合,我们必须把全部产品[铁、木材]分解为新加劳动和机器。即使假定这里新加劳动占产品的 2/3,机器占 1/3,应当消费掉的也只是 14/12 码,还会剩下 7/12 码作为不变资本,属于机器所占的部分;这 7/12 码还要回到机器制造业者手里。

因此,12 码的余数包括:机器制造业者必须支付给自己以补偿他自己的机器损耗的 1/3+1/4 码;以及制铁业者和木材业者为补偿机器而归还给机器制造业者的 7/12 码。于是,1/3+1/4=4/12+3/12=7/12;再加上制铁业者和木材业者归还的 7/12(共计 14/12,也就是 1+(2/12)或 1+(1/6))。

同织布业者、纺纱业者和亚麻种植业者完全一样,制铁业者和木材业者也必须向机器制造业者购买机器和工具。假定在 7/12 码中 1/3(2/12 码)是新加劳动。因而这 2/12 码也能够被消费掉。剩下的 5/12(其实是 4/12 和[2/3/12],不过在这里不必这样准确)代表伐木者的斧头和制铁业者的机器中包含的不变资本,而且 3/4 用于生铁、木材等等,1/4 用来补偿机器的损耗。(从 14/12 码中剩下 12/12 码,或 1 码=3 劳动小时=3 先令。)因而,在 1 码中,1/4 码用来补偿机器制造机,3/4 码用于木材、铁等等。

这样,用来补偿机器制造机损耗的是 7/12 码+1/4 码=7/12+3/12=10/12 码。另一方面,把木材和铁所占的 3/4 码再分解为它的组成部分,并把这些部分中的一部分重新还给机器制造业者,机器制造业者又把这一部分中的一部分还给制铁业者[293]和木材业者,这是徒劳无益的。始终会有一个余额,并且将无止境地演进下去。

\tsubsectionnonum{[(c)生产资料生产者中间资本同资本的交换。一年生产的劳动产品和一年新加劳动的产品]}

我们就来考察一下现在摆在我们面前的这个问题。

机器制造业者把 10/12(或 5/6)码的价值留给自己,来补偿机器的损耗。3/4(或 9/12)码代表木材和铁的相应价值。机器制造业者把它交给制铁业者和木材业者,来补偿自己的原料。麻布的总余额[不必再进一步分解为麻布的各组成部分]是 19/12(或 1+(7/12))码。

机器制造业者为补偿自己机器的损耗而给自己留出的 5/6 码余额,等于 15/6 先令=15/6 劳动小时,因而等于 2+(3/6)=2+(1/2)先令或 2+(1/2)劳动小时。这个价值不能以麻布的形式补偿给机器制造业者;因为他必须再卖掉这些麻布,以便用这 2+(1/2)先令来补偿自己机器的损耗,一句话,以便生产新的机器制造机。但是,这些麻布能够卖给谁呢\fontbox{?}卖给其他产品(铁和木材除外)的生产者吗\fontbox{?}但是,这些生产者能够以麻布的形式消费的一切,他们都已经以麻布的形式消费了。只有构成织布业者的工资和利润的那 4 码,能够同其他产品相交换(加入织布业者的不变资本的那些产品除外,或者说,由这个不变资本分解成的那种劳动除外)。可是这 4 码,我们已经计算过了。或者,机器制造业者也许会把这些麻布支付给工人\fontbox{?}但是由劳动加到产品上的一切,我们已经从他的产品中扣除了,并且按照我们的假定,这一切都以麻布的形式消费掉了。

我们换一个方式来说明问题:

\todo{}

为了计算简单起见,假定:4 码=12 先令=12 劳动小时,其中劳动(利润和工资)占 1/3,即 4/3 码或 1+(1/3)码。

剩下 2+(2/3)码为不变资本,其中 3/4 用于原料,1/4 补偿机器的损耗。2+(2/3)=8/3=32/12。这个量的 1/4 等于 8/12 码。

补偿机器损耗的这 8/12 码,就是机器制造业者手中剩下来的全部,因为 24/12 码(2 码)他要支付给铁生产者和木材生产者,以取得原料。

[294]如果让铁生产者和木材生产者再次支付机器,那是错误的,因为他们在机器上必须补偿的所有东西即 7/12 码,已经列到机器制造业者的项下了。他们生产铁和木材所必需的全部机器都已算在这一项了,所以这些机器不能再次列入计算。这样,用来支付铁和木材的最后 2 码(2+(8/12)的余额)就完全归结为劳动(因为这里没有原料),因此能够以麻布的形式消费掉。

可见,剩下来的全部余额为 8/12 码(2/3 码),它用于补偿机器制造业者使用的机器的损耗。

整个问题有一部分是这样解决的:土地耕种者的既不归结为新加劳动又不归结为机器的那部分\textbf{不变资本},根本不加入流通,事先就被扣除了;它在它自身的生产中自己补偿自己,因而,土地耕种者全部\textbf{加入流通}的产品,扣除机器之后,都分解为工资和利润,因此能够以麻布的形式消费掉。这是已经解决的一部分。

另一部分是这样解决的:在一个生产领域中表现为不变资本的东西,在其他生产领域中表现为同年内加进的新劳动。在织布业者手中表现为不变资本的东西,有很大一部分归结为纺纱业者、机器制造业者、亚麻种植业者、制铁业者、木材业者(煤炭业者等等;但是为使问题简单起见,我们不把后面这些计算在内)的收入。(这是非常明显的,例如,同一个工厂主又纺又织,他的不变资本就比织布业者的少,而他加进的劳动,即他的产品中归结为新加劳动,归结为收入即利润和工资的那部分,则比织布业者的多。例如织布业者的收入等于 4 码=12 先令,不变资本等于 8 码=24 先令。如果他同时又纺又织,他的收入就=6 码,他的不变资本也=6 码;即 2 码用于织机,3 码用于亚麻,1 码用于纺机。)

第三,直到现在我们所找到的解决办法是:为生产最终加入个人消费的产品提供原料和劳动工具的一切生产者,都不是以自己产品的形式来消费自己的收入,即代表新加劳动的利润和工资。他们只能以这里所说的可直接消费的产品形式,或者同样可以说,以交换来的、具有同等价值的、其他生产者的可直接消费的产品形式,来消费他们的产品中归结为收入的那部分价值。原料和劳动工具的生产者的新加劳动,作为价值组成部分加入最终产品,只有以最终产品的形式才被消费掉,而从使用价值来看,这个新加劳动则作为原料或消费了的机器包含在最终产品中。

因此,问题中有待解决的部分就归结为这样一点:用来补偿机器制造业者的机器制造机损耗的那 2/3 码将会怎样呢\fontbox{?}(这里所谈的正是这种机器的损耗,而不是织布业者、纺纱业者、亚麻种植业者、制铁业者、木材业者使用的工作机的损耗,因为这些工作机归结为新劳动,也就是归结为这样一种新劳动,它使本身不再在原料上有所花费的那种原料获得新机器的形式。)或者换句话说,机器制造业者在什么条件下才能以麻布的形式消费这 2/3 码(等于 2 先令或 2 劳动小时),同时又补偿自己的机器\fontbox{?}这就是问题的实质所在。这种情况确实是有的。它是必然要发生的。因而我们的任务是说明这种现象。

[295]转化为新资本(无论流动资本还是固定资本;无论可变资本还是不变资本)的那部分利润,我们在这里可以完全不去注意。它和我们的问题毫无关系,因为在这种场合,新的可变资本和新的不变资本一样,都是由\textbf{新}劳动(剩余劳动的一部分)创造和补偿的。

总之,如果把这一点撇开不谈,那末加进的(例如一年内加进的)新劳动的总额——等于利润和工资总额,即年收入总额——就统统花在那些加入个人消费的产品如食物、衣服、燃料、住宅、家具等等上面。

这些加入消费的产品总额,按其价值来说,等于一年新加劳动的总额(收入的价值总额)。这个劳动量应当等于这些产品中包含的新加劳动和过去劳动的总额。在购买这些产品时,必须不仅支付其中包含的新加劳动,而且支付其中包含的不变资本。如上所说,它们的价值等于利润和工资总额。当我们举麻布作例子时,麻布对于我们来说,代表一年内加入个人消费的产品总额。这个麻布不仅按其价值来说必须等于它的全部价值要素,而且它的全部使用价值对各个分得麻布的生产者来说必定是可消费的。它的全部价值必然分解为利润和工资,即分解为一年新加劳动的各个组成部分,虽然这个麻布是由新加劳动和不变资本构成的。

如上所说,这个问题部分地可以这样来解释:

\textbf{第一},生产麻布所必需的不变资本的一部分,既不作为使用价值,也不作为交换价值加入麻布。这就是归结为种子等等的那部分亚麻;农产品的这一部分不变资本不进入流通,而是直接或间接地归还给生产,归还给土地。这一部分自己补偿自己,因而不需要用麻布偿还。\fontbox{~\{}农民可能把自己收获的谷物,比方说,120 夸特全部卖掉。但是在这种情况下,他就必须向别的农民购买种子(例如 12 夸特)。这样一来,别的农民需要从自己产品(120 夸特)中留作种子的就不是 12 夸特,而是 24 夸特,不是 1/10,而是 1/5 了。因而,即使在这种情况下,在 240 夸特中,作为种子归还给土地的也是 24 夸特。不过,这在流通领域中确实有差别。在前一场合,每人留出 1/10,进入流通的是 216 夸特。在后一场合,进入流通的是第一个农民的 120 夸特和第二个农民的 108 夸特,共计 228 夸特。而实际消费者所占用的和原先一样,只有 216 夸特。可见,在这里我们已经有了一个例子,表明“实业家”和“实业家”之间流通的价值总额可以大于“实业家”和消费者之间流通的价值总额。\endnote{马克思这里批评的是斯密的(为图克接受的)论点:“各种实业家之间流通的商品的价值,绝不能超过实业家和消费者之间流通的商品的价值。”见前面第 111 页。——第 131 页。}\fontbox{\}~}(其次,每当一部分利润转化为新资本时,都存在着这样的差别;再其次,当“实业家”和“实业家”之间的交易持续多年时也是这样,等等。)

由此可见,生产麻布即生产可供个人消费的产品所必需的很大一部分不变资本,无需用麻布来补偿。

\textbf{第二},麻布即一年内生产出来的消费品所必需的很大一部分不变资本,在一个阶段上表现为不变资本,在另一个阶段上则表现为新加劳动,因而实际上分解为利润和工资,成为一些人的收入,而同一价值额对另一些人来说则表现为资本。例如,[织布业者的]一部分不变资本归结为纺纱业者的[新加]劳动,等等。

[296]\textbf{第三},在生产可消费的产品所必需的一切中间生产阶段,除原料和某些辅助材料之外,很大一部分产品从来不加入消费品的使用价值,而只是作为价值组成部分加入消费品;机器、煤炭、机油、油脂、皮带等等就是如此。就这些中间生产阶段由于社会分工而作为单独的部门出现来说,它们事实上都只是为下一阶段生产不变资本,而每一中间生产阶段的产品都分解为两部分:一部分代表新加劳动(这部分归结为利润和工资,并在前面指出的限定的条件\endnote{马克思是指他在第 129—130 页上所作的说明,他说他在这里把“转化为新资本的那部分利润”撇开不谈。——第 131 页。}下,归结为收入),另一部分代表消费了的不变资本的价值。因此很清楚,在每一个这样的生产领域,生产者本人也只能消费分解为工资和利润的那部分产品,即扣除等于本领域产品所包含的不变资本价值的产品量之后剩下来的那部分产品。但是这些生产者谁也不能消费上一阶段产品中的任何一部分,不能消费事实上只为下一阶段生产不变资本的所有阶段的产品中的任何一部分。

这样,虽然最终产品(麻布,在这里代表全部可消费的产品)由新加劳动和不变资本构成,以致这种消费品的最后生产者只能消费产品中归结为最后阶段上加进的劳动,即归结为工资和利润总额,归结为他们的收入的那一部分,但是,一切不变资本的生产者也都只是以可消费的产品形式来消费即实现自己的新加劳动。虽然可消费的产品由新加劳动和不变资本构成,但是它的购买价格(除等于最后阶段上新加的劳动量的那部分产品以外)体现着在这个产品的不变资本的生产过程中加进的劳动总量。不变资本的生产者不是以自己产品的形式,而是以可消费的产品的形式,来实现他们加进的全部劳动,所以,结果就好比这个可消费的产品仅仅由工资和利润,由加进的劳动构成。

麻布生产者在自己的生产领域最后制成了消费品麻布,他们从消费品麻布中自己留出一部分产品,这部分产品等于他们的收入,等于最后生产阶段上加进的劳动,等于工资和利润总额(消费品相互交换和商品事先转化为货币,对问题毫无影响)。而他们生产的另一部分消费品,他们则用来支付直接供给他们不变资本的生产者所应得的价值组成部分。因此,他们生产的这一部分消费品,全都用来抵补直接供给这种不变资本的生产者的收入和不变资本的价值。但这种不变资本的生产者又只留出价值等于他们的收入的那部分可消费的产品。另一部分他们又用来支付他们的不变资本的生产者,而这个不变资本又等于收入加不变资本。\textbf{但是,只有}当麻布这种可消费的产品的最后部分仅仅用来补偿收入,补偿新加劳动,而不再补偿不变资本\textbf{的时候,计算才能完结}。因为按照假定,麻布只加入消费,而不再构成其他生产阶段的不变资本。

对于一部分农产品来说,这一点已经得到了证明。

一般说来,只有作为原料加入最终产品的那些产品,才可以说它们是作为产品被消费的。其他的产品则只是作为价值组成部分加入可消费的产品。可消费的产品是用收入,也就是用工资和利润来购买的。因而,它的价值总额必定全部分解为工资和利润,即分解为在这种产品所经过的一切阶段上加进的不同的劳动量。现在要问:除了由生产者本人归还给生产的那部分农产品[297](种子、牲畜、粪肥等等)以外,是否还有另一部分不变资本,不作为价值组成部分加入可消费的产品,而在生产过程中以实物形式自己补偿自己呢\fontbox{?}

当然,这里可能指各种形式的固定资本,只要这种固定资本的价值本身加入生产,并在生产中被消费。

不仅在农业中(其中包括由人工进行再生产的畜牧业、养鱼业、林业),——因而不仅在衣服和食品的各种原料以及很大一部分加入工业固定资本的产品如帆、绳、皮带等等的生产中,——而且在采矿业中,也有一部分不变资本以实物形式从自己生产的产品中得到补偿。因此,这一部分不变资本就不应由进入流通的那部分产品来补偿。例如在煤炭生产中,就有一部分煤炭用来发动排水或提升煤炭用的蒸汽机。

由此看来,年产品的价值,有一部分等于采煤过程中消费的煤炭所包含的过去劳动,另一部分等于当时新加的劳动量(机器的损耗等等撇开不谈)。但是,本身由煤炭构成的那部分不变资本,会直接从这个总产品中留出,归还给生产。谁也不应把这部分补偿给生产者,因为生产者会自己补偿自己。如果劳动生产率既没有提高也没有降低,这部分产品所代表的那部分价值就保持不变,它等于产品中一部分作为过去劳动、一部分作为一年内新加劳动存在的那个劳动量的某个相应部分。在采矿工业的其他部门中,不变资本也有一部分是以实物形式得到补偿的。

产品的废料,例如飞花等等,可当作肥料归还给土地,或者可当作原料用于其他生产部门;例如破碎麻布可用来造纸。在前一种情况下,一个生产部门的一部分不变资本,就可以直接同另一个生产部门的不变资本相交换。例如棉花同用作肥料的飞花相交换。

一般说来,机器的制造和原料(煤炭、铁、木材)的生产,同其他生产阶段之间有一个主要的差别。在其他生产阶段上不发生相互作用。麻布不能成为纺纱业者的不变资本的一部分。纱本身不能成为亚麻种植业者或机器厂主的不变资本的一部分。但是,充当机器的原料的,不仅有皮带、绳子等等取自农产原料的产品,而且有木材、铁、煤炭;另一方面,机器又作为生产资料加入木材、铁、煤炭等等生产者的不变资本。由此可见,这两个部门事实上是以实物形式互相补偿自己的一部分不变资本。这里发生的是不变资本同不变资本的交换。

这里不单单是计算的问题。铁生产者要把生产铁时使用的机器的损耗算到机器制造业者的项下,而机器厂主要把他制造机器时使用的机器的损耗算到铁生产者的项下。假定铁生产者和煤炭生产者是一个人。第一,我们已经看到,他自己补偿自己的煤炭。第二,他的总产品(铁和煤炭)的价值,等于新加劳动创造的价值加机器损耗部分所包含的过去劳动创造的价值。从这个总产品中扣除补偿机器价值的铁量,剩下来的就是归结为新加劳动的铁量。后面这部分构成机器厂主、工具生产者等等的原料。对于后面这部分,机器厂主用麻布支付给铁生产者;而为了换取前一部分,机器厂主供给他机器,以补偿他的设备的损耗。

另一方面,机器制造业者的不变资本中有一部分代表他的机器制造机、工具等等的损耗,所以,这一部分既不能归结为原料(这里我们不谈[生产煤炭和铁时]使用的机器[298]和自己补偿自己的那部分煤炭),也不能归结为新加劳动,因而,既不能归结为工资,也不能归结为利润;这种损耗实际上是靠机器制造业者从自己的机器中给自己留下一部或几部机器当作机器制造机而得到补偿的。对于他的这一部分产品来说,问题只是:为了制造这一部分产品,要有一个原料的追加量。这一部分产品不代表新加劳动,因为在劳动的总产品中,一定数量的机器等于新加劳动创造的价值,另一数量的机器等于原料的价值,再一个数量的机器等于机器制造机所包含的价值组成部分。诚然,最后这个组成部分事实上也包含新加劳动。但是从价值方面来说,这种劳动等于零,因为在代表新加劳动的那一部分机器中,没有计算原料和已损耗的机器所包含的劳动;补偿原料的第二部分机器中,没有计算补偿新劳动和机器的部分;因而,从价值方面来看,第三部分机器既不包含新加劳动,也不包含原料;这一部分只代表机器的损耗。

机器厂主自己所需要的机器不出卖。它以实物形式得到补偿,由机器制造业者从总产品中留出来。这样,机器厂主所出卖的机器只代表原料(如果原料生产者的机器损耗已经算到机器厂主的项下,这些原料就只归结为劳动)和新加劳动;因而这些机器无论对机器厂主自己,还是对原料生产者,都只归结为麻布。如果专门就机器厂主和原料生产者之间的关系来说,那末原料生产者为了补偿自己机器的损耗部分,已经把相当于损耗部分的价值的铁量留出。他用这个铁量同机器厂主相交换,这样他们两人就以实物形式互相付清,这个过程同他们之间收入的分配也就毫无关系。

这个问题就是如此,我们在考察资本流通时还要回过来谈谈。\endnote{见马克思《资本论》第 2 卷第 20 章第 6 节。——第 136 页。}

不变资本实际上是这样得到补偿的:它不断地重新生产出来,并且有一部分是自己再生产自己。但是,加入可消费的产品的那部分不变资本,则由加入不可消费的产品的活劳动来支付。正因为这种劳动不由它本身的产品支付,所以全部可消费的产品都可以归结为收入。不变资本的一部分,作为年产品的一部分来考察,只是外表上的不变资本。另一部分,虽然也加入总产品,但是它既不作为价值组成部分,也不作为使用价值加入可消费的产品,而是以实物形式得到补偿,始终作为不可缺少的生产要素保留下来。

在这里,我们已经考察了全部可消费的产品如何分配,如何分解为产品中包含的各个价值组成部分和生产条件。

但是,可消费的产品(就它分解为工资这一点说,它等于资本的可变部分),可消费的产品的生产,以及生产可消费的产品所必需的不变资本各部分的生产(不管这个不变资本是否加入可消费的产品),它们总是同时并存的。因此,任何资本总是分为不变资本和可变资本,资本同时由这两部分构成;而且,资本的不变部分虽然象可变部分一样,不断由新产品来补偿,但是,只要生产以同一方式继续下去,这个不变部分就会始终以同样的形式存在下去。

[299]机器厂主和原料生产者(铁、木材等等的生产者)之间的关系是,他们事实上各用自己的一部分不变资本来互相交换(这种情况和一个人的一部分不变资本变成另一个人的收入毫无共同之处\endnote{马克思在《资本论》第二卷中批判了下面这种资产阶级观点:“对一个人是资本的东西,对另一个人就是收入;反过来说也一样。”(见马克思《资本论》第 2 卷第 20 章第 10 节)参看《资本论》第 2 卷第 19 章第 2 节第 4 小节、第 3 节和第 3 卷第 49 章。——第 137 页。}),并且在这两个互相联系的生产者当中,每一个人的产品(虽然一个产品是另一个产品的前一阶段),都作为生产资料互相加入对方的不变资本。铁、木材等等的生产者,为了换取他们所需要的机器,把等于待补偿的机器价值的铁、木材等等交给机器制造业者。机器制造业者的这一部分不变资本对于他自己来说,就好比种子对于农民一样。这是他的年产品中由他自己以实物形式补偿自己并且不构成他的收入的那一部分。另一方面,通过这种交换,以原料形式给机器制造业者不仅补偿了铁生产者使用的机器中包含的原料,而且补偿了这个机器中由新加劳动和机器制造业者自己的机器损耗构成的那部分价值。因此,对于机器制造业者来说,不仅补偿了相当于他自己的机器损耗的部分,而且还补偿了可以(当作补偿)算做别的机器包含的一部分损耗的部分。

诚然,卖给铁生产者的这种机器,也包含等于原料和新加劳动的价值组成部分。但是在别的机器中会相应地少算补偿损耗的部分。因此,铁生产者等等的这部分不变资本,即他们的年劳动产品中只补偿不变资本中代表损耗的价值组成部分的这部分产品,不加入机器制造业者卖给其他工业家的机器。至于这些别的机器的损耗,当然是用前面说过的 2/3 码麻布(=2 劳动小时)补偿给机器制造业者。机器制造业者用这些麻布购买同一价值额的生铁、木材等等,并且以自己不变资本的另一种形式,即生铁的形式,来补偿自己的损耗。由此可见,对于机器制造业者来说,他的一部分原料,除补偿原料的价值之外,还补偿他的机器损耗的价值。而在生铁生产者等等方面,这种原料只归结为新加劳动,因为这些原料(铁、木材、煤炭等等)的生产者的机器在前面已经算过了。

由此可见,麻布的一切要素归结为一定量劳动的总额,它等于新加劳动的总额,但决不等于不变资本中包含的、由于再生产而永远保留下去的全部劳动的总额。

而且,说构成一年内加入个人消费的商品总额的那个劳动量(一部分为活劳动,一部分为过去劳动),因而也就是作为收入被消费的那个劳动量,不能超过一年内新加的劳动,这种论点是同义反复。因为收入等于利润和工资总额,等于新加劳动的总额,等于包含这个劳动量的商品总额。

铁生产者和机器制造业者的例子,只是个别的例子。在其他以自己的产品互相提供生产资料的不同生产领域之间,也同样以实物形式交换它们的不变资本(虽然这种交换被一系列货币交易掩盖着)。只要有这种情况存在,加入消费的最终产品的消费者就不应补偿这种不变资本,因为它已经得到了补偿。[299]

[304]\fontbox{~\{}例如在制造机车时,每天都有成车皮的铁屑剩下。把铁屑收集起来,再卖给(或赊给)那个向机车制造厂主提供主要原料的制铁厂主。制铁厂主把这些铁屑重新制成块状,在它们上面加进新的劳动。他以这种形式把铁屑送回机车制造厂主手里,这些铁屑便成为产品价值中补偿原料的部分。就这样这些铁屑往返于这两个工厂之间,——当然,不会是同一些铁屑,但总是一定量的铁屑。这个部分不断交替地成为两个工业部门的原料,并且,从价值方面来看,始终只是从一个企业移到另一个企业。因此,它不加入最终产品,而是不变资本在实物形式上的补偿。

实际上,机器制造厂主供应的每一部机器,如果从它的价值来考察,都分解为原料、新加劳动和机器的损耗。但是加入其他领域生产的这些机器的总数,按其价值来说,只能等于机器的总价值减去不断在机器制造厂主和制铁厂主之间来回转移的那部分不变资本。

农民卖掉的任何一夸特小麦,同其他任何一夸特小麦值一样多的钱。卖掉的一夸特小麦,丝毫也不比作为种子归还给土地的那一夸特小麦便宜。然而,如果产品等于 6 夸特,每一夸特等于 3 镑,而且每一夸特都包含新加劳动、原料和机器这几个价值组成部分,如果农民必须用一夸特作种子,那末他就只卖给消费者 5 夸特,由此得到 15 镑。因而,消费者没有必要支付一夸特种子包含的价值组成部分。但是被卖掉的产品的价值等于其中包含的全部价值要素,即新加劳动和不变资本,消费者怎么能够不支付不变资本而又把这个产品买去呢\fontbox{?}问题的关键就在这里。\fontbox{\}~}\endnote{花括号内的这几段话是在手稿第 304 页,属于第四章。我们把这几段移至第三章是根据马克思在这些话开头所加的注:“接第 300 页”。手稿第 300 页有几段关于萨伊的话,开头写着:“对前一段话的补充”。把这两处加以对照,引人注意的是以下这一情况,即第 304 页的那几段话结尾提出一个问题:“消费者怎么能够……把这个产品买去呢”等等。而在关于萨伊的那几段的结尾,对这个问题做了回答:“仅仅由新加劳动构成的收入,能够支付这个……产品”等等。根据这一点,我们把手稿第 304 页的那几段话放到作为第三章第十节全节结尾部分关于萨伊的那几段之前。——第 139 页。}[304]

[300]\fontbox{~\{}对前一段话的补充。

下面这段引文表明,庸俗的萨伊对这个问题多么不了解:

\begin{quote}“要完全了解这个关于收入的问题,就必须注意,产品的全部价值分解为各种人的收入,因为任何产品的总价值,都是由促成它的生产的土地所有者、资本家和勤劳者的利润相加而成的。因此,社会的收入和生产的\textbf{总价值}相等,而不象某派经济学家\endnote{“经济学家”是十八世纪下半叶和十九世纪上半叶在法国对重农学派的称呼。——第 38、139、223、411 页。}所认为的那样,只和土地的\textbf{纯产品}相等……如果一个国家的收入只是生产出来的价值超过消费掉的价值的余额,那末从这里就会得出一个完全荒谬的结论:如果一个国家在一年内消费的价值等于它生产出来的价值,这个国家就没有任何收入了。”(同上,第 2 卷第 63—64 页)\end{quote}

实际上,这个国家在过去一年会有某些收入,但在下一年就没有任何收入了。说\textbf{一年生产的劳动产品}(\textbf{当年新加劳动的产品}只构成其中的一部分)都归结为收入,这是不对的。相反,只有说加入一年个人消费的那部分产品都归结为收入,才是对的。仅仅由新加劳动构成的收入,能够支付这个一部分由新加劳动、一部分由过去劳动构成的产品,换句话说,新加劳动在这些产品中不仅能够自己支付自己,而且能够支付过去劳动,——这是因为同样由新加劳动和过去劳动构成的另一部分产品,只补偿过去劳动,只补偿不变资本。\fontbox{\}~}

\tsectionnonum{[(11)补充:斯密在价值尺度问题上的混乱;斯密的矛盾的一般性质]}

\fontbox{~\{}对于这里考察的亚当·斯密理论的各点,还应补充如下:他在价值规定上的动摇,除了工资问题上的明显矛盾\endnote{马克思是指他在前面谈过的斯密关于“工资的自然价格”这一见解中的循环论证(见第 77 页)。——第 140 页。}以外,还有一条:混淆概念。他把作为内在尺度同时又构成价值实体的那个价值尺度,同货币称为价值尺度那种意义上的价值尺度混淆起来。由此就试图找到一个价值不变的商品作为后一种意义上的尺度,把它当作衡量其他商品的不变尺度——这是一个化圆为方问题\authornote{化圆为方问题,是古希腊的一个著名问题,即求作一正方形,使其面积等于一已知圆的面积,一般指难以解决的问题。——译者注}。关于货币意义上的价值尺度同价值由劳动时间决定这一价值规定之间的关系,请看我的著作第一部分\endnote{指《政治经济学批判》第一分册。见《马克思恩格斯全集》中文版第 13 卷第 54—66 页。——第 140 页。}。这种混淆现象在李嘉图的著作中,有些地方也可以碰到。\fontbox{\}~}[300]

\centerbox{※     ※     ※}

[299]亚·斯密的矛盾的重要意义在于:这些矛盾包含的问题,他固然没有解决,但是,他通过自相矛盾而提出了这些问题。后来的经济学家们互相争论时,时而接受斯密的这一方面,时而接受斯密的那一方面,这种情况最好不过地证明斯密在这方面的正确本能。\endnote{说明斯密矛盾的一般性质的这一段话,在本版作为结束语放在第三章结尾。这是同这段话在马克思手稿中所占的位置一致的,因为手稿上紧接这一段之后便是下一章的开始。——第 141 页。}

\tchapternonum{[第四章]关于生产劳动和非生产劳动的理论}

现在,我们转过来谈谈分析亚·斯密的观点时必须加以考察的最后一个争论点,即[300]\textbf{生产劳动和非生产劳动的区分问题}。

直到现在为止,我们看到,亚·斯密对一切问题的见解都具有二重性,他在区分\textbf{生产劳动和非生产劳动}时给生产劳动所下的定义也是如此。我们发现,在他的著作中,他称为生产劳动的东西总有两种定义混淆在一起。我们先来考察第一种正确的定义。

\tsectionnonum{[(1)资本主义制度下的生产劳动是创造剩余价值的劳动]}

从资本主义生产的意义上说,生产劳动是这样一种雇佣劳动,它同资本的可变部分(花在工资上的那部分资本)相交换,不仅把这部分资本(也就是自己劳动能力的价值)再生产出来,而且,除此之外,还为资本家生产剩余价值。仅仅由于这一点,商品或货币才转化为资本,才作为资本生产出来。只有生产资本的雇佣劳动才是生产劳动。(这就是说,雇佣劳动把花在它身上的价值额以增大了的数额再生产出来,换句话说,它归还的劳动大于它以工资形式取得的劳动。因而,只有创造的价值大于本身价值的劳动能力才是生产的。)

资本家阶级的存在,从而资本的存在本身,是以劳动生产率为基础的,但不是以绝对的劳动生产率为基础,而是以相对的劳动生产率为基础。如果一个工作日只够维持一个劳动者的生活,也就是说,只够把他的劳动能力再生产出来,[301]那末,绝对地说,这一劳动是生产的,因为它能够再生产即不断补偿它所消费的价值(这个价值额等于它自己的劳动能力的价值)。但是,从资本主义意义上来说,这种劳动就不是生产的,因为它不生产任何剩余价值。(它实际上不生产任何新价值,而只补偿原有价值;它以一种形式消费价值,为的是以另一种形式把价值再生产出来。有人说,一个劳动者,如果他的产品等于他自己的消费,他就是生产劳动者,如果他消费的东西多于他再生产的东西,他就是非生产劳动者,也就是从这个意义上说的。)

这种生产率是以相对的生产率为基础的,即工人不仅补偿原有价值,而且创造新价值;他在自己的产品中物化的劳动时间,比维持他作为一个工人生存所需的产品中物化的劳动时间要多。这种生产的雇佣劳动也就是资本存在的基础。

\fontbox{~\{}但是,假定不存在任何资本,而工人自己占有自己的剩余劳动,即他创造的价值超过他消费的价值的余额。只有在这种情况下才可以说,这种工人的劳动是真正生产的,也就是说,它创造新价值。\fontbox{\}~}

\tsectionnonum{[(2)重农学派和重商学派对生产劳动问题的提法]}

对生产劳动的这种观点,是从亚·斯密对剩余价值的起源的看法,因而是从他对资本的实质的看法,自然而然地得出来的。只要他对生产劳动持有这种观点,他就沿着重农学派甚至重商学派走过的一个方向走,不过使这个方向摆脱了错误的表述方式,从而揭示出它的内核。重农学派错误地认为,只有农业劳动才是生产的,但是他们坚持了正确的见解,即从资本主义观点来看,只有创造剩余价值的劳动,并且不是为自己而是为生产条件所有者创造剩余价值的劳动,才是生产的;只有不是为自己而是为土地所有者创造“纯产品”的劳动,才是生产的。因为剩余价值或剩余劳动时间是物化在剩余产品或“纯产品”中的。(重农学派对“纯产品”又理解错误。他们所以把它当作纯产品,是因为例如收获的小麦比工人和租地农场主吃掉的要多;可是生产出来的呢绒也比呢绒生产者即工人和企业主的衣着所需的要多。)他们对剩余价值本身的理解是错误的,因为他们对价值有不正确的看法,他们把价值归结为劳动的使用价值,而不是归结为劳动时间,不是归结为没有质的差别的社会劳动。不过,尽管如此,他们还是有一个正确的定义:雇佣劳动只有当它所创造的价值大于它本身所花费的价值的时候才是生产的。亚·斯密使这个定义摆脱了错误的表述方式,而在重农学派那里,这个定义是同错误的表述方式联系在一起的。

如果我们从重农学派追溯到重商学派,在重商学派那里也有对生产劳动的同样见解的一面,尽管他们对这一点是无意识的。重商学派的基本观点是:劳动只有在产品出口给国家带回的货币多于这些产品所值的货币(或者多于为换得这些产品而必须出口的货币)的那些生产部门,因而只有在使国家有可能在更大的程度上分沾当时新开采的金银矿的产品的那些生产部门,才是生产的。他们看到,在这些国家中已经出现了财富和中等阶级迅速增长的情况。金的这种影响事实上究竟是由什么原因造成的呢\fontbox{?}工资的增长赶不上商品价格的上涨;因此工资下降了,从而相对剩余劳动增加了,利润率提高了,但这不是因为工人的生产能力更大了,而是因为绝对工资(即工人得到的生活资料总额)降低了,总之,因为工人的状况恶化了。这样一来,在这些国家里,对使用劳动的企业主来说,劳动的生产能力实际上更大了。这个事实和贵金属的流入有关,这也就是促使重商学派把这种生产部门使用的劳动称为唯一生产劳动的原因,虽然这个原因仅仅是隐约地被意识到的。

\begin{quote}[302]“最近五、六十年以来,几乎在整个欧洲都发生了人口的惊人增加,其主要原因也许是美洲矿山生产率的增长。贵金属的大大过剩〈这当然是它们的实际价值下降的结果〉,使商品的价格比劳动的价格提高得更多;它使工人的状况恶化,同时却使雇主的利润增加,因此雇主能使用更多的流动资本来雇用工人,这就促进了人口的增加……马尔萨斯指出,美洲矿山的发现,使谷物价格提高了两三倍,而使劳动的价格只提高了一倍……供国内消费的商品的价格(例如谷物价格)不是马上跟着货币的流入就提高的;但由于农业中的利润率同工业中的利润率相比下降了,资本就从农业转到工业。这样,一切资本都开始获得比以前更高的利润,而利润的提高总是等于工资的下降。”(\textbf{约翰·巴顿}《论影响社会上劳动阶级状况的环境》1817 年伦敦版第 29 页及以下各页)\end{quote}

因此,第一,按巴顿的说法,十六世纪最后三十多年和十七世纪曾推动重商主义体系的那个现象,在十八世纪下半叶重新出现了。第二,因为只有出口的商品才按金银的已经降低的价值衡量,而供国内消费的商品仍按金银的原有价值衡量(直到资本家之间的竞争把这种用两个不同尺度衡量的现象消除为止),所以在为出口服务的生产部门中,由于工资降到原有水平之下,劳动就表现为直接生产的劳动,即创造剩余价值的劳动。

\tsectionnonum{[(3)斯密关于生产劳动的见解的二重性。对问题的第一种解释:把生产劳动看成同资本交换的劳动]}

斯密对于生产劳动所阐述的第二种见解即错误的见解,同正确的见解完全交错在一起,以致这两种见解在同一段文字中接连交替出现。所以,为了说明第一种见解,我们不得不在有些地方把引文分割成许多段。

\begin{quote}“有一种劳动加到对象上,就能使这个对象的价值增加,另一种劳动则没有这种作用。前一种劳动因为\textbf{它生产价值},可以称为\textbf{生产劳动},后一种劳动可以称为\textbf{非生产劳动}。例如,制造业工人的劳动,通常把\textbf{自己的生活费的价值和他的主人的利润,加到}他所加工的材料的价值上。相反,家仆的劳动不能使价值有任何增加。虽然主人也向制造业工人\textbf{预付}工资,但后者\textbf{实际上并没有使主人花费什么},因为由工人投入劳动的对象的价值增加了,通常通过这个增加了的价值,就把工资的价值\textbf{连同利润一起}偿还给主人了。相反,家仆的生活费永远得不到偿还。一个人,要是雇用许多制造业工人,\textbf{就会变富};要是维持许多家仆,就会变穷。”(《国民财富的性质和原因的研究》,麦克库洛赫版,第 2 卷第 2 篇第 3 章第 93 和 94 页)\end{quote}

在这段话中,——而在下面我们就要引用的紧接着的那段文字里,相互矛盾的定义更加穿插在一起,——生产劳动主要是指这样一种劳动,它除了再生产“自己的〈即雇佣工人的〉生活费”的价值之外,还生产剩余价值——“他的主人的利润”。如果制造业工人除了他自己的生活费的价值以外,不再创造剩余价值,工业家也就不能由于“雇用许多制造业工人”而\textbf{变富}。

但是,第二,亚·斯密在这里所说的生产劳动是指一般“生产价值”的劳动。我们暂且不谈这[303]后一种解释,先引证另外几段话,那里斯密的第一种见解一部分被重复了,一部分表述得更鲜明,并且主要是得到了进一步的发挥。

\begin{quote}“如果把非生产劳动者……消费的那个数量的食物和衣服,分配给生产劳动者,后者就会把他们所消费的东西的全部价值\textbf{连同利润一起}再生产出来。”(同上,第 109 页;第 2 篇第 3 章)\end{quote}

这里,生产劳动者十分明确是指这样的劳动者,他不仅把包含在工资中的生活资料的全部价值为资本家再生产出来,而且把这个价值“连同利润一起”为资本家再生产出来。

只有生产资本的劳动才是生产劳动。但是,商品或货币之所以变为资本,是因为它们直接同劳动能力交换,而且这种交换的目的,只是为了有一个比它们本身包含的劳动量更大的劳动量来补偿它们。因为劳动能力的使用价值对资本家本身来说,不在于它的\textbf{实际}使用价值,不在于某种具体劳动的效用,不在于这是纺纱者的劳动、织布者的劳动等等,正如这种劳动产品的使用价值本身并不使资本家感到兴趣一样,因为产品在他看来是商品(并且是第一形态变化之前的商品),而不是消费品。使资本家对商品感兴趣的仅仅是:商品具有的交换价值大于资本家为商品支付的交换价值。因此,劳动的使用价值在他看来就是:他收回的劳动时间量大于他以工资形式支付的劳动时间量。自然,所有以这种或那种方式参加商品生产的人,从真正的工人到(有别于资本家的)经理、工程师,都属于生产劳动者的范围。正因为如此,最近的英国官方工厂报告“\textbf{十分明确地}”把在工厂和工厂办事处就业的所有人员,除了工厂主本人以外,全都列入雇佣劳动者的范畴(见这个臭报告结尾部分以前的话)。

这里,从资本主义生产的观点给生产劳动下了定义,亚·斯密在这里触及了问题的本质,抓住了要领。他的巨大科学功绩之一(如马尔萨斯正确指出的\endnote{马克思指马尔萨斯的这两句话:对生产劳动和非生产劳动的区分是亚当·斯密著作的基石,是他的论述的主要思路的基础(马尔萨斯《政治经济学原理》1836 年伦敦第 2 版第 44 页)。——第 148 页。},斯密对生产劳动和非生产劳动的区分,仍然是全部资产阶级政治经济学的基础)就在于,他下了生产劳动是\textbf{直接同资本交换的劳动}这样一个定义,也就是说,他根据这样一种交换来给生产劳动下定义,只有通过这种交换,劳动的生产条件和一般价值即货币或商品,才转化为资本(而劳动则转化为科学意义上的雇佣劳动)。

什么是\textbf{非生产劳动},因此也绝对地确定下来了。那就是不同资本交换,而\textbf{直接}同收入即工资或利润交换的劳动(当然也包括同那些靠资本家的利润存在的不同项目,如利息和地租交换的劳动)。凡是在劳动一部分还是自己支付自己(例如徭役农民的农业劳动),一部分直接同收入交换(例如亚洲城市中的制造业劳动)的地方,不存在资产阶级政治经济学意义上的资本和雇佣劳动。因此,这些定义不是从劳动的物质规定性(不是从劳动产品的性质,不是从劳动作为具体劳动所固有的特性)得出来的,而是从一定的社会形式,从这个劳动借以实现的社会生产关系得出来的。例如一个演员,哪怕是丑角,只要他被资本家(剧院老板)雇用,他偿还给资本家的劳动,多于他以工资形式从资本家那里取得的劳动,那末,他就是生产劳动者;而一个缝补工,他来到资本家家里,给资本家缝补裤子,只为资本家创造使用价值,他就是非生产劳动者。前者的劳动同资本交换,后者的劳动同收入交换。前一种劳动创造剩余价值;后一种劳动消费收入。

这里,生产劳动和非生产劳动始终是\textbf{从货币所有者、资本家的角度}来区分的,不是从\textbf{劳动者}的角度来区分的,而加尼耳等人的荒谬论调正是从这里产生的,他们根本不懂问题的实质,竟然问道:妓女、仆役等等的劳动,或服务,或职能,会不会带来货币\fontbox{?}[303]

[304]作家所以是生产劳动者,并不是因为他生产出观念,而是因为他使出版他的著作的书商发财,也就是说,只有在他作为某一资本家的雇佣劳动者的时候,他才是生产的。

体现生产工人的劳动的商品,其使用价值可能是最微不足道的。劳动的这种物质规定性同劳动作为生产劳动的特性毫无关系,相反,劳动作为生产劳动的特性只表现一定的社会生产关系。我们在这里指的劳动的这种规定性,不是从劳动的内容或劳动的结果产生的,而是从劳动的一定的社会形式产生的。

另一方面,假定资本已掌握了全部生产,也就是说,\textbf{商品}(必须把它同单纯的使用价值区别开来)已不再由拥有这个商品的生产条件的劳动者来生产,因而只有资本家才是\textbf{商品}(只有一种商品——劳动能力除外)的生产者,那末,在这种情况下,收入必须\textbf{或者}同完全由资本来生产和出卖的商品交换,\textbf{或者}同这样一种劳动交换,购买它和购买那些商品一样,是为了消费,换句话说,仅仅是由于这种劳动所固有的物质规定性,由于这种劳动的使用价值,由于这种劳动以自己的物质规定性给自己的买者和消费者提供\textbf{服务}。对于提供这些服务的生产者来说,服务就是商品。服务有一定的使用价值(想象的或现实的)和一定的交换价值。但是对买者来说,这些服务只是使用价值,只是[305]他借以消费自己收入的对象。这些非生产劳动者并不是不付代价地从收入(工资和利润)中取得自己的一份,从生产劳动创造的商品中取得自己的一份,他们必须购买这一份,但是,他们同这些商品的生产毫无关系。

但在任何情况下,有一点是很清楚的:花在资本所生产的商品上的收入(工资和利润)愈多,能用来购买非生产劳动者的服务的收入就愈少,反过来也是一样。

劳动的物质规定性,从而劳动产品的物质规定性本身,同生产劳动和非生产劳动之间的这种区分毫无关系。例如,饭店里的厨师和侍者是生产劳动者,因为他们的劳动转化为饭店老板的资本。这些人作为家仆,就是非生产劳动者,因为我没有从他们的服务中创造出资本,而是把自己的收入花在这些服务上。但是,事实上,这些人,对我这个消费者来说,即使在饭店里也是非生产劳动者。

\begin{quote}“\textbf{无论在哪一个国家},土地和劳动的年产品中\textbf{补偿资本}的那部分,始终只\textbf{直接}用来维持生产劳动者的生活。它只\textbf{支付生产劳动的工资}。而\textbf{直接}用来构成收入(不管作为利润还是作为地租)的那部分,则既可以用来维持生产劳动者的生活,也可以用来维持非生产劳动者的生活。一个人无论把自己的哪一部分基金用作资本,他总是希望这部分基金能得到补偿并带来利润。因此,他只用它来维持\textbf{生产劳动者}的生活;这部分基金为资本家执行了资本的职能之后,便成为生产劳动者的收入。每当资本家用他的一部分基金来\textbf{维持}任何一种\textbf{非生产劳动者的生活},这部分基金便立即从他的资本中抽出,加入他用于直接消费的基金。”(同上[麦克库洛赫版第 2 卷],第 98 页)\end{quote}

显然,随着资本日益掌握全部生产,从而随着家庭工业和小工业——总之,为本身消费进行生产而产品不是商品的那种工业——逐渐消失,非生产劳动者,即以服务直接同收入交换的劳动者,绝大部分就只提供\textbf{个人}服务,他们中间只有极小部分(例如厨师、女裁缝、缝补工等)生产物质的使用价值。他们不生产\textbf{商品}是理所当然的。因为商品本身从来不是直接的消费对象,而是交换价值的承担者。因此,在资本主义生产方式发达的条件下,这些非生产劳动者只有极小部分能够直接参加物质生产。这一部分人只有用自己的服务同收入交换,才参加物质生产。正如亚·斯密所指出的,这不妨碍这些非生产劳动者的服务的价值,是由并且可以由决定生产劳动者的价值的同样方法(或类似方法)来决定。这就是说,由维持他们的生活或者说把他们生产出来所必需的生产费用来决定。这里还牵涉到别的一些不归这里考察的情况。

[306]生产劳动者的劳动能力,对他本人来说是商品。非生产劳动者的劳动能力也是这样。但是,生产劳动者为他的劳动能力的买者生产商品。而非生产劳动者为买者生产的只是使用价值,想象的或现实的使用价值,而决不是商品。非生产劳动者的特点是,他不为自己的买者生产商品,却从买者那里获得商品。

\begin{quote}“某些最受尊敬的社会阶层的劳动,象家仆的劳动一样,不生产任何价值……例如,君主和他的全部文武官员、全体陆海军,都是非生产劳动者。他们是社会的公仆,靠别人劳动的一部分年产品生活……应当列入这一类的,还有……教士、律师、医生、各种文人;演员、丑角、音乐家、歌唱家、舞蹈家等等。”(同上,第 94—95 页)\end{quote}

生产劳动和非生产劳动的这种区分本身,正如前面已经说过的,既同劳动独有的特殊性毫无关系,也同劳动的这种特殊性借以体现的特殊使用价值毫无关系。在一种情况下劳动同资本交换,在另一种情况下劳动同收入交换。在一种情况下,劳动转化为资本,并为资本家创造利润;在另一种情况下,它是一种支出,是花费收入的一个项目。例如,钢琴制造厂主的工人是生产劳动者。他的劳动不仅补偿他所消费的工资,而且在他的产品钢琴中,在厂主出售的商品中,除了工资的价值之外,还包含剩余价值。相反,假定我买到制造钢琴所必需的全部材料(或者甚至假定工人自己就有这种材料),我不是到商店去买钢琴,而是请工人到我家里来制造钢琴。在这种情况下,钢琴匠就是非生产劳动者,因为他的劳动直接同我的收入相交换。

\tsectionnonum{[(4)斯密对问题的第二种解释:生产劳动是物化在商品中的劳动]}

然而,有一点是清楚的:随着资本掌握全部生产,——因而一切商品的生产都是为了出卖,而不是为了直接消费,劳动生产率也相应地增长,——生产劳动者和非生产劳动者之间的物质差别也就愈来愈明显地表现出来,因为前一种人,除极少数以外,将仅仅生产商品,而后一种人,也是除极少数以外,将仅仅从事个人服务。因此,第一种人将生产直接的、物质的、由\textbf{商品}构成的财富,生产一切不是由劳动能力本身构成的商品。这就是促使亚·斯密除了作为基本定义的第一种特征以外,又加上另一些特征的理由之一。

这样,由于斯密的各种不同的想法交织在一起,就有了下面这一段话:

\begin{quote}“家仆的劳动〈与制造业工人的劳动不同〉……\textbf{不能使价值有任何增加}……家仆的生活费\textbf{永远得不到偿还}。一个人,要是雇用许多制造业工人,就会变富;要是维持许多家仆,就会变穷。然而\textbf{后者的劳动}也同前者的劳动一样,\textbf{有它的价值},理应得到报酬。不过,制造业工人的劳动\textbf{固定和物化在一个特定的对象或可以出卖的商品中,而这个对象或商品在劳动结束后,至少还存在若干时候}。可以说,这是在其物化过程中积累并储藏起来,准备必要时在另一场合拿来利用的一定量劳动。这个对象,或者可以说,这个对象的价格,后来到必要时,能够把一个同原先生产它所花费的劳动相等的劳动量推动起来。相反,家仆的[307]劳动不\textbf{固定}或不\textbf{物化在一个特定的对象或可以出卖的商品中。他的服务通常一经提供随即消失,很少留下某种痕迹或某种以后}能够用来取得同量服务的\textbf{价值}。某些最受尊敬的社会阶层的劳动,象家仆的劳动一样,\textbf{不生产任何价值,不固定或不物化在任何耐久的对象或可以出卖的商品中}。”(同上,第 93—94 页)\end{quote}

我们在这里看到,用来说明非生产劳动者的特点的有以下这些定义,这些定义同时显露了亚·斯密内在思想进程的各个环节:

非生产劳动者的劳动“不生产任何价值”,“不能使价值有任何增加”,“〈非生产劳动者的〉生活费永远得不到偿还”,“它不\textbf{固定}或不\textbf{物化在一个特定的对象或可以出卖的商品中}”。相反,“他的服务通常一经提供随即消失,很少留下某种痕迹或某种\textbf{以后}能够用来取得同量服务的价值”。最后,“它不固定或不物化在任何\textbf{耐久的对象或可以出卖的商品中}”。

这里,“生产的”和“非生产的”这些术语是在和原来不同的意义上说的。这里谈的已经不是剩余价值的生产——剩余价值的生产本身就意味着为已消费的价值再生产出一个等价。这里谈的是:一个劳动者,只要他用自己的劳动把他的工资所包含的那样多的价值量加到某种材料上,提供一个等价来代替已消费的价值,他的劳动就是生产劳动。这里就越出了和社会形式有关的那个定义的范围,越出了用劳动者对资本主义生产的关系来给生产劳动者和非生产劳动者下定义的范围。从第四篇第九章(亚·斯密在这里批判了重农学派的学说)可以看出,斯密走入这条歧途,是因为他在阐述自己的见解时一方面反对重农学派,另一方面又受到重农学派的影响。如果工人在一年内只补偿自己工资的等价,那末,他对资本家来说就不是生产劳动者。固然,他会给资本家补偿自己的工资即自己劳动的购买价格。但是这笔交易就好比资本家购买这个劳动者所生产的商品一样。资本家支付了用来生产这个商品的不变资本和工资所包含的劳动。他现在以商品形式占有的劳动和以前以货币形式占有的劳动是同一个量。他的货币没有因此而转化为资本。这种情况,就好比工人本人是自己的生产条件的占有者一样。他每年必须从自己年产品的价值中留出生产条件的价值,以便补偿它们。他一年内消费的,或者说,可以消费的,是他的产品中等于他当年加在自己不变资本上的新劳动的那部分价值。因此,在这种情况下,也就不会有资本主义生产了。

亚·斯密把这种劳动称为“生产的”,第一个理由是因为重农学派把它称为“不生产的”和“不结果实的”。

斯密在这一章里对我们说:

\begin{quote}“第一,[重农学派]承认,这个阶级〈即不从事农业的那些工业阶级〉\textbf{每年再生产出}自己的年消费\textbf{价值,\CJKunderdot{并且至少保持使他们能够就业}和生存的\CJKunderdot{基金或资本}}……诚然,租地农场主和农业工人,除了使他们能够就业和生存的资本以外,每年还再生产出一个\textbf{纯产品},即土地所有者的纯地租……租地农场主和农业工人的劳动,无疑要比商人、手工业者和制造业者的劳动具有更大的生产能力。但是,一个阶级的产品超过另一个阶级的产品,并不能使另一个阶级成为\textbf{不生产}的和\textbf{不结果实的}。”(同上,第 3 卷第 530 页[加尔涅的法译本])\end{quote}

可见,亚·斯密在这里回到重农学派的[308]观点上去了。农业劳动是生产剩余价值,因而也是生产“纯产品”的真正的“生产劳动”。斯密放弃了自己的剩余价值观点,接受了重农学派的观点。同时他又反对重农学派,提出制造业劳动(他认为还有商业劳动)也还是生产的,尽管不是就这个词的最高意义来说的。因此,斯密越出了和社会形式有关的那个定义的范围,越出了从资本主义生产的观点来给“生产劳动者”下定义的范围;他提出这样一个论点来反对重农学派:不从事农业的阶级,工业阶级,会把自己的工资再生产出来,因而还是会把一个等于他的消费的价值生产出来,从而“至少保持使他们能够就业的基金或资本”。这样,在重农学派的影响下,同时在反对重农学派的情况下,便产生了他对“生产劳动”的第二个定义。

\begin{quote}亚·斯密说:“第二,因此,象看待家仆那样来看待手工业者、制造业者和商人,是根本不正确的。\textbf{家仆的劳动不能保持使他能够就业和生存的基金。家仆完全是靠他主人的开支来就业和维持生活的,他所完成的劳动不是那种能补偿这些开支的劳动}。这种劳动是\textbf{服务,通常一经提供随即消失;它不固定和不物化在一个能够补偿他们的生活费和工资的价值的可以出卖的商品中}。相反,手工业者、制造业者和商人的劳动却自然地\textbf{固定和物化在可以出卖或交换的对象中}。正因为如此,我在论\textbf{生产劳动}和\textbf{非生产劳动}那一章中,把手工业者、制造业者和商人算作\textbf{生产的}劳动者,而把家仆算作\textbf{不生产的}和\textbf{不结果实的}劳动者。”(同上,第 531 页)\end{quote}

一旦资本掌握了全部生产,收入只要同劳动交换,它便不是直接同生产\textbf{商品}的劳动交换,而是同单纯的\textbf{服务}交换。收入的一部分同充当使用价值的\textbf{商品}交换,一部分同作为使用价值来消费的\textbf{服务}本身交换。

\textbf{商品}和劳动能力本身不同,它是以物质形式同人对立着的物,它对人有一定的效用,在它身上固定了、物化了一定量的劳动。

这样,我们就得出一个实质上已经包含在第一点中的定义:用自己的劳动\textbf{生产商品}的工人是生产的,并且这个工人消费的商品不多于他生产的东西,不多于他的劳动所值。他的劳动固定和物化在“\textbf{可以出卖或交换的对象中}”,“\textbf{一个能够补偿他们}〈即生产这些商品的工人〉\textbf{的生活费和工资的价值的可以出卖的商品中}”。生产工人生产商品,从而把他以工资形式不断消费的可变资本不断再生产出来。他把支付给他的“使他能够就业和生存的”基金不断生产出来。

\textbf{第一},亚·斯密自然把直接耗费在物质生产中的各类脑力劳动,算作“固定和物化在可以出卖或交换的商品中”的劳动。斯密在这里不仅指直接的手工工人或机器工人的劳动,而且指监工、工程师、经理、伙计等等的劳动,总之,指在一定物质生产领域内为生产某一商品所需要的一切人员的劳动,这些人员的共同劳动(协作)是制造商品所必需的。的确,他们把自己的全部劳动加到不变资本上,并使产品的价值提高这么多。(这在多大的程度上适用于银行家\endnote{关于银行家和他们在资本主义社会中的寄生作用,见马克思《资本论》第 3 卷第 30、32 和 33 章。——第 156 页。}等人呢\fontbox{?})

[309]\textbf{第二},亚·斯密说,非生产劳动者的劳动\textbf{通常}不是这样。亚·斯密非常清楚地知道,即使资本掌握了物质生产,因而家庭工业基本上消失了,直接到消费者家里为他创造使用价值的小手工业者的劳动消失了,——即使在这种情况下,我叫到家里来缝制衬衣的女裁缝,或修理家具的工人,或清扫、收拾房子等等的仆人,或烹调肉食等等的女厨师,他们也完全和在工厂做工的女裁缝、修理机器的机械师、洗刷机器的工人以及作为资本家的雇佣工人在饭店干活的女厨师一样,把自己的劳动固定在某种物上,并且确实使这些物的价值提高了。他们所生产的使用价值,从可能性来讲,也是商品:衬衣可能拿到当铺去当掉,房子可能卖掉,家具可能拍卖等等。因此,上述人员从可能性来讲,也生产了商品,把价值加到了自己的劳动对象上。但他们是非生产劳动者中极少的一部分人,那些适用于他们的说法,对广大家仆、牧师、政府官吏、士兵、音乐家等等则是不适用的。

然而,不管这些“非生产劳动者”人数有多少,有一点无论如何是清楚的(斯密也承认这一点,他说了一句有限制的话:“这些服务\textbf{通常}一经提供随即消失”),那就是:使劳动成为“生产的”或“非生产的”劳动的,既不一定是劳动的这种或那种特殊形式,也不是劳动产品的这种或那种表现形式。同一劳动可以是生产的,只要我作为资本家、作为生产者来购买它,为的是用它来为我增加价值;它也可以是非生产的,只要我作为消费者来购买它,只要我花费收入是为了消费它的(劳动的)使用价值,不管这个使用价值是随着劳动能力本身活动的停止而消失,还是物化、固定在某个物中。

对于一个以资本家身分购买女厨师的劳动的人来说,即对于一个饭店老板来说,女厨师在饭店里是生产商品。羊肉饼的消费者应当对她的劳动付钱,而这个劳动为饭店老板补偿(撇开利润不谈)他用以继续支付女厨师的基金。相反,如果女厨师为我烹调肉食等等,我购买女厨师的劳动,不是为了把这个劳动当作一般劳动来获取剩余价值,而是为了把它当作这种特定的具体劳动来使用;那末,在这种情况下,她的劳动就是非生产的,虽然这种劳动也固定在物质产品中,而且同样可能成为(从结果来看)可以出卖的商品,就象它对饭店老板来说确实是商品一样。可是,这里仍然有重大的差别(实质上的差别):女厨师并不补偿我(私人)用以支付她的基金。因为我购买她的劳动,不是把它作为构成价值的要素,而完全是为了它的使用价值。她的劳动不补偿我用以支付她的基金,即不补偿我给她的工资,这就好比我在饭店里吃的一顿午餐本身,不能使我再购买和吃一顿相同的午餐一样。但这种差别在商品中间也是存在的。资本家为补偿自己的不变资本而购买的商品(例如棉布,假如他有一个棉布印花工厂的话),会以印花布形式为资本家补偿它的价值。相反,如果资本家购买这个商品是为了把印花布本身消费掉,那末,这个商品就不会补偿他的开支。

其实,社会上人数最多的一部分人——工人阶级——都必须为自己进行这种非生产劳动;但是,他们只有先进行了“生产的”劳动,才能从事这种非生产劳动。工人只有生产了可以支付肉价的工资,才能给自己煮肉;他只有生产了家具、房租、靴子的价值,才能把自己的家具和住房收拾干净,把自己的靴子擦干净。因此,从这个生产工人阶级本身来说,他们为自己进行的劳动就是“非生产劳动”。如果他们不先进行生产劳动,这种非生产劳动是决不能使他们[310]重新进行同样的非生产劳动的。

\textbf{第三},另一方面,剧院、歌舞场、妓院等等的老板,购买对演员、音乐家、妓女等等的劳动能力的暂时支配权(事实上通过了曲折的途径,这个途径只有从经济形式的观点来看才有意义,它不影响过程的结果);他们购买这种所谓“非生产劳动”,它的“服务一经提供随即消失”,不固定或不物化在一个“耐久的〈换句话说,“特殊的”〉对象或可以出卖的商品中”(在这些服务本身以外)。把这些服务出卖给公众,就为老板补偿工资并提供利润。他这样买到的这些服务,使他能够重新去购买它们,也就是说,这些服务会自行更新用以支付它们的基金。同样的情况也适用于例如律师事务所的书记的劳动,所不同的只是,书记的服务大部分还体现在十分庞大的“特殊对象”上,即大堆的文件这个形式上。

不错,对老板本身来说,这些服务是由公众的收入支付的。但同样不错的是,一切产品,只要它们用于个人消费,情况也完全是这样。固然,国家不能出口这些服务本身;但它能出口提供这些服务的人。例如,法国出口舞蹈教员、厨师等等,德国出口学校教师。当然,随着舞蹈教员和学校教师的出口,也出口了他们的收入,可是舞鞋和书本的出口,却给国家提供了一笔补偿它们的价值。

因此,从一方面说,所谓非生产劳动有一部分体现在物质的使用价值中,这些使用价值同样可能成为商品(“可以出卖的商品”),从另一方面说,一部分纯粹的服务(它不采取实物的形式,不作为物而离开服务者独立存在,不作为价值组成部分加入某一商品),能够(由\textbf{直接}购买劳动的人)用资本来购买,能够补偿自己的工资并提供利润。总之,这些服务的生产有一部分从属于资本,就象体现在有用物品中的劳动有一部分直接用收入来购买,不从属于资本主义生产一样。

\textbf{第四},整个“商品”世界可以分为两大部分:第一,劳动能力;第二,不同于劳动能力本身的商品。有一些服务是训练、保持劳动能力,使劳动能力改变形态等等的,总之,是使劳动能力具有专门性,或者仅仅使劳动能力保持下去的,例如学校教师的服务(只要他是“产业上必要的”或有用的)、医生的服务(只要他能保护健康,保持一切价值的源泉即劳动能力本身)——购买这些服务,也就是购买提供“可以出卖的商品等等”,即提供劳动能力本身来代替自己的服务,这些服务应加入劳动能力的生产费用或再生产费用。不过,亚·斯密知道,“教育”费在工人群众的生产费用中是微不足道的。在任何情况下,医生的服务都属于生产上的非生产费用\authornote{不直接参加生产过程,但在一定条件下又非有不可的辅助费用。——编者注}。可以把它算入劳动能力的修理费。假定工资和利润由于某种原因同时下降,从总价值来看下降了(例如由于民族变懒),从使用价值来看也下降了(由于歉收等等,劳动的生产能力降低);总之,假定由于上一年新加劳动减少和新加劳动的生产能力降低,产品中价值等于收入的那一部分减少了。这时,如果资本家和工人还想以物质产品的形式消费原先那样大的价值量,他们就要少购买医生、教师等等的服务。如果他们对医生和教师必须继续花费以前那样大的开支,他们就要减少对其他物品的消费。因此,很明显,医生和教师的劳动不直接创造用来支付他们报酬的基金,尽管他们的劳动加入一般说来是创造一切价值的那个基金的生产费用,即加入劳动能力的生产费用。

[311]亚·斯密继续写道:

\begin{quote}“第三,说手工业者、制造业者和商人的劳动不增加社会的\textbf{实际收入},从任何角度来看都是不对的。例如,即使我们象这里考察的理论所假定的那样假定,这个阶级每日、每月、每年消费的价值,恰好等于它当日、当月、当年生产的价值,也决不能由此得出结论说,他们的劳动丝毫没有增加社会的实际收入,没有增加社会的土地和劳动的年产品的实际价值。例如,一个手工业者在收获后 6 个月内完成了价值 10 镑的劳动,即使他在这段时间也消费了价值 10 镑的谷物和其他生存资料,他事实上也已给社会的土地和劳动的年产品增加了 10 镑价值。他把价值 10 镑的半年收入消费在谷物和其他生存资料上,同时又用自己的劳动生产了一个相等的价值,用这个价值可以为他本人或任何别人购买同样多的半年收入。因此,这 6 个月内所消费和生产的价值不等于 10 镑,而等于 20 镑。当然,完全可能,在任何时候现有的这个价值都不超过 10 镑。但是,如果手工业者消费的这价值 10 镑的谷物和其他生存资料,由士兵或家仆来消费,那末,到 6 个月末存在的这部分年产品的价值,就会比由于有手工业者的劳动而实际存在的少 10 镑。可见,即使假定手工业者生产的价值从来没有超过他消费的价值,但在任何时候市场上现有的商品的总价值,都会由于有他的劳动而比没有他的劳动时要大。”(同上,第 4 篇第 9 章第 531—533 页[加尔涅的法译本第 3 卷])\end{quote}

难道任何时候市场上现有的商品的[总]价值,不是由于有“非生产劳动”而比没有这种劳动时要大吗\fontbox{?}难道任何时候市场上除了小麦、肉类等等之外,不是还有妓女、律师、布道、歌舞场、剧院、士兵、政治家等等吗\fontbox{?}这帮人得到谷物和其他生存资料或享乐并不是无代价的。为了得到这些东西,他们把自己的服务提供给或强加给别人,这些服务本身有使用价值,由于它们的生产费用,也有交换价值。任何时候,在消费品中,除了以商品形式存在的消费品以外,还包括一定量的以服务形式存在的消费品。因此,消费品的总额,任何时候都比没有可消费的服务存在时要大。其次,价值也大了,因为它等于维持这些服务的商品的价值和这些服务本身的价值。要知道,在这里就象每次商品和商品相交换一样,是等价物换等价物,因而同一价值具有二重的形式:一次在买者一方,另一次在卖者一方。

\fontbox{~\{}亚·斯密关于重农学派继续写道:

\begin{quote}“当这一体系的拥护者断言,手工业者、制造业者和商人的\textbf{消费等于他们所生产的东西的价值}时,他们大概仅仅是指这一情况:这些劳动者的\textbf{收入},或者说,\textbf{供他们消费的基金,等于这个价值}〈即他们所生产的东西的价值〉。”(同上,第 533 页)\end{quote}

如果把工人和企业主放在一起来看,重农学派在这一点上是对的;企业主的利润包括地租,地租只是企业主利润的一个特殊项目。\fontbox{\}~}

[312]\fontbox{~\{}\textbf{亚·斯密}在同一个场合,即在批判重农学派的场合——第四篇第九章[加尔涅的译本第 3 卷]——指出:

\begin{quote}“一个社会的土地和劳动的年产品,只能用两种办法增加:\textbf{第一,改善}当时在这个社会发生作用的\textbf{有用劳动的生产能力};或者\textbf{第二,增加这种劳动的量}。要使有用劳动的生产能力有所改善或增长,就必需\textbf{改进工人的技能或改进他用来劳动的机器}……当时在社会上使用的\textbf{有用劳动的量的增加},完全取决于\textbf{把这种劳动推动起来的资本的增加,而这种资本的增加,又必定恰好等于}管理这一资本的人或把资本借给他们的另一些人从自己的收入中\textbf{节约下来的数额}。”(第 534—535 页)\end{quote}

这里是双重的循环论证。\textbf{第一},年产品的增加是由于劳动生产率的提高。而提高劳动生产率的一切手段(只要这种提高不是由自然的偶然情况,如特别有利的天气等等引起的)都要求增加资本。但是,要增加资本,又必需增加劳动的年产品。这是第一个循环论证。\textbf{第二},年产品可以通过增加所使用的劳动量来增加。但是,只有先增加“把这种劳动推动起来”的资本,才能增加所使用的劳动量。这是第二个循环论证。斯密试图靠“\textbf{节约}”来摆脱这两个循环论证。节约一词,他指的是收入转化为资本。

把全部利润看成资本家的“收入”,这种看法本身就是错误的。相反,资本主义生产的规律要求把工人完成的剩余劳动即无酬劳动的一部分转化为资本。当单个资本家作为资本家即作为资本职能的执行者行动的时候,利润转化为资本,当然,在他本人看来可能象是一种节约,但即使对他本人来说,这种转化也是以必需有准备金的形式表现出来的。然而劳动量的增加不仅取决于工人人数,而且取决于工作日的长度。因而,即使转化为工资的那部分资本不增加,劳动的量也可能增加。在这种情况下也不需增加机器等等的数量(虽然机器磨损得快一些,但并不会使这里的问题有所改变)。唯一必需增加的,是用作种子等等的那部分原料。同时,这一点仍然是对的:在一个国家里(如果把对外贸易撇开不谈),剩余劳动首先必须在农业中出现,然后才有可能在从农业取得原料的那些工业部门中出现。一部分原料——煤、铁、木材、鱼(例如,作为肥料)等等,一切非动物性的肥料,可以用单纯增加劳动(工人的人数不变)的办法取得。因此,这些原料是不会缺乏的。另一方面,前面已经指出,生产率的提高最初总是只以资本的积聚为前提,而不是以资本的积累为前提。\endnote{关于资本的积聚是劳动生产率提高的最初条件,马克思是在他的 1861—1863 年手稿第 IV 本第 171—172 页(《相对剩余价值》一节,《分工》一小节中)谈到的。——第 162 页。}但以后这两个过程是相互补充的。\fontbox{\}~}

\fontbox{~\{}斯密在下面一段话里正确地指出了促使重农学派宣传自由放任\endnote{自由放任(原文是:laissezfaire,laissezaller,亦译听之任之)是重农学派的口号。重农学派认为,经济生活是受自然规律调节的,国家不得对经济事务进行干涉和监督;国家用各种规章进行干涉,不仅无益,而且有害;他们要求实行自由主义的经济政策。——第 27、42、162 页。},即自由竞争的原因:

\begin{quote}“两个不同的居民集团〈城市和乡村〉之间的贸易,归根到底,是一定量的原产品同一定量的制造业产品交换。因此,后者愈贵,前者愈贱,凡是在一个国家里能提高制造业产品价格的东西,都会降低土地的原产品的价格,从而使农业发展缓慢。”但是,加在制造业和对外贸易上的一切约束和限制,都会使制造业产品等等变贵。因此,等等。(\textbf{斯密},同上[加尔涅的法译本第 3 卷],第 554—556 页)\fontbox{\}~}\end{quote}

\centerbox{※     ※     ※}

[313]这样,斯密对“生产劳动”和“非生产劳动”的第二种见解(更确切地说,同上述他的另一种见解交错在一起的见解)可归结如下:生产劳动就是生产\textbf{商品}的劳动,非生产劳动就是不生产“任何商品”的劳动。斯密不否认,这两种劳动\textbf{都是商品}。请看前面讲的\authornote{见本册第 152 页。——编者注}:“后者的劳动也同前者的劳动一样,有它的价值,理应得到报酬”(就是说,从经济学来看;无论对这种劳动还是那种劳动,都谈不上从道德等等观点来看)。商品的概念本身包含着劳动体现、物化和实现在自己的产品中的意思。劳动本身,在它的直接存在上,在它的活生生的存在上,不能直接看作商品,只有劳动能力才能看作商品,劳动本身是劳动能力的暂时表现。只有这种观点才能使我们既弄清楚真正的雇佣劳动的概念,又弄清楚“非生产劳动”的概念,而亚·斯密到处都用生产“非生产劳动者”所必需的生产费用来给非生产劳动下定义。由此可见,\textbf{商品}必须看作一种和劳动本身不同的存在。这样,商品世界就分为两大类:

一方面是劳动能力。

另一方面是商品本身。

但是,对劳动的物化等等,不应当象亚·斯密那样按苏格兰方式去理解。如果我们从商品的交换价值来看,说商品是劳动的化身,那仅仅是指商品的一个想象的即纯粹社会的存在形式,这种存在形式和商品的物体实在性毫无关系;商品代表一定量的社会劳动或货币。使商品产生出来的那种具体劳动,在商品上可能不留任何痕迹。从制造业商品来说,这个痕迹保留在原料所取得的外形上。而在农业等等部门,例如小麦、公牛等等商品所取得的形式,虽然也是人类劳动的产品,而且是一代一代传下来、一代一代补充的劳动的产品,但这一点在产品上是看不出来的。还有这样的产业劳动部门,在那里,劳动的目的决不是改变物的形式,而仅仅是改变物的位置。例如,把商品从中国运到英国等等,在物本身谁也看不出运输时花费的劳动所留下的痕迹(除非有人想起这种东西不是英国货)。因此,决不能象上面所说的那样去理解劳动在商品中的物化。(这里所以产生迷误,是因为社会关系表现为物的形式。)

虽然如此,商品表现为过去的、物化的劳动这个说法还是对的,因而,如果它不表现为物的形式,它就只能表现为劳动能力本身的形式,但永远不能直接表现为活劳动本身(只有通过某种曲折的途径,才能表现为活劳动本身,这种途径在实践上似乎是无关紧要的,但在确定各种不同的工资的时候,则不然)。由此可见,斯密本应承认,生产劳动或者是生产商品的劳动,或者是直接把劳动能力本身生产、训练、发展、维持、再生产出来的劳动。亚·斯密把后一种劳动从他的生产劳动项目中除去了;他是任意这样做的,但他受某种正确的本能支配,意识到,如果他在这里把后一种劳动包括进去,那他就为各种冒充生产劳动的谬论敞开了大门。

因此,如果我们把劳动能力本身撇开不谈,生产劳动就可以归结为生产商品、生产物质产品的劳动,而商品、物质产品的生产,要花费一定量的劳动或劳动时间。一切艺术和科学的产品,书籍、绘画、雕塑等等,只要它们表现为物,就都包括在这些物质产品中。但是,其次,劳动产品必须是这种意义上的\textbf{商品}:它是“可以出卖的商品”,也就是还需要通过形态变化的第一种形式的商品。(假定一个工厂主买不到一部现成的机器,他可以自己制造一部机器,不是为了出卖,而是为了把它当作使用价值来利用。但是,在这种情况下,他把机器当作自己的不变资本的一部分来使用,因而他是通过由机器协助生产出来的产品的形式一部分一部分地把机器出卖的。)

[314]可见,虽然家仆的某些劳动完全可能表现为\textbf{商品(从可能性来讲)},从物质方面来看,甚至可能表现为同样的使用价值,但这不是生产劳动,因为实际上他们生产的不是“商品”,而是直接“\textbf{使用价值}”。而有些劳动,对它们的买者或雇主来说是生产的,例如演员的劳动对剧院老板来说是生产的,但这些劳动看起来象是非生产劳动,因为它们的买者不能以商品的形式,而只能以活动本身的形式把它们卖给观众。

如果把这一点撇开不谈,那末[按照斯密的第二个定义],生产劳动就是生产\textbf{商品}的劳动,\textbf{非生产劳动}就是生产个人服务的劳动。前一种劳动表现为某种可以出卖的物品;后一种劳动在它进行的时候就要被消费掉。前一种劳动(创造劳动能力本身的劳动除外)包括一切以物的形式存在的物质财富和精神财富,既包括肉,也包括书籍;后一种劳动包括一切满足个人某种想象的或实际的需要的劳动,甚至违背个人意志而强加给个人的劳动。

商品是资产阶级财富的最基本的元素形式。因此,把“生产劳动”解释为生产“商品”的劳动,比起把生产劳动解释为生产资本的劳动来,符合更基本得多的观点。

亚·斯密的反对者无视他的第一种解释即符合问题本质的解释,而抓住第二种解释,并强调这里不可避免的矛盾和不一贯的地方。而且他们把注意力集中在劳动的物质内容,特别是集中在劳动必须固定在一个比较\textbf{耐久的}产品中那个定义,用这个办法为自己的论战制造方便。我们马上就会看到,这场特别激烈的论战,究竟是由什么引起的。

还要先指出一点。亚·斯密认为,提出下面这个论点,是重农主义体系的巨大功绩:

\begin{quote}“各国的财富不在于不可消费的金和银,而在于每年由社会劳动再生产出来的可消费的货物。”([加尔涅的法译本]第 3 卷第 4 篇第 9 章第 538 页)\end{quote}

这里,我们看到了斯密关于生产劳动的第二个定义的来源。如何给剩余价值下定义,自然取决于所理解的价值本身具有什么形式。因此,剩余价值在货币主义和重商主义体系中,表现为货币;在重农学派那里,表现为土地的产品,农产品;最后,在亚·斯密那里,表现为一般\textbf{商品}。重农学派只要接触到价值实体,就把价值仅仅归结为使用价值(物质、实物),正如重商学派把价值仅仅归结为价值形式,归结为产品借以\textbf{表现}为一般社会劳动的那种形式即货币一样。在亚·斯密那里,商品的两个条件,使用价值和交换价值,合并在一起,所以在他看来,凡是表现在一种使用价值即有用产品中的劳动,都是生产的。表现在有用产品中的劳动就是生产劳动这一观点,就已经包含着这样的意思:这个产品同时等于一定量的一般社会劳动。亚·斯密同重农学派相反,重新提出产品的价值是构成资产阶级财富的实质的东西;但是另一方面,又使价值摆脱了纯粹幻想的形式——金银的形式,即在重商学派看来价值借以表现的形式。任何商品\textbf{从可能性来说}就是货币。不可否认,亚·斯密在这里同时又或多或少地回到重商学派关于这些或那些劳动产品的“耐久性”(实际上是“非直接消费性”)的观点上去。这里使人想起配第的一段话(见我的第 1 分册第 109 页\endnote{马克思指《政治经济学批判》第一分册。马克思提到的配第著作的引文,见《马克思恩格斯全集》中文版第 13 卷第 119 页。——第 167 页。},那里引用了配第《政治算术》中的一段话),在这段话里,财富是按照它不会毁坏的程度、耐久的程度来估价的,归根结蒂,金银被当作“长久的财富”而放在高于一切的地位。

\begin{quote}阿·布朗基说:“斯密把\textbf{财富}的范围仅仅限于固定在物质实体中的那些价值,这样就把无限多的非物质价值,文明国家的\textbf{精神资本}之女,全都从生产的账本中勾销了”,等等。(《欧洲政治经济学从古代到现代的历史》1839 年布鲁塞尔版第 152 页)\end{quote}

\tsectionnonum{[(5)资产阶级政治经济学在生产劳动问题上的庸俗化过程]}

反对亚·斯密提出的关于生产劳动和非生产劳动的区分的论战,主要是由二流人物(其中施托尔希还算是最出名的人物)进行的;我们在任何一个重要的经济学家那里,[315]在任何一个可以说在政治经济学上有所发现的人那里,都没有看到这种论战;然而这种论战对于第二流人物,特别是对于充满学究气的编书家和纲要编写者,以及对于在这方面舞文弄墨的业余爱好者和庸俗化者来说,却是一种嗜好。反对亚·斯密的这场论战,主要是由以下几种情况引起的。

有一大批所谓“高级”劳动者,如国家官吏、军人、艺术家、医生、牧师、法官、律师等等,他们的劳动有一部分不仅不是生产的,而且实质上是破坏性的,但他们善于依靠出卖自己的“非物质”商品或把这些商品强加于人,而占有很大部分的“物质”财富。对于这一批人来说,在\textbf{经济学上}被列入丑角、家仆一类,被说成靠真正的生产者(更确切地说,靠生产当事人)养活的食客、寄生者,决不是一件愉快的事。这对于那些向来显出灵光、备受膜拜的职务,恰恰是一种非同寻常的亵渎。政治经济学在其古典时期,就象资产阶级本身在其发家时期一样,曾以严格的批判态度对待国家机器等等。后来它理解到——这在它的实践中也表现出来——并且根据经验认识到,这种继承下来的所有这些在某种程度上完全非生产的阶级的社会结合的必要性,就是由资产阶级自己的组织中产生出来的。

如果上述“非生产劳动者”不生产享受,因此对他们的服务的需求不完全取决于生产当事人想如何花掉自己的工资或利润;相反,如果他们成为必要,或自己使自己成为必要,部分地是因为存在肉体上的疾病(如医生)或精神上的虚弱(如牧师),部分地是因为个人利益的冲突和民族利益的冲突(如政治家、一切法学家、警察、士兵);如果这样,那末,在亚·斯密看来,就象在产业资本家本身和工人阶级看来一样,他们就表现为生产上的非生产费用,因此必须尽可能地把这种非生产费用缩减到最低限度,尽可能地使它便宜。资产阶级社会把它曾经反对过的一切具有封建形式或专制形式的东西,以它自己所特有的形式再生产出来。因此,对这个社会阿谀奉承的人,尤其是对这个社会的上层阶级阿谀奉承的人,他们的首要业务就是,在理论上甚至为这些“非生产劳动者”中纯粹寄生的部分恢复地位,或者为其中不可缺少的部分的过分要求提供根据。事实上这就宣告了意识形态阶级等等是\textbf{依附于资本家}的。

但是,\textbf{第二},有一部分生产当事人(物质生产本身的当事人),时而被这一些经济学家,时而被那一些经济学家称为“非生产的”。例如,代表工业资本利益的那部分经济学家(李嘉图)把土地所有者称为“非生产的”。另一些经济学家(例如凯里)把本来意义的商人称为“非生产的”劳动者。后来甚至又有一些人把“资本家”本人也称为非生产的,或者至少企图把资本家对物质财富的要求归结为“工资”,即归结为一个“生产劳动者”所取得的报酬。脑力劳动者中间的许多人,看来都倾向于对资本家的生产性持这种怀疑观点。因此,已经是作出妥协并且承认不直接包括在物质生产当事人范围内的一切阶级都具有“生产性”的时候了。大家互相帮忙,并且,象《蜜蜂的寓言》\endnote{指英国作家孟德维尔的讽刺作品《蜜蜂的寓言,或个人劣行即公共利益》。该书于 1705 年出第一版,1728 年出第五版。——第 169 页。}中那样,必须证明,即使根据“生产的”、经济学的观点,资产阶级世界连同它的所有“非生产劳动者”一起,也是所有世界中最美好的世界;何况一些“非生产劳动者”从自己方面已经对那些根本是“为享受果实而生的”\authornote{贺雷西《书信集》。——编者注}阶级的生产性,或者对那些如土地所有者那样无所事事的生产当事人等等作出了批判的考察。\textbf{无所事事的人}也好,他们的\textbf{寄生者}也好,都必须在这个最美好的世界中找到自己的地位。

\textbf{第三},随着资本的统治的发展,随着那些和创造物质财富没有直接关系的生产领域实际上也日益依附于资本,——尤其是在实证科学(自然科学)被用来为物质生产服务的时候,——[316]政治经济学上的阿谀奉承的侍臣们便认为,对任何一个活动领域都必须加以推崇并给以辩护,说它是同物质财富的生产“联系着”的,说它是生产物质财富的手段;他们对每一个人都表示敬意,说他是“第一种”意义的“生产劳动者”,即为资本服务的、在这一或那一方面对资本家发财致富有用的劳动者,等等。

这里,应当首先提出的是马尔萨斯之流,他们直接为“\textbf{非生产}劳动者”和明显的寄生者辩护,说这些人是必要的和有用的。

\tsectionnonum{[(6)斯密关于生产劳动问题的见解的拥护者。有关这个问题的历史]}

\centerbox{\textbf{[(a)第一种解释的拥护者:李嘉图、西斯蒙第]}}

不值得花时间来详细考察热·加尔涅(斯密著作的译者)、罗德戴尔伯爵、布鲁姆、萨伊、施托尔希以及后来的西尼耳、罗西等人关于这一点的庸俗见解。只要引用一些典型的话就够了。

我们还要先举出\textbf{李嘉图}的一段话,他在其中证明,剩余价值(利润,地租)的所有者把剩余价值消费在“非生产劳动者”(例如家仆)身上,比他们把剩余价值花在“生产工人”所创造的奢侈品上,对于“生产工人”要有益得多。

\fontbox{~\{}\textbf{西斯蒙第}在《政治经济学新原理》(第 1 卷第 148 页)中,接受了斯密进行区分时的正确解释(这在李嘉图的著作中也是不言而喻的):生产阶级和非生产阶级的实际区别在于,

\begin{quote}“前者总是以自己的劳动同国民资本交换,后者总是以自己的劳动同一部分国民收入交换”。\end{quote}

\textbf{西斯蒙第}也是按照亚·斯密的见解来看剩余价值的:

\begin{quote}“虽然工人通过自己每天的劳动所生产的东西,远远超过他每天的支出,但是在他同土地所有者和资本家进行分配以后,除了维持生活最必需的东西以外,很少有剩余。”(\textbf{西斯蒙第}《政治经济学新原理》第 1 卷第 87 页)\fontbox{\}~}\end{quote}

李嘉图说:

\begin{quote}“如果土地所有者或资本家象古代贵族那样,把自己的收入用来供养很多的侍从或家仆,而不把它花费在华丽的衣服或昂贵的家具、马车、马或其他奢侈品上,那末他雇用的劳动人数就会多得多。在这两种情况下,纯收入是相同的,总收入也是相同的,但是纯收入实现在不同的商品上。如果我的收入是 1 万镑,那末,无论这 1 万镑是实现在华丽的衣服、昂贵的家具等等上,还是实现在同一价值的一定量食物和一般衣着上,所使用的生产劳动的数量差不多相等。但是,如果我把收入实现在前一类商品上,那\textbf{以后}就不会有对劳动的新的需求了:我将享用我的家具和衣服,事情就到这里为止。相反,如果我把收入实现在食物和一般衣着上,而且希望雇用家仆,那末,\textbf{除了原有对工人的需求之外,还会加上}对我用 1 万镑收入(或以这笔收入购买到的食物和一般衣着)所能雇用的所有那些人的需求。而需求的这种增加,只是因为我选择了第二种花费我的收入的方式。工人都关心\textbf{对劳动的需求},所以他们当然希望把用在购买奢侈品方面的收入尽量转用来维持家仆。”(\textbf{李嘉图}《原理》1821 年第 3 版第 475—476 页)\end{quote}

\centerbox{\textbf{[(b)区分生产劳动和非生产劳动的最初尝试(戴韦南特、配第)]}}

\textbf{戴韦南特}引用了一位老统计学家格雷哥里·金的一个图表,题为《1688 年英格兰不同家庭的收支表》。大学者金在表中把全体人民分成两个主要阶级:一个是“\textbf{增加}王国财富”的阶级,共计 2675520 人,一个是“\textbf{减少}王国财富”的阶级,共计 2825000 人;因此,前一个阶级是“生产的”,后一个阶级是“非生产的”。“\textbf{生产的}”阶级包括:勋爵、从男爵、骑士、乡绅、贵族、大小官吏、从事海上贸易的商人、法律家、教士、土地所有者、租地农场主、自由职业者、大小商人、手工业者、海陆军军官。相反,“\textbf{非生产的}”阶级包括:水手、农业工人和制造业短工、农民(在戴韦南特时代还占英格兰全部人口的 1/5)、[317]士兵、赤贫者、茨冈人、盗贼、乞丐和一般流浪者。戴韦南特这样来解释大学者金的这个等级表:

\begin{quote}“他的意思是说,前一个阶级的人靠土地、手艺和勤劳来养活自己,并且每年都给国民资本增加一些东西,此外,每年还从自己的剩余中分出一定数额来养活别人。在后一个阶级中,有一部分人靠自己的劳动养活自己,而其余的人和他们的妻子儿女,都要靠别人来养活;这是社会的负担,因为不然的话,他们每年消费的东西就可加到国民总资本中去。”(\textbf{戴韦南特}《论使一国人民在贸易差额中成为得利者的可能的方法》1699 年伦敦版第 23 和 50 页)\end{quote}

此外,戴韦南特的下面这段话,最能说明重商学派对剩余价值的看法的特点:

\begin{quote}“出口我们本国的产品,必定会使英国富裕;为了有贸易顺差,我们必须出口本国的产品,用它们去购买本国消费所必需的外国出产的物品,这里我们会有一个\textbf{余额},它或者采取贵金属的形式,或者采取我们可以用来卖给其他国家的商品的形式;\textbf{这个余额}就是\textbf{一国从贸易中取得的利润}。它的大小决定于出口国人民的自然节约〈荷兰人而不是英国人所特有的那种节约——同上,第 46 和 47 页〉,还决定于他们的劳动和制造业产品的低廉价格,这种低廉价格,使他们能\textbf{在国外市场上比所有的竞争者都便宜地出售这些产品}。”(\textbf{戴韦南特},同上第 45—46 页)\fontbox{~\{}“在国内消费产品时,一个人的赢利不过是另一个人的亏损,整个国家丝毫不会变富;但在国外消费的一切东西,却是明显的和可靠的利润。”(《论东印度贸易》1697 年伦敦版[第 31 页])\fontbox{\}~}\end{quote}

\fontbox{~\{}这本书是以戴韦南特的另一著作\endnote{指戴韦南特匿名出版的著作《论公共收入和英国贸易》1698 年伦敦版第二部分,其中载有戴韦南特一年前写的著作《论东印度贸易》。正文中引用的这段话的译文,同马克思在他的札记本中关于戴韦南特所说的话是一致的,马克思在正文中所用的戴韦南特著作的全部引文,都取自札记本(这一札记本的封面上马克思注明:“曼彻斯特。1845 年 7 月”)。——第 172 页。}(它是为了给这本书辩护而写的)的\textbf{附录形式刊印的},并不是麦克库洛赫引用过的那本《论东印度贸易》(1701 年版)。\fontbox{\}~}

可是,不应当象后来的庸俗自由贸易论者那样,把这些重商主义者说得那么愚蠢。戴韦南特在他的《论公共收入和英国贸易》第二卷(1698 年伦敦版)中曾说:

\begin{quote}“金和银实际上是贸易的尺度,但各国人民贸易的源泉和起源,却是一国自然的产物或人工的产物,即一国的土地或该国人民的劳动和勤勉所生产的东西。的确,一个民族由于某种情况可能完全丧失各种货币,但是只要它人口众多,热爱劳动,精于贸易,擅长航海,有良好的港湾,有生产各种产品的土地,它就仍然能够进行贸易,并且在短时间内拥有大量金银。所以,一国真正的实际的财富是它本国的产物。”(第 15 页)“金和银远不是能够称为一国的财宝或财富的唯一物品,因而货币实际上不过是人们在交易上习惯使用的计算筹码。”(第 16 页)“我们所说的财富,是指能使君主及其人民富裕、幸福、安全的东西;同样,财宝是指为了人们的需要用金银换来转化成建筑物和土壤改良的东西;还指\textbf{可以换成}这些金属的其他物品,如土地的果实和工业的产物,或外国的商品和商船……甚至那些不耐久的物品也能看成是国家的财富,只要它们\textbf{能够换成}金银——哪怕它们还\textbf{没有进行交换};并且我们认为,它们不仅在个人和个人之间的关系上是财富,而且在一国和别国之间的关系上也是财富。”(第 60—61 页)“平民\authornote{“平民”一词在这里是指革命前的法国称为“第三等级”的人,即同僧侣和贵族相对立的所有居民。——编者注}是国家身体中的胃。在西班牙,这个胃没有恰当地消受货币,[318]没有消化货币……工商业是能够保障消化和分配金银的唯一手段,而这将供给国家身体以必要的营养物。”(第 62—63 页)\end{quote}

其实,配第也已经有了\textbf{生产劳动者}的概念(不过他把士兵也包括在内):

\begin{quote}“土地耕种者、海员、士兵、手工业者和商人,是任何一个社会的真正的支柱。所有其他的大职业\textbf{都是由于这些人的孱弱和过失而产生的};海员身兼上述四者中的三者〈航海者、商人和士兵〉。”(《政治算术》1699 年伦敦版第 177 页)“海员的劳动和船只的运费,按其性质来说,始终是一种出口商品,出口超过进口的\textbf{余额}就给本国带回货币等等。”(同上,第 179 页)\end{quote}

在这一点上,配第又证明分工的好处:

\begin{quote}“在海上贸易中占支配地位的人们,即使在运费较低廉的情况下,也能比别人在较高〈运费较贵〉的情况下获得更多的利润;这是因为,就象做衣服一样,如果一个人完成一道工序,另一个人完成另一道工序,等等,衣服的价钱就比较便宜,在海上贸易中占支配地位的人们也是这样,他们可以建造各种不同用途的船只:海船、江船、商船、战船等等,这是荷兰人所以能够以低于他们邻国人的价格来运货的一个主要原因,因为他们能够为每个特定贸易部门提供特定种类的船只。”(同上,第 179—180 页)\end{quote}

此外,从配第的下面这些话里可以听到完全是斯密的调子:

\begin{quote}“如果向工业家等人收税,以便把货币供给那些按其职业来说一般\textbf{不}生产\textbf{物质品}即对\textbf{社会}有实际效用和价值的\textbf{物品}的人们,那末社会的财富就会减少。至于使精神得到消遣和恢复的活动,则又当别论,这些活动只要利用得当,就会使人能够并愿意去做本身具有更重要意义的事情。”(同上,第 198 页)“当计算好需要多少人从事生产劳动之后,剩下来的人就可以安全地、对社会无害地被用来从事娱乐和装饰方面的技艺和工作,\textbf{而其中最重大的,就是增进自然知识}。”(同上,第 199 页)“工业的收益比农业多,而商业的收益又比工业多。”(第 172 页)“一个海员相当于三个农民。”(第 178 页)\end{quote}

\centerbox{※     ※     ※}

[VIII—346]\textbf{配第。剩余价值}。从配第著作的一段话中,可以看到对\textbf{剩余价值}的性质的猜测,尽管他只是从地租的形式来考察剩余价值的。尤其是把这段话同下面几段话作一对比,就更清楚了。在下面几段话里,他用花费同样多劳动时间生产的银和谷物的相对量,来决定银和谷物的相对价值:

\begin{quote}“假定有人从秘鲁地下获得 1 盎斯银并带到伦敦来,他所用的时间和他生产 1 蒲式耳谷物所需要的时间相等,那末,前者就是后者的自然价格;假定现在由于开采更富的新矿,获得 2 盎斯银象以前获得 1 盎斯银花费一样多,那末在其他条件相同的情况下,现在 1 蒲式耳谷物值 10 先令的价格,就和它以前值 5 先令的价格一样便宜。”“假定让 100 个人在 10 年内生产谷物,又让同样数目的人在同一时间内开采银;我认为,银的纯产量将是谷物全部纯收获量的价格,前者的同样部分就是后者的同样部分的价格。”“100 个土地耕种者所能做的工作,如果由 200 个土地耕种者来做,谷物就会贵 1 倍。”(《赋税论》1662 年版)(1679 年版第 32、24、67 页)\end{quote}

我上面指的[关于剩余价值的性质的]一段话是这样说的:

\begin{quote}“如果商业和工艺发展了,那末,农业将要衰落,或者土地耕种者工资必将提高,\textbf{因而}地租就会下降……如果英格兰的商业和工业发展了,也就是说,如果从事工商业的人口比过去多了,而且现在谷物的价格不比以前有较多的人从事农业而较少的人从事工商业的时候高,那末仅仅由于这一个原因……地租就必定会下跌。例如,假定 1 蒲式耳小麦的价格为 5 先令或 60 便士;如果生长小麦的土地的地租为三分之一捆〈即收成的三分之一〉,那末在 60 便士中,就要有 20 便士归土地,40 便士归土地耕种者;但是,如果后者的工资提高 1/8,也就是从每天 8 便士提高到 9 便士,那末,在 1 蒲式耳小麦中,土地耕种者分得的份额就会由 40 便士增加到 45 便士,结果地租就要由 20 便士下降到 15 便士,因为我们假定\textbf{小麦价格仍然不变};更何况\textbf{我们不能够把小麦价格提高},因为如果我们试图把小麦价格提高,谷物就会从农业状况没有发生变化的外国输入我国[347](就象输入荷兰那样)。”(《政治算术》1699 年伦敦版第 193—194 页)[VIII—347]\end{quote}

\centerbox{※     ※     ※}

[VIII—364]\fontbox{~\{}\textbf{配第}。应当把上面引证的配第的一段话同下面这段话对比一下,在下面这段话里,地租表现为一般剩余价值,即表现为“纯产品”:

\begin{quote}“假定一个人用自己的双手在一块土地上种植谷物,耕地、播种、耙地、收割、搬运、脱粒,总之,干了农业上所需要的一切。我认为,这个人从他的收成中扣除自己的种子,并扣除自己食用的部分以及为换取衣服和其他必需品而给别人的部分之后,剩下的谷物就是当年真正的地租;而 7 年的\textbf{平均数},或者更确切地说,形成歉收和丰收循环周期的若干年的平均数,就是种植谷物的这块土地的通常的地租。但是,这里可能发生一个尽管是附带的、但需要进一步解决的问题:这种谷物或这种地租值多少货币呢\fontbox{?}我的回答是,值多少货币,要看另一个把\textbf{自己的全部时间}花在下述活动的人手中剩下多少货币:前往产银地区,在那里开采这种金属,把它提炼、铸成硬币,并把它运到第一个人播种和收获自己谷物的地方来。第二个人扣除他的全部费用之后手中剩下的货币量,将同土地耕种者手中剩下的那些谷物在价值上完全相等。”(《赋税论》\endnote{威廉·配第《赋税论》中的这段话,马克思在这里引自沙尔·加尼耳《论政治经济学的各种体系》一书第二卷第 36—37 页(1821 年巴黎版),这段话在这本书中已由加尼耳译成法文。这一段的法译文同马克思在手稿第 XXII 本中引用的英文原文有些不同(见本册第 381—382 页)。——第 176 页。}第 23 页)\fontbox{\}~}[VIII—364]\end{quote}

\centerbox{\textbf{[(c)斯密对生产劳动的第二种解释的拥护者——约翰·斯图亚特·穆勒]}}

[VII—318]\textbf{约翰·斯图亚特·穆勒}先生在《略论政治经济学的某些有待解决的问题》(1844 年伦敦版)一书中,也苦心研究生产劳动和非生产劳动的问题;但事实上他除了断言把劳动能力本身生产出来的那种劳动也是生产的以外,对斯密的(第二种)解释没有增添什么东西。

\begin{quote}“\textbf{享受的源泉}可以积累和积蓄,享受本身却不能这样。一国的财富由该国拥有的物质的或非物质的耐久的享受源泉的总和构成。用来增加或保存这些耐久的源泉的劳动或开支,都应称为\textbf{生产的}。”(同上,第 82 页)“机械师或纺纱者在学习手艺时所消费的东西,是用于生产消费,换句话说,他们消费的目的,不是减少而是增加国内耐久的享受源泉,因为他们新创造的享受源泉,在数量上超过消费掉的数额。”(同上,第 83 页)\end{quote}

\centerbox{※     ※     ※}

现在,我们简略地考察一下反对亚·斯密的那些关于生产劳动和非生产劳动的胡说八道。

\tsectionnonum{[(7)]热尔门·加尔涅[把斯密和重农学派的理论庸俗化]}

[319]在热尔门·加尔涅翻译的斯密《国富论》第五卷(1802 年巴黎版)中,载有译者的注释:

关于最高意义上的“生产劳动”问题,加尔涅是同意重农学派的观点的,只不过把这种观点略为缓和了。他反对斯密下面的观点:

\begin{quote}“生产劳动是这样的劳动,它物化在某种对象中,把自己活动的痕迹留下来,它的产品能够出卖或交换。”(同上,第 5 卷第 169 页)\endnote{手稿在加尔涅的这段引文之后,是篇幅很长的关于约翰·斯图亚特·穆勒的插入部分(手稿第 319—345 页)、一段不长的关于马尔萨斯的评论(第 345—346 页)和篇幅不大的关于配第的补充部分(第 346—347 页)。关于约·斯·穆勒的插入部分开头这样说:“在分析加尔涅的观点之前,我们要在这里附带地[即以补充部分的形式]就前面引证过的小穆勒说几句话。这里我们要说的话本来应放到后面论李嘉图剩余价值理论的地方去谈,而不在这里谈,这里我们还是考察亚当·斯密。”在手稿第 XIV 本目录中(见本册第 5 页)以及在这个稿本的正文中,论约·斯·穆勒一节是在《李嘉图学派的解体》一章内。根据所有这些理由,本版将关于约·斯·穆勒的补充部分移至《剩余价值理论》第三册《李嘉图学派的解体》一章。关于马尔萨斯的评论移至论马尔萨斯一章,关于配第的补充部分放在前面第 174—176 页。在所有这些插入部分之后,手稿上(第 VIII 本第 347 页)写道:“现在我们回过头来谈生产劳动和非生产劳动的问题。加尔涅。见手稿第 VII 本第 319 页。”接着便是对加尔涅观点的分析,现刊印在第 176—199 页上。——第 177 页。}[VII—319]\end{quote}

\centerbox{\textbf{[(a)把同资本交换的劳动和同收入交换的劳动混淆起来。关于全部资本由消费者的收入补偿的错误见解]}}

[VIII—347]加尔涅提出了反对亚·斯密的各种理由(其中一部分为后来的著作家们一再重复)。

\textbf{第一},

\begin{quote}“这种区分是错误的,因为它所根据的是不存在的差别。从作者所理解的\textbf{生产劳动}来看,\textbf{任何一种劳动都是生产劳动}。从劳动对支付其代价的人提供某种享受、某种方便或某种效用来看,一种劳动和另一种劳动一样,都是生产劳动。否则任何劳动都不会有报酬了。”\end{quote}

\fontbox{~\{}可见,劳动所以是生产的,是因为它生产某种使用价值,它可以出卖,它具有交换价值,也就是说,它本身就是商品。\fontbox{\}~}

但是,在发挥这个意思时,加尔涅举了许多例子来说明,在这些例子中,“非生产劳动者”和“生产劳动者”一样,做\textbf{同样的事情},生产同样的或同类的使用价值。例如:

\begin{quote}“服侍我的仆役,为我生火,为我卷发,为我洗衣服和整理家具,为我做饭等等。这个仆役和下面这些人提供的是\textbf{完全同类的服务}:如洗衣女工或缝纫女工为顾客洗濯和修补衬衣……如小饭馆主人或小酒馆主人为愿意光临的顾客烹调食物;如理发师、美容师等等\end{quote}

(但在亚·斯密看来,这种人大部分象家仆一样,都不属于生产劳动者的范畴)

\begin{quote}提供直接的服务;最后,如泥瓦匠、屋顶匠、木匠、玻璃匠、火炉匠等等,以及被人请去修缮房屋的大量建筑工人,后者从简单的修缮劳动中获得的年收入,同从新建房屋的劳动中获得的一样多。”\end{quote}

(亚·斯密在任何地方都没有说,修理劳动不能象生产新物品的劳动那样固定在比较耐久的物品上。)

\begin{quote}“这种劳动与其说是生产物品,不如说是保存物品;它的目的与其说是增加它所加工的物品的价值,不如说是防止这些物品的损坏。所有这些劳动者,包括家仆在内,\textbf{都能使付给他们报酬的人节约维护自己财物的劳动}。”\end{quote}

(因此,可以把他们看成是保存价值的机器,或者确切些说,保存使用价值的机器。这种“节约”劳动的观点被\textbf{德斯杜特·德·特拉西}进一步发挥了。这一点以后再谈。一个人的非生产劳动,决不能由于使另一个人省去\textbf{非生产劳动}而变成生产劳动。这种非生产劳动总得由其中的一个人来完成。斯密所说的非生产劳动,有一部分由于分工而成为必要,这只是指消费物品时绝对必要的并且可以说是属于\textbf{消费费用}的那一部分,而且它只有在使生产劳动者节约这部分时间的时候,才成为必要的。不过,亚·斯密并不否认这种“分工”。如果每个人本来不得不既完成生产劳动,又完成非生产劳动,而由于两个人之间实行这种分工,生产劳动和非生产劳动都能完成得更好,那末,按照斯密的说法,这丝毫也不会改变一种劳动是生产劳动,而另一种劳动是非生产劳动这个事实。)

\begin{quote}“在大多数情况下,他们都是由于这一点,并且仅仅是由于这一点而被雇用的\end{quote}

(为使一个人节约自己服侍自己的劳动,必须有 10 个人来服侍他,这真是一种奇特的“节约”劳动的方法;而且,这种“非生产劳动”大部分又恰恰是由那些无所事事的人来使用的);

\begin{quote}因此,或者他们都是\textbf{生产的},或者他们都不是生产的。”(同上,第 171—172 页)\end{quote}

[348]\textbf{第二},法国人不会忘掉“桥梁和公路”\authornote{在法国,这是指交通主管部门。——编者注}。他说:

\begin{quote}为什么“一个私人工商业企业的监督人或经理的劳动”,应当称为生产的,“而一个负责维持公路、运河、港口、货币制度和其他活跃商业的重要机构的秩序,保障交通运输的安全,监督契约的执行等等,并完全有权被认为是\textbf{大社会工厂监督人}的政府官吏的劳动,就应当称为\textbf{非生产的}呢\fontbox{?}这完全是同类的劳动,只不过规模更大罢了”。(第 172—173 页)\end{quote}

只要这个小伙子参加物质品的生产(或保存和再生产),并且这些物质品不是掌握在国家手里而是\textbf{可以出卖的},斯密就会把他的劳动称为“生产的”。“大社会工厂监督人”——这纯粹是法国的创造。

\textbf{第三},在这里,加尔涅热衷于“道德”。为什么“诱惑我的嗅觉的香水制造者”应当认为是生产劳动者,而“陶醉我的听觉”的音乐家应当是非生产劳动者呢\fontbox{?}(第 173 页)斯密会回答说,因为一个提供物质产品,另一个不提供物质产品。道德和这两个人的“功绩”一样,同这里的区分毫无关系。

\textbf{第四},认为“提琴制造者、风琴制造者、乐器商人、布景师等等”是生产的,而以他们的劳动为“准备阶段”的那些职业则是非生产的,难道这不是矛盾吗\fontbox{?}

\begin{quote}“这两种人\textbf{劳动}的最终目的是提供\textbf{同一种消费}。如果一种人劳动的最终结果不应当算作社会劳动的\textbf{产品},那末,为什么偏要对不过是\textbf{达到这种结果的手段}另眼看待呢\fontbox{?}”(同上,第 173 页)\end{quote}

如果这样来谈问题,那就会得出结论说:吃粮食的人和生产粮食的人一样,也是生产的。因为,为什么生产粮食呢\fontbox{?}就是为了吃。因此,如果吃粮食是非生产劳动,那末,为什么种粮食这种不过是达到这个目的的手段,却是生产的呢\fontbox{?}而且,吃粮食的人会生产脑子、肌肉等等,难道这不是象大麦或小麦一样贵重的产品吗\fontbox{?}——某位被激怒的人类之友,也许会这样质问亚·斯密。

第一,亚·斯密并不否认非生产劳动者会生产某种产品。否则,他根本就不是劳动者了。第二,开药方的医生不是生产劳动者,而配药的药剂师却是生产劳动者,这看起来好象是奇怪的。同样,制造提琴的乐器制造者是生产劳动者,而演奏提琴的提琴师却不是。这只能证明,某些“生产劳动者”提供的产品,其唯一的目的是充当非生产劳动者的生产资料。但这并不比这样的事实更奇怪:归根到底,一切生产劳动者,第一,提供支付非生产劳动者的资金,第二,提供产品,让\textbf{不从事任何劳动}的人消费。

在这些批评意见中,第二点完全符合怎么也忘不了“桥梁和公路”的法国人的精神;第三点归结为道德;第四点,或者是包含一种胡说,即认为消费和生产一样是生产的(这对于资产阶级社会来说是错误的,因为在这个社会里,一种人生产而另一种人消费),或者是说明,生产劳动的一部分只为非生产劳动提供材料,而这一点,亚·斯密从来没有否认过。只有第一点包含着正确的意思,即亚·斯密在他的第二个定义中,把\textbf{同一种}劳动既称为生产劳动又称为非生产劳动,[349]或者确切些说,按照他自己的定义,他本来应该把他的“非生产”劳动中的某一个较小的部分称为\textbf{生产的}。——可见,这并不是反对\textbf{区分}本身,而是反对这种区分\textbf{包括的范围},或者说,反对这种区分\textbf{适用的范围}。

提了所有这些批评意见之后,大学者加尔涅终于谈到本题:

\begin{quote}“看来在斯密所想象出来的两个阶级之间,能够找到的唯一的总的区别就是:就他所谓的\textbf{生产}阶级来说,\textbf{物品制造者和物品消费者之间}有或者\textbf{总会}有一个\textbf{中介人存在};而就他所谓的\textbf{非生产}阶级来说,\textbf{不会有任何中介人存在},这里\textbf{劳动者和消费者}之间的关系\textbf{必然是直接的、没有中介的}。很明显,那些享受医生的经验、外科医师的手术、律师的知识、音乐家或演员的天才以及家仆的服务的人,在所有这些不同的劳动者从事这种劳动时,\textbf{必然}同他们发生一种直接的没有中介的关系;相反,在另一个阶级的职业中,\textbf{供消费的物品是物质的、可以感觉的},因此,在它们从制造者手里转到消费者手里之前,\textbf{就能够成为一系列中间性交换行为的对象}。”(第 174 页)\end{quote}

加尔涅无意中用后面几句话表明,在斯密的第一种区分(同资本交换的劳动和同收入交换的劳动)和第二种区分(固定在物质的可以出卖的商品上的劳动和不固定在这种商品上的劳动)之间,存在着多么隐蔽的思想联系。不固定在商品上的种种劳动,按其性质来说,大多数\textbf{不能}从属于资本主义生产方式;其他各种劳动,则可能从属于资本主义生产方式。更不用说,\textbf{在资本主义生产的基础上},大部分物质商品,即“物质的、可以感觉的物品”,是在资本的支配下由雇佣工人生产的,那些[非生产]劳动(或服务,无论是妓女的服务,还是罗马教皇的服务),只能由生产工人的工资或他们的雇主(和分享利润的人)的利润来支付;也不必谈这样一个事实,即这些生产工人创造着养活非生产劳动者,因而使他们得以生存的物质基础。但这条饶舌的法国狗有一个特点,他自认为是政治经济学家,即资本主义生产的研究者,却把那种使生产成为资本主义生产的东西(即资本同雇佣劳动相交换,而不是收入同雇佣劳动直接交换,或劳动者自己直接把收入支付给自己)看成是\textbf{非本质的东西}。因此,在加尔涅看来,资本主义生产本身是一种非本质的形式,而不是一种发展社会劳动生产力,并使劳动变为社会劳动的必然形式,尽管只是历史的也就是暂时的必然形式。

\begin{quote}“此外,还应当从他的\textbf{生产}阶级中除掉所有这样的工人,他们的劳动只是洗刷、保存或修理成品,而不是使任何新产品进入流通。”(第 175 页)\end{quote}

(斯密从来没有认为,劳动或劳动产品必须加入流动资本。劳动能够直接加入固定资本,例如在工厂中修理机器的机械师的劳动就是如此。但在这种情况下,这种劳动的\textbf{价值}会加入产品即商品的流通。如果从事修理等等的劳动者是在主顾家里干活,那他们[350]的劳动就不是同资本交换,而是同收入交换。)

\begin{quote}“正是由于这种区别,\textbf{非生产}阶级,象斯密所指出的,只是靠收入而生存。事实上,因为对于这个阶级来说,在他们和他们产品的消费者即他们劳动的享受者之间,不可能有中介人存在,所以这个阶级直接由消费者支付;\textbf{而这种消费者只能用自己的收入来支付}。相反,\textbf{生产}阶级的劳动者通常由\textbf{中介人}支付,\textbf{中介人的目的是从他们的劳动中吸取利润};因此,\textbf{他们大多由资本支付}。但是,这个资本归根到底总要由消费者的收入来补偿;否则它就不能流通,因而也就不能给它的所有者带来利润。”[第 175 页]\end{quote}

最后这个“但是”十分幼稚。第一,资本的一部分就由资本补偿,而不是由收入补偿——不管资本的这一部分进入流通还是不进入流通(例如种子的补偿就是后一种情况)。

\centerbox{\textbf{[(b)在资本同资本交换的过程中不变资本的补偿问题]}}

如果煤矿向制铁厂供应煤炭,并从制铁厂得到铁,铁作为生产资料加入煤矿的生产过程,那末煤炭就按照铁的价值额同资本交换,反过来说,铁也按照自己的价值额作为资本同煤炭交换。煤炭和铁(作为使用价值)都是新劳动的产品,虽然这种新劳动是用已有的生产资料来进行的。但是,年劳动产品的价值,并不就是这一年新加劳动的产品。它还要补偿已经物化在生产资料中的过去劳动的价值。因而,总产品中和过去劳动的价值相等的那一部分,并不是当年劳动产品的一部分,而是过去劳动的再生产。

我们举煤矿、制铁厂、木材厂和机器制造厂的日劳动产品为例。假设这些企业的不变资本都等于产品价值所有组成部分的 1/3,即过去劳动和活劳动之比等于 1∶2。假定这些企业的日产品是 X、X′、X″、X′″。这些产品是一定量的煤炭、铁、木材和机器。作为这样的产品,它们都是日劳动的产品(但同样是协助当日生产而在一日内消费的原料、燃料、机器设备等等的产品)。假定它们的价值是 Z、Z′、Z″、Z′″。这些价值并不是当日劳动的产品,因为 Z/3、Z′/3、Z″/3、Z′″/3 只不过等于 Z、Z′、Z″、Z′″的不变要素在加入当日劳动过程以前具有的价值。所以 X/3、X′/3、X″/3、X′″/3,即生产出来的使用价值的 1/3,也只是代表过去劳动的价值,并且不断补偿这个价值。\fontbox{~\{}这里发生的过去劳动和活劳动产品之间的交换,按其性质来说,完全不同于劳动能力和作为资本存在的劳动条件之间的交换。\fontbox{\}~}

X=Z;但是这个 Z 是整个 X 的价值,\endnote{在这之前马克思一直用字母 x 代表作为使用价值来考察的产品,用字母 z 代表产品价值。从这里起马克思改换了字母符号:用 x 代表价值,用 z 代表使用价值。本版各处都采取马克思最初使用的字母符号。——第 183 页。}而(1/3)Z 是整个 X 中包含的原料等等的价值。因此,X/3 是劳动的日产品的一部分\fontbox{~\{}但决不是日劳动的产品,而是和日劳动相结合的往日劳动的产品,总之,是过去劳动的产品\fontbox{\}~},和日劳动相结合的过去劳动,就在这一部分中得到再现和补偿。不过,只表示实在产品(铁、煤炭等等)的量的这个 X,它的任何一部分按其价值来说都是 1/3 代表过去劳动,2/3 代表当日完成和加进的劳动。过去劳动和当日劳动以什么比例加入产品总额,也就以什么比例加入作为总额组成部分的每一个产品。但如果我把全部产品分成两部分,一边是 1/3,另一边是 2/3,那就等于说,前 1/3 只代表过去劳动,其余的 2/3 只代表当日劳动。事实上,前 1/3 代表加入总产品的全部过去劳动,即代表消费掉的生产资料的全部价值。因而,扣除这 1/3 之后,剩下的 2/3 就可以只代表日劳动的产品。它们事实上也代表一日内加到生产资料上的全部劳动量。

因此,后 2/3 等于生产者的收入(工资和利润)。生产者可以消费它们,即把它们花在个人消费品上。假设一日内采掘的这 2/3 煤炭由消费者或者说由买者购买,不是用货币来购买,而是用商品来交换,他们为了购买煤炭,预先把这些商限品转化为货币。在这 2/3 中,有一部分煤炭加入煤炭生产者本人的个人消费,用于家庭取暖等等,因而这一部分不进入流通,即使它已经进入流通,[351]也会被它的生产者从流通中取回来。从 2/3 煤炭中扣除煤炭生产者本人消费的这部分之后,其余所有的量(如果生产者想把它消费掉)必须拿去同个人消费品交换。

在进行这种交换时,煤炭生产者根本不问消费品的卖者究竟用什么来同煤炭交换,用他们的资本还是用他们的收入;换句话说,根本不问:例如,是毛织厂主为了自己的住宅取暖而用自己的呢绒来同煤炭交换(在这种情况下,煤炭对于他也是消费品,他用他的收入,用代表利润的一定量的呢绒,来支付煤炭),还是毛织厂主的仆役詹姆斯,用他作为工资得到的呢绒来同煤炭交换(在这种情况下,煤炭也是消费品,并且是同毛织厂主的收入交换,不过毛织厂主已经把自己的收入同仆役的非生产劳动交换过了),还是毛织厂主为了补偿它的工厂所必需的、已经消费掉的煤炭,而用呢绒来同煤炭交换。(在最后这种情况下,毛织厂主用来交换的呢绒,对他来说代表不变资本,代表他的一种生产资料的价值;煤炭对他来说不只是价值,而且是一定的、实物形式的生产资料。而呢绒对于煤炭业者来说是消费品,呢绒和煤炭对他来说都代表收入:煤炭是他的未实现形式的收入,呢绒是他的已实现形式的收入。)

至于最后的 1/3 煤炭,煤炭业者不能把它花在他的个人消费品上,不能把它当作收入来花费。它属于生产过程(或再生产过程),它必须转化为铁、木材、机器,转化为构成他的不变资本各组成部分的那些物品,没有这些物品,煤炭的生产就不能更新或继续下去。当然,他也可以用这 1/3 同消费品交换(或者同样可以说,同这些消费品的生产者的货币交换),但这只是为了再用这些消费品去换回铁、木材、机器,这样它们既不加入他本人的消费,也不加入他的收入的支出,而是加入木材、铁、机器生产者的收入的消费和支出,而所有这些木材、铁等等的生产者自己又处于这样的情况:他们不能把他们产品的 1/3 花在个人消费品上。

现在假设,煤炭加入铁生产者、木材生产者、机器制造业者的不变资本。另一方面,铁、木材、机器加入煤炭业者的不变资本。这样一来,既然他们的这些产品以相同的价值额彼此加入[他们的不变资本],那它们就是以实物形式互相补偿,交易的一方只须将买进的东西超过卖出的东西的差额支付给对方就行了。实际上,货币在这里的实践中(通过期票等等)也只是作为\textbf{支付手段}出现,而不是作为铸币,不是作为流通手段出现,它们只是用来支付差额。煤炭生产者需要用这 1/3 煤炭中的某一部分来进行他自己的再生产,正象他在 2/3 中留下某一部分来供他自己消费一样。

通过不变资本同不变资本的交换,即通过一种实物形式的不变资本同另一种实物形式的不变资本的交换来互相补偿的这全部数量的煤炭、铁、木材和机器,既与收入同不变资本的交换毫无关系,也与收入同收入的交换毫无关系。这一部分产品所起的作用,完全象农业中的种子或畜牧业中的种畜所起的作用一样。这是\textbf{劳动的年产品}(不是\textbf{当年新加劳动}的产品,而是新加劳动和过去劳动的产品)的一部分,是(在生产条件不变的情况下)每年作为生产资料,作为不变资本来自我补偿的那一部分。它除了加入一些“实业家”和另一些“实业家”之间的流通以外,不加入任何别的流通,也不影响加入“实业家”和“消费者”之间流通的那部分产品的\textbf{价值}。\authornote{见本册第 111、130—131 页。——编者注}

假定这全部 1/3 煤炭按上述方式以实物形式同自己的生产要素即铁、木材、机器相交换。\fontbox{~\{}也可能是这样:例如它直接只同机器交换,但机器制造业者又把它作为不变资本,不仅同自己的不变资本交换,而且同铁生产者和木材业者的不变资本交换。\fontbox{\}~}在这种情况下,煤炭业者[352]用来同消费品交换的,即作为收入去交换的那 2/3 产品中的每担煤炭,固然也象全部产品一样,按其价值来说由两部分组成:1/3 担等于生产 1 担煤炭时消费的生产资料的价值,2/3 担等于煤炭生产者新加到这 1/3 上的劳动。但是,如果煤炭业者的全部产品,比如说,是 3 万担,那末他只把 2 万担作为收入去交换,其余 1 万担,根据假定,由铁、木材、机器等等补偿;一句话,3 万担中包含的生产资料的全部价值,由同样种类和同等价值的生产资料以实物形式补偿。

这样,2 万担的买者事实上对于 2 万担中包含的过去劳动的价值没有支付一文钱;因为 2 万担只代表总产品价值中体现新加劳动的那 2/3。同样可以说,这 2 万担只代表(例如一年的)新加劳动,完全不代表过去劳动。这样,买者虽然支付每担的全部价值,即过去劳动加新加劳动,但他又是只支付新加劳动;这正是因为他只购买 2 万担,即只购买全部产品中等于全部新加劳动价值的那一部分。这就好比买者除了支付他吃的小麦以外,不支付土地耕种者的种子一样。铁、木材、机器等等的生产者互相补偿了这一部分产品,所以不必再由买者补偿它。生产者用自己产品的一部分补偿了它,这一部分固然是他们劳动的年产品,但决不是他们当年新加劳动的产品,而是他们的年产品中代表过去劳动的那一部分。没有新劳动就不会有产品;但是没有物化在生产资料中的劳动也不会有产品。如果它只是新劳动的产品,那末它的价值就会比现在小,而产品的任何一部分就不需要归还给生产了。但是,如果另一种劳动方式[即以使用生产资料为基础的劳动方式]没有更大的生产能力,不会提供更多的产品,——虽然一部分产品必须归还给生产,——那就没有人会采用它了。

虽然这 1/3 煤炭中没有一个价值组成部分会加入当作收入来出卖的 2 万担煤炭,但是这 1/3 即 1 万担所代表的不变资本的任何价值变动,都会引起当作收入来出卖的其余 2/3 的价值变动。假定铁、木材、机器等等的生产,一句话,上述 1/3 产品所分解成的那些生产要素的生产变贵了。而开采煤炭的劳动生产率仍然不变。花费同量的铁、木材、煤炭、机器和劳动,仍然生产出 3 万担煤炭。但是因为铁、木材和机器贵了,要比以前花更多的劳动时间,所以为换取它们就必须付出比以前更多的煤炭。

[353]假定产品仍然是 3 万担煤炭。煤矿的劳动生产率保持不变。用同量的活劳动和同量的木材、铁、机器等等,仍然生产出 3 万担煤炭。活劳动仍然表现为同一个价值,例如 2 万镑(用货币表现)。相反,木材、铁等等,一句话,不变资本,现在值 16000 镑,而不是 1 万镑,就是说,它们包含的劳动时间增加了 6/10 或 60\%。

这样,全部产品的价值现在等于 36000 镑,而不是以前的 3 万镑了;因而价值增加了 1/5 或 20\%。因此,产品的每一个部分现在也比以前多值 1/5 或 20\%。以前每 1 担煤炭值 1 镑,现在每 1 担值 1 镑+1/5 镑=1 镑 4 先令。以前总产品中 1/3 或 3/9 等于不变资本,2/3 等于新加劳动。现在不变资本和总产品价值之比等于 16000∶36000=4/9。因此,现在不变资本比以前多占了[总产品价值的]1/9。等于新加劳动价值的那部分产品,以前占产品的 2/3 或 6/9,现在占 5/9。

\textbf{这样,我们就有}:

采煤工人的劳动生产能力并没有降低,但是花在采煤上的总劳动(采煤工人的劳动加过去劳动)的生产能力却降低了;就是说,为了补偿[354]不变资本所占的价值组成部分,现在需要比以前多 1/9 的总产品,而新加劳动的价值在产品中则少占了 1/9。铁、木材等等的生产者现在也象以前一样只支付 1 万担煤炭。这个数量的煤炭,以前要花他们 1 万镑,现在要花他们 12000 镑。这样,由于他们必须按照提高了的价格支付他们用铁等等来交换的那部分煤炭,不变资本费用的一部分也就会得到补偿。但是,煤炭生产者必须向他们购买 16000 镑的原料等等。因而煤炭生产者必须支付 4000 镑的差额,即 3333+(1/3)担煤炭。这样,他仍然向消费者提供 16666+(2/3)+3333+(1/3)=20000 担煤炭,即 2/3 产品;但消费者现在为这 2 万担必须支付 24000 镑,而不是 2 万镑。消费者用这个数额为煤炭生产者不仅补偿新加劳动,而且还补偿不变资本的一部分。

对于消费者来说,问题很简单。如果他们要想消费以前那样多的煤炭量,他们就必须多支付 1/5,因此,在每一个生产部门的生产费用都照旧不变的前提下,他们就必须在自己的收入中少用 1/5 购买别的产品。困难只是在于:如果铁、木材等等的生产者不再需要煤炭,那末煤炭生产者怎样才能支付这 4000 镑的铁、木材等等呢\fontbox{?}他把他的等于这 4000 镑的 3333+(1/3)担煤炭卖给了煤炭消费者,并由此取得各种各样的商品。但是这些商品既不能加入他个人的消费,也不能加入他的工人的消费,而必须由铁、木材等等的生产者消费,因为他必须以这些物品的形式来补偿他的 3333+(1/3)担煤炭的价值。人们会说:问题很简单。现在所有的煤炭消费者必须少消费 1/5 的所有其他商品,或者说,必须从每个人自己的商品中多拿出 1/5 来支付煤炭。正是这 1/5 用来增加木材、铁等等生产者的消费。但是,制铁厂、机器制造业、木材业等等生产率的减低,究竟怎样使铁、机器、木材的生产者能够消费比以前更大的收入,这一点乍看起来是不明白的;因为我们假设,他们产品的价格等于产品的价值,因而产品价格的提高,只同他们劳动生产率的减低成比例。

我们曾假设,铁、木材、机器的价值提高了 3/5 即 60\%。这只能由两个原因引起。或者,铁、木材等等的生产部门的生产能力减低是由于这些生产部门中使用的活劳动的生产能力减低,以致生产同一产品必须使用更多的劳动。在这种情况下,铁、木材、机器的生产者必须比以前多使用 3/5 劳动。因为劳动生产率的降低只是暂时涉及个别的产品,所以工资率保持不变。因而,剩余价值率也保持不变。生产者以前需要 15 个工作日的地方,现在需要 24 个工作日,但是在这 24 个工作日中,他照旧每天只支付工人 10 劳动小时,照旧迫使工人每天无偿地劳动 2 小时。这样,以前 15 个工人为自己劳动 150 小时,为企业主劳动 30 小时,而现在 24 个工人则为自己劳动 240 小时,为企业主劳动 48 小时。(在这里我们不问利润率如何。)只有当工资花在铁、木材、机器设备等等上面的时候,工资才会降低,而这种情况实际上是没有的。现在 24 个工人会比以前 15 个工人多消费 3/5。因而煤炭生产者现在能够把 3333+(1/3)担的相应加大的那部分价值销售给这些工人(也就是说,销售给向他们支付工资的老板)。

或者,制铁业、木材业等等生产率的减低是由于它们的一部分不变资本即生产资料变贵了。这时[其他生产部门]又会面临同样的抉择,而归根到底,生产率的减低必定会造成使用的活劳动量的增加,因而也会造成工资的增加,煤炭消费者会以上述 4000 镑的形式,把这些工资部分地支付给煤炭业者。

在那些使用追加劳动量的生产部门中,由于雇用的工人人数增加,剩余价值量也会增加。另一方面,随着这些部门本身的产品加入这些部门不变资本的各个要素(不管这些部门自己是把本身的一部分产品当作生产资料使用,还是象煤炭的情形那样,把自己的产品传为生产资料加入这些部门本身的生产资料中)[的价值的增加],利润率会相应地降低。但是,如果它们花在工资上的流动资本比需要补偿的那部分不变资本增加得更多,那末它们的利润率也会提高,它们[355]也会参加上述 4000 镑某些部分的消费。

不变资本价值的提高(由供应这种不变资本的劳动部门生产率的减低引起),会使包含这种不变资本的产品的价值提高,会使补偿新加劳动的那部分产品(在实物形式上)减少,因而就会使这种劳动的生产能力降低,如果这种劳动是用它本身的产品来表示的话。对于以实物形式自行交换的那部分不变资本来说,一切照旧不变。照旧是同量的铁、木材、煤炭以实物形式自行交换,以补偿用掉的铁、木材、煤炭;价格的上涨在这里会互相抵销。但是,现在形成煤炭业者的一部分不变资本并且不加入这种实物交换的那一煤炭余额,照旧要同收入交换(在上面谈到的情况下,它的一部分不仅同工资交换,而且同利润交换),区别只是在于,这种收入已经不属于以前的消费者,而属于那些在使用更多劳动量,也就是工人人数增加了的生产领域中工作的生产者。

如果某个生产部门生产的产品只加入个人消费,既不作为生产资料加入任何别的生产部门(在这里生产资料始终是指不变资本),也不加入自己的再生产(例如在农业、畜牧业、煤炭业中就有这种情形,在煤炭业中,煤炭本身作为辅助材料加入生产),那末这个部门的年产品\fontbox{~\{}超过年产品的可能的余额对这里的问题没有意义\fontbox{\}~}就始终要由收入支付,即由工资或利润支付。

我们拿前面举过的麻布的例子\authornote{见本册第 92 页及以下各页。——编者注}来看。在 3 码麻布中,2/3 是不变资本,1/3 是新加劳动。因而 1 码麻布代表新加劳动。如果剩余价值等于 25\%,那末 1 码的 1/5 就代表利润。其余的 4/5 码代表再生产出来的工资。1/5 由工厂主本人消费,或者由其他人消费,——这些人把它的价值支付给工厂主,工厂主又以这些人的商品或其他商品的形式来消费这个价值,——结果是一样的。\fontbox{~\{}为简单起见,这里不正确地把全部利润看作收入。\fontbox{\}~}其余的 4/5 码由工厂主再以工资的形式支付出去;他的工人把这些当作自己的收入来消费——或者直接消费,或者交换其他消费品,而这些消费品的所有者则消费麻布。

这两部分加在一起,就是 3 码麻布中生产者自己能当作收入来消费的全部份额——1 码麻布。其余的 2 码代表工厂主的不变资本;它们必须再转化为麻布的生产条件——纱、机器等等。从工厂主的角度来看,2 码麻布的交换是不变资本的交换,但是他可以把这些麻布只同别人的收入交换。例如,他用 2 码的 4/5,即 8/5 码,支付纱,用 2/5 码支付机器。纺纱业者和机器制造业者又可以在这些麻布量中各自消费 1/3,就是说,一个人可以在 8/5 码中消费 8/15 码,另一个人可以在 2/5 码中消费 2/15 码。共计 10/15 或 2/3 码。其余的 20/15 或 4/3 码必须补偿他们的原料——亚麻、铁、煤炭等等,而这些物品中的每一种,又都分解为代表收入(新加劳动)的部分和代表不变资本(原料和固定资本等等)的部分。

但是,这最后的 4/3 码麻布可以只当作收入来消费。因而,那种最终以纱和机器的形式表现为不变资本并由纺纱业者和机器制造业者用来补偿亚麻、铁、煤炭等等的东西(我们把机器制造业者用机器补偿的那部分铁、煤炭等等撇开不谈),可以只代表形成亚麻、铁、煤炭生产者的收入,因而不需补偿不变资本的那部分亚麻、铁、煤炭;换句话说,以纱和机器的形式表现为不变资本的东西,必须属于亚麻、铁、煤炭等等生产者的产品中如上面所说的不包含任何不变资本份额的那一部分。但是,铁、煤炭等等的生产者会把他们的表现为铁、煤炭、亚麻等等的收入,以麻布的形式或其他消费品的形式来消费,因为他们自己的产品本身完全不加入或只有极小部分加入他们的个人消费。这样,铁、亚麻等等的一部分就可以同只加入个人消费的产品——麻布——交换,而由于同这种产品交换,对纺纱业者来说,全部补偿了他们的不变资本,对机器制造业者来说,部分地补偿了他们的不变资本,同时,纺纱业者和机器制造业者又是拿出代表他们收入的那部分纱和机器换取麻布来消费的,从而就补偿了织布业者的不变资本。

这样,实际上全部麻布都归结为织布业者、纺纱业者、机器制造业者、亚麻种植业者、煤炭生产者和铁生产者的利润和工资;同时他们又给麻织厂主和纺纱业者补偿全部不变资本。如果后面这些原料生产者必须通过同麻布交换来补偿自己的不变资本,那末计算就完结不了,因为麻布是个人消费品,不能作为生产资料、作为[356]不变资本的一部分,加入任何生产领域。计算所以会完结,是因为亚麻种植业者、煤炭业者、制铁业者、机器制造业者等等用他们的产品购买的麻布,只补偿他们的产品中对于他们是收入而对于他们的买者是不变资本的那一部分。这种情况所以可能,只是因为他们的产品中不归结为收入、因而不能同消费品交换的那一部分,由他们以实物形式补偿,也就是通过不变资本同不变资本相交换来补偿。

在前面举的例子中,曾假设一个生产部门的劳动生产率保持不变,但用这个部门本身的产品来表示该部门所使用的活劳动的生产率时,劳动生产率却降低了;这个假设可能使人感到奇怪。但是事情解释起来却很简单。

假设纺纱业者的劳动产品是 5 磅棉纱。假定纺纱业者为了生产这些产品只需要 5 磅棉花(就是说没有一点飞花);假定 1 磅棉纱值 1 先令(我们不谈机器设备,也就是说,假设它的价值没有下降,也没有上涨,因而它对于我们考察的情况来说等于零)。1 磅棉花值 8 便士。在表示 5 磅棉纱价值的 5 先令当中,棉花占 40 便士(5×8 便士)或 3 先令 4 便士,新加劳动占 5×4 便士,即 20 便士或 1 先令 8 便士。因而在全部产品中,3+(1/3)磅棉纱(价值为 3 先令 4 便士)是不变资本所占的部分,1+(2/3)磅棉纱是劳动所占的部分。所以,5 磅棉纱的 2/3 补偿不变资本,5 磅棉纱的 1/3,即 1+(2/3)磅,是支付劳动的那部分产品。

现在假定 1 磅棉花的价格上涨了 50\%,从 8 便士上涨到 12 便士,即上涨到 1 先令。那末 5 磅棉纱就值:5 磅棉花所值的 5 先令和新加劳动所值的 1 先令 8 便士(新加劳动的量,因而用货币表现的价值,保持不变)。这样,5 磅棉纱现在值 5 先令+1 先令 8 便士=6 先令 8 便士。在这 6 先令 8 便士中,现在原料占 5 先令,劳动占 1 先令 8 便士。

6 先令 8 便士=80 便士,其中 60 便士为原料,20 便士为劳动。在 5 磅棉纱的总价值(80 便士)中劳动现在只占 20 便士,或 1/4 即 25\%;而以前是占[33+(1/3)]\%。另一方面,原料占 60 便士,即 3/4 或 75\%,而以前只占[66+(2/3)]\%。因为 5 磅棉纱现在值 80 便士,所以 1 磅值 80/5 即 16 便士。这样,在 5 磅棉纱中,代表[新加]劳动创造的价值的 20 便士占 1+(1/4)磅棉纱;其余的 3+(3/4)磅棉纱是原料所占的部分。以前[新加]劳动(利润和工资)占 1+(2/3)磅,不变资本占 3+(1/3)磅。因而,用劳动本身的产品来估计,劳动的生产能力降低了,虽然劳动生产率没有变,而只是原料涨价了。劳动仍然保持自己原来的生产率,因为同一劳动在同一时间内把 5 磅棉花变成 5 磅棉纱,因为这种劳动的真正产品(从使用价值来看),只不过是棉花所获得的\textbf{棉纱形式}。5 磅棉花象以前一样由于同一劳动而获得棉纱形式。但是构成实在产品的不只是这种棉纱形式,而且还有棉花,即获得这种形式的物质,这种物质的价值现在和赋予这种形式的劳动相比,在总产品中占了更大的部分。因此,纺纱工人的同量劳动现在是由较少量的棉纱来支付了,换句话说,补偿这种劳动的那部分产品减少了。

这个问题就是如此。

\centerbox{\textbf{[(c)加尔涅反驳斯密时的庸俗前提。加尔涅回到重农学派的见解。比重农学派后退一步:把非生产劳动者的消费看成生产的源泉]}}

因此,第一,加尔涅断言全部资本归根到底总要由消费者的收入来补偿,是错误的;因为资本的一部分只能由资本补偿,不能由收入补偿。第二,这种说法本身也是荒谬的,因为收入本身,只要不是工资(或由工资支付的工资,即由工资派生的收入),就是资本的利润(或由资本的利润派生的收入)。最后,加尔涅断言\authornote{见本册第 182 页。——编者注}那一部分不流通(意即不由消费者的收入补偿)的资本,“不能给它的所有者带来利润”,也是荒谬的。这一部分——在生产条件不变的情况下——事实上不带来利润(确切些说,不带来剩余价值)。但是没有这一部分,资本就根本不可能生产自己的利润。

\begin{quote}[357]“从这种区别中只能得出这样的结论:为了雇用生产劳动者,不仅需要享用他们劳动的人的收入,而且还需要给中介人带来利润的资本;而为了雇用非生产劳动者,在大多数情况下只要有支付他们报酬的人的收入就够了。”(同上,第 175 页)\end{quote}

单单这一段话就已经是一派胡言乱语,从中可以看出,亚·斯密著作的译者加尔涅,实质上对亚·斯密是一无所知,甚至一点也没有看出《国富论》中最本质的东西,那就是:认为资本主义生产方式是最生产的(同以前的那些形式比较起来,它无疑是这样的)。

首先,为反驳斯密所说的直接由收入支付的劳动是非生产劳动这一点,而提出什么“为了雇用\textbf{非生产}劳动者,在大多数情况下只要有支付他们报酬的人的收入就够了”,这是愚蠢到了极点。其次,还有相对的命题:“为了雇用生产劳动者,\textbf{不仅}需要\textbf{享用}他们劳动\textbf{的人的收入},而且还需要\textbf{给中介人带来利润的资本}”!(那末,加尔涅先生的农业劳动该具有多么大的生产能力,对这种劳动来说,除了消费土地产品的人的收入之外,还需要一个不仅给中介人带来利润,而且给土地所有者带来地租的资本!)

说“为了雇用生产劳动者”,需有第一,使用他们的资本,第二,消费他们的劳动的收入,这是不对的;为此,只需有创造收入来消费他们的劳动成果的资本就够了。如果我以缝纫业资本家的身分把 100 镑花在工资上,这 100 镑会给我带来譬如说 120 镑。它为我创造出 20 镑的收入。只要我愿意,我现在也可以用这 20 镑来消费把衣料做成“上衣”的裁缝的劳动。如果相反,我用 20 镑买一件衣服穿,那就十分明显,并不是这件衣服为我创造了用来购买它的 20 镑。如果我把一个裁缝叫到家里,要他为我缝一件价值 20 镑的衣服,情形也是一样。在第一种情况下,我会比原有的多得 20 镑,在第二种情况下,我在交易后会比交易前少 20 镑。而且,我还会很快发现,直接用我的收入支付给裁缝来做上衣,不如我从“中介人”那里购买上衣便宜。

加尔涅以为,利润是由消费者支付的。消费者支付商品的“价值”;虽然商品中也包含资本家的利润,但是这种商品对于消费者来说,比起他直接花费自己的收入去购买劳动,让受雇的劳动者小规模地生产物品,以满足雇主的个人需要,是便宜的。这里显然暴露出加尔涅对什么是资本一窍不通。

他接着说:

\begin{quote}“其次,不是有许多\textbf{非生产}劳动者,例如演员、音乐家等等,在大多数情况下通过经理来取得自己的工资吗\fontbox{?}而这些经理是从投入这类企业的资本中吸取利润的。”(同上,第 175—176 页)\end{quote}

这个意见是对的。但这不过表明,有一部分劳动者,即亚·斯密按照他的第二个定义称为非生产劳动者的,按照他的第一个定义却应当是生产劳动者。

\begin{quote}“由此应当承认,在\textbf{生产}阶级人数众多的社会里,中介人或企业主手里有大量的资本积累。”(同上,第 176 页)\end{quote}

的确,雇佣劳动的大量存在,只是资本的大量存在的另一种表现。

\begin{quote}“因此,并不是象斯密所认为的那样,资本量和收入量之间的比例决定\textbf{生产}阶级和\textbf{非生产}阶级之间的比例。后面这种比例看来在极大程度上取决于国民的风俗习惯,取决于该国工业发展水平的高低。”(第 177 页)\end{quote}

如果生产劳动者是由资本支付的劳动者,非生产劳动者是由收入支付的劳动者,那末十分明显,生产阶级和非生产阶级之比等于资本和收入之比。但是这两个阶级的比例的增加,不仅仅取决于资本量和收入量之间的现有比例。它还取决于增长着的收入(利润)以怎样的比例转化为资本,并以怎样的比例当作收入来花费。虽然资产阶级起初很节约,但是随着资本的生产率即劳动者的生产率的增长,它就开始仿效[358]封建主豢养大批侍从。根据最近的(1861 或 1862 年)工厂报告,联合王国真正在工厂工作的总人数(包括管理人员)只有 775534 人\authornote{《答可尊敬的下院 1861 年 4 月 24 日的质问》(1862 年 2 月 11 日刊印)。},而女仆单是英格兰一处就有 100 万。让工厂女工每天在工厂流汗 12 小时,以便工厂主可以拿她的无酬劳动的一部分雇她的姊妹当佣人,雇她的兄弟当马夫,雇她的堂兄弟当士兵或警察,来为他个人服务,这真是一种绝妙的安排!

加尔涅最后加的一句话是庸俗的同义反复。他说生产阶级和非生产阶级之间的比例不取决于资本和收入之间的比例,或者更确切地说,不取决于以资本形式支出的现有商品量和以收入形式支出的商品量之间的比例,而(!\fontbox{?})取决于国民的风俗习惯,取决于该国的工业发展水平。事实上,资本主义生产只有在工业发展的一定阶段才出现。

加尔涅作为波拿巴的参议员,当然热衷于有侍从和一般仆人。

\begin{quote}“在人数相等的情况下,没有一个阶级象家仆那样促使来自\textbf{收入}的金额转化为资本。”(第 181 页)\end{quote}

事实上,没有一个阶级会给小资产阶级提供更为卑贱的分子。加尔涅不懂,斯密这个“具有如此洞察力的人”,怎么不能较高地评价

\begin{quote}“这种依附于富人、\textbf{捡拾}被富人挥霍浪费的收入的残余的中介人”等等。(同上,第 183 页)\end{quote}

但是,加尔涅自己在这里也说,中介人只是“捡拾”“收入”的残余。这种收入由什么构成呢\fontbox{?}由生产工人的无酬劳动构成。

加尔涅在对斯密作了所有这些毫无用处的反驳之后,就滚回到重农主义去了,他宣布农业劳动是唯一的生产劳动!为什么呢\fontbox{?}因为它

\begin{quote}“还创造一个新价值,这个价值在这种劳动开始活动时在社会上并\textbf{不存在},甚至作为等价物也不存在,正是这个价值为土地所有者提供地租”。(同上,第 184 页)\end{quote}

那末,什么是生产劳动呢\fontbox{?}就是创造剩余价值的劳动,即除了它以工资形式取得的等价之外还创造新价值的劳动。\textbf{资本同劳动}相交换,\textbf{只不过}是具有和一定量劳动相等的一定价值的商品同比这个商品本身包含的更大的劳动量相交换,因此“创造一个新价值,这个价值在这种劳动开始活动时在社会上并不存在,甚至作为等价物也不存在”。如果加尔涅不了解这一点,那末,这并不是斯密的过错。[VIII—358]

\centerbox{※     ※     ※}

[IX—400]\textbf{热·加尔涅先生于 1796 年}在巴黎出版了《政治经济学原理概论》。书中除了认为只有农业是生产的这种重农主义观点以外,我们还看到另一个观点(这个观点很能说明他对亚·斯密的反驳),那就是,认为消费(由“非生产劳动者”出色地代表的消费)是生产的源泉,生产的大小由消费的大小来衡量。非生产劳动者满足“人为的需要”并消费物质产品,所以据说他们在各方面都是有用的。因此,加尔涅也反驳节省(节约)。在他的序言第 XIII 页上可以读到这样的话:

\begin{quote}“个人的财富由于节约而增加,\textbf{相反},社会的财富则由于消费增加而增长。”\end{quote}

在第 240 页,论国债这一章中,加尔涅说:

\begin{quote}“农业的改进和扩大,因而工商业的进步,除了人为的需要扩大之外,没有\textbf{别的原因}。”\end{quote}

他由此得出结论说,国债是好事情,因为它会使这种需要增加。\endnote{这几段话是论热尔门·加尔涅那一小节的补充,取自手稿第 IX 本,在论萨伊那一小节和论德斯杜特·德·特拉西那一小节之间。加尔涅《政治经济学原理概论》一书中的话,马克思引自德斯杜特·德·特拉西的著作《思想的要素》第四、五部分,1826 年巴黎版第 250—251 页。——第 199 页。}[IX—400]

\centerbox{※     ※     ※}

[IX—421]\textbf{施马尔茨}。这个德国的重农主义余孽,批评斯密对生产劳动和非生产劳动的区分说(他的书德文版\textbf{在 1818 年}出版):

\begin{quote}“我只指出……斯密对\textbf{生产}劳动和\textbf{非生产}劳动所作的区分,不应当看成是重要的和十分准确的,如果考虑到别人的劳动始终只是使我们节省时间,而这种时间的节省就是构成\textbf{劳动价值}和\textbf{劳动价格}的一切。”\end{quote}

\fontbox{~\{}这里他搞乱了。事情并不是由分工引起的时间的节省决定物的价值和价格,而是我用同一价值得到更多的使用价值,劳动的生产能力更大了,因为在同一时间内创造出更多的产品;但是,作为重农学派的余波,他当然不能在劳动时间本身找到价值。\fontbox{\}~}

\begin{quote}“例如,为我做桌子的木匠,和把我的信送到邮局、给我洗衣服和张罗我所需要的物品的仆人,他们两者对我的服务在性质上是完全一样的:他们既节约我亲自干这些事情所必须花费的时间,又节约我为获得做这些事情的技能和本领所必须花费的时间。”(\textbf{施马尔茨}《政治经济学》,昂利·茹弗鲁瓦译自德文,1826 年版第 1 卷第 304 页)\end{quote}

在这个粗制滥造者施马尔茨的作品中,我们还发现,下面这种意见对于了解加尔涅——例如他的消费主义(以及浪费的经济效用)——同重农主义之间的联系,是很重要的:

\begin{quote}“这个主义〈魁奈主义〉认为,手工业者,甚至\textbf{纯粹消费}的人,消费有功,理由是他们的消费可以促进(虽然是间接地促进)国民收入的增加,\textbf{因为没有这种消费,被消费的物品就不会由土地生产出来,也不能加到土地所有者的收入中去}。”(同上,第 321 页)\endnote{以“施马尔茨”为总标题的这几段话是手稿第 IX 本结尾部分的附笔。就其内容来说,它们是该稿本第 400 页(见本册第 199 页)关于加尔涅的补充评论的补充。——第 200 页。}[IX—421]\end{quote}

\tsectionnonum{[(8)]沙尔·加尼耳[关于交换和交换价值的重商主义观点。把一切得到报酬的劳动都纳入生产劳动的概念]}

[VIII—358]\textbf{沙·加尼耳}的《论政治经济学的各种体系》是一本很糟糕、很肤浅的拙劣作品。第一版\textbf{于 1809 年在}巴黎出版;第二版\textbf{于 1821 年}出版(我引用的是第二版)。他的胡诌是同他所反驳的加尔涅直接衔接的。

\fontbox{~\{}\textbf{卡纳尔}在《政治经济学原理》中下定义说:“\textbf{财富是多余劳动的积累}。”\endnote{卡纳尔给财富所下的定义,马克思引自加尼耳《论政治经济学的各种体系》一书(第 2 版第 1 卷第 75 页)。这个定义在卡纳尔的著作第 4 页。——第 201 页。}如果他说,财富是超过工人维持他作为工人的生活而多余下来的劳动,那末这个定义就对了。\fontbox{\}~}

加尼耳先生的出发点是这样一个基本的论点:商品是资产阶级财富的元素,也就是说,为了生产财富,劳动必须生产商品,\textbf{必须}出卖它自己或自己的产品。加尼耳说:

\begin{quote}“在现代文明状态下,劳动只有通过交换才能为我们认识。”(同上,第 1 卷第 79 页)“没有交换,劳动就不能生产任何财富。”(第 81 页)\end{quote}

加尼耳先生从这里一下子就跳到重商主义体系上去了。因为没有交换,劳动就不会创造资产阶级财富,所以“财富完全是来源于商业”(第 84 页)。或者象他在后面所说的:

\begin{quote}“只有交换或商业才使物具有价值。”(第 98 页)这个“价值和财富等同的原则……是关于一般劳动的生产性的学说的基础”。(同上,第 93 页)\end{quote}

加尼耳自己解释说:

\begin{quote}[359]“商业主义”——他把它叫做货币主义的纯粹“变态”——“从劳动的交换价值中引出私人的和公共的财富,不管这种价值是否固定在耐久的和不变的物质对象上。”(第 95 页)\end{quote}

由此可见,就象加尔涅回到重农主义体系的见解上去一样,加尼耳回到重商主义体系的见解上去了。因此,他的毫无用处的废话,对于说明重商主义体系及其“剩余价值”观点的特征,却是非常有用的,特别是因为他提出这些观点来反对斯密、李嘉图等等。

财富是交换价值;因此,凡是生产交换价值或本身具有交换价值的劳动,都生产财富。表明加尼耳看问题比别的重商主义者略深一些的唯一的说法,就是“一般劳动”这个术语。个人的劳动,或者更确切地说,个人劳动的产品,必须采取\textbf{一般}劳动的形式。只有这样,劳动的产品才变成交换价值,变成\textbf{货币}。实际上,加尼耳回到财富就是货币这个观点上去了;不过在加尼耳看来,财富不仅是金银,而且是商品本身,因为商品就是\textbf{货币}。加尼耳说:

\begin{quote}“\textbf{商业主义},或者说,\textbf{一般劳动}的价值的交换。”(第 98 页)\end{quote}

后一说法是荒谬的。产品所以是价值,因为它是一般劳动的存在,是这种劳动的化身,而不是“一般劳动的\textbf{价值}”,否则就等于说价值的价值了。但是假定商品已经被确认为价值,如果愿意,还可以假定它甚至已经具有货币形式,已经完成自己的形态变化。现在商品是交换价值。但是这个商品的价值量有多大呢\fontbox{?}一切商品都是交换价值,它们在这方面彼此没有区别。但是什么东西决定这个商品的交换价值呢\fontbox{?}这里加尼耳停留在最表面的现象上。商品 A 如果同大量的商品 B、C、D 等等相交换,它就是大的交换价值。

加尼耳反对李嘉图和大多数政治经济学家时说,虽然他们的体系和一切资产阶级的体系一样,以交换价值为基础,可是他们在考察劳动时,却把交换置之不顾;这个意见是完全正确的。但是,他们所以这样做,唯一的原因在于,他们认为产品的商品\textbf{形式}是不言而喻的事情,因此他们只考察\textbf{价值量}。在交换中,个人的产品所以表现为一般劳动的产品,仅仅是因为它们表现为\textbf{货币}。而这种相对性的根源已经在于:它们必须表现为一般劳动的存在,并且只有作为相对的、仅仅在量上不同的社会劳动的表现,才能归结为这种存在。不过,交换本身不会使它们具有\textbf{价值量}。在交换中,它们表现为一般社会劳动,而它们究竟能在多大程度上这样表现,要看它们本身能在多大程度上表现为社会劳动,也就是说,要看它们能够交换的商品的数量,因而要看市场的规模、贸易的规模,要看它们借以表现为交换价值的商品系列。例如,如果只有四个不同的生产部门,那末,这四个生产者中的每一个人都会有很大一部分产品是为自己生产。如果有几千个生产部门,那末,每一个人就可以把他的全部产品都当作商品来生产。他的全部产品都可以加入交换。

但是,加尼耳同重商学派一道认为,\textbf{价值量本身是交换的产物},其实,产品通过交换得到的只不过是价值形式或\textbf{商品}形式。

\begin{quote}“交换使\textbf{物}具有价值;没有交换,物就没有价值。”(第 102 页)\end{quote}

如果这是说,\textbf{物},使用价值,只有作为社会劳动的相对表现,才变成价值,才获得这种形式,那是同义反复。如果这是说,它们通过交换,比没有交换前取得更大的价值,那显然是胡说,因为交换要提高商品 A 的价值量,就只有降低商品 B 的价值量。它使商品 A 的价值比交换前增加多少,它也就使商品 B 的价值减少多少。因此,\textbf{A+B 无论在交换前还是在交换后都具有同样的价值}。

\begin{quote}“最有用的产品可能没有价值,如果交换不使它们具有价值……”\end{quote}

(首先,如果这些物是“产品”,那末它们一开始就是劳动的产品,而不象空气等等那样是一般的自然赐予。如果它们是“最有用的”,那末它们就是最高意义上的使用价值,是所有的人都需要的使用价值。如果交换\textbf{不}使它们具有价值,那末,这只有在每个人都为自己生产这些东西的条件下才是可能的;但这同[360]它们是为交换而生产出来的这个前提相矛盾;因此,整个前提就是荒谬的。)

\begin{quote}……“而最无用的产品可能有很大的价值,如果交换对它们有利。”(第 104 页)\end{quote}

在加尼耳先生看来,“交换”是一个神秘人物。如果“最无用的产品”没有一点用处,如果它们没有任何使用价值,谁还会购买它们呢\fontbox{?}可见,对于买者来说,它们无论如何必须有某种“有用性”,哪怕只是想象的“有用性”。如果买者不是傻子,他又何必要为它们支付较高的价钱呢\fontbox{?}可见,造成它们昂贵的原因,无论如何不会是它们的“无用性”。也许是它们的“稀有性”吧\fontbox{?}可是加尼耳把它们叫做“最无用的\textbf{产品}”。既然它们是产品,那末,为什么它们有大的“交换价值”,人们却不以更大的规模生产它们呢\fontbox{?}如果说前面那个用许多货币去购买对他本人既没有实际的使用价值,又没有想象的使用价值的东西的买者,是傻子;那末这里,生产交换价值小的有用品,而不生产交换价值大的无用品的卖者,也是傻子。由此看来,如果它们的使用价值很小(假定使用价值由人们的自然需要决定),但交换价值很大,那末,这必定不是由交换先生的情况造成的,而是由产品本身的情况造成的。\textbf{可见,产品的高的交换价值并不是交换的产物,它不过是在交换中表现出来而已}。

\begin{quote}“物的已交换价值[Valeuréchangée],而不是物的可交换价值[Valeuréchangeable],构成\textbf{真正价值},即和财富等同的那种真正价值。”(同上,第 104 页)\end{quote}

但是,可交换价值是一种物同它所能交换的另一种物的比例。\fontbox{~\{}这里有正确的一点作为基础:商品之所以不得不转化为货币,是由于商品必须作为可交换价值,作为交换价值进入交换,而商品之所以是交换价值,只是交换的结果。\fontbox{\}~}相反,A 的已交换价值是一定量的产品 B、C、D 等等。因此,这已经不是价值(按照加尼耳先生的说法),而是“没有交换的物”。B、C、D 等等在同 A 交换以前都不是“价值”。A 变成价值是由于这些非价值取代了它的位置(作为已交换价值)。于是,由于单纯的换位,这些“物”突然变成价值,而在换位以后,它们就退出交换,处于和以前一样的状况。

\begin{quote}“可见,既不是物的实际效用,也不是物的\textbf{内在}价值,使物成为财富;是交换确定和决定它们的价值,就是这种价值使它们成为和财富等同的东西。”(同上,第 105 页)\end{quote}

交换先生确定和决定的或者是某种已经存在的东西,或者是某种不存在的东西。如果是交换首先创造物的价值,那末,交换本身一停止,这个价值,这个交换的产物,也就会消失。因此,它创造什么,它也就消灭什么。我用 A 同 B+C+D 相交换。A 就在这个交换行为中得到价值。这个行为一结束,B+C+D 就站在 A 方面,A 则站在 B+C+D 方面。它们各站一方,都站在交换(只不过换位)先生的范围以外。B+C+D 现在是“物”,但不是价值。A 也是这样。或者,交换就是名副其实地“确定和决定”,就象测力计确定和决定我们的肌肉力,而不是创造我们的肌肉力一样。但如果是这样,价值就不是由交换创造的了。

\begin{quote}“对于个人和对于各国人民来说,事实上只有在下面的情况下才会有财富:每人为大家劳动\end{quote}

(就是说,每人的劳动表现为\textbf{一般社会劳动}。如果不这样解释,这句话就是荒谬的;因为要是撇开这种一般社会劳动形式,那就应该说,制铁业者不是为大家劳动,而\textbf{只是}为铁的消费者劳动),

\begin{quote}大家为每人劳动”\end{quote}

(如果这里谈的是使用价值,那又是荒谬的;因为大家的产品全都是特殊的产品,每人需要的也只是某种特殊的产品;因此,这里的意思又只能是:每种特殊的产品都采取\textbf{为每人而存在}的形式,而它以这样的形式存在,并不是由于它作为特殊的产品和“每人”的产品不同,而只是由于它和这些产品等同;我们在这里看到的,又是在商品生产基础上表现出来的那种社会劳动形式)。(同上,第 108 页)

[361]加尼耳从这个规定——交换价值是孤立的个人的劳动作为一般社会劳动的表现,又滚到最粗俗的看法上去了:交换价值是商品 A 同商品 B、C、D 等等相交换的比例,如果用许多 B、C、D 来交换 A,那末商品 A 的交换价值就大;但在这种情况下,也就是用少量的 A 来交换 B、C、D。财富由交换价值构成。交换价值是产品相互交换的比例。这样,全部产品的总和也就没有任何交换价值了,因为这个总和不同任何东西交换。因此,财富由交换价值构成的社会,也就没有任何财富了。由此不仅可以得出加尼耳本人所做的结论:“由劳动的交换价值构成的国民财富”(第 108 页)按其交换价值来说,永远不会增加,也不会减少(因此,\textbf{也就没有任何剩余价值});而且还可以得出这样的结论:这个财富根本没有交换价值,从而不是财富,因为财富只由交换价值构成。

\begin{quote}“如果谷物极为丰裕,因而\textbf{谷物价值下降},那末土地耕种者的财富就会减少,因为现在他们只有较少的交换价值可以用来获取生活上必需的、有用的或喜爱的物品;不过,土地耕种者亏损多少,谷物消费者就恰好得利多少;一些人的亏损为另一些人的得利所补偿,总财富不会有任何变化。”(第 108—109 页)\end{quote}

对不起!谷物消费者消费的是谷物,而不是谷物的交换价值。他们是在食物上,而不是在交换价值上变得更富了。他们用自己少量的产品交换谷物,而这些产品,由于同它们所交换的谷物比起来数量较少,所以具有\textbf{高的交换价值}。土地耕种者现在得到了高的交换价值,而消费者得到了具有较低交换价值的许多谷物,因而现在消费者是穷人,土地耕种者是富人。

其次,总和(交换价值的社会总和)愈变为交换价值的总和,就愈丧失它作为交换价值的性质。A、B、C、D、E、F 只有互相交换,才有交换价值。一旦它们已经交换,它们对于它们的消费者,买者来说,就都是产品了。它们经过转手,就不再是交换价值了。这样一来,“由交换价值构成的”社会财富也就消失了。商品 A 的价值是相对的;这个价值是 A 对 B、C 等等的交换比例。A+B 具有较少的交换价值,因为它们的交换价值只不过是它们对 C、D、E、F 的比例。而 A、B、C、D、E、F 的总和根本没有任何交换价值,因为这个总和不表示任何比例。商品的总和不同别的商品交换。因此,由交换价值构成的社会财富没有任何交换价值,因而也不是财富。

\begin{quote}“由此可见,一国要靠国内贸易来致富是困难的,也许是不可能的。进行对外贸易的各国人民的情况就有所不同了。”(同上,第 109 页)\end{quote}

这是老重商主义。据说价值就在于,我得到的不单单是等价,而是多于等价。可是加尼耳同时又认为根本没有等价,因为等价的前提是:商品 A 的价值和商品 B 的价值,不是由 A 对 B 的比例或 B 对 A 的比例决定,而是由一个第三者决定,在这个第三者身上,A 和 B 是等同的。如果没有等价,也就没有超过等价的余额。我用铁换得的金比用金换得的铁少。现在我有较多的铁,我又用这些铁换得较少的金。因而,如果说起初由于较少的金等于较多的铁,我得利了,那末,现在因为较多的铁等于较少的金,我同样亏损了。

\centerbox{※     ※     ※}

\begin{quote}“任何劳动,不管其性质如何,只要它具有交换价值,就是生产劳动,就是生产财富的劳动。”(同上,第 119 页)“交换既不考虑产品的量,也不考虑产品的物质性,也不考虑产品的耐久性。”(同上,第 121 页)“所有的〈劳动〉从它们生产出它们本身已经与之交换的\textbf{总额}这一点来说,\textbf{同样都是生产的}。”(第 121—122 页)\end{quote}

起先,说这些劳动同样都是生产的,是从它们补偿上述\textbf{总额},即补偿支付它们的\textbf{价格}(它们工资的\textbf{价值})这一点来说的。但是,加尼耳马上又进一步。他宣称,非物质劳动生产出它本身与之交换的物质产品,以致看起来象是物质劳动生产出非物质劳动的产品。

\begin{quote}[362]“一个是制造柜子的工人,用这个柜子换得 1 舍费耳谷物,另一个是流浪音乐家,用他的劳动也换得 1 舍费耳谷物;这两个人的劳动没有任何区别。在这两种情况下都是生产出了 1 舍费耳谷物,在一种情况下,生产出 1 舍费耳谷物是为了支付柜子,在另一种情况下,生产出 1 舍费耳谷物是为了支付流浪音乐家提供的娱乐。诚然,木匠把 1 舍费耳谷物消费以后,留下一个柜子,而音乐家把 1 舍费耳谷物消费以后,什么也没有留下。可是,有多少被认为是生产劳动的劳动都是这样的情况啊!……判断一种劳动是生产的还是不生产的,不能根据消费以后究竟留下什么,而应当\textbf{根据交换或根据这种劳动所引起的生产}。因为音乐家的劳动同木匠的劳动一样,都是\textbf{生产 1 舍费耳谷物的原因,所以他们两人劳动的生产性同样由 1 舍费耳谷物来衡量},虽然一个人的劳动在干完以后不固定、不物化在某种耐久的对象上,而另一个人的劳动则固定、物化在某种耐久的对象上。”(同上,第 122—123 页)“亚·斯密想减少从事无用劳动的劳动者人数,以便增加从事有用劳动的劳动者人数;但是持这种观点的人没有看到,如果这种愿望能够实现,那就不可能有任何财富了,因为生产者将会缺乏消费者,而没有消费掉的剩余物也就不会再生产出来。生产阶级把自己劳动的产品供给\textbf{那些不创造物质产品的阶级}并不是无代价的。”\end{quote}

(可见,在这里他自己还是把生产物质产品的劳动和不生产物质产品的劳动区分开来了。)

\begin{quote}“生产阶级把自己的产品供给他们,从他们那里换得方便、娱乐、享受,\textbf{而为了能够把自己的产品供给他们,生产阶级就不得不生产这些产品}。如果劳动的物质产品不用来支付那些不创造物质产品的劳动,那它们就找不到消费者,它们的\textbf{再生产}就要停止。因此,生产娱乐的劳动,\textbf{也象}那种被认为是最生产的劳动\textbf{一样有效地参加生产}。”(同上,第 123—124 页)“他们〈各国人民〉所追求的方便、娱乐或享受,几乎总是\textbf{跟在必须用来支付它们的产品后面,而不是走在这些产品前面}。”(同上,第 125 页)\end{quote}

(可见,它们看来与其说是“必须用来支付它们的”那些产品的原因,不如说是那些产品的结果。)

\begin{quote}“如果\textbf{生产阶级不需要}为娱乐、奢侈或豪华服务的劳动〈可见,在这里加尼耳本人也做了这样的区分〉,而又\textbf{不得不}支付这种劳动,并把自己的需求削减相应的数额,那末情况就不同了。在这种情况下,这样的被迫支付就不会引起产品量的增加。”(第 125 页)“除了这种情况之外……任何劳动都必然是生产的,并在不同程度上有助于整个财富的形成和增长,因为\textbf{任何劳动都必然会引起用以支付它的那些产品的生产}。”(同上,第 126 页)\end{quote}

\fontbox{~\{}可见,按照这种说法,“各种非生产劳动”所以是生产的,既不是因为它们有所值,就是说,不是由于它们的交换价值;也不是因为它们生产某种娱乐,就是说,不是由于它们的使用价值;而是因为它们生产生产劳动。\fontbox{\}~}

\fontbox{~\{}如果按照亚·斯密的说法,直接同资本交换的劳动是生产的,那末,同劳动交换的资本,除了它的形式之外,它的物质组成部分也应加以考虑。这个资本归结为必要的生存资料,因而大部分归结为商品,归结为物质品。工人不得不从这种工资中支付给国家和教会的东西,则是为支付那强加于他的服务而作的一种扣除;工人支出在教育上的东西是微不足道的;凡是工人有这种支出的时候,这种支出都是生产的,因为教育会生产劳动能力;工人在医生、律师、牧师的服务上支出,算是他倒霉;还有一些别的非生产劳动或服务也花费工人的工资,不过那是很少的,特别是因为同消费有关的工作(做饭,打扫房屋,甚至在大部分情况下各种各样的修理工作)都是工人自己干的。\fontbox{\}~}

加尼耳的下面这段话是最典型的:

\begin{quote}“如果交换使仆人的劳动具有 1000 法郎的价值,而使土地耕种者或工业工人的劳动只具有 500 法郎的价值,那末由此就应该得出结论说,仆人的劳动对于\textbf{财富生产}的贡献,比土地耕种者或工业工人的劳动大一倍;只要仆人的劳动得到的物质产品报酬比土地耕种者或工业工人的劳动得到的大一倍,就不可能是另外的情况。怎么能够说财富是来源于交换价值最少、因而得到的报酬最低的劳动呢!”(同上,第 293—294 页)\end{quote}

[363]如果工业工人或农业工人的工资等于 500 法郎,他所创造的剩余价值(利润和地租)等于 40\%,那末这种工人的“纯产品”就等于 200 法郎,必须有 5 个这种工人的劳动才能生产出仆人的 1000 法郎工资。如果交换先生一年中不愿购买仆人,而愿意用 1 万法郎去购买一个姘妇,那就需要 50 个这种生产工人的“纯产品”了。既然姘妇的非生产劳动给她带来的交换价值即报酬,比生产工人的工资大 19 倍,那末,按照加尼耳的看法,这个女人对于“财富生产”的贡献就大 19 倍,而且一国向仆人和姘妇支付的东西愈多,它生产的财富也就愈多。加尼耳先生忘记了,只有工业劳动和农业劳动的生产性,总之,只有生产工人创造的、但没有被支付过代价的那个余额,才提供支付非生产劳动者的基金。但加尼耳是这样计算的:1000 法郎工资加上它的以仆人或姘妇的劳动形式存在的等价,一共是 2000 法郎。而实际上,仆人和姘妇的价值,即他们的生产费用,完全取决于生产工人的“纯产品”。甚至仆人和姘妇作为特殊的一类人存在本身,也取决于这种“\textbf{纯产品}”。他们的价格和他们的价值之间极少共同之处。

即使假定,仆人的价值(生产费用)比生产工人的价值或生产费用大一倍。那也应当看到,工人的生产率(象机器的生产率一样)和他的价值是完全不同的东西,它们两者甚至是成反比的。机器所费的价值,对于它的生产率始终是一个负数。

\begin{quote}“有人徒劳地反驳说:如果仆人的劳动,也象土地耕种者的劳动和工业工人的劳动一样,是生产的,那我们就不明白,为什么不能把一国的一般积蓄拿来养活仆人,这样不但不会把这些积蓄浪费掉,而且还会不断地增加它们的价值。这种反驳之所以看来好象是正确的,只不过因为它假设任何劳动的生产性都是它\textbf{参与物质品生产}的结果,\textbf{物质生产创造财富,生产和财富完全是等同的东西}。他们忘记了,\textbf{任何产品的生产都只是由于这些产品被消费,才成为财富},[注:\fontbox{~\{}因此,这个家伙在下一页上说:“任何劳动\textbf{生产}财富,都同自己的由供求决定的交换价值成比例〈这样一来,劳动\textbf{生产}财富,不是同它生产多少交换价值成比例,而是同它本身的交换价值是多少成比例,也就是说,不是同它生产多少成比例,而是同它值多少成比例〉,而且只有把劳动的价值使劳动有权从产品总额中取得的那些产品\textbf{节约下来不消费},劳动的价值才会有助于资本的积累。”\fontbox{\}~}]并且是交换决定生产究竟在什么程度上\textbf{有助于财富的形成}。如果我们记住:在任何国家,各种劳动都直接或间接有助于整个生产;交换既然确定每种劳动的价值,也就决定每种劳动参加生产的份额;\textbf{产品的消费}实现着交换使它具有的价值;生产同消费相比出现的剩余或不足,决定着各国人民富裕或贫穷的程度;——那末,我们就会明白,\textbf{孤立地看}每种劳动,根据它\textbf{参加物质生产}的程度来衡量它的效果性和生产性,\textbf{而不考虑}[364]\textbf{这种劳动的消费},该是多么不彻底。\textbf{只有消费才使劳动具有价值},没有这种价值,财富就不能存在。”(同上,第 294—295 页)\end{quote}

这个家伙,一方面认为财富取决于生产超过消费的余额,另一方面又认为只有消费才使劳动具有价值。而且,根据这种看法,一个消费 1000 法郎的仆人对于创造价值的贡献,比消费 500 法郎的农民大一倍。

加尼耳一方面承认这种非生产劳动不直接参加物质财富的形成,而斯密的主张也无非是这样;另一方面,他又力图证明,这些非生产劳动,同他自己承认的不创造物质财富相反,是创造物质财富的。

在所有这些反驳亚·斯密的人那里,我们一方面看到他们对物质生产采取高傲态度,另一方面又看到他们力图为非物质生产——甚至根本不是生产,如象仆役的劳动——辩护,把它冒充物质生产。无论“纯收入”的所有者把这种收入花在仆役身上,花在姘妇身上,还是花在油炸馅饼上,那是完全无关紧要的。但是认为为了不使产品的价值去见鬼,余额就必须由仆人来消费,而不能由生产工人本人来消费,这种想法是可笑的。马尔萨斯也就是这样宣扬非生产的消费者存在的必要——一旦余额掌握在“有闲者”手里,实际上也就有这种必要。[364]

\tsectionnonum{[(9)加尼耳和李嘉图论“纯收入”。加尼耳主张减少生产人口;李嘉图主张资本积累和提高生产力]}

[364]\textbf{加尼耳}断言,他在他的《政治经济学理论》一书(我不知道这本书)中提出了一个理论,这个理论在他之后被李嘉图复制了。\endnote{加尼耳的这一说法,在他的著作《论政治经济学的各种体系》(1821 年巴黎第 2 版)第一卷第 213 页。加尼耳的《政治经济学理论》一书于 1815 年出版,比李嘉图的《政治经济学和赋税原理》第一版早两年。——第 212 页。}这个理论就是:财富不取决于总产品,而取决于纯产品,即取决于利润和地租的高低。(这决不是加尼耳的发现;但他表述这一点的方式确实有独特之处。)

剩余价值表现在(实际存在于)剩余产品中,表现在产品超过仅仅补偿产品原有要素、因而加入产品生产费用的那部分产品(如果把不变资本和可变资本合起来计算,这部分产品就等于预付在生产中的资本)的余额中。资本主义生产的目的是剩余价值,而不是产品。工人的必要劳动时间——以及产品中用来支付这个时间的等价物——只有在提供剩余劳动的情况下,才是必要的。否则,这个时间对于资本家就是\textbf{非生产的}。

剩余价值等于剩余价值率 m/v 乘同时使用的工作日数或所使用的工人人数 n。因而,M=m/v×n。这样,就有两种方法可以使这个剩余价值增加或减少。例如,[m/v/2]×n 等于 2m/v×n,即 2M。这里 M 增加了[365]1 倍,因为剩余价值率增加了 1 倍,[m/v/2]就是 2m/v,而这比 m/v 大 1 倍。另一方面,m/v×2n 也同样等于 2mn/v,因而也是 2M。可变资本 V 等于 1 工作日的价格乘所使用的工人人数。如果雇用 800 工人,每个工人花费 1 镑,那末,V=800 镑,即 1 镑×800,这里 n=800。如果剩余价值是 160,那末,剩余价值率=160/1 镑×800=160/800=16/80=1/5=20\%。但剩余价值本身是[160/1 镑×800]×800,即[M 镑/1 镑×n]×n。

如果工作日的长度已定,这个剩余价值就只有靠提高生产率才能增加;如果生产率已定,这个剩余价值就只有靠延长劳动时间才能增加。

但是,这里重要的是:2M=[m/v/2]×n,和 2M=(m/v)×2n。如果工人人数减少一半(不是 2n 而是 n),而工人每日的剩余劳动比以前增加一倍,剩余价值(它的总额)就仍然不变。在这样的假设下,有两个东西不变:第一,生产出来的产品总量不变;第二,剩余产品或“纯产品”的总量不变。但发生变化的是:第一,可变资本额或花在工资上的那部分流动资本减少了一半。由原料构成的那部分不变资本也不变,因为虽然工人人数少了一半,但加工的原料数量照旧。相反,由固定资本构成的那部分增加了。

如果原来花在工资上的资本为 300 镑(1 个工人 1 镑),那末现在是 150 镑。原来花在原料上的资本是 310 镑,现在仍然是 310 镑。假定机器的价值比资本的其余部分大 3 倍,那末现在,它就是 1600 镑。\endnote{严格地说,假设机器的价值比资本的其余部分即 460 镑(150+310)大 3 倍,机器的价值就应当是 1840 镑。但这个数目会使计算大大复杂化。因此,马克思为了使计算简便起见,就假定机器价值等于 1600 镑。——第 214 页。}这样,如果机器在 10 年内磨损完,那末,每年加入产品的机器价值就是 160 镑。假定原来每年花在生产工具上的资本是 40 镑,即只是现在的 1/4。

在这种条件下,得出如下的计算:

\todo{}

在这种情况下,利润率提高了,因为总资本减少了:花在工资上的资本减少了 150,而固定资本价值额只增加了 120,即总共比原来少花了 30 镑。

如果把剩下的 30 镑也按照同样的方式花掉,即把这个数额的 31/62(或 1/2)花在原料上,16/62 花在机器上,15/62 花在工资上,那就得出:

\todo{}

因而,合计起来是:

\todo{}

\textbf{所花费的资本总额是 650 镑,和以前一样。全部产品}是 807 镑 5 先令 6 便士。

产品的总价值增加了,所花费的资本总价值仍然不变;这里,不仅全部产品的价值增加了,而且全部产品的数量也增加了,因为比原来多 15 镑的原料转化成了产品。

[366]在加尼耳的书中我们读到:

\begin{quote}“当一国没有机器的帮助,它的劳动只靠手的力量进行时,劳动阶级几乎把自己的全部产品都消费掉。随着工业取得成就,随着工业由于分工、工人熟练和机器的发明而日臻完善,生产费用就减少,换句话说,需要较少的工人就获得较多的产品。”(同上[《论政治经济学的各种体系》1821 年第 2 版],第 1 卷第 211—212 页)\end{quote}

这就是说,随着劳动的生产能力愈来愈大,用于工资的生产费用就减少。同产品相比,工人人数就减少;因而在该产品中他们吃掉的部分就更小。

如果没有机器,一个工人生产自己的生活资料需要 10 小时,而有了机器只需要 6 小时;那末,在前一种场合,他(在工作日为 12 小时的情况下)为自己劳动 10 小时,为资本家劳动 2 小时,资本家从 12 小时劳动的全部产品中获得 1/6。在前一种场合,10 个工人为自己劳动 100 小时(为 10 个工人生产产品),为资本家劳动 20 小时。资本家从 120 单位的价值中获得 1/6,即 20 单位。在后一种场合,5 个工人为自己劳动 30 小时(为 5 个工人生产产品),为资本家劳动 30 小时。现在资本家从 60 小时中获得 30 小时即 1/2,比以前多 2 倍。剩余价值量也增加了,从 20 增加到 30,即增加 1/3。60 日(我在其中占有 1/2)所提供的剩余价值,比 120 日(我在其中占有 1/6)所提供的剩余价值,多 1/3。

其次,资本家获得的总产品的一半,在数量上也比他以前获得的多了。因为现在 6 小时提供的产品同以前 10 小时提供的一样多;就是说,1 小时提供以前的 10/6,或 1+(4/6)=1+(2/3)。因此,现在 30 剩余小时所表现的产品量,就是以前 30(1+2/3)=30+60/3=50 小时所表现的产品量。6 小时提供的产品同以前 10 小时提供的一样多;因此,30(或 5×6)小时提供的产品,同以前 5×10 小时提供的一样多。

因此,资本家的剩余价值增加了,他的剩余产品(如果他自己以实物形式消费这种产品,或者就他以实物形式消费的那部分产品来说)也增加了。即使总产品量不增加,剩余价值也可能增加。事实上,剩余价值的增加意味着:工人能在比以前少的时间内,生产自己的生活资料;因此,工人所消费的商品的价值就减少,就代表较少的劳动时间;这样,一定的价值,例如等于 6 小时,就代表比以前多的使用价值。工人得到的产品量同以前一样,但这个量是总产品的较小部分,这个量的价值则表现工作日产品的较小部分。在产品既不直接加入也不间接加入工人消费品生产的那些生产部门,由于生产率的增减不会改变必要劳动同剩余劳动的比例,生产力的任何增长都不能产生这样的结果,——尽管如此,对这些生产部门来说,结果也会是一样的,不过这种结果不是由它们本身的生产率的变化引起。它们产品的相对价值的提高(如果它们的生产率保持不变)同别的商品的相对价值的降低恰成同一比例;因此,这些产品的相应的较小部分,或者说,物化在这些产品中的、工人所花费的劳动时间的较小部分,就会给工人提供同以前一样多的生活资料。就是说,在这些部门剩余价值的提高,也会完全象在其他部门一样。

但 5 个被解雇的工人将怎样呢\fontbox{?}人们会说,有一笔资本也游离出来了,这就是原先用来支付那 5 个被解雇工人的资本(被解雇的工人每天劳动 12 小时,各自得到 10 小时报酬),即游离出来了 50 小时的资本;用这笔资本以前能够支付 5 个工人,而现在工资降低到 6 小时,就可以支付 50/6 即 8+(1/3)工作日。这样看来,用游离出来的 50 劳动小时的资本,现在能够雇用的工人人数,多于被解雇的工人人数。

但是,并不是全部 50 劳动小时的资本都游离出来。因为,即使假定材料也是按它现在在同一劳动时间内多加工多少而便宜多少,就是说,即使假定这个生产部门的生产力有了同样的增长,毕竟还要留下一笔费用来购买新机器。假定新机器正好值 50 劳动小时;那末,生产新机器所能使用的人数,决不会同被解雇的人数一样多。这 50 劳动小时原来是完全花在工资上的,是用来雇 5 个工人的。而等于 50 劳动小时的机器价值中,则包括利润和工资,包括有酬劳动时间和无酬劳动时间。此外,机器的价值中还包括不变资本。而且,被雇来制造新机器的那些机器制造工人(人数比被解雇的工人少),也并不就是那些被解雇的[367]工人。机器制造业对工人的需求的增加,最多只能影响到后来一批工人的分配,即让刚开始劳动的那一代工人比上一代工人更大量地进到这个部门中来。这对被解雇的工人不会发生影响。除此之外,对机器制造工人的年需求的增加,也决不会和投在机器上的新资本相等。例如,机器可以使用 10 年。这就是说,由机器造成的经常需求,每年只等于机器中包含的工资的 1/10。除这 1/10 之外,还要加上 10 年内的修理劳动和煤炭、机油、各种辅助材料的日常消费;这一切加在一起,也许又占 2/10。

\fontbox{~\{}如果游离出来的资本等于 60 小时,那末,这 60 小时现在就代表 10 小时剩余劳动和仅仅 50 小时的必要劳动。因此,这 60 小时以前都花在工资上,用来雇 6 个工人,而现在就只雇 5 个工人了。\fontbox{\}~}

\fontbox{~\{}在某一单个生产部门,由于采用机器等等使生产力增长从而引起劳动和资本的转移,总是在以后才能发生。这就是说,\textbf{增加的人数},即\textbf{新涌现的一批工人},将以另外的方式分配;这批人也许是被抛上街头的工人的子女,但不是他们自己。他们自己长时期靠旧职业糊口,在最不利的条件下干活,因为他们的必要劳动时间大于社会必要劳动时间;他们或者成为赤贫者,或者在使用比较低级的劳动的部门中找到工作。\fontbox{\}~}

\fontbox{~\{}赤贫者也象资本家(食利者)一样,靠国家的收入过活,不加入产品的生产费用。因此,照加尼耳先生看来,这种人和监狱中养活的犯人完全一样,是交换价值的代表。有很大一部分“非生产劳动者”——领干薪的挂名官员等等,不过是高贵的赤贫者。\fontbox{\}~}

\fontbox{~\{}假定劳动生产率大大提高,以前是 2/3 人口直接参加物质生产,现在只要 1/3 人口参加就行了。以前是 2/3 人口为 3/3 人口提供生活资料;现在是 1/3 人口为 3/3 人口提供生活资料。以前“纯收入”(和劳动者的收入不同)是 1/3;现在是 2/3。现在国民——撇开[阶级]对立不谈——应该用在直接生产上的时间,不再是以前的 2/3,而是 1/3。如果平均分配,所有的人就都会有更多的(即 2/3 的)非生产劳动时间和余暇。但是,在资本主义生产条件下,一切看来都是对抗的,而事实上也是这样。我们的假定并不意味着人口始终是停滞的。因为 3/3 在增长,1/3 也会增长,所以按照\textbf{数量}来说,从事生产劳动的人数可能不断增加。但是相对地,按照同总人口的比例来说,他们还是比以前少 50\%。现在 2/3 的人口中一部分是利润和地租的所有者,一部分是非生产劳动者(由于竞争,非生产劳动者的报酬也差了),这些非生产劳动者帮助前者把收入吃掉,并且把服务作为等价提供给前者或者(例如政治的非生产劳动者)强加给前者。我们可以设想:除了家仆、士兵、水手、警察、下级官吏等等、姘妇、马夫、小丑和丑角之外,这些非生产劳动者一般会有比以前高的教育程度;并且,特别是报酬菲薄的艺术家、音乐家、律师、医生、学者、教师、发明家等等的人数将会增加。

而在生产阶级本身,商业中介人的人数会增加,特别是在机器制造业、铁路修建业、采矿工业中就业的人数会增加;其次,在农业中从事畜牧业,制造化肥、矿肥等等的工人人数会增加。再其次,生产工业原料的土地耕种者同生产食物的土地耕种者相比会增加;为家畜生产饲料的人数同为人生产粮食的人数相比会增加。\textbf{不变资本不断增加,从事不变资本再生产的总劳动的相对量也就不断增加。}直接生产生活资料的那部分工人,虽然人数减少了,可是现在生产出来[368]的产品比以前多。他们的劳动的生产能力更大了。\textbf{在个别资本中,同资本不变部分相比资本可变部分的减少},直接表现为花在工资上的那部分资本的减少,\textbf{同样},从资本的总量来说,——在资本\textbf{再生产}时,——可变资本所占份额的减少,必定表现为所使用的工人总数中相对地有更大的部分从事生产资料的再生产,也就是说,从事机器设备(包括交通运输工具,以及建筑物)、辅助材料(煤炭、煤气、机油、传动皮带等)和充当工业品原料的植物的再生产,而不从事产品本身的再生产。农业工人的人数同工业工人的人数相比会减少。最后,从事奢侈品生产的工人人数会增加,因为收入已经提高,现在会消费更多的奢侈品。\fontbox{\}~}

\centerbox{※     ※     ※}

\fontbox{~\{}可变资本转化为收入:第一,转化为工资;第二,转化为利润。因此,如果把资本同收入对立起来理解,不变资本就表现为\textbf{本来意义上的}资本,表现为总产品中属于生产并加入生产费用,而不被任何个人消费的部分(役畜例外)。即使在个别场合这一部分完全是从利润和工资产生的。归根到底,这一部分决不可能只从这个源泉产生;它是劳动的产品,但它是这样一种劳动的产品,这种劳动把生产工具本身当作收入,就象野蛮人把自己的弓当作收入一样。但是,产品的这一部分一经转化为不变资本,就不再归结为工资和利润,虽然它的再生产也会提供工资和利润。产品中有一定份额属于这一部分。任何后来的产品都是这种过去劳动和现在劳动的产品。现在劳动只有把总产品的某一部分归还给生产,才能继续下去。它必须\textbf{以实物形式}补偿不变资本。如果劳动的生产能力变大了,那末它所补偿的就是相应的产品,而不是产品的价值,因为产品的价值已因此减少。如果劳动的生产能力变小了,那末产品的价值就会提高。在前一种情况下,总产品中需要用来补偿过去劳动的那个相应部分就减少,在后一种情况下,这个部分就增加。在前一种情况下,活劳动的生产能力变大了,在后一种情况下,活劳动的生产能力变小了。\fontbox{\}~}

\fontbox{~\{}原料的改进,也是使\textbf{不变资本}费用降低的一个条件。例如,在同一时间内,用好棉纺纱和用次棉纺纱,纺出的数量就不可能一样,更不用说飞花等等的相对量了。种子等等的质量也具有同样的意义。\fontbox{\}~}

\fontbox{~\{}\textbf{联合化}的例子,即工厂主自己把他原先的不变资本的一部分生产出来,或者他自己使原先作为不变资本从他的生产领域转到别的生产领域去的那些原产品具有进一步的形式(前面已经指出\authornote{见本册第 128—129 页,以及本卷第 3 册第 20 章第 7 节(马克思手稿第 332—334 页)。——编者注},这一切始终只归结为利润的积聚)。\textbf{前者的例子}是纺和织的联合。\textbf{后者的例子}是北明翰市郊的矿主,他们自己承担生产铁的\textbf{全部}过程,而在以前这是由不同的企业主和所有者分别承担的。\fontbox{\}~}

\centerbox{※     ※     ※}

加尼耳接着说:

\begin{quote}“当分工还没有在一切部门中实行时,当构成劳动的、有手艺的人口的一切阶级还没有达到充分发展时,某些工业部门中机器的发明和应用,只会使那些被机器游离出来的资本和工人流向能够利用它们的其他劳动部门。但是十分明显,当一切劳动部门都拥有它们所必需的资本和工人时,能够缩短劳动的任何进一步的改良,任何新的机器,都必定会使劳动人口减少;而因为劳动人口的减少决不会使生产缩减,所以在此之后,社会仍然拥有的那部分产品,就会或者使资本的利润增加,或者使地租增加;因此,采用机器所造成的自然的和必然的结果,就是靠总产品过活的那些雇佣阶级人数减少,靠纯产品过活的那些阶级人数增加。”(同上,第 212 页)[369]“由\textbf{工业进步必然}引起的一国\textbf{人口成分的变化},是现代各国人民繁荣、强大和文明的真正原因。社会下层阶级的人数愈减少,社会就愈不必为这些不幸阶级的贫困、无知、轻信和迷信所不断造成的危险担心;上层阶级的人数愈多,国家所能支配的臣民的人数也就愈多,国家也就愈强盛,开化、理智和文明也就愈能遍于整个人口。”(同上,第 213 页)\end{quote}

\fontbox{~\{}萨伊以下述方式把产品的总价值归结为收入。萨伊在他给李嘉图的《原理》(康斯坦西奥的法译本)第二十六章所加的一个注释中这样说:

\begin{quote}“私人的纯收入,是由他参加生产的……\textbf{那个产品的价值}减去他的费用构成的。但因为他的全部费用是他支付给另一些人的\textbf{收入的各个部分},所以\textbf{产品的总价值是用来支付各项收入的}。一国的总收入是由一国的总产品构成的,也就是由各个生产者之间分配的一国全部产品的总价值构成的。”\endnote{萨伊给李嘉图《政治经济学和赋税原理》第二十六章加的注释,马克思引自加尼耳的著作(第 1 卷第 216 页)。——第 222 页。}\end{quote}

后面这个论点如果这样表述就对了:一国的总收入是由一国的总产品中作为收入在各个生产者之间分配的那一部分构成的,也就是由全部产品中作为收入在各个生产者之间分配的那一部分的总价值构成的,换句话说,它等于扣除了全部产品中补偿每个生产部门的生产资料的那一部分之外的总产品。但是,如果这样表述,萨伊的论点也就自己把自己推翻了。

萨伊接着说:

\begin{quote}“这个价值经过一系列的交换行为之后,可能在它产生的一年内就被完全消费掉,但它仍然是一国的收入,就象一个私人拥有年收入 2 万法郎,即使他在一年内把这些收入全部吃掉,他也仍然拥有 2 万法郎的年收入一样。他的收入不是单单由他的节约构成的。”\end{quote}

他的收入从来不由他的节约构成,虽然这种节约始终由他的收入构成。为了证明一个国家在一年内既可以吃掉自己的资本,又可以吃掉自己的收入,萨伊就拿一个国家和一个不动用自己的资本而在一年内只消费自己收入的私人作比较。如果这个私人在一年内既吃掉自己的资本 20 万法郎,又吃掉自己的收入 2 万法郎,那末,下一年他就没有什么东西可吃了。如果一国的全部资本,从而一国产品的全部总价值,都分解为各项收入,那末萨伊就对了。私人吃掉自己的 2 万法郎收入。他所没有吃掉的他的 20 万法郎资本,由其他私人的收入构成,其中每人都吃掉自己的部分,结果到年终这全部资本就会被吃光。萨伊也许会反驳说:难道这个资本不会在它被消费的同时再生产出来,这样不是就得到补偿了吗\fontbox{?}但是,这个私人所以每年能够再生产出自己的 2 万法郎收入,正是因为他不吃掉自己的 20 万法郎资本。如果别人把这笔资本消费掉,他们也就没有资本来再生产收入了。\fontbox{\}~}

\begin{quote}加尼耳说:“只有\textbf{纯产品}和消费纯产品的人,才构成它的〈国家的〉财富和力量,对它的繁荣、荣誉和强大做出贡献。”(同上,第 218 页)\end{quote}

接着,加尼耳引证萨伊给李嘉图《原理》(康斯坦西奥的译本)第二十六章所加的注释。李嘉图在这里说,如果一国有 1200 万居民,那末,500 万生产工人为 1200 万人劳动,比 700 万生产工人为 1200 万人劳动,对一国的财富更为有利。在前一种情况下,“纯产品”由 700 万非生产人口赖以生活的剩余产品构成,在后一种情况下,“纯产品”由 500 万人赖以生活的剩余产品构成。关于这一点,萨伊指出:

\begin{quote}“这使我们清楚地想起十八世纪经济学家\endnote{“经济学家”是十八世纪下半叶和十九世纪上半叶在法国对重农学派的称呼。——第 38、139、223、411 页。}的学说,他们认为,制造业对于国家的财富毫无贡献,因为\textbf{雇佣阶级}生产多少价值,就消费[370]多少价值,不会为他们的〈经济学家的〉著名的\textbf{纯产品}增添任何东西。”\end{quote}

关于这一点,加尼耳指出(第 219—220 页):

\begin{quote}“经济学家认为,\textbf{工业阶级生产多少价值就消费多少价值},李嘉图先生认为,\textbf{工人的工资不能计算在国家的收入中},很难发现经济学家的看法和李嘉图的理论之间的联系。”\end{quote}

加尼耳在这里也没有抓住问题的实质。经济学家的错误在于,他们把工业劳动者只看作“\textbf{雇佣阶级}”。这是他们和李嘉图不同之处。其次,他们的错误是,认为“\textbf{雇佣阶级}”所生产的只是他们所消费的。和他们相反,李嘉图正确地指出,“纯产品”正是由雇佣工人生产的,而所以会生产纯产品,恰恰是因为他们的消费(即他们的工资)不等于他们的全部劳动时间,只等于他们为生产自己的工资所花的劳动时间;换句话说,因为他们从自己的产品中得到的,只是和他们的必要消费相等的部分,即从他们自己的产品中得到的,只是他们本身的必要消费的等价。经济学家认为,整个工业阶级(业主和工人)都处于他们所说的这种情况。在他们看来,只有地租才是生产出来的产品超过工资的余额。因此他们认为,地租是唯一的财富。而李嘉图说,利润和地租构成这个余额,因而也构成唯一的财富,因此,他尽管同重农学派有分歧,但仍赞同他们的这种见解:只有“纯产品”,即体现剩余价值的那个产品,才构成一国的财富(虽然李嘉图比重农学派更清楚地理解这种剩余价值的性质)。而且他还认为,只有作为超过工资的余额的那部分收入才是财富。他和经济学家不同之处,不是对“纯产品”的解释,而是对工资的解释,经济学家错误地把利润也归入工资的范畴。

萨伊接着反驳李嘉图说:

\begin{quote}“700 万完全就业的工人比 500 万人会有更大的节约。”\end{quote}

针对这一点,加尼耳正确地指出:

\begin{quote}“这就是认为,\textbf{从工资中节约比通过停止支付工资来节约}可取”……“说什么给不生产任何纯产品的工人支付 4 亿法郎工资,只是为了使他们得到从自己的工资中节约的机会和手段,这真是太荒谬了。”(同上,第 221 页)“随着文明的每一进步,劳动变得比较不繁重,而它的生产能力变大了;注定要从事生产和消费的各阶级的人数在减少,而管理劳动,安抚(!)、宽慰(!)和开导全体居民的各阶级的人数在增加,\textbf{而且愈来愈多,\CJKunderdot{他们占有因劳动费用减少}}、商品丰富和消费品价格低廉而\CJKunderdot{\textbf{产生的全部利益}}。人类沿着这个方向正在升入……由于\textbf{社会下层阶级的人数不断减少和上层阶级的人数不断增加的这种趋势}……市民社会会变得更幸福、更强大”等等。(同上,第 224 页)“如果……在业工人人数是 700 万,那末他们的工资就是 14 亿法郎;但如果这 14 亿法郎……不能比 500 万工人得到的 10 亿法郎提供更多的纯产品,那末\textbf{真正的节约,就在于停止向那些不提供任何纯产品的 200 万工人支付 4 亿法郎工资},而决不在于这 200 万工人能够从自己的 4 亿法郎工资中节约。”(第 221 页)\end{quote}

李嘉图在第二十六章中指出:

\begin{quote}“\textbf{亚当·斯密}经常夸大一个国家从大量总收入中得到的利益,而不是从大量纯收入中得到的利益……如果一个国家无论使用多少劳动量,它的纯地租和纯利润加在一起始终是那么多,那末,使用大量生产劳动对于该国又有什么好处呢\fontbox{?}……一个国家无论是使用 500 万还是 700 万生产工人[371]来生产其余 500 万人赖以生活的纯收入……纯收入仍然是 500 万人的食物和衣着。使用更多的人既不能使我们的陆海军增加一名士兵,也不能使赋税多收一个基尼。”(同上,第 215 页)\endnote{马克思指加尼耳著作第一卷的一页。马克思这里从该页摘引了李嘉图《政治经济学和赋税原理》第二十六章的若干片断(康斯坦西奥的法译文)。后来在他的手稿第 377 页又引证了李嘉图《原理》中的这些话,但已经是英文(引自英文第 3 版),而且引得更完全(见本册第 228—229 页)。——第 225 页。}\end{quote}

这使我们想起了古日耳曼人,他们轮换着由一部分居民去打仗,另一部分居民种地。需要留下来种地的人愈少,能够去打仗的人就愈多。如果以前种地需要 500 人,现在需要 1000 人,那末,人口即使增加 1/3,例如从 1000 人增加到 1500 人,对这些日耳曼人也毫无好处。他们所能拥有的军队人数仍然只有 500 人。相反,如果他们的劳动生产力增长了,耕地只需要 250 人就够了,那末,在 1000 人中就有 750 人可以去打仗,而在劳动生产率减低时,1500 人中也只有 500 人可以去打仗。

这里首先应当指出,李嘉图所说的“纯收入”或“纯产品”,并不是指全部产品中超过必须作为生产资料(原料或工具)归还给生产的那一部分的余额。相反,他赞同把总产品归结为总收入的这种错误见解。他所说的“纯产品”或“纯收入”,是指剩余价值,指总收入中超过由工资即由工人收入构成的那一部分的余额。而这种工人收入等于可变资本,等于流动资本中由工人不断消费和不断再生产出来的那一部分,即工人的产品中由工人本身消费的部分。

既然李嘉图并不认为资本家是完全无用的人,就是说,既然他把资本家本人也看作生产当事人,因而把他们的一部分利润归结为工资,那末,他必定把他们的一部分收入从“纯收入”中扣除,并且宣布,所有这些人,只有当他们的工资在他们的利润中构成尽量小的部分时,才会促进财富的增加。但无论如何,这些人作为生产当事人,他们的时间至少有一部分是生产本身不可分割的部分。在他们的时间属于生产的情况下,他们不可能被用于社会或国家的其他目的。他们除了从事生产领导者的业务以外剩下的空闲时间愈多,他们的利润就愈不取决于他们的工资。和他们相反,那些只靠利息过活的资本家,还有那些靠地租过活的人,却完全可能被用于社会和国家的目的,而且他们的收入除了用来再生产他们贵体的那一部分之外,任何一部分也不加入生产费用。这样看来,为了\textbf{国家}的利益,李嘉图必定也希望靠减少利润来增加地租(最纯粹的“纯收入”)了;但李嘉图的观点决不是这样。为什么呢\fontbox{?}因为这有害于资本的积累,或者在某种程度上也可以说,因为这是靠损害生产劳动者的利益来增加非生产劳动者。

李嘉图完全赞同斯密对生产劳动和非生产劳动的下面这种区分,即生产劳动直接同资本交换,而[非生产劳动]直接同收入交换。但李嘉图已经没有斯密那种对生产工人的温情和幻想了。成为生产工人,这是一种不幸。生产工人是生产\textbf{别人的}财富的工人。只有在他充当生产别人财富的生产工具时,他的生存才有意义。因此,如果同量的别人财富能够由较少的生产工人创造出来,那末把多余的生产工人解雇,是完全恰当的。你们,不是为了你们。\endnote{你们,不是为了你们(Vos,nonvobis),出自味吉尔的《警言诗》:“你们,鸟儿们,作巢不是为了你们自己;你们,绵羊们,蓄毛不是为了你们自己;你们,蜜蜂们,酿蜜不是为了你们自己;你们,犍牛们,拉犁不是为了你们自己。”——第 227 页。}不过,李嘉图对这种\textbf{解雇}的理解,并不象加尼耳那样;他并不认为单是解雇这些工人就可以增加收入,就可以\textbf{把以前当作可变资本}(即以工资的形式)消费的\textbf{东西当作收入}来消费。随着生产工人人数的减少,这些被排挤的工人本身所消费的和本身所生产的产品量也会消失,这些工人的等价也会消失。李嘉图并不象加尼耳那样认为仍然会有同量的产品生产出来。不过“纯产品”的量会依然不变。如果工人消费 200,他们所生产的剩余产品是 100,那末总产品就是 300,而剩余产品等于总产品的 1/3。如果工人消费 100,他们所生产的剩余产品照旧等于 100,那末总产品就是 200,而剩余产品就等于总产品的 1/2。这样,总产品就会减少 1/3,即 100 个被解雇的工人过去所消费的那个产品量,而“纯[372]产品”仍然\textbf{不变},因为 200/2 等于 300/3。因此,李嘉图对于总产品的量是漠不关心的,只要总产品中构成“纯产品”的那部分保持不变或者增加了,至少不减少就行。

他这样说:\endnote{马克思在这里引用的李嘉图《政治经济学和赋税原理》第二十六章的话,起先用康斯坦西奥的法译文(引自加尼耳的著作第 1 卷第 214 页),后来用英文原文(引自第 3 版第 416 页)。——第 227 页。}

\begin{quote}“对于一个拥有 2 万镑资本,每年获得利润 2000 镑的人来说,只要他的利润不低于 2000 镑,不管他的资本是雇 100 个工人还是雇 1000 个工人,不管生产的商品是卖 1 万镑还是卖 2 万镑,都是一样的。一个国家的实际利益不也是这样吗\fontbox{?}”\endnote{接着,手稿上用铅笔划去了四页半(第 372—376 页),在这几页马克思详细分析了李嘉图所举的“一个拥有 2 万镑资本的人”的例子中包含的数字材料。马克思指出,这些数字材料是不合情理的。在一种情况下,2 万镑资本的所有者使用 100 工人和按 1 万镑出卖生产出来的商品。在另一种情况下,他使用 1000 工人和按 2 万镑出卖生产出来的商品。李嘉图断言,在这两种情况下 2 万镑资本的利润会是相等的,即都是 2000 镑。马克思作了仔细的计算,这些计算表明在上述前提下这种结果是不可能的。因此,马克思提出了以下这个一般的原理:“例解中的前提不得自相矛盾。提出的前提必须是现实的前提,现实的假设,而不是预先作出的荒谬之谈,也不是假设的非现实性和不可能性。”(第 373 页)李嘉图著作所举的例子所以不能令人满意,还在于这个例子仅仅指明使用的工人人数,而没有指明在两种情况下所生产的总产品的数量。马克思在手稿这个划去的地方的结尾(第 376 页)写道:“这一计算必须停止。没有理由把时间浪费在摆弄李嘉图的这些废话上。”——第 227 页。}[VIII—372]\end{quote}

\centerbox{※     ※     ※}

[IX—377]前面从李嘉图的《原理》第二十六章引证的一段话是:

\begin{quote}“亚当·斯密经常夸大一个国家从大量总收入中得到的利益,而不是从大量纯收入中得到的利益〈因为亚当说:“资本所推动的生产劳动量就愈大”〉……如果一个国家无论使用多少劳动量,它的纯地租和纯利润加在一起始终是那么多,那末,使用大量生产劳动对于该国又有什么好处呢\fontbox{?}”……\end{quote}

\fontbox{~\{}可见,这只不过是说,较大量劳动所生产的剩余价值和较少量劳动所生产的剩余价值一样多。这又只不过是说,在剩余价值率较小时使用大量工人,或者在剩余价值率较大时使用较少量工人,对于一个国家来说是一样的。n×1/2 等于 2n×1/4,这里 n 表示工人人数,1/2 和 1/4 表示剩余劳动。“生产工人”本身只不过是创造剩余价值的生产工具,只要结果一样,这种“生产工人”较多就只会是一个累赘。\fontbox{\}~}

\begin{quote}……“对于一个拥有 2 万镑资本,每年获得利润 2000 镑的人来说,只要他的利润不低于 2000 镑,不管他的资本是雇 100 个工人还是雇 1000 个工人,不管生产的商品是卖 1 万镑还是卖 2 万镑,都是一样的。”\end{quote}

\fontbox{~\{}这句话同下面一段话所表明的一样,具有十分庸俗的意味。例如,一个酒商投资 2 万镑,每年把价值 12000 镑的酒放在酒窖里,把价值 8000 镑的酒拿去卖得 1 万镑,这个酒商使用少量的人而得到 10\%的利润。如此等等。还有银行家呢!\fontbox{\}~}

\begin{quote}“一个国家的实际利益不也是这样吗\fontbox{?}\textbf{只要这个国家的实际纯收入,它的地租和利润不变,这个国家的人口有 1000 万还是有 1200 万,都是无关紧要的}。一国维持海陆军\textbf{以及各种非生产劳动}的能力\end{quote}

(这段话表明,李嘉图赞同亚·斯密关于生产劳动和非生产劳动的见解,尽管他已经不赞同后者对生产工人的那种基于幻想的温情了),

\begin{quote}必须同它的纯收入成比例,而不同它的总收入成比例。如果 500 万人能够生产 1000 万人所必需的食物和衣着,那末 500 万人的食物和衣着便是纯收入。假如\textbf{生产同量的纯收入}需要 700 万人,也就是说,要用 700 万人来生产足够 1200 万人用的食物和衣着,那对于国家又有什么好处呢\fontbox{?}纯收入仍然是 500 万人的食物和衣着。使用更多的人既不能使我们的陆海军增加一名士兵,也不能使赋税多收一个基尼。”(\textbf{李嘉图}《政治经济学和赋税原理》1821 年伦敦第 3 版第 415—417 页)\end{quote}

同总产品\textbf{相比},一个国家的生产人口愈少,国家就愈富;对于单个资本家来说也完全是这样,为了生产同量的剩余价值,他必须使用的工人愈少愈好。在产品量相同的情况下,同非生产人口相比,一个国家的生产人口愈少,国家就愈富。因为生产人口相对的少,不过是劳动生产率相对的高的另一种表现。

一方面,资本的趋势是把生产商品所必要的劳动时间,因而也就是把同产品量相比的生产人口的人数,减到愈来愈小的最低限度。另一方面则相反,资本主义生产方式的趋势是积累,把利润转化为资本,尽量占有更多的别人劳动。资本主义生产方式力求降低必要劳动的比率,但在这个比率已定时,就要尽量使用更多的生产劳动。在这方面,产品同人口之比是无关紧要的。谷物和棉花可以同酒、钻石等等相交换,[378]或者,工人可以被使用来进行不直接往消费品上添加任何东西的那种生产劳动(例如修建铁路等等)。

如果由于某项发明,资本家向自己企业投资可以不是原来的 2 万镑,而只是 1 万镑(因为有这 1 万镑已经足够了),如果这 1 万镑提供的利润不是以前的 10\%,而是 20\%,也就是说,和以前 2 万镑提供的利润一样多;那末,这并不能使资本家因此就把 1 万镑当作收入来花费,而不象以前那样把它当作资本来使用。(只有讲到国债,才谈得上资本直接转化为收入。)他会把它投入别的企业;此外,他还会把自己的一部分利润化为资本。

在政治经济学家们(从某方面说,李嘉图也包括在内)的著作中,我们看到了现实本身存在的二律背反。机器排挤劳动和增加“纯收入”(特别是,它们始终会增加李嘉图在这里称为“纯收入”的东西,即收入借以被消费的那些产品的量);机器减少工人人数和增加产品量(这些产品现在一部分由非生产劳动者消费,一部分在国外交换等等)。这种情况似乎是合乎心愿了。然而,不。应当证明,它们,这些机器,不会使工人找不到饭吃。怎样证明这一点呢\fontbox{?}这样证明:机器经过某种动荡之后(对于这种动荡,恰受其害的那个阶层的居民也许无力反抗),又会比采用机器前使用更多的人,结果“生产工人”的数量又增加起来,以前的不合比例的现象重新出现。

实际上也有这样的情况。因此,尽管劳动生产率不断增长,工人人口还是会不断增加,这不是同产品相比来说的增加(产品是和人口一道增加的,而且比人口增加得更快),而是同全部人口相比来说的增加,例如,如果同时发生资本积聚,从而生产阶级原来的一些组成部分沦为无产阶级。无产阶级的一小部分上升为中等阶级。但是非生产阶级所注意的是不使无产阶级得到过多的生存资料。利润的不断再转化为资本,会在愈来愈广泛的基础上使同样的循环不断再现。

但在李嘉图那里,对积累的关心,比对纯利润的关心更多,他热心地把纯利润称颂为积累的手段。由此也就产生了一面告诫工人,一面又安慰工人的矛盾现象。说什么资本积累对工人的利害关系最大,因为对工人的需求取决于资本积累。如果对工人的需求增加了,劳动的价格也就会提高。因而,他们自己应当愿意降低工资,以便从他们那里夺去的剩余价值,经过资本过滤,再被用来购买他们的新劳动,并提高他们的工资。不过,提高工资是有害的,因为这会阻碍积累。一方面,工人们不应当生育子女。这样,劳动的供给就会减少,也就是说,劳动的价格会提高。但是,劳动价格的提高又会减低积累率,从而减少对工人的需求,并降低劳动的价格。随着劳动的供给的减少,资本[积累]会更迅速地减少。如果工人生育子女,那末他们就会使劳动的供给增加,并使劳动的价格降低,结果利润率就提高,资本积累也随之增加。但工人人口应当和资本积累步调一致;也就是说,必须使工人人口正好同资本家所要求的一样多,——实际上本来也就是这样的。

加尼耳先生在崇拜“纯产品”方面,并不是始终一贯的。他引用萨伊的话:

\begin{quote}“我毫不怀疑,在奴隶劳动的条件下,产品超过消费的余额,比在自由人劳动的条件下更大……奴隶的劳动除了奴隶的体力耗尽之外,没有别的界限……奴隶〈自由工人也是一样〉\textbf{为满足无限的需要即主人的贪欲而劳动}。”(\textbf{萨伊}的著作第 1 版第 215—216 页)\end{quote}

[379]关于这一点,加尼耳指出:

\begin{quote}“自由工人不可能比奴隶花费多,而生产少……任何一项费用都必须有一个为支付这项费用而生产出来的等价。如果自由工人比奴隶花费多,那末,他的劳动产品也一定会比奴隶的劳动产品多。”(\textbf{加尼耳}的著作第 1 卷第 234 页)\end{quote}

仿佛工资的大小\textbf{只}取决于劳动者的生产率,而不取决于一定的生产率条件下产品在劳动者和雇主之间的分配。

\begin{quote}加尼耳接着说:“我知道,人们有某些理由可以说,\textbf{主人在奴隶的费用方面实行节约}〈可见,这里仍是“在奴隶的报酬方面的节约”〉,可以增加他个人的费用”等等。……“但是对总财富来说,社会的一切阶级普遍富裕,比少数的个人拥有过多的财富更有利。”(第 234—235 页)\end{quote}

这种话怎么能同“纯产品”协调起来呢\fontbox{?}而且,加尼耳先生马上就收回了他的自由主义言论(同上,第 236—237 页)。他赞同殖民地的黑奴制度。他持自由主义见解,就只限于不主张在欧洲恢复奴隶制度,因为他已经明白,这里的自由工人就是奴隶,他们之所以存在,只是为了给资本家、土地所有者和他们的仆人生产“纯产品”。

\begin{quote}“他〈\textbf{魁奈}〉坚决反驳雇佣阶级的节约能够增加资本这一点;他的理由是,这些阶级不会有节约的可能,如果他们有了\textbf{剩余},有了\textbf{余额},那也只能是由社会经济中的错误或紊乱造成的。”(同上,第 274 页)\end{quote}

为了证实这一点,加尼耳从魁奈那里引用了下面一段话:

\begin{quote}“如果不生产阶级为了增加自己的现金而实行节约……那末,他们的劳动和他们的报酬就会按同一比例减少,这个阶级就会没落下去。”(《重农主义》第 321 页)\end{quote}

蠢驴!他不懂魁奈的意思。

加尼耳先生用下面这句话作为结尾:

\begin{quote}“它〈工资〉愈多,社会的收入就愈少\end{quote}

(雇佣工人是“社会”立足的基础,但他们本身竟不包括在“社会”之中),

\begin{quote}政府的全部艺术都应当用来减少工资的数额……\textbf{我们所生活的文明世纪应该担负的……一项任务}。”(同上,第 2 卷第 24 页)\end{quote}

\centerbox{※     ※     ※}

关于生产劳动和非生产劳动的问题,还要简略地考察一下\textbf{罗德戴尔}(在这之后,布鲁姆平淡无味的玩笑就不值得考察了)、(费里埃\fontbox{?})、\textbf{托克维尔、施托尔希、西尼耳}和\textbf{罗西}的观点。

\tsectionnonum{[(10)]收入和资本的交换[简单再生产条件下年产品总量的补偿:(a)收入同收入的交换;(b)收入同资本的交换;(c)资本同资本的交换]}

\fontbox{~\{}应当区分:(1)\textbf{转化为新资本的}那部分\textbf{收入};也就是重新资本化的那部分利润。这一部分我们在这里完全不去考虑。这是论积累那一篇要谈的问题。(2)同生产中消费了的资本相交换的收入;通过这种交换,不会形成新资本,只会补偿旧资本,一句话,保存旧资本。因此,在这里的研究中,我们可以把转化为新资本的那部分收入看作等于零,情况就好比所有的收入不是抵补了收入,就是抵补了已消费的资本。

这样,年产品总量就分为两部分:一部分作为收入被消费,另一部分以实物形式补偿已消费的不变资本。

收入同收入交换,例如,麻布生产者从代表他们的利润和工资,即代表他们的收入的那部分产品麻布中,拿出一部分来同代表土地耕种者的一部分利润和[380]工资的谷物交换。因此,这里是麻布同谷物的交换,是加入个人消费的两种商品的交换,是麻布形式的收入同谷物形式的收入的交换。这里没有任何困难。只要可供个人消费的产品按照符合需要的比例生产出来,因而,只要生产这些产品所必需的社会劳动的相应量也按比例分配\fontbox{~\{}当然,决不会丝毫不差地按比例;总会有偏离,有不合比例的情况,这种情况本身会得到平衡;但不断的平衡运动本身是以不断的不合比例现象为前提的\fontbox{\}~},那末收入,比如说以麻布形式存在的数量,恰好就是它作为消费品所需要的数量,也就是说,恰好就是它被其他生产者的消费品所补偿的数量。麻布生产者以谷物等等形式消费的东西,由农民等等以麻布形式消费掉。因此,他用来换取其他商品(消费品)的、代表他的收入的那部分产品,被其他商品的生产者作为消费品换去。他以其他产品形式消费的东西,由别人以他的产品形式消费掉。

顺便指出:在一个单位产品上花费的劳动时间不超过社会必要劳动时间,即不超过生产这个商品平均所需要的时间,这是资本主义生产的结果,资本主义生产甚至不断降低必要劳动时间的最低额。但为此资本主义生产必须在不断扩大的规模上进行。

如果 1 码麻布的价值只等于 1 小时,并且这就是社会为满足自己对 1 码麻布的需要所必须花费的必要劳动时间,那末,由此还决不能得出结论说:如果生产 1200 万码麻布,从而花费 1200 万劳动小时,或者同样可以说,花费 100 万工作日,使用 100 万工人来织麻布,那末,社会“必须”花费在麻布织造业上的,就正好是社会劳动时间的这样一个部分。如果必要劳动时间已知,就是说,一日内所能生产的麻布量已知,那还要问,究竟有多少这样的日数必须花费在麻布生产上。例如一年内花费在一定产品总量上的劳动时间等于:这种使用价值的一定量,例如 1 码麻布(假定这个量=1 工作日),乘所花费的总工作日数。虽然产品每一部分包含的只是生产这一部分所必要的劳动时间,或者说,虽然所花费的劳动时间的每一部分都是创造总产品的相应部分所必要的,但是,一定生产部门所花费的劳动时间总量对社会所拥有的全部劳动时间的百分比,仍然可能低于或高于应有的比例。

从这个观点来看,必要劳动时间就有了另外的意义。现在要问:必要劳动时间究竟按怎样的量在不同的生产领域中分配\fontbox{?}竞争不断地调节这种分配,正象它不断地打乱这种分配一样。如果某个部门花费的社会劳动时间量过大,那末,就只能按照应该花费的社会劳动时间量来支付等价。因此,在这种情况下,总产品——即总产品的价值——就不等于它本身所包含的劳动时间,而等于这个领域的总产品同其他领域的产品保持应有的比例时按比例应当花费的劳动时间。但是,这个领域总产品的价格比它的价值降低多少,总产品的每一部分的价格也降低多少。如果原来生产 4000 码麻布,现在生产 6000 码,而 6000 码的价值是 12000 先令,那末它们还会按 8000 先令出卖。每码的价格将是 1+(1/3)先令,而不是 2 先令,即比价值低 1/3。可见,这就好比在每码的生产上比必须花费的劳动时间多花费了 1/3。因此,在商品的使用价值已定时,商品价格降低到商品价值以下的事实证明,虽然花费在产品的每一部分上的只是社会必要劳动时间\fontbox{~\{}这里假定生产条件不变\fontbox{\}~},但花费在整个这一生产部门中的社会劳动总量过多了,超过必要量了。

由生产条件的变化[381]引起的商品相对价值的降低,完全是另外一回事。已经在市场上的这块麻布,过去值 2 先令,假定等于 1 工作日。但是现在,每天能用 1 先令把它再生产出来。因为价值决定于社会必要劳动时间,而不决定于个别生产者要用的劳动时间,所以,生产者生产 1 码要用的 1 日,只等于 1 个社会规定日的一半。他的 1 码麻布的价格从 2 先令降低到 1 先令,即 1 码麻布的价格降低到他在这块麻布上\textbf{花费的}价值以下,这不过表明生产条件发生了变化,也就是表明必要劳动时间本身发生了变化。另一方面,如果麻布的生产费用不变,而所有其他物品,除了金即货币材料以外,生产费用都提高了,或者,只有某些物品如小麦、铜等等,一句话,不加入麻布组成部分的物品的生产费用提高了,那末 1 码麻布就仍然会等于 2 先令。它的\textbf{价格}不会降低,但是它以小麦、铜等等表示的相对价值降低了。

\centerbox{※     ※     ※}

某一生产部门(生产可供个人消费的商品的生产部门)的一部分收入以另一生产部门的收入的形式被消费,关于这一部分收入,可以说,需求同它本身的供给相等(在生产按照应有的\textbf{比例}进行的情况下)。这就好比这些生产部门各自消费了自己的这一部分收入。这里只有形式上的商品形态变化:W—G—W′。麻布—货币—小麦。

这里互相交换的两种商品,只代表一年内新加劳动的一部分。但是,第一,很明显,这种交换——两个生产者各自以对方的商品形式消费自己产品中代表收入的那一部分——只能在生产消费品,即生产直接加入个人消费因而收入能够借以作为收入来花费的物品的那些生产部门中发生。第二,同样很明显,只有就\textbf{这一部分}产品交换来说,生产者的供给等于他对他想要消费的其他产品的需求这种说法,才是正确的。这里涉及的实际上不过是简单商品交换。生产者不是为自己生产生活资料,而是为别人生产生活资料,而别人又为他生产生活资料。这里不包括收入同资本的任何关系。一种消费品形式的收入同另一种消费品形式的收入交换,事实上也就是消费品同消费品交换。它们的交换过程不决定于它们两者都是收入,而决定于它们两者都是消费品。从形式来说它们都是收入,这种情况在这里是毫无关系的。诚然,这种情况在相互交换的商品的使用价值上,在它们两者都加入个人消费这一点上会显露出来,但这也无非是说明,一部分消费品同另一部分消费品交换。

收入的形式只有在资本的形式同它对立的地方,才能表现出来或显露出来。但即使在我们所考察的情况下,萨伊\endnote{马克思指萨伊的下述论断(在他的《给马尔萨斯的信》1820 年巴黎版第 15 页):例如,如果英国商品充斥意大利市场,那末,原因就在于能够同英国商品交换的意大利商品生产不足。萨伊的这些论断在匿名著作《论马尔萨斯先生近来提倡的关于需求的性质和消费的必要性的原理》(1821 年伦敦版第 15 页)中引证过,在马克思的第 VII 本札记本第 12 页对这部著作所作的摘录中也有这些论断。并参看马克思在本册第 276 页分析的萨伊的这一论点:“某些产品的滞销,是由另一些产品太少引起的。”——第 237 页。}和其他庸俗经济学家们的主张也是错误的。他们断言,如果 A 不能把自己的麻布即他自己想要作为收入来消费的那部分麻布卖掉,或者说,如果他只能低于麻布的价格把麻布卖掉,那末,这是由于 B、C 等等生产的小麦、肉等等太少。这可能是由于他们生产的这些东西数量不够。但这也可能是由于 A 生产的麻布太多;因为即使假定 B、C 等等有足够的小麦等等可以用来购买 A 的全部麻布,他们也还是不会把全部麻布买来,因为他们\textbf{消费的}仅仅是一定量的麻布。或者,这还可能是由于 A 生产的麻布在数量上比他们的收入中一般说来能够用在衣料上的那部分还多,因而总的说来是由于每个人都只能把自己的一定量的产品作为收入来花费,而 A 的麻布生产却以一个比实有额大的收入为前提。但可笑的是,在只牵涉到收入同收入交换的地方,就假定需求的对象不是产品的使用价值,而是这个使用价值的量,也就是说,又忘记了在\textbf{这种}交换中涉及的只是需要的满足,而不象在谈交换价值时那样涉及的是量。

然而每个人有某种产品都宁愿多些,而不愿少些!如果这种说法能够解决困难,那[382]就绝对不能理解,为什么麻布生产者不是采取更简单的过程,以多余的麻布形式消费自己的一部分收入,而是用他的麻布交换其他消费品,并把这些东西大量堆积起来。为什么他总是把自己的收入由麻布形式转化为其他形式呢\fontbox{?}因为他除了需要麻布以外,还有别的需要必须满足。为什么他自己只是消费一定部分的麻布呢\fontbox{?}因为只有一定数量的麻布对他有使用价值。不过,这种说法也适用于 B、C 等等。如果 B 卖酒,C 卖书,D 卖镜子,那末他们每一个人也许都宁愿以自己的产品形式,即以酒、书、镜子的形式,而不以麻布形式消费自己多余的收入。所以不能说,如果 A 完全不能(或不能按照价值)把自己的由麻布构成的收入转化为酒、书、镜子,那末这就绝对必然地意味着酒、书、镜子生产得太少。然而更加可笑的是,把这种收入同收入的交换——只是商品交换的一部分——说成是全部的商品交换。

这样,我们就已经把产品的一部分处理了。消费品的一部分在这些消费品的生产者本身之间转手。这些生产者每人都不以自己的产品形式,而以别人的产品形式消费自己收入(利润和工资)的一部分。他所以能够这样做,只是因为别人也不消费自己的产品,而消费他人的可消费的产品。这就好比每个人都把自己的可消费的产品中代表自己收入的那部分消费掉一样。

至于说到产品的所有其余部分,却出现了更为复杂的关系,只有在这里,互相交换的商品才作为收入同资本彼此对立,因而不只是作为收入彼此对立。

首先必须作如下的区分。在一切生产部门中,总产品的一部分代表收入,即(一年内的)新加劳动:利润和工资。\fontbox{~\{}地租、利息等等都是利润的一部分;混蛋官吏的收入是利润和工资的一部分;其他非生产劳动者的收入,是他们用自己的非生产劳动购买的利润和工资的一部分,因此,这种收入不增加以利润和工资的形式存在的产品,它只决定这种产品中有多大一部分由这些非生产劳动者消费,有多大一部分由工人和资本家自己消费。\fontbox{\}~}但是,只有在某些生产领域,代表收入的那部分产品才能够直接以实物形式成为收入的组成部分,就是说,才能够按其\textbf{使用价值}作为收入来消费。所有只代表生产资料的产品都不能以实物形式,以直接的形式,作为收入来消费,而只能按其\textbf{价值}作为收入来消费。但是这个价值必须在那些生产直接消费品的生产部门中消费。有一部分生产资料,根据用途可以充当这种或那种直接消费品,例如马、大车等等。有一部分直接消费品可以充当生产资料,例如用来酿酒的谷物、用作种子的小麦等等。几乎所有消费品本身都可以作为消费的废料重新加入生产过程,例如,用坏了的破烂麻布可以用来造纸。但是,无论谁生产麻布,也不是为了把它作为破布充当造纸的原料。它只有在麻布织造业的产品本身已经加入消费之后,才取得这种形式。它只有作为这个消费的废料,作为消费过程的残余和产品,才能作为生产资料重新加入其他生产领域。因此,这种情况不是这里要讨论的问题。

总之,有一些产品,它们的生产者只能按其价值而不能按其使用价值消费其中代表收入的部分。因而这些生产者为了消费他们的代表工资和利润的那部分产品,例如机器,就必须把这些产品卖掉,因为他们不能用这些机器本身来直接满足任何的个人需要。同样,这些产品也不能由其他产品的生产者消费,不能加入他们的个人消费,从而不能属于他们借以花费自己收入的那些产品之列,因为这同这些商品的使用价值相矛盾,这些商品的使用价值按其性质来说是\textbf{排斥}个人消费的。所以,这些不可直接消费的产品的生产者只能消费产品的\textbf{交换价值};也就是说,他们必须先把产品转化为货币,然后把货币再转化为可直接消费的商品。但他们应当[383]把这些产品卖给谁呢\fontbox{?}卖给其他非个人消费的产品的生产者吗\fontbox{?}在这种情况下,他们不过是得到一种不可直接消费的产品来代替另一种不可直接消费的产品而已。然而我们曾假定,这部分产品构成他们的收入,他们卖掉这些产品,是为了以消费品的形式消费这些产品的价值。因而他们只能把这些产品卖给可供个人消费的产品的生产者。

这一部分商品交换代表一个人的资本同另一个人的收入的交换,或一个人的收入同另一个人的资本的交换。消费品生产者的总产品中,只有一部分代表收入;另一部分代表不变资本。后面这一部分,他既不能自己消费,也不能用来同其他生产者的可直接消费的产品交换。他既不能以实物形式消费自己这部分产品的使用价值,也不能把这部分产品换成其他消费品而消费其价值。相反,他必须把这部分产品再转化为他的不变资本的实物要素。他必须把自己的这部分产品\textbf{用于生产消费},即作为生产资料来消费。但是他的产品按其使用价值来说只能加入个人消费;因此这种产品的生产者不能以实物形式把它再转化为他自己的生产要素。这种产品的使用价值的性质排斥\textbf{生产消费}。因此,这种产品的生产者只能把自己产品的\textbf{价值}用于生产消费,办法是把这种产品卖给它的上述各生产要素的生产者。他不能以实物形式消费自己的这部分产品;他也不能通过同其他个人消费品的交换,来消费这部分产品的价值。他的这部分产品不能加入他自己的收入,同样不能由其他个人消费品的生产者的收入来补偿,因为要能进行这种补偿,只有他用自己的产品去同这些生产者的产品交换,就是说,只有他把自己产品的价值\textbf{吃掉},而按照假定,这样的事情是不会发生的。但是,因为他的这部分产品象他的另一部分作为收入消费的产品一样,按其使用价值来说只能作为收入来消费,必须加入个人消费,而不能补偿不变资本,所以,这部分产品必须加入不可直接消费的产品的生产者的收入,必须用来同这些生产者的产品中能够由他们消费产品价值的或代表他们收入的那一部分相交换。

如果从交换双方分别来考察这种交换,那末,对 A 这个消费品生产者来说,这种交换表现从资本到资本的转化。通过这种交换,生产者 A 把他的总产品中等于它所包含的不变资本价值的那一部分,再转化为能够执行不变资本的职能的实物形式。无论在交换以前,还是在交换以后,这部分产品按其价值来说都只代表不变资本。相反,对 B 这个不可直接消费的产品的生产者来说,这种交换只表现收入从一种形式到另一种形式的转化。在这里,生产者 B 首先把他的总产品中构成他的收入的那一部分,即总产品中代表这个生产领域的新加劳动(必要劳动和剩余劳动)的那一部分,转化为能够作为收入来消费的实物形式。无论在交换以前,还是在交换以后,这部分产品按其价值来说都只代表他的收入。

如果从交换双方同时来考察这种关系,那就是 A 用他的不变资本去交换 B 的收入,而 B 用他的收入去交换 A 的不变资本。B 的收入补偿 A 的不变资本,而 A 的不变资本补偿 B 的收入。

在这种交换中\fontbox{~\{}交换双方所追求的目的撇开不谈\fontbox{\}~},相互对立的只是商品,发生的是简单的商品交换,这些商品只是作为商品彼此发生关系,对于它们来说,“收入”和“资本”的标志是无关紧要的。只有这些商品的\textbf{使用价值}的不同性质表明,一些商品只能用于和加入生产消费,另一些商品只能用于和加入个人消费。但是,各种商品的各种使用价值在用途上的不同,是属于消费范围内的问题,同它们作为商品进行的交换过程毫无关系。当资本家的资本转化为工资,而劳动转化为资本时,情形就完全不同了。这里商品不是作为简单的商品相互对立,而是资本作为资本出现。在刚才考察的交换中,卖者和买者只是作为卖者和买者,只是作为简单的商品所有者相互对立。

其次,很明显:凡是只用于个人消费的产品,或者说,凡是加入个人消费的产品,在它加入这种消费的范围内,都只能同收入交换。它不能用于生产消费,这一点正好说明,它只能作为收入来消费,即只能用于个人消费。\fontbox{~\{}前面已经指出,我们在这里撇开利润转化为资本的情况不谈。\fontbox{\}~}

假定 A 是某种只用于个人消费的产品的生产者,他的收入等于他的总产品的 1/3,他的不变资本等于总产品的 2/3。按照假定,前 1/3 由他自己消费,不管他[384]是全部还是部分以实物形式消费它,还是完全不以实物形式消费它,而以其他消费品形式消费它的价值;在后面这种情况下,其他消费品的卖者就是以 A 的产品形式消费自己的收入。由此可见,各种消费品中代表自己的生产者的收入的那一部分,或者直接地由生产者消费,或者间接地,通过生产者所需要的消费品的相互交换,由生产者消费。所以,就这一部分来说,发生的是\textbf{收入同收入的交换}。这里的情形就好比 A 代表所有消费品的生产者一样。这些产品总量的 1/3,即代表他的收入的那部分,由他自己消费。但是,这一部分正好代表 A 部类在一年内加到自己的不变资本上的劳动量,而这个量等于 A 部类在一年内生产的工资和利润的总额。

A 部类总产品的其余 2/3 等于不变资本的价值;因而它们必须由 B 部类的年劳动产品来补偿,B 部类提供的是非个人消费的、只加入生产消费、作为生产资料加入生产过程的产品。但是,因为 A 的总产品的这 2/3,象前面那 1/3 一样,必须加入个人消费,所以它们要由 B 部类的生产者用代表他们收入的那部分产品来交换。这样,A 部类就用自己总产品的不变部分换得具有这个不变部分的原来实物形式的产品,即把这个不变部分再转化为 B 部类新创造的产品;而 B 部类用来支付的,只是代表它的收入的那部分产品,但它自己又只能以 A 部类的产品形式消费这一部分产品。因而,B 部类事实上用来支付的是它的新加劳动,这种新加劳动全部表现为 B 用来同产品 A 的后 2/3 交换的那部分产品。这样,全部产品 A 就同收入交换,或者说,全部加入个人消费。另一方面(因为按照假定,对于收入转化为资本在这里不加考察,可以把它看作等于零),社会的\textbf{全部收入}都花费在 A 的产品上;因为 A 的生产者以产品 A 形式消费自己的收入,B 部类的生产者也以产品 A 形式消费自己的收入。而除了这两个部类之外,再也没有别的部类了。

A 的产品全部被消费,虽然这种产品的 2/3 包含不变资本,不能由 A 的生产者消费,而必须再转化为这种产品的各生产要素的实物形式。A 的总产品等于社会的总收入。而社会的总收入代表社会在一年内加到现有不变资本上的劳动时间的总和。这样,虽然 A 的总产品只有 1/3 由新加劳动构成,2/3 则由过去的待补偿的劳动构成,但它仍然能够全部用新加劳动来购买,因为这全部年劳动的 2/3 不能以它本身的产品形式来消费,必须以产品 A 形式来消费。用来补偿产品 A 的新加劳动,比产品 A 本身包含的新加劳动多 2/3,因为这 2/3 是 B 的新加劳动,而 B 只能把这 2/3 用于个人消费,即以产品 A 形式消费,正如 A 只能把这 2/3 用于生产消费,即以产品 B 形式消费一样。可见,第一,A 的总产品能够全部作为收入消费掉,同时,它的不变资本也能够得到补偿。或者更确切地说,A 的总产品所以能够全部作为收入来消费,只是因为它的 2/3 由不变资本的生产者补偿,这些生产者不能以实物形式消费他们那部分代表他们收入的产品,而必须以产品 A 形式,就是说,通过同 A 的 2/3 交换来消费它。

这样,我们就把 A 的后 2/3 处理了。

很明显,如果有第三部类 C,它的产品既能用于生产消费,又能用于个人消费,例如谷物可以充当人的食物或牲畜的饲料,也可以用来做种子或烤面包,又如大车、马、牲畜等等;那末,这丝毫也不会使问题有所改变。就这些产品加入个人消费的那部分来说,它们必须由它们自己的生产者,或者由它们所包含的那部分不变资本的(直接或间接的)生产者,作为收入直接或间接地消费掉。因而在这种情况下,它们属于 A 部类。就这些产品不加入个人消费的那部分来说,它们属于 B 部类。

在这第二种交换过程中,不是收入同收入交换,而是资本同收入交换;在这里,全部不变资本归根到底必须归结为收入,因而归结为新加劳动。这种交换过程可以从两方面来看。假定 A 的产品是麻布。和 A 的不变资本相等的 2/3 麻布(或这些麻布的价值)用来支付纱、机器、辅助材料。但纺纱厂主和机器厂主[385]只能消费这一产品中代表他们自己收入的那一部分。麻织厂主用自己的 2/3 产品支付纱和机器的全部价格。这样他就补偿了纺纱业者和机器制造业者作为不变资本加入麻布的全部产品。但纺纱业者和机器制造业者的这个总产品本身,又等于不变资本和收入,等于这样两部分的总和:一部分是纺纱业者和机器制造业者的新加劳动,另一部分代表他们自己的生产资料的价值,即从纺纱业者来说是亚麻、机油、机器、煤炭等等的价值,从机器制造业者来说是煤炭、铁、机器等等的价值。这样,和 A 的不变资本相等的 2/3 麻布,就补偿了纺纱业者和机器制造业者的总产品,补偿了他们的不变资本加他们的新加劳动,他们的资本加他们的收入。但纺纱业者和机器制造业者只能以产品 A 形式消费自己的收入。他们从 A 的 2/3 中扣除等于他们的收入的那部分之后,便用余额来支付自己的原料和机器。但是按照假定,这些原料和机器的生产者已经不必再补偿任何不变资本了。他们的产品中能够加入产品 A 的数量,即加入充当 A 的生产资料的产品的数量,只是同 A 所能支付的数量一样多。但是,A 用自己产品的 2/3 能够支付的数量,只是同 B 用自己的收入能够购买的数量一样多,也就是说,同 B 换来的产品所代表的收入,所代表的新加劳动一样多。如果 A 的最后一些生产要素的生产者必须卖给纺纱业者[和机器制造业者]的那个产品数量,还代表着他们自己的一部分不变资本,即代表着比他们加到自己的不变资本上的劳动更多的东西,那末,他们就不能以产品 A 形式得到支付,因为这一部分产品是他们不能消费的。因而这里发生的是相反的情况。

现在我们按反过来的顺序来看一看。假定全部麻布等于 12 日。亚麻种植业者、制铁业者等等的产品等于 4 日;这个产品卖给纺纱业者和机器制造业者,他们又给这个产品加上 4 日;然后他们把产品卖给织布业者,织布业者又加上 4 日。麻布织造业者可以自己消费自己产品的 1/3;8 日用来补偿他的不变资本,支付纺纱业者和机器制造业者的产品;纺纱业者和机器制造业者在 8 日中可以消费 4 日,把其余的 4 日支付给亚麻种植业者等等,以此来补偿自己的不变资本;亚麻种植业者、制铁业者等等应当用物化在麻布中的最后 4 日只补偿自己的劳动。

虽然收入在这三种场合假定都是一样的(等于 4 日),但它在参与生产产品 A 的三类生产者的产品中,却占不同的比例。在织布业者那里,它占产品的 1/3(12 的 1/3),在纺纱业者和机器制造业者那里,它占产品的 1/2(8 的 1/2),在亚麻种植业者那里,它和产品相等,等于 4 日。对总产品来说,所有这些生产者的收入都是完全一样的:都等于 12 的 1/3,即 4 日。但是在织布业者那里,纺纱业者、机器制造业者和亚麻种植业者的新加劳动表现为不变资本;在纺纱业者和机器制造业者那里,他们自己的和亚麻种植业者的新加劳动表现为总产品,而亚麻种植业者的劳动时间表现为不变资本。在亚麻种植业者那里,不变资本的现象就不存在了。因此,例如纺纱业者,能够同织布业者按一样的比例使用机器,使用不变资本。例如,比例都是 1/3∶2/3。但是第一,纺纱业使用的资本量(总额)必定比织布业使用的资本量小,因为纺纱业的全部产品都作为不变资本加入织布业。第二,如果在纺纱业者那里,比例正好也是 1/3∶2/3,那末,他的不变资本就将等于 16/3,他的新加劳动将等于 8/3;不变资本将等于 5+(1/3)工作日,新加劳动将等于 2+(2/3)工作日。这样,向他提供亚麻等等的那个部门就将包含相对地更多的工作日。因此,在这里他就要用 5+(1/3)日,而不是用 4 日来支付新加劳动时间了。

不言而喻,A 部类的不变资本中,只有在 A 那里加入价值形成过程的那部分,即在这个 A 的劳动过程中被消费的那部分,才必须由新劳动来补偿。原料、辅助材料和固定资本的损耗全部加入价值形成过程。固定资本的其余部分不加入这个过程,因此不需要补偿。

可见,现有不变资本的很大一部分(其大小决定于固定资本同总资本之比),不需要每年由新劳动补偿。因此,虽然[每年被补偿的固定资本价值]量可能很大(绝对数字),但同总产品(年产品)相比,仍然是不大的。A 部类和 B 部类中\textbf{不变资本的}上述\textbf{整个部分}(在剩余价值已知时)都参加决定利润率,但不参加决定固定资本的实际再生产。同总资本相比,这个部分愈大,即在生产中使用现有的已存在的固定资本的规模愈大,则用来补偿损耗的固定资本的\textbf{再生产的实际\CJKunderdot{量}}也就愈大;但\textbf{同}总资本\textbf{相比},这种再生产的量可能相对地愈小。

假设各种固定资本的再生产时间(\textbf{平均})等于 10 年。[386]假定各种固定资本周转一次各需 20、17、15、12、11、10、8、6、4、3、2、1、4/6 和 2/6 年(共 14 种),那末,固定资本就\textbf{平均}10 年\endnote{马克思用整数 10,为的是不使以后的计算复杂化。如果按照正文中采用的数字(14 种固定资本周转时间的总数为 110 年),对固定资本的平均周转时间进行准确的计算(假定所有 14 种固定资本的数目一样多),那末得出的就不是 10 年,而只是 7.86 年。——第 247 页。}周转一次。

因此,固定资本平均应当在 10 年内得到补偿。如果全部固定资本占总资本的 1/10,每年要补偿的 1/10 固定资本,就只占总资本的 1/100。

如果固定资本占总资本的 1/3,每年就要补偿总资本的 1/30。

但我们现在拿再生产时期不同的两个固定资本加以比较,例如一个固定资本的再生产需要 20 年,另一个资本需要 1/3 年。

那个在 20 年内再生产出来的固定资本,每年只要补偿 1/20。因此,如果它占总资本的 1/2,每年就只要补偿总资本的 1/40,即使它占总资本的 4/5,每年要补偿的也只是总资本的 4/100,即 1/25。相反,那个需要 2/6 年再生产出来即一年周转三次的固定资本,如果只占资本的 1/10,那末固定资本每年就要补偿三次;因而每年就要补偿总资本的 3/10,即几乎是 1/3。平均说来,同总资本相比,固定资本愈大,它的\textbf{相对的}(不是绝对的)再生产时间就愈长,固定资本愈小,它的\textbf{相对的}再生产时间就愈短。在手工生产条件下工具占资本的部分,比在机器生产条件下机器占资本的部分小得多。但是手工业工具的损耗,比机器的损耗快得多。

虽然随着固定资本的绝对量的增加,它的再生产的绝对量或它的损耗也会增加,但是它的再生产的相对量在大多数情况下会减少,因为固定资本的周转时间,它的存在时间,在大多数情况下会同它的规模成比例地增加。这也就证明,再生产机器即固定资本的劳动量,决不会同原先生产这些机器的劳动量(在生产条件不变的情况下)成比例,因为需要补偿的只是每年的损耗。如果象这个部门常常发生的那样,劳动生产率增长了,那末,再生产这部分不变资本所必需的劳动量就减少得更多。诚然,这里还应当算上充当机器每日消费资料的那些东西(不过这些东西同机器制造本身所花费的劳动毫无直接关系)。但机器只消费煤炭和少量的机油或油脂,它的维持费比工人——不但比它所代替的工人,而且比把它本身制造出来的工人——的生活费不知要低多少。

\centerbox{※     ※     ※}

这样,我们就把整个 A 部类的产品和 B 部类的一部分产品处理了。产品 A 全部被消费:1/3 由它自己的生产者消费;2/3 由 B 的生产者消费,B 的生产者不能以自己的产品形式消费自己的收入。B 的生产者以 2/3 的产品 A 形式消费自己产品中代表收入的那部分价值,这 2/3 同时以实物形式补偿 A 的生产者的不变资本,即为他们提供\textbf{用于生产消费}的那些商品。但是,随着产品 A 全部被吃掉以及 A 的 2/3 由产品 B 作为不变资本补偿,全部年产品中代表一年内新加劳动的那一部分,也就\textbf{全部}处理了。因而这个劳动不能购买总产品的任何其他部分了。事实上,一年内全部新加劳动(撇开利润的资本化不谈)就等于 A\textbf{包含的劳动}。因为 A 的 1/3,即由它自己的生产者消费的部分,代表他们在一年内加到 A 的 2/3 上,即加到构成 A 部类不变资本的那部分产品上的新加劳动。除了他们以自己的产品形式消费的这种劳动以外,他们没有进行任何其他的劳动。A 的其余 2/3,即由 B 部类的产品补偿并由产品 B 的生产者消费的部分,代表 B 的生产者加到自己的不变资本上的全部劳动时间。他们没有加入任何更多的劳动,他们也没有更多的东西可消费。[387]

产品 A 按其\textbf{使用价值}来说,代表全部年产品中每年加入个人消费的整个部分。按其\textbf{交换价值}来说,它代表生产者在一年内新加的劳动总量。

但是除了这一切之外,我们还剩下作为\textbf{余额}的总产品的第三部分,它的组成部分在交换时既不能代表收入同收入的交换,也不能代表资本同收入或收入同资本的交换。这就是产品 B 中代表 B 的不变资本的那一部分。这部分不加入 B 的收入;所以它不能由产品 A 补偿,或者说,不能同产品 A 交换,因而也不能作为组成部分加入 A 的不变资本。既然这一部分在 B 部类中不但加入劳动过程,而且加入价值形成过程,那末这一部分也要被消费掉,被用于生产消费。因此,这一部分也完全象总产品的所有其他部分一样,必须\textbf{按照它形成总产品组成部分的比例}得到补偿,而且必须由同类的\textbf{新}产品以实物形式补偿。另一方面,它又不是由任何新劳动补偿。因为新加劳动的总量等于产品 A 所包含的劳动时间,并且这种劳动时间所以全部得到补偿,只是因为 B 以 2/3 的产品 A 形式消费自己的收入,并在交换过程中给 A 提供生产资料,来代替 A 领域中消费了的、待补偿的一切东西。而产品 A 的前 1/3,即由它自己的生产者消费的部分,按其交换价值来说,只由他们本身的新加劳动构成,不包含任何不变资本。

我们就来看看这个余额。

它由以下不变资本构成:第一,加入原料的不变资本;第二,加入固定资本形成过程的不变资本,第三,加入辅助材料的不变资本。

\textbf{第一,原料}。首先,花费在原料生产上的不变资本,归结为固定资本,如机器、工具、建筑物,以及作为机器的消费资料的那些辅助材料。对于可直接消费的那部分原料(如牲畜、谷物、葡萄等等)来说,不会发生上述困难。从这方面来说,它们属于 A 类。它们所包含的那部分不变资本加入 A 的 2/3 即不变部分,这个部分作为资本同不可直接消费的产品 B 交换,或者说,B 以这个部分的形式消费自己的收入。一些不管在生产过程中通过多少中间阶段,都以实物形式加入消费品的不可直接消费的原料,一般也是这样的情况。先转化为纱、然后转化为麻布的那部分亚麻,就全部加入消费品。

但是一部分这样的\textbf{有机原料},如木材、亚麻、大麻、皮革等,它们一部分直接加入固定资本的构成要素,一部分加入固定资本的辅助材料。例如以机油、油脂等等形式。

\textbf{其次,种子}也属于花费在原料生产上的不变资本。植物性的物质和动物性的物质自己再生产自己:植物蕃殖和动物生殖。种子应当是指本来意义上的种子,其次是指作为厩肥再投到土地中去的牲畜饲料,以及种畜等等。年产品中——或年产品的不变部分中——这很大的一部分,直接充当再生自己的材料,它们自己再生产自己。

\textbf{无机原料}——金属、石头等等。这种原料的价值只由两部分构成,因为这里没有在农业中代表原料的种子。无机原料的价值,只由新加的劳动和消费掉的机器(包括机器的消费资料)构成。因此,除了代表新加劳动、因而加入 B 和 2/3A 之间的交换的那部分产品之外,需要补偿的只是固定资本的损耗和固定资本的消费资料(如煤炭、机油等等)。但是这种无机原料构成不变资本的主要组成部分——固定资本(机器、劳动工具、建筑物等等)。所以,无机原料通过[资本同资本的]交换以实物形式补偿自己的不变资本。

[388]\textbf{第二,固定资本(机器、建筑物、劳动工具、各种器皿)}。

它们的不变资本由以下各部分构成:(1)它们的原料,金属、石头、有机原料(如木材、皮带、绳索等等)。它们的这些原料形成它们(机器、工具、建筑物等等)的原料,而它们自己又作为劳动工具加入这种原料的采制过程。因此,它们以实物形式彼此补偿。制铁业者必须补偿机器,机器制造业者必须补偿铁。在采石场中有机器的损耗,而在工厂建筑物中有建筑石材的损耗,等等。(2)\textbf{机器制造机的损耗}。这些机器制造机必须在一定时期内由同种新产品补偿。同种产品自然可以自己补偿自己。(3)\textbf{机器的消费资料}(辅助材料)。机器消费煤炭,但煤炭也消费机器,等等。各种机器以器皿、管筒、软管等等形式加入机器的消费资料的生产,例如加入油脂、肥皂、煤气\fontbox{~\{}用于照明\fontbox{\}~}的生产。可见,甚至在这里,这些领域的产品也都彼此加入对方的不变资本,因而以实物形式互相补偿。

如果把役畜也算作机器,那末,对役畜就需要补偿饲料,并且在一定条件下需要补偿厩舍(建筑物)。但是,饲料加入牲畜的生产,牲畜也加入饲料的生产。

\textbf{第三,辅助材料}。其中一部分,如机油、肥皂、油脂、煤气等等需要原料。另一方面,它们一部分会以肥料等等形式重新加入这种原料的形成过程。制造煤气需要煤炭,而生产煤炭又使用煤气照明,等等。另一些\textbf{辅助材料}只由新加劳动和固定资本(机器、器皿等等)构成。煤炭必须补偿生产煤炭时使用的蒸汽机的损耗。但蒸汽机也消费煤炭。煤炭本身加入煤炭的生产资料。因此,在这里煤炭以实物形式自己补偿自己。煤炭的铁路运输加入煤炭的生产费用,但煤炭又加入机车的生产费用。

\textbf{以后关于化学工厂还要专门补充一下},所有这些工厂在不同程度上制造辅助材料、器皿的原料(例如玻璃、瓷),以及直接加入消费的物品。

一切染料都是辅助材料。但它们不仅按其价值来说会加入产品,例如象工厂中消费的煤炭加入棉布那样,而且会在产品取得的形式上(产品的色彩上)再现出来。

\textbf{辅助材料}可以是\textbf{机器的消费资料},——在这里,它们或者充当发动机的燃料,或者用作减轻工作机摩擦的手段等等,如油脂、肥皂、机油等等,——它们也可以是建筑用的辅助材料,如水泥等等,最后,它们还可以是实现生产过程一般所必需的辅助材料,如照明、取暖等等(在这种情况下,它们就是工人本身为了能够劳动所必需的辅助材料)。

或者,\textbf{这是}一些加入原料形成过程的\textbf{辅助材料},如各种肥料以及原料所消费的一切化学产品。

或者,\textbf{这是}一些加入成品的\textbf{辅助材料},如颜料、漆等等。

\centerbox{※     ※     ※}

\textbf{因而,结果是}:

通过同非个人消费的产品 B 中代表 B 的收入的那部分相交换,即通过同 B 部类一年内的新加劳动相交换,A 补偿了自己的不变资本,即产品的 2/3。但是 A 不补偿 B 的不变资本。B 部类必须用同类新产品以实物形式自己补偿这个不变资本。但 B 部类已经没有任何劳动时间来补偿它们了。因为 B 部类的全部新加劳动时间构成它的收入,因而已经由产品 B 中作为不变资本加入 A 的那一部分来代表了。那末,B 的不变资本怎样补偿呢\fontbox{?}

B 的不变资本部分地通过\textbf{本身的}(植物性的或动物性的)\textbf{再生产}来补偿,农业和畜牧业的所有部门的情形,就是如此;部分地通过一种不变资本的一部分同另一种不变资本的一部分\textbf{以实物形式交换}来补偿,这里,一个领域的产品作为原料或生产资料加入另一个领域,反过来也是如此。可见,这里不同生产领域的产品,[389]不同种类的不变资本,彼此作为生产条件以实物形式加入对方。

非个人消费品的生产者为个人消费品的生产者生产不变资本。但同时,他们的产品还相互充当彼此的不变资本的要素或因素。这就是说,他们相互把彼此的产品用于\textbf{生产}消费。

产品 A 全部由个人消费。因而其中包含的全部不变资本也都由个人消费。(1/3)A 由 A 的生产者消费,(2/3)A 由非个人消费的产品 B 的生产者消费。A 的不变资本由构成 B 的收入的 B 的产品来补偿。事实上,这是不变资本中由\textbf{新加劳动}来补偿的唯一部分,这一部分所以由这种劳动来补偿,是因为代表 B 部类新加劳动的 B 的那个产品量,不由 B 部类消费,相反地由 A 部类用于生产消费,而 B 部类则把(2/3)A 用于个人消费。

如果 A 等于 3 工作日;那末,按照假定,它的不变资本等于 2 工作日。B 补偿产品 A 的 2/3,也就是提供等于 2 工作日的非个人消费品。现在 3 工作日已经被吃掉,还剩下 2 工作日。换句话说,A 的过去的 2 工作日由 B 的新加的 2 工作日补偿,但这只是因为,B 的新加的 2 工作日按价值来说是以产品 A 的形式,而不是以产品 B 本身的形式消费的。

B 部类的不变资本,由于它加入 B 的总产品,也必须由同类的新产品,也就是由 B 部类的\textbf{生产}消费所必需的产品以实物形式补偿。它虽然也是由一年内新花费的劳动时间的\textbf{产品}补偿,但不是由\textbf{新}的劳动时间补偿。

假定在 B 的总产品中,全部不变资本占 2/3。那末,如果新加劳动(等于工资和利润的总额)等于 1,则充当其劳动材料和劳动资料的过去劳动就等于 2。这个 2 是怎样补偿的呢\fontbox{?}在 B 部类的不同生产领域中,可变资本同不变资本之比,可能是极不相同的。但是按照假定,平均比例等于 1/3∶2/3,或 1∶2。B 部类的每个生产者都有 2/3 的产品——如煤炭、铁、亚麻、机器、牲畜、小麦(指不加入个人消费的那部分牲畜和小麦)等等——需要补偿它们的生产要素,或者说,这 2/3 的产品必须再转化为自己生产要素的实物形式。但所有这些产品本身都重新加入生产消费。小麦(作为种子)同时又成了它自己的原料,养大的一部分牲畜补偿消费掉的牲畜,也就是自己补偿自己。这样,在 B 部类的这些生产领域(农业和畜牧业)中,它们的一部分产品就是以自己的实物形式补偿自己的不变资本。可见,这种产品的一部分不进入流通(至少它不一定要进入流通,它可能只是在形式上进入流通)。这些产品中的其他产品,如亚麻、大麻等等,煤炭、铁、木材、机器,都部分地作为生产资料加入自己的生产。就象农业中的种子一样,煤炭加入煤炭的生产,机器加入机器的生产。可见,由机器和煤炭构成的产品的一部分,并且是这一产品中代表它的不变资本的那个份额中的一部分,是自己补偿自己的,它在生产过程中只是改变自己的位置。它不再是产品,而变成了自己的生产资料。

这些和那些产品的其余部分,则彼此作为生产要素相互加入对方:机器加入铁和木材,木材和铁加入机器;机油加入机器,机器加入机油;煤炭加入铁,铁(作为铁轨等等)加入煤炭,等等。这样,B 部类的这些产品的 2/3,就其不是自己补偿自己,即不以自己的实物形式再加入自己的生产这部分(因此,产品 B 的一部分由自己的生产者直接用于生产消费,就象产品 A 的一部分由自己的生产者直接用于个人消费一样)来说,B 部类各生产者的产品是彼此作为生产资料相互补偿的。生产者 a 的产品加入生产者 b 的生产消费,而生产者 b 的产品加入生产者 a 的生产消费,或者通过间接的方式:生产者 a 的产品加入生产者 b 的生产消费,生产者 b 的产品加入生产者 c 的生产消费,而生产者 c 的产品加入生产者 a 的生产消费。这样,在 B 部类的一个生产领域中作为不变资本消费的东西,就在另一个生产领域中重新生产出来,而在后一个生产领域中消费的东西,又在前一个生产领域中生产出来。在一个领域中从机器和煤炭的形式变为铁的形式的东西,在另一个领域中则从铁和煤炭的形式变为机器的形式,等等。

[390]必须使 B 部类的不变资本以实物形式得到补偿。如果考察一下 B 部类的总产品,那末它正好代表各种实物形式的全部不变资本。当 B 部类的某个特殊领域的产品不能以实物形式补偿自己的不变资本时,买和卖,相互转手,在这里会使一切重新就序。

总之,这里发生的是不变资本由不变资本补偿;只要这种补偿不是直接的,不是不经过交换的,那就是\textbf{资本同资本相交换},按使用价值来说就是产品同产品相交换,这些产品彼此加入对方相应的生产过程,因而每个这样的产品都由相应的其他产品的生产者用于生产消费。

这部分资本既不归结为利润,也不归结为工资。它不包含任何新加劳动。它不同收入交换。它既不直接也不间接由消费者支付。资本的这种相互补偿不论有没有商人(即商人资本)作中介,都丝毫不会使问题有所改变。

但是,既然这些产品(彼此相互补偿的机器、铁、煤炭、木材等等)是新的产品,既然它们是当年劳动的产品,例如用作种子的小麦,完全象加入个人消费的小麦一样,是新劳动的产品,等等,那又怎么能够说,这些产品中不包含任何新加劳动呢\fontbox{?}此外,它们的形式不是十分令人信服地表明情况恰好相反吗\fontbox{?}如果说在小麦或牲畜身上还看不出这一点,那末机器,机器的形式,却直接证明了把它从铁等等变成机器的那种劳动。如此等等。

这个问题在前面已经解决了。\authornote{见本册第 89—140、182—195 页和第 220—221 页。——编者注}这里不需要回过头去再谈了。

\fontbox{~\{}可见,亚·斯密认为“实业家”和“实业家”之间的贸易规模同“实业家”和消费者(消费者是指直接消费者,不是指生产消费者,因为斯密本人把生产消费者列入“实业家”的范畴)之间的贸易规模相等,这个命题是错误的。这个命题建立在他的一个错误的论点上,按照这个论点,全部产品都归结为收入,而事实上这不过是说,由资本和收入的交换构成的那部分商品交换,等于全部商品交换。因此,图克根据这个命题对于货币流通(特别是对于“实业家”之间流通的货币量同“实业家”和消费者之间流通的货币量之比)所做出的实际结论,和这个命题一样,也是错误的。

如果我们把购买产品 A 的商人当作同消费者对立的最后一个“实业家”,那末,这些产品在他手里,就会由生产者 A 的收入(等于(1/3)A)和生产者 B 的收入(等于(2/3)A)买去。这些收入补偿他的商人资本。这些收入的总和必定抵补他的资本。(这个家伙赚到的利润必定是这样得来的:他把一部分 A 留给自己,把较小量的 A 按照全部 A 的价值来出卖。无论是把这个家伙看作必要的生产当事人,还是把他看作寄生的中间人,都完全不会使问题有所改变。)经营产品 A 的“实业家”和产品 A 的消费者之间的交换,按其价值来说,则抵补经营产品 A 的人和所有参加产品 A 生产的人之间的交换,因而抵补这些生产者相互之间的全部交易。

商人购买麻布。这是“实业家”和“实业家”之间的最后一次交易。麻布织造业者购买纱、机器、煤炭等等。这是“实业家”和“实业家”之间的倒数第二次交易。纺纱业者购买亚麻、机器、煤炭等等。这是“实业家”和“实业家”之间的倒数第三次交易。亚麻种植业者和机器制造业者购买机器、铁等等,依此类推。但是亚麻、机器、铁、煤炭的生产者之间为补偿他们的不变资本而进行的交易以及这些交易的价值,都不加入产品 A 所经过的那些交易(不管这是收入和收入之间的交换,还是收入和不变资本之间的交换)。这些交易,——不是 B 的生产者和 A 的生产者之间的交易,而只是 B 的生产者们相互之间的交易,——就象产品 B 的这一部分的价值完全不加入产品 A 的价值一样,完全不需要由产品 A 的买者对产品 A 的卖者进行补偿。这些交易也需要货币,也要以商人为中介。但专门属于这个领域的那部分货币流通同“实业家”和消费者之间的货币流通是完全分开的。\fontbox{\}~}

[391]剩下要解决的还有两个问题:

(1)在以上的论述中,我们把工资看作收入,没有把它同利润区分开来。现在要问,工资同时表现为资本家的流动资本的一部分这种情况,在这里会有多大的意义\fontbox{?}

(2)直到现在我们假定,全部收入都作为收入花掉。因此,应当考察收入即利润的一部分化为资本时发生的变化。这事实上同对积累过程的考察——但不是从它的形式上考察——是一致的。代表剩余价值的那部分产品一部分再转化为工资,一部分再转化为不变资本,这是很简单的。但这里必须研究,这种情况怎样影响前面所分析的各个项目的商品交换,——在这些项目下,商品交换可以从它的承担者方面来考察,——即:收入同收入的交换,收入同资本的交换,以及资本同资本的交换。\fontbox{\}~}

\fontbox{~\{}这样,这一幕间曲就必须穿插在这个历史批判部分,一直演奏到结束。\endnote{马克思在他的手稿第 X 本中,由于分析魁奈的《经济表》,又回过头来分析了这“幕间曲”中涉及的某些问题(见本册第 6 章)。而对前面所提出的两个问题,在《资本论》第二卷(特别是在第二十章第十节《资本和收入:可变资本和工资》,以及第二十一章《积累和扩大再生产》)作了详细和系统的回答。马克思在《剩余价值理论》第二册论李嘉图的积累理论一章中,又回过头来分析了这一“幕间曲”中所分析的问题。马克思在《剩余价值理论》第三册《反对政治经济学家的无产阶级反对派》一章(批判分析匿名小册子《根据政治经济学基本原理得出的国民困难的原因及其解决办法》)和论舍尔比利埃一章(论述作为扩大再生产的积累问题),又回过头来分析了资本和收入之间的交换问题。——第 258 页。}\fontbox{\}~}

\tsectionnonum{[(11)]费里埃[费里埃对斯密的生产劳动和资本积累理论的反驳的保护关税性质。斯密在积累问题上的混乱。斯密关于“生产劳动者”的见解中的庸俗成分]}

弗·路·奥·费里埃(\textbf{海关副督察})著有《论政府和贸易的相互关系》(1805 年巴黎版)一书。(这本书是弗·李斯特论据的主要来源。)此人是\textbf{波拿巴王室}的禁止性关税制度等等的赞颂者。实际上,他认为政府(因而国家官吏这些非生产劳动者)具有重要意义,说政府是直接干预生产的领导者。所以,这个海关官吏对亚·斯密把国家官吏叫做非生产劳动者这一点非常恼火。

\begin{quote}“斯密\textbf{确定的国家节约的}原则,是以生产劳动和非生产劳动之间的区分为根据的……”\end{quote}

\fontbox{~\{}这正是因为斯密希望,将产品中尽可能大的部分当作资本花费,即用来同生产劳动交换,而将尽可能小的部分当作收入花费,用来同非生产劳动交换。\fontbox{\}~}

\begin{quote}“这一区分实质上是错误的。\textbf{根本就没有非生产劳动}。”(第 141 页)“因此,国家有节约和浪费之别;但国家是浪费还是节约,只能从该国同\textbf{他}国的关系来看,问题也正是必须这样来看。”(同上,第 143 页)\end{quote}

现在我们把费里埃所憎恨的亚·斯密的论断拿来与此对照一下。

\begin{quote}费里埃说:“国家的节约是有的,但跟斯密所说的完全不同。国家的节约在于,购买外国产品的数量不超过能用本国产品支付的限度。有时这种节约在于完全不要外国产品。”(同上,第 174—175 页)\end{quote}

\fontbox{~\{}\textbf{亚·斯密}在第一篇第六章(加尔涅的译本,第 1 卷第 108—109 页)《论商品价格的构成部分》的结尾说:

\begin{quote}“因为在一个文明国家里,\textbf{只有极少数商品的交换价值仅由劳动产生,绝大多数商品的交换价值中有大量地租和利润加入},所以,\textbf{这个国家的劳动的年产品所能购买和支配的劳动量,比这个产品的制造、加工和运到市场所必须使用的劳动量总要大得多}。如果\textbf{社会每年使用了它每年所能购买的全部劳动,那末,由于这个劳动量逐年会有很大的增加},每一年的产品就会比上一年的产品具有大得多的价值。但是,没有一个国家\textbf{会把全部年产品都用于}工人的生活费。在任何地方,产品很大一部分都是归有闲者消费的。年产品的一般的或平均的价值,究竟是增加,减少还是年年不变,就必然要看这个产品按怎样的比例在这两个不同的阶级之间分配。”\end{quote}

在斯密实际上想解开积累之谜的这段话里,各式各样的混乱看法是不少的。

首先,我们在这里又看到那个错误的前提:劳动的年产品的“交换价值”,也就是“\textbf{劳动的年产品}”,全部分解为工资和利润(地租也包括在内)。我们不想回过头来谈这个荒谬的观点。我们只想指出下面一点。年产品总量——或构成劳动的年产品的商品总额、商品储备,——按其实物形式来说,很大一部分[392]是由只能作为不变资本的要素\fontbox{~\{}各种原料、种子、机器等等\fontbox{\}~}加入不变资本的,即只能用于生产消费的商品构成的。关于这些商品(而这是加入不变资本的大部分商品),它们的\textbf{使用价值}本身就已经表明,它们不能用于个人消费,因而收入——无论是工资,利润还是地租——不能花在它们身上。固然,一部分原料(只要不是这些原料本身的再生产所必需的,或者不是作为辅助材料或作为直接组成部分加入固定资本的)在以后会取得可消费的形式,但这只是由于加上了当年劳动。作为去年劳动的产品,甚至这些原料也不能成为收入的任何一部分。只有产品的可消费部分才能被消费,才能加入个人消费,从而才能构成收入;但是,甚至可消费的产品也有某一部分不能被消费,否则就会使再生产成为不可能。因此,就连商品的可消费部分中,也要拿出一部分来,这一部分必须用于\textbf{生产消费},即必须成为劳动材料、种子等等,不能成为生活资料——不管是工人的还是资本家的生活资料。因而这部分产品一开始就必须从亚·斯密的计算中扣除,或者更确切地说,必须加入这个计算。只要\textbf{劳动生产率保持不变,产品中}不分解为收入的那个部分也每年不变,也就是说,只要劳动生产率保持不变,花费在生产上的劳动时间量也和以前一样。

如果假定每年使用比以前\textbf{更大的}劳动\textbf{量},那就必须考察在这种情况下不变资本的状况。有一点是无疑的:为了能够使用更大的劳动量,单单支配\textbf{更大的劳动量}并\textbf{支付这个更大的量},即把更多的资金用在工资上,是不够的,还必须拥有可以吸收这个更大劳动量的劳动资料(原料和固定资本)。因此,在阐明亚·斯密考察的各点\textbf{之后},还必须把这一点分析一下。

这样,让我们再一次看看他的第一句话:

\begin{quote}“因为在一个文明国家里,只有极少数商品的交换价值\textbf{仅由劳动产生},绝大多数商品的交换价值中\textbf{有大量地租和利润加入},所以,\textbf{这个国家的劳动的年产品所能购买和支配的劳动量},比\textbf{这个产品的制造}、加工和运到市场〈换句话说,产品的生产〉\textbf{所必须使用的劳动量}总要大得多。”\end{quote}

这里,显然是把各种不同的东西混杂在一起了。加入全部年产品的交换价值的不仅有活劳动,即当年耗费的活劳动,而且还有过去劳动,即往年劳动的产品。不仅有活的形式的劳动,而且有物化形式的劳动。产品的交换价值等于产品所包含的劳动时间的总和,其中一部分由活劳动构成,一部分由物化劳动构成。

假定活劳动和物化劳动之比是 1/3∶2/3,即 1∶2。那末,全部产品的价值就等于 3,其中 2 是物化劳动时间,1 是活劳动时间。因此,如果只从物化劳动和活劳动作为等价物相互交换,一定量的物化劳动只能支配等量的活劳动这个前提出发,那末,全部产品的\textbf{价值}所能购买的活劳动,就比它本身包含的活劳动多。因为产品等于 3 工作日,而它包含的活劳动时间等于 1 工作日。为了生产产品(实际上不过是为了使产品的要素具有最终形式),只要 1 日的活劳动就够了。但是,产品中包含 3 工作日。可见,如果把这个产品全部用来同活劳动时间交换,如果只把它用来“购买和支配”活劳动量,那末它就能支配、购买 3 工作日。

然而亚·斯密指的显然不是这个意思;在他看来,这是一个完全无用的前提。他想说的是:产品的交换价值有很大一部分不分解为工资,而分解为利润和地租,或者为了简单起见,我们可以说分解为利润(斯密不用“分解”这个词,他用的是另一种\textbf{错误的}说法,这是由我们在前面已经指出过的概念的混淆\authornote{见本册第 75—78 页。——编者注}造成的)。换句话说:产品中和当年加进的劳动量相等的那部分价值——实际上就是真正由当年劳动生产的那部分产品——第一,支付工人,第二,加入资本家的收入,加入资本家的消费基金。总产品的这部分全部由劳动产生,并且仅仅由劳动产生;但它包括有酬劳动和无酬劳动。工资等于有酬劳动的总和,利润[393]等于无酬劳动的总和。因此,如果把这全部产品都花在工资上,它所能推动的劳动量,自然就会比生产这个产品的劳动量大;而且,产品所能推动的更大的劳动时间量和产品本身包含的劳动时间量的比例,恰好决定于工作日分为有酬劳动时间和无酬劳动时间的比例。

假定有酬劳动时间和无酬劳动时间的比例是:工人在 6 小时即半个工作日内生产或再生产自己的工资。其余 6 小时即半个工作日,形成剩余时间。因此,有一个产品例如包含 100 工作日[新加劳动],这 100 工作日等于 50 镑(如果 1 工作日等于 10 先令,那末 100 工作日就等于 1000 先令,即 50 镑)。其中 25 镑为工资,25 镑为利润(地租)。用这笔等于 50 工作日的 25 镑,可以支付 100 工人,这 100 工人有一半劳动时间是进行无代价的劳动,换句话说,就是为自己的老板劳动。因此,如果把这全部产品(100 工作日)都花在工资上,那末 50 镑就能推动 200 工人,他们每一个人都象以前一样,得到工资 5 先令,即自己劳动产品的一半。这些工人的产品是 100 镑(也就是说,200 工作日等于 2000 先令,即 100 镑),用这 100 镑又能推动 400 工人(每个工人得到 5 先令,400 工人得到 2000 先令),他们的产品等于 200 镑,依此类推。

亚·斯密说“劳动的年产品所能购买和支配的劳动量”,比产品的生产所使用的劳动量总要“大得多”,就是指这个意思。(如果把工人劳动的全部产品都支付给工人,也就是说,如果是 100 工作日,就支付给他 50 镑,那末这 50 镑也就只能推动 100 工作日。)正是在这个意义上,亚·斯密接着说:

\begin{quote}“如果社会每年使用了它每年所能购买的全部劳动,那末,由于这个劳动量逐年会有很大的增加,每一年的产品就会比上一年的产品具有大得多的价值。”\end{quote}

但是,这个产品的一部分被利润和地租的所有者吃掉,另一部分被他们的食客吃掉。因此,能够重新用在劳动(生产劳动)上的那部分产品究竟有多少,就取决于产品中没有被资本家、租金所得者和他们的食客(同时也是非生产劳动者)吃掉的那部分究竟有多少。

然而,这样一来,这里总还有一笔新的基金(新的工资基金),以便用去年劳动的产品在本年推动更多的工人。因为年产品的价值决定于所花费的劳动时间量,所以年产品的价值将会逐年增长。

不言而喻,如果市场上没有更大量的劳动,即使有一笔基金,它能够“\textbf{购买和支配}的劳动量”比去年“大得多”,也是没有用处的。即使我有更多的货币可以购买某种商品,如果市场上没有更多的这种商品,对我也是没有什么用处的。假定从 50 镑中拿出一个数目,这个数目不是推动 200 工人以代替原先的 100 工人(他们得到 25 镑),而是只推动 150 工人,这时,资本家自己吃掉的是 12+(1/2)镑,而不是 25 镑。在这种情况下,150 工人(他们得到 37+(1/2)镑)就会提供 150 工作日,即等于 1500 先令或 75 镑。但是,如果可使用的工人人数照旧只有 100 人,那末,这 100 人现在得到的工资就会是 37+(1/2)镑,而不是原先的 25 镑,可是他们的产品仍然只有 50 镑。这样一来,资本家的收入就会从 25 镑降到 12+(1/2)镑,因为工资增加了 50\%。但是亚·斯密知道,要增加的劳动量是会有的。一方面是由于人口每年增长(诚然,按照斯密的意见,原有的工资总额是这种增长的前提)。另一方面是由于存在着失业的赤贫者、半失业的工人等等。其次,大量非生产劳动者中间有一部分人因剩余产品使用上的改变而能够变成\textbf{生产}工人。最后,同样数量的工人可以提供\textbf{更大的}劳动\textbf{量}。因为我雇用 125 工人来代替 100 工人,或者让 100 工人每天劳动 15 小时而不是劳动 12 小时,是完全一样的。

此外,说随着生产资本的增加,——或者说随着用于再生产的那部分年产品的增加,——\textbf{所使用的劳动}(活劳动,花费在工资上的那部分资本)也会按同样的比例增加,这是亚·斯密的错误,这个错误同他认为全部产品分解为各种收入的观点有着最密切的联系。

[394]总之,斯密首先肯定说,有可供个人消费的生活资料基金,这个基金在本年内能够“购买和支配”的劳动量比去年大。有更多的劳动,同时又有供这个劳动用的更多的生活资料。现在应当考察一下,这个追加的劳动量如何实现。\fontbox{\}~}

如果亚·斯密完全自觉地、始终一贯地坚持他实质上已有的那种对剩余价值的分析,即认为剩余价值只有在资本同雇佣劳动的交换中才会创造出来,那末,他就会发现,只有同资本交换的劳动才是生产劳动,而同收入本身交换的劳动决不是生产劳动。为了同生产劳动交换,收入必须先转化为资本。

但斯密同时又从片面的传统观点出发,认为生产劳动就是一般直接生产物质财富的劳动;并且把自己的区分(根据资本同劳动交换还是收入同劳动交换作出的区分)同这种观点结合起来,所以在他看来,下面这样的定义是可能的:同资本交换的那种劳动始终是生产劳动(始终创造物质财富等等);而同收入交换的那种劳动既可能是生产劳动,也可能是非生产劳动,但是,花费自己收入的人,在大多数情况下,都宁愿使用某种直接的非生产劳动,而不愿使用生产劳动。这里可以看出,亚·斯密由于把自己的两种区分混在一起,就把主要的区分大大削弱并冲淡了。

下面这段引文表明,亚·斯密并没有把劳动的固定化完全归结为纯粹的表面的固定化;在这段引文中,有他列举的固定资本各个组成部分中的一条:

\begin{quote}“(4)居民或社会成员所获得的有用才能。要获得这种才能,总得支出一笔实在的费用,供获得才能的人在他受教育、实习或学习期间维持生活,而这笔费用可以说就是固定和物化在他个人身上的资本。这种才能是他的财产的一部分,也是他所在的那个社会的财产的一部分。可以把工人的提高了的技能,同减轻和缩短劳动的机器或工具一样看待,在这些东西上虽然要支出一笔费用,但它们会偿还这笔费用,并提供利润。”(加尔涅的译本,第 2 卷第 204—205 页)\end{quote}

\textbf{奇怪的积累来源和积累的必要性}:

\begin{quote}“在社会的原始状态中,没有任何分工,几乎不发生交换,每一个人都用自己的手去谋得他所需要的一切东西。在这种状态中,\textbf{没有必要为了维持社会经济生活而把资财预先积累或储存起来}”\end{quote}

(其实这里一开始就假定不存在任何社会)。

\begin{quote}“每一个人都努力以自己的活动来获得满足自身随时产生的需要的手段。他饿了,便到森林去打猎”等等。(同上,第 2 卷第 191—192 页)(第 2 篇\textbf{序论})“但是,一旦分工普遍实行,一个人用他个人的劳动就只能满足他当时产生的需要的极小部分。他的需要的绝大部分都要靠\textbf{别人劳动的产品}来满足,他用自己劳动的产品,或者说用自己产品的价格去购买别人劳动的产品。但是,在实行这种\textbf{购买}之前,他必须有时间不仅\textbf{完全制成}并且\textbf{还要卖掉他的劳动产品}。”\end{quote}

(即使在前一种场合,他不先打到兔子,也吃不到兔肉,而他不先制成古“弓”或类似的东西,就不可能打到兔子。所以在后一种场合,唯一的新条件并不是必须有什么“储存”,而是必须“有时间……\textbf{卖掉}他的劳动产品”。)

\begin{quote}“因此,至少在他能够完成这两件事以前,必须在某个地方\textbf{预先储存各种物品},以维持他的生活,并供给他劳动所必需的原料和工具。一个织布业者,在他把麻布织成并且卖掉以前,如果\textbf{在他手里或别人手里}没有预先\textbf{储存}的物品,以维持他的生活,并供给他劳动所需的工具和材料,他是\textbf{不能全力从事}自己的专业的。十分明显,在他能够从事这项工作并把它完成\textbf{以前,必须有积累}……按照事物的本性,\textbf{\CJKunderdot{资本}的积累是分工的必要的先决条件}。”(同上,第 192—193 页)\end{quote}

(另一方面,按照斯密在序论中的第一句话,好象在分工\textbf{以前}没有任何资本积累,而现在他却完全相反,断言在资本积累以前没有任何分工。)

斯密继续说道:

\begin{quote}“只有预先积累的资本愈来愈多,分工才会愈来愈细。分工愈细,\textbf{同样数目的人所能加工的原料数量就会大大增加};因为这时每一个工人的操作愈来愈简单,所以减轻和[395]缩短劳动的新机器就大量发明出来。因此,随着分工的发展,为了经常雇用同样数目的工人,就必须\textbf{预先积累同样多的生活资料},以及比分工不发达时\textbf{更多的原料和劳动工具}。”(同上,第 193—194 页)“劳动生产力的大大提高,\textbf{非有预先的资本积累}不可,同样,资本的积累也自然会引起劳动生产力的大大提高。\textbf{凡是使用自己的资本来雇用工人的人}当然希望,他这样做会使工人完成尽可能多的工作。因此,他力求在自己的工人中间最恰当地进行分工,并把他所能发明或购买的最好的机器供给工人使用。他在这两方面成功的可能性如何,通常要看他有多少资本,或者说,要看这个资本能够雇用多少工人。因此,\textbf{在一个国家里},不仅\textbf{劳动量随着}推动劳动的\textbf{资本的扩大而增加},而且同一\textbf{劳动量所生产的产品,也由于资本的扩大而大大增加}。”(同上,第 194—195 页)\end{quote}

亚·斯密完全象他论述生产劳动和非生产劳动那样,论述已经加入消费基金的物品。例如,他说:

\begin{quote}“住房不会给居住者带来任何收入;虽然这所住房无疑对他说来是非常有用的,但这不过是象他的衣服和家具一样,衣服和家具对他说来也是十分有用的,但这不过是他的开支的一部分,不是收入的一部分。”(同上,第 2 卷第 201—202 页)相反,属于固定资本的有“一切有用的建筑物,它们不仅对收取租金的建筑物所有者来说,是获得收入的手段,甚至对支付租金的建筑物承租人来说,也是获得收入的手段;例如店铺、仓库、工场以及有各种必要的设备、厩舍、粮仓等等的租地农场,就是如此。这种建筑物和纯粹的住房大不相同;它们是一种生产工具”。(同上,第 203—204 页)(第 2 篇第 1 章)“一切技术成就,使得同一数量的工人能够用比以前更简单、更便宜的机器来完成同样的工作量,这始终被认为是对社会很有利的。以前用来维持较复杂、较昂贵的机器的一定数量的原料和一定数量的工人,现在就可以用来增大工作量,而这些或那些机器是为了进行这种工作而制造出来的。”(同上,第 2 卷第 216—217 页)(第 2 篇第 2 章)“\textbf{固定资本}的维持费……要从社会纯收入中排除掉。”(同上,第 2 卷第 218 页)“不减低劳动生产力的\textbf{固定资本}的维持费的任何节约,就必定会增加推动企业的基金,因而必定会增加土地和劳动的年产品,增加每个社会的实际收入。”(同上,第 2 卷第 226—227 页)被银行券(一般说,纸币)排挤到国外的金银币,——如果花在“购买供国内消费的外国货”上,——或者用来购买奢侈品,如外国葡萄酒、丝织品等等,一句话,购买“供什么也不生产的……\textbf{有闲者}消费的商品……或者……用来购买\textbf{追加的原料、劳动工具和生活资料,以维持和雇用追加的勤劳者,这些勤劳者会把他们每年消费的价值再生产出来,并提供一笔利润}”。(同上,第 2 卷第 231—232 页)斯密说,前一种使用货币的方法增加浪费,“增加开支和消费,丝毫不会增加生产,也不会创造抵补这些开支的固定基金,所以从各方面来讲,对社会都是有害的”。(同上,第 2 卷第 232 页)相反,“用后一种方法支出的货币,就相应地扩大生产规模,尽管它也增加社会的消费,但是它开辟了维持消费的固定来源,因为\textbf{消费这一生活资料的追加量的人会把他们每年消费的全部价值再生产出来,并提供一笔利润}”。(第 2 卷第 232 页)“一笔资本能推动多少生产劳动,显然要看它能给多少工人提供符合他们劳动性质的原料、劳动工具和生活资料。”(同上,第 2 卷第 235 页)(第 2 篇第 2 章)\end{quote}

[396]我们\textbf{在第二篇第三章}(同上,第 2 卷第 314 页及以下各页)读到:

\begin{quote}“生产劳动者、非生产劳动者以及那些根本不劳动的人,同样都是靠该国土地和劳动的年产品维持生活。这种产品……必然是有限的。因此,根据一年内用来维持非生产劳动者的那部分产品是较少或者较多的不同情况,为生产劳动者留下的产品就会是较多或者较少,与此相适应,下一年的产品也会增加或减少……虽然每一个国家的土地和劳动的全部年产品……归根到底都是供国内居民消费,并给他们带来收入,\textbf{但是,当}产品从土地或从生产工人手中生产出来\textbf{的时候},它就自然分成两部分。其中一部分,而且往往是最大的部分,首先用来\textbf{补偿资本,更新那些}已经从资本中取出的\textbf{\CJKunderdot{生活资料}、原料和成品};另一部分则用来形成收入,——或是作为这个资本的所有者的资本的利润,或是作为另一个人的土地的地租……\textbf{每一个国家的土地和劳动的年产品中补偿资本的那一部分},决不能直接用来维持生产工人以外的其他任何雇佣人员;这一部分只能给生产劳动支付工资。直接形成收入的那部分产品……既可以用来维持生产劳动者,也可以用来维持非生产劳动者……非生产劳动者和那些根本不劳动的人,都\textbf{靠收入}维持生活;或者,第一,靠年产品中一开始就形成某些人的收入(不是作为地租,便是作为资本利润)的那一部分;或者第二,靠年产品中的另一部分,这一部分虽然是供补偿资本和仅仅维持生产工人用的,但一到生产工人手里,除了维持他们的生活所必需的部分之外,其余部分就能用来既维持生产人员,也维持非生产人员。例如,一个普通工人,如果他的工资高,他就能……雇个仆人,或者有时去看看喜剧或木偶戏,这样,他就用自己的一部分收入来帮助维持一类非生产劳动者;或者最后,他能交纳一些税,从而帮助维持另一类……同样是非生产的劳动者。但是,一开始就供补偿资本用的那部分年产品,在它没有把与它相应的生产劳动量全部推动之前,是决不能用来维持非生产劳动者的……工人必须先做工,挣得了工资,然后才能把哪怕是极小的一部分收入,支出在非生产劳动上……地租和资本利润……在任何地方,都是非生产劳动者赖以取得生活费的主要源泉……这两种收入,既可以维持生产劳动者,也可以维持非生产劳动者;但是,这些收入的所有者,看来总是更喜欢把它们用在后者身上……总之,年产品从土地或从生产工人手中生产出来以后,一部分供补偿资本用,另一部分形成收入(作为地租或利润)。在每一个国家里,生产劳动者和非生产劳动者之间的比例,主要决定于年产品的这两部分之间的比例,而这个比例在富国和贫国是极不相同的。”\end{quote}

斯密接着把情况作了对比:

\begin{quote}“在欧洲各富国”,今天“土地产品的很大一部分,而且往往是最大的部分,\textbf{都用来补偿富有的独立的租地农场主的资本}”;过去的情况则相反,“在封建制度统治时期,产品的极小部分就足以补偿耕地使用的资本”。商业和工业中的情形也是这样。现在商业和工业中使用大资本;而以前,资本是极小的,但它们带来的利润很大。“利息在任何地方都不低于 10\%,要支付如此高的利息,资本利润必定非常大。目前在欧洲比较发达的国家,利息在任何地方都不超过 6\%,而在最富的国家,利息则等于 4\%、3\%、2\%。居民由利润得来的那部分收入,在富国总是比在贫国大得多,这是因为富国的资本大得多;但利润同资本相比,富国的利润通常又低得多。由此可见,年产品从土地或从生产工人手中生产出来以后,供补偿[397]资本用的那部分,在富国不仅比在贫国大得多,而且同直接形成收入(作为地租或利润)的那部分相比,也大得多。用来维持生产劳动的基金,在富国不仅比在贫国大得多,而且同那种既能用来维持生产劳动者,又能用来维持非生产劳动者,但通常主要是用来维持后者的基金相比,也大得多。”\end{quote}

(斯密犯了这样的错误:他把生产资本的量同用来维持生产劳动的\textbf{那部分生产资本的量}等同起来。但这同他所了解的大工业实际上还只处在萌芽状态有关系。)

\begin{quote}“这两种不同基金之间的比例,在每一个国家,必然会决定该国居民的一般性格是勤劳,还是懒惰。”斯密说,例如,“在英国和荷兰的工业城市里,人民的下层阶级主要依靠所使用的资本过活,他们一般来说都是勤劳的、刻苦的和节俭的。相反,在宫廷所在地的都城等等,人民的下层阶级依靠上层阶级挥霍收入来生活,他们一般来说都是懒惰的、放荡的和贫困的;例如,罗马、凡尔赛等地就是这样……”“由此看来,资本总额和收入总额之间的比例,在任何地方都决定勤劳和懒散之间的比例:在资本占优势的地方,多勤劳;在收入占优势的地方,多懒散。因此,\textbf{资本量的每一增减},自然会引起生产活动量、生产工人人数的实际的增减,从而引起该国土地和劳动年产品的交换价值、该国全体居民的财富和实际收入的增减……一年内节约下来的东西,象一年内支出的东西一样,照例是会被消费的,而且几乎在同时被消费;不过,它是被另一类人消费的。一年内支出的那部分收入,由家仆、无用的食客等等消费,这些人决不会留下任何一点东西来补偿他们的消费。而一年内节约下来的那部分收入,由工人消费,这些工人会把自己一年消费的价值再生产出来,并提供一笔利润……消费是一样的,但消费者不一样……”\end{quote}

以下就开始了斯密关于节约的人的说教(同一章,下面第 328、329 页及以下各页),他说这种人靠自己每年的节约,可以为追加的生产工人建立一个公共工场,

\begin{quote}“设立一种永久的基金来维持相应数量的生产工人”,而“浪费者却使维持生产劳动的基金总数减少……如果把非生产人员这样〈由于浪费者挥霍〉消费的食物和衣服,分配给生产工人,这些生产工人就会把他们所消费的全部价值\textbf{再生产出来,并提供一笔利润}……”\end{quote}

斯密的这种说教的结语是:这(节约和浪费)会在私人中间相抵,并且实际上“理智”占上风。大国

\begin{quote}“从来不会因私人的浪费和妄为而变穷,虽然有时会因政府的浪费和妄为而变穷。在大多数国家,国民收入全部或几乎全部用来维持非生产人员。这些人包括宫廷人员、教会人士、海军、陆军,他们在平时什么也不生产,在战时也不能获得任何东西,来抵偿他们即使只是在战争期间的生活费用。\textbf{这种人自己什么也不生产,全靠别人的劳动产品来养活}。因此,如果他们人数的增加超过了必要的数量,他们在一年内就能消费很大一部分产品,以致剩下来的产品不足以维持必须在下一年把产品再生产出来的生产工人……”[同上,第 2 卷第 314—336 页]\end{quote}

斯密在第二篇第四章写道:

\begin{quote}“因为用来维持生产劳动的基金逐日增加,所以对生产劳动的需求也与日俱增:工人[398]容易找到工作,而资本所有者却难以找到他们能够雇用的工人。资本家的竞争使工资提高,利润下降。”(同上,第 2 卷第 359 页)\end{quote}

斯密在第二篇第五章《\textbf{论资本的各种用途}》中,根据各种资本所雇用的生产劳动量的大小,从而按照它们所增加的年产品“交换价值”的多少,对资本进行了分类。斯密放在第一位的是\textbf{农业},其次是\textbf{制造业},然后是\textbf{商业},最后是\textbf{零售商业}。这就是斯密根据资本所推动的生产劳动量排列的资本用途的顺序。这里我们又得到一个关于“生产劳动者”的全新的定义:

\begin{quote}“凡是把资本用于这四种用途之一的人,自己就是\textbf{生产劳动者}。他们的劳动,如果使用得当,会固定和物化在它所加工的物品或商品上,通常至少也会把他们维持自己生活和个人消费的价值加在商品的价格上。”(同上,第 2 卷第 374 页)\end{quote}

(总之,斯密把他们的生产性归结为他们推动生产劳动这一点。)

关于\textbf{租地农场主},他说:

\begin{quote}“没有一个同量的资本能比租地农场主的资本推动更大量的\textbf{生产劳动}。不仅他的雇工是生产劳动者,而且\textbf{他的役畜也是生产劳动者}。”[同上,第 2 卷第 376 页]\end{quote}

可见,最后连牛也成了生产劳动者。

\tsectionnonum{[(12)]罗德戴尔伯爵[把统治阶级看成各种最重要生产劳动的代表的辩护论观点]}

\textbf{罗德戴尔(伯爵)}《论公共财富的性质和起源》1804 年伦敦版(法译本:Recherchessurlanatureetl’originedelarichessepubliqueetc.,1808 年巴黎版)。

罗德戴尔提出的为利润辩护的理由,要放到后面第三篇\endnote{马克思这里说的“第三章”是指关于“资本一般”的研究的第三部分。这一章的标题应为:《资本的生产过程和流通过程的统一,或资本和利润》。以后(例如,见第 IX 本第 398 页和第 XI 本第 526 页)马克思不用“第三章”而用“第三篇”(《dritterAbschnitt》)。后来他就把这第三章称作“第三册”(例如,在 1865 年 7 月 31 日给恩格斯的信中)。关于“资本一般”的研究的“第三章”马克思是在第 XVI 本开始的。从这“第三章”或“第三篇”的计划草稿(见本册第 447 页)中可以看出,马克思打算在那里写两篇专门关于利润理论的历史补充部分。但是马克思在写作《剩余价值理论》的过程中,就已在自己的这一历史批判研究的范围内,详细地批判分析了各种资产阶级经济学家对利润的看法。因此,马克思在《剩余价值理论》中,特别是在这一著作的第二册和第三册中,就已进一步更充分地揭示了由于把剩余价值和利润混淆起来而产生的理论谬误。——第 7、87、272 页。}去考察。按照这种辩护论观点,利润是由资本本身产生的,因为资本“\textbf{代替}”劳动。资本之所以得到报酬,是因为它做了人没有它就得自己去做的事,或者做了人不借助于它就根本做不到的事。

\begin{quote}“现在很明白,资本利润的取得,总是或者因为资本代替了人必须用自己的手去完成的劳动;或者因为资本完成了人的个人力量不能胜任和人自己不能完成的劳动。”(法译本第 119 页)\end{quote}

“伯爵”先生极力反对斯密关于积累和节约的学说。他也极力反对斯密提出的对\textbf{生产劳动者和非生产劳动者}的区分;但是,按照他的意见,斯密叫做“劳动生产力”的东西只不过是“资本生产力”。他直接否认斯密提出的对剩余价值的解释,理由是:

\begin{quote}“如果对资本利润的这种理解真正正确的话,那就会得出结论说:利润不是收入的原始源泉,而只是派生源泉,并且,决不能把资本看作财富的源泉之一,因为资本带来的利润不过是收入从工人的口袋转到资本家的口袋而已。”(同上,第 116—117 页)\end{quote}

显然,在这种前提下,罗德戴尔在同斯密的论战中,抓住的也是最肤浅的东西。例如,他说:

\begin{quote}“由此可见,同一种劳动可以是生产的,也可以是非生产的,这要看劳动所加工的那个物品以后的用途如何。例如,如果我的厨师制成一个大蛋糕,我马上把它吃掉,那末,他就是非生产劳动者,他干的活就是不生产的劳动,因为他的服务一经提供随即消失。但如果这种劳动是在糕点店里完成的,那末它就成了生产劳动。”(同上,第 110 页)\end{quote}

(这里\textbf{加尔涅}应享有专利权,因为他出版的那本附有他的注释的斯密著作,是在 1802 年,即比罗德戴尔的书早两年问世的。)

\begin{quote}“这种新奇的区分,仅仅以所提供的服务的耐久性为根据,它把那些在社会上担任最重要职务的人,都归到非生产劳动者里面去。君主、一切神职人员、司法人员、国家保卫者以及一切用自己的技能……保护国民健康或使国民受到教育的人,都被视为非生产劳动者。”(同上,第 110—111 页)(或者象亚·斯密排列的一个很好的次序:“教士、律师、医生、各种文人;演员、丑角、音乐家、歌唱家、舞蹈家等等。”\authornote{加尔涅的译本,第 2 卷第 313 页。})“如果承认交换价值是财富的基础,那末,就没有必要用冗长的议论来证明这个学说的错误。最能[399]证明这个学说错误的是,如果根据这些服务所取得的报酬来判断,人们对这些服务是尊敬的。”(\textbf{罗德戴尔},同上第 111 页)\end{quote}

其次:

\begin{quote}“制造业工人的劳动固定和物化在某种可以出卖的商品上……\textbf{仆人的劳动}也好,由流动资本节约的劳动也好\fontbox{~\{}罗德戴尔在这里所说的“流动资本”是指\textbf{货币}\fontbox{\}~},当然都不能形成积累,不能形成以一定价值从一个人手里转到另一个人手里的基金。它们所提供的利益,同样都是由于它们\textbf{节约主人}或所有者的\textbf{劳动}造成的。既然它们产生如此相同的结果,那末,把其中一个称为非生产劳动,也就必然要把另一个称为非生产劳动。”\fontbox{~\{}他接着引了斯密在第二篇第二章中说的一段话\endnote{指下面这段话:“在一国内流通的金币和银币,作为本国土地和劳动的年产品在适当的消费者之间流通和分配的手段,就象个别商人的现金一样,是死资本。这是一国资本的极有价值的部分,但不为本国生产任何东西。”——第 273 页。}\fontbox{\}~}(\textbf{罗德戴尔},同上第 144—145 页)\end{quote}

\centerbox{※     ※     ※}

因此,可以排一个队:费里埃、加尔涅、罗德戴尔、加尼耳。\textbf{托克维尔}特别爱用最后那句关于“\textbf{节约劳动}”的话。

\tsectionnonum{[(13)萨伊对“非物质产品”的见解。为非生产劳动的不可遏止的增长辩护]}

在加尔涅之后,出版了庸俗的让·巴·萨伊的《论政治经济学》一书。萨伊非难斯密,说他

\begin{quote}“不把医生、音乐家、演员等人这类活动的\textbf{结果}叫做\textbf{产品}。他把这些人从事的劳动称为\textbf{非生产劳动}”。(第 3 版第 1 卷第 117 页)\end{quote}

斯密完全不否认“这类活动”会产生某种“结果”,某种“产品”。他甚至直接提到:

\begin{quote}“国家的安全、安定和保卫”是〈“国家公务人员”〉“年劳动的结果”。(\textbf{斯密}的著作第 2 篇第 3 章;加尔涅的译本,第 2 卷第 313 页)\end{quote}

萨伊也坚持斯密的补充定义:这些“服务”以及它们的产品“通常一经提供,一经生产,随即消失”。(\textbf{斯密},同一章)萨伊先生把这样消费掉的“服务”或它的产品,它的结果,一句话,它的使用价值,称为“非物质产品或一生产出来就被消费掉的价值”。他不把提供这种服务的人叫做“非生产劳动者”,而叫做“生产非物质产品的人”。他用了另一个名称。但是他在下面又说:

\begin{quote}“他们不是用来增加国民资本的。”(第 1 卷第 119 页)“一个国家有许多音乐家、教士、官吏,可能有很好的娱乐,精通宗教教义,并且治理得井井有条;但不过如此而已。国家的资本不会由于这些人的劳动而有任何直接的增加,因为他们的产品一生产出来就被消费掉。”(同上,第 119 页)\end{quote}

由此可见,萨伊先生只是从斯密的定义的最有限的意义上把这类劳动称为\textbf{非生产劳动}。但同时他又想把加尔涅的“进步”据为己有。所以他给各种非生产劳动发明了一个新名称。这就是他的独创性、生产性和发现方式。可是,他又以他惯有的逻辑把自己推翻了。他说:

\begin{quote}“不能同意加尔涅先生的意见,他根据医生、律师等等的劳动是生产劳动这一点,就得出结论说,增加这种劳动和增加其他任何劳动一样,对国家有利。”(同上,第 120 页)\end{quote}

但是,既然一种劳动和另一种劳动一样是生产劳动,既然生产劳动的增加一般都“对国家有利”,为什么不能同意这种意见呢\fontbox{?}为什么增加这种劳动,不象增加其他任何劳动那样有利呢\fontbox{?}萨伊用他特有的深奥想法回答说,因为增加任何一种生产劳动,如果超过了人们对这种劳动的需要,一般都是不利的。如此说来,加尔涅倒是对了。如此说来,增加这种劳动象增加其他劳动一样,超过了一定的限度,就一样有利——也就是一样不利了。

\begin{quote}萨伊继续说道:“这种情况就好比人们花费在产品上的体力劳动,超出了制造该产品所必要的劳动。”\end{quote}

(做一张桌子所花费的木匠劳动,不应超出生产桌子所必要的劳动。同样,修补病体所花费的劳动,也不应超出治好病体所必要的劳动。因此,律师和医生应当花费的只是制成自己的“非物质产品”所必要的劳动。)

\begin{quote}“生产非物质产品的劳动,\textbf{也象其他任何劳动一样},只有在增加产品的效用从而增加产品的价值〈即增加产品的使用价值,但萨伊把效用同交换价值混为一谈〉的时候,才是生产劳动;一旦超出这个界限,它就成为纯粹的非生产劳动了。”(同上,第 120 页)\end{quote}

可见,萨伊的逻辑是这样的:

增加“非物质产品生产者”的人数,\textbf{并不象}增加物质产品生产者的人数\textbf{那样}对国家\textbf{有用}。\textbf{论据}:无论哪种产品(物质产品或非物质产品)生产者人数的增加超过了需要,都是绝对无用的。\textbf{所以},增加无用的物质产品生产者的人数,比增加无用的非物质产品生产者的人数更有用。在这两种场合,都不能得出结论说:增加所有这些生产者的人数是无用的。只能得出结论说:增加某一部门内某种产品生产者的人数是无用的。

按照萨伊的意见,物质产品[400]决不会生产过多,非物质产品也是一样。但是,多样化使人愉快。所以这两个部门必须生产各式各样的产品。此外,萨伊先生教导说:

\begin{quote}“某些产品的滞销,是由另一些产品太少引起的。”[同上,第 1 卷第 438 页]\end{quote}

这就是说,桌子决不会生产过多,至多也许是可以放在桌子上的如碗之类的东西太少了。如果医生人数增加太多,那末错误不在于他们提供的服务过多,而大概在于其他“非物质产品”生产者,例如妓女,提供的服务太少(同上,第 123 页,那里,搬运工人、妓女等等的劳动被归成一类,萨伊还大胆断言,妓女的“训练时间等于零”)。

归根到底,在萨伊的书中,优势是在“非生产劳动者”方面。在一定的生产条件下,人们能准确地知道,做一张桌子,需要多少工人,制成某种产品,需要某种劳动量应多大。许多“非物质产品”的情况却不是这样。这里,达到某种结果所需要的某种劳动量多大,和结果本身一样,要靠猜测。二十个教士在一起对犯罪者的感化,也许是一个教士做不到的;六个医生会诊,能找到的有效药方,也许是一个医生找不到的。一个审判团,也许比一个无人监督的审判官能做出更为公正的裁判。保卫国家需要多少士兵,维持国内秩序需要多少警察,治理好国家需要多少官吏,等等,所有这些都是大可研究的问题,例如在英国议会中,这些问题就经常引起争论,虽然在英国,人们都很准确地知道,生产 1000 磅纱所必需的纺工劳动量有多大。至于另一些这类“生产”劳动者,他们的概念本身就包含着这样的意思:他们产生的效用,恰好只取决于他们的人数,只在于他们的人数本身。例如仆役就是这样,他们是他们主人有钱有势的证据。他们人数越多,他们“生产”的效果就越大。因此,萨伊先生始终认为:“非生产劳动者”的人数决不会增加到充分的程度。[400]

\tsectionnonum{[(14)]德斯杜特·德·特拉西伯爵[关于利润起源的庸俗见解。宣称“产业资本家”是唯一的最高意义上的生产劳动者]}

[400]\textbf{德斯杜特·德·特拉西伯爵}《思想的要素》,第四、五部分《论意志及其作用》1826 年巴黎版(第一版 1815 年出版)。

\begin{quote}“任何有用劳动都是真正的生产劳动,社会上的任何劳动阶级都同样应当称为\textbf{生产}阶级。”(第 87 页)\end{quote}

但是在这种生产阶级中,德斯杜特·德·特拉西又分出一个

\begin{quote}“\textbf{直接生产}我们的一切财富的劳动阶级”(第 88 页),\end{quote}

这也就是斯密所谓的生产工人。

相反,\textbf{不生产}阶级包括消费自己的土地的租金或货币的租金的富人,这是“\textbf{有闲阶级}”。

\begin{quote}“真正的\textbf{不生产}阶级是有闲者阶级,他们无所事事,专靠在他们以前业已完成的劳动的产品,过着所谓\textbf{养尊处优的}生活,而这些产品或者是物化在一些地产中,由他们把这些地产出租即\textbf{租借}给某个劳动者;或者是一些货币或物品,由他们借出去,取得一定的报酬,这也是\textbf{租借}。这种人是蜂房里真正的雄蜂(为享受果实而生的人们\authornote{见贺雷西《书信集》。——编者注})。”(第 87 页)这些有闲者“只能花费自己的\textbf{收入}。如果他们花费[401]自己的资本,那末资本就将无法补偿,而他们的消费,在短期间内过度增加之后,就会完全停止”。(第 237 页)“这种\textbf{收入}不外是……劳动市民的活动的产品的扣除部分。”(第 236 页)\end{quote}

这些“有闲者”直接使用的劳动者情况又怎样呢\fontbox{?}从有闲者消费商品这一点来说,他们不是直接消费劳动,而是消费生产工人的劳动产品。因而这里谈的劳动者,是指那些由有闲者直接花费收入来购买其劳动的劳动者;所以,是指那些直接从收入而不是从资本取得自己工资的劳动者。

\begin{quote}“因为占有它〈收入〉的那些人是有闲者,所以很明显,他们不\textbf{管理任何生产劳动}。一切由他们支付报酬的劳动者唯一的用处是为他们提供享受。享受当然是各种各样的……整个有闲阶级的开支……用于维持大批人口的生活,这批人的生活由此得到保证,但他们的劳动是完全不生产的……这种开支的某些部分也许多少有点用处,例如建筑房屋,改良土地。但这是例外;在这种情况下,有闲者会暂时成为生产劳动的领导者。撇开这些微不足道的例外情况,从再生产的角度来看,这类资本家的全部消费在一切方面都是一种纯损失,是已生产出来的财富的相应扣除部分。”(第 236 页)\end{quote}

\fontbox{~\{}具有真正斯密精神的政治经济学把资本家只看成人格化的资本,看成 G—W—G,看成生产当事人。但究竟谁来消费产品呢\fontbox{?}工人吗\fontbox{?}不,不是工人。资本家自己吗\fontbox{?}那他就成了大消费者、“有闲者”,而不是资本家了。土地租金和货币租金的所有者吗\fontbox{?}但他们不会把他们消费的东西再生产出来,因而只会损害财富。不过,在这种把资本家只看做现实的货币贮藏者,而不是看做象真正货币贮藏者那样的幻想的货币贮藏者的矛盾看法中,有两点是正确的:(1)资本(从而也就是资本家,资本的人格化)只被看做促使生产力和生产发展的当事人;(2)这里表现了上升的资本主义社会的观点,对这种社会具有意义的不是使用价值,而是交换价值,不是享受,而是财富。当上升的资本主义社会本身还没有学会把剥削和消费结合起来,还没有使享用的财富从属于自己时,享用的财富对它来说,是一种过度的奢侈。\fontbox{\}~}

\begin{quote}“要发现这种收入〈有闲者赖以生活的收入〉怎样形成,始终必须追溯到\textbf{产业资本家}。”(第 237 页注)“\textbf{产业资本家}〈第二种资本家〉包括所有经济部门的一切企业主,即一切\textbf{拥有资本}的人……他们把自己的才能和劳动用于自己利用资本,而不是把资本借给别人。因此,这种人不是靠工资过活,也不是靠收入过活,而是靠\textbf{利润}过活。”(第 237 页)\end{quote}

在德斯杜特的著作中可以明显地看出,正象在亚·斯密的著作中已经明显地看出那样,表面上是在赞美生产工人,实际上不过是赞美那些与土地所有者和单靠自己的收入过活的货币资本家相对立的“\textbf{产业资本家}”。

\begin{quote}“产业资本家……几乎把社会的全部财富掌握在自己手中……这些人在一年中,不仅支出这些财富的租金,而且支出资本本身;有时,如果事业的进展相当迅速,他们还有可能在一年内作几次这样的支出。因为他们作为实业家,支出只是为了使支出带着利润回到他们手里,所以,他们在这种条件下能够支出愈多,他们的利润就愈大。”(第 237—238 页)\end{quote}

至于他们的个人消费,那末,这种消费是同有闲资本家的个人消费一样的。不过,他们的个人消费

\begin{quote}“一般说来是适度的,因为实业家通常是简朴的”。(第 238 页)他们的生产消费就不一样了。“这种消费决不是最后的消费,它会带着利润回到他们手里。”(同上)他们的利润必定是相当大的,不仅足以供他们“个人消费,而且还”足以支付“从有闲资本家那里租借来的土地和货币的租金”。(第 238 页)\end{quote}

德斯杜特在这一点上是对的。土地的和货币的租金只不过是产业利润中的“\textbf{扣除部分}”,是产业资本家从自己的总利润中交给土地所有者和货币资本家的那一部分产业利润。

\begin{quote}“有闲的富人的收入,只不过是从生产中取得的租金;只有生产才创造这种收入。”(第 248 页)产业资本家“花费租金,租借他们〈有闲资本家〉的土地、房屋和货币,并以某种方式加以利用,由此\textbf{取得超出租金之上的利润}”,[第 237 页]就是说,他们支付给有闲者的这种租金,只是利润的一部分。他们这样支付给有闲资本家的这种租金,是“这些有闲者的唯一收入,是他们常年支出的唯一基金”。(第 238 页)\end{quote}

直到现在为止,谈的都很好。但“\textbf{雇佣工人}”(即产业资本家使用的生产工人)情况又怎样呢\fontbox{?}

\begin{quote}“他们除了自己日常的劳动之外,没有任何别的贮藏。这种劳动为他们提供工资……但工资是从哪里来的呢\fontbox{?}很明显,是从购买雇佣工人\textbf{出卖的劳动}的[402]那些人的财产中来的,也就是从事先掌握在雇主手中、\textbf{不外是积累起来的以前劳动的产品}的那种基金中来的。由此可以得出结论说,虽然由这种财富支付的消费,从它维持雇佣工人生活这个意义上来说,就是雇佣工人的消费,但实质上\textbf{支付消费的并不是雇佣工人},或者至少可以说,雇佣工人只是用\textbf{事先掌握在他们雇主手中的那种基金}来支付消费。因此,必须把他们的消费看作是雇用他们的那些人的消费。雇佣工人只不过是这只手拿进来,那只手还回去……不仅应当把他们〈雇佣工人〉所支出的一切,而且应当把他们所取得的一切,看成\textbf{购买他们的劳动的人}的实际支出和\textbf{这些人本身的消费}。这是千真万确的,所以要确定这种消费究竟是对现有财富会造成或多或少的损失,还是相反地会促使现有财富增加……就必须知道\textbf{资本家如何使用他们所购买的劳动},因为一切都取决于这一点。”(第 234—235 页)\end{quote}

好极了。但是企业主能够用来给自己和有闲资本家等等支付收入的利润,是从哪里来的呢\fontbox{?}

\begin{quote}“有人问我,这些产业主怎么能赚取这样大的利润,他们能够从谁手里取得这样大的利润。我回答说:那是\textbf{因为他们按高于生产成本的价格出卖他们生产的一切产品}。”(第 239 页)\end{quote}

但是他们把这一切卖得比成本贵,是卖给谁呢\fontbox{?}

\begin{quote}“(1)他们彼此出售用来满足他们需要的全部消费品;他们用自己的一部分利润来支付这些消费品;(2)他们把产品卖给他们自己雇用的和有闲资本家雇用的雇佣劳动者。通过这种途径,他们\textbf{从雇佣劳动者那里收回劳动者的全部工资},或许只有劳动者的少量积蓄除外;(3)他们把产品卖给有闲资本家。这种资本家把还没有付给自己直接雇用的雇佣劳动者的\textbf{那一部分收入支付给他们}。他们每年付给这种资本家的全部租金,就是通过这种或那种途径,再流回他们手里的。”(同上,第 239 页)\end{quote}

现在我们进一步考察这里划分的三项售卖。

(1)产业资本家自己吃掉自己产品(或利润)的\textbf{一部分}。他们决不能因互相欺骗、互相把自己的产品卖得比所\textbf{花费}的成本\textbf{贵}而发财致富。而且谁也不能用这种办法欺骗别人。如果 A 把自己的由产业资本家 B 吃掉的产品卖得过贵,那末 B 也会把自己的由产业资本家 A 吃掉的产品卖得过贵。这就好比 A 和 B 都按照实际价值互相出卖产品一样。这第一项告诉我们,资本家如何支出自己的一部分利润;它并没有告诉我们,资本家从哪里取得这部分利润。无论如何,他们“\textbf{彼此}”“按\textbf{高于}生产成本的价格\textbf{出卖}他们生产的一切产品”,并不能获得任何利润。

(2)同样,从他们按照\textbf{超过生产费用}的价格卖给自己工人的那部分产品中,他们也不能取得任何利润。根据假定,工人的全部消费实际上都是“购买他们的劳动的人本身的消费”。此外,德斯杜特还补充说,资本家把自己的产品卖给雇佣劳动者(他自己的劳动者和有闲资本家的劳动者)时,只是“收回劳动者的全部工资”。甚至不是全部,而是工人的积蓄除外。资本家把产品究竟是贱卖还是贵卖给工人,都是完全一样的,因为资本家始终只是收回他们给工人的东西,正如前面所说的,雇佣工人“只不过是这只手拿进来,那只手还回去”。资本家先把\textbf{货币}作为工资付给工人。然后他把自己的产品“过贵”卖给工人,从而收回货币。但是,因为工人还给资本家的货币,不能多于他从资本家那里取得的货币,所以,资本家把自己的产品卖给工人,也决不能\textbf{贵于}他\textbf{支付}工人劳动的代价。他在出卖自己的产品时从工人那里所能收回的货币,始终只和他给工人劳动支付的货币一样多。一文钱也多不了。资本家的货币量,怎么能由于这种“流通”而增加呢\fontbox{?}

[403]此外,德斯杜特还有一个荒谬看法。资本家 C 把周工资 1 镑支付给工人 A,然后卖给工人价值 1 镑的商品,从而取回这 1 镑。特拉西认为,通过这种办法,资本家就取回了全部工资。但是,他先给工人 1 镑货币,然后又给工人价值 1 镑的商品。可见,他事实上给工人的是 2 镑:1 镑商品和 1 镑货币。在这 2 镑当中,他以货币形式取回了 1 镑。因此,他事实上没有从 1 镑工资中取回一文钱。如果他要通过这样“取回”工资的办法(而不是通过工人用劳动把资本家以商品预付给工人的东西还给资本家的办法)来发财致富,那他很快就会碰壁。

这里,高贵的德斯杜特把货币流通和实际的商品流通混为一谈了。因为资本家不是直接给工人价值 1 镑的商品,而是给工人 1 镑货币,使工人现在能够自己决定购买什么商品,又因为工人在获得他应得的那份商品之后,会以货币形式把资本家拨给他的商品归还给资本家,所以德斯杜特就以为,当同一货币流回到资本家手里时,资本家就把工资“取回”了。就在同一页上,德斯杜特先生还说,流通现象“没有被很好地理解”(第 239 页)。的确,他自己就根本没有理解这种现象。如果德斯杜特不是用这种奇特的方式来说明“取回全部工资”,那末,这种荒谬看法,象我们马上就要提到的那样,至少还是可以想象的。

(但是,在这以前还要举下面这个例子来说明他的绝顶聪明。如果我走进一家店铺,店铺老板给我 1 镑,我用这 1 镑在他的店里购买价值 1 镑的商品,那末,他就取回了这 1 镑。谁也不会硬说店铺老板由于这桩买卖就变富了。他以前有 1 镑货币和价值 1 镑的商品,现在只有 1 镑货币。即使他的商品的价值只是 10 先令,而他按 1 镑的价钱卖给我,他也仍然比出卖商品前少了 10 先令,虽然他把他拿出的 1 镑全部取回了。)

如果资本家 C 给工人 1 镑工资,然后又把价值 10 先令的商品按 1 镑的价钱卖给工人,那末,他当然会得到 10 先令的利润,因为他卖给工人的商品贵 10 先令。但是从德斯杜特先生的观点来看,即使在这种情况下,也还是不能理解资本家的利润怎么会必然由此产生。(据说利润的产生,是由于资本家支付给工人的是降低了的工资,资本家在同工人的劳动交换时,事实上付给工人的那部分产品比\textbf{名义上}付给工人的少。)如果他给工人 10 先令,再把自己的商品按 10 先令的价钱卖给工人,那他就象给工人 1 镑,再把自己的价值 10 先令的商品按 1 镑的价钱卖给工人时一样富。而且德斯杜特的推论是从必要工资的前提出发的。在这里,对利润的全部说明,顶多也只能归结为工资上的诈骗勾当。

总之,这第二种情况表明,德斯杜特完全忘记了什么是生产工人,他对利润的源泉一窍不通。最多也只能说,在资本家不是把产品卖给自己的雇佣工人,而是卖给有闲资本家的雇佣劳动者的时候,他用高于产品价值出卖产品的办法来创造利润。但因为非生产劳动者的消费,事实上只是有闲资本家的消费的一部分,所以我们现在要考察第三种情况。

(3)第三,产业资本家把自己的产品高于产品价值过“贵”卖给

\begin{quote}“有闲资本家。这种资本家把还没有付给自己直接雇用的雇佣劳动者的那一部分收入支付给他们。他们每年付给这种资本家的全部租金,就是通过这种或那种途径,再流回他们手里的”。\end{quote}

这里谈的租金回流等等,就象前面谈的取回全部工资一样,又是一种幼稚的看法。例如,假定 C 把 100 镑土地的或货币的租金支付给 O(有闲资本家)。这 100 镑对于 C 来说,是支付手段。对于 O 来说,是购买手段。O 用这些货币从 C 的仓库里取得价值 100 镑的商品。这样,这 100 镑就作为 C 的商品的转化形式回到他手里。但是他现在比以前少了价值 100 镑的商品。他没有直接把商品给 O,而是把 100 镑货币给了 O,O 用这些货币向他购买价值 100 镑的商品。但 O 是用 C 的货币,不是用自己的基金来购买价值 100 镑的这些商品。而特拉西以为,通过这种办法,C 支付给 O 的租金会回到 C 手里。多么愚蠢!这是第一个荒谬之处。

第二,德斯杜特自己对我们说过,土地的和货币的租金只不过是产业资本家利润中的扣除部分,因而只不过是这种利润中交给有闲资本家的部分。如果现在假定,C 用某种诡计把这部分全部收回来[404](不过,靠特拉西所说的两种办法中的任何一种,都是绝对收不回来的),换句话说,如果假定资本家 C 既不向土地所有者,也不向货币资本家支付任何租金,他把自己的\textbf{全部}利润\textbf{都}留给自己,那末,正好需要说明的是,他究竟\textbf{从哪里}得到利润,他怎样创造利润,利润是如何产生的。如果说他没有把利润的一部分交给土地所有者和货币资本家,因而\textbf{拥有}利润或把利润\textbf{留给了自己}的说法不能说明这个问题,那末,同样,他以某种方法把以前在某种名义下交给有闲资本家的那部分利润,全部或部分地从有闲资本家口袋里取回来的说法也不能说明这个问题。这是第二个荒谬之处。

但是我们且把这些荒谬之处撇开不谈。C 由于向 O(有闲资本家)租借土地或货币,必须付给 O100 镑的租金。他从自己的利润中支付这 100 镑(而利润是从哪里产生的,我们还是不知道)。然后他把自己的产品卖给 O,这些产品或是直接由 O 自己吃掉,或是由他的食客(非生产的雇佣劳动者)吃掉;C 把这些产品\textbf{过贵}卖给他,例如比价值高 25\%。他把价值 80 镑的产品按 100 镑卖给 O。在这种情况下,C 无疑赚得 20 镑利润。C 给了 O 一张价值 100 镑商品的票据。可是当 O 拿这张票据去兑现时,C 付给他的只是价值 80 镑的商品,因为他把自己商品的名义价格比价值提高了 25\%。即使 O 满足于这种状况,即消费价值 80 镑的商品而支付 100 镑,C 的利润也决不会超过 25\%。这种价格,这种欺骗,会逐年继续下去。但是 O 想吃掉价值 100 镑的商品。如果他是土地所有者,他会怎么办呢\fontbox{?}他会把一块土地按 25 镑抵押给资本家 C,C 为此供给他价值 20 镑的商品,因为 C 是按高于商品价值 25\%(1/4)的价格出卖商品。如果 O 是货币贷放者,他就会从自己的资本中给资本家 C25 镑,资本家 C 为此供给他价值 20 镑的商品。

假定资本(或土地价值)按 5\%的利息借出。那时资本是 2000 镑。现在资本只有 1975 镑。这个 O 的租金现在等于 98+(3/4)镑。这种情形会继续下去:O 始终消费 100 镑实际的商品价值,他的租金则不断减少,因为要获得价值 100 镑的商品,他就始终必须吃掉自己资本的愈来愈大的部分。这样,C 就会逐渐把 O 的全部资本拿到自己手里,并且把他的租金,即把他用借来的资本获得的利润中原先交给有闲资本家 O 的那部分,连同资本一道据为己有。显然,德斯杜特先生也想到了这个过程,因为他接着说:

\begin{quote}“但是,有人会说,如果情形是这样,如果产业主确实\textbf{每年收获的比播种的多},那末,在很短的时间内,他们就一定会占有\textbf{全部社会财富},在国内很快就会只剩下没有财产的雇佣劳动者和资本主义企业主了。\textbf{这是对的}。只要企业主或他们的继承人在发财之后不是放弃经营,因而不是不断地补充有闲资本家阶级的队伍,情形的确会如此;尽管常有这种变迁,我们仍然可以看到,当一国的生产在一段时间内有所发展而没有发生太大的震动时,企业主的资本总是会不断增加,这种增加不仅同总财产的增加成比例,而且还大大超过……还可以补充一句,如果没有历届政府每年以赋税形式加给产业阶级的庞大负担,这种结果还会显著得多。”(第 240—241 页)\end{quote}

德斯杜特先生的话在某种程度上说是完全对的,不过他想要说明的那种东西根本不是这样。在中世纪衰落和资本主义生产上升的时期,“产业资本家”迅速致富,其中一部分原因,就是他们直接欺骗土地所有者。由于美洲的发现,货币价值降低了,租地农场主名义上而不是实际上继续向土地所有者支付原来的租金,而工业家却不仅按照提高了的货币价值,而且甚至高于商品的价值,把商品卖给这些土地所有者。同样,在国家的主要收入以地租形式掌握在土地所有者、君主等等手里的所有那些国家里,例如在亚洲国家,\textbf{人数不多}因而不受竞争影响的工业家,按照垄断价格把自己的商品卖给土地所有者、君主等等,从而把这些人的一部分收入据为己有,[405]他们不仅由于把“无酬”劳动卖给这些人,而且由于按照比商品中包含的更大的劳动量出卖商品而发财致富。不过,德斯杜特先生认为出借货币的资本家也同样受到欺骗,这就又不对了。相反,这些资本家取得高额利息,他们直接或间接地分享这种高额利润,参加这种欺骗。

下面这段话表明,德斯杜特先生脑中也浮现过这种现象:

\begin{quote}“我们只要看一看这样的事实:三、四百年前,在整个欧洲,同各种有权势的人物的庞大财富相比,他们〈产业资本家〉是弱小的,可是今天,他们的人数增加了,力量增强了,而那些人的财富却减少了。”(同上,第 241 页)\end{quote}

德斯杜特先生想给我们说明产业资本的\textbf{利润},而且是\textbf{高额利润}。他对这个问题做了双重的说明。第一,这些资本家以工资和租金形式支付的\textbf{货币},会流回他们手里,因为这些工资和租金会被用来购买他们的产品。而事实上,这只不过说明了,为什么他们不是\textbf{双重地}支付工资和租金——先是以货币形式支付,然后再以同一货币额的商品形式支付。第二个说明是,他们高于商品价格\textbf{过贵}出卖自己的商品:第一,贵卖给\textbf{自己},也就是欺骗自己和互相欺骗;第二,贵卖给工人,这又是欺骗自己,因为德斯杜特先生对我们说过,“必须把”雇佣工人的消费“看作是雇用他们的那些人的消费”(第 235 页);最后,第三,贵卖给\textbf{租金所得者},也就是欺骗这些人;这确实能说明,为什么产业资本家会把自己利润的愈来愈大的部分留给自己,而不把它交给有闲资本家。这能表明,为什么\textbf{总利润}在产业资本家和非产业资本家之间的\textbf{分配},会愈来愈牺牲后者而有利于前者。但这丝毫不能帮助我们理解这个\textbf{总利润是从哪里}来的。即使假定产业资本家占有了全部利润,也仍然有这一个问题:利润是从哪里来的\fontbox{?}

可见,德斯杜特什么也没有回答,他只不过暴露了,他把货币的回流看作商品本身的回流。这种\textbf{货币的回流}仅仅表示,资本家起初不是用商品而是用货币支付工资和租金;然后这些货币被用来购买他们的商品,这样,他们也就是以间接的方式用商品来支付了。因此,这些货币不断地流回他们手中,但只有在同一货币价值额的商品最终地从他们手里取走,而加入雇佣工人和租金所得者的消费的条件下,才会流回他们手中。

德斯杜特先生(纯粹按法国方式,我们在蒲鲁东那里也看到这种自我惊叹)完全是在叹赏

\begin{quote}“对我们财富的消费的这种考察……把社会整个运动解释得多么清晰。这种一致和这种清晰是从哪里来的呢\fontbox{?}来自我们遇到了真理。这使人想起了镜子的作用。如果我们站在适当的角度,事物就会清楚地并按照它们的正确比例反映出来。如果离得太近或太远,一切事物就会显得是混乱的和歪曲的”。(第 242—243 页)\end{quote}

下面,德斯杜特先生完全是附带地想起了亚·斯密书中所谈的事物的实际状态,但实质上他只是用一些词句复述这种事物的实际状态,并不了解其真正含意;不然的话,这位法国研究院\endnote{法国研究院——法国的最高科学机构,它由几个分院即学院组成;1795 年成立。德斯杜特·德·特拉西是伦理学和政治学学院院士。——第 288 页。}的院士就决不会放射出上述的“光流”来。德斯杜特写道(第 246 页):

\begin{quote}“这些有闲者的收入是从哪里来的呢\fontbox{?}不是来自租金吗\fontbox{?}而租金是由那些\textbf{使有闲者的资本发挥作用}的人,也就是由那些用有闲者的基金\textbf{雇用劳动,从而生产出比劳动本身的费用更多的产品}的人,一句话,由产业家从自己的\textbf{利润}中支付给有闲者的。”\end{quote}

\fontbox{~\{}啊哈!这就是说,产业资本家由于向有闲资本家借用基金而支付给后者的租金(以及他们自己的利润)是这样得到的:他们用这种基金雇用劳动,从而“\textbf{生产出比劳动本身的费用更多的产品}”,也就是说,这种劳动的产品比完成这种劳动的工人所得到的代价具有更大的价值;可见,利润来自雇佣工人所生产的、超过维持自己生活的费用的东西,即来自剩余产品。产业资本家占有这种剩余产品,只把其中的某一部分交给土地租金和货币租金的所得者。\fontbox{\}~}

但德斯杜特先生由此得出的结论是:不应追溯到这些生产工人,而应追溯到使用这些生产工人的资本家。他说:

\begin{quote}“实际上正是他们养活有闲者所雇用的雇佣劳动者。”(第 246 页)\end{quote}

这是不言而喻的。既然他们直接剥削劳动,而有闲资本家只是通过他们作中介来剥削劳动。在这个意义上,把产业资本看作财富的源泉是正确的。

\begin{quote}[406]“所以,要寻找一切财富的源泉,总是要追溯到这种人〈产业资本家〉。”(第 246 页)“久而久之,\textbf{财富就积累到相当的数量,因为以前劳动的成果不会一生产出来就都消费掉}。在这些财富的所有者当中,有一部分人满足于从财富取得租金并消费这些租金。这就是我们所谓的有闲资本家。另一部分比较积极的人,把自己的和从别人那里借来的基金运用起来。他们用这些基金\textbf{来支付劳动的报酬,而劳动把这些基金再生产出来,同时带来利润}。”\end{quote}

\fontbox{~\{}可见,这里不仅是把这些基金再生产出来,而且把构成\textbf{利润}的那个余额也生产出来了。\fontbox{\}~}

\begin{quote}“他们用这种利润支付他们自己的消费和支付别人的消费。由于这种消费〈他们自己以及有闲资本家的消费吗\fontbox{?}这又是以前那种荒谬说法〉,他们的基金回到他们手中,并有所增加,然后他们再从头开始。这也就是流通。”(第 246—247 页)\end{quote}

关于“生产工人”的研究及其结果——只有被产业资本家购买的工人,只有用劳动为劳动的直接购买者生产利润的工人,才是生产工人——使德斯杜特先生得出这样的结论:\textbf{产业资本家}实际上是\textbf{唯一的}最高意义上的\textbf{生产劳动者}。他说:

\begin{quote}“靠利润生活的人〈产业资本家〉养活其他一切人,只有他们能够增加公共财富,创造我们的全部享受资料。情况必定是这样,\textbf{因为劳动是一切财富的源泉},因为只有他们这些人才\textbf{有利地运用积累的劳动},从而\textbf{给现时的劳动指出有用的方向}。”(第 242 页)\end{quote}

说他们“给现时的劳动指出有用的方向”,事实上只不过是说,他们雇用有用劳动,雇用会生产出使用价值的劳动。但是,说他们“有利地运用积累的劳动”,如果这不应当还是指上面那个意思,即他们使用积累的财富来从事生产,来生产使用价值,那就是指他们“有利地运用积累的劳动”来购买比它们本身所包含的更多的现时的劳动。在刚刚引用的这句话中,德斯杜特天真地概括了构成资本主义生产实质的矛盾。因为劳动是一切财富的源泉,所以资本是一切财富的源泉;并且日益增长的财富的真正创造者不是从事劳动的人,而是从别人的劳动中取得利润的人。劳动的生产力就是资本的生产力。

\begin{quote}“我们的能力是我们唯一的原始财富;我们的劳动生产其他一切财富,而任何一种受到良好管理的劳动都是生产的。”(第 243 页)\end{quote}

从这里,按照德斯杜特的看法,当然可以得出结论说:产业资本家“养活其他一切人,只有他们能够增加公共财富,创造全部享受资料”。我们的能力是我们唯一的原始财富,所以劳动能力不是财富。劳动生产其他一切财富;这就是说,劳动为自己以外的其他一切人生产财富,而它本身不是财富,只有它的产品才是财富。任何一种受到良好管理的劳动都是生产的;这就是说,任何一种生产劳动,任何一种给资本家带来利润的劳动,都是受到良好管理的。

德斯杜特下面一些话不是就\textbf{消费者的不同阶级}说的,而是就\textbf{消费品的不同性质}说的,这些话很好地转述了亚·斯密的看法。亚·斯密在第二篇第三章末尾研究了哪一种(非生产)支出,即哪一种个人消费,收入的消费比较有利,哪一种支出比较不利。他在研究开始时用了这样的话(加尔涅的译本,第 2 卷第 345 页):

\begin{quote}“如果说节约增加资本总量,浪费减少资本总量,那末,收支相抵的人的行为,既不积累也不损及自己的基金,既不增加也不减少资本总量。不过,有一些花费收入的方法,看来比别的方法更能促进普遍福利的增长。”\end{quote}

德斯杜特这样概括斯密的论述:

\begin{quote}“消费因消费者的性质不同而大不相同,消费还因消费品的性质不同而有所变化。的确,一切物品都代表劳动,但是,劳动的价值固定在一种物品上的时间,比固定在另一种物品上的时间更持久。制造焰火可能与开采和琢磨钻石花费同样多的辛劳,因而前者可能和后者具有同样的价值。但是,当我把这两者买来,付了价钱并加以使用时,焰火过半小时就无影无踪了,而钻石过一百年还可能成为我的子孙的财富源泉……[407]有人〈即萨伊先生〉称为非物质产品的东西也是如此。\textbf{某种发现具有永久的效用}。某种文学作品,某一幅画,也具有相当长久的效用,而一个舞会、一个音乐会、一出戏剧的效用则是短暂的,转瞬即逝的。关于医生、律师、士兵、家仆以及所有一般称为\textbf{雇员}的人的\textbf{个人服务},也可以这样说。他们的效用只在需要他们的瞬间才存在……最快的消费是最有破坏性的消费,因为它会在同样的时间内消灭最大量的劳动,或者在最短的时间内消灭同量的劳动;和这种消费相比,任何一种较慢的消费都是一种\textbf{贮藏},因为它使我们有可能把今天放弃使用的部分留待将来享用……每个人都知道,如果\textbf{价格相同},买一件可以穿三年的衣服,比买一件只能穿三个月的衣服,要经济得多。”(第 243—244 页)\end{quote}

\tsectionnonum{[(15)对斯密关于生产劳动和非生产劳动的区分的反驳的一般特点。把非生产消费看成对生产的必要刺激的辩护论观点]}

大多数反驳斯密关于生产劳动和非生产劳动的区分的著作家,都把\textbf{消费}看作对生产的必要刺激。\textbf{因此},在他们看来,那些靠收入来生活的\textbf{雇佣劳动者},即非生产劳动者(对他们的雇用并不生产财富,而雇用本身却是财富的新的消费),\textbf{甚至从创造物质财富的意义来说},也和生产工人一样是生产劳动者,因为他们扩大物质消费的范围,从而扩大生产的范围。可见,这种看法大部分是从资产阶级经济学观点出发,一方面为有闲的富人和提供\textbf{服务}给富人消费的“非生产劳动者”辩护,另一方面为开支庞大的“强大政府”辩护,为国债的增加,为占有教会和国家的肥缺的人、各种领干薪的人等等辩护。因为所有这些非生产劳动者——他们的服务体现为有闲的富人的一部分支出——都有一个共同点,就是他们\textbf{生产“非物质产品”},但消费“\textbf{物质产品}”即生产工人的劳动产品。

另一些政治经济学家,例如马尔萨斯,虽然承认生产劳动者和非生产劳动者有区别,但是又向“产业资本家”证明,甚至就生产物质财富来说,非生产劳动者也象生产劳动者一样对他是必要的。

在这里,说生产和消费是等同的,或者说消费是一切生产的目的或生产是一切消费的前提,都毫无用处。撇开上述倾向不谈,作为全部争论的基础的,倒是下面这些:

工人的消费,平均起来只等于他的生产费用,而不等于他的产品。因此,全部余额都是工人为别人生产的,所以工人的这部分产品全是\textbf{为别人而生产}。其次,“产业资本家”迫使工人进行这种\textbf{剩余生产}(即超过工人本身生活需要的生产),并且运用一切手段来尽量增加这种同必要生产相对立的相对\textbf{剩余生产},直接把剩余产品据为己有。但是,作为人格化的资本,他是为生产而生产,想为发财而发财。既然他是资本职能的单纯执行者,即资本主义生产的承担者,他所关心的就是交换价值和它的增加,而不是使用价值和它的数量的增加。他只关心抽象财富的增加,对别人劳动的愈来愈多的占有。他象货币贮藏者一样,完全受发财的绝对欲望支配,所不同的只是,他并不以形成金银财宝的幻想形式来满足这种欲望,而是以形成资本的形式即实际生产的形式来满足这种欲望。工人的剩余生产是\textbf{为别人而生产},正常的资本家,即“产业资本家”的生产则是\textbf{为生产而生产}。当然,他的财富愈增加,他也就愈背弃这种理想而成为挥霍者,哪怕是为了显示一下自己的财富也好。不过,他始终是昧着良心、怀着精打细算的念头去享用财富。“产业资本家”无论怎样挥霍,他实质上仍然和货币贮藏者一样吝啬。

西斯蒙第说,劳动生产力的发展使工人有可能得到愈来愈多的享受,但这些享受如果给了工人,就使他(作为雇佣工人)不适宜于劳动了。[注:\textbf{西斯蒙第}说:“由于工业和科学的进步,每个工人每天所能生产的远远超过他自己所必需消费的。但在他的劳动生产财富的同时,这种财富如果供他自己消费,就使他不适宜于劳动了。”(《新原理》第 1 卷第 85 页)]如果是这样,那末,同样可以正确地说,“产业资本家”一旦成为享用财富的代表,一旦开始追求享受的积累,而不是积累的享受,他就或多或少不能执行自己的职能了。

可见,“产业资本家”也是\textbf{剩余生产}即\textbf{为别人而生产}的生产者。一方面有这种剩余生产,与此相对,另一方面必定有剩余消费,一方面是为生产而生产,与此相对,另一方面必定是为消费而消费。“产业资本家”必须交给地租所有者、国家、国债债权人、教会等等只消费收入的人的东西[408],固然绝对减少他的财富,但是使他发财的贪欲旺盛不衰,从而保存他的资本主义灵魂。如果土地租金和货币租金的所得者等等也把自己的收入花费在生产劳动上,而不花费在非生产劳动上,目的就不会达到。他们自己就会成为“产业资本家”,而不再代表消费的职能。以后我们还会知道,一个李嘉图主义者和一个马尔萨斯主义者之间,曾就这个问题展开过一场极为滑稽的争论。\endnote{马克思在他的手稿第 XIV 本(这一稿本收入本卷第三册)中,分析了马尔萨斯的观点之后,谈到两本匿名著作,其中一本从李嘉图的立场出发反对马尔萨斯,另一本维护马尔萨斯的观点,反对李嘉图学派。第一本题为《论马尔萨斯先生近来提倡的关于需求的性质和消费的必要性的原理;从这一原理所得的结论是:税收和供养非生产的消费者可以导致财富的增长》1821 年伦敦版。第二本题为《政治经济学大纲》1832 年伦敦版。——第 293 页。}

生产和消费是\textbf{内在地}[ansich]不可分离的。由此可以得出结论:因为它们在资本主义生产体系内实际上是分离的,所以它们的统一要通过它们的对立来恢复,就是说,如果 A 必须为 B 生产,B 就必须为 A 消费。正如每个资本家从他这方面说,都希望分享他的收入的人有所浪费一样,整个老重商主义体系也是以这样的观念为根据:一个国家从自己这方面必须节俭,但是必须为别的沉湎于享受的国家生产奢侈品。这里始终是这样的观念:一方是为生产而生产,因此另一方就是消费别国的产品。这种重商主义体系的观念在佩利博士的《道德哲学》一书第二卷第十一章中也表现出来:

\begin{quote}“节俭而勤劳的民族,用自己的活动去满足沉湎于奢侈的富有国家的需要。”\endnote{威廉·佩利《道德哲学和政治哲学原理》一书(1785 年伦敦版)的这段话,马克思引自托·罗·马尔萨斯《人口原理》法文本,比埃尔·普雷沃和吉约姆·普雷沃译自英文第五版,1836 年巴黎和日内瓦法文第三版第四卷第 109 页。——第 294 页。}\end{quote}

\centerbox{※     ※     ※}

\begin{quote}德斯杜特说:“他们〈“我们的政治家”,即加尔涅等人〉提出这样的总原则:消费是生产的原因,因而消费愈多愈好。他们硬说,正是这一点造成社会经济和私人经济之间的巨大差别。”(同上,第 249—250 页)\end{quote}

下面这句话也很好:

\begin{quote}“在\textbf{贫国},人民是安乐的,在\textbf{富国},人民通常是贫苦的。”(同上,第 231 页)\end{quote}

\tsectionnonum{[(16)]昂利·施托尔希[对物质生产和精神生产相互关系问题的反历史态度。关于统治阶级的“非物质劳动”的见解]}

\textbf{昂利·施托尔希}《政治经济学教程》,让·巴·萨伊出版,1823 年巴黎版(这是为尼古拉大公讲授的讲义,完成于 1815 年。)\textbf{第三卷}。

在加尔涅之后,施托尔希事实上是第一个试图以新的论据来反驳斯密对生产劳动和非生产劳动的区分的人。

他把“\textbf{内在财富}即文明要素”同物质生产的组成部分——物质财富区别开来,“文明论”应该研究文明要素的生产规律(同上,第 3 卷第 217 页)。(在第一卷第 136 页上,我们读到:

\begin{quote}“显然,人在没有内在财富之前,即在尚未发展其体力、智力和道德力之前,是决不会生产财富的,而要发展这些能力,必须先有手段,如各种\textbf{社会设施}等等。因此,一国人民愈文明,该国国民财富就愈能增加。”反过来也一样。)\end{quote}

他反对斯密说:

\begin{quote}“斯密……把一切不\textbf{直接}参加财富生产的人排除在\textbf{生产劳动}之外;不过他所指的只是国民\textbf{财富}……他的错误在于,他没有对\textbf{非物质}价值和\textbf{财富}作出应有的区分。”(第 3 卷第 218 页)\end{quote}

事情其实就此完了。生产劳动和非生产劳动的区分,对于斯密所考察的东西——物质财富的生产,而且是这种生产的一定形式即资本主义生产方式——具有决定性的意义。在精神生产中,表现为生产劳动的是另一种劳动,但斯密没有考察它。最后,两种生产的相互作用和内在联系,也不在斯密的考察范围之内;而且,物质生产只有从它本身的角度来考察,才不致流于空谈。如果说斯密曾谈到并非直接生产的劳动者,那只是因为这些人\textbf{直接}参加物质财富的消费,而不是参加物质财富的生产。

从施托尔希的著作本身来看,他的“\textbf{文明论}”虽然有一些机智的见解,例如说物质分工是精神分工的前提,但是依然脱不掉陈词滥调。\textbf{仅仅}由\textbf{一个}情况就可以看出,施托尔希的著作\textbf{必然}会如此,他甚至连\textbf{表述}这个问题都还远远没有做到,更不用说解决这个问题了。要研究精神生产[409]和物质生产之间的联系,首先必须把这种物质生产本身不是当作一般范畴来考察,而是从\textbf{一定的历史的}形式来考察。例如,与资本主义生产方式相适应的精神生产,就和与中世纪生产方式相适应的精神生产不同。如果物质生产本身不从它的\textbf{特殊的历史的}形式来看,那就不可能理解与它相适应的精神生产的特征以及这两种生产的相互作用。从而也就不能超出庸俗的见解。这一切都是由于“文明”的空话而说的。

其次,从物质生产的一定形式产生:第一,一定的社会结构;第二,人对自然的一定关系。人们的国家制度和人们的精神方式由这两者决定,因而人们的精神生产的性质也由这两者决定。

\textbf{最后},施托尔希所理解的精神生产,还包括统治阶级中专门执行社会职能的各个阶层的职业活动。这些阶层的存在以及他们的职能,只有根据他们生产关系的一定的历史结构才能够理解。

因为施托尔希不是\textbf{历史地}考察物质生产本身,他把物质生产当作一般的物质财富的生产来考察,而不是当作这种生产的一定的、历史地发展的和特殊的形式来考察,所以他就失去了理解的基础,而只有在这种基础上,才能够既理解统治阶级的意识形态组成部分,也理解一定社会形态下自由的精神生产。他没有能够超出泛泛的毫无内容的空谈。而且,这种关系本身也完全不象他原先设想的那样简单。例如资本主义生产就同某些精神生产部门如艺术和诗歌相敌对。不考虑这些,就会坠入莱辛巧妙地嘲笑过的十八世纪法国人的幻想。\endnote{马克思指莱辛在他的《汉堡戏剧论》(1767—1768 年)中同伏尔泰的论战。——第 296 页。}既然我们在力学等等方面已经远远超过了古代人,为什么我们不能也创作出自己的史诗来呢\fontbox{?}于是出现了《亨利亚特》\endnote{《亨利亚特》是伏尔泰写的关于法国国王亨利四世的长诗,于 1723 年第一次出版。——第 296 页。}来代替《伊利亚特》。

但是,施托尔希在专门反对加尔涅这个最早对斯密进行\textbf{这种}反驳的人的时候,所强调指出的东西则是正确的。那就是:他强调指出反对斯密的人把问题完全弄错了。

\begin{quote}“批评斯密的人做些什么呢\fontbox{?}他们完全没有弄清这种区分〈“非物质价值”和“财富”之间的区分〉,他们把这两种显然不同的价值完全混淆起来。〈他们硬说,精神产品的生产或服务的生产就是\textbf{物质}生产。〉他们把非物质劳动看做\textbf{生产劳动},认为这种劳动\textbf{生产}〈即直接生产〉\textbf{财富},即物质的、可交换的价值;其实,这种劳动只生产非物质的、直接的价值;批评斯密的人则根据这样的假定,即非物质劳动的产品也象物质劳动的产品一样,受同一规律支配;其实,支配前者的原则和支配后者的原则并不相同。”(第 3 卷第 218 页)\end{quote}

我们要指出施托尔希的下面这些常被后来的著作家抄引的论点:

\begin{quote}“因为内在财富有一部分是服务的产品,所以人们便断言,内在财富不比服务本身更耐久,它们必然是随生产随消费。”(第 3 卷第 234 页)“原始的内在财富决不会因为它们被使用而消灭,它们会由于不断运用而增加并扩大起来,所以,它们的\textbf{消费}本身会增加它们的价值。”(同上,第 236 页)“内在财富也象一般财富一样,可以积累起来,能够形成资本,而这种资本可以用来进行再生产”等等。(同上,第 236 页)“在人们能够开始考虑非物质劳动的分工以前,必须先有物质劳动的分工和物质劳动产品的积累。”(第 241 页)\end{quote}

这一切只不过是精神财富和物质财富之间的最一般的表面的类比和对照。例如他的下面那种说法也是如此,他说,不发达的国家从外国\textbf{吸取}自己的精神资本,就象物质上不发达的国家从外国吸取自己的物质资本一样(同上,第 306 页);他还说,非物质劳动的分工决定于对这种劳动的需求,一句话,决定于市场,等等(第 246 页)。

下面这些话是直接抄来的:

\begin{quote}[410]“内在财富的\textbf{生产}决不会因为它所需要的物质产品的消费而使国民财富减少,相反,它是促进国民财富增加的有力手段”,反过来也是一样,“财富的生产也是增进文明的有力手段”。(同上,第 517 页)“国民福利因这两种生产的平衡而不断增长。”(第 521 页)\end{quote}

施托尔希认为,医生生产健康(但他也生产疾病),教授和作家生产文化(但他们也生产蒙昧),诗人、画家等等生产趣味(但他们也生产乏味),道德家等等生产道德,传教士生产宗教,君主的劳动生产安全,等等(第 347—350 页)。但是同样完全可以说,疾病生产医生,愚昧生产教授和作家,乏味生产诗人和画家,不道德生产道德家,迷信生产传教士,普遍的不安全生产君主。这种说法事实上是说,所有这些活动,这些“服务”,都生产现实的或想象的使用价值;后来的著作家不断重复这种说法,用以证明上述这些人都是斯密所谓的生产劳动者,也就是说,他们直接生产的不是特殊种类的产品,而是物质劳动的产品,所以他们直接生产财富。在施托尔希的书中还没有这种荒谬说法。其实这种荒谬说法完全可以由下面各点来说明:

(1)在资产阶级社会中,各种职能是互为前提的;

(2)物质生产领域中的对立,使得由各个意识形态阶层构成的上层建筑成为必要,这些阶层的活动不管是好是坏,因为是必要的,所以总是好的;

(3)一切职能都是为资本家服务,为资本家谋“福利”;

(4)连最高的精神生产,也只是由于被描绘为、被错误地解释为物质财富的直接生产者,才得到承认,在资产者眼中才成为\textbf{可以原谅的}。

\tsectionnonum{[(17)]纳骚·西尼耳[宣称对资产阶级有用的一切职能都是生产的。对资产阶级和资产阶级国家阿谀奉承]}

\textbf{威·纳骚·西尼耳}《政治经济学基本原理》,让·阿里瓦本译,1836 年巴黎版。

纳骚·西尼耳摆出一副高傲的样子说:

\begin{quote}“照斯密看来,犹太人的立法者是非生产劳动者。”(同上,第 198 页)\end{quote}

这是指埃及的摩西,还是指摩西·门德尔森\fontbox{?}摩西将会因自己被称为斯密所谓的“生产劳动者”,而十分感谢西尼耳先生吧。这些人如此拘守于自己的资产阶级固定观念,以致认为,如果把亚里士多德或尤利乌斯·凯撒称为“非生产劳动者”,那就是侮辱他们。其实,单是“劳动者”这个名称,就会使亚里士多德和凯撒感到侮辱了。

\begin{quote}“一个医生开药方把病孩治好,从而使他的生命延续好多年,这个医生难道不是生产持久的结果吗\fontbox{?}”(同上)\end{quote}

胡说八道!如果孩子死了,结果同样是持久的。如果孩子的病没有治好,医生的\textbf{服务}还是要得到报酬的。照纳骚看来,医生只有把病治好,律师只有把官司打赢,士兵只有把仗打胜,才能得到报酬了。

但是他现在变得真正崇高起来了,他说:

\begin{quote}“起来反抗西班牙人的暴政的荷兰人,或者起义反对有可能变得更可怕的暴政的英国人,难道只生产了短暂的结果吗\fontbox{?}”(同上,第 198 页)\end{quote}

真是美文学式的废话!荷兰人和英国人举行起义是靠自己负担费用。谁也没有为了他们“在革命中”劳动而给他们支付代价。在关于生产劳动者或非生产劳动者的问题上涉及的始终是劳动的买者和卖者。多么愚蠢!

这些家伙在对斯密的反驳中发表的庸俗的美文学,不过表明他们是“有教养的资本家”的代表,而斯密则是露骨粗鲁的资产者暴发户的\textbf{解释者}。有教养的资产者及其代言人非常愚蠢,竟用对钱袋的[411]影响来衡量每一种活动的意义。另一方面,他们又很有教养,连那些同财富的生产毫不相干的职能和活动,也加以\textbf{承认},而且他们之所以加以承认,是因为这些活动会“间接地”使他们的财富增加等等,总之会执行一种对财富“有用的”职能。

人本身是他自己的物质生产的基础,也是他进行的其他各种生产的基础。因此,所有对人这个生产\textbf{主体}发生影响的情况,都会在或大或小的程度上改变人的各种职能和活动,从而也会改变人作为物质财富、商品的创造者所执行的各种职能和活动。在这个意义上,确实可以证明,所有人的关系和职能,不管它们以什么形式和在什么地方表现出来,都会影响物质生产,并对物质生产发生或多或少是决定的作用。

\begin{quote}“有些国家,没有士兵的守卫便根本不可能耕种土地。可是,按照斯密的分类法,收成并不是扶犁的人和手执武器守卫在他旁边的人共同劳动的产品;照斯密的说法,只有土地耕种者才是生产劳动者,士兵的活动则是非生产的。”(同上,第 202 页)\end{quote}

第一,这是错误的。斯密会说,士兵的活动生产保卫,但不生产谷物。如果国内建立了秩序,那末土地耕种者就会象以前一样继续生产谷物,但不必另行生产士兵的给养,从而不必生产士兵的生命。士兵象很大一部分非生产劳动者一样,属于生产上的非生产费用\authornote{见第 159 页脚注。——编者注},这些非生产劳动者,无论在精神生产领域还是在物质生产领域,都什么也不生产,他们只是由于社会结构的缺陷,才成为有用的和必要的,他们的存在,只能归因于社会的弊端。

但纳骚会说,如果发明一种机器,使 20 个工人中有 19 个人成为多余的,那末这 19 个人也就成了生产上的非生产费用了。但是,尽管\textbf{生产的物质条件}、耕作本身的条件保持不变,士兵也可能成为多余的。而 19 个工人却只有在剩下的那 1 个工人的劳动的生产能力提高到 20 倍之后,因而只有在生产的现有物质条件发生革命之后,才能成为多余的。而且\textbf{布坎南}已经指出:

\begin{quote}“比方说,如果士兵由于他的劳动有助于生产,便应当被称为生产劳动者,那末生产工人就有同样的权利要求得到军人的荣誉了,因为毫无疑问,没有生产工人的协助,任何军队也不能上战场去打仗并取得胜利。”(\textbf{大·布坎南}《论斯密博士的〈国民财富的性质和原因的研究〉的内容》1814 年爱丁堡版第 132 页)“一个国家的财富不取决于生产\textbf{服务}的人和生产\textbf{价值}的人之间的人数比例,而取决于这两种人之间最能使每种人的劳动具有最大生产能力的那种比例。”(\textbf{西尼耳},同上第 204 页)\end{quote}

斯密从来没有否认这一点,因为他想使国家官吏、律师、教士等等这些“必要的”非生产劳动者,减少到非有他们的服务不可的\textbf{限度}。无论如何,这也就是他们能使生产工人的劳动具有最大的生产能力的那种“比例”。至于其他的“非生产劳动者”,因为他们的劳动是每一个人为了享用他们的\textbf{服务}而\textbf{随意}购买的,也就是说,是每一个人把它作为随便挑选的消费品来购买的,所以,就可能出现各种不同的情况。这些靠收入过活的劳动者的人数,同“生产”工人的人数相比可能很多,\textbf{第一},是因为财富一般说来并不很多或者带有片面性,例如,中世纪的贵族及其仆从的情形就是这样。他们和他们的仆从不是消费相当数量的工业品,而是吃掉自己的农产品。当他们不是这样而开始消费工业品的时候,他们的仆从就不得不从事劳动了。靠收入过活的人所以这样多,只是因为有很大一部分年产品不是\textbf{为了再生产}而消费。但是,人口的总数并不很多。\textbf{第二},靠收入过活的人数可能很多,是因为生产工人的生产率高,即他们生产的用来养活仆从的剩余产品多。在这种情况下,不是因为有这么多的仆从,生产工人的劳动才是生产的,相反,是因为生产工人的劳动具有这样大的生产能力,所以才有这么多的仆从。

如果两个国家人口相等,劳动生产力的发展水平相同,那就始终有充分的理由可以同亚·斯密一起说:两国的财富应由生产劳动者和非生产劳动者之间的比例来衡量。因为这不过表明,在生产工人人数较多的国家里,有较大量的年收入是为了再生产而消费,因而每年会生产较大量的价值。可见,西尼耳先生只不过是复述[412]亚当的论点,并没有提出什么新思想来同亚当相对立。后来,他自己在这里也区分了“服务的生产者”和“价值的生产者”,这样一来,他也就和大多数反对斯密的区分的人们一样:他们接受并且自己也采用了他们所反驳的那种区分。

值得注意的是:一切在自己的专业方面毫无创造的“非生产的”经济学家,都反对生产劳动和非生产劳动的区分。但是,对于资产者来说,“非生产的”经济学家们的这种立场,一方面表示阿谀奉承,力图把一切职能都说成是为资产者生产财富服务的职能;另一方面表示力图证明资产阶级世界是最美好的世界,在这个世界中一切都是有用的,而资产者本人又是如此有教养,以致能理解这一点。

对于工人来说,这种看法是要人们确信,非生产人员消费大量产品完全是理所当然的,因为非生产消费者象工人一样能促进财富的生产,不过是以自己特殊的方式罢了。

但是,纳骚终于说漏了嘴,表明他对斯密所做的本质区分一窍不通。他说:

\begin{quote}“看来,实际上在这种场合,斯密把注意力完全集中在\textbf{大土地所有者}的状况方面。他关于非生产阶级的意见一般只适用于这一种人。否则,我就无法解释他的这种论断:\textbf{资本只用来维持生产劳动者,而非生产劳动者则靠收入过活}。在他突出地称为非生产劳动者的人们当中,最大一部分人,象教师、治理国家的人,都是\textbf{靠资本}维持,即\textbf{靠预付在再生产中的资金}维持。”(同上,第 204—205 页)\end{quote}

这里,确实使人惊讶得目瞪口呆。纳骚先生关于国家和学校教师靠资本生活,而不是靠收入生活的发现,无须作进一步的注解。如果西尼耳先生想用这些话告诉我们,他们是靠资本的利润生活,在这个意义上也就是靠资本生活,那末,他只是忘记了,资本的收入并不是资本本身,这个收入,这个资本主义生产的结果,并不预付在再生产中,相反,它本身倒是再生产的结果。或者,西尼耳这样想是因为有一些税收加入某些商品的生产费用,因而加入某些生产的开支\fontbox{?}那他应该知道,这只是对收入课税的一种形式。

关于施托尔希,纳骚·西尼耳这个自作聪明的家伙还指出:

\begin{quote}“施托尔希先生断言,这些\textbf{结果}〈健康、趣味等等〉象其他有价值的物品一样,是拥有这些结果的人的\textbf{收入}的一部分,并且同样可以交换〈就是说,可以从它们的生产者手里购买〉;他无疑是错了。如果是这样,如果趣味、道德、宗教确实是可以\textbf{购买}的\textbf{物品},那末,财富的意义就和经济学家们……所说的完全不同了。我们购买的决不是健康、知识和虔诚。医生、牧师、教师……只能生产那种用以多少可靠地和完善地把这些进一步的结果生产出来的手段……既然在每一种情况下,为了取得成就,都要使用最合适的手段,那末,即使没有取得成就,没有达到预期的结果,这些\textbf{手段}的生产者也有权得到报酬。一旦提出了劝告或授了课,因此得到了报酬,交换也就完成了。”(同上,第 288—289 页)\end{quote}

最后,伟大的纳骚自己又接受了斯密的区分。就是说,他以“生产消费和非生产消费”的区分(第 206 页)来代替生产劳动和非生产劳动的区分。而消费品要么是商品,——但这里不是谈商品,——要么直接是劳动。

根据西尼耳的说法,生产消费,是指使用这样一种劳动的消费,这种劳动或者再生产劳动能力本身(例如教师或医生的劳动),或者\textbf{再生产}用来购买这种劳动的那些商品的价值。非生产消费,则是指既不生产前者也不生产后者的那种劳动的消费。而斯密说:只能用于生产的(即产业的)消费的劳动,我称为生产劳动,能够用于非生产的消费的劳动(这种劳动的消费按其性质来说不是生产消费),我称为非生产劳动。可见,西尼耳先生在这里是靠事物的新名称来证明自己的才智。

总的说来,纳骚是抄袭施托尔希的著作。

\tsectionnonum{[(18)]佩·罗西[对经济现象的社会形式的忽视。关于非生产劳动者“节约劳动”的庸俗见解]}

[413]\textbf{佩·罗西}《政治经济学教程》(1836—1837 年讲授)1842 年布鲁塞尔版。

聪明就在这里!

\begin{quote}“\textbf{间接的}〈生产〉\textbf{手段}包括一切能促进生产,有助于消除障碍,使生产更有效、更迅速、更简便的东西。〈在此之前,他在第 268 页上说:“有直接的生产手段和间接的生产手段。也就是说,有些生产手段是取得我们所关心的结果的\textbf{必要}条件,是\textbf{完成}这种生产的力量;另一些生产手段有助于生产,但不是进行生产。前者甚至能够\textbf{单独}起作用,后者只能在生产过程中帮助前者。”〉……任何一种政府劳动都是间接的生产手段……制造这顶帽子的人必须承认,在街上巡逻的宪兵、坐在法庭上的法官、关押犯人的狱吏、守卫国境防止敌人侵犯的军队,所有这些人都促进生产。”(第 272 页)\end{quote}

制造帽子的人意识到,为了使他能够生产和出卖这顶帽子,全世界都动了起来,他是多么高兴!罗西让这个狱吏等等\textbf{间接地}——不是\textbf{直接地}——促进物质生产,事实上就是作了和亚当同样的区分(见第十二讲)。

罗西在下一讲即第十三讲里专门攻击斯密——其实罗西同他的先辈们几乎一样。

他说,对生产劳动者和非生产劳动者的错误区分,是由三个原因造成的。

\begin{quote}(1)“在\textbf{买者}当中,一部分人购买产品或\textbf{劳动},是\textbf{为了个人直接消费它们};另一部分人购买它们,只是为了把他们用购得的产品和买到的劳动制造的新产品出卖。对于前一种人来说,有决定意义的是\textbf{使用价值},对于后一种人来说,有决定意义的是交换价值。”当人们只注意交换价值时,就会犯斯密的错误。“我的仆人的劳动对我来说是非生产的,——暂且承认这一点;但是,难道这种劳动对他自己来说也是非生产的吗\fontbox{?}”(同上,第 275—276 页)\end{quote}

既然整个资本主义生产的基础是:直接购买劳动,以便在生产过程中\textbf{不经购买}而占有所使用的劳动的一部分,然后又以产品形式把这一部分\textbf{卖掉};既然这是资本存在的基础,是资本的实质,那末,生产资本的劳动和不生产资本的劳动二者之间的区分,不就是理解资本主义生产过程的基础吗\fontbox{?}斯密并不否认,仆人的劳动对\textbf{他}自己来说是生产的。每种服务对它的卖者来说都是生产的。假誓约对那个靠假誓约获得现金的人来说是生产的。伪造文件对那个靠伪造文件赚钱的人来说是生产的。杀人对那个因杀人而得到报酬的人来说是生产的。诬陷者、告密者、食客、寄生者、谄媚者,只要他们的这种“服务”不是无酬的,他们的这些勾当对他们来说就都是生产的。按照罗西的看法,所有这些人都是“生产劳动者”,不仅是财富的生产者,而且是资本的生产者。自己给自己支付报酬的骗子手,——同法官和国家所做的完全一样,——也是“按照一定的方式,使用一种力量,生产一种满足人的需要的结果”[同上,第 275 页],就是说,满足盗贼的需要,也许还满足他的妻子儿女的需要。这样说来,如果全部问题只在于生产一种满足“需要”的“结果”,或者说,如果一个人只要出卖自己的“服务”就可以把这种服务算作“生产的”,就象上述情况那样,那末,这个骗子手就是生产劳动者了。

\begin{quote}(2)“第二个错误是没有区分直接生产和间接生产。因此,在亚·斯密看来,官吏是非生产的。如果〈没有官吏的劳动〉生产就几乎不可能进行,那就很清楚,这种劳动对于生产是有帮助的,即使没有直接的物质的帮助,至少还有不应忽视的间接的作用。”(同上,第 276 页)\end{quote}

这种间接参加生产的劳动(它不过是非生产劳动的一部分),我们也称为非生产劳动。否则就必须说,因为官吏没有农民就绝对不能生活,所以农民是司法等等的“间接生产者”。真是胡说八道!还有一个同分工问题有关的观点,等以后再谈。

\begin{quote}[(3)]“没有仔细区分生产现象的三个基本事实:\textbf{力量即生产手段},这种力量的\textbf{使用,结果}。”我们向钟表业者买一只表;这时我们关心的只是劳动的\textbf{结果}。或者我们向裁缝买一件上衣;情况也是一样。但是“还有一种老古板的人,他们不是这样对待事物。他们叫一个工人到家里来,供给他材料和一切必需的东西,要他做一件衣服。这些老古板的人所购买的是什么呢\fontbox{?}他们购买的是力量\fontbox{~\{}但还有“这种力量的使用”\fontbox{\}~},是冒风险生产某种结果的手段……契约的对象是对力量的购买”。\end{quote}

(然而问题正是在于,这些“老古板的人”所使用的生产方式同资本主义的生产方式毫无共同之处,在这种生产方式下,不可能有资本主义生产所带来的劳动生产力的全部发展。值得注意的是,这种特殊区别在罗西之流看来是非本质的区别。)

\begin{quote}“在雇用一个仆人的场合,我是购买一种力量,这种力量可以被利用来完成多种多样的服务,这种力量活动的结果取决于我如何使用它。”(第 276 页)\end{quote}

这一切都同问题毫无关系。

[414]

\begin{quote}“可以购买或雇用……对某种力量的一定使用权……在这种情况下,您购买的就不是产品,不是您心目中的结果了。”律师的辩护词也许能,也许不能使我打赢官司。“无论如何,您和您的律师之间的交易,都是他为取得一定价值而在某日某地替您说话,为您的利益而运用他的智力。”(第 276 页)\end{quote}

\fontbox{~\{}对此罗西还有一点意见。他在第十二讲(第 273 页)中说:

\begin{quote}“我决不认为只有靠生产棉布或制作靴子生活的人才是生产者。无论哪种劳动我都尊重……但这种尊重不应成为\textbf{体力劳动者}独占的特权。”\end{quote}

亚·斯密不是这样看的。他认为从事写作、绘画、作曲、雕塑的人是第二种意义的“生产劳动者”,虽然即兴诗人、演说家、音乐家等等不是这样的劳动者。而“服务”只要是直接加入生产的,亚·斯密就把它看作是物化在产品中的,不管这是体力劳动者的劳动,还是经理、店员、工程师的劳动,甚至学者的劳动(只要这个学者是个发明家,是在工场内或在工场外劳动的工场劳动者)。斯密在谈到分工的时候,曾说明这些业务如何在各种人员之间分配,并指出产品、商品是他们共同劳动的结果,不是其中某一个人劳动的结果。不过,象罗西这样的“精神的”劳动者所关心的,是如何为他们从物质生产中取得的那个巨大的份额辩护。\fontbox{\}~}

发了这段议论之后,罗西接着说:

\begin{quote}“这样,在交换行为中,人们把注意力集中在生产的三个基本事实的某一个上面。但是\textbf{这些不同的交换形式}是否能使某些\textbf{产品}失去\textbf{财富}的性质,使\textbf{某一生产者阶级的努力}失去\textbf{生产劳动的性质}呢\fontbox{?}显然,在这些观念之间并没有任何可以证实这种结论的联系。难道因为我不是购买某种结果,而是购买生产这种结果所必要的力量,这个\textbf{力量的活动就不会是生产的,产品就不会是财富了吗}\fontbox{?}我们再以裁缝为例。无论是向裁缝买一件现成的衣服,还是把材料和工钱给裁缝工人,要他缝一件衣服,这两种情况从结果来看始终是一样的。谁也不会说第一种劳动是\textbf{生产劳动},第二种劳动是\textbf{非生产劳动};区别只是,在第二种情况下,\textbf{想要得到衣服}的人是\textbf{他自己的雇主}。但是,从生产力方面来看,您叫到家里来的裁缝工人和您的仆人之间又有什么区别呢\fontbox{?}没有任何区别。”(同上,第 277 页)\end{quote}

这位妄自尊大的空谈家,他的全部假聪明的精华就在这里!如果亚·斯密根据他的第二个比较浅薄的见解,即根据劳动是否直接物化在劳动的买者可以出卖的商品中这一点,来区分生产劳动和非生产劳动,那末,他就会把这两种情况下的裁缝都叫做生产劳动者。但是按照他的较为深刻的见解来看,上述第二种情况下的裁缝就是“非生产劳动者”。罗西只不过表明,他“显然”不懂亚·斯密的意思。

罗西以为“\textbf{交换形式}”是无关紧要的,就好比生理学家说,一定的生命形式是无关紧要的,因为它们都只是有机物的形式。但当问题是要了解某一社会生产方式的特殊性质时,恰好只有这些形式才是重要的。上衣就是上衣。但如果它是在第一种交换形式下生产出来的,那就是资本主义生产和现代资产阶级社会;如果它是在第二种交换形式下生产出来的,那就是某种甚至和亚洲关系或中世纪关系等等相适应的手工劳动形式。所以,这些\textbf{形式}对于物质财富本身是有决定作用的。

上衣就是上衣,罗西的绝顶聪明就表现在这一点上。但是,在第一种情况下,裁缝工人不只生产上衣,他生产资本,就是说,也生产利润;他把自己的雇主作为资本家生产出来,也把自己作为雇佣工人生产出来。如果我把裁缝工人叫到家里来为我个人缝上衣,我决不因为这一点而成为自己的\textbf{企业主}(从一定经济范畴的意义上说),就象\textbf{缝纫企业主}决不是因为[415]他把他的工人缝的上衣拿来自己穿和自己消费而成为企业主一样。在一种情况下,裁缝劳动的买者和裁缝工人是作为单纯的买者和卖者相对立。一个支付货币,另一个供给商品,我的货币就转化为这个商品的使用价值。这种情形和我从商店里买一件上衣毫无区别。卖者和买者在这里,是单纯作为卖者和买者相对立。相反,在另一种情况下,他们则是作为资本和雇佣劳动相对立。至于仆人,他同第二种情况下的裁缝工人(在这种情况下,我购买他的劳动是为了它的使用价值)有共同之处,那就是他们两者具有同样的社会形式。两者都是单纯的买者和卖者。区别只在于,这里由于在利用所购买的使用价值上的特殊方式,还发生一种宗法制的关系,主人和奴仆的关系,这就使这种单纯买卖的关系在内容上——即使不是在经济形式上——发生形态变化,成为令人厌恶的事情。

此外,罗西不过是用另一种说法重复加尔涅的意见。

\begin{quote}“我们坦率地说,当斯密断言仆人的劳动不会留下任何痕迹时,他犯了他这样的人所不应当犯的大错误。假定有一个工厂主,他自己管理一个需要严加监督的大工厂……这个人不容许在自己的身边有非生产劳动者,不雇用家仆。因而,他不得不\textbf{自己服侍自己}……当他必须从事这种所谓非生产劳动的时候,他将怎样进行他的生产劳动呢\fontbox{?}您的仆人所完成的工作使您能够从事更适合于您的能力的劳动,这难道还不明白吗\fontbox{?}因此,怎么能够说仆人的服务不会留下任何痕迹呢\fontbox{?}您所做的,以及没有仆人替您服侍贵体和收拾家务您就不可能做到的,这一切都会留下来的。”(同上,第 277 页)\end{quote}

这又是加尔涅、罗德戴尔和加尼耳已经说过的\textbf{节约劳动}。按照这种看法,非生产劳动只要在如下的情况下就是生产的:它们节约劳动,并且使“产业资本家”或者生产工人有更多的时间从事自己的劳动,由于别人代替他们去完成价值较小的劳动,他们就能完成价值较大的劳动。即使这样,仍然有很大一部分非生产劳动者不能包括在内,例如只当作奢侈品的那些家仆,以及所有这样的非生产劳动者:他们只生产享受,并且只有在我\textbf{为享用他们的劳动而花费的时间同这种劳动的卖者为生产这种劳动}(完成这种劳动)\textbf{而花费的时间一样多的时候},我才能享用他们的劳动。在这两种情况下,都谈不到“节约”劳动。最后,甚至真正节约劳动的个人服务,也只有在它们的消费者是生产劳动者的情况下,才是生产的。如果它们的消费者是个有闲资本家,那末它们节约他的劳动,不过意味着让他可以什么事都不干。例如,猪一样脏的懒女人自己不动手,而叫别人替她梳头、剪指甲;乡绅自己不照管马匹,而雇用一个马夫;一个专讲吃喝的人自己不做饭,而雇用一个厨师。

施托尔希(在前面引用的著作中)所说的那些生产“\textbf{余暇}”,因而使人有空闲时间来享乐、从事脑力劳动等等的人们,也属于这类劳动者。警察节约我为自己当宪兵的时间,士兵节约我自卫的时间,政府官吏节约我管理自己的时间,擦皮靴的人节约我自己擦靴子的时间,教士节约思考的时间,等等。

在这个问题上正确的一点是\textbf{分工}的思想。每个人除了自己从事生产劳动或对生产劳动进行剥削之外,还必须执行大量非生产的并且部分地加入消费费用的职能。(真正的生产工人必须自己负担这些消费费用,自己替自己完成非生产劳动。)如果这种“服务”是令人愉快的,主人就往往代替奴仆去做,例如初夜权或者早就由主人担任的管理劳动等等,都证明了这一点。但这决没有消除生产劳动和非生产劳动的区分;相反,这种区分本身表现为\textbf{分工}的结果,从而促进一般劳动生产率的发展,因为分工使非生产劳动变成一部分人的专门职能,使生产劳动变成另一部分人的专门职能。

但是罗西断言,就连专门用来使主人摆阔、满足主人虚荣心的那些家仆的“\textbf{劳动}”,也“不是非生产劳动”。为什么呢\fontbox{?}因为它生产\textbf{某种东西}:满足虚荣心,使主人能够吹嘘、摆阔(同上,第 277 页)。这里我们又看到了那种胡说八道,好象每种服务都生产某种东西:妓女生产淫欲,杀人犯生产杀人行为等等。而且,据说斯密说过,这些污秽的东西每一种都有自己的\textbf{价值}。就差[416]说这些“服务”是无酬的了。问题并不在这里。但是,即使这些服务是无酬的,它们也不会使财富(物质财富)增加一文钱。

然后又是一段美文学式的胡言乱语:

\begin{quote}“有人硬说,歌手唱完歌,不给我们留下什么东西。不,他留下回忆!〈妙极了!〉你喝完香槟酒留下了什么呢\fontbox{?}……消费是否紧紧跟随生产,消费进行得快还是慢,固然会使经济结果有所不同,但消费这个事实本身无论怎样也不会使产品丧失财富的性质。某些非物质产品比某些物质产品存在更长久。一座宫殿会长期存在,但《\textbf{伊利亚特}》是更长久的享受来源。”(第 277—278 页)\end{quote}

多么荒唐!

从这里罗西所理解的财富的意义,即从使用价值的意义来说,情况甚至是这样的:只有\textbf{消费}才使产品成为财富,而不管这种消费是快还是慢(消费的快慢决定于消费本身的性质和消费品的性质)。使用价值只对消费有意义,而且对消费来说,使用价值的存在,只是作为一种消费品的存在,只是使用价值在消费中的存在。喝香槟酒虽然生产“头昏”,但不是生产的消费,同样,听音乐虽然留下“回忆”,但也不是生产的消费。如果音乐很好,听者也懂音乐,那末消费音乐就比消费香槟酒高尚,虽然香槟酒的生产是“生产劳动”,而音乐的生产是非生产劳动。

\centerbox{※     ※     ※}

把反对斯密关于生产劳动和非生产劳动的区分的所有胡说八道总括一下,可以说,加尔涅,也许还有罗德戴尔和加尼耳(但后者没有提出什么新东西),已经把这种反驳的全部内容都表达出来了。后来的著作家(施托尔希没有成功的尝试除外)只不过发一些美文学式的议论,讲一些有教养的空话而已。加尔涅是督政府和执政时代的经济学家,费里埃和加尼耳是帝国的经济学家。另一方面,罗德戴尔是伯爵大人,他尤其愿意把\textbf{消费者当作“非生产劳动”的生产者加以辩护}。对奴仆、仆役的\textbf{颂扬},对征税人、寄生虫的\textbf{赞美},贯穿在所有这些畜生的作品中。和这些相比,古典政治经济学粗率嘲笑的性质,倒显得是对现有制度的批判。

\tsectionnonum{[(19)马尔萨斯主义者查默斯为富人浪费辩护的论点]}

\textbf{托·查默斯牧师}是最狂热的马尔萨斯主义者之一,他是神学教授,著有《\textbf{论政治经济学和社会的道德状况、道德远景的关系}》一书(1832 年伦敦第 2 版)。按照查默斯的意见,要消除一切社会弊端,没有别的手段,只有对工人阶级进行宗教教育(他指的是通过基督教的粉饰和教士的感化来灌输马尔萨斯的人口论)。同时,他竭力为各种浪费、国家的无谓开支、教士的巨额俸禄、富人的极度挥霍辩护。他对(第 260 页及以下各页)“时代精神”和“严酷的忍饥挨饿的节约”感到痛心;他要求实行重税,让那些“高级的”非生产劳动者,教士等等可以大吃大喝(同上);当然,他对斯密的区分是极为反对的。他用整整一章(第十一章)的篇幅来谈这个区分,不过其中除了断言节约等等对“生产劳动者”只有害处以外,没有任何新的东西。下面这些话可以概括说明这一章的倾向:

\begin{quote}“这种区分是荒谬的,而且应用起来是有害的。”(同上,第 344 页)\end{quote}

害处在哪里呢\fontbox{?}

\begin{quote}“我们所以要这样详细地谈这个问题,是因为我们认为,\textbf{今日的政治经济学对教会过于严厉、过于敌视了},我们不怀疑,\textbf{斯密的有害的区分}大大促进了这一点。”(第 346 页)\end{quote}

这位牧师所说的“教会”是指他自己的教会,作为“法定”教会的英国国教会。而且,他还是把这个“教会”推行到爱尔兰的那帮家伙中的一个。至少,这个牧师是很坦率的。

\tsectionnonum{[(20)关于亚当·斯密及其对生产劳动和非生产劳动的看法的总结性评论]}

[417]在结束关于亚当·斯密的部分之前,我们还要引用他书中的两段话:在第一段话中,他发泄了自己对非生产的政府的憎恨;在第二段话中,他力图证明,为什么工业等等的进步要以自由劳动为前提。关于\textbf{斯密对牧师的憎恨}!\endnote{关于亚当·斯密对牧师的敌对态度,见马克思《资本论》第 1 卷第 23 章注 75。——第 314 页。}

第一段话说:

\begin{quote}“因此,国王和大臣们要求监督私人的节约,并以反奢侈法令或禁止外国奢侈品进口的办法来限制私人开支,这是他们最无耻、最专横的行为。他们自己始终是并且毫无例外地是社会上最大的浪费者。他们还是好好地注意他们自己的开支吧,私人的开支尽可以让私人自己去管。如果他们自己的浪费不会使国家破产,那末,他们臣民的浪费也决不会使国家破产。”(第 2 篇第 3 章,麦克库洛赫版,第 2 卷第 122 页)\end{quote}

再引下面这段话\authornote{见本册第 151,152 和 273 页。——编者注}:

\begin{quote}“某些最受尊敬的社会阶层的劳动,象\textbf{家仆的劳动}一样,不生产\textbf{任何价值}\fontbox{~\{}它有价值,因而值一个等价,但不生产任何价值\fontbox{\}~},不固定或不物化在任何耐久的对象或可以出卖的商品中……例如,君主和他的全部文武官员、全体陆海军,都是\textbf{非生产劳动者}。他们是社会的\textbf{公仆},靠\textbf{别人劳动}的一部分年产品生活……应当列入\textbf{这一类的},还有……教士、律师、医生、各种文人;演员、丑角、音乐家、歌唱家、舞蹈家等等。”(同上,第 94—95 页)\end{quote}

这是还具有革命性的资产阶级说的话,那时它还没有把整个社会、国家等等置于自己支配之下。所有这些卓越的历来受人尊敬的职业——君主、法官、军官、教士等等,所有由这些职业产生的各个旧的意识形态阶层,所有属于这些阶层的学者、学士、教士……\textbf{在经济学上}被放在与他们自己的、由资产阶级以及有闲财富的代表(土地贵族和有闲资本家)豢养的大批仆从和丑角同样的地位。他们不过是社会的\textbf{仆人},就象别人是他们的仆人一样。他们靠\textbf{别人劳动}的产品生活。因此,他们的人数必须减到必不可少的最低限度。国家、教会等等,只有在它们是管理和处理生产的资产者的共同利益的委员会这个情况下,才是正当的;这些机构的费用必须缩减到必要的最低限度,因为这些费用本身属于生产上的非生产费用\authornote{见第 159 页脚注。——编者注}。这种观点具有历史的意义,一方面,它同古代的见解形成尖锐的对立,在古代,物质生产劳动带有奴隶制的烙印,这种劳动被看作仅仅是有闲的市民的立足基石;另一方面,它又同由于中世纪瓦解而产生的专制君主国或贵族君主立宪国的见解形成尖锐的对立,就连孟德斯鸠自己都还拘泥于这种见解,他天真不过地把它表达如下(《论法的精神》第 7 篇第 4 章):

\begin{quote}“富人不多花费,穷人就要饿死。”\end{quote}

相反,一旦资产阶级占领了地盘,一方面自己掌握国家,一方面又同以前掌握国家的人妥协;一旦资产阶级把意识形态阶层看作自己的亲骨肉,到处按照自己的本性把他们改造成为自己的伙计;一旦资产阶级自己不再作为生产劳动的代表来同这些人对立,而真正的生产工人起来反对资产阶级,并且同样说它是靠别人劳动生活的;一旦资产阶级有了足够的教养,不是一心一意从事生产,而是也想从事“有教养的”消费;一旦连精神劳动本身也愈来愈为资产阶级\textbf{服务},为资本主义生产服务;——一旦发生了这些情况,事情就反过来了。这时资产阶级从自己的立场出发,力求“在经济学上”证明它从前批判过的东西是合理的。加尔涅等人就是资产阶级在这方面的代言人和良心安慰者。此外,这些经济学家(他们本人就是教士、教授等等)也热衷于证明自己“在生产上的”有用性,“在经济学上”证明自己的薪金的合理性。

[418]第二段话讲到奴隶制,他说:

\begin{quote}“这类职业〈手工业者和制造业劳动者的职业,在许多古代国家〉被看作只适宜于奴隶,而市民则不准从事这类职业。就连没有这种禁令的国家如雅典和罗马,事实上人民也不从事今天城市居民的下层阶级通常所从事的各种职业。在罗马和雅典,富人的奴隶从事这些职业,而且他们是为了主人的利益从事这些职业的。富人有钱有势,并且得到保护,这就使贫穷的自由民在自己的制品和富人奴隶的制品竞争时,几乎不可能为自己的制品找到销路。但是奴隶很少有发明;工业上一切减轻劳动和缩短劳动的最重要的改良,无论是机器还是更好的劳动组织和分工,都是自由民发明的。即使有的奴隶想出了并且提议实行这类改良,他的主人也会认为这是懒惰的表现,是奴隶企图牺牲主人的利益来减轻自己的劳动。可怜的奴隶不但不能由此得到报酬,还多半会遭到辱骂,甚至惩罚。因此,同使用自由民劳动的企业相比,使用奴隶劳动的企业,为了完成同量的工作,通常要花费更多的劳动。因此,后一类企业的制品通常总要比前一类企业的制品贵。孟德斯鸠指出,匈牙利矿山虽然不比邻近的土耳其矿山富,但是开采起来始终费用较小,因而利润较大。土耳其矿山靠奴隶开采,\textbf{奴隶的双手是土耳其人}想到使用的\textbf{唯一机器}。匈牙利矿山是靠自由民开采的,他们为了减轻和缩短自己的劳动使用了大量的机器。根据我们所知道的关于希腊和罗马时代工业品价格的不多的资料,精制的工业品看来是非常贵的。”(同上,第 4 篇第 9 章;加尔涅的译本,第 3 卷第 549—551 页)\end{quote}

\centerbox{※     ※     ※}

\textbf{亚·斯密}自己在第四篇第一章中\endnote{斯密在这一章中考察了重商主义的一般理论观点。——第 316 页。}写道:

\begin{quote}“洛克先生曾指出货币和其他各种动产的区别。他说,其他一切动产\textbf{按其性质来说是这样容易消耗},以致由这些动产构成的财富是极不可靠的……相反,货币却是一个可靠的朋友”等等。(同上,第 3 卷第 5 页)\end{quote}

接着在同一章第 24—25 页上说:

\begin{quote}“有人说,消费品很快就消灭了,而金和银\textbf{按其性质来说比较耐久},只要不把这些金属不断输出国外,这些金属就可以一个世纪一个世纪地积累起来,使一国的实际财富得到难以置信的增加。”\end{quote}

货币主义者醉心于金银,因为金银是\textbf{货币},是交换价值的独立的存在,是交换价值的可感觉的存在,而且只要不让它们成为流通手段这种不过是商品交换价值的转瞬即逝的形式,它们就是不会毁坏的、永久的存在。因此,积累金银,积蓄金银,贮藏货币,成了货币主义所宣扬的致富之道。正象我引用配第的话所指出的那样,\endnote{马克思指《政治经济学批判》第一分册《货币贮藏》那一小节,那里他引了配第《政治算术》中的话。马克思在前面第 167 页也引了同样的话,在这一页他指出斯密部分地回到了重商学派的观点。——第 317 页。}连其他商品在这里也只是根据它们的耐久程度,即根据它们作为交换价值存在多久来估价的。

现在,\textbf{第一},亚·斯密是在重复他在一个地方曾说过的关于商品耐久程度相对大小的意见,在那里他曾说,消费对于财富的形成究竟是较有利还是较不利,要看消费品存在的时间是较长还是较短。\endnote{马克思指斯密《国富论》第二篇第三章最后六段,斯密在那里研究:收入以何种方式支出对促进社会财富的增长作用比较大,以何种方式支出则作用比较小。斯密认为这取决于消费品的不同性质,取决于它们的耐久程度。马克思在前面论德斯杜特·德·特拉西那一小节(第 290—291 页)提到过斯密的这个观点。——第 317 页。}因而,这里可以看出他的货币主义观点,而这也是必然的,因为即使在直接消费时,拥有商品的人也始终盘算着使[419]消费品继续是\textbf{财富},是商品,因而是使用价值和交换价值的统一;而这又取决于使用价值的耐久程度,因而取决于消费是否只是逐渐地、缓慢地使这个使用价值失去作为\textbf{商品}或作为交换价值承担者的可能性。

\textbf{第二},斯密在他关于生产劳动和非生产劳动的第二种区分上,完全回到——在更广泛的形式上——货币主义的区分上去了。

\begin{quote}生产劳动“固定和物化在一个特定的对象或可以出卖的商品中,而\textbf{这个对象或商品在劳动结束后,至少还存在若干时候}。可以说,这是在其物化过程中积累并储藏起来,准备必要时在另一场合拿来利用的一定量劳动”。相反,非生产劳动的结果或非生产劳动的服务“通常一经提供随即消失,很少留下某种痕迹或某种以后能够用来取得同量服务的\textbf{价值}”。(第 2 篇第 3 章,麦克库洛赫版,第 2 卷第 94 页)\end{quote}

可见,斯密区分商品和服务,就象货币主义区分金银和其他一切商品一样。斯密也是从积累的角度来区分的,不过积累已经不再被看作货币贮藏的形式,而是被看作再生产的实际形式了。商品在消费中消灭,但同时它会重新生出具有更高价值的商品来,或者,如果不这样使用,商品本身就是可以用来购买其他商品的价值。劳动产品本身的属性是:它作为一个或多或少耐久的、因而可以再让渡出去的使用价值存在,它作为这样一种使用价值存在,即它是可以出卖的有用品,是交换价值的承担者,\textbf{是商品},或者说,实质上是\textbf{货币}。非生产劳动者的服务不会再变成\textbf{货币}。我对律师、医生、教士、音乐家等等、国家活动家、士兵等等的服务支付了报酬,但是,我既不能用这些服务来还债,也不能用它们来购买商品,也不能用它们来购买创造剩余价值的劳动。这些服务完全象容易消失的消费品一样消失了。

可见,斯密所说的实质上同货币主义所说的一样。货币主义认为,只有生产\textbf{货币},生产金银的劳动,才是生产的。在斯密看来,只有为自己的买者生产\textbf{货币}的劳动才是生产的。所不同的只是,斯密在一切商品中都看出了它们具有的货币性质,不管这种性质在商品中怎样隐蔽,而货币主义则只有在作为交换价值的独立存在的商品中才看出这种性质。

这种区分是以资产阶级生产实质本身为基础的,因为财富不等于使用价值,只有\textbf{商品},只有作为交换价值承担者、作为货币的使用价值,才是财富。货币主义不懂得,这些货币的创造和增加,是靠商品的消费,而不是靠商品变为金银,商品以金银的形式结晶为独立的交换价值,但是,商品在金银的形式上不仅丧失了它们的使用价值,而且没有改变它们的\textbf{价值量}。

\tchapternonum{[第五章]奈克尔}

\vicetitle{[试图把资本主义制度下的阶级对立描绘成贫富之间的对立]}

前面已经引过的兰盖的一些话表明,他对资本主义生产的性质是清楚的。\endnote{马克思在手稿第 V 本第 181 页(第一章第三节《相对剩余价值》,《分工》一小节)引了兰盖的下面一段话:“贪婪的吝啬鬼不放心地监视着他〈短工〉,只要他稍一中断工作,就大加叱责。只要他休息一下,就硬说是偷窃了他。”([兰盖]《民法论》1767 年伦敦版第 2 卷第 466 页)马克思在手稿第 X 本第 439 页论兰盖一章中引了这些话(见本册第 371 页)。在《资本论》第一卷中作为第八章注 39 引了这些话,但有删节。——第 319 页。}然而这里在谈完奈克尔之后,还可以再提一下兰盖。\endnote{尽管兰盖的著作《民法论》(1767 年)发表在马克思这里所考察的奈克尔的《论立法和谷物贸易》(1775 年)和《论法国财政的管理》(1784 年)这两本著作之前,马克思却把论兰盖一章放在论奈克尔一章之后。马克思把材料作这样的编排,是因为从理解资本主义生产的性质来说,兰盖的著作超过奈克尔的上述两本著作。——第 319 页。}

奈克尔在他的《论立法和谷物贸易》(1775 年初版)和《论法国财政的管理》这两部著作中,指出劳动生产力的发展只不过使工人用\textbf{较少的时间}再生产自己的工资,从而用\textbf{较多的时间无代价地}为自己的雇主劳动。同时奈克尔正确地用\textbf{平均工资},用最低限度的工资作基础。但是,实际上他关心的不是劳动本身转化为资本,也不是资本通过这个过程得到积累,而宁可说是贫富之间、贫困和奢侈之间对立的一般发展。这种发展的基础是:随着生产必要生活资料所需的劳动量愈来愈少,有愈来愈大的一部分劳动成为剩余的,因而可以用来生产者侈品,可以用在别的生产领域。这种奢侈品的一部分是能够保存的;这样,奢侈品就在支配剩余劳动的人手里一个世纪一个世纪地积累起来,上述对立因此也就愈来愈严重。

重要的是,奈克尔一般认为非劳动阶层的财富[420]——利润和地租——来源于剩余劳动。在考察剩余价值时,他注意到相对剩余价值,即不是从延长整个工作日而是从缩短\textbf{必要劳动时间}得出的剩余价值。劳动生产力变成劳动条件所有者的生产力。而这种生产力本身则表现为得到一定结果所必要的劳动时间的缩短。主要的几段话如下:

\textbf{第一},《论法国财政的管理》(《奈克尔著作集》1789 年洛桑和巴黎版第 2 卷):

\begin{quote}“我看到社会上的一个阶级,它的收入几乎始终不变;我注意到另一个阶级,它的财富必然增长。这样,由对比和比较而来的奢侈现象,必然随着这种不平衡的发展而发展起来,并随着时间的推移而日益显著……”(同上,第 285—286 页)\end{quote}

(这里已经很好地指出了\textbf{两个阶级}之间的\textbf{阶级}对立。)

\begin{quote}“社会的一个阶级的命运好象已经由社会的法律\textbf{固定了},所有属于这个阶级的人都\textbf{靠自己双手劳动过活},被迫服从\textbf{所有者}〈生产条件所有者〉的法律,不得不以领取\textbf{相当于最迫切的生活需要的工资}为满足;他们之间的竞争和\textbf{贫困的压迫},使\textbf{他们处于从属地位};而且这种状况是不能改变的。”(同上,第 286 页)

“\textbf{使一切机械工艺简单化的新工具不断发明},因而\textbf{增加了所有者的财富和财产};其中一部分工具\textbf{减少了土地耕作费用},使土地所有者所能支配的\textbf{收入增加了};人类天才的另一部分发明\textbf{大大地减轻了}工业中的劳动,以致\textbf{在生存资料的分配者}〈即资本家〉\textbf{手下劳动的人们},能够\textbf{在同样的时间内,拿同样的工资},生产出更多的各种制品。”(同上,第 287 页)“假设在上一世纪,必须有 10 万工人才能完成今天 8 万工人就能完成的工作;那末,现在剩下来的 2 万人为了取得工资,就不得不投身于\textbf{别的职业};由此创造出来的新的手工制品,就会增加富人的享受和奢侈。”(第 287—288 页)奈克尔接着说:“因为不应当忽视,一切不需要特殊技艺的劳动的报酬,总是同\textbf{每个工人所必需的生存资料的价格}成比例的;所以,知识一旦普及,\textbf{制作速度的加快就丝毫不会有利于劳动者,而只会增加}用以满足拥有土地产品的人们的趣味和虚荣心的\textbf{手段}。”(同上,第 288 页)“靠人的技艺成型和改变形态的种种自然财物中,有许多按其耐久程度来说是大大超过人的通常寿命的,因此每一代都继承前几代劳动创造物的一部分\end{quote}

\fontbox{~\{}奈克尔这里所考察的,只是亚·斯密称为消费基金的那种东西的积累\fontbox{\}~},

\begin{quote}并且在各个国家都有愈来愈多的工艺制品逐渐\textbf{积累起来};因为这一切制品总是在所有者中间分配,所以这些人的享受和人数众多的市民阶级的享受之间的不平衡,必然愈来愈大,愈来愈明显。”(第 289 页)\end{quote}

所以:

\begin{quote}“能增加大地上奢侈品和装饰品的\textbf{工业劳动速度的加快,这些奢侈品和装饰品能够积累起来的时期的延续,以及使这些财物只集中在一个社会阶级手中的财产法}……奢侈的这许多源泉,不管流通中的货币量有多少,都是始终存在的。”(第 291 页)\end{quote}

(最后一句话,是反驳那些认为奢侈来源于货币量日益增加的人的。)

\textbf{第二},《论立法和谷物贸易》(《奈克尔著作集》第 4 卷)说:

\begin{quote}“手工业者或土地耕种者一旦\textbf{丧失储备},他们就无能为力了;他们必须\textbf{今天劳动,才不致明天饿死};在所有者和工人之间的[421]这种利益斗争中,一方用自己的生命和全家的生命作赌注,另一方只不过延缓一下自己奢侈的发展而已。”(同上,第 63 页)\end{quote}

这种不劳动的富和为生活而劳动的贫之间的对立,又造成了知识的对立。知识和劳动彼此分离,于是知识作为资本或富人的奢侈品同劳动相对立:

\begin{quote}“认识和理解的能力是一般天赋,但这种能力只有通过教育才能发展;如果财产是平等分配的,那末每个人\textbf{就会适度地劳动}\end{quote}

(可见,起决定作用的又是劳动时间的量),

\begin{quote}\textbf{并且,每个人都会有一些知识},因为每个人都剩下\textbf{一定量的时间}〈空闲的时间〉来学习和思考;但是在社会制度所造成的财产不平等的情况下,所有那些生下来就没有财产的人,\textbf{根本没有受教育的机会}。因为一切生存资料都掌握在占有\textbf{货币或土地}的那部分国民手里。因为谁也不会白给东西,所以生下来除了自己的力气之外便没有别的储备的人,不得不在刚有点力气的时候,就用来为所有者服务,并且要一天又一天地干一辈子,每天从日出一直干到筋疲力尽,干到为了恢复精力必需睡眠时为止。”(第 112 页)“最后,为了维持所有那些\textbf{造成知识不平等的}社会的不平等,这种知识的不平等已经成了必要的了,这一点难道不是无可怀疑的吗\fontbox{?}”(同上,第 113 页;参看第 118、119 页)\end{quote}

奈克尔嘲笑经济概念的混淆。重农学派对于土地就有这种混淆,后来的所有经济学家对于资本的各物质要素也有这种混淆。有了这种混淆,生产条件的所有者就受到赞扬,因为生产条件(但决不是生产条件的所有者本人)对于劳动过程和财富的生产是必要的。奈克尔说:

\begin{quote}“人们一开始就把土地所有者(一个非常容易执行的职能)的意义同土地的意义混淆起来。”(同上,第 126 页)[IX—421]\end{quote}

\tchapternonum{[第六章]魁奈的经济表(插入部分)}

\tsectionnonum{[(1)魁奈表述总资本的再生产和流通的过程的尝试]}

一年总产品 50 亿(图尔利弗尔)租地农场主以原预付和年预付形式支出

为了使这个表更加清楚起见,凡是魁奈认为是流通的出发点的地方,我就标上 a、a′、a″,这个流通的下一环节则标上 b、c、d 以及相应的 b′、b″。\endnote{马克思在这里使用的字母符号(和标记)使《经济表》一目了然,无论在施马尔茨的著作中还是在魁奈的著作中都没有这样清楚。用两个字母(a—b,a—c,c—d 等等)来标明每一条线,使人能确定线的方向,即这条线是从哪个阶级到哪个阶级(方向按字母表上字母的顺序确定,a—b,a—c,c—d 等等)。例如,a—b 线表示土地所有者阶级和生产阶级(租地农场主)之间的流通以土地所有者阶级为出发点(后者向租地农场主购买食物)。用两个字母来标明每一条线,同时表明了货币的运动和商品的运动。例如,a—b 线表示货币的运动(土地所有者阶级向生产阶级支付 10 亿货币);但是这条线从相反的方向(b—a)来看,就表明商品的运动(生产阶级交给土地所有者阶级 10 亿食物)。虚线 a—b—c—d 由以下几个环节组成:(1)a—b 段表示土地所有者和生产阶级之间的流通(土地所有者向租地农场主购买 10 亿食物);(2)a—c 段表示土地所有者和不生产阶级——工业家之间的流通(土地所有者向工业家购买 10 亿工业品);(3)c—d 段表示不生产阶级和生产阶级之间的流通(工业家向租地农场主购买 10 亿食物)。a′—b′线表示生产阶级和不生产阶级之间的流通(租地农场主向工业家购买 10 亿工业品)。a″—b″线表示不生产阶级和生产阶级之间的最后的流通(工业家向租地农场主购买工业生产所必需的 10 亿原料)。——第 324、349、352 页。}

这个表上首先值得注意并且不能不使同时代人留下深刻印象的,是这样一个方式:货币流通在这里表现为完全是由商品流通和商品再生产决定的,实际上是由资本的流通过程决定的。

\tsectionnonum{[(2)租地农场主和土地所有者之间的流通。货币流回租地农场主手里,不表现再生产]}

租地农场主首先把 20 亿货币支付给土地所有者。后者用其中 10 亿货币向租地农场主购买食物。因此,10 亿货币流回租地农场主手里,同时总产品的 1/5 得到实现,最终由流通领域转入消费领域。

然后,土地所有者用 10 亿货币购买价值 10 亿的工业品,非农产品。从而又有 1/5 的产品(以已经加工的形式)从流通领域转入消费领域。这 10 亿货币现在落到不生产阶级手里,这个阶级用它向租地农场主购买食物。于是,租地农场主以地租形式付给土地所有者的第二个 10 亿,又流回租地农场主手里。另一方面,租地农场主的产品中另一个 1/5 归不生产阶级,由流通领域转入消费领域。因而,到这第一个运动结束时,这 20 亿货币又在租地农场主手里。这 20 亿货币完成了四个不同的流通过程。

\textbf{第一},它们用作支付地租的支付手段。在执行这个职能时,它们并不使年产品的任何一部分流通,它们只是用来支取总产品中等于地租的那一部分的流通支票。

\textbf{第二},土地所有者用 20 亿的半数 10 亿向租地农场主购买食物;因而土地所有者把自己的 10 亿实现为食物。租地农场主得到这 10 亿货币,实际上只是收回了他开给土地所有者用来支取他的产品 2/5 的那张支票的半数。在这种情况下,这 10 亿由于用作购买手段,就使同额商品进入流通,这批商品最终转入消费。在这里,这 10 亿对于土地所有者来说,只是\textbf{购买手段}:土地所有者把货币再转化为使用价值(转化为商品,然而这是最终转入消费、作为使用价值被购买的商品)。

如果我们只考察这个单独的行为,那末,这里的货币对租地农场主来说,只是起了它作为购买手段对卖者始终所起的作用,也就是成为卖者的商品的转化形式。土地所有者把他的 10 亿货币转化为谷物,而租地农场主把价格为 10 亿的谷物转化为货币,实现了谷物的价格。但是,我们把这个行为同前面的流通行为联系起来看,货币在这里,就不是表现为租地农场主的商品的单纯形态变化,不是表现为他的商品的金等价物。这 10 亿货币本来只是租地农场主[423]以地租形式支付给土地所有者的 20 亿货币的半数。诚然,租地农场主以 10 亿商品的代价取得了 10 亿货币,但\textbf{实际上他这样只是赎回他用来向土地所有者支付地租的货币;换句话说,土地所有者用他从租地农场主那里得来的 10 亿,向租地农场主购买价值 10 亿的商品。土地所有者用他不给等价物}而从租地农场主那里取得的货币,\textbf{付给租地农场主}。

货币流回租地农场主手里的这种回流,首先使这里的货币(同第一个行为联系起来看)对租地农场主来说,并不是简单的流通手段。其次,这种回流同表现再生产过程的货币流回出发点的运动有本质的区别。

例如,一个资本家,或者,为了完全撇开资本主义再生产的特征,就说一个生产者,为了取得他在全部劳动时间所必需的原料、劳动工具和生活资料,支出 100 镑。假定他加到生产资料上的劳动,不比他花费在生活资料即他支付给自己的工资上的劳动多。如果原料等等,等于 80 镑,而新加劳动等于 20 镑(他消费了的生活资料也等于 20 镑),那末产品就等于 100 镑。如果生产者再把产品卖掉,那末 100 镑货币又流回到他的手里——如此周而复始。货币流回到它的出发点的这种回流在这里不是表现别的,正是表现不断的再生产。这里是单纯的形态变化 G—W—G,货币转化为商品,商品再转化为货币。商品和货币的这种单纯的形式变换,在这里同时又表现再生产过程。货币转化为商品——\textbf{生产资料}和生活资料;然后,这些商品作为要素进入劳动过程,又作为产品从劳动过程出来;这样,从完成的产品再进入流通过程,因而再作为商品同货币相对立的时候起,商品又是过程的结果;最后,商品再转化为货币,因为完成的商品只有在它先转化为货币之后,才能重新同它的生产要素交换。

货币不断流回它的出发点,在这里,不仅表现从货币到商品和从商品到货币的形式上的转化,象它在简单流通过程或简单商品交换中所表现的那样,\textbf{同时也表现同一个生产者进行的商品的不断再生产}。交换价值(货币)转化为商品,而这些商品进入消费,作为使用价值被利用,但它们是进入再生产消费或生产消费,从而恢复了原有价值,因此又转化为\textbf{同一个}货币额(在上述例子中,生产者只为维持自己的生活而劳动)。这里,公式 G—W—G 表明,G 不仅是形式上转化为 W,而且 W 实际上作为使用价值被消费,从流通领域转入消费领域,但这是生产消费,所以商品的价值在消费中保存着,并且再生产出来,因此,G 在过程的终点又出现了,它在 G—W—G 的运动中保存了自己。

相反,在上述的货币从土地所有者流回租地农场主手里的那种回流中,没有发生任何再生产过程。这就好比租地农场主给土地所有者开了 10 亿产品的凭证或者票券。一旦土地所有者把这些票券付兑,票券便流回租地农场主手里,后者又予以承兑。如果土地所有者同意一半地租直接以实物支付,那末在这种情况下,就不会发生任何货币流通。整个流通就会限于简单的转手,即产品从租地农场主手里转到土地所有者手里。但是,起先租地农场主付给土地所有者的不是商品而是货币,然后,土地所有者又把货币付还给租地农场主,以取得商品本身。货币对租地农场主来说,作为\textbf{支付手段}付给土地所有者;货币对土地所有者来说,作为\textbf{购买手段}付给租地农场主。在执行第一种职能时,货币从租地农场主那里离开,在执行第二种职能时,货币又回到租地农场主手里。

凡是在生产者不把自己的一部分产品而把这种产品的价值用货币支付给他的债权人的时候,都必定会发生货币流回生产者手里的这种回流;这里,凡是共同占有他的剩余产品的人都表现为债权人。可以举这样一个例子。一切税收都是由生产者用货币支付的。这里,货币对生产者来说,作为支付手段付给国家。国家用这些货币向生产者购买商品。在国家手里,货币成为购买手段,这样就流回生产者手里,有多少商品从生产者那里出去,就有多少货币流回生产者手里。

货币回流这个环节——这种特殊的、不由再生产决定的货币回流——每当收入同资本交换时,都一定要发生。这里,引起货币回流的不是再生产,而是消费。收入用货币支付,但是收入只能以商品形式消费。因而,从生产者那里作为收入所取得的货币必须付还给生产者,才能从生产者那里取得同等价值的商品,也就是说,才有可能消费收入。用来支付收入的货币,例如租金、利息或税收,都具有支付手段的一般形式。[424]\fontbox{~\{}产业资本家自己用产品来支付自己的收入,或者在产品出卖后,把产品换得的、构成他的收入的那部分货币支付给自己。\fontbox{\}~}这里假定支付收入的人事先从自己的债权人那里得到自己产品的一部分,譬如说,租地农场主事先得到产品的 2/5(按魁奈的说法,这 2/5 产品构成地租)。租地农场主仅仅是这 2/5 产品的名义所有者,或者说,defacto\authornote{defacto(事实上),以区别于 dejure(法律上、依据法律)。——编者注}的掌握者。

因此,租地农场主用来支付地租的那一部分产品,为了在租地农场主和土地所有者之间流通,只需要一个和产品价值相等的货币额,虽然这个价值流通两次。首先,租地农场主用货币支付地租,然后土地所有者用同一笔货币购买产品。第一种情况是货币的简单的转移,因为这里货币只起\textbf{支付手段}的作用;因而这里是假定用货币支付的那个商品已经为货币支付者占有,而货币对他来说不是购买手段,他没有用货币换得等价物,倒是这个等价物早已在他的手中。相反,在第二种情况下,货币执行购买手段、商品流通手段的职能。这好比租地农场主用他支付地租的货币,从土地所有者那里赎回产品中属于土地所有者的一份。土地所有者用从租地农场主那里得到的同一笔货币(但实际上这笔货币是租地农场主在没有得到等价物的情况下交出的),向租地农场主买回产品。

因此,生产者以支付手段形式向收入所有者支付的同一个货币额,对收入所有者来说,是向生产者购买商品的购买手段。这样,货币从生产者手里到达收入所有者手里以及从收入所有者手里回到生产者手里这两次位置变换,仅仅表现了商品的一次位置变换,即商品从生产者手里到达收入所有者手里。因为假定生产者——就他的一部分产品来说——是收入所有者的债务人,所以生产者向收入所有者支付货币地租,实际上只是事后支付他(生产者)已经占有的商品的价值。商品在他的手里,但商品不属于他。因而,生产者用他以收入形式支付的货币把该商品购进,归自己所有。所以商品没有转手。货币的转手只不过表明商品\textbf{所有权的变换},商品仍然在生产者手里。由此发生了商品仅一次转手而货币却两次变换位置的情况。货币流通两次,是为了使商品流通一次。但是货币作为流通手段(购买手段)也只流通一次,另外一次它是作为支付手段流通的;在后面这种流通中,正如我在前面已经指出的,不发生商品和货币同时变换位置的情况。

事实上,如果租地农场主没有货币,只有产品,那末他只有在先出卖自己的商品之后,才能支付自己的产品;因而,在租地农场主能够以货币向土地所有者支付自己的商品之前,这个商品就必须已经完成它的第一形态变化。但是,即使在这种情况下,货币的位置变换还是多于商品。首先是 W—G:2/5 商品被卖掉,变成了货币。这里商品和货币同时变换位置。但是,后来同一笔货币从租地农场主手里转到土地所有者手里,商品却没有变换位置。这里有货币的位置变换,而没有商品的位置变换。这好比租地农场主有一个合伙人。租地农场主卖得了货币,但必须同他的合伙人分货币。更确切些说,从这 2/5 来看,就好比租地农场主的伙计卖得了货币。这个伙计必须把货币交给租地农场主,他无权把货币留在自己的口袋里。这里,货币的转手不表现商品的任何形态变化,而只是货币从它的直接掌握者手里转到它的所有者手里。可见,如果第一个收款人只是一个替自己雇主收款的代理人,情况就会如此。在这种情况下,货币甚至不是支付手段;货币仅仅是简单地从收款人(货币不属于他)手里转到货币所有者手里。

货币的这种位置变换,就象一种货币简单地兑换成另一种货币时所发生的位置变换一样,同商品的形态变化绝对无关。但是,当货币执行支付手段的职能的时候,总是假定支付人先取得了商品,以后才进行支付。至于租地农场主等等,他\textbf{不是取得了}这种商品:这种商品落到土地所有者手里以前,就在租地农场主手里,而且是\textbf{他的}产品的一部分。然而,\textbf{从法律上来说},租地农场主只有在他把用商品换得的货币交给土地所有者的时候,才成为这种商品的所有者。他对商品的权利发生了变化;商品本身仍然在他的手里。但是,以前商品在他手里是作为他\textbf{掌握}的东西,商品的所有者是土地所有者。而现在商品在他手里是作为归他自己所有的东西。仍然保留在同一个人手里的商品所发生的法律形式的变化,自然不会引起商品本身的转手。

\tsectionnonum{[(3)资本家和工人之间的货币流通问题]}

\tsubsectionnonum{[(a)把工资看成资本家对工人的预付的荒谬见解。把利润看成风险费的资产阶级观点]}

[425]\fontbox{~\{}由上面所说的还可以看出,用所谓资本家在把他的商品变为货币之前就已向工人预付货币这一点来“解释”资本家的利润,是多么荒谬。

\textbf{第一},如果我购买商品供自己消费,那末我是买者,而商品所有者是“卖者”,我的商品具有货币形式,他的商品还有待于变为货币,我决不会因此而取得任何“利润”。资本家只是在他消费了劳动之后,才对劳动支付代价,而其他商品则在被消费之前就得到支付。这个情况的产生是由于资本家购买的商品具有特殊的性质,这种商品实际上只是在被消费之后,才完全转到买者手里。货币在这里是作为支付手段出现的。资本家把“劳动”这个商品占为己有,总是\textbf{在}对它支付代价\textbf{之前}。他购买劳动,只是为了从出卖劳动产品中获得利润,但是这决不能成为他获得这笔利润的\textbf{理由}。这只是一个动机。而且在这种情况下无非是说:资本家购买雇佣劳动之所以获得利润,\textbf{是因为}他想从出卖这个劳动产品中获得利润。

\textbf{第}二,但是,有人会说,资本家毕竟把作为工资归工人的那一部分产品以货币形式预付给了工人,这样就使工人不必为了亲自把作为工资归他所有的那部分商品变为货币而备受辛苦、承担风险和花费时间。对于这种辛苦、风险和时间,工人难道不该向资本家支付一笔报酬吗\fontbox{?}因而,工人得到的产品份额难道不该比在其他情况下应得到的产品份额少一些吗\fontbox{?}

如果这样提问题,雇佣劳动和资本的全部关系就被抹杀了,从经济学上对剩余价值的解释也就勾销了。诚然,过程的结果是,资本家用来支付雇佣工人的基金实际上只是后者自己的产品,因此资本家和工人\textbf{事实上}是按一定的比例分享产品。但这个事实上的结果,同资本和雇佣劳动之间的交易(由商品交换本身的规律得出的从经济学上对剩余价值的论证,是以这种交易为基础的),是绝对没有关系的。资本家购买的是对劳动能力的暂时支配权,并且他只是在劳动能力发挥了作用、物化在产品中以后,才对这种支配权进行支付。就象货币作为支付手段起作用的一切地方一样,这里买和卖也是发生在货币实际脱离买者之前。但是,从这个在真正的生产过程开始前就已完成的交易以后,劳动已经\textbf{属于}资本家。作为产品从这个过程出来的\textbf{商品}也完全属于资本家。资本家用属于他的生产资料以及他所购买的(虽然还没有支付过代价)因而也属于他的劳动,生产了这个商品。这就好比他根本没有消费他人的劳动来生产商品一样。

资本家获得的利润,他实现的剩余价值,正是这样来的:工人作为商品卖给他的,不是物化在商品中的劳动,而是自己的劳动能力本身。如果工人作为第一种形式的商品所有者\endnote{马克思把以自己的劳动力为自己唯一商品的工人同“第一种形式的商品所有者”,即拥有供出卖的“不同于劳动能力本身的商品”的商品所有者对立起来。(参看前面第 159 和 111—113 页)。——第 334 页。}同资本家相对立,那末,资本家就不可能获得任何利润,不可能实现任何剩余价值,因为按照价值规律,是等价物同等价物相交换,等量劳动同等量劳动相交换。资本家的剩余价值正是这样来的:他向工人购买的不是商品,而是工人的劳动能力本身,而劳动能力所具有的价值比它的产品所具有的价值小,或者同样可以说,劳动能力所实现的物化劳动量比实现在劳动能力自身的物化劳动量大。但是,现在为了替利润辩护,利润的源泉本身被掩盖起来了,利润借以产生的整个交易也被抛开了。因为实际上(只要过程是连续不断的)资本家只是用工人自己的产品支付工人,工人支取的只是工人自己的产品的一部分,因而预付纯粹是假象,所以现在有人说:\textbf{在产品变为货币之前},工人已把产品中归自己所有的那一份卖给资本家了。(或许甚至是在产品有可能变为货币之前,因为工人的劳动虽然已物化在某一产品中,但在当时也许只造出了可出卖的商品的一部分,例如,房屋的一部分。)如果这样看问题,资本家就不成其为产品的所有者了,他借以\textbf{无偿}占有别人劳动的整个过程也就消失了。这样一来,互相对立的就都是商品所有者。资本家手里有货币,而工人卖给资本家的不是他的劳动能力,而是商品,即他自己的劳动借以物化的那部分产品。

这样,工人就会对资本家说:“在这 5 磅产品(譬如说纱)中,3/5 代表不变资本,属于你。2/5 即 2 磅代表我的新加劳动。因此你应当支付我 2 磅纱。现在就请付给我这两磅的价值吧。”在这种情况下,工人装进口袋的就不仅是工资,而且还有利润,简单说,就是和他以 2 磅纱的形式新加的物化劳动量相等的全部货币额。

资本家说:“难道我没有预付不变资本吗\fontbox{?}”

工人回答说:“对呀,正因为这样,你才拿走 3 磅,只付给我 2 磅。”

资本家坚持说:“但是,如果没有我的棉花和我的纱锭,你就不可能使你的劳动物化,不可能纺纱。所以,你应当另外付一笔报酬。”

工人回答说:“得了!如果我不用你的棉花和纱锭纺纱,棉花就会烂掉,纱锭就会生锈。[426]不错,你给自己扣下的 3 磅纱,只代表在 5 磅纱生产过程中耗费的,因而包含在这 5 磅纱内的你的棉花和纱锭的价值。但是,只有我的劳动把这些生产资料作为生产资料来消费之后,才保存了棉花和纱锭的价值。我并没有由于我的劳动有保存价值的能力而向你索取分文,因为除了纺纱(由于纺纱我得到 2 磅)之外,这种能力并没有花费我什么额外的劳动时间。这是我的劳动的一种自然赐予,它并不花费我什么,却保存了不变资本的价值。既然我并不因此向你索取分文,那末,你也不能因为我没有纱锭和棉花不能纺纱这一点而向我索取报酬。如果我不纺纱,你的纱锭和棉花就一钱不值。”

资本家无可奈何,就说:“2 磅纱,的确值 2 先令,这正代表了你的劳动时间的数量。但是,我在把这两磅纱卖掉之前,就得付给你这两磅的钱。也许我根本卖不掉。这是第一个风险。第二,我可能卖得比它的价格低。这是第二个风险。还有第三,无论如何,为了把它卖掉,必须花费时间。难道我就应当\textbf{无代价地}为你承担这两个风险,再加时间的损失吗\fontbox{?}天下绝没有无代价的事。”

工人回答说:“等一等,我们究竟是什么关系\fontbox{?}我们是作为\textbf{商品所有者}相对立的,\textbf{你是买者,我们是卖者},因为你想买产品中我们的一份即 2 磅,而这 2 磅中所包含的,实际上只是我们自己的物化劳动时间。现在你说,我们必须把自己的商品\textbf{低于}它的价值卖给你,使你因此得到的商品形式的价值比你现在拥有的货币形式的价值多。我们的商品的价值等于 2 先令。你想只给 1 先令,这样一来,——因为 1 先令所包含的劳动时间和 1 磅纱所包含的一样多,——你换进的价值就比你换出的价值多 1 倍。相反,我们得到的不是等价,而是等价的一半,不是 2 磅纱的等价,而只是 1 磅纱的等价。你凭什么提出这个违反价值规律、违反商品按价值交换的规律的要求\fontbox{?}凭什么\fontbox{?}凭你是买者,我们是卖者,凭我们的价值是以纱的形式、商品的形式存在,而你的价值是以货币的形式存在,凭纱形式的一定价值与货币形式的同一价值相对立。但是,老兄!这不过是形式的变换,这种变换和价值的\textbf{表现}有关,但并不使\textbf{价值量}发生变化。或者你有一种幼稚的看法,认为任何商品在货币形式上都具有\textbf{较大的}价值,所以任何商品都必须\textbf{低于}它的价格——就是说,低于代表它的价值的货币额——出卖\fontbox{?}但是,不对,老兄,商品在货币形式上并不具有较大的价值;它的价值量并没有变化,它不过纯粹表现为交换价值而已。

想想看,老兄,你使自己处于多么尴尬的境地!你的主张是:卖者始终必须把商品\textbf{低于}它的价值卖给\textbf{买者}。从前,当我们卖给你的还不是我们制造的商品,而是我们的劳动能力本身的时候,在你那里,情况确实是这样。那时,你购买劳动能力固然按照它的价值,但是你购买我们的劳动本身却\textbf{低于}它所体现的价值。不过,我们抛开这种不愉快的回忆吧。谢天谢地,自从你自己作出决定,要我们不再把劳动能力当作商品卖给你,而把商品本身即我们劳动的产品卖给你以来,我们就不再处于这种地位了。我们回过来谈谈你所处的那种尴尬境地吧。按照你提出的新规律,卖者为了把他的商品变为货币,不是简单地付出他的商品,不是简单地用他的商品来同货币交换,而是把商品\textbf{低于}它的价格出卖。按照这个规律,买者总是欺骗和诈取卖者,而这个规律应该对任何买者和卖者都是同样有效的。假定我们同意你的建议,但有一个条件,你自己也要服从你臆想出来的规律,即买者替卖者把商品变为货币,卖者必须把自己的商品的一部分\textbf{无代价地}送给买者。例如,你买我们的 2 磅纱,价值 2 先令,只付给 1 先令,于是你赚了 1 先令即 100\%。但是,现在,你从我们这儿买了属于我们的 2 磅纱之后,你手里就有 5 磅纱,价值 5 先令。当然,你打算做一桩有利可图的买卖。5 磅纱只花了你 4 先令,而你想按 5 先令把它卖出去。\textbf{你的买者}说:‘且慢!你的 5 磅纱是商品,你是卖者。我有同一价值的货币,我是买者。因此,按照你承认的规律,同你进行的交易应当给我带来 100\%的利润。所以你必须把这 5 磅纱低于它的价值 50\%,即按 2+(1/2)先令,卖给我。我给你 2+(1/2)先令,换你价值 5 先令的商品,这样一来,就从你那里赚了 100\%,因为同一道理,人人适用。’

[工人继续说道:]老兄,你自己看看,你的新规律会得出什么结果;你只会落得自己欺骗自己,虽然你一时是买者,但过后又成了卖者。在这种情况下,你作为卖者所损失的,会比你作为买者所得到的更多。你好好想想吧!你现在想从我们手上买去 2 磅纱,但是在这 2 磅纱生产出来之前,难道你没有买过东西吗\fontbox{?}如果没有买,就根本不会有这 5 磅纱。[426a]难道你没有预先购买现在由 3 磅纱代表的棉花和纱锭吗\fontbox{?}在购买这些东西时,利物浦的棉花批发商和奥尔丹的纺机厂主作为\textbf{卖者}同你对立,而你作为\textbf{买者}同他们对立;他们是商品的代表,而你是货币的代表,这同现时我们有幸或不幸地相互对立着的关系是完全一样的。如果你根据你替他们把商品变为货币,而他们替你把货币变为商品,他们是卖者,你是买者这一点理由,就要求他们把一部分棉花和纱锭\textbf{无代价地}让给你,或者同样可以说,要求他们把这些商品低于它们的价格(和它们的价值)卖给你,那末,那个狡猾的棉花商人和你的爱逗乐的奥尔丹同行不会嘲笑你吗\fontbox{?}他们可没有冒什么风险嘛,他们得到了现金,得到了纯粹独立形式的交换价值。而你这方面要担多少风险呀!首先,要用纱锭和棉花制成纱,经历生产过程的一切风险;然后把纱卖出,再变为货币,也要担风险。纱能否按它的价值卖出,是高于还是低于价值卖出,这是一个风险。也许它根本卖不出去,根本不能再变为货币,这又是一个风险。至于纱本身,你一点也不感兴趣。你不能拿纱来吃、来喝,除了把它卖出去以外,没有任何用处。而且无论如何,为了把纱再变为货币(这里还包括把纱锭和棉花再变为货币),还要有时间的损失!你的同行将反驳你说:老兄!别装傻了!别说废话了!你想怎样利用我们的棉花和纱锭,你要用它们做什么,这关我们屁事!把它们烧掉、扔掉,随你的便,只要把它们的价钱付了!想得倒好!你当上了棉纺厂主,我们就得把我们的商品白白送给你,看来你感到这一行不很顺手,你把这一行的风险和危险太夸大啦!那就别经营棉纺业啦,不然就别抱着这些荒唐的想法到市场上来!”

听了工人这番话,资本家带着轻蔑的微笑回答说:“可见,你们这些家伙只听钟声响,不知钟声哪里来。你们谈的是你们根本不懂的事情。你们以为我把现金付给利物浦坏蛋和奥尔丹小子吗\fontbox{?}决不是!我付给他们的是期票,而在期票到期之前,利物浦坏蛋的棉花实际上已经加工成纱,并且卖出去了。你们的情况却完全不同。你们是要拿现金的。”

工人说:“好极了,但是利物浦坏蛋和奥尔丹小子把你的期票拿去干什么啦\fontbox{?}”

资本家喊道:“他们拿去干什么\fontbox{?}!问得多蠢!他们拿去找银行家贴现呗!”

“他们付给银行家多少\fontbox{?}”

“多少\fontbox{?}货币现在十分便宜。我想他们大约付 3\%的贴现利息,就是说,不是付期票金额的 3\%,而是根据距期票到期的时间,按年利 3\%计算。”

工人说:“那就更好啦,付给我们 2 先令,这是我们商品的价值。或者付给我们 12 先令吧,因为我们想按周计算,不按天计算。所以,就依年利 3\%从这个金额中扣除 14 天的利息吧。”

资本家说:“但这张期票太小,没有一个银行家肯贴现的。”

工人回答说:“好吧,我们是 100 人。这也就是说,你应当付给我们 1200 先令。给我们开一张期票吧。合 60 镑,拿去贴现,不算太小的数目了。而且,是你自己给它贴现,这个数目对你来说,一定不会太小,因为你想从我们身上赚得的利润,也正是这个数目。扣除的是微不足道的数目。既然这样一来我们能够把我们产品的大部分完全拿到手,那我们很快也就不再需要你贴现了。自然,我们给你的信贷,不会比交易所经纪人给你的多——总共只有 14 天。”

如果认为工资的来源(在完全歪曲真实关系的情况下),是对总产品中属于工人的那部分价值的贴现,即资本家提前用\textbf{货币}把这部分支付给工人,那末资本家势必开给他们期限很短的期票,就象他开给做棉花生意的商人等等的那种短期期票一样。工人就会得到自己产品的大部分,而资本家很快就会不成其为资本家了。对于工人来说,资本家就会从产品的所有者变成单纯的银行家。

此外,如果说资本家有商品低于其[427]价值出卖的风险,那他也有商品高于其价值出卖的机会。而如果产品卖不出去,工人就会被抛到街头。如果产品价格长期低于市场价格,工人的工资就要下降到平均水平之下,工厂就开工不足。所以,工人承担的风险最大。

\textbf{第三},租地农场主用货币支付地租,工业资本家用货币支付利息,他们为了有可能实行这些支付,必须事先把自己的产品变为货币,可是谁也不会想到,仅仅因为这个缘故,他们就可以从地租或利息中扣下一部分。\fontbox{\}~}

\tsubsectionnonum{[(b)工人向资本家购买商品。不表现再生产的货币回流]}

工业资本家和工人之间流通的那部分资本(即构成可变资本的那部分流动资本),也发生货币流回它的出发点的现象。资本家以货币支付工人的工资;工人用这笔货币向资本家购买商品,于是货币流回资本家手里。(在实践中,是回到资本家的银行家手里。但是,事实上,银行家对单个资本家来说是总资本的代表,因为总资本表现为\textbf{货币}。)这种货币回流按其本身来说根本不表现再生产。资本家用货币向工人购买劳动,工人用这笔货币向资本家购买商品。同一笔货币,最初表现为购买劳动的手段,后来表现为购买商品的手段。这笔货币之所以流回资本家手里,是因为资本家对同样一些人来说,起先作为买者出现,后来又作为卖者出现。他作为买者,货币从他那里出去,他作为卖者,货币又回到他手里。相反,工人起先作为卖者出现,后来又作为买者出现,他起先收进货币,后来支出货币,而资本家与工人相反,起先支出货币,后来收回货币。

在资本家方面,这里发生了 G—W—G 运动。他用货币购买商品(劳动能力)。他用这种劳动能力的产品(商品)购买货币,或者说,他把这个产品卖回给从前作为卖者同他相对立的工人。相反,工人代表 W—G—W 流通。他卖出自己的商品(劳动能力),并用卖得的货币买回自己的产品(他生产的商品)的一部分。当然,有人会说:工人卖出商品(劳动能力)换得货币,用这笔货币购买商品,然后又出卖自己的劳动能力,所以,从工人方面说,也出现 G—W—G 过程;并且,因为货币是在工人和资本家之间不断来回流通的,所以要看从哪一方面来考察这个过程,同样可以说,工人也象资本家那样代表 G—W—G 运动。然而,资本家是买者。过程是从资本家而不是从工人重新开始的;货币必然流回资本家手里,是因为工人必须购买生活资料。这里,正象在一方的流通形式是 G—W—G,另一方的流通形式是 W—G—W 的一切过程中一样,可以看出,交换过程的目的,对一方来说,是交换价值,货币,因而也就是价值的增殖;对另一方来说,则是使用价值,是消费。这种情况也发生在上面提到的第一种场合的货币回流上。在那种场合,租地农场主一方实现 G—W—G 过程,而土地所有者一方实现 W—G—W;这是很明显的,只要我们注意到,土地所有者用来向租地农场主购买商品的 G 是地租的货币形式,因而已经是 W—G 的结果,是以实物形式实际属于土地所有者的那部分产品的转化形式。

这种 G—W—G 过程在工人和资本家的关系中,只是资本家用来支付工资的那笔货币流回资本家手里的表现,它本身不表现再生产过程,只表现买者对同样一些人来说又是卖者这一情况。它也不表现作为资本的货币,就是说,不表现 G—W—G′过程,在那里,第二个 G′是比第一个 G 大的货币额,因而 G 是自行增殖的价值(资本)。相反,这只是表示同一货币额(往往还少一些)在形式上流回出发点。(这里所说的资本家,当然是指资本家阶级。)因此,我在第一部分\endnote{马克思指《政治经济学批判》第一分册第二章第三节《货币》的头两段(见《马克思恩格斯全集》中文版第 13 卷第 112—113 页)。——第 342 页。}中说 G—W—G 的形式必定是 G—W—G′,是不对的。这种形式可以只表现货币回流的形式,我在同一个地方用买者同时又成为卖者的情况来说明货币流回同一出发点的循环运动时,已暗示了这一点。\endnote{马克思指《政治经济学批判》第一分册的下面这段话:“他们[商品所有者]作为买者付出的货币,一到他们重新作为商品的卖者出现时,又回到他们手里。因此,商品流通的不断更新就反映成:货币不仅在资产阶级社会的整个表面上不断地转手,而且同时画出许多不同的小循环,从无数不同地点出发,又回到这些地点,以便重新再作同样的运动。”(见《马克思恩格斯全集》中文版第 13 卷第 90 页)——第 342 页。}

但资本家不是靠\textbf{这种}货币回流发财的。例如,他付出了 10 先令工资。工人用这 10 先令向他购买商品。资本家给工人 10 先令商品,换取他的劳动能力。如果资本家用实物形式给工人以价格 10 先令的生活资料,这里就根本不发生货币流通,因而也就不发生货币回流。可见,这种货币回流现象同资本家发财无关。资本家所以发财,是因为资本家在生产过程中占有的劳动,比他以工资形式支出的劳动多,因此他的产品大于这个产品的生产费用,可是,资本家支付给工人的货币也决不可能少于工人用来向资本家购买商品的货币。这里,这种形式上的货币回流同发财毫无关系,因而并没有表现[428]作为资本的 G,正如当用来支付地租、利息和税收的货币流回支付地租、利息和税收的人手里的时候,这种回流本身不包含价值的增殖或者[哪怕是]价值的补偿一样。

只要 G—W—G 表示货币流回资本家手里的形式上的回流,那末它只是表明,资本家开出的货币形式的支票由他自己的商品来兑现。

对这种货币回流(即货币回到它的出发点)作错误解释的例子,见前面论德斯杜特·德·特拉西那一节\endnote{见本册第 277—291 页。并见马克思《资本论》第 2 卷第 20 章第 13 节:《德斯杜特·德·特拉西的再生产理论》。——第 343、364 页。}。作为第二个例子——特别适用于工人和资本家之间的货币流通——后面要引用布雷的一段话\endnote{论布雷一节在手稿第 X 本第 441—444 页。这一节没有写完,也没有接触到布雷对工人和资本家之间的货币流通问题的观点。关于布雷对货币的本质和作用的观点,见马克思 1847 年手稿《工资》(《马克思恩格斯全集》中文版第 6 卷第 641 页);《政治经济学批判大纲》1939 年莫斯科德文版第 55、690、754 页;马克思 1858 年 4 月 2 日给恩格斯的信;《政治经济学批判》(《马克思恩格斯全集》中文版第 13 卷第 76 页)。——第 343 页。}。最后关于贷放货币的资本家,要引用\textbf{蒲鲁东}\endnote{后面,在手稿第 X 本第 428 和 437 页,马克思对蒲鲁东在这个问题上的观点作了简短的说明(见本册第 344—345 页)。——第 343 页。}。

G—W—G 这种回流形式,凡是在买者又成为卖者的地方都存在;因而它在整个商业资本中都存在。在商业资本中,一切商人彼此之间都是为卖而买和为买而卖。可能,买者——G——不能把商品(例如米)卖得比买来时贵;他可能甚至不得不低于它的价格把它出卖。在这种情况下,就只发生简单的货币回流,因为这里买变为卖并没有使货币表现为自行增殖的价值即资本。

例如,在不变资本交换时,也发生这种情况。机器制造业者向铁生产者买铁,又把机器卖给铁生产者。在这种情况下,货币流回机器制造业者手里。货币作为买铁的购买手段被付出。后来,它成为铁生产者买机器的购买手段,因此流回机器制造业者手里。机器制造业者付出货币换进铁,又收进货币付出了机器。这里,同一数量的货币可以使两倍的价值流通。例如,机器制造业者用 1000 镑买铁;铁生产者用这 1000 镑买机器。铁和机器的价值加在一起等于 2000 镑。但是,这样一来,就必须有 3000 镑在运动:1000 镑货币,1000 镑机器和 1000 镑铁。如果资本家进行实物交换,商品转手就不必有一个法寻流通。

如果资本家彼此实行结算,货币对他们起支付手段的作用,那末情况也是如此。如果流通的是纸币或信用货币(银行券),那末,情况有一点变化。在这时,还有 1000 镑银行券,但它没有“内在价值”。不管怎样,这里也有 3000 镑:1000 镑铁,1000 镑机器和 1000 镑银行券。但这 3000 镑之所以存在,如同第一种情况一样,只是因为机器制造业者手里有 2000 镑:1000 镑机器和 1000 镑货币(金银或银行券)。在这两种情况下,铁生产者还给机器制造业者的,都只是后者(即货币),而铁生产者之所以得到货币,只是因为作为买者的机器制造业者并不直接又是卖者,他不是用商品而是用货币来支付第一种商品铁。当他用商品来支付,即把自己的商品卖给铁生产者的时候,后者就把货币还给他。因为支付并不是双重的——一次用货币,再一次用商品。

在这两种情况下,金或银行券都代表由机器制造业者买去的商品的转化形式,或者代表后来由别人买去的商品的转化形式;或者代表虽然没有被购买但已转化为货币的那种商品,就象土地所有者(他的祖先等等)\endnote{括号里的话是马克思打算以后加以发挥的思想。马克思很可能指魁奈在土地私有制问题上的辩护论观点。按照这种观点,土地所有者所以对土地持有权利,是因为他们的祖先使处女地变成了适于耕种的土地。在马克思写的《反杜林论》第二编第十章中,对重农学派的这一观点作了如下说明:“魁奈认为,……按照‘自然法’说来,土地所有者的真正职能正是在于‘关心良好的管理和关心维持他们世袭财产所必需的费用’,或者……在于 avancesfoncières,即用来准备土地并供给农场以一切必需东西的费用,这些费用,使租地农场主可以把其全部资本只用在真正的耕种事业上。”(《马克思恩格斯全集》中文版第 20 卷第 274 页)——第 344 页。}获得收入时的情况那样。因而这里货币的回流只是表示:原来为换得商品而把货币付出即投入流通的人,由于卖出他投入流通的另一种商品,又把货币从流通中抽回来。

刚才谈的这 1000 镑,在资本家之间流通,可以在一天之内经过四五十只手,这只不过是资本的转手而已。机器转到铁生产者手里,铁转到农民手里,谷物转到淀粉厂主或酒精厂主手里,等等。最后,这 1000 镑可能又落到机器制造业者手里,再经过后者转到铁生产者手里等等。由此可见,用这 1000 镑就可以使 40000 镑以上的资本流通,而且货币可能不断回到第一个把它投入流通的人手里。用这 40000 镑赚得的利润,有一部分会转化为利息,由各个资本家支付,例如机器制造业者向贷给他 1000 镑的人付利息,铁生产者向借给他 1000 镑(这笔钱他早已用在煤炭等等或工资等等上面了)的人付利息,等等。蒲鲁东先生由此得出结论说,这 1000 镑带来了由 40000 镑赚得的\textbf{全部利息}。这样一来,如果利率等于 5\%,利息就等于 2000 镑。根据这一假设,他正确地算出 1000 镑生出 200\%的利息。这位鼎鼎大名的政治经济学批评家原来如此!\authornote{[437}上面提到的蒲鲁东的那段话,是这样说的:

“抵押债务总额,据最了解情况的作者说,达到 120 亿,据其他一些人说,达 160 亿;期票债务总额至少有 60 亿,股票大约 20 亿,国债 80 亿;总计 280 亿。必须指出,所有这些债务,是按 4\%、5\%、6\%、8\%、12\%,甚至 15\%的利息借来的或视同借来的货币。我假定前三类债务的平均利息是 6\%;200 亿就有利息 12 亿。此外还加上国债利息约 4 亿,总计:10 亿资本的年息为 16 亿。”(第 152 页)因而是 160\%。因为“在法国,现金总额,——我不想说一般存在的,我说的是在流通中的货币总额,包括银行的库存现金在内,——按照最流行的估计,不超过 10 亿”。(第 151 页)“当交换结束时,货币又空出来了,因而又可以重新贷出……货币资本从一次交换到另一次交换,总是不断回到它的出发点。由此可见,每次由同一个人的手重新把这些货币贷出,总能给同一个人带来利润。”(第 153—154 页)(《无息信贷。弗·巴师夏先生和蒲鲁东先生的辩论》1850 年巴黎版\endnote{马克思在《剩余价值理论》第三册补充部分《收入及其源泉。庸俗政治经济学》(手稿第 935—937 页)中,批判了蒲鲁东在《无息信贷》一书中发挥的关于货币资本的作用和利息本质的庸俗观点。并见马克思《资本论》第 3 卷第 21 章。——第 345 页。})[437]]

但是,虽然 G—W—G 过程在代表资本家和工人之间的货币流通时本身不表现任何再生产行为,这个过程的经常重复,货币的不断回流却表示再生产。任何一个买者,如果不把他卖出的商品再生产出来,他就根本不可能经常作为卖者出现。诚然,所有不靠地租、利息或税收生活的人都是这样。但是就一部分人来说,在过程结束时总发生货币的回流 G—W—G,例如在资本家对工人,或对土地所有者,或对食利者的关系上就是如此(从后两者来说是单纯的货币回流)。就另一部分人来说,过程结束则购得了商品,即发生 W—G—W 过程,例如在工人方面就是这样。工人使这个过程不断周而复始。工人总是作为卖者,而不是作为买者开始行动。[429]仅仅表明花费收入的整个货币流通,也是这种情况。例如,资本家自己每年就消费一定数量的产品。他把自己的商品变为货币,以便用这些货币来购买他要想最终消费的商品。这里是 W—G—W,并不发生货币流回资本家手里的现象,但是货币流回卖者(例如店主)手里,而卖者的资本靠资本家花费其收入来补偿。

但是,我们还看到收入同收入交换,即收入之间的流通。屠宰业者向面包业者买面包,面包业者向屠宰业者买肉;他们两者都消费自己的收入。屠宰业者自己吃肉是不付钱的,同样,面包业者自己吃面包也不付钱。他们都是以实物形式消费收入的这一部分。但可能有这种情况:面包业者向屠宰业者买的肉,对屠宰业者来说不是补偿资本,而是补偿收入,即补偿他出卖的肉中不是纯粹代表他的利润,而是代表他的利润中他自己想当作收入消费掉的那一部分。屠宰业者向面包业者买面包,对屠宰业者来说,也是花费他的收入。双方结算的时候,只要有一个人支付差额就行了。他们相互买卖彼此抵销的部分不加入货币流通。但是,假定面包业者要支付差额,而且这个差额对屠宰业者来说代表收入。那末屠宰业者就要用面包业者的货币去买其他消费品。假定这是 10 镑,他支付给裁缝。如果这 10 镑对裁缝来说代表收入,那末裁缝也以类似的方式花费这 10 镑。他又用这 10 镑去买面包等等。这样,货币就流回面包业者手里,但对面包业者来说,这笔货币所补偿的已经不是收入,而是资本了。

还可以提出这样一个问题:在由资本家进行的、代表自行增殖的价值的 G—W—G 过程中,资本家从流通中抽出的货币比他投入流通的货币多。(这曾经是货币贮藏者的真正愿望,但没有实现。因为他以金银形式从流通领域抽出的价值,并不比他以商品形式投入流通的价值多。他现在有更多的货币形式的价值,而过去他有更多的商品形式的价值。)假定资本家的商品的全部生产费用等于 1000 镑。他按 1200 镑把自己的商品卖出去,因为现在他的商品中包含 20\%即 1/5 的无酬劳动,这种劳动是资本家虽然没有支付过代价却拿去出卖的。全体资本家即工业资本家阶级不断从流通中抽出的货币,怎么可能比他们投入流通的货币多呢\fontbox{?}从另一方面可以说,资本家不断投入流通的东西比他从流通中抽出的东西多。他要支付他的固定资本。但是,他出卖固定资本是随着固定资本消费的程度,一部分一部分地进行的。固定资本虽然完全进入商品的生产过程,但它始终是以小得多的部分加到商品的\textbf{价值}中去。假定固定资本的流通期间为 10 年,那末每年加到商品中去的只是它的 1/10,而其余 9/10 不加入货币流通,因为这 9/10 根本没有以商品形式进入流通。这是一个问题。

这个问题我们以后再来考察,\endnote{马克思在《资本论》第二卷第十七章、第二十章第五节和第十二节、第二十一章(特别是第一节中的第一小节《贮藏货币的形成》)中全面考察了这个问题。——第 347、349、365 页。}现在回过头来谈魁奈。

但是,还有一个问题,先得说一说。银行券流回到办理期票贴现和用银行券贷款的银行,是同上面考察的货币回流完全不同的现象。在这种情况下,商品转化为货币是预先实现的。商品取得货币形式还在它出卖以前,甚至在它生产出来以前。但也有可能,它已经卖出了(凭期票)。\textbf{无论如何},它还没有\textbf{被支付},还没有再转化为货币。因此,这种向货币的转化,无论在哪一种情况下,都是预先实现的。一旦商品被卖掉(或\textbf{被认为}已卖掉),货币就流回银行:或者以本银行的银行券的形式流回,于是这种银行券便退出流通;或者以别家银行的银行券的形式流回,这种银行券便同本银行的银行券相交换(在银行家之间),这样一来,两种银行券都退出流通,回到它们的出发点;或者以金银的形式流回。如果这些金银被用来兑换第三者手中的银行券,那末银行券就回到银行。如果银行券不要求兑换,那末流通中的金银就会减少,减少的正是代替银行券存放在银行库里的金银的数目。

在所有这些情况下,过程是这样的:

货币的现有存在(商品向货币转化)已经预先实现。当商品真正转化为货币时,它是第二次向货币转化。但商品的这种第二次货币存在会回到出发点,抵偿、替代商品的第一次货币存在,离开流通,回到银行。很可能,表现商品的这种第二次货币存在的,就是表现过商品的第一次货币存在的\textbf{同一}批银行券。例如,给纺纱厂主贴现了一张期票。这张期票是他从织布业者那里得到的。纺纱厂主用贴现得来的 1000 镑支付煤炭、棉花等等。这些银行券被用来支付商品,经过各种不同的人的手,最后被用于购买麻布,于是银行券落到织布业者手里。期票到期,织布业者就把这些银行券付给纺纱业者,后者则把它们归还银行。在商品预先实现向货币转化之后发生的第二次的(后遗的)向货币转化,除了第一次的货币以外,[430]完全不需要另外的货币。这样,似乎纺纱业者实际上什么也没有得到,因为开始他借了银行券,而在过程结束时,他收回银行券,把它们归还银行。但是,实际上,同一批银行券在这个时期内起了流通手段和支付手段的作用,而纺纱业者把其中一部分用来偿付债务,一部分用来购买再生产纱所必需的商品,从而也实现了通过剥削工人而创造出来的剩余价值,他现在可以把剩余价值的一部分付给银行。而且也是用货币,因为流回他手里的货币要比他支出的、预付的和花费的货币多。怎么发生的呢\fontbox{?}这又属于我们留到后面再考察的那个问题。\endnote{马克思在《资本论》第二卷第十七章、第二十章第五节和第十二节、第二十一章(特别是第一节中的第一小节《贮藏货币的形成》)中全面考察了这个问题。——第 347、349、365 页。}

\tsectionnonum{[(4)《经济表》上租地农场主和工业家之间的流通]}

现在我们回过来讲魁奈。我们要考察第三和第四个流通行为。

P(土地所有者)向 S\endnote{马克思在这里用如下符号代表魁奈著作中出现的三个阶级:P——classedesPropriétaires(土地所有者阶级),S——classeStérile(不生产阶级——工业家),F——Fermiers,classeproductive(租地农场主,生产阶级)。——第 349 页。}(从事工业的“不生产”阶级)购买 10 亿工业品(在表上是 a—c 线\endnote{马克思在这里使用的字母符号(和标记)使《经济表》一目了然,无论在施马尔茨的著作中还是在魁奈的著作中都没有这样清楚。用两个字母(a—b,a—c,c—d 等等)来标明每一条线,使人能确定线的方向,即这条线是从哪个阶级到哪个阶级(方向按字母表上字母的顺序确定,a—b,a—c,c—d 等等)。例如,a—b 线表示土地所有者阶级和生产阶级(租地农场主)之间的流通以土地所有者阶级为出发点(后者向租地农场主购买食物)。用两个字母来标明每一条线,同时表明了货币的运动和商品的运动。例如,a—b 线表示货币的运动(土地所有者阶级向生产阶级支付 10 亿货币);但是这条线从相反的方向(b—a)来看,就表明商品的运动(生产阶级交给土地所有者阶级 10 亿食物)。虚线 a—b—c—d 由以下几个环节组成:(1)a—b 段表示土地所有者和生产阶级之间的流通(土地所有者向租地农场主购买 10 亿食物);(2)a—c 段表示土地所有者和不生产阶级——工业家之间的流通(土地所有者向工业家购买 10 亿工业品);(3)c—d 段表示不生产阶级和生产阶级之间的流通(工业家向租地农场主购买 10 亿食物)。a′—b′线表示生产阶级和不生产阶级之间的流通(租地农场主向工业家购买 10 亿工业品)。a″—b″线表示不生产阶级和生产阶级之间的最后的流通(工业家向租地农场主购买工业生产所必需的 10 亿原料)。——第 324、349、352 页。})。这里,10 亿货币使同额商品进入流通。\fontbox{~\{}因为在这种情况下发生的是一次交换。如果 P 是分次向 S 购买商品,而 P 同样是分次从 F(租地农场主)那里收到地租,那末,这 10 亿工业品就可能,比方说,用 1 亿购买。因为 P 向 S 购买 1 亿工业品,S 向 F 购买 1 亿食物,F 向 P 支付 1 亿地租;如果这样重复 10 次,那末就有 10×1 亿的商品从 S 转到 P 和从 F 转到 S,而有 10×1 亿的地租从 F 转到 P。于是,整个流通用 1 亿就完成了。但是,如果 F 一次支付全部地租,那末,S 手里的 10 亿和流回到 F 手里的 10 亿,就可能有一部分存放着,一部分在流通。\fontbox{\}~}现在有 10 亿商品从 S 转到 P,相反,有 10 亿货币从 P 转到 S。这是简单流通。货币和商品只是按相反方向转手。但是,除了租地农场主已经卖给 P 因而进入消费的 10 亿食物以外,还有 10 亿工业品由 S 卖给 P 而进入消费。必须指出,这些商品在新的收获以前就存在了(否则,P 就不能用新收获的产品购买它们)。

S 再用 10 亿向 F 购买食物。于是\textbf{总产品的第二个}1/5 离开流通,进入消费。在 S 和 F 之间,这 10 亿执行了流通手段的职能。但同时,这里发生了两种现象,这两种现象是在 S 和 P 之间的过程中没有发生的。在后面这个过程中,S 把他的产品的一部分,即 10 亿工业品,再转化为货币。但是,在同 F 交换中,他把货币再转化为食物(在魁奈那里,就是转化为工资),从而补偿他投在工资上的、已消费的资本。10 亿变为生活资料的这种再转化,在 P 那里表示单纯的消费,而在 S 那里则表示生产的消费,表示再生产,因为 S 把他的商品的一部分再转化为商品的生产要素之一,即生活资料。因此,商品的这一形态变化,它从货币到商品的再转化,在这里同时表示商品的\textbf{实际的}(而不仅是\textbf{形式上的})形态变化的开始,表示商品的再生产的开始,商品变为它自己的生产要素的再转化的开始。这里同时也发生资本的形态变化。相反,从 P 这方面来说,只是收入从货币形式转化为商品形式。这只是表示消费。

第二,当 S 向 F 购买 10 亿食物的时候,F 作为货币地租付给 P 的第二个 10 亿,就回到 F 手里。不过,它们之所以回到 F 手里,只是因为 F 用价值 10 亿的商品等价物把它们从流通中再抽回,赎回。这就好比土地所有者向他(除了第一个 10 亿以外)买了 10 亿食物一样,就是说,好比土地所有者以商品形式从租地农场主那里获得了他的货币地租的第二部分,然后又用这些商品换了 S 的商品一样。S 不过是代替 P 以商品形式提取 F 已经用货币付给 P 的那 20 亿的第二部分。如果用实物支付,那就是 F 给 P20 亿食物,P 自己消费其中 10 亿,用另外 10 亿向 S 交换工业品。在这种情况下,只会是:(1)20 亿食物从 F 转到 P;(2)P 和 S 之间进行物物交换,前者用 10 亿食物去换 10 亿工业品,后者则相反。

实际上不是这样,而是发生了四个行为:[431](1)20 亿货币从 F 转到 P;(2)P 向 F 购买 10 亿食物;货币流回 F 手里,执行流通手段的职能;(3)P 用 10 亿货币向 S 购买工业品;货币执行流通手段的职能,按与商品运动相反的方向转手;(4)S 用这 10 亿货币向 F 购买食物;货币执行流通手段的职能。对于 S 来说,货币同时作为资本流通。它流回 F 手里,因为现在那第二个 10 亿的食物——土地所有者从 F 那里得到过这 10 亿支票——被提取了。但是,货币不是直接从土地所有者那里流回 F 手里的,货币先在 P 和 S 之间起了流通手段的作用,货币在提取 10 亿食物之前,中途先提取了 10 亿工业品,并把它们从工业家手里转给土地所有者,在这以后,货币才流回 F 手里。由工业家的商品转化为货币(在同土地所有者交换中),和接着而来的由货币转化为食物(在同租地农场主交换中),在 S 方面,都是他的资本的形态变化,先是变成货币的形式,然后变成资本再生产所必需的构成要素的形式。

因此,以上四个流通行为的结果是:土地所有者花完了他的收入,一半花在食物上,一半花在工业品上。这样一来,他以货币地租形式得到的 20 亿就花完了。其中一半从他那里直接流回租地农场主手里,另一半通过 S 间接流回租地农场主手里。而 S 把他的成品的一部分脱了手,用食物,也就是用再生产的一个要素来补偿。通过这些过程,结束了有土地所有者出现的流通。离开流通进入消费(一部分是非生产消费,一部分是生产消费,因为土地所有者已经用他的收入部分地补偿了 S 的资本)的是:(1)10 亿食物(新收获的产品);(2)10 亿工业品(上年收获的产品);(3)10 亿食物,这个 10 亿是加入再生产的,就是用来生产 S 在次年拿去同土地所有者的一半地租相交换的那些商品的。

20 亿货币又在租地农场主手里了。租地农场主现在为了补偿他的“年预付和原预付”(因为它们一部分由劳动工具等构成,一部分由生产中所消费的其他工业品构成),向 S 购买 10 亿商品。这是简单的流通过程。于是 10 亿转到 S 手里,而 S 的以商品形式存在的产品的第二部分转化为货币。双方都发生资本的形态变化。租地农场主的 10 亿再转化为再生产过程所必需的生产要素。S 的成品再转化为货币,完成了从商品到货币的\textbf{形式上的}形态变化,没有这种形态变化,资本就不能再转化为自己的生产要素,因而也就不能进行再生产。这是第五个流通过程。有\textbf{10 亿工业品}(上年收获的产品)离开流通,进入再生产消费(a′—b′)\endnote{马克思在这里使用的字母符号(和标记)使《经济表》一目了然,无论在施马尔茨的著作中还是在魁奈的著作中都没有这样清楚。用两个字母(a—b,a—c,c—d 等等)来标明每一条线,使人能确定线的方向,即这条线是从哪个阶级到哪个阶级(方向按字母表上字母的顺序确定,a—b,a—c,c—d 等等)。例如,a—b 线表示土地所有者阶级和生产阶级(租地农场主)之间的流通以土地所有者阶级为出发点(后者向租地农场主购买食物)。用两个字母来标明每一条线,同时表明了货币的运动和商品的运动。例如,a—b 线表示货币的运动(土地所有者阶级向生产阶级支付 10 亿货币);但是这条线从相反的方向(b—a)来看,就表明商品的运动(生产阶级交给土地所有者阶级 10 亿食物)。虚线 a—b—c—d 由以下几个环节组成:(1)a—b 段表示土地所有者和生产阶级之间的流通(土地所有者向租地农场主购买 10 亿食物);(2)a—c 段表示土地所有者和不生产阶级——工业家之间的流通(土地所有者向工业家购买 10 亿工业品);(3)c—d 段表示不生产阶级和生产阶级之间的流通(工业家向租地农场主购买 10 亿食物)。a′—b′线表示生产阶级和不生产阶级之间的流通(租地农场主向工业家购买 10 亿工业品)。a″—b″线表示不生产阶级和生产阶级之间的最后的流通(工业家向租地农场主购买工业生产所必需的 10 亿原料)。——第 324、349、352 页。}。

最后,S 把这 10 亿货币——他的一半商品现在以这 10 亿货币的形式存在——再转化为商品的生产条件的另一半,即原料等等(a″—b″)。这是简单流通。这对 S 来说,同时也是他的资本变为适于再生产的形式的形态变化,而对 F 来说,是他的产品变为货币的再转化,现在,“总产品”的最后 1/5 离开流通,进入消费。

总之:1/5 加入租地农场主的再生产过程,不进入流通,1/5 被土地所有者消费掉;合计 2/5;2/5 由 S 取得;共计 4/5。\endnote{马克思在这里和在后面都采用魁奈的说法:只有 1/5 的农业总产品不进入流通,而由生产阶级以实物形式享用。马克思在手稿第 XXIII 本第 1433—1434 页(见本册第 405—406 页)和他写的《反杜林论》第二编第十章中又回过头来谈这个问题。他在这一章中对魁奈关于农业中流动资本的补偿的观点作了如下详细说明:“价值五十亿的全部总产品因而掌握在生产阶级的手中,也就是说,首先是掌握在租地农场主的手中,这些租地农场主每年花费二十亿经营资本(与一百亿基本投资相适应)来生产全部总产品。为了补偿经营资本,因而也为了维持一切直接从事农业的人所需要的农产品、生活资料、原料等等,是以实物形式从总收成中拿出来的,并且花费在新的农业生产上。因为,正如前面所说,是以一次规定了的标准的固定价格和简单再生产为前提,所以总产品中预先拿出去的部分的货币价值,等于二十亿利弗尔。因此,这一部分没有进入一般的流通,因为正如已经指出的,任何发生于每一个别阶级的范围之内而不是发生于各阶级相互之间的流通,都没有列入表内。”(《马克思恩格斯全集》中文版第 20 卷第 270—271 页)因此,按照魁奈的说法,应当说租地农场主以实物形式补偿他们的流动资本的那部分产品,占他们的全部总产品的 2/5。——第 352、406 页。}

在这里,这笔账显然有缺陷。看来,魁奈是这样计算的:F 给 P10 亿食物(a—b 线)。F 用 10 亿原料补偿 S 的资本(a″—b″)。10 亿食物构成 S 的工资,这笔工资的价值就是 S 加在商品上并在加的过程中耗费在食物上的价值(c—d)。10 亿留在再生产过程中(a′),不进入流通。最后,10 亿产品补偿“预付”(a′—b′)。可是,魁奈没有看到:S 既不是用这价值 10 亿的工业品向租地农场主购买食物,也不是用它向租地农场主购买原料;S 在同租地农场主的货币结算中,倒是用租地农场主自己的货币偿还租地农场主。要知道,魁奈一开始就是从下面这个假设出发的,即租地农场主除了他的总产品以外还有 20 亿货币,这 20 亿,总的说来,是一个基金,流通的货币是从这里汲取的。

此外,魁奈忘记了,除了这 50 亿总产品以外,还有 20 亿总产品,即在新收获前就已制造出来了的工业品。因为 50 亿只代表租地农场主的全部年产品,[432]租地农场主得到的全部收成,而决不代表要由这个收成补偿自己再生产要素的工业总产品。

因此,现有:(1)租地农场主方面——20 亿货币;(2)50 亿土地总产品;(3)价值 20 亿的工业品。就是说,有 20 亿货币和 70 亿产品(农产品和工业品)。流通过程可以概括如下(F——租地农场主,P——土地所有者,S——工业家,不生产阶级):

F 付给 P20 亿货币地租,P 向 F 购买 10 亿食物。这样就实现了租地农场主的总产品的 1/5。同时有 10 亿货币流回他的手里。其次,P 向 S 购买 10 亿商品。这样就实现了 S 的总产品的 1/2。S 卖得 10 亿货币。他用这笔货币向 F 购买价值 10 亿的食物。从而 S 补偿了他的资本的再生产要素的 1/2。这样就实现了租地农场主的总产品的又一个 1/5。同时,租地农场主又有了 20 亿货币,这是他卖给 P 和 S 的 20 亿食物的价格。然后,F 向 S 购买 10 亿商品,以补偿自己的“预付”的一半。这样就实现了工业家的总产品的另一半。最后,S 用最后这 10 亿货币向租地农场主购买原料。这样就实现了租地农场主的总产品的第三个 1/5,S 的资本的再生产要素的另一半得到补偿,而 10 亿又流回租地农场主手里。租地农场主又有了 20 亿,这是合乎情理的,因为魁奈把租地农场主看作资本家,在同租地农场主的关系上,P 只是收入所得者,S 只是工资所得者。如果租地农场主直接用他的产品支付 P 和 S,他就不付出任何货币。但是,因为他支付了货币,P 和 S 就用这些货币来买他的产品,货币就流回他手里。这是这样一种形式上的货币回流,即货币流回到以买者资格开始全部业务并将它完成的工业资本家手里。其次,租地农场主的总产品中补偿他的“预付”的那 1/5 属于再生产。还剩下 1/5 食物要实现,这是完全不进入流通的。

\tsectionnonum{[(5)《经济表》上的商品流通和货币流通。货币流回出发点的各种情况]}

S 向租地农场主购买 10 亿食物和 10 亿原料,相反,F 只向 S 购买 10 亿商品以补偿他的“预付”。因此,S 要支付 10 亿的差额,而这个差额最终要用 S 从 P 那里得到的 10 亿来支付。看来,魁奈把向 F\textbf{支付}这 10 亿,同向 F\textbf{购买}10 亿产品混淆起来了。至于魁奈怎样考虑这一点,应该参阅勃多的解说\endnote{马克思指勃多的注释:《经济表说明》(载于《重农学派》,附欧·德尔的绪论和评注,1846 年巴黎版第 2 部第 822—867 页)。——第 354 页。}。

实际上(按我们的计算),20 亿只起下列作用:(1)以货币支付 20 亿的地租;(2)使租地农场主的 30 亿总产品流通(其中 10 亿食物给 P,20 亿食物和原料给 S),并使 S 的 20 亿总产品流通(其中 10 亿给 P 用于消费,10 亿给 F 用于再生产的消费)。

最后一次购买(a″—b″)是 S 向 F 购买原料,S 以货币支付 F。

[433]再说一次吧:

S 从 P 那里取得了 10 亿货币。他用这 10 亿货币向 F 购买价值 10 亿的食物。F 用这 10 亿货币向 S 购买商品。S 又用这 10 亿货币向 F 购买原料。

或者:S 向 F 购买 10 亿货币的原料和 10 亿货币的食物。F 向 S 购买 10 亿货币的商品。在这种情况下,10 亿流回 S 手里,但是,所以如此,只是因为假定 S 除了从土地所有者那里取得的 10 亿货币和他需要出卖的 10 亿商品之外,还有他自己投入流通的 10 亿货币。按照这个假定,为了使商品在 S 和租地农场主之间流通,就需要 20 亿货币,而不是 10 亿货币。结果有 10 亿回到 S 手里。这是因为 S 用 20 亿货币向租地农场主购买。而租地农场主向 S 购买 10 亿,把从 S 那里得到的货币的半数付还给 S。

在第一种情况下,S 分两次购买。第一次他付出 10 亿;这 10 亿从 F 流回他的手里;然后他再次把这 10 亿最后付给 F,这样就不再流回了。

相反,在第二种情况下,S 一次就购买 20 亿,当 F 再向 S 购买 10 亿的时候,这 10 亿就留在 S 手里。在这种情况下,流通就需要 20 亿,而不是 10 亿。在第一种情况下,10 亿货币经过两次流通,实现了 20 亿商品。在第二种情况下,20 亿货币经过一次流通,也实现了 20 亿商品。当租地农场主现在支付 10 亿给工业家 S 时,S 由此得到的货币并不比第一种情况下多。因为 S 除了把 10 亿商品投入流通之外,还从他自己的在流通过程开始前就存在的基金中,拿出 10 亿货币投入流通。他为流通投放了货币,因而货币流回他手里。

在第一种情况下:S 用 10 亿货币向 F 购买 10 亿商品。F 用 10 亿货币向 S 购买 10 亿商品。S 又用 10 亿货币向 F 购买 10 亿商品,于是 10 亿货币留在 F 手里。

在第二种情况下:S 用 20 亿货币向 F 购买 20 亿商品。F 用 10 亿货币向 S 购买 10 亿商品。同第一种情况一样,租地农场主手里留下 10 亿货币。但 S 收回 10 亿,这 10 亿是以前从他这方面预付到流通中的资本,现在从流通中回到他手里。S 向 F 购买 20 亿商品。F 向 S 购买 10 亿商品。因此,S 在任何情况下都要支付 10 亿差额,但是不会更多。既然 S 由于这种流通方式的特点,为上述差额付给了 F20 亿,那末 F 就付还给 S10 亿,而在第一种情况下,F 则不付还给 S 任何货币。

就是说,在第一种情况下,S 向 F 购买 20 亿,F 向 S 购买 10 亿。因此,差额仍然是应付给 F10 亿。但是,这个差额是这样支付给 F 的,就是 F 自己的货币又流回 F 手里,因为 S 先向 F 购买 10 亿,然后 F 向 S 购买 10 亿,最后 S 向 F 购买 10 亿。10 亿在这里使 30 亿流通。但是总的说来,流通中存在过的价值(如果货币是实在货币)等于 40 亿:30 亿是商品,10 亿是货币。流通的和最初(支付给租地农场主)投入流通的货币额,从来不超过 10 亿,就是说,不超过 S 应付给 F 的差额。由于在 S 第二次向 F 购买 10 亿以前,F 已向 S 购买 10 亿,S 就可以用这 10 亿支付他应付的差额。

在第二种情况下,S 把 20 亿投入流通。诚然,S 用这 20 亿向 F 购买了价值 20 亿的商品。这 20 亿在这里用作流通手段,它被支出是要换得商品形式的等价物。但是,F 又向 S 购买 10 亿。这样就有 10 亿回到 S 手里,因为 S 应付给 F 的差额只是 10 亿,而不是 20 亿。S 现在已经用商品给 F 补偿了 10 亿,因此 F 必须付还给 S10 亿,这 10 亿\textbf{现在}看来是 S 以货币形式多付给 F 的。这个情况很值得注意,要稍微费点时间谈一谈。

前面所假定的 30 亿商品(其中 20 亿是食物[和原料],10 亿是工业品)的流通,可以有几种不同情况;但是,这里要注意:\textbf{第一},按照魁奈的假定,当 S 和 F 之间的流通过程开始的时候,有 10 亿货币在 S 手里,10 亿货币在 F 手里;\textbf{第二},为了举例说明,我们假定 S 除了从 P 那里得到 10 亿以外,在钱柜里还有 10 亿货币。

[434](I)\textbf{第一},情况就象魁奈假设的那样。S 用 10 亿货币向 F 购买 10 亿商品;F 用从 S 那里得来的 10 亿货币,向 S 购买 10 亿商品;最后,S 用这样收回的 10 亿货币,向 F 购买 10 亿商品。因此,有 10 亿货币留在 F 手里,这笔货币对 F 说来,代表资本(实际上,这 10 亿货币同 F 从 P 那里收回的另外 10 亿货币一起,形成他下年度用来重新支付货币地租的收入,即 20 亿货币)。在这里,10 亿货币流通三次(从 S 到 F,从 F 到 S,再从 S 到 F),每次偿付 10 亿商品,因而总共偿付 30 亿商品。如果货币本身具有价值,在流通中就有 40 亿价值。货币在这里只执行流通手段的职能,但是对于 F(货币最后留在他的手里)说来,却转化为货币,而且可能转化为资本。

(II)\textbf{第二},货币只执行支付手段的职能。在这种情况下,S 向 F 购买 20 亿商品,F 向 S 购买 10 亿商品,他们彼此进行结算。在交易结束时,S 要用货币支付 10 亿差额。同前面情况一样,10 亿货币落入 F 的钱柜,但它一直没有起过流通手段的作用。这笔货币对 F 说来是资本的转移,因为它只给 F 补偿一笔 10 亿商品的资本。这样一来,同第一种情况一样,有 40 亿价值加入流通。但是,10 亿货币只发生一次运动,而不是三次运动,货币只支付同额的商品价值。而在第一种情况下,货币则支付了三倍于它本身的价值。同第一种情况相比,省去了两次多余的流通行为。

(III)\textbf{第三},F 首先作为买者出现,用 10 亿货币(他从 P 那里得来)向 S 购买 10 亿商品。这 10 亿货币不是当作贮藏货币闲放在 F 身边到下年度支付地租,而是现在就进入流通。于是,S 有了 20 亿货币(10 亿货币从 P 那里得来,10 亿货币从 F 那里得来)。他用这 20 亿货币向 F 购买了价值 20 亿的商品。现在流通中有 50 亿价值(30 亿商品和 20 亿货币)。发生了 10 亿货币和 10 亿商品的流通以及 20 亿货币和 20 亿商品的流通。在这 20 亿货币中,来自 F 的 10 亿流通两次,来自 S 的 10 亿只流通一次。现在 20 亿货币回到 F 手里,但是其中只有 10 亿货币是向他结算差额的,另外 10 亿货币,即他从前因首先作为买者出现而投入流通的那 10 亿,则通过流通过程流回他的手里。

(IV)\textbf{第四},S 用 20 亿货币(10 亿货币从 P 那里得来,10 亿从他自己的钱柜中取出投入流通)一次就向 F 购买价值 20 亿的商品。F 又向 S 购买 10 亿商品,因而把 10 亿货币还给 S,同前面情况一样,F 还有 10 亿货币留在手上,用来结算他同 S 之间的差额。这里投入流通的有 50 亿价值。流通行为是两次。

在 III 的情况下,在 S 还给 F 的 20 亿货币中,10 亿代表 F 自己投入流通的货币,只有 10 亿代表 S 投入流通的货币。这里回到 F 手里的是 20 亿货币,而不是 10 亿货币,但实际上他得到的只是 10 亿,因为另外 10 亿是他自己投入流通的。在 IV 的情况下,有 10 亿货币回到 S 手里,但是这 10 亿货币,是他自己从钱柜中取出投入流通的,而不是向租地农场主出卖自己的商品得来的。

如果说,在 I 的情况下和在 II 的情况下,流通中的货币都是从来不超过 10 亿,可是在 I 的情况下货币流通三次,转手三次,而在 II 的情况下只流通一次,转手一次,那末,这不过是由于在 II 的情况下假定有发达的信用,因此节约了支付的次数,而在 I 的情况下则发生急速的运动,货币每次都作为流通手段出现,因此价值每一次都要以二重形式在两极出现,一极以货币形式,一极以商品形式。如果说,在 III、IV 的情况下有 20 亿货币流通,不象在 I、II 的情况下是 10 亿货币流通,那末,这是因为在 III、IV 的情况下,都有 20 亿的商品价值一次就进入流通过程(在 III 的情况下是 S 作为买者结束流通过程,在 H 的情况下是 S 作为买者开始流通过程);总之,20 亿商品一次就进入流通,并且假定它们立即被购买,而不是结算后才支付。

但不管怎样,在这个运动中最有意思的是,在 III 的情况下 10 亿货币留在租地农场主手里,而在 IV 的情况下 10 亿货币则留在工业家手里,虽然在两种情况下 10 亿货币的差额都是付给租地农场主,而租地农场主在 III 的情况下没有多得分文,在 H 的情况下也没有少得分文。自然,这里总是等价物交换,如果我们谈到差额,它所指的不过是用货币而不是用商品支付的价值等价物。

在 III 的情况下,F 把 10 亿货币投入流通,从 S 那里换得商品等价物,即得到 10 亿商品。但是,后来 S 用 20 亿货币向 F 购买商品。这样,F 投入流通的第一个 10 亿货币回到 F 手里,然而有 10 亿商品离开 F。这 10 亿商品是用 F 以前支出的货币付给 F 的。从对第二个 10 亿商品的支付中 F 得到第二个 10 亿货币。这一货币差额由 F 收进,因为 F 总共只买进 10 亿货币的商品,而从 F 那里买去的却是价值 20 亿的商品。

[435]在 IV 的情况下,S 把 20 亿货币一次投入流通,从 F 那里换得 20 亿商品。F 又用 S 自己支出的这笔货币向 S 购买 10 亿商品,于是 10 亿货币回到 S 手里。

在 IV 的情况下:S 实际上以商品形式给 F10 亿商品(等于 10 亿货币),以货币形式给 F20 亿货币,因此,总共是 30 亿货币。S 从 F 那里得到的只是 20 亿商品。因此,F 应还给 S10 亿货币。

在 III 的情况下:F 以商品形式给 S20 亿商品(等于 20 亿货币),以货币形式给 S10 亿货币,因此,总共是 30 亿货币,而 F 从 S 那里得到的,只是 10 亿商品,等于 10 亿货币。因此,S 应还给 F20 亿货币,其中 10 亿,S 用 F 自己投入流通的货币来付还,另外 10 亿是 S 自己投入流通。F 留下 10 亿货币的差额,但不可能留下 20 亿货币。

在这两种情况下,S 都是得到 20 亿商品,F 都是得到 10 亿商品加 10 亿货币即货币差额。如果说在 G 的情况下,另外还有 10 亿货币流到 F 手里,那末,这只不过是他投入流通的超过他因出卖商品而从流通中抽出的数额的货币。在 H 的情况下,S 的情况也是一样。

在这两种情况下,S 都要用货币支付 10 亿货币的差额,因为他从流通中抽出 20 亿商品,而投入流通的只有 10 亿商品。在这两种情况下,F 都要以货币形式收进 10 亿货币的差额,因为他把 20 亿商品投入流通,而从流通中只抽出 10 亿商品,因此,对第二个 10 亿商品,必须用货币向他结算。在这两种情况下,最后能够转手的只有这 10 亿货币。但是,因为流通中有 20 亿货币,所以这 10 亿货币就不得不流回原来把它投入流通的人手里,不管他是 F(他从流通中收进 10 亿货币的差额,此外还把另外 10 亿货币投进了流通)还是 S(他应支付的只有 10 亿货币的差额,此外还把 10 亿货币投进了流通)。

在 III 的情况下,进入流通的货币,比其他情况下使这个商品量流通所必需的货币量多 10 亿,因为 F 首先作为买者出现,不管最后结算情况如何,他都必须把货币投入流通。在 IV 的情况下,同样有 20 亿货币进入流通,不象在 II 的情况下只有 10 亿,因为在 IV 的情况下,第一,一开始 S 就作为买者出现;第二,他一次就买进 20 亿商品。在 III、IV 的情况下,买者和卖者之间\textbf{流通}的货币,最后只能等于其中一方应付的差额。因为 S 或 F 付出的超过这个数额的货币,都要付还给 S 或 F 本人。

假定 F 向 S 购买 20 亿商品。那末,情况就会变成这样,F 给 S10 亿货币以交换商品。S 向 F 购买 20 亿货币的商品,因此,第一个 10 亿货币就回到 F 手里,还外加 10 亿货币。F 再用 10 亿货币向 S 购买商品,于是这笔货币又回到 S 手里。在过程结束时,F 有 20 亿货币的商品和流通过程开始之前他原来就有的 10 亿货币;而 S 有 20 亿商品和他同样是原来就有的 10 亿货币。F 的 10 亿货币和 S 的 10 亿货币只起了流通手段的作用,后来则作为货币或者在这种情况下也作为资本,流回到把它们投入流通的双方手里。如果双方都把货币用作支付手段的话,他们进行结算,就是 20 亿商品对 20 亿商品;他们彼此销账;双方之间连一分钱也不流通。

因此,在互为买者和卖者两次对立的双方之间作为流通手段流通的货币,都是流回的;这些货币的流通可能有三种情况。

[\textbf{第一},]投入流通的商品价值彼此相等。在这种情况下,货币流回那个把它预付到流通过程中去,并且这样以自己资本开支流通费用的人手里。例如,如果 F 和 S 各向对方购买 20 亿商品,S 先开头,那末,S 就一次向 F 购买 20 亿货币的商品。然后,F 向 S 购买 20 亿商品,把这 20 亿货币还给 S。这样,S 在交易前和交易后,都有 20 亿商品和 20 亿货币。或者,如果象前面提到的情况那样,双方都预付等量流通手段,那末,双方预付到流通过程中的流通手段都回到各自的手里;就象前面 10 亿货币回到 F 手里,10 亿货币回到 S 手里那样。

\textbf{第二},双方交换的商品价值不等。出现一个要用货币支付的差额。如果商品流通象前面 I 的情况那样,进入\textbf{流通}的货币量没有超过为支付这个差额所必需的货币额,——因为始终只有这个货币额在双方之间来回,——那末,这笔货币最终落入收进这个差额的最后卖者手里。

\textbf{第三},双方交换的商品价值不等;有一个差额要支付;但是,商品流通进行时,流通的货币多于为支付这个差额所必需的货币;在这种情况下,超出这个差额的货币流回预付货币的一方。在 III 的情况下,流回收进差额的人手里,在 IV 的情况下,流回应付差额的人手里。

在上面“第二”所说的情况下,只有当收进差额的人是第一个买者的时候,象在工人和资本家的例子中那样,货币才\textbf{流回}收进差额的人手里。如果象在 II 的情况下那样,另一方首先作为买者出现,那末,货币就离开他而到他的对方手里去了。

[436]\fontbox{~\{}自然,这一切只限于这样的假设:一定的商品量在同一些人之间买卖,其中每一个人交替地在对方面前作为买者和卖者出现。相反,我们假定,有 3000 商品平分给商品所有者-卖者 A、A′、A″,同他们相对立的是买者 B、B′、B″。如果这里三次购买行为同时发生,因而在空间上并行地发生,那末,就要有 3000 货币流通,才能使每一个 A 都有 1000 货币,而每一个 B 都有 1000 商品。如果几次购买行为一个接着一个,在时间上连续发生,那末,只有商品形态变化串连在一起,也就是说,只有一些人既作为买者,又作为卖者出现,虽然不是象上面说的那样,对同一些人既作为买者又作为卖者出现,而是对一些人作为买者出现,对另一些人又作为卖者出现,只有这样,同一个 1000 货币的流通才能完成这几次购买行为。例如:(1)A 卖给 B1000 货币的商品;(2)A 用这 1000 货币向 B′购买;(3)B′用这 1000 货币向 A′购买;(4)A′用这 1000 货币向 B″购买;(5)B″用这 1000 货币向 A″购买。货币在 6 个人之间转手 5 次,但是也使 5000 货币的商品得以流通。如果只使价值 3000 货币的商品流通,那就是:(1)A 用 1000 货币向 B 购买;(2)B 用 1000 货币向 A′购买;(3)A′用 1000 货币向 B′购买。在 4 个人之间转手 3 次。这是 G—W 过程。\fontbox{\}~}

上面阐明的各种情况,并不违背以前所阐明的规律,即:

\begin{quote}“已知货币流通速度,已知商品价格总额,流通手段量就已决定。”(第 1 分册第 85 页)\endnote{指《政治经济学批判》第一分册。见《马克思恩格斯全集》中文版第 13 卷第 96 页。——第 363 页。}\end{quote}

在前面 I 的情况的例子中,1000 货币\endnote{在魁奈的《经济表》中是 1000 个百万(即 10 亿)图尔利弗尔,马克思在这里用 1000 货币单位来代替,这丝毫不改变问题的本质。——第 363 页。}流通了 3 次,使 3000 货币的商品进入运动。因此,流通的货币量

\centerbox{=[3000(价格总额)/3(流通速度)]或[3000(价格总额)/3 次流通]=1000 货币}

在 III 或 IV 的情况下,流通的商品的价格总额固然是同一数额(3000 货币),但流通速度不同。2000 货币流通 1 次,即 1000 货币加 1000 货币。但是,这 2000 货币中有 1000 再流通了 1 次。2000 货币使价值 3000 的商品的 2/3 流通,而 2000 货币的半数使价值 1000 的商品即余下的 1/3 流通;一个 1000 货币流通 2 次,但另一个 1000 货币仅仅流通 1 次。1000 货币的 2 次流通实现了等于 2000 货币的商品价格,1000 货币的 1 次流通实现了等于 1000 货币的商品价格,加在一起等于 3000 商品。那末,同这些商品——货币使之流通的商品——有关的货币流通速度怎样呢\fontbox{?}这 2000 货币流通 1+(1/2)次(这就是说\textbf{首先}全额流通 1 次,然后其中半数又流通 1 次),等于 3/2 次。实际是:

\centerbox{[3000(价格总额)/3/2 次流通]=2000 货币}

但是,这里货币流通的\textbf{不同速度}是由什么决定的呢\fontbox{?}

不论 III 或 IV 的情况,它们与 I 的情况的不同是这样引起的。在 I 的情况下,每次流通的商品的价格总额,始终不大于也不小于一般进入流通的商品总量的价格总额的 1/3。始终只有价值 1000 货币的商品在流通。在 III 和 IV 的情况下则相反,一次是 2000 货币的商品流通,一次是 1000 货币的商品流通;因此,一次是现有商品总量的 2/3 流通,一次是它的 1/3 流通。由于同一个理由,批发商业中流通的铸币,必定大于零售商业中流通的铸币。

我已经指出过(第 1 分册,《货币的流通》),货币的回流首先表明,\textbf{买者又变成了卖者},\endnote{见《马克思恩格斯全集》中文版第 13 卷第 89—90 页。——第 364 页。}至于他是否卖给他曾经向之买过东西的人,实际上是无关紧要的。可是,如果事情发生在同一些人之间,那就会出现一些现象,这些现象曾经引起许多混乱(德斯杜特·德·特拉西)\endnote{见本册第 277—291 页。并见马克思《资本论》第 2 卷第 20 章第 13 节:《德斯杜特·德·特拉西的再生产理论》。——第 343、364 页。}。买者变成卖者表明,要出卖的是新的商品。商品流通的继续,就是商品流通的不断更新(第 1 分册第 78 页)\endnote{马克思引的是《政治经济学批判》第一分册(见《马克思恩格斯全集》中文版第 13 卷第 88—89 页)。参看注 100。——第 364 页。},——因而,这里有一个再生产过程。买者可以再变成卖者(如厂主对工人)而不表示任何再生产行为。只有这种货币回流的继续、重复,才表示再生产过程。

当货币的回流代表资本再转化为资本的货币形式的时候,如果资本继续作为资本运动,这种货币回流必然表示一个周转的结束和再生产过程的重新开始。在这里,同在其他一切场合一样,资本家是卖者,W—G,然后变成买者,G—W,但是,他的资本只有变为 G,才重新具有能够同它的再生产要素相交换的形式,而 W 在这里代表这些再生产要素。G—W 在这里代表货币资本转化为生产资本,或产业资本。

其次,我们说过,货币流回它的出发点的这种回流可能表示,在一系列的买卖之后,货币差额由首先开始这一系列过程的买者收进。F 向 S 购买 1000 货币的商品。S 向 F 购买 2000 货币的商品。这里,有 1000 货币流回 F 手里。至于另一个 1000,那不过是货币在 S 和 F 之间简单变换位置。

[437]最后,货币流回出发点的回流可能不表示支付差额,这在下述两种场合都可能发生:(1)双方支付平衡,因此没有任何差额要用货币支付,(2)双方支付\textbf{不}平衡,因此需要支付一个货币差额。请参看上面分析的各种情况。在所有这些情况下,譬如说,同 F 相对的 S 是不是同一个人,那是无关紧要的;这里,S 对于 F 和 F 对于 S,都是代表向他出卖和向他购买的人的总数(完全象在货币回流表现支付差额的例子中一样)。在所有这些情况下,货币都流回到把货币可以说预付到流通过程中去的人手里。货币同银行券一样,在流通中完事之后,就回到把它投入流通的人手里。\textbf{在这里货币只是流通手段。最后出现的资本家彼此支付,这样,货币就回到首先把它投入流通的人手里}。

还有一个问题,即资本家从流通中抽出的货币多于他投入流通的货币的问题,留待以后解决。\endnote{马克思在《资本论》第二卷第十七章、第二十章第五节和第十二节、第二十一章(特别是第一节中的第一小节《贮藏货币的形成》)中全面考察了这个问题。——第 347、349、365 页。}

\tsectionnonum{[(6)《经济表》在政治经济学史上的意义]}

现在回过头来讲魁奈。

亚·斯密带着几分讽刺意味引用了米拉波侯爵的夸张说法:

\begin{quote}“自从世界形成以来,有三大发明……第一是\textbf{文字}的发明……第二是\textbf{货币的发明}〈!〉……第三是《\textbf{经济表}》,这个表是前两者的结果和完成。”(\textbf{加尔涅}的译本,第 3 卷第 4 篇第 9 章第 540 页)\end{quote}

但是,实际上,这是一种尝试:把资本的整个生产过程表现为\textbf{再生产过程},把流通表现为仅仅是这个再生产过程的形式;把货币流通表现为仅仅是资本流通的一个要素;同时,把收入的起源、资本和收入之间的交换、再生产消费对最终消费的关系都包括到这个再生产过程中,把生产者和消费者之间(实际上是资本和收入之间)的流通包括到资本流通中;最后,把生产劳动的两大部门——原料生产和工业——之间的流通表现为这个再生产过程的要素,而且把这一切总结在一张《\textbf{表}》上,这张表实际上只有五条线,连结着六个出发点或归宿点。这个尝试是在十八世纪三十至六十年代政治经济学幼年时期做出的,这是一个极有天才的思想,毫无疑问是政治经济学至今所提出的一切思想中最有天才的思想。

至于资本流通、资本的再生产过程、资本在这个再生产过程中采取的各种不同的形式、资本流通同一般流通的联系,也就是说,不仅资本同资本的交换,而且资本同收入的交换,那末,斯密实际上只是接受了重农学派的遗产,对财产目录的各个项目作了更严格的分类和更详细的描述,但是对于过程的整体未必叙述和说明得象《经济表》大体上描绘的那样正确,尽管魁奈的前提是错误的。

此外,斯密评论重农学派说:

\begin{quote}“他们的著作肯定对他们的国家有些贡献。”(同上,第 538 页)\end{quote}

这样的评价,对于一个比如象杜尔哥(法国革命的直接先导之一)这样的人所起的作用来说,未免失之过稳罢。[437]

\tchapternonum{[第七章]兰盖}

\vicetitle{[对关于工人“自由”的资产阶级自由主义观点的最初批判]}

[438]\textbf{兰盖}《民法论》1767 年伦敦版。

按照我的写作计划,社会主义的和共产主义的著作家都不包括在历史的评论之内。这种历史的评论不过是要指出,一方面,政治经济学家们以怎样的形式自行批判,另一方面,政治经济学规律最先以怎样的历史路标的形式被揭示出来并得到进一步发展。因此,在考察剩余价值时,我把布里索、葛德文等等这样的十八世纪著作家,以及十九世纪的社会主义者和共产主义者,都放在一边了。至于我在这个评论中以后要说到的少数几个社会主义著作家\endnote{在《剩余价值理论》第三册(手稿第 XIV 本和 XV 本,第 852—890 页)有一章:《以李嘉图理论为依据反对政治经济学家的无产阶级反对派》。第 X 本(第 441—444 页)中未完成的论布雷一节和第 XVIII 本(第 1084—1086 页)中论霍吉斯金一节的结尾部分也属于这一章。——第 367 页。},他们不是本身站在资产阶级政治经济学的立场上,便是从资产阶级政治经济学的观点出发去同资产阶级政治经济学作斗争。

然而兰盖并不是社会主义者。他反对他同时代的启蒙运动者的资产阶级自由主义理想,反对资产阶级刚刚开始的统治,他的抨击半是认真半是嘲弄地采取反动的外观。他维护亚洲的专制主义,反对文明的欧洲形式的专制主义,他捍卫奴隶制,反对雇佣劳动。

第一卷。他的一句反对孟德斯鸠的话:

\begin{quote}“法律的精神就是所有权”\endnote{[兰盖,尼]《民法论,或社会的基本原理》1767 年伦敦版第 1 卷第 236 页。——第 368 页。}\end{quote}

就表明了他的见解的深刻。

兰盖碰到的和他对立的唯一的一批政治经济学家,是重农学派。

兰盖证明,富人占有一切生产条件;这是\textbf{生产条件的异化},而最简单形式的生产条件是自然要素本身。

\begin{quote}“在我们的各个文明国家里,一切自然要素都成了奴隶。”(第 188 页)\end{quote}

要取得这些被富人占有的财宝的一部分,必须用增加富人财富的繁重劳动来购买这一部分。

\begin{quote}“这样,整个被俘虏的自然,就不再向自己的儿女提供容易得到的维持生命的源泉了。自然的恩赐必须以辛苦的努力为代价,自然的赐予必须以顽强的劳动来取得。”\end{quote}

(这里,在“自然的赐予”这个词上,露出了重农学派的见解。)

\begin{quote}“\textbf{独占这些财宝的}富人,只有取得这种代价,才同意把财宝的极小部分还给大家使用。\textbf{为了得到分享他的财宝的许可,必须努力劳动来增加财宝}。”(第 189 页)“这样,就必须放弃自由的幻想。”(第 190 页)法律的存在是为了“批准〈对私有财产〉最初的夺取”,并“防止以后的夺取”。(第 192 页)“法律可以说是一种反对人类最大多数〈即无产者〉的阴谋。”(同上[第 195 页])“是社会创造了法律,而不是法律创造了社会。”(第 230 页)“所有权先于法律。”(第 236 页)\end{quote}

“社会”本身——人生活在社会中,而不是作为独立自主的个人——是所有权、建立在所有权基础上的法律以及由所有权必然产生的奴隶制的根源。

\begin{quote}一方面是土地耕种者和牧人过着和平的、孤立的生活。另一方面,还有“猎人,他们习惯于靠屠杀取得生活资料,习惯于成群结队,以便于围猎他们吃的野兽和瓜分猎物”。(第 279 页)“正是在猎人当中出现了最初的社会标志。”(第 278 页)“\textbf{真正的社会是牺牲牧人和土地耕种者的利益而形成的,是以}”联合起来的一伙猎人“\textbf{对他们进行奴役为基础的}”。(第 289 页)社会上的一切义务可归结为命令和服从。“人类一部分的地位降低,先是产生了社会,然后再产生法律。”(第 294 页)\end{quote}

贫困迫使丧失生产条件的工人为生活而劳动,去增加别人的财富。

\begin{quote}“只是因为没有别的活路,我们的短工才不得不耕种土地而自己享受不到它的果实,我们的石匠才不得不修建房屋而自己不能居住。贫穷把他们赶到市场上,等待主人开恩购买他们。\textbf{贫穷迫使他们跪在富人面前,央求富人准许他们使他发财}。”(第 274 页)“可见,奴役是产生社会的第一个原因,暴力是社会的第一个纽带。”(第 302 页)“他们〈人们〉关心的第一件事,无疑是获得自己的食物……关心的第二件事,就是想方设法\textbf{不劳动而获得自己的食物}。”(第 307—308 页)“他们只有\textbf{占有别人劳动的果实},才能做到这一点。”(第 308 页)“最初的征服者们,只是为了不受惩罚地过游手好闲的生活,才实行统治;他们成为国王,只是为了拥有生存资料。这就使统治的观念……大大缩小和简化了。”(第 309 页)“社会由暴力产生,所有权由夺取产生。”(第 347 页)“主人和奴隶一出现,社会就形成了。”(第 343 页)“市民社会一开始就有两个[439]柱石,一方面是大部分男子的奴隶地位,另一方面是全部女子的奴隶地位……社会靠四分之三的人口来保证少数有产者的幸福、财产、闲暇,社会关心的只是这少数人。”(第 365 页)\end{quote}

第二卷。

\begin{quote}“因此,问题不是要弄清奴隶制是否同自然本身有矛盾,而是要弄清奴隶制是否同社会的本性有矛盾……奴隶制是同社会的存在分不开的。”(第 256 页)“社会和市民的受奴役同时产生。”(第 257 页)“终身奴隶制是社会的不可毁灭的基础。”(第 347 页)“被迫靠某个人的施舍才获得生存资料的人们,只是在\textbf{这个人由于从他们手里夺得财物而大大富裕起来},以致有可能把其中的一小部分\textbf{归还}给他们的时候才出现的。此人虚伪的慷慨,不过是\textbf{把他占有的别人劳动果实的一部分归还给别人而已}。”(第 242 页)“人们被迫耕种而自己得不到收获物,为了别人的幸福而牺牲自己的幸福,被迫进行无希望的劳动,这不\textbf{就是奴隶制}吗\fontbox{?}人们被迫在鞭打下劳动,而回到畜栏只得到一点燕麦,奴隶制的真正历史不就是从这个时候开始的吗\fontbox{?}只有在发达的社会中,生存资料对\textbf{饥饿的}贫民来说才是他们的自由的充分\textbf{等价物};在发展初期的社会里,这样不平等的交换,在自由人看来是骇人听闻的事情。只有对\textbf{战俘}才能这样做。只有剥夺了他们享有任何财产的权利之后,才能使这样的交换对他们说来是必然的。”(第 244—245 页)“\textbf{社会的本质……就是使富人免除劳动},使富人获得新的器官、获得不会疲倦的肢体来担负一切繁重劳动,\textbf{而劳动果实则由富人据为己有}。这就是奴隶制使富人轻而易举地达到的目的。他购买了那些必须为他服务的人们。”(第 461 页)“奴隶制废除了,但人们决不会废弃财富和财富的好处……因此,除了名称改变以外,一切都必须照旧。最大多数人总是必须靠工资生活,依赖于\textbf{把全部财物据为己有}的极少数人。这样,奴隶制就在世上永存下来,不过名称更加动听。它现在以仆人的美名出现于我们中间。”(第 462 页)\end{quote}

兰盖说,这里所说的“仆人”不是指仆役等等。

\begin{quote}“城市和乡村住满了另一种仆人,他们人数更多、更有用、更勤劳,他们被称为《journaliers》(短工),‘\textbf{手工工人}’等等。他们没有用奢侈的虚饰来玷污自己;他们穿着令人厌恶的破烂衣衫,穿着这种贫穷的\textbf{制服}在呻吟。\textbf{在他们的劳动所创造的丰裕财富中,他们从来分不到一丝一毫}。当财富竟肯接受\textbf{他们赠送的礼物}时,财富就象是对他们开恩一样。他们还必须\textbf{为他们能够向财富服务}表示感激。当他们抱着财富的双膝,请求\textbf{允许他们对财富做点有用的事时},财富用最侮辱人的轻蔑态度对待他们。财富迫使人们去央求得到这种允许,并且\textbf{在真慷慨同假恩惠之间进行的这种独特的交换中,受惠者方面}是傲慢的和轻蔑的,\textbf{给予者方面}则是驯服的、焦虑的和勤恳的。事实上正是这样的仆人在我们这里取代了奴隶。”(第 463—464 页)“必须弄清\textbf{奴隶制的废除}实际上给他们带来了什么利益。我要沉痛而直率地说:这全部利益就是他们永远经受着饿死的恐怖,这种不幸,至少他们的处在人类社会这一底层的先辈是没有遭受到的。”(第 464 页)“你们说,他[工人]是自由的。唉!他的不幸也正是在这里。他同任何人无关,任何人也同他无关。当需要他的时候,人们就\textbf{用最低的价钱雇用}他。人们答应给他的微不足道的\textbf{工资},只够\textbf{他交换出去的一个工作日所必需的生存资料的价格}。人们叫\textbf{监工督促他尽快劳动};人们催他,赶他,唯恐他想出一种偷懒的法子来少花一半力气,人们生怕他想要\textbf{拉长劳动时间}就会使他的手变得不灵巧,会把他的工具弄钝。\textbf{贪婪的吝啬鬼不放心地监视着他,只要他稍一中断工作,就大加叱责}。只要他休息一下,\textbf{就硬说是偷窃了他}。工作一完,他就被解雇,人们解雇他时,象雇用他时一样冷淡,丝毫也不考虑,[440]\textbf{如果他明天找不到工作},他劳苦了一天所得到的 20 或 30 苏够不够维持生活。”(第 466—467 页)“他是自由了!正因为如此,我才怜惜他。正因为如此,人们雇用他来干活时才极端不爱惜他。正因为如此,人们才更加肆无忌惮地浪费他的生命。奴隶对于自己的主人来说是一种有价值的东西,因为主人为他花了钱。而工人并没有使雇用他的富裕的享乐者花费什么。在奴隶制时期,人的血是有一定价格的。人的价值至少等于他们在市场上被卖的那个数额。自从停止贩卖人口以来,人实际上也就没有任何内在价值了。在军队中,对工兵的估价,比对辎重马的估价低得多,因为马的价钱很贵,而工兵不用花钱就能弄来。奴隶制的废除,使这样的估价方法从军队生活被搬用到市民生活中来;\textbf{从此以后,没有一个富裕的市民不是象威武的勇士那样来评价人了}。”(第 467 页)“短工为了替财富服务而出生、成长、受教育,这就象财富在自己的领地内打死的野兽一样,不用花财富分文。好象财富真的掌握了倒霉的庞培瞎吹嘘的那个秘密似的。只要财富往地上一跺脚,就会从地里钻出一大群勤劳的人,争先恐后地要为他服务。在这一大群为他造房子或管花园的雇佣者当中,如果少了一个人,空缺是看不出来的,它会马上被填补起来,不用任何人过问。大河里失掉一滴水是没有什么可惜的,因为新水流不断流进来。工人的情形也是这样;要找代替的人很容易,所以\textbf{富人}对待他们是冷酷无情的。”\end{quote}

(这是兰盖的说法,他提的还不是资本家)(第 468 页)

\begin{quote}“人们说,他们没有主人……但这是明显的滥用词句。他们没有主人,是什么意思呢\fontbox{?}他们有一个主人,而且是一切主人中最可怕、最专制的主人,这就是\textbf{贫困}。贫困使他们陷入最残酷的奴隶地位。\textbf{他们不是听命于某一个个别的人,而是听命于所有一切人}。他们非去讨好和巴结不可的,不只是某一个统治他们的暴君,否则他们的奴隶地位就有一定的界限了,也比较好忍受了。\textbf{他们成了每一个有钱人的仆人},因此,他们的奴隶地位就是没有界限的,极端严酷的了。有人说:如果他们在一个主人那里过得不好,那末他们至少有一点是可以告慰的,那就是可以向主人申述,并且另找一个主人;而奴隶就不能这样。可见奴隶是更不幸的。什么样的诡辩啊!请想一想,\textbf{迫使别人劳动的人}是很少很少的,而劳动者很多。”(第 470—471 页)“你们授予他们的那个虚幻的自由,对他们会有什么结果呢\fontbox{?}\textbf{他们只能靠出租自己的双手来生活。可见,他们必须找到一个雇用他们的人,要不就饿死。难道这就是自由了吗}\fontbox{?}”(第 472 页)“最可怕的是,他们工资的菲薄竟成了工资进一步下降的原因。短工愈穷,他就愈便宜地出卖自己。他穷得愈厉害,他的劳动的报酬就愈低。当他含泪哀求他面前象暴君一样的人接受他的服务时,这些人一点也不脸红,好象在摸摸他的脉搏,判断他剩下的力气还有多少。他们根据他衰弱的程度,来确定给他多少价钱。在他们看来,他愈是虚弱得濒于死亡,他们就愈是削减还能救他命的那些东西。这些野蛮人给他的东西,与其说用来延长他的生命,不如说用来推迟他的死期。”(第 482—483 页)“〈短工的〉独立……是我们时代的精巧性所造成的最有害的灾祸之一。这种独立使富人愈富,穷人愈穷。富人所积蓄的,都是穷人为维持生活所花费的;穷人不是从剩余中节省,而是不得不从最需要的东西中节省。”(第 483 页)“今天,竟如此容易维持庞大的军队,这些军队连同奢侈一起导致人类的毁灭。这只能归功于奴隶制的废除……自从不再有奴隶以来,放荡和赤贫才造成一日得 5 苏的勇士。”(第 484—485 页)“我认为,它〈亚洲的奴隶制〉对于被迫用每天的劳动去谋生的人们来说,要比所有别的生存方式强百倍。”(第 496 页)“他们〈奴隶和雇佣工人〉的锁链是用同样的材料制成的,只不过颜色不同。一种人的锁链是黑色的,看起来比较重;另一种人的锁链不那么黑,看起来比较轻;但是如果不偏不倚地把它们衡量一下,就会发现它们之间没有丝毫差别,两者都同样是由贫困制成的。它们的重量完全一样,而且,如果说有一种更重一些,那恰好就是从表面看起来比较轻的那一种。”(第 510 页)\end{quote}

兰盖就工人问题向法国启蒙运动者们大声疾呼:

\begin{quote}“难道你们没有看见,这一大群羊的驯服,直率地说,绝对顺从,创造了牧人的财富吗\fontbox{?}……请相信我,为了他〈牧人〉的利益,为了你们的利益,甚至为了它们〈羊〉自己的利益,还是让它们抱定它们历来的信念:相信一只向它们吠叫的狗要比所有的羊加在一起都强大吧。让它们一看见狗的影子就不知所以地逃跑吧。大家都可以由此得到好处……你们可以更容易地把它们赶去剪毛。它们可以更容易地避免被狼吃掉的危险。[441]诚然,它们避免这种危险,只是为了给人当食物。但是,自从它们一进入畜栏,它们的命运就已注定如此了。在谈论把它们从畜栏引出去以前,你们应当先把畜栏即社会砸毁。”(第 512—513 页)[X—441]\end{quote}

\tchapternonum{附录}

\tsectionnonum{[(1)霍布斯论劳动,论价值,论科学的经济作用]}

[XX—1291a]霍布斯认为技艺之母是\textbf{科学},而不是\textbf{实行者的劳动}:

\begin{quote}“对社会有意义的技艺,如修筑要塞、制造兵器和其他战争工具,是一种力量,因为它们有助于防卫和胜利;虽然它们的真正母亲是\textbf{科学,即数学},但由于它们是在工匠手里产生出来的,它们就被看成是工匠的产物,就象老百姓把助产婆叫做母亲一样。”(《利维坦》,载于《托马斯·霍布斯英文著作选》,摩耳斯沃思出版,1839—1844 年伦敦版第 3 卷第 75 页)\end{quote}

对脑力劳动的产物——科学——的估价,总是比它的价值低得多,因为再生产科学所必要的劳动时间,同最初生产科学所需要的劳动时间是无法相比的,例如学生在一小时内就能学会二项式定理。

\textbf{劳动能力}:

\begin{quote}“\textbf{人的价值},和其他一切物的价值一样,等于他的价格,就是说,等于\textbf{对他的能力的使用}所付的报酬。”(同上,第 76 页)“\textbf{人的劳动}〈因而人的劳动力的使用〉也是\textbf{商品},人们可以有利地交换它,就象交换其他任何\textbf{物品}一样。”(同上,第 233 页)\end{quote}

\textbf{生产劳动和非生产劳动}:

\begin{quote}“人仅仅为了维持自己的生活而\textbf{劳动}是不够的。他还应当在必要时为\textbf{保卫自己的劳动}而\textbf{战斗}。人们或者必须象犹太人被俘归来后重建神殿那样,一手建设,一手拿剑;或者要雇用别人来为他们战斗。”(同上,第 333 页)[XX—1291a]\end{quote}

\tsectionnonum{[(2)]历史方面:配第}

\vicetitle{[对于非生产职业的否定态度。劳动价值论的萌芽。在价值论的基础上说明工资、地租、土地价格和利息的尝试]}

[XXII—1346]\textbf{配第}《赋税论》1667 年伦敦版。

我们的朋友配第\endnote{关于配第的某些观点,马克思已经在前面《关于生产劳动和非生产劳动的理论》一章中,即在该章论及区分生产劳动和非生产劳动的最初尝试的那部分谈到过(见本册第 173—176 页)。——第 378 页。}的“人口论”,同马尔萨斯的完全不同。按照配第的意见,应该制止牧师的“繁殖”能力,让他们恢复独身生活。

这一切属于\textbf{生产劳动和非生产劳动}那一节\endnote{指马克思在手稿第 XVIII 本第 1140 页起草的计划所拟定的《资本论》第一部分最后一节即第九节(见本册第 446 页《资本论》第一部分的计划)。——第 378 页。}。

(a)\textbf{牧师}:

\begin{quote}“由于在英国男人比女人多……所以让牧师\textbf{恢复独身生活},换句话说,不让结了婚的人当牧师,是有好处的……这样一来,我们的\textbf{不结婚的牧师}就能够以他们现有俸禄的一半,来维持他们现在用全部俸禄所过的生活。”(第 7—8 页)\end{quote}

(b)\textbf{批发商和零售商}:

\begin{quote}“这些人很大一部分也可以削减。他们\textbf{本来就不配从社会得到什么},因为他们不过是一种\textbf{互相}[1347]\textbf{以贫民劳动为赌注的赌徒},他们自己什么也不生产,只是象静脉和动脉那样,把社会机体的血液和营养液,即工农业产品,\textbf{分配}到各方。”(第 10 页)\end{quote}

(c)\textbf{律师、医生、官吏等等}:

\begin{quote}“如果将有关\textbf{行政、司法}和\textbf{教会}方面的许多职务和费用削减,并且将那些为社会\textbf{工作极少}而所得\textbf{报酬极高}的牧师、律师、医生、批发商、零售商的人数削减,那末公共经费就会很容易地得到抵补。”(第 11 页)\end{quote}

(d)\textbf{贫民:“多余的人”}[supernumeraries]:

\begin{quote}“谁来供养这些人呢\fontbox{?}我的回答是:每一个人……在我看来,很明白,既不应该让他们饿死,也不应该将他们绞死,也不应该把他们送出国外”等等。(第 12 页)必须把“多余的东西”给他们,如果没有多余的东西,“如果\textbf{没有剩余}……则可把别人的丰美食物在数量或质量上\textbf{缩减一点}”。(第 12—13 页)什么劳动都可以让这些“多余的人”担负,只要这种劳动“无需耗用外国的商品”。主要的是,“使他们的精神养成守纪律和服从的习惯,使他们的肉体在必要时担当得了更加有利的劳动”。(第 13 页)“最好利用他们去筑路、架桥和开矿等等。”(第 12 页)\end{quote}

\textbf{人口——财富}:

\begin{quote}“\textbf{人口少是真正的贫穷};有 800 万人口的国家,要比领土面积相同而只有 400 万人口的国家富裕 1 倍以上。”(第 16 页)\end{quote}

关于上述(a)(\textbf{牧师})方面。配第对牧师进行了巧妙的讽刺:

\begin{quote}“牧师最守苦行的时候,宗教最繁荣,正如律师最清闲的时候……法律……最昌明一样。”(第 57 页)他在任何情况下都劝告牧师“\textbf{不要生出多于}现有\textbf{牧师俸禄}所能吸收的\textbf{牧师}”。例如,假定在英格兰和威尔士有 12000 份牧师俸禄。那末“生出 24000 个牧师,是不明智的”。因为,这样一来,12000 个无以为生的人就会同受俸牧师竞争,“他们要做到这一点,最容易的方法就是,向人们游说:那 12000 个受俸牧师在毒害人们的灵魂,使这些灵魂饿死〈这是暗指英国宗教战争〉,把他们引入歧途,使他们无法升入天国”。(第 57 页)\end{quote}

\textbf{剩余价值的起源和计算}。这个问题的叙述有些杂乱无章,但是,在苦苦思索寻求适当表达的过程中,分散在各处的中肯的见解就构成某种有联系的整体。

配第区分了“自然价格”、“政治价格”和“真正的市场价格”。(第 67 页)他所说的“\textbf{自然价格}”实际上是指\textbf{价值},这是我们在这里唯一感兴趣的东西,因为[1348]剩余价值的规定\textbf{取决于价值规定}。

配第在这本著作中,实际上用商品中包含的\textbf{劳动}的比较\textbf{量}来确定\textbf{商品的价值}。

\begin{quote}“但是在我们详细地论述\textbf{各种租金}之前,我们试图一方面联系\textbf{货币(它的租金叫做利息)},另一方面联系\textbf{土地和房屋}的租金,来说明租金的神秘性质。”(第 23 页)\end{quote}

首先要问,什么是商品的\textbf{价值},具体地说,什么是谷物的\textbf{价值}\fontbox{?}

\begin{quote}(α)“假定有人从秘鲁地下获得 1 盎斯银并带到伦敦来,他所用的时间和他生产 1 蒲式耳谷物所需要的\textbf{时间相等},那末,前者就是后者的自然价格;假定现在由于开采更富的新矿,获得 2 盎斯银象以前获得 1 盎斯银花费一样多,那末在其他条件相同的情况下,现在 1 蒲式耳谷物值 10 先令的价格,就和它以前值 5 先令的价格一样便宜。”(第 31 页)“假定生产 1 蒲式耳谷物和\textbf{生产 1 盎斯银}要用\textbf{相等的劳动}。”(第 66 页)这首先是“计算商品价格的真实的而不是想象的方法”。(第 66 页)\end{quote}

(β)现在要研究的第二点是\textbf{劳动的价值}。

\begin{quote}“法律……\textbf{应该使工人得到仅仅最必要的生活资料},因为,如果给工人双倍的生活资料,那末,工人做的工作,将只有他本来能做的并且在工资不加倍时实际所做的一半。\textbf{这对社会说来,就损失了同量劳动所创造的产品}。”(第 64 页)\end{quote}

可见,劳动的价值是由必要的生活资料决定的。工人之所以注定要生产剩余产品,提供剩余劳动,不过是因为人们强迫他用尽他全部可以利用的劳动力,以使他本人得到\textbf{仅仅最必要的生活资料}。因此,他的劳动的贵贱决定于两种情况:自然肥力和因气候影响而造成的费用(需要)大小:

\begin{quote}“自然的\textbf{贵贱}取决于\textbf{需要用多少人手来满足自然需要}(所以,谷物在 1 个人能为 10 个人生产的地方,比 1 个人只能为 6 个人生产的地方便宜);此外还取决于气候,因为它使人们或者多花费一些,或者少花费一些。”(第 67 页)\end{quote}

(γ)在配第看来,\textbf{剩余价值}只有两种形式:\textbf{土地的租金}和\textbf{货币的租金}(利息)。他是从前者推出后者的。前者,在他看来,正如后来在重农学派看来一样,是剩余价值的\textbf{真正的形式}。(可是配第同时声明,谷物应该指

\begin{quote}“一切生活必需品,就象主祷文中的‘面包’一词那样”。)\end{quote}

他在叙述中不仅把租金(剩余价值)说成是雇主抽取的超过必要劳动时间的余额,并且把它说成是生产者本人超出他的工资和他自己的资本的补偿额之上的剩余劳动的余额。

\begin{quote}“假定一个人用自己的双手在一块土地上种植谷物,就是说,他干了耕种这块土地所要干的活,如翻地或犁田、耙地、除草、收割、搬运回家、脱粒、簸扬等等,并且假定他有播种这块土地所需的\textbf{种子}。我认为,这个人\textbf{从他的收成中扣除自己的种子}〈就是说,第一,从产品中扣除了不变资本的等价物〉,[1349]并扣除自己食用的部分以及为换取衣服和其他必需品而给别人的部分之后,\textbf{剩下的谷物}就是\textbf{当年自然的和真正的地租};而 7 年的平均数,或者更确切地说,形成歉收和丰收循环周期的若干年的平均数,就是用谷物表示的这块土地的通常的地租。”(第 23—24 页)\end{quote}

可见,在配第看来,因为谷物的价值决定于它所包含的劳动时间,而地租等于总产品减去工资和种子,所以地租等于剩余劳动借以体现的剩余产品。这里,地租包括利润;利润还没有同地租分开。

接着,配第又以同样机智的方式问道:

\begin{quote}“但是,这里可能发生一个尽管是附带的、但需要进一步解决的问题:\textbf{这种谷物或这种地租值多少英国货币呢}\fontbox{?}我的回答是:值多少货币,要看另一个在同一时间内完全从事货币生产的人,除去自己全部费用之外还能剩下多少货币。也就是说,假定这个人前往产银地区,在那里采掘和提炼银,然后把它运到第一个人种植谷物的地方铸成银币,等等;并且假定这个人在他生产银的全部时间内,同时也谋得生活所必需的食物和衣服等等。这样,我认为\textbf{一个人的银和另一个人的谷物在价值上必定相等}。假定银是 20 盎斯,谷物是 20 蒲式耳,那末,1 盎斯银就是 1 蒲式耳谷物的价格。”(第 24 页)\end{quote}

此外,配第明确地指出,劳动种类的差别在这里是毫无意义的——一切只取决于\textbf{劳动时间}。

\begin{quote}“即使生产银比生产谷物可能需要更多的技术和冒更大的风险,但是结局总是一样的。假定让 100 个人\textbf{在 10 年内}生产谷物,又让\textbf{同样数目的人在同一时间内}开采银;我认为,银的\textbf{纯产量}将是\textbf{谷物全部纯收获量的价格},前者的同样部分就是后者的同样部分的价格。”(第 24 页)\end{quote}

配第在这样确定了\textbf{地租}(这里等于包括利润在内的全部\textbf{剩余价值})和找到了地租的货币表现之后,就着手确定\textbf{土地的货币价值},这又是很有天才的。

\begin{quote}“因此,如果我们能够发现可以自由买卖的土地的\textbf{自然价值},即使这种发现不见得比我们发现上述 ususfructus\authornote{ususfructus 指对别人财产(主要是地产)的使用权;这里是指土地的纯收入。——编者注}的价值好多少,我们也会觉得欣慰……在我们发现了\textbf{地租或一年 ususfructus 的价值}之后,产生了一个问题:一块可以自由买卖的土地的自然价值等于(用我们平常的说法)\textbf{多少年的年租}\fontbox{?}如果我们说年数无限,那就是说 1 英亩土地的价值等于 1000 英亩同样土地的价值(因为 1 的无限等于 1000 的无限),这是荒谬的。因此,我们必须选定一个\textbf{有限的}年\textbf{数}。我认为这个年数就是一个 50 岁的人、一个 28 岁的人、一个 7 岁的人可以同时生活的年数,也就是祖、父、子三代可以同时生活的年数。因为很少有人会为再下一代的子孙操心……所以我认为,\textbf{构成任何一块土地的自然价值的年租的年数},等于这三代人通常[1350]可以同时生活的年数。我们估计在英国这三代人可以同时生活 21 年,因而\textbf{土地的价值}也大约等于\textbf{21 年的年租}。”(第 25—26 页)\end{quote}

配第把地租归结为\textbf{剩余劳动},因而归结为\textbf{剩余价值}之后说道,土地不过是资本化的地租,即\textbf{一定年数的年租},或者说,一定年数的地租总额。

实际上,地租是\textbf{这样资本化}的,或者说,是\textbf{这样}作为\textbf{土地的价值}计算的:

假定 1 英亩土地每年带来 10 镑地租。如果利率等于 5\%,10 镑就代表 200 镑资本的利息,又因为利率是 5\%时利息在 20 年内就补偿了资本,所以,1 英亩土地的价值等于 200 镑(20×10 镑)。可见,地租的资本化取决于利率的高低。如果利率等于 10\%,10 镑就代表 100 镑资本或者说 10 年收入总额的利息。

但是,因为配第是从作为包括利润在内的剩余价值一般形式的\textbf{地租}出发的,所以他不能把资本的利息作为既定的东西,反而必须把利息当作地租的\textbf{特殊形式}从地租中推出来(杜尔哥也是这样做的,这从他的观点来看是完全合乎逻辑的)。但是,这里如何确定形成\textbf{土地的价值}的年数即年租的年数呢\fontbox{?}一个人[配第推论说]有兴趣购买的年租的年数,只是他要为自己和自己最近的后代“操心”的年数,就是说,只是一个\textbf{平均人}——祖、父、子三代——生活的年数。这个年数按“英国的”估计是 21 年。因此,21 年《ususfructus》以外的东西,对他毫无价值。因此他支付 21 年《ususfructus》的代价,而这也就形成\textbf{土地的价值}。

配第用这种巧妙的方式使自己摆脱了困难;但是,这里重要的是:

第一,\textbf{地租},作为全部\textbf{农业剩余价值}的表现,不是从土地,而是从劳动中引出来的,并且被说成劳动所创造的、超过劳动者维持生活所必需的东西的余额;

第二,\textbf{土地的价值}不外是预购的一定年数的地租,是地租本身的\textbf{转化}形式,在这种形式中,若干年(例如 21 年)的剩余价值(或剩余劳动)表现为\textbf{土地的价值};总之,\textbf{土地的价值}无非是\textbf{资本化的地租}。

配第如此深刻地看到问题的实质。因此,从地租购买者(即土地购买者)的观点来看,地租只表现为他用来购买地租的他的资本的利息,而在这种形式中,地租已经变得完全无法辨认,并且表现为资本利息了。

配第这样确定了\textbf{土地的价值}和\textbf{年租的价值}之后,就能够把“货币的租金”即利息当作派生的形式引出来了。

\begin{quote}“至于\textbf{利息},在保证没有问题的地方,它至少要同\textbf{贷出的货币所能买到的那么多土地的租金}相等。”(第 28 页)\end{quote}

这里,利息由\textbf{地租的价格}决定,而实际上相反,\textbf{地租的价格}即\textbf{土地的购买价值}是由利息决定的。但这样说是完全合乎逻辑的,因为配第把\textbf{地租}说成是剩余价值的一般形式,所以必然把\textbf{货币的利息}当作派生的形式从地租引出来。

\textbf{级差地租}。在配第的著作中,我们也看到关于级差地租的最初概念。他不是从同样大小地段的\textbf{不同}肥力引出级差地租,而是从\textbf{同等肥力的地段的不同位置}、从它们对市场的不同距离引出级差地租,大家知道,后者是级差地租的一个要素。他说:

\begin{quote}[1351]“正如对货币的需求大就会提高货币行市一样,对谷物的需求大也会提高\textbf{谷物的价格,从而提高种植谷物的土地的租金}\end{quote}

(可见,这里直接说出了谷物的\textbf{价格}决定地租,正如在前面的阐述中已经包含着地租不决定谷物\textbf{价值}的意思一样),

\begin{quote}\textbf{最后还提高土地本身的价格}。例如,如果供应伦敦或某一支军队的谷物必须从 40 英里远的地方运来,那末,在\textbf{离伦敦}或这支军队驻地\textbf{只有 1 英里的地方种植的谷物,它的自然价格还要加上}把谷物运输 39 英里的费用……由此产生的结果是,在靠近需要由广大地区供应粮食的人口稠密地方的土地,由于这个原因,比距离远而\textbf{土质相同的土地},不仅提供\textbf{更多的地租},并且所值的年租总额也更多”等等。(第 29 页)\end{quote}

配第也提到级差地租的第二个原因,即土地的\textbf{不同肥力},以及由此而来的同等面积的土地上劳动的\textbf{不同生产率}:

\begin{quote}“土地的好坏,或土地的价值,取决于人们\textbf{为利用土地而支付的产品的或大或小的部分对\CJKunderdot{生产上述产品所花费的简单劳动的比例}}。”(第 67 页)\end{quote}

由此可见,配第\textbf{比亚当·斯密更好地}阐明了级差地租。[XXII—1351)

\centerbox{※     ※     ※}

[XXII—1397]《\textbf{赋税论}》(1667 年版)\textbf{补录}。

(1)\textbf{关于}一国所必需的\textbf{流通货币}量,第 16—17 页。

配第对于\textbf{总生产}的看法,可以从下面的话里看出来:

\begin{quote}“如果某地有 1000 人,其中 100 人能够生产全体 1000 人所必需的食物和衣服;另外 200 人生产的商品和别国用来交换的商品或货币一样多,另外 400 人为全体居民的装饰、娱乐和华丽服务,如果还有 200 人是行政官吏、牧师、法官、医生、批发商和零售商,共计 900 人,那末就有一个问题”等等,接着讲了关于贫民(“多余的人”)的事。(第 12 页)\end{quote}

配第在阐述地租和地租的货币表现时——这里他以“\textbf{相等的劳动}”\authornote{见本册第 380 页。——编者注}(相等的劳动量)为基础,——说道:

\begin{quote}“我断定,这一点是\textbf{平衡和衡量各个价值的基础};但是在它的上层建筑和实际应用中,我承认情况是多种多样的和错综复杂的。”(第 25 页)\end{quote}

[1398](2)配第十分注意“\textbf{土地和劳动之间的自然的等同关系}”。

\begin{quote}“我们的银币和金币有各种不同的名称,例如,在英国叫做镑、先令和便士;所有这些铸币都可以用这三种名称中的任何一种来称呼,来理解。但是,关于这个问题,我要指出的是:一切东西都应由\textbf{两个自然单位——土地和劳动来评定价值};换句话,我们应该说:一条船或者一件衣服的价值等于若干土地的价值加上若干数量的劳动,因为船和衣服都是\textbf{土地和人类劳动的产物}。既然这样,我们就很想找出\textbf{土地和劳动之间的自然的等同关系},使我们单用土地或单用劳动来表现价值能够和同时用二者来表现价值一样好(甚至更好),而且能够象把便士折合为镑那样容易和可靠地将一个单位折合成另一个单位。”\end{quote}

因此,配第在找到地租的货币表现之后,又去找“可以自由买卖的土地的\textbf{自然价值}”(第 25 页)。

在配第那里有三种规定混在一起:

(a)由等量劳动时间决定的\textbf{价值量},在这里,\textbf{劳动}被看作\textbf{价值的源泉}。

(b)作为社会劳动的形式的\textbf{价值}。因此,货币表现为\textbf{价值的真正形式},虽然配第在其他地方抛弃了货币主义的一切幻想。所以在他的著作里形成\textbf{定义}。

(c)把作为交换价值的源泉的劳动和作为以自然物质(土地)为前提的使用价值的源泉的劳动混为一谈。实际上,当他建立劳动和土地之间的“等同关系”的时候,他把可以自由买卖的土地说成是\textbf{资本化的地租},因而这里他谈的,不是作为同实在劳动有关的自然物质的土地。

(3)关于\textbf{利率},配第说:

\begin{quote}“我在别处已经说到,制定违反\textbf{自然法}〈就是由资产阶级生产本性产生的法律〉的\textbf{成文民法}是徒劳无益的。”(第 29 页)\end{quote}

(4)关于\textbf{地租}:由于\textbf{劳动生产率较高}而产生的\textbf{剩余价值}:

\begin{quote}“如果上述那些郡,用比现在更多的劳动(如用翻地代替犁田,用点种代替散播,用选种代替任意取种,用浸种代替事先不作准备,用盐类代替腐草施肥等等)能够获得更大的丰产,那末,\textbf{增加的收入超过增加的劳动}越多,\textbf{地租}也上涨得越多。”(第 32 页)\end{quote}

(这里说的增加的劳动是指上涨了的“\textbf{劳动价格}”或\textbf{工资}。)

(5)\textbf{由国家提高货币的价值}(第 14 章)。

(6)前面引过的\authornote{见本册第 380 页。——编者注}一句话“如果给工人双倍的生活资料,那末,工人做的工作,将只有……一半”,应该这样理解:如果工人劳动 6 小时,得到他在这 6 小时内创造的价值,那他就得到他现在所得的二倍,——现在,他劳动 12 小时,只得到他在 6 小时内创造的价值。这样一来,他就会只劳动 6 小时了,“这对社会说来,就损失了”等等。

\centerbox{※     ※     ※}

\textbf{配第}《论人类的增殖》(1682 年)。《分工》(第 35—36 页)。

\centerbox{※     ※     ※}

\textbf{配第}《爱尔兰政治剖视》(1672 年)和《献给英明人士》(1691 年伦敦版)。

\begin{quote}(1)“这使我遇到\textbf{政治经济学}上最重要的一个问题,就是,怎样建立土地和劳动之间的\textbf{等同关系和等式},以便用这两个因素之一来表示任何东西的价值。”(第 63—64 页)\end{quote}

实际上,作为提出这一问题的基础的只是把\textbf{土地}本身的\textbf{价值}归结为\textbf{劳动}。

[1399](2)这部著作比前面考察过的著作\endnote{指 1662 年第一次发表的《赋税论》。——第 388 页。}写得晚。

\begin{quote}“\textbf{价值的一般尺度},是平均\textbf{一个成年男人的一天食物},而\textbf{不是他的一天劳动};这个尺度同纯银的价值一样有规则,一样稳定……因此,我认为一所爱尔兰茅屋的\textbf{价值,是用建筑茅屋的人在建筑时消费了多少天的食物来确定的}。”(第 65 页)\end{quote}

最后这一段,完全是重农主义的调子。

\begin{quote}“有些人吃得比别人多,这是无关紧要的,因为我们所说的一天食物,是指 100 个各种各样的、体格不同的人为生活、劳动和繁殖所吃的食物的 1\%。”(第 64 页)\end{quote}

但是,这里配第在爱尔兰\textbf{统计}中所找的,不是价值的“\textbf{一般}尺度”,而是\textbf{货币}是价值尺度这个意义上的\textbf{价值}尺度。

(3)\textbf{货币数量和国家财富}(《献给英明人士》第 13 页)。

(4)\textbf{资本}:

\begin{quote}“我们称为\textbf{一个国家的财富、资本或储备}并且是\textbf{以前或过去劳动}的成果的东西,不应看成\textbf{同现在发挥作用的力量有区别的}东西。”(第 9 页)\end{quote}

(5)\textbf{劳动的生产力}:

\begin{quote}“我们说过,人口的一半,花费不多的劳动就可以使王国大大富足……这些人应该把劳动用于生产什么呢\fontbox{?}对于这个问题,我的一般回答是:\textbf{由少数人}为全国居民生产食物和必需品;或者靠\textbf{更紧张的劳动},或者靠\textbf{采用节省和减轻劳动的手段};这样得到的结果,和人们徒然希望从\textbf{一夫多妻制}得到的结果相同。因为一个人能做五个人的工作,得到的结果就好比他生了四个成年劳动者。”(第 22 页)“当\textbf{生产}食物\textbf{使用的人手比任何别的地方都少}的时候……\textbf{食物将最便宜}。”(第 23 页)\end{quote}

(6)人的目的和目标(第 24 页)。

(7)关于\textbf{货币},也可参考《货币略论》(1682 年)。[XXII—1399]

\tsectionnonum{[(3)]配第、达德利·诺思爵士、洛克}

[XXII—1397]把诺思和洛克的著作同配第的《货币略论》(1682 年)、《赋税论》(1662 年)以及《爱尔兰政治剖视》(1672 年)比较一下,就可以看出,在(1)关于\textbf{利率降低}的问题,(2)关于国家提高和降低货币价值的问题,(3)\textbf{诺思}的把利息称为“货币的租金”等等问题上,诺思和洛克两人都是追随配第的。

\textbf{诺思}和\textbf{洛克}就同一个问题即\textbf{利率降低}和\textbf{国家提高货币价值}的问题,同时写了他们的著作\endnote{指诺思的《贸易论》和洛克的《论降低利息和提高货币价值的后果》。这两本著作都写于 1691 年,在伦敦出版:前者于 1691 年出版,后者于 1692 年出版。——第 389 页。}。但是他们阐明的观点是完全对立的。洛克认为\textbf{缺乏货币}是高利率的原因,一般说来是货物不能按它们的实际价格出卖并带来应有的收入的原因。相反,诺思指出,原因不是流通中缺乏货币,而是缺乏资本或收入。在他的著作中,第一次出现关于 stock\authornote{储备、基金、资金。——编者注}即\textbf{资本}的明确概念,或者更确切地说,关于不作为流通手段只作为\textbf{资本的形式的货币}的明确概念。在\textbf{达德利·诺思}爵士那里,我们看见同洛克的观点对立的关于利息的第一个正确的概念。[XXII—1397]

\tsectionnonum{[(4)]洛克}

\vicetitle{[从资产阶级自然法理论观点来解释地租和利息]}

[XX—1291a]如果我们把洛克关于劳动的一般观点同他关于\textbf{利息}和\textbf{地租的起源}的观点(因为在洛克那里,剩余价值只表现为利息和地租这两种特定形式)对照一下,那末,剩余价值无非是土地和资本这些劳动条件使它们的所有者能够去占有的\textbf{别人劳动},剩余劳动。在洛克看来,如果劳动条件的数量大于一个人用自己的劳动所能利用的数量,那末,对这些劳动条件的所有权,就是一种同私有制的自然法基础相矛盾的[1292a]\textbf{政治}发明。

\fontbox{~\{}在\textbf{霍布斯}那里,除了处于直接可供消费状态的自然赐予之外,劳动也是一切财富的唯一源泉。上帝(自然)

\begin{quote}“或者把必要的东西\textbf{无代价地赐给}人类,或者\textbf{要劳动作交换卖给}人类”。(《利维坦》[第 232 页])\end{quote}

但是,在霍布斯那里,土地所有权由君主随意分配。\fontbox{\}~}

下面是洛克著作中与此有关的几段话:

\begin{quote}“虽然\textbf{土地}和一切低等生物是一切人所共有的,但是每一个人仍然有一个所有物,就是他自己的人身,对于这个所有物,除了他自己以外,别人是没有任何权利的。他的身体的劳动和他的双手的创作,我们可以说,是理应属于他的。他把他从自然创造并提供给他的东西中取得的一切,同自己的劳动溶合起来,同一种属于他的东西溶合起来;他以这种方式使这一切成为自己所有。”(《论政府的两篇论文》第 2 篇第 5 章;《约翰·洛克著作集》1768 年第 7 版第 2 卷第 229 页)“这一切在自然手里,都是公共所有,都一视同仁地属于自然的全体子女;人的劳动把这一切从自然手里拿过来,从而把它们据为己有。”(同上,第 230 页)“以这种方式给予我们所有权的这一自然法,同时也限制了这个所有权的范围……一个人在对他的生活有某种用处的东西损坏之前能够使用它多少,他用自己的劳动可以使它变为自己所有的也就多少;超出这个限度的,就是超过他的份额而属于别人的东西。”(同上)“但是,现在所有权的主要对象不是\textbf{土地的果实}等等,而是\textbf{土地}本身……一个人能够耕作、播种、施肥和种植多大的土地,能够享用多大土地的产品,多大的土地就是他的所有物。人就好比是用自己的劳动把它从公共财产中圈出来。”(同上)“我们看到,开垦或耕作土地和占有土地,是互相连结在一起的。前者为后者提供了权利。”(同上,第 231 页)“自然已经\textbf{按照人的劳动}以及人的生活方便\textbf{所能达到的程度},正确地确定了所有权的尺度:谁都不可能用自己的劳动征服或占有一切;谁都不可能为了满足自己的需要而消费比这一小部分更多的东西;因此谁都不可能用这种形式侵犯别人的权利,或者为自己取得所有权而损害邻人的利益……早先,这个尺度使每个人的占有限于非常小的一份,限于他自己能够占有而不损害别人利益的范围……\textbf{就是现在},尽管全世界似乎挤满了人,\textbf{仍然可以承认}同一尺度而不损害任何人的利益。”(第 231—232 页)\end{quote}

劳动几乎提供了一切东西的全部价值\fontbox{~\{}在洛克那里,价值等于使用价值,劳动是指具体劳动,不是指劳动的量;但是,交换价值以劳动为尺度,实际上是以劳动者创造使用价值为基础的\fontbox{\}~}。不能归结为劳动的使用价值余额,在洛克看来,是自然的赐予,因而,就它本身来说是\textbf{公共所有物}。因此,洛克想要证明的,不是除劳动之外还可以通过其他办法获得所有权这个同他原来的观点相矛盾的论题,而是怎样才能通过个人劳动创造个人所有权,尽管自然是公共所有物。

\begin{quote}“实际上正是\textbf{劳动决定一切东西的价值的差别}……对人的生活有用的土地产品……有 99%完全要记在劳动的账上。”(第 234 页)“因此,劳动决定土地价值的最大部分。”(第 235 页)“虽然自然的一切东西是给予一切人共同所有的,可是,人是\textbf{他自己的主人},是\textbf{他自身}及其活动或劳动的\textbf{占有者},作为这样的一个人,他本身已经包含着所有权的重大基础。”(第 235 页)\end{quote}

所以,所有权的一个界限是\textbf{个人劳动的界限};另一个界限是,一个人储存的东西不多于他能够使用的东西。后一个界限由于把容易损坏的产品同\textbf{货币}交换(撇开别种交换不说)而扩大了:

\begin{quote}“这种\textbf{耐久的}东西,一个人愿意储存多少,就可以储存多少;所谓超出他的正当所有权的界限\fontbox{~\{}撇开他个人劳动的\textbf{界限}不谈\fontbox{\}~},不是指他有很多东西,而是指其中一部分东西损坏了,对他没有用了。于是货币这种耐久的东西就被采用,人们可以把它储存起来而不致于损坏,并根据相互协议,把它用于[1293a]交换真正有用但容易损坏的生存资料。”(第 236 页)\end{quote}

这样就产生了个人所有权的不均等,但是\textbf{个人劳动}这一\textbf{尺度}仍然有效。

\begin{quote}“人们之所以可能超过社会确定的界限,不经协议,把财物分成不均等的私人财产,只是因为他们使金银具有了价值,默认货币的使用。”(第 237 页)\end{quote}

应该把这段话同洛克关于利息的著作\endnote{这一著作是《论降低利息和提高货币价值的后果》(1691 年)。——第 392 页。}中的下面一段话加以对比,不要忘记,照他看来,自然法使\textbf{个人劳动}成为所有权的界限:

\begin{quote}“现在我们就来考察一下,货币怎么会具有同土地一样的性质,提供我们称作利钱或利息的一定年收入。因为土地自然地生产某种新的、有用的和对人类有价值的东西;相反,货币是不结果实的,它不会生产任何东西,但是,它通过相互协议,\textbf{把作为一个人的劳动报酬的利润转入另一个人的口袋}。这种情况是由货币分配的不均等引起的;这种不均等对土地产生的影响,同它对货币产生的影响一样……土地分配的这种不均等(你的土地多于你能够耕种或愿意耕种的,而另一个人的土地却少于他能够耕种或愿意耕种的)会为你招来一个租种你的土地的佃户;而货币分配的这种不均等……会为我招来一个借用我的货币的债户;这样一来,\textbf{我的货币靠债务人的勤劳,能够}在他的营业中为他带来多于 6%的收入,正如你的土地\textbf{靠佃户的劳动}能够生产一个大于他的地租的收益。”(《约翰·洛克著作集》1740 年对开本版第 2 卷[第 19 页])\endnote{马克思这里引用的洛克的话,取自马西的著作《论决定自然利息率的原因》第 10—11 页的引文。在 1768 年出版的洛克著作集中,这段话在第 2 卷第 24 页。——第 393 页。}\end{quote}

在洛克的这段话里,部分含有论战的意思,他想向土地所有者指出,他们的地租同高利贷者所取得的利息完全没有区别。地租和利息都由于生产条件的不均等的分配而“把作为一个人的劳动报酬的利润转入另一个人的口袋”。

因为洛克是同封建社会相对立的资产阶级社会的法权观念的经典表达者;此外,洛克哲学成了以后整个英国政治经济学的一切观念的基础,所以他的观点就更加重要。[XX—1293a]

\tsectionnonum{[(5)]诺思}

\vicetitle{[作为资本的货币。商业的发展是利率下降的原因]}

[XXIII—1418]\textbf{达德利·诺思爵士}《贸易论》1691 年伦敦版(补充本 C)\endnote{马克思指他的 1861—1863 年手稿的“补充本”(Beihefte)之一,他在 1863 年 5 月 29 日曾写信给恩格斯说,1863 年春他在这些补充本中“摘录了同我已写好的部分有关的政治经济学史方面的各种材料”。现在有补充本 A、B、C、D、E、F、G、H。诺思著作的摘要在补充本 C 第 12—14 页。——第 394 页。}。

这部著作同洛克的经济学著作完全一样,也和配第的著作直接有关,并直接以配第的著作为依据。

诺思的著作主要是研究\textbf{商业资本},就这点来说,和这里要讨论的东西无关。诺思在他所研究的问题的范围内,表现了行家的谙练。

非常值得注意的是,从查理二世复辟时期到十八世纪中叶,地主对地租下降不断发出怨言(因为小麦价格特别从\fontbox{?}年\endnote{手稿这里本来写了“从 1688 年起”,但后来 1688 这个数字被勾掉,而代之以问号。马克思在他的 1861—1863 年手稿第 XI 本第 507—508 页引用了 1641 年起小麦价格变动的资料。1641—1649 年小麦的平均价格是每夸特 60 先令 5+(2/3)便士,十七世纪下半叶平均价格降到每夸特 44 先令 2+(1/5)便士,十八世纪上半叶降到每夸特 35 先令 9+(29/50)便士。——第 394 页。}起,不断下降)。虽然(从卡耳佩珀和约瑟亚·柴尔德爵士以来)工业资本家阶级积极参与强制压低利率,但是这个措施的真正鼓吹者是\textbf{土地所有者}。“\textbf{土地的价值}”和“提高这个价值的方法”作为国民利益被提到首位。(正如,相反的,大约从 1760 年起,地租、土地价值、谷物和其他食物的价格的上涨以及工业资本家对此发出的怨言,成为经济学上对这个对象进行研究的基础一样。)

从 1650 年到 1750 年的整个世纪,除了少数例外,不断发生货币所有者和土地所有者之间的斗争,因为生活阔绰的贵族,看到高利贷者把他们抓在手里,又看到自从十七世纪末建立了现代信用制度和国债制度以后,高利贷者在立法等方面占了他们的上风,心中十分不快。

\textbf{配第}已经谈到地主对地租下降发出怨言和他们反对农业改良的事情(参看有关的段落)。\endnote{马克思显然是指他在《剩余价值理论》第二册论洛贝尔图斯一章(手稿第 494 页)所引用的配第《政治算术》(1676 年)第四章的一段话。参看《资本论》第 3 卷第 39 章:“在配第和戴韦南特时期,农民和土地所有者对改良和开发发出怨言;较好土地的地租下降……”——第 395 页。}他维护高利贷者而反对地主,他把货币的租金同土地的租金相提并论。

\textbf{洛克}把这两者都归结为对劳动的剥削。他同配第站在同一立场。他们两人都反对对利息作强制性的调整。土地所有者注意到,如果利息下降,\textbf{土地的价值}就上升。在地租量已定时,地租的\textbf{资本化的表现}即土地的价值,其升降就同利率的高低成反比。

\textbf{达德利·诺思爵士}在上述著作中是配第路线的第三个代表。

这是\textbf{资本}起来反抗\textbf{土地所有权}的最初形式。因为实际上,\textbf{高利贷},即土地所有者的一部分收入转入高利贷者手中,是资本积累的主要手段之一。但是,工业资本和商业资本或多或少同土地所有者携手共同反对资本的这种古旧的形式。

\begin{quote}“正如土地所有者出租他的土地一样,这些人(他们有\textbf{资本}[stock]可用于商业,可是由于没有必要的才干或由于怕辛苦而没有用于商业)就出借他们的\textbf{资本}。他们从中得到的东西叫做\textbf{利息},但是,利息不过是资本的\textbf{租金}\end{quote}

\fontbox{~\{}在这里如同在配第那里一样,我们看到,对于刚从中世纪走出来的人,租金[1419]表现为剩余价值的原始形式\fontbox{\}~},

\begin{quote}就象土地所有者的收入是土地的租金一样。在几种语言中,借货币和租土地是通用的说法:英国有几个郡的情况也是如此。因此,当一个\textbf{地主}[landlord]或当一个\textbf{财主}[stocklord]是一回事;地主有利的地方,只在于他的佃户不能把土地带走,而资本的债户却很容易把资本带走;因此,土地提供的\textbf{利润}应当比冒更大风险借出的资本提供的利润\textbf{少}。”(第 4 页)\end{quote}

\textbf{利息}。诺思看来是第一个正确理解利息的人,因为,从后面引用的几段话可以看到,stock 这个词他不仅指货币,并且指资本(同样,配第也把 stock 和\textbf{货币}区别开来。在洛克那里,利息完全决定于流通中的货币量,在配第那里也是这样。\textbf{参看马西关于这个问题的论述})。

\begin{quote}“如果放债人多于借债人,利息将下降……不是低利息使商业活跃,而是\textbf{在商业发展时国民资本}使利息下降。”(第 4 页)“金、银和用金银铸造的货币无非是重量和尺度,有它们比没有它们更便于进行交易;此外,它们又是适于\textbf{存放多余资本}的基金。”(第 16 页)\end{quote}

\textbf{价格和货币}。因为价格无非是用\textbf{货币}表示并且由货币实现的(在我们说的是\textbf{卖}的情况下)商品的\textbf{等价物},也就是说,因为\textbf{商品}先在价格中表现为\textbf{交换价值},以便以后再转化为使用价值,所以,在经济思想方面迈出的最初的步伐之一,就是认为金银在这里只作为商品本身的\textbf{交换价值的存在形式},作为\textbf{商品形态变化的一个因素}出现,而不作为金银本身出现。就诺思那个时代来说,诺思把这一点说得很巧妙:

\begin{quote}“因为货币……是买和卖的普遍的尺度,所以每一个要卖东西而找不到买者的人,总以为他的商品卖不出去是因为\textbf{国内缺乏货币};因此到处都叫嚷缺乏货币。然而这是一个大错误……那些叫嚷缺乏货币的人究竟要什么呢\fontbox{?}我从\textbf{乞丐}说起……他要的不是货币,而是面包和其他生活必需品……租地农民抱怨缺乏货币……他以为,如果国内有较多的货币,他的货物就可以卖到好价钱。看来,他要的不是货币,而是他想卖但又卖不出去的谷物和牲畜的好价钱……为什么他卖不到好价钱呢\fontbox{?}……(1)或者是因为国内谷物和牲畜太多,到市场上来的人大多数都象他那样要卖,但只有少数人要买。(2)或者是因为通常的出口停滞,例如在战时,贸易不安全或不准进行。(3)或者是因为消费缩减,例如,人们由于贫困,不能再花费过去那样多的生活费用。可见,有助于租地农民出售货物的,不是增加货币,而是消除这三个真正造成市场缩减的原因中的任何一个原因。批发商和零售商也同样要货币,就是说,因为市场停滞,他们要把他们经营的货物销售出去。”(第 11—12 页)\endnote{这段引文(从“我从乞丐说起”开始),据马克思在 1861—1863 年手稿第 XXIII 本第 1419 页注明,系取自补充本 C 第 12—13 页。——第 397 页。}\end{quote}

其次,\textbf{资本}是\textbf{自行增殖的价值},而\textbf{货币贮藏}却以\textbf{交换价值的结晶形式}本身为目的。因此,古典政治经济学最早的发现之一,是它认识了\textbf{货币贮藏}和\textbf{货币自行增殖}之间的对立,也就是说,它论述了\textbf{作为资本的货币}。

\begin{quote}“谁也不会因为用货币、金银器等形式把自己的全部财产留在身边而变富,相反,倒会因此而变穷。只有财产\textbf{正在增长}的人才是最富的人,不管他的财产是租出去的土地,还是放出去生息的货币,还是投入商业的货物。”(第 11 页)\end{quote}

\fontbox{~\{}约翰·贝勒斯在他所著《论贫民、工业、贸易、殖民地和道德堕落》一书(1699 年伦敦版)中谈到这个问题时说道:

\begin{quote}“货币只有放出去才能\textbf{增殖},才有用处;正如一个私人除非用货币去换取某种更有价值的东西,否则货币对他就无利可图,同样,超过国内贸易绝对需要的全部货币量,对于一个王国或一个民族来说,是死资本,它不会给让货币停滞不动的国家带来任何利润。”(第 13 页)\fontbox{\}~}“虽然每一个人都愿意有它〈货币〉,可是没有一个人或者只有很少的人愿意把它保存起来,大家都力求把它立刻花出去;因为大家都知道,从一切放着不用的货币中,不能得到任何利润,只会受到损失。”(\textbf{诺思},同上第 21 页)\end{quote}

[1420]\textbf{作为世界货币的货币}。

\begin{quote}“从商业来说,一个国家在世界上的地位,无论哪一方面都同一个城市在一个王国中的地位,或者一个家庭在一个城市中的地位一样。”(第 14 页)“在这种商业交往中,金银同其他商品毫无区别,人们从金银过多的人手里把金银拿来,转交给缺少金银或需要金银的人。”(第 13 页)\end{quote}

\textbf{能够流通的货币量决定于商品交换}。

\begin{quote}“不论从国外带进多少货币,或者在国内铸造多少货币,凡是超过一国商业的需要的,\textbf{都只是金银条块},并且只有拿它当作金银条块对待;而且铸造的货币就象旧金银器一样,只有按它所包含的金属成色出卖。”(第 17—18 页)\end{quote}

货币变成金银条块和相反的情况(第 18 页)(补充本 C,第 13 页)。货币的\textbf{估价}和\textbf{衡量}。上下波动(补充本 C,第 14 页)\endnote{在补充本 C 第 14 页有诺思著作的摘录,诺思谈到一国货币流通中的“涨落”。马克思在《资本论》第一卷第三章注 95 中引用了其中一部分摘要。——第 398 页。}。

\textbf{高利贷、土地所有者和商业}:

\begin{quote}“在我国,取息的货币,\textbf{放给商人}去经营业务的还不到 1/10;大部分是借给这样一些人去维持奢侈生活和其他开销的,这些人虽然是大地产的所有者,但是,他们花费收入比他们的地产带来收入快,他们不愿出卖自己的财产,宁愿拿财产去抵押。”(\textbf{诺思},同上第 6—7 页)[XXIII—1420]\end{quote}

\tsectionnonum{[(6)贝克莱论勤劳是财富的源泉]}

[XIII—670a]“难道认为\textbf{土地本身}就是\textbf{财富}不是错误的吗\fontbox{?}难道我们不应当首先把人民的勤劳看成这样一种东西,它形成财富,甚至使那些除了作为勤劳的\textbf{手段和刺激}以外便毫无价值的土地和白银变成财富\fontbox{?}”(\textbf{乔·贝克莱}博士《提问者》1750 年伦敦版。第 38 个问题)[XIII—670a]

\tsectionnonum{[(7)]休谟和马西}

\tsubsectionnonum{[(a)马西和休谟著作中的利息问题]}

[XX—1293a]马西的匿名著作《论决定自然利息率的原因》于 1750 年出版;休谟的《论丛》第二卷,其中有《论利息》,于 1752 年出版,比前书迟了两年。因此马西在先。休谟反对洛克,而马西反对配第和洛克,配第和洛克两人还抱着这样的观点,即认为利息率的高低取决于流通中的货币量,认为真正被拿来贷放的东西实际上是货币(而不是资本)。

马西比休谟更坚决地宣称,\textbf{利息}只不过是利润的一部分。休谟主要证明货币的价值对利息率的高低没有意义,因为在利息和货币资本之间的比率已知(譬如说 6\%)的情况下,6 镑的价值同 100 镑(也可以说 1 镑)的价值一起升降,但并不影响用 6 这个数字表示的比率。

\tsubsectionnonum{[(b)休谟。由于商业和工业增长而引起的利润和利息的降低]}

我们从休谟谈起。

\begin{quote}“世上一切都是用劳动购买的。”(《论丛》第 1 卷第 2 部分,1764 年伦敦版[《论商业》]第 289 页)\end{quote}

在休谟看来,利息率的高低取决于借债人的需求和放债人的供给,即取决于供求。但是后来,它本质上取决于

\begin{quote}“从商业中产生出来的利润”的高低。(同上[《论利息》],第 329 页)“劳动储备和商品储备的多少,对于利息必定有重大影响,因为我们出利息借货币,借的实际上就是劳动和商品。”(同上,第 337 页)“在可以得到高利息的地方,没有人会以低利润为满足,而在可以得到高利润的地方,也没有人会以低利息为满足。”(同上,第 335 页)\end{quote}

高利息和高利润这两者是

\begin{quote}“商业和工业不够发达”的表现,“而不是缺乏金银”的表现,“低利息则表明相反的情况”。(同上,第 329 页)[1294a]“因此,在一个只有土地所有者〈或者象休谟后来说的,“地主和农民”〉的国家,借债人必定多,利息必定高”(第 330 页),\end{quote}

因为代表只供享用的财富的人出于无聊,追求享乐,而另一方面,除了农业以外,生产非常有限。一旦商业发展起来,情况就相反。商人完全被获利的欲望支配。他

\begin{quote}“\textbf{除了看到他的财产一天天增加以外,不知道还有什么更大的享乐}”。\end{quote}

(在这里,对交换价值、对抽象财富的追求大大超过对使用价值的追求。)

\begin{quote}“这就是为什么商业扩大节约,为什么在商人中守财奴大大超过挥霍者,而在土地所有者中情况则相反的原因。”(第 333 页)\end{quote}

\fontbox{~\{}\textbf{非生产劳动}:

\begin{quote}“律师和医生不产生任何生产活动;而且他们的财富是靠牺牲别人得来的;这样,他们使自己的财富增加多少,就一定使某些同胞的财富减少多少。”(第 333—334 页)\fontbox{\}~}“因而,商业的增长造成放债人数目的增加,因此\textbf{引起利息率的降低}。”(第 334 页)“\textbf{低利息}和\textbf{商业}中的\textbf{低利润},是彼此互相促进的两件事,\textbf{两者都来源于}商业的扩展,商业的扩展产生富商,使货币所有者增加。商人有了大笔的资本,不管这些资本是由少量的铸币还是由大量的铸币代表,都必然要常常发生这种情况:当他们倦于经商,或者他们的后代不喜欢或没有才干经商的时候,有很大一部分资本就自然地寻求一个常年的可靠的收入。供应多了就使价格降低,使放债人接受低利息。这种考虑迫使许多人宁愿把他们的资本留在商业中,满足于低利润,而不愿把他们的货币按更低的利息贷放出去。另一方面,当商业有了很大的扩展并且运用大量资本的时候,必然\textbf{产生商人之间的竞争},这种竞争使\textbf{商业利润减少},同时也使商业本身规模扩大。商业中的利润降低,使商人宁肯在离开商业,开始过清闲日子时接受低利息。因此,研究\textbf{低利息}和\textbf{低利润}这两种情况中,究竟哪一个是\textbf{原因},哪一个是\textbf{结果},是\textbf{没有用处}的。两者都是从大大扩展了的商业中产生的,并且彼此促进……大大扩展了的商业产生大量资本,因此,它既降低利息又降低利润;每当它降低利息的时候,总有利润的相应降低来促进它,反之也是一样。我可以补充说一句,正如\textbf{商业和工业的增长}引起低利润一样,低利润反过来又促使商业进一步增长,因为低利润使商品便宜,鼓励消费,促进工业的发展。由此可见……\textbf{利息}是\textbf{国家状况的真正的晴雨表,低利息率}是人民兴旺的几乎屡试不爽的标志。”(同上,第 334—336 页)\end{quote}

\tsubsectionnonum{[(c)马西。利息是利润的一部分。用利润率说明利息的高低]}

[\textbf{约·马西}]《论决定自然利息率的原因。对威廉·配第爵士和洛克先生关于这个问题的见解的考察》1750 年伦敦版。

\begin{quote}“从这些引文\endnote{在这段话前面,马西引用了配第的《政治算术》和洛克的《论降低利息和提高货币价值的后果》两书的摘要。——第 402 页。}中可以看到,洛克先生认为,自然\textbf{利息率}一方面决定于一国货币量同一国居民相互间的债务之比,另一方面决定于一国货币量同一国商业之比,威廉·配第爵士则认为,自然利息率只决定于一国货币量,因此,他们只在债务这一点上有不同意见。”(第 14—15 页)[XX—1294a][XXI—1300]富人“不是自己使用自己的货币,而是把自己的货币借给别人去营利,让别人把这样\textbf{得来的利润拿出一部分}交给货币所有者。但是,如果一国的财富平均分配给许多人,以致国内很少有人能够靠把货币投入商业的办法来供养两个家庭,那末,就\textbf{只能有很少的货币借贷了}:如果 2000 镑属于一个人,它就会被贷出,因为它带来的利息足以供养一个家庭;如果 2000 镑属于 10 个人,它就不会被贷出,因为它的利息不能供养 10 个家庭”。(第 23—24 页)“\textbf{根据政府为所借货币支付的利息率}来推断自然利息率的任何尝试,都是必然要失败的。经验表明,这两种利息率彼此既不一致,又不保持一定的关系;理性告诉我们,它们决不可能是这样,因为\textbf{自然利息率是以利润为基础,而国债的利息率是以需要为基础},利润有界限,而需要没有界限。借货币去改良自己土地的贵族,借货币去经营企业的商人或工业家,都有他们不能超越的一定界限:如果他们用借来的货币能赚得 10\%的利润,他们可以为所借货币付给放债人 5\%;但是他们不会付给 10\%;相反,如果谁由于有迫切需要而借债,那就一切只取决于他的需要的程度,而需要是不承认任何戒律的。”(第 31—32 页)“收取利息的合理性,不取决于借债人是否赚得\textbf{利润},而取决于这些货币如果正确地加以使用,能够带来利润。”(第 49 页)“既然\textbf{借债人}为所借货币支付的利息,是\CJKunderdot{\textbf{所借货币能够带来的利润的一部分}},那末,这个\textbf{利息}总是要由这个\textbf{利润}决定。”(第 49 页)“在这个利润中,多大一部分归借债人,多大一部分归放债人才算合理呢\fontbox{?}这一般地只有根据借贷双方的意见来决定。因为在这方面合理不合理,仅仅是大家同意的结果。”(第 49 页)“可是,这一条\textbf{利润分配}规则,并不是对每一个放债人和借债人都适用,而只是对放债人和借债人总的来说适用……特大的利润和特小的利润是对业务熟练和业务不熟练的报酬,这是\textbf{同放债人绝无关系的};因为他们既不会因业务不熟练而吃亏,也不会因业务熟练而得利。在这里,适用于\textbf{同一工商业部门各个人}的话,也适用于\textbf{各个不同的工商业部门}。”(第 50 页)“\textbf{自然利息率}是由\textbf{工商业企业的利润}决定的。”(第 51 页)\end{quote}

为什么英国现在的利息率是 4\%,而过去是 8\%\fontbox{?}因为那时候英国商人

\begin{quote}“赚得的利润比现在多一倍”。\end{quote}

为什么利息率在荷兰是 3\%,在法国、德国和葡萄牙是 5—6\%,在西印度和东印度是 9\%,在土耳其是 12\%\fontbox{?}

\begin{quote}“对于所有这些情况,只要总的答复一下就够了,就是说,这些国家的商业利润和我国的商业利润不同,并且如此不同,以致产生了上述各种不同的利息率。”(第 51 页)\end{quote}

但是,为什么利润会下降呢\fontbox{?}那是由于国外和国内的竞争:

\begin{quote}“由于对外贸易〈因国外竞争〉减少,或者由于\textbf{商人彼此竞相压低自己商品的价格}……因为他们有必要把东西卖掉,或者因为他们利欲熏心想尽量多卖一些”。(第 52—53 页)“商业利润一般决定于\textbf{商人数目}同\textbf{商业规模}之比。”(第 55 页)在荷兰,“从事商业的人数在人口总数中占的比例最大……\textbf{利息最低}”;在土耳其,这种比例最小,利息最高。(第 55—56 页)[1301]“\textbf{商业规模同商人数目之比}是由什么决定的呢\fontbox{?}”(第 57 页)“由商业的动机决定”:由自然的必要性、自由、私人权利的保护、社会安全来决定。(第 58 页)“没有两个国家能够\textbf{以等量的劳动耗费},同样丰富地提供数目相等的必要生活资料。人的需要的增减取决于人生活在其中的气候的严寒或温暖;所以不同国家的居民必须经营的\textbf{商业的规模}不能不有所差别,只有根据冷热的程度才能知道这种差别的程度。由此可以得出一个一般的结论:维持一定人口生活所需要的\textbf{劳动量},在气候寒冷的地方最大,在气候炎热的地方最小,因为在寒冷的地方,不仅人需要较多的衣服,而且土地也必须耕作得更好。”(第 59 页)“荷兰具有发展商业的特殊必要性……这种必要性是由国内人口过剩引起的;这种情况,再加上\textbf{必须花费很多劳动去筑堤和排水},就使荷兰经营商业的必要性比世界上其他任何可以居住的地方都大。”(第 60 页)\end{quote}

\tsubsectionnonum{[(d)结束语]}

马西比休谟更加明确地说明利息只不过是\textbf{利润的一部分};他们两人都用资本积累(马西特别讲到竞争)和由此产生的利润下降,来说明利息的下降。两人同样很少谈到“\textbf{商业利润}”本身的\textbf{源泉}问题。[XXI—1301]

\tsectionnonum{[(8)对论重农学派的各章的补充]}

\tsubsectionnonum{[(a)对《经济表》的补充意见。魁奈的错误前提]}

[XXIII—1433]

生产阶级

这是《经济表》的最简单的形式。\endnote{马克思在这虽引用(并略加简化)的《经济表》是魁奈在《经济表的分析》(德尔出版的《重农学派》第 1 部第 65 页)中用的那种《经济表》图式。——第 405 页。}

(1)\textbf{货币流通}(假定每年只支付一次)。货币流通的出发点是花钱的阶级,即土地所有者阶级,他们没有任何\textbf{商品}要卖,他们只买不卖。

土地所有者用 10 亿向生产阶级购买,把生产阶级用来付地租的 10 亿货币还给生产阶级。(从而实现了农产品的 1/5。)他们用 10 亿向不生产阶级购买,于是 10 亿货币流到不生产阶级手里。(同时实现了工业品的 1/2。)不生产阶级用这 10 亿向生产阶级购买食物,于是又有 10 亿货币流回生产阶级手里。(从而实现了农产品的另一个 1/5。)生产阶级用这同一个 10 亿货币购买价值 10 亿的工业品,以此补偿他们的“预付”的半数。(同时实现了工业品的另一个 1/2。)不生产阶级[1434]用同一个 10 亿货币购买原料。(从而实现了农产品的又一个 1/5。)这样一来,20 亿货币流回生产阶级手里。

因而还剩下农产品的 2/5。1/5 以实物形式消费,但是第二个 1/5 以什么形式积累起来呢\fontbox{?}这个问题到后面再研究。\endnote{马克思在这里和在后面都采用魁奈的说法:只有 1/5 的农业总产品不进入流通,而由生产阶级以实物形式享用。马克思在手稿第 XXIII 本第 1433—1434 页(见本册第 405—406 页)和他写的《反杜林论》第二编第十章中又回过头来谈这个问题。他在这一章中对魁奈关于农业中流动资本的补偿的观点作了如下详细说明:“价值五十亿的全部总产品因而掌握在生产阶级的手中,也就是说,首先是掌握在租地农场主的手中,这些租地农场主每年花费二十亿经营资本(与一百亿基本投资相适应)来生产全部总产品。为了补偿经营资本,因而也为了维持一切直接从事农业的人所需要的农产品、生活资料、原料等等,是以实物形式从总收成中拿出来的,并且花费在新的农业生产上。因为,正如前面所说,是以一次规定了的标准的固定价格和简单再生产为前提,所以总产品中预先拿出去的部分的货币价值,等于二十亿利弗尔。因此,这一部分没有进入一般的流通,因为正如已经指出的,任何发生于每一个别阶级的范围之内而不是发生于各阶级相互之间的流通,都没有列入表内。”(《马克思恩格斯全集》中文版第 20 卷第 270—271 页)因此,按照魁奈的说法,应当说租地农场主以实物形式补偿他们的流动资本的那部分产品,占他们的全部总产品的 2/5。——第 352、406 页。}

(2)即使从魁奈本人的观点出发(按照他的观点,整个不生产阶级实际上只不过是雇佣劳动者),也已经可以看出,《经济表》的前提是错误的。

这里假定在生产阶级那里,“原预付”(固定资本)是“年预付”数额的 5 倍。在不生产阶级那里,这一项根本没有提及,这当然并不妨碍它的存在。

此外,说再生产等于 50 亿,是错误的。从《经济表》本身来看,再生产等于 70 亿:生产阶级方面 50 亿,不生产阶级方面 20 亿。

\tsubsectionnonum{[(b)个别重农主义者局部地回到重商主义的观点。重农主义者要求竞争自由]}

不生产阶级的产品等于 20 亿。这个产品是由 10 亿原料(这些原料一部分加入产品,一部分补偿加入产品价值的机器的损耗)和在原料加工期间被消费了的 10 亿食物组成的。

不生产阶级把这全部产品卖给土地所有者阶级和生产阶级,以便\textbf{第一},补偿“预付”(以原料形式),\textbf{第二},取得农产生活资料。这样,不生产阶级就\textbf{丝毫没有}留下\textbf{一点工业品}供他们自己消费,更不用说利息和利润了。勃多(或列特隆)看到了这一点,他这样来解释:不生产阶级\textbf{高于}产品的\textbf{价值}出卖他们的产品,因而他们卖 20 亿的东西等于 20 亿减去 x。因此,利润,甚至这个阶级\textbf{本身}所消费的、属于它所必需的生活资料的工业品,按照上面的解释,就只被归结为这个阶级\textbf{把自己商品的价格抬得高于它们的价值}\endnote{重农主义者勃多在他的《经济表说明》第三章第十二节(德尔出版的《重农学派》第 2 部第 852—854 页)发挥了这一观点。——第 407 页。}。可见,重农学派在这里必然回到重商主义体系,回到“\textbf{让渡利润}”的概念。

因此,他们也认为,工业家之间的自由竞争是完全必要的,这样可以使工业家不致过分欺骗生产阶级即农业家。另一方面,这种自由竞争之所以必要,是因为这样就可以使农产品卖得一个“\textbf{好}价钱”,就是说,通过输出国外把农产品的价格抬得\textbf{高于}它原来的本国价格,因为这里假定的是一个出口小麦等等的国家。

\tsubsectionnonum{[(c)关于价值不可能在交换中增殖的最初提法]}

\begin{quote}“每次买都是卖,每次卖都是买。”(\textbf{魁奈}《关于商业和手工业者劳动的问答》,德尔出版,\endnote{德尔出版的《重农学派》第一部在这个标题下把魁奈的两篇问答《关于商业。H 先生和 N 先生的第一次问答》和《关于手工业者劳动。第二次问答》合在一起。马克思的引文取自第一次问答。——第 407 页。}第 170 页)“买就是卖,卖就是买。”(\textbf{魁奈},见\textbf{杜邦·德·奈穆尔}《论近代科学的起源和进步》第 392 页)\endnote{马克思所引的这句话不在杜邦·德·奈穆尔的著作《论近代科学的起源和进步》本文内,而在内容同该著作衔接的《魁奈医生的学说,或他的社会经济学原理概述》一文内。——第 407 页。}“\textbf{价格总是先于买卖}。如果卖者和买者的竞争没有引起任何变化,价格就仍然是由同商业\textbf{无关的}其他原因所确定的那个价格。”(第 148 页)\endnote{引文取自魁奈的《关于商业的问答》。——第 407 页。}“始终可以假定,它〈交换〉对于双方〈当事人〉都是有利的,因为双方都保证自己有可能享受他们只有通过交换才能得到的财富。但是这里所讲的,始终只是具有\textbf{一定价值}的财富同具有\textbf{同一价值}的另一财富交换,因而,\textbf{不可能有财富的任何实际的增加}〈应该说:不可能有价值的任何实际的增加〉。”(同上,第 197 页)\endnote{取自《关于手工业者劳动的问答》。——第 407 页。}\end{quote}

明确地把“\textbf{预付}”和“\textbf{资本}”等同起来。把\textbf{资本积累}作为主要条件。

\begin{quote}“因此,\textbf{增加资本是增加劳动的主要手段},这会给\textbf{社会}带来\textbf{最大的好处}”等等。(\textbf{魁奈},见\textbf{杜邦·德·奈穆尔},同上第 391 页)\endnote{取自《魁奈医生的学说》。——第 407 页。}[XXIII—1434]\end{quote}

\tsectionnonum{[(9)重农学派的追随者毕阿伯爵对土地贵族的赞美]}

[XXII—1399]\textbf{毕阿(伯爵)}《政治要素,或社会经济真正原则的研究》(六卷集)1773 年伦敦版。

这个低能的废话连篇的著作家,把重农主义的外观看成重农主义的实质,竭力赞扬土地贵族,事实上,只有当重农主义符合这个目的时,他才接受重农主义。要不是他的著作中有象后来李嘉图的著作中那样露骨地表现出来的粗俗的资产阶级性质,根本就不会提到他。认为“纯产品”只限于地租的错误看法,不会使问题有丝毫改变。

毕阿伯爵所说的东西,就是李嘉图后来对一般“纯产品”所重复提到的东西\endnote{马克思指李嘉图的《政治经济学和赋税原理》第二十六章(《论总收入与纯收入》)。——第 408、438 页。}。工人属于非生产费用\authornote{见第 159 页脚注。——编者注},他们之所以存在,只是为了使“纯产品”所有者得以“组成社会”(见有关的地方)\endnote{马克思指他在补充本 A(见注 122)第 27—32 页所作的毕阿伯爵著作摘录。在后面正文所用的引文中,马克思注明的不是补充本的页码,而是毕阿伯爵著作的页码。——第 408 页。}。自由工人的地位被他看成只不过是奴隶制的改变了的形式,然而在他看来,这种改变了的形式对于上层组成“社会”来说是必要的。\fontbox{~\{}连\textbf{阿瑟·杨格}也把“纯产品”,即剩余价值,说成生产的目的。\endnote{关于“剩余产品的狂热的崇拜者”杨格,见《资本论》第 1 卷第 7 章注 34。——第 408 页。}\fontbox{\}~}

[1400]由此可以使人想起李嘉图同斯密争论的一段话,\endnote{马克思指李嘉图的《政治经济学和赋税原理》第二十六章(《论总收入与纯收入》)。——第 408、438 页。}他不同意斯密把使用工人最多的资本看成生产能力最大的资本。参看毕阿的著作第 6 卷第 51—52、68—70 页;其次,关于工人阶级和奴隶制,参看第 2 卷第 288、297、309 页;第 3 卷第 74、95—96、103 页;第 6 卷第 43、51 页;关于这些工人被迫进行剩余劳动,以及什么叫做“最必要的生存资料”,参看第 6 卷第 52—53 页。

我们在这里只引一段话,因为这段话对于所谓资本家总是冒\textbf{风险}的空谈,作了很好的反驳:

\begin{quote}“据说他们〈商人〉为了多赚钱而冒很多风险。不过,他们或者拿人去冒险,或者拿商品和货币去冒险。如果他们为了发财而让别人陷于明显的危险境地,那他们就是干了极坏的事情。至于谈到商品,一个人把商品生产出来,是有功绩的;但是,为了一个人的发财致富而拿这些商品去冒险,就不可能是什么功绩了”等等。(第 2 卷第 297 页)[XXII—1400]\end{quote}

\tsectionnonum{[(10)从重农学派的观点出发反驳土地贵族(英国的一个匿名作者)]}

[XXIII—1449]《国民财富基本原理的说明。驳亚当·斯密博士等人的某些错误论点》1797 年伦敦版。\endnote{后来查明,马克思在这里分析的匿名著作的作者是一个叫约翰·格雷(JohnGray)的人,此人生卒年月不详。1802 年这位作者在伦敦还发表了一部关于所得税的著作。——第 410 页。}

这本书的作者知道安德森的著作,因为他在该书的附录中,转载了安德森关于阿贝丁郡的农业报告的片段。

这是英国的一本可直接算在重农主义学说内的\textbf{唯一重要}著作。\textbf{威廉·斯宾斯}的《不列颠不依靠商业》一书(1807 年版)只不过是一幅讽刺画。这个斯宾斯在 1814—1815 年间,是土地所有者的最狂热的维护者之一,他根据主张……贸易自由的重农主义学说来维护土地所有者的利益。不要把这个家伙同\textbf{土地私有制}的死敌\textbf{托马斯·斯宾斯}混淆起来。

《基本原理》一书首先包含着对重农主义学说的卓越而简洁的概括。

作者正确地指出重农主义的观点来源于\textbf{洛克}和\textbf{范德林特}的观点。他把重农学派说成是这样的著作家,他们

\begin{quote}“\textbf{虽不是完全正确地}但很有系统地阐明了”自己的学说。(第 4 页)这点还可参看第 6 页(摘录在\textbf{稿本}H 第 32—33 页\endnote{马克思指他的补充本 H(见注 122)。后面正文引用的是补充本 H 第 32—33 页上对匿名著作第 6 页所作的几乎全部摘录。——第 410 页。})。\end{quote}

从匿名作者对重农主义学说的概括中,可以十分明显地看出,被后来的辩护论者——斯密就已经部分地这样做了——当作资本形成的基础的\textbf{节欲论},是直接从重农学派的这样一个见解产生的:工业等等不创造\textbf{任何剩余价值}。

\begin{quote}“用于使用和维持手工业者、制造业者\endnote{这位英国匿名作者所说的“制造业者”是指制造业工人(他有时把他们叫做“劳动的制造业者”)和工业家-企业主(有时他称他们为“企业老板”)。而“手工业者”一词,这位作者是指雇佣工人和本来意义上的手工业者。——第 411 页。}和商人的费用,结果只能\textbf{保持支出的数额的价值},因而是非生产的〈因为它不生产剩余价值〉。除非手工业者、制造业者和商人\textbf{从本来供他们维持每日生活的东西中节约和积累下来一部分},否则社会的财富靠他们不可能得到丝毫\textbf{增加}。可见,他们\textbf{只有通过节欲和节约}〈西尼耳的节欲论和亚当·斯密的节约论〉才能使总资本有所增加。相反,土地耕种者能够消费自己的全部收入,同时又使国家致富;因为他们的活动会提供叫做地租的剩余产品。”(第 6 页)“有一个阶级,他们的劳动虽然也生产一些东西,但所生产的并不比维持他们的劳动所花费的多,理所当然可以把他们叫做\textbf{非生产阶级}。”(第 10 页)\end{quote}

\textbf{应当把剩余价值的生产同剩余价值的“转手”严格区分开来}。

\begin{quote}“收入的\textbf{增加}〈即\textbf{积累}〉只间接地是经济学家\endnote{“经济学家”是十八世纪下半叶和十九世纪上半叶在法国对重农学派的称呼。——第 38、139、223、411 页。}的研究对象……他们的研究对象是\textbf{收入的生产和再生产}。”(第 18 页)\end{quote}

这正是重农主义的巨大功绩。重农学派给自己提出的问题是,\textbf{剩余价值}(匿名作者把剩余价值叫做“收入”)是怎样生产和再生产出来的。关于剩余价值怎样\textbf{以更大的规模再生产出来},即剩余价值怎样增加的问题,只是属于第二位的问题。首先必须揭示剩余价值的\textbf{范畴},[1450]揭示剩余价值生产的秘密。

\textbf{剩余价值和商业资本}:

\begin{quote}“在谈到收入的\textbf{生产}时,用\textbf{收入的转手}这个问题来替换,是完全不合逻辑的,只有\textbf{一切商业交易}才归结为收入的转手。”(第 22 页)“\textbf{商业}这个词的意思不过是指\textbf{商品的交换}……有时,这种交换对一方比对另一方更有利;但一个人的赢利,总是另一个人的亏损,所以他们之间的商业交易实际上\textbf{不会造成财富的任何增加}。”(第 23 页)“如果一个犹太人把 1 克朗卖了 10 先令,或者把安女王时代的 1 法寻卖了 1 基尼,\endnote{1 克朗是 5 先令的铸币,1 法寻是 1/4 便士,1 基尼等于 21 先令。——第 411 页。}那他毫无疑问会增加自己的收入,但他并不会因此而增加\textbf{现有的贵金属量};而且,无论喜爱古玩的买者是同旧币的卖者住在一条街上,还是住在法国或中国,这种商业交易的性质始终是一样的。”(第 23 页)\end{quote}

\textbf{在重农学派的著作中,工业利润被看成“让渡利润”}(即按重商主义来解释)。\textbf{因此,这个英国人作出正确的结论说,只有当工业品在国外出卖时,这种利润才是真正的利润。他从重商主义的前提出发作出正确的重商主义的结论。}

\begin{quote}“任何一个制造业者,如果他的商品是在国内出售和消费,那末,无论他自己获得多少赢利,也不会使国民收入增加分毫;因为\textbf{买者的亏损……同制造业者的赢利正好一样多}……这里是卖者和买者之间的\textbf{交换},而不是财富的增加。”(第 26 页)“为了\textbf{弥补盈余的缺乏}……企业主从自己支付的工资中提取 50%的利润,或者说,从他们支付给制造业工人的每 1 先令中提取 6 便士……如果商品在国外出卖”,那末这就是若干数量的“手艺人”所提供的“\textbf{国民利润}”。(第 27 页)\end{quote}

\textbf{作者很好地说明了荷兰财富的原因}。渔业(还应当指出畜牧业)。对东方香料的垄断。海运业。向外国人贷款(补充本 H 第 36—37 页)\endnote{在补充本 H 第 36—37 页是对匿名著作第 31—33 页所作的摘录。——第 412 页。}。

这位作者写道,制造业者“是一个\textbf{必要的}阶级”,但他们不是“\textbf{生产阶级}”。(同上,第 35 页)他们“只是使土地耕种者早已取得的收入\textbf{替换}或\textbf{转手},而他们采取的办法是,使这种收入在一种新形式上具有\textbf{耐久性}”。(第 38 页)

只有四个必要的阶级:(1)生产阶级或土地耕种者;(2)制造业者;(3)国家保卫者;(4)“教师阶级”,他用教师来代替重农学派所说的“什一税所得者”即牧师。

\begin{quote}“因为任何市民社会都需要吃饭、穿衣、保卫和教育。”(第 50—51 页)“经济学家”的错误在于,“他们把\textbf{作为单纯的租金所得者的地租所得者}看成社会的\textbf{生产阶级}……他们暗示,教会和国王必定要靠土地所有者获得的地租来维持生活,因而在某种程度上改正了自己的错误。斯密博士……让它〈“经济学家”的上述错误〉贯穿\textbf{他的全部著作}〈这是对的〉,他的批判正好针对着经济学家体系的正确部分”。(第 8 页)\end{quote}

[1451]土地所有者本身不仅不是\textbf{生产}阶级,甚至不是社会的\textbf{必要阶级}:

\begin{quote}“\textbf{土地所有者}作为单纯的地租所得者,\textbf{并不是社会的必要阶级}……\textbf{只要地租脱离宪法所规定的目的——为保卫国家服务},这种地租的所得者就不再是必要阶级,而成为社会上最不需要的、最麻烦的阶级之一。”(第 51 页)关于这一点的进一步内容,见补充本 H 第 38—39 页\endnote{在补充本 H 第 38—39 页是对匿名著作第 51—54 页所作的摘录。在后面正文所用的引文中,马克思注明的不是补充本 H 的页码,而是匿名著作的页码。——第 413 页。}。\end{quote}

所有这些都很好,这种从重农学派观点出发对地租所得者的反驳,\textbf{作为重农学派学说的完成是很重要的}。

作者指出,真正的\textbf{土地税}是土耳其人所特有的。(同上,第 59 页)

\textbf{土地所有者}不仅对现有的“土地改良”\textbf{收税},而且往往对“推测中的将来的改良”也收税。(第 63—64 页)地租税。(第 65 页)

在税收方面,重农主义理论在英格兰、爱尔兰、封建的欧洲、莫卧儿帝国\textbf{早就}实现了。(第 93—94 页)

土地所有者是收税人。(第 118 页)

\textbf{重农主义的局限性表现在下述看法上}(对\textbf{分工}缺乏\textbf{理解}):

\begin{quote}假设一个钟表业者或棉布厂主不能把他的钟表或棉布卖掉;他就陷入困难的境地。这表明,“制造业者只有成为\textbf{卖者}才能发财致富\end{quote}

(实际上,这只是表明,他把自己的产品作为\textbf{商品}生产出来),

\begin{quote}一旦他不再成为\textbf{卖者},他的\textbf{利润}也就立即终止\end{quote}

(而本身不是\textbf{卖者}的租地农场主的利润又是怎么一回事呢\fontbox{?}),

\begin{quote}因为这些利润不是自然的而是人为的利润。土地耕种者……不\textbf{出卖}任何东西就\textbf{能生存}、兴旺和增加自己的财富”(第 38—39 页)\end{quote}

(但在这种场合,他必须同时又是制造业者)。

为什么作者只谈钟表业者或棉布厂主呢\fontbox{?}同样,也可以假设煤炭、铁、亚麻、靛蓝等等的生产者不能把这些产品卖掉,或者连小麦的生产者也不能把自己的小麦卖掉。关于这一点,前面提到过的贝阿尔岱·德·拉贝伊讲得很好。\endnote{马克思在他的 1861—1863 年手稿第 1446 页(第 XXIII 本)提到了贝阿尔岱·德·拉贝伊的旨在反对重农学派的著作《关于取消税收办法的研究》1770 年阿姆斯特丹版。这一著作的摘录在补充本 H 第 10—11 页。马克思指的贝阿尔岱·德·拉贝伊的那段话在该书第 43 页。——第 414 页。}匿名作者不得不提出以\textbf{直接}消费为目的的生产来反对\textbf{商品生产},这是同下面的情况非常矛盾的:对于重农学派来说,最主要的问题倒是\textbf{交换价值}。然而后面这点也贯穿在我们所说的这个人的著作中,这是囿于资产阶级前的\textbf{思考方式}的一种对事物的\textbf{资产阶级}见解。\endnote{在最后几段论述匿名作者的“重农主义局限性”的俄译文中,对马克思在引用所分析的著作的文字(引自该著作第 38—39 页)中加入的某些插话在编排上略加改动。马克思对引文作了删节。本版按原作恢复了删去的文字。——第 414 页。}

这位匿名作者反对阿瑟·杨格认为\textbf{高价格对农业繁荣很重要}的看法;\textbf{但是这样反对杨格同时也就是反驳重农主义}。(同上,第 65—78 页和第 118 页)

\textbf{由卖者在名义上提高价格不能得出剩余价值}:

\begin{quote}“靠提高\textbf{产品的名义价值……卖者不会致富}……因为他们作为卖者所得的利益,在他们作为买者时又如数付出。”(第 66 页)\end{quote}

下面这段话是按\textbf{范德林特}的精神写的:

\begin{quote}“只要能为每个失业者找到一块可耕的土地,任何一个失业者就都不会没有土地了。劳动的房屋是好东西;但劳动的田地更好得多。”(第 47 页)\end{quote}

匿名作者反对一切租佃制,不过他认为长期租佃比短期租佃好,因为如果实行短期租佃,土地所有权只会妨碍生产和阻碍土地改良。(第 118—123 页)(\textbf{爱尔兰的租佃权}。)\endnote{关于“爱尔兰的租佃权”,见马克思发表在 1853 年 7 月 11 日《纽约每日论坛报》上的文章(《马克思恩格斯全集》中文版第 9 卷第 177—183 页)。——第 414 页。}[XXIII—1451]

\tsectionnonum{[(11)关于一切职业都具有生产性的辩护论见解]}

[V—182]哲学家生产观念,诗人生产诗,牧师生产说教,教授生产讲授提纲,等等。罪犯生产罪行。如果我们仔细考察一下最后这个生产部门同整个社会的联系,那就可以摆脱许多偏见。罪犯不仅生产罪行,而且还生产刑法,因而还生产讲授刑法的教授,以及这个教授用来把自己的讲课作为“商品”投到一般商品市场上去的必不可少的讲授提纲。据说这就会使国民财富增加,更不用说象权威证人罗雪尔教授先生所说的,这种讲授提纲的手稿给作者本人带来的个人快乐了。

其次,罪犯生产全体警察和全部刑事司法、侦探、法官、刽子手、陪审官等等,而在所有这些不同职业中,每一种职业都是社会分工中的一定部门,这些不同职业发展着不同的人类精神能力,创造新的需要和满足新需要的新方式。单是刑讯一项就推动了最巧妙的机械的发明,并保证使大量从事刑具生产的可敬的手工业者有工可做。

罪犯生产印象,有时是道德上有教益的印象,有时是悲惨的印象,看情况而定;而且在唤起公众的道德感和审美感这个意义上说也提供一种“服务”。他不仅生产刑法讲授提纲,不仅生产刑法典,因而不仅生产这方面的立法者,而且还生产艺术、文艺——小说,甚至悲剧;不仅缪尔纳的《罪》和席勒的《强盗》,而且《奥狄浦斯王》和《理查三世》都证明了这一点。罪犯打破了资产阶级生活的单调和日常的太平景况。这样,他就防止了资产阶级生活的停滞,造成了令人不安的紧张和动荡,而没有这些东西,连竞争的刺激都会减弱。因此,他就推动了生产力。一方面,犯罪使劳动市场去掉了一部分过剩人口,从而减少了工人之间的竞争,在一定程度上阻止工资降到某种最低额以下;另一方面,反对犯罪的斗争又会吸收另一部分过剩人口。这样一来,罪犯成了一种自然“平衡器”,它可以建立适当的水平并为一系列“有用”职业开辟场所。

罪犯对生产力的发展的影响,可以研究得很细致。如果没有小偷,锁是否能达到今天的完善程度\fontbox{?}如果没有[183]伪造钞票的人,银行券的印制是否能象现在这样完善\fontbox{?}如果商业中没有欺骗,显微镜是否会应用于通常的商业领域(见拜比吉的书)\fontbox{?}应用化学不是也应当把自己取得的成就,象归功于诚实生产者的热情那样,归功于商品的伪造和为发现这种伪造所作的努力吗\fontbox{?}犯罪使侵夺财产的手段不断翻新,从而也使保护财产的手段日益更新,这就象罢工推动机器的发明一样,促进了生产。而且,离开私人犯罪的领域来说,如果没有国家的犯罪,能不能产生世界市场\fontbox{?}如果没有国家的犯罪,能不能产生民族本身\fontbox{?}难道从亚当的时候起,罪恶树不同时就是知善恶树吗\fontbox{?}

孟德维尔在他的《蜜蜂的寓言》(1705 年版)中,已经证明任何一种职业都具有生产性等等,在他的书中,已经可以看到这全部议论的一般倾向:

\begin{quote}“我们在这个世界上称之为恶的东西,不论道德上的恶,还是身体上的恶,都是使我们成为社会生物的伟大原则,是毫无例外的\textbf{一切职业和事业}的牢固基础、\textbf{生命力和支柱};我们应当在这里寻找一切艺术和科学的真正源泉;一旦不再有恶,社会即使不完全毁灭,也一定要衰落。”\endnote{[贝·孟德维尔]《蜜蜂的寓言,或个人劣行即公共利益》1728 年伦敦第 5 版第 428 页。该书第一版于 1705 年出版。——第 417 页。}\end{quote}

当然,只有孟德维尔才比充满庸人精神的资产阶级社会的辩护论者勇敢得多、诚实得多。[V—183]

\tsectionnonum{[(12)]资本的生产性。生产劳动和非生产劳动}

\tsubsectionnonum{[(a)资本的生产力是社会劳动生产力的资本主义表现]}

[XXI—1317]我们不仅看到了资本是怎样进行生产的,而且看到了资本本身是怎样被生产出来的,资本作为一种发生了本质变化的关系,是怎样从生产过程中产生并在生产过程中发展起来的。\endnote{马克思指同《资本的生产性。生产劳动和非生产劳动》这一节紧接的前一节《劳动对资本的形式上的隶属和实际上的隶属。过渡形式》(手稿第 XXI 本,第 1306—1316 页)。关于劳动对资本的形式上的隶属和实际上的隶属的问题,见马克思《资本论》第 1 卷第 14 章和第 24 章第 3 节。——第 418 页。}一方面,资本改变着生产方式的形态,另一方面,生产方式的这种被改变了的形态和物质生产力的这种特殊发展阶段,是资本本身的基础和条件,是资本本身形成的前提。

因为活劳动——由于资本同工人之间的交换——被并入资本,从劳动过程一开始就作为属于资本的活动出现,所以社会劳动的一切生产力都表现为资本的生产力,就和劳动的一般社会形式在货币上表现为一种物的属性的情况完全一样。同样,现在社会劳动的生产力和社会劳动的特殊形式,表现为资本的生产力和形式,即\textbf{物化}劳动的,劳动的物的条件(它们作为这种独立的要素,人格化为资本家,同活劳动相对立)的生产力和形式。这里,我们又遇到关系的颠倒,我们在考察货币时,已经把这种关系颠倒的表现称为\textbf{拜物教}。\endnote{马克思在《政治经济学批判》第一分册(1859 年)中就已指出,在资产阶级社会中,社会关系的神秘化在货币上表现得特别显著,财富结晶为贵金属形式的拜物教是资产阶级生产所固有的(见《马克思恩格斯全集》中文版第 13 卷第 37—39 和 144—146 页)。马克思在《剩余价值理论》第三册补充部分《收入及其源泉。庸俗政治经济学》(手稿第 891—899 和 910—919 页)中对资产阶级关系的拜物教化过程作了分析。——第 418 页。}

资本家本身只有作为\textbf{资本的人格化}才是统治者。(在意大利式簿记中,他作为\textbf{资本家},作为人格化资本的这一作用,总是同他作为单纯的个人相对立,而他作为单纯的个人就是仅仅作为私人消费者,作为他自己的资本的债务人出现。)

资本的\textbf{生产性}(即使仅仅考察劳动对资本的\textbf{形式上的}隶属),首先在于\textbf{强迫进行剩余劳动},强迫进行超过直接需要的劳动。这种强迫,是资本主义生产方式和以前的生产方式所共有的,但是,资本主义生产方式是以更加有利于生产的方式实行并采用这种强迫的。

即使考察这种纯粹形式上的关系,考察资本主义生产的较不发达阶段和较为发达阶段所共有的\textbf{一般}形式,\textbf{生产资料},劳动的物的条件——劳动材料、劳动资料(以及生活资料)——也不是从属于工人,相反,是工人从属于它们。不是工人使用它们,而是它们使用工人。正因为这样,它们才是资本。“资本\textbf{使用}劳动。”对工人来说,它们不是生产产品的手段,不论这些产品采取直接生存资料的形式,还是采取交换手段,商品的形式。相反,工人对它们来说倒是一个手段,它们依靠这个手段,一方面保存自己的价值,另方面使自己的价值转化为资本,也就是说,吸收剩余劳动,使自己的价值增殖。

这种关系在它的简单形式中就已经是一种颠倒,是物的人格化和人的物化;因为这个形式和以前一切形式不同的地方就在于,资本家不是作为这种或那种个人属性的体现者来统治工人,他只在他是“资本”的范围内统治工人;他的统治只不过是物化劳动对活劳动的统治,工人制造的产品对工人本身的统治。

但是,这种关系所以变得更加复杂,显得更加神秘,是因为随着特殊的资本主义生产方式的发展,不仅这些直接物质的东西\fontbox{~\{}它们都是劳动产品;从使用价值来看,它们是劳动产品,又是劳动的物的条件;从交换价值来看,它们是物化的一般劳动时间或货币\fontbox{\}~}起来反对工人,作为“资本”同工人相对立,就连社会地发展了的劳动的形式——协作、工场手工业(作为分工的形式)、工厂(作为以机器体系为自己的物质基础的社会劳动形式)——都表现为\textbf{资本的发展形式},因此,从这些社会劳动形式发展起来的劳动生产力,从而还有科学和自然力,也表现为\textbf{资本的生产力}。事实上,协作中同种劳动的统一,分工中异种劳动的结合,机器工业中自然力、科学和劳动产品的用于生产,所有这一切,都作为某种\textbf{异己的、物的}东西,纯粹作为不依赖于工人而支配着工人的劳动资料的存在形式,同单个工人相对立,正如劳动资料本身在它们作为材料、工具等简单可见的形式上,作为\textbf{资本}的职能,因而作为\textbf{资本家}的职能,同单个工人相对立一样。

工人自己的劳动的社会形式,或者说,工人自己的[1318]社会劳动的形式,是完全不以单个工人为转移而形成的关系;工人从属于资本,变成这些社会构成的要素,但是这些社会构成并不属于工人。因而,这些社会构成,作为资本本身的\textbf{形态},作为不同于每个工人的单个劳动能力的、属于资本的、从资本中产生并被并入资本的结合,同工人相对立。并且这一点随着下述情况的发展越来越具有实在的形式。这些情况是:一方面,工人的劳动能力本身由于上述社会形式而发生了形态变化,以致它在独立存在时,也就是说,\textbf{处在}这种资本主义联系\textbf{之外}时,就变得无能为力,它的独立的生产能力被破坏了;另一方面,随着机器生产的发展,劳动条件在工艺方面也表现为统治劳动的力量,同时又代替劳动,压迫劳动,使独立形式的劳动成为多余的东西。

工人的劳动的\textbf{社会}性质作为从某种意义上说\textbf{资本化的}东西同工人相对立(例如,在机器生产部门,劳动的可见产品表现为劳动的统治者),在这个过程中,各种自然力和科学——历史发展总过程的产物,它抽象地表现了这一发展总过程的精华——自然也发生同样的情况:它们作为资本的\textbf{力量}同工人相对立。科学及其应用,事实上同单个工人的技能和知识分离了,虽然它们——从它们的源泉来看——又是劳动的产品,然而在它们进入劳动过程的一切地方,它们都表现为\textbf{被并入资本的东西}。使用机器的资本家不必懂得机器(见尤尔的著作)。\endnote{马克思在《资本论》第一卷第十三章注 108 中写道:“科学不费资本家‘分文’,但这丝毫不妨碍他们去利用科学。资本象吞并别人的劳动一样,吞并‘别人的’科学。但是,科学或物质财富的‘资本主义的’占有和‘个人的’占有,是截然不同的两件事。尤尔博士本人曾哀叹他的亲爱的、使用机器的工厂主对力学一窍不通……”——第 421 页。}但是,\textbf{在机器上}实现了的科学,作为\textbf{资本}同工人相对立。而事实上,以\textbf{社会劳动}为基础的所有这些对科学、自然力和大量劳动产品的应用本身,只表现为\textbf{剥削}劳动的\textbf{手段},表现为占有剩余劳动的手段,因而,表现为属于资本而同劳动对立的\textbf{力量}。资本使用这一切手段,当然只是为了剥削劳动,但是为了剥削劳动,资本必然要在生产过程中使用这些手段。所以,劳动的\textbf{社会}生产力的发展和这个发展的条件就表现为\textbf{资本的行为},这种行为不仅是不管单个工人的意志如何而完成的,而且是直接反对单个工人的。

因为资本是由商品组成的,所以资本本身具有二重性:

(1)\textbf{交换价值}(货币);但是,它是\textbf{自行增殖的价值},是——因为它是\textbf{价值}——创造价值、\textbf{作为价值而增殖}、取得一个增殖额的价值。这种价值增殖归结为一定量物化劳动同较大量活劳动的交换。

(2)\textbf{使用价值};这里,资本是按照它在劳动过程中所具有的一定关系出现的。但是,正是在这里,资本不仅仅是\textbf{劳动}所归属的、把劳动并入自身的劳动材料和劳动资料:资本还把劳动的\textbf{社会结合}以及与这些社会结合相适应的劳动资料的发展程度,连同劳动一起并入它自身。资本主义生产第一次大规模地发展了劳动过程的物的条件和主观条件,把这些条件同单个的独立的劳动者分割开来,但是资本是把这些条件作为统治\textbf{单个工人}的、对单个工人来说是\textbf{异己的}力量来发展的。

这一切使资本变成一种非常神秘的存在。[1318]\endnote{马克思把 1861—1863 年手稿第 1318 页(除了最后 9 行)从第 XXI 本剪下来贴到《资本论》第一卷倒数第二稿第 490 页(这个倒数第二稿的第六章载于《马克思恩格斯文库》1933 年版第 2(7)卷)。后面第 1318 页、第 1319 页和第 1320 页前半页的正文,马克思在手稿页边(第 1318 页末尾和第 1320 页开头)曾两次注上“利润”字样,显然打算把它用在论利润的一节。——第 422 页。}

\centerbox{※     ※     ※}

[1320]因此,资本(1)作为\textbf{强迫}进行剩余劳动的力量,(2)作为吸收和占有社会劳动生产力和一般社会生产力(如科学)的力量(作为这些生产力的人格化),它是生产的。

试问:既然劳动的生产力已经转给了资本,而同一生产力不能计算两次,一次作为劳动的生产力,另一次作为资本的生产力,那末,同资本相对立的劳动,怎样或者说为什么表现为生产的,表现为\textbf{生产劳动}呢\fontbox{?}\fontbox{~\{}劳动的生产力就是资本的生产力。而\textbf{劳动能力}所以是生产的,是由于它的\textbf{价值}和\textbf{它创造的价值}之间有\textbf{差别}。\fontbox{\}~}

\tsubsectionnonum{[(b)资本主义生产体系中的生产劳动]}

只有把生产的资本主义形式当作生产的绝对形式、因而当作生产的永恒的自然形式的资产阶级狭隘眼界,才会把从资本的观点来看什么是生产劳动的问题,同一般说来哪一种劳动是生产的或什么是\textbf{生产劳动}的问题混为一谈,并且因此自作聪明地回答说,凡是生产某种东西、取得某种结果的劳动,都是生产劳动。

只\textbf{有直接转化为资本的}劳动,也就是说,只有使可变资本成为可变的量,因而使整个资本 C 等于 C+△\endnote{马克思在这里用数学上表示增量的希腊字母Δ代表剩余价值。马克思在后面正文中用拉丁字母 h 表示同一意义。——第 422 页。}的劳动,才是\textbf{生产的}。假定可变资本在同劳动交换之前等于 x,这样,我们得到等式 y=x,那末,把 x 变为 x+h、把等式 y=x 变为等式 y′=x+h 的那种劳动,是生产劳动。这是需要说明的\textbf{第一}点。这里谈的是创造剩余价值的劳动,或者说,是作为使资本能够形成剩余价值,因而能够表现为资本,表现为自行增殖的价值的因素来为资本服务的劳动。

\textbf{第二},劳动的社会的和一般的生产力,是资本的生产力;但是这种生产力只同劳动过程有关,或者说,只涉及使用价值。它表现为作为物的资本所固有的属性,表现为资本的使用价值。它不直接涉及\textbf{交换价值}。无论是 100 个工人一起劳动,还是他们各自单独劳动,他们所生产的产品的价值都等于 100 个工作日,不管这些工作日表现为许多产品或很少产品;换句话说,这些产品的价值不取决于劳动生产率。

[1321]劳动生产率的差别只在一个方面涉及交换价值。

举例来说,如果劳动生产率在某一个生产部门有了发展,例如用机器织机代替手工织机来生产布,已经不是例外的情况,用机器织机织 1 码布所需的劳动时间,只是用手工织机织 1 码布所需的劳动时间的一半,那末,一个手工织工的 12 小时就不再表现为 12 小时的价值,而只是表现为 6 小时的价值,因为\textbf{必要}劳动时间现在缩短为 6 小时了。手工织工虽然同以前一样劳动 12 小时,但他的 12 小时现在只等于 6 小时的社会劳动时间。

但是,这里谈的不是这一点。相反,如果我们拿另一个生产部门例如排字来看,在这里还没有使用机器,那末这个部门中的 12 小时创造的价值,同机器等等最发达的生产部门中的 12 小时创造的\textbf{价值}完全一样多。因此,作为\textbf{价值}的创造者,劳动总是\textbf{单个工人}的劳动,不过表现为\textbf{一般劳动}。因此,生产劳动,作为生产价值的劳动,总是作为单个劳动能力的劳动、\textbf{单个工人}的劳动同资本相对立,而不管这些工人在生产过程中参加什么样的社会结合。所以,同工人相对立的资本,代表劳动的社会生产力,而同资本相对立的工人的生产劳动,始终只代表\textbf{单个工人}的劳动。

\textbf{第}三,如果说榨取工人的剩余劳动和占有劳动的社会生产力,看来是资本的自然属性,因而看来是从资本的使用价值中产生的属性,那末,反过来说,把劳动自己的社会生产力表现为资本的生产力,把劳动生产的剩余产品表现为资本生产的剩余价值、资本的自行增殖,看来就是劳动的自然属性。

这三点现在要详细探讨一下,并从中得出生产劳动和非生产劳动的差别。

\textbf{关于第一点}。资本的生产性在于资本同作为雇佣劳动的劳动相对立,而劳动的生产性在于劳动同作为资本的劳动资料相对立。

我们已经看到,货币转化为资本,就是说,一定的交换价值转化为自行增殖的交换价值,转化为价值加剩余价值,是由于这个交换价值有一部分转化为在劳动过程中用作劳动资料(原料、工具,总之,劳动的物的条件)的商品,而另一部分则用于购买劳动能力。但是使货币转化为资本的,不是货币和劳动能力的最初交换,不是购买劳动能力这一事实本身。这种购买把被使用的劳动能力在一定时间内并入资本;换句话说,使一定量的活劳动成为资本本身的存在形式之一,可以说,成为资本本身的隐德来希\authornote{希腊文?ντελ?χεια的音译,古希腊哲学家亚里士多德的用语;他认为每一事物所要完成或达到的目的即其潜能的实现,就是隐德来希。在这里有活动、现实、效能的意思。——译者注}。

在实际生产过程中,活劳动转化为资本,是由于活劳动一方面把工资再生产出来,也就是把可变资本的价值再生产出来,另一方面又创造一个剩余价值;由于这个转化过程,整个[预付的]货币额就都转化为资本,虽然这个货币额中直接发生变化的部分,只是用于工资的那一部分。如果原先价值等于 C+v,它现在就等于 C+(v+X),或者同样可以说,(C+v)+X\endnote{马克思在这里以及在后面用拉丁字母 x 代表剩余价值。——第 425 页。};换句话说,原来的货币额,原来的价值量,在劳动过程中已经增殖,表现为既保存自己同时又增大自己的价值。

\fontbox{~\{}必须指出:只有资本的\textbf{可变部分}才创造资本的增殖额,这种情况丝毫也不改变以下事实,即通过这个过程,全部原有价值增大了,增加了一个剩余价值量;因此,仍然是全部原有的货币额都转化为资本。因为原有价值等于 C+v(不变资本和可变资本)。在上述过程中,这个价值转化为 C+(v+X);v+X 是再生产出来的部分,是通过活劳动转化为物化劳动产生的,而这个转化是由 v 同劳动能力的交换,由可变资本转化为工资所决定和引起的。但是,C+(v+X)=(C+v)(原有资本)+X。此外,v 所以能转化为 v+X,也就是说,(C+v)所以能转化为(C+v)+X,只是由于货币的一部分已转化为 C。一部分货币所以能转化为\textbf{可变}资本,只是由于另一部分货币转化为不变资本。\fontbox{\}~}

劳动在实际生产过程中\textbf{实际上}转化为资本,但是,这个转化是由货币同劳动能力的最初交换决定的。只是由于劳动\textbf{直接}转化为不属于工人而属于资本家的\textbf{物化}劳动,货币才转化为资本,就连已经取得生产资料即劳动条件的形式的那一部分货币也是这样。在此以前,货币不论以它本身的形式存在,或者以那种在实物形式上可以充当生产新商品所必需的生产资料的商品(产品)的形式存在,都只不过\textbf{从可能性来说}是资本。

[1322]只有这种对劳动的一定\textbf{关系}才使货币或商品转化为资本,只有由于自己对生产条件的上述关系(在实际生产过程中有一定的关系同这个关系相适应)使货币或商品转化为资本的\textbf{劳动},才是\textbf{生产劳动};换句话说,只有使那种同劳动能力相对立的、独立化了的\textbf{物化}劳动的价值保存并增殖的劳动,才是生产劳动。生产劳动不过是对劳动能力出现在资本主义生产过程中所具有的整个关系和方式的简称。但是,把生产劳动同\textbf{其他}种类的劳动区分开来是十分重要的,因为这种区分恰恰表现了那种作为整个资本主义生产方式以及资本本身的基础的劳动的形式规定性。

由此可见,在资本主义生产体系中,\textbf{生产劳动}是给使用劳动的人生产\textbf{剩余价值}的劳动,或者说,是把客观劳动条件转化为资本、把客观劳动条件的所有者转化为资本家的劳动,所以,这是把自己的产品作为资本生产出来的劳动。

因此,我们所说的\textbf{生产劳动},是指\textbf{社会地规定了的}劳动,这种劳动包含着劳动的买者和卖者之间的一个十分确定的关系。

虽然劳动能力的买者手中的货币或商品(生产资料和工人的生活资料),只有经过上述过程,只有在上述过程中才转化为资本(这些东西在进入过程之前并不是资本,而只是必将变成资本),但是,它们\textbf{从可能性来说}是资本。它们所以是资本,是由于它们作为某种独立的东西同劳动能力相对立,而劳动能力也作为某种独立的东西同它们相对立,在这里有一种关系,它决定着并保证着它们同劳动能力的交换以及随后发生的劳动实际上转化为资本的过程。在这里,生产资料和生活资料在它们同工人的关系中,从一开始就具有\textbf{一种社会规定性},这种社会规定性使它们变成资本,给它们以支配劳动的权力。因此,它们在作为资本同劳动相对立的情况下,是劳动的\textbf{前提}。

因此,\textbf{生产劳动}可以说是直接同\textbf{作为资本的货币}交换的劳动,或者说,是直接同\textbf{资本}交换的劳动(这不过是前一说法的简化),也就是直接同这样的货币交换的劳动,这种货币从可能性来说就是资本,预定要执行资本的职能,换句话说,作为\textbf{资本}同劳动能力相对立。“\textbf{直接}同\textbf{资本}交换的劳动”,这句话的意思是指劳动同作为\textbf{资本}的货币交换,并使这些货币在实际上转化为资本。从“\textbf{直接}”一词产生什么后果,现在就要作更详细的说明。

因此,生产劳动是这样的劳动,它为工人仅仅再生产出事先已经确定了的他的劳动能力的价值,可是同时,它作为创造价值的活动却增大资本的价值,换句话说,它把它所创造的价值作为资本同工人本身相对立。

\tsubsectionnonum{[(c)在资本同劳动的交换中两个本质上不同的环节]}

我们在考察生产过程时\endnote{马克思指《资本和劳动之间的交换。劳动过程。价值增殖过程》一节(手稿第 I 本第 15—53 页),其中有一小节:《劳动过程和价值增殖过程的统一(资本主义生产过程)》(第 49—53 页)。——第 427 页。}已经看到,在资本同劳动的交换中,应该区别两个互相制约但本质上不同的环节。

\textbf{第一},劳动同资本的最初交换是一个\textbf{形式上的过程},其中资本作为货币出现,劳动能力作为\textbf{商品}出现。在这第一个过程中,劳动能力的出卖是观念上或法律上的出卖,尽管劳动要等到完成之后,也就是要在一日、一周等等末了才\textbf{支付报酬}。这种情况对于\textbf{出卖}劳动能力的交易并无影响。这里\textbf{直接}被出卖的,不是包含已经物化了的劳动的商品,而是\textbf{劳动能力本身的使用},因此,实际上是\textbf{劳动本身},因为劳动能力的使用表现在它的动作——劳动上。也就是说,这里不是通过商品同商品的交换而完成的劳动同劳动的交换。如果 A 把靴子卖给 B,那末他们两人交换的是劳动,一个换出的是物化在靴子中的劳动,另一个换出的是物化在货币中的劳动。但这里拿来交换的,在一方,是一般社会形式的,即作为\textbf{货币}的\textbf{物化劳动},另一方,是\textbf{还只作为劳动能力存在着的劳动};在这里,虽然被出卖的商品的\textbf{价值}不是劳动的价值(一个不合理的用语),而是劳动能力的\textbf{价值},但是,被买卖的对象却是这个劳动能力的使用,即劳动本身。因此,这里发生的是\textbf{物化}劳动同实际上化为活劳动的\textbf{劳动能力}的直接交换,也就是物化劳动同活劳动的交换。因此,工资——劳动能力的价值——如前所说,就表现为\textbf{劳动的价格},\endnote{指以下两小节:《劳动能力的价值。最低限度的工资,或平均工资》(手稿第 I 本第 21—25 页)和《货币和劳动能力之间的交换》(同上,第 25—34 页)。马克思在第 XXI 本第 1312—1314 页又回过头来谈“劳动的价格”问题。——第 428 页。}表现为劳动的直接的购买价格。

在这第一个环节中,工人和资本家的关系是商品的卖者和买者的关系。资本家支付劳动能力的\textbf{价值},即他所购买的商品的\textbf{价值}。

但是,同时,劳动能力所以被购买,只是因为这个劳动能力能够完成和有义务完成的劳动量比再生产劳动能力所需要的劳动量大;因此,这个劳动能力所完成的劳动,表现为一个比劳动能力的价值大的价值。

[1323]\textbf{第二},资本同劳动的\textbf{交换}的第二个环节,实际上同第一个环节毫无关系,严格地说,这个环节根本不是\textbf{交换}。

第一个环节的特点是货币同商品的交换——等价物的交换;在这里,工人和资本家仅仅作为商品所有者彼此对立。交换的是等价物(就是说,交换实际上\textbf{在什么时候}实现,并不会使这个关系有丝毫变化;劳动的价格究竟\textbf{高于}或\textbf{低于}劳动能力的\textbf{价值},还是\textbf{等于}后者的\textbf{价值},并不会使这个交易的性质有丝毫变化。因此,这个交易\textbf{可以}按照商品交换的一般规律来进行)。

第二个环节的特点是根本不发生任何交换。货币所有者不再是商品的买者,而工人也不再是商品的卖者。货币所有者现在执行资本家的职能。他消费他所购买的商品;工人则提供这个商品,因为他的劳动能力的使用就是他的\textbf{劳动}本身。通过前一个交易,劳动本身变成了物质财富的一部分。工人完成这个劳动,但是他的这个劳动是\textbf{属于}资本的,从此以后,只是资本的一种职能而已。因此,这个劳动是在资本的直接监督和管理之下完成的;而这个劳动借以物化的产品,是资本借以表现的新形式,或者更确切地说,是资本实际上借以\textbf{实现}为资本的新形式。因此,劳动通过第一个交易已经\textbf{在形式上}被并入资本之后,在这个过程中,就直接\textbf{物化}为资本,\textbf{直接}转化为资本。在这里,转化为资本的劳动量,比以前用于购买劳动能力的资本量大。在这个过程中,一定量的无酬劳动被占有了,只是因为这个缘故,货币才转化为资本。

虽然这里事实上没有发生交换,可是,如果撇开中介不谈,我们看到,结果是,在这个过程中——把两个环节结合在一起——一定量的物化劳动同较大量的活劳动相交换。整个过程的结果表现为:物化在自己产品中的劳动,大于物化在劳动能力中的劳动,因而大于作为工资支付给工人的物化劳动;换句话说:过程的实际结果在于,资本家不仅收回了他花在工资上的那部分资本,而且得到了一个完全是无偿占有的剩余价值。劳动同资本的\textbf{直接}交换在这里的意思是:(1)劳动直接转化为资本,变成资本的物质组成部分,这个转化是在生产过程中完成的;(2)一定量的物化劳动与等量活劳动加一个\textbf{不经过交换}而占有的活劳动的追加量相交换。

“\textbf{生产劳动}是\textbf{直接}同\textbf{资本}交换的劳动”这个说法,包括上述所有环节,它不过是从下面这个论点派生出来的提法:生产劳动是这样的\textbf{劳动},它把货币转化为资本,它同作为\textbf{资本}的生产条件相交换;因而,它也决不是简单地作为不带特殊社会规定性的\textbf{劳动}同这些生产条件——在这里不是简单地只作为生产条件出现——发生关系。

这包括:(1)货币和劳动能力作为商品彼此对立的关系,货币所有者和劳动能力所有者之间的买和卖;(2)劳动直接隶属于资本;(3)劳动在生产过程中实际转化为资本,或者同样可以说,为资本创造剩余价值。这里发生了\textbf{劳动和资本之间的双重的交换}。第一种交换只表示对劳动能力的购买,所以,从实际结果来看,就是对劳动的购买,因而也是对劳动产品的购买。第二种交换是活劳动直接转化为资本,或者说,作为资本的实现的活劳动的物化。

\tsubsectionnonum{[(d)生产劳动对资本的特殊使用价值]}

资本主义生产过程的结果,既不是单纯的产品(使用价值),也不是\textbf{商品},即具有一定交换价值的使用价值。它的结果,它的产品,是为资本创造\textbf{剩余价值},因而,是货币或商品实际\textbf{转化}为资本;而在生产过程之前,货币或商品仅仅从自己的目的来说,从可能性来说,从自己的使命来说,才是资本。生产过程吸收的劳动量,比购买的劳动量大。在生产过程中完成的这种对别人无酬劳动的吸收、[1324]\textbf{占有},是资本主义生产过程的\textbf{直接目的};因为资本本身(因而资本家本身)的任务,既不是生产直接供自己消费的使用价值,也不是生产用来转化为货币再转化为使用价值的商品。资本主义生产的目的是\textbf{发财致富},是\textbf{价值的增殖},是价值的\textbf{增大},因而是保存原有价值并创造剩余价值。资本只有在同劳动交换(这种劳动因而被称为\textbf{生产劳动})之后,才能得到在资本主义生产过程中\textbf{生产出来的}这种\textbf{特殊的产品}。

用来生产\textbf{商品}的劳动必须是有用劳动,必须生产某种\textbf{使用价值},必须表现为某种\textbf{使用价值}。所以,只有表现为\textbf{商品}、也就是表现为使用价值的劳动,才是同资本交换的劳动。这是不言而喻的前提。但是,不是劳动的这种具体性质,不是劳动的使用价值本身,因而,不是由于劳动是例如裁缝的劳动、鞋匠的劳动、纺工的劳动、织工的劳动等等——不是这一点构成劳动对资本的特殊使用价值,不是这一点使劳动在资本主义生产体系中打上\textbf{生产劳动}的印记。构成劳动对资本的\textbf{特殊使用价值}的,不是劳动的一定的有用性质,也不是劳动借以物化的产品的特殊有用性质。劳动对资本的使用价值,是由这种劳动作为创造交换价值的因素的性质决定的,是由这种劳动固有的抽象劳动的性质决定的;但是,问题不在于劳动一般地代表着这种一般劳动的一定量,而在于劳动代表着一个比劳动价格即\textbf{劳动能力的价值所包含的}抽象劳动\textbf{量大的}抽象劳动量。

对资本来说,劳动能力的使用价值,在于劳动能力提供的劳动量超过物化在劳动能力本身因而为再生产劳动能力所需要的劳动量的余额。劳动当然是以它作为特殊的有用劳动(如纺纱劳动、织布劳动等等)所固有的\textbf{一定形式}被提供的。但是,劳动的这种使自己能表现为商品的具体性质,不是劳动对资本的\textbf{特殊使用价值}。对资本来说,劳动的这种特殊使用价值,在于劳动作为一般劳动所提供的劳动量,并且在于所完成的劳动量\textbf{超过}构成劳动报酬的劳动量的余额。

一定的货币额 x 变成资本,是由于它在它的产品中表现为 x+h,也就是由于作为产品的货币额所包含的劳动量,大于这个货币额原来包含的劳动量。而这是货币同生产劳动相交换的结果,换句话说,只有那种在同物化劳动交换时能使物化劳动表现为一个增大了的物化劳动量的劳动,才是\textbf{生产劳动}。

因此,资本主义生产过程并不单纯是商品生产。它是一个吸收无酬劳动的过程,是一个使生产资料(材料和劳动资料)变为吸收无酬劳动的手段的过程。

从上述一切可以看出,“生产劳动”是对劳动所下的同劳动的\textbf{一定内容},同劳动的特殊效用或劳动所借以表现的特殊使用价值绝对没有任何直接关系的定义。

\textbf{同一}种劳动可以是\textbf{生产劳动},也可以是\textbf{非生产劳动}。

例如,密尔顿创作《失乐园》得到 5 镑,他是\textbf{非生产劳动者}。相反,为书商提供工厂式劳动的作家,则是\textbf{生产劳动者}。密尔顿出于同春蚕吐丝一样的必要而创作《失乐园》。那是\textbf{他的}天性的能动表现。后来,他把作品卖了 5 镑。但是,在书商指示下编写书籍(例如政治经济学大纲)的莱比锡的一位无产者作家却是\textbf{生产劳动者},因为他的产品从一开始就从属于资本,只是为了增加资本的价值才完成的。一个自行卖唱的歌女是\textbf{非生产劳动者}。但是,同一个歌女,被剧院老板雇用,老板为了赚钱而让她去唱歌,她就是\textbf{生产劳动者},因为她生产资本。

\tsubsectionnonum{[(e)非生产劳动是提供服务的劳动。资本主义条件下对服务的购买。把资本和劳动的关系看成服务的交换的庸俗观点]}

[1325]这里产生了不同的问题,不能混为一谈。

我买一条现成的裤子呢,还是买布请一个裁缝到家里来做一条裤子,我对他的\textbf{服务}(即他的缝纫劳动)支付报酬,——这对我是完全无关紧要的,因为对我来说,重要的是裤子本身。我不请裁缝到家里来,而是到服装商人那里去买裤子,是因为前一种方式花费大,而缝纫业资本家生产的裤子,比裁缝在我家做的裤子,花费的劳动少,也就便宜。但是在这两种情况下,我都不是把我用来买裤子的货币变成资本,而是变成裤子;在这两种情况下,对我来说,都是把货币单纯用作流通手段,即把货币转化为一定的使用价值。因此,虽然在一种情况下,货币同\textbf{商品}交换,在另一种情况下,货币购买作为\textbf{商品}的\textbf{劳动}本身,但是,货币在这里都不是执行资本的职能。它只是执行货币的职能,确切些说,执行流通手段的职能。

另一方面,那个在我家里劳动的裁缝不是\textbf{生产劳动者},虽然他的劳动给我提供产品——裤子,而给他自己提供他的劳动的价格——货币。可能,这个裁缝提供的劳动量,大于他从我这里得到的报酬中包含的劳动量;这甚至是完全可能的,因为他的劳动的价格,是由作为\textbf{生产劳动者}的裁缝所取得的价格决定的。但是,这对我是完全无关紧要的。价格一经确定之后,他劳动 8 小时还是劳动 10 小时,对我都一样。对我来说,有意义的只是\textbf{使用价值}——裤子,并且,不论我用前一种方式或后一种方式购买裤子,我所关心的当然是尽量少支付;在第一种情况下和第二种情况下,我同样关心的是:\textbf{我支付的价格在两种情况下,都不应该超过正常价格}。这是用于我的消费的一笔\textbf{支出},这不是我的货币的增加,倒是我的货币的减少。这决不是发财致富的手段,正如用于我\textbf{个人消费}的任何一笔货币支出,都不是发财致富的手段一样。

保尔·德·科克小说中的一位“学者”会对我说,如果没有这种购买,就象不购买面包一样,我就不能生活,因而也就不能\textbf{发财致富}了;因此,这种购买是我发财致富的一个间接手段,或者说,至少是一个条件。根据同样的理由,可以认为我的血液循环、我的呼吸过程也是我发财致富的条件。但是,无论我的血液循环,还是我的呼吸过程,就其本身而论,都决不能使我发财致富,相反,两者都是以代价昂贵的新陈代谢为前提的,如果完全不需要这种新陈代谢,世界上也就没有穷人了。因此,货币和劳动之间的单纯的、\textbf{直接的}交换,既不会使货币转化为资本,也不会使劳动转化为生产劳动。

什么是这种交换的最大特点呢\fontbox{?}这种交换与货币和生产劳动之间的交换有什么不同呢\fontbox{?}不同之处在于,一方面,这里\textbf{货币}是\textbf{作为货币},作为交换价值的独立形式支出的,这个交换价值应该转化为某种\textbf{使用价值},生活资料,个人消费品。在这里,货币不变成资本,相反,为了作为使用价值来消费,它不再作为交换价值而存在。另一方面,在这里,劳动只是作为使用价值,作为把布做成裤子的\textbf{服务},作为依靠它的一定有用性质给我提供的服务,才使我感到兴趣。

相反,同一个裁缝向雇用他的缝纫业资本家提供的服务,决不在于他把布做成裤子,而在于物化在裤子中的必要劳动时间等于 12 小时,而裁缝所得的工资只等于 6 小时。因此,裁缝向资本家提供的服务,在于他无偿地劳动了 6 小时。这件事以缝制裤子的形式出现,只是\textbf{掩盖}了实际的关系。因此,缝纫业资本家一有可能就设法把裤子再转化为货币,就是说,转化为这样一种形式,在这种形式中,缝纫劳动的一定性质完全消失,而已经提供的服务就不是表现为[1326]由一定货币额代表的 6 小时劳动时间,而是表现为由加倍的货币额代表的 12 小时劳动时间。

我购买缝纫劳动,是为了取得它作为\textbf{缝纫劳动}所提供的服务,来满足我穿衣的需要,也就是为我的一种\textbf{需要}服务。缝纫业资本家购买缝纫劳动,是把它当作使一个塔勒变成两个塔勒的手段。我购买缝纫劳动,是因为它生产一定的使用价值,提供一定的服务。资本家购买缝纫劳动,是因为它提供的交换价值额大于花在它上面的费用,也就是说,是因为它对资本家说来,纯粹是一个用较少劳动交换较多劳动的手段。

凡是货币直接同不生产资本的劳动即\textbf{非生产}劳动相交换的地方,这种劳动都是作为\textbf{服务}被购买的。服务这个名词,一般地说,不过是指这种劳动所提供的特殊使用价值,就象其他一切商品也提供自己的特殊使用价值一样;但是,这种劳动的特殊使用价值在这里取得了“服务”这个特殊名称,是因为劳动不是作为\textbf{物},而是作为\textbf{活动}提供服务的,可是,这一点并不使它例如同某种机器(如钟表)有什么区别。我给为了你做,我做为了你做,我做为了你给,我给为了你给,\endnote{这是罗马法上的契约关系的四种公式。原文是:Doutfacias,facioutfacias,facioutdes,doutdes。参看马克思《资本论》第 1 卷第 17 章。——第 435 页。}在这里是同一关系的、意义完全相同的几种形式,而在资本主义生产中,我给为了你做这个形式所表示的,是被付出的具有物的形式的价值同被占有的活的活动之间的极为特殊的关系。因此,既然对\textbf{服务}的购买中完全不包含劳动和资本的特殊关系(在这里,这个关系或者完全消失了,或者根本不存在),那末,对服务的购买,自然成为萨伊和巴师夏之流最喜欢用来表现\textbf{资本和劳动之间的关系}的形式。

这些服务的\textbf{价值}如何确定,这个\textbf{价值}本身如何由工资规律决定,这是同我们这里研究的关系完全无关的问题,这个问题要在工资那一章考察。

由此可见,单是货币同劳动的交换,还不能使劳动转化为\textbf{生产劳动},另一方面,这种劳动的\textbf{内容}最初是无关紧要的。

工人自己可以购买劳动,就是购买以服务形式提供的商品,他的工资花在这些服务上,同他的工资花在购买其他任何商品上,是没有什么不同的。他购买的服务,可以是相当必要的,也可以是不太必要的:例如,他可以购买医生的服务,也可以购买牧师的服务,就象他可以购买面包,也可以购买烧酒一样。工人作为买者,即作为同商品对立的货币的代表,同仅仅作为买者出现,即仅仅把货币换成商品形式的资本家,完全属于同一个范畴。这些服务的价格怎样决定,这种价格同真正的工资有什么关系,它在什么程度上受工资规律的调节,在什么程度上不受工资规律的调节,这些问题,应当在研究工资时加以考察,同当前的研究完全无关。

可见,如果说单是货币同劳动的交换还不能使劳动转化为\textbf{生产劳动},或者同样可以说,还不能使货币转化为资本,那末,劳动的\textbf{内容}、它的具体性质、它的特殊效用,看来最初也是无关紧要的:我们前面已经看到,同一个裁缝的同样的劳动,在一种情况下表现为生产劳动,在另一种情况下却表现为非生产劳动。

某些\textbf{服务},或者说,作为某些活动或劳动的结果的\textbf{使用价值},体现为\textbf{商品},相反,其他一些服务却不留下任何可以捉摸的、同提供这些服务的人\textbf{分开存在的}结果,或者说,其他一些服务的结果不是\textbf{可以出卖的商品}。例如,一个歌唱家为我提供的服务,满足了我的审美的需要;但是,我所享受的,只是同歌唱家本身分不开的活动,他的劳动即歌唱一停止,我的享受也就结束;我所享受的是活动本身,是它引起的我的听觉的反应。这些服务本身,同我买的商品一样,可以是确实必要的,或者仅仅看来是必要的,例如士兵、医生和律师的服务,——或者它们可以是给我提供享受的服务。但是,这丝毫不改变它们的经济性质。如果我身体健康,用不着医生,或者我有幸不必去打官司,那我就会象避开瘟疫一样,避免把货币花在医生或律师的服务上。

[1328]\endnote{这里马克思把手稿页码“1327”误写为“1328”。——第 437 页。}有些\textbf{服务}也可以是强加于人的,例如\textbf{官吏的服务}等等。

如果我自己购买,或者别人为我购买一个教师的服务,其目的不是发展我的才智,而是让我学会赚钱的本领,而我又真的学到了一些东西(这件事就它本身来说,完全同对于教师的服务支付报酬无关),那末,这笔学费同我的生活费完全一样,应归入我的劳动能力的生产费用。但是,这种服务的特殊效用\textbf{丝毫不改变现有的经济关系};在这里,货币没有转化为资本,换句话说,我对这个提供服务的人即教师来说,并没有成为\textbf{资本家},没有成为他的主人。因此,医生是否把我的病治好了,教师的教导是否有成效,律师是否使我打赢了官司,对于这种关系的\textbf{经济性质}来说,也完全是无关紧要的。在这里,被支付报酬的是服务本身,而就服务的性质来说,其结果是不能由提供服务的人保证的。很大一部分\textbf{服务}的报酬,属于同商品的\textbf{消费}有关的费用,如女厨师、女佣人等等的服务。

一切\textbf{非生产劳动}的特点是,支配多少非生产劳动——象购买其他一切供消费的商品的情况一样——是同剥削多少\textbf{生产工人}成比例的。因此,\textbf{生产工人}支配非生产劳动者的\textbf{服务}的可能性,比一切人都要少,虽然他们对\textbf{强加于}他们的服务(国家、赋税)支付报酬最多。相反,我使用\textbf{生产工人}的劳动的可能性,同我使用\textbf{非生产劳动者}的劳动决不是成比例地增长;相反,这里是成反比例。

对我来说,甚至\textbf{生产工人}也可以是\textbf{非生产劳动者}。例如,如果我请人来把我的房子裱糊一下,而这些裱糊工人是完成我的这项定货的老板的雇佣工人,那末,这个情况,对我来说,就好比是我买了一所裱糊好的房子,也就是说,好比是我把货币花费在一个供我消费的商品上。可是,对于叫这些工人来裱糊的那位老板来说,他们是生产工人,因为他们为他生产剩余价值。[1328]

\centerbox{※     ※     ※}

[1333]如果一个工人虽然生产了可以出卖的商品,但是,他生产的数额仅仅相当于他自己的劳动能力的价值,因而没有为资本生产出剩余价值,那末,从资本主义生产的观点看来,这种工人\textbf{不是生产的},这一点,从李嘉图的话里已经可以看出来,他说,这种人的存在本身就是一个累赘。\endnote{马克思指李嘉图的《政治经济学和赋税原理》第二十六章(《论总收入与纯收入》)。——第 408、438 页。}这就是资本的理论和实践:

\begin{quote}“关于资本的理论,以及\textbf{使劳动停在}除工人生活费用之外还能为资本家生产\textbf{利润的那个点上的实践},看来,都是同调节生产的自然法相违背的。”(\textbf{托·霍吉斯金}《通俗政治经济学》1827 年伦敦版第 238 页)[1333]\end{quote}

\centerbox{※     ※     ※}

[1336]我们已经看到:资本主义生产过程不仅是\textbf{商品}的生产过程,而且是\textbf{剩余价值}的生产过程,是剩余劳动的吸收,因而是资本的生产过程。货币和劳动之间或者说资本和劳动之间的最初的、形式上的交换行为,仅仅\textbf{从可能性来说}是通过物化劳动对别人的活劳动的占有。实际的占有过程只是在实际的生产过程中才完成的,对这个生产过程说来,资本家和工人\textbf{单纯作为商品所有者}相互对立、作为买者和卖者彼此发生关系的那种最初的、形式上的交易,已经是过去的阶段了。因此,一切庸俗经济学家,例如巴师夏,都只停留在这种最初的、形式上的交易上,其目的正是要用欺骗手法摆脱特殊的资本主义关系。在货币同非生产劳动的交换中,这个区别就十分清楚地表现出来了。这里,货币和劳动\textbf{只}作为商品相互交换。因此,这种交换不形成资本,这种交换是\textbf{收入的支出}。[1336]

\tsubsectionnonum{[(f)资本主义社会中手工业者和农民的劳动]}

[1328]那些不雇用工人因而不是作为资本家来进行生产的独立的手工业者或农民的情况又怎样呢\fontbox{?}他们——这是农民的典型情况\fontbox{~\{}但是,比方说,我在家里雇用的园丁却不是这样\fontbox{\}~}——可以是\textbf{商品生产者},而我向他们购买\textbf{商品},至于手工业者按定货供应商品,农民按自己资金的多少供应商品,这些情况并不会使问题有丝毫改变。在这种场合,他们是作为商品的卖者,而不是作为劳动的卖者同我发生一定的关系,所以,这种关系与资本和劳动之间的交换毫无共同之处,因此,在这里也就用不上\textbf{生产劳动}和\textbf{非生产劳动}的区分——这种区分的基础在于,劳动是同作为货币的货币相交换,还是同作为资本的货币相交换。因此,农民和手工业者虽然也是商品生产者,却既不属于\textbf{生产劳动者}的范畴,又不属于\textbf{非生产劳动者}的范畴。但是,他们是自己的生产不从属于资本主义生产方式的商品生产者。

可能,这些用自己的生产资料进行劳动的生产者,不仅再生产了自己的劳动能力,而且创造了剩余价值,并且,他们的地位容许他们占有自己的剩余劳动或剩余劳动的一部分(因为其余部分以税收等形式从他们那里拿走了)。这里我们遇到的是这样一个社会所具有的特点,在这个社会中一定的生产方式占支配地位,但是还不是这个社会的一切生产关系都从属于它。例如,在封建社会中,连那些同封建主义的实质相距很远的关系,也具有封建的外貌(这一点我们在英国可以看得最清楚,因为封建制度是现成地从诺曼底移进英国的,而它的形式给英国固有的、许多方面都和它不同的社会制度打上了印记)。例如,同领主和陪臣相互间的亲身服务毫无关系的单纯货币关系,也具有了封建的外貌。关于小农凭封地权占有自己土地的虚构,也是一个例证。

在资本主义生产方式下,情况也完全一样。独立农民或手工业者分裂为两重身分。[注:“在小企业中……\textbf{企业主}常常是他\textbf{自己的工人}。”(\textbf{施托尔希}[《政治经济学教程》]彼得堡版第 1 卷第 242 页)]作为生产资料的所有者,他是资本家;作为劳动者,他是他自己的雇佣工人。因此,他作为资本家,自己给自己支付工资,从自己的资本中取得利润,就是说,剥削他自己这个雇佣工人,他以剩余价值的形式向自己支付那应由劳动向资本交付的贡物。同样,也许他还向作为土地所有者的自己支付第三个部分(地租),就象我们后面要看到的\endnote{见马克思《资本论》第 3 卷第 23 章。——第 440 页。}工业资本家那样,工业资本家在自己企业中使用自己的[1329]资本,向自己支付利息,并且把这种利息看成他不是作为工业资本家而是单纯作为资本家所应得的东西。

在资本主义生产中,生产资料(它们表现一定的\textbf{生产关系})所具有的\textbf{社会规定性}同生产资料本身的物质存在是这样地结合在一起,而在资产阶级社会的观念中,这种社会规定性同这种物质存在是这样地不可分离,以致这种社会规定性(即范畴的规定性)甚至也被用到同它直接矛盾的那些关系上去了。生产资料只有当它独立化,作为独立的力量来反对劳动的时候,才成为资本。而在我们所考察的场合,生产者——劳动者——是自己的生产资料的占有者、所有者。因此,这些生产资料不是资本,而劳动者也不是作为雇佣工人同这些生产资料相对立。然而,这些生产资料被看作资本,而劳动者自己分裂为两重身分,结果就是\textbf{他}作为资本家来雇用他自己这个工人。

这种表现方式,初看起来虽然很不合理,可是从下述意义来看,实际上还是表现了某种正确的东西。在我们考察的场合,生产者的确创造他自己的剩余价值\fontbox{~\{}假定生产者按商品的价值出卖他的商品\fontbox{\}~},或者说,物化在他的全部产品中的,只是他自己的劳动。但是,他能够\textbf{自己}占有他自己劳动的全部产品,他的产品价值超过他的譬如一天劳动的平均价格的余额没有被第三者即\textbf{老板}占有,这并不是靠他的劳动(就这方面来说,他同其他工人毫无区别),而是仅仅靠他占有生产资料。因此,仅仅由于他是生产资料所有者,他自己的剩余劳动才归他所有,从这个意义上说,他作为他自己的资本家同他自己这个雇佣工人发生关系。

在现在这个社会中,\textbf{分离}表现为正常的关系。因此,在实际上没有分离的地方,也假定有分离,并且象刚才已经指出的,这在一定意义上是正确的;因为(和例如古罗马、挪威以及美国西北部的社会关系不同)在这里,\textbf{结合}表现为某种偶然的东西,而\textbf{分离}却表现为某种正常的东西,因此,即使在各种不同的职能结合在一个人身上的地方,分离还是被作为一定的关系来坚持。这里表现得非常明显:资本家本身不过是资本的职能,工人本身不过是劳动能力的职能。并且这是一条规律:在经济发展过程中,这些职能分配在不同的人身上,而且用自己的生产资料进行生产的手工业者或农民,不是逐渐变成剥削别人劳动的小资本家,就是丧失自己的生产资料\fontbox{~\{}最常见的是后一种情况,即使他仍然是生产资料的\textbf{名义上的}所有者,例如农民在抵押借款的时候就是这样\fontbox{\}~},变成雇佣工人。这是资本主义生产方式占支配地位的社会形式中的发展趋势。

\tsubsectionnonum{[(g)关于生产劳动的补充定义:生产劳动是物化在物质财富中的劳动]}

因此,在考察资本主义生产的本质关系时,可以假定\fontbox{~\{}因为资本主义生产越来越接近这个情况;因为这是过程的基本方向,而且只有在这种情况下,劳动生产力的发展才达到最高峰\fontbox{\}~},整个商品世界,物质生产即物质财富生产的一切领域,都(在形式上或者实际上)从属于资本主义生产方式。这个假定表示上述过程的极限,并且越来越接近于现实情况的正确表述。按照这个假定,一切从事商品生产的工人都是雇佣工人,而生产资料在所有物质生产领域中,都作为资本同他们相对立。在这种情况下,可以认为,\textbf{生产工人}即生产资本的工人的特点,是他们的劳动物化在\textbf{商品}中,物化在物质财富中。这样一来,\textbf{生产劳动},除了它那个与\textbf{劳动内容}完全无关、不以劳动内容为转移的具有决定意义的特征之外,又得到了与这个特征不同的第二个定义,补充的定义。

\tsubsectionnonum{[(h)非物质生产领域中的资本主义表现]}

在非物质生产中,甚至当这种生产纯粹为交换而进行,因而纯粹生产\textbf{商品}的时候,也可能有两种情况:

(1)生产的结果是\textbf{商品},是使用价值,它们具有离开生产者和消费者而独立的形式,因而能在生产和消费之间的一段时间内存在,并能在这段时间内作为\textbf{可以出卖的商品}而流通,如书、画以及一切脱离艺术家的艺术活动而单独存在的艺术作品。在这里,资本主义生产只是在很有限的规模上被应用,例如,一个作家在编一部集体著作百科全书时,把其他许多作家当作助手来剥削。[1330]这里的大多数情况,都还只局限于\textbf{向资本主义生产过渡的形式},就是说,从事各种科学或艺术的生产的人,工匠或行家,为书商的总的商业资本而劳动,这种关系同真正的资本主义生产方式无关,甚至在形式上也还没有从属于它。在这些过渡形式中,恰恰对劳动的剥削最厉害,但这一点并不改变事情的本质。

(2)产品同生产行为不能分离,如一切表演艺术家、演说家、演员、教员、医生、牧师等等的情况。在这里,资本主义生产方式也只是在很小的范围内能够应用,并且就事物的本性来说,只能在某些领域中应用。例如,在学校中,教师对于学校老板,可以是纯粹的雇佣劳动者,这种教育工厂在英国多得很。这些教师对学生来说虽然不是\textbf{生产工人},但是对雇佣他们的老板来说却是生产工人。老板用他的资本交换教师的劳动能力,通过这个过程使自己发财。戏院、娱乐场所等等的老板也是用这种办法发财致富。在这里,演员对观众说来,是艺术家,但是对自己的企业主说来,是\textbf{生产工人}。资本主义生产在这个领域中的所有这些表现,同整个生产比起来是微不足道的,因此可以完全置之不理。

\tsubsectionnonum{[(i)从物质生产总过程的角度看生产劳动问题]}

在特殊的资本主义生产方式中,许多工人共同生产同一个商品;随着这种生产方式的发展,这些或那些工人的劳动同生产对象之间直接存在的关系,自然是各种各样的。例如,前面提到过的那些工厂小工\endnote{马克思在同一稿本(第 XXI 本)第 1308 页写了关于工厂小工的劳动。——第 443 页。},同原料的加工毫无直接关系;监督直接进行原料加工的工人的那些监工,就更远一步;工程师又有另一种关系,他主要只是从事脑力劳动,如此等等。但是,\textbf{所有这些}具有不同价值的劳动能力(虽然使用的劳动量大致是在同一水平上)的\textbf{劳动者的总体}进行生产的结果——从单纯的劳动过程的\textbf{结果}来看——表现为\textbf{商品}或一个\textbf{物质产品}。所有这些劳动者合在一起,作为一个生产集体,是生产这种\textbf{产品}的活机器,就象从整个生产过程来看,他们用自己的劳动同资本交换,把资本家的货币作为资本再生产出来,就是说,作为自行增殖的价值,自行增大的价值再生产出来。

资本主义生产方式的特点,恰恰在于它把各种不同的劳动,因而也把脑力劳动和体力劳动,或者说,把以脑力劳动为主或者以体力劳动为主的各种劳动分离开来,分配给不同的人。但是,这一点并不妨碍物质产品是所有这些人的\textbf{共同劳动的产品},或者说,并不妨碍他们的共同劳动的产品体现在物质财富中;另一方面,这一分离也丝毫不妨碍:这些人中的每一个人对资本的关系是雇佣劳动者的关系,是在这个特定意义上的\textbf{生产工人}的关系。所有这些人不仅\textbf{直接}从事物质财富的生产,并且用自己的劳动\textbf{直接}同作为资本的货币交换,因而不仅把自己的工资再生产出来,并且还直接为资本家创造剩余价值。他们的劳动是由有酬劳动加无酬的剩余劳动组成的。

\tsubsectionnonum{[(k)运输业是一个物质生产领域。运输业中的生产劳动]}

除了采掘工业、农业和加工工业以外,还存在着第四个物质生产领域,这个领域在自己的发展中,也经历了几个不同的生产阶段:手工业生产阶段、工场手工业生产阶段、机器生产阶段。这就是\textbf{运输业},不论它是客运还是货运。在这里,\textbf{生产劳动}对资本家的关系,也就是说,雇佣工人对资本家的关系,同其他物质生产领域是完全一样的。其次,在这里,劳动对象发生某种物质变化——\textbf{空间的}、位置的变化。至于客运,这种位置变化只不过是企业主向乘客提供的\textbf{服务}。但是,这种\textbf{服务}的买者和卖者的关系,就象纱的卖者和买者的关系一样,同生产工人对资本的关系是毫无共同之处的。

如果我们就商品来考察这个过程,那末[1331]这里在劳动过程中,劳动对象,\textbf{商品},确实发生了某种变化。它的位置改变了,从而它的使用价值也起了变化,因为这个使用价值的位置改变了。商品的交换价值增加了,增加的数量等于使商品的使用价值发生这种变化所需要的劳动量。这个劳动量,一部分决定于不变资本的消耗,即加入商品的物化劳动量,另一部分决定于活劳动量,这同其他一切商品的价值增殖过程的情况是一样的。

商品一到达目的地,它的使用价值所发生的这个变化也就消失,这个变化只表现为商品的交换价值提高了,商品变贵了。虽然在这里,实在劳动在使用价值上没有留下一点痕迹,可是这个劳动已经实现在这个物质产品的交换价值中。可见,凡是适用于其他一切物质生产领域的,同样适用于运输业:在这个领域里,劳动也体现在\textbf{商品}中,虽然它在商品的使用价值上并不留下任何可见的痕迹。

\centerbox{※     ※     ※}

我们在这里研究的还只是\textbf{生产资本},就是说,还只是用于\textbf{直接生产过程}中的资本。后面我们还要谈到\textbf{流通过程}中的资本。只有到后面研究资本作为\textbf{商业资本}所采取的特殊形式时,才能答复这样的问题:商业资本所雇用的工人在什么范围内是生产的,在什么范围内是非生产的。\endnote{见马克思《资本论》第 2 卷第 6 章和第 3 卷第 17 章。——第 445 页。}[XXI—1331]

\tsectionnonum{[(13)《资本论》第一部分和第三部分的计划草稿]}

\tsubsectionnonum{[(a)《资本论》第一部分或第一篇的计划]}

[XVIII—1140]第一篇\endnote{马克思把《资本论》的三个理论部分最初称为“章”,然后称为“篇”,最后称为“册”。参看注 13。——第 446 页。}——《\textbf{资本的生产过程}》——分为:\endnote{这些计划草稿马克思写于 1863 年 1 月。它们在 1861—1863 年手稿第 XVIII 本论舍尔比利埃和理查·琼斯两章的正文之间(手稿上用粗体的方括号把它们同这两章的正文隔开)。——第 446 页。}

(1)导言:商品,货币。

(2)货币转化为资本。

(3)绝对剩余价值:(a)劳动过程和价值增殖过程;(b)不变资本和可变资本;(c)绝对剩余价值;(d)争取正常工作日的斗争;(e)同一时间的工作日(同时雇用的工人人数)。剩余价值额和剩余价值率(大小和高低\fontbox{?})。

(4)相对剩余价值:(a 简单协作;(b)分工;(c)机器等等。

(5)绝对剩余价值和相对剩余价值的结合。雇佣劳动和剩余价值的比例。劳动对资本的形式上的隶属和实际上的隶属。资本的生产性。生产劳动和非生产劳动。

(6)剩余价值再转化为资本。原始积累。威克菲尔德的殖民学说。

(7)生产过程的结果。

(占有规律的表现中的变革可以在第 6 点或第 7 点中考察。)

(8)剩余价值理论。

(9)关于生产劳动和非生产劳动的理论。[XVIII—1140]

\tsubsectionnonum{[(b)《资本论》第三部分或第三篇的计划]}

[XVIII—1139]第三篇——《\textbf{资本和利润}》——分为:

(1)剩余价值转化为利润。不同于剩余价值率的利润率。

(2)利润转化为平均利润。一般利润率的形成。价值转化为生产价格。

(3)亚·斯密和李嘉图关于利润和生产价格的理论。

(4)地租(价值和生产价格的区别的例解)。

(5)所谓李嘉图地租规律的历史。

(6)利润率下降的规律。亚·斯密、李嘉图、凯里。

(7)利润理论。

(问题:是不是还应该把西斯蒙第和马尔萨斯包括在《剩余价值理论》里\fontbox{?})

(8)利润分为产业利润和利息。商业资本。货币资本。

(9)收入及其源泉。这里也包括生产过程和分配过程之间的关系问题。

(10)资本主义生产总过程中货币的回流运动。

(11)庸俗政治经济学。

(12)结论。资本和雇佣劳动。[XVIII—1139]

\tsubsectionnonum{[(c)《资本论》第三部分第二章的计划]}

[XVIII—1109]研究《\textbf{资本和利润}》的第三部分第二章论述一般利润率的形成。这里要研究以下几个问题:

(1)资本的不同的有机构成。它部分是由可变资本和不变资本之间的[比例的]差别决定的,因为这个差别是从一定的\textbf{生产发展阶段}产生的,是从机器和原料同推动它们的劳动量之间的绝对的\textbf{数量上}的比例产生的。这些差别同劳动过程有关。同样,在这里还必须考察从流通过程产生的固定资本和流动资本的差别,考察它们如何使一定时期内不同领域中的资本的价值增殖发生变化。

(2)不同资本的组成部分的价值比例的差别,这些差别不是由不同资本的有机构成产生的。这主要是由原料价值的差别产生的,即使假定在两个不同的领域中,原料吸收的劳动量相等。

(3)在资本主义生产的不同领域中,由于上述差别而产生的利润率的差异。不同领域中的利润率相等,以及利润量同所使用的资本量成正比,这只有对同一构成的资本来说才是正确的。

(4)第一章论述的一切,适用于总资本。在资本主义生产中,每一个资本都作为总资本的一部分,作为它的某个份额出现。一般利润率的形成(竞争)。

(5)价值转化为生产价格。价值、成本价格和生产价格之间的差别。

(6)为了还要包括对李嘉图关于这个论题的观点的分析,补充以下一点:工资的一般变动对一般利润率的影响,从而对生产价格的影响。[XVIII—1109]
