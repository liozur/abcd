%!TEX program = xelatex

% TODO: 尝试让 \endnote 的标记使用黑体数字
% TODO: 尝试让 \endnote 命令可以用于 \title 之中
% TODO: 尝试让 \tdoc 与 \tpart 在目录中可以显示页码范围
% TODO: \authornote 与 \editornote 之间的垂直间距可能需要微调

\documentclass[
  zihao=5,
  punct=kaiming,
  linespread=1.4,
  sub4section,
]{ctexbook}

% ------------------------------------------------------------------------------
% 导言区
% ------------------------------------------------------------------------------

% ----------------------------------------------------------
% 宏包加载
% ----------------------------------------------------------

\usepackage{bigfoot}
\usepackage{calc}
\usepackage{CJKfntef}
\usepackage{enotez}
\usepackage{etoolbox}
\usepackage{fancyhdr}
\usepackage{fontspec}
\usepackage{geometry}
\usepackage{graphicx}
\usepackage[hidelinks,hyperfootnotes=true]{hyperref}
\usepackage{lipsum}
\usepackage{perpage}
\usepackage{scrextend}
\usepackage{tikz}
\usepackage{titlesec}
\usepackage{titletoc}
\usepackage{truncate}
\usepackage{ulem}
\usepackage{xcolor}
\usepackage{xunicode-addon}

\raggedbottom
\newsavebox{\headerbox}

\usepackage{xeCJKfntef}
\xeCJKsetup{underdot/symbol={\raisebox{-5pt}{◦}}}
\newcommand{\dotemph}[1]{\CJKunderdot{#1}}

% ----------------------------------------------------------
% 字体定义
% ----------------------------------------------------------

\setCJKmainfont{FZShuSong}
[
  Path           = fonts/,
  BoldFont       = FZHei,
  ItalicFont     = FZKai,
  BoldItalicFont = FZHei,
]
\setCJKsansfont{FZHei}[Path=fonts/]
\setCJKmonofont{HYFangSong}[Path=fonts/]

\setCJKfamilyfont{ss}{FZShuSong}[Path=fonts/]
\setCJKfamilyfont{fs}{FZFangSong}[Path=fonts/]
\setCJKfamilyfont{ht}{FZHei}[Path=fonts/]
\setCJKfamilyfont{kt}{FZKai}[Path=fonts/]
\setCJKfamilyfont{xb}{FZXiaoBiaoSong}[Path=fonts/]
\setCJKfamilyfont{xh}{FZXiHeiI}[Path=fonts/]
\setCJKfamilyfont{sz}{Berthold Baskerville Book}[Path=fonts/]

% 为封面设计的压缩字体
\newCJKfontfamily\coverxb{FZXiaoBiaoSong}[Path=fonts/, FakeStretch=0.85]
\newCJKfontfamily\coverxh{FZXiHeiI}[Path=fonts/, FakeStretch=0.85]

% 定义经过轻微压缩的字体族 (90%)
\newCJKfontfamily\modifiedss{FZShuSong}[Path=fonts/, FakeStretch=0.9]
\newCJKfontfamily\modifiedfs{FZFangSong}[Path=fonts/, FakeStretch=0.9]
\newCJKfontfamily\modifiedht{FZHei}[Path=fonts/, FakeStretch=0.9]
\newCJKfontfamily\modifiedkt{FZKai}[Path=fonts/, FakeStretch=0.9]
\newCJKfontfamily\modifiedxb{FZXiaoBiaoSong}[Path=fonts/, FakeStretch=0.9]
\newCJKfontfamily\modifiedxh{FZXiHeiI}[Path=fonts/, FakeStretch=0.9]

% 定义经过更轻微压缩的字体族 (95%)
\newCJKfontfamily\modifiedmodifiedss{FZShuSong}[Path=fonts/, FakeStretch=0.95]
\newCJKfontfamily\modifiedmodifiedfs{FZFangSong}[Path=fonts/, FakeStretch=0.95]
\newCJKfontfamily\modifiedmodifiedht{FZHei}[Path=fonts/, FakeStretch=0.95]
\newCJKfontfamily\modifiedmodifiedkt{FZKai}[Path=fonts/, FakeStretch=0.95]
\newCJKfontfamily\modifiedmodifiedxb{FZXiaoBiaoSong}[Path=fonts/, FakeStretch=0.95]
\newCJKfontfamily\modifiedmodifiedxh{FZXiHeiI}[Path=fonts/, FakeStretch=0.95]

% 标点符号样式设置
\xeCJKEditPunctStyle{kaiming}
{
  fixed-margin-ratio = 0.000,
  mixed-margin-ratio = 1.000,
  middle-margin-ratio = 0.000,
}

% ----------------------------------------------------------
% 版面与边距
% ----------------------------------------------------------

\geometry{
  paperwidth  = 437bp,
  paperheight = 613bp,
  top         = 94.3bp,
  bottom      = 60.0bp,
  left        = 76.5bp,
  right       = 66.5bp,
  headsep     = 25.0bp,
  footskip    = 17.0bp,
  footnotesep = 11bp,
}

% ----------------------------------------------------------
% 页眉、页脚与页码
% ----------------------------------------------------------

\pagestyle{fancy}

\fancyhf{}

% --------------------------------------
% 页眉与页脚
% --------------------------------------

\renewcommand{\headrulewidth}{0bp}

\fancyhead[RO]{\rightmark}
\fancyhead[LE]{\leftmark}

\renewcommand{\chaptermark}[1]{%
  \sbox{\headerbox}{\small\modifiedxh#1}%
  \ifdim\wd\headerbox > 20em
  \def\theheadermark{\truncate[……]{20em}{\small\modifiedxh#1}}%
  \else
  \def\theheadermark{\small\modifiedxh#1}%
  \fi
  \markboth{\theheadermark}{}%
}

\renewcommand{\sectionmark}[1]{%
  \sbox{\headerbox}{\small\modifiedxh#1}%
  \ifdim\wd\headerbox > 20em
  \def\theheadermark{\truncate[……]{20em}{\small\modifiedxh#1}}%
  \else
  \def\theheadermark{\small\modifiedxh#1}%
  \fi
  \markright{\theheadermark}%
}

% --------------------------------------
% 页码
% --------------------------------------

\fancyfoot[RO]{\fontspec{Berthold Baskerville Book}[Path=fonts/]\thepage}
\fancyfoot[LE]{\fontspec{Berthold Baskerville Book}[Path=fonts/]\thepage}

\assignpagestyle{\part}{empty}
\fancypagestyle{empty}{%
  \fancyhf{}%
  \fancyfoot{}%
}

\fancypagestyle{plain}{%
  \fancyhf{}%
  \fancyfoot[RO]{%
    \makebox[0pt][l]{%
      \hspace{3.5bp}%
      \fontspec{Berthold Baskerville Book}[Path=fonts/]\thepage
    }%
  }%
  \fancyfoot[LE]{\fontspec{Berthold Baskerville Book}[Path=fonts/]\thepage}%
}

% ----------------------------------------------------------
% 注释
% ----------------------------------------------------------

\AtBeginUTFCommand[\textcircled]{\begroup\EnclosedNumbers}
\AtEndUTFCommand[\textcircled]{\endgroup}

\xeCJKDeclareCharClass{Default}{"24EA, "2460->"2473, "3251->"32BF}
\newfontfamily\EnclosedNumbers{FZShuSong}[Path=fonts/]

\makeatletter

\setlength{\skip\footins}{2\baselineskip plus 2\baselineskip minus 2\baselineskip}

% --------------------------------------
% authornote
% --------------------------------------

\DeclareNewFootnote{B}

\newcommand\defaultfootnoterule{\vskip 8.0bp\hrule width 74bp height 0.3bp\vskip 8.0bp}

\let\fn@footnoteB\thefootnoteB
\let\fn@fnfootnoteB\footnoteB

\renewcommand{\thefootnoteB}{%
  \CJKfamily{fs}%
  \fontsize{8.0bp}{8.0bp}\selectfont%
  (\fn@footnoteB)%
}

\newcommand{\authornote}[1]{%
  \deffootnote[5em]{0em}{0em}{%
    \raisebox{0.3bp}{%
      \thefootnotemark\hspace{0.5em}%
    }%
  }%
  \fn@fnfootnoteB{%
    \CJKfamily{fs}%
    \fontsize{9.0bp}{9.5bp}\selectfont%
    #1%
  }%
  \hspace{-0.2em}%
}

% --------------------------------------
% editornote
% --------------------------------------

\DeclareNewFootnote{C}

\renewcommand\extrafootnoterule{\vskip -4.0bp\hrule width \textwidth height 0.3bp\vskip 8.0bp}

\let\fn@fnfootnoteC\footnoteC

\renewcommand\thefootnoteC{%
  \textcircled{%
    \arabic{footnoteC}%
  }%
}

\newcommand{\editornote}[1]{%
  \hspace{0.1em}%
  \deffootnote{2em}{0em}{%
    \raisebox{0.4bp}{%
      \makebox[2em][l]{\thefootnotemark}%
    }%
  }%
  \fn@fnfootnoteC{%
    \CJKfamily{fs}%
    \fontsize{9.0bp}{9.5bp}\selectfont%
    #1%
  }%
}

\MakePerPage{footnoteC}

\makeatother

% --------------------------------------
% endnote
% --------------------------------------

\setenotez{
  list-heading=\chapter*{\CJKfamily{kt}\fontsize{16.5bp}{16.5bp}\selectfont\ziju{1.500}{注释}},
  backref=true
}

\renewcommand\enmark[1]{%
  {\CJKfamily{ss}\fontsize{9.0bp}{9.75bp}\selectfont #1}\hspace{1em}%
}

\let\originalendnote\endnote

\renewcommand{\endnote}[1]{%
  \hspace{0.05em}%
  \originalendnote{%
    \begingroup
    \leftskip=2.9em\relax%
    \rightskip=0.1em\relax%
    \CJKfamily{ss}%
    \fontsize{9.0bp}{9.75bp}\selectfont%
    \ziju{0.050}%
    #1%
    \par%
    \endgroup%
    \vspace{-0.6em}%
  }%
  \hspace{-0.05em}%
}

\fancypagestyle{endnotestyle}{%
  \fancyhf{}%
  \renewcommand{\headrulewidth}{0bp}%

  \fancyhead[RO]{\small\CJKfamily{xh}\scalebox{0.9}[1.0]{\ziju{1.5pt}{注释}}}
  \fancyhead[LE]{\small\CJKfamily{xh}\scalebox{0.9}[1.0]{\ziju{1.5pt}{注释}}}

  \fancyfoot[RO]{%
    \makebox[0pt][l]{%
      \hspace{3.5bp}%
      \fontspec{Berthold Baskerville Book}[Path=fonts/]\thepage
    }%
  }
  \fancyfoot[LE]{\hspace{-4.5bp}\makebox[0pt][r]{\fontspec{Berthold
  Baskerville Book}[Path=fonts/]\thepage}}
}

% --------------------------------------

% ----------------------------------------------------------
% 自定义命令
% ----------------------------------------------------------

% --- 副标题 (vicetitle) ---
\newcommand{\vicetitle}[1]{%
  \vspace{-3.0bp}%
  \begin{center}%
    \begin{minipage}[t]{\textwidth}%
      \CJKfamily{fs}%
      \fontsize{10.5bp}{10.5bp}\selectfont\ziju{0.010}%
      \centering%
      (#1)
    \end{minipage}%
    \vspace{18.5bp}%
  \end{center}%
}

% --- 引言 (quote) ---
\renewenvironment{quote}
{%
  \list{}{\rightmargin=0pt \leftmargin=0pt}%
\item[]\relax\vspace{0.2em}
  \fontsize{9.0bp}{9.75bp}\selectfont
  \hspace*{2em}%
}
{%
  \vspace{-0.2em}\endlist
}

% --- 落款 (closing) ---
\newcommand{\closing}[2]{%
  \vspace{10.5bp}%
  \begin{flushright}%
    \CJKfamily{fs}%
    \fontsize{9.0bp}{9.5bp}\selectfont\ziju{0.010}%
    #1
    \if\relax#2\relax%
    \else%
    \par#2%
    \fi%
  \end{flushright}%
}

% --- 信息栏 (info) ---
\newcommand{\info}[4]{%
  \vspace{-4.0bp}%
  \begin{center}%
    \begin{minipage}[t]{\textwidth}%
      \begin{minipage}[t]{0.40\textwidth}%
        \begin{minipage}[t]{\textwidth}%
          \CJKfamily{ss}%
          \fontsize{9.0bp}{9.5bp}\selectfont\ziju{0.010}%
          #1%
        \end{minipage}%
        \\%
        \raisebox{-10bp}{%
          \begin{minipage}[t]{\textwidth}%
            \CJKfamily{ss}%
            \fontsize{9.0bp}{9.5bp}\selectfont\ziju{0.010}%
            #2%
          \end{minipage}%
        }%
      \end{minipage}%
      \hfill%
      \begin{minipage}[t]{0.40\textwidth}%
        \begin{minipage}[t]{\textwidth}%
          \CJKfamily{ss}%
          \fontsize{9.0bp}{9.5bp}\selectfont\ziju{0.010}%
          #3%
        \end{minipage}%
        \\%
        \raisebox{-10bp}{%
          \begin{minipage}[t]{\textwidth}%
            \CJKfamily{ss}%
            \fontsize{9.0bp}{9.5bp}\selectfont\ziju{0.010}%
            #4%
          \end{minipage}%
        }%
      \end{minipage}%
    \end{minipage}%
  \end{center}%
}

% --- 占位框 (todo) ---
\newcommand{\todo}{%
  \par\vspace{12bp}%
  \noindent%
  \makebox[\textwidth][c]{%
    \begin{tikzpicture}%
      \node[%
        draw,%
        rectangle,%
        line width=1pt,%
        minimum height=5\baselineskip,%
        text width=0.8\textwidth,%
        align=center%
      ]%
      {这里应该添加一张图像或者一段文字};%
    \end{tikzpicture}%
  }%
  \par\vspace{4bp}%
}

% --- 居中盒子 (centerbox) ---
\newcommand{\centerbox}[1]{%
  \par\removelastskip\vspace{\dimexpr\baselineskip - 1em\relax}%
  \noindent%
  \makebox[\textwidth][c]{%
    \begin{minipage}{\textwidth}%
      \centering%
      #1%
    \end{minipage}%
  }%
  \par\removelastskip\vspace{\dimexpr\baselineskip - 1em\relax}%
}

% --- 图像盒子 (imagebox) ---
\newcommand{\imagebox}[1]{%
  \par\vspace{\dimexpr\baselineskip - 1em\relax}%
  \noindent%
  \makebox[\textwidth][c]{%
    \includegraphics[width=\textwidth]{#1}%
  }%
  \par\vspace{\dimexpr\baselineskip - 1.8em\relax}%
}

% --- 字体盒子 (fontbox) ---
\newcommand{\fontbox}[1]{%
  \raisebox{0.15ex}{#1}%
}

% --- 悬挂段落 (indentpara) ---
\newcommand{\indentpara}[1]{%
  \noindent\hangindent=#1 \hangafter=0%
}

% ----------------------------------------------------------
% 目录与标题
% ----------------------------------------------------------

\ctexset{
  tocdepth    = 10,
  secnumdepth = 6,
}

\renewcommand{\contentsname}{%
  \CJKfamily{xb}%
  \fontsize{16.5bp}{16.5bp}\selectfont\ziju{1.500}%
  目录%
}

% --------------------------------------
% part
% --------------------------------------

\titlecontents{part}[0bp]%
{%
  \fontspec{Berthold Baskerville Book}[Path=fonts/]%
  \fontsize{14.5bp}{14.5bp}\selectfont\ziju{0.100}%
  \centering%
  \vspace{7.5bp}%
}%
{}%
{}%
{}[\vspace{5.5bp}]%

\newcommand\partbox{%
  \centering%
  \begin{minipage}{0.80\textwidth}%
    \fontspec{Berthold Baskerville Book}[Path=fonts/]%
    \fontsize{32.0bp}{32.0bp}\selectfont\ziju{0.100}%
    \centering%
  }

\titleformat{\part}[display]%
  {\partbox}{}{0bp}%
  {}[%
\end{minipage}]

\ctexset{
  part / name   = ,
  part / number = \hspace{-1em},
}

\ctexset{
  part / fixskip    = true,
  part / afterskip  = ,
  part / beforeskip = ,
}

% --------------------------------------
% chapter
% --------------------------------------

\titlecontents{chapter}[0bp]%
{%
  \CJKfamily{xb}%
  \fontsize{9.5bp}{9.5bp}\selectfont\ziju{0.150}%
  \vspace{3bp}%
}%
{}%
{}%
{\hspace{3bp}\titlerule*[.4em]{$\cdot$}\contentspage}

\newcommand\chapterbox{%
  \centering%
  \begin{minipage}{0.80\textwidth}%
    \modifiedmodifiedxb%
    \fontsize{16.5bp}{18.0bp}\selectfont\ziju{0.050}%
    \centering%
  }

\titleformat{\chapter}[display]%
  {\chapterbox}{}{0bp}%
  {}[%
\end{minipage}]

\titlespacing{\chapter}%
{0bp}%
{0bp}%
{18.0bp}%

\ctexset{
  chapter / fixskip    = true,
  chapter / afterskip  = ,
  chapter / beforeskip = ,
}

% --------------------------------------
% chapterx
% --------------------------------------

\titleclass{\chapterx}{straight}[\part]

\newcounter{chapterx}[part]
\renewcommand\thechapterx{\hspace{-1em}}

\titlecontents{chapterx}[0bp]%
{%
  \CJKfamily{fs}%
  \fontsize{10.5bp}{10.5bp}\selectfont\ziju{0.050}%
  \vspace{3bp}%
}%
{}%
{}%
{\hspace{3bp}\titlerule*[.4em]{$\cdot$}\contentspage}

\newcommand\chapterxbox{%
  \centering%
  \begin{minipage}{0.80\textwidth}%
    \modifiedmodifiedxb%
    \fontsize{16.5bp}{18.0bp}\selectfont\ziju{0.050}%
    \centering%
  }

\titleformat{\chapterx}[display]%
  {\chapterxbox}{}{0bp}%
  {}[%
\end{minipage}]

\titlespacing{\chapterx}%
{0bp}%
{0bp}%
{18.0bp}%

% --------------------------------------
% section
% --------------------------------------

\titlecontents{section}[2em]%
{%
  \CJKfamily{fs}%
  \fontsize{9.5bp}{9.5bp}\selectfont\ziju{-0.050}%
  \vspace{3bp}%
}%
{\contentspush{\thecontentslabel\hspace{1em}}}%
{}%
{\hspace{3bp}\titlerule*[.4em]{$\cdot$}\contentspage}

\newcommand\sectionbox{%
  \centering%
  \begin{minipage}{0.50\textwidth}%
    \modifiedmodifiedxh%
    \fontsize{14.5bp}{15.5bp}\selectfont\ziju{0.025}%
    \centering%
  }

\titleformat{\section}[display]%
  {\sectionbox}%
  {\ziju{0.5}第\chinese{section}篇}%
  {5.0bp}%
  {}%
  [%
\end{minipage}]

\ctexset{
  section / name   = {第,篇},
  section / number = \chinese{section},
}

\ctexset{
  section / fixskip    = true,
  section / afterskip  = ,
  section / beforeskip = ,
}

\titlespacing{\section}%
{0bp}%
{28.0bp}%
{26.0bp}%

% --------------------------------------
% subsection
% --------------------------------------

\titlecontents{subsection}[2em]%
{%
  \CJKfamily{ss}%
  \fontsize{9.0bp}{9.0bp}\selectfont\ziju{0.010}%
  \vspace{3bp}%
}%
{\contentspush{\thecontentslabel\hspace{1em}}}%
{}%
{\hspace{3bp}\titlerule*[.4em]{$\cdot$}\contentspage}

\newcommand\subsectionbox{%
  \centering%
  \begin{minipage}{0.70\textwidth}%
    \modifiedmodifiedss%
    \fontsize{14.0bp}{15.0bp}\selectfont\ziju{0.025}%
    \centering%
  }

\titleformat{\subsection}[display]%
  {\subsectionbox}%
  {\ziju{0.5}第\chinese{subsection}章}%
  {5.0bp}%
  {}%
  [%
\end{minipage}]

\ctexset{
  subsection / name   = {第,章},
  subsection / number = \chinese{subsection},
}

\ctexset{
  subsection / fixskip    = true,
  subsection / afterskip  = ,
  subsection / beforeskip = ,
}

\titlespacing{\subsection}%
{0bp}%
{14.5bp}%
{13.0bp}%

% --------------------------------------
% subsubsection
% --------------------------------------

\titlecontents{subsubsection}[4em]%
{%
  \CJKfamily{ss}%
  \fontsize{9.0bp}{9.0bp}\selectfont\ziju{0.010}%
  \vspace{3bp}%
}%
{\contentspush{\thecontentslabel\hspace{1em}}}%
{}%
{\hspace{3bp}\titlerule*[.4em]{$\cdot$}\contentspage}

\newcommand\subsubsectionbox[1]{%
  \begin{center}%
    \begin{minipage}{0.60\textwidth}%
      \modifiedmodifiedss%
      \fontsize{12.5bp}{14.5bp}\selectfont\ziju{0.025}%
      \centering%
      #1%
    \end{minipage}%
  \end{center}%
}

\ctexset{
  subsubsection / name       = {,.},
  subsubsection / number     = \arabic{subsubsection},
  subsubsection / format     = \subsubsectionbox,
  subsubsection / aftername  = ,
}

\ctexset{
  subsubsection / fixskip    = true,
  subsubsection / afterskip  = ,
  subsubsection / beforeskip = ,
}

\titlespacing{\subsubsection}%
{0bp}%
{19.5bp}%
{9.0bp}%

% --------------------------------------
% paragraph
% --------------------------------------

\titlecontents{paragraph}[6em]%
{%
  \CJKfamily{ss}%
  \fontsize{9.0bp}{9.0bp}\selectfont\ziju{0.010}%
  \vspace{3bp}%
}%
{\contentspush{\thecontentslabel\hspace{1em}}}%
{}%
{\hspace{3bp}\titlerule*[.4em]{$\cdot$}\contentspage}

\newcommand\paragraphbox[1]{%
  \begin{center}%
    \begin{minipage}{0.80\textwidth}%
      \modifiedmodifiedss%
      \fontsize{12.5bp}{12.5bp}\selectfont\ziju{0.010}%
      \centering%
      #1%
    \end{minipage}%
  \end{center}%
}

\ctexset{
  paragraph / name       = {,.},
  paragraph / number     = \Alph{paragraph},
  paragraph / format     = \paragraphbox,
  paragraph / aftername  = ,
}

\ctexset{
  paragraph / fixskip    = true,
  paragraph / afterskip  = ,
  paragraph / beforeskip = ,
}

\titlespacing{\paragraph}%
{0bp}%
{18.0bp}%
{9.0bp}%

% --------------------------------------
% subparagraph
% --------------------------------------

\titlecontents{subparagraph}[8em]%
{%
  \CJKfamily{ss}%
  \fontsize{9.0bp}{9.0bp}\selectfont\ziju{0.010}%
  \vspace{3bp}%
}%
{\contentspush{\thecontentslabel\hspace{1em}}}%
{}%
{\hspace{3bp}\titlerule*[.4em]{$\cdot$}\contentspage}

\newcommand\subparagraphbox[1]{%
  \begin{center}%
    \begin{minipage}{0.80\textwidth}%
      \CJKfamily{ht}%
      \fontsize{10.5bp}{10.5bp}\selectfont\ziju{0.010}%
      \centering%
      #1%
    \end{minipage}%
  \end{center}%
}

\ctexset{
  subparagraph / name         = {\CJKfamily{ss}(,\CJKfamily{ss})},
  subparagraph / number       = \arabic{subparagraph},
  subparagraph / format       = \subparagraphbox,
  subparagraph / numberformat = \bf,
  subparagraph / aftername    = ,
}

\ctexset{
  subparagraph / fixskip      = true,
  subparagraph / afterskip    = ,
  subparagraph / beforeskip   = ,
}

\titlespacing{\subparagraph}%
{0bp}%
{0bp}%
{-1bp}%

% ------------------------------------------------------------------------------
% 重命名
%
% \tyear          -> 年份标题 (可选)
% \tdoc           -> 文献标题
% \tpart          -> 篇 (可选)
% \tchapter       -> 章
% \tsection       -> 节 (如: 1.)
% \tsubsection    -> 子节 (如: A.)
% \tsubsubsection -> 小节 (如: (1))
% ------------------------------------------------------------------------------

\newcommand{\tyear}[1]{%
  \cleardoublepage%
  \thispagestyle{empty}%
  \part{#1}\label{#1}%
}

\newcommand{\tdoc}[1]{%
  \cleardoublepage%
  \thispagestyle{empty}%
  \let\oldthispagestyle\thispagestyle%
  \renewcommand{\thispagestyle}[1]{}%
  \chapter{#1}\label{#1}%
  \let\thispagestyle\oldthispagestyle%
}

\newcommand{\tpart}[1]{%
  \cleardoublepage%
  \section{#1}\label{#1}%
}

\newcommand{\tpartnonum}[1]{%
  \cleardoublepage%
  \section*{#1}\label{#1}%
  \addcontentsline{toc}{section}{#1}%
  \partmark{#1}%
}

\newcommand{\tchapter}[1]{%
  \clearpage%
  \subsection{#1}\label{#1}%
}

\newcommand{\tchapternonum}[1]{%
  \clearpage%
  \subsection*{#1}\label{#1}%
  \addcontentsline{toc}{subsection}{#1}%
  \chaptermark{#1}%
}

\newcommand{\tsection}[1]{%
  \subsubsection{#1}\label{#1}%
}

\newcommand{\tsectionnonum}[1]{%
  \subsubsection*{#1}\label{#1}%
  \addcontentsline{to}{subsubsection}{#1}%
  \sectionmark{#1}%
}

\newcommand{\tsubsection}[1]{%
  \paragraph{#1}\label{#1}%
}

\newcommand{\tsubsectionnonum}[1]{%
  \paragraph*{#1}\label{#1}%
  \addcontentsline{toc}{paragraph}{#1}%
}

\newcommand{\tsubsubsection}[1]{%
  \subparagraph{#1}\label{#1}%
}

\newcommand{\tsubsubsectionnonum}[1]{%
  \subparagraph*{#1}\label{#1}%
  \addcontentsline{toc}{subparagraph}{#1}%
}


% ------------------------------------------------------------------------------
% 正文
% ------------------------------------------------------------------------------

\begin{document}

% --------------------------------------

% ------------------
% 封面
% ------------------

\cleardoublepage%
\thispagestyle{empty}%
{%
  \vspace*{\fill}%
  \begin{center}%
    \vspace{-12.5em}
    {\CJKfamily{coverxb}\fontsize{38bp}{47.5bp}\selectfont\ziju{0.200} 剩余价值理论\par}
  \end{center}
  \vspace*{\fill}%
}
\cleardoublepage%

% ------------------
% 宣言
% ------------------

\cleardoublepage%
\thispagestyle{empty}%
\definecolor{deepred}{rgb}{0.6, 0, 0}
{\color{deepred}\CJKfamily{xb}\fontsize{14bp}{14.5bp}\selectfont\ziju{0.100}%
  \center 全世界无产者,联合起来!
\par}

\frontmatter
\pagenumbering{arabic}


% --------------------------------------

\xeCJKsetup{CJKglue={\hskip 0bp plus 0.02\baselineskip minus 0.000\baselineskip}}
\setlength{\parskip}{0bp}

\clearpage
\fancyhf{}
\renewcommand{\headrulewidth}{0bp}

\fancyhead[RO]{\small\CJKfamily{xh}\scalebox{0.9}[1.0]{\ziju{1.5pt}{目录}}}
\fancyhead[LE]{\small\CJKfamily{xh}\scalebox{0.9}[1.0]{\ziju{1.5pt}{目录}}}
\fancyfoot[RO]{%
  \makebox[0pt][l]{%
    \hspace{3.5bp}%
    \fontspec{Berthold Baskerville Book}[Path=fonts/]\thepage
  }%
}
\fancyfoot[LE]{\hspace{-4.5bp}\makebox[0pt][r]{\fontspec{Berthold Baskerville Book}[Path=fonts/]\thepage}}
\pagestyle{fancy}

\tableofcontents

\mainmatter

\clearpage
\fancyhf{}
\renewcommand{\headrulewidth}{0bp}

\fancyhead[RO]{\rightmark}
\fancyhead[LE]{\leftmark}
\fancyfoot[RO]{%
  \makebox[0pt][l]{%
    \hspace{3.5bp}%
    \fontspec{Berthold Baskerville Book}[Path=fonts/]\thepage
  }%
}
\fancyfoot[LE]{\hspace{-4.5bp}\makebox[0pt][r]{\fontspec{Berthold Baskerville Book}[Path=fonts/]\thepage}}
\pagestyle{fancy}

\markboth{}{}

% --------------------------------------

\tdoc{《剩余价值理论》}

\tpartnonum{《剩余价值理论》第一册}

\tchapternonum{[《剩余价值理论》手稿目录]}

\indentpara{0em}[\endnote{《剩余价值理论》是马克思的主要著作《资本论》的第四卷。马克思把《资本论》的前三卷称为理论部分,把第四卷称为历史部分、历史批判部分或历史文献部分。在这一卷中,马克思围绕着剩余价值理论这个政治经济学的核心问题,对各派资产阶级经济学家的理论进行了系统的、历史的分析批判,同时以论战的形式阐述了自己的政治经济学理论的许多重要方面。马克思从十九世纪四十年代起就开始研究政治经济学,并计划写一部批判现存制度和资产阶级政治经济学的巨著。经过长期系统研究,于 1857—1858 年写了一部经济学手稿,在这个手稿的基础上于 1859 年出版了《政治经济学批判》(第一分册)。从 1861 年 8 月到 1863 年 7 月,又写了一部篇幅很大的手稿。这部手稿的大部分,也是整理得最细致的部分,构成《剩余价值理论》。手稿的其余部分,即理论部分,后来经马克思重新修改和补充,形成了《资本论》前三卷的内容,而《剩余价值理论》这一历史部分没有重新加工,仍保持着原来的样子。马克思生前出版了《资本论》第一卷。他逝世后,恩格斯整理出版了《资本论》第二卷和第三卷,但是没有来得及整理出版《资本论》第四卷即《剩余价值理论》。1905—1910 年卡尔·考茨基编辑出版了《剩余价值理论》,他对马克思的手稿做了许多删改和变动。1954—1961 年按马克思的手稿次序编辑出版了《剩余价值理论》俄文新版本;1956—1962 年出版了该书德文新版本;1962—1964 年则作为《马克思恩格斯全集》俄文第二版第二十六卷(共三册)出版。《剩余价值理论》的章节标题大部分是由俄文版编者拟定的。编者加的标题和文字,用方括号[]标出。马克思手稿中使用的方括号则改用花括号\fontbox{~\{} \fontbox{\}~}。马克思手稿的稿本编号和页码,一律用方括号标出,括号中的罗马数字表示稿本编号,阿拉伯数字表示页码。——第 1 页。}VI—219b]第 VI 本目录:\endnote{《剩余价值理论》手稿的这一目录是马克思写在 1861—1863 年手稿第 VI—XV 各稿本的封面上的。其中有几本的目录比正文先写,这从后来写完相应稿本的正文时对目录所作的修改和删节中可以看出。第 XIV 本封面上的目录并不符合稿本的实际内容:这一目录是第 XIV、XV 和 XVIII 本中所完成的计划。——第 3 页。}

\indentpara{2em}(5)剩余价值理论\endnote{马克思在《剩余价值理论》这一标题之前写了阿拉伯数字“5”。这表示关于资本的研究的第一章第五节,即最后一节;马克思打算把这一研究作为论述商品和货币的《政治经济学批判》第一分册的直接继续来发表。在这第五节之前,在手稿第 I—V 本中有三节概述:(1)货币转化为资本,(2)绝对剩余价值,(3)相对剩余价值。在第 V 本第 184 页,马克思指出:“在相对剩余价值之后,应当考察绝对剩余价值和相对剩余价值两者的结合。”这一考察本应构成第四节,但当时并未写成。马克思还没有写完第三节就马上转写第五节,即《剩余价值理论》。——第 3 页。}

\indentpara{4em}(a)詹姆斯·斯图亚特爵士

\indentpara{4em}(b)重农学派

\indentpara{4em}(c)亚·斯密[VI—219b]

\indentpara{0em}[VII—272b][第 VII 本目录]

\indentpara{2em}(5)剩余价值理论

\indentpara{4em}(c)亚·斯密(续篇)

\indentpara{6em}(研究年利润和年工资怎样才能购买一年内生产的、除利润和工资外还包括不变资本的商品)[VII—272b]

\indentpara{0em}[VIII—331b][第 VIII 本目录]

\indentpara{2em}(5)剩余价值理论

\indentpara{4em}(c)亚·斯密(结尾)\endnote{实际上这并不是论斯密这一节的“结尾”,而只是“续篇”。这一节的结尾部分在下一本即手稿第 IX 本中。——第 3 页。}[VIII—331b]

\indentpara{0em}[IX—376b][第 IX 本目录]

\indentpara{2em}(5)剩余价值理论

\indentpara{4em}(c)亚·斯密。结尾

\indentpara{4em}(d)奈克尔[IX—376b]

\indentpara{0em}[X—421c][第 X 本目录]

\indentpara{2em}(5)剩余价值理论

\indentpara{6em}插入部分。魁奈的经济表

\indentpara{4em}(e)兰盖

\indentpara{4em}(f)布雷

\indentpara{4em}(g)洛贝尔图斯先生。插入部分。新的地租理论[X—421c]

\indentpara{0em}[XI—490a][第 XI 本目录]

\indentpara{2em}(5)剩余价值理论

\indentpara{4em}(g)洛贝尔图斯

\indentpara{6em}插入部分。评所谓李嘉图规律的发现史

\indentpara{4em}(h)李嘉图

\indentpara{6em}李嘉图和亚·斯密的费用价格理论(批驳部分)

\indentpara{6em}李嘉图的地租理论

\indentpara{6em}级差地租表及其说明[XI—490a]

\indentpara{0em}[XII—580b][第 XII 本目录]

\indentpara{2em}(5)剩余价值理论

\indentpara{4em}(h)李嘉图

\indentpara{6em}级差地租表及其说明

\indentpara{6em}(考察生活资料和原料的价值——以及机器的价值——的变动对资本有机构成的影响)

\indentpara{6em}李嘉图的地租理论

\indentpara{6em}亚·斯密的地租理论

\indentpara{6em}李嘉图的剩余价值理论

\indentpara{6em}李嘉图的利润理论[XII—580b]

\indentpara{0em}[XIII—670a][第 XIII 本目录]

\indentpara{2em}(5)剩余价值理论及其他

\indentpara{4em}(h)李嘉图

\indentpara{6em}李嘉图的利润理论

\indentpara{6em}李嘉图的积累理论。对这个理论的批判。(从资本的基本形式得出危机)

\indentpara{6em}李嘉图的其他方面。论李嘉图这一节的结尾(约翰·巴顿)

\indentpara{4em}(i)马尔萨斯[XIII—670a]

\indentpara{0em}[XIV—771a][第 XIV 本目录和《剩余价值理论》最后几章的计划]

\indentpara{2em}(5)剩余价值理论

\indentpara{4em}(i)马尔萨斯

\indentpara{4em}(k)李嘉图学派的解体(托伦斯、詹姆斯·穆勒、普雷沃、几部论战著作、麦克库洛赫、威克菲尔德、斯特林、约·斯·穆勒)

\indentpara{4em}(l)政治经济学家的反对派\endnote{关于政治经济学家的反对派一章在手稿第 XIV 本中仅仅是开始。这一章的续篇在第 XV 本前半部分。——第 5 页。}

\indentpara{6em}(政治经济学家的反对派布雷)\endnote{布雷《劳动中的不公正现象及其消除办法》一书的摘录和马克思所加的为数不多的评语包含在手稿第 X 本中。——第 5 页。}

\indentpara{4em}(m)拉姆赛

\indentpara{4em}(n)舍尔比利埃

\indentpara{4em}(o)理查·琼斯。\endnote{论拉姆赛、舍尔比利埃和理·琼斯的几章包含在手稿第 XVIII 本中。——第 5 页。}(这第五部分结束)

\indentpara{6em}补充部分:收入及其源泉\endnote{马克思在手稿第 XV 本后半部分论述了收入及其源泉,在这方面揭示了庸俗政治经济学的阶级根源和认识论根源。这个“补充部分”马克思后来决定放在《资本论》第三部分,这从他在 1863 年 1 月拟定的这一部分的计划可以看出;按照这一计划,第九章的标题应该是《收入及其源泉》(见本册第 447 页)。——第 5 页。}[XIV—771a]

\indentpara{0em}[XV—862a][第 XV 本目录]

\indentpara{2em}(5)剩余价值理论

\indentpara{4em}(1)以李嘉图理论为依据的无产阶级反对派

\indentpara{4em}(2)莱文斯顿。结尾\endnote{论莱文斯顿一节是在前一本(手稿第 XIV 本)第 861 页开始的。第 XIV 本中在这一节之前有标以数码“1”的一节,即论匿名小册子《根据政治经济学基本原理得出的国民困难的原因及其解决办法》。——第 6 页。}

\indentpara{4em}(3)和(4)霍吉斯金\endnote{论霍吉斯金一节的结尾包含在手稿第 XVIII 本中(第 1084—1086 页)。——第 6 页。}

\indentpara{4em}(现存财富同生产运动的关系)

\indentpara{4em}所谓积累不过是流通现象(储备等等是流通的蓄水池)

\indentpara{4em}(复利;根据复利说明利润率的下降)

\indentpara{6em}庸俗政治经济学\endnote{马克思在手稿第 XV 本中研究收入及其源泉的问题时,对庸俗政治经济学进行了分析。他在这一本第 935 页注明参看“论庸俗经济学家一节”,即他的著作中尚未完成的一章,说他在这一章里“将回过头来谈”顺便涉及的蒲鲁东和巴师夏之间的论战。这一提示表明马克思打算专门写一章来批判庸俗政治经济学,但是没有写成。手稿第 XVIII 本中,马克思在结束对霍吉斯金观点的分析并提到后者对资产阶级辩护士的理论的反驳时,注明:“要在论庸俗经济学家一章中谈这一点”(第 1086 页)。这句话也证明马克思打算以后专门写一章来论述庸俗政治经济学。在 1863 年 1 月拟定的《资本论》第三部分的计划中,倒数第二章即第十一章的标题是《庸俗政治经济学》(见本册第 447 页)。——第 6 页。}

\indentpara{4em}(生息资本在资本主义生产基础上的发展)

\indentpara{4em}(生息资本和商业资本同产业资本的关系。更为古老的形式。派生的形式)

\indentpara{4em}(高利贷。路德等等)\endnote{马克思在手稿第 XV 本封面上写下了这一本的目录,目录中的某些标题是写在旁边或上面的。在本卷发表的目录中,这些标题是按照符合稿本实际内容的次序排列的。——第 6 页。}[XV—862a]

\tchapternonum{[总的评论]}

[VI—220]所有经济学家都犯了一个错误:他们不是就剩余价值的纯粹形式,不是就剩余价值本身,而是就利润和地租这些特殊形式来考察剩余价值。由此必然会产生哪些理论谬误,这将在第三章中得到更充分的揭示,那里要分析以利润形式出现的剩余价值所采取的完全转化了的形式。\endnote{马克思这里说的“第三章”是指关于“资本一般”的研究的第三部分。这一章的标题应为:《资本的生产过程和流通过程的统一,或资本和利润》。以后(例如,见第 IX 本第 398 页和第 XI 本第 526 页)马克思不用“第三章”而用“第三篇”(《dritterAbschnitt》)。后来他就把这第三章称作“第三册”(例如,在 1865 年 7 月 31 日给恩格斯的信中)。关于“资本一般”的研究的“第三章”马克思是在第 XVI 本开始的。从这“第三章”或“第三篇”的计划草稿(见本册第 447 页)中可以看出,马克思打算在那里写两篇专门关于利润理论的历史补充部分。但是马克思在写作《剩余价值理论》的过程中,就已在自己的这一历史批判研究的范围内,详细地批判分析了各种资产阶级经济学家对利润的看法。因此,马克思在《剩余价值理论》中,特别是在这一著作的第二册和第三册中,就已进一步更充分地揭示了由于把剩余价值和利润混淆起来而产生的理论谬误。——第 7、87、272 页。}

\tchapternonum{[第一章]詹姆斯·斯图亚特爵士}

\vicetitle{[区分“让渡利润”和财富的绝对增加]}

在重农学派以前,剩余价值——即利润,利润形式的剩余价值——完全是用\textbf{交换},用商品高于它的价值出卖来解释的。詹姆斯·斯图亚特爵士,总的说来,并没有超出这种狭隘看法;甚至可以更确切地说,正是斯图亚特科学地复制了这种看法。我说:“科学地”复制。因为斯图亚特不同意这种幻想:单个资本家由于商品高于它的价值出卖而获得的剩余价值,就是新财富的创造。因此,他把\textbf{绝对}利润和\textbf{相对}利润区分开来:

\begin{quote}“\textbf{绝对利润}对谁都不意味着亏损;它是劳动、勤勉或技能的\textbf{增进}的结果,它能引起\textbf{社会财富}的扩大或增加……\textbf{相对利润}对有的人意味着亏损;它表示财富的天平在有关双方之间的摆动,但并不意味着\textbf{总基金的任何增加}……\textbf{混合}利润很容易理解:这种利润……一部分是\textbf{相对的},一部分是\textbf{绝对的}……二者能够不可分割地存在于同一交易中。”(《政治经济学原理研究》,由其子詹姆斯·斯图亚特爵士将军汇编的《詹姆斯·斯图亚特爵士著作集》(六卷集)第 1 卷,1805 年伦敦版第 275—276 页)\end{quote}

\textbf{绝对}利润是由“劳动、勤勉和技能的增进”产生的。究竟它怎样由这种增进产生,斯图亚特并没有想弄清楚。他所加的关于这个利润能引起“\textbf{社会财富}”的扩大和增加的这句话,看来,可以使人得出这样的结论:斯图亚特所指的,仅仅是由劳动生产力的发展造成的使用价值量的增加,他完全离开总是以交换价值的增加为前提的资本家的利润来考察这个绝对利润。这样的解释完全被他进一步的叙述证实了。

他是这样说的:

\begin{quote}“在商品的\textbf{价格}中,我认为有两个东西是实际存在而又彼此\textbf{完全不同}的:商品的\textbf{实际价值}和\textbf{让渡利润}。”(第 244 页)\end{quote}

可见,商品的价格包含着两个彼此完全不同的要素:第一,商品的\textbf{实际价值};第二,“\textbf{让渡利润}”,即让出或卖出商品时实现的利润。

[221]因此,这个“\textbf{让渡利润}”是由于商品的价格高于商品的实际价值而产生的,换句话说,是由于商品\textbf{高于}它的价值出卖而产生的。这里,一方的赢利总是意味着另一方的亏损。不会造成“总基金的增加”。利润——应该说是剩余价值——是相对的,并且归结为“财富的天平在有关双方之间的摆动”。斯图亚特自己舍弃可以用这种办法来说明剩余价值的看法。他的关于“财富的天平在有关双方之间的摆动”的理论,虽然丝毫没有触及剩余价值本身的性质和起源问题,但是对于考察剩余价值在不同阶级之间按利润、利息、地租这些不同项目进行的分配,有重要的意义。

从下面的引文中可以看出,斯图亚特认为,单个资本家的全部利润只限于这种“相对利润”,“让渡利润”。

\begin{quote}他说:“实际价值”决定于“该国一个劳动者平常……在一天、一周、一月……平均能够完成的”劳动“量”。第二,决定于“劳动者用以满足他个人的需要和……购置适合于他的职业的工具的生存资料和必要费用的价值;这些同样也必须平均计算”……第三,决定于“材料的价值”。(第 244—245 页)“如果这三项是已知的,产品的价格就确定了。它不能低于这三项的总和,即不能低于\textbf{实际价值。凡是超过实际价值的,就是厂主的利润}。这个利润将始终同\textbf{需求}成比例,因此它将随情况而变动。”(同上,第 245 页)“由此看来,为了促进制造业的繁荣,必须有大规模的需求……工业家是按照他们有把握取得的利润,来安排自己的开支和自己的生活方式的。”(同上,第 246 页)\end{quote}

从这里可以清楚地看出,“厂主”即单个资本家的利润,总是“相对利润”,总是“让渡利润”,总是由于商品的价格高于商品的实际价值,由于\textbf{商品高于它的价值出卖}而产生的。因此,如果一切商品都按它的\textbf{价值}出卖,那就不会有任何利润了。

关于这个问题,斯图亚特写了专门的一章,他详细地研究

\begin{quote}“利润怎样同生产费用结成一体”。(同上,第 3 卷第 11 页及以下各页)\end{quote}

一方面,斯图亚特抛弃了货币主义和重商主义体系的这样一种看法,即认为商品高于它的价值出卖以及由此产生的利润,形成剩余价值,造成财富的绝对增加;\authornote{其实,连货币主义也认为,这个利润不是在国内产生,而只是在同其他国家的交换中产生。重商主义体系只看到,这个价值表现为货币(金和银),因此剩余价值表现为用货币结算的贸易差额。}另一方面,他仍然维护它们的这样一种观点,即单个资本家的利润无非是价格超过[222]价值的这个余额——“让渡利润”,不过按照他的意见,这种利润只是\textbf{相对的},一方的赢利相当于另一方的亏损,因此,利润的运动归结为“财富的天平在有关双方之间的摆动”。

可见,在这方面,斯图亚特是货币主义和重商主义体系的\textbf{合理的}表达者。

在对资本的理解方面,他的功绩在于:他指出了生产条件作为一定阶级的财产同劳动能力\endnote{原文是:《Arbeitsvermögen》(“劳动能力”)。马克思在 1861—1863 年手稿中在绝大多数场合都用《Arbeitsvermögen》(“劳动能力”)这个术语,而没有用《Arbeitskraft》(“劳动力”)这个术语。在《资本论》第一卷中,马克思把这两个术语当作同一概念使用:“我们把劳动力或劳动能力,理解为人的身体即活的人体中存在的、每当人生产某种使用价值时就运用的体力和智力的总和。”(见马克思《资本论》第 1 卷第 4 章第 3 节)——第 13 页。}分离的过程是怎样发生的。斯图亚特十分注意资本的这个\textbf{产生过程};诚然,他还没有把这个过程直接理解为资本的产生过程,但是,他仍然把这个过程看成是大工业存在的条件。斯图亚特特别在农业中考察了这个过程,并且正确地认为,只是因为农业中发生了这个分离过程,真正的制造业才产生出来。在亚·斯密的著作里,是以这个分离过程已经完成为前提的。

(斯图亚特的书于 1767 年在伦敦[出版],\textbf{杜尔哥}的书[写于]1766 年,亚当·斯密的书——1775 年。)

\tchapternonum{[第二章]重农学派}

\tsectionnonum{[(1)把剩余价值的起源问题从流通领域转到生产领域。把地租看成剩余价值的唯一形式]}

重农学派的重大功绩在于,他们在资产阶级视野以内对\textbf{资本}进行了分析。正是这个功绩,使他们成为现代政治经济学的真正鼻祖。首先,他们分析了资本在劳动过程中借以存在并分解成的各种\textbf{物质组成部分}。决不能责备重农学派,说他们和他们所有的后继者一样,把资本存在的这些物质形式——工具、原料等等,当作跟它们在资本主义生产中出现时的社会条件脱离的资本来理解,简言之,不管劳动过程的社会形式如何,只从它们是一般劳动过程的要素这个形式来理解;从而,把生产的资本主义形式变成生产的一种永恒的自然形式。对于他们来说,生产的资产阶级形式必然以生产的自然形式出现。重农学派的巨大功绩是,他们把这些形式看成社会的生理形式,即从生产本身的自然必然性产生的,不以意志、政策等等为转移的形式。这是物质规律;错误只在于,他们把社会的一个特定历史阶段的物质规律看成同样支配着一切社会形式的抽象规律。

除了对资本在劳动过程中借以组成的物质要素进行这种分析以外,重农学派还研究了资本在流通中所采取的形式(固定资本、流动资本,不过重农学派用的是别的术语),并且一般地确定了资本的流通过程和再生产过程之间的联系。这一点在论流通那一章\endnote{指关于“资本一般”的研究的第二章,这一章最后发展成为《资本论》第二卷。《资本论》第二卷第十章(《关于固定资本和流动资本的理论。重农学派和亚当·斯密》)包含对重农学派关于固定资本和流动资本的观点的分析。在《社会总资本的再生产和流通》一篇的第十九章《前人对这个问题的阐述》中有专门论重农学派的一节。——第 16 页。}再谈。

在这两个要点上,亚·斯密继承了重农学派的遗产。他的功绩,在这方面,不过是把抽象范畴固定下来,对重农学派所分析的差别采用了更稳定的名称。

[223]我们已经看到\endnote{马克思指他的 1861—1863 年手稿第 II 本第 58—60 页(《货币转化为资本》一节,《转化过程的两个组成部分》一小节)。——第 16 页。},资本主义生产发展的基础,一般说来,是\textbf{劳动能力}这种属于工人的\textbf{商品}同劳动条件这种固着于资本形式并脱离工人而独立存在的商品相对立。劳动能力作为商品,它的\textbf{价值}规定具有极重要的意义。这个价值等于把再生产劳动能力所必需的生活资料创造出来的劳动时间,或者说,等于工人作为工人生存所必需的生活资料的价格。只有在这个基础上,才出现劳动能力的\textbf{价值}和这个劳动能力\textbf{所创造的价值}之间的差额,——任何别的商品都没有这个差额,因为任何别的商品的使用价值,从而它的使用,都不能提高它的\textbf{交换价值}或提高从它得到的交换价值。

因此,从事分析资本主义生产的现代政治经济学的基础,就是把\textbf{劳动能力的价值}看作某种固定的东西,已知的量,而实际上它在每一个特定的场合,也就是一个已知量。所以,\textbf{最低限度的工资}理所当然地构成重农学派的学说的轴心。虽然他们还不了解价值本身的性质,他们却能够确定最低限度的工资的概念,这是因为这个\textbf{劳动能力的价值}表现为必要生活资料的价格,因而表现为一定使用价值的总和。他们尽管没有弄清一般价值的性质,但仍然能够在他们的研究所必需的范围内,把劳动能力的价值理解为一定的量。其次,如果说,他们错误地把这个\textbf{最低限度}看作不变的量,在他们看来,这个量完全决定于自然,而不决定于本身就是一个变量的历史发展阶段,那末,这丝毫也不影响他们的结论的抽象正确性,因为劳动能力的价值和这个劳动能力所创造的价值之间的差额,同我们假定劳动能力的价值是大是小毫无关系。

重农学派把关于剩余价值起源的研究从流通领域转到直接生产领域,这样就为分析资本主义生产奠定了基础。

他们完全正确地提出了这样一个基本论点:只有创造\textbf{剩余价值}的劳动,即只有劳动产品中包含的价值超过生产该产品时消费的价值总和的那种劳动,才是\textbf{生产的}。既然原料和材料的价值是已知的,劳动能力的价值又等于最低限度的工资,那末很明显,这个剩余价值只能由工人向资本家提供的劳动超过工人以工资形式得到的劳动量的余额构成。当然,在重农学派那里,剩余价值还不是以这种形式出现的,因为他们还没有把一般价值归结为它的简单实体:劳动量,或劳动时间。

[224]自然,重农学派的表述方式必然决定于他们对价值性质的一般看法,按照他们的理解,价值不是人的活动(劳动)的一定的社会存在方式,而是由土地即自然所提供的物质以及这个物质的各种变态构成的。

劳动能力的\textbf{价值}和这个劳动能力\textbf{所创造的价值}之间的差额,也就是劳动能力使用者由于购买劳动能力而取得的剩余价值,无论在哪个\textbf{生产部门}都不如在\textbf{农业}这个最初的生产部门表现得这样显而易见,这样无可争辩。劳动者逐年消费的生活资料总量,或者说,他消费的物质总量,小于他所生产的生活资料总量。在工业中,一般既不能直接看到工人生产自己的生活资料,也不能直接看到他还生产超过这个生活资料的余额。在这里,过程以买卖为中介,以各种流通行为为中介,而要理解这个过程,就必须分析价值。在农业中,过程在生产出的使用价值超过劳动者消费的使用价值的余额上直接表现出来,因此,不分析价值,不弄清价值的性质,也能够理解这个过程。因此,在把价值归结为使用价值,又把使用价值归结为一般物质的情况下,也能够理解这个过程。所以在重农学派看来,农业劳动是唯一的\textbf{生产劳动},因为按照他们的意见,这是唯一\textbf{创造剩余价值}的劳动,而\textbf{地租}是他们所知道的\textbf{剩余价值的唯一形式}。他们认为,在工业中,工人并不增加物质的量:他只改变物质的形式。材料——物质总量——是农业供给他的。他诚然把价值加到物质上,但这不是靠他的劳动,而是靠他的劳动的生产费用,也就是靠他在劳动期间所消费的、等于他从农业得到的最低限度工资的生活资料总额。既然农业劳动被看成唯一的生产劳动,那末,把农业劳动同工业劳动区别开来的剩余价值形式,即\textbf{地租},就被看成剩余价值的唯一形式。

因此,在重农学派那里不存在资本的\textbf{利润}——真正的利润,而地租本身只不过是这种利润的一个分枝。重农学派认为利润只是一种较高的工资,这种工资由土地所有者支付,并且由资本家作为收入来消费(因此,它完全象普通工人所得的最低限度的工资一样,加入生产费用),它增大原料的价值,因为它加入资本家即工业家在生产产品、变原料为新产品时的消费费用。

因此,某些重农主义者,例如老米拉波,把\textbf{货币利息}形式的剩余价值——利润的另一分枝——称为违反自然的高利贷。相反,杜尔哥认为货币利息是正当的,因为货币资本家本来可以购买土地,即购买地租,所以他的货币资本应当使他得到他把这笔资本变成地产时所能得到的那样多的剩余价值。由此可见,根据这种看法,连货币利息也不是新创造的价值,不是剩余价值;这里只是说明土地所有者得到的剩余价值的一部分为什么会以利息形式流到货币资本家手里,正如用别的理由[225]说明这个剩余价值的一部分为什么会以利润形式流到工业资本家手里一样。按照重农学派的意见,既然\textbf{农业劳动}是唯一的生产劳动,是唯一创造剩余价值的劳动,那末,把农业劳动同其他一切劳动部门区别开来的\textbf{剩余价值形式},即\textbf{地租},就是\textbf{剩余价值的一般形式}。工业利润和货币利息只是地租依以进行分配的各个不同项目,地租按照这些项目以一定的份额从土地所有者手里转到其他阶级手里。这同从亚当·斯密开始的后来的政治经济学家所持的观点完全相反,因为这些政治经济学家正确地把\textbf{工业利润}看成剩余价值\textbf{最初}为资本占有的\textbf{形式},从而看成剩余价值的最初的一般形式,而把利息和地租仅仅解释为由工业资本家分配给剩余价值共同占有者各阶级的工业利润的分枝。

除了上面所说的理由,即农业劳动是一种使剩余价值的创造在物质上显而易见,并且可以不经过流通过程就表现出来的劳动,重农学派还有一些别的理由说明他们的观点。

\textbf{第一},在农业中,地租表现为第三要素,表现为一种在工业中或者根本不存在,或者只是转瞬即逝的剩余价值形式。这是超过剩余价值(超过利润)的剩余价值,因此是最显而易见和最引人注目的剩余价值形式,是二次方的剩余价值。

\begin{quote}粗俗的政治经济学家\textbf{卡尔·阿伦德}(《合乎自然的国民经济学》1845 年哈瑙版第 461—462 页)说:“农业以地租形式创造一种在工业和商业中遇不到的价值:一种在补偿全部支付了的工资和全部消耗了的资本利润之后剩下来的价值。”\end{quote}

\textbf{第二},如果撇开对外贸易(重农学派为了抽象地考察资产阶级社会,完全正确地这样做了,而且应当这样做),那末很明显,从事加工工业等等而完全脱离农业的工人(斯图亚特称之为“自由人手”)的数目,取决于农业劳动者所生产的超过自己消费的农产品的数量。

\begin{quote}“显然,不从事农业劳动而能生活的人的相对数,完全取决于土地耕种者的劳动生产率。”(\textbf{理查·琼斯}《论财富的分配》1831 年伦敦版第 159—160 页)\end{quote}

可见,农业劳动不仅对于农业领域本身的剩余劳动来说是自然基础(关于这一点见前面的一个稿本)\endnote{马克思指他的 1861—1863 年手稿第 III 本第 105—106 页,在那里他也顺便提到了重农学派(《绝对剩余价值》一节,《剩余劳动的性质》一小节)。——第 22 页。},而且对于其他一切劳动部门之变为独立劳动部门,从而对于这些部门中创造的剩余价值来说,也是自然基础;因此很明显,只要价值实体被认为是一定的具体劳动,而不是抽象劳动及其尺度即劳动时间,农业劳动就必定被看作是剩余价值的创造者。

[226]\textbf{第三},一切剩余价值,不仅相对剩余价值,而且绝对剩余价值,都是以一定的劳动生产率为基础的。如果劳动生产率只达到这样的发展程度:一个人的劳动时间只够维持他本人的生活,只够生产和再生产他本人的生活资料,那就没有任何剩余劳动和任何剩余价值,就根本没有劳动能力的价值和这个劳动能力所创造的价值之间的差额了。因此,剩余劳动和剩余价值的可能性要以一定的劳动生产率为条件,这个生产率使劳动能力能够创造出超过本身价值的新价值,能够生产比维持生活过程所必需的更多的东西。而且,正象我们在\textbf{第二}点已经看到的,这个生产率,这个作为出发前提的生产率阶段,必定首先存在于农业劳动中,因而表现为\textbf{自然的赐予,自然的生产力}。在这里,在农业中,自然力的协助——通过运用和开发自动发生作用的自然力来提高人的劳动力,从一开始就具有广大的规模。在工业中,自然力的这种大规模的利用是随着大工业的发展才出现的。农业的一定发展阶段,不管是本国的还是外国的,是资本发展的基础。就这点来说,绝对剩余价值同相对剩余价值是一致的。(连重农学派的大敌\textbf{布坎南}都用这一点来反对亚·斯密,力图证明,甚至在现代城市工业产生之前,已先有农业的发展。)

\textbf{第四},因为重农学派的功绩和特征在于,它不是从流通中而是从生产中引出价值和剩余价值,所以它同货币主义和重商主义体系相反,必然从这样的生产部门开始,这个生产部门一般可以同流通、交换脱离开来单独考察,并且是不以人和人之间的交换为前提,而只以人和自然之间的交换为前提的。

\tsectionnonum{[(2)重农学派体系的矛盾:这个体系的封建主义外貌和它的资产阶级实质;对剩余价值的解释中的二重性]}

从上述情况也就产生了重农学派体系的矛盾。

实际上这是第一个对资本主义生产进行分析,并把资本在其中被生产出来又在其中进行生产的那些条件当作生产的永恒自然规律来表述的体系。但是另一方面,这个体系宁可说是封建制度即土地所有权统治的资产阶级式的再现;而资本最先得到独立发展的工业部门,在它看来却是“非生产的”劳动部门,只不过是农业的附庸而已。资本发展的第一个条件,是土地所有权同劳动分离,是土地——这个劳动的最初条件——作为独立的力量,作为掌握在特殊阶级手中的力量,开始同自由劳动者相对立。因此,在重农学派的解释中,土地所有者表现为真正的资本家,即剩余劳动的占有者。可见,在这里,封建主义是从资产阶级生产的角度来加以表述和说明的,而农业则被解释成唯一进行资本主义生产即剩余价值生产的生产部门。这样,封建主义就具有了资产阶级的性质,资产阶级社会获得了封建主义的外观。

这个外观曾迷惑了魁奈医生的贵族出身的门徒们,例如守旧的怪人老\textbf{米拉波}。在那些眼光比较远大的重农主义体系[227]代表者那里,特别是在\textbf{杜尔哥}那里,这个外观完全消失了,重农主义体系就成为在封建社会的框子里为自己开辟道路的新的资本主义社会的表现了。因而,这个体系是同刚从封建主义中孵化出来的资产阶级社会相适应的。所以出发点是在法国这个以农业为主的国家,而不是在英国这个以工业、商业和航海业为主的国家。在英国,目光自然集中到流通过程,看到的是产品只有作为一般社会劳动的表现,作为货币,才取得价值,变成商品。因此,只要问题涉及的不是价值形式,而是价值量和价值增殖,那末在这里首先看到的就是“\textbf{让渡利润}”,即斯图亚特所描述的相对利润。但是,如果要证明剩余价值是在生产领域本身创造的,那末,首先必须从剩余价值不依赖流通过程就能表现出来的劳动部门即农业着手。因而这方面的首创精神,是在一个以农业为主的国家中表现出来的。在重农学派的前辈老作家中,已经可以零星地看到近似重农学派的思想,例如在法国的布阿吉尔贝尔那里就可以部分地看到。但是这些思想只有在重农学派那里,才成为标志着科学新阶段的体系。

农业劳动者只能得到最低限度的工资,即“最必需品”,而他们再生产出来的东西却多于这个“最必需品”,这个余额就是地租,就是由劳动的基本条件——自然——的所有者占有的\textbf{剩余价值}。因此,重农学派不是说:劳动者是超过再生产他的劳动能力所必需的劳动时间进行劳动的,所以他创造的价值高于他的劳动能力的价值,换句话说,他付出的劳动大于他以工资形式得到的劳动量。但是他们说:劳动者在生产时消费的使用价值的总和小于他所生产的使用价值的总和,因而剩下一个使用价值的余额。——如果他只用再生产自己的劳动能力所必需的时间来进行劳动,那就没有什么余额了。但是重农学派只抓住这样一点:土地的生产力使劳动者能够在一个工作日(假定为已知量)生产出多于他维持生活所必需消费的东西。这样一来,这个剩余价值就表现为\textbf{自然的赐予},在自然的协助下,一定量的有机物(种子、畜群)使劳动能够把更多的无机物变为有机物。

另一方面,不言而喻,这里是假定土地所有者作为资本家同劳动者相对立的。土地所有者向劳动者支付劳动能力的代价,——这种劳动能力是劳动者当作商品提供给他的,——而作为补偿,他不但得到一个等价物,而且占有这种劳动能力所创造的价值增殖额。在这个交换中,必须以劳动的物质条件和劳动能力本身彼此脱离为前提。出发点是封建土地所有者,但他表现为一个资本家,表现为一个纯粹的商品所有者,他使他用来同劳动交换的商品的价值增殖,并且不仅收回这些商品的等价物,还收回超过这个等价物的余额,因为他把劳动能力只当作商品来支付代价。他作为商品所有者而同自由工人相对立。换句话说,这个土地所有者实质上是资本家。在这方面重农主义体系也是对的,因为劳动者同土地和土地所有权的分离[228]是资本主义生产和资本的生产的基本条件。

因此,在这一体系中就产生了以下矛盾:它最先试图用对于别人劳动的占有来解释\textbf{剩余价值},并且根据商品交换来解释这种占有,但是在它看来,价值不是社会劳动的形式,剩余价值不是剩余劳动;价值只是使用价值,只是物质,而剩余价值只是自然的赐予,——自然还给劳动的不是既定量的有机物,而是较大量的有机物。一方面,地租,即土地所有权的实际经济形式,脱去了土地所有权的封建外壳,归结为超出工资之上的纯粹的剩余价值。另一方面,这个剩余价值——又按封建主义的精神——是从自然而不是从社会,是从对土地的关系而不是从社会关系引伸出来的。价值本身只不过归结为使用价值,从而归结为物质。而在这个物质中,重农学派所关心的只是量的方面,即生产出来的使用价值超过消费掉的使用价值的余额,因而只是使用价值相互之间的量的关系,只是它们的最终要归结为劳动时间的交换价值。

这一切都是资本主义生产初期的矛盾,那时资本主义生产正从封建社会内部挣脱出来,暂时还只能给这个封建社会本身以资产阶级的解释,还没有找到它本身的形式;这正象哲学一样,哲学最初在意识的宗教形式中形成,从而一方面它消灭宗教本身,另一方面从它的积极内容说来,它自己还只在这个理想化的、化为思想的宗教领域内活动。

因此,在重农学派本身得出的结论中,对土地所有权的表面上的推崇,也就变成了对土地所有权的经济上的否定和对资本主义生产的肯定。一方面,全部赋税都转到地租上,换句话说,土地所有权部分地被没收了——而这正是法国革命制定的法律打算实施的办法,也是李嘉图学派的充分发展的现代政治经济学\endnote{马克思指激进的李嘉图学派。这个学派从李嘉图的理论中得出了反对土地私有制存在的实际结论,建议把这一制度(全部或部分)转变为资产阶级国家所有制。属于这个激进的李嘉图学派的有詹姆斯·穆勒、约翰·斯图亚特·穆勒、希尔迪奇,在一定程度上也有舍尔比利埃。关于这一点见本卷第二册,马克思手稿第 458 页;第三册,马克思手稿第 791、1120 和 1139 页;并见《哲学的贫困》(《马克思恩格斯全集》中文版第 4 卷第 187 页)和马克思 1881 年 6 月 20 日给左尔格的信(见《马克思恩格斯全集》中文版第 35 卷第 190—194 页)。——第 26、42 页。}的最终结论。因为地租被认为是唯一的剩余价值,并且根据这一点,一切赋税都落到地租身上,所以对其他形式的收入课税,只不过是对土地所有权采取间接的、因而在经济上有害的、妨碍生产的课税办法。结果,赋税的负担,从而国家的各种干涉,都落不到工业身上,工业也就摆脱了国家的任何干涉。这样做,表面上是有利于土地所有权,不是为了工业的利益而是为了土地所有权的利益。

与此有关的是:自由放任\endnote{自由放任(原文是:laissezfaire,laissezaller,亦译听之任之)是重农学派的口号。重农学派认为,经济生活是受自然规律调节的,国家不得对经济事务进行干涉和监督;国家用各种规章进行干涉,不仅无益,而且有害;他们要求实行自由主义的经济政策。——第 27、42、162 页。},无拘无束的自由竞争,工业摆脱国家的任何干涉,取消垄断等等。按照重农学派的意见,既然工业什么也不创造,只是把农业提供给它的价值变成另一种形式;既然它没有在这个价值上增加任何新价值,只是把提供给它的价值以另一种形式作为等价物归还,那末,很自然,最好是这个转变过程不受干扰地、最便宜地进行,而要达到这一点,只有通过自由竞争,听任资本主义生产自行其是。这样一来,把资产阶级社会从建立在封建社会废墟上的君主专制下解放出来,就只是为了[229]已经变成资本家并一心一意想发财致富的封建土地所有者的利益。资本家成为仅仅为了土地所有者的利益的资本家,正象进一步发展了的政治经济学让资本家成为仅仅为了工人阶级的利益的资本家一样。

从上述一切可以看到,现代的经济学家,如出版了重农学派的著作和自己论述重农学派的得奖论文的欧仁·德尔先生,认为重农学派关于只有农业劳动才具有生产性、关于地租是剩余价值的唯一形式、关于土地所有者在生产体系中占杰出地位这些独特的论点,同重农学派的自由竞争的宣传、大工业和资本主义生产的原则毫无联系,只是偶然地凑合在一起,——他们这种看法是多么不了解重农学派。同时也就可以理解,这个体系的封建主义外观——完全象启蒙时代的贵族腔调——必然会使不少的封建老爷成为这个实质上是宣告在封建废墟上建立资产阶级生产制度的体系的狂热的拥护者和传播者。

\tsectionnonum{[(3)魁奈论社会的三个阶级。杜尔哥对重农主义理论的进一步发展:对资本主义关系作更深入分析的因素]}

现在我们来考察几段引文,一方面为了阐明上述论点,一方面为了给以证明。

在\textbf{魁奈}的《经济表的分析》一书中,国民由三个市民阶级组成:

\begin{quote}“\textbf{生产阶级}〈农业劳动者〉、\textbf{土地所有者阶级}”和“\textbf{不生产阶级}〈“所有从事农业以外的其他职务和其他工作的市民”〉”。\authornote{本卷引文中凡是尖括号〈〉和花括号\fontbox{~\{}\fontbox{\}~}内的话都是马克思加的。——译者注}(《重农学派》,欧仁·德尔出版,1846 年巴黎版第 1 部第 58 页)\end{quote}

只有农业劳动者才是生产阶级,创造剩余价值的阶级,土地所有者就不是。土地所有者阶级不是“不生产的”,因为它代表“剩余价值”,这个阶级的重要并不是由于它创造这个剩余价值,而完全是由于它占有这个剩余价值。

在\textbf{杜尔哥}那里,重农主义体系发展到最高峰。他的著作中某些地方甚至把“纯粹的自然赐予”看作\textbf{剩余劳动},另一方面,他用劳动者脱离劳动条件、劳动条件作为拿这些条件做买卖的那个阶级的财产同劳动者相对立这种情况,来说明工人提供的东西必须超过维持生活的工资。

说明只有农业劳动是生产劳动的第一个理由是:农业劳动是其他一切劳动得以独立存在的自然基础和前提。

\begin{quote}“他的〈土地耕种者的〉劳动,在社会不同成员所分担的各种劳动中占着首要地位……正象在社会分工以前,人为取得食物而必须进行的劳动,在他为满足自己的各种需要而不得不进行的各种劳动中占着首要地位一样。这不是在荣誉或尊严的意义上的首要地位;这是由\textbf{生理的必然性}决定的首要地位……土地耕种者的劳动使土地生产出超过他本人需要的东西,这些东西是社会其他一切成员用自己的劳动换来的工资的唯一基金。当后者利用从这种交换中得来的报酬再来购买土地耕种者的产品时,他们归还土地耕种者的〈在物质形式上〉,恰好只是他们原来得到的。这就是这两种劳动之间的本质[230]差别。”(《关于财富的形成和分配的考察》(1766 年),载于德尔出版的《杜尔哥全集》1844 年巴黎版第 1 卷第 9—10 页)\end{quote}

剩余价值究竟是怎样产生的呢\fontbox{?}它不是从流通中产生的,但是它在流通中实现。产品是按自己的价值出卖的,不是\textbf{高于}自己的价值出卖的。这里没有价格超过价值的余额。但是,因为产品按它的价值出卖,卖者就实现了剩余价值。这种情况所以可能,只是因为卖者本人对他所卖的价值没有全部支付过代价,换句话说,因为产品中包含着卖者没有支付过代价的、没有用等价物补偿的价值组成部分。农业劳动的情况正是这样。卖者出卖他没有买过的东西。杜尔哥最初把这个没有买过的东西描绘成“\textbf{纯粹的自然赐予}”。但是我们将会看到,这个“纯粹的自然赐予”在他那里,不知不觉地变成土地所有者没有买过而以农产品形式出卖的土地耕种者的剩余劳动。

\begin{quote}“土地耕种者的劳动一旦\textbf{生产出超过}他的需要\textbf{的东西},他就可以用这个余额——\textbf{自然给他的}超过他的劳动报酬的\textbf{纯粹的赐予}——去购买社会其他成员的劳动。后者向他出卖自己的劳动时所得到的只能维持生活;而土地耕种者除了自己的生存资料以外,还得到一种独立的和可以自由支配的财富,这是\textbf{他没有买却拿去卖的}财富。因此,他是财富(财富通过自己的流通使社会上一切劳动活跃起来)的唯一源泉,\textbf{因为他的劳动是唯一生产出超过劳动报酬的东西的劳动}。”(同上,第 11 页)\end{quote}

在这第一个解释中,第一,掌握了剩余价值的本质,就是说,剩余价值是卖者没有支付过等价物,即没有买过而拿去出卖时实现的价值。它是\textbf{没有支付过代价的价值}。但是第二,这个超过“\textbf{劳动报酬}”的余额被看成是“纯粹的自然赐予”,因为劳动者在他的工作日中所能生产的东西,比再生产他的劳动能力所必需的东西多,比他的工资多,这种情况一般地说就是自然的赐予,是取决于自然的生产率的。按照这第一个解释,全部产品还是归劳动者本人占有。但它分成两部分。第一部分形成劳动者的工资——他被看作是自己的雇佣劳动者,他把再生产他的劳动能力,维持他的生活所必需的那部分产品支付给自己。除此以外的第二部分是\textbf{自然的赐予},形成剩余价值。但是,只要抛开“土地耕种者-土地所有者”这个前提,而产品的两部分即工资和剩余价值分别属于不同的阶级,一部分属于雇佣劳动者,另一部分属于土地所有者,那末,这个剩余价值的性质,这个“纯粹的自然赐予”的性质,就更清楚地表现出来了。

不论工业还是农业本身,要形成雇佣劳动者阶级(最初,一切从事工业的人只表现为“土地耕种者-土地所有者”的雇工,即雇佣劳动者),劳动条件必须同劳动能力分离,而这种分离的基础是,土地本身表现为社会上一部分人的私有财产,以致社会上另一部分人失去了借以运用自己劳动的这个物质条件。

\begin{quote}“在最初的时代,土地所有者同土地耕种者还没有区别……在那个最初的时代,每一个勤劳的人要多少土地,就可以找到多少土地,[231]谁也不会想到\textbf{为别人劳动}……但是,到了最后,每一块土地都有了主人,那些没有得到土地的人最初没有别的出路,只好从事\textbf{雇佣}阶级的职业〈即手工业者阶级,一句话,一切非农业劳动者阶级〉,\textbf{用自己双手的劳动去换取}土地耕种者-土地所有者的产品余额。”(第 12 页)“土地耕种者-土地所有者”可以用土地对其劳动所赐予的“相当多的余额支付别人,要别人为他耕种土地。对于靠工资过活的人来说,无论从事哪种劳动来挣工资,都是一样。\textbf{因此,土地所有权必定要同农业劳动分离,而且不久也真的分离了}……土地所有者开始把耕种土地的劳动交给雇佣的土地耕种者去担负”。(同上,第 13 页)\end{quote}

这样,资本和雇佣劳动的关系就在农业中出现了。只有当一定数量的人丧失对劳动条件——首先是土地——的所有权,并且除了自己的劳动之外再也没有什么可以出卖的时候,这种关系才会出现。

现在,对于已经不能生产任何商品而不得不出卖自己的劳动的雇佣工人来说,\textbf{最低限度}的工资,即必要生活资料的等价物,必然成为他同劳动条件的所有者交换时的规律。

\begin{quote}“只凭双手和勤劳的普通工人,除了能够把他的劳动出卖给别人以外,就一无所有……在一切劳动部门,工人的工资都必定是,而实际上也是限于维持他的生活所必需的东西。”(同上,第 10 页)\end{quote}

而且,雇佣劳动一出现,

\begin{quote}“土地产品就分成两部分:一部分包括土地耕种者的生存资料和利润,这是他的劳动的报酬,也是他耕种土地所有者的土地的条件;余下的就是那个独立的可以自由支配的部分,这是\textbf{土地作为纯粹的赐予交给耕种土地的人的}、超过他的预付和他的劳动报酬的部分;这是归土地所有者的份额,或者说,是土地所有者赖以不劳动而生活并且可以任意花费的收入”。(同上,第 14 页)\end{quote}

但是,这个“纯粹的土地赐予”现在已经明确地表现为土地给“耕种土地的人”的礼物,即土地给劳动的礼物,表现为用在土地上的劳动的生产力,这种生产力是劳动由于利用自然的生产力所具有的,从而是劳动从土地中吸取的,是劳动只作为劳动从土地中吸取的。因此,在土地所有者手中,余额已经不再表现为“自然的赐予”,而表现为对于别人劳动的——不给等价物的——占有,后者的劳动由于自然的生产率,能够生产出超过本身需要的生存资料,但是它由于是雇佣劳动,在全部劳动产品中只能占有“维持他的生活所必需的东西”。

\begin{quote}“\textbf{土地耕种者}生产\textbf{他自己的工资},此外还生产用来支付整个手工业者和其他雇佣人员阶级的收入。\textbf{土地所有者没有土地耕种者的劳动,就一无所有}〈可见不是靠“纯粹的自然赐予”〉;他从土地耕种者那里[232]得到他的生存资料和用来支付其他雇佣人员劳动的东西……土地耕种者需要土地所有者,却仅仅由于习俗和法律。”(同上,第 15 页)\end{quote}

可见,在这里,剩余价值直接被描绘成土地所有者不给等价物而占有的土地耕种者劳动的一部分,因而这部分劳动的产品是他没有买过而拿去出卖的。但是,杜尔哥所指的不是交换价值本身,不是劳动时间本身,而是土地耕种者的劳动超出他自己的工资之上提供给土地所有者的产品余额;但这个产品余额,只不过是土地耕种者在他为再生产自己的工资而劳动的时间以外,白白地为土地所有者劳动的那一定量时间的体现。

因此,我们看到,重农学派在\textbf{农业劳动范围内}是正确地理解剩余价值的,他们把剩余价值看成雇佣劳动者的劳动产品,虽然对于这种劳动本身,他们又是从它表现为使用价值的具体形式来考察的。

顺便指出,杜尔哥认为,农业的资本主义经营方式——“土地出租”是

\begin{quote}“一切方式中最有利的方式,但是采用这种方式应以已经富庶的地区为前提”。(同上,第 21 页)\end{quote}

\fontbox{~\{}在考察剩余价值时,必须从流通领域转到生产领域,就是说,不是简单地从商品同商品的交换中,而是从劳动条件的所有者和工人之间在生产范围内进行的交换中,引出剩余价值。而劳动条件的所有者和工人又是作为商品所有者彼此对立的,因此,这里决不是以脱离交换的生产为前提。\fontbox{\}~}

\fontbox{~\{}在重农主义体系中,土地所有者是“雇主”,而其他一切生产部门的工人和企业主是“工资所得者”,或者说,“雇佣人员”。由此也就有了“管理者”和“被管理者”。\fontbox{\}~}

杜尔哥这样分析劳动条件:

\begin{quote}“在任何劳动部门,劳动者事先都要有劳动工具,都要有足够数量的材料作为他的劳动对象;而且都要在他的成品出卖之前有可能维持生活。”(同上,第 34 页)\end{quote}

所有这些“预付”,这些使劳动有可能进行,因而成为劳动过程的\textbf{前提}的条件,最初是由土地无偿提供的:

\begin{quote}“在土地完全没有耕种以前,土地就提供了最初的预付基金”,如果实、鱼、兽类等等,还有工具——例如树枝、石块、家畜,后者的数量由于繁殖而增加起来,它们每年还提供“乳、毛、皮和其他材料,这些产品连同从森林里采伐来的木材一起,成了工业生产的最初基金”。(同上,第 34 页)\end{quote}

这些劳动条件,这些“预付”,一旦必须由第三者预付给工人,就变成\textbf{资本},而这种情况,从工人除了本身的劳动能力外一无所有的时候起,就出现了。

\begin{quote}“\textbf{当}社会上大部分成员\textbf{只靠自己的双手谋生的时候},这些靠工资生活的人,无论是为了取得加工的原料,还是为了在发工资之前维持生活,都必须\textbf{事先}得到\textbf{一些东西}。”(同上,第 37—38 页)\end{quote}

[233]杜尔哥给“\textbf{资本}”下的定义是

\begin{quote}“积累起来的流动的价值”。(同上,第 38 页)最初,土地所有者或土地耕种者每天直接把工资和材料付给,比如说,纺麻女工。随着工业的发展,必须使用较大量的“预付”,并保证这个生产过程的不断进行。于是这件事就由“资本所有者”担当起来。这些“资本所有者”必须在自己产品的价格中收回他的全部“\textbf{预付}”,取得等于“假定他用货币购买一块〈土地〉而给他带来的东西”的一笔利润,还要取得他的“工资”,“因为,毫无疑问,如果利润一样多,他就宁可毫不费力地靠那笔资本能够买到的土地的收入来生活了”。(第 38—39 页)\end{quote}

“工业雇佣阶级”又划分为

\begin{quote}“企业资本家和普通工人”等等。(第 39 页)\end{quote}

“租地农场企业主”的情形也和这些企业资本家的情形一样。他们也象上述情况一样,必须收回全部“预付”,同时取得利润。

\begin{quote}“所有这一切都必须从土地产品的价格中事先扣除;\textbf{余下的部分}由土地耕种者交给土地所有者,作为允许他利用后者的土地来建立企业的报酬。这就是租金,就是土地所有者的收入,就是\textbf{纯产品},因为土地生产出来补偿各种预付和这些预付的提供者的利润的全部东西,不能看成收入,而只能看成\textbf{土地耕作费用的补偿};要知道,如果土地耕种者收不回这些费用,他就不会花费自己的资金和劳动去耕种别人的土地。”(同上,第 40 页)\end{quote}

最后:

\begin{quote}“虽然资本有一部分是由劳动者阶级的利润积蓄而成,但是,既然这些利润总是来自土地(因为所有这些利润不是由收入来支付,便是由生产这种收入的费用来支付),那末很明显,资本也完全象收入一样,来自土地;或者更确切地说,资本不外是土地所生产的一部分价值的积累,这一部分价值是收入的所有者或分享者可以每年储存起来,而不用来满足自己的需要的。”(第 66 页)\end{quote}

不言而喻,既然地租成为剩余价值的唯一形式,那末唯有地租才是资本积累的源泉。资本家在此以外积累的东西,是从他们的“工资”中(从供他们消费的收入中,因为利润正是被看成这种收入)积攒下来的。

因为利润和工资一样,算在土地耕作费用中,只有余下的部分才成为土地所有者的收入,所以土地所有者尽管被摆在可敬的地位,事实上,丝毫不分摊土地耕作费用,因而他不再是生产当事人——这一点同李嘉图学派的看法完全一样。

重农主义的产生,既同反对柯尔培尔主义\endnote{指法国路易十四时期柯尔培尔的重商主义经济政策。柯尔培尔当时任财政总稽核,他采取的财政经济政策是为了巩固专制国家的。例如,改革税收制度,建立垄断性的对外贸易公司,通过统一关税率来促进国内贸易,建立国家工场手工业,以及修建道路和港口。柯尔培尔主义客观上促进了新兴的资本主义经济方式的发展。它是法国资本原始积累的一个工具。但是随着资本主义生产方式日益强大,国家的这些强制性措施就由于日益阻碍经济发展而失去作用。——第 35、42 页。}有关系,又特别是同罗氏制度的破产\endnote{英国银行家和经济学家约翰·罗于 1716 年在巴黎创办一家私人银行,该银行于 1718 年改组为国家银行。罗力图依靠这家银行来实现他的荒唐主张,即国家通过发行不可兑的银行券来增加国内财富。罗氏银行无限制地发行纸币,同时回收金属货币。结果,交易所买空卖空和投机倒把活动空前盛行。到 1720 年,国家银行倒闭,罗氏“制度”也就彻底破产。罗逃往国外。——第 35、40 页。}有关系。

\tsectionnonum{[(4)把价值同自然物质混淆起来(帕奥累蒂)]}

[234]把价值同自然物质混淆起来,或者确切些说,把两者等同起来的看法,以及这种看法同重农学派的整套见解的联系,在后面这段引文中表现得很清楚。这段引文摘自\textbf{斐迪南多·帕奥累蒂}的著作《谋求幸福社会的真正手段》(这部著作一部分是针对维里的,维里在他的《政治经济学研究》(1771 年)中曾经反对重农学派)。(托斯卡纳的帕奥累蒂所写的这部著作,见库斯托第出版的《意大利政治经济学名家文集》(现代部分)第 20 卷。)

\begin{quote}象“土地产品”这样的“\textbf{物质数量倍增的情况}”,“在工业中无疑是没有的,而且永远也不可能发生,因为工业只给物质以形式,仅仅使物质发生形态变化;所以工业什么也不创造。但是,有人反驳我说,工业既给物质以形式,那它就是生产的;因为它即使不是物质的生产,也还是形式的生产。好吧,我不否认这一点;可是,\textbf{这不是财富的创造,相反,这无非是一种支出}……作为政治经济学的前提和研究对象的,是物质的和实在的生产,而这种生产只能在农业中发生,因为只有农业才使构成财富的物质和产品的数量倍增……工业从农业购买原料,以便把它加工。工业劳动,前面已经说过,只给这个原料以形式,但什么也不给它添加,不能使它倍增”。(第 196—197 页)“给厨师一定数量的豌豆,要他用来准备午餐;他好好烹调之后,将烧好的豌豆端到你桌上,但是数量同他拿去的一样;相反,把同量的豌豆交给种菜人,让他把豌豆拜托给土地,到时候,他归还给你的至少比他领去的多 3 倍。这才是真正的唯一的生产。”(第 197 页)“物由于人的需要才有价值。因此,商品的价值,或这个价值的增加,不是工业劳动的结果,而是劳动者支出的结果。”(第 198 页)“一种最新的工业品刚一出现,它很快就在国内外风行起来;可是,其他工业家、商人的竞争会\textbf{极快地}把它的价格压低到它应有的水平,这个水平……决定于原料和工人生存资料的价值。”(第 204—205 页)\end{quote}

\tsectionnonum{[(5)亚当•斯密著作中重农主义理论的因素]}

把自然力大规模地使用于生产过程,在农业中要比在其他一切生产部门中早。自然力在工业中的使用,只是在工业发展到比较高的阶段才明显。从后面的引文可以看出,亚·斯密在这里还反映大工业的史前时期,因此他表达的是重农主义的观点,而李嘉图则从现代工业的观点来回答他。

[235]亚·斯密在他的著作《国民财富的性质和原因的研究》第二篇第五章中,谈到地租时说道:

\begin{quote}“地租是扣除或补偿一切可以看作人工产物的东西之后所留下的自然的产物。它很少少于总产品的四分之一,而常常多于总产品的三分之一。制造业中使用的等量生产劳动,决不可能引起这样大的再生产。\textbf{在制造业中,自然什么也没有做,一切都是人做的};并且再生产必须始终和实行再生产的当事人的力量成比例。”\end{quote}

对于这一点,李嘉图在他的《政治经济学和赋税原理》(1819 年第 2 版第 61—62 页上的注)中作了回答:

\begin{quote}“在工业中,自然没有替人做什么吗\fontbox{?}那些推动我们的机器和船只的风力和水力,不算数吗\fontbox{?}那些使我们能开动最惊人的机器的大气压力和蒸汽压力,不是自然的赐予吗\fontbox{?}至于在软化和溶化金属时热的作用以及在染色和发酵过程中大气的作用,就更不用提了。在人们所能举出的任何一个工业部门中,自然都给人以帮助,并且是慷慨而无代价的帮助。”\end{quote}

至于重农学派把利润只看成是地租的扣除部分:

\begin{quote}“重农学派说,例如一幅花边的价格,它的一部分只补偿工人的消费,而另一部分则由一个人\fontbox{~\{}也就是土地所有者\fontbox{\}~}的口袋转入另一个人的口袋。”(《论马尔萨斯先生近来提倡的关于需求的性质和消费的必要性的原理》1821 年伦敦版第 96 页)\end{quote}

重农学派认为利润(包括利息)只是用于资本家消费的收入,从这种见解也产生了亚·斯密和追随他的经济学家的以下观点:资本的积累应归功于资本家个人的节俭、节约和节欲。重农学派所以作出这个论断,是因为他们认为,只有地租才是真正的、经济的、可以说是合法的积累源泉。

\begin{quote}杜尔哥说:“它〈即土地耕种者的劳动〉是唯一生产出\textbf{超过劳动报酬}的东西的劳动。”(\textbf{杜尔哥},同上第 11 页)\end{quote}

可见,利润在这里完全包括在“劳动报酬”之中。

\begin{quote}[236]“土地耕种者除了这个补偿〈补偿他自己的工资〉以外,还生产出土地所有者的收入,而手工业者既不为自己也不为别人生产任何收入。”(同上,第 16 页)“土地生产出来补偿各种预付和这些预付的提供者的利润的全部东西,\textbf{不能看成收入},而只能看成\textbf{土地耕作费用的补偿}。”(同上,第 40 页)\end{quote}

阿·布朗基在《欧洲政治经济学从古代到现代的历史》(1839 年布鲁塞尔版第 139 页)中,谈到重农学派时说道:

\begin{quote}“他们认为,用于耕种土地的劳动,不仅生产出劳动者在整个劳动期间为维持本人生活所必需的东西,而且还生产出一个可以加到已有财富量上的\textbf{价值余额}〈剩余价值〉,他们把这个余额称为\textbf{纯产品}。”\end{quote}

(因而他们是从剩余价值借以表现的使用价值的形式来看剩余价值的。)

\begin{quote}“从他们的观点来看,纯产品必定属于土地所有者,并且成为他手中完全可以由他支配的收入。那末什么是其他劳动部门的纯产品呢\fontbox{?}……工业家、商人、工人——他们都被看成是农业的伙计、\textbf{雇佣劳动者},而农业是一切财富的至高无上的创造者和分配者。根据经济学家\endnote{“经济学家”是十八世纪下半叶和十九世纪上半叶在法国对重农学派的称呼。——第 38、139、223、411 页。}的体系,所有这些人的劳动产品只代表他们在劳动期间消费掉的东西的等价物,因此,在劳动完成之后,\textbf{除非工人或业主把他们有权消费的东西储存下来},也就是说\CJKunderdot{\textbf{节约下来}},财富的总量同以前是完全一样的。因此,只有用在土地上的劳动,才能生产财富,而其他生产部门的劳动是\CJKunderdot{\textbf{不生产的}},因为\textbf{它不能使社会资本有任何增加}。”\end{quote}

\fontbox{~\{}总之,重农学派认为,资本主义生产的实质在于剩余价值的生产。他们应当解释的正是这种现象。在他们驳倒了重商主义的“让渡利润”之后,问题也就在这里。

\begin{quote}\textbf{里维埃尔的迈尔西埃}说:“要有货币,人就必须购买货币,在这种购买之后,他并不比以前更富;他不过是把他以商品形式付出去的同一个价值,以货币形式取回来。”(\textbf{里维埃尔的迈尔西埃}《政治社会天然固有的秩序》第 2 卷第 338 页)\end{quote}

这一点适用于[237]买,也适用于卖,同样适用于商品的整个形态变化的结果,即买卖的结果;适用于各种商品按其价值进行的交换,即等价物的交换。但在这种情况下,剩余价值是从哪里来的呢\fontbox{?}也就是说,资本是从哪里来的呢\fontbox{?}摆在重农学派面前的正是这个问题。他们的错误在于,他们把那种由于植物自然生长和动物自然繁殖而使农业和畜牧业有别于工业的\textbf{物质增加},同\textbf{交换价值的增殖}混淆起来了。在他们看来,使用价值是基础。而一切商品的使用价值(如果用烦琐哲学家的术语来说,则归结为一般实质),在他们看来,就是自然物质本身,而自然物质在其既定形式上的增加,只有在农业中才会发生。\fontbox{\}~}

亚·斯密著作的翻译者热·加尔涅本人是一个重农主义者,他正确地叙述了重农主义的\textbf{节约论}等。首先他告诉我们,工业——而重商学派认为是一切生产部门——\textbf{只有}靠“让渡利润”,靠商品高于其价值出卖,才能创造剩余价值,因此,发生的只是已创造的价值的新分配,而不是已创造的价值的新增加。

\begin{quote}“手工业者和工业家的劳动不开辟财富的任何新源泉,它只有靠\textbf{有利的交换才能获得利润},并且只具有纯粹相对的价值,这种价值,如果\textbf{靠交换获利}的机会不再出现,也就不会再有了。”(亚·斯密《国民财富的性质和原因的研究》,加尔涅的译本,1802 年巴黎版第 5 卷第 266 页)\endnote{热尔门·加尔涅翻译的亚当·斯密的《国富论》法译本(1802 年版)第五卷中包含有《译者的注释》,即热尔门·加尔涅的注释。——第 39 页。}\end{quote}

或者说,他们的节约——除去开支以外给自己保留下来的价值——必须依靠缩减自己的消费来实现。

\begin{quote}“虽然除了雇佣工人和资本家的节约以外,手工业者和工业家的劳动不可能把其他任何东西加到社会财富的总量上去,但是它依靠这种节约,能促使社会富裕。”(同上,第 266 页)\end{quote}

下面一段话说得更详细:

\begin{quote}“农业劳动者正是以自己的劳动产品使国家富裕;相反,工商业劳动者\textbf{只有节约自己的消费}才能够使国家富裕。经济学家的这个论断是从他们对农业劳动和工业劳动所作的区分得出的,并且同这种区分本身一样是无可争议的。事实上,手工业者和工业家的劳动可以加到物质价值上去的,仅仅是他们自己劳动的\textbf{价值},也就是这种劳动根据国内当时通行的工资率[238]和利润率必定带来的工资和利润的价值。这种工资,无论是高是低,都是劳动的报酬;这是工人有权消费并且假定正在消费的东西;因为只有通过消费,他才能享受自己的劳动果实,而这种享受实际上也就是他的全部报酬。同样,利润无论是高是低,也被看成资本家天天消费的东西,当然,假定资本家也是按照资本带给他的收入的多少来安排自己的享受的。总之,如果工人不放弃他按照适合于\textbf{他的劳动}的通行的工资率有权享受的一部分福利,如果资本家不把资本带给他的一部分收入储蓄起来,那末,工人和资本家在完成劳动时,也就消费了这个劳动带来的全部价值。因此,在他们的劳动完成之后,社会财富的总量依然和以前一样,\textbf{除非他们}把他们有权消费并且能够消费而不致被指责为浪费的一部分东西\textbf{节约下来};在后一场合,社会财富总量就增加了\textbf{这种节约的全部价值}。因此,完全可以说,从事工商业的人\textbf{只有通过个人的节俭}才能\textbf{增加社会现有的财富总量}。”(同上,第 263—264 页)\end{quote}

加尔涅也完全正确地觉察到,亚·斯密关于通过节约进行积累的理论,是建立在这个重农主义基础上的(亚·斯密深受重农主义的影响,这种影响在他对重农主义的批判上表现得最明显)。加尔涅说:

\begin{quote}“最后,如果经济学家曾经断言,工业和商业只有通过节俭才能增加国民财富,那末,斯密同样说过,如果经济不通过本身的节约来增加资本,工业就会白白经营,一国的资本也就永远不会增加(第 2 篇第 3 章)。由此可见,斯密完全同意经济学家的意见”等等。(同上,第 270 页)\end{quote}

\tsectionnonum{[(6)重农学派是资本主义大农业的拥护者]}

[239]阿·布朗基在前面引用过的著作中指出,促使重农主义传播、甚至促使它产生的一个直接历史情况是:

\begin{quote}“在\textbf{制度}〈罗氏制度\endnote{英国银行家和经济学家约翰·罗于 1716 年在巴黎创办一家私人银行,该银行于 1718 年改组为国家银行。罗力图依靠这家银行来实现他的荒唐主张,即国家通过发行不可兑的银行券来增加国内财富。罗氏银行无限制地发行纸币,同时回收金属货币。结果,交易所买空卖空和投机倒把活动空前盛行。到 1720 年,国家银行倒闭,罗氏“制度”也就彻底破产。罗逃往国外。——第 35、40 页。}〉的狂热气氛中猛长起来的一切价值,除了毁灭、荒芜、破产之外,毫无所留。\textbf{唯独土地所有权}在这次风暴中未受损伤。”\end{quote}

\fontbox{~\{}因此,蒲鲁东先生在《贫困的哲学》中,也让土地所有权跟在信贷后面出现。\fontbox{\}~}

\begin{quote}“它的地位甚至改善了,因为它——也许是从封建时代以来\textbf{第一次}——转了手,并且\textbf{被大规模地分割了}。”(同上,第 138 页)\end{quote}

这就是说:

\begin{quote}“在该制度的影响下发生的无数次转手,开始了土地所有权的分割……土地所有权第一次摆脱了封建制度长期来使它所处的僵化状态。对农业来说,这是土地所有权的真正的苏醒……它〈土地〉从死手制度转入了流通制度。”(第 137—138 页)\end{quote}

正象\textbf{魁奈}和他的其他门徒一样,杜尔哥也主张农业中的\textbf{资本主义}生产。例如,杜尔哥说:

\begin{quote}“土地的出租……最后这种方式〈以现代租佃制为基础的大农业〉是一切方式中最有利的方式,但是采用这种方式应以已经富庶的地区为前提。”(见\textbf{杜尔哥},同上第 21 页)\end{quote}

魁奈在他的《农业国经济管理的一般原则》中说:

\begin{quote}“用于种植谷物的土地应当尽可能地联合成由富裕的土地耕种者〈即资本家〉经营的大农场,因为大农业企业与小农业企业相比,建筑物的维修费较低,生产费用也相应地少得多,而纯产品多得多。”[《重农学派》,德尔出版,第 1 部第 96—97 页]\end{quote}

同时,魁奈在上述地方承认:农业劳动生产率提高的成果,应当归“纯收入”,因而首先落到土地所有者手里,即剩余价值占有者手里;剩余价值的相对增加不是由土地产生的,而是由提高劳动生产率的社会措施和其他措施产生的。[240]他在上述地方说:

\begin{quote}“可以利用动物、机器、水力等等进行的劳动,它的任何有利的\fontbox{~\{}对“纯产品”有利的\fontbox{\}~}节约,都造福于居民[和国家,因为较大量的纯产品能保证从事其他职业和工作的人有较多的工资]。”\end{quote}

同时,里维埃尔的迈尔西埃(同上,第 2 卷第 407 页)模糊地猜到:剩余价值至少在工业中(前面已经指出,杜尔哥把这一点推广到一切生产部门)同工业工人本身有某种关系。他在这个地方大声疾呼:

\begin{quote}“盲目崇拜工业的虚假产品的人们,请把你们的狂喜劲儿收敛一下吧!在你们赞赏工业奇迹之前,睁开眼睛看看,那些有手艺把 20 苏变为 1000 埃巨价值的工人是多么贫穷,至少是多么拮据!\textbf{价值的这个巨大的增殖额落到谁手里去呢\fontbox{?}请看:亲手创造价值增殖额的人却过不了宽裕日子!请注意这个对照吧}!”\end{quote}

\tsectionnonum{[(7)重农学派政治观点中的矛盾。重农学派和法国革命]}

经济学家的整个体系的矛盾。魁奈是君主专制的拥护者之一。

\begin{quote}“政权应当是统一的……在政体上,保持各种相互对抗的力量的制度是有害的,它只证明上层不和睦和下层受压迫。”(见前面引证的《农业国经济管理的一般原则》[《重农学派》,德尔出版,第 1 部第 81 页])\end{quote}

里维埃尔的迈尔西埃写道:

\begin{quote}“人注定要在社会内生活,单单这一点就注定他要在专制制度的统治下生活。”(《政治社会天然固有的秩序》第 1 卷第 281 页)\end{quote}

这里还有“人民之友”\endnote{老米拉波活着的时候,人们根据他的一本著作的标题称他为《L’Amideshommes》(“人民之友”、“人类之友”)。——第 42 页。}米拉波侯爵,老米拉波!正是这个学派以自己的自由放任\endnote{自由放任(原文是:laissezfaire,laissezaller,亦译听之任之)是重农学派的口号。重农学派认为,经济生活是受自然规律调节的,国家不得对经济事务进行干涉和监督;国家用各种规章进行干涉,不仅无益,而且有害;他们要求实行自由主义的经济政策。——第 27、42、162 页。}口号推翻了柯尔培尔主义\endnote{指法国路易十四时期柯尔培尔的重商主义经济政策。柯尔培尔当时任财政总稽核,他采取的财政经济政策是为了巩固专制国家的。例如,改革税收制度,建立垄断性的对外贸易公司,通过统一关税率来促进国内贸易,建立国家工场手工业,以及修建道路和港口。柯尔培尔主义客观上促进了新兴的资本主义经济方式的发展。它是法国资本原始积累的一个工具。但是随着资本主义生产方式日益强大,国家的这些强制性措施就由于日益阻碍经济发展而失去作用。——第 35、42 页。},并根本否定政府对市民社会活动的任何干涉。它只让国家在这个社会的缝隙中生活,就象伊壁鸠鲁让神在世界的缝隙中存在\endnote{古希腊哲学家伊壁鸠鲁认为,神存在于世界与世界之间的空隙、间隙中,它对宇宙的发展和人类的生活没有任何影响。——第 42 页。}一样!对土地所有权的颂扬,在实践中变成了把赋税全都转到地租上的要求,这就包含着国家没收地产的可能性,——这一点完全同李嘉图学派的激进分子\endnote{马克思指激进的李嘉图学派。这个学派从李嘉图的理论中得出了反对土地私有制存在的实际结论,建议把这一制度(全部或部分)转变为资产阶级国家所有制。属于这个激进的李嘉图学派的有詹姆斯·穆勒、约翰·斯图亚特·穆勒、希尔迪奇,在一定程度上也有舍尔比利埃。关于这一点见本卷第二册,马克思手稿第 458 页;第三册,马克思手稿第 791、1120 和 1139 页;并见《哲学的贫困》(《马克思恩格斯全集》中文版第 4 卷第 187 页)和马克思 1881 年 6 月 20 日给左尔格的信(见《马克思恩格斯全集》中文版第 35 卷第 190—194 页)。——第 26、42 页。}一样。法国革命不顾勒代雷和其他人的反对,采纳了这种赋税理论。

杜尔哥本人是给法国革命引路的激进资产阶级大臣。重农学派虽然有它的假封建主义外貌,但他们同百科全书派\endnote{指法国《百科全书或科学、艺术和工艺详解辞典》(1751 至 1772 年出版,共 28 卷)的作者。百科全书是十八世纪最著名的法国启蒙运动者的著作。主编是狄德罗。参加编篡工作的还有:达兰贝尔、霍尔巴赫、爱尔维修和拉美特利等。孟德斯鸠、伏尔泰和毕丰参与撰写自然科学的条目,孔狄亚克参与撰写哲学的条目。魁奈和杜尔哥在他们撰写的政治经济学条目中阐述了重农主义体系。百科全书派是由具有不同政治观点的人组成的。这部著作为法国革命的思想准备作出了贡献。——第 42 页。}齐心协力地工作。[240]

[241]杜尔哥试图预先采取法国革命的措施。他以\textbf{1776 年二月敕令}废除了行会。(这个敕令在颁布三个月后就撤销了。)同样,杜尔哥还使农民摆脱了筑路义务,并试图对地租实行单一税。\endnote{手稿中这一段是在下面三段之后(仍在第 241 页)。它被用横线同上下文隔开,同前后两段都没有直接联系。因此本版把这一段移至第 240 页末,就其内容来说,它直接同这一页有关。——第 42 页。}

[241]后面,我们还要回过头来谈重农学派在分析资本方面的巨大功绩。\endnote{参看前面第 15—16 页和那里的注 15。在《剩余价值理论》中,马克思在手稿第 X 本中又回过头来谈重农学派,那里有题为《魁奈的经济表》的长篇“插入部分”(见本册第 323—366 页)。——第 43 页。}

这里暂时先指出一点。按照重农学派的意见,剩余价值的产生有赖于特种劳动的生产率即农业的生产率。而这种特殊的生产率的存在,总的说来,有赖于自然本身。

根据重商主义体系,剩余价值只是相对的:一人赢利就是他人亏损。“让渡利润”,或者说“财富的天平在有关双方之间的摆动”。\authornote{见本册第 11—13 页。——编者注}因此,从一国总资本来看,在这个国家内部,实际上并没有形成剩余价值。剩余价值只有在一个国家同另一些国家的关系中才能形成。一国所实现的超过另一国的余额表现在货币上(贸易差额),因为货币正是交换价值的直接的和独立的形式。与此相反,——因为重商主义体系事实上否定绝对剩余价值的形成,——重农主义愿意把绝对剩余价值解释为“\textbf{纯产品}”。因为重农学派把注意力集中在使用价值上,所以他们认为农业是\textbf{这种“纯产品”的唯一创造者}。

\tsectionnonum{[(8)普鲁士反动分子施马尔茨把重农主义学说庸俗化]}

我们看到,追查蛊惑者\endnote{蛊惑者是十九世纪二十年代在德国对本国知识分子中间反政府运动的参加者的称呼。这个词是在 1819 年 8 月举行的德意志各邦大臣卡尔斯巴德联席会议通过了一项对付“蛊惑者的阴谋”的专门决议之后流行开来的。——第 43 页。}的老手,普鲁士王国枢密顾问施马尔茨是重农主义的最幼稚的代表之一——他同杜尔哥相差不知多远!例如,施马尔茨说:

\begin{quote}“既然自然付给他〈土地所有者〉\textbf{比合法货币利息多一倍的利息},那末,根据什么明显的理由可以剥夺他的这种收入呢\fontbox{?}”(《政治经济学》,昂利·茹弗鲁瓦译自德文,1826 年巴黎版第 1 卷第 90 页)\endnote{施马尔茨的著作德文原本于 1818 年在柏林出版,题为《政治经济学。致德意志某王储书柬》第一册和第二册。——第 44 页。}\end{quote}

关于最低限度的工资,重农学派是这样表述的:工人的消费(或开支)等于\textbf{他们所得的}工资。或者象施马尔茨先生那样把这一点一般地表述为:

\begin{quote}“某一职业的平均工资,等于从事这一职业的人在他劳动期间的平均消费额。”(同上,第 120 页)\end{quote}

[接着,我们在施马尔茨的书里读到:]

\begin{quote}“\textbf{地租}是国民收入的唯一要素;[242]投资的利息和各种劳动的工资,都不过是把这个地租的产品从一个人的手里转到另一个人的手里。”(同上,第 309—310 页)“国民财富仅仅在于土地每年生产地租的能力。”(同上,第 310 页)“一切东西,不管它们是什么样的,如果要追究它们的\textbf{价值}的基础或原始要素,那就必须承认,这个价值无非是纯粹的自然产品的价值。这就是说,虽然劳动使这些东西具有新价值,因而增加了它们的价格,但这个新价值,或者说,这个增加了的价格,仍然不过是为了使这些东西具有新形式,而由工人以各种方式毁坏、消费或用掉的一切自然产品的价值的总和。”(同上,第 313 页)“这种劳动〈真正的农业〉是唯一有助于生产\textbf{新\CJKunderdot{物体}}的劳动,因而是唯一在某种程度上可以称为生产劳动的劳动。至于加工工业的劳动……它只赋予自然所生产的物体以新的形式。”(同上,第 15—16 页)\end{quote}

\tsectionnonum{[(9)对重农学派在农业问题上的偏见的最初批判(维里)]}

反对重农学派的偏见。

\textbf{维里(彼得罗)}《政治经济学研究》(1771 年第一次刊印),见库斯托第出版的《意大利政治经济学名家文集》现代部分,第十五卷。

\begin{quote}“宇宙的一切现象,不论是由人手创造的,还是由物理学的一般规律引起的,都不是真正的\textbf{新创造},而只是物质的\textbf{形态变化}。\textbf{结合}和\textbf{分离}是人的智慧在分析\textbf{再生产}的观念时一再发现的唯一要素;\textbf{价值和财富的再生产},如土地、空气和水在田地上变成谷物,或者昆虫的分泌物经过人的手变成丝绸,或者一些金属片被装配成钟表,也是这样。”(第 21—22 页)\end{quote}

接着他写道:

\begin{quote}重农学派把“工业劳动者阶级称为\textbf{不生产}阶级,因为按照他们的意见,\textbf{工业品的价值等于原料加上工业劳动者在加工这个原料}时所消费的\textbf{食品}”。(第 25 页)\end{quote}

[243]相反,维里却注意到土地耕种者经常贫穷,而工业劳动者日益富裕,然后他继续写道:

\begin{quote}“这证明,工业家从他卖得的价格中不仅获得\textbf{消费的补偿,而且在这个补偿之外多得一部分,而这一部分就是}一年生产中\textbf{所创造的新的价值量}。”(第 26 页)“新创造的价值,就是农产品或工业品的价格中\textbf{超过}物质和物质加工时所必要的消费费用的\textbf{原有价值的余额}。在农业中必须扣除种子和土地耕种者的消费;在工业中同样要扣除原料和劳动者的消费,而每年所创造的\textbf{新价值和扣除后的余额}一样多。”(第 26—27 页)\end{quote}

\tchapternonum{[第三章]亚当·斯密}

\tsectionnonum{[(1)斯密著作中两种不同的价值规定:价值决定于商品中包含的已耗费的劳动量;价值决定于用这个商品可以买到的活劳动量]}

亚·斯密和一切值得一谈的经济学家一样,从重农学派那里接受了平均工资的概念,他把平均工资叫做“工资的自然价格”:

\begin{quote}“一个人总要靠自己的劳动来生活,他的工资至少要够维持他的生存。在大多数情况下,他的工资甚至应略高于这个水平,否则,工人就不可能养活一家人,这些工人就不能传宗接代。”([加尔涅的译本]第 1 卷第 1 篇第 8 章第 136 页)\end{quote}

亚·斯密十分明确地断定,劳动生产力的发展,对工人本身并没有好处。例如我们在他的著作中读到(麦克库洛赫版,第 1 篇第 8 章,1828 年伦敦版):

\begin{quote}“劳动的产品构成劳动的自然报酬或工资。在\textbf{土地私有制产生和资本积累}之前的社会原始状态中,全部劳动产品都属于劳动者。既没有土地所有者,也没有老板来同他分享。假如社会的这种状态保持下去,那末工资\textbf{就会随着分工}引起的\textbf{劳动生产力的增长而增长}。一切东西就会逐渐便宜起来。”\end{quote}

\fontbox{~\{}无论如何,在再生产时需要劳动量较少的一切东西,都是如此。但是,它们不仅“会”便宜起来,实际上已经便宜了。\fontbox{\}~}

\begin{quote}“它们将会由较少量的劳动生产出来;而因为在这种状态下同量劳动生产的商品自然会相互交换,所以它们也就可以用劳动量较少的产品[244]来购买……但是,这种由劳动者享有自己的全部劳动产品的社会原始状态,在\textbf{土地私有制产生和资本积累之后,不可能保持下去}。因此,这种状态在劳动生产力取得最重大发展之前早就不存在了,所以,进一步研究这种状态对劳动报酬或工资可能发生什么影响,就没有用处了。”(第 1 卷第 107—109 页)\end{quote}

亚·斯密在这里非常确切地指出,劳动生产力真正大规模的发展,只是从劳动变为雇佣劳动,而劳动条件作为土地所有权和作为资本同劳动相对立的时刻才开始的。因而劳动生产力的发展只是在劳动者自己再也不能占有这一发展成果的条件下才开始的。因此,研究生产力的这种增长在假定劳动产品(或这个产品的价值)属于劳动者本人的情况下对“工资”——在这里等于劳动产品——会有(或应当有)什么影响,就完全没有用处了。

亚·斯密深受重农主义观点的影响,在他的著作中,往往夹杂着许多属于重农学派而同他自己提出的观点完全矛盾的东西。例如地租学说等等,就是如此。斯密著作的这些部分并不表现他的特点,他在这些地方纯粹是一个重农主义者,\endnote{马克思在《剩余价值理论》第二册(手稿第 628—632 页,《亚·斯密的地租理论》一章)中对斯密的地租观点中的重农主义因素作了批判的分析。参看前面《重农学派》一章,第 36—40 页。——第 47 页。}从我们的研究目的来说,这些部分可以完全不去注意。

我在这部著作的第一部分分析商品时已经指出,\endnote{马克思指《政治经济学批判》第一分册。见《马克思恩格斯全集》中文版第 13 卷第 49—50 页。——第 47 页。}亚·斯密在两种不同的交换价值规定之间摇摆不定:一方面认为\textbf{商品}的价值决定于生产商品所必要的劳动量,另一方面又认为商品的价值决定于可以买到商品的活劳动量,或者同样可以说,决定于可以买到一定量活劳动的商品量;他时而把第一种规定同第二种规定混淆起来,时而以后者顶替前者。在第二种规定中,斯密把劳动的\textbf{交换价值},实际上就是把\textbf{工资}当作商品的价值尺度,因为工资等于用一定量活劳动可以购得的商品量,或者说,等于用一定量商品可以买到的劳动量。但是,劳动的价值,或者确切些说,劳动能力的价值,也和其他任何商品的价值一样,是变化的,它和其他商品的价值没有什么特殊的区别。这里把价值本身当作价值标准和说明价值存在的理由,因此成了循环论证。

但是,从后面的叙述中可以看到,斯密的这种摇摆不定以及把完全不同的规定混为一谈,并不妨碍他对剩余价值的性质和来源的探讨,因为斯密凡是在发挥他的论点的地方,实际上甚至不自觉地坚持了商品交换价值的正确规定,即商品的交换价值决定于商品中包含的已耗费的劳动量或劳动时间。[244]

[VII—283a]\fontbox{~\{}可以举出许多例子证明,亚·斯密在他的整部著作中,凡是说明真正事实的地方,往往把产品中包含的劳动量理解为价值和决定价值的因素。这方面的材料,在李嘉图的著作中引用了一部分。\endnote{指李嘉图的《政治经济学和赋税原理》第一篇第一章。——第 48 页。}斯密关于分工和机器改良对商品价格的影响的全部学说,就是建立在这个基础上的。这里只引一个地方就够了。亚·斯密在第一篇第十一章谈到,他那个时代同前几个世纪比较,有许多工业品便宜了,关于前几个世纪,他指出:

\begin{quote}“那时,为了制造这些商品供应市场,要花费多得多的[283B]劳动量,因此商品上市以后,在交换中必定买回或得到一个多得多的劳动量的价格。”([加尔涅的译本]第 2 卷第 156 页)\fontbox{\}~}[VII—283b]\end{quote}

[VI—245]其次,亚·斯密著作中的上述矛盾以及他从一种解释方法到另一种解释方法的转变,是有更深刻的基础的。(李嘉图发现了斯密的矛盾,但没有觉察到这个更深刻的基础,没有对他所发现的矛盾做出正确的评价,因此也没有解决这个矛盾。)假定所有劳动者都是商品生产者,他们不仅生产自己的商品,而且出卖这些商品。这些商品的价值决定于商品中包含的必要劳动时间。因此,如果商品按它们的价值出卖,那末劳动者用一个作为 12 小时劳动时间的产品的商品,仍然可以买到以另一个商品为形式的 12 小时劳动时间,即物化在另一个使用价值中的 12 小时劳动时间。由此看来,他的劳动的价值等于他的商品的价值,即等于 12 小时劳动时间的产品。卖和随之而来的买,总之,整个交换过程——商品的形态变化——在这里没有引起任何改变。它所改变的只是表现这 12 小时劳动时间的使用价值的形态。因此,劳动的价值等于劳动产品的价值。第一,以商品形式相交换的——只要商品按它们的价值进行交换——是等量物化劳动。而第二,一定量活劳动同等量物化劳动相交换,因为一方面,活劳动物化在属于劳动者的产品即商品中,另一方面,这个商品又同包含等量劳动的另一个商品相交换。因而实际上是一定量活劳动同等量物化劳动相交换。由此可见,不仅是商品同商品按照它们所代表的等量物化劳动时间的比例相交换,而且是一定量活劳动与代表同量物化劳动的商品相交换。

在这种前提下,劳动的价值(用一定量劳动可以买到的商品量,或者说,用一定量商品可以买到的劳动量),就和商品中包含的劳动量完全一样,可以看作商品的价值尺度。这是因为,劳动的价值所表现的物化劳动量总是等于生产这个商品所必要的活劳动量,换句话说,一定量的活劳动时间总是支配着代表同样多的物化劳动时间的商品量。但是,在劳动的物质条件属于一个阶级(或几个阶级),而只有劳动能力属于另一个阶级即工人阶级的一切生产方式下——特别是在资本主义生产方式下——情况正好相反。劳动产品或劳动产品的价值不属于工人。一定量活劳动支配的不是同它等量的物化劳动;换句话说,一定量物化在商品中的劳动所支配的活劳动量,大于该商品本身包含的活劳动量。

但是,因为亚·斯密完全正确地从商品以及商品交换出发,从而生产者最初只是作为商品所有者——商品的卖者和买者——相互对立,所以,他发现(他以为),在资本和雇佣劳动的交换、[246]物化劳动和活劳动的交换中,一般规律立即失效了,商品(因为劳动既然被买卖,那它也是商品)已经不按照它们所代表的劳动量来交换了。\textbf{由此}他得出结论:一旦劳动条件以土地所有权和资本的形式同雇佣工人相对立,劳动时间就不再是调节商品交换价值的内在尺度了。正如李嘉图正确地评论他的那样,斯密倒是应当做出相反的结论:“劳动的量”和“劳动的价值”这两个用语不再是等同的了,因而,商品的相对价值,虽然由商品中包含的劳动时间调节,但已经不再由劳动的价值调节了,因为后一个用语只有在同前一个用语等同的时候,才是正确的。以后谈到马尔萨斯的时候,\endnote{在《剩余价值理论》第三册《托·罗·马尔萨斯》一章(手稿第 753—781 页)中,马克思对马尔萨斯的价值观点和剩余价值观点作了详细的批判(手稿第 753—767 页)。——第 50、67 页。}将会证明,即使在劳动者占有自己的产品即自己产品的价值的情况下,把这个价值或劳动的价值当作象劳动时间或劳动本身作为价值尺度和创造价值的要素那种意义的价值尺度,这本身就是错误的和荒谬的。即使在这种情况下,一种商品可以买到的劳动,也不能当作与商品中所包含的劳动有同样意义的尺度,其中的一个只不过是另一个的指数。

无论如何,亚·斯密感到,从决定商品交换的规律中很难引伸出资本和劳动之间的交换,后者显然是建立在同这一规律完全对立和矛盾的原则上的。只要资本直接同劳动相对立,而不是同劳动能力相对立,这种矛盾就无法解释。亚·斯密知道得很清楚,再生产和维持劳动能力所耗费的劳动时间,与劳动能力本身所能提供的劳动是大不相同的。关于这个问题,他甚至引证康替龙的著作《试论一般商业的性质》。

\begin{quote}斯密谈到康替龙时写道:“这位作者补充说,强壮奴隶的劳动据估计有两倍于他的生活费用的价值,而一个最弱工人的劳动所具有的价值,在他看来,也不会比强壮奴隶的劳动少。”(第 1 篇第 8 章第 137 页,\textbf{加尔涅}的译本,第 1 卷)\end{quote}

另一方面,奇怪的是,亚·斯密竟不了解,他的疑问同调节商品交换的规律没有什么关系。商品 A 和商品 B 按它们所包含的劳动时间的比例进行交换,这丝毫不会由于产品 A 或产品 B 的生产者相互之间分配产品 A 和产品 B(或者确切些说,分配它们的价值)的比例而受到破坏。如果产品 A 的一部分归土地所有者,第二部分归资本家,第三部分归工人,那末,无论他们所得的份额是多少,丝毫也不会改变 A 本身是按其价值同 B 相交换的情况。A 和 B 这两种商品所包含的劳动时间的比例,完全不因 A 或 B 所包含的劳动时间如何由不同的人占有而受到影响。

\begin{quote}“当呢绒和麻布进行交换的时候,呢绒的生产者就会在麻布上恰恰占有他们以前在呢绒上所占有的那一份。”(《哲学的贫困》第 29 页)\endnote{马克思引用的是他的著作《哲学的贫困》法文第一版(1847 年巴黎—布鲁塞尔版)。见《马克思恩格斯全集》中文版第 4 卷第 95—96 页。——第 51 页。}\end{quote}

这也就是李嘉图学派后来完全正当地提出来反对[247]亚·斯密的论据。马尔萨斯主义者约翰·卡泽诺夫同样写道:

\begin{quote}“商品的交换和商品的分配应当分开来考察……对其中一个有影响的情况并不总是对另一个也有影响。例如,某一种商品的生产费用的减少,会改变它对其他一切商品的比例;但不一定会改变这种商品本身的分配,或者根本不会影响其他商品的分配。另一方面,\textbf{对一切}商品\textbf{同样发生影响的}价值普遍下降,不会改变商品之间的比例。它可能影响——但也可能不影响——它们的分配”等等。(\textbf{约翰·卡泽诺夫}《为\textbf{马尔萨斯}的〈政治经济学定义〉所写的序言》1853 年伦敦版)\end{quote}

但是,因为在资本家和工人之间进行的产品价值的“分配”本身,是以商品交换——商品和劳动能力之间的交换——为基础的,所以这就自然引起亚·斯密的混乱。亚·斯密还把劳动的价值或某一商品(或货币)对劳动的购买力当作价值尺度,这就使他在阐述价格理论、研究竞争对利润率的影响等等地方乱了思路,使他的著作在总的方面失去了任何统一性,甚至使他把许多重大问题排除在研究范围之外。然而,我们在后面马上就会看到,这并没有影响他关于\textbf{剩余价值的一般}思路,因为斯密在这里始终坚持了价值决定于各种商品中包含的已耗费的劳动时间这一正确规定。

现在我们就来谈谈他对问题的阐述。

不过,还要先提到一个情况。亚·斯密把不同的东西混淆起来了。首先,他在第一篇第五章中说:

\begin{quote}“一个人是富是贫,要看他能取得必需品、舒适品和娱乐品的程度如何。但是,自从各个部门的分工确立之后,一个人依靠自己的劳动能够取得的只是这些物品中的极小部分,极大部分必须\textbf{仰给于他人的劳动};所以他是富是贫,\textbf{就要看他能够支配或买到的劳动量有多大}。因此,任何一种商品,对于占有这种商品而不打算自己使用或消费,却打算\textbf{用它交换其他商品的人来说},\textbf{它的价值等于}这个\textbf{商品能够买到或支配的劳动量}。由此看来,劳动是一切商品的\textbf{交换价值的真实的}尺度。”(第 1 卷第 59—60 页)\end{quote}

接着,他说:

\begin{quote}“\textbf{它们〈商品〉}包含着\textbf{一定量劳动的价值,我们就用这一定量的劳动去同假定}[248]\textbf{在当时包含着同量劳动的价值的东西相交换}……世界上的一切财富原先都不是用金或银,而只是用劳动购买的;这些财富的价值,对于占有它们并想用它们交换什么新产品的人来说,恰好等于他能够买到或支配的劳动量。”(第 1 篇第 5 章第 60—61 页)\end{quote}

最后,他说:

\begin{quote}“霍布斯先生说,\textbf{财富}就是\textbf{权力};但是,获得或继承了大宗财产的人,不一定因此就得到民政的或军事的政治权力……财富直接提供给他的权力,无非是购买的权力;这是一种支配\textbf{当时市场上有的一切他人劳动\CJKunderdot{或者说}他人劳动的一切产品}的权力。”(同上,第 61 页)\end{quote}

我们看到,在所有这些地方,斯密都把“\textbf{他人劳动}”同“\textbf{他人劳动的产品}”混淆起来了。自从分工确立之后,属于某一个人的商品的交换价值,就表现为这个人所能买到的别人的商品,也就是表现为这些商品中包含的别人劳动的量,即物化了的别人劳动的量。而别人劳动的这个量等于他自己的商品中包含的劳动量。斯密十分明确地说:

\begin{quote}“商品包含着一定量劳动的价值,我们就用这一定量的劳动去同假定在当时包含着\textbf{同量劳动的价值}的东西相交换。”\end{quote}

这里的重点在于\textbf{分工}所引起的变化,它表现在:财富已经不再由本人劳动的产品构成,而由这个产品支配的别人劳动的量构成,也就是由它能够买到的并由它本身包含的劳动量决定的那个社会劳动的量构成。其实,这里只包含着交换价值的概念——我的劳动只有作为社会劳动才决定我的财富,因而我的财富是由使我能够支配等量社会劳动的我的劳动产品决定的。我的商品包含着一定量必要劳动时间,它使我能够支配任何其他具有相等价值的商品,因而支配物化在其他使用价值中的等量的别人劳动。这里的重点在于分工和交换价值引起的对\textbf{我的}劳动和\textbf{别人}劳动的同等看待,换句话说,对社会劳动的同等看待(亚当忽略了一点:连\textbf{我的}劳动,或者我的商品中包含的劳动,也已经被\textbf{社会地}规定,它已经根本改变了自己的性质),而决不在于\textbf{物化}劳动同\textbf{活}劳动之间的差别和两者交换的特殊规律。事实上,亚·斯密在这里谈的仅仅是:商品的价值决定于它们所包含的劳动时间,商品所有者的财富由他所支配的社会劳动量构成。

然而,把\textbf{劳动}和\textbf{劳动的产品}等同起来[249],的确在这里已经为混淆两种不同的价值规定——商品价值决定于它们所包含的劳动量;商品价值决定于用这些商品可以买到的活劳动量,即商品价值决定于劳动的价值——提供了最初的根据。既然亚·斯密说:

\begin{quote}“一个人财富的多少同这个权力的大小恰成比例,也就是说,同他能够支配的他人劳动量成比例,或者同样可以说〈这里就错误地等同起来!〉,同他能够买到的他人劳动的产品成比例。”(同上,第 61 页)\end{quote}

那末,他同样可以说:一个人的财富同他自己的商品即他的“财富”所包含的社会劳动量成比例。斯密也指出了这一点:

\begin{quote}“它们〈商品〉包含着一定量劳动的价值,我们就用这一定量的劳动去同假定在当时包含着\textbf{同量劳动的}价值的东西相交换。”(“\textbf{价值}”一词在这里是多余的,没有意义的。)\end{quote}

错误的结论已经在这第五章中表现出来,例如他说:

\begin{quote}“这样看来,劳动\textbf{本身的价值}决不改变,因而劳动是在任何时候和任何地方都可以用来衡量和比较一切商品的价值的唯一真实的和最终的尺度。”(同上,第 66 页)\end{quote}

在这里,把适用于劳动本身,因而也适用于劳动尺度即劳动时间的话——无论\textbf{劳动价值}如何变化,商品价值总是同物化在商品中的劳动时间成比例——硬用于这个变化不定的劳动价值本身。

在这里,亚·斯密只是考察一般商品交换:交换价值、分工以及货币的性质。交换者还只是作为商品所有者相对立。他们是以购买商品的形式购买别人的劳动,就象他们本人的劳动也是以商品的形式出现一样。因此,他们所支配的社会劳动量,等于他们自己用来购买东西的那个商品所包含的劳动量。但是,亚·斯密在以后几章谈到物化劳动和活劳动之间的交换、资本家和工人之间的交换,而且\textbf{强调指出},现在商品的价值已经不再决定于商品本身所包含的劳动量,而决定于这个商品可以支配即可以买到的、和商品本身包含的劳动量不同的别人活劳动的量,实际上他这种说法决不意味着,商品本身现在已经不按照商品所包含的劳动时间来进行交换。这只是意味着,\textbf{发财致富},商品所包含的价值的增殖以及这种增殖的程度,取决于物化劳动所推动的活劳动量的大小。只有这样理解,这才是正确的,但斯密在这里仍然没有弄清楚。

\tsectionnonum{[(2)斯密对剩余价值的一般见解。把利润、地租和利息看成工人劳动产品的扣除部分]}

[250]在第一篇第六章,亚·斯密从假定生产者只作为商品出卖者和商品所有者互相对立的关系,转到劳动条件所有者和单纯的劳动能力所有者之间进行交换的关系。

\begin{quote}“在\textbf{资本积累和土地私有制产生之前的}社会原始不发达状态中,\textbf{为获得各种交换对象所必要的劳动量},看来是能够提供交换准则的唯一根据……通常需要两天或两小时劳动制造的产品,自然比通常需要一天或一小时劳动制造的产品,有加倍的价值。”(加尔涅的译本,第 1 卷第 1 篇第 6 章第 94—95 页)\end{quote}

因此,生产各种商品所必要的劳动时间,决定着商品相互交换的比例,换句话说,决定着它们的\textbf{交换价值}。

\begin{quote}“在这种情况下,全部劳动产品属于劳动者,通常为获得或生产某一商品所耗费的劳动量,是决定用这个商品通常可以买到、支配或换得的那个劳动量的唯一条件。”(同上,第 96 页)\end{quote}

由此可见,在这种前提下,劳动者是单纯的商品出卖者,一个人只有在他用自己的商品购买别人的商品的时候,才能支配别人的劳动。因此,他用自己的商品所支配的别人劳动的量,只有他自己的商品中包含的那样多,因为他们两个人彼此交换的只是商品,而商品的交换价值是由商品中包含的劳动时间或劳动量决定的。

但是,亚当继续说道:

\begin{quote}“一旦\textbf{资本在个别人手中积累起来},其中某些人自然就利用它使勤劳者去劳动,向他们提供材料和生活资料,\textbf{以便从他们的劳动产品的出售中,\CJKunderdot{或者说},从这些工人的劳动加到那些加工材料价值上的东西中,取得利润}。”(同上,第 96 页)\end{quote}

在继续往下读以前,我们先在这里停一下。首先,既无生存资料,又无劳动材料的“勤劳者”——失去了立足之地的人,究竟是从哪里来的呢\fontbox{?}如果把斯密说法中的天真形式去掉,它的含义就是:资本主义生产是在劳动条件归一个阶级所有,而另一个阶级仅仅支配劳动能力的时刻开始的。劳动和劳动条件的这种分离成为资本主义生产的前提。

但是,第二,亚·斯密说,“雇主”使用工人,“以便\textbf{从他们的劳动产品的出售中,\CJKunderdot{或者说}},从这些工人的劳动[251]加到那些加工材料价值上的东西中,\textbf{取得利润}”,这句话是什么意思呢\fontbox{?}他是不是说:利润从\textbf{出售}中产生,商品\textbf{高于}它的价值出售,因此利润是斯图亚特所说的“让渡利润”,它无非是“财富的天平在有关双方之间的摆动”\authornote{见本册第 11—13 页。——编者注}\fontbox{?}下面就是他自己的回答:

\begin{quote}“在用\textbf{成品}同货币或\textbf{劳动}〈这里是新的错误的根源〉或其他商品交换时,除了偿付材料价格和工人工资以外,\textbf{还必须有一些东西},作为在这个事业上冒风险投资的企业主的利润。”(同上)\end{quote}

至于这个“风险”,在后面谈到对利润的辩护论的解释那一章(见第 VII 本札记本第 173 页)\endnote{马克思引用的是他摘录所读过的著作的“札记本”之一。在第 VII 本札记本第 173 页(根据第 VII 本这一部分的报纸摘录来判断,第 173 页写于 1860 年 1 月),马克思引用了斯密的《国富论》第一篇第六章中的话,并加了批语,指出企图从“企业主的风险”中得出利润是荒谬的。至于“对利润的辩护论的解释那一章”,马克思原来是打算为他的关于“资本一般”的研究的第三部分写的。在《剩余价值理论》第三册(手稿第 777 页)中,马克思在同一意义上提到要写的《对资本和雇佣劳动的关系的辩护论的解释》一节。马克思在 1861—1863 年手稿第 X 本中分析魁奈的《经济表》时,对于把利润看成“风险费”的资产阶级观点也进行了批判(见本册第 332—340 页)。——第 57 页。}再讲。“在用成品交换时作为企业主的\textbf{利润}的一些东西”,是不是由于商品高于它的价值出售而产生的呢\fontbox{?}它是不是斯图亚特的“让渡利润”呢\fontbox{?}

\begin{quote}亚当紧接着说:“因此,\textbf{工人加到材料上的价值,这时}〈从资本主义生产发生的时候起〉分成\textbf{两部分,一部分支付工人的工资,另一部分支付企业主的利润,作为他预付工资和加工材料的资本总额的报酬}。”(同上,第 96—97 页)\end{quote}

可见,斯密在这里说得十分明确:出售“成品”时所得的利润,不是从\textbf{出售本身}产生的,不是由于商品\textbf{高于}它的价值出售而产生的,它不是“让渡利润”。情况恰恰相反。工人加到材料上的价值即劳动量分成两部分。一部分支付工人的工资,或者说,已经用工人得到的工资支付。工人交还的这一部分劳动量,只等于他们以工资形式得到的劳动量。另一部分构成资本家的利润,它是资本家没有支付过代价而拿去出售的一定量劳动。因此,如果资本家按商品价值即按商品中包含的劳动时间来出售商品,换句话说,如果这一商品按价值规律同别的商品相交换,那末,资本家的利润就由于资本家对商品中包含的一部分劳动没有\textbf{支付过代价却拿去出售}而产生。这样一来,亚·斯密自己就驳倒了自己的这种想法,即认为当工人的全部劳动产品已不再属于工人自己,他不得不同资本所有者分享这种产品或产品价值的时候,商品相互交换的比例即商品的交换价值决定于物化在商品中的劳动时间量这一规律就会失效。何况他自己就认为,正因为资本家对加到商品上的一部分劳动没有支付过代价,所以产生了他在出售商品时得到的利润。后面我们将会看到,斯密后来更直接地从工人超出他用来\textbf{支付}(即用等价物来补偿)自己工资的那个劳动量之上所完成的劳动,引伸出利润。从而斯密认识到了剩余价值的真正起源。同时他还十分明确地指出,剩余价值不是从[252]预付基金中产生的,无论预付基金在现实的劳动过程中如何有用,它的价值不过是在产品中再现而已。剩余价值仅仅是在新的生产过程中从“工人\textbf{加到材料上的}”新劳动中产生的,在这个新的生产过程中,预付基金表现为劳动资料或劳动工具。

相反,“在用成品同货币\textbf{或劳动}或其他商品交换时”这句话,是不对的(并且是由于前面提到的混淆产生的)。

在资本家用商品同货币或商品交换的时候,他的利润所以产生,是因为他拿去出售的劳动量多于他支付过代价的劳动量,就是说,资本家没有用等量的物化劳动去交换等量的活劳动。因此,亚·斯密不该把成品“同货币或其他商品交换”和“成品同劳动交换”相提并论。因为在前一类交换中,剩余价值所以产生,是由于商品按它们的价值交换,按它们包含的劳动时间交换,但是这个劳动时间中有一部分是\textbf{没有支付过代价}的。这里的前提是:资本家不是用等量的过去劳动交换等量的活劳动;他占有的活劳动量大于他支付过代价的活劳动量。否则工人的工资就会等于他的产品的价值了。因此,在用“成品”同货币或商品交换时,即在它们按它们的价值交换时,利润所以产生,是因为“成品”同活劳动的交换服从于另外的规律,这里不是等价物相交换。因而这两种情况不能混为一谈。

可见,利润不是别的,正是工人加到劳动材料上的价值中的扣除部分。但工人加到材料上的无非是新的劳动量。所以,工人的劳动时间分为两部分:其中一部分,工人用来向资本家换得一个等价物,即自己的工资;另一部分,由工人无偿地交给资本家,从而构成\textbf{利润}。亚·斯密正确地强调指出,只有工人新加到材料上的那部分劳动(价值)才分解为工资和利润;所以,新创造的剩余价值本身,同花费在材料和工具上的那部分资本,是毫不相干的。

亚·斯密这样把利润归结为对无酬的别人劳动的占有之后,接着说:

\begin{quote}“可是,也许有人会说,资本的利润不过是特种劳动即监督和管理的劳动的工资的别名。”(第 97 页)\end{quote}

他也反驳了这种关于“监督和管理的劳动”的错误观点。我们在后面另一章还要谈到这个问题。\endnote{马克思在《剩余价值理论》第三册(论拉姆赛的一章和补充部分《收入及其源泉。庸俗政治经济学》)中,对于把企业主的收入看成资本家因进行“监督和管理的劳动”而取得的工资这种辩护论观点进行了批判。并见马克思《资本论》第 1 卷第 11 章和第 3 卷第 23 章。——第 59 页。}这里重要的只是指出,亚·斯密清楚地看到并且坚决地强调,他的关于利润起源的观点是同这种辩护论观点对立的。在强调这种对立之后,他继续说道:

\begin{quote}[253]“在这种情况下,劳动者的全部劳动产品并不总是属于劳动者。在大多数场合,他必须同雇用他的\textbf{资本所有者}一起分享劳动产品。在这种情况下,通常为获得或生产某一商品所耗费的劳动量,不再是决定用这个商品通常可以买到、支配或换得的那个劳动量的唯一条件。显然,这里还应当有一个劳动的\textbf{追加量},价为预付工资和给工人提供材料的资本的利润。”(同上,第 99 页)\end{quote}

这完全正确。如果我们谈的是资本主义生产,那末表现为货币或商品的物化劳动所买到的,除了它本身包含的劳动量之外,总还有一个活劳动的“追加量”,“作为资本的利润”,但是,换句话说,这不过意味着,物化劳动无偿地占有,不付代价地占有一部分活劳动。斯密胜过李嘉图的地方是,他有力地强调指出,这个变化是随着资本主义生产而出现的。相反,斯密不如李嘉图的地方是,他总不能摆脱掉被他自己在研究过程中驳倒了的那种观点,即认为由于物化劳动和活劳动相交换时发生的这种新关系,商品(它们彼此不过代表物化劳动,代表已知量的实现了的劳动)的相对价值规定也就发生变化。

斯密把剩余价值的一种形式,利润形式,表述为工人超出他补偿自己工资的那部分劳动之上所完成的劳动部分以后,对于剩余价值的另一种形式——\textbf{地租},也作了同样的表述。从劳动那里夺走因而作为别人的财产同劳动相对立的劳动的一个物质条件是\textbf{资本};另一个物质条件是\textbf{土地}本身,是作为\textbf{地产}的土地。所以,亚·斯密谈完了“\textbf{资本所有者}”之后接着说:

\begin{quote}“一旦一个国家的土地全部变成了私有财产,土地所有者也象\textbf{所有其他人}一样,喜欢在他们未曾播种的地方得到收获,甚至对土地的自然成果也索取\textbf{地租}……他〈劳动者〉必须\textbf{把用自己的劳动收集或生产的东西}让给土地所有者一部分,这\textbf{一部分},或者说,这一部分的价格,就构成\textbf{地租}……”(同上,第 99—100 页)\end{quote}

因此,象真正的工业利润一样,地租只不过是工人加到材料上的一部分劳动,也就是“\textbf{他让给}”土地所有者、无偿地给予土地所有者的一部分劳动;因此它只不过是超出工人补偿自己工资(或为工资中包含的劳动时间提供等价物)的那部分劳动时间之上所完成的剩余劳动部分。

可见,亚·斯密把\textbf{剩余价值},即剩余劳动——已经完成并物化在商品中的劳动超过有酬劳动即超过以工资形式取得自己等价物的劳动的余额——理解为\textbf{一般范畴},[254]而本来意义上的利润和地租只是这一般范畴的分枝。然而,他并没有把剩余价值本身作为一个专门范畴同它在利润和地租中所具有的特殊形式区别开来。斯密尤其是李嘉图在研究中的许多错误和缺点,都是由此而产生的。

剩余价值的另一种表现形式是\textbf{资本利息},借贷利息(货币的利息)。但这种

\begin{quote}“\textbf{货币的利息总是}〈斯密在同一章说道〉\textbf{一种派生的收入},如果它不从使用这些货币所取得的\textbf{利润}中支付,那也一定是从他种收入源泉中支付\end{quote}

(因此,不是从地租中支付,就是从工资中支付;在后一种情况下,假定平均工资是已知的,利息就不是从剩余价值中取得,而是工资中的扣除部分,或者说,不过是利润的另一种形式——在以后的研究过程中我们将会看到,在资本主义生产不发达的条件下,利息就是以这种形式出现的)\endnote{马克思在补充部分《收入及其源泉。庸俗政治经济学》中考察了资本的“洪水期前的形式”这一问题(手稿第 899—901 页)。并见《资本论》第 3 卷第 36 章《资本主义以前的状态》。——第 61 页。},

\begin{quote}除非借债人是靠借新债来还旧债利息的浪费者。”(同上,第 105—106 页)\end{quote}

可见,利息或者是用借来的资本赚得的\textbf{利润}的一部分;在这种情况下,利息就是利润本身的派生形式,是它的一个分枝,因而只是以利润形式占有的剩余价值在不同的人之间的进一步分配。利息或者是从地租中支付,那末情况也是一样。最后,利息或者是由借债人从自己的资本或别人的资本中支付;在这种情况下,利息就根本不是剩余价值,而只是已有财富的另一次分配,是“财富的天平在有关双方之间的摆动”,就象“让渡利润”一样。除了利息根本不是剩余价值的形式这最后一种情况之外(并且除了利息是工资中的扣除部分,或者说,它本身是利润的一种形式的情况之外;最后这种情况,亚当根本没有谈到),利息只是剩余价值的派生形式,只是利润或地租的一部分(这只同利润和地租的分配有关);因而利息也无非表现了无酬的剩余劳动的一部分。

\begin{quote}“放债人总是把借出生息的货币资金\textbf{看成}资本。他希望货币资金能按期归还,而借债人在这个期间将为使用这笔货币资金,付给放债人一定的年金。借债人可以把这笔资金当作\textbf{资本}来使用,也可以当作\textbf{直接消费基金}来使用。如果他把这笔资金当作资本来使用,他就用它们来维持生产工人的生活,\textbf{而工人则再生产出它们的价值,并提供利润}。在这种情况下,他不转让和动用任何其他的收入源泉,就可以归还资本,支付利息。如果他把这笔资金用于直接消费,他就成了一个浪费者,把那些原来应该维持勤劳者生活的东西挥霍在有闲者身上。在这种情况下,他不转让或动用别的收入源泉,如地产或地租,就不能归还资本,也不能支付利息。”(麦克库洛赫版,第 2 卷第 2 篇第 4 章第 127 页)\end{quote}

[255]因此,借进货币的人——从这里来看,是指借进资本的人——或者他自己把货币用作资本,从中取得利润。在这种情况下,他付给放债人的利息,不过是利润的一个具有\textbf{特殊名称}的部分。或者他把借来的货币浪费掉,那末,他就会使放债人的财富增加,而使自己的财富减少。这里发生的仅仅是财富的另一次分配,财富从浪费者手里转到高利贷者手里,但在这里没有创造剩余价值的过程。由此看来,只要利息一般代表剩余价值,它就不过是利润的一部分,而利润本身又无非是剩余价值即无酬劳动的一定形式。

最后,亚·斯密指出,连靠税收生活的人的一切收入,也是或者由工资支付,即成为工资本身的扣除部分,或者来源于利润和地租,因而只是意味着各个不同社会阶层分享利润和地租的权利,而利润和地租本身只是剩余价值的不同形式。

\begin{quote}“一切税收和以税收为基础的一切收入——薪俸、津贴、各种年金——归根到底都是从收入的这三个原始源泉中得来的,都是直接或间接地从工资、资本的利润或者地租中支付的。”(加尔涅的译本,第 1 篇第 6 章第 106 页)\end{quote}

因此,货币利息以及税收和由税收而来的收入,只要不是工资本身的扣除部分,那就只是利润和地租的分成而已,而利润和地租又归结为剩余价值,即无酬的劳动时间。

这就是亚·斯密的一般剩余价值理论。

亚·斯密又一次把自己的整个见解加以总括。这里看得特别清楚,他并不打算哪怕是稍微证明一下:工人加到产品上的价值(在扣除生产费用即原料和劳动工具的价值之后),由于工人不是全部占有这个价值,而是被迫同资本家和土地所有者分享这个价值或产品,似乎已不再由包含在产品中的劳动时间决定了。商品的价值以什么方式在商品生产者之间分配,这当然丝毫不会改变这个价值的性质,以及商品与商品之间的价值比例。

\begin{quote}“一旦土地成为私有财产,对劳动者在这块土地上所能生产和收集的几乎一切产品,土地所有者都要求得到一份。\textbf{他的地租是对耕种土地的劳动所生产的产品的第一个扣除}。但是,种地人在收获以前很少有维持自己生活的资金。他的生活费通常是从他的雇主即租地农场主的资本中预付的,如果租地农场主不能从劳动者的劳动的产品中得到一份,或者说,如果他的资本不能得到补偿并带来利润,他就没有兴趣雇人了。\textbf{这种利润是}[256]\textbf{对耕种土地的劳动所生产的产品的第二个扣除}。几乎所有其他劳动的产品都要作\textbf{这样的扣除,来支付利润}。在所有手工业和制造业中,大多数劳动者都需要雇主预付给他们劳动材料以及工资和生活费,直到劳动完成的时候为止。\textbf{这个雇主从他们劳动的产品中得到一份,或者说,从他们的劳动加到加工材料上的价值中得到一份,这一份也就是雇主的利润}。”(麦克库洛赫版,第 1 卷第 1 篇第 8 章第 109—110 页)\end{quote}

总之,亚·斯密在这里直截了当地把地租和资本的利润称为纯粹是工人产品中的\textbf{扣除部分},或者说,是与工人加到原料上的劳动量相等的产品价值中的\textbf{扣除部分}。但是,正如亚·斯密自己在前面证明过的,这个扣除部分只能由工人加到原料上的、超过只支付他的工资或只提供他的工资等价物的劳动量的那部分劳动构成;因而这个扣除部分是由工人的剩余劳动,即工人劳动的无酬部分构成。(因此,顺便指出,利润和地租,或者说,资本和地产,决不可能是“\textbf{价值的源泉}”。)

\tsectionnonum{[(3)斯密把剩余价值的概念推广到社会劳动的一切领域]}

我们看到,在对剩余价值的分析上,因而在对资本的分析上,亚·斯密比重农学派前进了一大步。在重农学派的著作中,创造剩余价值的,仅仅是一个特定种类的实在劳动——农业劳动。因此,他们考察的是劳动的使用价值,而不是劳动时间,不是作为价值的唯一源泉的一般社会劳动。而在这特定种类的劳动中,实际上创造剩余价值的又是\textbf{自然},是土地,剩余价值被理解为物质(有机物质)的量的增加,理解为生产出来的物质超过消费了的物质的余额。他们还只是在十分狭隘的形式中考察问题,因而夹杂着空想的观念。相反,在亚·斯密的著作中,创造价值的,是一般社会劳动(不管它表现为哪一种使用价值),仅仅是必要劳动的量。剩余价值,无论它表现为利润、地租的形式,还是表现为派生的利息形式,都不过是劳动的物质条件的所有者在同活劳动交换过程中占有的这种必要劳动的一部分。因此,在重农学派看来,剩余价值只表现为地租形式,而在亚·斯密看来,地租、利润和利息都不过是剩余价值的不同形式。

我把与预付资本总额相联系的剩余价值,称为\textbf{资本的利润},我所以这样称谓,是因为直接参与生产的资本家\textbf{直接}占有剩余劳动,不管他以后还要把这个剩余价值分成哪些项目,也不管是同土地所有者分享,还是同资本的出借人分享。例如租地农场主直接向土地所有者支付;例如工厂主从他占有的剩余价值中向工厂地基所有者支付地租,向出借资本给他的资本家支付利息。

[257]\fontbox{~\{}现在还有以下几点要考察:(1)亚·斯密把剩余价值和利润混淆起来;(2)他关于生产劳动的观点;(3)他如何把地租和利润变为\textbf{价值的源泉},他对商品的“自然价格”的分析如何错误,他认为,在商品的“自然价格”中,原料和劳动工具的价值不应离开三个“收入源泉”的“价格”而存在,因而不应单独加以考察。\fontbox{\}~}

\tsectionnonum{[(4)斯密不懂得价值规律在资本同雇佣劳动的交换中的特殊作用]}

工资,或者说,资本家用来购买对劳动能力的暂时支配权的等价物,不是直接形式的商品,而是经过了形态变化的商品,是货币,即作为交换价值,作为社会劳动、一般劳动时间的直接化身的独立形式的商品。当然,工人和任何其他货币所有者一样,按照同样的价格用这些货币购买商品\fontbox{~\{}关于那些细节,例如工人是在对他比较不利的条件和情况下购买,等等,这里撇开不谈\fontbox{\}~}。工人象任何其他货币所有者一样,作为买者同商品的卖者相对立。在商品流通过程本身,工人不是作为工人,而是作为同商品极相对立的货币极,作为随时可以交换的一般形式的商品的所有者出现。他的货币又转化为给他充当使用价值的商品,他在这个交换过程中,按市场上出卖商品的价格,一般说来,按商品的价值购买商品。他在这里只完成 G—W 的行为,从其一般形式来看,这个行为表示的是形式的改变,而决不是价值量的改变。但是,因为工人通过他自己的物化在产品中的劳动,不仅加进了包含在他获得的货币中的那个劳动时间量,不仅支付了等价物,而且还无偿地提供了恰恰成为利润源泉的剩余劳动,所以,\textbf{实际上}(从结果来看,包含在劳动能力出卖中的中介运动不见了)工人提供的价值,高于作为他的工资的那个货币额的价值。他用更多的劳动时间购得了作为工资流到他手里的货币所体现的那个劳动量。因此可以说,工人购买由他挣得的货币(这只是一定量社会劳动时间的独立表现)转化成的那一切商品,间接地用了比这些商品包含的劳动时间更多的劳动时间,虽然他和任何其他买者一样,或者说,和完成了第一转化的商品的所有者一样,按照同样的价格购买商品。相反,资本家用来购买劳动的货币包含的劳动量或劳动时间,比工人生产的商品包含的劳动量或劳动时间要少。除了作为工资的那个货币额所包含的劳动量之外,资本家还买到一个他没有支付过代价的追加劳动量,即超出他支付的货币所包含的劳动量的余额。这个追加劳动量也就构成资本所创造的剩余价值。

但是,因为[258]资本家用来购买劳动(从结果来看,实际上是购买劳动,虽然这里也是通过同劳动能力的交换作为中介,而不是直接同劳动交换)的货币,无非是\textbf{其他一切商品}的转化形式,是其他一切商品作为交换价值的独立存在,所以也可以说,一切商品在同活劳动相交换时买到的劳动多于这些商品本身所包含的劳动。这个追加量也就构成剩余价值。

亚·斯密的巨大功绩在于:他正是在第一篇的几章(第六、七、八章)中,即在从简单商品交换及其固有的价值规律转到物化劳动同活劳动之间的交换,转到资本和雇佣劳动之间的交换,转到从一般形式来考察利润和地租,总之,转到剩余价值的起源问题的那几章中,就感觉到已经出现了缺口;他感觉到,——不管他所不理解的中介环节是怎样的,——从结果来看,规律实际上是失效了:较大量的劳动同较小量的劳动相交换(从工人方面说),较小量的劳动同较大量的劳动相交换(从资本家方面说)。斯密的功绩在于,他强调指出了下面这一点(而这一点也把他弄糊涂了):\textbf{随着资本积累和土地所有权的产生},因而随着同劳动本身相对立的劳动条件的独立化,发生了一个转变,价值规律似乎变成了(从结果来看,也确实变成了)它的对立面。如果说,亚·斯密的理论的长处在于,他感觉到并强调了这个矛盾,那末,他的理论的短处在于,这个矛盾甚至在他考察一般规律如何运用于简单商品交换的时候也把他弄糊涂了;他不懂得,这个矛盾之所以产生,是由于劳动能力本身成了商品,作为这种特殊的商品,它的使用价值本身(因而同它的交换价值毫无关系)是一种创造交换价值的能力。李嘉图胜过亚·斯密的地方是,这个似乎存在而从结果来看也确实存在的矛盾,并没有把他弄糊涂。但是,他不如亚·斯密的地方是,他竟从来没有料到这里有问题,因此价值规律随着资本的出现而发生的\textbf{特殊}发展,丝毫没有引起他的不安,更没有促使他去研究。后面我们将会看到,亚·斯密著作中的天才的东西,到马尔萨斯著作中怎样变成了攻击李嘉图观点的反动的东西。\endnote{在《剩余价值理论》第三册《托·罗·马尔萨斯》一章(手稿第 753—781 页)中,马克思对马尔萨斯的价值观点和剩余价值观点作了详细的批判(手稿第 753—767 页)。——第 50、67 页。}

当然,正是这个观点,使亚·斯密摇摆不定、没有把握,它抽掉了他脚下的坚实基础,使他和李嘉图相反,不能做到对资产阶级制度的抽象的一般基础有一个连贯的理论见解。

[259]前面提到,亚·斯密说,商品买到的劳动多于商品本身所包含的劳动,或者说,工人为商品支付的价值大于商品所包含的价值,这个论点在霍吉斯金的《通俗政治经济学》一书中是这样表达的:

\begin{quote}“\textbf{自然价格}(或\textbf{必要}价格)意味着自然为生产某一商品而要求于人的总\textbf{劳动量}……在我们同自然的相互关系中,劳动是最初的并且永远是唯一的购买手段。不管生产某一商品所必需的劳动量有多大,在现代社会状态下,工人为了获得并占有这个商品所必须付出的劳动,总是比向自然直接购买时所必需的劳动多得多。这样增加了(对工人来说)的自然价格,就是\textbf{社会价格}。必须随时把这两种价格区别开来。”(\textbf{托马斯·霍吉斯金}《通俗政治经济学》1827 年伦敦版第 219—220 页)\end{quote}

霍吉斯金的这种看法既反映了亚·斯密的见解中正确的东西,又反映了使斯密本人糊涂也使别人糊涂的东西。

\tsectionnonum{[(5)斯密把剩余价值同利润混淆起来。斯密理论中的庸俗成分]}

我们已经看到,亚·斯密如何考察一般剩余价值,而地租和利润只不过是\textbf{剩余价值}的不同形式和组成部分。按照他的解释,由原料和生产资料构成的那部分资本,同剩余价值的创造没有任何直接的关系。剩余价值完全是从工人提供的\textbf{超出}仅仅构成他的工资等价物的那部分劳动\textbf{之上}的追加劳动量产生的。因而,剩余价值只是直接由花费在工资上的那部分资本产生的,因为这是资本中唯一不仅再生产自己,而且生产一个“余额”的部分。相反,在利润的形式上,剩余价值是按照预付资本总额计算的,而且,除了这一个形态变化之外,由于资本主义生产各个不同领域中利润的平均化,还有一些新的形态变化。

亚当虽然实质上是考察剩余价值,但是他没有清楚地用一个不同于剩余价值特殊形式的特定范畴来阐明剩余价值,因此,后来他不通过任何中介环节,直接就把剩余价值同更发展的形式即利润混淆起来了。这个错误,在李嘉图和以后的所有经济学家的著作中,仍然存在。由此就产生了一系列不一贯的说法、没有解决的矛盾和荒谬的东西(在李嘉图的著作中,这种情况更加突出,因为他更加系统而一致地、始终如一地贯彻了价值的基本规律,所以不一贯的说法和矛盾表现得更为突出),对于这一切,李嘉图学派企图用烦琐论证的办法,靠玩弄词句来加以解决(我们在后面关于利润那一篇中将会看到这一点)\endnote{见注 13。马克思在继续写作《剩余价值理论》的过程中,也对李嘉图学派的利润观点进行了批判。在《剩余价值理论》第三册《李嘉图学派的解体》一章中,马克思专门谈到李嘉图主义者詹姆斯·穆勒想用烦琐论证的方法来解决李嘉图的利润理论的矛盾,谈到约翰·斯图亚特·穆勒徒劳无益地试图直接从价值理论中得出李嘉图关于利润率和工资额成反比的论点。——第 69 页。}。粗俗的经验主义变成了虚伪的形而上学,变成了烦琐哲学,它绞尽脑汁,想用简单的形式抽象,直接从一般规律中得出不可否认的经验现象,或者巧妙地使经验现象去迁就一般规律。在这里,在分析斯密的剩余价值观点的时候,我们就举一个这方面的例子,因为斯密的混乱不是发生在他专门谈论利润和地租这些剩余价值的特殊形式的地方,而是发生在他把利润和地租仅仅当作剩余价值的一般形式,当作“工人加到材料上的劳动中的\textbf{扣除部分}”的地方。

[260]亚·斯密在第一篇第六章中说:

\begin{quote}“因此,工人\textbf{加到}材料上的价值,这时分成两部分,一部分支付工人的工资,另一部分支付企业主的利润,作为他预付工资和加工材料的资本总额的报酬。”[加尔涅的译本,第 1 卷第 96—97 页]\end{quote}

然后他接着说:

\begin{quote}“如果他〈企业主〉从出卖工人生产的产品中,除了用于补偿他的资本所必需的数额以外,不指望再多得一个余额,他就不会有兴趣雇用这些工人了;同样,如果他的利润不同所使用的资本的量成一定的比例,他就不会有兴趣使用较大的资本来代替较小的资本。”[同上,第 97 页]\end{quote}

首先我们要注意下面一点:亚·斯密起初把剩余价值,即“企业主”除了“用于补偿他的资本”所必需的价值量以外得到的那个“余额”,归结为工人加到材料上的劳动中超出补偿他们工资的劳动之上的部分;因而,斯密完全是从花费在工资上的那部分资本中得出这个余额的。但是,随后他马上就从利润的形式来考察这个余额,也就是说,不把这个余额同它所由产生的那部分资本联系起来看,而认为它是超出预付资本总价值,即超出“预付工资\textbf{和}加工材料〈由于疏忽,这里遗漏了生产工具〉的资本总额”之上的余额。因此,他是直接从利润的形式来考察剩余价值的。从而立刻就产生了困难。

亚·斯密说:如果资本家“从出卖工人生产的产品中,除了用于补偿他的资本所必需的数额以外,不指望\textbf{再多得一个余额},他就不会有兴趣雇用这些工人了”。只要以资本主义关系为前提,那末这句话是完全正确的。资本家不是为了用产品满足自己的需要而生产,一般说来,消费并不是他生产的直接目的。他是为生产剩余价值而生产。但是,亚·斯密不象他后来的许多无能的门徒,他并不是用这个仅仅表明在资本主义生产条件下资本家是为了剩余价值而生产的前提来\textbf{说明剩余价值}的。这就是说,他不是用资本家的兴趣,资本家追求剩余价值的愿望,来说明剩余价值的存在。相反,他已经从工人“加到材料上的、超出他为补偿所得工资而加的价值之上”的那个价值,得出了剩余价值。但是,紧接着他又说:如果资本家的利润不同预付资本的量成一定的比例,他就不会有兴趣使用较大的资本来代替较小的资本。这里,已经不是用剩余价值的本质,而是用资本家的“兴趣”来说明利润了。这是庸俗的和荒谬的。

亚·斯密没有感觉到,他这样直接地把剩余价值同利润,又把利润同剩余价值混淆起来,也就推翻了刚才由他提出的剩余价值起源的规律。[261]如果剩余价值只是“工人\textbf{所加}的、\textbf{超出}他为补偿自己工资而加到材料上的那个数额之上”的“那部分价值〈或劳动量〉”,那末,为什么这部分价值会直接因为预付资本的价值在一种场合下比在另一种场合下大,就一定增加呢\fontbox{?}亚·斯密紧接着自己举了一个例子,来驳斥那种说利润是所谓“监督的劳动”的工资的意见,在这个例子中,上述的矛盾就更加明显了。

他是这样说的:

\begin{quote}“但是,它〈资本利润〉与工资根本不同;它受完全不同的原则支配,并且同这种所谓监督和管理的劳动的数量和性质不成任何比例。\textbf{它完全取决于所使用的资本的价值},它的大小同这一资本的多少成比例。例如,假定某地工业中的\textbf{资本利润每年通常为 10\%},有两个不同的制造厂,各雇 20 个工人,每个工人每年工资 15 镑,这样,每个制造厂每年支出的工资为 300 镑。再假定,其中一个制造厂加工低等材料,每年只花费 700 镑,另一个制造厂加工比较高等的材料,值 7000 镑。在这种情况下,前一个制造厂每年所使用的资本总共只有 1000 镑,而后一个制造厂所使用的资本达 7300 镑。按 10\%的比率,前一个厂主只能指望得到年利润约 100 镑,后一个厂主则可指望得到年利润约 730 镑。尽管他们的利润额相差这样大,但他们的监督和管理的劳动却可能是一样的,或者几乎一样的。”[加尔涅的法译本,第 1 卷第 97—98 页]\end{quote}

在这里,我们一下子就从一般形式的剩余价值转到同问题没有直接关系的一般利润率上来了。我们且往前看!两个工厂都使用 20 个工人的劳动,两处工人的工资都是 300 镑。可见,并不是其中的一个厂比另一个厂用了更高级的劳动,以致一个厂的一小时劳动,也就是说一小时剩余劳动,等于另一个厂的几小时剩余劳动。相反,两处都是假定用同样的平均劳动,这已由两个厂的工资相等这一点表明了。可是,为什么一个厂的工人在他们的工资价格之外加上的剩余劳动会比另一个厂高 6 倍呢\fontbox{?}或者说,虽然两处工人获得同样的工资,因而两处工人劳动了同样多的时间来[262]再生产这笔工资,可是,为什么一个厂仅仅因为加工的材料比另一个厂贵 6 倍,它的工人提供的剩余劳动就一定会多 6 倍呢\fontbox{?}

可见,一个厂比另一个厂所得到的利润多 6 倍这一情况,或者一般地说,利润同预付资本的量成比例的规律,乍一看来同剩余价值仅仅表明工人的无酬剩余劳动这个剩余价值规律,或者说利润规律(因为亚·斯密直接把剩余价值和利润等同起来),是矛盾的。亚·斯密极其天真地、不加思索地说出了这一点,丝毫没有想到这里产生的矛盾。所有后来的经济学家——由于他们当中没有一个人离开剩余价值的特定形式而从一般形式来考察剩余价值——在这方面都信守斯密的思想。前面已经指出,这一点在李嘉图的著作中,只不过表现得更突出罢了。

因为亚·斯密不仅把剩余价值归结为利润,而且归结为地租,——这是剩余价值的两个特殊形式,它们的运动取决于完全不同的规律,——所以仅仅这一点本来就应当使他意识到,决不能不通过任何中介环节,而把一般的抽象形式同它的任何一个特殊形式混淆起来。不论是斯密,还是后来所有的资产阶级经济学家,照例都缺乏对于阐明经济关系的形式差别所必要的理论认识,——他们都是粗略地抓住现成的经验材料,只对这些材料感兴趣。正由于这个原因,在问题纯粹涉及在价值量不变的条件下交换价值形式的各种变化的地方,他们也就不能正确地理解货币的本质。

\tsectionnonum{[(6)斯密把利润、地租和工资看成价值源泉的错误观点]}

\textbf{罗德戴尔}在《论公共财富的性质和起源》(拉让蒂·德·拉瓦伊斯译,1808 年巴黎版)一书中,反对斯密对剩余价值的理解,——他说斯密的理解同洛克早已提出的观点一致,——他指责说:按照这种理解,资本就不象斯密所说的那样是财富的原始源泉,而只是派生源泉。关于这个问题,罗德戴尔说道:

\begin{quote}“一百多年以前,洛克已经提出了〈同亚·斯密〉几乎一样的观点……洛克说,货币是不结果实的,它不会生产任何东西;从货币中得到的全部好处,就是它通过相互协议,把作为一个人的劳动报酬的利润转入另一个人的口袋。”(\textbf{罗德戴尔}的著作第 116 页)“如果对资本利润的这种理解真正正确的话,那就会得出结论说:利润不是收入的原始源泉,而只是派生源泉,并且,决不能把资本看作财富的源泉之一,因为资本带来的利润不过是收入从工人的口袋转到资本家的口袋而已。”(同上,第 116—117 页)\end{quote}

从资本的价值再现在产品中这一点来说,不能把资本称为“财富的源泉”。在这里,资本仅仅作为积累的劳动,作为一定量的物化劳动,把自己的价值加到产品上。

资本只有作为一种\textbf{关系},——从资本作为对雇佣劳动的强制力量,迫使雇佣劳动提供剩余劳动,或者促使劳动生产力去创造相对剩余价值这一点来说,——才生产价值。在这两种情况下,资本都只是[263]作为劳动本身的物质条件所具有的从劳动异化的而又支配劳动的力量,总之,只是作为雇佣劳动本身的一种形式,作为雇佣劳动的条件,才生产价值。按照经济学家通常的理解,资本是以货币和商品形式存在的积累的劳动,它象一切劳动条件(包括不花钱的自然力在内)一样,在劳动过程中,在创造使用价值时,发挥生产性的作用,但它永远不会成为价值的源泉。资本不创造任何新价值,一般地说,它把交换价值加到产品上,只是由于它本身具有交换价值,也就是说,由于它本身可归结为物化劳动时间,因而由于劳动是它的价值的源泉。

罗德戴尔说得对:亚·斯密在研究了剩余价值和价值的本质之后,错误地把资本和土地说成是交换价值的独立源泉。资本和土地,只有成为占有工人超过为补偿他的工资所必需的劳动时间而被迫完成的那一定量剩余劳动的根据,才是它们的所有者的收入的源泉。例如,亚·斯密说:

\begin{quote}“\textbf{工资、利润和地租},是一切收入的\textbf{三个原始源泉,也是一切交换价值的三个原始源泉}。”(第 1 篇第 6 章)\end{quote}

说它们是“一切收入的三个原始源泉”,这是对的;说它们“也是\textbf{一切交换价值}的三个原始源泉”,就不对了,因为商品的价值是完全由商品中包含的劳动时间决定的。亚·斯密刚说了地租和利润纯粹是工人加到原料上的价值或劳动中的“扣除部分”,怎么可以随后又把它们称为“交换价值的原始源泉”呢\fontbox{?}(只有在推动“原始源泉”,即强迫工人完成剩余劳动这个意义上说,它们才能起这种作用。)它们只有成为占有一部分价值即一部分物化在商品中的劳动的根据(条件),才是它们的所有者的收入的源泉。但是价值的分配,或者说,价值的占有,决不是被占有的价值的源泉。如果没有这种占有,工人以工资形式得到自己劳动的全部产品,那末,虽然土地所有者和资本家没有来分享,生产出来的商品的价值仍然同从前一样。

土地所有权和资本,对于它们的所有者来说,是收入的源泉,也就是说,使它们的所有者有权占有劳动创造的价值的一部分,可是它们并不因此就成为它们的所有者占有的价值的源泉。但是,说工资构成交换价值的原始源泉,同样是不正确的,虽然工资,或者确切些说,劳动能力的不断出卖,也构成工人的收入的源泉。创造价值的是工人的劳动,而不是工人的工资。工资只不过是已经存在的价值,或者从整个生产来看,只不过是工人创造的价值中由工人自己占有的那一部分。但是这种占有并不创造价值。因此,工人的工资可以增减,但并不影响他们生产的商品的价值。[263]

[265]\fontbox{~\{}对于上述内容,还要补充以下一段引文,以证明亚·斯密把商品价值被占有时分成的不同项目说成这个价值的源泉。他在驳斥了那种认为利润只是资本家的工资或“监督劳动的工资”的别名的看法之后,得出结论说:

\begin{quote}“因此,在商品\textbf{价格}中,基金即资本的利润是与工资根本\textbf{不同的价值源泉},它受完全不同的原则支配。”(第 1 篇第 6 章)\end{quote}

然而,斯密刚才还证明,工人加到材料上的价值在工人和资本家之间以工资和利润的形式分配;因此,劳动是唯一的\textbf{价值源泉},“工资价格”和“利润价格”都是从这个价值源泉产生出来的。但是这些“价格”本身——无论工资还是利润——都不是\textbf{价值源泉}。\fontbox{\}~}[265]

\tsectionnonum{[(7)斯密对价值和收入的关系的看法的二重性。斯密关于“自然价格”是工资、利润和地租的总和这一见解中的循环论证]}

[263]在这里,我们想完全不谈亚·斯密在多大程度上把地租看作商品价格的构成要素。这个问题在这里对我们的研究更是无关紧要,因为斯密把地租看成和利润完全一样,纯粹是剩余价值的一部分,即“工人加到原料上的劳动中的扣除部分”。从而,斯密[264]实质上也把地租理解为“利润中的扣除部分”,因为全部无酬剩余劳动是由同劳动对立的资本家\textbf{直接}占有,不管他以后还要同生产条件所有者(无论土地所有者还是资本出借人)按哪些项目分享这个剩余价值。所以,为简单起见,我们将只谈工资和利润,作为新创造的价值分成的两个项目。

假定某一商品体现 12 小时的劳动时间(消费在这个商品上的原料和劳动工具的价值\textbf{撇开不谈})。这个商品的价值本身,我们只能用货币来表现。再假定 5 先令也体现 12 小时的劳动时间。在这种情况下,商品价值就等于 5 先令。亚·斯密所理解的“商品的自然价格”不是别的,正是以货币表现的商品价值。(商品的市场价格当然高于或低于商品的价值。甚至商品的平均价格也\textbf{总是不同于}商品的价值,这一点我将在后面说明。\endnote{“平均价格”(《Durchschnittspreis》)这一术语,马克思这里是指“生产价格”,就是指生产费用(C+v)加平均利润。马克思在《剩余价值理论》第二册——论洛贝尔图斯一章和《李嘉图和亚当·斯密的费用价格理论》一章中,考察了商品价值和商品的“平均价格”之间的相互关系问题。“平均价格”这一术语本身说明,这里所指的,正如马克思在手稿第 605 页(《李嘉图的地租理论[结尾]》一章)所解释的那样,是“一个相当长的时期内的平均市场价格,或者市场价格所趋向的中心”。——第 76 页。}但是亚·斯密在考察“自然价格”时,根本没有提到这个问题。况且,如果没有对价值本性的正确看法作基础,那就不能理解商品的市场价格,更不能理解商品的平均价格的波动。)

如果商品中包含的剩余价值,占商品总价值的 20\%,或者同样可以说,占商品中包含的必要劳动的 25\%,那末,这 5 先令价值,即商品的“自然价格”,就可以分为 4 先令工资和 1 先令剩余价值(在这里我们仿效亚·斯密,也把剩余价值叫做利润)。说不依赖于工资和利润而决定的商品价值量,或商品的“自然价格”,可以分为 4 先令工资(“劳动价格”)和 1 先令利润(“利润价格”),这是对的。但是,说商品价值由不受商品价值调节的工资价格和利润价格相加或合计而成,那就错了。在后一情况下,就找不到任何理由说明:为什么商品总价值不会随着人们假定工资等于 5 先令、利润等于 3 先令等等情况而成为 8 先令、10 先令等等。

亚·斯密在研究工资的“自然率”或工资的“自然价格”时所遵循的指导线索是什么呢\fontbox{?}是再生产劳动能力所必需的生活资料的自然价格。但是,他又用什么来决定这些生活资料的自然价格呢\fontbox{?}当他一般地决定这个价格时,他又回到正确的价值规定上来,也就是说,回到价值决定于生产这些生活资料所必要的劳动时间这个规定上来。但是,只要斯密离开这条正确的道路,他就陷入循环论证。决定工资自然价格的这些生活资料的自然价格,他是用什么来决定的呢\fontbox{?}用“工资”、“利润”、“地租”的自然价格;这三者构成这些生活资料的自然价格,也构成一切商品的自然价格。如此反复,以至无穷。关于供求规律的空谈,当然无助于摆脱这种循环论证。因为“自然价格”,或者说,与商品价值相适应的价格,恰好发生在供求彼此相符的时候,也就是在商品价格不因供求的波动而高于或低于商品价值的时候,换句话说,在商品的费用价格\endnote{“费用价格”(《Kostenpreis》或《Kostpreis》,《costprice》)这一术语,马克思用在三种不同的意义上:(1)资本家的生产费用(C+v),(2)同商品的价值一致的商品的“内在的生产费用”(C+v+m),(3)生产价格(C+v+平均利润)。这里,这一术语是用在第二种意义上,也就是指内在的生产费用。在《剩余价值理论》第二册中,“费用价格”这一术语马克思是用在第三种意义上,即生产价格,或“平均价格”。在那里马克思直接把这些术语等同了起来。例如,在手稿第 529 页,马克思写道:“……不同于价值本身的平均价格,即我们后面所说的费用价格,这个费用价格不直接决定于商品价值,而决定于预付在这些商品上的资本加平均利润。”在第 624 页,马克思指出:“价格是提供商品的必要条件,是使商品生产出来并作为商品出现在市场上的必要条件,它当然是商品的生产价格或费用价格。”在《剩余价值理论》第三册中,《Kostenpreis》这一术语马克思有时用在生产价格的意义上,有时用在资本家的生产费用的意义上,也就是指 C+v。《Kostenpreis》这一术语所以有三种用法,是由于《Kosten》(“费用”、“生产费用”)这个词在经济科学中被用在三种意思上,正如马克思在《剩余价值理论》第三册(1861—1863 年手稿第 788—790 页和第 928 页)特别指出的,这三种意思是:(1)资本家预付的东西,(2)预付资本的价格加平均利润,(3)商品本身的实在的(或内在的)生产费用。除了资产阶级政治经济学古典作家使用的这三种意思以外,“生产费用”这一术语还有第四种庸俗的意思,即让·巴·萨伊给“生产费用”下的定义:“生产费用是为劳动、资本和土地的生产性服务支付的东西。”(让·巴·萨伊《论政治经济学》1814 年巴黎第 2 版第 2 卷第 453 页)马克思坚决否定了对“生产费用”的这种庸俗的理解(例如见《剩余价值理论》第 2 册手稿第 506 页和第 693—694 页)。——第 77 页。}(或卖者供应的商品的价值)同时就是需求所支付的价格的时候。

[265]但是,前面已经说过,亚·斯密在研究工资的自然价格时,他事实上——至少在一些地方——又回到商品的正确的价值规定上来了。相反,在关于利润的自然率或利润的自然价格的那一章,就本应解决的题目来说,他却在毫无意义的老生常谈和同义反复之中迷失了方向。事实上,他原来是用商品价值来调节工资、利润和地租的。但是后来,他反过来了(这更接近于经验的外观和平常的印象),企图用工资、利润和地租的自然价格的相加数来决定商品的自然价格。李嘉图的主要功绩之一,就是消除了这种混乱。以后我们讲到李嘉图的时候,还要简单地谈谈这一点。\endnote{在《剩余价值理论》第二册中,篇幅巨大的论李嘉图的那一节是在马克思手稿第 XI、XII 和 XIII 本,其中有一章《李嘉图和亚·斯密的费用价格理论(批驳部分)》,马克思在那里又回过头来分析斯密的“自然价格”观点(手稿第 549—560 页)。——第 78 页。}

在这里我们还要指出的只是下面一点:作为支付工资和利润的基金的商品价值的\textbf{已知量},在工业家面前,从经验上看,表现为这样的形式:不管工资有什么波动,商品的一定的市场价格在一个时期内保持不变。

总之,应当注意亚·斯密书中的奇怪的思路:起先他研究商品的价值,在一些地方正确地规定价值,而且正确到这样的程度,大体上说,他找到了剩余价值及其特殊形式的源泉——他从商品价值推出工资和利润。但是后来,他走上了相反的道路,又想倒过来从工资、利润和地租的自然价格的相加数来推出商品价值(他已经从商品价值推出了工资和利润)。正由于后面这种情况,斯密对于工资、利润等等的波动给予商品价格的影响,没有一个地方做出了正确的分析,因为他没有基础。[VI—265]

\centerbox{※     ※     ※}

[VIII—364]\fontbox{~\{}\textbf{亚·斯密。价值及其组成部分}。斯密违反他原来的正确观点而发挥的错误看法(见前面),也表现在下面这段话里:

\begin{quote}“地租成为……商品价格的组成部分,但与利润和工资完全不同。利润和工资的高低是\textbf{谷物价格高低的原因,而地租的多少是这一价格的结果}。”(《国富论》第 1 篇第 11 章)\endnote{马克思在《剩余价值理论》第二册《亚·斯密的地租理论》一章(手稿第 620—625 页)中,对斯密关于地租以不同于利润和工资的方式加入产品价格的论点作了批判的分析。斯密《国富论》的这段引文,马克思引自加尼耳的《论政治经济学的各种体系》一书(1821 年巴黎版第 2 卷第 3 页)。——第 78 页。}\fontbox{\}~}[VIII—364]\end{quote}

\tsectionnonum{[(8)斯密的错误——把社会产品的全部价值归结为收入。斯密关于总收入和纯收入的看法的矛盾]}

[VI—265]现在我们来谈谈同商品价格或商品价值(这里还假定它们两者是同一个东西)的分解有关的另一个问题。假定亚·斯密正确地作了计算,也就是说,他以商品价值为出发点,把商品价值分解成这个价值在不同的生产当事人之间进行分配的各个组成部分,而不想倒过来从这些组成部分的价格推出价值,——这一点这里撇开不谈。我们也不谈他的片面看法,即把工资和利润只当作分配的形式,因而在同等意义上把这两者描写成由它们的所有者消费的收入。撇开这一切不谈,应当指出,亚·斯密自己[对于把产品的全部价值归结为收入]曾提出某种疑问;这里他胜过李嘉图的地方,仍然不是他正确地解决了他所提出的疑问,而是他一般地提出了这种疑问。

[266]亚·斯密是这样说的:

\begin{quote}“这三部分〈工资、利润、土地所有者的地租〉看来直接地或最终地构成谷物的\textbf{全部}价格〈指一般商品的全部价格,亚·斯密在这里说谷物,是因为在他看来,有许多商品的价格并不包括地租这一组成部分〉。也许有人以为必须有\textbf{第四个部分},用来补偿租地农场主的资本,或者说,补偿他的役畜和其他农具的损耗。但是必须考虑到,任何一种农具的价格,例如一匹役马的价格,本身又是由上述三个部分构成:养马用的土地的地租,养马的\textbf{劳动},预付这块土地的地租和这种劳动的工资的租地农场主的利润。\fontbox{~\{}这里利润表现为原始形式,也把地租包括在内。\fontbox{\}~}因此,谷物的价格虽然要补偿马的价格和给养费用,但\textbf{全部}价格仍然直接地或最终地分解为这三个部分:地租、劳动和利润。”(第 1 篇第 6 章)(在这里,斯密突然不说“工资”,而说“劳动”,可是他又说“地租”和“利润”,而不说“土地所有权”和“资本”,这是完全荒谬的。)\end{quote}

但同样明显的是,这里要注意到下面的情况:正象租地农场主把马和犁的价格包括在小麦的价格中一样,那些把马和犁卖给租地农场主的养马人和制犁人,也会把生产工具的价格(例如养马人可能把另一匹马的价格)和原料(饲料和铁)的价格包括在马和犁的价格中,而养马人和制犁人用来\textbf{支付}工资和利润(和地租)的基金,仅仅由他们在自己的生产领域中加到现有不变资本价值额上的新劳动组成。因此,如果亚·斯密在谈到租地农场主的时候,承认在他的谷物价格中,除了他支付给自己和别人的工资、利润和地租以外,还包括一个\textbf{不同于这些部分的第四个组成部分},即租地农场主使用的不变资本的价值,例如马、农具等等的价值,那末,对于养马人和农具制造人来说,这也是适用的,斯密把我们从本丢推给彼拉多\authornote{此语出自福音书路加福音第 23 章。本丢和彼拉多是罗马的一个犹太总督的名和姓。据福音书记载,耶稣被解送到本丢那里受审,本丢知道耶稣是加利利人,属希律所管,就把他送交给希律,希律拒绝审讯,又把他送回彼拉多。人们沿用此语时省去希律,而说“从本丢推给彼拉多”,意思是推来推去,不解决问题。——译者注}完全是徒劳无益的。而且选用租地农场主的例子,把我们推来推去,尤其不恰当,因为在这个例子中,不变资本项目中包括了完全不必向别人购买的东西,即种子,难道价值的这一组成部分会分解成谁的工资、利润和地租吗\fontbox{?}

但是,我们且往前走,先看看斯密是否始终贯彻了自己的观点:一切商品的价值都可以归结为某一收入源泉或全部收入源泉——工资、利润、地租,也就是说,一切商品都可以作为供消费用的产品来消费掉,或者说,无论如何都可以这样或那样地用于个人需要(而不是用于生产消费)。不过[267]还要先说明一点。例如在采集浆果等等的时候,浆果等等的价值可以只归结为工资,虽然在大多数情况下也要有篮筐等等用具作为劳动资料。可是这里谈的是资本主义生产,这一类的例子是根本不相干的。

最初他又重复第一篇第六章说过的观点。

在\textbf{第二篇第二章}(\textbf{加尔涅}的译本,第 2 卷第 212—213 页)中说:

\begin{quote}“我们说过……\textbf{大部分商品的价格}都分解为三部分,其中一部分支付工资,第二部分支付资本利润,第三部分支付地租。”\end{quote}

按照这种说法,一切商品的全部价值都可分解为各种收入,并且作为消费基金而归于依靠这种收入过活的这个或那个阶级。但是,既然一国的总产量,例如年产量,只由已生产出来的商品的价值总额构成,而这些商品中的单个商品的价值又分解为各种收入,那末,商品的总额——劳动的年产品,即总收入,也就能够在一年内以这种形式消费掉。可是斯密马上起来反驳自己:

\begin{quote}“既然就每一个特殊商品分别来说是如此,那末,就形成每一个国家的土地和劳动的全部年产品的一切商品\textbf{整体}来说也必然是如此。这个年产品的\textbf{全部价格\CJKunderdot{或}交换价值},必须分解为同样三个部分,在国内不同居民之间进行分配,或是作为他们的劳动的工资,或是作为他们的资本的利润,或是作为他们占有的土地的地租。”(同上,第 213 页)\end{quote}

这确实是必然的结论。适用于单个商品的必定适用于商品总额。但是亚当说并非如此。他接着说:

\begin{quote}“虽然一国土地和劳动的年产品的总价值这样在国内不同居民之间分配,构成他们的收入,但是,就象我们把私人地产的收入区分为\textbf{总收入}和\textbf{纯收入}一样,我们也可以对一个大国\textbf{全体居民}的收入作这样的区分。”(同上,第 213 页)\end{quote}

(等一等!他在前面对我们说的恰好相反:在单个租地农场主的产品中,例如在小麦中,我们还可以在这一产品价值分解成的各部分中,分出第四个部分,就是只补偿已使用的不变资本的部分;这对于单个租地农场主\textbf{直接地说}是正确的,但如果我们再往前走,就会看到,作为租地农场主的不变资本的那一部分,在更早的阶段——在别人手里的时候,在成为租地农场主的资本以前,就分解为工资、利润等等,一句话,分解为各种收入。因此,如果说商品从它们在单个生产者手中来考察,包含一部分不构成收入的价值,是正确的,那末从“一个大国全体居民”来说,在他看来就是不正确的了,因为在一个人手中成为不变资本的东西之所以具有价值,是由于它作为工资、利润和地租的总价格来自别人手中。现在他说的恰好相反。)

亚·斯密接着说:

\begin{quote}[268]“私人地产的\textbf{总}收入包括租地农场主所支付的一切;\textbf{纯收入}则是扣除管理、修理的开支以及其他一切\textbf{必要费用}之后,留归\textbf{土地所有者}的东西,换句话说,是他不损及自己的财产而可以归入用于直接消费即吃喝等等的基金的东西。土地所有者的实际财富不同他的\textbf{总}收入成比例,而同他的\textbf{纯}收入成比例。”\end{quote}

(第一,斯密在这里谈的是不相干的东西。租地农场主作为地租支付给土地所有者的,和他作为工资支付给工人的毫无差别,都同他自己的利润一样,是商品价值或价格中分解为各种收入的那一部分。问题在于,商品是否还包括价值的另一个组成部分。他在这里是承认这一点的,正象他在谈到租地农场主时曾经不得不承认这一点一样,不过这种承认并没有妨碍他宣称,租地农场主生产出来的谷物(即他的谷物的价格\textbf{或}交换价值)只分解为各种收入。第二,我们要顺便指出下面一点。单个租地农场主作为\textbf{租地农场主}能够支配的实际财富,取决于他的利润。但另一方面,他作为商品所有者,可以把他的农场卖掉,或者说,如果土地不属于他,可以把土地上的全部不变资本如役畜、农具等卖掉。他由此所能实现的价值,从而,他所能支配的财富,就取决于他的不变资本的价值,也就是取决于这个不变资本的大小。但是他只能把这些东西再卖给另一个租地农场主,而在后者手中,这些东西并不是可以自由支配的财富,而是不变资本。因此,我们仍然没有前进一步。)

\begin{quote}“一个大国全体居民的\textbf{总}收入,包括他们的土地和劳动的\textbf{全部}年产品\end{quote}

(前面我们听到,这全部产品——它的价值——都分解为工资、利润和地租,也就是说,仅仅分解为各种形式的纯收入);

\begin{quote}\textbf{纯}收入是在先扣除\textbf{固定资本}的维持费用,再扣除\textbf{流动资本}的维持费用之后,余下供他们使用的部分\end{quote}

(可见,斯密现在把劳动工具和原料扣除了),

\begin{quote}或者说,是他们可以列入\textbf{直接消费基金}……而不侵占资本的部分。”\end{quote}

(因此,我们现在知道,商品总量的价格或交换价值,无论就单个资本家来说,还是就全国来说,都还包含第四个部分,这部分对任何人都不构成收入,既不能归结为工资、利润,也不能归结为地租。)

\begin{quote}“维持\textbf{固定资本}的全部费用,显然要从社会纯收入中排除掉。无论是为维持有用机器、生产工具、经营用的建筑物等等\textbf{所必需的材料},还是为使这些材料转化为适当的形式\textbf{所必需的劳动的产品},从来都不可能成为社会\textbf{纯}收入的一部分。\textbf{这种劳动的价格},当然可以是社会纯收入的一部分,因为从事这种劳动的工人,可以把[269]他们\textbf{工资的全部价值}用在他们的\textbf{直接消费基金}上。但是,在其他各种劳动中,\textbf{劳动的价格和劳动的产品二者都加入这个消费基金};劳动的价格加入工人的消费基金,劳动的产品则加入另一些人的消费基金,这些人靠这种工人的劳动来增加自己的生活必需品、舒适品和享乐品。”(同上,第 214—215 页)\authornote{马克思在这里用铅笔加了一句话:“这毕竟是比其他经济学家的看法更接近正确的观点”。——编者注}\end{quote}

这里,亚·斯密又避开了他应该回答的问题——关于商品全部价格的第四个部分,即不归结为工资、利润、地租的那一部分的问题。首先我们指出一个大错误。要知道,在机器厂主那里,也象在其他所有工业资本家那里一样,把机器等等的原料变成适当形式的劳动分解为必要劳动和剩余劳动,因而不仅分解为工人的工资,而且分解为资本家的利润。但原料的价值和工人把这些原料变成适当形式时使用的工具的价值,既不归结为工资,也不归结为利润。那些从性质来说不用于个人消费而用于生产消费的产品并不加入直接消费基金,这一点,是与问题毫无关系的。例如种子(播种用的那部分小麦),从性质来说也可以加入消费基金,但是从经济上说必须加入生产基金。其次,说用于个人消费的产品的全部价格同产品一起都加入\textbf{消费基金},是完全错误的。例如麻布,如果不是用来作帆或用于别的生产目的,它就作为产品全部加入消费;但是对于麻布的价格却不能这样说,因为这个价格的一部分补偿麻纱,另一部分补偿织机等等,麻布的价格只有一部分归结为这种或那种收入。

亚当刚对我们说过,机器、经营用的建筑物等等所必需的材料,也同由这些材料制造的机器等等一样,“从来都不可能成为\textbf{纯}收入的一部分”;这就是说,它们加入总收入。但是就在这些话后面不远,就在第二篇第二章第 220 页上,他却说出相反的话:

\begin{quote}“机器和工具等等构成个人或社会的\textbf{固定资本},它们既不构成\textbf{个人或社会的总收入}的一部分,也不构成\textbf{个人或社会的纯收入}的一部分,\textbf{货币}也是一样”等等。\end{quote}

亚当的混乱、矛盾、离题,证明他既然把工资、利润、地租当作产品的交换价值或全部价格的组成部分,在这里就必然寸步难行、陷入困境。

\tsectionnonum{[(9)萨伊是斯密理论的庸俗化者。萨伊把社会总产品和社会收入等同起来。施托尔希和拉姆赛试图把这两者区别开来]}

萨伊把亚·斯密的不一贯的说法和错误的意见化为十分一般的词句,来掩饰他自己的陈腐的浅薄见解。我们在他的著作中读到:

\begin{quote}“从整个国家来看,根本没有纯产品。因为\textbf{产品}的价值等于产品的生产\textbf{费用},所以,如果我们把这些\textbf{费用}扣除,也就把全部\textbf{产品价值}扣除……\textbf{年收入}就是\textbf{总收入}。”(《论政治经济学》1817 年巴黎第 3 版第 2 卷第 469 页)\end{quote}

年产品总额的价值等于物化在这些产品中的[270]劳动时间量。如果从年产品中把这个总价值扣除,那末实际上——就价值来说——就没有任何价值留下来了,因而无论纯收入还是总收入统统都没有了。但是,萨伊认为,每年生产的价值,当年会消费掉。所以,对整个国家来说,根本不存在纯产品,只存在总产品。第一,说每年生产的价值,当年会消费掉,这是错误的。固定资本的大部分就不是这种情况。一年内生产的价值大部分进入劳动过程,而不进入价值形成过程;这就是说,并不是这些东西的总价值全部在一年内消费掉。第二,每年消费的价值中有一部分是由不加入消费基金而作为生产资料来消费的那种价值构成的,这些生产资料从生产过程出来,又以自身的实物形式或以等价物的形式,重新回到生产过程中去。另一部分则由扣除上述第一部分之后能够加入个人消费的那种价值构成;这部分价值就构成“纯产品”。

关于萨伊的这种胡言乱语,施托尔希说:

\begin{quote}“很明显,年产品的价值分成资本和利润两部分。\textbf{年产品价值的这两部分}中,每一部分\textbf{都要有规则地用来购买国民所需要的产品},以便维持该国的资本和更新它的消费基金。”(\textbf{施托尔希}《政治经济学教程》第 5 卷:《论国民收入的性质》1824 年巴黎版第 134—135 页)“我们要问,靠自己的劳动来满足自己的全部需要的家庭(我们在俄国可以看到许多这样的例子)……其\textbf{收入}是否等于这个家庭的土地、资本和劳动的总产品\fontbox{?}难道一家人能够住自己的粮仓和畜棚,吃自己的谷种和饲料,穿自己役畜的毛皮,用自己的农具当娱乐品吗\fontbox{?}按照萨伊先生的论点,对所有这些问题必须作肯定的回答。”(同上,第 135—136 页)“萨伊把总产品看成社会的收入,并由此得出结论说,社会可以把等于这个产品的价值消费掉。”(同上,第 145 页)“一国的纯收入,不是由已生产出来的价值\textbf{超过消费了的价值总额的}余额构成,就象萨伊所描写的那样,而只是由已生产出来的价值\textbf{超过为生产目的而消费了的价值的}余额构成。因此,如果一个国家在一年内消费这全部余额,那末它就是消费自己的全部纯收入。”(同上,第 146 页)“如果承认一个国家的收入等于该国的总产品,就是说不必扣除任何\textbf{资本},那末也必须承认,这个国家可以把年产品的全部价值非生产地消费掉,而丝毫无损于该国的未来收入。”(同上,第 147 页)“\textbf{构成一个国家的〈不变〉资本的产品,是不能消费的}。”(同上,第 150 页)\end{quote}

\textbf{拉姆赛}(\textbf{乔治})在《论财富的分配》(1836 年爱丁堡版)中,对于亚·斯密称为“全部价格的第四个组成部分”的东西,也就是我为了同花在工资上的资本相区别而称为不变资本的东西,提出如下意见:

\begin{quote}[271]他说:“李嘉图忘记了,全部产品不仅分为工资和利润,而且还必须有一部分补偿固定资本。”(第 174 页注)\end{quote}

拉姆赛理解的“固定资本”,不仅包括生产工具等等,而且包括原料,总之,就是我在各生产领域内称为不变资本的东西。当李嘉图谈到产品分为利润和工资的时候,他总是假定,预付在生产上并在生产中消费了的资本已经扣除。然而拉姆赛基本上是对的。李嘉图对资本的不变部分没有作任何进一步的分析,忽视了它,犯了重大的错误,特别是把利润和剩余价值混淆起来,其次在研究利润率的波动等等问题上也犯了错误。

现在我们听听拉姆赛本人是怎样说的:

\begin{quote}“怎样才能把产品和花费在产品上的资本加以比较呢\fontbox{?}……如果指整个国家而言……那末很清楚,花费了的资本的各个不同要素应当在这个或那个经济部门再生产出来,否则国家的生产就不能继续以原有的规模进行。工业的原料,工业和农业中使用的工具,工业中无数复杂的机器,生产和贮存产品所必需的建筑物,这一切都应当是一个国家总产品的组成部分,同时也应当是一个国家资本主义企业主的全部预付的组成部分。因此,总产品的量可以同全部预付的量相比较,因为每一项物品都可以看成是与同类的其他物品并列的。”(同上,第 137—139 页)“至于单个资本家,由于他不是以实物来补偿自己的支出,他的支出的大部分必须通过交换来取得,而交换就需要一定份额的产品,由于这种情况,单个资本主义企业主不得不把更大的注意力放在自己产品的交换价值上,而不是放在产品的量上。”(同上,第 145—146 页)“他的\textbf{产品的价值}愈高于预付\textbf{资本的价值},他的利润就愈大。因此,资本家计算利润时,是拿价值同价值相比,而不是拿量同量相比……利润的上升或下降,同总产品或它的\textbf{价值}中用来\textbf{补偿必要预付}的那个份额的下降或上升成比例……因此,利润率决定于以下两个因素:第一,全部产品中归工人所得的那个份额;第二,为了以实物形式或通过交换来补偿固定资本而必须储存的那个份额。”(同上,第 146—148 页)\end{quote}

\fontbox{~\{}拉姆赛在这里谈的关于利润率的意见,要放到第三章(关于利润)去考察。\endnote{马克思这里说的“第三章”是指关于“资本一般”的研究的第三部分。这一章的标题应为:《资本的生产过程和流通过程的统一,或资本和利润》。以后(例如,见第 IX 本第 398 页和第 XI 本第 526 页)马克思不用“第三章”而用“第三篇”(《dritterAbschnitt》)。后来他就把这第三章称作“第三册”(例如,在 1865 年 7 月 31 日给恩格斯的信中)。关于“资本一般”的研究的“第三章”马克思是在第 XVI 本开始的。从这“第三章”或“第三篇”的计划草稿(见本册第 447 页)中可以看出,马克思打算在那里写两篇专门关于利润理论的历史补充部分。但是马克思在写作《剩余价值理论》的过程中,就已在自己的这一历史批判研究的范围内,详细地批判分析了各种资产阶级经济学家对利润的看法。因此,马克思在《剩余价值理论》中,特别是在这一著作的第二册和第三册中,就已进一步更充分地揭示了由于把剩余价值和利润混淆起来而产生的理论谬误。——第 7、87、272 页。}重要的是,他正确地强调指出了这个要素。从一方面来看,李嘉图说,构成不变资本(拉姆赛说的“固定资本”,就是指这个)的那些商品的落价,总会使现有资本的一部分价值下降,这是对的。这种说法特别适用于真正的固定资本——机器等等。同全部资本相比剩余价值的增加,如果是由资本家的不变资本总价值下降引起的(资本家在总价值下降之前就占有了这些不变资本),这对单个资本家来说毫无利益可言。不过这种说法仅仅在极小的程度上适用于由原料或成品(不加入固定资本的成品)构成的那部分资本。原料或成品的现有量可能发生价值下降,但这个现有量同总产量相比,始终只是一个微不足道的量。在单个资本家那里,这种说法仅仅在很小的程度上适用于他投入流动资本的那部分资本。从另一方面来看,很清楚,因为利润等于剩余价值和总预付资本之比,因为可以被吸收的劳动量不取决于价值,而取决于原料的量和生产资料的效率,不取决于它们的交换价值,而取决于它们的使用价值,所以,其产品[272]构成不变资本要素的那些部门中的劳动的生产能力愈高,生产一定量剩余价值所必需的不变资本的支出愈少,这个剩余价值和全部预付资本之比就愈大,从而,在剩余价值量已知的情况下,利润率就愈高。\fontbox{\}~}

(被拉姆赛当作两个独立现象来考察的东西——在再生产过程中,就全国而言,是以产品补偿产品,就单个资本家而言,是以价值补偿价值——可归结为两个观点,这两个观点即使对于单个资本来说,在分析\textbf{资本的流通过程,同时也就是再生产过程}时,都是应当加以考虑的。)

拉姆赛没有解决亚·斯密所研究的并使他陷入重重矛盾的实际困难。为了直截了当地讲清楚这个困难,我们把它表述如下:整个资本(作为价值)都归结为劳动,它无非是一定量的物化劳动。但是,有酬劳动等于工人的工资,无酬劳动等于资本家的利润。因此,整个资本都可以直接地或间接地归结为工资和利润。也许,什么地方在完成这样一种劳动,它既不归结为工资,也不归结为利润,它的目的只是为了补偿在生产过程中消费了的、同时又是作为再生产条件的那种价值\fontbox{?}但是谁来完成这种劳动呢\fontbox{?}要知道,工人的一切劳动都分为两部分,一部分用来恢复他自身的生产能力,另一部分构成资本利润。

\tsectionnonum{[(10)]研究年利润和年工资怎样才能购买一年内生产的、除利润和工资外还包含不变资本的商品}

\tsubsectionnonum{[(a)靠消费品生产者之间的交换不可能补偿消费品生产者的不变资本]}

为了把各种虚假的搀杂的东西从问题中清除出去,首先还要指出下面一点。当资本家把自己的利润即自己的收入的一部分转化为资本,转化为劳动资料和劳动材料的时候,他是用工人为他无偿地完成的那部分劳动来支付这两者的。这里有一个新的劳动量,它构成一个新商品量的等价物,而这些商品按其使用价值来说就是劳动资料和劳动材料。所以,这种情况属于资本积累的问题,它本身不包含任何困难;这里我们碰到的,是不变资本超过它原有界限的增长,或者说,超出已经存在的和待补偿的不变资本量之上的新不变资本的形成。困难在于\textbf{已经存在的}不变资本的再生产,而不在于超出有待再生产的不变资本量之上的新不变资本的形成。新的不变资本显然来源于利润;它以收入的形式存在极短时间,随后即转化为资本。这部分利润归结为\textbf{剩余劳动时间,即使没有资本存在,社会也必须不断地完成这个剩余劳动时间,以便能支配一个所谓发展基金——仅仅人口的增长,就已使这个发展基金成为必要的了}。\endnote{马克思在《资本论》第三卷第四十九章对这里所提出的问题作了如下的表述:“怎样才能使工人用他的工资、资本家用他的利润、土地所有者用他的地租买到各自不是只包含这三个组成部分之一,而是包含所有这三个组成部分的商品\fontbox{?}怎样才能使工资、利润、地租,即收入的三个源泉加在一起的价值总额买到构成这些收入所得者的全部消费的商品(这些商品除了价值的这三个组成部分之外,还包含价值的另一个组成部分,也就是不变资本)\fontbox{?}他们怎样才能用三部分的价值买到四部分的价值\fontbox{?}”接着马克思写道:“我们已经在第二卷第三篇作了分析。”这是指《社会总资本的再生产和流通》一篇(见马克思《资本论》第 2 卷第 3 篇)。——第 89 页。}

\fontbox{~\{}在拉姆赛的著作第 166 页上,我们看到他对不变资本作了很好的说明,不过他仅仅是从使用价值方面来谈的。他说:

\begin{quote}“无论总产品〈例如租地农场主的总产品〉的数额是多少,其中用来补偿在生产过程中以不同形式消费了的全部东西的那个量,不应当有任何变动。只要生产以原有的规模进行,这个量就必须认为是\textbf{不变的}。”\fontbox{\}~}\end{quote}

所以,首先必须从以下事实出发:和已经存在的不变资本的再生产不同,不变资本的新形成是以利润为源泉;这里假定:一方面,工资只够用来再生产劳动能力,另一方面,全部剩余价值统统归入“利润”范畴,因为\textbf{直接占有}全部剩余价值的不是别人,正是产业资本家,不管他以后在什么地方还得把其中的一部分分给谁。

\begin{quote}\fontbox{~\{}“资本主义企业主是财富的总分配者:他付给工人工资,付给资本家(货币资本家)利息,付给土地所有者地租。”(\textbf{拉姆赛}的著作第 218—219 页)\end{quote}

我们把全部剩余价值称为利润,是把资本家看成这样一种人:(1)他直接占有生产出来的全部剩余价值,(2)他拿这种剩余价值在他自己、货币资本家和土地所有者之间进行分配。\fontbox{\}~}

[VII—273]然而,这个新的不变资本由利润产生,无非是说,新的不变资本来源于工人的一部分剩余劳动。这好比野蛮人除了打猎的时间以外,还必须花费一定的时间来制造弓箭,又好比农民在宗法式农业条件下,除了耕地的时间以外,还要花费一定量的劳动时间来制造他的大部分工具。

但是,这里问题在于:究竟由谁劳动,以补偿生产中已经耗费的不变资本的等价\fontbox{?}工人为自己完成的那部分劳动,补偿他的工资,或者从整个生产来看,就是创造他的工资。相反,他的剩余劳动,即构成利润的劳动,一部分成为资本家的消费基金,一部分转化为追加资本。但是,资本家并不是用这个剩余劳动或利润,来补偿他自己生产中已耗费的资本。\fontbox{~\{}如果情况是这样的话,那末剩余价值就不会成为形成新资本的基金,而变成保存旧资本的基金了。\fontbox{\}~}然而,形成工资的必要劳动和形成利润的剩余劳动,已经构成了整个工作日,再没有任何其他的劳动存在的余地了。(就算资本家担任的“监督劳动”也归入工资之内。从这方面看,资本家是雇佣工人,不过不是别的资本家的雇佣工人,而是他自己的资本的雇佣工人。)那末,补偿不变资本的那个源泉,那个劳动,究竟从何而来呢\fontbox{?}

花在工资上的那部分资本,由新的生产来补偿(这里不谈剩余劳动)。工人消费工资,但他耗费掉多少旧劳动,他就加进多少新劳动。如果我们考察整个工人阶级,而不受分工的干扰,那末就会看到,工人不仅会再生产出同一个价值,而且会再生产出同样的使用价值。这样,根据工人的劳动生产率的不同,同一个价值,同一个劳动量,会以较大量或较小量的同样的使用价值形式再生产出来。

拿社会来说,无论什么时候我们都会看到,有一定的不变资本作为生产条件,以极不相同的比例,同时存在于一切生产领域,它永远属于生产,并且必须归还给生产,就象种子要归还给土地一样。这个不变部分的\textbf{价值}固然可能降低或提高,这取决于构成这个不变部分的那些商品的再生产是更便宜,还是更贵。但是,这种\textbf{价值变动}决不会影响下面这一点:作为生产条件进入生产过程的资本不变部分,在生产过程中是一个事先已知的价值,它必须再现在产品的价值中。因此,不变资本本身的价值变动,在这里可以不加考虑。在一切场合,不变资本在这里都表现为一定量的\textbf{过去的、物化的}劳动,这一定量的劳动要作为决定产品价值的因素之一转移到产品价值中去。因此,为了更明确地说明问题,我们假定资本不变部分的生产费用\endnote{“生产费用”(《Produktionskosten》)这一术语这里是用在“内在的生产费用”的意义上,即指 C+v+m。参看注 41。——第 91 页。}或价值也保持不变,始终一样。又如,不变资本的价值在一年内不是全部都转移到产品中,而是(就固定资本的价值而言)在许多年内才转移到这个时期所生产的产品总量中,这种情况也不会使问题发生任何变化。因为这里谈的仅仅是一年内实际消费的,因而必须在当年得到补偿的那部分不变资本。

十分明显,不变资本的再生产问题,要在关于资本再生产过程或流通过程的那一篇里谈,但这并不妨碍在这里就把基本问题弄清楚。

[274]先谈工人的工资。工人为资本家一天劳动 12 小时,得到一定量的货币,假定这笔货币体现 10 劳动小时。这笔工资转化为生活资料。这些生活资料全都是商品。假定这些商品的价格同它们的价值相等。但是,在这些商品的价值中,有一个组成部分,是抵补商品中包含的原料和损耗了的生产资料的价值的。但是,这些商品的价值的所有组成部分合在一起,也象工人支出的工资一样,只包含 10 劳动小时。假定这些商品的价值的 2/3 由它们包含的不变资本的价值构成,1/3 由完成生产过程并把产品变为消费品的劳动构成。这样,工人是用自己的 10 小时活劳动补偿 2/3 的不变资本和 1/3(当年加到对象上去的)活劳动。如果在生活资料中,即在工人购买的商品中,根本不包含不变资本;如果这些商品的原料不花费什么,并且生产这些商品时不需要任何劳动工具,那就可能出现以下两种情况之一。或者,商品象原先一样,还是包含 10 小时劳动。在这种情况下,工人就是用 10 小时活劳动补偿 10 小时活劳动。或者,由工人的工资转化成的、工人再生产其劳动能力所必需的那个使用价值量,只值 3+(1/3)小时劳动(假定没有劳动工具和那种本身就是劳动产品的原料)。在这种情况下,工人的必要劳动就只是 3+(1/3)小时,他的工资事实上就会降低到 3+(1/3)小时物化劳动时间。

假定商品是麻布,12 码麻布(在这里,我们当然根本不必注意实际价格如何)=36 先令,或 1 镑 16 先令。其中 1/3 为新加劳动,2/3 用于原料(纱)和机器的损耗。假定必要劳动时间等于 10 小时;因而剩余劳动就等于 2 小时。1 劳动小时用货币来表现,等于 1 先令。在这种情况下,12 劳动小时=12 先令,工资=10 先令,利润=2 先令。假定工人和资本家购买麻布作为消费品时,花掉全部工资和全部利润(共 12 先令),换句话说,花掉加到原料和机器上的全部价值,即在纱变为麻布的过程中物化的全部新的劳动时间量。(可能以后购买自己生产的产品要花费一个工作日以上的时间。)1 码麻布值 3 先令。把工资和利润加在一起,工人和资本家用 12 先令只能买到 4 码麻布。这 4 码麻布包含 12 劳动小时,然而其中只有 4 小时代表新加劳动,8 小时则代表物化在不变资本中的劳动。工资和利润加在一起,用 12 劳动小时只能买到自己总产品的 1/3,因为这个总产品的 2/3 由不变资本构成。12 劳动小时分为 4+8,其中 4 小时自己补偿自己,8 小时补偿那个同织布过程中加进的劳动无关的、以物化了的形式即纱和机器的形式加入织布过程的劳动。

因此,谈到用工资和利润交换或购买来作为消费品(或者是为了再生产本身的某种目的,因为购买商品的目的丝毫不会改变这里的问题)的那部分产品或商品,那末很清楚,这种产品的价值中相当于不变资本的部分,是由分解为工资和利润的新加劳动基金支付的。有多少不变资本以及有多少在最后生产过程中加进的劳动是用工资和利润加在一起来购买的;在生产的最后阶段加进的劳动按什么比例来支付,物化在不变资本中的劳动按什么比例来支付;——这一切都取决于它们作为价值组成部分加入成品的最初比例。为了简单起见,我们假定这个比例是:物化在不变资本中的劳动为 2/3,新加劳动为 1/3。

[275]因此,有两点是清楚的:

\textbf{第一},我们为麻布所假设的比例,也就是为工人和资本家把工资和利润实现在自己生产的商品上,即他们买回自己产品的一部分这种情况所假设的比例,——这个比例,即使工人和资本家把同一个价值额花在其他产品上,也保持不变。根据上面的假设,每个商品都包含 2/3 不变资本和 1/3 新加劳动,工资和利润加在一起,始终只能购买产品的 1/3。12 小时劳动=4 码麻布。如果这 4 码麻布转化为货币,它就以 12 先令的形式存在。如果这 12 先令又转化为麻布以外的其他商品,这笔货币购买的也是价值为 12 劳动小时的商品,其中 4 小时是新加劳动,8 小时是物化在不变资本中的劳动。因此,假设在其他商品中也象在麻布中一样,在生产的最后阶段加进的劳动和物化在不变资本中的劳动之间保持同样的最初比例,这个比例就是普遍的。

\textbf{第二},如果一天的新加劳动等于 12 小时,那末在这 12 小时中,只有 4 小时自己补偿自己,即补偿活的、新加的劳动,而 8 小时支付物化在不变资本中的劳动。但是,不由活劳动本身来补偿的这 8 小时活劳动究竟由谁来支付呢\fontbox{?}正是由包含在不变资本中并同 8 小时活劳动相交换的那 8 小时物化劳动来支付。

因此,毫无疑问,用工资和利润总额(两者加在一起,只不过代表新加到不变资本上的劳动总量)购买的那部分成品,都以它的各个要素的形式得到补偿。这部分成品所包含的新加劳动得到补偿,不变资本所包含的劳动量也得到补偿。其次,毫无疑问,不变资本所包含的劳动在这里从新加到不变资本上的活劳动基金中得到了自己的等价。

但是,在这里就发生困难了。12 小时织布劳动的总产品(这个总产品与织布劳动本身所生产的大不相同),等于 12 码麻布,价值为 36 劳动小时或 36 先令。但是,工资和利润合起来,或者说 12 小时总劳动时间,只能从这 36 劳动小时中买回 12 小时;换句话说,只能从总产品中买回 4 码,多 1 码也不行。那末,其余 8 码又将怎样呢\fontbox{?}(\textbf{福尔卡德、蒲鲁东}。)\endnote{“(福尔卡德、蒲鲁东)”这两个名字是马克思在手稿上用铅笔加上的。马克思这里指的是他在第 XVI 本札记本中从法国资产阶级政论家、庸俗经济学家福尔卡德发表在 1848 年《两大陆评论》杂志(第 24 卷第 998—999 页)上的文章《社会主义的战争》(第二篇)摘录的一段话。福尔卡德在这段话中批判了蒲鲁东的“工人不能买回自己的产品,因为其中包含加入产品成本的利息”这一论点(见蒲鲁东的《什么是财产》1840 年巴黎版第 4 章第 5 节)。福尔卡德概括了蒲鲁东以极其狭隘的形式提出的困难,并指出商品价格包含着一个不仅超过工资而且超过利润的余额,因为它还包含原料等等的价值。福尔卡德企图以这种概括的形式来解决问题,他以“国民资本的不断增长”为理由,似乎就能解释上述“买回”。马克思在《资本论》第三卷第四十九章注 53 指出了福尔卡德这样以资本增长为理由是荒谬的,并痛斥这种说法是“资产阶级无知的乐观主义”。《两大陆评论》是资产阶级文学、艺术和政论双周刊,从 1829 年起在巴黎出版。——第 95 页。}

首先我们要指出,这 8 码代表的无非是已耗费的不变资本。但这个不变资本的使用价值的形式已经改变了。它是以新产品的形式,不再以纱、织机等等的形式,而以麻布的形式存在了。这 8 码麻布,也象用工资和利润购买的其余 4 码麻布一样,按价值来说,1/3 由织布过程中加进的劳动构成,2/3 由过去的、物化在不变资本中的劳动构成。而在前面那 4 码麻布的情况下,新加劳动的 1/3 抵补了 4 码麻布中包含的织布劳动,也就是自己抵补了自己,织布劳动其余的 2/3 抵补了 4 码麻布中包含的不变资本;而现在正相反,8 码麻布包含的不变资本由不变资本的 2/3 来抵补,它包含的新加劳动由不变资本的 1/3 来抵补。

这 8 码麻布本身包含了、吸收了整个不变资本的价值,——这个价值在 12 小时的织布劳动期间,转移到产品中,加入到产品的生产过程中,而现在以供直接的个人消费(不是生产消费)的产品形式存在,——这 8 码麻布本身又将怎样呢\fontbox{?}

这 8 码属于资本家。如果资本家想自己把这 8 码消费掉,就象他把代表他的利润的 2/3 码消费掉一样,[276]那他就不能把加入 12 小时织布过程的不变资本再生产出来了;他也就根本不能——就这里所谈的加入这 12 小时过程的资本来说——继续执行资本家的职能了。所以,他要卖掉 8 码麻布,把它们变成 24 先令货币或 24 劳动小时。但在这里我们又遇到了困难。他把这 8 码麻布卖给谁呢\fontbox{?}他把这些麻布转化为谁的货币呢\fontbox{?}我们很快将回过头来谈这个问题,现在让我们先看看下一段过程。

资本家一旦把 8 码麻布,即他的产品中和他预付的不变资本相等的那部分价值,转化为货币,把它们卖掉,使它们转为交换价值形式,他就用这些货币重新购买和原先构成他的不变资本的那些商品同类的(按使用价值来说是同类的)商品,他购买纱、织机等等。他按照生产新麻布所必需的比例,把 24 先令分别用来购买原料和生产工具。

由此可见,按使用价值来说,他的不变资本原先由哪种劳动的产品构成,现在就用那种劳动的新产品来补偿。资本家再生产了不变资本。但这些新的纱、新的织机等等,同样(按照假定)也是 2/3 由不变资本构成,1/3 由新加劳动构成。因而,如果说前 4 码麻布(新加劳动和不变资本)完全由新加劳动来支付,那末这 8 码麻布就由生产本身所必需的新生产出来的各个要素来补偿,而这些新生产出来的要素也是一部分由新加劳动,一部分由不变资本构成的。这样,看来至少有一部分不变资本要同另一种形式的不变资本相交换。产品的补偿是实在的事情,因为在纱被加工为麻布的同时,亚麻正被加工为纱,亚麻的种子正种成亚麻;同样,在织机受到磨损的同时,新的织机正在制造出来,在制造新织机的时候,新的木材和铁正在开采出来。某些要素在某一生产领域内被生产出来的时候,在另一生产领域内正在对它们进行加工。在所有这些\textbf{同时进行的}生产过程中,虽然每个过程都代表制造产品的一个更高的阶段,但是不变资本却以各种不同的比例同时被消费。

\textbf{总之,麻布这一成品的价值分为两部分},一部分用来重新购买这个时期生产出来的不变资本各个要素,另一部分则用在消费品上。为了简单起见,这里我们完全撇开一部分利润再转化为资本的问题,也就是说,正象在这整个研究中一样,我们假定,工资加利润,即加到不变资本上的全部劳动量,都作为收入被消费掉。

尚待回答的问题只是:有一部分总产品的价值被用来重新购买这个时期新生产的不变资本各个要素,究竟由谁购买这部分总产品\fontbox{?}谁购买 8 码麻布\fontbox{?}为了切断各种遁词的后路,我们假定,这是一种专供个人消费的麻布,而不是供生产消费(如制帆)的麻布。这里还必须把纯粹中间性的商业活动撇开,因为这些活动只起中介作用。例如,8 码麻布被卖给商人,并且不是经过一个商人的手,而是经过整整二十个商人的手,经过二十次买而再卖;那末,在第二十次,麻布终究还是要被商人卖给实际消费者,因此,实际消费者事实上或者支付给生产者,或者支付给\textbf{最后一个}即第二十个商人,而这个商人对消费者来说,是代表\textbf{第一个商人}即实际生产者。这些中间交易只会把最终交易推迟,或者也可以说,只会为最终交易起中介作用,但是不能说明最终交易。无论我们是问,谁从麻织厂主手里购买这 8 码麻布,还是问,[277]谁从第二十个商人手里(麻布经过一系列交换行为才落到他手里)购买这 8 码麻布——问题仍然是一样的。

这 8 码麻布和前 4 码麻布完全一样,必定要转入消费基金。这就是说,它们只能由工资和利润来支付,因为工资和利润是生产者的收入的唯一源泉,而在这里只有这些生产者才以消费者的身分出现。8 码麻布包含 24 劳动小时。我们假定(以 12 劳动小时为通行的正常工作日),其他两个部门的工人和资本家把自己的全部工资和利润花在麻布上,就象织布业中的工人和资本家把自己的整个工作日(工人把自己的 10 小时,资本家把他靠工人赚得的,就是靠 10 小时赚得的 2 小时剩余价值)都花在麻布上一样。在这种情况下,麻织厂主就会卖掉自己的 8 码麻布;这样,用于织造这 12 码的不变资本的\textbf{价值}就会得到补偿,这个价值可以重新花在构成不变资本的那些商品上,\textbf{因为}所有这些商品,如纱、织机等等,在市场上都有,它们在纱和织机被加工为麻布的时候已经生产出来了。纱和织机作为产品\textbf{生产出来}的过程,同它们作为产品加入(而不是作为产品从中出来)的生产过程\textbf{同时进行},这种情况说明,为什么麻布的\textbf{价值}中和被加工的材料、织机等等的价值相等的那部分,能够重新转化为纱、织机等等。如果麻布的各个要素的生产和麻布本身的生产不是同时进行,8 码麻布即使卖掉了,即使转化为货币,也不能从货币再转化为麻布的各个不变要素。\authornote{例如,目前由于美国内战,棉纺织厂主的棉纱和棉布就发生了这样的情况。即使他们的产品卖出去了,也不能保证实现前面说的那种再转化,因为市场上没有棉花。}

但是,另一方面,尽管市场上有新的纱、新的织机等等,因而新的纱、新的织机等等的生产是和已有的纱、已有的织机转化为麻布同时进行的;尽管纱和织机是和麻布同时生产出来,但是,在这 8 码麻布没有卖掉,没有转化为货币以前,它们也不能再转化为织布生产的不变资本的这些物质要素。因此,在我们还没有弄清楚,购买 8 码麻布,使它们重新具有货币形式即独立的交换价值形式所必需的基金从哪里来以前,麻布各个要素的不断的现实的生产始终和麻布本身的生产同时并进这个事实,还是不能向我们说明不变资本的再生产。

为了解决这个最后的困难,我们假定有 B 和 C(比方说,一个是制鞋业者,一个是屠宰业者)把自己的工资和利润总额,即他们所支配的 24 小时劳动时间,全都花在麻布上。这样,关于麻布织造业者 A,我们就摆脱了困难了。他的全部产品 12 码麻布(其中物化了 36 小时劳动),完全由工资和利润来补偿了,也就是说,由 A、B 和 C 这三个生产领域中新加到不变资本上的全部劳动时间来补偿了。麻布中包含的全部劳动时间,无论是原先就已体现在它的不变资本中的,还是在织布过程中新加的,现在都同这样一个劳动时间相交换了,这个劳动时间不是以不变资本的形式原先就存在于某一个生产领域中的,而是在上述 A、B、C 三个生产领域中\textbf{在生产的最后阶段}同时加到不变资本上去的。

因此,如果说麻布的原有价值只分解为工资和利润,还是错误的,——因为它实际上分解为两部分,一部分是同工资和利润总额相等的价值,即 12 小时织布劳动,一部分是同织布过程无关的,包含在纱、织机中的,总之,包含在不变资本中的 24 劳动小时,——那末,相反,说 12 码麻布的等价物,即 12 码麻布卖得的 36 先令,只分解为工资和利润,也就是说,不仅织布劳动,而且包含在纱和织机中的劳动,都完全由新加劳动来补偿,就是由 A 的 12 小时劳动、B 的 12 小时劳动和 C 的 12 小时劳动来补偿,却是正确的。

卖出的商品本身具有的价值,分解为[278]新加劳动(工资和利润)和过去劳动(不变资本的价值);这就是卖者的商品的价值(这也是商品的实际价值)。相反,购买这个商品的价值,即买者给予卖者的等价物,从我们的例子来看,只归结为新加劳动,归结为工资和利润。但是,如果由于任何商品在卖出以前都是待卖的商品,而它只有通过单纯的形式变化才转化为货币,就认为任何商品作为出卖的商品时的价值组成部分不同于它作为购买的商品(作为货币)时的价值组成部分,那是荒谬的。其次,认为社会例如在一年内完成的劳动不仅可以自己抵补自己,——这样,如果把全部商品量分为两个相等的部分,年劳动的一半就成为另一半的等价,——而且年产品所包含的总劳动中由当年劳动构成的 1/3 的劳动,可以抵补 3/3 的劳动,也就是说,它的大小等于自己的 3 倍,那就更加荒谬了。

在上述例子中,我们把困难推移了,从 A 移到了 B 和 C。但是困难并没有减少,反而增加了。\textbf{第一},我们在谈 A 的时候,有一个解决办法,就是 4 码所包含的劳动时间恰好等于加在纱上的劳动时间,也就是 A 领域的利润和工资总额,这 4 码是以麻布的形式,以自己劳动产品的形式消费的。B 和 C 的情况却不是这样,因为这两个领域是以 A 领域的产品麻布的形式,而不是以 B 和 C 的产品的形式,来消费它们加进的劳动时间总量,即工资和利润总额的。因此,它们不仅要卖掉代表不变资本包含的 24 劳动小时的那部分产品,而且要卖掉代表新加到不变资本上的 12 小时劳动时间的那部分产品。B 领域要卖掉 36 劳动小时,而不是象 A 领域那样只卖掉 24 劳动小时。C 的情况也是这样。\textbf{第二},为了把 A 领域的不变资本卖掉,推销出去,转化为货币,不仅需要 B 领域的全部新加劳动,而且需要 C 领域的全部新加劳动。\textbf{第三},B 和 C 不能把自己产品中的任何部分卖给 A 领域,因为 A 产品中归结为收入的部分,已经由 A 产品的生产者全部花在 A 领域本身了。B 和 C 也不可能用自己产品中的任何部分来补偿 A 的不变部分,因为根据假定,他们的产品不是 A 的生产要素,而是加入个人消费的商品。每前进一步,困难都在增加。

为了使 A 产品包含的 36 小时(就是说,不变资本包含的 2/3 即 24 小时,新加劳动包含的 1/3 即 12 小时)可以完全同加在不变资本上的劳动相交换,就必须使 A 的工资和利润,即 A 领域中加进的 12 小时劳动,自己消费掉本领域的产品的 1/3。总产品的其余 2/3,即 24 小时,代表不变资本包含的价值。这个价值已同 B 和 C 领域的工资和利润总额,也就是同 B 和 C 领域的新加劳动相交换。但是,为了使 B 和 C 能够用他们的产品包含的、归结为工资[和利润]的 24 小时来购买麻布,他们就要以他们自己的产品形式卖掉这 24 小时。此外,他们还要以他们自己生产的产品形式卖掉 48 小时来补偿不变资本。因此,他们就要卖掉共 72 小时的 B 和 C 的产品,来同其他生产领域 D、E 等等的利润和工资总额相交换;同时(假定正常工作日等于 12 小时),为了购买 B 和 C 的产品,就必须花 12×6=72 小时,即其他 6 个生产领域中的新加劳动;[279]因而,必须花费 D、E、F、G、H、I 这些领域的利润和工资,即这些领域中加到各自不变资本上的全部劳动量。

在这种情况下,B+C 总产品的价值,就会完全由 D、E、F、G、H、I 这 6 个生产领域的新加劳动,即工资和利润总额来支付。但这 6 个领域的全部总产品也要卖掉(因为它们的产品的任何部分都不是由它们的生产者自己消费,这些人已经把自己的全部收入投在 B 和 C 的产品上了),并且这个总产品的任何部分都不能在这些领域内部实现。因此,这里就牵涉到 6×36 劳动小时=216 劳动小时的产品,其中 144 劳动小时为不变资本,72(即 6×12)劳动小时为新加劳动。现在为了使 D 等等的产品也按同样的方式转化为工资和利润,即转化为新加劳动,就必须使$K^{1}$—$K^{18}$这 18 个生产领域的全部新加劳动,即这 18 个领域的工资和利润总额,统统都花在 D、E、F、G、H、I 这些领域的产品上。$K^{1}$—$K^{18}$这 18 个领域并不消费自己产品的任何部分,相反,它们已经把自己的全部收入花在 D—I 这 6 个领域中,所以它们又要卖掉 18×36 劳动小时=648 劳动小时,其中 18×12(216 小时)代表新加劳动,432 小时代表不变资本包含的劳动。因此,为了把$K^{1}$—$K^{18}$的这个总产品归结为其他领域的新加劳动,换句话说,归结为工资和利润总额,就需要有$L^{1}$—$L^{54}$领域的新加劳动,也就是 12×54=648 劳动小时。$L^{1}$—$L^{54}$领域的总产品等于 1944 小时(其中 648=12×54 为新加劳动,1296 劳动小时等于不变资本所包含的劳动),为了使这个总产品同新加劳动相交换,这些领域就要吸收$M^{1}$—$M^{162}$领域的新加劳动,因为 162×12=1944,而$M^{1}$—$M^{162}$领域又要吸收$N^{1}$—$N^{486}$领域的新加劳动,依此类推。

这是一个美妙的无止境的演进,如果我们认为,一切产品的价值都归结为工资和利润,即归结为新加劳动,同时,不仅新加到商品上的劳动,而且这个商品所包含的不变资本,都必须由其他某个生产领域的新加劳动来支付,那末,我们就会陷入这种境况。

为了把 A 产品包含的劳动时间,即 36 小时(1/3 为新加劳动,2/3 为不变资本),归结为新加劳动,也就是说,为了假定这个劳动时间由工资和利润来支付,我们首先就假定,产品的 1/3(这 1/3 的价值等于工资和利润总额)由 A 领域的生产者自己消费,或者同样可以说,由他们自己购买。以后的进程如下\endnote{后面,马克思在保留前面引用的数字材料的同时,改换了生产领域的字母符号(A 除外)。马克思用 B1—B2(或 B1-2)代替 B 和 C;用 C1—C6(或 C1-6)代替 D、E、F、G、H、I;用 D1—D18(或 D1-18)代替 K1—K18;用 E1—E54(或 E1-54)代替 L1—L54;用 F1—F162(或 F1-162)代替 M1—M162;用 G1—G486(或 G1-486)代替 N1—N486。——第 102 页。}:

(1)\textbf{生产领域}A。产品=36 劳动小时。24 劳动小时为不变资本。12 小时为新加劳动。产品的 1/3 由参加这 12 小时分配的双方——工资和利润,即工人和资本家来消费。A 产品的 2/3,等于不变资本包含的 24 劳动小时,则有待卖出。

(2)\textbf{生产领域}$B^{1}$—$B^{2}$。产品=72 劳动小时;其中 24 为新加劳动,48 为不变资本。这些领域用新加劳动购买 A 产品的 2/3 即补偿 A 的不变资本价值的那部分产品。但是,$B^{1}$—$B^{2}$领域共计要卖掉构成它们总产品价值的 72 劳动小时。

(3)\textbf{生产领域}$C^{1}$—$C^{6}$。产品=216 劳动小时,其中 72 小时为新加劳动(工资和利润)。它们用新加劳动购买$B^{1}$—$B^{2}$的全部产品。但是,它们要卖掉 216,其中 144 为不变资本。

[280](4)\textbf{生产领域}$D^{1}$—$D^{18}$。产品=648 劳动小时;216 为新加劳动,432 为不变资本。它们用新加劳动购买生产领域$C^{1}$—$C^{6}$的总产品=216。但是,它们要卖掉 648。(5)\textbf{生产领域}$E^{1}$—$E^{54}$。\textbf{产品}=1944 劳动小时;648 为新加劳动,1296 为不变资本。它们购买生产领域$D^{1}$-$D^{18}$的总产品,但是要卖掉 1944。

(6)\textbf{生产领域}$F^{1}$—$F^{162}$。\textbf{产品}=5832,其中 1944 为新加劳动,3888 为不变资本。它们用 1944 购买$E^{1}$—$E^{54}$的产品。而它们要卖掉 5832。

(7)\textbf{生产领域}$G^{1}$—$G^{486}$。

为了简单起见,假定每一个生产领域每次都只有一个工作日——12 小时——为资本家和工人所分享。增加这种工作日的数目并不能解决问题,反而会毫无必要地使问题复杂化。

这样,为了使这个序列的规律能够看得更清楚,可以写成:

(1)A。\textbf{产品}=36 小时;不变资本=24 小时。\textbf{工资和利润总额},或者说,\textbf{新加劳动}=12 小时。后者以 A 领域本身的产品形式由资本和劳动消费掉。

A 的待卖产品=它的不变资本=24 小时。

(2)$B^{1}$—$B^{2}$。这里我们需要 2 工作日,因而需要 2 个生产领域,以便支付 A 领域的 24 小时。

产品=2×36=72 小时,其中 24 小时为新加劳动,48 小时为不变资本。

$B^{1}$—$B^{2}$的待卖产品=72 劳动小时;这一产品的任何部分都不会在这些领域本身消费掉。

(6)$C^{1}$—$C^{6}$。这里我们需要 6 工作日,因为 72=12×6,$B^{1}$—$B^{2}$的全部产品必须由$C^{1}$—$C^{6}$加进的劳动消费掉。产品=6×36=216 劳动小时,其中 72 为新加劳动,144 为不变资本。

(18)$D^{1}$—$D^{18}$。这里我们需要 18 工作日,因为 216=12×18。既然每个工作日应有 2/3 不变资本,所以总产品=18×36=648(432 为不变资本)。

依此类推。

每段开头的 1、2 等数字,是指工作日的数目或不同生产领域的不同劳动种类的数目,因为我们假定每一个领域只有一个工作日。

可见,(1)A。产品为 36 小时。新加劳动为 12 小时。\textbf{待卖产品}(不变资本)=24 小时。

或者说:

(1)A。\textbf{待卖产品}即\textbf{不变资本}=24 小时。总产品为 36 小时。新加劳动为\textbf{12 小时}。后者\textbf{在 A 领域本身被消费掉}。

(2)$B^{1}$—$B^{2}$。这些领域用新加劳动购买 A 的 24 小时。\textbf{不变资本}为 48 小时。\textbf{总产品}为 72 小时。

(6)$C^{1}$—$C^{6}$。它们用新加劳动购买$B^{1}$—$B^{2}$的\textbf{72}小时(=12×6)。\textbf{不变资本}为 144,总产品为 216。依此类推。

[281]因此:

(1)A。产品=3 工作日(36 小时)。12 小时为新加劳动。\textbf{24 小时}为不变资本。

(2)$B^{1-2}$。\textbf{产品}=2×3=6 工作日(72 小时)。新加劳动=\textbf{12×2=24 小时}。\textbf{不变资本}=48=2×24 小时。

(6)$C^{1-6}$。\textbf{产品}=3×6 工作日=3×72 小时=216 劳动小时。\textbf{新加劳动}=6×12 劳动小时(=72 劳动小时)。\textbf{不变资本}=2×72=144。

(18)$D^{1-18}$。\textbf{产品}=3×3×6 工作日=3×18 工作日=54 工作日=648 劳动小时。新加劳动=12×18=\textbf{216}。不变资本=432 劳动小时。

(54)$E^{1-54}$。\textbf{产品}=3×54 工作日=162 工作日=1944 劳动小时。新加劳动=54 工作日=648 劳动小时;不变资本=1296 劳动小时。

(162)$F^{1}$—$F^{162}$。\textbf{产品}=3×162 工作日=486 工作日=5832 劳动小时,其中 162 工作日即 1944 劳动小时为新加劳动,3888 劳动小时为不变资本。

(486)$G^{1-486}$。\textbf{产品=3×486 工作日},其中 486 工作日即 5832 劳动小时为新加劳动,11664 劳动小时为不变资本。

依此类推。

这里我们已经有了一个相当可观的数目,它由 729 个不同生产领域的不同工作日 1+2+6+18+54+162+486 合计而成。这已经是一个分工相当精细的社会了。

在 A 领域中,加到不变资本 2 工作日上的,只有 12 小时劳动即 1 工作日,而工资和利润是消费它自己的产品;为了从 A 领域的总产品中仅仅卖掉不变资本 24 小时,并且又只是同新加劳动即工资和利润相交换,我们就需要:

$B^{1}$和$B^{2}$的 2 工作日。但是这 2 工作日又应有不变资本 4 工作日,这样,$B^{1-2}$的总产品等于 6 工作日。这 6 工作日必须\textbf{全部}卖掉,因为\textbf{从这里起},就假定后一个领域都不消费自己的产品,而只是把自己的利润和工资花在前一个领域的产品上。为了补偿$B^{1-2}$产品所包含的这 6 工作日,就必需有 6 工作日,而这 6 工作日又要有不变资本 12 工作日。因此$C^{1-6}$的总产品等于 18 工作日。为了用新加劳动来补偿它们,就必需有 18 工作日($D^{1-18}$),而这 18 工作日又要有不变资本 36 工作日。因而产品等于 54 工作日。为了补偿这些工作日又需要$E^{1-54}$的 54 工作日,而这 54 工作日又要有不变资本 108 工作日。产品就会等于 162 工作日。最后,为了补偿这些工作日,就需有 162 工作日,而这 162 工作日又要有不变资本 324 工作日;因而总产品就是 486 工作日。这也就是$F^{1}$—$F^{162}$的产品。最后,为了补偿这$F^{1-162}$的产品,就需有 486 工作日($G^{1-486}$),而这 486 工作日又要有不变资本 972 工作日。因此,$G^{1-486}$的总产品=972+486=1458 工作日。

但现在假定,到 G 领域,我们已到了尽头,再也不能推移下去了。[282]在任何一个社会,上述这种从一个领域到另一领域的推移,也会很快达到尽头的。这时的情况怎样呢\fontbox{?}我们有这样一种产品,它包含 1458 工作日,其中 486 日为新加劳动,972 日为物化在不变资本中的劳动。新加劳动 486 日可以在前一个领域$F^{1-162}$交换。但是不变资本包含的 972 工作日用什么东西来购买呢\fontbox{?}在$G^{486}$领域以外再也没有任何新的生产领域,因而也就没有任何新的交换领域了。在它前面的各个领域,除了$F^{1-162}$以外,什么也交换不到。而且$G^{1-486}$领域本身又把它包含的工资和利润都花在$F^{1-162}$领域了。由此看来,物化在$G^{1-486}$产品中、等于它包含的不变资本价值的 972 工作日,是不可能卖出去了。可见,我们把我们碰到的困难——A 领域中的 8 码麻布,或者说,这个领域的产品中代表不变资本价值的 24 劳动小时,即 2 工作日——推移到了将近 800 个生产部门,还是无济于事。

有人认为,如果 A 领域不是把自己的全部利润和工资花在麻布上,而是把其中一部分花在 B 和 C 的产品上,那计算就会不同了。这种想法也是无济于事的。A、B 和 C 包含的新加劳动时间量是支出的界限,因此,这些领域在任何情况下都只能支配同它们的新加劳动相等的劳动时间量。它们多买这一种产品,便会少买那一种产品。这只会把计算搅乱,丝毫也不会改变结果。

那末,我们怎么办呢\fontbox{?}

在上述计算中,我们看到:

总之,在我们的计算中,相当于新加劳动的 243 工作日事实上是可以消费掉的。最后一种产品的价值等于 486 工作日,而 A—F 所包含的全部不变资本的价值也等于 486 工作日,两者正好相等。为了说明这些工作日,我们假定 G 有 486 日的新劳动。这样一来,我们就不必再去分析 486 日的不变资本的问题,[283]可是现在,我们还得说明 G 产品中包含的 972 工作日的不变资本,因为 G 产品等于 1458 工作日(972 不变资本+486 劳动)。如果我们想要摆脱困难,假定 G 领域在劳动过程中不使用不变资本,因此产品仅仅等于 486 日的新加劳动,那末,我们的计算当然也就算清了。但是,产品中构成不变资本的价值组成部分究竟由谁支付的问题所以能解决,只是因为我们假定不变资本等于零,因而它不构成产品的价值组成部分。

为了使 A 的全部产品都能卖掉,同新加劳动相交换;为了有可能把它归结为利润和工资;A、B、C 的\textbf{全部新加劳动}\endnote{符号 B 和 C,马克思这里是在第 102 页以前使用的意义上使用的(见注 47)。马克思这里是指两个生产领域,其中每一个领域的新加劳动都是一个工作日。A、B 和 C 三个领域的新加劳动总额等于三个工作日,即等于物化在 A 领域的产品中的劳动。——第 108 页。}就必须以 A 领域的劳动产品的形式花掉。同样,为了使 B+C 的全部产品能够卖掉,就必须拿出$D^{1}$—$D^{18}$的全部新加劳动来同它相交换。\endnote{马克思在这里使用的字母符号 B 和 C 已经不是指两个生产领域,因为两个生产领域的产品总共只有 6 工作日,而马克思这里说的是 18 工作日。但是马克思用这些符号也不是指 B1—B2 和 C1—C6(马克思用 B1—B2 表示由两个生产领域组成的一组,用 C1—C6 表示由 6 个生产领域组成的一组;这 8 个领域的总产品是 24 工作日)。马克思在这里是指由 6 个生产领域组成的一组。它们的总产品为 18 工作日,因而可以同 D1—D18 的也等于 18 工作日的新加劳动相交换。——第 108 页。}同样,为了购买$D^{1}$—$D^{18}$的全部产品,必须有$E^{1-54}$的全部新加劳动。为了购买$E^{1-54}$的全部产品,必须有$F^{1-162}$的全部新加劳动。最后,为了购买$F^{1-162}$的全部产品,就需要有$G^{1-486}$的全部新加劳动时间。归根到底,由$G^{1-486}$领域代表的这 486 个生产领域中,全部新加劳动时间等于 162 个 F 领域的全部产品,而 F 领域由劳动来补偿的全部产品又等于 A、$B^{1-2}$、$C^{1-6}$、$D^{1-18}$、$E^{1-54}$、$F^{1-162}$领域的不变资本。但是,G 领域的不变资本(它比 A—$F^{162}$领域使用的不变资本多 1 倍)仍然没有得到补偿,而且也不可能得到补偿。

事实上我们发现,因为根据我们的假定,在每一个生产领域,新加劳动和过去劳动之比等于 1∶2,所以,为了购买前面那些领域的产品,每次都需要[比前面所有领域加在一起的数目]\endnote{方括号中的话是根据马克思的整个思想进程加的。按照马克思的计算,在每下一组生产领域中,生产领域的数目都比前面所有领域的总数大倍。例如,在有 18 个生产领域的 D1-18 这一组中,生产领域的数目比前面所有各组的领域的数目加在一起大 1 倍(A——1 个领域,B1-2——2 个领域,C1-6——6 个领域;共计 9 个领域)。因此,马克思在 D1-18 符号后面,在括号内写着 2×9。——第 108 页。}多 1 倍的新领域拿出全部新加劳动来。为了购买 A 的总产品,需要 A 和$B^{1-2}$的新加劳动,为了购买$C^{1-6}$的产品,需要 18 个 D 即$D^{1-18}$(即 2×9)的新加劳动,依此类推。简单说,总是需要比产品本身包含的新加劳动多 1 倍的新加劳动量,因此,对最后一个生产领域 G 来说,情况就是:为了购买这个领域的全部产品,就需要比已有的多 1 倍的新加劳动量。总之,在终点 G 上我们碰到的,恰恰是在起点 A 上已经存在的那种情况:新加劳动无论如何不可能从自己的产品中购买一个比它本身大的量,它\textbf{不可能}购买不变资本中包含的过去劳动。

因此,要用收入的价值抵补整个产品的价值是不可能的。因为除了收入以外,\textbf{没有任何基金可以用来支付生产者卖给(个人)消费者的产品},所以,整个产品的价值减去收入的价值之后,根本不可能被卖掉、被支付或被(个人)消费。但是,另一方面,任何产品都必须卖掉,并按其价格得到支付(按照假定,这里价格等于价值)。

此外,从一开始就可以预见到,把中间性交换行为即各种商品或各个生产领域的产品的卖和买加进来,并不能使我们前进一步。在考察 A 领域即第一种商品麻布时,我们有 1/3 或[283a]12 小时新加劳动和 2×12(或 24)小时包含在[不变]资本中的过去劳动。工资和利润只能从 A 商品中,因而也只能从作为 A 商品的等价物的其他某种产品中买回等于 12 劳动小时的商品。它们不可能买回自己的 24 小时的不变资本,因而也不可能买回其他某种商品形式的这个不变资本的等价物。

在 B 商品中,新加劳动和不变资本之间的比例可能不同。但是,在各个生产领域中,不变资本和新加劳动之间的比例无论怎样不同,我们总能算出这个比例的平均数。我们可以说,在整个社会或整个资本家阶级的产品中,在资本的总产品中,新加劳动等于 a,作为不变资本存在的过去劳动等于 B;换句话说,我们对 A 即麻布所假定的比例 1∶2,只不过是 a∶B 的一种象征,只不过说明在这两个要素之间,在本年内或任何一段时间内加进的活劳动和作为不变资本存在的过去劳动之间,存在着某种无论如何是确定的并且是可以确定的比例。如果加在纱上的 12 小时不是全都用来购买麻布,如果购买麻布例如只用 4 小时,那末其余 8 小时就可以用来购买其他任何产品;但是购买的总数决不能超过 12 小时。如果 8 小时用来购买其他产品,那末 A 就必须卖掉 32 小时的麻布。因此,A 的例子,对整个社会的总资本也是适用的,把不同商品的中间性交换行为加进来,只会把问题搅乱,但丝毫不会改变问题的实质。

我们假定 A 是社会的总产品;在这种情况下,这个总产品的 1/3 可以由生产者买来自己消费,生产者会用他们的等于新加劳动量即总收入额的工资和利润总额来支付它。但是要支付、购买和消费其余的 2/3,他们就没有必要的基金了。因此,新加劳动即完全分解为利润和工资的 1/3 总劳动,用它自己的产品抵补自己,或者说,只是把包含 1/3 总劳动即新加劳动或它的等价物的那部分产品价值留下来,同样,属于过去劳动的那 2/3 也必须用这种过去劳动本身的产品来抵补。换句话说,不变资本始终等于它自己,它由总产品中代表它自己的那部分价值来补偿。不同商品之间的交换,不同生产领域之间的一系列买和卖的行为,只是从这样一种意义来说会引起某种形式上的区别,即不同生产领域的不变资本都按照它们原来在这些生产领域中保持的比例相互抵补。

现在我们必须更详细地考察这一点。[283a]

\tsubsectionnonum{[(b)靠消费品生产者和生产资料生产者之间的交换不可能补偿全部社会不变资本]}

[283b]亚当·斯密在第二篇第二章考察货币流通和信用制度时(后面将同\textbf{图克}的说法相对比)提出了同样的看法,即认为一国的年产品分解为工资和利润(地租、利息等等包括在利润之中)。在那里,他说:

\begin{quote}“每一个国家的流通都可以认为是分成两个不同的领域:实业家(dealers)〈这里注明:dealers 是指“全体商人、制造业者、手工业者等等,一句话,是指一国工商业的全体当事人”〉之间的流通,实业家和消费者之间的流通。即使同一些货币,纸币或者金属货币,可以时而用于这个流通领域,时而用于那个流通领域,但这两个流通过程是不断同时进行的,因此,要使流通进行下去,各自需要有一定量的这种或那种货币。\textbf{各种实业家之间流通的商品的价值,绝不能超过实业家和消费者之间流通的商品的价值,因为无论实业家购买什么,最终必然会卖给消费者}。”(第 2 卷第 2 篇第 2 章第 292—293 页)\endnote{马克思这里引的是斯密著作的加尔涅译本。马克思在尖括号内提到的关于《dealers》这一术语的说明是加尔涅加的。——第 111 页。}\end{quote}

这个问题,还有图克,以后再谈。\endnote{对斯密和图克的这一错误论点,马克思在后面第 130—131 和 256—257 页分别作了评论。在《资本论》第二卷第二十章中,马克思指出,斯密和图克的“年收入流通所需要的货币,也足以使全部年产品流通”的观点,同斯密把社会产品的全部价值归结为收入的教条是密切相关的(马克思《资本论》第 2 卷第 20 章第 12 节)。并见马克思《资本论》第 3 卷第 49 章。——第 111 页。}

现在回过来谈我们的例子。把麻纱变为麻布的 A 领域,它的日产品等于 12 码,或 36 先令,或 36 劳动小时。在这 36 小时中,12 小时新加劳动分解为工资和利润,24 小时或 2 日等于不变资本的价值。不过后者现在已经不是以原来的纱和织机的形式存在,而是以麻布的形式存在,并且是以等于 24 小时或 24 先令的麻布量存在。这个麻布量所包含的劳动量,同麻布现在所代替的纱和织机包含的劳动量一样多。因此,用这个麻布量去交换,可以重新买回同样数量的纱和织机(假定纱和织机的价值照旧,这些工业部门的劳动生产率不变)。纺纱业者和织机厂主必须把他们的全部产品——年产品或日产品(在这里,对于我们的目的来说,是无关紧要的)——卖给织布业者,因为他们的商品只对织布业者才有使用价值,织布业者是这种商品的唯一消费者。

如果织布业者的不变资本(他每天消费的不变资本)等于 2 工作日,那末织布业者的 1 工作日就需要有纺纱业者和机器厂主的 2 工作日,这 2 工作日又按极不相同的比例分解为新加劳动和不变资本。但纺纱业者和机器厂主两人加在一起的总的日产品(假定机器厂主只生产织机),也就是说,不变资本和新加劳动合在一起,不可能多于 2 工作日,而织布业者的日产品,由于他新加进 12 小时劳动,则是 3 工作日。纺纱业者和机器厂主消费的活劳动时间,可能同织布业者消费的一样多;在这种情况下,他们的不变资本包含的劳动时间必定少些。无论如何,纺纱业者和机器厂主使用的劳动量,即物化劳动和活劳动的量(总合起来),决不可能同织布业者使用的一样多。织布业者使用的活劳动时间,可能比纺纱业者少(例如,后者一定会比亚麻种植业者少);在这种情况下,他的不变资本就会超过资本的可变部分更多。

[284]于是,织布业者的不变资本就补偿纺纱业者和织机厂主的全部资本,不仅补偿后两者自己的不变资本,而且补偿纺纱过程和机器生产过程中新加的劳动。这样一来,在这里新的不变资本就完全补偿其他的不变资本,此外还补偿全部新加到不变资本上的劳动。纺纱业者和织机厂主由于把自己的商品卖给织布业者,就不仅补偿了自己的不变资本,而且得到了自己新加劳动的报酬。织布业者的不变资本补偿纺纱业者和织机厂主自己的不变资本,并实现他们的收入(工资和利润的总额)。既然织布业者的不变资本只是补偿纺纱业者和织机厂主自己的、以纱和织机形式交给织布业者的不变资本,那末这只是一种形式的不变资本同另一种形式的不变资本相交换。实际上不变资本的价值没有任何变化。

但是,我们再往回走。纺纱业者的产品也分解为两部分:一部分是亚麻、纱锭、煤炭等等,一句话,纺纱业者的不变资本;另一部分是新加劳动。机器厂主的总产品也是这样。当纺纱业者补偿自己的不变资本时,他不仅支付纱锭制造厂主等等的全部资本,而且支付亚麻种植业者的全部资本。纺纱业者的不变资本支付这些人的资本的不变部分加上新加劳动。至于亚麻种植业者,他的不变资本,扣除农具等等之后,就归结为种子、肥料等等。我们假定,租地农场主的这部分不变资本,构成他自己的产品中每年的扣除部分(这种情况在农业中总是或多或少间接地发生),这一扣除部分每年从租地农场主自己的产品中归还给土地,即归还给生产本身。在这里,我们发现有一部分不变资本是自己补偿自己,从来不拿去出卖,因而从来不被支付,也从来不被消费,不加入个人消费。种子等东西等于一定量的劳动时间。种子等等的价值加入总产品的价值;但同一价值——因为这里指的是同一产品量(假定劳动生产率不变)——又从总产品中留出,归还给生产,而不进入流通。

这里我们看到,至少有一部分不变资本,即可以看成农业原料的东西,是自己补偿自己的。因此,这里我们有一个年生产的巨大部门(按规模和投入的资本量来说是最巨大的部门),其中很大一部分由原料(人造肥料等等除外)构成的不变资本,是自己补偿自己,不加入流通的,也就是说,不是由任何形式的收入来补偿的。因此,纺纱业者不必为了补偿(由亚麻种植业者自行补偿、自行支付的)这一部分不变资本,而把它偿还给亚麻种植业者,同样,织布业者也不必把它偿还给纺纱业者,麻布购买者也不必把它偿还给织布业者。

我们假定,所有直接或间接参加生产 12 码麻布(=36 先令=3 工作日,即 36 劳动小时)的人,都以麻布本身的形式得到补偿。首先很清楚,麻布各要素的生产者,即麻布的不变资本的生产者,\textbf{不可能消费他们自己的产品},因为这种产品是为了进一步生产而生产出来的,不加入直接[285]消费。因此,这些生产者必须用自己的工资和利润来购买麻布,购买最终加入个人消费的产品。凡是他们不以麻布的形式来消费的东西,他们必定要以麻布换来的其他可消费的产品的形式来消费。因此,其他产品的生产者消费的麻布,同麻布各要素的生产者不消费麻布而消费的其他消费品,(按价值来说)正好一样多。结果就好比麻布各要素的生产者自己以麻布的形式来消费,因为他们消费多少其他产品,其他产品的生产者就消费多少麻布。因此,完全不问交换,只考察这 12 码麻布如何在所有生产者——参加麻布本身的生产或麻布各要素的生产的人——之间分配,这整个谜就能解开。

在纱和织机中,纺纱业者和织机厂主(假定他同时又是纺机厂主)的新加劳动占 1/3,他们的不变资本占 2/3。因此,在补偿他们全部产品的 8 码麻布(即 24 小时)或 24 先令中,他们能够消费 8/3 码麻布(2+(2/3)),即 8 小时劳动或 8 先令。这样,剩下来要说明的是 5+(1/3)码或 16 劳动小时。

5+(1/3)码麻布或 16 劳动小时代表纺纱业者和织机厂主的不变资本。如果我们假定,在纺纱业者的不变资本中,原料(在这里是亚麻)占 2/3,那末亚麻种植业者就能够以麻布的形式完全消费这 2/3,因为他根本没有把自己的不变资本\fontbox{~\{}我们在这里假定,他的劳动工具等等的损耗=0\fontbox{\}~}投入流通,相反,他已经把自己的不变资本从产品中扣除,并且为了再生产而保存起来。因此,他能够从 5+(1/3)码麻布\endnote{根据前面的计算,5+(1/3)码麻布代表纺纱业者和织机厂主的全部不变资本。因此,为了确定亚麻种植业者的份额,就不应当以 5+(1/3)码,而应当以更少的麻布量为初量。后来马克思纠正了这个不确切的地方,假定只有 4 码麻布代表纺纱业者的不变资本。——第 115 页。},或者说,从 16 劳动小时中,购买 2/3;这部分等于 3+(5/9)码或 10+(2/3)劳动小时。这样,剩下来要说明的就只是 5+(1/3)码减 3+(5/9)码,或 16 劳动小时减 10+(2/3)小时,即 1+(7/9)码或 5+(1/3)劳动小时。这 1+(7/9)码或 5+(1/3)劳动小时又分解为两部分:织机厂主的不变资本和纺机厂主的总产品,这里同时假定,织机厂主和纺机厂主是一个人。[286]

我们再说一遍:

\todo{}

由此可见,在补偿织布业者不变资本的 8 码中,2 码(=6 先令=6 小时)由纺纱业者消费,2/3 码(2 先令,2 劳动小时)由制造织机和其他生产工具的厂主消费。

这样,剩下来要说明的是 8 减 2+(2/3),即 5+(1/3)码(=16 先令=16 劳动小时)。这剩下来的 5+(1/3)码(=16 先令=16 劳动小时)分配如下。假定,在代表纺纱业者的不变资本,也就是代表他的纱的各要素的 4 码中,3/4 等于亚麻,1/4 等于纺机。纺机的各要素,[287]我们在以后考察织机厂主的不变资本时再计算。纺机厂主和织机厂主是一个人。

这样,在补偿纺纱业者不变资本的 4 码中,3/4 即 3 码是\textbf{亚麻}。但是生产亚麻时使用的很大一部分不变资本无需补偿,因为它已由亚麻种植业者本人以\textbf{种子、肥料、饲料、牲畜}等等形式归还给农业生产。因此,在亚麻种植业者出卖的那部分产品中,我们在计算时应当仅仅把他的劳动工具等等的损耗算作不变资本。这里我们必须把新加劳动估计为至少等于 2/3,待补偿的不变资本至多等于 1/3。

于是:

\todo{}

因此,剩下要我们计算的是:

1 码(3 先令,3 劳动小时)=亚麻种植业者的不变资本;

1+(1/3)码(4 先令,4 劳动小时)=织机的不变资本;

最后,1 码(3 先令,3 劳动小时)应付给机器制造业者,以支付他的包含在纺机中的\textbf{总产品}。

因此,首先应当扣除机器制造业者用他制造的纺机来交换的可消费部分:

\todo{}

其次,我们把\textbf{农业机器}的价值,亚麻种植业者的不变资本,分为可消费部分和其他部分。

\todo{}

我们把织布业者的总产品中属于机器的部分合在一起,就得出:织机 2 码,纺机 1 码,农业机器 1 码,共 4 码(12 先令,12 劳动小时,或全部产品 12 码麻布的 1/3)。在这 4 码中,织机厂主可以消费 2/3 码,纺机厂主可以消费 1/3 码,农业机器厂主也可以消费 1/3 码,共 1+(1/3)码。剩下 2+(2/3)码,即:织机的不变资本 4/3 码,纺机的不变资本 2/3 码,农业机器的不变资本 2/3 码,共 8/3=2+(2/3)码(=8 先令=8 劳动小时)。剩下的这个数字就是机器制造业者的待补偿的不变资本。这个不变资本又分解为哪些部分呢\fontbox{?}一方面,分解为原料——铁、木材、皮带等等。另一方面,分解为他生产机器时所必需的工作机的损耗部分(假定这种工作机由机器制造业者自己制造)。我们假定,原料占不变资本的 2/3,机器制造机的损耗占 1/3(我们在后面再考察这 1/3)。用于木材和铁的 2/3,[288]是 2+(2/3)码(2+(2/3)=8/3=24/9)的 2/3。2+(2/3)的 1/3 是 8/9。因此,2/3 是 16/9 码。

假定在这里[在木材和铁的生产中],机器占 1/3,新加劳动占 2/3(因为这里原料不占份额)。在这种情况下,16/9 码的 2/3 补偿新加劳动,1/3 补偿机器。因此,用来补偿机器的是 16/27 码。木材生产者和铁生产者(一句话,采掘工业)的不变资本不是由原料构成,而仅仅是由我们在这里统称为机器的生产工具构成。

由此可见,8/9 码用来补偿机器制造机,16/27 码用来补偿铁生产者和木材生产者使用的机器。这样,24/27+16/27=40/27=1+(13/27)码。这个麻布量还应付给机器制造业者。

\textbf{机器}。24/27 码是对机器制造机的补偿。但机器制造机又分解为原料(铁、木材等等)、生产机器制造机时机器设备的损耗部分以及新加劳动。因而,假定这些要素各等于 1/3,那末 8/27 码就属于新加劳动,而剩下的 16/27 码,用来补偿机器制造机的\textbf{不变资本},也就是说,8/27 码用于原料,8/27 码补偿加工这种原料时机器损耗的那个价值组成部分(共 16/27 码)。

另一方面,用来补偿铁生产者和木材生产者的机器的那 16/27 码,也分解为原料、机器和新加劳动。如果新加劳动等于 1/3,那末它就等于 16/27×3=16/81 码,而这部分机器的不变资本则表现为 32/81 码,其中 16/81 码用于原料,16/81 码补偿机器的损耗。

由此可见,在机器制造业者手中,——为了补偿他的机器的损耗,——仍剩下作为不变资本的 8/27 码(他用这部分补偿他的机器制造机的损耗)以及 16/81 码(用于那些必须由铁生产者和木材生产者补偿的机器的损耗)。

另一方面,机器制造业者必须从自己的不变资本中拿出 8/27 码来补偿机器制造机中包含的原料,拿出 16/81 码来补偿铁生产者和木材生产者的机器中包含的原料。在这个麻布量中,2/3 又归结为新加劳动,1/3 归结为机器损耗。这样一来,在 24/81+16/81(=40/81)中,有 2/3 即[26+(2/3)/81]支付劳动。在这原料中[289]又剩下[13+(1/3)/81]码以补偿机器。因此,这[13+(1/3)/81]码麻布是归还给机器制造业者的。

现在,机器制造业者手中又有:8/27 码,用来补偿机器制造机的损耗,16/81 码,用来补偿铁生产者和木材生产者使用的机器的损耗;[13+(1/3)/81]属于机器制造业者的原料(铁等等)中用来补偿机器的那个价值组成部分。

我们可以这样无止境地计算下去,分数愈来愈小,但是我们这 12 码麻布永远也分不尽。

我们把以上的研究进程简略地概括一下。

起初我们说过,在不同的生产领域中,新加劳动(它一部分补偿花在工资上的可变资本,一部分构成利润即无酬剩余劳动)同有新加劳动加于其上的不变资本之间,存在着不同的比例。但是,我们可以假定 a(新加劳动)和 B(不变资本)之间的平均比例;例如,可以假定后者和前者之比平均为 2∶1=2/3∶1/3。接着,我们还说过,如果资本的每个生产领域中都是这样的比例,那末在某一个生产领域中,新加劳动(工资和利润一起)始终只能购买自己产品的 1/3,因为工资和利润加在一起,只构成物化在产品中的总劳动时间的 1/3。补偿资本家不变资本的那 2/3 产品,当然也属于资本家。但是,资本家要想继续进行生产,就必须补偿自己的不变资本,因而,必须把自己 2/3 的产品再转化为不变资本。为此,他就必须卖掉这 2/3。

但是卖给谁呢\fontbox{?}可以用利润和工资总额来购买的那 1/3 产品,已经被我们扣除。如果这个总额代表 1 工作日或 12 小时,那末价值等于不变资本的那部分产品,就代表 2 工作日或 24 小时。因此我们假定,[第二个]1/3 产品由另一生产部门的利润和工资总额来购买,最后的 1/3 又由第三个生产部门的利润和工资总额来购买。但是,在这种情况下,我们就把产品 I 的不变资本完全同工资和利润相交换,即同新加劳动相交换,办法是让产品 II 和 III 中包含的全部新加劳动去消费产品 I。而产品 II 和 III 包含的 6 工作日(不仅是新加劳动,还有过去劳动)中,任何一个工作日都既没有用产品 I 包含的劳动来补偿或购买,也没有用产品 II 和 III 包含的劳动来补偿或购买。因此,我们必须再假设,其他产品的生产者花费自己的全部新加劳动来购买产品 II 和 III,依此类推。最后,我们必须停在某种产品 X 上,它的新加劳动等于以前一切产品的不变资本的总和;可是占这个产品 2/3 的不变资本仍然卖不出去。可见,对于解决问题来说,仍然没有前进一步。在产品 X 上,就象在产品 I 上一样,问题仍然是那一个:补偿不变资本的那部分产品卖给谁\fontbox{?}难道占产品 1/3 的新加劳动能补偿产品中包含的 1/3 新劳动和 2/3 过去劳动吗\fontbox{?}难道 1/3=3/3 吗\fontbox{?}

由此可见,把困难从产品 I 推到产品 II,并依此类推下去,一句话,把仅仅作为商品交换的中介环节引进来,是不能说明任何问题的。

[290]我们不得不换个方式提出问题。

我们假定,12 码麻布(=36 先令=36 劳动小时)这个产品包含织布业者的 12 劳动小时,或者说,他的 1 工作日(必要劳动和剩余劳动合在一起,即利润和工资总额),而 2/3 代表麻布中包含的不变资本(纱和机器等)的价值。其次,为了切断各种遁词的后路并避免把中间交易引进来,我们又假定,我们的这种麻布只用于个人消费,因此不能再用作某种新产品的原料。从而我们也就假定,这种产品必须由工资和利润来支付,必须同收入相交换。最后,为了简单起见,我们假定,利润中没有一个部分再转化为资本,全部利润都作为收入来花费。

至于前 4 码,即产品的第一个 1/3(等于织布业者加进的 12 劳动小时),我们很快就把它的问题解决了。它分解为工资和利润;它们的价值等于织布业者的利润和工资总额的价值。因而它由织布企业主和他的工人自己消费。对于这 4 码来说,问题是彻底解决了。事实上,如果利润和工资不是以麻布的形式,而是以其他某种产品的形式消费掉,那末,这只是因为其他产品的生产者以麻布的形式,而不是以自己产品的形式消费本来用于他自己消费的那部分产品。例如,在 4 码麻布中,如果织布业者自己只消费 1 码,而其余 3 码,他以肉、面包、呢绒的形式来消费,那末,4 码麻布的价值仍旧是被织布业者自己消费掉,只有一点不同,就是织布业者是以其他商品的形式消费这一价值的 3/4,而其他商品的生产者则以麻布的形式来消费那些可以作为工资和利润被他们消费的肉、面包、呢绒。\fontbox{~\{}当然,在这里也象在全部研究中一样,我们始终假定,商品能够卖出去,而且是按它的价值卖出去的。\fontbox{\}~}

现在我们接触到问题本身。织布业者的不变资本现在以 8 码麻布(=24 劳动小时=24 先令)的形式存在。如果织布业者想要继续进行生产,他就必须把这 8 码麻布转化为货币,转化为 24 先令,并用这 24 先令购买市场上现有的、新生产出来的、构成他的不变资本的那些商品。为使问题简单起见,我们假定,织布业者不是过若干年后一下子补偿自己的机器,而是每天用他出卖自己产品得到的货币,以实物形式补偿机器的一部分,即这些机器每天损耗的那部分价值。等于生产过程中消费的不变资本价值的那部分产品,织布业者必须用该不变资本的各个要素来补偿,也就是用织布劳动所必需的物质生产条件来补偿。另一方面,织布业者的产品麻布,并不作为生产条件加入其他任何生产领域,它只加入个人消费。因此,织布业者为了能够补偿他的产品中代表他的不变资本的部分,只有一种办法,即把这一部分产品同收入相交换,也就是同其他生产者的产品中归结为工资和利润,因而归结为新加劳动的那部分价值相交换。这样问题就正确地提出来了。现在只是要问:在什么条件下问题才能够解决。

我们第一次提出问题时发生的一个困难,现在已经部分地消除了。虽然在每个生产领域中新加劳动都等于 1/3,而不变资本根据假定等于 2/3,但是这个由新加劳动构成的 1/3,或者说,收入(工资和利润;前面已经说明,我们在这里不谈再转化为资本的那部分利润)的价值总额,只能以直接为个人消费而进行生产的那些部门的产品的形式来消费。其余一切生产部门的产品只能作为资本来消费,只能加入生产消费。

[291]8 码(=24 小时=24 先令)所代表的不变资本,由纱(原料)和机器构成,比方说,其中 3/4 为原料,1/4 为机器。(在这里,也可以把全部辅助材料如机油、煤炭等等,都算在原料中,但为简单起见,最好把它们完全撇开不谈。)在这种情况下,纱值 18 先令或 18 劳动小时=6 码,而机器值 6 先令=6 劳动小时=2 码。

因此,如果织布业者用他的 8 码购买值 6 码的纱和值 2 码的机器,他用自己的 8 码不变资本就不仅抵补了纺纱业者和织机厂主的不变资本,而且抵补了他们的新加劳动。由此可见,在织布业者那里表现为不变资本的价值的一部分,在纺纱业者和织机厂主方面就表现为新加劳动,因此,这部分对他们来说,并不归结为资本,而归结为收入。

在 6 码麻布中,纺纱业者可以自己消费 1/3,即 2 码(=新加劳动,即利润和工资),而 4 码只是补偿他的亚麻和机器。比方说,3 码用于亚麻,1 码用于机器。当他重新购买时,他又必须用这 4 码来支付。在[得自织布业者的]2 码中,机器制造业者可以自己消费 2/3 码,其余 4/3 码只是补偿他的铁和木材即原料,以及生产机器时使用的机器设备。比方说,在 4/3 码中,1 码用于原料,1/3 码用于机器。

到现在为止,在 12 码中:(1)4 码被织布业者消费;(2)2 码被纺纱业者消费;(3)2/3 码被机器制造业者消费。总共 6+(2/3)码。因而,还剩下 5+(1/3)码。它们分配如下:

纺纱业者必须从 4 码的价值中拿出 3 码补偿亚麻,拿出 1 码补偿机器。

机器制造业者必须从 4/3 码的价值中拿出 1 码补偿铁等等,拿出 1/3 码补偿机器(他自己在制造机器时使用的那些机器)。

由此可见,纺纱业者购买亚麻,把 3 码支付给亚麻种植业者。但是亚麻种植业者有一个特点,就是他的不变资本的一部分(即种子、肥料等等,一句话,所有由亚麻种植业者再归还给土地的土地产品)完全不加入流通,因而无需从他出卖的产品中扣除;他出卖的产品(除了补偿机器、人造肥料等等的那部分以外)只代表新加劳动,所以,这个产品只分解为工资和利润。如果我们在这里也象前面那样,假定新加劳动占总产品的 1/3,那末 3 码中就有 1 码属于新加劳动这一范畴。至于其余的 2 码,我们象以前一样假定,其中 1/4 用于机器;它等于 2/4 码。相反,其余的 6/4 码仍然不得不归入新加劳动,因为亚麻种植业者的这一部分产品不包含不变资本——这种不变资本已经被他事先扣除了。因此,在亚麻种植业者那里,属于工资和利润的是 2+(2/4)码,用来补偿机器的是 2/4 码。\fontbox{~\{}这样,按照我们的计算,在有待消费的 5+(1/3)码中,已经用掉 2+(2/4)(5+(4/12)-2+(6/12)=2+(10/12)=2+(5/6)码)。\fontbox{\}~}因此,最后的这 2/4 码,亚麻种植业者必定会用来购买机器。

机器制造业者的账目现在是这样:从用于织机的不变资本中,他把 1 码花在铁等等上,把 1/3 码花在生产织机过程中机器制造机的损耗上。

但是,其次,纺纱业者会用 1 码向机器制造业者购买纺机,亚麻种植业者会用 2/4 码向机器制造业者购买农具。在这 6/4 码中,机器制造业者要消费 1/3 来补偿他的新加劳动,2/3 则用来补偿投入纺机和农具的不变资本。而 6/4=18/12。因此,机器制造业者[292]又要消费 6/12 码,而把 12/12 即 1 码转化为不变资本。(这样,在尚未消费的 2+(5/6)码麻布中,减去 1/2 码;剩下 14/6 码,即 2+(2/6)或 2+(1/3)码。)

在机器制造业者手中剩下来补偿他的不变资本的 1 码麻布中,机器制造业者必须把 3/4 花在原料即铁、木材等等上,把 1/4 支付给自己,以补偿机器制造机。

全部计算现在是这样:

\todo{}

这样,1+(3/4)码被用来向制铁业者和木材业者购买价值相等的铁和木材。7/4=21/12。但是这里产生了新问题。在亚麻种植业者那里,一部分不变资本(原料)并没有加入他所出卖的产品,因为事先已经扣除了。而在我们现在所考察的这种场合,我们必须把全部产品[铁、木材]分解为新加劳动和机器。即使假定这里新加劳动占产品的 2/3,机器占 1/3,应当消费掉的也只是 14/12 码,还会剩下 7/12 码作为不变资本,属于机器所占的部分;这 7/12 码还要回到机器制造业者手里。

因此,12 码的余数包括:机器制造业者必须支付给自己以补偿他自己的机器损耗的 1/3+1/4 码;以及制铁业者和木材业者为补偿机器而归还给机器制造业者的 7/12 码。于是,1/3+1/4=4/12+3/12=7/12;再加上制铁业者和木材业者归还的 7/12(共计 14/12,也就是 1+(2/12)或 1+(1/6))。

同织布业者、纺纱业者和亚麻种植业者完全一样,制铁业者和木材业者也必须向机器制造业者购买机器和工具。假定在 7/12 码中 1/3(2/12 码)是新加劳动。因而这 2/12 码也能够被消费掉。剩下的 5/12(其实是 4/12 和[2/3/12],不过在这里不必这样准确)代表伐木者的斧头和制铁业者的机器中包含的不变资本,而且 3/4 用于生铁、木材等等,1/4 用来补偿机器的损耗。(从 14/12 码中剩下 12/12 码,或 1 码=3 劳动小时=3 先令。)因而,在 1 码中,1/4 码用来补偿机器制造机,3/4 码用于木材、铁等等。

这样,用来补偿机器制造机损耗的是 7/12 码+1/4 码=7/12+3/12=10/12 码。另一方面,把木材和铁所占的 3/4 码再分解为它的组成部分,并把这些部分中的一部分重新还给机器制造业者,机器制造业者又把这一部分中的一部分还给制铁业者[293]和木材业者,这是徒劳无益的。始终会有一个余额,并且将无止境地演进下去。

\tsubsectionnonum{[(c)生产资料生产者中间资本同资本的交换。一年生产的劳动产品和一年新加劳动的产品]}

我们就来考察一下现在摆在我们面前的这个问题。

机器制造业者把 10/12(或 5/6)码的价值留给自己,来补偿机器的损耗。3/4(或 9/12)码代表木材和铁的相应价值。机器制造业者把它交给制铁业者和木材业者,来补偿自己的原料。麻布的总余额[不必再进一步分解为麻布的各组成部分]是 19/12(或 1+(7/12))码。

机器制造业者为补偿自己机器的损耗而给自己留出的 5/6 码余额,等于 15/6 先令=15/6 劳动小时,因而等于 2+(3/6)=2+(1/2)先令或 2+(1/2)劳动小时。这个价值不能以麻布的形式补偿给机器制造业者;因为他必须再卖掉这些麻布,以便用这 2+(1/2)先令来补偿自己机器的损耗,一句话,以便生产新的机器制造机。但是,这些麻布能够卖给谁呢\fontbox{?}卖给其他产品(铁和木材除外)的生产者吗\fontbox{?}但是,这些生产者能够以麻布的形式消费的一切,他们都已经以麻布的形式消费了。只有构成织布业者的工资和利润的那 4 码,能够同其他产品相交换(加入织布业者的不变资本的那些产品除外,或者说,由这个不变资本分解成的那种劳动除外)。可是这 4 码,我们已经计算过了。或者,机器制造业者也许会把这些麻布支付给工人\fontbox{?}但是由劳动加到产品上的一切,我们已经从他的产品中扣除了,并且按照我们的假定,这一切都以麻布的形式消费掉了。

我们换一个方式来说明问题:

\todo{}

为了计算简单起见,假定:4 码=12 先令=12 劳动小时,其中劳动(利润和工资)占 1/3,即 4/3 码或 1+(1/3)码。

剩下 2+(2/3)码为不变资本,其中 3/4 用于原料,1/4 补偿机器的损耗。2+(2/3)=8/3=32/12。这个量的 1/4 等于 8/12 码。

补偿机器损耗的这 8/12 码,就是机器制造业者手中剩下来的全部,因为 24/12 码(2 码)他要支付给铁生产者和木材生产者,以取得原料。

[294]如果让铁生产者和木材生产者再次支付机器,那是错误的,因为他们在机器上必须补偿的所有东西即 7/12 码,已经列到机器制造业者的项下了。他们生产铁和木材所必需的全部机器都已算在这一项了,所以这些机器不能再次列入计算。这样,用来支付铁和木材的最后 2 码(2+(8/12)的余额)就完全归结为劳动(因为这里没有原料),因此能够以麻布的形式消费掉。

可见,剩下来的全部余额为 8/12 码(2/3 码),它用于补偿机器制造业者使用的机器的损耗。

整个问题有一部分是这样解决的:土地耕种者的既不归结为新加劳动又不归结为机器的那部分\textbf{不变资本},根本不加入流通,事先就被扣除了;它在它自身的生产中自己补偿自己,因而,土地耕种者全部\textbf{加入流通}的产品,扣除机器之后,都分解为工资和利润,因此能够以麻布的形式消费掉。这是已经解决的一部分。

另一部分是这样解决的:在一个生产领域中表现为不变资本的东西,在其他生产领域中表现为同年内加进的新劳动。在织布业者手中表现为不变资本的东西,有很大一部分归结为纺纱业者、机器制造业者、亚麻种植业者、制铁业者、木材业者(煤炭业者等等;但是为使问题简单起见,我们不把后面这些计算在内)的收入。(这是非常明显的,例如,同一个工厂主又纺又织,他的不变资本就比织布业者的少,而他加进的劳动,即他的产品中归结为新加劳动,归结为收入即利润和工资的那部分,则比织布业者的多。例如织布业者的收入等于 4 码=12 先令,不变资本等于 8 码=24 先令。如果他同时又纺又织,他的收入就=6 码,他的不变资本也=6 码;即 2 码用于织机,3 码用于亚麻,1 码用于纺机。)

第三,直到现在我们所找到的解决办法是:为生产最终加入个人消费的产品提供原料和劳动工具的一切生产者,都不是以自己产品的形式来消费自己的收入,即代表新加劳动的利润和工资。他们只能以这里所说的可直接消费的产品形式,或者同样可以说,以交换来的、具有同等价值的、其他生产者的可直接消费的产品形式,来消费他们的产品中归结为收入的那部分价值。原料和劳动工具的生产者的新加劳动,作为价值组成部分加入最终产品,只有以最终产品的形式才被消费掉,而从使用价值来看,这个新加劳动则作为原料或消费了的机器包含在最终产品中。

因此,问题中有待解决的部分就归结为这样一点:用来补偿机器制造业者的机器制造机损耗的那 2/3 码将会怎样呢\fontbox{?}(这里所谈的正是这种机器的损耗,而不是织布业者、纺纱业者、亚麻种植业者、制铁业者、木材业者使用的工作机的损耗,因为这些工作机归结为新劳动,也就是归结为这样一种新劳动,它使本身不再在原料上有所花费的那种原料获得新机器的形式。)或者换句话说,机器制造业者在什么条件下才能以麻布的形式消费这 2/3 码(等于 2 先令或 2 劳动小时),同时又补偿自己的机器\fontbox{?}这就是问题的实质所在。这种情况确实是有的。它是必然要发生的。因而我们的任务是说明这种现象。

[295]转化为新资本(无论流动资本还是固定资本;无论可变资本还是不变资本)的那部分利润,我们在这里可以完全不去注意。它和我们的问题毫无关系,因为在这种场合,新的可变资本和新的不变资本一样,都是由\textbf{新}劳动(剩余劳动的一部分)创造和补偿的。

总之,如果把这一点撇开不谈,那末加进的(例如一年内加进的)新劳动的总额——等于利润和工资总额,即年收入总额——就统统花在那些加入个人消费的产品如食物、衣服、燃料、住宅、家具等等上面。

这些加入消费的产品总额,按其价值来说,等于一年新加劳动的总额(收入的价值总额)。这个劳动量应当等于这些产品中包含的新加劳动和过去劳动的总额。在购买这些产品时,必须不仅支付其中包含的新加劳动,而且支付其中包含的不变资本。如上所说,它们的价值等于利润和工资总额。当我们举麻布作例子时,麻布对于我们来说,代表一年内加入个人消费的产品总额。这个麻布不仅按其价值来说必须等于它的全部价值要素,而且它的全部使用价值对各个分得麻布的生产者来说必定是可消费的。它的全部价值必然分解为利润和工资,即分解为一年新加劳动的各个组成部分,虽然这个麻布是由新加劳动和不变资本构成的。

如上所说,这个问题部分地可以这样来解释:

\textbf{第一},生产麻布所必需的不变资本的一部分,既不作为使用价值,也不作为交换价值加入麻布。这就是归结为种子等等的那部分亚麻;农产品的这一部分不变资本不进入流通,而是直接或间接地归还给生产,归还给土地。这一部分自己补偿自己,因而不需要用麻布偿还。\fontbox{~\{}农民可能把自己收获的谷物,比方说,120 夸特全部卖掉。但是在这种情况下,他就必须向别的农民购买种子(例如 12 夸特)。这样一来,别的农民需要从自己产品(120 夸特)中留作种子的就不是 12 夸特,而是 24 夸特,不是 1/10,而是 1/5 了。因而,即使在这种情况下,在 240 夸特中,作为种子归还给土地的也是 24 夸特。不过,这在流通领域中确实有差别。在前一场合,每人留出 1/10,进入流通的是 216 夸特。在后一场合,进入流通的是第一个农民的 120 夸特和第二个农民的 108 夸特,共计 228 夸特。而实际消费者所占用的和原先一样,只有 216 夸特。可见,在这里我们已经有了一个例子,表明“实业家”和“实业家”之间流通的价值总额可以大于“实业家”和消费者之间流通的价值总额。\endnote{马克思这里批评的是斯密的(为图克接受的)论点:“各种实业家之间流通的商品的价值,绝不能超过实业家和消费者之间流通的商品的价值。”见前面第 111 页。——第 131 页。}\fontbox{\}~}(其次,每当一部分利润转化为新资本时,都存在着这样的差别;再其次,当“实业家”和“实业家”之间的交易持续多年时也是这样,等等。)

由此可见,生产麻布即生产可供个人消费的产品所必需的很大一部分不变资本,无需用麻布来补偿。

\textbf{第二},麻布即一年内生产出来的消费品所必需的很大一部分不变资本,在一个阶段上表现为不变资本,在另一个阶段上则表现为新加劳动,因而实际上分解为利润和工资,成为一些人的收入,而同一价值额对另一些人来说则表现为资本。例如,[织布业者的]一部分不变资本归结为纺纱业者的[新加]劳动,等等。

[296]\textbf{第三},在生产可消费的产品所必需的一切中间生产阶段,除原料和某些辅助材料之外,很大一部分产品从来不加入消费品的使用价值,而只是作为价值组成部分加入消费品;机器、煤炭、机油、油脂、皮带等等就是如此。就这些中间生产阶段由于社会分工而作为单独的部门出现来说,它们事实上都只是为下一阶段生产不变资本,而每一中间生产阶段的产品都分解为两部分:一部分代表新加劳动(这部分归结为利润和工资,并在前面指出的限定的条件\endnote{马克思是指他在第 129—130 页上所作的说明,他说他在这里把“转化为新资本的那部分利润”撇开不谈。——第 131 页。}下,归结为收入),另一部分代表消费了的不变资本的价值。因此很清楚,在每一个这样的生产领域,生产者本人也只能消费分解为工资和利润的那部分产品,即扣除等于本领域产品所包含的不变资本价值的产品量之后剩下来的那部分产品。但是这些生产者谁也不能消费上一阶段产品中的任何一部分,不能消费事实上只为下一阶段生产不变资本的所有阶段的产品中的任何一部分。

这样,虽然最终产品(麻布,在这里代表全部可消费的产品)由新加劳动和不变资本构成,以致这种消费品的最后生产者只能消费产品中归结为最后阶段上加进的劳动,即归结为工资和利润总额,归结为他们的收入的那一部分,但是,一切不变资本的生产者也都只是以可消费的产品形式来消费即实现自己的新加劳动。虽然可消费的产品由新加劳动和不变资本构成,但是它的购买价格(除等于最后阶段上新加的劳动量的那部分产品以外)体现着在这个产品的不变资本的生产过程中加进的劳动总量。不变资本的生产者不是以自己产品的形式,而是以可消费的产品的形式,来实现他们加进的全部劳动,所以,结果就好比这个可消费的产品仅仅由工资和利润,由加进的劳动构成。

麻布生产者在自己的生产领域最后制成了消费品麻布,他们从消费品麻布中自己留出一部分产品,这部分产品等于他们的收入,等于最后生产阶段上加进的劳动,等于工资和利润总额(消费品相互交换和商品事先转化为货币,对问题毫无影响)。而他们生产的另一部分消费品,他们则用来支付直接供给他们不变资本的生产者所应得的价值组成部分。因此,他们生产的这一部分消费品,全都用来抵补直接供给这种不变资本的生产者的收入和不变资本的价值。但这种不变资本的生产者又只留出价值等于他们的收入的那部分可消费的产品。另一部分他们又用来支付他们的不变资本的生产者,而这个不变资本又等于收入加不变资本。\textbf{但是,只有}当麻布这种可消费的产品的最后部分仅仅用来补偿收入,补偿新加劳动,而不再补偿不变资本\textbf{的时候,计算才能完结}。因为按照假定,麻布只加入消费,而不再构成其他生产阶段的不变资本。

对于一部分农产品来说,这一点已经得到了证明。

一般说来,只有作为原料加入最终产品的那些产品,才可以说它们是作为产品被消费的。其他的产品则只是作为价值组成部分加入可消费的产品。可消费的产品是用收入,也就是用工资和利润来购买的。因而,它的价值总额必定全部分解为工资和利润,即分解为在这种产品所经过的一切阶段上加进的不同的劳动量。现在要问:除了由生产者本人归还给生产的那部分农产品[297](种子、牲畜、粪肥等等)以外,是否还有另一部分不变资本,不作为价值组成部分加入可消费的产品,而在生产过程中以实物形式自己补偿自己呢\fontbox{?}

当然,这里可能指各种形式的固定资本,只要这种固定资本的价值本身加入生产,并在生产中被消费。

不仅在农业中(其中包括由人工进行再生产的畜牧业、养鱼业、林业),——因而不仅在衣服和食品的各种原料以及很大一部分加入工业固定资本的产品如帆、绳、皮带等等的生产中,——而且在采矿业中,也有一部分不变资本以实物形式从自己生产的产品中得到补偿。因此,这一部分不变资本就不应由进入流通的那部分产品来补偿。例如在煤炭生产中,就有一部分煤炭用来发动排水或提升煤炭用的蒸汽机。

由此看来,年产品的价值,有一部分等于采煤过程中消费的煤炭所包含的过去劳动,另一部分等于当时新加的劳动量(机器的损耗等等撇开不谈)。但是,本身由煤炭构成的那部分不变资本,会直接从这个总产品中留出,归还给生产。谁也不应把这部分补偿给生产者,因为生产者会自己补偿自己。如果劳动生产率既没有提高也没有降低,这部分产品所代表的那部分价值就保持不变,它等于产品中一部分作为过去劳动、一部分作为一年内新加劳动存在的那个劳动量的某个相应部分。在采矿工业的其他部门中,不变资本也有一部分是以实物形式得到补偿的。

产品的废料,例如飞花等等,可当作肥料归还给土地,或者可当作原料用于其他生产部门;例如破碎麻布可用来造纸。在前一种情况下,一个生产部门的一部分不变资本,就可以直接同另一个生产部门的不变资本相交换。例如棉花同用作肥料的飞花相交换。

一般说来,机器的制造和原料(煤炭、铁、木材)的生产,同其他生产阶段之间有一个主要的差别。在其他生产阶段上不发生相互作用。麻布不能成为纺纱业者的不变资本的一部分。纱本身不能成为亚麻种植业者或机器厂主的不变资本的一部分。但是,充当机器的原料的,不仅有皮带、绳子等等取自农产原料的产品,而且有木材、铁、煤炭;另一方面,机器又作为生产资料加入木材、铁、煤炭等等生产者的不变资本。由此可见,这两个部门事实上是以实物形式互相补偿自己的一部分不变资本。这里发生的是不变资本同不变资本的交换。

这里不单单是计算的问题。铁生产者要把生产铁时使用的机器的损耗算到机器制造业者的项下,而机器厂主要把他制造机器时使用的机器的损耗算到铁生产者的项下。假定铁生产者和煤炭生产者是一个人。第一,我们已经看到,他自己补偿自己的煤炭。第二,他的总产品(铁和煤炭)的价值,等于新加劳动创造的价值加机器损耗部分所包含的过去劳动创造的价值。从这个总产品中扣除补偿机器价值的铁量,剩下来的就是归结为新加劳动的铁量。后面这部分构成机器厂主、工具生产者等等的原料。对于后面这部分,机器厂主用麻布支付给铁生产者;而为了换取前一部分,机器厂主供给他机器,以补偿他的设备的损耗。

另一方面,机器制造业者的不变资本中有一部分代表他的机器制造机、工具等等的损耗,所以,这一部分既不能归结为原料(这里我们不谈[生产煤炭和铁时]使用的机器[298]和自己补偿自己的那部分煤炭),也不能归结为新加劳动,因而,既不能归结为工资,也不能归结为利润;这种损耗实际上是靠机器制造业者从自己的机器中给自己留下一部或几部机器当作机器制造机而得到补偿的。对于他的这一部分产品来说,问题只是:为了制造这一部分产品,要有一个原料的追加量。这一部分产品不代表新加劳动,因为在劳动的总产品中,一定数量的机器等于新加劳动创造的价值,另一数量的机器等于原料的价值,再一个数量的机器等于机器制造机所包含的价值组成部分。诚然,最后这个组成部分事实上也包含新加劳动。但是从价值方面来说,这种劳动等于零,因为在代表新加劳动的那一部分机器中,没有计算原料和已损耗的机器所包含的劳动;补偿原料的第二部分机器中,没有计算补偿新劳动和机器的部分;因而,从价值方面来看,第三部分机器既不包含新加劳动,也不包含原料;这一部分只代表机器的损耗。

机器厂主自己所需要的机器不出卖。它以实物形式得到补偿,由机器制造业者从总产品中留出来。这样,机器厂主所出卖的机器只代表原料(如果原料生产者的机器损耗已经算到机器厂主的项下,这些原料就只归结为劳动)和新加劳动;因而这些机器无论对机器厂主自己,还是对原料生产者,都只归结为麻布。如果专门就机器厂主和原料生产者之间的关系来说,那末原料生产者为了补偿自己机器的损耗部分,已经把相当于损耗部分的价值的铁量留出。他用这个铁量同机器厂主相交换,这样他们两人就以实物形式互相付清,这个过程同他们之间收入的分配也就毫无关系。

这个问题就是如此,我们在考察资本流通时还要回过来谈谈。\endnote{见马克思《资本论》第 2 卷第 20 章第 6 节。——第 136 页。}

不变资本实际上是这样得到补偿的:它不断地重新生产出来,并且有一部分是自己再生产自己。但是,加入可消费的产品的那部分不变资本,则由加入不可消费的产品的活劳动来支付。正因为这种劳动不由它本身的产品支付,所以全部可消费的产品都可以归结为收入。不变资本的一部分,作为年产品的一部分来考察,只是外表上的不变资本。另一部分,虽然也加入总产品,但是它既不作为价值组成部分,也不作为使用价值加入可消费的产品,而是以实物形式得到补偿,始终作为不可缺少的生产要素保留下来。

在这里,我们已经考察了全部可消费的产品如何分配,如何分解为产品中包含的各个价值组成部分和生产条件。

但是,可消费的产品(就它分解为工资这一点说,它等于资本的可变部分),可消费的产品的生产,以及生产可消费的产品所必需的不变资本各部分的生产(不管这个不变资本是否加入可消费的产品),它们总是同时并存的。因此,任何资本总是分为不变资本和可变资本,资本同时由这两部分构成;而且,资本的不变部分虽然象可变部分一样,不断由新产品来补偿,但是,只要生产以同一方式继续下去,这个不变部分就会始终以同样的形式存在下去。

[299]机器厂主和原料生产者(铁、木材等等的生产者)之间的关系是,他们事实上各用自己的一部分不变资本来互相交换(这种情况和一个人的一部分不变资本变成另一个人的收入毫无共同之处\endnote{马克思在《资本论》第二卷中批判了下面这种资产阶级观点:“对一个人是资本的东西,对另一个人就是收入;反过来说也一样。”(见马克思《资本论》第 2 卷第 20 章第 10 节)参看《资本论》第 2 卷第 19 章第 2 节第 4 小节、第 3 节和第 3 卷第 49 章。——第 137 页。}),并且在这两个互相联系的生产者当中,每一个人的产品(虽然一个产品是另一个产品的前一阶段),都作为生产资料互相加入对方的不变资本。铁、木材等等的生产者,为了换取他们所需要的机器,把等于待补偿的机器价值的铁、木材等等交给机器制造业者。机器制造业者的这一部分不变资本对于他自己来说,就好比种子对于农民一样。这是他的年产品中由他自己以实物形式补偿自己并且不构成他的收入的那一部分。另一方面,通过这种交换,以原料形式给机器制造业者不仅补偿了铁生产者使用的机器中包含的原料,而且补偿了这个机器中由新加劳动和机器制造业者自己的机器损耗构成的那部分价值。因此,对于机器制造业者来说,不仅补偿了相当于他自己的机器损耗的部分,而且还补偿了可以(当作补偿)算做别的机器包含的一部分损耗的部分。

诚然,卖给铁生产者的这种机器,也包含等于原料和新加劳动的价值组成部分。但是在别的机器中会相应地少算补偿损耗的部分。因此,铁生产者等等的这部分不变资本,即他们的年劳动产品中只补偿不变资本中代表损耗的价值组成部分的这部分产品,不加入机器制造业者卖给其他工业家的机器。至于这些别的机器的损耗,当然是用前面说过的 2/3 码麻布(=2 劳动小时)补偿给机器制造业者。机器制造业者用这些麻布购买同一价值额的生铁、木材等等,并且以自己不变资本的另一种形式,即生铁的形式,来补偿自己的损耗。由此可见,对于机器制造业者来说,他的一部分原料,除补偿原料的价值之外,还补偿他的机器损耗的价值。而在生铁生产者等等方面,这种原料只归结为新加劳动,因为这些原料(铁、木材、煤炭等等)的生产者的机器在前面已经算过了。

由此可见,麻布的一切要素归结为一定量劳动的总额,它等于新加劳动的总额,但决不等于不变资本中包含的、由于再生产而永远保留下去的全部劳动的总额。

而且,说构成一年内加入个人消费的商品总额的那个劳动量(一部分为活劳动,一部分为过去劳动),因而也就是作为收入被消费的那个劳动量,不能超过一年内新加的劳动,这种论点是同义反复。因为收入等于利润和工资总额,等于新加劳动的总额,等于包含这个劳动量的商品总额。

铁生产者和机器制造业者的例子,只是个别的例子。在其他以自己的产品互相提供生产资料的不同生产领域之间,也同样以实物形式交换它们的不变资本(虽然这种交换被一系列货币交易掩盖着)。只要有这种情况存在,加入消费的最终产品的消费者就不应补偿这种不变资本,因为它已经得到了补偿。[299]

[304]\fontbox{~\{}例如在制造机车时,每天都有成车皮的铁屑剩下。把铁屑收集起来,再卖给(或赊给)那个向机车制造厂主提供主要原料的制铁厂主。制铁厂主把这些铁屑重新制成块状,在它们上面加进新的劳动。他以这种形式把铁屑送回机车制造厂主手里,这些铁屑便成为产品价值中补偿原料的部分。就这样这些铁屑往返于这两个工厂之间,——当然,不会是同一些铁屑,但总是一定量的铁屑。这个部分不断交替地成为两个工业部门的原料,并且,从价值方面来看,始终只是从一个企业移到另一个企业。因此,它不加入最终产品,而是不变资本在实物形式上的补偿。

实际上,机器制造厂主供应的每一部机器,如果从它的价值来考察,都分解为原料、新加劳动和机器的损耗。但是加入其他领域生产的这些机器的总数,按其价值来说,只能等于机器的总价值减去不断在机器制造厂主和制铁厂主之间来回转移的那部分不变资本。

农民卖掉的任何一夸特小麦,同其他任何一夸特小麦值一样多的钱。卖掉的一夸特小麦,丝毫也不比作为种子归还给土地的那一夸特小麦便宜。然而,如果产品等于 6 夸特,每一夸特等于 3 镑,而且每一夸特都包含新加劳动、原料和机器这几个价值组成部分,如果农民必须用一夸特作种子,那末他就只卖给消费者 5 夸特,由此得到 15 镑。因而,消费者没有必要支付一夸特种子包含的价值组成部分。但是被卖掉的产品的价值等于其中包含的全部价值要素,即新加劳动和不变资本,消费者怎么能够不支付不变资本而又把这个产品买去呢\fontbox{?}问题的关键就在这里。\fontbox{\}~}\endnote{花括号内的这几段话是在手稿第 304 页,属于第四章。我们把这几段移至第三章是根据马克思在这些话开头所加的注:“接第 300 页”。手稿第 300 页有几段关于萨伊的话,开头写着:“对前一段话的补充”。把这两处加以对照,引人注意的是以下这一情况,即第 304 页的那几段话结尾提出一个问题:“消费者怎么能够……把这个产品买去呢”等等。而在关于萨伊的那几段的结尾,对这个问题做了回答:“仅仅由新加劳动构成的收入,能够支付这个……产品”等等。根据这一点,我们把手稿第 304 页的那几段话放到作为第三章第十节全节结尾部分关于萨伊的那几段之前。——第 139 页。}[304]

[300]\fontbox{~\{}对前一段话的补充。

下面这段引文表明,庸俗的萨伊对这个问题多么不了解:

\begin{quote}“要完全了解这个关于收入的问题,就必须注意,产品的全部价值分解为各种人的收入,因为任何产品的总价值,都是由促成它的生产的土地所有者、资本家和勤劳者的利润相加而成的。因此,社会的收入和生产的\textbf{总价值}相等,而不象某派经济学家\endnote{“经济学家”是十八世纪下半叶和十九世纪上半叶在法国对重农学派的称呼。——第 38、139、223、411 页。}所认为的那样,只和土地的\textbf{纯产品}相等……如果一个国家的收入只是生产出来的价值超过消费掉的价值的余额,那末从这里就会得出一个完全荒谬的结论:如果一个国家在一年内消费的价值等于它生产出来的价值,这个国家就没有任何收入了。”(同上,第 2 卷第 63—64 页)\end{quote}

实际上,这个国家在过去一年会有某些收入,但在下一年就没有任何收入了。说\textbf{一年生产的劳动产品}(\textbf{当年新加劳动的产品}只构成其中的一部分)都归结为收入,这是不对的。相反,只有说加入一年个人消费的那部分产品都归结为收入,才是对的。仅仅由新加劳动构成的收入,能够支付这个一部分由新加劳动、一部分由过去劳动构成的产品,换句话说,新加劳动在这些产品中不仅能够自己支付自己,而且能够支付过去劳动,——这是因为同样由新加劳动和过去劳动构成的另一部分产品,只补偿过去劳动,只补偿不变资本。\fontbox{\}~}

\tsectionnonum{[(11)补充:斯密在价值尺度问题上的混乱;斯密的矛盾的一般性质]}

\fontbox{~\{}对于这里考察的亚当·斯密理论的各点,还应补充如下:他在价值规定上的动摇,除了工资问题上的明显矛盾\endnote{马克思是指他在前面谈过的斯密关于“工资的自然价格”这一见解中的循环论证(见第 77 页)。——第 140 页。}以外,还有一条:混淆概念。他把作为内在尺度同时又构成价值实体的那个价值尺度,同货币称为价值尺度那种意义上的价值尺度混淆起来。由此就试图找到一个价值不变的商品作为后一种意义上的尺度,把它当作衡量其他商品的不变尺度——这是一个化圆为方问题\authornote{化圆为方问题,是古希腊的一个著名问题,即求作一正方形,使其面积等于一已知圆的面积,一般指难以解决的问题。——译者注}。关于货币意义上的价值尺度同价值由劳动时间决定这一价值规定之间的关系,请看我的著作第一部分\endnote{指《政治经济学批判》第一分册。见《马克思恩格斯全集》中文版第 13 卷第 54—66 页。——第 140 页。}。这种混淆现象在李嘉图的著作中,有些地方也可以碰到。\fontbox{\}~}[300]

\centerbox{※     ※     ※}

[299]亚·斯密的矛盾的重要意义在于:这些矛盾包含的问题,他固然没有解决,但是,他通过自相矛盾而提出了这些问题。后来的经济学家们互相争论时,时而接受斯密的这一方面,时而接受斯密的那一方面,这种情况最好不过地证明斯密在这方面的正确本能。\endnote{说明斯密矛盾的一般性质的这一段话,在本版作为结束语放在第三章结尾。这是同这段话在马克思手稿中所占的位置一致的,因为手稿上紧接这一段之后便是下一章的开始。——第 141 页。}

\tchapternonum{[第四章]关于生产劳动和非生产劳动的理论}

现在,我们转过来谈谈分析亚·斯密的观点时必须加以考察的最后一个争论点,即[300]\textbf{生产劳动和非生产劳动的区分问题}。

直到现在为止,我们看到,亚·斯密对一切问题的见解都具有二重性,他在区分\textbf{生产劳动和非生产劳动}时给生产劳动所下的定义也是如此。我们发现,在他的著作中,他称为生产劳动的东西总有两种定义混淆在一起。我们先来考察第一种正确的定义。

\tsectionnonum{[(1)资本主义制度下的生产劳动是创造剩余价值的劳动]}

从资本主义生产的意义上说,生产劳动是这样一种雇佣劳动,它同资本的可变部分(花在工资上的那部分资本)相交换,不仅把这部分资本(也就是自己劳动能力的价值)再生产出来,而且,除此之外,还为资本家生产剩余价值。仅仅由于这一点,商品或货币才转化为资本,才作为资本生产出来。只有生产资本的雇佣劳动才是生产劳动。(这就是说,雇佣劳动把花在它身上的价值额以增大了的数额再生产出来,换句话说,它归还的劳动大于它以工资形式取得的劳动。因而,只有创造的价值大于本身价值的劳动能力才是生产的。)

资本家阶级的存在,从而资本的存在本身,是以劳动生产率为基础的,但不是以绝对的劳动生产率为基础,而是以相对的劳动生产率为基础。如果一个工作日只够维持一个劳动者的生活,也就是说,只够把他的劳动能力再生产出来,[301]那末,绝对地说,这一劳动是生产的,因为它能够再生产即不断补偿它所消费的价值(这个价值额等于它自己的劳动能力的价值)。但是,从资本主义意义上来说,这种劳动就不是生产的,因为它不生产任何剩余价值。(它实际上不生产任何新价值,而只补偿原有价值;它以一种形式消费价值,为的是以另一种形式把价值再生产出来。有人说,一个劳动者,如果他的产品等于他自己的消费,他就是生产劳动者,如果他消费的东西多于他再生产的东西,他就是非生产劳动者,也就是从这个意义上说的。)

这种生产率是以相对的生产率为基础的,即工人不仅补偿原有价值,而且创造新价值;他在自己的产品中物化的劳动时间,比维持他作为一个工人生存所需的产品中物化的劳动时间要多。这种生产的雇佣劳动也就是资本存在的基础。

\fontbox{~\{}但是,假定不存在任何资本,而工人自己占有自己的剩余劳动,即他创造的价值超过他消费的价值的余额。只有在这种情况下才可以说,这种工人的劳动是真正生产的,也就是说,它创造新价值。\fontbox{\}~}

\tsectionnonum{[(2)重农学派和重商学派对生产劳动问题的提法]}

对生产劳动的这种观点,是从亚·斯密对剩余价值的起源的看法,因而是从他对资本的实质的看法,自然而然地得出来的。只要他对生产劳动持有这种观点,他就沿着重农学派甚至重商学派走过的一个方向走,不过使这个方向摆脱了错误的表述方式,从而揭示出它的内核。重农学派错误地认为,只有农业劳动才是生产的,但是他们坚持了正确的见解,即从资本主义观点来看,只有创造剩余价值的劳动,并且不是为自己而是为生产条件所有者创造剩余价值的劳动,才是生产的;只有不是为自己而是为土地所有者创造“纯产品”的劳动,才是生产的。因为剩余价值或剩余劳动时间是物化在剩余产品或“纯产品”中的。(重农学派对“纯产品”又理解错误。他们所以把它当作纯产品,是因为例如收获的小麦比工人和租地农场主吃掉的要多;可是生产出来的呢绒也比呢绒生产者即工人和企业主的衣着所需的要多。)他们对剩余价值本身的理解是错误的,因为他们对价值有不正确的看法,他们把价值归结为劳动的使用价值,而不是归结为劳动时间,不是归结为没有质的差别的社会劳动。不过,尽管如此,他们还是有一个正确的定义:雇佣劳动只有当它所创造的价值大于它本身所花费的价值的时候才是生产的。亚·斯密使这个定义摆脱了错误的表述方式,而在重农学派那里,这个定义是同错误的表述方式联系在一起的。

如果我们从重农学派追溯到重商学派,在重商学派那里也有对生产劳动的同样见解的一面,尽管他们对这一点是无意识的。重商学派的基本观点是:劳动只有在产品出口给国家带回的货币多于这些产品所值的货币(或者多于为换得这些产品而必须出口的货币)的那些生产部门,因而只有在使国家有可能在更大的程度上分沾当时新开采的金银矿的产品的那些生产部门,才是生产的。他们看到,在这些国家中已经出现了财富和中等阶级迅速增长的情况。金的这种影响事实上究竟是由什么原因造成的呢\fontbox{?}工资的增长赶不上商品价格的上涨;因此工资下降了,从而相对剩余劳动增加了,利润率提高了,但这不是因为工人的生产能力更大了,而是因为绝对工资(即工人得到的生活资料总额)降低了,总之,因为工人的状况恶化了。这样一来,在这些国家里,对使用劳动的企业主来说,劳动的生产能力实际上更大了。这个事实和贵金属的流入有关,这也就是促使重商学派把这种生产部门使用的劳动称为唯一生产劳动的原因,虽然这个原因仅仅是隐约地被意识到的。

\begin{quote}[302]“最近五、六十年以来,几乎在整个欧洲都发生了人口的惊人增加,其主要原因也许是美洲矿山生产率的增长。贵金属的大大过剩〈这当然是它们的实际价值下降的结果〉,使商品的价格比劳动的价格提高得更多;它使工人的状况恶化,同时却使雇主的利润增加,因此雇主能使用更多的流动资本来雇用工人,这就促进了人口的增加……马尔萨斯指出,美洲矿山的发现,使谷物价格提高了两三倍,而使劳动的价格只提高了一倍……供国内消费的商品的价格(例如谷物价格)不是马上跟着货币的流入就提高的;但由于农业中的利润率同工业中的利润率相比下降了,资本就从农业转到工业。这样,一切资本都开始获得比以前更高的利润,而利润的提高总是等于工资的下降。”(\textbf{约翰·巴顿}《论影响社会上劳动阶级状况的环境》1817 年伦敦版第 29 页及以下各页)\end{quote}

因此,第一,按巴顿的说法,十六世纪最后三十多年和十七世纪曾推动重商主义体系的那个现象,在十八世纪下半叶重新出现了。第二,因为只有出口的商品才按金银的已经降低的价值衡量,而供国内消费的商品仍按金银的原有价值衡量(直到资本家之间的竞争把这种用两个不同尺度衡量的现象消除为止),所以在为出口服务的生产部门中,由于工资降到原有水平之下,劳动就表现为直接生产的劳动,即创造剩余价值的劳动。

\tsectionnonum{[(3)斯密关于生产劳动的见解的二重性。对问题的第一种解释:把生产劳动看成同资本交换的劳动]}

斯密对于生产劳动所阐述的第二种见解即错误的见解,同正确的见解完全交错在一起,以致这两种见解在同一段文字中接连交替出现。所以,为了说明第一种见解,我们不得不在有些地方把引文分割成许多段。

\begin{quote}“有一种劳动加到对象上,就能使这个对象的价值增加,另一种劳动则没有这种作用。前一种劳动因为\textbf{它生产价值},可以称为\textbf{生产劳动},后一种劳动可以称为\textbf{非生产劳动}。例如,制造业工人的劳动,通常把\textbf{自己的生活费的价值和他的主人的利润,加到}他所加工的材料的价值上。相反,家仆的劳动不能使价值有任何增加。虽然主人也向制造业工人\textbf{预付}工资,但后者\textbf{实际上并没有使主人花费什么},因为由工人投入劳动的对象的价值增加了,通常通过这个增加了的价值,就把工资的价值\textbf{连同利润一起}偿还给主人了。相反,家仆的生活费永远得不到偿还。一个人,要是雇用许多制造业工人,\textbf{就会变富};要是维持许多家仆,就会变穷。”(《国民财富的性质和原因的研究》,麦克库洛赫版,第 2 卷第 2 篇第 3 章第 93 和 94 页)\end{quote}

在这段话中,——而在下面我们就要引用的紧接着的那段文字里,相互矛盾的定义更加穿插在一起,——生产劳动主要是指这样一种劳动,它除了再生产“自己的〈即雇佣工人的〉生活费”的价值之外,还生产剩余价值——“他的主人的利润”。如果制造业工人除了他自己的生活费的价值以外,不再创造剩余价值,工业家也就不能由于“雇用许多制造业工人”而\textbf{变富}。

但是,第二,亚·斯密在这里所说的生产劳动是指一般“生产价值”的劳动。我们暂且不谈这[303]后一种解释,先引证另外几段话,那里斯密的第一种见解一部分被重复了,一部分表述得更鲜明,并且主要是得到了进一步的发挥。

\begin{quote}“如果把非生产劳动者……消费的那个数量的食物和衣服,分配给生产劳动者,后者就会把他们所消费的东西的全部价值\textbf{连同利润一起}再生产出来。”(同上,第 109 页;第 2 篇第 3 章)\end{quote}

这里,生产劳动者十分明确是指这样的劳动者,他不仅把包含在工资中的生活资料的全部价值为资本家再生产出来,而且把这个价值“连同利润一起”为资本家再生产出来。

只有生产资本的劳动才是生产劳动。但是,商品或货币之所以变为资本,是因为它们直接同劳动能力交换,而且这种交换的目的,只是为了有一个比它们本身包含的劳动量更大的劳动量来补偿它们。因为劳动能力的使用价值对资本家本身来说,不在于它的\textbf{实际}使用价值,不在于某种具体劳动的效用,不在于这是纺纱者的劳动、织布者的劳动等等,正如这种劳动产品的使用价值本身并不使资本家感到兴趣一样,因为产品在他看来是商品(并且是第一形态变化之前的商品),而不是消费品。使资本家对商品感兴趣的仅仅是:商品具有的交换价值大于资本家为商品支付的交换价值。因此,劳动的使用价值在他看来就是:他收回的劳动时间量大于他以工资形式支付的劳动时间量。自然,所有以这种或那种方式参加商品生产的人,从真正的工人到(有别于资本家的)经理、工程师,都属于生产劳动者的范围。正因为如此,最近的英国官方工厂报告“\textbf{十分明确地}”把在工厂和工厂办事处就业的所有人员,除了工厂主本人以外,全都列入雇佣劳动者的范畴(见这个臭报告结尾部分以前的话)。

这里,从资本主义生产的观点给生产劳动下了定义,亚·斯密在这里触及了问题的本质,抓住了要领。他的巨大科学功绩之一(如马尔萨斯正确指出的\endnote{马克思指马尔萨斯的这两句话:对生产劳动和非生产劳动的区分是亚当·斯密著作的基石,是他的论述的主要思路的基础(马尔萨斯《政治经济学原理》1836 年伦敦第 2 版第 44 页)。——第 148 页。},斯密对生产劳动和非生产劳动的区分,仍然是全部资产阶级政治经济学的基础)就在于,他下了生产劳动是\textbf{直接同资本交换的劳动}这样一个定义,也就是说,他根据这样一种交换来给生产劳动下定义,只有通过这种交换,劳动的生产条件和一般价值即货币或商品,才转化为资本(而劳动则转化为科学意义上的雇佣劳动)。

什么是\textbf{非生产劳动},因此也绝对地确定下来了。那就是不同资本交换,而\textbf{直接}同收入即工资或利润交换的劳动(当然也包括同那些靠资本家的利润存在的不同项目,如利息和地租交换的劳动)。凡是在劳动一部分还是自己支付自己(例如徭役农民的农业劳动),一部分直接同收入交换(例如亚洲城市中的制造业劳动)的地方,不存在资产阶级政治经济学意义上的资本和雇佣劳动。因此,这些定义不是从劳动的物质规定性(不是从劳动产品的性质,不是从劳动作为具体劳动所固有的特性)得出来的,而是从一定的社会形式,从这个劳动借以实现的社会生产关系得出来的。例如一个演员,哪怕是丑角,只要他被资本家(剧院老板)雇用,他偿还给资本家的劳动,多于他以工资形式从资本家那里取得的劳动,那末,他就是生产劳动者;而一个缝补工,他来到资本家家里,给资本家缝补裤子,只为资本家创造使用价值,他就是非生产劳动者。前者的劳动同资本交换,后者的劳动同收入交换。前一种劳动创造剩余价值;后一种劳动消费收入。

这里,生产劳动和非生产劳动始终是\textbf{从货币所有者、资本家的角度}来区分的,不是从\textbf{劳动者}的角度来区分的,而加尼耳等人的荒谬论调正是从这里产生的,他们根本不懂问题的实质,竟然问道:妓女、仆役等等的劳动,或服务,或职能,会不会带来货币\fontbox{?}[303]

[304]作家所以是生产劳动者,并不是因为他生产出观念,而是因为他使出版他的著作的书商发财,也就是说,只有在他作为某一资本家的雇佣劳动者的时候,他才是生产的。

体现生产工人的劳动的商品,其使用价值可能是最微不足道的。劳动的这种物质规定性同劳动作为生产劳动的特性毫无关系,相反,劳动作为生产劳动的特性只表现一定的社会生产关系。我们在这里指的劳动的这种规定性,不是从劳动的内容或劳动的结果产生的,而是从劳动的一定的社会形式产生的。

另一方面,假定资本已掌握了全部生产,也就是说,\textbf{商品}(必须把它同单纯的使用价值区别开来)已不再由拥有这个商品的生产条件的劳动者来生产,因而只有资本家才是\textbf{商品}(只有一种商品——劳动能力除外)的生产者,那末,在这种情况下,收入必须\textbf{或者}同完全由资本来生产和出卖的商品交换,\textbf{或者}同这样一种劳动交换,购买它和购买那些商品一样,是为了消费,换句话说,仅仅是由于这种劳动所固有的物质规定性,由于这种劳动的使用价值,由于这种劳动以自己的物质规定性给自己的买者和消费者提供\textbf{服务}。对于提供这些服务的生产者来说,服务就是商品。服务有一定的使用价值(想象的或现实的)和一定的交换价值。但是对买者来说,这些服务只是使用价值,只是[305]他借以消费自己收入的对象。这些非生产劳动者并不是不付代价地从收入(工资和利润)中取得自己的一份,从生产劳动创造的商品中取得自己的一份,他们必须购买这一份,但是,他们同这些商品的生产毫无关系。

但在任何情况下,有一点是很清楚的:花在资本所生产的商品上的收入(工资和利润)愈多,能用来购买非生产劳动者的服务的收入就愈少,反过来也是一样。

劳动的物质规定性,从而劳动产品的物质规定性本身,同生产劳动和非生产劳动之间的这种区分毫无关系。例如,饭店里的厨师和侍者是生产劳动者,因为他们的劳动转化为饭店老板的资本。这些人作为家仆,就是非生产劳动者,因为我没有从他们的服务中创造出资本,而是把自己的收入花在这些服务上。但是,事实上,这些人,对我这个消费者来说,即使在饭店里也是非生产劳动者。

\begin{quote}“\textbf{无论在哪一个国家},土地和劳动的年产品中\textbf{补偿资本}的那部分,始终只\textbf{直接}用来维持生产劳动者的生活。它只\textbf{支付生产劳动的工资}。而\textbf{直接}用来构成收入(不管作为利润还是作为地租)的那部分,则既可以用来维持生产劳动者的生活,也可以用来维持非生产劳动者的生活。一个人无论把自己的哪一部分基金用作资本,他总是希望这部分基金能得到补偿并带来利润。因此,他只用它来维持\textbf{生产劳动者}的生活;这部分基金为资本家执行了资本的职能之后,便成为生产劳动者的收入。每当资本家用他的一部分基金来\textbf{维持}任何一种\textbf{非生产劳动者的生活},这部分基金便立即从他的资本中抽出,加入他用于直接消费的基金。”(同上[麦克库洛赫版第 2 卷],第 98 页)\end{quote}

显然,随着资本日益掌握全部生产,从而随着家庭工业和小工业——总之,为本身消费进行生产而产品不是商品的那种工业——逐渐消失,非生产劳动者,即以服务直接同收入交换的劳动者,绝大部分就只提供\textbf{个人}服务,他们中间只有极小部分(例如厨师、女裁缝、缝补工等)生产物质的使用价值。他们不生产\textbf{商品}是理所当然的。因为商品本身从来不是直接的消费对象,而是交换价值的承担者。因此,在资本主义生产方式发达的条件下,这些非生产劳动者只有极小部分能够直接参加物质生产。这一部分人只有用自己的服务同收入交换,才参加物质生产。正如亚·斯密所指出的,这不妨碍这些非生产劳动者的服务的价值,是由并且可以由决定生产劳动者的价值的同样方法(或类似方法)来决定。这就是说,由维持他们的生活或者说把他们生产出来所必需的生产费用来决定。这里还牵涉到别的一些不归这里考察的情况。

[306]生产劳动者的劳动能力,对他本人来说是商品。非生产劳动者的劳动能力也是这样。但是,生产劳动者为他的劳动能力的买者生产商品。而非生产劳动者为买者生产的只是使用价值,想象的或现实的使用价值,而决不是商品。非生产劳动者的特点是,他不为自己的买者生产商品,却从买者那里获得商品。

\begin{quote}“某些最受尊敬的社会阶层的劳动,象家仆的劳动一样,不生产任何价值……例如,君主和他的全部文武官员、全体陆海军,都是非生产劳动者。他们是社会的公仆,靠别人劳动的一部分年产品生活……应当列入这一类的,还有……教士、律师、医生、各种文人;演员、丑角、音乐家、歌唱家、舞蹈家等等。”(同上,第 94—95 页)\end{quote}

生产劳动和非生产劳动的这种区分本身,正如前面已经说过的,既同劳动独有的特殊性毫无关系,也同劳动的这种特殊性借以体现的特殊使用价值毫无关系。在一种情况下劳动同资本交换,在另一种情况下劳动同收入交换。在一种情况下,劳动转化为资本,并为资本家创造利润;在另一种情况下,它是一种支出,是花费收入的一个项目。例如,钢琴制造厂主的工人是生产劳动者。他的劳动不仅补偿他所消费的工资,而且在他的产品钢琴中,在厂主出售的商品中,除了工资的价值之外,还包含剩余价值。相反,假定我买到制造钢琴所必需的全部材料(或者甚至假定工人自己就有这种材料),我不是到商店去买钢琴,而是请工人到我家里来制造钢琴。在这种情况下,钢琴匠就是非生产劳动者,因为他的劳动直接同我的收入相交换。

\tsectionnonum{[(4)斯密对问题的第二种解释:生产劳动是物化在商品中的劳动]}

然而,有一点是清楚的:随着资本掌握全部生产,——因而一切商品的生产都是为了出卖,而不是为了直接消费,劳动生产率也相应地增长,——生产劳动者和非生产劳动者之间的物质差别也就愈来愈明显地表现出来,因为前一种人,除极少数以外,将仅仅生产商品,而后一种人,也是除极少数以外,将仅仅从事个人服务。因此,第一种人将生产直接的、物质的、由\textbf{商品}构成的财富,生产一切不是由劳动能力本身构成的商品。这就是促使亚·斯密除了作为基本定义的第一种特征以外,又加上另一些特征的理由之一。

这样,由于斯密的各种不同的想法交织在一起,就有了下面这一段话:

\begin{quote}“家仆的劳动〈与制造业工人的劳动不同〉……\textbf{不能使价值有任何增加}……家仆的生活费\textbf{永远得不到偿还}。一个人,要是雇用许多制造业工人,就会变富;要是维持许多家仆,就会变穷。然而\textbf{后者的劳动}也同前者的劳动一样,\textbf{有它的价值},理应得到报酬。不过,制造业工人的劳动\textbf{固定和物化在一个特定的对象或可以出卖的商品中,而这个对象或商品在劳动结束后,至少还存在若干时候}。可以说,这是在其物化过程中积累并储藏起来,准备必要时在另一场合拿来利用的一定量劳动。这个对象,或者可以说,这个对象的价格,后来到必要时,能够把一个同原先生产它所花费的劳动相等的劳动量推动起来。相反,家仆的[307]劳动不\textbf{固定}或不\textbf{物化在一个特定的对象或可以出卖的商品中。他的服务通常一经提供随即消失,很少留下某种痕迹或某种以后}能够用来取得同量服务的\textbf{价值}。某些最受尊敬的社会阶层的劳动,象家仆的劳动一样,\textbf{不生产任何价值,不固定或不物化在任何耐久的对象或可以出卖的商品中}。”(同上,第 93—94 页)\end{quote}

我们在这里看到,用来说明非生产劳动者的特点的有以下这些定义,这些定义同时显露了亚·斯密内在思想进程的各个环节:

非生产劳动者的劳动“不生产任何价值”,“不能使价值有任何增加”,“〈非生产劳动者的〉生活费永远得不到偿还”,“它不\textbf{固定}或不\textbf{物化在一个特定的对象或可以出卖的商品中}”。相反,“他的服务通常一经提供随即消失,很少留下某种痕迹或某种\textbf{以后}能够用来取得同量服务的价值”。最后,“它不固定或不物化在任何\textbf{耐久的对象或可以出卖的商品中}”。

这里,“生产的”和“非生产的”这些术语是在和原来不同的意义上说的。这里谈的已经不是剩余价值的生产——剩余价值的生产本身就意味着为已消费的价值再生产出一个等价。这里谈的是:一个劳动者,只要他用自己的劳动把他的工资所包含的那样多的价值量加到某种材料上,提供一个等价来代替已消费的价值,他的劳动就是生产劳动。这里就越出了和社会形式有关的那个定义的范围,越出了用劳动者对资本主义生产的关系来给生产劳动者和非生产劳动者下定义的范围。从第四篇第九章(亚·斯密在这里批判了重农学派的学说)可以看出,斯密走入这条歧途,是因为他在阐述自己的见解时一方面反对重农学派,另一方面又受到重农学派的影响。如果工人在一年内只补偿自己工资的等价,那末,他对资本家来说就不是生产劳动者。固然,他会给资本家补偿自己的工资即自己劳动的购买价格。但是这笔交易就好比资本家购买这个劳动者所生产的商品一样。资本家支付了用来生产这个商品的不变资本和工资所包含的劳动。他现在以商品形式占有的劳动和以前以货币形式占有的劳动是同一个量。他的货币没有因此而转化为资本。这种情况,就好比工人本人是自己的生产条件的占有者一样。他每年必须从自己年产品的价值中留出生产条件的价值,以便补偿它们。他一年内消费的,或者说,可以消费的,是他的产品中等于他当年加在自己不变资本上的新劳动的那部分价值。因此,在这种情况下,也就不会有资本主义生产了。

亚·斯密把这种劳动称为“生产的”,第一个理由是因为重农学派把它称为“不生产的”和“不结果实的”。

斯密在这一章里对我们说:

\begin{quote}“第一,[重农学派]承认,这个阶级〈即不从事农业的那些工业阶级〉\textbf{每年再生产出}自己的年消费\textbf{价值,\CJKunderdot{并且至少保持使他们能够就业}和生存的\CJKunderdot{基金或资本}}……诚然,租地农场主和农业工人,除了使他们能够就业和生存的资本以外,每年还再生产出一个\textbf{纯产品},即土地所有者的纯地租……租地农场主和农业工人的劳动,无疑要比商人、手工业者和制造业者的劳动具有更大的生产能力。但是,一个阶级的产品超过另一个阶级的产品,并不能使另一个阶级成为\textbf{不生产}的和\textbf{不结果实的}。”(同上,第 3 卷第 530 页[加尔涅的法译本])\end{quote}

可见,亚·斯密在这里回到重农学派的[308]观点上去了。农业劳动是生产剩余价值,因而也是生产“纯产品”的真正的“生产劳动”。斯密放弃了自己的剩余价值观点,接受了重农学派的观点。同时他又反对重农学派,提出制造业劳动(他认为还有商业劳动)也还是生产的,尽管不是就这个词的最高意义来说的。因此,斯密越出了和社会形式有关的那个定义的范围,越出了从资本主义生产的观点来给“生产劳动者”下定义的范围;他提出这样一个论点来反对重农学派:不从事农业的阶级,工业阶级,会把自己的工资再生产出来,因而还是会把一个等于他的消费的价值生产出来,从而“至少保持使他们能够就业的基金或资本”。这样,在重农学派的影响下,同时在反对重农学派的情况下,便产生了他对“生产劳动”的第二个定义。

\begin{quote}亚·斯密说:“第二,因此,象看待家仆那样来看待手工业者、制造业者和商人,是根本不正确的。\textbf{家仆的劳动不能保持使他能够就业和生存的基金。家仆完全是靠他主人的开支来就业和维持生活的,他所完成的劳动不是那种能补偿这些开支的劳动}。这种劳动是\textbf{服务,通常一经提供随即消失;它不固定和不物化在一个能够补偿他们的生活费和工资的价值的可以出卖的商品中}。相反,手工业者、制造业者和商人的劳动却自然地\textbf{固定和物化在可以出卖或交换的对象中}。正因为如此,我在论\textbf{生产劳动}和\textbf{非生产劳动}那一章中,把手工业者、制造业者和商人算作\textbf{生产的}劳动者,而把家仆算作\textbf{不生产的}和\textbf{不结果实的}劳动者。”(同上,第 531 页)\end{quote}

一旦资本掌握了全部生产,收入只要同劳动交换,它便不是直接同生产\textbf{商品}的劳动交换,而是同单纯的\textbf{服务}交换。收入的一部分同充当使用价值的\textbf{商品}交换,一部分同作为使用价值来消费的\textbf{服务}本身交换。

\textbf{商品}和劳动能力本身不同,它是以物质形式同人对立着的物,它对人有一定的效用,在它身上固定了、物化了一定量的劳动。

这样,我们就得出一个实质上已经包含在第一点中的定义:用自己的劳动\textbf{生产商品}的工人是生产的,并且这个工人消费的商品不多于他生产的东西,不多于他的劳动所值。他的劳动固定和物化在“\textbf{可以出卖或交换的对象中}”,“\textbf{一个能够补偿他们}〈即生产这些商品的工人〉\textbf{的生活费和工资的价值的可以出卖的商品中}”。生产工人生产商品,从而把他以工资形式不断消费的可变资本不断再生产出来。他把支付给他的“使他能够就业和生存的”基金不断生产出来。

\textbf{第一},亚·斯密自然把直接耗费在物质生产中的各类脑力劳动,算作“固定和物化在可以出卖或交换的商品中”的劳动。斯密在这里不仅指直接的手工工人或机器工人的劳动,而且指监工、工程师、经理、伙计等等的劳动,总之,指在一定物质生产领域内为生产某一商品所需要的一切人员的劳动,这些人员的共同劳动(协作)是制造商品所必需的。的确,他们把自己的全部劳动加到不变资本上,并使产品的价值提高这么多。(这在多大的程度上适用于银行家\endnote{关于银行家和他们在资本主义社会中的寄生作用,见马克思《资本论》第 3 卷第 30、32 和 33 章。——第 156 页。}等人呢\fontbox{?})

[309]\textbf{第二},亚·斯密说,非生产劳动者的劳动\textbf{通常}不是这样。亚·斯密非常清楚地知道,即使资本掌握了物质生产,因而家庭工业基本上消失了,直接到消费者家里为他创造使用价值的小手工业者的劳动消失了,——即使在这种情况下,我叫到家里来缝制衬衣的女裁缝,或修理家具的工人,或清扫、收拾房子等等的仆人,或烹调肉食等等的女厨师,他们也完全和在工厂做工的女裁缝、修理机器的机械师、洗刷机器的工人以及作为资本家的雇佣工人在饭店干活的女厨师一样,把自己的劳动固定在某种物上,并且确实使这些物的价值提高了。他们所生产的使用价值,从可能性来讲,也是商品:衬衣可能拿到当铺去当掉,房子可能卖掉,家具可能拍卖等等。因此,上述人员从可能性来讲,也生产了商品,把价值加到了自己的劳动对象上。但他们是非生产劳动者中极少的一部分人,那些适用于他们的说法,对广大家仆、牧师、政府官吏、士兵、音乐家等等则是不适用的。

然而,不管这些“非生产劳动者”人数有多少,有一点无论如何是清楚的(斯密也承认这一点,他说了一句有限制的话:“这些服务\textbf{通常}一经提供随即消失”),那就是:使劳动成为“生产的”或“非生产的”劳动的,既不一定是劳动的这种或那种特殊形式,也不是劳动产品的这种或那种表现形式。同一劳动可以是生产的,只要我作为资本家、作为生产者来购买它,为的是用它来为我增加价值;它也可以是非生产的,只要我作为消费者来购买它,只要我花费收入是为了消费它的(劳动的)使用价值,不管这个使用价值是随着劳动能力本身活动的停止而消失,还是物化、固定在某个物中。

对于一个以资本家身分购买女厨师的劳动的人来说,即对于一个饭店老板来说,女厨师在饭店里是生产商品。羊肉饼的消费者应当对她的劳动付钱,而这个劳动为饭店老板补偿(撇开利润不谈)他用以继续支付女厨师的基金。相反,如果女厨师为我烹调肉食等等,我购买女厨师的劳动,不是为了把这个劳动当作一般劳动来获取剩余价值,而是为了把它当作这种特定的具体劳动来使用;那末,在这种情况下,她的劳动就是非生产的,虽然这种劳动也固定在物质产品中,而且同样可能成为(从结果来看)可以出卖的商品,就象它对饭店老板来说确实是商品一样。可是,这里仍然有重大的差别(实质上的差别):女厨师并不补偿我(私人)用以支付她的基金。因为我购买她的劳动,不是把它作为构成价值的要素,而完全是为了它的使用价值。她的劳动不补偿我用以支付她的基金,即不补偿我给她的工资,这就好比我在饭店里吃的一顿午餐本身,不能使我再购买和吃一顿相同的午餐一样。但这种差别在商品中间也是存在的。资本家为补偿自己的不变资本而购买的商品(例如棉布,假如他有一个棉布印花工厂的话),会以印花布形式为资本家补偿它的价值。相反,如果资本家购买这个商品是为了把印花布本身消费掉,那末,这个商品就不会补偿他的开支。

其实,社会上人数最多的一部分人——工人阶级——都必须为自己进行这种非生产劳动;但是,他们只有先进行了“生产的”劳动,才能从事这种非生产劳动。工人只有生产了可以支付肉价的工资,才能给自己煮肉;他只有生产了家具、房租、靴子的价值,才能把自己的家具和住房收拾干净,把自己的靴子擦干净。因此,从这个生产工人阶级本身来说,他们为自己进行的劳动就是“非生产劳动”。如果他们不先进行生产劳动,这种非生产劳动是决不能使他们[310]重新进行同样的非生产劳动的。

\textbf{第三},另一方面,剧院、歌舞场、妓院等等的老板,购买对演员、音乐家、妓女等等的劳动能力的暂时支配权(事实上通过了曲折的途径,这个途径只有从经济形式的观点来看才有意义,它不影响过程的结果);他们购买这种所谓“非生产劳动”,它的“服务一经提供随即消失”,不固定或不物化在一个“耐久的〈换句话说,“特殊的”〉对象或可以出卖的商品中”(在这些服务本身以外)。把这些服务出卖给公众,就为老板补偿工资并提供利润。他这样买到的这些服务,使他能够重新去购买它们,也就是说,这些服务会自行更新用以支付它们的基金。同样的情况也适用于例如律师事务所的书记的劳动,所不同的只是,书记的服务大部分还体现在十分庞大的“特殊对象”上,即大堆的文件这个形式上。

不错,对老板本身来说,这些服务是由公众的收入支付的。但同样不错的是,一切产品,只要它们用于个人消费,情况也完全是这样。固然,国家不能出口这些服务本身;但它能出口提供这些服务的人。例如,法国出口舞蹈教员、厨师等等,德国出口学校教师。当然,随着舞蹈教员和学校教师的出口,也出口了他们的收入,可是舞鞋和书本的出口,却给国家提供了一笔补偿它们的价值。

因此,从一方面说,所谓非生产劳动有一部分体现在物质的使用价值中,这些使用价值同样可能成为商品(“可以出卖的商品”),从另一方面说,一部分纯粹的服务(它不采取实物的形式,不作为物而离开服务者独立存在,不作为价值组成部分加入某一商品),能够(由\textbf{直接}购买劳动的人)用资本来购买,能够补偿自己的工资并提供利润。总之,这些服务的生产有一部分从属于资本,就象体现在有用物品中的劳动有一部分直接用收入来购买,不从属于资本主义生产一样。

\textbf{第四},整个“商品”世界可以分为两大部分:第一,劳动能力;第二,不同于劳动能力本身的商品。有一些服务是训练、保持劳动能力,使劳动能力改变形态等等的,总之,是使劳动能力具有专门性,或者仅仅使劳动能力保持下去的,例如学校教师的服务(只要他是“产业上必要的”或有用的)、医生的服务(只要他能保护健康,保持一切价值的源泉即劳动能力本身)——购买这些服务,也就是购买提供“可以出卖的商品等等”,即提供劳动能力本身来代替自己的服务,这些服务应加入劳动能力的生产费用或再生产费用。不过,亚·斯密知道,“教育”费在工人群众的生产费用中是微不足道的。在任何情况下,医生的服务都属于生产上的非生产费用\authornote{不直接参加生产过程,但在一定条件下又非有不可的辅助费用。——编者注}。可以把它算入劳动能力的修理费。假定工资和利润由于某种原因同时下降,从总价值来看下降了(例如由于民族变懒),从使用价值来看也下降了(由于歉收等等,劳动的生产能力降低);总之,假定由于上一年新加劳动减少和新加劳动的生产能力降低,产品中价值等于收入的那一部分减少了。这时,如果资本家和工人还想以物质产品的形式消费原先那样大的价值量,他们就要少购买医生、教师等等的服务。如果他们对医生和教师必须继续花费以前那样大的开支,他们就要减少对其他物品的消费。因此,很明显,医生和教师的劳动不直接创造用来支付他们报酬的基金,尽管他们的劳动加入一般说来是创造一切价值的那个基金的生产费用,即加入劳动能力的生产费用。

[311]亚·斯密继续写道:

\begin{quote}“第三,说手工业者、制造业者和商人的劳动不增加社会的\textbf{实际收入},从任何角度来看都是不对的。例如,即使我们象这里考察的理论所假定的那样假定,这个阶级每日、每月、每年消费的价值,恰好等于它当日、当月、当年生产的价值,也决不能由此得出结论说,他们的劳动丝毫没有增加社会的实际收入,没有增加社会的土地和劳动的年产品的实际价值。例如,一个手工业者在收获后 6 个月内完成了价值 10 镑的劳动,即使他在这段时间也消费了价值 10 镑的谷物和其他生存资料,他事实上也已给社会的土地和劳动的年产品增加了 10 镑价值。他把价值 10 镑的半年收入消费在谷物和其他生存资料上,同时又用自己的劳动生产了一个相等的价值,用这个价值可以为他本人或任何别人购买同样多的半年收入。因此,这 6 个月内所消费和生产的价值不等于 10 镑,而等于 20 镑。当然,完全可能,在任何时候现有的这个价值都不超过 10 镑。但是,如果手工业者消费的这价值 10 镑的谷物和其他生存资料,由士兵或家仆来消费,那末,到 6 个月末存在的这部分年产品的价值,就会比由于有手工业者的劳动而实际存在的少 10 镑。可见,即使假定手工业者生产的价值从来没有超过他消费的价值,但在任何时候市场上现有的商品的总价值,都会由于有他的劳动而比没有他的劳动时要大。”(同上,第 4 篇第 9 章第 531—533 页[加尔涅的法译本第 3 卷])\end{quote}

难道任何时候市场上现有的商品的[总]价值,不是由于有“非生产劳动”而比没有这种劳动时要大吗\fontbox{?}难道任何时候市场上除了小麦、肉类等等之外,不是还有妓女、律师、布道、歌舞场、剧院、士兵、政治家等等吗\fontbox{?}这帮人得到谷物和其他生存资料或享乐并不是无代价的。为了得到这些东西,他们把自己的服务提供给或强加给别人,这些服务本身有使用价值,由于它们的生产费用,也有交换价值。任何时候,在消费品中,除了以商品形式存在的消费品以外,还包括一定量的以服务形式存在的消费品。因此,消费品的总额,任何时候都比没有可消费的服务存在时要大。其次,价值也大了,因为它等于维持这些服务的商品的价值和这些服务本身的价值。要知道,在这里就象每次商品和商品相交换一样,是等价物换等价物,因而同一价值具有二重的形式:一次在买者一方,另一次在卖者一方。

\fontbox{~\{}亚·斯密关于重农学派继续写道:

\begin{quote}“当这一体系的拥护者断言,手工业者、制造业者和商人的\textbf{消费等于他们所生产的东西的价值}时,他们大概仅仅是指这一情况:这些劳动者的\textbf{收入},或者说,\textbf{供他们消费的基金,等于这个价值}〈即他们所生产的东西的价值〉。”(同上,第 533 页)\end{quote}

如果把工人和企业主放在一起来看,重农学派在这一点上是对的;企业主的利润包括地租,地租只是企业主利润的一个特殊项目。\fontbox{\}~}

[312]\fontbox{~\{}\textbf{亚·斯密}在同一个场合,即在批判重农学派的场合——第四篇第九章[加尔涅的译本第 3 卷]——指出:

\begin{quote}“一个社会的土地和劳动的年产品,只能用两种办法增加:\textbf{第一,改善}当时在这个社会发生作用的\textbf{有用劳动的生产能力};或者\textbf{第二,增加这种劳动的量}。要使有用劳动的生产能力有所改善或增长,就必需\textbf{改进工人的技能或改进他用来劳动的机器}……当时在社会上使用的\textbf{有用劳动的量的增加},完全取决于\textbf{把这种劳动推动起来的资本的增加,而这种资本的增加,又必定恰好等于}管理这一资本的人或把资本借给他们的另一些人从自己的收入中\textbf{节约下来的数额}。”(第 534—535 页)\end{quote}

这里是双重的循环论证。\textbf{第一},年产品的增加是由于劳动生产率的提高。而提高劳动生产率的一切手段(只要这种提高不是由自然的偶然情况,如特别有利的天气等等引起的)都要求增加资本。但是,要增加资本,又必需增加劳动的年产品。这是第一个循环论证。\textbf{第二},年产品可以通过增加所使用的劳动量来增加。但是,只有先增加“把这种劳动推动起来”的资本,才能增加所使用的劳动量。这是第二个循环论证。斯密试图靠“\textbf{节约}”来摆脱这两个循环论证。节约一词,他指的是收入转化为资本。

把全部利润看成资本家的“收入”,这种看法本身就是错误的。相反,资本主义生产的规律要求把工人完成的剩余劳动即无酬劳动的一部分转化为资本。当单个资本家作为资本家即作为资本职能的执行者行动的时候,利润转化为资本,当然,在他本人看来可能象是一种节约,但即使对他本人来说,这种转化也是以必需有准备金的形式表现出来的。然而劳动量的增加不仅取决于工人人数,而且取决于工作日的长度。因而,即使转化为工资的那部分资本不增加,劳动的量也可能增加。在这种情况下也不需增加机器等等的数量(虽然机器磨损得快一些,但并不会使这里的问题有所改变)。唯一必需增加的,是用作种子等等的那部分原料。同时,这一点仍然是对的:在一个国家里(如果把对外贸易撇开不谈),剩余劳动首先必须在农业中出现,然后才有可能在从农业取得原料的那些工业部门中出现。一部分原料——煤、铁、木材、鱼(例如,作为肥料)等等,一切非动物性的肥料,可以用单纯增加劳动(工人的人数不变)的办法取得。因此,这些原料是不会缺乏的。另一方面,前面已经指出,生产率的提高最初总是只以资本的积聚为前提,而不是以资本的积累为前提。\endnote{关于资本的积聚是劳动生产率提高的最初条件,马克思是在他的 1861—1863 年手稿第 IV 本第 171—172 页(《相对剩余价值》一节,《分工》一小节中)谈到的。——第 162 页。}但以后这两个过程是相互补充的。\fontbox{\}~}

\fontbox{~\{}斯密在下面一段话里正确地指出了促使重农学派宣传自由放任\endnote{自由放任(原文是:laissezfaire,laissezaller,亦译听之任之)是重农学派的口号。重农学派认为,经济生活是受自然规律调节的,国家不得对经济事务进行干涉和监督;国家用各种规章进行干涉,不仅无益,而且有害;他们要求实行自由主义的经济政策。——第 27、42、162 页。},即自由竞争的原因:

\begin{quote}“两个不同的居民集团〈城市和乡村〉之间的贸易,归根到底,是一定量的原产品同一定量的制造业产品交换。因此,后者愈贵,前者愈贱,凡是在一个国家里能提高制造业产品价格的东西,都会降低土地的原产品的价格,从而使农业发展缓慢。”但是,加在制造业和对外贸易上的一切约束和限制,都会使制造业产品等等变贵。因此,等等。(\textbf{斯密},同上[加尔涅的法译本第 3 卷],第 554—556 页)\fontbox{\}~}\end{quote}

\centerbox{※     ※     ※}

[313]这样,斯密对“生产劳动”和“非生产劳动”的第二种见解(更确切地说,同上述他的另一种见解交错在一起的见解)可归结如下:生产劳动就是生产\textbf{商品}的劳动,非生产劳动就是不生产“任何商品”的劳动。斯密不否认,这两种劳动\textbf{都是商品}。请看前面讲的\authornote{见本册第 152 页。——编者注}:“后者的劳动也同前者的劳动一样,有它的价值,理应得到报酬”(就是说,从经济学来看;无论对这种劳动还是那种劳动,都谈不上从道德等等观点来看)。商品的概念本身包含着劳动体现、物化和实现在自己的产品中的意思。劳动本身,在它的直接存在上,在它的活生生的存在上,不能直接看作商品,只有劳动能力才能看作商品,劳动本身是劳动能力的暂时表现。只有这种观点才能使我们既弄清楚真正的雇佣劳动的概念,又弄清楚“非生产劳动”的概念,而亚·斯密到处都用生产“非生产劳动者”所必需的生产费用来给非生产劳动下定义。由此可见,\textbf{商品}必须看作一种和劳动本身不同的存在。这样,商品世界就分为两大类:

一方面是劳动能力。

另一方面是商品本身。

但是,对劳动的物化等等,不应当象亚·斯密那样按苏格兰方式去理解。如果我们从商品的交换价值来看,说商品是劳动的化身,那仅仅是指商品的一个想象的即纯粹社会的存在形式,这种存在形式和商品的物体实在性毫无关系;商品代表一定量的社会劳动或货币。使商品产生出来的那种具体劳动,在商品上可能不留任何痕迹。从制造业商品来说,这个痕迹保留在原料所取得的外形上。而在农业等等部门,例如小麦、公牛等等商品所取得的形式,虽然也是人类劳动的产品,而且是一代一代传下来、一代一代补充的劳动的产品,但这一点在产品上是看不出来的。还有这样的产业劳动部门,在那里,劳动的目的决不是改变物的形式,而仅仅是改变物的位置。例如,把商品从中国运到英国等等,在物本身谁也看不出运输时花费的劳动所留下的痕迹(除非有人想起这种东西不是英国货)。因此,决不能象上面所说的那样去理解劳动在商品中的物化。(这里所以产生迷误,是因为社会关系表现为物的形式。)

虽然如此,商品表现为过去的、物化的劳动这个说法还是对的,因而,如果它不表现为物的形式,它就只能表现为劳动能力本身的形式,但永远不能直接表现为活劳动本身(只有通过某种曲折的途径,才能表现为活劳动本身,这种途径在实践上似乎是无关紧要的,但在确定各种不同的工资的时候,则不然)。由此可见,斯密本应承认,生产劳动或者是生产商品的劳动,或者是直接把劳动能力本身生产、训练、发展、维持、再生产出来的劳动。亚·斯密把后一种劳动从他的生产劳动项目中除去了;他是任意这样做的,但他受某种正确的本能支配,意识到,如果他在这里把后一种劳动包括进去,那他就为各种冒充生产劳动的谬论敞开了大门。

因此,如果我们把劳动能力本身撇开不谈,生产劳动就可以归结为生产商品、生产物质产品的劳动,而商品、物质产品的生产,要花费一定量的劳动或劳动时间。一切艺术和科学的产品,书籍、绘画、雕塑等等,只要它们表现为物,就都包括在这些物质产品中。但是,其次,劳动产品必须是这种意义上的\textbf{商品}:它是“可以出卖的商品”,也就是还需要通过形态变化的第一种形式的商品。(假定一个工厂主买不到一部现成的机器,他可以自己制造一部机器,不是为了出卖,而是为了把它当作使用价值来利用。但是,在这种情况下,他把机器当作自己的不变资本的一部分来使用,因而他是通过由机器协助生产出来的产品的形式一部分一部分地把机器出卖的。)

[314]可见,虽然家仆的某些劳动完全可能表现为\textbf{商品(从可能性来讲)},从物质方面来看,甚至可能表现为同样的使用价值,但这不是生产劳动,因为实际上他们生产的不是“商品”,而是直接“\textbf{使用价值}”。而有些劳动,对它们的买者或雇主来说是生产的,例如演员的劳动对剧院老板来说是生产的,但这些劳动看起来象是非生产劳动,因为它们的买者不能以商品的形式,而只能以活动本身的形式把它们卖给观众。

如果把这一点撇开不谈,那末[按照斯密的第二个定义],生产劳动就是生产\textbf{商品}的劳动,\textbf{非生产劳动}就是生产个人服务的劳动。前一种劳动表现为某种可以出卖的物品;后一种劳动在它进行的时候就要被消费掉。前一种劳动(创造劳动能力本身的劳动除外)包括一切以物的形式存在的物质财富和精神财富,既包括肉,也包括书籍;后一种劳动包括一切满足个人某种想象的或实际的需要的劳动,甚至违背个人意志而强加给个人的劳动。

商品是资产阶级财富的最基本的元素形式。因此,把“生产劳动”解释为生产“商品”的劳动,比起把生产劳动解释为生产资本的劳动来,符合更基本得多的观点。

亚·斯密的反对者无视他的第一种解释即符合问题本质的解释,而抓住第二种解释,并强调这里不可避免的矛盾和不一贯的地方。而且他们把注意力集中在劳动的物质内容,特别是集中在劳动必须固定在一个比较\textbf{耐久的}产品中那个定义,用这个办法为自己的论战制造方便。我们马上就会看到,这场特别激烈的论战,究竟是由什么引起的。

还要先指出一点。亚·斯密认为,提出下面这个论点,是重农主义体系的巨大功绩:

\begin{quote}“各国的财富不在于不可消费的金和银,而在于每年由社会劳动再生产出来的可消费的货物。”([加尔涅的法译本]第 3 卷第 4 篇第 9 章第 538 页)\end{quote}

这里,我们看到了斯密关于生产劳动的第二个定义的来源。如何给剩余价值下定义,自然取决于所理解的价值本身具有什么形式。因此,剩余价值在货币主义和重商主义体系中,表现为货币;在重农学派那里,表现为土地的产品,农产品;最后,在亚·斯密那里,表现为一般\textbf{商品}。重农学派只要接触到价值实体,就把价值仅仅归结为使用价值(物质、实物),正如重商学派把价值仅仅归结为价值形式,归结为产品借以\textbf{表现}为一般社会劳动的那种形式即货币一样。在亚·斯密那里,商品的两个条件,使用价值和交换价值,合并在一起,所以在他看来,凡是表现在一种使用价值即有用产品中的劳动,都是生产的。表现在有用产品中的劳动就是生产劳动这一观点,就已经包含着这样的意思:这个产品同时等于一定量的一般社会劳动。亚·斯密同重农学派相反,重新提出产品的价值是构成资产阶级财富的实质的东西;但是另一方面,又使价值摆脱了纯粹幻想的形式——金银的形式,即在重商学派看来价值借以表现的形式。任何商品\textbf{从可能性来说}就是货币。不可否认,亚·斯密在这里同时又或多或少地回到重商学派关于这些或那些劳动产品的“耐久性”(实际上是“非直接消费性”)的观点上去。这里使人想起配第的一段话(见我的第 1 分册第 109 页\endnote{马克思指《政治经济学批判》第一分册。马克思提到的配第著作的引文,见《马克思恩格斯全集》中文版第 13 卷第 119 页。——第 167 页。},那里引用了配第《政治算术》中的一段话),在这段话里,财富是按照它不会毁坏的程度、耐久的程度来估价的,归根结蒂,金银被当作“长久的财富”而放在高于一切的地位。

\begin{quote}阿·布朗基说:“斯密把\textbf{财富}的范围仅仅限于固定在物质实体中的那些价值,这样就把无限多的非物质价值,文明国家的\textbf{精神资本}之女,全都从生产的账本中勾销了”,等等。(《欧洲政治经济学从古代到现代的历史》1839 年布鲁塞尔版第 152 页)\end{quote}

\tsectionnonum{[(5)资产阶级政治经济学在生产劳动问题上的庸俗化过程]}

反对亚·斯密提出的关于生产劳动和非生产劳动的区分的论战,主要是由二流人物(其中施托尔希还算是最出名的人物)进行的;我们在任何一个重要的经济学家那里,[315]在任何一个可以说在政治经济学上有所发现的人那里,都没有看到这种论战;然而这种论战对于第二流人物,特别是对于充满学究气的编书家和纲要编写者,以及对于在这方面舞文弄墨的业余爱好者和庸俗化者来说,却是一种嗜好。反对亚·斯密的这场论战,主要是由以下几种情况引起的。

有一大批所谓“高级”劳动者,如国家官吏、军人、艺术家、医生、牧师、法官、律师等等,他们的劳动有一部分不仅不是生产的,而且实质上是破坏性的,但他们善于依靠出卖自己的“非物质”商品或把这些商品强加于人,而占有很大部分的“物质”财富。对于这一批人来说,在\textbf{经济学上}被列入丑角、家仆一类,被说成靠真正的生产者(更确切地说,靠生产当事人)养活的食客、寄生者,决不是一件愉快的事。这对于那些向来显出灵光、备受膜拜的职务,恰恰是一种非同寻常的亵渎。政治经济学在其古典时期,就象资产阶级本身在其发家时期一样,曾以严格的批判态度对待国家机器等等。后来它理解到——这在它的实践中也表现出来——并且根据经验认识到,这种继承下来的所有这些在某种程度上完全非生产的阶级的社会结合的必要性,就是由资产阶级自己的组织中产生出来的。

如果上述“非生产劳动者”不生产享受,因此对他们的服务的需求不完全取决于生产当事人想如何花掉自己的工资或利润;相反,如果他们成为必要,或自己使自己成为必要,部分地是因为存在肉体上的疾病(如医生)或精神上的虚弱(如牧师),部分地是因为个人利益的冲突和民族利益的冲突(如政治家、一切法学家、警察、士兵);如果这样,那末,在亚·斯密看来,就象在产业资本家本身和工人阶级看来一样,他们就表现为生产上的非生产费用,因此必须尽可能地把这种非生产费用缩减到最低限度,尽可能地使它便宜。资产阶级社会把它曾经反对过的一切具有封建形式或专制形式的东西,以它自己所特有的形式再生产出来。因此,对这个社会阿谀奉承的人,尤其是对这个社会的上层阶级阿谀奉承的人,他们的首要业务就是,在理论上甚至为这些“非生产劳动者”中纯粹寄生的部分恢复地位,或者为其中不可缺少的部分的过分要求提供根据。事实上这就宣告了意识形态阶级等等是\textbf{依附于资本家}的。

但是,\textbf{第二},有一部分生产当事人(物质生产本身的当事人),时而被这一些经济学家,时而被那一些经济学家称为“非生产的”。例如,代表工业资本利益的那部分经济学家(李嘉图)把土地所有者称为“非生产的”。另一些经济学家(例如凯里)把本来意义的商人称为“非生产的”劳动者。后来甚至又有一些人把“资本家”本人也称为非生产的,或者至少企图把资本家对物质财富的要求归结为“工资”,即归结为一个“生产劳动者”所取得的报酬。脑力劳动者中间的许多人,看来都倾向于对资本家的生产性持这种怀疑观点。因此,已经是作出妥协并且承认不直接包括在物质生产当事人范围内的一切阶级都具有“生产性”的时候了。大家互相帮忙,并且,象《蜜蜂的寓言》\endnote{指英国作家孟德维尔的讽刺作品《蜜蜂的寓言,或个人劣行即公共利益》。该书于 1705 年出第一版,1728 年出第五版。——第 169 页。}中那样,必须证明,即使根据“生产的”、经济学的观点,资产阶级世界连同它的所有“非生产劳动者”一起,也是所有世界中最美好的世界;何况一些“非生产劳动者”从自己方面已经对那些根本是“为享受果实而生的”\authornote{贺雷西《书信集》。——编者注}阶级的生产性,或者对那些如土地所有者那样无所事事的生产当事人等等作出了批判的考察。\textbf{无所事事的人}也好,他们的\textbf{寄生者}也好,都必须在这个最美好的世界中找到自己的地位。

\textbf{第三},随着资本的统治的发展,随着那些和创造物质财富没有直接关系的生产领域实际上也日益依附于资本,——尤其是在实证科学(自然科学)被用来为物质生产服务的时候,——[316]政治经济学上的阿谀奉承的侍臣们便认为,对任何一个活动领域都必须加以推崇并给以辩护,说它是同物质财富的生产“联系着”的,说它是生产物质财富的手段;他们对每一个人都表示敬意,说他是“第一种”意义的“生产劳动者”,即为资本服务的、在这一或那一方面对资本家发财致富有用的劳动者,等等。

这里,应当首先提出的是马尔萨斯之流,他们直接为“\textbf{非生产}劳动者”和明显的寄生者辩护,说这些人是必要的和有用的。

\tsectionnonum{[(6)斯密关于生产劳动问题的见解的拥护者。有关这个问题的历史]}

\centerbox{\textbf{[(a)第一种解释的拥护者:李嘉图、西斯蒙第]}}

不值得花时间来详细考察热·加尔涅(斯密著作的译者)、罗德戴尔伯爵、布鲁姆、萨伊、施托尔希以及后来的西尼耳、罗西等人关于这一点的庸俗见解。只要引用一些典型的话就够了。

我们还要先举出\textbf{李嘉图}的一段话,他在其中证明,剩余价值(利润,地租)的所有者把剩余价值消费在“非生产劳动者”(例如家仆)身上,比他们把剩余价值花在“生产工人”所创造的奢侈品上,对于“生产工人”要有益得多。

\fontbox{~\{}\textbf{西斯蒙第}在《政治经济学新原理》(第 1 卷第 148 页)中,接受了斯密进行区分时的正确解释(这在李嘉图的著作中也是不言而喻的):生产阶级和非生产阶级的实际区别在于,

\begin{quote}“前者总是以自己的劳动同国民资本交换,后者总是以自己的劳动同一部分国民收入交换”。\end{quote}

\textbf{西斯蒙第}也是按照亚·斯密的见解来看剩余价值的:

\begin{quote}“虽然工人通过自己每天的劳动所生产的东西,远远超过他每天的支出,但是在他同土地所有者和资本家进行分配以后,除了维持生活最必需的东西以外,很少有剩余。”(\textbf{西斯蒙第}《政治经济学新原理》第 1 卷第 87 页)\fontbox{\}~}\end{quote}

李嘉图说:

\begin{quote}“如果土地所有者或资本家象古代贵族那样,把自己的收入用来供养很多的侍从或家仆,而不把它花费在华丽的衣服或昂贵的家具、马车、马或其他奢侈品上,那末他雇用的劳动人数就会多得多。在这两种情况下,纯收入是相同的,总收入也是相同的,但是纯收入实现在不同的商品上。如果我的收入是 1 万镑,那末,无论这 1 万镑是实现在华丽的衣服、昂贵的家具等等上,还是实现在同一价值的一定量食物和一般衣着上,所使用的生产劳动的数量差不多相等。但是,如果我把收入实现在前一类商品上,那\textbf{以后}就不会有对劳动的新的需求了:我将享用我的家具和衣服,事情就到这里为止。相反,如果我把收入实现在食物和一般衣着上,而且希望雇用家仆,那末,\textbf{除了原有对工人的需求之外,还会加上}对我用 1 万镑收入(或以这笔收入购买到的食物和一般衣着)所能雇用的所有那些人的需求。而需求的这种增加,只是因为我选择了第二种花费我的收入的方式。工人都关心\textbf{对劳动的需求},所以他们当然希望把用在购买奢侈品方面的收入尽量转用来维持家仆。”(\textbf{李嘉图}《原理》1821 年第 3 版第 475—476 页)\end{quote}

\centerbox{\textbf{[(b)区分生产劳动和非生产劳动的最初尝试(戴韦南特、配第)]}}

\textbf{戴韦南特}引用了一位老统计学家格雷哥里·金的一个图表,题为《1688 年英格兰不同家庭的收支表》。大学者金在表中把全体人民分成两个主要阶级:一个是“\textbf{增加}王国财富”的阶级,共计 2675520 人,一个是“\textbf{减少}王国财富”的阶级,共计 2825000 人;因此,前一个阶级是“生产的”,后一个阶级是“非生产的”。“\textbf{生产的}”阶级包括:勋爵、从男爵、骑士、乡绅、贵族、大小官吏、从事海上贸易的商人、法律家、教士、土地所有者、租地农场主、自由职业者、大小商人、手工业者、海陆军军官。相反,“\textbf{非生产的}”阶级包括:水手、农业工人和制造业短工、农民(在戴韦南特时代还占英格兰全部人口的 1/5)、[317]士兵、赤贫者、茨冈人、盗贼、乞丐和一般流浪者。戴韦南特这样来解释大学者金的这个等级表:

\begin{quote}“他的意思是说,前一个阶级的人靠土地、手艺和勤劳来养活自己,并且每年都给国民资本增加一些东西,此外,每年还从自己的剩余中分出一定数额来养活别人。在后一个阶级中,有一部分人靠自己的劳动养活自己,而其余的人和他们的妻子儿女,都要靠别人来养活;这是社会的负担,因为不然的话,他们每年消费的东西就可加到国民总资本中去。”(\textbf{戴韦南特}《论使一国人民在贸易差额中成为得利者的可能的方法》1699 年伦敦版第 23 和 50 页)\end{quote}

此外,戴韦南特的下面这段话,最能说明重商学派对剩余价值的看法的特点:

\begin{quote}“出口我们本国的产品,必定会使英国富裕;为了有贸易顺差,我们必须出口本国的产品,用它们去购买本国消费所必需的外国出产的物品,这里我们会有一个\textbf{余额},它或者采取贵金属的形式,或者采取我们可以用来卖给其他国家的商品的形式;\textbf{这个余额}就是\textbf{一国从贸易中取得的利润}。它的大小决定于出口国人民的自然节约〈荷兰人而不是英国人所特有的那种节约——同上,第 46 和 47 页〉,还决定于他们的劳动和制造业产品的低廉价格,这种低廉价格,使他们能\textbf{在国外市场上比所有的竞争者都便宜地出售这些产品}。”(\textbf{戴韦南特},同上第 45—46 页)\fontbox{~\{}“在国内消费产品时,一个人的赢利不过是另一个人的亏损,整个国家丝毫不会变富;但在国外消费的一切东西,却是明显的和可靠的利润。”(《论东印度贸易》1697 年伦敦版[第 31 页])\fontbox{\}~}\end{quote}

\fontbox{~\{}这本书是以戴韦南特的另一著作\endnote{指戴韦南特匿名出版的著作《论公共收入和英国贸易》1698 年伦敦版第二部分,其中载有戴韦南特一年前写的著作《论东印度贸易》。正文中引用的这段话的译文,同马克思在他的札记本中关于戴韦南特所说的话是一致的,马克思在正文中所用的戴韦南特著作的全部引文,都取自札记本(这一札记本的封面上马克思注明:“曼彻斯特。1845 年 7 月”)。——第 172 页。}(它是为了给这本书辩护而写的)的\textbf{附录形式刊印的},并不是麦克库洛赫引用过的那本《论东印度贸易》(1701 年版)。\fontbox{\}~}

可是,不应当象后来的庸俗自由贸易论者那样,把这些重商主义者说得那么愚蠢。戴韦南特在他的《论公共收入和英国贸易》第二卷(1698 年伦敦版)中曾说:

\begin{quote}“金和银实际上是贸易的尺度,但各国人民贸易的源泉和起源,却是一国自然的产物或人工的产物,即一国的土地或该国人民的劳动和勤勉所生产的东西。的确,一个民族由于某种情况可能完全丧失各种货币,但是只要它人口众多,热爱劳动,精于贸易,擅长航海,有良好的港湾,有生产各种产品的土地,它就仍然能够进行贸易,并且在短时间内拥有大量金银。所以,一国真正的实际的财富是它本国的产物。”(第 15 页)“金和银远不是能够称为一国的财宝或财富的唯一物品,因而货币实际上不过是人们在交易上习惯使用的计算筹码。”(第 16 页)“我们所说的财富,是指能使君主及其人民富裕、幸福、安全的东西;同样,财宝是指为了人们的需要用金银换来转化成建筑物和土壤改良的东西;还指\textbf{可以换成}这些金属的其他物品,如土地的果实和工业的产物,或外国的商品和商船……甚至那些不耐久的物品也能看成是国家的财富,只要它们\textbf{能够换成}金银——哪怕它们还\textbf{没有进行交换};并且我们认为,它们不仅在个人和个人之间的关系上是财富,而且在一国和别国之间的关系上也是财富。”(第 60—61 页)“平民\authornote{“平民”一词在这里是指革命前的法国称为“第三等级”的人,即同僧侣和贵族相对立的所有居民。——编者注}是国家身体中的胃。在西班牙,这个胃没有恰当地消受货币,[318]没有消化货币……工商业是能够保障消化和分配金银的唯一手段,而这将供给国家身体以必要的营养物。”(第 62—63 页)\end{quote}

其实,配第也已经有了\textbf{生产劳动者}的概念(不过他把士兵也包括在内):

\begin{quote}“土地耕种者、海员、士兵、手工业者和商人,是任何一个社会的真正的支柱。所有其他的大职业\textbf{都是由于这些人的孱弱和过失而产生的};海员身兼上述四者中的三者〈航海者、商人和士兵〉。”(《政治算术》1699 年伦敦版第 177 页)“海员的劳动和船只的运费,按其性质来说,始终是一种出口商品,出口超过进口的\textbf{余额}就给本国带回货币等等。”(同上,第 179 页)\end{quote}

在这一点上,配第又证明分工的好处:

\begin{quote}“在海上贸易中占支配地位的人们,即使在运费较低廉的情况下,也能比别人在较高〈运费较贵〉的情况下获得更多的利润;这是因为,就象做衣服一样,如果一个人完成一道工序,另一个人完成另一道工序,等等,衣服的价钱就比较便宜,在海上贸易中占支配地位的人们也是这样,他们可以建造各种不同用途的船只:海船、江船、商船、战船等等,这是荷兰人所以能够以低于他们邻国人的价格来运货的一个主要原因,因为他们能够为每个特定贸易部门提供特定种类的船只。”(同上,第 179—180 页)\end{quote}

此外,从配第的下面这些话里可以听到完全是斯密的调子:

\begin{quote}“如果向工业家等人收税,以便把货币供给那些按其职业来说一般\textbf{不}生产\textbf{物质品}即对\textbf{社会}有实际效用和价值的\textbf{物品}的人们,那末社会的财富就会减少。至于使精神得到消遣和恢复的活动,则又当别论,这些活动只要利用得当,就会使人能够并愿意去做本身具有更重要意义的事情。”(同上,第 198 页)“当计算好需要多少人从事生产劳动之后,剩下来的人就可以安全地、对社会无害地被用来从事娱乐和装饰方面的技艺和工作,\textbf{而其中最重大的,就是增进自然知识}。”(同上,第 199 页)“工业的收益比农业多,而商业的收益又比工业多。”(第 172 页)“一个海员相当于三个农民。”(第 178 页)\end{quote}

\centerbox{※     ※     ※}

[VIII—346]\textbf{配第。剩余价值}。从配第著作的一段话中,可以看到对\textbf{剩余价值}的性质的猜测,尽管他只是从地租的形式来考察剩余价值的。尤其是把这段话同下面几段话作一对比,就更清楚了。在下面几段话里,他用花费同样多劳动时间生产的银和谷物的相对量,来决定银和谷物的相对价值:

\begin{quote}“假定有人从秘鲁地下获得 1 盎斯银并带到伦敦来,他所用的时间和他生产 1 蒲式耳谷物所需要的时间相等,那末,前者就是后者的自然价格;假定现在由于开采更富的新矿,获得 2 盎斯银象以前获得 1 盎斯银花费一样多,那末在其他条件相同的情况下,现在 1 蒲式耳谷物值 10 先令的价格,就和它以前值 5 先令的价格一样便宜。”“假定让 100 个人在 10 年内生产谷物,又让同样数目的人在同一时间内开采银;我认为,银的纯产量将是谷物全部纯收获量的价格,前者的同样部分就是后者的同样部分的价格。”“100 个土地耕种者所能做的工作,如果由 200 个土地耕种者来做,谷物就会贵 1 倍。”(《赋税论》1662 年版)(1679 年版第 32、24、67 页)\end{quote}

我上面指的[关于剩余价值的性质的]一段话是这样说的:

\begin{quote}“如果商业和工艺发展了,那末,农业将要衰落,或者土地耕种者工资必将提高,\textbf{因而}地租就会下降……如果英格兰的商业和工业发展了,也就是说,如果从事工商业的人口比过去多了,而且现在谷物的价格不比以前有较多的人从事农业而较少的人从事工商业的时候高,那末仅仅由于这一个原因……地租就必定会下跌。例如,假定 1 蒲式耳小麦的价格为 5 先令或 60 便士;如果生长小麦的土地的地租为三分之一捆〈即收成的三分之一〉,那末在 60 便士中,就要有 20 便士归土地,40 便士归土地耕种者;但是,如果后者的工资提高 1/8,也就是从每天 8 便士提高到 9 便士,那末,在 1 蒲式耳小麦中,土地耕种者分得的份额就会由 40 便士增加到 45 便士,结果地租就要由 20 便士下降到 15 便士,因为我们假定\textbf{小麦价格仍然不变};更何况\textbf{我们不能够把小麦价格提高},因为如果我们试图把小麦价格提高,谷物就会从农业状况没有发生变化的外国输入我国[347](就象输入荷兰那样)。”(《政治算术》1699 年伦敦版第 193—194 页)[VIII—347]\end{quote}

\centerbox{※     ※     ※}

[VIII—364]\fontbox{~\{}\textbf{配第}。应当把上面引证的配第的一段话同下面这段话对比一下,在下面这段话里,地租表现为一般剩余价值,即表现为“纯产品”:

\begin{quote}“假定一个人用自己的双手在一块土地上种植谷物,耕地、播种、耙地、收割、搬运、脱粒,总之,干了农业上所需要的一切。我认为,这个人从他的收成中扣除自己的种子,并扣除自己食用的部分以及为换取衣服和其他必需品而给别人的部分之后,剩下的谷物就是当年真正的地租;而 7 年的\textbf{平均数},或者更确切地说,形成歉收和丰收循环周期的若干年的平均数,就是种植谷物的这块土地的通常的地租。但是,这里可能发生一个尽管是附带的、但需要进一步解决的问题:这种谷物或这种地租值多少货币呢\fontbox{?}我的回答是,值多少货币,要看另一个把\textbf{自己的全部时间}花在下述活动的人手中剩下多少货币:前往产银地区,在那里开采这种金属,把它提炼、铸成硬币,并把它运到第一个人播种和收获自己谷物的地方来。第二个人扣除他的全部费用之后手中剩下的货币量,将同土地耕种者手中剩下的那些谷物在价值上完全相等。”(《赋税论》\endnote{威廉·配第《赋税论》中的这段话,马克思在这里引自沙尔·加尼耳《论政治经济学的各种体系》一书第二卷第 36—37 页(1821 年巴黎版),这段话在这本书中已由加尼耳译成法文。这一段的法译文同马克思在手稿第 XXII 本中引用的英文原文有些不同(见本册第 381—382 页)。——第 176 页。}第 23 页)\fontbox{\}~}[VIII—364]\end{quote}

\centerbox{\textbf{[(c)斯密对生产劳动的第二种解释的拥护者——约翰·斯图亚特·穆勒]}}

[VII—318]\textbf{约翰·斯图亚特·穆勒}先生在《略论政治经济学的某些有待解决的问题》(1844 年伦敦版)一书中,也苦心研究生产劳动和非生产劳动的问题;但事实上他除了断言把劳动能力本身生产出来的那种劳动也是生产的以外,对斯密的(第二种)解释没有增添什么东西。

\begin{quote}“\textbf{享受的源泉}可以积累和积蓄,享受本身却不能这样。一国的财富由该国拥有的物质的或非物质的耐久的享受源泉的总和构成。用来增加或保存这些耐久的源泉的劳动或开支,都应称为\textbf{生产的}。”(同上,第 82 页)“机械师或纺纱者在学习手艺时所消费的东西,是用于生产消费,换句话说,他们消费的目的,不是减少而是增加国内耐久的享受源泉,因为他们新创造的享受源泉,在数量上超过消费掉的数额。”(同上,第 83 页)\end{quote}

\centerbox{※     ※     ※}

现在,我们简略地考察一下反对亚·斯密的那些关于生产劳动和非生产劳动的胡说八道。

\tsectionnonum{[(7)]热尔门·加尔涅[把斯密和重农学派的理论庸俗化]}

[319]在热尔门·加尔涅翻译的斯密《国富论》第五卷(1802 年巴黎版)中,载有译者的注释:

关于最高意义上的“生产劳动”问题,加尔涅是同意重农学派的观点的,只不过把这种观点略为缓和了。他反对斯密下面的观点:

\begin{quote}“生产劳动是这样的劳动,它物化在某种对象中,把自己活动的痕迹留下来,它的产品能够出卖或交换。”(同上,第 5 卷第 169 页)\endnote{手稿在加尔涅的这段引文之后,是篇幅很长的关于约翰·斯图亚特·穆勒的插入部分(手稿第 319—345 页)、一段不长的关于马尔萨斯的评论(第 345—346 页)和篇幅不大的关于配第的补充部分(第 346—347 页)。关于约·斯·穆勒的插入部分开头这样说:“在分析加尔涅的观点之前,我们要在这里附带地[即以补充部分的形式]就前面引证过的小穆勒说几句话。这里我们要说的话本来应放到后面论李嘉图剩余价值理论的地方去谈,而不在这里谈,这里我们还是考察亚当·斯密。”在手稿第 XIV 本目录中(见本册第 5 页)以及在这个稿本的正文中,论约·斯·穆勒一节是在《李嘉图学派的解体》一章内。根据所有这些理由,本版将关于约·斯·穆勒的补充部分移至《剩余价值理论》第三册《李嘉图学派的解体》一章。关于马尔萨斯的评论移至论马尔萨斯一章,关于配第的补充部分放在前面第 174—176 页。在所有这些插入部分之后,手稿上(第 VIII 本第 347 页)写道:“现在我们回过头来谈生产劳动和非生产劳动的问题。加尔涅。见手稿第 VII 本第 319 页。”接着便是对加尔涅观点的分析,现刊印在第 176—199 页上。——第 177 页。}[VII—319]\end{quote}

\centerbox{\textbf{[(a)把同资本交换的劳动和同收入交换的劳动混淆起来。关于全部资本由消费者的收入补偿的错误见解]}}

[VIII—347]加尔涅提出了反对亚·斯密的各种理由(其中一部分为后来的著作家们一再重复)。

\textbf{第一},

\begin{quote}“这种区分是错误的,因为它所根据的是不存在的差别。从作者所理解的\textbf{生产劳动}来看,\textbf{任何一种劳动都是生产劳动}。从劳动对支付其代价的人提供某种享受、某种方便或某种效用来看,一种劳动和另一种劳动一样,都是生产劳动。否则任何劳动都不会有报酬了。”\end{quote}

\fontbox{~\{}可见,劳动所以是生产的,是因为它生产某种使用价值,它可以出卖,它具有交换价值,也就是说,它本身就是商品。\fontbox{\}~}

但是,在发挥这个意思时,加尔涅举了许多例子来说明,在这些例子中,“非生产劳动者”和“生产劳动者”一样,做\textbf{同样的事情},生产同样的或同类的使用价值。例如:

\begin{quote}“服侍我的仆役,为我生火,为我卷发,为我洗衣服和整理家具,为我做饭等等。这个仆役和下面这些人提供的是\textbf{完全同类的服务}:如洗衣女工或缝纫女工为顾客洗濯和修补衬衣……如小饭馆主人或小酒馆主人为愿意光临的顾客烹调食物;如理发师、美容师等等\end{quote}

(但在亚·斯密看来,这种人大部分象家仆一样,都不属于生产劳动者的范畴)

\begin{quote}提供直接的服务;最后,如泥瓦匠、屋顶匠、木匠、玻璃匠、火炉匠等等,以及被人请去修缮房屋的大量建筑工人,后者从简单的修缮劳动中获得的年收入,同从新建房屋的劳动中获得的一样多。”\end{quote}

(亚·斯密在任何地方都没有说,修理劳动不能象生产新物品的劳动那样固定在比较耐久的物品上。)

\begin{quote}“这种劳动与其说是生产物品,不如说是保存物品;它的目的与其说是增加它所加工的物品的价值,不如说是防止这些物品的损坏。所有这些劳动者,包括家仆在内,\textbf{都能使付给他们报酬的人节约维护自己财物的劳动}。”\end{quote}

(因此,可以把他们看成是保存价值的机器,或者确切些说,保存使用价值的机器。这种“节约”劳动的观点被\textbf{德斯杜特·德·特拉西}进一步发挥了。这一点以后再谈。一个人的非生产劳动,决不能由于使另一个人省去\textbf{非生产劳动}而变成生产劳动。这种非生产劳动总得由其中的一个人来完成。斯密所说的非生产劳动,有一部分由于分工而成为必要,这只是指消费物品时绝对必要的并且可以说是属于\textbf{消费费用}的那一部分,而且它只有在使生产劳动者节约这部分时间的时候,才成为必要的。不过,亚·斯密并不否认这种“分工”。如果每个人本来不得不既完成生产劳动,又完成非生产劳动,而由于两个人之间实行这种分工,生产劳动和非生产劳动都能完成得更好,那末,按照斯密的说法,这丝毫也不会改变一种劳动是生产劳动,而另一种劳动是非生产劳动这个事实。)

\begin{quote}“在大多数情况下,他们都是由于这一点,并且仅仅是由于这一点而被雇用的\end{quote}

(为使一个人节约自己服侍自己的劳动,必须有 10 个人来服侍他,这真是一种奇特的“节约”劳动的方法;而且,这种“非生产劳动”大部分又恰恰是由那些无所事事的人来使用的);

\begin{quote}因此,或者他们都是\textbf{生产的},或者他们都不是生产的。”(同上,第 171—172 页)\end{quote}

[348]\textbf{第二},法国人不会忘掉“桥梁和公路”\authornote{在法国,这是指交通主管部门。——编者注}。他说:

\begin{quote}为什么“一个私人工商业企业的监督人或经理的劳动”,应当称为生产的,“而一个负责维持公路、运河、港口、货币制度和其他活跃商业的重要机构的秩序,保障交通运输的安全,监督契约的执行等等,并完全有权被认为是\textbf{大社会工厂监督人}的政府官吏的劳动,就应当称为\textbf{非生产的}呢\fontbox{?}这完全是同类的劳动,只不过规模更大罢了”。(第 172—173 页)\end{quote}

只要这个小伙子参加物质品的生产(或保存和再生产),并且这些物质品不是掌握在国家手里而是\textbf{可以出卖的},斯密就会把他的劳动称为“生产的”。“大社会工厂监督人”——这纯粹是法国的创造。

\textbf{第三},在这里,加尔涅热衷于“道德”。为什么“诱惑我的嗅觉的香水制造者”应当认为是生产劳动者,而“陶醉我的听觉”的音乐家应当是非生产劳动者呢\fontbox{?}(第 173 页)斯密会回答说,因为一个提供物质产品,另一个不提供物质产品。道德和这两个人的“功绩”一样,同这里的区分毫无关系。

\textbf{第四},认为“提琴制造者、风琴制造者、乐器商人、布景师等等”是生产的,而以他们的劳动为“准备阶段”的那些职业则是非生产的,难道这不是矛盾吗\fontbox{?}

\begin{quote}“这两种人\textbf{劳动}的最终目的是提供\textbf{同一种消费}。如果一种人劳动的最终结果不应当算作社会劳动的\textbf{产品},那末,为什么偏要对不过是\textbf{达到这种结果的手段}另眼看待呢\fontbox{?}”(同上,第 173 页)\end{quote}

如果这样来谈问题,那就会得出结论说:吃粮食的人和生产粮食的人一样,也是生产的。因为,为什么生产粮食呢\fontbox{?}就是为了吃。因此,如果吃粮食是非生产劳动,那末,为什么种粮食这种不过是达到这个目的的手段,却是生产的呢\fontbox{?}而且,吃粮食的人会生产脑子、肌肉等等,难道这不是象大麦或小麦一样贵重的产品吗\fontbox{?}——某位被激怒的人类之友,也许会这样质问亚·斯密。

第一,亚·斯密并不否认非生产劳动者会生产某种产品。否则,他根本就不是劳动者了。第二,开药方的医生不是生产劳动者,而配药的药剂师却是生产劳动者,这看起来好象是奇怪的。同样,制造提琴的乐器制造者是生产劳动者,而演奏提琴的提琴师却不是。这只能证明,某些“生产劳动者”提供的产品,其唯一的目的是充当非生产劳动者的生产资料。但这并不比这样的事实更奇怪:归根到底,一切生产劳动者,第一,提供支付非生产劳动者的资金,第二,提供产品,让\textbf{不从事任何劳动}的人消费。

在这些批评意见中,第二点完全符合怎么也忘不了“桥梁和公路”的法国人的精神;第三点归结为道德;第四点,或者是包含一种胡说,即认为消费和生产一样是生产的(这对于资产阶级社会来说是错误的,因为在这个社会里,一种人生产而另一种人消费),或者是说明,生产劳动的一部分只为非生产劳动提供材料,而这一点,亚·斯密从来没有否认过。只有第一点包含着正确的意思,即亚·斯密在他的第二个定义中,把\textbf{同一种}劳动既称为生产劳动又称为非生产劳动,[349]或者确切些说,按照他自己的定义,他本来应该把他的“非生产”劳动中的某一个较小的部分称为\textbf{生产的}。——可见,这并不是反对\textbf{区分}本身,而是反对这种区分\textbf{包括的范围},或者说,反对这种区分\textbf{适用的范围}。

提了所有这些批评意见之后,大学者加尔涅终于谈到本题:

\begin{quote}“看来在斯密所想象出来的两个阶级之间,能够找到的唯一的总的区别就是:就他所谓的\textbf{生产}阶级来说,\textbf{物品制造者和物品消费者之间}有或者\textbf{总会}有一个\textbf{中介人存在};而就他所谓的\textbf{非生产}阶级来说,\textbf{不会有任何中介人存在},这里\textbf{劳动者和消费者}之间的关系\textbf{必然是直接的、没有中介的}。很明显,那些享受医生的经验、外科医师的手术、律师的知识、音乐家或演员的天才以及家仆的服务的人,在所有这些不同的劳动者从事这种劳动时,\textbf{必然}同他们发生一种直接的没有中介的关系;相反,在另一个阶级的职业中,\textbf{供消费的物品是物质的、可以感觉的},因此,在它们从制造者手里转到消费者手里之前,\textbf{就能够成为一系列中间性交换行为的对象}。”(第 174 页)\end{quote}

加尔涅无意中用后面几句话表明,在斯密的第一种区分(同资本交换的劳动和同收入交换的劳动)和第二种区分(固定在物质的可以出卖的商品上的劳动和不固定在这种商品上的劳动)之间,存在着多么隐蔽的思想联系。不固定在商品上的种种劳动,按其性质来说,大多数\textbf{不能}从属于资本主义生产方式;其他各种劳动,则可能从属于资本主义生产方式。更不用说,\textbf{在资本主义生产的基础上},大部分物质商品,即“物质的、可以感觉的物品”,是在资本的支配下由雇佣工人生产的,那些[非生产]劳动(或服务,无论是妓女的服务,还是罗马教皇的服务),只能由生产工人的工资或他们的雇主(和分享利润的人)的利润来支付;也不必谈这样一个事实,即这些生产工人创造着养活非生产劳动者,因而使他们得以生存的物质基础。但这条饶舌的法国狗有一个特点,他自认为是政治经济学家,即资本主义生产的研究者,却把那种使生产成为资本主义生产的东西(即资本同雇佣劳动相交换,而不是收入同雇佣劳动直接交换,或劳动者自己直接把收入支付给自己)看成是\textbf{非本质的东西}。因此,在加尔涅看来,资本主义生产本身是一种非本质的形式,而不是一种发展社会劳动生产力,并使劳动变为社会劳动的必然形式,尽管只是历史的也就是暂时的必然形式。

\begin{quote}“此外,还应当从他的\textbf{生产}阶级中除掉所有这样的工人,他们的劳动只是洗刷、保存或修理成品,而不是使任何新产品进入流通。”(第 175 页)\end{quote}

(斯密从来没有认为,劳动或劳动产品必须加入流动资本。劳动能够直接加入固定资本,例如在工厂中修理机器的机械师的劳动就是如此。但在这种情况下,这种劳动的\textbf{价值}会加入产品即商品的流通。如果从事修理等等的劳动者是在主顾家里干活,那他们[350]的劳动就不是同资本交换,而是同收入交换。)

\begin{quote}“正是由于这种区别,\textbf{非生产}阶级,象斯密所指出的,只是靠收入而生存。事实上,因为对于这个阶级来说,在他们和他们产品的消费者即他们劳动的享受者之间,不可能有中介人存在,所以这个阶级直接由消费者支付;\textbf{而这种消费者只能用自己的收入来支付}。相反,\textbf{生产}阶级的劳动者通常由\textbf{中介人}支付,\textbf{中介人的目的是从他们的劳动中吸取利润};因此,\textbf{他们大多由资本支付}。但是,这个资本归根到底总要由消费者的收入来补偿;否则它就不能流通,因而也就不能给它的所有者带来利润。”[第 175 页]\end{quote}

最后这个“但是”十分幼稚。第一,资本的一部分就由资本补偿,而不是由收入补偿——不管资本的这一部分进入流通还是不进入流通(例如种子的补偿就是后一种情况)。

\centerbox{\textbf{[(b)在资本同资本交换的过程中不变资本的补偿问题]}}

如果煤矿向制铁厂供应煤炭,并从制铁厂得到铁,铁作为生产资料加入煤矿的生产过程,那末煤炭就按照铁的价值额同资本交换,反过来说,铁也按照自己的价值额作为资本同煤炭交换。煤炭和铁(作为使用价值)都是新劳动的产品,虽然这种新劳动是用已有的生产资料来进行的。但是,年劳动产品的价值,并不就是这一年新加劳动的产品。它还要补偿已经物化在生产资料中的过去劳动的价值。因而,总产品中和过去劳动的价值相等的那一部分,并不是当年劳动产品的一部分,而是过去劳动的再生产。

我们举煤矿、制铁厂、木材厂和机器制造厂的日劳动产品为例。假设这些企业的不变资本都等于产品价值所有组成部分的 1/3,即过去劳动和活劳动之比等于 1∶2。假定这些企业的日产品是 X、X′、X″、X′″。这些产品是一定量的煤炭、铁、木材和机器。作为这样的产品,它们都是日劳动的产品(但同样是协助当日生产而在一日内消费的原料、燃料、机器设备等等的产品)。假定它们的价值是 Z、Z′、Z″、Z′″。这些价值并不是当日劳动的产品,因为 Z/3、Z′/3、Z″/3、Z′″/3 只不过等于 Z、Z′、Z″、Z′″的不变要素在加入当日劳动过程以前具有的价值。所以 X/3、X′/3、X″/3、X′″/3,即生产出来的使用价值的 1/3,也只是代表过去劳动的价值,并且不断补偿这个价值。\fontbox{~\{}这里发生的过去劳动和活劳动产品之间的交换,按其性质来说,完全不同于劳动能力和作为资本存在的劳动条件之间的交换。\fontbox{\}~}

X=Z;但是这个 Z 是整个 X 的价值,\endnote{在这之前马克思一直用字母 x 代表作为使用价值来考察的产品,用字母 z 代表产品价值。从这里起马克思改换了字母符号:用 x 代表价值,用 z 代表使用价值。本版各处都采取马克思最初使用的字母符号。——第 183 页。}而(1/3)Z 是整个 X 中包含的原料等等的价值。因此,X/3 是劳动的日产品的一部分\fontbox{~\{}但决不是日劳动的产品,而是和日劳动相结合的往日劳动的产品,总之,是过去劳动的产品\fontbox{\}~},和日劳动相结合的过去劳动,就在这一部分中得到再现和补偿。不过,只表示实在产品(铁、煤炭等等)的量的这个 X,它的任何一部分按其价值来说都是 1/3 代表过去劳动,2/3 代表当日完成和加进的劳动。过去劳动和当日劳动以什么比例加入产品总额,也就以什么比例加入作为总额组成部分的每一个产品。但如果我把全部产品分成两部分,一边是 1/3,另一边是 2/3,那就等于说,前 1/3 只代表过去劳动,其余的 2/3 只代表当日劳动。事实上,前 1/3 代表加入总产品的全部过去劳动,即代表消费掉的生产资料的全部价值。因而,扣除这 1/3 之后,剩下的 2/3 就可以只代表日劳动的产品。它们事实上也代表一日内加到生产资料上的全部劳动量。

因此,后 2/3 等于生产者的收入(工资和利润)。生产者可以消费它们,即把它们花在个人消费品上。假设一日内采掘的这 2/3 煤炭由消费者或者说由买者购买,不是用货币来购买,而是用商品来交换,他们为了购买煤炭,预先把这些商限品转化为货币。在这 2/3 中,有一部分煤炭加入煤炭生产者本人的个人消费,用于家庭取暖等等,因而这一部分不进入流通,即使它已经进入流通,[351]也会被它的生产者从流通中取回来。从 2/3 煤炭中扣除煤炭生产者本人消费的这部分之后,其余所有的量(如果生产者想把它消费掉)必须拿去同个人消费品交换。

在进行这种交换时,煤炭生产者根本不问消费品的卖者究竟用什么来同煤炭交换,用他们的资本还是用他们的收入;换句话说,根本不问:例如,是毛织厂主为了自己的住宅取暖而用自己的呢绒来同煤炭交换(在这种情况下,煤炭对于他也是消费品,他用他的收入,用代表利润的一定量的呢绒,来支付煤炭),还是毛织厂主的仆役詹姆斯,用他作为工资得到的呢绒来同煤炭交换(在这种情况下,煤炭也是消费品,并且是同毛织厂主的收入交换,不过毛织厂主已经把自己的收入同仆役的非生产劳动交换过了),还是毛织厂主为了补偿它的工厂所必需的、已经消费掉的煤炭,而用呢绒来同煤炭交换。(在最后这种情况下,毛织厂主用来交换的呢绒,对他来说代表不变资本,代表他的一种生产资料的价值;煤炭对他来说不只是价值,而且是一定的、实物形式的生产资料。而呢绒对于煤炭业者来说是消费品,呢绒和煤炭对他来说都代表收入:煤炭是他的未实现形式的收入,呢绒是他的已实现形式的收入。)

至于最后的 1/3 煤炭,煤炭业者不能把它花在他的个人消费品上,不能把它当作收入来花费。它属于生产过程(或再生产过程),它必须转化为铁、木材、机器,转化为构成他的不变资本各组成部分的那些物品,没有这些物品,煤炭的生产就不能更新或继续下去。当然,他也可以用这 1/3 同消费品交换(或者同样可以说,同这些消费品的生产者的货币交换),但这只是为了再用这些消费品去换回铁、木材、机器,这样它们既不加入他本人的消费,也不加入他的收入的支出,而是加入木材、铁、机器生产者的收入的消费和支出,而所有这些木材、铁等等的生产者自己又处于这样的情况:他们不能把他们产品的 1/3 花在个人消费品上。

现在假设,煤炭加入铁生产者、木材生产者、机器制造业者的不变资本。另一方面,铁、木材、机器加入煤炭业者的不变资本。这样一来,既然他们的这些产品以相同的价值额彼此加入[他们的不变资本],那它们就是以实物形式互相补偿,交易的一方只须将买进的东西超过卖出的东西的差额支付给对方就行了。实际上,货币在这里的实践中(通过期票等等)也只是作为\textbf{支付手段}出现,而不是作为铸币,不是作为流通手段出现,它们只是用来支付差额。煤炭生产者需要用这 1/3 煤炭中的某一部分来进行他自己的再生产,正象他在 2/3 中留下某一部分来供他自己消费一样。

通过不变资本同不变资本的交换,即通过一种实物形式的不变资本同另一种实物形式的不变资本的交换来互相补偿的这全部数量的煤炭、铁、木材和机器,既与收入同不变资本的交换毫无关系,也与收入同收入的交换毫无关系。这一部分产品所起的作用,完全象农业中的种子或畜牧业中的种畜所起的作用一样。这是\textbf{劳动的年产品}(不是\textbf{当年新加劳动}的产品,而是新加劳动和过去劳动的产品)的一部分,是(在生产条件不变的情况下)每年作为生产资料,作为不变资本来自我补偿的那一部分。它除了加入一些“实业家”和另一些“实业家”之间的流通以外,不加入任何别的流通,也不影响加入“实业家”和“消费者”之间流通的那部分产品的\textbf{价值}。\authornote{见本册第 111、130—131 页。——编者注}

假定这全部 1/3 煤炭按上述方式以实物形式同自己的生产要素即铁、木材、机器相交换。\fontbox{~\{}也可能是这样:例如它直接只同机器交换,但机器制造业者又把它作为不变资本,不仅同自己的不变资本交换,而且同铁生产者和木材业者的不变资本交换。\fontbox{\}~}在这种情况下,煤炭业者[352]用来同消费品交换的,即作为收入去交换的那 2/3 产品中的每担煤炭,固然也象全部产品一样,按其价值来说由两部分组成:1/3 担等于生产 1 担煤炭时消费的生产资料的价值,2/3 担等于煤炭生产者新加到这 1/3 上的劳动。但是,如果煤炭业者的全部产品,比如说,是 3 万担,那末他只把 2 万担作为收入去交换,其余 1 万担,根据假定,由铁、木材、机器等等补偿;一句话,3 万担中包含的生产资料的全部价值,由同样种类和同等价值的生产资料以实物形式补偿。

这样,2 万担的买者事实上对于 2 万担中包含的过去劳动的价值没有支付一文钱;因为 2 万担只代表总产品价值中体现新加劳动的那 2/3。同样可以说,这 2 万担只代表(例如一年的)新加劳动,完全不代表过去劳动。这样,买者虽然支付每担的全部价值,即过去劳动加新加劳动,但他又是只支付新加劳动;这正是因为他只购买 2 万担,即只购买全部产品中等于全部新加劳动价值的那一部分。这就好比买者除了支付他吃的小麦以外,不支付土地耕种者的种子一样。铁、木材、机器等等的生产者互相补偿了这一部分产品,所以不必再由买者补偿它。生产者用自己产品的一部分补偿了它,这一部分固然是他们劳动的年产品,但决不是他们当年新加劳动的产品,而是他们的年产品中代表过去劳动的那一部分。没有新劳动就不会有产品;但是没有物化在生产资料中的劳动也不会有产品。如果它只是新劳动的产品,那末它的价值就会比现在小,而产品的任何一部分就不需要归还给生产了。但是,如果另一种劳动方式[即以使用生产资料为基础的劳动方式]没有更大的生产能力,不会提供更多的产品,——虽然一部分产品必须归还给生产,——那就没有人会采用它了。

虽然这 1/3 煤炭中没有一个价值组成部分会加入当作收入来出卖的 2 万担煤炭,但是这 1/3 即 1 万担所代表的不变资本的任何价值变动,都会引起当作收入来出卖的其余 2/3 的价值变动。假定铁、木材、机器等等的生产,一句话,上述 1/3 产品所分解成的那些生产要素的生产变贵了。而开采煤炭的劳动生产率仍然不变。花费同量的铁、木材、煤炭、机器和劳动,仍然生产出 3 万担煤炭。但是因为铁、木材和机器贵了,要比以前花更多的劳动时间,所以为换取它们就必须付出比以前更多的煤炭。

[353]假定产品仍然是 3 万担煤炭。煤矿的劳动生产率保持不变。用同量的活劳动和同量的木材、铁、机器等等,仍然生产出 3 万担煤炭。活劳动仍然表现为同一个价值,例如 2 万镑(用货币表现)。相反,木材、铁等等,一句话,不变资本,现在值 16000 镑,而不是 1 万镑,就是说,它们包含的劳动时间增加了 6/10 或 60\%。

这样,全部产品的价值现在等于 36000 镑,而不是以前的 3 万镑了;因而价值增加了 1/5 或 20\%。因此,产品的每一个部分现在也比以前多值 1/5 或 20\%。以前每 1 担煤炭值 1 镑,现在每 1 担值 1 镑+1/5 镑=1 镑 4 先令。以前总产品中 1/3 或 3/9 等于不变资本,2/3 等于新加劳动。现在不变资本和总产品价值之比等于 16000∶36000=4/9。因此,现在不变资本比以前多占了[总产品价值的]1/9。等于新加劳动价值的那部分产品,以前占产品的 2/3 或 6/9,现在占 5/9。

\textbf{这样,我们就有}:

采煤工人的劳动生产能力并没有降低,但是花在采煤上的总劳动(采煤工人的劳动加过去劳动)的生产能力却降低了;就是说,为了补偿[354]不变资本所占的价值组成部分,现在需要比以前多 1/9 的总产品,而新加劳动的价值在产品中则少占了 1/9。铁、木材等等的生产者现在也象以前一样只支付 1 万担煤炭。这个数量的煤炭,以前要花他们 1 万镑,现在要花他们 12000 镑。这样,由于他们必须按照提高了的价格支付他们用铁等等来交换的那部分煤炭,不变资本费用的一部分也就会得到补偿。但是,煤炭生产者必须向他们购买 16000 镑的原料等等。因而煤炭生产者必须支付 4000 镑的差额,即 3333+(1/3)担煤炭。这样,他仍然向消费者提供 16666+(2/3)+3333+(1/3)=20000 担煤炭,即 2/3 产品;但消费者现在为这 2 万担必须支付 24000 镑,而不是 2 万镑。消费者用这个数额为煤炭生产者不仅补偿新加劳动,而且还补偿不变资本的一部分。

对于消费者来说,问题很简单。如果他们要想消费以前那样多的煤炭量,他们就必须多支付 1/5,因此,在每一个生产部门的生产费用都照旧不变的前提下,他们就必须在自己的收入中少用 1/5 购买别的产品。困难只是在于:如果铁、木材等等的生产者不再需要煤炭,那末煤炭生产者怎样才能支付这 4000 镑的铁、木材等等呢\fontbox{?}他把他的等于这 4000 镑的 3333+(1/3)担煤炭卖给了煤炭消费者,并由此取得各种各样的商品。但是这些商品既不能加入他个人的消费,也不能加入他的工人的消费,而必须由铁、木材等等的生产者消费,因为他必须以这些物品的形式来补偿他的 3333+(1/3)担煤炭的价值。人们会说:问题很简单。现在所有的煤炭消费者必须少消费 1/5 的所有其他商品,或者说,必须从每个人自己的商品中多拿出 1/5 来支付煤炭。正是这 1/5 用来增加木材、铁等等生产者的消费。但是,制铁厂、机器制造业、木材业等等生产率的减低,究竟怎样使铁、机器、木材的生产者能够消费比以前更大的收入,这一点乍看起来是不明白的;因为我们假设,他们产品的价格等于产品的价值,因而产品价格的提高,只同他们劳动生产率的减低成比例。

我们曾假设,铁、木材、机器的价值提高了 3/5 即 60\%。这只能由两个原因引起。或者,铁、木材等等的生产部门的生产能力减低是由于这些生产部门中使用的活劳动的生产能力减低,以致生产同一产品必须使用更多的劳动。在这种情况下,铁、木材、机器的生产者必须比以前多使用 3/5 劳动。因为劳动生产率的降低只是暂时涉及个别的产品,所以工资率保持不变。因而,剩余价值率也保持不变。生产者以前需要 15 个工作日的地方,现在需要 24 个工作日,但是在这 24 个工作日中,他照旧每天只支付工人 10 劳动小时,照旧迫使工人每天无偿地劳动 2 小时。这样,以前 15 个工人为自己劳动 150 小时,为企业主劳动 30 小时,而现在 24 个工人则为自己劳动 240 小时,为企业主劳动 48 小时。(在这里我们不问利润率如何。)只有当工资花在铁、木材、机器设备等等上面的时候,工资才会降低,而这种情况实际上是没有的。现在 24 个工人会比以前 15 个工人多消费 3/5。因而煤炭生产者现在能够把 3333+(1/3)担的相应加大的那部分价值销售给这些工人(也就是说,销售给向他们支付工资的老板)。

或者,制铁业、木材业等等生产率的减低是由于它们的一部分不变资本即生产资料变贵了。这时[其他生产部门]又会面临同样的抉择,而归根到底,生产率的减低必定会造成使用的活劳动量的增加,因而也会造成工资的增加,煤炭消费者会以上述 4000 镑的形式,把这些工资部分地支付给煤炭业者。

在那些使用追加劳动量的生产部门中,由于雇用的工人人数增加,剩余价值量也会增加。另一方面,随着这些部门本身的产品加入这些部门不变资本的各个要素(不管这些部门自己是把本身的一部分产品当作生产资料使用,还是象煤炭的情形那样,把自己的产品传为生产资料加入这些部门本身的生产资料中)[的价值的增加],利润率会相应地降低。但是,如果它们花在工资上的流动资本比需要补偿的那部分不变资本增加得更多,那末它们的利润率也会提高,它们[355]也会参加上述 4000 镑某些部分的消费。

不变资本价值的提高(由供应这种不变资本的劳动部门生产率的减低引起),会使包含这种不变资本的产品的价值提高,会使补偿新加劳动的那部分产品(在实物形式上)减少,因而就会使这种劳动的生产能力降低,如果这种劳动是用它本身的产品来表示的话。对于以实物形式自行交换的那部分不变资本来说,一切照旧不变。照旧是同量的铁、木材、煤炭以实物形式自行交换,以补偿用掉的铁、木材、煤炭;价格的上涨在这里会互相抵销。但是,现在形成煤炭业者的一部分不变资本并且不加入这种实物交换的那一煤炭余额,照旧要同收入交换(在上面谈到的情况下,它的一部分不仅同工资交换,而且同利润交换),区别只是在于,这种收入已经不属于以前的消费者,而属于那些在使用更多劳动量,也就是工人人数增加了的生产领域中工作的生产者。

如果某个生产部门生产的产品只加入个人消费,既不作为生产资料加入任何别的生产部门(在这里生产资料始终是指不变资本),也不加入自己的再生产(例如在农业、畜牧业、煤炭业中就有这种情形,在煤炭业中,煤炭本身作为辅助材料加入生产),那末这个部门的年产品\fontbox{~\{}超过年产品的可能的余额对这里的问题没有意义\fontbox{\}~}就始终要由收入支付,即由工资或利润支付。

我们拿前面举过的麻布的例子\authornote{见本册第 92 页及以下各页。——编者注}来看。在 3 码麻布中,2/3 是不变资本,1/3 是新加劳动。因而 1 码麻布代表新加劳动。如果剩余价值等于 25\%,那末 1 码的 1/5 就代表利润。其余的 4/5 码代表再生产出来的工资。1/5 由工厂主本人消费,或者由其他人消费,——这些人把它的价值支付给工厂主,工厂主又以这些人的商品或其他商品的形式来消费这个价值,——结果是一样的。\fontbox{~\{}为简单起见,这里不正确地把全部利润看作收入。\fontbox{\}~}其余的 4/5 码由工厂主再以工资的形式支付出去;他的工人把这些当作自己的收入来消费——或者直接消费,或者交换其他消费品,而这些消费品的所有者则消费麻布。

这两部分加在一起,就是 3 码麻布中生产者自己能当作收入来消费的全部份额——1 码麻布。其余的 2 码代表工厂主的不变资本;它们必须再转化为麻布的生产条件——纱、机器等等。从工厂主的角度来看,2 码麻布的交换是不变资本的交换,但是他可以把这些麻布只同别人的收入交换。例如,他用 2 码的 4/5,即 8/5 码,支付纱,用 2/5 码支付机器。纺纱业者和机器制造业者又可以在这些麻布量中各自消费 1/3,就是说,一个人可以在 8/5 码中消费 8/15 码,另一个人可以在 2/5 码中消费 2/15 码。共计 10/15 或 2/3 码。其余的 20/15 或 4/3 码必须补偿他们的原料——亚麻、铁、煤炭等等,而这些物品中的每一种,又都分解为代表收入(新加劳动)的部分和代表不变资本(原料和固定资本等等)的部分。

但是,这最后的 4/3 码麻布可以只当作收入来消费。因而,那种最终以纱和机器的形式表现为不变资本并由纺纱业者和机器制造业者用来补偿亚麻、铁、煤炭等等的东西(我们把机器制造业者用机器补偿的那部分铁、煤炭等等撇开不谈),可以只代表形成亚麻、铁、煤炭生产者的收入,因而不需补偿不变资本的那部分亚麻、铁、煤炭;换句话说,以纱和机器的形式表现为不变资本的东西,必须属于亚麻、铁、煤炭等等生产者的产品中如上面所说的不包含任何不变资本份额的那一部分。但是,铁、煤炭等等的生产者会把他们的表现为铁、煤炭、亚麻等等的收入,以麻布的形式或其他消费品的形式来消费,因为他们自己的产品本身完全不加入或只有极小部分加入他们的个人消费。这样,铁、亚麻等等的一部分就可以同只加入个人消费的产品——麻布——交换,而由于同这种产品交换,对纺纱业者来说,全部补偿了他们的不变资本,对机器制造业者来说,部分地补偿了他们的不变资本,同时,纺纱业者和机器制造业者又是拿出代表他们收入的那部分纱和机器换取麻布来消费的,从而就补偿了织布业者的不变资本。

这样,实际上全部麻布都归结为织布业者、纺纱业者、机器制造业者、亚麻种植业者、煤炭生产者和铁生产者的利润和工资;同时他们又给麻织厂主和纺纱业者补偿全部不变资本。如果后面这些原料生产者必须通过同麻布交换来补偿自己的不变资本,那末计算就完结不了,因为麻布是个人消费品,不能作为生产资料、作为[356]不变资本的一部分,加入任何生产领域。计算所以会完结,是因为亚麻种植业者、煤炭业者、制铁业者、机器制造业者等等用他们的产品购买的麻布,只补偿他们的产品中对于他们是收入而对于他们的买者是不变资本的那一部分。这种情况所以可能,只是因为他们的产品中不归结为收入、因而不能同消费品交换的那一部分,由他们以实物形式补偿,也就是通过不变资本同不变资本相交换来补偿。

在前面举的例子中,曾假设一个生产部门的劳动生产率保持不变,但用这个部门本身的产品来表示该部门所使用的活劳动的生产率时,劳动生产率却降低了;这个假设可能使人感到奇怪。但是事情解释起来却很简单。

假设纺纱业者的劳动产品是 5 磅棉纱。假定纺纱业者为了生产这些产品只需要 5 磅棉花(就是说没有一点飞花);假定 1 磅棉纱值 1 先令(我们不谈机器设备,也就是说,假设它的价值没有下降,也没有上涨,因而它对于我们考察的情况来说等于零)。1 磅棉花值 8 便士。在表示 5 磅棉纱价值的 5 先令当中,棉花占 40 便士(5×8 便士)或 3 先令 4 便士,新加劳动占 5×4 便士,即 20 便士或 1 先令 8 便士。因而在全部产品中,3+(1/3)磅棉纱(价值为 3 先令 4 便士)是不变资本所占的部分,1+(2/3)磅棉纱是劳动所占的部分。所以,5 磅棉纱的 2/3 补偿不变资本,5 磅棉纱的 1/3,即 1+(2/3)磅,是支付劳动的那部分产品。

现在假定 1 磅棉花的价格上涨了 50\%,从 8 便士上涨到 12 便士,即上涨到 1 先令。那末 5 磅棉纱就值:5 磅棉花所值的 5 先令和新加劳动所值的 1 先令 8 便士(新加劳动的量,因而用货币表现的价值,保持不变)。这样,5 磅棉纱现在值 5 先令+1 先令 8 便士=6 先令 8 便士。在这 6 先令 8 便士中,现在原料占 5 先令,劳动占 1 先令 8 便士。

6 先令 8 便士=80 便士,其中 60 便士为原料,20 便士为劳动。在 5 磅棉纱的总价值(80 便士)中劳动现在只占 20 便士,或 1/4 即 25\%;而以前是占[33+(1/3)]\%。另一方面,原料占 60 便士,即 3/4 或 75\%,而以前只占[66+(2/3)]\%。因为 5 磅棉纱现在值 80 便士,所以 1 磅值 80/5 即 16 便士。这样,在 5 磅棉纱中,代表[新加]劳动创造的价值的 20 便士占 1+(1/4)磅棉纱;其余的 3+(3/4)磅棉纱是原料所占的部分。以前[新加]劳动(利润和工资)占 1+(2/3)磅,不变资本占 3+(1/3)磅。因而,用劳动本身的产品来估计,劳动的生产能力降低了,虽然劳动生产率没有变,而只是原料涨价了。劳动仍然保持自己原来的生产率,因为同一劳动在同一时间内把 5 磅棉花变成 5 磅棉纱,因为这种劳动的真正产品(从使用价值来看),只不过是棉花所获得的\textbf{棉纱形式}。5 磅棉花象以前一样由于同一劳动而获得棉纱形式。但是构成实在产品的不只是这种棉纱形式,而且还有棉花,即获得这种形式的物质,这种物质的价值现在和赋予这种形式的劳动相比,在总产品中占了更大的部分。因此,纺纱工人的同量劳动现在是由较少量的棉纱来支付了,换句话说,补偿这种劳动的那部分产品减少了。

这个问题就是如此。

\centerbox{\textbf{[(c)加尔涅反驳斯密时的庸俗前提。加尔涅回到重农学派的见解。比重农学派后退一步:把非生产劳动者的消费看成生产的源泉]}}

因此,第一,加尔涅断言全部资本归根到底总要由消费者的收入来补偿,是错误的;因为资本的一部分只能由资本补偿,不能由收入补偿。第二,这种说法本身也是荒谬的,因为收入本身,只要不是工资(或由工资支付的工资,即由工资派生的收入),就是资本的利润(或由资本的利润派生的收入)。最后,加尔涅断言\authornote{见本册第 182 页。——编者注}那一部分不流通(意即不由消费者的收入补偿)的资本,“不能给它的所有者带来利润”,也是荒谬的。这一部分——在生产条件不变的情况下——事实上不带来利润(确切些说,不带来剩余价值)。但是没有这一部分,资本就根本不可能生产自己的利润。

\begin{quote}[357]“从这种区别中只能得出这样的结论:为了雇用生产劳动者,不仅需要享用他们劳动的人的收入,而且还需要给中介人带来利润的资本;而为了雇用非生产劳动者,在大多数情况下只要有支付他们报酬的人的收入就够了。”(同上,第 175 页)\end{quote}

单单这一段话就已经是一派胡言乱语,从中可以看出,亚·斯密著作的译者加尔涅,实质上对亚·斯密是一无所知,甚至一点也没有看出《国富论》中最本质的东西,那就是:认为资本主义生产方式是最生产的(同以前的那些形式比较起来,它无疑是这样的)。

首先,为反驳斯密所说的直接由收入支付的劳动是非生产劳动这一点,而提出什么“为了雇用\textbf{非生产}劳动者,在大多数情况下只要有支付他们报酬的人的收入就够了”,这是愚蠢到了极点。其次,还有相对的命题:“为了雇用生产劳动者,\textbf{不仅}需要\textbf{享用}他们劳动\textbf{的人的收入},而且还需要\textbf{给中介人带来利润的资本}”!(那末,加尔涅先生的农业劳动该具有多么大的生产能力,对这种劳动来说,除了消费土地产品的人的收入之外,还需要一个不仅给中介人带来利润,而且给土地所有者带来地租的资本!)

说“为了雇用生产劳动者”,需有第一,使用他们的资本,第二,消费他们的劳动的收入,这是不对的;为此,只需有创造收入来消费他们的劳动成果的资本就够了。如果我以缝纫业资本家的身分把 100 镑花在工资上,这 100 镑会给我带来譬如说 120 镑。它为我创造出 20 镑的收入。只要我愿意,我现在也可以用这 20 镑来消费把衣料做成“上衣”的裁缝的劳动。如果相反,我用 20 镑买一件衣服穿,那就十分明显,并不是这件衣服为我创造了用来购买它的 20 镑。如果我把一个裁缝叫到家里,要他为我缝一件价值 20 镑的衣服,情形也是一样。在第一种情况下,我会比原有的多得 20 镑,在第二种情况下,我在交易后会比交易前少 20 镑。而且,我还会很快发现,直接用我的收入支付给裁缝来做上衣,不如我从“中介人”那里购买上衣便宜。

加尔涅以为,利润是由消费者支付的。消费者支付商品的“价值”;虽然商品中也包含资本家的利润,但是这种商品对于消费者来说,比起他直接花费自己的收入去购买劳动,让受雇的劳动者小规模地生产物品,以满足雇主的个人需要,是便宜的。这里显然暴露出加尔涅对什么是资本一窍不通。

他接着说:

\begin{quote}“其次,不是有许多\textbf{非生产}劳动者,例如演员、音乐家等等,在大多数情况下通过经理来取得自己的工资吗\fontbox{?}而这些经理是从投入这类企业的资本中吸取利润的。”(同上,第 175—176 页)\end{quote}

这个意见是对的。但这不过表明,有一部分劳动者,即亚·斯密按照他的第二个定义称为非生产劳动者的,按照他的第一个定义却应当是生产劳动者。

\begin{quote}“由此应当承认,在\textbf{生产}阶级人数众多的社会里,中介人或企业主手里有大量的资本积累。”(同上,第 176 页)\end{quote}

的确,雇佣劳动的大量存在,只是资本的大量存在的另一种表现。

\begin{quote}“因此,并不是象斯密所认为的那样,资本量和收入量之间的比例决定\textbf{生产}阶级和\textbf{非生产}阶级之间的比例。后面这种比例看来在极大程度上取决于国民的风俗习惯,取决于该国工业发展水平的高低。”(第 177 页)\end{quote}

如果生产劳动者是由资本支付的劳动者,非生产劳动者是由收入支付的劳动者,那末十分明显,生产阶级和非生产阶级之比等于资本和收入之比。但是这两个阶级的比例的增加,不仅仅取决于资本量和收入量之间的现有比例。它还取决于增长着的收入(利润)以怎样的比例转化为资本,并以怎样的比例当作收入来花费。虽然资产阶级起初很节约,但是随着资本的生产率即劳动者的生产率的增长,它就开始仿效[358]封建主豢养大批侍从。根据最近的(1861 或 1862 年)工厂报告,联合王国真正在工厂工作的总人数(包括管理人员)只有 775534 人\authornote{《答可尊敬的下院 1861 年 4 月 24 日的质问》(1862 年 2 月 11 日刊印)。},而女仆单是英格兰一处就有 100 万。让工厂女工每天在工厂流汗 12 小时,以便工厂主可以拿她的无酬劳动的一部分雇她的姊妹当佣人,雇她的兄弟当马夫,雇她的堂兄弟当士兵或警察,来为他个人服务,这真是一种绝妙的安排!

加尔涅最后加的一句话是庸俗的同义反复。他说生产阶级和非生产阶级之间的比例不取决于资本和收入之间的比例,或者更确切地说,不取决于以资本形式支出的现有商品量和以收入形式支出的商品量之间的比例,而(!\fontbox{?})取决于国民的风俗习惯,取决于该国的工业发展水平。事实上,资本主义生产只有在工业发展的一定阶段才出现。

加尔涅作为波拿巴的参议员,当然热衷于有侍从和一般仆人。

\begin{quote}“在人数相等的情况下,没有一个阶级象家仆那样促使来自\textbf{收入}的金额转化为资本。”(第 181 页)\end{quote}

事实上,没有一个阶级会给小资产阶级提供更为卑贱的分子。加尔涅不懂,斯密这个“具有如此洞察力的人”,怎么不能较高地评价

\begin{quote}“这种依附于富人、\textbf{捡拾}被富人挥霍浪费的收入的残余的中介人”等等。(同上,第 183 页)\end{quote}

但是,加尔涅自己在这里也说,中介人只是“捡拾”“收入”的残余。这种收入由什么构成呢\fontbox{?}由生产工人的无酬劳动构成。

加尔涅在对斯密作了所有这些毫无用处的反驳之后,就滚回到重农主义去了,他宣布农业劳动是唯一的生产劳动!为什么呢\fontbox{?}因为它

\begin{quote}“还创造一个新价值,这个价值在这种劳动开始活动时在社会上并\textbf{不存在},甚至作为等价物也不存在,正是这个价值为土地所有者提供地租”。(同上,第 184 页)\end{quote}

那末,什么是生产劳动呢\fontbox{?}就是创造剩余价值的劳动,即除了它以工资形式取得的等价之外还创造新价值的劳动。\textbf{资本同劳动}相交换,\textbf{只不过}是具有和一定量劳动相等的一定价值的商品同比这个商品本身包含的更大的劳动量相交换,因此“创造一个新价值,这个价值在这种劳动开始活动时在社会上并不存在,甚至作为等价物也不存在”。如果加尔涅不了解这一点,那末,这并不是斯密的过错。[VIII—358]

\centerbox{※     ※     ※}

[IX—400]\textbf{热·加尔涅先生于 1796 年}在巴黎出版了《政治经济学原理概论》。书中除了认为只有农业是生产的这种重农主义观点以外,我们还看到另一个观点(这个观点很能说明他对亚·斯密的反驳),那就是,认为消费(由“非生产劳动者”出色地代表的消费)是生产的源泉,生产的大小由消费的大小来衡量。非生产劳动者满足“人为的需要”并消费物质产品,所以据说他们在各方面都是有用的。因此,加尔涅也反驳节省(节约)。在他的序言第 XIII 页上可以读到这样的话:

\begin{quote}“个人的财富由于节约而增加,\textbf{相反},社会的财富则由于消费增加而增长。”\end{quote}

在第 240 页,论国债这一章中,加尔涅说:

\begin{quote}“农业的改进和扩大,因而工商业的进步,除了人为的需要扩大之外,没有\textbf{别的原因}。”\end{quote}

他由此得出结论说,国债是好事情,因为它会使这种需要增加。\endnote{这几段话是论热尔门·加尔涅那一小节的补充,取自手稿第 IX 本,在论萨伊那一小节和论德斯杜特·德·特拉西那一小节之间。加尔涅《政治经济学原理概论》一书中的话,马克思引自德斯杜特·德·特拉西的著作《思想的要素》第四、五部分,1826 年巴黎版第 250—251 页。——第 199 页。}[IX—400]

\centerbox{※     ※     ※}

[IX—421]\textbf{施马尔茨}。这个德国的重农主义余孽,批评斯密对生产劳动和非生产劳动的区分说(他的书德文版\textbf{在 1818 年}出版):

\begin{quote}“我只指出……斯密对\textbf{生产}劳动和\textbf{非生产}劳动所作的区分,不应当看成是重要的和十分准确的,如果考虑到别人的劳动始终只是使我们节省时间,而这种时间的节省就是构成\textbf{劳动价值}和\textbf{劳动价格}的一切。”\end{quote}

\fontbox{~\{}这里他搞乱了。事情并不是由分工引起的时间的节省决定物的价值和价格,而是我用同一价值得到更多的使用价值,劳动的生产能力更大了,因为在同一时间内创造出更多的产品;但是,作为重农学派的余波,他当然不能在劳动时间本身找到价值。\fontbox{\}~}

\begin{quote}“例如,为我做桌子的木匠,和把我的信送到邮局、给我洗衣服和张罗我所需要的物品的仆人,他们两者对我的服务在性质上是完全一样的:他们既节约我亲自干这些事情所必须花费的时间,又节约我为获得做这些事情的技能和本领所必须花费的时间。”(\textbf{施马尔茨}《政治经济学》,昂利·茹弗鲁瓦译自德文,1826 年版第 1 卷第 304 页)\end{quote}

在这个粗制滥造者施马尔茨的作品中,我们还发现,下面这种意见对于了解加尔涅——例如他的消费主义(以及浪费的经济效用)——同重农主义之间的联系,是很重要的:

\begin{quote}“这个主义〈魁奈主义〉认为,手工业者,甚至\textbf{纯粹消费}的人,消费有功,理由是他们的消费可以促进(虽然是间接地促进)国民收入的增加,\textbf{因为没有这种消费,被消费的物品就不会由土地生产出来,也不能加到土地所有者的收入中去}。”(同上,第 321 页)\endnote{以“施马尔茨”为总标题的这几段话是手稿第 IX 本结尾部分的附笔。就其内容来说,它们是该稿本第 400 页(见本册第 199 页)关于加尔涅的补充评论的补充。——第 200 页。}[IX—421]\end{quote}

\tsectionnonum{[(8)]沙尔·加尼耳[关于交换和交换价值的重商主义观点。把一切得到报酬的劳动都纳入生产劳动的概念]}

[VIII—358]\textbf{沙·加尼耳}的《论政治经济学的各种体系》是一本很糟糕、很肤浅的拙劣作品。第一版\textbf{于 1809 年在}巴黎出版;第二版\textbf{于 1821 年}出版(我引用的是第二版)。他的胡诌是同他所反驳的加尔涅直接衔接的。

\fontbox{~\{}\textbf{卡纳尔}在《政治经济学原理》中下定义说:“\textbf{财富是多余劳动的积累}。”\endnote{卡纳尔给财富所下的定义,马克思引自加尼耳《论政治经济学的各种体系》一书(第 2 版第 1 卷第 75 页)。这个定义在卡纳尔的著作第 4 页。——第 201 页。}如果他说,财富是超过工人维持他作为工人的生活而多余下来的劳动,那末这个定义就对了。\fontbox{\}~}

加尼耳先生的出发点是这样一个基本的论点:商品是资产阶级财富的元素,也就是说,为了生产财富,劳动必须生产商品,\textbf{必须}出卖它自己或自己的产品。加尼耳说:

\begin{quote}“在现代文明状态下,劳动只有通过交换才能为我们认识。”(同上,第 1 卷第 79 页)“没有交换,劳动就不能生产任何财富。”(第 81 页)\end{quote}

加尼耳先生从这里一下子就跳到重商主义体系上去了。因为没有交换,劳动就不会创造资产阶级财富,所以“财富完全是来源于商业”(第 84 页)。或者象他在后面所说的:

\begin{quote}“只有交换或商业才使物具有价值。”(第 98 页)这个“价值和财富等同的原则……是关于一般劳动的生产性的学说的基础”。(同上,第 93 页)\end{quote}

加尼耳自己解释说:

\begin{quote}[359]“商业主义”——他把它叫做货币主义的纯粹“变态”——“从劳动的交换价值中引出私人的和公共的财富,不管这种价值是否固定在耐久的和不变的物质对象上。”(第 95 页)\end{quote}

由此可见,就象加尔涅回到重农主义体系的见解上去一样,加尼耳回到重商主义体系的见解上去了。因此,他的毫无用处的废话,对于说明重商主义体系及其“剩余价值”观点的特征,却是非常有用的,特别是因为他提出这些观点来反对斯密、李嘉图等等。

财富是交换价值;因此,凡是生产交换价值或本身具有交换价值的劳动,都生产财富。表明加尼耳看问题比别的重商主义者略深一些的唯一的说法,就是“一般劳动”这个术语。个人的劳动,或者更确切地说,个人劳动的产品,必须采取\textbf{一般}劳动的形式。只有这样,劳动的产品才变成交换价值,变成\textbf{货币}。实际上,加尼耳回到财富就是货币这个观点上去了;不过在加尼耳看来,财富不仅是金银,而且是商品本身,因为商品就是\textbf{货币}。加尼耳说:

\begin{quote}“\textbf{商业主义},或者说,\textbf{一般劳动}的价值的交换。”(第 98 页)\end{quote}

后一说法是荒谬的。产品所以是价值,因为它是一般劳动的存在,是这种劳动的化身,而不是“一般劳动的\textbf{价值}”,否则就等于说价值的价值了。但是假定商品已经被确认为价值,如果愿意,还可以假定它甚至已经具有货币形式,已经完成自己的形态变化。现在商品是交换价值。但是这个商品的价值量有多大呢\fontbox{?}一切商品都是交换价值,它们在这方面彼此没有区别。但是什么东西决定这个商品的交换价值呢\fontbox{?}这里加尼耳停留在最表面的现象上。商品 A 如果同大量的商品 B、C、D 等等相交换,它就是大的交换价值。

加尼耳反对李嘉图和大多数政治经济学家时说,虽然他们的体系和一切资产阶级的体系一样,以交换价值为基础,可是他们在考察劳动时,却把交换置之不顾;这个意见是完全正确的。但是,他们所以这样做,唯一的原因在于,他们认为产品的商品\textbf{形式}是不言而喻的事情,因此他们只考察\textbf{价值量}。在交换中,个人的产品所以表现为一般劳动的产品,仅仅是因为它们表现为\textbf{货币}。而这种相对性的根源已经在于:它们必须表现为一般劳动的存在,并且只有作为相对的、仅仅在量上不同的社会劳动的表现,才能归结为这种存在。不过,交换本身不会使它们具有\textbf{价值量}。在交换中,它们表现为一般社会劳动,而它们究竟能在多大程度上这样表现,要看它们本身能在多大程度上表现为社会劳动,也就是说,要看它们能够交换的商品的数量,因而要看市场的规模、贸易的规模,要看它们借以表现为交换价值的商品系列。例如,如果只有四个不同的生产部门,那末,这四个生产者中的每一个人都会有很大一部分产品是为自己生产。如果有几千个生产部门,那末,每一个人就可以把他的全部产品都当作商品来生产。他的全部产品都可以加入交换。

但是,加尼耳同重商学派一道认为,\textbf{价值量本身是交换的产物},其实,产品通过交换得到的只不过是价值形式或\textbf{商品}形式。

\begin{quote}“交换使\textbf{物}具有价值;没有交换,物就没有价值。”(第 102 页)\end{quote}

如果这是说,\textbf{物},使用价值,只有作为社会劳动的相对表现,才变成价值,才获得这种形式,那是同义反复。如果这是说,它们通过交换,比没有交换前取得更大的价值,那显然是胡说,因为交换要提高商品 A 的价值量,就只有降低商品 B 的价值量。它使商品 A 的价值比交换前增加多少,它也就使商品 B 的价值减少多少。因此,\textbf{A+B 无论在交换前还是在交换后都具有同样的价值}。

\begin{quote}“最有用的产品可能没有价值,如果交换不使它们具有价值……”\end{quote}

(首先,如果这些物是“产品”,那末它们一开始就是劳动的产品,而不象空气等等那样是一般的自然赐予。如果它们是“最有用的”,那末它们就是最高意义上的使用价值,是所有的人都需要的使用价值。如果交换\textbf{不}使它们具有价值,那末,这只有在每个人都为自己生产这些东西的条件下才是可能的;但这同[360]它们是为交换而生产出来的这个前提相矛盾;因此,整个前提就是荒谬的。)

\begin{quote}……“而最无用的产品可能有很大的价值,如果交换对它们有利。”(第 104 页)\end{quote}

在加尼耳先生看来,“交换”是一个神秘人物。如果“最无用的产品”没有一点用处,如果它们没有任何使用价值,谁还会购买它们呢\fontbox{?}可见,对于买者来说,它们无论如何必须有某种“有用性”,哪怕只是想象的“有用性”。如果买者不是傻子,他又何必要为它们支付较高的价钱呢\fontbox{?}可见,造成它们昂贵的原因,无论如何不会是它们的“无用性”。也许是它们的“稀有性”吧\fontbox{?}可是加尼耳把它们叫做“最无用的\textbf{产品}”。既然它们是产品,那末,为什么它们有大的“交换价值”,人们却不以更大的规模生产它们呢\fontbox{?}如果说前面那个用许多货币去购买对他本人既没有实际的使用价值,又没有想象的使用价值的东西的买者,是傻子;那末这里,生产交换价值小的有用品,而不生产交换价值大的无用品的卖者,也是傻子。由此看来,如果它们的使用价值很小(假定使用价值由人们的自然需要决定),但交换价值很大,那末,这必定不是由交换先生的情况造成的,而是由产品本身的情况造成的。\textbf{可见,产品的高的交换价值并不是交换的产物,它不过是在交换中表现出来而已}。

\begin{quote}“物的已交换价值[Valeuréchangée],而不是物的可交换价值[Valeuréchangeable],构成\textbf{真正价值},即和财富等同的那种真正价值。”(同上,第 104 页)\end{quote}

但是,可交换价值是一种物同它所能交换的另一种物的比例。\fontbox{~\{}这里有正确的一点作为基础:商品之所以不得不转化为货币,是由于商品必须作为可交换价值,作为交换价值进入交换,而商品之所以是交换价值,只是交换的结果。\fontbox{\}~}相反,A 的已交换价值是一定量的产品 B、C、D 等等。因此,这已经不是价值(按照加尼耳先生的说法),而是“没有交换的物”。B、C、D 等等在同 A 交换以前都不是“价值”。A 变成价值是由于这些非价值取代了它的位置(作为已交换价值)。于是,由于单纯的换位,这些“物”突然变成价值,而在换位以后,它们就退出交换,处于和以前一样的状况。

\begin{quote}“可见,既不是物的实际效用,也不是物的\textbf{内在}价值,使物成为财富;是交换确定和决定它们的价值,就是这种价值使它们成为和财富等同的东西。”(同上,第 105 页)\end{quote}

交换先生确定和决定的或者是某种已经存在的东西,或者是某种不存在的东西。如果是交换首先创造物的价值,那末,交换本身一停止,这个价值,这个交换的产物,也就会消失。因此,它创造什么,它也就消灭什么。我用 A 同 B+C+D 相交换。A 就在这个交换行为中得到价值。这个行为一结束,B+C+D 就站在 A 方面,A 则站在 B+C+D 方面。它们各站一方,都站在交换(只不过换位)先生的范围以外。B+C+D 现在是“物”,但不是价值。A 也是这样。或者,交换就是名副其实地“确定和决定”,就象测力计确定和决定我们的肌肉力,而不是创造我们的肌肉力一样。但如果是这样,价值就不是由交换创造的了。

\begin{quote}“对于个人和对于各国人民来说,事实上只有在下面的情况下才会有财富:每人为大家劳动\end{quote}

(就是说,每人的劳动表现为\textbf{一般社会劳动}。如果不这样解释,这句话就是荒谬的;因为要是撇开这种一般社会劳动形式,那就应该说,制铁业者不是为大家劳动,而\textbf{只是}为铁的消费者劳动),

\begin{quote}大家为每人劳动”\end{quote}

(如果这里谈的是使用价值,那又是荒谬的;因为大家的产品全都是特殊的产品,每人需要的也只是某种特殊的产品;因此,这里的意思又只能是:每种特殊的产品都采取\textbf{为每人而存在}的形式,而它以这样的形式存在,并不是由于它作为特殊的产品和“每人”的产品不同,而只是由于它和这些产品等同;我们在这里看到的,又是在商品生产基础上表现出来的那种社会劳动形式)。(同上,第 108 页)

[361]加尼耳从这个规定——交换价值是孤立的个人的劳动作为一般社会劳动的表现,又滚到最粗俗的看法上去了:交换价值是商品 A 同商品 B、C、D 等等相交换的比例,如果用许多 B、C、D 来交换 A,那末商品 A 的交换价值就大;但在这种情况下,也就是用少量的 A 来交换 B、C、D。财富由交换价值构成。交换价值是产品相互交换的比例。这样,全部产品的总和也就没有任何交换价值了,因为这个总和不同任何东西交换。因此,财富由交换价值构成的社会,也就没有任何财富了。由此不仅可以得出加尼耳本人所做的结论:“由劳动的交换价值构成的国民财富”(第 108 页)按其交换价值来说,永远不会增加,也不会减少(因此,\textbf{也就没有任何剩余价值});而且还可以得出这样的结论:这个财富根本没有交换价值,从而不是财富,因为财富只由交换价值构成。

\begin{quote}“如果谷物极为丰裕,因而\textbf{谷物价值下降},那末土地耕种者的财富就会减少,因为现在他们只有较少的交换价值可以用来获取生活上必需的、有用的或喜爱的物品;不过,土地耕种者亏损多少,谷物消费者就恰好得利多少;一些人的亏损为另一些人的得利所补偿,总财富不会有任何变化。”(第 108—109 页)\end{quote}

对不起!谷物消费者消费的是谷物,而不是谷物的交换价值。他们是在食物上,而不是在交换价值上变得更富了。他们用自己少量的产品交换谷物,而这些产品,由于同它们所交换的谷物比起来数量较少,所以具有\textbf{高的交换价值}。土地耕种者现在得到了高的交换价值,而消费者得到了具有较低交换价值的许多谷物,因而现在消费者是穷人,土地耕种者是富人。

其次,总和(交换价值的社会总和)愈变为交换价值的总和,就愈丧失它作为交换价值的性质。A、B、C、D、E、F 只有互相交换,才有交换价值。一旦它们已经交换,它们对于它们的消费者,买者来说,就都是产品了。它们经过转手,就不再是交换价值了。这样一来,“由交换价值构成的”社会财富也就消失了。商品 A 的价值是相对的;这个价值是 A 对 B、C 等等的交换比例。A+B 具有较少的交换价值,因为它们的交换价值只不过是它们对 C、D、E、F 的比例。而 A、B、C、D、E、F 的总和根本没有任何交换价值,因为这个总和不表示任何比例。商品的总和不同别的商品交换。因此,由交换价值构成的社会财富没有任何交换价值,因而也不是财富。

\begin{quote}“由此可见,一国要靠国内贸易来致富是困难的,也许是不可能的。进行对外贸易的各国人民的情况就有所不同了。”(同上,第 109 页)\end{quote}

这是老重商主义。据说价值就在于,我得到的不单单是等价,而是多于等价。可是加尼耳同时又认为根本没有等价,因为等价的前提是:商品 A 的价值和商品 B 的价值,不是由 A 对 B 的比例或 B 对 A 的比例决定,而是由一个第三者决定,在这个第三者身上,A 和 B 是等同的。如果没有等价,也就没有超过等价的余额。我用铁换得的金比用金换得的铁少。现在我有较多的铁,我又用这些铁换得较少的金。因而,如果说起初由于较少的金等于较多的铁,我得利了,那末,现在因为较多的铁等于较少的金,我同样亏损了。

\centerbox{※     ※     ※}

\begin{quote}“任何劳动,不管其性质如何,只要它具有交换价值,就是生产劳动,就是生产财富的劳动。”(同上,第 119 页)“交换既不考虑产品的量,也不考虑产品的物质性,也不考虑产品的耐久性。”(同上,第 121 页)“所有的〈劳动〉从它们生产出它们本身已经与之交换的\textbf{总额}这一点来说,\textbf{同样都是生产的}。”(第 121—122 页)\end{quote}

起先,说这些劳动同样都是生产的,是从它们补偿上述\textbf{总额},即补偿支付它们的\textbf{价格}(它们工资的\textbf{价值})这一点来说的。但是,加尼耳马上又进一步。他宣称,非物质劳动生产出它本身与之交换的物质产品,以致看起来象是物质劳动生产出非物质劳动的产品。

\begin{quote}[362]“一个是制造柜子的工人,用这个柜子换得 1 舍费耳谷物,另一个是流浪音乐家,用他的劳动也换得 1 舍费耳谷物;这两个人的劳动没有任何区别。在这两种情况下都是生产出了 1 舍费耳谷物,在一种情况下,生产出 1 舍费耳谷物是为了支付柜子,在另一种情况下,生产出 1 舍费耳谷物是为了支付流浪音乐家提供的娱乐。诚然,木匠把 1 舍费耳谷物消费以后,留下一个柜子,而音乐家把 1 舍费耳谷物消费以后,什么也没有留下。可是,有多少被认为是生产劳动的劳动都是这样的情况啊!……判断一种劳动是生产的还是不生产的,不能根据消费以后究竟留下什么,而应当\textbf{根据交换或根据这种劳动所引起的生产}。因为音乐家的劳动同木匠的劳动一样,都是\textbf{生产 1 舍费耳谷物的原因,所以他们两人劳动的生产性同样由 1 舍费耳谷物来衡量},虽然一个人的劳动在干完以后不固定、不物化在某种耐久的对象上,而另一个人的劳动则固定、物化在某种耐久的对象上。”(同上,第 122—123 页)“亚·斯密想减少从事无用劳动的劳动者人数,以便增加从事有用劳动的劳动者人数;但是持这种观点的人没有看到,如果这种愿望能够实现,那就不可能有任何财富了,因为生产者将会缺乏消费者,而没有消费掉的剩余物也就不会再生产出来。生产阶级把自己劳动的产品供给\textbf{那些不创造物质产品的阶级}并不是无代价的。”\end{quote}

(可见,在这里他自己还是把生产物质产品的劳动和不生产物质产品的劳动区分开来了。)

\begin{quote}“生产阶级把自己的产品供给他们,从他们那里换得方便、娱乐、享受,\textbf{而为了能够把自己的产品供给他们,生产阶级就不得不生产这些产品}。如果劳动的物质产品不用来支付那些不创造物质产品的劳动,那它们就找不到消费者,它们的\textbf{再生产}就要停止。因此,生产娱乐的劳动,\textbf{也象}那种被认为是最生产的劳动\textbf{一样有效地参加生产}。”(同上,第 123—124 页)“他们〈各国人民〉所追求的方便、娱乐或享受,几乎总是\textbf{跟在必须用来支付它们的产品后面,而不是走在这些产品前面}。”(同上,第 125 页)\end{quote}

(可见,它们看来与其说是“必须用来支付它们的”那些产品的原因,不如说是那些产品的结果。)

\begin{quote}“如果\textbf{生产阶级不需要}为娱乐、奢侈或豪华服务的劳动〈可见,在这里加尼耳本人也做了这样的区分〉,而又\textbf{不得不}支付这种劳动,并把自己的需求削减相应的数额,那末情况就不同了。在这种情况下,这样的被迫支付就不会引起产品量的增加。”(第 125 页)“除了这种情况之外……任何劳动都必然是生产的,并在不同程度上有助于整个财富的形成和增长,因为\textbf{任何劳动都必然会引起用以支付它的那些产品的生产}。”(同上,第 126 页)\end{quote}

\fontbox{~\{}可见,按照这种说法,“各种非生产劳动”所以是生产的,既不是因为它们有所值,就是说,不是由于它们的交换价值;也不是因为它们生产某种娱乐,就是说,不是由于它们的使用价值;而是因为它们生产生产劳动。\fontbox{\}~}

\fontbox{~\{}如果按照亚·斯密的说法,直接同资本交换的劳动是生产的,那末,同劳动交换的资本,除了它的形式之外,它的物质组成部分也应加以考虑。这个资本归结为必要的生存资料,因而大部分归结为商品,归结为物质品。工人不得不从这种工资中支付给国家和教会的东西,则是为支付那强加于他的服务而作的一种扣除;工人支出在教育上的东西是微不足道的;凡是工人有这种支出的时候,这种支出都是生产的,因为教育会生产劳动能力;工人在医生、律师、牧师的服务上支出,算是他倒霉;还有一些别的非生产劳动或服务也花费工人的工资,不过那是很少的,特别是因为同消费有关的工作(做饭,打扫房屋,甚至在大部分情况下各种各样的修理工作)都是工人自己干的。\fontbox{\}~}

加尼耳的下面这段话是最典型的:

\begin{quote}“如果交换使仆人的劳动具有 1000 法郎的价值,而使土地耕种者或工业工人的劳动只具有 500 法郎的价值,那末由此就应该得出结论说,仆人的劳动对于\textbf{财富生产}的贡献,比土地耕种者或工业工人的劳动大一倍;只要仆人的劳动得到的物质产品报酬比土地耕种者或工业工人的劳动得到的大一倍,就不可能是另外的情况。怎么能够说财富是来源于交换价值最少、因而得到的报酬最低的劳动呢!”(同上,第 293—294 页)\end{quote}

[363]如果工业工人或农业工人的工资等于 500 法郎,他所创造的剩余价值(利润和地租)等于 40\%,那末这种工人的“纯产品”就等于 200 法郎,必须有 5 个这种工人的劳动才能生产出仆人的 1000 法郎工资。如果交换先生一年中不愿购买仆人,而愿意用 1 万法郎去购买一个姘妇,那就需要 50 个这种生产工人的“纯产品”了。既然姘妇的非生产劳动给她带来的交换价值即报酬,比生产工人的工资大 19 倍,那末,按照加尼耳的看法,这个女人对于“财富生产”的贡献就大 19 倍,而且一国向仆人和姘妇支付的东西愈多,它生产的财富也就愈多。加尼耳先生忘记了,只有工业劳动和农业劳动的生产性,总之,只有生产工人创造的、但没有被支付过代价的那个余额,才提供支付非生产劳动者的基金。但加尼耳是这样计算的:1000 法郎工资加上它的以仆人或姘妇的劳动形式存在的等价,一共是 2000 法郎。而实际上,仆人和姘妇的价值,即他们的生产费用,完全取决于生产工人的“纯产品”。甚至仆人和姘妇作为特殊的一类人存在本身,也取决于这种“\textbf{纯产品}”。他们的价格和他们的价值之间极少共同之处。

即使假定,仆人的价值(生产费用)比生产工人的价值或生产费用大一倍。那也应当看到,工人的生产率(象机器的生产率一样)和他的价值是完全不同的东西,它们两者甚至是成反比的。机器所费的价值,对于它的生产率始终是一个负数。

\begin{quote}“有人徒劳地反驳说:如果仆人的劳动,也象土地耕种者的劳动和工业工人的劳动一样,是生产的,那我们就不明白,为什么不能把一国的一般积蓄拿来养活仆人,这样不但不会把这些积蓄浪费掉,而且还会不断地增加它们的价值。这种反驳之所以看来好象是正确的,只不过因为它假设任何劳动的生产性都是它\textbf{参与物质品生产}的结果,\textbf{物质生产创造财富,生产和财富完全是等同的东西}。他们忘记了,\textbf{任何产品的生产都只是由于这些产品被消费,才成为财富},[注:\fontbox{~\{}因此,这个家伙在下一页上说:“任何劳动\textbf{生产}财富,都同自己的由供求决定的交换价值成比例〈这样一来,劳动\textbf{生产}财富,不是同它生产多少交换价值成比例,而是同它本身的交换价值是多少成比例,也就是说,不是同它生产多少成比例,而是同它值多少成比例〉,而且只有把劳动的价值使劳动有权从产品总额中取得的那些产品\textbf{节约下来不消费},劳动的价值才会有助于资本的积累。”\fontbox{\}~}]并且是交换决定生产究竟在什么程度上\textbf{有助于财富的形成}。如果我们记住:在任何国家,各种劳动都直接或间接有助于整个生产;交换既然确定每种劳动的价值,也就决定每种劳动参加生产的份额;\textbf{产品的消费}实现着交换使它具有的价值;生产同消费相比出现的剩余或不足,决定着各国人民富裕或贫穷的程度;——那末,我们就会明白,\textbf{孤立地看}每种劳动,根据它\textbf{参加物质生产}的程度来衡量它的效果性和生产性,\textbf{而不考虑}[364]\textbf{这种劳动的消费},该是多么不彻底。\textbf{只有消费才使劳动具有价值},没有这种价值,财富就不能存在。”(同上,第 294—295 页)\end{quote}

这个家伙,一方面认为财富取决于生产超过消费的余额,另一方面又认为只有消费才使劳动具有价值。而且,根据这种看法,一个消费 1000 法郎的仆人对于创造价值的贡献,比消费 500 法郎的农民大一倍。

加尼耳一方面承认这种非生产劳动不直接参加物质财富的形成,而斯密的主张也无非是这样;另一方面,他又力图证明,这些非生产劳动,同他自己承认的不创造物质财富相反,是创造物质财富的。

在所有这些反驳亚·斯密的人那里,我们一方面看到他们对物质生产采取高傲态度,另一方面又看到他们力图为非物质生产——甚至根本不是生产,如象仆役的劳动——辩护,把它冒充物质生产。无论“纯收入”的所有者把这种收入花在仆役身上,花在姘妇身上,还是花在油炸馅饼上,那是完全无关紧要的。但是认为为了不使产品的价值去见鬼,余额就必须由仆人来消费,而不能由生产工人本人来消费,这种想法是可笑的。马尔萨斯也就是这样宣扬非生产的消费者存在的必要——一旦余额掌握在“有闲者”手里,实际上也就有这种必要。[364]

\tsectionnonum{[(9)加尼耳和李嘉图论“纯收入”。加尼耳主张减少生产人口;李嘉图主张资本积累和提高生产力]}

[364]\textbf{加尼耳}断言,他在他的《政治经济学理论》一书(我不知道这本书)中提出了一个理论,这个理论在他之后被李嘉图复制了。\endnote{加尼耳的这一说法,在他的著作《论政治经济学的各种体系》(1821 年巴黎第 2 版)第一卷第 213 页。加尼耳的《政治经济学理论》一书于 1815 年出版,比李嘉图的《政治经济学和赋税原理》第一版早两年。——第 212 页。}这个理论就是:财富不取决于总产品,而取决于纯产品,即取决于利润和地租的高低。(这决不是加尼耳的发现;但他表述这一点的方式确实有独特之处。)

剩余价值表现在(实际存在于)剩余产品中,表现在产品超过仅仅补偿产品原有要素、因而加入产品生产费用的那部分产品(如果把不变资本和可变资本合起来计算,这部分产品就等于预付在生产中的资本)的余额中。资本主义生产的目的是剩余价值,而不是产品。工人的必要劳动时间——以及产品中用来支付这个时间的等价物——只有在提供剩余劳动的情况下,才是必要的。否则,这个时间对于资本家就是\textbf{非生产的}。

剩余价值等于剩余价值率 m/v 乘同时使用的工作日数或所使用的工人人数 n。因而,M=m/v×n。这样,就有两种方法可以使这个剩余价值增加或减少。例如,[m/v/2]×n 等于 2m/v×n,即 2M。这里 M 增加了[365]1 倍,因为剩余价值率增加了 1 倍,[m/v/2]就是 2m/v,而这比 m/v 大 1 倍。另一方面,m/v×2n 也同样等于 2mn/v,因而也是 2M。可变资本 V 等于 1 工作日的价格乘所使用的工人人数。如果雇用 800 工人,每个工人花费 1 镑,那末,V=800 镑,即 1 镑×800,这里 n=800。如果剩余价值是 160,那末,剩余价值率=160/1 镑×800=160/800=16/80=1/5=20\%。但剩余价值本身是[160/1 镑×800]×800,即[M 镑/1 镑×n]×n。

如果工作日的长度已定,这个剩余价值就只有靠提高生产率才能增加;如果生产率已定,这个剩余价值就只有靠延长劳动时间才能增加。

但是,这里重要的是:2M=[m/v/2]×n,和 2M=(m/v)×2n。如果工人人数减少一半(不是 2n 而是 n),而工人每日的剩余劳动比以前增加一倍,剩余价值(它的总额)就仍然不变。在这样的假设下,有两个东西不变:第一,生产出来的产品总量不变;第二,剩余产品或“纯产品”的总量不变。但发生变化的是:第一,可变资本额或花在工资上的那部分流动资本减少了一半。由原料构成的那部分不变资本也不变,因为虽然工人人数少了一半,但加工的原料数量照旧。相反,由固定资本构成的那部分增加了。

如果原来花在工资上的资本为 300 镑(1 个工人 1 镑),那末现在是 150 镑。原来花在原料上的资本是 310 镑,现在仍然是 310 镑。假定机器的价值比资本的其余部分大 3 倍,那末现在,它就是 1600 镑。\endnote{严格地说,假设机器的价值比资本的其余部分即 460 镑(150+310)大 3 倍,机器的价值就应当是 1840 镑。但这个数目会使计算大大复杂化。因此,马克思为了使计算简便起见,就假定机器价值等于 1600 镑。——第 214 页。}这样,如果机器在 10 年内磨损完,那末,每年加入产品的机器价值就是 160 镑。假定原来每年花在生产工具上的资本是 40 镑,即只是现在的 1/4。

在这种条件下,得出如下的计算:

\todo{}

在这种情况下,利润率提高了,因为总资本减少了:花在工资上的资本减少了 150,而固定资本价值额只增加了 120,即总共比原来少花了 30 镑。

如果把剩下的 30 镑也按照同样的方式花掉,即把这个数额的 31/62(或 1/2)花在原料上,16/62 花在机器上,15/62 花在工资上,那就得出:

\todo{}

因而,合计起来是:

\todo{}

\textbf{所花费的资本总额是 650 镑,和以前一样。全部产品}是 807 镑 5 先令 6 便士。

产品的总价值增加了,所花费的资本总价值仍然不变;这里,不仅全部产品的价值增加了,而且全部产品的数量也增加了,因为比原来多 15 镑的原料转化成了产品。

[366]在加尼耳的书中我们读到:

\begin{quote}“当一国没有机器的帮助,它的劳动只靠手的力量进行时,劳动阶级几乎把自己的全部产品都消费掉。随着工业取得成就,随着工业由于分工、工人熟练和机器的发明而日臻完善,生产费用就减少,换句话说,需要较少的工人就获得较多的产品。”(同上[《论政治经济学的各种体系》1821 年第 2 版],第 1 卷第 211—212 页)\end{quote}

这就是说,随着劳动的生产能力愈来愈大,用于工资的生产费用就减少。同产品相比,工人人数就减少;因而在该产品中他们吃掉的部分就更小。

如果没有机器,一个工人生产自己的生活资料需要 10 小时,而有了机器只需要 6 小时;那末,在前一种场合,他(在工作日为 12 小时的情况下)为自己劳动 10 小时,为资本家劳动 2 小时,资本家从 12 小时劳动的全部产品中获得 1/6。在前一种场合,10 个工人为自己劳动 100 小时(为 10 个工人生产产品),为资本家劳动 20 小时。资本家从 120 单位的价值中获得 1/6,即 20 单位。在后一种场合,5 个工人为自己劳动 30 小时(为 5 个工人生产产品),为资本家劳动 30 小时。现在资本家从 60 小时中获得 30 小时即 1/2,比以前多 2 倍。剩余价值量也增加了,从 20 增加到 30,即增加 1/3。60 日(我在其中占有 1/2)所提供的剩余价值,比 120 日(我在其中占有 1/6)所提供的剩余价值,多 1/3。

其次,资本家获得的总产品的一半,在数量上也比他以前获得的多了。因为现在 6 小时提供的产品同以前 10 小时提供的一样多;就是说,1 小时提供以前的 10/6,或 1+(4/6)=1+(2/3)。因此,现在 30 剩余小时所表现的产品量,就是以前 30(1+2/3)=30+60/3=50 小时所表现的产品量。6 小时提供的产品同以前 10 小时提供的一样多;因此,30(或 5×6)小时提供的产品,同以前 5×10 小时提供的一样多。

因此,资本家的剩余价值增加了,他的剩余产品(如果他自己以实物形式消费这种产品,或者就他以实物形式消费的那部分产品来说)也增加了。即使总产品量不增加,剩余价值也可能增加。事实上,剩余价值的增加意味着:工人能在比以前少的时间内,生产自己的生活资料;因此,工人所消费的商品的价值就减少,就代表较少的劳动时间;这样,一定的价值,例如等于 6 小时,就代表比以前多的使用价值。工人得到的产品量同以前一样,但这个量是总产品的较小部分,这个量的价值则表现工作日产品的较小部分。在产品既不直接加入也不间接加入工人消费品生产的那些生产部门,由于生产率的增减不会改变必要劳动同剩余劳动的比例,生产力的任何增长都不能产生这样的结果,——尽管如此,对这些生产部门来说,结果也会是一样的,不过这种结果不是由它们本身的生产率的变化引起。它们产品的相对价值的提高(如果它们的生产率保持不变)同别的商品的相对价值的降低恰成同一比例;因此,这些产品的相应的较小部分,或者说,物化在这些产品中的、工人所花费的劳动时间的较小部分,就会给工人提供同以前一样多的生活资料。就是说,在这些部门剩余价值的提高,也会完全象在其他部门一样。

但 5 个被解雇的工人将怎样呢\fontbox{?}人们会说,有一笔资本也游离出来了,这就是原先用来支付那 5 个被解雇工人的资本(被解雇的工人每天劳动 12 小时,各自得到 10 小时报酬),即游离出来了 50 小时的资本;用这笔资本以前能够支付 5 个工人,而现在工资降低到 6 小时,就可以支付 50/6 即 8+(1/3)工作日。这样看来,用游离出来的 50 劳动小时的资本,现在能够雇用的工人人数,多于被解雇的工人人数。

但是,并不是全部 50 劳动小时的资本都游离出来。因为,即使假定材料也是按它现在在同一劳动时间内多加工多少而便宜多少,就是说,即使假定这个生产部门的生产力有了同样的增长,毕竟还要留下一笔费用来购买新机器。假定新机器正好值 50 劳动小时;那末,生产新机器所能使用的人数,决不会同被解雇的人数一样多。这 50 劳动小时原来是完全花在工资上的,是用来雇 5 个工人的。而等于 50 劳动小时的机器价值中,则包括利润和工资,包括有酬劳动时间和无酬劳动时间。此外,机器的价值中还包括不变资本。而且,被雇来制造新机器的那些机器制造工人(人数比被解雇的工人少),也并不就是那些被解雇的[367]工人。机器制造业对工人的需求的增加,最多只能影响到后来一批工人的分配,即让刚开始劳动的那一代工人比上一代工人更大量地进到这个部门中来。这对被解雇的工人不会发生影响。除此之外,对机器制造工人的年需求的增加,也决不会和投在机器上的新资本相等。例如,机器可以使用 10 年。这就是说,由机器造成的经常需求,每年只等于机器中包含的工资的 1/10。除这 1/10 之外,还要加上 10 年内的修理劳动和煤炭、机油、各种辅助材料的日常消费;这一切加在一起,也许又占 2/10。

\fontbox{~\{}如果游离出来的资本等于 60 小时,那末,这 60 小时现在就代表 10 小时剩余劳动和仅仅 50 小时的必要劳动。因此,这 60 小时以前都花在工资上,用来雇 6 个工人,而现在就只雇 5 个工人了。\fontbox{\}~}

\fontbox{~\{}在某一单个生产部门,由于采用机器等等使生产力增长从而引起劳动和资本的转移,总是在以后才能发生。这就是说,\textbf{增加的人数},即\textbf{新涌现的一批工人},将以另外的方式分配;这批人也许是被抛上街头的工人的子女,但不是他们自己。他们自己长时期靠旧职业糊口,在最不利的条件下干活,因为他们的必要劳动时间大于社会必要劳动时间;他们或者成为赤贫者,或者在使用比较低级的劳动的部门中找到工作。\fontbox{\}~}

\fontbox{~\{}赤贫者也象资本家(食利者)一样,靠国家的收入过活,不加入产品的生产费用。因此,照加尼耳先生看来,这种人和监狱中养活的犯人完全一样,是交换价值的代表。有很大一部分“非生产劳动者”——领干薪的挂名官员等等,不过是高贵的赤贫者。\fontbox{\}~}

\fontbox{~\{}假定劳动生产率大大提高,以前是 2/3 人口直接参加物质生产,现在只要 1/3 人口参加就行了。以前是 2/3 人口为 3/3 人口提供生活资料;现在是 1/3 人口为 3/3 人口提供生活资料。以前“纯收入”(和劳动者的收入不同)是 1/3;现在是 2/3。现在国民——撇开[阶级]对立不谈——应该用在直接生产上的时间,不再是以前的 2/3,而是 1/3。如果平均分配,所有的人就都会有更多的(即 2/3 的)非生产劳动时间和余暇。但是,在资本主义生产条件下,一切看来都是对抗的,而事实上也是这样。我们的假定并不意味着人口始终是停滞的。因为 3/3 在增长,1/3 也会增长,所以按照\textbf{数量}来说,从事生产劳动的人数可能不断增加。但是相对地,按照同总人口的比例来说,他们还是比以前少 50\%。现在 2/3 的人口中一部分是利润和地租的所有者,一部分是非生产劳动者(由于竞争,非生产劳动者的报酬也差了),这些非生产劳动者帮助前者把收入吃掉,并且把服务作为等价提供给前者或者(例如政治的非生产劳动者)强加给前者。我们可以设想:除了家仆、士兵、水手、警察、下级官吏等等、姘妇、马夫、小丑和丑角之外,这些非生产劳动者一般会有比以前高的教育程度;并且,特别是报酬菲薄的艺术家、音乐家、律师、医生、学者、教师、发明家等等的人数将会增加。

而在生产阶级本身,商业中介人的人数会增加,特别是在机器制造业、铁路修建业、采矿工业中就业的人数会增加;其次,在农业中从事畜牧业,制造化肥、矿肥等等的工人人数会增加。再其次,生产工业原料的土地耕种者同生产食物的土地耕种者相比会增加;为家畜生产饲料的人数同为人生产粮食的人数相比会增加。\textbf{不变资本不断增加,从事不变资本再生产的总劳动的相对量也就不断增加。}直接生产生活资料的那部分工人,虽然人数减少了,可是现在生产出来[368]的产品比以前多。他们的劳动的生产能力更大了。\textbf{在个别资本中,同资本不变部分相比资本可变部分的减少},直接表现为花在工资上的那部分资本的减少,\textbf{同样},从资本的总量来说,——在资本\textbf{再生产}时,——可变资本所占份额的减少,必定表现为所使用的工人总数中相对地有更大的部分从事生产资料的再生产,也就是说,从事机器设备(包括交通运输工具,以及建筑物)、辅助材料(煤炭、煤气、机油、传动皮带等)和充当工业品原料的植物的再生产,而不从事产品本身的再生产。农业工人的人数同工业工人的人数相比会减少。最后,从事奢侈品生产的工人人数会增加,因为收入已经提高,现在会消费更多的奢侈品。\fontbox{\}~}

\centerbox{※     ※     ※}

\fontbox{~\{}可变资本转化为收入:第一,转化为工资;第二,转化为利润。因此,如果把资本同收入对立起来理解,不变资本就表现为\textbf{本来意义上的}资本,表现为总产品中属于生产并加入生产费用,而不被任何个人消费的部分(役畜例外)。即使在个别场合这一部分完全是从利润和工资产生的。归根到底,这一部分决不可能只从这个源泉产生;它是劳动的产品,但它是这样一种劳动的产品,这种劳动把生产工具本身当作收入,就象野蛮人把自己的弓当作收入一样。但是,产品的这一部分一经转化为不变资本,就不再归结为工资和利润,虽然它的再生产也会提供工资和利润。产品中有一定份额属于这一部分。任何后来的产品都是这种过去劳动和现在劳动的产品。现在劳动只有把总产品的某一部分归还给生产,才能继续下去。它必须\textbf{以实物形式}补偿不变资本。如果劳动的生产能力变大了,那末它所补偿的就是相应的产品,而不是产品的价值,因为产品的价值已因此减少。如果劳动的生产能力变小了,那末产品的价值就会提高。在前一种情况下,总产品中需要用来补偿过去劳动的那个相应部分就减少,在后一种情况下,这个部分就增加。在前一种情况下,活劳动的生产能力变大了,在后一种情况下,活劳动的生产能力变小了。\fontbox{\}~}

\fontbox{~\{}原料的改进,也是使\textbf{不变资本}费用降低的一个条件。例如,在同一时间内,用好棉纺纱和用次棉纺纱,纺出的数量就不可能一样,更不用说飞花等等的相对量了。种子等等的质量也具有同样的意义。\fontbox{\}~}

\fontbox{~\{}\textbf{联合化}的例子,即工厂主自己把他原先的不变资本的一部分生产出来,或者他自己使原先作为不变资本从他的生产领域转到别的生产领域去的那些原产品具有进一步的形式(前面已经指出\authornote{见本册第 128—129 页,以及本卷第 3 册第 20 章第 7 节(马克思手稿第 332—334 页)。——编者注},这一切始终只归结为利润的积聚)。\textbf{前者的例子}是纺和织的联合。\textbf{后者的例子}是北明翰市郊的矿主,他们自己承担生产铁的\textbf{全部}过程,而在以前这是由不同的企业主和所有者分别承担的。\fontbox{\}~}

\centerbox{※     ※     ※}

加尼耳接着说:

\begin{quote}“当分工还没有在一切部门中实行时,当构成劳动的、有手艺的人口的一切阶级还没有达到充分发展时,某些工业部门中机器的发明和应用,只会使那些被机器游离出来的资本和工人流向能够利用它们的其他劳动部门。但是十分明显,当一切劳动部门都拥有它们所必需的资本和工人时,能够缩短劳动的任何进一步的改良,任何新的机器,都必定会使劳动人口减少;而因为劳动人口的减少决不会使生产缩减,所以在此之后,社会仍然拥有的那部分产品,就会或者使资本的利润增加,或者使地租增加;因此,采用机器所造成的自然的和必然的结果,就是靠总产品过活的那些雇佣阶级人数减少,靠纯产品过活的那些阶级人数增加。”(同上,第 212 页)[369]“由\textbf{工业进步必然}引起的一国\textbf{人口成分的变化},是现代各国人民繁荣、强大和文明的真正原因。社会下层阶级的人数愈减少,社会就愈不必为这些不幸阶级的贫困、无知、轻信和迷信所不断造成的危险担心;上层阶级的人数愈多,国家所能支配的臣民的人数也就愈多,国家也就愈强盛,开化、理智和文明也就愈能遍于整个人口。”(同上,第 213 页)\end{quote}

\fontbox{~\{}萨伊以下述方式把产品的总价值归结为收入。萨伊在他给李嘉图的《原理》(康斯坦西奥的法译本)第二十六章所加的一个注释中这样说:

\begin{quote}“私人的纯收入,是由他参加生产的……\textbf{那个产品的价值}减去他的费用构成的。但因为他的全部费用是他支付给另一些人的\textbf{收入的各个部分},所以\textbf{产品的总价值是用来支付各项收入的}。一国的总收入是由一国的总产品构成的,也就是由各个生产者之间分配的一国全部产品的总价值构成的。”\endnote{萨伊给李嘉图《政治经济学和赋税原理》第二十六章加的注释,马克思引自加尼耳的著作(第 1 卷第 216 页)。——第 222 页。}\end{quote}

后面这个论点如果这样表述就对了:一国的总收入是由一国的总产品中作为收入在各个生产者之间分配的那一部分构成的,也就是由全部产品中作为收入在各个生产者之间分配的那一部分的总价值构成的,换句话说,它等于扣除了全部产品中补偿每个生产部门的生产资料的那一部分之外的总产品。但是,如果这样表述,萨伊的论点也就自己把自己推翻了。

萨伊接着说:

\begin{quote}“这个价值经过一系列的交换行为之后,可能在它产生的一年内就被完全消费掉,但它仍然是一国的收入,就象一个私人拥有年收入 2 万法郎,即使他在一年内把这些收入全部吃掉,他也仍然拥有 2 万法郎的年收入一样。他的收入不是单单由他的节约构成的。”\end{quote}

他的收入从来不由他的节约构成,虽然这种节约始终由他的收入构成。为了证明一个国家在一年内既可以吃掉自己的资本,又可以吃掉自己的收入,萨伊就拿一个国家和一个不动用自己的资本而在一年内只消费自己收入的私人作比较。如果这个私人在一年内既吃掉自己的资本 20 万法郎,又吃掉自己的收入 2 万法郎,那末,下一年他就没有什么东西可吃了。如果一国的全部资本,从而一国产品的全部总价值,都分解为各项收入,那末萨伊就对了。私人吃掉自己的 2 万法郎收入。他所没有吃掉的他的 20 万法郎资本,由其他私人的收入构成,其中每人都吃掉自己的部分,结果到年终这全部资本就会被吃光。萨伊也许会反驳说:难道这个资本不会在它被消费的同时再生产出来,这样不是就得到补偿了吗\fontbox{?}但是,这个私人所以每年能够再生产出自己的 2 万法郎收入,正是因为他不吃掉自己的 20 万法郎资本。如果别人把这笔资本消费掉,他们也就没有资本来再生产收入了。\fontbox{\}~}

\begin{quote}加尼耳说:“只有\textbf{纯产品}和消费纯产品的人,才构成它的〈国家的〉财富和力量,对它的繁荣、荣誉和强大做出贡献。”(同上,第 218 页)\end{quote}

接着,加尼耳引证萨伊给李嘉图《原理》(康斯坦西奥的译本)第二十六章所加的注释。李嘉图在这里说,如果一国有 1200 万居民,那末,500 万生产工人为 1200 万人劳动,比 700 万生产工人为 1200 万人劳动,对一国的财富更为有利。在前一种情况下,“纯产品”由 700 万非生产人口赖以生活的剩余产品构成,在后一种情况下,“纯产品”由 500 万人赖以生活的剩余产品构成。关于这一点,萨伊指出:

\begin{quote}“这使我们清楚地想起十八世纪经济学家\endnote{“经济学家”是十八世纪下半叶和十九世纪上半叶在法国对重农学派的称呼。——第 38、139、223、411 页。}的学说,他们认为,制造业对于国家的财富毫无贡献,因为\textbf{雇佣阶级}生产多少价值,就消费[370]多少价值,不会为他们的〈经济学家的〉著名的\textbf{纯产品}增添任何东西。”\end{quote}

关于这一点,加尼耳指出(第 219—220 页):

\begin{quote}“经济学家认为,\textbf{工业阶级生产多少价值就消费多少价值},李嘉图先生认为,\textbf{工人的工资不能计算在国家的收入中},很难发现经济学家的看法和李嘉图的理论之间的联系。”\end{quote}

加尼耳在这里也没有抓住问题的实质。经济学家的错误在于,他们把工业劳动者只看作“\textbf{雇佣阶级}”。这是他们和李嘉图不同之处。其次,他们的错误是,认为“\textbf{雇佣阶级}”所生产的只是他们所消费的。和他们相反,李嘉图正确地指出,“纯产品”正是由雇佣工人生产的,而所以会生产纯产品,恰恰是因为他们的消费(即他们的工资)不等于他们的全部劳动时间,只等于他们为生产自己的工资所花的劳动时间;换句话说,因为他们从自己的产品中得到的,只是和他们的必要消费相等的部分,即从他们自己的产品中得到的,只是他们本身的必要消费的等价。经济学家认为,整个工业阶级(业主和工人)都处于他们所说的这种情况。在他们看来,只有地租才是生产出来的产品超过工资的余额。因此他们认为,地租是唯一的财富。而李嘉图说,利润和地租构成这个余额,因而也构成唯一的财富,因此,他尽管同重农学派有分歧,但仍赞同他们的这种见解:只有“纯产品”,即体现剩余价值的那个产品,才构成一国的财富(虽然李嘉图比重农学派更清楚地理解这种剩余价值的性质)。而且他还认为,只有作为超过工资的余额的那部分收入才是财富。他和经济学家不同之处,不是对“纯产品”的解释,而是对工资的解释,经济学家错误地把利润也归入工资的范畴。

萨伊接着反驳李嘉图说:

\begin{quote}“700 万完全就业的工人比 500 万人会有更大的节约。”\end{quote}

针对这一点,加尼耳正确地指出:

\begin{quote}“这就是认为,\textbf{从工资中节约比通过停止支付工资来节约}可取”……“说什么给不生产任何纯产品的工人支付 4 亿法郎工资,只是为了使他们得到从自己的工资中节约的机会和手段,这真是太荒谬了。”(同上,第 221 页)“随着文明的每一进步,劳动变得比较不繁重,而它的生产能力变大了;注定要从事生产和消费的各阶级的人数在减少,而管理劳动,安抚(!)、宽慰(!)和开导全体居民的各阶级的人数在增加,\textbf{而且愈来愈多,\CJKunderdot{他们占有因劳动费用减少}}、商品丰富和消费品价格低廉而\CJKunderdot{\textbf{产生的全部利益}}。人类沿着这个方向正在升入……由于\textbf{社会下层阶级的人数不断减少和上层阶级的人数不断增加的这种趋势}……市民社会会变得更幸福、更强大”等等。(同上,第 224 页)“如果……在业工人人数是 700 万,那末他们的工资就是 14 亿法郎;但如果这 14 亿法郎……不能比 500 万工人得到的 10 亿法郎提供更多的纯产品,那末\textbf{真正的节约,就在于停止向那些不提供任何纯产品的 200 万工人支付 4 亿法郎工资},而决不在于这 200 万工人能够从自己的 4 亿法郎工资中节约。”(第 221 页)\end{quote}

李嘉图在第二十六章中指出:

\begin{quote}“\textbf{亚当·斯密}经常夸大一个国家从大量总收入中得到的利益,而不是从大量纯收入中得到的利益……如果一个国家无论使用多少劳动量,它的纯地租和纯利润加在一起始终是那么多,那末,使用大量生产劳动对于该国又有什么好处呢\fontbox{?}……一个国家无论是使用 500 万还是 700 万生产工人[371]来生产其余 500 万人赖以生活的纯收入……纯收入仍然是 500 万人的食物和衣着。使用更多的人既不能使我们的陆海军增加一名士兵,也不能使赋税多收一个基尼。”(同上,第 215 页)\endnote{马克思指加尼耳著作第一卷的一页。马克思这里从该页摘引了李嘉图《政治经济学和赋税原理》第二十六章的若干片断(康斯坦西奥的法译文)。后来在他的手稿第 377 页又引证了李嘉图《原理》中的这些话,但已经是英文(引自英文第 3 版),而且引得更完全(见本册第 228—229 页)。——第 225 页。}\end{quote}

这使我们想起了古日耳曼人,他们轮换着由一部分居民去打仗,另一部分居民种地。需要留下来种地的人愈少,能够去打仗的人就愈多。如果以前种地需要 500 人,现在需要 1000 人,那末,人口即使增加 1/3,例如从 1000 人增加到 1500 人,对这些日耳曼人也毫无好处。他们所能拥有的军队人数仍然只有 500 人。相反,如果他们的劳动生产力增长了,耕地只需要 250 人就够了,那末,在 1000 人中就有 750 人可以去打仗,而在劳动生产率减低时,1500 人中也只有 500 人可以去打仗。

这里首先应当指出,李嘉图所说的“纯收入”或“纯产品”,并不是指全部产品中超过必须作为生产资料(原料或工具)归还给生产的那一部分的余额。相反,他赞同把总产品归结为总收入的这种错误见解。他所说的“纯产品”或“纯收入”,是指剩余价值,指总收入中超过由工资即由工人收入构成的那一部分的余额。而这种工人收入等于可变资本,等于流动资本中由工人不断消费和不断再生产出来的那一部分,即工人的产品中由工人本身消费的部分。

既然李嘉图并不认为资本家是完全无用的人,就是说,既然他把资本家本人也看作生产当事人,因而把他们的一部分利润归结为工资,那末,他必定把他们的一部分收入从“纯收入”中扣除,并且宣布,所有这些人,只有当他们的工资在他们的利润中构成尽量小的部分时,才会促进财富的增加。但无论如何,这些人作为生产当事人,他们的时间至少有一部分是生产本身不可分割的部分。在他们的时间属于生产的情况下,他们不可能被用于社会或国家的其他目的。他们除了从事生产领导者的业务以外剩下的空闲时间愈多,他们的利润就愈不取决于他们的工资。和他们相反,那些只靠利息过活的资本家,还有那些靠地租过活的人,却完全可能被用于社会和国家的目的,而且他们的收入除了用来再生产他们贵体的那一部分之外,任何一部分也不加入生产费用。这样看来,为了\textbf{国家}的利益,李嘉图必定也希望靠减少利润来增加地租(最纯粹的“纯收入”)了;但李嘉图的观点决不是这样。为什么呢\fontbox{?}因为这有害于资本的积累,或者在某种程度上也可以说,因为这是靠损害生产劳动者的利益来增加非生产劳动者。

李嘉图完全赞同斯密对生产劳动和非生产劳动的下面这种区分,即生产劳动直接同资本交换,而[非生产劳动]直接同收入交换。但李嘉图已经没有斯密那种对生产工人的温情和幻想了。成为生产工人,这是一种不幸。生产工人是生产\textbf{别人的}财富的工人。只有在他充当生产别人财富的生产工具时,他的生存才有意义。因此,如果同量的别人财富能够由较少的生产工人创造出来,那末把多余的生产工人解雇,是完全恰当的。你们,不是为了你们。\endnote{你们,不是为了你们(Vos,nonvobis),出自味吉尔的《警言诗》:“你们,鸟儿们,作巢不是为了你们自己;你们,绵羊们,蓄毛不是为了你们自己;你们,蜜蜂们,酿蜜不是为了你们自己;你们,犍牛们,拉犁不是为了你们自己。”——第 227 页。}不过,李嘉图对这种\textbf{解雇}的理解,并不象加尼耳那样;他并不认为单是解雇这些工人就可以增加收入,就可以\textbf{把以前当作可变资本}(即以工资的形式)消费的\textbf{东西当作收入}来消费。随着生产工人人数的减少,这些被排挤的工人本身所消费的和本身所生产的产品量也会消失,这些工人的等价也会消失。李嘉图并不象加尼耳那样认为仍然会有同量的产品生产出来。不过“纯产品”的量会依然不变。如果工人消费 200,他们所生产的剩余产品是 100,那末总产品就是 300,而剩余产品等于总产品的 1/3。如果工人消费 100,他们所生产的剩余产品照旧等于 100,那末总产品就是 200,而剩余产品就等于总产品的 1/2。这样,总产品就会减少 1/3,即 100 个被解雇的工人过去所消费的那个产品量,而“纯[372]产品”仍然\textbf{不变},因为 200/2 等于 300/3。因此,李嘉图对于总产品的量是漠不关心的,只要总产品中构成“纯产品”的那部分保持不变或者增加了,至少不减少就行。

他这样说:\endnote{马克思在这里引用的李嘉图《政治经济学和赋税原理》第二十六章的话,起先用康斯坦西奥的法译文(引自加尼耳的著作第 1 卷第 214 页),后来用英文原文(引自第 3 版第 416 页)。——第 227 页。}

\begin{quote}“对于一个拥有 2 万镑资本,每年获得利润 2000 镑的人来说,只要他的利润不低于 2000 镑,不管他的资本是雇 100 个工人还是雇 1000 个工人,不管生产的商品是卖 1 万镑还是卖 2 万镑,都是一样的。一个国家的实际利益不也是这样吗\fontbox{?}”\endnote{接着,手稿上用铅笔划去了四页半(第 372—376 页),在这几页马克思详细分析了李嘉图所举的“一个拥有 2 万镑资本的人”的例子中包含的数字材料。马克思指出,这些数字材料是不合情理的。在一种情况下,2 万镑资本的所有者使用 100 工人和按 1 万镑出卖生产出来的商品。在另一种情况下,他使用 1000 工人和按 2 万镑出卖生产出来的商品。李嘉图断言,在这两种情况下 2 万镑资本的利润会是相等的,即都是 2000 镑。马克思作了仔细的计算,这些计算表明在上述前提下这种结果是不可能的。因此,马克思提出了以下这个一般的原理:“例解中的前提不得自相矛盾。提出的前提必须是现实的前提,现实的假设,而不是预先作出的荒谬之谈,也不是假设的非现实性和不可能性。”(第 373 页)李嘉图著作所举的例子所以不能令人满意,还在于这个例子仅仅指明使用的工人人数,而没有指明在两种情况下所生产的总产品的数量。马克思在手稿这个划去的地方的结尾(第 376 页)写道:“这一计算必须停止。没有理由把时间浪费在摆弄李嘉图的这些废话上。”——第 227 页。}[VIII—372]\end{quote}

\centerbox{※     ※     ※}

[IX—377]前面从李嘉图的《原理》第二十六章引证的一段话是:

\begin{quote}“亚当·斯密经常夸大一个国家从大量总收入中得到的利益,而不是从大量纯收入中得到的利益〈因为亚当说:“资本所推动的生产劳动量就愈大”〉……如果一个国家无论使用多少劳动量,它的纯地租和纯利润加在一起始终是那么多,那末,使用大量生产劳动对于该国又有什么好处呢\fontbox{?}”……\end{quote}

\fontbox{~\{}可见,这只不过是说,较大量劳动所生产的剩余价值和较少量劳动所生产的剩余价值一样多。这又只不过是说,在剩余价值率较小时使用大量工人,或者在剩余价值率较大时使用较少量工人,对于一个国家来说是一样的。n×1/2 等于 2n×1/4,这里 n 表示工人人数,1/2 和 1/4 表示剩余劳动。“生产工人”本身只不过是创造剩余价值的生产工具,只要结果一样,这种“生产工人”较多就只会是一个累赘。\fontbox{\}~}

\begin{quote}……“对于一个拥有 2 万镑资本,每年获得利润 2000 镑的人来说,只要他的利润不低于 2000 镑,不管他的资本是雇 100 个工人还是雇 1000 个工人,不管生产的商品是卖 1 万镑还是卖 2 万镑,都是一样的。”\end{quote}

\fontbox{~\{}这句话同下面一段话所表明的一样,具有十分庸俗的意味。例如,一个酒商投资 2 万镑,每年把价值 12000 镑的酒放在酒窖里,把价值 8000 镑的酒拿去卖得 1 万镑,这个酒商使用少量的人而得到 10\%的利润。如此等等。还有银行家呢!\fontbox{\}~}

\begin{quote}“一个国家的实际利益不也是这样吗\fontbox{?}\textbf{只要这个国家的实际纯收入,它的地租和利润不变,这个国家的人口有 1000 万还是有 1200 万,都是无关紧要的}。一国维持海陆军\textbf{以及各种非生产劳动}的能力\end{quote}

(这段话表明,李嘉图赞同亚·斯密关于生产劳动和非生产劳动的见解,尽管他已经不赞同后者对生产工人的那种基于幻想的温情了),

\begin{quote}必须同它的纯收入成比例,而不同它的总收入成比例。如果 500 万人能够生产 1000 万人所必需的食物和衣着,那末 500 万人的食物和衣着便是纯收入。假如\textbf{生产同量的纯收入}需要 700 万人,也就是说,要用 700 万人来生产足够 1200 万人用的食物和衣着,那对于国家又有什么好处呢\fontbox{?}纯收入仍然是 500 万人的食物和衣着。使用更多的人既不能使我们的陆海军增加一名士兵,也不能使赋税多收一个基尼。”(\textbf{李嘉图}《政治经济学和赋税原理》1821 年伦敦第 3 版第 415—417 页)\end{quote}

同总产品\textbf{相比},一个国家的生产人口愈少,国家就愈富;对于单个资本家来说也完全是这样,为了生产同量的剩余价值,他必须使用的工人愈少愈好。在产品量相同的情况下,同非生产人口相比,一个国家的生产人口愈少,国家就愈富。因为生产人口相对的少,不过是劳动生产率相对的高的另一种表现。

一方面,资本的趋势是把生产商品所必要的劳动时间,因而也就是把同产品量相比的生产人口的人数,减到愈来愈小的最低限度。另一方面则相反,资本主义生产方式的趋势是积累,把利润转化为资本,尽量占有更多的别人劳动。资本主义生产方式力求降低必要劳动的比率,但在这个比率已定时,就要尽量使用更多的生产劳动。在这方面,产品同人口之比是无关紧要的。谷物和棉花可以同酒、钻石等等相交换,[378]或者,工人可以被使用来进行不直接往消费品上添加任何东西的那种生产劳动(例如修建铁路等等)。

如果由于某项发明,资本家向自己企业投资可以不是原来的 2 万镑,而只是 1 万镑(因为有这 1 万镑已经足够了),如果这 1 万镑提供的利润不是以前的 10\%,而是 20\%,也就是说,和以前 2 万镑提供的利润一样多;那末,这并不能使资本家因此就把 1 万镑当作收入来花费,而不象以前那样把它当作资本来使用。(只有讲到国债,才谈得上资本直接转化为收入。)他会把它投入别的企业;此外,他还会把自己的一部分利润化为资本。

在政治经济学家们(从某方面说,李嘉图也包括在内)的著作中,我们看到了现实本身存在的二律背反。机器排挤劳动和增加“纯收入”(特别是,它们始终会增加李嘉图在这里称为“纯收入”的东西,即收入借以被消费的那些产品的量);机器减少工人人数和增加产品量(这些产品现在一部分由非生产劳动者消费,一部分在国外交换等等)。这种情况似乎是合乎心愿了。然而,不。应当证明,它们,这些机器,不会使工人找不到饭吃。怎样证明这一点呢\fontbox{?}这样证明:机器经过某种动荡之后(对于这种动荡,恰受其害的那个阶层的居民也许无力反抗),又会比采用机器前使用更多的人,结果“生产工人”的数量又增加起来,以前的不合比例的现象重新出现。

实际上也有这样的情况。因此,尽管劳动生产率不断增长,工人人口还是会不断增加,这不是同产品相比来说的增加(产品是和人口一道增加的,而且比人口增加得更快),而是同全部人口相比来说的增加,例如,如果同时发生资本积聚,从而生产阶级原来的一些组成部分沦为无产阶级。无产阶级的一小部分上升为中等阶级。但是非生产阶级所注意的是不使无产阶级得到过多的生存资料。利润的不断再转化为资本,会在愈来愈广泛的基础上使同样的循环不断再现。

但在李嘉图那里,对积累的关心,比对纯利润的关心更多,他热心地把纯利润称颂为积累的手段。由此也就产生了一面告诫工人,一面又安慰工人的矛盾现象。说什么资本积累对工人的利害关系最大,因为对工人的需求取决于资本积累。如果对工人的需求增加了,劳动的价格也就会提高。因而,他们自己应当愿意降低工资,以便从他们那里夺去的剩余价值,经过资本过滤,再被用来购买他们的新劳动,并提高他们的工资。不过,提高工资是有害的,因为这会阻碍积累。一方面,工人们不应当生育子女。这样,劳动的供给就会减少,也就是说,劳动的价格会提高。但是,劳动价格的提高又会减低积累率,从而减少对工人的需求,并降低劳动的价格。随着劳动的供给的减少,资本[积累]会更迅速地减少。如果工人生育子女,那末他们就会使劳动的供给增加,并使劳动的价格降低,结果利润率就提高,资本积累也随之增加。但工人人口应当和资本积累步调一致;也就是说,必须使工人人口正好同资本家所要求的一样多,——实际上本来也就是这样的。

加尼耳先生在崇拜“纯产品”方面,并不是始终一贯的。他引用萨伊的话:

\begin{quote}“我毫不怀疑,在奴隶劳动的条件下,产品超过消费的余额,比在自由人劳动的条件下更大……奴隶的劳动除了奴隶的体力耗尽之外,没有别的界限……奴隶〈自由工人也是一样〉\textbf{为满足无限的需要即主人的贪欲而劳动}。”(\textbf{萨伊}的著作第 1 版第 215—216 页)\end{quote}

[379]关于这一点,加尼耳指出:

\begin{quote}“自由工人不可能比奴隶花费多,而生产少……任何一项费用都必须有一个为支付这项费用而生产出来的等价。如果自由工人比奴隶花费多,那末,他的劳动产品也一定会比奴隶的劳动产品多。”(\textbf{加尼耳}的著作第 1 卷第 234 页)\end{quote}

仿佛工资的大小\textbf{只}取决于劳动者的生产率,而不取决于一定的生产率条件下产品在劳动者和雇主之间的分配。

\begin{quote}加尼耳接着说:“我知道,人们有某些理由可以说,\textbf{主人在奴隶的费用方面实行节约}〈可见,这里仍是“在奴隶的报酬方面的节约”〉,可以增加他个人的费用”等等。……“但是对总财富来说,社会的一切阶级普遍富裕,比少数的个人拥有过多的财富更有利。”(第 234—235 页)\end{quote}

这种话怎么能同“纯产品”协调起来呢\fontbox{?}而且,加尼耳先生马上就收回了他的自由主义言论(同上,第 236—237 页)。他赞同殖民地的黑奴制度。他持自由主义见解,就只限于不主张在欧洲恢复奴隶制度,因为他已经明白,这里的自由工人就是奴隶,他们之所以存在,只是为了给资本家、土地所有者和他们的仆人生产“纯产品”。

\begin{quote}“他〈\textbf{魁奈}〉坚决反驳雇佣阶级的节约能够增加资本这一点;他的理由是,这些阶级不会有节约的可能,如果他们有了\textbf{剩余},有了\textbf{余额},那也只能是由社会经济中的错误或紊乱造成的。”(同上,第 274 页)\end{quote}

为了证实这一点,加尼耳从魁奈那里引用了下面一段话:

\begin{quote}“如果不生产阶级为了增加自己的现金而实行节约……那末,他们的劳动和他们的报酬就会按同一比例减少,这个阶级就会没落下去。”(《重农主义》第 321 页)\end{quote}

蠢驴!他不懂魁奈的意思。

加尼耳先生用下面这句话作为结尾:

\begin{quote}“它〈工资〉愈多,社会的收入就愈少\end{quote}

(雇佣工人是“社会”立足的基础,但他们本身竟不包括在“社会”之中),

\begin{quote}政府的全部艺术都应当用来减少工资的数额……\textbf{我们所生活的文明世纪应该担负的……一项任务}。”(同上,第 2 卷第 24 页)\end{quote}

\centerbox{※     ※     ※}

关于生产劳动和非生产劳动的问题,还要简略地考察一下\textbf{罗德戴尔}(在这之后,布鲁姆平淡无味的玩笑就不值得考察了)、(费里埃\fontbox{?})、\textbf{托克维尔、施托尔希、西尼耳}和\textbf{罗西}的观点。

\tsectionnonum{[(10)]收入和资本的交换[简单再生产条件下年产品总量的补偿:(a)收入同收入的交换;(b)收入同资本的交换;(c)资本同资本的交换]}

\fontbox{~\{}应当区分:(1)\textbf{转化为新资本的}那部分\textbf{收入};也就是重新资本化的那部分利润。这一部分我们在这里完全不去考虑。这是论积累那一篇要谈的问题。(2)同生产中消费了的资本相交换的收入;通过这种交换,不会形成新资本,只会补偿旧资本,一句话,保存旧资本。因此,在这里的研究中,我们可以把转化为新资本的那部分收入看作等于零,情况就好比所有的收入不是抵补了收入,就是抵补了已消费的资本。

这样,年产品总量就分为两部分:一部分作为收入被消费,另一部分以实物形式补偿已消费的不变资本。

收入同收入交换,例如,麻布生产者从代表他们的利润和工资,即代表他们的收入的那部分产品麻布中,拿出一部分来同代表土地耕种者的一部分利润和[380]工资的谷物交换。因此,这里是麻布同谷物的交换,是加入个人消费的两种商品的交换,是麻布形式的收入同谷物形式的收入的交换。这里没有任何困难。只要可供个人消费的产品按照符合需要的比例生产出来,因而,只要生产这些产品所必需的社会劳动的相应量也按比例分配\fontbox{~\{}当然,决不会丝毫不差地按比例;总会有偏离,有不合比例的情况,这种情况本身会得到平衡;但不断的平衡运动本身是以不断的不合比例现象为前提的\fontbox{\}~},那末收入,比如说以麻布形式存在的数量,恰好就是它作为消费品所需要的数量,也就是说,恰好就是它被其他生产者的消费品所补偿的数量。麻布生产者以谷物等等形式消费的东西,由农民等等以麻布形式消费掉。因此,他用来换取其他商品(消费品)的、代表他的收入的那部分产品,被其他商品的生产者作为消费品换去。他以其他产品形式消费的东西,由别人以他的产品形式消费掉。

顺便指出:在一个单位产品上花费的劳动时间不超过社会必要劳动时间,即不超过生产这个商品平均所需要的时间,这是资本主义生产的结果,资本主义生产甚至不断降低必要劳动时间的最低额。但为此资本主义生产必须在不断扩大的规模上进行。

如果 1 码麻布的价值只等于 1 小时,并且这就是社会为满足自己对 1 码麻布的需要所必须花费的必要劳动时间,那末,由此还决不能得出结论说:如果生产 1200 万码麻布,从而花费 1200 万劳动小时,或者同样可以说,花费 100 万工作日,使用 100 万工人来织麻布,那末,社会“必须”花费在麻布织造业上的,就正好是社会劳动时间的这样一个部分。如果必要劳动时间已知,就是说,一日内所能生产的麻布量已知,那还要问,究竟有多少这样的日数必须花费在麻布生产上。例如一年内花费在一定产品总量上的劳动时间等于:这种使用价值的一定量,例如 1 码麻布(假定这个量=1 工作日),乘所花费的总工作日数。虽然产品每一部分包含的只是生产这一部分所必要的劳动时间,或者说,虽然所花费的劳动时间的每一部分都是创造总产品的相应部分所必要的,但是,一定生产部门所花费的劳动时间总量对社会所拥有的全部劳动时间的百分比,仍然可能低于或高于应有的比例。

从这个观点来看,必要劳动时间就有了另外的意义。现在要问:必要劳动时间究竟按怎样的量在不同的生产领域中分配\fontbox{?}竞争不断地调节这种分配,正象它不断地打乱这种分配一样。如果某个部门花费的社会劳动时间量过大,那末,就只能按照应该花费的社会劳动时间量来支付等价。因此,在这种情况下,总产品——即总产品的价值——就不等于它本身所包含的劳动时间,而等于这个领域的总产品同其他领域的产品保持应有的比例时按比例应当花费的劳动时间。但是,这个领域总产品的价格比它的价值降低多少,总产品的每一部分的价格也降低多少。如果原来生产 4000 码麻布,现在生产 6000 码,而 6000 码的价值是 12000 先令,那末它们还会按 8000 先令出卖。每码的价格将是 1+(1/3)先令,而不是 2 先令,即比价值低 1/3。可见,这就好比在每码的生产上比必须花费的劳动时间多花费了 1/3。因此,在商品的使用价值已定时,商品价格降低到商品价值以下的事实证明,虽然花费在产品的每一部分上的只是社会必要劳动时间\fontbox{~\{}这里假定生产条件不变\fontbox{\}~},但花费在整个这一生产部门中的社会劳动总量过多了,超过必要量了。

由生产条件的变化[381]引起的商品相对价值的降低,完全是另外一回事。已经在市场上的这块麻布,过去值 2 先令,假定等于 1 工作日。但是现在,每天能用 1 先令把它再生产出来。因为价值决定于社会必要劳动时间,而不决定于个别生产者要用的劳动时间,所以,生产者生产 1 码要用的 1 日,只等于 1 个社会规定日的一半。他的 1 码麻布的价格从 2 先令降低到 1 先令,即 1 码麻布的价格降低到他在这块麻布上\textbf{花费的}价值以下,这不过表明生产条件发生了变化,也就是表明必要劳动时间本身发生了变化。另一方面,如果麻布的生产费用不变,而所有其他物品,除了金即货币材料以外,生产费用都提高了,或者,只有某些物品如小麦、铜等等,一句话,不加入麻布组成部分的物品的生产费用提高了,那末 1 码麻布就仍然会等于 2 先令。它的\textbf{价格}不会降低,但是它以小麦、铜等等表示的相对价值降低了。

\centerbox{※     ※     ※}

某一生产部门(生产可供个人消费的商品的生产部门)的一部分收入以另一生产部门的收入的形式被消费,关于这一部分收入,可以说,需求同它本身的供给相等(在生产按照应有的\textbf{比例}进行的情况下)。这就好比这些生产部门各自消费了自己的这一部分收入。这里只有形式上的商品形态变化:W—G—W′。麻布—货币—小麦。

这里互相交换的两种商品,只代表一年内新加劳动的一部分。但是,第一,很明显,这种交换——两个生产者各自以对方的商品形式消费自己产品中代表收入的那一部分——只能在生产消费品,即生产直接加入个人消费因而收入能够借以作为收入来花费的物品的那些生产部门中发生。第二,同样很明显,只有就\textbf{这一部分}产品交换来说,生产者的供给等于他对他想要消费的其他产品的需求这种说法,才是正确的。这里涉及的实际上不过是简单商品交换。生产者不是为自己生产生活资料,而是为别人生产生活资料,而别人又为他生产生活资料。这里不包括收入同资本的任何关系。一种消费品形式的收入同另一种消费品形式的收入交换,事实上也就是消费品同消费品交换。它们的交换过程不决定于它们两者都是收入,而决定于它们两者都是消费品。从形式来说它们都是收入,这种情况在这里是毫无关系的。诚然,这种情况在相互交换的商品的使用价值上,在它们两者都加入个人消费这一点上会显露出来,但这也无非是说明,一部分消费品同另一部分消费品交换。

收入的形式只有在资本的形式同它对立的地方,才能表现出来或显露出来。但即使在我们所考察的情况下,萨伊\endnote{马克思指萨伊的下述论断(在他的《给马尔萨斯的信》1820 年巴黎版第 15 页):例如,如果英国商品充斥意大利市场,那末,原因就在于能够同英国商品交换的意大利商品生产不足。萨伊的这些论断在匿名著作《论马尔萨斯先生近来提倡的关于需求的性质和消费的必要性的原理》(1821 年伦敦版第 15 页)中引证过,在马克思的第 VII 本札记本第 12 页对这部著作所作的摘录中也有这些论断。并参看马克思在本册第 276 页分析的萨伊的这一论点:“某些产品的滞销,是由另一些产品太少引起的。”——第 237 页。}和其他庸俗经济学家们的主张也是错误的。他们断言,如果 A 不能把自己的麻布即他自己想要作为收入来消费的那部分麻布卖掉,或者说,如果他只能低于麻布的价格把麻布卖掉,那末,这是由于 B、C 等等生产的小麦、肉等等太少。这可能是由于他们生产的这些东西数量不够。但这也可能是由于 A 生产的麻布太多;因为即使假定 B、C 等等有足够的小麦等等可以用来购买 A 的全部麻布,他们也还是不会把全部麻布买来,因为他们\textbf{消费的}仅仅是一定量的麻布。或者,这还可能是由于 A 生产的麻布在数量上比他们的收入中一般说来能够用在衣料上的那部分还多,因而总的说来是由于每个人都只能把自己的一定量的产品作为收入来花费,而 A 的麻布生产却以一个比实有额大的收入为前提。但可笑的是,在只牵涉到收入同收入交换的地方,就假定需求的对象不是产品的使用价值,而是这个使用价值的量,也就是说,又忘记了在\textbf{这种}交换中涉及的只是需要的满足,而不象在谈交换价值时那样涉及的是量。

然而每个人有某种产品都宁愿多些,而不愿少些!如果这种说法能够解决困难,那[382]就绝对不能理解,为什么麻布生产者不是采取更简单的过程,以多余的麻布形式消费自己的一部分收入,而是用他的麻布交换其他消费品,并把这些东西大量堆积起来。为什么他总是把自己的收入由麻布形式转化为其他形式呢\fontbox{?}因为他除了需要麻布以外,还有别的需要必须满足。为什么他自己只是消费一定部分的麻布呢\fontbox{?}因为只有一定数量的麻布对他有使用价值。不过,这种说法也适用于 B、C 等等。如果 B 卖酒,C 卖书,D 卖镜子,那末他们每一个人也许都宁愿以自己的产品形式,即以酒、书、镜子的形式,而不以麻布形式消费自己多余的收入。所以不能说,如果 A 完全不能(或不能按照价值)把自己的由麻布构成的收入转化为酒、书、镜子,那末这就绝对必然地意味着酒、书、镜子生产得太少。然而更加可笑的是,把这种收入同收入的交换——只是商品交换的一部分——说成是全部的商品交换。

这样,我们就已经把产品的一部分处理了。消费品的一部分在这些消费品的生产者本身之间转手。这些生产者每人都不以自己的产品形式,而以别人的产品形式消费自己收入(利润和工资)的一部分。他所以能够这样做,只是因为别人也不消费自己的产品,而消费他人的可消费的产品。这就好比每个人都把自己的可消费的产品中代表自己收入的那部分消费掉一样。

至于说到产品的所有其余部分,却出现了更为复杂的关系,只有在这里,互相交换的商品才作为收入同资本彼此对立,因而不只是作为收入彼此对立。

首先必须作如下的区分。在一切生产部门中,总产品的一部分代表收入,即(一年内的)新加劳动:利润和工资。\fontbox{~\{}地租、利息等等都是利润的一部分;混蛋官吏的收入是利润和工资的一部分;其他非生产劳动者的收入,是他们用自己的非生产劳动购买的利润和工资的一部分,因此,这种收入不增加以利润和工资的形式存在的产品,它只决定这种产品中有多大一部分由这些非生产劳动者消费,有多大一部分由工人和资本家自己消费。\fontbox{\}~}但是,只有在某些生产领域,代表收入的那部分产品才能够直接以实物形式成为收入的组成部分,就是说,才能够按其\textbf{使用价值}作为收入来消费。所有只代表生产资料的产品都不能以实物形式,以直接的形式,作为收入来消费,而只能按其\textbf{价值}作为收入来消费。但是这个价值必须在那些生产直接消费品的生产部门中消费。有一部分生产资料,根据用途可以充当这种或那种直接消费品,例如马、大车等等。有一部分直接消费品可以充当生产资料,例如用来酿酒的谷物、用作种子的小麦等等。几乎所有消费品本身都可以作为消费的废料重新加入生产过程,例如,用坏了的破烂麻布可以用来造纸。但是,无论谁生产麻布,也不是为了把它作为破布充当造纸的原料。它只有在麻布织造业的产品本身已经加入消费之后,才取得这种形式。它只有作为这个消费的废料,作为消费过程的残余和产品,才能作为生产资料重新加入其他生产领域。因此,这种情况不是这里要讨论的问题。

总之,有一些产品,它们的生产者只能按其价值而不能按其使用价值消费其中代表收入的部分。因而这些生产者为了消费他们的代表工资和利润的那部分产品,例如机器,就必须把这些产品卖掉,因为他们不能用这些机器本身来直接满足任何的个人需要。同样,这些产品也不能由其他产品的生产者消费,不能加入他们的个人消费,从而不能属于他们借以花费自己收入的那些产品之列,因为这同这些商品的使用价值相矛盾,这些商品的使用价值按其性质来说是\textbf{排斥}个人消费的。所以,这些不可直接消费的产品的生产者只能消费产品的\textbf{交换价值};也就是说,他们必须先把产品转化为货币,然后把货币再转化为可直接消费的商品。但他们应当[383]把这些产品卖给谁呢\fontbox{?}卖给其他非个人消费的产品的生产者吗\fontbox{?}在这种情况下,他们不过是得到一种不可直接消费的产品来代替另一种不可直接消费的产品而已。然而我们曾假定,这部分产品构成他们的收入,他们卖掉这些产品,是为了以消费品的形式消费这些产品的价值。因而他们只能把这些产品卖给可供个人消费的产品的生产者。

这一部分商品交换代表一个人的资本同另一个人的收入的交换,或一个人的收入同另一个人的资本的交换。消费品生产者的总产品中,只有一部分代表收入;另一部分代表不变资本。后面这一部分,他既不能自己消费,也不能用来同其他生产者的可直接消费的产品交换。他既不能以实物形式消费自己这部分产品的使用价值,也不能把这部分产品换成其他消费品而消费其价值。相反,他必须把这部分产品再转化为他的不变资本的实物要素。他必须把自己的这部分产品\textbf{用于生产消费},即作为生产资料来消费。但是他的产品按其使用价值来说只能加入个人消费;因此这种产品的生产者不能以实物形式把它再转化为他自己的生产要素。这种产品的使用价值的性质排斥\textbf{生产消费}。因此,这种产品的生产者只能把自己产品的\textbf{价值}用于生产消费,办法是把这种产品卖给它的上述各生产要素的生产者。他不能以实物形式消费自己的这部分产品;他也不能通过同其他个人消费品的交换,来消费这部分产品的价值。他的这部分产品不能加入他自己的收入,同样不能由其他个人消费品的生产者的收入来补偿,因为要能进行这种补偿,只有他用自己的产品去同这些生产者的产品交换,就是说,只有他把自己产品的价值\textbf{吃掉},而按照假定,这样的事情是不会发生的。但是,因为他的这部分产品象他的另一部分作为收入消费的产品一样,按其使用价值来说只能作为收入来消费,必须加入个人消费,而不能补偿不变资本,所以,这部分产品必须加入不可直接消费的产品的生产者的收入,必须用来同这些生产者的产品中能够由他们消费产品价值的或代表他们收入的那一部分相交换。

如果从交换双方分别来考察这种交换,那末,对 A 这个消费品生产者来说,这种交换表现从资本到资本的转化。通过这种交换,生产者 A 把他的总产品中等于它所包含的不变资本价值的那一部分,再转化为能够执行不变资本的职能的实物形式。无论在交换以前,还是在交换以后,这部分产品按其价值来说都只代表不变资本。相反,对 B 这个不可直接消费的产品的生产者来说,这种交换只表现收入从一种形式到另一种形式的转化。在这里,生产者 B 首先把他的总产品中构成他的收入的那一部分,即总产品中代表这个生产领域的新加劳动(必要劳动和剩余劳动)的那一部分,转化为能够作为收入来消费的实物形式。无论在交换以前,还是在交换以后,这部分产品按其价值来说都只代表他的收入。

如果从交换双方同时来考察这种关系,那就是 A 用他的不变资本去交换 B 的收入,而 B 用他的收入去交换 A 的不变资本。B 的收入补偿 A 的不变资本,而 A 的不变资本补偿 B 的收入。

在这种交换中\fontbox{~\{}交换双方所追求的目的撇开不谈\fontbox{\}~},相互对立的只是商品,发生的是简单的商品交换,这些商品只是作为商品彼此发生关系,对于它们来说,“收入”和“资本”的标志是无关紧要的。只有这些商品的\textbf{使用价值}的不同性质表明,一些商品只能用于和加入生产消费,另一些商品只能用于和加入个人消费。但是,各种商品的各种使用价值在用途上的不同,是属于消费范围内的问题,同它们作为商品进行的交换过程毫无关系。当资本家的资本转化为工资,而劳动转化为资本时,情形就完全不同了。这里商品不是作为简单的商品相互对立,而是资本作为资本出现。在刚才考察的交换中,卖者和买者只是作为卖者和买者,只是作为简单的商品所有者相互对立。

其次,很明显:凡是只用于个人消费的产品,或者说,凡是加入个人消费的产品,在它加入这种消费的范围内,都只能同收入交换。它不能用于生产消费,这一点正好说明,它只能作为收入来消费,即只能用于个人消费。\fontbox{~\{}前面已经指出,我们在这里撇开利润转化为资本的情况不谈。\fontbox{\}~}

假定 A 是某种只用于个人消费的产品的生产者,他的收入等于他的总产品的 1/3,他的不变资本等于总产品的 2/3。按照假定,前 1/3 由他自己消费,不管他[384]是全部还是部分以实物形式消费它,还是完全不以实物形式消费它,而以其他消费品形式消费它的价值;在后面这种情况下,其他消费品的卖者就是以 A 的产品形式消费自己的收入。由此可见,各种消费品中代表自己的生产者的收入的那一部分,或者直接地由生产者消费,或者间接地,通过生产者所需要的消费品的相互交换,由生产者消费。所以,就这一部分来说,发生的是\textbf{收入同收入的交换}。这里的情形就好比 A 代表所有消费品的生产者一样。这些产品总量的 1/3,即代表他的收入的那部分,由他自己消费。但是,这一部分正好代表 A 部类在一年内加到自己的不变资本上的劳动量,而这个量等于 A 部类在一年内生产的工资和利润的总额。

A 部类总产品的其余 2/3 等于不变资本的价值;因而它们必须由 B 部类的年劳动产品来补偿,B 部类提供的是非个人消费的、只加入生产消费、作为生产资料加入生产过程的产品。但是,因为 A 的总产品的这 2/3,象前面那 1/3 一样,必须加入个人消费,所以它们要由 B 部类的生产者用代表他们收入的那部分产品来交换。这样,A 部类就用自己总产品的不变部分换得具有这个不变部分的原来实物形式的产品,即把这个不变部分再转化为 B 部类新创造的产品;而 B 部类用来支付的,只是代表它的收入的那部分产品,但它自己又只能以 A 部类的产品形式消费这一部分产品。因而,B 部类事实上用来支付的是它的新加劳动,这种新加劳动全部表现为 B 用来同产品 A 的后 2/3 交换的那部分产品。这样,全部产品 A 就同收入交换,或者说,全部加入个人消费。另一方面(因为按照假定,对于收入转化为资本在这里不加考察,可以把它看作等于零),社会的\textbf{全部收入}都花费在 A 的产品上;因为 A 的生产者以产品 A 形式消费自己的收入,B 部类的生产者也以产品 A 形式消费自己的收入。而除了这两个部类之外,再也没有别的部类了。

A 的产品全部被消费,虽然这种产品的 2/3 包含不变资本,不能由 A 的生产者消费,而必须再转化为这种产品的各生产要素的实物形式。A 的总产品等于社会的总收入。而社会的总收入代表社会在一年内加到现有不变资本上的劳动时间的总和。这样,虽然 A 的总产品只有 1/3 由新加劳动构成,2/3 则由过去的待补偿的劳动构成,但它仍然能够全部用新加劳动来购买,因为这全部年劳动的 2/3 不能以它本身的产品形式来消费,必须以产品 A 形式来消费。用来补偿产品 A 的新加劳动,比产品 A 本身包含的新加劳动多 2/3,因为这 2/3 是 B 的新加劳动,而 B 只能把这 2/3 用于个人消费,即以产品 A 形式消费,正如 A 只能把这 2/3 用于生产消费,即以产品 B 形式消费一样。可见,第一,A 的总产品能够全部作为收入消费掉,同时,它的不变资本也能够得到补偿。或者更确切地说,A 的总产品所以能够全部作为收入来消费,只是因为它的 2/3 由不变资本的生产者补偿,这些生产者不能以实物形式消费他们那部分代表他们收入的产品,而必须以产品 A 形式,就是说,通过同 A 的 2/3 交换来消费它。

这样,我们就把 A 的后 2/3 处理了。

很明显,如果有第三部类 C,它的产品既能用于生产消费,又能用于个人消费,例如谷物可以充当人的食物或牲畜的饲料,也可以用来做种子或烤面包,又如大车、马、牲畜等等;那末,这丝毫也不会使问题有所改变。就这些产品加入个人消费的那部分来说,它们必须由它们自己的生产者,或者由它们所包含的那部分不变资本的(直接或间接的)生产者,作为收入直接或间接地消费掉。因而在这种情况下,它们属于 A 部类。就这些产品不加入个人消费的那部分来说,它们属于 B 部类。

在这第二种交换过程中,不是收入同收入交换,而是资本同收入交换;在这里,全部不变资本归根到底必须归结为收入,因而归结为新加劳动。这种交换过程可以从两方面来看。假定 A 的产品是麻布。和 A 的不变资本相等的 2/3 麻布(或这些麻布的价值)用来支付纱、机器、辅助材料。但纺纱厂主和机器厂主[385]只能消费这一产品中代表他们自己收入的那一部分。麻织厂主用自己的 2/3 产品支付纱和机器的全部价格。这样他就补偿了纺纱业者和机器制造业者作为不变资本加入麻布的全部产品。但纺纱业者和机器制造业者的这个总产品本身,又等于不变资本和收入,等于这样两部分的总和:一部分是纺纱业者和机器制造业者的新加劳动,另一部分代表他们自己的生产资料的价值,即从纺纱业者来说是亚麻、机油、机器、煤炭等等的价值,从机器制造业者来说是煤炭、铁、机器等等的价值。这样,和 A 的不变资本相等的 2/3 麻布,就补偿了纺纱业者和机器制造业者的总产品,补偿了他们的不变资本加他们的新加劳动,他们的资本加他们的收入。但纺纱业者和机器制造业者只能以产品 A 形式消费自己的收入。他们从 A 的 2/3 中扣除等于他们的收入的那部分之后,便用余额来支付自己的原料和机器。但是按照假定,这些原料和机器的生产者已经不必再补偿任何不变资本了。他们的产品中能够加入产品 A 的数量,即加入充当 A 的生产资料的产品的数量,只是同 A 所能支付的数量一样多。但是,A 用自己产品的 2/3 能够支付的数量,只是同 B 用自己的收入能够购买的数量一样多,也就是说,同 B 换来的产品所代表的收入,所代表的新加劳动一样多。如果 A 的最后一些生产要素的生产者必须卖给纺纱业者[和机器制造业者]的那个产品数量,还代表着他们自己的一部分不变资本,即代表着比他们加到自己的不变资本上的劳动更多的东西,那末,他们就不能以产品 A 形式得到支付,因为这一部分产品是他们不能消费的。因而这里发生的是相反的情况。

现在我们按反过来的顺序来看一看。假定全部麻布等于 12 日。亚麻种植业者、制铁业者等等的产品等于 4 日;这个产品卖给纺纱业者和机器制造业者,他们又给这个产品加上 4 日;然后他们把产品卖给织布业者,织布业者又加上 4 日。麻布织造业者可以自己消费自己产品的 1/3;8 日用来补偿他的不变资本,支付纺纱业者和机器制造业者的产品;纺纱业者和机器制造业者在 8 日中可以消费 4 日,把其余的 4 日支付给亚麻种植业者等等,以此来补偿自己的不变资本;亚麻种植业者、制铁业者等等应当用物化在麻布中的最后 4 日只补偿自己的劳动。

虽然收入在这三种场合假定都是一样的(等于 4 日),但它在参与生产产品 A 的三类生产者的产品中,却占不同的比例。在织布业者那里,它占产品的 1/3(12 的 1/3),在纺纱业者和机器制造业者那里,它占产品的 1/2(8 的 1/2),在亚麻种植业者那里,它和产品相等,等于 4 日。对总产品来说,所有这些生产者的收入都是完全一样的:都等于 12 的 1/3,即 4 日。但是在织布业者那里,纺纱业者、机器制造业者和亚麻种植业者的新加劳动表现为不变资本;在纺纱业者和机器制造业者那里,他们自己的和亚麻种植业者的新加劳动表现为总产品,而亚麻种植业者的劳动时间表现为不变资本。在亚麻种植业者那里,不变资本的现象就不存在了。因此,例如纺纱业者,能够同织布业者按一样的比例使用机器,使用不变资本。例如,比例都是 1/3∶2/3。但是第一,纺纱业使用的资本量(总额)必定比织布业使用的资本量小,因为纺纱业的全部产品都作为不变资本加入织布业。第二,如果在纺纱业者那里,比例正好也是 1/3∶2/3,那末,他的不变资本就将等于 16/3,他的新加劳动将等于 8/3;不变资本将等于 5+(1/3)工作日,新加劳动将等于 2+(2/3)工作日。这样,向他提供亚麻等等的那个部门就将包含相对地更多的工作日。因此,在这里他就要用 5+(1/3)日,而不是用 4 日来支付新加劳动时间了。

不言而喻,A 部类的不变资本中,只有在 A 那里加入价值形成过程的那部分,即在这个 A 的劳动过程中被消费的那部分,才必须由新劳动来补偿。原料、辅助材料和固定资本的损耗全部加入价值形成过程。固定资本的其余部分不加入这个过程,因此不需要补偿。

可见,现有不变资本的很大一部分(其大小决定于固定资本同总资本之比),不需要每年由新劳动补偿。因此,虽然[每年被补偿的固定资本价值]量可能很大(绝对数字),但同总产品(年产品)相比,仍然是不大的。A 部类和 B 部类中\textbf{不变资本的}上述\textbf{整个部分}(在剩余价值已知时)都参加决定利润率,但不参加决定固定资本的实际再生产。同总资本相比,这个部分愈大,即在生产中使用现有的已存在的固定资本的规模愈大,则用来补偿损耗的固定资本的\textbf{再生产的实际\CJKunderdot{量}}也就愈大;但\textbf{同}总资本\textbf{相比},这种再生产的量可能相对地愈小。

假设各种固定资本的再生产时间(\textbf{平均})等于 10 年。[386]假定各种固定资本周转一次各需 20、17、15、12、11、10、8、6、4、3、2、1、4/6 和 2/6 年(共 14 种),那末,固定资本就\textbf{平均}10 年\endnote{马克思用整数 10,为的是不使以后的计算复杂化。如果按照正文中采用的数字(14 种固定资本周转时间的总数为 110 年),对固定资本的平均周转时间进行准确的计算(假定所有 14 种固定资本的数目一样多),那末得出的就不是 10 年,而只是 7.86 年。——第 247 页。}周转一次。

因此,固定资本平均应当在 10 年内得到补偿。如果全部固定资本占总资本的 1/10,每年要补偿的 1/10 固定资本,就只占总资本的 1/100。

如果固定资本占总资本的 1/3,每年就要补偿总资本的 1/30。

但我们现在拿再生产时期不同的两个固定资本加以比较,例如一个固定资本的再生产需要 20 年,另一个资本需要 1/3 年。

那个在 20 年内再生产出来的固定资本,每年只要补偿 1/20。因此,如果它占总资本的 1/2,每年就只要补偿总资本的 1/40,即使它占总资本的 4/5,每年要补偿的也只是总资本的 4/100,即 1/25。相反,那个需要 2/6 年再生产出来即一年周转三次的固定资本,如果只占资本的 1/10,那末固定资本每年就要补偿三次;因而每年就要补偿总资本的 3/10,即几乎是 1/3。平均说来,同总资本相比,固定资本愈大,它的\textbf{相对的}(不是绝对的)再生产时间就愈长,固定资本愈小,它的\textbf{相对的}再生产时间就愈短。在手工生产条件下工具占资本的部分,比在机器生产条件下机器占资本的部分小得多。但是手工业工具的损耗,比机器的损耗快得多。

虽然随着固定资本的绝对量的增加,它的再生产的绝对量或它的损耗也会增加,但是它的再生产的相对量在大多数情况下会减少,因为固定资本的周转时间,它的存在时间,在大多数情况下会同它的规模成比例地增加。这也就证明,再生产机器即固定资本的劳动量,决不会同原先生产这些机器的劳动量(在生产条件不变的情况下)成比例,因为需要补偿的只是每年的损耗。如果象这个部门常常发生的那样,劳动生产率增长了,那末,再生产这部分不变资本所必需的劳动量就减少得更多。诚然,这里还应当算上充当机器每日消费资料的那些东西(不过这些东西同机器制造本身所花费的劳动毫无直接关系)。但机器只消费煤炭和少量的机油或油脂,它的维持费比工人——不但比它所代替的工人,而且比把它本身制造出来的工人——的生活费不知要低多少。

\centerbox{※     ※     ※}

这样,我们就把整个 A 部类的产品和 B 部类的一部分产品处理了。产品 A 全部被消费:1/3 由它自己的生产者消费;2/3 由 B 的生产者消费,B 的生产者不能以自己的产品形式消费自己的收入。B 的生产者以 2/3 的产品 A 形式消费自己产品中代表收入的那部分价值,这 2/3 同时以实物形式补偿 A 的生产者的不变资本,即为他们提供\textbf{用于生产消费}的那些商品。但是,随着产品 A 全部被吃掉以及 A 的 2/3 由产品 B 作为不变资本补偿,全部年产品中代表一年内新加劳动的那一部分,也就\textbf{全部}处理了。因而这个劳动不能购买总产品的任何其他部分了。事实上,一年内全部新加劳动(撇开利润的资本化不谈)就等于 A\textbf{包含的劳动}。因为 A 的 1/3,即由它自己的生产者消费的部分,代表他们在一年内加到 A 的 2/3 上,即加到构成 A 部类不变资本的那部分产品上的新加劳动。除了他们以自己的产品形式消费的这种劳动以外,他们没有进行任何其他的劳动。A 的其余 2/3,即由 B 部类的产品补偿并由产品 B 的生产者消费的部分,代表 B 的生产者加到自己的不变资本上的全部劳动时间。他们没有加入任何更多的劳动,他们也没有更多的东西可消费。[387]

产品 A 按其\textbf{使用价值}来说,代表全部年产品中每年加入个人消费的整个部分。按其\textbf{交换价值}来说,它代表生产者在一年内新加的劳动总量。

但是除了这一切之外,我们还剩下作为\textbf{余额}的总产品的第三部分,它的组成部分在交换时既不能代表收入同收入的交换,也不能代表资本同收入或收入同资本的交换。这就是产品 B 中代表 B 的不变资本的那一部分。这部分不加入 B 的收入;所以它不能由产品 A 补偿,或者说,不能同产品 A 交换,因而也不能作为组成部分加入 A 的不变资本。既然这一部分在 B 部类中不但加入劳动过程,而且加入价值形成过程,那末这一部分也要被消费掉,被用于生产消费。因此,这一部分也完全象总产品的所有其他部分一样,必须\textbf{按照它形成总产品组成部分的比例}得到补偿,而且必须由同类的\textbf{新}产品以实物形式补偿。另一方面,它又不是由任何新劳动补偿。因为新加劳动的总量等于产品 A 所包含的劳动时间,并且这种劳动时间所以全部得到补偿,只是因为 B 以 2/3 的产品 A 形式消费自己的收入,并在交换过程中给 A 提供生产资料,来代替 A 领域中消费了的、待补偿的一切东西。而产品 A 的前 1/3,即由它自己的生产者消费的部分,按其交换价值来说,只由他们本身的新加劳动构成,不包含任何不变资本。

我们就来看看这个余额。

它由以下不变资本构成:第一,加入原料的不变资本;第二,加入固定资本形成过程的不变资本,第三,加入辅助材料的不变资本。

\textbf{第一,原料}。首先,花费在原料生产上的不变资本,归结为固定资本,如机器、工具、建筑物,以及作为机器的消费资料的那些辅助材料。对于可直接消费的那部分原料(如牲畜、谷物、葡萄等等)来说,不会发生上述困难。从这方面来说,它们属于 A 类。它们所包含的那部分不变资本加入 A 的 2/3 即不变部分,这个部分作为资本同不可直接消费的产品 B 交换,或者说,B 以这个部分的形式消费自己的收入。一些不管在生产过程中通过多少中间阶段,都以实物形式加入消费品的不可直接消费的原料,一般也是这样的情况。先转化为纱、然后转化为麻布的那部分亚麻,就全部加入消费品。

但是一部分这样的\textbf{有机原料},如木材、亚麻、大麻、皮革等,它们一部分直接加入固定资本的构成要素,一部分加入固定资本的辅助材料。例如以机油、油脂等等形式。

\textbf{其次,种子}也属于花费在原料生产上的不变资本。植物性的物质和动物性的物质自己再生产自己:植物蕃殖和动物生殖。种子应当是指本来意义上的种子,其次是指作为厩肥再投到土地中去的牲畜饲料,以及种畜等等。年产品中——或年产品的不变部分中——这很大的一部分,直接充当再生自己的材料,它们自己再生产自己。

\textbf{无机原料}——金属、石头等等。这种原料的价值只由两部分构成,因为这里没有在农业中代表原料的种子。无机原料的价值,只由新加的劳动和消费掉的机器(包括机器的消费资料)构成。因此,除了代表新加劳动、因而加入 B 和 2/3A 之间的交换的那部分产品之外,需要补偿的只是固定资本的损耗和固定资本的消费资料(如煤炭、机油等等)。但是这种无机原料构成不变资本的主要组成部分——固定资本(机器、劳动工具、建筑物等等)。所以,无机原料通过[资本同资本的]交换以实物形式补偿自己的不变资本。

[388]\textbf{第二,固定资本(机器、建筑物、劳动工具、各种器皿)}。

它们的不变资本由以下各部分构成:(1)它们的原料,金属、石头、有机原料(如木材、皮带、绳索等等)。它们的这些原料形成它们(机器、工具、建筑物等等)的原料,而它们自己又作为劳动工具加入这种原料的采制过程。因此,它们以实物形式彼此补偿。制铁业者必须补偿机器,机器制造业者必须补偿铁。在采石场中有机器的损耗,而在工厂建筑物中有建筑石材的损耗,等等。(2)\textbf{机器制造机的损耗}。这些机器制造机必须在一定时期内由同种新产品补偿。同种产品自然可以自己补偿自己。(3)\textbf{机器的消费资料}(辅助材料)。机器消费煤炭,但煤炭也消费机器,等等。各种机器以器皿、管筒、软管等等形式加入机器的消费资料的生产,例如加入油脂、肥皂、煤气\fontbox{~\{}用于照明\fontbox{\}~}的生产。可见,甚至在这里,这些领域的产品也都彼此加入对方的不变资本,因而以实物形式互相补偿。

如果把役畜也算作机器,那末,对役畜就需要补偿饲料,并且在一定条件下需要补偿厩舍(建筑物)。但是,饲料加入牲畜的生产,牲畜也加入饲料的生产。

\textbf{第三,辅助材料}。其中一部分,如机油、肥皂、油脂、煤气等等需要原料。另一方面,它们一部分会以肥料等等形式重新加入这种原料的形成过程。制造煤气需要煤炭,而生产煤炭又使用煤气照明,等等。另一些\textbf{辅助材料}只由新加劳动和固定资本(机器、器皿等等)构成。煤炭必须补偿生产煤炭时使用的蒸汽机的损耗。但蒸汽机也消费煤炭。煤炭本身加入煤炭的生产资料。因此,在这里煤炭以实物形式自己补偿自己。煤炭的铁路运输加入煤炭的生产费用,但煤炭又加入机车的生产费用。

\textbf{以后关于化学工厂还要专门补充一下},所有这些工厂在不同程度上制造辅助材料、器皿的原料(例如玻璃、瓷),以及直接加入消费的物品。

一切染料都是辅助材料。但它们不仅按其价值来说会加入产品,例如象工厂中消费的煤炭加入棉布那样,而且会在产品取得的形式上(产品的色彩上)再现出来。

\textbf{辅助材料}可以是\textbf{机器的消费资料},——在这里,它们或者充当发动机的燃料,或者用作减轻工作机摩擦的手段等等,如油脂、肥皂、机油等等,——它们也可以是建筑用的辅助材料,如水泥等等,最后,它们还可以是实现生产过程一般所必需的辅助材料,如照明、取暖等等(在这种情况下,它们就是工人本身为了能够劳动所必需的辅助材料)。

或者,\textbf{这是}一些加入原料形成过程的\textbf{辅助材料},如各种肥料以及原料所消费的一切化学产品。

或者,\textbf{这是}一些加入成品的\textbf{辅助材料},如颜料、漆等等。

\centerbox{※     ※     ※}

\textbf{因而,结果是}:

通过同非个人消费的产品 B 中代表 B 的收入的那部分相交换,即通过同 B 部类一年内的新加劳动相交换,A 补偿了自己的不变资本,即产品的 2/3。但是 A 不补偿 B 的不变资本。B 部类必须用同类新产品以实物形式自己补偿这个不变资本。但 B 部类已经没有任何劳动时间来补偿它们了。因为 B 部类的全部新加劳动时间构成它的收入,因而已经由产品 B 中作为不变资本加入 A 的那一部分来代表了。那末,B 的不变资本怎样补偿呢\fontbox{?}

B 的不变资本部分地通过\textbf{本身的}(植物性的或动物性的)\textbf{再生产}来补偿,农业和畜牧业的所有部门的情形,就是如此;部分地通过一种不变资本的一部分同另一种不变资本的一部分\textbf{以实物形式交换}来补偿,这里,一个领域的产品作为原料或生产资料加入另一个领域,反过来也是如此。可见,这里不同生产领域的产品,[389]不同种类的不变资本,彼此作为生产条件以实物形式加入对方。

非个人消费品的生产者为个人消费品的生产者生产不变资本。但同时,他们的产品还相互充当彼此的不变资本的要素或因素。这就是说,他们相互把彼此的产品用于\textbf{生产}消费。

产品 A 全部由个人消费。因而其中包含的全部不变资本也都由个人消费。(1/3)A 由 A 的生产者消费,(2/3)A 由非个人消费的产品 B 的生产者消费。A 的不变资本由构成 B 的收入的 B 的产品来补偿。事实上,这是不变资本中由\textbf{新加劳动}来补偿的唯一部分,这一部分所以由这种劳动来补偿,是因为代表 B 部类新加劳动的 B 的那个产品量,不由 B 部类消费,相反地由 A 部类用于生产消费,而 B 部类则把(2/3)A 用于个人消费。

如果 A 等于 3 工作日;那末,按照假定,它的不变资本等于 2 工作日。B 补偿产品 A 的 2/3,也就是提供等于 2 工作日的非个人消费品。现在 3 工作日已经被吃掉,还剩下 2 工作日。换句话说,A 的过去的 2 工作日由 B 的新加的 2 工作日补偿,但这只是因为,B 的新加的 2 工作日按价值来说是以产品 A 的形式,而不是以产品 B 本身的形式消费的。

B 部类的不变资本,由于它加入 B 的总产品,也必须由同类的新产品,也就是由 B 部类的\textbf{生产}消费所必需的产品以实物形式补偿。它虽然也是由一年内新花费的劳动时间的\textbf{产品}补偿,但不是由\textbf{新}的劳动时间补偿。

假定在 B 的总产品中,全部不变资本占 2/3。那末,如果新加劳动(等于工资和利润的总额)等于 1,则充当其劳动材料和劳动资料的过去劳动就等于 2。这个 2 是怎样补偿的呢\fontbox{?}在 B 部类的不同生产领域中,可变资本同不变资本之比,可能是极不相同的。但是按照假定,平均比例等于 1/3∶2/3,或 1∶2。B 部类的每个生产者都有 2/3 的产品——如煤炭、铁、亚麻、机器、牲畜、小麦(指不加入个人消费的那部分牲畜和小麦)等等——需要补偿它们的生产要素,或者说,这 2/3 的产品必须再转化为自己生产要素的实物形式。但所有这些产品本身都重新加入生产消费。小麦(作为种子)同时又成了它自己的原料,养大的一部分牲畜补偿消费掉的牲畜,也就是自己补偿自己。这样,在 B 部类的这些生产领域(农业和畜牧业)中,它们的一部分产品就是以自己的实物形式补偿自己的不变资本。可见,这种产品的一部分不进入流通(至少它不一定要进入流通,它可能只是在形式上进入流通)。这些产品中的其他产品,如亚麻、大麻等等,煤炭、铁、木材、机器,都部分地作为生产资料加入自己的生产。就象农业中的种子一样,煤炭加入煤炭的生产,机器加入机器的生产。可见,由机器和煤炭构成的产品的一部分,并且是这一产品中代表它的不变资本的那个份额中的一部分,是自己补偿自己的,它在生产过程中只是改变自己的位置。它不再是产品,而变成了自己的生产资料。

这些和那些产品的其余部分,则彼此作为生产要素相互加入对方:机器加入铁和木材,木材和铁加入机器;机油加入机器,机器加入机油;煤炭加入铁,铁(作为铁轨等等)加入煤炭,等等。这样,B 部类的这些产品的 2/3,就其不是自己补偿自己,即不以自己的实物形式再加入自己的生产这部分(因此,产品 B 的一部分由自己的生产者直接用于生产消费,就象产品 A 的一部分由自己的生产者直接用于个人消费一样)来说,B 部类各生产者的产品是彼此作为生产资料相互补偿的。生产者 a 的产品加入生产者 b 的生产消费,而生产者 b 的产品加入生产者 a 的生产消费,或者通过间接的方式:生产者 a 的产品加入生产者 b 的生产消费,生产者 b 的产品加入生产者 c 的生产消费,而生产者 c 的产品加入生产者 a 的生产消费。这样,在 B 部类的一个生产领域中作为不变资本消费的东西,就在另一个生产领域中重新生产出来,而在后一个生产领域中消费的东西,又在前一个生产领域中生产出来。在一个领域中从机器和煤炭的形式变为铁的形式的东西,在另一个领域中则从铁和煤炭的形式变为机器的形式,等等。

[390]必须使 B 部类的不变资本以实物形式得到补偿。如果考察一下 B 部类的总产品,那末它正好代表各种实物形式的全部不变资本。当 B 部类的某个特殊领域的产品不能以实物形式补偿自己的不变资本时,买和卖,相互转手,在这里会使一切重新就序。

总之,这里发生的是不变资本由不变资本补偿;只要这种补偿不是直接的,不是不经过交换的,那就是\textbf{资本同资本相交换},按使用价值来说就是产品同产品相交换,这些产品彼此加入对方相应的生产过程,因而每个这样的产品都由相应的其他产品的生产者用于生产消费。

这部分资本既不归结为利润,也不归结为工资。它不包含任何新加劳动。它不同收入交换。它既不直接也不间接由消费者支付。资本的这种相互补偿不论有没有商人(即商人资本)作中介,都丝毫不会使问题有所改变。

但是,既然这些产品(彼此相互补偿的机器、铁、煤炭、木材等等)是新的产品,既然它们是当年劳动的产品,例如用作种子的小麦,完全象加入个人消费的小麦一样,是新劳动的产品,等等,那又怎么能够说,这些产品中不包含任何新加劳动呢\fontbox{?}此外,它们的形式不是十分令人信服地表明情况恰好相反吗\fontbox{?}如果说在小麦或牲畜身上还看不出这一点,那末机器,机器的形式,却直接证明了把它从铁等等变成机器的那种劳动。如此等等。

这个问题在前面已经解决了。\authornote{见本册第 89—140、182—195 页和第 220—221 页。——编者注}这里不需要回过头去再谈了。

\fontbox{~\{}可见,亚·斯密认为“实业家”和“实业家”之间的贸易规模同“实业家”和消费者(消费者是指直接消费者,不是指生产消费者,因为斯密本人把生产消费者列入“实业家”的范畴)之间的贸易规模相等,这个命题是错误的。这个命题建立在他的一个错误的论点上,按照这个论点,全部产品都归结为收入,而事实上这不过是说,由资本和收入的交换构成的那部分商品交换,等于全部商品交换。因此,图克根据这个命题对于货币流通(特别是对于“实业家”之间流通的货币量同“实业家”和消费者之间流通的货币量之比)所做出的实际结论,和这个命题一样,也是错误的。

如果我们把购买产品 A 的商人当作同消费者对立的最后一个“实业家”,那末,这些产品在他手里,就会由生产者 A 的收入(等于(1/3)A)和生产者 B 的收入(等于(2/3)A)买去。这些收入补偿他的商人资本。这些收入的总和必定抵补他的资本。(这个家伙赚到的利润必定是这样得来的:他把一部分 A 留给自己,把较小量的 A 按照全部 A 的价值来出卖。无论是把这个家伙看作必要的生产当事人,还是把他看作寄生的中间人,都完全不会使问题有所改变。)经营产品 A 的“实业家”和产品 A 的消费者之间的交换,按其价值来说,则抵补经营产品 A 的人和所有参加产品 A 生产的人之间的交换,因而抵补这些生产者相互之间的全部交易。

商人购买麻布。这是“实业家”和“实业家”之间的最后一次交易。麻布织造业者购买纱、机器、煤炭等等。这是“实业家”和“实业家”之间的倒数第二次交易。纺纱业者购买亚麻、机器、煤炭等等。这是“实业家”和“实业家”之间的倒数第三次交易。亚麻种植业者和机器制造业者购买机器、铁等等,依此类推。但是亚麻、机器、铁、煤炭的生产者之间为补偿他们的不变资本而进行的交易以及这些交易的价值,都不加入产品 A 所经过的那些交易(不管这是收入和收入之间的交换,还是收入和不变资本之间的交换)。这些交易,——不是 B 的生产者和 A 的生产者之间的交易,而只是 B 的生产者们相互之间的交易,——就象产品 B 的这一部分的价值完全不加入产品 A 的价值一样,完全不需要由产品 A 的买者对产品 A 的卖者进行补偿。这些交易也需要货币,也要以商人为中介。但专门属于这个领域的那部分货币流通同“实业家”和消费者之间的货币流通是完全分开的。\fontbox{\}~}

[391]剩下要解决的还有两个问题:

(1)在以上的论述中,我们把工资看作收入,没有把它同利润区分开来。现在要问,工资同时表现为资本家的流动资本的一部分这种情况,在这里会有多大的意义\fontbox{?}

(2)直到现在我们假定,全部收入都作为收入花掉。因此,应当考察收入即利润的一部分化为资本时发生的变化。这事实上同对积累过程的考察——但不是从它的形式上考察——是一致的。代表剩余价值的那部分产品一部分再转化为工资,一部分再转化为不变资本,这是很简单的。但这里必须研究,这种情况怎样影响前面所分析的各个项目的商品交换,——在这些项目下,商品交换可以从它的承担者方面来考察,——即:收入同收入的交换,收入同资本的交换,以及资本同资本的交换。\fontbox{\}~}

\fontbox{~\{}这样,这一幕间曲就必须穿插在这个历史批判部分,一直演奏到结束。\endnote{马克思在他的手稿第 X 本中,由于分析魁奈的《经济表》,又回过头来分析了这“幕间曲”中涉及的某些问题(见本册第 6 章)。而对前面所提出的两个问题,在《资本论》第二卷(特别是在第二十章第十节《资本和收入:可变资本和工资》,以及第二十一章《积累和扩大再生产》)作了详细和系统的回答。马克思在《剩余价值理论》第二册论李嘉图的积累理论一章中,又回过头来分析了这一“幕间曲”中所分析的问题。马克思在《剩余价值理论》第三册《反对政治经济学家的无产阶级反对派》一章(批判分析匿名小册子《根据政治经济学基本原理得出的国民困难的原因及其解决办法》)和论舍尔比利埃一章(论述作为扩大再生产的积累问题),又回过头来分析了资本和收入之间的交换问题。——第 258 页。}\fontbox{\}~}

\tsectionnonum{[(11)]费里埃[费里埃对斯密的生产劳动和资本积累理论的反驳的保护关税性质。斯密在积累问题上的混乱。斯密关于“生产劳动者”的见解中的庸俗成分]}

弗·路·奥·费里埃(\textbf{海关副督察})著有《论政府和贸易的相互关系》(1805 年巴黎版)一书。(这本书是弗·李斯特论据的主要来源。)此人是\textbf{波拿巴王室}的禁止性关税制度等等的赞颂者。实际上,他认为政府(因而国家官吏这些非生产劳动者)具有重要意义,说政府是直接干预生产的领导者。所以,这个海关官吏对亚·斯密把国家官吏叫做非生产劳动者这一点非常恼火。

\begin{quote}“斯密\textbf{确定的国家节约的}原则,是以生产劳动和非生产劳动之间的区分为根据的……”\end{quote}

\fontbox{~\{}这正是因为斯密希望,将产品中尽可能大的部分当作资本花费,即用来同生产劳动交换,而将尽可能小的部分当作收入花费,用来同非生产劳动交换。\fontbox{\}~}

\begin{quote}“这一区分实质上是错误的。\textbf{根本就没有非生产劳动}。”(第 141 页)“因此,国家有节约和浪费之别;但国家是浪费还是节约,只能从该国同\textbf{他}国的关系来看,问题也正是必须这样来看。”(同上,第 143 页)\end{quote}

现在我们把费里埃所憎恨的亚·斯密的论断拿来与此对照一下。

\begin{quote}费里埃说:“国家的节约是有的,但跟斯密所说的完全不同。国家的节约在于,购买外国产品的数量不超过能用本国产品支付的限度。有时这种节约在于完全不要外国产品。”(同上,第 174—175 页)\end{quote}

\fontbox{~\{}\textbf{亚·斯密}在第一篇第六章(加尔涅的译本,第 1 卷第 108—109 页)《论商品价格的构成部分》的结尾说:

\begin{quote}“因为在一个文明国家里,\textbf{只有极少数商品的交换价值仅由劳动产生,绝大多数商品的交换价值中有大量地租和利润加入},所以,\textbf{这个国家的劳动的年产品所能购买和支配的劳动量,比这个产品的制造、加工和运到市场所必须使用的劳动量总要大得多}。如果\textbf{社会每年使用了它每年所能购买的全部劳动,那末,由于这个劳动量逐年会有很大的增加},每一年的产品就会比上一年的产品具有大得多的价值。但是,没有一个国家\textbf{会把全部年产品都用于}工人的生活费。在任何地方,产品很大一部分都是归有闲者消费的。年产品的一般的或平均的价值,究竟是增加,减少还是年年不变,就必然要看这个产品按怎样的比例在这两个不同的阶级之间分配。”\end{quote}

在斯密实际上想解开积累之谜的这段话里,各式各样的混乱看法是不少的。

首先,我们在这里又看到那个错误的前提:劳动的年产品的“交换价值”,也就是“\textbf{劳动的年产品}”,全部分解为工资和利润(地租也包括在内)。我们不想回过头来谈这个荒谬的观点。我们只想指出下面一点。年产品总量——或构成劳动的年产品的商品总额、商品储备,——按其实物形式来说,很大一部分[392]是由只能作为不变资本的要素\fontbox{~\{}各种原料、种子、机器等等\fontbox{\}~}加入不变资本的,即只能用于生产消费的商品构成的。关于这些商品(而这是加入不变资本的大部分商品),它们的\textbf{使用价值}本身就已经表明,它们不能用于个人消费,因而收入——无论是工资,利润还是地租——不能花在它们身上。固然,一部分原料(只要不是这些原料本身的再生产所必需的,或者不是作为辅助材料或作为直接组成部分加入固定资本的)在以后会取得可消费的形式,但这只是由于加上了当年劳动。作为去年劳动的产品,甚至这些原料也不能成为收入的任何一部分。只有产品的可消费部分才能被消费,才能加入个人消费,从而才能构成收入;但是,甚至可消费的产品也有某一部分不能被消费,否则就会使再生产成为不可能。因此,就连商品的可消费部分中,也要拿出一部分来,这一部分必须用于\textbf{生产消费},即必须成为劳动材料、种子等等,不能成为生活资料——不管是工人的还是资本家的生活资料。因而这部分产品一开始就必须从亚·斯密的计算中扣除,或者更确切地说,必须加入这个计算。只要\textbf{劳动生产率保持不变,产品中}不分解为收入的那个部分也每年不变,也就是说,只要劳动生产率保持不变,花费在生产上的劳动时间量也和以前一样。

如果假定每年使用比以前\textbf{更大的}劳动\textbf{量},那就必须考察在这种情况下不变资本的状况。有一点是无疑的:为了能够使用更大的劳动量,单单支配\textbf{更大的劳动量}并\textbf{支付这个更大的量},即把更多的资金用在工资上,是不够的,还必须拥有可以吸收这个更大劳动量的劳动资料(原料和固定资本)。因此,在阐明亚·斯密考察的各点\textbf{之后},还必须把这一点分析一下。

这样,让我们再一次看看他的第一句话:

\begin{quote}“因为在一个文明国家里,只有极少数商品的交换价值\textbf{仅由劳动产生},绝大多数商品的交换价值中\textbf{有大量地租和利润加入},所以,\textbf{这个国家的劳动的年产品所能购买和支配的劳动量},比\textbf{这个产品的制造}、加工和运到市场〈换句话说,产品的生产〉\textbf{所必须使用的劳动量}总要大得多。”\end{quote}

这里,显然是把各种不同的东西混杂在一起了。加入全部年产品的交换价值的不仅有活劳动,即当年耗费的活劳动,而且还有过去劳动,即往年劳动的产品。不仅有活的形式的劳动,而且有物化形式的劳动。产品的交换价值等于产品所包含的劳动时间的总和,其中一部分由活劳动构成,一部分由物化劳动构成。

假定活劳动和物化劳动之比是 1/3∶2/3,即 1∶2。那末,全部产品的价值就等于 3,其中 2 是物化劳动时间,1 是活劳动时间。因此,如果只从物化劳动和活劳动作为等价物相互交换,一定量的物化劳动只能支配等量的活劳动这个前提出发,那末,全部产品的\textbf{价值}所能购买的活劳动,就比它本身包含的活劳动多。因为产品等于 3 工作日,而它包含的活劳动时间等于 1 工作日。为了生产产品(实际上不过是为了使产品的要素具有最终形式),只要 1 日的活劳动就够了。但是,产品中包含 3 工作日。可见,如果把这个产品全部用来同活劳动时间交换,如果只把它用来“购买和支配”活劳动量,那末它就能支配、购买 3 工作日。

然而亚·斯密指的显然不是这个意思;在他看来,这是一个完全无用的前提。他想说的是:产品的交换价值有很大一部分不分解为工资,而分解为利润和地租,或者为了简单起见,我们可以说分解为利润(斯密不用“分解”这个词,他用的是另一种\textbf{错误的}说法,这是由我们在前面已经指出过的概念的混淆\authornote{见本册第 75—78 页。——编者注}造成的)。换句话说:产品中和当年加进的劳动量相等的那部分价值——实际上就是真正由当年劳动生产的那部分产品——第一,支付工人,第二,加入资本家的收入,加入资本家的消费基金。总产品的这部分全部由劳动产生,并且仅仅由劳动产生;但它包括有酬劳动和无酬劳动。工资等于有酬劳动的总和,利润[393]等于无酬劳动的总和。因此,如果把这全部产品都花在工资上,它所能推动的劳动量,自然就会比生产这个产品的劳动量大;而且,产品所能推动的更大的劳动时间量和产品本身包含的劳动时间量的比例,恰好决定于工作日分为有酬劳动时间和无酬劳动时间的比例。

假定有酬劳动时间和无酬劳动时间的比例是:工人在 6 小时即半个工作日内生产或再生产自己的工资。其余 6 小时即半个工作日,形成剩余时间。因此,有一个产品例如包含 100 工作日[新加劳动],这 100 工作日等于 50 镑(如果 1 工作日等于 10 先令,那末 100 工作日就等于 1000 先令,即 50 镑)。其中 25 镑为工资,25 镑为利润(地租)。用这笔等于 50 工作日的 25 镑,可以支付 100 工人,这 100 工人有一半劳动时间是进行无代价的劳动,换句话说,就是为自己的老板劳动。因此,如果把这全部产品(100 工作日)都花在工资上,那末 50 镑就能推动 200 工人,他们每一个人都象以前一样,得到工资 5 先令,即自己劳动产品的一半。这些工人的产品是 100 镑(也就是说,200 工作日等于 2000 先令,即 100 镑),用这 100 镑又能推动 400 工人(每个工人得到 5 先令,400 工人得到 2000 先令),他们的产品等于 200 镑,依此类推。

亚·斯密说“劳动的年产品所能购买和支配的劳动量”,比产品的生产所使用的劳动量总要“大得多”,就是指这个意思。(如果把工人劳动的全部产品都支付给工人,也就是说,如果是 100 工作日,就支付给他 50 镑,那末这 50 镑也就只能推动 100 工作日。)正是在这个意义上,亚·斯密接着说:

\begin{quote}“如果社会每年使用了它每年所能购买的全部劳动,那末,由于这个劳动量逐年会有很大的增加,每一年的产品就会比上一年的产品具有大得多的价值。”\end{quote}

但是,这个产品的一部分被利润和地租的所有者吃掉,另一部分被他们的食客吃掉。因此,能够重新用在劳动(生产劳动)上的那部分产品究竟有多少,就取决于产品中没有被资本家、租金所得者和他们的食客(同时也是非生产劳动者)吃掉的那部分究竟有多少。

然而,这样一来,这里总还有一笔新的基金(新的工资基金),以便用去年劳动的产品在本年推动更多的工人。因为年产品的价值决定于所花费的劳动时间量,所以年产品的价值将会逐年增长。

不言而喻,如果市场上没有更大量的劳动,即使有一笔基金,它能够“\textbf{购买和支配}的劳动量”比去年“大得多”,也是没有用处的。即使我有更多的货币可以购买某种商品,如果市场上没有更多的这种商品,对我也是没有什么用处的。假定从 50 镑中拿出一个数目,这个数目不是推动 200 工人以代替原先的 100 工人(他们得到 25 镑),而是只推动 150 工人,这时,资本家自己吃掉的是 12+(1/2)镑,而不是 25 镑。在这种情况下,150 工人(他们得到 37+(1/2)镑)就会提供 150 工作日,即等于 1500 先令或 75 镑。但是,如果可使用的工人人数照旧只有 100 人,那末,这 100 人现在得到的工资就会是 37+(1/2)镑,而不是原先的 25 镑,可是他们的产品仍然只有 50 镑。这样一来,资本家的收入就会从 25 镑降到 12+(1/2)镑,因为工资增加了 50\%。但是亚·斯密知道,要增加的劳动量是会有的。一方面是由于人口每年增长(诚然,按照斯密的意见,原有的工资总额是这种增长的前提)。另一方面是由于存在着失业的赤贫者、半失业的工人等等。其次,大量非生产劳动者中间有一部分人因剩余产品使用上的改变而能够变成\textbf{生产}工人。最后,同样数量的工人可以提供\textbf{更大的}劳动\textbf{量}。因为我雇用 125 工人来代替 100 工人,或者让 100 工人每天劳动 15 小时而不是劳动 12 小时,是完全一样的。

此外,说随着生产资本的增加,——或者说随着用于再生产的那部分年产品的增加,——\textbf{所使用的劳动}(活劳动,花费在工资上的那部分资本)也会按同样的比例增加,这是亚·斯密的错误,这个错误同他认为全部产品分解为各种收入的观点有着最密切的联系。

[394]总之,斯密首先肯定说,有可供个人消费的生活资料基金,这个基金在本年内能够“购买和支配”的劳动量比去年大。有更多的劳动,同时又有供这个劳动用的更多的生活资料。现在应当考察一下,这个追加的劳动量如何实现。\fontbox{\}~}

如果亚·斯密完全自觉地、始终一贯地坚持他实质上已有的那种对剩余价值的分析,即认为剩余价值只有在资本同雇佣劳动的交换中才会创造出来,那末,他就会发现,只有同资本交换的劳动才是生产劳动,而同收入本身交换的劳动决不是生产劳动。为了同生产劳动交换,收入必须先转化为资本。

但斯密同时又从片面的传统观点出发,认为生产劳动就是一般直接生产物质财富的劳动;并且把自己的区分(根据资本同劳动交换还是收入同劳动交换作出的区分)同这种观点结合起来,所以在他看来,下面这样的定义是可能的:同资本交换的那种劳动始终是生产劳动(始终创造物质财富等等);而同收入交换的那种劳动既可能是生产劳动,也可能是非生产劳动,但是,花费自己收入的人,在大多数情况下,都宁愿使用某种直接的非生产劳动,而不愿使用生产劳动。这里可以看出,亚·斯密由于把自己的两种区分混在一起,就把主要的区分大大削弱并冲淡了。

下面这段引文表明,亚·斯密并没有把劳动的固定化完全归结为纯粹的表面的固定化;在这段引文中,有他列举的固定资本各个组成部分中的一条:

\begin{quote}“(4)居民或社会成员所获得的有用才能。要获得这种才能,总得支出一笔实在的费用,供获得才能的人在他受教育、实习或学习期间维持生活,而这笔费用可以说就是固定和物化在他个人身上的资本。这种才能是他的财产的一部分,也是他所在的那个社会的财产的一部分。可以把工人的提高了的技能,同减轻和缩短劳动的机器或工具一样看待,在这些东西上虽然要支出一笔费用,但它们会偿还这笔费用,并提供利润。”(加尔涅的译本,第 2 卷第 204—205 页)\end{quote}

\textbf{奇怪的积累来源和积累的必要性}:

\begin{quote}“在社会的原始状态中,没有任何分工,几乎不发生交换,每一个人都用自己的手去谋得他所需要的一切东西。在这种状态中,\textbf{没有必要为了维持社会经济生活而把资财预先积累或储存起来}”\end{quote}

(其实这里一开始就假定不存在任何社会)。

\begin{quote}“每一个人都努力以自己的活动来获得满足自身随时产生的需要的手段。他饿了,便到森林去打猎”等等。(同上,第 2 卷第 191—192 页)(第 2 篇\textbf{序论})“但是,一旦分工普遍实行,一个人用他个人的劳动就只能满足他当时产生的需要的极小部分。他的需要的绝大部分都要靠\textbf{别人劳动的产品}来满足,他用自己劳动的产品,或者说用自己产品的价格去购买别人劳动的产品。但是,在实行这种\textbf{购买}之前,他必须有时间不仅\textbf{完全制成}并且\textbf{还要卖掉他的劳动产品}。”\end{quote}

(即使在前一种场合,他不先打到兔子,也吃不到兔肉,而他不先制成古“弓”或类似的东西,就不可能打到兔子。所以在后一种场合,唯一的新条件并不是必须有什么“储存”,而是必须“有时间……\textbf{卖掉}他的劳动产品”。)

\begin{quote}“因此,至少在他能够完成这两件事以前,必须在某个地方\textbf{预先储存各种物品},以维持他的生活,并供给他劳动所必需的原料和工具。一个织布业者,在他把麻布织成并且卖掉以前,如果\textbf{在他手里或别人手里}没有预先\textbf{储存}的物品,以维持他的生活,并供给他劳动所需的工具和材料,他是\textbf{不能全力从事}自己的专业的。十分明显,在他能够从事这项工作并把它完成\textbf{以前,必须有积累}……按照事物的本性,\textbf{\CJKunderdot{资本}的积累是分工的必要的先决条件}。”(同上,第 192—193 页)\end{quote}

(另一方面,按照斯密在序论中的第一句话,好象在分工\textbf{以前}没有任何资本积累,而现在他却完全相反,断言在资本积累以前没有任何分工。)

斯密继续说道:

\begin{quote}“只有预先积累的资本愈来愈多,分工才会愈来愈细。分工愈细,\textbf{同样数目的人所能加工的原料数量就会大大增加};因为这时每一个工人的操作愈来愈简单,所以减轻和[395]缩短劳动的新机器就大量发明出来。因此,随着分工的发展,为了经常雇用同样数目的工人,就必须\textbf{预先积累同样多的生活资料},以及比分工不发达时\textbf{更多的原料和劳动工具}。”(同上,第 193—194 页)“劳动生产力的大大提高,\textbf{非有预先的资本积累}不可,同样,资本的积累也自然会引起劳动生产力的大大提高。\textbf{凡是使用自己的资本来雇用工人的人}当然希望,他这样做会使工人完成尽可能多的工作。因此,他力求在自己的工人中间最恰当地进行分工,并把他所能发明或购买的最好的机器供给工人使用。他在这两方面成功的可能性如何,通常要看他有多少资本,或者说,要看这个资本能够雇用多少工人。因此,\textbf{在一个国家里},不仅\textbf{劳动量随着}推动劳动的\textbf{资本的扩大而增加},而且同一\textbf{劳动量所生产的产品,也由于资本的扩大而大大增加}。”(同上,第 194—195 页)\end{quote}

亚·斯密完全象他论述生产劳动和非生产劳动那样,论述已经加入消费基金的物品。例如,他说:

\begin{quote}“住房不会给居住者带来任何收入;虽然这所住房无疑对他说来是非常有用的,但这不过是象他的衣服和家具一样,衣服和家具对他说来也是十分有用的,但这不过是他的开支的一部分,不是收入的一部分。”(同上,第 2 卷第 201—202 页)相反,属于固定资本的有“一切有用的建筑物,它们不仅对收取租金的建筑物所有者来说,是获得收入的手段,甚至对支付租金的建筑物承租人来说,也是获得收入的手段;例如店铺、仓库、工场以及有各种必要的设备、厩舍、粮仓等等的租地农场,就是如此。这种建筑物和纯粹的住房大不相同;它们是一种生产工具”。(同上,第 203—204 页)(第 2 篇第 1 章)“一切技术成就,使得同一数量的工人能够用比以前更简单、更便宜的机器来完成同样的工作量,这始终被认为是对社会很有利的。以前用来维持较复杂、较昂贵的机器的一定数量的原料和一定数量的工人,现在就可以用来增大工作量,而这些或那些机器是为了进行这种工作而制造出来的。”(同上,第 2 卷第 216—217 页)(第 2 篇第 2 章)“\textbf{固定资本}的维持费……要从社会纯收入中排除掉。”(同上,第 2 卷第 218 页)“不减低劳动生产力的\textbf{固定资本}的维持费的任何节约,就必定会增加推动企业的基金,因而必定会增加土地和劳动的年产品,增加每个社会的实际收入。”(同上,第 2 卷第 226—227 页)被银行券(一般说,纸币)排挤到国外的金银币,——如果花在“购买供国内消费的外国货”上,——或者用来购买奢侈品,如外国葡萄酒、丝织品等等,一句话,购买“供什么也不生产的……\textbf{有闲者}消费的商品……或者……用来购买\textbf{追加的原料、劳动工具和生活资料,以维持和雇用追加的勤劳者,这些勤劳者会把他们每年消费的价值再生产出来,并提供一笔利润}”。(同上,第 2 卷第 231—232 页)斯密说,前一种使用货币的方法增加浪费,“增加开支和消费,丝毫不会增加生产,也不会创造抵补这些开支的固定基金,所以从各方面来讲,对社会都是有害的”。(同上,第 2 卷第 232 页)相反,“用后一种方法支出的货币,就相应地扩大生产规模,尽管它也增加社会的消费,但是它开辟了维持消费的固定来源,因为\textbf{消费这一生活资料的追加量的人会把他们每年消费的全部价值再生产出来,并提供一笔利润}”。(第 2 卷第 232 页)“一笔资本能推动多少生产劳动,显然要看它能给多少工人提供符合他们劳动性质的原料、劳动工具和生活资料。”(同上,第 2 卷第 235 页)(第 2 篇第 2 章)\end{quote}

[396]我们\textbf{在第二篇第三章}(同上,第 2 卷第 314 页及以下各页)读到:

\begin{quote}“生产劳动者、非生产劳动者以及那些根本不劳动的人,同样都是靠该国土地和劳动的年产品维持生活。这种产品……必然是有限的。因此,根据一年内用来维持非生产劳动者的那部分产品是较少或者较多的不同情况,为生产劳动者留下的产品就会是较多或者较少,与此相适应,下一年的产品也会增加或减少……虽然每一个国家的土地和劳动的全部年产品……归根到底都是供国内居民消费,并给他们带来收入,\textbf{但是,当}产品从土地或从生产工人手中生产出来\textbf{的时候},它就自然分成两部分。其中一部分,而且往往是最大的部分,首先用来\textbf{补偿资本,更新那些}已经从资本中取出的\textbf{\CJKunderdot{生活资料}、原料和成品};另一部分则用来形成收入,——或是作为这个资本的所有者的资本的利润,或是作为另一个人的土地的地租……\textbf{每一个国家的土地和劳动的年产品中补偿资本的那一部分},决不能直接用来维持生产工人以外的其他任何雇佣人员;这一部分只能给生产劳动支付工资。直接形成收入的那部分产品……既可以用来维持生产劳动者,也可以用来维持非生产劳动者……非生产劳动者和那些根本不劳动的人,都\textbf{靠收入}维持生活;或者,第一,靠年产品中一开始就形成某些人的收入(不是作为地租,便是作为资本利润)的那一部分;或者第二,靠年产品中的另一部分,这一部分虽然是供补偿资本和仅仅维持生产工人用的,但一到生产工人手里,除了维持他们的生活所必需的部分之外,其余部分就能用来既维持生产人员,也维持非生产人员。例如,一个普通工人,如果他的工资高,他就能……雇个仆人,或者有时去看看喜剧或木偶戏,这样,他就用自己的一部分收入来帮助维持一类非生产劳动者;或者最后,他能交纳一些税,从而帮助维持另一类……同样是非生产的劳动者。但是,一开始就供补偿资本用的那部分年产品,在它没有把与它相应的生产劳动量全部推动之前,是决不能用来维持非生产劳动者的……工人必须先做工,挣得了工资,然后才能把哪怕是极小的一部分收入,支出在非生产劳动上……地租和资本利润……在任何地方,都是非生产劳动者赖以取得生活费的主要源泉……这两种收入,既可以维持生产劳动者,也可以维持非生产劳动者;但是,这些收入的所有者,看来总是更喜欢把它们用在后者身上……总之,年产品从土地或从生产工人手中生产出来以后,一部分供补偿资本用,另一部分形成收入(作为地租或利润)。在每一个国家里,生产劳动者和非生产劳动者之间的比例,主要决定于年产品的这两部分之间的比例,而这个比例在富国和贫国是极不相同的。”\end{quote}

斯密接着把情况作了对比:

\begin{quote}“在欧洲各富国”,今天“土地产品的很大一部分,而且往往是最大的部分,\textbf{都用来补偿富有的独立的租地农场主的资本}”;过去的情况则相反,“在封建制度统治时期,产品的极小部分就足以补偿耕地使用的资本”。商业和工业中的情形也是这样。现在商业和工业中使用大资本;而以前,资本是极小的,但它们带来的利润很大。“利息在任何地方都不低于 10\%,要支付如此高的利息,资本利润必定非常大。目前在欧洲比较发达的国家,利息在任何地方都不超过 6\%,而在最富的国家,利息则等于 4\%、3\%、2\%。居民由利润得来的那部分收入,在富国总是比在贫国大得多,这是因为富国的资本大得多;但利润同资本相比,富国的利润通常又低得多。由此可见,年产品从土地或从生产工人手中生产出来以后,供补偿[397]资本用的那部分,在富国不仅比在贫国大得多,而且同直接形成收入(作为地租或利润)的那部分相比,也大得多。用来维持生产劳动的基金,在富国不仅比在贫国大得多,而且同那种既能用来维持生产劳动者,又能用来维持非生产劳动者,但通常主要是用来维持后者的基金相比,也大得多。”\end{quote}

(斯密犯了这样的错误:他把生产资本的量同用来维持生产劳动的\textbf{那部分生产资本的量}等同起来。但这同他所了解的大工业实际上还只处在萌芽状态有关系。)

\begin{quote}“这两种不同基金之间的比例,在每一个国家,必然会决定该国居民的一般性格是勤劳,还是懒惰。”斯密说,例如,“在英国和荷兰的工业城市里,人民的下层阶级主要依靠所使用的资本过活,他们一般来说都是勤劳的、刻苦的和节俭的。相反,在宫廷所在地的都城等等,人民的下层阶级依靠上层阶级挥霍收入来生活,他们一般来说都是懒惰的、放荡的和贫困的;例如,罗马、凡尔赛等地就是这样……”“由此看来,资本总额和收入总额之间的比例,在任何地方都决定勤劳和懒散之间的比例:在资本占优势的地方,多勤劳;在收入占优势的地方,多懒散。因此,\textbf{资本量的每一增减},自然会引起生产活动量、生产工人人数的实际的增减,从而引起该国土地和劳动年产品的交换价值、该国全体居民的财富和实际收入的增减……一年内节约下来的东西,象一年内支出的东西一样,照例是会被消费的,而且几乎在同时被消费;不过,它是被另一类人消费的。一年内支出的那部分收入,由家仆、无用的食客等等消费,这些人决不会留下任何一点东西来补偿他们的消费。而一年内节约下来的那部分收入,由工人消费,这些工人会把自己一年消费的价值再生产出来,并提供一笔利润……消费是一样的,但消费者不一样……”\end{quote}

以下就开始了斯密关于节约的人的说教(同一章,下面第 328、329 页及以下各页),他说这种人靠自己每年的节约,可以为追加的生产工人建立一个公共工场,

\begin{quote}“设立一种永久的基金来维持相应数量的生产工人”,而“浪费者却使维持生产劳动的基金总数减少……如果把非生产人员这样〈由于浪费者挥霍〉消费的食物和衣服,分配给生产工人,这些生产工人就会把他们所消费的全部价值\textbf{再生产出来,并提供一笔利润}……”\end{quote}

斯密的这种说教的结语是:这(节约和浪费)会在私人中间相抵,并且实际上“理智”占上风。大国

\begin{quote}“从来不会因私人的浪费和妄为而变穷,虽然有时会因政府的浪费和妄为而变穷。在大多数国家,国民收入全部或几乎全部用来维持非生产人员。这些人包括宫廷人员、教会人士、海军、陆军,他们在平时什么也不生产,在战时也不能获得任何东西,来抵偿他们即使只是在战争期间的生活费用。\textbf{这种人自己什么也不生产,全靠别人的劳动产品来养活}。因此,如果他们人数的增加超过了必要的数量,他们在一年内就能消费很大一部分产品,以致剩下来的产品不足以维持必须在下一年把产品再生产出来的生产工人……”[同上,第 2 卷第 314—336 页]\end{quote}

斯密在第二篇第四章写道:

\begin{quote}“因为用来维持生产劳动的基金逐日增加,所以对生产劳动的需求也与日俱增:工人[398]容易找到工作,而资本所有者却难以找到他们能够雇用的工人。资本家的竞争使工资提高,利润下降。”(同上,第 2 卷第 359 页)\end{quote}

斯密在第二篇第五章《\textbf{论资本的各种用途}》中,根据各种资本所雇用的生产劳动量的大小,从而按照它们所增加的年产品“交换价值”的多少,对资本进行了分类。斯密放在第一位的是\textbf{农业},其次是\textbf{制造业},然后是\textbf{商业},最后是\textbf{零售商业}。这就是斯密根据资本所推动的生产劳动量排列的资本用途的顺序。这里我们又得到一个关于“生产劳动者”的全新的定义:

\begin{quote}“凡是把资本用于这四种用途之一的人,自己就是\textbf{生产劳动者}。他们的劳动,如果使用得当,会固定和物化在它所加工的物品或商品上,通常至少也会把他们维持自己生活和个人消费的价值加在商品的价格上。”(同上,第 2 卷第 374 页)\end{quote}

(总之,斯密把他们的生产性归结为他们推动生产劳动这一点。)

关于\textbf{租地农场主},他说:

\begin{quote}“没有一个同量的资本能比租地农场主的资本推动更大量的\textbf{生产劳动}。不仅他的雇工是生产劳动者,而且\textbf{他的役畜也是生产劳动者}。”[同上,第 2 卷第 376 页]\end{quote}

可见,最后连牛也成了生产劳动者。

\tsectionnonum{[(12)]罗德戴尔伯爵[把统治阶级看成各种最重要生产劳动的代表的辩护论观点]}

\textbf{罗德戴尔(伯爵)}《论公共财富的性质和起源》1804 年伦敦版(法译本:Recherchessurlanatureetl’originedelarichessepubliqueetc.,1808 年巴黎版)。

罗德戴尔提出的为利润辩护的理由,要放到后面第三篇\endnote{马克思这里说的“第三章”是指关于“资本一般”的研究的第三部分。这一章的标题应为:《资本的生产过程和流通过程的统一,或资本和利润》。以后(例如,见第 IX 本第 398 页和第 XI 本第 526 页)马克思不用“第三章”而用“第三篇”(《dritterAbschnitt》)。后来他就把这第三章称作“第三册”(例如,在 1865 年 7 月 31 日给恩格斯的信中)。关于“资本一般”的研究的“第三章”马克思是在第 XVI 本开始的。从这“第三章”或“第三篇”的计划草稿(见本册第 447 页)中可以看出,马克思打算在那里写两篇专门关于利润理论的历史补充部分。但是马克思在写作《剩余价值理论》的过程中,就已在自己的这一历史批判研究的范围内,详细地批判分析了各种资产阶级经济学家对利润的看法。因此,马克思在《剩余价值理论》中,特别是在这一著作的第二册和第三册中,就已进一步更充分地揭示了由于把剩余价值和利润混淆起来而产生的理论谬误。——第 7、87、272 页。}去考察。按照这种辩护论观点,利润是由资本本身产生的,因为资本“\textbf{代替}”劳动。资本之所以得到报酬,是因为它做了人没有它就得自己去做的事,或者做了人不借助于它就根本做不到的事。

\begin{quote}“现在很明白,资本利润的取得,总是或者因为资本代替了人必须用自己的手去完成的劳动;或者因为资本完成了人的个人力量不能胜任和人自己不能完成的劳动。”(法译本第 119 页)\end{quote}

“伯爵”先生极力反对斯密关于积累和节约的学说。他也极力反对斯密提出的对\textbf{生产劳动者和非生产劳动者}的区分;但是,按照他的意见,斯密叫做“劳动生产力”的东西只不过是“资本生产力”。他直接否认斯密提出的对剩余价值的解释,理由是:

\begin{quote}“如果对资本利润的这种理解真正正确的话,那就会得出结论说:利润不是收入的原始源泉,而只是派生源泉,并且,决不能把资本看作财富的源泉之一,因为资本带来的利润不过是收入从工人的口袋转到资本家的口袋而已。”(同上,第 116—117 页)\end{quote}

显然,在这种前提下,罗德戴尔在同斯密的论战中,抓住的也是最肤浅的东西。例如,他说:

\begin{quote}“由此可见,同一种劳动可以是生产的,也可以是非生产的,这要看劳动所加工的那个物品以后的用途如何。例如,如果我的厨师制成一个大蛋糕,我马上把它吃掉,那末,他就是非生产劳动者,他干的活就是不生产的劳动,因为他的服务一经提供随即消失。但如果这种劳动是在糕点店里完成的,那末它就成了生产劳动。”(同上,第 110 页)\end{quote}

(这里\textbf{加尔涅}应享有专利权,因为他出版的那本附有他的注释的斯密著作,是在 1802 年,即比罗德戴尔的书早两年问世的。)

\begin{quote}“这种新奇的区分,仅仅以所提供的服务的耐久性为根据,它把那些在社会上担任最重要职务的人,都归到非生产劳动者里面去。君主、一切神职人员、司法人员、国家保卫者以及一切用自己的技能……保护国民健康或使国民受到教育的人,都被视为非生产劳动者。”(同上,第 110—111 页)(或者象亚·斯密排列的一个很好的次序:“教士、律师、医生、各种文人;演员、丑角、音乐家、歌唱家、舞蹈家等等。”\authornote{加尔涅的译本,第 2 卷第 313 页。})“如果承认交换价值是财富的基础,那末,就没有必要用冗长的议论来证明这个学说的错误。最能[399]证明这个学说错误的是,如果根据这些服务所取得的报酬来判断,人们对这些服务是尊敬的。”(\textbf{罗德戴尔},同上第 111 页)\end{quote}

其次:

\begin{quote}“制造业工人的劳动固定和物化在某种可以出卖的商品上……\textbf{仆人的劳动}也好,由流动资本节约的劳动也好\fontbox{~\{}罗德戴尔在这里所说的“流动资本”是指\textbf{货币}\fontbox{\}~},当然都不能形成积累,不能形成以一定价值从一个人手里转到另一个人手里的基金。它们所提供的利益,同样都是由于它们\textbf{节约主人}或所有者的\textbf{劳动}造成的。既然它们产生如此相同的结果,那末,把其中一个称为非生产劳动,也就必然要把另一个称为非生产劳动。”\fontbox{~\{}他接着引了斯密在第二篇第二章中说的一段话\endnote{指下面这段话:“在一国内流通的金币和银币,作为本国土地和劳动的年产品在适当的消费者之间流通和分配的手段,就象个别商人的现金一样,是死资本。这是一国资本的极有价值的部分,但不为本国生产任何东西。”——第 273 页。}\fontbox{\}~}(\textbf{罗德戴尔},同上第 144—145 页)\end{quote}

\centerbox{※     ※     ※}

因此,可以排一个队:费里埃、加尔涅、罗德戴尔、加尼耳。\textbf{托克维尔}特别爱用最后那句关于“\textbf{节约劳动}”的话。

\tsectionnonum{[(13)萨伊对“非物质产品”的见解。为非生产劳动的不可遏止的增长辩护]}

在加尔涅之后,出版了庸俗的让·巴·萨伊的《论政治经济学》一书。萨伊非难斯密,说他

\begin{quote}“不把医生、音乐家、演员等人这类活动的\textbf{结果}叫做\textbf{产品}。他把这些人从事的劳动称为\textbf{非生产劳动}”。(第 3 版第 1 卷第 117 页)\end{quote}

斯密完全不否认“这类活动”会产生某种“结果”,某种“产品”。他甚至直接提到:

\begin{quote}“国家的安全、安定和保卫”是〈“国家公务人员”〉“年劳动的结果”。(\textbf{斯密}的著作第 2 篇第 3 章;加尔涅的译本,第 2 卷第 313 页)\end{quote}

萨伊也坚持斯密的补充定义:这些“服务”以及它们的产品“通常一经提供,一经生产,随即消失”。(\textbf{斯密},同一章)萨伊先生把这样消费掉的“服务”或它的产品,它的结果,一句话,它的使用价值,称为“非物质产品或一生产出来就被消费掉的价值”。他不把提供这种服务的人叫做“非生产劳动者”,而叫做“生产非物质产品的人”。他用了另一个名称。但是他在下面又说:

\begin{quote}“他们不是用来增加国民资本的。”(第 1 卷第 119 页)“一个国家有许多音乐家、教士、官吏,可能有很好的娱乐,精通宗教教义,并且治理得井井有条;但不过如此而已。国家的资本不会由于这些人的劳动而有任何直接的增加,因为他们的产品一生产出来就被消费掉。”(同上,第 119 页)\end{quote}

由此可见,萨伊先生只是从斯密的定义的最有限的意义上把这类劳动称为\textbf{非生产劳动}。但同时他又想把加尔涅的“进步”据为己有。所以他给各种非生产劳动发明了一个新名称。这就是他的独创性、生产性和发现方式。可是,他又以他惯有的逻辑把自己推翻了。他说:

\begin{quote}“不能同意加尔涅先生的意见,他根据医生、律师等等的劳动是生产劳动这一点,就得出结论说,增加这种劳动和增加其他任何劳动一样,对国家有利。”(同上,第 120 页)\end{quote}

但是,既然一种劳动和另一种劳动一样是生产劳动,既然生产劳动的增加一般都“对国家有利”,为什么不能同意这种意见呢\fontbox{?}为什么增加这种劳动,不象增加其他任何劳动那样有利呢\fontbox{?}萨伊用他特有的深奥想法回答说,因为增加任何一种生产劳动,如果超过了人们对这种劳动的需要,一般都是不利的。如此说来,加尔涅倒是对了。如此说来,增加这种劳动象增加其他劳动一样,超过了一定的限度,就一样有利——也就是一样不利了。

\begin{quote}萨伊继续说道:“这种情况就好比人们花费在产品上的体力劳动,超出了制造该产品所必要的劳动。”\end{quote}

(做一张桌子所花费的木匠劳动,不应超出生产桌子所必要的劳动。同样,修补病体所花费的劳动,也不应超出治好病体所必要的劳动。因此,律师和医生应当花费的只是制成自己的“非物质产品”所必要的劳动。)

\begin{quote}“生产非物质产品的劳动,\textbf{也象其他任何劳动一样},只有在增加产品的效用从而增加产品的价值〈即增加产品的使用价值,但萨伊把效用同交换价值混为一谈〉的时候,才是生产劳动;一旦超出这个界限,它就成为纯粹的非生产劳动了。”(同上,第 120 页)\end{quote}

可见,萨伊的逻辑是这样的:

增加“非物质产品生产者”的人数,\textbf{并不象}增加物质产品生产者的人数\textbf{那样}对国家\textbf{有用}。\textbf{论据}:无论哪种产品(物质产品或非物质产品)生产者人数的增加超过了需要,都是绝对无用的。\textbf{所以},增加无用的物质产品生产者的人数,比增加无用的非物质产品生产者的人数更有用。在这两种场合,都不能得出结论说:增加所有这些生产者的人数是无用的。只能得出结论说:增加某一部门内某种产品生产者的人数是无用的。

按照萨伊的意见,物质产品[400]决不会生产过多,非物质产品也是一样。但是,多样化使人愉快。所以这两个部门必须生产各式各样的产品。此外,萨伊先生教导说:

\begin{quote}“某些产品的滞销,是由另一些产品太少引起的。”[同上,第 1 卷第 438 页]\end{quote}

这就是说,桌子决不会生产过多,至多也许是可以放在桌子上的如碗之类的东西太少了。如果医生人数增加太多,那末错误不在于他们提供的服务过多,而大概在于其他“非物质产品”生产者,例如妓女,提供的服务太少(同上,第 123 页,那里,搬运工人、妓女等等的劳动被归成一类,萨伊还大胆断言,妓女的“训练时间等于零”)。

归根到底,在萨伊的书中,优势是在“非生产劳动者”方面。在一定的生产条件下,人们能准确地知道,做一张桌子,需要多少工人,制成某种产品,需要某种劳动量应多大。许多“非物质产品”的情况却不是这样。这里,达到某种结果所需要的某种劳动量多大,和结果本身一样,要靠猜测。二十个教士在一起对犯罪者的感化,也许是一个教士做不到的;六个医生会诊,能找到的有效药方,也许是一个医生找不到的。一个审判团,也许比一个无人监督的审判官能做出更为公正的裁判。保卫国家需要多少士兵,维持国内秩序需要多少警察,治理好国家需要多少官吏,等等,所有这些都是大可研究的问题,例如在英国议会中,这些问题就经常引起争论,虽然在英国,人们都很准确地知道,生产 1000 磅纱所必需的纺工劳动量有多大。至于另一些这类“生产”劳动者,他们的概念本身就包含着这样的意思:他们产生的效用,恰好只取决于他们的人数,只在于他们的人数本身。例如仆役就是这样,他们是他们主人有钱有势的证据。他们人数越多,他们“生产”的效果就越大。因此,萨伊先生始终认为:“非生产劳动者”的人数决不会增加到充分的程度。[400]

\tsectionnonum{[(14)]德斯杜特·德·特拉西伯爵[关于利润起源的庸俗见解。宣称“产业资本家”是唯一的最高意义上的生产劳动者]}

[400]\textbf{德斯杜特·德·特拉西伯爵}《思想的要素》,第四、五部分《论意志及其作用》1826 年巴黎版(第一版 1815 年出版)。

\begin{quote}“任何有用劳动都是真正的生产劳动,社会上的任何劳动阶级都同样应当称为\textbf{生产}阶级。”(第 87 页)\end{quote}

但是在这种生产阶级中,德斯杜特·德·特拉西又分出一个

\begin{quote}“\textbf{直接生产}我们的一切财富的劳动阶级”(第 88 页),\end{quote}

这也就是斯密所谓的生产工人。

相反,\textbf{不生产}阶级包括消费自己的土地的租金或货币的租金的富人,这是“\textbf{有闲阶级}”。

\begin{quote}“真正的\textbf{不生产}阶级是有闲者阶级,他们无所事事,专靠在他们以前业已完成的劳动的产品,过着所谓\textbf{养尊处优的}生活,而这些产品或者是物化在一些地产中,由他们把这些地产出租即\textbf{租借}给某个劳动者;或者是一些货币或物品,由他们借出去,取得一定的报酬,这也是\textbf{租借}。这种人是蜂房里真正的雄蜂(为享受果实而生的人们\authornote{见贺雷西《书信集》。——编者注})。”(第 87 页)这些有闲者“只能花费自己的\textbf{收入}。如果他们花费[401]自己的资本,那末资本就将无法补偿,而他们的消费,在短期间内过度增加之后,就会完全停止”。(第 237 页)“这种\textbf{收入}不外是……劳动市民的活动的产品的扣除部分。”(第 236 页)\end{quote}

这些“有闲者”直接使用的劳动者情况又怎样呢\fontbox{?}从有闲者消费商品这一点来说,他们不是直接消费劳动,而是消费生产工人的劳动产品。因而这里谈的劳动者,是指那些由有闲者直接花费收入来购买其劳动的劳动者;所以,是指那些直接从收入而不是从资本取得自己工资的劳动者。

\begin{quote}“因为占有它〈收入〉的那些人是有闲者,所以很明显,他们不\textbf{管理任何生产劳动}。一切由他们支付报酬的劳动者唯一的用处是为他们提供享受。享受当然是各种各样的……整个有闲阶级的开支……用于维持大批人口的生活,这批人的生活由此得到保证,但他们的劳动是完全不生产的……这种开支的某些部分也许多少有点用处,例如建筑房屋,改良土地。但这是例外;在这种情况下,有闲者会暂时成为生产劳动的领导者。撇开这些微不足道的例外情况,从再生产的角度来看,这类资本家的全部消费在一切方面都是一种纯损失,是已生产出来的财富的相应扣除部分。”(第 236 页)\end{quote}

\fontbox{~\{}具有真正斯密精神的政治经济学把资本家只看成人格化的资本,看成 G—W—G,看成生产当事人。但究竟谁来消费产品呢\fontbox{?}工人吗\fontbox{?}不,不是工人。资本家自己吗\fontbox{?}那他就成了大消费者、“有闲者”,而不是资本家了。土地租金和货币租金的所有者吗\fontbox{?}但他们不会把他们消费的东西再生产出来,因而只会损害财富。不过,在这种把资本家只看做现实的货币贮藏者,而不是看做象真正货币贮藏者那样的幻想的货币贮藏者的矛盾看法中,有两点是正确的:(1)资本(从而也就是资本家,资本的人格化)只被看做促使生产力和生产发展的当事人;(2)这里表现了上升的资本主义社会的观点,对这种社会具有意义的不是使用价值,而是交换价值,不是享受,而是财富。当上升的资本主义社会本身还没有学会把剥削和消费结合起来,还没有使享用的财富从属于自己时,享用的财富对它来说,是一种过度的奢侈。\fontbox{\}~}

\begin{quote}“要发现这种收入〈有闲者赖以生活的收入〉怎样形成,始终必须追溯到\textbf{产业资本家}。”(第 237 页注)“\textbf{产业资本家}〈第二种资本家〉包括所有经济部门的一切企业主,即一切\textbf{拥有资本}的人……他们把自己的才能和劳动用于自己利用资本,而不是把资本借给别人。因此,这种人不是靠工资过活,也不是靠收入过活,而是靠\textbf{利润}过活。”(第 237 页)\end{quote}

在德斯杜特的著作中可以明显地看出,正象在亚·斯密的著作中已经明显地看出那样,表面上是在赞美生产工人,实际上不过是赞美那些与土地所有者和单靠自己的收入过活的货币资本家相对立的“\textbf{产业资本家}”。

\begin{quote}“产业资本家……几乎把社会的全部财富掌握在自己手中……这些人在一年中,不仅支出这些财富的租金,而且支出资本本身;有时,如果事业的进展相当迅速,他们还有可能在一年内作几次这样的支出。因为他们作为实业家,支出只是为了使支出带着利润回到他们手里,所以,他们在这种条件下能够支出愈多,他们的利润就愈大。”(第 237—238 页)\end{quote}

至于他们的个人消费,那末,这种消费是同有闲资本家的个人消费一样的。不过,他们的个人消费

\begin{quote}“一般说来是适度的,因为实业家通常是简朴的”。(第 238 页)他们的生产消费就不一样了。“这种消费决不是最后的消费,它会带着利润回到他们手里。”(同上)他们的利润必定是相当大的,不仅足以供他们“个人消费,而且还”足以支付“从有闲资本家那里租借来的土地和货币的租金”。(第 238 页)\end{quote}

德斯杜特在这一点上是对的。土地的和货币的租金只不过是产业利润中的“\textbf{扣除部分}”,是产业资本家从自己的总利润中交给土地所有者和货币资本家的那一部分产业利润。

\begin{quote}“有闲的富人的收入,只不过是从生产中取得的租金;只有生产才创造这种收入。”(第 248 页)产业资本家“花费租金,租借他们〈有闲资本家〉的土地、房屋和货币,并以某种方式加以利用,由此\textbf{取得超出租金之上的利润}”,[第 237 页]就是说,他们支付给有闲者的这种租金,只是利润的一部分。他们这样支付给有闲资本家的这种租金,是“这些有闲者的唯一收入,是他们常年支出的唯一基金”。(第 238 页)\end{quote}

直到现在为止,谈的都很好。但“\textbf{雇佣工人}”(即产业资本家使用的生产工人)情况又怎样呢\fontbox{?}

\begin{quote}“他们除了自己日常的劳动之外,没有任何别的贮藏。这种劳动为他们提供工资……但工资是从哪里来的呢\fontbox{?}很明显,是从购买雇佣工人\textbf{出卖的劳动}的[402]那些人的财产中来的,也就是从事先掌握在雇主手中、\textbf{不外是积累起来的以前劳动的产品}的那种基金中来的。由此可以得出结论说,虽然由这种财富支付的消费,从它维持雇佣工人生活这个意义上来说,就是雇佣工人的消费,但实质上\textbf{支付消费的并不是雇佣工人},或者至少可以说,雇佣工人只是用\textbf{事先掌握在他们雇主手中的那种基金}来支付消费。因此,必须把他们的消费看作是雇用他们的那些人的消费。雇佣工人只不过是这只手拿进来,那只手还回去……不仅应当把他们〈雇佣工人〉所支出的一切,而且应当把他们所取得的一切,看成\textbf{购买他们的劳动的人}的实际支出和\textbf{这些人本身的消费}。这是千真万确的,所以要确定这种消费究竟是对现有财富会造成或多或少的损失,还是相反地会促使现有财富增加……就必须知道\textbf{资本家如何使用他们所购买的劳动},因为一切都取决于这一点。”(第 234—235 页)\end{quote}

好极了。但是企业主能够用来给自己和有闲资本家等等支付收入的利润,是从哪里来的呢\fontbox{?}

\begin{quote}“有人问我,这些产业主怎么能赚取这样大的利润,他们能够从谁手里取得这样大的利润。我回答说:那是\textbf{因为他们按高于生产成本的价格出卖他们生产的一切产品}。”(第 239 页)\end{quote}

但是他们把这一切卖得比成本贵,是卖给谁呢\fontbox{?}

\begin{quote}“(1)他们彼此出售用来满足他们需要的全部消费品;他们用自己的一部分利润来支付这些消费品;(2)他们把产品卖给他们自己雇用的和有闲资本家雇用的雇佣劳动者。通过这种途径,他们\textbf{从雇佣劳动者那里收回劳动者的全部工资},或许只有劳动者的少量积蓄除外;(3)他们把产品卖给有闲资本家。这种资本家把还没有付给自己直接雇用的雇佣劳动者的\textbf{那一部分收入支付给他们}。他们每年付给这种资本家的全部租金,就是通过这种或那种途径,再流回他们手里的。”(同上,第 239 页)\end{quote}

现在我们进一步考察这里划分的三项售卖。

(1)产业资本家自己吃掉自己产品(或利润)的\textbf{一部分}。他们决不能因互相欺骗、互相把自己的产品卖得比所\textbf{花费}的成本\textbf{贵}而发财致富。而且谁也不能用这种办法欺骗别人。如果 A 把自己的由产业资本家 B 吃掉的产品卖得过贵,那末 B 也会把自己的由产业资本家 A 吃掉的产品卖得过贵。这就好比 A 和 B 都按照实际价值互相出卖产品一样。这第一项告诉我们,资本家如何支出自己的一部分利润;它并没有告诉我们,资本家从哪里取得这部分利润。无论如何,他们“\textbf{彼此}”“按\textbf{高于}生产成本的价格\textbf{出卖}他们生产的一切产品”,并不能获得任何利润。

(2)同样,从他们按照\textbf{超过生产费用}的价格卖给自己工人的那部分产品中,他们也不能取得任何利润。根据假定,工人的全部消费实际上都是“购买他们的劳动的人本身的消费”。此外,德斯杜特还补充说,资本家把自己的产品卖给雇佣劳动者(他自己的劳动者和有闲资本家的劳动者)时,只是“收回劳动者的全部工资”。甚至不是全部,而是工人的积蓄除外。资本家把产品究竟是贱卖还是贵卖给工人,都是完全一样的,因为资本家始终只是收回他们给工人的东西,正如前面所说的,雇佣工人“只不过是这只手拿进来,那只手还回去”。资本家先把\textbf{货币}作为工资付给工人。然后他把自己的产品“过贵”卖给工人,从而收回货币。但是,因为工人还给资本家的货币,不能多于他从资本家那里取得的货币,所以,资本家把自己的产品卖给工人,也决不能\textbf{贵于}他\textbf{支付}工人劳动的代价。他在出卖自己的产品时从工人那里所能收回的货币,始终只和他给工人劳动支付的货币一样多。一文钱也多不了。资本家的货币量,怎么能由于这种“流通”而增加呢\fontbox{?}

[403]此外,德斯杜特还有一个荒谬看法。资本家 C 把周工资 1 镑支付给工人 A,然后卖给工人价值 1 镑的商品,从而取回这 1 镑。特拉西认为,通过这种办法,资本家就取回了全部工资。但是,他先给工人 1 镑货币,然后又给工人价值 1 镑的商品。可见,他事实上给工人的是 2 镑:1 镑商品和 1 镑货币。在这 2 镑当中,他以货币形式取回了 1 镑。因此,他事实上没有从 1 镑工资中取回一文钱。如果他要通过这样“取回”工资的办法(而不是通过工人用劳动把资本家以商品预付给工人的东西还给资本家的办法)来发财致富,那他很快就会碰壁。

这里,高贵的德斯杜特把货币流通和实际的商品流通混为一谈了。因为资本家不是直接给工人价值 1 镑的商品,而是给工人 1 镑货币,使工人现在能够自己决定购买什么商品,又因为工人在获得他应得的那份商品之后,会以货币形式把资本家拨给他的商品归还给资本家,所以德斯杜特就以为,当同一货币流回到资本家手里时,资本家就把工资“取回”了。就在同一页上,德斯杜特先生还说,流通现象“没有被很好地理解”(第 239 页)。的确,他自己就根本没有理解这种现象。如果德斯杜特不是用这种奇特的方式来说明“取回全部工资”,那末,这种荒谬看法,象我们马上就要提到的那样,至少还是可以想象的。

(但是,在这以前还要举下面这个例子来说明他的绝顶聪明。如果我走进一家店铺,店铺老板给我 1 镑,我用这 1 镑在他的店里购买价值 1 镑的商品,那末,他就取回了这 1 镑。谁也不会硬说店铺老板由于这桩买卖就变富了。他以前有 1 镑货币和价值 1 镑的商品,现在只有 1 镑货币。即使他的商品的价值只是 10 先令,而他按 1 镑的价钱卖给我,他也仍然比出卖商品前少了 10 先令,虽然他把他拿出的 1 镑全部取回了。)

如果资本家 C 给工人 1 镑工资,然后又把价值 10 先令的商品按 1 镑的价钱卖给工人,那末,他当然会得到 10 先令的利润,因为他卖给工人的商品贵 10 先令。但是从德斯杜特先生的观点来看,即使在这种情况下,也还是不能理解资本家的利润怎么会必然由此产生。(据说利润的产生,是由于资本家支付给工人的是降低了的工资,资本家在同工人的劳动交换时,事实上付给工人的那部分产品比\textbf{名义上}付给工人的少。)如果他给工人 10 先令,再把自己的商品按 10 先令的价钱卖给工人,那他就象给工人 1 镑,再把自己的价值 10 先令的商品按 1 镑的价钱卖给工人时一样富。而且德斯杜特的推论是从必要工资的前提出发的。在这里,对利润的全部说明,顶多也只能归结为工资上的诈骗勾当。

总之,这第二种情况表明,德斯杜特完全忘记了什么是生产工人,他对利润的源泉一窍不通。最多也只能说,在资本家不是把产品卖给自己的雇佣工人,而是卖给有闲资本家的雇佣劳动者的时候,他用高于产品价值出卖产品的办法来创造利润。但因为非生产劳动者的消费,事实上只是有闲资本家的消费的一部分,所以我们现在要考察第三种情况。

(3)第三,产业资本家把自己的产品高于产品价值过“贵”卖给

\begin{quote}“有闲资本家。这种资本家把还没有付给自己直接雇用的雇佣劳动者的那一部分收入支付给他们。他们每年付给这种资本家的全部租金,就是通过这种或那种途径,再流回他们手里的”。\end{quote}

这里谈的租金回流等等,就象前面谈的取回全部工资一样,又是一种幼稚的看法。例如,假定 C 把 100 镑土地的或货币的租金支付给 O(有闲资本家)。这 100 镑对于 C 来说,是支付手段。对于 O 来说,是购买手段。O 用这些货币从 C 的仓库里取得价值 100 镑的商品。这样,这 100 镑就作为 C 的商品的转化形式回到他手里。但是他现在比以前少了价值 100 镑的商品。他没有直接把商品给 O,而是把 100 镑货币给了 O,O 用这些货币向他购买价值 100 镑的商品。但 O 是用 C 的货币,不是用自己的基金来购买价值 100 镑的这些商品。而特拉西以为,通过这种办法,C 支付给 O 的租金会回到 C 手里。多么愚蠢!这是第一个荒谬之处。

第二,德斯杜特自己对我们说过,土地的和货币的租金只不过是产业资本家利润中的扣除部分,因而只不过是这种利润中交给有闲资本家的部分。如果现在假定,C 用某种诡计把这部分全部收回来[404](不过,靠特拉西所说的两种办法中的任何一种,都是绝对收不回来的),换句话说,如果假定资本家 C 既不向土地所有者,也不向货币资本家支付任何租金,他把自己的\textbf{全部}利润\textbf{都}留给自己,那末,正好需要说明的是,他究竟\textbf{从哪里}得到利润,他怎样创造利润,利润是如何产生的。如果说他没有把利润的一部分交给土地所有者和货币资本家,因而\textbf{拥有}利润或把利润\textbf{留给了自己}的说法不能说明这个问题,那末,同样,他以某种方法把以前在某种名义下交给有闲资本家的那部分利润,全部或部分地从有闲资本家口袋里取回来的说法也不能说明这个问题。这是第二个荒谬之处。

但是我们且把这些荒谬之处撇开不谈。C 由于向 O(有闲资本家)租借土地或货币,必须付给 O100 镑的租金。他从自己的利润中支付这 100 镑(而利润是从哪里产生的,我们还是不知道)。然后他把自己的产品卖给 O,这些产品或是直接由 O 自己吃掉,或是由他的食客(非生产的雇佣劳动者)吃掉;C 把这些产品\textbf{过贵}卖给他,例如比价值高 25\%。他把价值 80 镑的产品按 100 镑卖给 O。在这种情况下,C 无疑赚得 20 镑利润。C 给了 O 一张价值 100 镑商品的票据。可是当 O 拿这张票据去兑现时,C 付给他的只是价值 80 镑的商品,因为他把自己商品的名义价格比价值提高了 25\%。即使 O 满足于这种状况,即消费价值 80 镑的商品而支付 100 镑,C 的利润也决不会超过 25\%。这种价格,这种欺骗,会逐年继续下去。但是 O 想吃掉价值 100 镑的商品。如果他是土地所有者,他会怎么办呢\fontbox{?}他会把一块土地按 25 镑抵押给资本家 C,C 为此供给他价值 20 镑的商品,因为 C 是按高于商品价值 25\%(1/4)的价格出卖商品。如果 O 是货币贷放者,他就会从自己的资本中给资本家 C25 镑,资本家 C 为此供给他价值 20 镑的商品。

假定资本(或土地价值)按 5\%的利息借出。那时资本是 2000 镑。现在资本只有 1975 镑。这个 O 的租金现在等于 98+(3/4)镑。这种情形会继续下去:O 始终消费 100 镑实际的商品价值,他的租金则不断减少,因为要获得价值 100 镑的商品,他就始终必须吃掉自己资本的愈来愈大的部分。这样,C 就会逐渐把 O 的全部资本拿到自己手里,并且把他的租金,即把他用借来的资本获得的利润中原先交给有闲资本家 O 的那部分,连同资本一道据为己有。显然,德斯杜特先生也想到了这个过程,因为他接着说:

\begin{quote}“但是,有人会说,如果情形是这样,如果产业主确实\textbf{每年收获的比播种的多},那末,在很短的时间内,他们就一定会占有\textbf{全部社会财富},在国内很快就会只剩下没有财产的雇佣劳动者和资本主义企业主了。\textbf{这是对的}。只要企业主或他们的继承人在发财之后不是放弃经营,因而不是不断地补充有闲资本家阶级的队伍,情形的确会如此;尽管常有这种变迁,我们仍然可以看到,当一国的生产在一段时间内有所发展而没有发生太大的震动时,企业主的资本总是会不断增加,这种增加不仅同总财产的增加成比例,而且还大大超过……还可以补充一句,如果没有历届政府每年以赋税形式加给产业阶级的庞大负担,这种结果还会显著得多。”(第 240—241 页)\end{quote}

德斯杜特先生的话在某种程度上说是完全对的,不过他想要说明的那种东西根本不是这样。在中世纪衰落和资本主义生产上升的时期,“产业资本家”迅速致富,其中一部分原因,就是他们直接欺骗土地所有者。由于美洲的发现,货币价值降低了,租地农场主名义上而不是实际上继续向土地所有者支付原来的租金,而工业家却不仅按照提高了的货币价值,而且甚至高于商品的价值,把商品卖给这些土地所有者。同样,在国家的主要收入以地租形式掌握在土地所有者、君主等等手里的所有那些国家里,例如在亚洲国家,\textbf{人数不多}因而不受竞争影响的工业家,按照垄断价格把自己的商品卖给土地所有者、君主等等,从而把这些人的一部分收入据为己有,[405]他们不仅由于把“无酬”劳动卖给这些人,而且由于按照比商品中包含的更大的劳动量出卖商品而发财致富。不过,德斯杜特先生认为出借货币的资本家也同样受到欺骗,这就又不对了。相反,这些资本家取得高额利息,他们直接或间接地分享这种高额利润,参加这种欺骗。

下面这段话表明,德斯杜特先生脑中也浮现过这种现象:

\begin{quote}“我们只要看一看这样的事实:三、四百年前,在整个欧洲,同各种有权势的人物的庞大财富相比,他们〈产业资本家〉是弱小的,可是今天,他们的人数增加了,力量增强了,而那些人的财富却减少了。”(同上,第 241 页)\end{quote}

德斯杜特先生想给我们说明产业资本的\textbf{利润},而且是\textbf{高额利润}。他对这个问题做了双重的说明。第一,这些资本家以工资和租金形式支付的\textbf{货币},会流回他们手里,因为这些工资和租金会被用来购买他们的产品。而事实上,这只不过说明了,为什么他们不是\textbf{双重地}支付工资和租金——先是以货币形式支付,然后再以同一货币额的商品形式支付。第二个说明是,他们高于商品价格\textbf{过贵}出卖自己的商品:第一,贵卖给\textbf{自己},也就是欺骗自己和互相欺骗;第二,贵卖给工人,这又是欺骗自己,因为德斯杜特先生对我们说过,“必须把”雇佣工人的消费“看作是雇用他们的那些人的消费”(第 235 页);最后,第三,贵卖给\textbf{租金所得者},也就是欺骗这些人;这确实能说明,为什么产业资本家会把自己利润的愈来愈大的部分留给自己,而不把它交给有闲资本家。这能表明,为什么\textbf{总利润}在产业资本家和非产业资本家之间的\textbf{分配},会愈来愈牺牲后者而有利于前者。但这丝毫不能帮助我们理解这个\textbf{总利润是从哪里}来的。即使假定产业资本家占有了全部利润,也仍然有这一个问题:利润是从哪里来的\fontbox{?}

可见,德斯杜特什么也没有回答,他只不过暴露了,他把货币的回流看作商品本身的回流。这种\textbf{货币的回流}仅仅表示,资本家起初不是用商品而是用货币支付工资和租金;然后这些货币被用来购买他们的商品,这样,他们也就是以间接的方式用商品来支付了。因此,这些货币不断地流回他们手中,但只有在同一货币价值额的商品最终地从他们手里取走,而加入雇佣工人和租金所得者的消费的条件下,才会流回他们手中。

德斯杜特先生(纯粹按法国方式,我们在蒲鲁东那里也看到这种自我惊叹)完全是在叹赏

\begin{quote}“对我们财富的消费的这种考察……把社会整个运动解释得多么清晰。这种一致和这种清晰是从哪里来的呢\fontbox{?}来自我们遇到了真理。这使人想起了镜子的作用。如果我们站在适当的角度,事物就会清楚地并按照它们的正确比例反映出来。如果离得太近或太远,一切事物就会显得是混乱的和歪曲的”。(第 242—243 页)\end{quote}

下面,德斯杜特先生完全是附带地想起了亚·斯密书中所谈的事物的实际状态,但实质上他只是用一些词句复述这种事物的实际状态,并不了解其真正含意;不然的话,这位法国研究院\endnote{法国研究院——法国的最高科学机构,它由几个分院即学院组成;1795 年成立。德斯杜特·德·特拉西是伦理学和政治学学院院士。——第 288 页。}的院士就决不会放射出上述的“光流”来。德斯杜特写道(第 246 页):

\begin{quote}“这些有闲者的收入是从哪里来的呢\fontbox{?}不是来自租金吗\fontbox{?}而租金是由那些\textbf{使有闲者的资本发挥作用}的人,也就是由那些用有闲者的基金\textbf{雇用劳动,从而生产出比劳动本身的费用更多的产品}的人,一句话,由产业家从自己的\textbf{利润}中支付给有闲者的。”\end{quote}

\fontbox{~\{}啊哈!这就是说,产业资本家由于向有闲资本家借用基金而支付给后者的租金(以及他们自己的利润)是这样得到的:他们用这种基金雇用劳动,从而“\textbf{生产出比劳动本身的费用更多的产品}”,也就是说,这种劳动的产品比完成这种劳动的工人所得到的代价具有更大的价值;可见,利润来自雇佣工人所生产的、超过维持自己生活的费用的东西,即来自剩余产品。产业资本家占有这种剩余产品,只把其中的某一部分交给土地租金和货币租金的所得者。\fontbox{\}~}

但德斯杜特先生由此得出的结论是:不应追溯到这些生产工人,而应追溯到使用这些生产工人的资本家。他说:

\begin{quote}“实际上正是他们养活有闲者所雇用的雇佣劳动者。”(第 246 页)\end{quote}

这是不言而喻的。既然他们直接剥削劳动,而有闲资本家只是通过他们作中介来剥削劳动。在这个意义上,把产业资本看作财富的源泉是正确的。

\begin{quote}[406]“所以,要寻找一切财富的源泉,总是要追溯到这种人〈产业资本家〉。”(第 246 页)“久而久之,\textbf{财富就积累到相当的数量,因为以前劳动的成果不会一生产出来就都消费掉}。在这些财富的所有者当中,有一部分人满足于从财富取得租金并消费这些租金。这就是我们所谓的有闲资本家。另一部分比较积极的人,把自己的和从别人那里借来的基金运用起来。他们用这些基金\textbf{来支付劳动的报酬,而劳动把这些基金再生产出来,同时带来利润}。”\end{quote}

\fontbox{~\{}可见,这里不仅是把这些基金再生产出来,而且把构成\textbf{利润}的那个余额也生产出来了。\fontbox{\}~}

\begin{quote}“他们用这种利润支付他们自己的消费和支付别人的消费。由于这种消费〈他们自己以及有闲资本家的消费吗\fontbox{?}这又是以前那种荒谬说法〉,他们的基金回到他们手中,并有所增加,然后他们再从头开始。这也就是流通。”(第 246—247 页)\end{quote}

关于“生产工人”的研究及其结果——只有被产业资本家购买的工人,只有用劳动为劳动的直接购买者生产利润的工人,才是生产工人——使德斯杜特先生得出这样的结论:\textbf{产业资本家}实际上是\textbf{唯一的}最高意义上的\textbf{生产劳动者}。他说:

\begin{quote}“靠利润生活的人〈产业资本家〉养活其他一切人,只有他们能够增加公共财富,创造我们的全部享受资料。情况必定是这样,\textbf{因为劳动是一切财富的源泉},因为只有他们这些人才\textbf{有利地运用积累的劳动},从而\textbf{给现时的劳动指出有用的方向}。”(第 242 页)\end{quote}

说他们“给现时的劳动指出有用的方向”,事实上只不过是说,他们雇用有用劳动,雇用会生产出使用价值的劳动。但是,说他们“有利地运用积累的劳动”,如果这不应当还是指上面那个意思,即他们使用积累的财富来从事生产,来生产使用价值,那就是指他们“有利地运用积累的劳动”来购买比它们本身所包含的更多的现时的劳动。在刚刚引用的这句话中,德斯杜特天真地概括了构成资本主义生产实质的矛盾。因为劳动是一切财富的源泉,所以资本是一切财富的源泉;并且日益增长的财富的真正创造者不是从事劳动的人,而是从别人的劳动中取得利润的人。劳动的生产力就是资本的生产力。

\begin{quote}“我们的能力是我们唯一的原始财富;我们的劳动生产其他一切财富,而任何一种受到良好管理的劳动都是生产的。”(第 243 页)\end{quote}

从这里,按照德斯杜特的看法,当然可以得出结论说:产业资本家“养活其他一切人,只有他们能够增加公共财富,创造全部享受资料”。我们的能力是我们唯一的原始财富,所以劳动能力不是财富。劳动生产其他一切财富;这就是说,劳动为自己以外的其他一切人生产财富,而它本身不是财富,只有它的产品才是财富。任何一种受到良好管理的劳动都是生产的;这就是说,任何一种生产劳动,任何一种给资本家带来利润的劳动,都是受到良好管理的。

德斯杜特下面一些话不是就\textbf{消费者的不同阶级}说的,而是就\textbf{消费品的不同性质}说的,这些话很好地转述了亚·斯密的看法。亚·斯密在第二篇第三章末尾研究了哪一种(非生产)支出,即哪一种个人消费,收入的消费比较有利,哪一种支出比较不利。他在研究开始时用了这样的话(加尔涅的译本,第 2 卷第 345 页):

\begin{quote}“如果说节约增加资本总量,浪费减少资本总量,那末,收支相抵的人的行为,既不积累也不损及自己的基金,既不增加也不减少资本总量。不过,有一些花费收入的方法,看来比别的方法更能促进普遍福利的增长。”\end{quote}

德斯杜特这样概括斯密的论述:

\begin{quote}“消费因消费者的性质不同而大不相同,消费还因消费品的性质不同而有所变化。的确,一切物品都代表劳动,但是,劳动的价值固定在一种物品上的时间,比固定在另一种物品上的时间更持久。制造焰火可能与开采和琢磨钻石花费同样多的辛劳,因而前者可能和后者具有同样的价值。但是,当我把这两者买来,付了价钱并加以使用时,焰火过半小时就无影无踪了,而钻石过一百年还可能成为我的子孙的财富源泉……[407]有人〈即萨伊先生〉称为非物质产品的东西也是如此。\textbf{某种发现具有永久的效用}。某种文学作品,某一幅画,也具有相当长久的效用,而一个舞会、一个音乐会、一出戏剧的效用则是短暂的,转瞬即逝的。关于医生、律师、士兵、家仆以及所有一般称为\textbf{雇员}的人的\textbf{个人服务},也可以这样说。他们的效用只在需要他们的瞬间才存在……最快的消费是最有破坏性的消费,因为它会在同样的时间内消灭最大量的劳动,或者在最短的时间内消灭同量的劳动;和这种消费相比,任何一种较慢的消费都是一种\textbf{贮藏},因为它使我们有可能把今天放弃使用的部分留待将来享用……每个人都知道,如果\textbf{价格相同},买一件可以穿三年的衣服,比买一件只能穿三个月的衣服,要经济得多。”(第 243—244 页)\end{quote}

\tsectionnonum{[(15)对斯密关于生产劳动和非生产劳动的区分的反驳的一般特点。把非生产消费看成对生产的必要刺激的辩护论观点]}

大多数反驳斯密关于生产劳动和非生产劳动的区分的著作家,都把\textbf{消费}看作对生产的必要刺激。\textbf{因此},在他们看来,那些靠收入来生活的\textbf{雇佣劳动者},即非生产劳动者(对他们的雇用并不生产财富,而雇用本身却是财富的新的消费),\textbf{甚至从创造物质财富的意义来说},也和生产工人一样是生产劳动者,因为他们扩大物质消费的范围,从而扩大生产的范围。可见,这种看法大部分是从资产阶级经济学观点出发,一方面为有闲的富人和提供\textbf{服务}给富人消费的“非生产劳动者”辩护,另一方面为开支庞大的“强大政府”辩护,为国债的增加,为占有教会和国家的肥缺的人、各种领干薪的人等等辩护。因为所有这些非生产劳动者——他们的服务体现为有闲的富人的一部分支出——都有一个共同点,就是他们\textbf{生产“非物质产品”},但消费“\textbf{物质产品}”即生产工人的劳动产品。

另一些政治经济学家,例如马尔萨斯,虽然承认生产劳动者和非生产劳动者有区别,但是又向“产业资本家”证明,甚至就生产物质财富来说,非生产劳动者也象生产劳动者一样对他是必要的。

在这里,说生产和消费是等同的,或者说消费是一切生产的目的或生产是一切消费的前提,都毫无用处。撇开上述倾向不谈,作为全部争论的基础的,倒是下面这些:

工人的消费,平均起来只等于他的生产费用,而不等于他的产品。因此,全部余额都是工人为别人生产的,所以工人的这部分产品全是\textbf{为别人而生产}。其次,“产业资本家”迫使工人进行这种\textbf{剩余生产}(即超过工人本身生活需要的生产),并且运用一切手段来尽量增加这种同必要生产相对立的相对\textbf{剩余生产},直接把剩余产品据为己有。但是,作为人格化的资本,他是为生产而生产,想为发财而发财。既然他是资本职能的单纯执行者,即资本主义生产的承担者,他所关心的就是交换价值和它的增加,而不是使用价值和它的数量的增加。他只关心抽象财富的增加,对别人劳动的愈来愈多的占有。他象货币贮藏者一样,完全受发财的绝对欲望支配,所不同的只是,他并不以形成金银财宝的幻想形式来满足这种欲望,而是以形成资本的形式即实际生产的形式来满足这种欲望。工人的剩余生产是\textbf{为别人而生产},正常的资本家,即“产业资本家”的生产则是\textbf{为生产而生产}。当然,他的财富愈增加,他也就愈背弃这种理想而成为挥霍者,哪怕是为了显示一下自己的财富也好。不过,他始终是昧着良心、怀着精打细算的念头去享用财富。“产业资本家”无论怎样挥霍,他实质上仍然和货币贮藏者一样吝啬。

西斯蒙第说,劳动生产力的发展使工人有可能得到愈来愈多的享受,但这些享受如果给了工人,就使他(作为雇佣工人)不适宜于劳动了。[注:\textbf{西斯蒙第}说:“由于工业和科学的进步,每个工人每天所能生产的远远超过他自己所必需消费的。但在他的劳动生产财富的同时,这种财富如果供他自己消费,就使他不适宜于劳动了。”(《新原理》第 1 卷第 85 页)]如果是这样,那末,同样可以正确地说,“产业资本家”一旦成为享用财富的代表,一旦开始追求享受的积累,而不是积累的享受,他就或多或少不能执行自己的职能了。

可见,“产业资本家”也是\textbf{剩余生产}即\textbf{为别人而生产}的生产者。一方面有这种剩余生产,与此相对,另一方面必定有剩余消费,一方面是为生产而生产,与此相对,另一方面必定是为消费而消费。“产业资本家”必须交给地租所有者、国家、国债债权人、教会等等只消费收入的人的东西[408],固然绝对减少他的财富,但是使他发财的贪欲旺盛不衰,从而保存他的资本主义灵魂。如果土地租金和货币租金的所得者等等也把自己的收入花费在生产劳动上,而不花费在非生产劳动上,目的就不会达到。他们自己就会成为“产业资本家”,而不再代表消费的职能。以后我们还会知道,一个李嘉图主义者和一个马尔萨斯主义者之间,曾就这个问题展开过一场极为滑稽的争论。\endnote{马克思在他的手稿第 XIV 本(这一稿本收入本卷第三册)中,分析了马尔萨斯的观点之后,谈到两本匿名著作,其中一本从李嘉图的立场出发反对马尔萨斯,另一本维护马尔萨斯的观点,反对李嘉图学派。第一本题为《论马尔萨斯先生近来提倡的关于需求的性质和消费的必要性的原理;从这一原理所得的结论是:税收和供养非生产的消费者可以导致财富的增长》1821 年伦敦版。第二本题为《政治经济学大纲》1832 年伦敦版。——第 293 页。}

生产和消费是\textbf{内在地}[ansich]不可分离的。由此可以得出结论:因为它们在资本主义生产体系内实际上是分离的,所以它们的统一要通过它们的对立来恢复,就是说,如果 A 必须为 B 生产,B 就必须为 A 消费。正如每个资本家从他这方面说,都希望分享他的收入的人有所浪费一样,整个老重商主义体系也是以这样的观念为根据:一个国家从自己这方面必须节俭,但是必须为别的沉湎于享受的国家生产奢侈品。这里始终是这样的观念:一方是为生产而生产,因此另一方就是消费别国的产品。这种重商主义体系的观念在佩利博士的《道德哲学》一书第二卷第十一章中也表现出来:

\begin{quote}“节俭而勤劳的民族,用自己的活动去满足沉湎于奢侈的富有国家的需要。”\endnote{威廉·佩利《道德哲学和政治哲学原理》一书(1785 年伦敦版)的这段话,马克思引自托·罗·马尔萨斯《人口原理》法文本,比埃尔·普雷沃和吉约姆·普雷沃译自英文第五版,1836 年巴黎和日内瓦法文第三版第四卷第 109 页。——第 294 页。}\end{quote}

\centerbox{※     ※     ※}

\begin{quote}德斯杜特说:“他们〈“我们的政治家”,即加尔涅等人〉提出这样的总原则:消费是生产的原因,因而消费愈多愈好。他们硬说,正是这一点造成社会经济和私人经济之间的巨大差别。”(同上,第 249—250 页)\end{quote}

下面这句话也很好:

\begin{quote}“在\textbf{贫国},人民是安乐的,在\textbf{富国},人民通常是贫苦的。”(同上,第 231 页)\end{quote}

\tsectionnonum{[(16)]昂利·施托尔希[对物质生产和精神生产相互关系问题的反历史态度。关于统治阶级的“非物质劳动”的见解]}

\textbf{昂利·施托尔希}《政治经济学教程》,让·巴·萨伊出版,1823 年巴黎版(这是为尼古拉大公讲授的讲义,完成于 1815 年。)\textbf{第三卷}。

在加尔涅之后,施托尔希事实上是第一个试图以新的论据来反驳斯密对生产劳动和非生产劳动的区分的人。

他把“\textbf{内在财富}即文明要素”同物质生产的组成部分——物质财富区别开来,“文明论”应该研究文明要素的生产规律(同上,第 3 卷第 217 页)。(在第一卷第 136 页上,我们读到:

\begin{quote}“显然,人在没有内在财富之前,即在尚未发展其体力、智力和道德力之前,是决不会生产财富的,而要发展这些能力,必须先有手段,如各种\textbf{社会设施}等等。因此,一国人民愈文明,该国国民财富就愈能增加。”反过来也一样。)\end{quote}

他反对斯密说:

\begin{quote}“斯密……把一切不\textbf{直接}参加财富生产的人排除在\textbf{生产劳动}之外;不过他所指的只是国民\textbf{财富}……他的错误在于,他没有对\textbf{非物质}价值和\textbf{财富}作出应有的区分。”(第 3 卷第 218 页)\end{quote}

事情其实就此完了。生产劳动和非生产劳动的区分,对于斯密所考察的东西——物质财富的生产,而且是这种生产的一定形式即资本主义生产方式——具有决定性的意义。在精神生产中,表现为生产劳动的是另一种劳动,但斯密没有考察它。最后,两种生产的相互作用和内在联系,也不在斯密的考察范围之内;而且,物质生产只有从它本身的角度来考察,才不致流于空谈。如果说斯密曾谈到并非直接生产的劳动者,那只是因为这些人\textbf{直接}参加物质财富的消费,而不是参加物质财富的生产。

从施托尔希的著作本身来看,他的“\textbf{文明论}”虽然有一些机智的见解,例如说物质分工是精神分工的前提,但是依然脱不掉陈词滥调。\textbf{仅仅}由\textbf{一个}情况就可以看出,施托尔希的著作\textbf{必然}会如此,他甚至连\textbf{表述}这个问题都还远远没有做到,更不用说解决这个问题了。要研究精神生产[409]和物质生产之间的联系,首先必须把这种物质生产本身不是当作一般范畴来考察,而是从\textbf{一定的历史的}形式来考察。例如,与资本主义生产方式相适应的精神生产,就和与中世纪生产方式相适应的精神生产不同。如果物质生产本身不从它的\textbf{特殊的历史的}形式来看,那就不可能理解与它相适应的精神生产的特征以及这两种生产的相互作用。从而也就不能超出庸俗的见解。这一切都是由于“文明”的空话而说的。

其次,从物质生产的一定形式产生:第一,一定的社会结构;第二,人对自然的一定关系。人们的国家制度和人们的精神方式由这两者决定,因而人们的精神生产的性质也由这两者决定。

\textbf{最后},施托尔希所理解的精神生产,还包括统治阶级中专门执行社会职能的各个阶层的职业活动。这些阶层的存在以及他们的职能,只有根据他们生产关系的一定的历史结构才能够理解。

因为施托尔希不是\textbf{历史地}考察物质生产本身,他把物质生产当作一般的物质财富的生产来考察,而不是当作这种生产的一定的、历史地发展的和特殊的形式来考察,所以他就失去了理解的基础,而只有在这种基础上,才能够既理解统治阶级的意识形态组成部分,也理解一定社会形态下自由的精神生产。他没有能够超出泛泛的毫无内容的空谈。而且,这种关系本身也完全不象他原先设想的那样简单。例如资本主义生产就同某些精神生产部门如艺术和诗歌相敌对。不考虑这些,就会坠入莱辛巧妙地嘲笑过的十八世纪法国人的幻想。\endnote{马克思指莱辛在他的《汉堡戏剧论》(1767—1768 年)中同伏尔泰的论战。——第 296 页。}既然我们在力学等等方面已经远远超过了古代人,为什么我们不能也创作出自己的史诗来呢\fontbox{?}于是出现了《亨利亚特》\endnote{《亨利亚特》是伏尔泰写的关于法国国王亨利四世的长诗,于 1723 年第一次出版。——第 296 页。}来代替《伊利亚特》。

但是,施托尔希在专门反对加尔涅这个最早对斯密进行\textbf{这种}反驳的人的时候,所强调指出的东西则是正确的。那就是:他强调指出反对斯密的人把问题完全弄错了。

\begin{quote}“批评斯密的人做些什么呢\fontbox{?}他们完全没有弄清这种区分〈“非物质价值”和“财富”之间的区分〉,他们把这两种显然不同的价值完全混淆起来。〈他们硬说,精神产品的生产或服务的生产就是\textbf{物质}生产。〉他们把非物质劳动看做\textbf{生产劳动},认为这种劳动\textbf{生产}〈即直接生产〉\textbf{财富},即物质的、可交换的价值;其实,这种劳动只生产非物质的、直接的价值;批评斯密的人则根据这样的假定,即非物质劳动的产品也象物质劳动的产品一样,受同一规律支配;其实,支配前者的原则和支配后者的原则并不相同。”(第 3 卷第 218 页)\end{quote}

我们要指出施托尔希的下面这些常被后来的著作家抄引的论点:

\begin{quote}“因为内在财富有一部分是服务的产品,所以人们便断言,内在财富不比服务本身更耐久,它们必然是随生产随消费。”(第 3 卷第 234 页)“原始的内在财富决不会因为它们被使用而消灭,它们会由于不断运用而增加并扩大起来,所以,它们的\textbf{消费}本身会增加它们的价值。”(同上,第 236 页)“内在财富也象一般财富一样,可以积累起来,能够形成资本,而这种资本可以用来进行再生产”等等。(同上,第 236 页)“在人们能够开始考虑非物质劳动的分工以前,必须先有物质劳动的分工和物质劳动产品的积累。”(第 241 页)\end{quote}

这一切只不过是精神财富和物质财富之间的最一般的表面的类比和对照。例如他的下面那种说法也是如此,他说,不发达的国家从外国\textbf{吸取}自己的精神资本,就象物质上不发达的国家从外国吸取自己的物质资本一样(同上,第 306 页);他还说,非物质劳动的分工决定于对这种劳动的需求,一句话,决定于市场,等等(第 246 页)。

下面这些话是直接抄来的:

\begin{quote}[410]“内在财富的\textbf{生产}决不会因为它所需要的物质产品的消费而使国民财富减少,相反,它是促进国民财富增加的有力手段”,反过来也是一样,“财富的生产也是增进文明的有力手段”。(同上,第 517 页)“国民福利因这两种生产的平衡而不断增长。”(第 521 页)\end{quote}

施托尔希认为,医生生产健康(但他也生产疾病),教授和作家生产文化(但他们也生产蒙昧),诗人、画家等等生产趣味(但他们也生产乏味),道德家等等生产道德,传教士生产宗教,君主的劳动生产安全,等等(第 347—350 页)。但是同样完全可以说,疾病生产医生,愚昧生产教授和作家,乏味生产诗人和画家,不道德生产道德家,迷信生产传教士,普遍的不安全生产君主。这种说法事实上是说,所有这些活动,这些“服务”,都生产现实的或想象的使用价值;后来的著作家不断重复这种说法,用以证明上述这些人都是斯密所谓的生产劳动者,也就是说,他们直接生产的不是特殊种类的产品,而是物质劳动的产品,所以他们直接生产财富。在施托尔希的书中还没有这种荒谬说法。其实这种荒谬说法完全可以由下面各点来说明:

(1)在资产阶级社会中,各种职能是互为前提的;

(2)物质生产领域中的对立,使得由各个意识形态阶层构成的上层建筑成为必要,这些阶层的活动不管是好是坏,因为是必要的,所以总是好的;

(3)一切职能都是为资本家服务,为资本家谋“福利”;

(4)连最高的精神生产,也只是由于被描绘为、被错误地解释为物质财富的直接生产者,才得到承认,在资产者眼中才成为\textbf{可以原谅的}。

\tsectionnonum{[(17)]纳骚·西尼耳[宣称对资产阶级有用的一切职能都是生产的。对资产阶级和资产阶级国家阿谀奉承]}

\textbf{威·纳骚·西尼耳}《政治经济学基本原理》,让·阿里瓦本译,1836 年巴黎版。

纳骚·西尼耳摆出一副高傲的样子说:

\begin{quote}“照斯密看来,犹太人的立法者是非生产劳动者。”(同上,第 198 页)\end{quote}

这是指埃及的摩西,还是指摩西·门德尔森\fontbox{?}摩西将会因自己被称为斯密所谓的“生产劳动者”,而十分感谢西尼耳先生吧。这些人如此拘守于自己的资产阶级固定观念,以致认为,如果把亚里士多德或尤利乌斯·凯撒称为“非生产劳动者”,那就是侮辱他们。其实,单是“劳动者”这个名称,就会使亚里士多德和凯撒感到侮辱了。

\begin{quote}“一个医生开药方把病孩治好,从而使他的生命延续好多年,这个医生难道不是生产持久的结果吗\fontbox{?}”(同上)\end{quote}

胡说八道!如果孩子死了,结果同样是持久的。如果孩子的病没有治好,医生的\textbf{服务}还是要得到报酬的。照纳骚看来,医生只有把病治好,律师只有把官司打赢,士兵只有把仗打胜,才能得到报酬了。

但是他现在变得真正崇高起来了,他说:

\begin{quote}“起来反抗西班牙人的暴政的荷兰人,或者起义反对有可能变得更可怕的暴政的英国人,难道只生产了短暂的结果吗\fontbox{?}”(同上,第 198 页)\end{quote}

真是美文学式的废话!荷兰人和英国人举行起义是靠自己负担费用。谁也没有为了他们“在革命中”劳动而给他们支付代价。在关于生产劳动者或非生产劳动者的问题上涉及的始终是劳动的买者和卖者。多么愚蠢!

这些家伙在对斯密的反驳中发表的庸俗的美文学,不过表明他们是“有教养的资本家”的代表,而斯密则是露骨粗鲁的资产者暴发户的\textbf{解释者}。有教养的资产者及其代言人非常愚蠢,竟用对钱袋的[411]影响来衡量每一种活动的意义。另一方面,他们又很有教养,连那些同财富的生产毫不相干的职能和活动,也加以\textbf{承认},而且他们之所以加以承认,是因为这些活动会“间接地”使他们的财富增加等等,总之会执行一种对财富“有用的”职能。

人本身是他自己的物质生产的基础,也是他进行的其他各种生产的基础。因此,所有对人这个生产\textbf{主体}发生影响的情况,都会在或大或小的程度上改变人的各种职能和活动,从而也会改变人作为物质财富、商品的创造者所执行的各种职能和活动。在这个意义上,确实可以证明,所有人的关系和职能,不管它们以什么形式和在什么地方表现出来,都会影响物质生产,并对物质生产发生或多或少是决定的作用。

\begin{quote}“有些国家,没有士兵的守卫便根本不可能耕种土地。可是,按照斯密的分类法,收成并不是扶犁的人和手执武器守卫在他旁边的人共同劳动的产品;照斯密的说法,只有土地耕种者才是生产劳动者,士兵的活动则是非生产的。”(同上,第 202 页)\end{quote}

第一,这是错误的。斯密会说,士兵的活动生产保卫,但不生产谷物。如果国内建立了秩序,那末土地耕种者就会象以前一样继续生产谷物,但不必另行生产士兵的给养,从而不必生产士兵的生命。士兵象很大一部分非生产劳动者一样,属于生产上的非生产费用\authornote{见第 159 页脚注。——编者注},这些非生产劳动者,无论在精神生产领域还是在物质生产领域,都什么也不生产,他们只是由于社会结构的缺陷,才成为有用的和必要的,他们的存在,只能归因于社会的弊端。

但纳骚会说,如果发明一种机器,使 20 个工人中有 19 个人成为多余的,那末这 19 个人也就成了生产上的非生产费用了。但是,尽管\textbf{生产的物质条件}、耕作本身的条件保持不变,士兵也可能成为多余的。而 19 个工人却只有在剩下的那 1 个工人的劳动的生产能力提高到 20 倍之后,因而只有在生产的现有物质条件发生革命之后,才能成为多余的。而且\textbf{布坎南}已经指出:

\begin{quote}“比方说,如果士兵由于他的劳动有助于生产,便应当被称为生产劳动者,那末生产工人就有同样的权利要求得到军人的荣誉了,因为毫无疑问,没有生产工人的协助,任何军队也不能上战场去打仗并取得胜利。”(\textbf{大·布坎南}《论斯密博士的〈国民财富的性质和原因的研究〉的内容》1814 年爱丁堡版第 132 页)“一个国家的财富不取决于生产\textbf{服务}的人和生产\textbf{价值}的人之间的人数比例,而取决于这两种人之间最能使每种人的劳动具有最大生产能力的那种比例。”(\textbf{西尼耳},同上第 204 页)\end{quote}

斯密从来没有否认这一点,因为他想使国家官吏、律师、教士等等这些“必要的”非生产劳动者,减少到非有他们的服务不可的\textbf{限度}。无论如何,这也就是他们能使生产工人的劳动具有最大的生产能力的那种“比例”。至于其他的“非生产劳动者”,因为他们的劳动是每一个人为了享用他们的\textbf{服务}而\textbf{随意}购买的,也就是说,是每一个人把它作为随便挑选的消费品来购买的,所以,就可能出现各种不同的情况。这些靠收入过活的劳动者的人数,同“生产”工人的人数相比可能很多,\textbf{第一},是因为财富一般说来并不很多或者带有片面性,例如,中世纪的贵族及其仆从的情形就是这样。他们和他们的仆从不是消费相当数量的工业品,而是吃掉自己的农产品。当他们不是这样而开始消费工业品的时候,他们的仆从就不得不从事劳动了。靠收入过活的人所以这样多,只是因为有很大一部分年产品不是\textbf{为了再生产}而消费。但是,人口的总数并不很多。\textbf{第二},靠收入过活的人数可能很多,是因为生产工人的生产率高,即他们生产的用来养活仆从的剩余产品多。在这种情况下,不是因为有这么多的仆从,生产工人的劳动才是生产的,相反,是因为生产工人的劳动具有这样大的生产能力,所以才有这么多的仆从。

如果两个国家人口相等,劳动生产力的发展水平相同,那就始终有充分的理由可以同亚·斯密一起说:两国的财富应由生产劳动者和非生产劳动者之间的比例来衡量。因为这不过表明,在生产工人人数较多的国家里,有较大量的年收入是为了再生产而消费,因而每年会生产较大量的价值。可见,西尼耳先生只不过是复述[412]亚当的论点,并没有提出什么新思想来同亚当相对立。后来,他自己在这里也区分了“服务的生产者”和“价值的生产者”,这样一来,他也就和大多数反对斯密的区分的人们一样:他们接受并且自己也采用了他们所反驳的那种区分。

值得注意的是:一切在自己的专业方面毫无创造的“非生产的”经济学家,都反对生产劳动和非生产劳动的区分。但是,对于资产者来说,“非生产的”经济学家们的这种立场,一方面表示阿谀奉承,力图把一切职能都说成是为资产者生产财富服务的职能;另一方面表示力图证明资产阶级世界是最美好的世界,在这个世界中一切都是有用的,而资产者本人又是如此有教养,以致能理解这一点。

对于工人来说,这种看法是要人们确信,非生产人员消费大量产品完全是理所当然的,因为非生产消费者象工人一样能促进财富的生产,不过是以自己特殊的方式罢了。

但是,纳骚终于说漏了嘴,表明他对斯密所做的本质区分一窍不通。他说:

\begin{quote}“看来,实际上在这种场合,斯密把注意力完全集中在\textbf{大土地所有者}的状况方面。他关于非生产阶级的意见一般只适用于这一种人。否则,我就无法解释他的这种论断:\textbf{资本只用来维持生产劳动者,而非生产劳动者则靠收入过活}。在他突出地称为非生产劳动者的人们当中,最大一部分人,象教师、治理国家的人,都是\textbf{靠资本}维持,即\textbf{靠预付在再生产中的资金}维持。”(同上,第 204—205 页)\end{quote}

这里,确实使人惊讶得目瞪口呆。纳骚先生关于国家和学校教师靠资本生活,而不是靠收入生活的发现,无须作进一步的注解。如果西尼耳先生想用这些话告诉我们,他们是靠资本的利润生活,在这个意义上也就是靠资本生活,那末,他只是忘记了,资本的收入并不是资本本身,这个收入,这个资本主义生产的结果,并不预付在再生产中,相反,它本身倒是再生产的结果。或者,西尼耳这样想是因为有一些税收加入某些商品的生产费用,因而加入某些生产的开支\fontbox{?}那他应该知道,这只是对收入课税的一种形式。

关于施托尔希,纳骚·西尼耳这个自作聪明的家伙还指出:

\begin{quote}“施托尔希先生断言,这些\textbf{结果}〈健康、趣味等等〉象其他有价值的物品一样,是拥有这些结果的人的\textbf{收入}的一部分,并且同样可以交换〈就是说,可以从它们的生产者手里购买〉;他无疑是错了。如果是这样,如果趣味、道德、宗教确实是可以\textbf{购买}的\textbf{物品},那末,财富的意义就和经济学家们……所说的完全不同了。我们购买的决不是健康、知识和虔诚。医生、牧师、教师……只能生产那种用以多少可靠地和完善地把这些进一步的结果生产出来的手段……既然在每一种情况下,为了取得成就,都要使用最合适的手段,那末,即使没有取得成就,没有达到预期的结果,这些\textbf{手段}的生产者也有权得到报酬。一旦提出了劝告或授了课,因此得到了报酬,交换也就完成了。”(同上,第 288—289 页)\end{quote}

最后,伟大的纳骚自己又接受了斯密的区分。就是说,他以“生产消费和非生产消费”的区分(第 206 页)来代替生产劳动和非生产劳动的区分。而消费品要么是商品,——但这里不是谈商品,——要么直接是劳动。

根据西尼耳的说法,生产消费,是指使用这样一种劳动的消费,这种劳动或者再生产劳动能力本身(例如教师或医生的劳动),或者\textbf{再生产}用来购买这种劳动的那些商品的价值。非生产消费,则是指既不生产前者也不生产后者的那种劳动的消费。而斯密说:只能用于生产的(即产业的)消费的劳动,我称为生产劳动,能够用于非生产的消费的劳动(这种劳动的消费按其性质来说不是生产消费),我称为非生产劳动。可见,西尼耳先生在这里是靠事物的新名称来证明自己的才智。

总的说来,纳骚是抄袭施托尔希的著作。

\tsectionnonum{[(18)]佩·罗西[对经济现象的社会形式的忽视。关于非生产劳动者“节约劳动”的庸俗见解]}

[413]\textbf{佩·罗西}《政治经济学教程》(1836—1837 年讲授)1842 年布鲁塞尔版。

聪明就在这里!

\begin{quote}“\textbf{间接的}〈生产〉\textbf{手段}包括一切能促进生产,有助于消除障碍,使生产更有效、更迅速、更简便的东西。〈在此之前,他在第 268 页上说:“有直接的生产手段和间接的生产手段。也就是说,有些生产手段是取得我们所关心的结果的\textbf{必要}条件,是\textbf{完成}这种生产的力量;另一些生产手段有助于生产,但不是进行生产。前者甚至能够\textbf{单独}起作用,后者只能在生产过程中帮助前者。”〉……任何一种政府劳动都是间接的生产手段……制造这顶帽子的人必须承认,在街上巡逻的宪兵、坐在法庭上的法官、关押犯人的狱吏、守卫国境防止敌人侵犯的军队,所有这些人都促进生产。”(第 272 页)\end{quote}

制造帽子的人意识到,为了使他能够生产和出卖这顶帽子,全世界都动了起来,他是多么高兴!罗西让这个狱吏等等\textbf{间接地}——不是\textbf{直接地}——促进物质生产,事实上就是作了和亚当同样的区分(见第十二讲)。

罗西在下一讲即第十三讲里专门攻击斯密——其实罗西同他的先辈们几乎一样。

他说,对生产劳动者和非生产劳动者的错误区分,是由三个原因造成的。

\begin{quote}(1)“在\textbf{买者}当中,一部分人购买产品或\textbf{劳动},是\textbf{为了个人直接消费它们};另一部分人购买它们,只是为了把他们用购得的产品和买到的劳动制造的新产品出卖。对于前一种人来说,有决定意义的是\textbf{使用价值},对于后一种人来说,有决定意义的是交换价值。”当人们只注意交换价值时,就会犯斯密的错误。“我的仆人的劳动对我来说是非生产的,——暂且承认这一点;但是,难道这种劳动对他自己来说也是非生产的吗\fontbox{?}”(同上,第 275—276 页)\end{quote}

既然整个资本主义生产的基础是:直接购买劳动,以便在生产过程中\textbf{不经购买}而占有所使用的劳动的一部分,然后又以产品形式把这一部分\textbf{卖掉};既然这是资本存在的基础,是资本的实质,那末,生产资本的劳动和不生产资本的劳动二者之间的区分,不就是理解资本主义生产过程的基础吗\fontbox{?}斯密并不否认,仆人的劳动对\textbf{他}自己来说是生产的。每种服务对它的卖者来说都是生产的。假誓约对那个靠假誓约获得现金的人来说是生产的。伪造文件对那个靠伪造文件赚钱的人来说是生产的。杀人对那个因杀人而得到报酬的人来说是生产的。诬陷者、告密者、食客、寄生者、谄媚者,只要他们的这种“服务”不是无酬的,他们的这些勾当对他们来说就都是生产的。按照罗西的看法,所有这些人都是“生产劳动者”,不仅是财富的生产者,而且是资本的生产者。自己给自己支付报酬的骗子手,——同法官和国家所做的完全一样,——也是“按照一定的方式,使用一种力量,生产一种满足人的需要的结果”[同上,第 275 页],就是说,满足盗贼的需要,也许还满足他的妻子儿女的需要。这样说来,如果全部问题只在于生产一种满足“需要”的“结果”,或者说,如果一个人只要出卖自己的“服务”就可以把这种服务算作“生产的”,就象上述情况那样,那末,这个骗子手就是生产劳动者了。

\begin{quote}(2)“第二个错误是没有区分直接生产和间接生产。因此,在亚·斯密看来,官吏是非生产的。如果〈没有官吏的劳动〉生产就几乎不可能进行,那就很清楚,这种劳动对于生产是有帮助的,即使没有直接的物质的帮助,至少还有不应忽视的间接的作用。”(同上,第 276 页)\end{quote}

这种间接参加生产的劳动(它不过是非生产劳动的一部分),我们也称为非生产劳动。否则就必须说,因为官吏没有农民就绝对不能生活,所以农民是司法等等的“间接生产者”。真是胡说八道!还有一个同分工问题有关的观点,等以后再谈。

\begin{quote}[(3)]“没有仔细区分生产现象的三个基本事实:\textbf{力量即生产手段},这种力量的\textbf{使用,结果}。”我们向钟表业者买一只表;这时我们关心的只是劳动的\textbf{结果}。或者我们向裁缝买一件上衣;情况也是一样。但是“还有一种老古板的人,他们不是这样对待事物。他们叫一个工人到家里来,供给他材料和一切必需的东西,要他做一件衣服。这些老古板的人所购买的是什么呢\fontbox{?}他们购买的是力量\fontbox{~\{}但还有“这种力量的使用”\fontbox{\}~},是冒风险生产某种结果的手段……契约的对象是对力量的购买”。\end{quote}

(然而问题正是在于,这些“老古板的人”所使用的生产方式同资本主义的生产方式毫无共同之处,在这种生产方式下,不可能有资本主义生产所带来的劳动生产力的全部发展。值得注意的是,这种特殊区别在罗西之流看来是非本质的区别。)

\begin{quote}“在雇用一个仆人的场合,我是购买一种力量,这种力量可以被利用来完成多种多样的服务,这种力量活动的结果取决于我如何使用它。”(第 276 页)\end{quote}

这一切都同问题毫无关系。

[414]

\begin{quote}“可以购买或雇用……对某种力量的一定使用权……在这种情况下,您购买的就不是产品,不是您心目中的结果了。”律师的辩护词也许能,也许不能使我打赢官司。“无论如何,您和您的律师之间的交易,都是他为取得一定价值而在某日某地替您说话,为您的利益而运用他的智力。”(第 276 页)\end{quote}

\fontbox{~\{}对此罗西还有一点意见。他在第十二讲(第 273 页)中说:

\begin{quote}“我决不认为只有靠生产棉布或制作靴子生活的人才是生产者。无论哪种劳动我都尊重……但这种尊重不应成为\textbf{体力劳动者}独占的特权。”\end{quote}

亚·斯密不是这样看的。他认为从事写作、绘画、作曲、雕塑的人是第二种意义的“生产劳动者”,虽然即兴诗人、演说家、音乐家等等不是这样的劳动者。而“服务”只要是直接加入生产的,亚·斯密就把它看作是物化在产品中的,不管这是体力劳动者的劳动,还是经理、店员、工程师的劳动,甚至学者的劳动(只要这个学者是个发明家,是在工场内或在工场外劳动的工场劳动者)。斯密在谈到分工的时候,曾说明这些业务如何在各种人员之间分配,并指出产品、商品是他们共同劳动的结果,不是其中某一个人劳动的结果。不过,象罗西这样的“精神的”劳动者所关心的,是如何为他们从物质生产中取得的那个巨大的份额辩护。\fontbox{\}~}

发了这段议论之后,罗西接着说:

\begin{quote}“这样,在交换行为中,人们把注意力集中在生产的三个基本事实的某一个上面。但是\textbf{这些不同的交换形式}是否能使某些\textbf{产品}失去\textbf{财富}的性质,使\textbf{某一生产者阶级的努力}失去\textbf{生产劳动的性质}呢\fontbox{?}显然,在这些观念之间并没有任何可以证实这种结论的联系。难道因为我不是购买某种结果,而是购买生产这种结果所必要的力量,这个\textbf{力量的活动就不会是生产的,产品就不会是财富了吗}\fontbox{?}我们再以裁缝为例。无论是向裁缝买一件现成的衣服,还是把材料和工钱给裁缝工人,要他缝一件衣服,这两种情况从结果来看始终是一样的。谁也不会说第一种劳动是\textbf{生产劳动},第二种劳动是\textbf{非生产劳动};区别只是,在第二种情况下,\textbf{想要得到衣服}的人是\textbf{他自己的雇主}。但是,从生产力方面来看,您叫到家里来的裁缝工人和您的仆人之间又有什么区别呢\fontbox{?}没有任何区别。”(同上,第 277 页)\end{quote}

这位妄自尊大的空谈家,他的全部假聪明的精华就在这里!如果亚·斯密根据他的第二个比较浅薄的见解,即根据劳动是否直接物化在劳动的买者可以出卖的商品中这一点,来区分生产劳动和非生产劳动,那末,他就会把这两种情况下的裁缝都叫做生产劳动者。但是按照他的较为深刻的见解来看,上述第二种情况下的裁缝就是“非生产劳动者”。罗西只不过表明,他“显然”不懂亚·斯密的意思。

罗西以为“\textbf{交换形式}”是无关紧要的,就好比生理学家说,一定的生命形式是无关紧要的,因为它们都只是有机物的形式。但当问题是要了解某一社会生产方式的特殊性质时,恰好只有这些形式才是重要的。上衣就是上衣。但如果它是在第一种交换形式下生产出来的,那就是资本主义生产和现代资产阶级社会;如果它是在第二种交换形式下生产出来的,那就是某种甚至和亚洲关系或中世纪关系等等相适应的手工劳动形式。所以,这些\textbf{形式}对于物质财富本身是有决定作用的。

上衣就是上衣,罗西的绝顶聪明就表现在这一点上。但是,在第一种情况下,裁缝工人不只生产上衣,他生产资本,就是说,也生产利润;他把自己的雇主作为资本家生产出来,也把自己作为雇佣工人生产出来。如果我把裁缝工人叫到家里来为我个人缝上衣,我决不因为这一点而成为自己的\textbf{企业主}(从一定经济范畴的意义上说),就象\textbf{缝纫企业主}决不是因为[415]他把他的工人缝的上衣拿来自己穿和自己消费而成为企业主一样。在一种情况下,裁缝劳动的买者和裁缝工人是作为单纯的买者和卖者相对立。一个支付货币,另一个供给商品,我的货币就转化为这个商品的使用价值。这种情形和我从商店里买一件上衣毫无区别。卖者和买者在这里,是单纯作为卖者和买者相对立。相反,在另一种情况下,他们则是作为资本和雇佣劳动相对立。至于仆人,他同第二种情况下的裁缝工人(在这种情况下,我购买他的劳动是为了它的使用价值)有共同之处,那就是他们两者具有同样的社会形式。两者都是单纯的买者和卖者。区别只在于,这里由于在利用所购买的使用价值上的特殊方式,还发生一种宗法制的关系,主人和奴仆的关系,这就使这种单纯买卖的关系在内容上——即使不是在经济形式上——发生形态变化,成为令人厌恶的事情。

此外,罗西不过是用另一种说法重复加尔涅的意见。

\begin{quote}“我们坦率地说,当斯密断言仆人的劳动不会留下任何痕迹时,他犯了他这样的人所不应当犯的大错误。假定有一个工厂主,他自己管理一个需要严加监督的大工厂……这个人不容许在自己的身边有非生产劳动者,不雇用家仆。因而,他不得不\textbf{自己服侍自己}……当他必须从事这种所谓非生产劳动的时候,他将怎样进行他的生产劳动呢\fontbox{?}您的仆人所完成的工作使您能够从事更适合于您的能力的劳动,这难道还不明白吗\fontbox{?}因此,怎么能够说仆人的服务不会留下任何痕迹呢\fontbox{?}您所做的,以及没有仆人替您服侍贵体和收拾家务您就不可能做到的,这一切都会留下来的。”(同上,第 277 页)\end{quote}

这又是加尔涅、罗德戴尔和加尼耳已经说过的\textbf{节约劳动}。按照这种看法,非生产劳动只要在如下的情况下就是生产的:它们节约劳动,并且使“产业资本家”或者生产工人有更多的时间从事自己的劳动,由于别人代替他们去完成价值较小的劳动,他们就能完成价值较大的劳动。即使这样,仍然有很大一部分非生产劳动者不能包括在内,例如只当作奢侈品的那些家仆,以及所有这样的非生产劳动者:他们只生产享受,并且只有在我\textbf{为享用他们的劳动而花费的时间同这种劳动的卖者为生产这种劳动}(完成这种劳动)\textbf{而花费的时间一样多的时候},我才能享用他们的劳动。在这两种情况下,都谈不到“节约”劳动。最后,甚至真正节约劳动的个人服务,也只有在它们的消费者是生产劳动者的情况下,才是生产的。如果它们的消费者是个有闲资本家,那末它们节约他的劳动,不过意味着让他可以什么事都不干。例如,猪一样脏的懒女人自己不动手,而叫别人替她梳头、剪指甲;乡绅自己不照管马匹,而雇用一个马夫;一个专讲吃喝的人自己不做饭,而雇用一个厨师。

施托尔希(在前面引用的著作中)所说的那些生产“\textbf{余暇}”,因而使人有空闲时间来享乐、从事脑力劳动等等的人们,也属于这类劳动者。警察节约我为自己当宪兵的时间,士兵节约我自卫的时间,政府官吏节约我管理自己的时间,擦皮靴的人节约我自己擦靴子的时间,教士节约思考的时间,等等。

在这个问题上正确的一点是\textbf{分工}的思想。每个人除了自己从事生产劳动或对生产劳动进行剥削之外,还必须执行大量非生产的并且部分地加入消费费用的职能。(真正的生产工人必须自己负担这些消费费用,自己替自己完成非生产劳动。)如果这种“服务”是令人愉快的,主人就往往代替奴仆去做,例如初夜权或者早就由主人担任的管理劳动等等,都证明了这一点。但这决没有消除生产劳动和非生产劳动的区分;相反,这种区分本身表现为\textbf{分工}的结果,从而促进一般劳动生产率的发展,因为分工使非生产劳动变成一部分人的专门职能,使生产劳动变成另一部分人的专门职能。

但是罗西断言,就连专门用来使主人摆阔、满足主人虚荣心的那些家仆的“\textbf{劳动}”,也“不是非生产劳动”。为什么呢\fontbox{?}因为它生产\textbf{某种东西}:满足虚荣心,使主人能够吹嘘、摆阔(同上,第 277 页)。这里我们又看到了那种胡说八道,好象每种服务都生产某种东西:妓女生产淫欲,杀人犯生产杀人行为等等。而且,据说斯密说过,这些污秽的东西每一种都有自己的\textbf{价值}。就差[416]说这些“服务”是无酬的了。问题并不在这里。但是,即使这些服务是无酬的,它们也不会使财富(物质财富)增加一文钱。

然后又是一段美文学式的胡言乱语:

\begin{quote}“有人硬说,歌手唱完歌,不给我们留下什么东西。不,他留下回忆!〈妙极了!〉你喝完香槟酒留下了什么呢\fontbox{?}……消费是否紧紧跟随生产,消费进行得快还是慢,固然会使经济结果有所不同,但消费这个事实本身无论怎样也不会使产品丧失财富的性质。某些非物质产品比某些物质产品存在更长久。一座宫殿会长期存在,但《\textbf{伊利亚特}》是更长久的享受来源。”(第 277—278 页)\end{quote}

多么荒唐!

从这里罗西所理解的财富的意义,即从使用价值的意义来说,情况甚至是这样的:只有\textbf{消费}才使产品成为财富,而不管这种消费是快还是慢(消费的快慢决定于消费本身的性质和消费品的性质)。使用价值只对消费有意义,而且对消费来说,使用价值的存在,只是作为一种消费品的存在,只是使用价值在消费中的存在。喝香槟酒虽然生产“头昏”,但不是生产的消费,同样,听音乐虽然留下“回忆”,但也不是生产的消费。如果音乐很好,听者也懂音乐,那末消费音乐就比消费香槟酒高尚,虽然香槟酒的生产是“生产劳动”,而音乐的生产是非生产劳动。

\centerbox{※     ※     ※}

把反对斯密关于生产劳动和非生产劳动的区分的所有胡说八道总括一下,可以说,加尔涅,也许还有罗德戴尔和加尼耳(但后者没有提出什么新东西),已经把这种反驳的全部内容都表达出来了。后来的著作家(施托尔希没有成功的尝试除外)只不过发一些美文学式的议论,讲一些有教养的空话而已。加尔涅是督政府和执政时代的经济学家,费里埃和加尼耳是帝国的经济学家。另一方面,罗德戴尔是伯爵大人,他尤其愿意把\textbf{消费者当作“非生产劳动”的生产者加以辩护}。对奴仆、仆役的\textbf{颂扬},对征税人、寄生虫的\textbf{赞美},贯穿在所有这些畜生的作品中。和这些相比,古典政治经济学粗率嘲笑的性质,倒显得是对现有制度的批判。

\tsectionnonum{[(19)马尔萨斯主义者查默斯为富人浪费辩护的论点]}

\textbf{托·查默斯牧师}是最狂热的马尔萨斯主义者之一,他是神学教授,著有《\textbf{论政治经济学和社会的道德状况、道德远景的关系}》一书(1832 年伦敦第 2 版)。按照查默斯的意见,要消除一切社会弊端,没有别的手段,只有对工人阶级进行宗教教育(他指的是通过基督教的粉饰和教士的感化来灌输马尔萨斯的人口论)。同时,他竭力为各种浪费、国家的无谓开支、教士的巨额俸禄、富人的极度挥霍辩护。他对(第 260 页及以下各页)“时代精神”和“严酷的忍饥挨饿的节约”感到痛心;他要求实行重税,让那些“高级的”非生产劳动者,教士等等可以大吃大喝(同上);当然,他对斯密的区分是极为反对的。他用整整一章(第十一章)的篇幅来谈这个区分,不过其中除了断言节约等等对“生产劳动者”只有害处以外,没有任何新的东西。下面这些话可以概括说明这一章的倾向:

\begin{quote}“这种区分是荒谬的,而且应用起来是有害的。”(同上,第 344 页)\end{quote}

害处在哪里呢\fontbox{?}

\begin{quote}“我们所以要这样详细地谈这个问题,是因为我们认为,\textbf{今日的政治经济学对教会过于严厉、过于敌视了},我们不怀疑,\textbf{斯密的有害的区分}大大促进了这一点。”(第 346 页)\end{quote}

这位牧师所说的“教会”是指他自己的教会,作为“法定”教会的英国国教会。而且,他还是把这个“教会”推行到爱尔兰的那帮家伙中的一个。至少,这个牧师是很坦率的。

\tsectionnonum{[(20)关于亚当·斯密及其对生产劳动和非生产劳动的看法的总结性评论]}

[417]在结束关于亚当·斯密的部分之前,我们还要引用他书中的两段话:在第一段话中,他发泄了自己对非生产的政府的憎恨;在第二段话中,他力图证明,为什么工业等等的进步要以自由劳动为前提。关于\textbf{斯密对牧师的憎恨}!\endnote{关于亚当·斯密对牧师的敌对态度,见马克思《资本论》第 1 卷第 23 章注 75。——第 314 页。}

第一段话说:

\begin{quote}“因此,国王和大臣们要求监督私人的节约,并以反奢侈法令或禁止外国奢侈品进口的办法来限制私人开支,这是他们最无耻、最专横的行为。他们自己始终是并且毫无例外地是社会上最大的浪费者。他们还是好好地注意他们自己的开支吧,私人的开支尽可以让私人自己去管。如果他们自己的浪费不会使国家破产,那末,他们臣民的浪费也决不会使国家破产。”(第 2 篇第 3 章,麦克库洛赫版,第 2 卷第 122 页)\end{quote}

再引下面这段话\authornote{见本册第 151,152 和 273 页。——编者注}:

\begin{quote}“某些最受尊敬的社会阶层的劳动,象\textbf{家仆的劳动}一样,不生产\textbf{任何价值}\fontbox{~\{}它有价值,因而值一个等价,但不生产任何价值\fontbox{\}~},不固定或不物化在任何耐久的对象或可以出卖的商品中……例如,君主和他的全部文武官员、全体陆海军,都是\textbf{非生产劳动者}。他们是社会的\textbf{公仆},靠\textbf{别人劳动}的一部分年产品生活……应当列入\textbf{这一类的},还有……教士、律师、医生、各种文人;演员、丑角、音乐家、歌唱家、舞蹈家等等。”(同上,第 94—95 页)\end{quote}

这是还具有革命性的资产阶级说的话,那时它还没有把整个社会、国家等等置于自己支配之下。所有这些卓越的历来受人尊敬的职业——君主、法官、军官、教士等等,所有由这些职业产生的各个旧的意识形态阶层,所有属于这些阶层的学者、学士、教士……\textbf{在经济学上}被放在与他们自己的、由资产阶级以及有闲财富的代表(土地贵族和有闲资本家)豢养的大批仆从和丑角同样的地位。他们不过是社会的\textbf{仆人},就象别人是他们的仆人一样。他们靠\textbf{别人劳动}的产品生活。因此,他们的人数必须减到必不可少的最低限度。国家、教会等等,只有在它们是管理和处理生产的资产者的共同利益的委员会这个情况下,才是正当的;这些机构的费用必须缩减到必要的最低限度,因为这些费用本身属于生产上的非生产费用\authornote{见第 159 页脚注。——编者注}。这种观点具有历史的意义,一方面,它同古代的见解形成尖锐的对立,在古代,物质生产劳动带有奴隶制的烙印,这种劳动被看作仅仅是有闲的市民的立足基石;另一方面,它又同由于中世纪瓦解而产生的专制君主国或贵族君主立宪国的见解形成尖锐的对立,就连孟德斯鸠自己都还拘泥于这种见解,他天真不过地把它表达如下(《论法的精神》第 7 篇第 4 章):

\begin{quote}“富人不多花费,穷人就要饿死。”\end{quote}

相反,一旦资产阶级占领了地盘,一方面自己掌握国家,一方面又同以前掌握国家的人妥协;一旦资产阶级把意识形态阶层看作自己的亲骨肉,到处按照自己的本性把他们改造成为自己的伙计;一旦资产阶级自己不再作为生产劳动的代表来同这些人对立,而真正的生产工人起来反对资产阶级,并且同样说它是靠别人劳动生活的;一旦资产阶级有了足够的教养,不是一心一意从事生产,而是也想从事“有教养的”消费;一旦连精神劳动本身也愈来愈为资产阶级\textbf{服务},为资本主义生产服务;——一旦发生了这些情况,事情就反过来了。这时资产阶级从自己的立场出发,力求“在经济学上”证明它从前批判过的东西是合理的。加尔涅等人就是资产阶级在这方面的代言人和良心安慰者。此外,这些经济学家(他们本人就是教士、教授等等)也热衷于证明自己“在生产上的”有用性,“在经济学上”证明自己的薪金的合理性。

[418]第二段话讲到奴隶制,他说:

\begin{quote}“这类职业〈手工业者和制造业劳动者的职业,在许多古代国家〉被看作只适宜于奴隶,而市民则不准从事这类职业。就连没有这种禁令的国家如雅典和罗马,事实上人民也不从事今天城市居民的下层阶级通常所从事的各种职业。在罗马和雅典,富人的奴隶从事这些职业,而且他们是为了主人的利益从事这些职业的。富人有钱有势,并且得到保护,这就使贫穷的自由民在自己的制品和富人奴隶的制品竞争时,几乎不可能为自己的制品找到销路。但是奴隶很少有发明;工业上一切减轻劳动和缩短劳动的最重要的改良,无论是机器还是更好的劳动组织和分工,都是自由民发明的。即使有的奴隶想出了并且提议实行这类改良,他的主人也会认为这是懒惰的表现,是奴隶企图牺牲主人的利益来减轻自己的劳动。可怜的奴隶不但不能由此得到报酬,还多半会遭到辱骂,甚至惩罚。因此,同使用自由民劳动的企业相比,使用奴隶劳动的企业,为了完成同量的工作,通常要花费更多的劳动。因此,后一类企业的制品通常总要比前一类企业的制品贵。孟德斯鸠指出,匈牙利矿山虽然不比邻近的土耳其矿山富,但是开采起来始终费用较小,因而利润较大。土耳其矿山靠奴隶开采,\textbf{奴隶的双手是土耳其人}想到使用的\textbf{唯一机器}。匈牙利矿山是靠自由民开采的,他们为了减轻和缩短自己的劳动使用了大量的机器。根据我们所知道的关于希腊和罗马时代工业品价格的不多的资料,精制的工业品看来是非常贵的。”(同上,第 4 篇第 9 章;加尔涅的译本,第 3 卷第 549—551 页)\end{quote}

\centerbox{※     ※     ※}

\textbf{亚·斯密}自己在第四篇第一章中\endnote{斯密在这一章中考察了重商主义的一般理论观点。——第 316 页。}写道:

\begin{quote}“洛克先生曾指出货币和其他各种动产的区别。他说,其他一切动产\textbf{按其性质来说是这样容易消耗},以致由这些动产构成的财富是极不可靠的……相反,货币却是一个可靠的朋友”等等。(同上,第 3 卷第 5 页)\end{quote}

接着在同一章第 24—25 页上说:

\begin{quote}“有人说,消费品很快就消灭了,而金和银\textbf{按其性质来说比较耐久},只要不把这些金属不断输出国外,这些金属就可以一个世纪一个世纪地积累起来,使一国的实际财富得到难以置信的增加。”\end{quote}

货币主义者醉心于金银,因为金银是\textbf{货币},是交换价值的独立的存在,是交换价值的可感觉的存在,而且只要不让它们成为流通手段这种不过是商品交换价值的转瞬即逝的形式,它们就是不会毁坏的、永久的存在。因此,积累金银,积蓄金银,贮藏货币,成了货币主义所宣扬的致富之道。正象我引用配第的话所指出的那样,\endnote{马克思指《政治经济学批判》第一分册《货币贮藏》那一小节,那里他引了配第《政治算术》中的话。马克思在前面第 167 页也引了同样的话,在这一页他指出斯密部分地回到了重商学派的观点。——第 317 页。}连其他商品在这里也只是根据它们的耐久程度,即根据它们作为交换价值存在多久来估价的。

现在,\textbf{第一},亚·斯密是在重复他在一个地方曾说过的关于商品耐久程度相对大小的意见,在那里他曾说,消费对于财富的形成究竟是较有利还是较不利,要看消费品存在的时间是较长还是较短。\endnote{马克思指斯密《国富论》第二篇第三章最后六段,斯密在那里研究:收入以何种方式支出对促进社会财富的增长作用比较大,以何种方式支出则作用比较小。斯密认为这取决于消费品的不同性质,取决于它们的耐久程度。马克思在前面论德斯杜特·德·特拉西那一小节(第 290—291 页)提到过斯密的这个观点。——第 317 页。}因而,这里可以看出他的货币主义观点,而这也是必然的,因为即使在直接消费时,拥有商品的人也始终盘算着使[419]消费品继续是\textbf{财富},是商品,因而是使用价值和交换价值的统一;而这又取决于使用价值的耐久程度,因而取决于消费是否只是逐渐地、缓慢地使这个使用价值失去作为\textbf{商品}或作为交换价值承担者的可能性。

\textbf{第二},斯密在他关于生产劳动和非生产劳动的第二种区分上,完全回到——在更广泛的形式上——货币主义的区分上去了。

\begin{quote}生产劳动“固定和物化在一个特定的对象或可以出卖的商品中,而\textbf{这个对象或商品在劳动结束后,至少还存在若干时候}。可以说,这是在其物化过程中积累并储藏起来,准备必要时在另一场合拿来利用的一定量劳动”。相反,非生产劳动的结果或非生产劳动的服务“通常一经提供随即消失,很少留下某种痕迹或某种以后能够用来取得同量服务的\textbf{价值}”。(第 2 篇第 3 章,麦克库洛赫版,第 2 卷第 94 页)\end{quote}

可见,斯密区分商品和服务,就象货币主义区分金银和其他一切商品一样。斯密也是从积累的角度来区分的,不过积累已经不再被看作货币贮藏的形式,而是被看作再生产的实际形式了。商品在消费中消灭,但同时它会重新生出具有更高价值的商品来,或者,如果不这样使用,商品本身就是可以用来购买其他商品的价值。劳动产品本身的属性是:它作为一个或多或少耐久的、因而可以再让渡出去的使用价值存在,它作为这样一种使用价值存在,即它是可以出卖的有用品,是交换价值的承担者,\textbf{是商品},或者说,实质上是\textbf{货币}。非生产劳动者的服务不会再变成\textbf{货币}。我对律师、医生、教士、音乐家等等、国家活动家、士兵等等的服务支付了报酬,但是,我既不能用这些服务来还债,也不能用它们来购买商品,也不能用它们来购买创造剩余价值的劳动。这些服务完全象容易消失的消费品一样消失了。

可见,斯密所说的实质上同货币主义所说的一样。货币主义认为,只有生产\textbf{货币},生产金银的劳动,才是生产的。在斯密看来,只有为自己的买者生产\textbf{货币}的劳动才是生产的。所不同的只是,斯密在一切商品中都看出了它们具有的货币性质,不管这种性质在商品中怎样隐蔽,而货币主义则只有在作为交换价值的独立存在的商品中才看出这种性质。

这种区分是以资产阶级生产实质本身为基础的,因为财富不等于使用价值,只有\textbf{商品},只有作为交换价值承担者、作为货币的使用价值,才是财富。货币主义不懂得,这些货币的创造和增加,是靠商品的消费,而不是靠商品变为金银,商品以金银的形式结晶为独立的交换价值,但是,商品在金银的形式上不仅丧失了它们的使用价值,而且没有改变它们的\textbf{价值量}。

\tchapternonum{[第五章]奈克尔}

\vicetitle{[试图把资本主义制度下的阶级对立描绘成贫富之间的对立]}

前面已经引过的兰盖的一些话表明,他对资本主义生产的性质是清楚的。\endnote{马克思在手稿第 V 本第 181 页(第一章第三节《相对剩余价值》,《分工》一小节)引了兰盖的下面一段话:“贪婪的吝啬鬼不放心地监视着他〈短工〉,只要他稍一中断工作,就大加叱责。只要他休息一下,就硬说是偷窃了他。”([兰盖]《民法论》1767 年伦敦版第 2 卷第 466 页)马克思在手稿第 X 本第 439 页论兰盖一章中引了这些话(见本册第 371 页)。在《资本论》第一卷中作为第八章注 39 引了这些话,但有删节。——第 319 页。}然而这里在谈完奈克尔之后,还可以再提一下兰盖。\endnote{尽管兰盖的著作《民法论》(1767 年)发表在马克思这里所考察的奈克尔的《论立法和谷物贸易》(1775 年)和《论法国财政的管理》(1784 年)这两本著作之前,马克思却把论兰盖一章放在论奈克尔一章之后。马克思把材料作这样的编排,是因为从理解资本主义生产的性质来说,兰盖的著作超过奈克尔的上述两本著作。——第 319 页。}

奈克尔在他的《论立法和谷物贸易》(1775 年初版)和《论法国财政的管理》这两部著作中,指出劳动生产力的发展只不过使工人用\textbf{较少的时间}再生产自己的工资,从而用\textbf{较多的时间无代价地}为自己的雇主劳动。同时奈克尔正确地用\textbf{平均工资},用最低限度的工资作基础。但是,实际上他关心的不是劳动本身转化为资本,也不是资本通过这个过程得到积累,而宁可说是贫富之间、贫困和奢侈之间对立的一般发展。这种发展的基础是:随着生产必要生活资料所需的劳动量愈来愈少,有愈来愈大的一部分劳动成为剩余的,因而可以用来生产者侈品,可以用在别的生产领域。这种奢侈品的一部分是能够保存的;这样,奢侈品就在支配剩余劳动的人手里一个世纪一个世纪地积累起来,上述对立因此也就愈来愈严重。

重要的是,奈克尔一般认为非劳动阶层的财富[420]——利润和地租——来源于剩余劳动。在考察剩余价值时,他注意到相对剩余价值,即不是从延长整个工作日而是从缩短\textbf{必要劳动时间}得出的剩余价值。劳动生产力变成劳动条件所有者的生产力。而这种生产力本身则表现为得到一定结果所必要的劳动时间的缩短。主要的几段话如下:

\textbf{第一},《论法国财政的管理》(《奈克尔著作集》1789 年洛桑和巴黎版第 2 卷):

\begin{quote}“我看到社会上的一个阶级,它的收入几乎始终不变;我注意到另一个阶级,它的财富必然增长。这样,由对比和比较而来的奢侈现象,必然随着这种不平衡的发展而发展起来,并随着时间的推移而日益显著……”(同上,第 285—286 页)\end{quote}

(这里已经很好地指出了\textbf{两个阶级}之间的\textbf{阶级}对立。)

\begin{quote}“社会的一个阶级的命运好象已经由社会的法律\textbf{固定了},所有属于这个阶级的人都\textbf{靠自己双手劳动过活},被迫服从\textbf{所有者}〈生产条件所有者〉的法律,不得不以领取\textbf{相当于最迫切的生活需要的工资}为满足;他们之间的竞争和\textbf{贫困的压迫},使\textbf{他们处于从属地位};而且这种状况是不能改变的。”(同上,第 286 页)

“\textbf{使一切机械工艺简单化的新工具不断发明},因而\textbf{增加了所有者的财富和财产};其中一部分工具\textbf{减少了土地耕作费用},使土地所有者所能支配的\textbf{收入增加了};人类天才的另一部分发明\textbf{大大地减轻了}工业中的劳动,以致\textbf{在生存资料的分配者}〈即资本家〉\textbf{手下劳动的人们},能够\textbf{在同样的时间内,拿同样的工资},生产出更多的各种制品。”(同上,第 287 页)“假设在上一世纪,必须有 10 万工人才能完成今天 8 万工人就能完成的工作;那末,现在剩下来的 2 万人为了取得工资,就不得不投身于\textbf{别的职业};由此创造出来的新的手工制品,就会增加富人的享受和奢侈。”(第 287—288 页)奈克尔接着说:“因为不应当忽视,一切不需要特殊技艺的劳动的报酬,总是同\textbf{每个工人所必需的生存资料的价格}成比例的;所以,知识一旦普及,\textbf{制作速度的加快就丝毫不会有利于劳动者,而只会增加}用以满足拥有土地产品的人们的趣味和虚荣心的\textbf{手段}。”(同上,第 288 页)“靠人的技艺成型和改变形态的种种自然财物中,有许多按其耐久程度来说是大大超过人的通常寿命的,因此每一代都继承前几代劳动创造物的一部分\end{quote}

\fontbox{~\{}奈克尔这里所考察的,只是亚·斯密称为消费基金的那种东西的积累\fontbox{\}~},

\begin{quote}并且在各个国家都有愈来愈多的工艺制品逐渐\textbf{积累起来};因为这一切制品总是在所有者中间分配,所以这些人的享受和人数众多的市民阶级的享受之间的不平衡,必然愈来愈大,愈来愈明显。”(第 289 页)\end{quote}

所以:

\begin{quote}“能增加大地上奢侈品和装饰品的\textbf{工业劳动速度的加快,这些奢侈品和装饰品能够积累起来的时期的延续,以及使这些财物只集中在一个社会阶级手中的财产法}……奢侈的这许多源泉,不管流通中的货币量有多少,都是始终存在的。”(第 291 页)\end{quote}

(最后一句话,是反驳那些认为奢侈来源于货币量日益增加的人的。)

\textbf{第二},《论立法和谷物贸易》(《奈克尔著作集》第 4 卷)说:

\begin{quote}“手工业者或土地耕种者一旦\textbf{丧失储备},他们就无能为力了;他们必须\textbf{今天劳动,才不致明天饿死};在所有者和工人之间的[421]这种利益斗争中,一方用自己的生命和全家的生命作赌注,另一方只不过延缓一下自己奢侈的发展而已。”(同上,第 63 页)\end{quote}

这种不劳动的富和为生活而劳动的贫之间的对立,又造成了知识的对立。知识和劳动彼此分离,于是知识作为资本或富人的奢侈品同劳动相对立:

\begin{quote}“认识和理解的能力是一般天赋,但这种能力只有通过教育才能发展;如果财产是平等分配的,那末每个人\textbf{就会适度地劳动}\end{quote}

(可见,起决定作用的又是劳动时间的量),

\begin{quote}\textbf{并且,每个人都会有一些知识},因为每个人都剩下\textbf{一定量的时间}〈空闲的时间〉来学习和思考;但是在社会制度所造成的财产不平等的情况下,所有那些生下来就没有财产的人,\textbf{根本没有受教育的机会}。因为一切生存资料都掌握在占有\textbf{货币或土地}的那部分国民手里。因为谁也不会白给东西,所以生下来除了自己的力气之外便没有别的储备的人,不得不在刚有点力气的时候,就用来为所有者服务,并且要一天又一天地干一辈子,每天从日出一直干到筋疲力尽,干到为了恢复精力必需睡眠时为止。”(第 112 页)“最后,为了维持所有那些\textbf{造成知识不平等的}社会的不平等,这种知识的不平等已经成了必要的了,这一点难道不是无可怀疑的吗\fontbox{?}”(同上,第 113 页;参看第 118、119 页)\end{quote}

奈克尔嘲笑经济概念的混淆。重农学派对于土地就有这种混淆,后来的所有经济学家对于资本的各物质要素也有这种混淆。有了这种混淆,生产条件的所有者就受到赞扬,因为生产条件(但决不是生产条件的所有者本人)对于劳动过程和财富的生产是必要的。奈克尔说:

\begin{quote}“人们一开始就把土地所有者(一个非常容易执行的职能)的意义同土地的意义混淆起来。”(同上,第 126 页)[IX—421]\end{quote}

\tchapternonum{[第六章]魁奈的经济表(插入部分)}

\tsectionnonum{[(1)魁奈表述总资本的再生产和流通的过程的尝试]}

一年总产品 50 亿(图尔利弗尔)租地农场主以原预付和年预付形式支出

为了使这个表更加清楚起见,凡是魁奈认为是流通的出发点的地方,我就标上 a、a′、a″,这个流通的下一环节则标上 b、c、d 以及相应的 b′、b″。\endnote{马克思在这里使用的字母符号(和标记)使《经济表》一目了然,无论在施马尔茨的著作中还是在魁奈的著作中都没有这样清楚。用两个字母(a—b,a—c,c—d 等等)来标明每一条线,使人能确定线的方向,即这条线是从哪个阶级到哪个阶级(方向按字母表上字母的顺序确定,a—b,a—c,c—d 等等)。例如,a—b 线表示土地所有者阶级和生产阶级(租地农场主)之间的流通以土地所有者阶级为出发点(后者向租地农场主购买食物)。用两个字母来标明每一条线,同时表明了货币的运动和商品的运动。例如,a—b 线表示货币的运动(土地所有者阶级向生产阶级支付 10 亿货币);但是这条线从相反的方向(b—a)来看,就表明商品的运动(生产阶级交给土地所有者阶级 10 亿食物)。虚线 a—b—c—d 由以下几个环节组成:(1)a—b 段表示土地所有者和生产阶级之间的流通(土地所有者向租地农场主购买 10 亿食物);(2)a—c 段表示土地所有者和不生产阶级——工业家之间的流通(土地所有者向工业家购买 10 亿工业品);(3)c—d 段表示不生产阶级和生产阶级之间的流通(工业家向租地农场主购买 10 亿食物)。a′—b′线表示生产阶级和不生产阶级之间的流通(租地农场主向工业家购买 10 亿工业品)。a″—b″线表示不生产阶级和生产阶级之间的最后的流通(工业家向租地农场主购买工业生产所必需的 10 亿原料)。——第 324、349、352 页。}

这个表上首先值得注意并且不能不使同时代人留下深刻印象的,是这样一个方式:货币流通在这里表现为完全是由商品流通和商品再生产决定的,实际上是由资本的流通过程决定的。

\tsectionnonum{[(2)租地农场主和土地所有者之间的流通。货币流回租地农场主手里,不表现再生产]}

租地农场主首先把 20 亿货币支付给土地所有者。后者用其中 10 亿货币向租地农场主购买食物。因此,10 亿货币流回租地农场主手里,同时总产品的 1/5 得到实现,最终由流通领域转入消费领域。

然后,土地所有者用 10 亿货币购买价值 10 亿的工业品,非农产品。从而又有 1/5 的产品(以已经加工的形式)从流通领域转入消费领域。这 10 亿货币现在落到不生产阶级手里,这个阶级用它向租地农场主购买食物。于是,租地农场主以地租形式付给土地所有者的第二个 10 亿,又流回租地农场主手里。另一方面,租地农场主的产品中另一个 1/5 归不生产阶级,由流通领域转入消费领域。因而,到这第一个运动结束时,这 20 亿货币又在租地农场主手里。这 20 亿货币完成了四个不同的流通过程。

\textbf{第一},它们用作支付地租的支付手段。在执行这个职能时,它们并不使年产品的任何一部分流通,它们只是用来支取总产品中等于地租的那一部分的流通支票。

\textbf{第二},土地所有者用 20 亿的半数 10 亿向租地农场主购买食物;因而土地所有者把自己的 10 亿实现为食物。租地农场主得到这 10 亿货币,实际上只是收回了他开给土地所有者用来支取他的产品 2/5 的那张支票的半数。在这种情况下,这 10 亿由于用作购买手段,就使同额商品进入流通,这批商品最终转入消费。在这里,这 10 亿对于土地所有者来说,只是\textbf{购买手段}:土地所有者把货币再转化为使用价值(转化为商品,然而这是最终转入消费、作为使用价值被购买的商品)。

如果我们只考察这个单独的行为,那末,这里的货币对租地农场主来说,只是起了它作为购买手段对卖者始终所起的作用,也就是成为卖者的商品的转化形式。土地所有者把他的 10 亿货币转化为谷物,而租地农场主把价格为 10 亿的谷物转化为货币,实现了谷物的价格。但是,我们把这个行为同前面的流通行为联系起来看,货币在这里,就不是表现为租地农场主的商品的单纯形态变化,不是表现为他的商品的金等价物。这 10 亿货币本来只是租地农场主[423]以地租形式支付给土地所有者的 20 亿货币的半数。诚然,租地农场主以 10 亿商品的代价取得了 10 亿货币,但\textbf{实际上他这样只是赎回他用来向土地所有者支付地租的货币;换句话说,土地所有者用他从租地农场主那里得来的 10 亿,向租地农场主购买价值 10 亿的商品。土地所有者用他不给等价物}而从租地农场主那里取得的货币,\textbf{付给租地农场主}。

货币流回租地农场主手里的这种回流,首先使这里的货币(同第一个行为联系起来看)对租地农场主来说,并不是简单的流通手段。其次,这种回流同表现再生产过程的货币流回出发点的运动有本质的区别。

例如,一个资本家,或者,为了完全撇开资本主义再生产的特征,就说一个生产者,为了取得他在全部劳动时间所必需的原料、劳动工具和生活资料,支出 100 镑。假定他加到生产资料上的劳动,不比他花费在生活资料即他支付给自己的工资上的劳动多。如果原料等等,等于 80 镑,而新加劳动等于 20 镑(他消费了的生活资料也等于 20 镑),那末产品就等于 100 镑。如果生产者再把产品卖掉,那末 100 镑货币又流回到他的手里——如此周而复始。货币流回到它的出发点的这种回流在这里不是表现别的,正是表现不断的再生产。这里是单纯的形态变化 G—W—G,货币转化为商品,商品再转化为货币。商品和货币的这种单纯的形式变换,在这里同时又表现再生产过程。货币转化为商品——\textbf{生产资料}和生活资料;然后,这些商品作为要素进入劳动过程,又作为产品从劳动过程出来;这样,从完成的产品再进入流通过程,因而再作为商品同货币相对立的时候起,商品又是过程的结果;最后,商品再转化为货币,因为完成的商品只有在它先转化为货币之后,才能重新同它的生产要素交换。

货币不断流回它的出发点,在这里,不仅表现从货币到商品和从商品到货币的形式上的转化,象它在简单流通过程或简单商品交换中所表现的那样,\textbf{同时也表现同一个生产者进行的商品的不断再生产}。交换价值(货币)转化为商品,而这些商品进入消费,作为使用价值被利用,但它们是进入再生产消费或生产消费,从而恢复了原有价值,因此又转化为\textbf{同一个}货币额(在上述例子中,生产者只为维持自己的生活而劳动)。这里,公式 G—W—G 表明,G 不仅是形式上转化为 W,而且 W 实际上作为使用价值被消费,从流通领域转入消费领域,但这是生产消费,所以商品的价值在消费中保存着,并且再生产出来,因此,G 在过程的终点又出现了,它在 G—W—G 的运动中保存了自己。

相反,在上述的货币从土地所有者流回租地农场主手里的那种回流中,没有发生任何再生产过程。这就好比租地农场主给土地所有者开了 10 亿产品的凭证或者票券。一旦土地所有者把这些票券付兑,票券便流回租地农场主手里,后者又予以承兑。如果土地所有者同意一半地租直接以实物支付,那末在这种情况下,就不会发生任何货币流通。整个流通就会限于简单的转手,即产品从租地农场主手里转到土地所有者手里。但是,起先租地农场主付给土地所有者的不是商品而是货币,然后,土地所有者又把货币付还给租地农场主,以取得商品本身。货币对租地农场主来说,作为\textbf{支付手段}付给土地所有者;货币对土地所有者来说,作为\textbf{购买手段}付给租地农场主。在执行第一种职能时,货币从租地农场主那里离开,在执行第二种职能时,货币又回到租地农场主手里。

凡是在生产者不把自己的一部分产品而把这种产品的价值用货币支付给他的债权人的时候,都必定会发生货币流回生产者手里的这种回流;这里,凡是共同占有他的剩余产品的人都表现为债权人。可以举这样一个例子。一切税收都是由生产者用货币支付的。这里,货币对生产者来说,作为支付手段付给国家。国家用这些货币向生产者购买商品。在国家手里,货币成为购买手段,这样就流回生产者手里,有多少商品从生产者那里出去,就有多少货币流回生产者手里。

货币回流这个环节——这种特殊的、不由再生产决定的货币回流——每当收入同资本交换时,都一定要发生。这里,引起货币回流的不是再生产,而是消费。收入用货币支付,但是收入只能以商品形式消费。因而,从生产者那里作为收入所取得的货币必须付还给生产者,才能从生产者那里取得同等价值的商品,也就是说,才有可能消费收入。用来支付收入的货币,例如租金、利息或税收,都具有支付手段的一般形式。[424]\fontbox{~\{}产业资本家自己用产品来支付自己的收入,或者在产品出卖后,把产品换得的、构成他的收入的那部分货币支付给自己。\fontbox{\}~}这里假定支付收入的人事先从自己的债权人那里得到自己产品的一部分,譬如说,租地农场主事先得到产品的 2/5(按魁奈的说法,这 2/5 产品构成地租)。租地农场主仅仅是这 2/5 产品的名义所有者,或者说,defacto\authornote{defacto(事实上),以区别于 dejure(法律上、依据法律)。——编者注}的掌握者。

因此,租地农场主用来支付地租的那一部分产品,为了在租地农场主和土地所有者之间流通,只需要一个和产品价值相等的货币额,虽然这个价值流通两次。首先,租地农场主用货币支付地租,然后土地所有者用同一笔货币购买产品。第一种情况是货币的简单的转移,因为这里货币只起\textbf{支付手段}的作用;因而这里是假定用货币支付的那个商品已经为货币支付者占有,而货币对他来说不是购买手段,他没有用货币换得等价物,倒是这个等价物早已在他的手中。相反,在第二种情况下,货币执行购买手段、商品流通手段的职能。这好比租地农场主用他支付地租的货币,从土地所有者那里赎回产品中属于土地所有者的一份。土地所有者用从租地农场主那里得到的同一笔货币(但实际上这笔货币是租地农场主在没有得到等价物的情况下交出的),向租地农场主买回产品。

因此,生产者以支付手段形式向收入所有者支付的同一个货币额,对收入所有者来说,是向生产者购买商品的购买手段。这样,货币从生产者手里到达收入所有者手里以及从收入所有者手里回到生产者手里这两次位置变换,仅仅表现了商品的一次位置变换,即商品从生产者手里到达收入所有者手里。因为假定生产者——就他的一部分产品来说——是收入所有者的债务人,所以生产者向收入所有者支付货币地租,实际上只是事后支付他(生产者)已经占有的商品的价值。商品在他的手里,但商品不属于他。因而,生产者用他以收入形式支付的货币把该商品购进,归自己所有。所以商品没有转手。货币的转手只不过表明商品\textbf{所有权的变换},商品仍然在生产者手里。由此发生了商品仅一次转手而货币却两次变换位置的情况。货币流通两次,是为了使商品流通一次。但是货币作为流通手段(购买手段)也只流通一次,另外一次它是作为支付手段流通的;在后面这种流通中,正如我在前面已经指出的,不发生商品和货币同时变换位置的情况。

事实上,如果租地农场主没有货币,只有产品,那末他只有在先出卖自己的商品之后,才能支付自己的产品;因而,在租地农场主能够以货币向土地所有者支付自己的商品之前,这个商品就必须已经完成它的第一形态变化。但是,即使在这种情况下,货币的位置变换还是多于商品。首先是 W—G:2/5 商品被卖掉,变成了货币。这里商品和货币同时变换位置。但是,后来同一笔货币从租地农场主手里转到土地所有者手里,商品却没有变换位置。这里有货币的位置变换,而没有商品的位置变换。这好比租地农场主有一个合伙人。租地农场主卖得了货币,但必须同他的合伙人分货币。更确切些说,从这 2/5 来看,就好比租地农场主的伙计卖得了货币。这个伙计必须把货币交给租地农场主,他无权把货币留在自己的口袋里。这里,货币的转手不表现商品的任何形态变化,而只是货币从它的直接掌握者手里转到它的所有者手里。可见,如果第一个收款人只是一个替自己雇主收款的代理人,情况就会如此。在这种情况下,货币甚至不是支付手段;货币仅仅是简单地从收款人(货币不属于他)手里转到货币所有者手里。

货币的这种位置变换,就象一种货币简单地兑换成另一种货币时所发生的位置变换一样,同商品的形态变化绝对无关。但是,当货币执行支付手段的职能的时候,总是假定支付人先取得了商品,以后才进行支付。至于租地农场主等等,他\textbf{不是取得了}这种商品:这种商品落到土地所有者手里以前,就在租地农场主手里,而且是\textbf{他的}产品的一部分。然而,\textbf{从法律上来说},租地农场主只有在他把用商品换得的货币交给土地所有者的时候,才成为这种商品的所有者。他对商品的权利发生了变化;商品本身仍然在他的手里。但是,以前商品在他手里是作为他\textbf{掌握}的东西,商品的所有者是土地所有者。而现在商品在他手里是作为归他自己所有的东西。仍然保留在同一个人手里的商品所发生的法律形式的变化,自然不会引起商品本身的转手。

\tsectionnonum{[(3)资本家和工人之间的货币流通问题]}

\tsubsectionnonum{[(a)把工资看成资本家对工人的预付的荒谬见解。把利润看成风险费的资产阶级观点]}

[425]\fontbox{~\{}由上面所说的还可以看出,用所谓资本家在把他的商品变为货币之前就已向工人预付货币这一点来“解释”资本家的利润,是多么荒谬。

\textbf{第一},如果我购买商品供自己消费,那末我是买者,而商品所有者是“卖者”,我的商品具有货币形式,他的商品还有待于变为货币,我决不会因此而取得任何“利润”。资本家只是在他消费了劳动之后,才对劳动支付代价,而其他商品则在被消费之前就得到支付。这个情况的产生是由于资本家购买的商品具有特殊的性质,这种商品实际上只是在被消费之后,才完全转到买者手里。货币在这里是作为支付手段出现的。资本家把“劳动”这个商品占为己有,总是\textbf{在}对它支付代价\textbf{之前}。他购买劳动,只是为了从出卖劳动产品中获得利润,但是这决不能成为他获得这笔利润的\textbf{理由}。这只是一个动机。而且在这种情况下无非是说:资本家购买雇佣劳动之所以获得利润,\textbf{是因为}他想从出卖这个劳动产品中获得利润。

\textbf{第}二,但是,有人会说,资本家毕竟把作为工资归工人的那一部分产品以货币形式预付给了工人,这样就使工人不必为了亲自把作为工资归他所有的那部分商品变为货币而备受辛苦、承担风险和花费时间。对于这种辛苦、风险和时间,工人难道不该向资本家支付一笔报酬吗\fontbox{?}因而,工人得到的产品份额难道不该比在其他情况下应得到的产品份额少一些吗\fontbox{?}

如果这样提问题,雇佣劳动和资本的全部关系就被抹杀了,从经济学上对剩余价值的解释也就勾销了。诚然,过程的结果是,资本家用来支付雇佣工人的基金实际上只是后者自己的产品,因此资本家和工人\textbf{事实上}是按一定的比例分享产品。但这个事实上的结果,同资本和雇佣劳动之间的交易(由商品交换本身的规律得出的从经济学上对剩余价值的论证,是以这种交易为基础的),是绝对没有关系的。资本家购买的是对劳动能力的暂时支配权,并且他只是在劳动能力发挥了作用、物化在产品中以后,才对这种支配权进行支付。就象货币作为支付手段起作用的一切地方一样,这里买和卖也是发生在货币实际脱离买者之前。但是,从这个在真正的生产过程开始前就已完成的交易以后,劳动已经\textbf{属于}资本家。作为产品从这个过程出来的\textbf{商品}也完全属于资本家。资本家用属于他的生产资料以及他所购买的(虽然还没有支付过代价)因而也属于他的劳动,生产了这个商品。这就好比他根本没有消费他人的劳动来生产商品一样。

资本家获得的利润,他实现的剩余价值,正是这样来的:工人作为商品卖给他的,不是物化在商品中的劳动,而是自己的劳动能力本身。如果工人作为第一种形式的商品所有者\endnote{马克思把以自己的劳动力为自己唯一商品的工人同“第一种形式的商品所有者”,即拥有供出卖的“不同于劳动能力本身的商品”的商品所有者对立起来。(参看前面第 159 和 111—113 页)。——第 334 页。}同资本家相对立,那末,资本家就不可能获得任何利润,不可能实现任何剩余价值,因为按照价值规律,是等价物同等价物相交换,等量劳动同等量劳动相交换。资本家的剩余价值正是这样来的:他向工人购买的不是商品,而是工人的劳动能力本身,而劳动能力所具有的价值比它的产品所具有的价值小,或者同样可以说,劳动能力所实现的物化劳动量比实现在劳动能力自身的物化劳动量大。但是,现在为了替利润辩护,利润的源泉本身被掩盖起来了,利润借以产生的整个交易也被抛开了。因为实际上(只要过程是连续不断的)资本家只是用工人自己的产品支付工人,工人支取的只是工人自己的产品的一部分,因而预付纯粹是假象,所以现在有人说:\textbf{在产品变为货币之前},工人已把产品中归自己所有的那一份卖给资本家了。(或许甚至是在产品有可能变为货币之前,因为工人的劳动虽然已物化在某一产品中,但在当时也许只造出了可出卖的商品的一部分,例如,房屋的一部分。)如果这样看问题,资本家就不成其为产品的所有者了,他借以\textbf{无偿}占有别人劳动的整个过程也就消失了。这样一来,互相对立的就都是商品所有者。资本家手里有货币,而工人卖给资本家的不是他的劳动能力,而是商品,即他自己的劳动借以物化的那部分产品。

这样,工人就会对资本家说:“在这 5 磅产品(譬如说纱)中,3/5 代表不变资本,属于你。2/5 即 2 磅代表我的新加劳动。因此你应当支付我 2 磅纱。现在就请付给我这两磅的价值吧。”在这种情况下,工人装进口袋的就不仅是工资,而且还有利润,简单说,就是和他以 2 磅纱的形式新加的物化劳动量相等的全部货币额。

资本家说:“难道我没有预付不变资本吗\fontbox{?}”

工人回答说:“对呀,正因为这样,你才拿走 3 磅,只付给我 2 磅。”

资本家坚持说:“但是,如果没有我的棉花和我的纱锭,你就不可能使你的劳动物化,不可能纺纱。所以,你应当另外付一笔报酬。”

工人回答说:“得了!如果我不用你的棉花和纱锭纺纱,棉花就会烂掉,纱锭就会生锈。[426]不错,你给自己扣下的 3 磅纱,只代表在 5 磅纱生产过程中耗费的,因而包含在这 5 磅纱内的你的棉花和纱锭的价值。但是,只有我的劳动把这些生产资料作为生产资料来消费之后,才保存了棉花和纱锭的价值。我并没有由于我的劳动有保存价值的能力而向你索取分文,因为除了纺纱(由于纺纱我得到 2 磅)之外,这种能力并没有花费我什么额外的劳动时间。这是我的劳动的一种自然赐予,它并不花费我什么,却保存了不变资本的价值。既然我并不因此向你索取分文,那末,你也不能因为我没有纱锭和棉花不能纺纱这一点而向我索取报酬。如果我不纺纱,你的纱锭和棉花就一钱不值。”

资本家无可奈何,就说:“2 磅纱,的确值 2 先令,这正代表了你的劳动时间的数量。但是,我在把这两磅纱卖掉之前,就得付给你这两磅的钱。也许我根本卖不掉。这是第一个风险。第二,我可能卖得比它的价格低。这是第二个风险。还有第三,无论如何,为了把它卖掉,必须花费时间。难道我就应当\textbf{无代价地}为你承担这两个风险,再加时间的损失吗\fontbox{?}天下绝没有无代价的事。”

工人回答说:“等一等,我们究竟是什么关系\fontbox{?}我们是作为\textbf{商品所有者}相对立的,\textbf{你是买者,我们是卖者},因为你想买产品中我们的一份即 2 磅,而这 2 磅中所包含的,实际上只是我们自己的物化劳动时间。现在你说,我们必须把自己的商品\textbf{低于}它的价值卖给你,使你因此得到的商品形式的价值比你现在拥有的货币形式的价值多。我们的商品的价值等于 2 先令。你想只给 1 先令,这样一来,——因为 1 先令所包含的劳动时间和 1 磅纱所包含的一样多,——你换进的价值就比你换出的价值多 1 倍。相反,我们得到的不是等价,而是等价的一半,不是 2 磅纱的等价,而只是 1 磅纱的等价。你凭什么提出这个违反价值规律、违反商品按价值交换的规律的要求\fontbox{?}凭什么\fontbox{?}凭你是买者,我们是卖者,凭我们的价值是以纱的形式、商品的形式存在,而你的价值是以货币的形式存在,凭纱形式的一定价值与货币形式的同一价值相对立。但是,老兄!这不过是形式的变换,这种变换和价值的\textbf{表现}有关,但并不使\textbf{价值量}发生变化。或者你有一种幼稚的看法,认为任何商品在货币形式上都具有\textbf{较大的}价值,所以任何商品都必须\textbf{低于}它的价格——就是说,低于代表它的价值的货币额——出卖\fontbox{?}但是,不对,老兄,商品在货币形式上并不具有较大的价值;它的价值量并没有变化,它不过纯粹表现为交换价值而已。

想想看,老兄,你使自己处于多么尴尬的境地!你的主张是:卖者始终必须把商品\textbf{低于}它的价值卖给\textbf{买者}。从前,当我们卖给你的还不是我们制造的商品,而是我们的劳动能力本身的时候,在你那里,情况确实是这样。那时,你购买劳动能力固然按照它的价值,但是你购买我们的劳动本身却\textbf{低于}它所体现的价值。不过,我们抛开这种不愉快的回忆吧。谢天谢地,自从你自己作出决定,要我们不再把劳动能力当作商品卖给你,而把商品本身即我们劳动的产品卖给你以来,我们就不再处于这种地位了。我们回过来谈谈你所处的那种尴尬境地吧。按照你提出的新规律,卖者为了把他的商品变为货币,不是简单地付出他的商品,不是简单地用他的商品来同货币交换,而是把商品\textbf{低于}它的价格出卖。按照这个规律,买者总是欺骗和诈取卖者,而这个规律应该对任何买者和卖者都是同样有效的。假定我们同意你的建议,但有一个条件,你自己也要服从你臆想出来的规律,即买者替卖者把商品变为货币,卖者必须把自己的商品的一部分\textbf{无代价地}送给买者。例如,你买我们的 2 磅纱,价值 2 先令,只付给 1 先令,于是你赚了 1 先令即 100\%。但是,现在,你从我们这儿买了属于我们的 2 磅纱之后,你手里就有 5 磅纱,价值 5 先令。当然,你打算做一桩有利可图的买卖。5 磅纱只花了你 4 先令,而你想按 5 先令把它卖出去。\textbf{你的买者}说:‘且慢!你的 5 磅纱是商品,你是卖者。我有同一价值的货币,我是买者。因此,按照你承认的规律,同你进行的交易应当给我带来 100\%的利润。所以你必须把这 5 磅纱低于它的价值 50\%,即按 2+(1/2)先令,卖给我。我给你 2+(1/2)先令,换你价值 5 先令的商品,这样一来,就从你那里赚了 100\%,因为同一道理,人人适用。’

[工人继续说道:]老兄,你自己看看,你的新规律会得出什么结果;你只会落得自己欺骗自己,虽然你一时是买者,但过后又成了卖者。在这种情况下,你作为卖者所损失的,会比你作为买者所得到的更多。你好好想想吧!你现在想从我们手上买去 2 磅纱,但是在这 2 磅纱生产出来之前,难道你没有买过东西吗\fontbox{?}如果没有买,就根本不会有这 5 磅纱。[426a]难道你没有预先购买现在由 3 磅纱代表的棉花和纱锭吗\fontbox{?}在购买这些东西时,利物浦的棉花批发商和奥尔丹的纺机厂主作为\textbf{卖者}同你对立,而你作为\textbf{买者}同他们对立;他们是商品的代表,而你是货币的代表,这同现时我们有幸或不幸地相互对立着的关系是完全一样的。如果你根据你替他们把商品变为货币,而他们替你把货币变为商品,他们是卖者,你是买者这一点理由,就要求他们把一部分棉花和纱锭\textbf{无代价地}让给你,或者同样可以说,要求他们把这些商品低于它们的价格(和它们的价值)卖给你,那末,那个狡猾的棉花商人和你的爱逗乐的奥尔丹同行不会嘲笑你吗\fontbox{?}他们可没有冒什么风险嘛,他们得到了现金,得到了纯粹独立形式的交换价值。而你这方面要担多少风险呀!首先,要用纱锭和棉花制成纱,经历生产过程的一切风险;然后把纱卖出,再变为货币,也要担风险。纱能否按它的价值卖出,是高于还是低于价值卖出,这是一个风险。也许它根本卖不出去,根本不能再变为货币,这又是一个风险。至于纱本身,你一点也不感兴趣。你不能拿纱来吃、来喝,除了把它卖出去以外,没有任何用处。而且无论如何,为了把纱再变为货币(这里还包括把纱锭和棉花再变为货币),还要有时间的损失!你的同行将反驳你说:老兄!别装傻了!别说废话了!你想怎样利用我们的棉花和纱锭,你要用它们做什么,这关我们屁事!把它们烧掉、扔掉,随你的便,只要把它们的价钱付了!想得倒好!你当上了棉纺厂主,我们就得把我们的商品白白送给你,看来你感到这一行不很顺手,你把这一行的风险和危险太夸大啦!那就别经营棉纺业啦,不然就别抱着这些荒唐的想法到市场上来!”

听了工人这番话,资本家带着轻蔑的微笑回答说:“可见,你们这些家伙只听钟声响,不知钟声哪里来。你们谈的是你们根本不懂的事情。你们以为我把现金付给利物浦坏蛋和奥尔丹小子吗\fontbox{?}决不是!我付给他们的是期票,而在期票到期之前,利物浦坏蛋的棉花实际上已经加工成纱,并且卖出去了。你们的情况却完全不同。你们是要拿现金的。”

工人说:“好极了,但是利物浦坏蛋和奥尔丹小子把你的期票拿去干什么啦\fontbox{?}”

资本家喊道:“他们拿去干什么\fontbox{?}!问得多蠢!他们拿去找银行家贴现呗!”

“他们付给银行家多少\fontbox{?}”

“多少\fontbox{?}货币现在十分便宜。我想他们大约付 3\%的贴现利息,就是说,不是付期票金额的 3\%,而是根据距期票到期的时间,按年利 3\%计算。”

工人说:“那就更好啦,付给我们 2 先令,这是我们商品的价值。或者付给我们 12 先令吧,因为我们想按周计算,不按天计算。所以,就依年利 3\%从这个金额中扣除 14 天的利息吧。”

资本家说:“但这张期票太小,没有一个银行家肯贴现的。”

工人回答说:“好吧,我们是 100 人。这也就是说,你应当付给我们 1200 先令。给我们开一张期票吧。合 60 镑,拿去贴现,不算太小的数目了。而且,是你自己给它贴现,这个数目对你来说,一定不会太小,因为你想从我们身上赚得的利润,也正是这个数目。扣除的是微不足道的数目。既然这样一来我们能够把我们产品的大部分完全拿到手,那我们很快也就不再需要你贴现了。自然,我们给你的信贷,不会比交易所经纪人给你的多——总共只有 14 天。”

如果认为工资的来源(在完全歪曲真实关系的情况下),是对总产品中属于工人的那部分价值的贴现,即资本家提前用\textbf{货币}把这部分支付给工人,那末资本家势必开给他们期限很短的期票,就象他开给做棉花生意的商人等等的那种短期期票一样。工人就会得到自己产品的大部分,而资本家很快就会不成其为资本家了。对于工人来说,资本家就会从产品的所有者变成单纯的银行家。

此外,如果说资本家有商品低于其[427]价值出卖的风险,那他也有商品高于其价值出卖的机会。而如果产品卖不出去,工人就会被抛到街头。如果产品价格长期低于市场价格,工人的工资就要下降到平均水平之下,工厂就开工不足。所以,工人承担的风险最大。

\textbf{第三},租地农场主用货币支付地租,工业资本家用货币支付利息,他们为了有可能实行这些支付,必须事先把自己的产品变为货币,可是谁也不会想到,仅仅因为这个缘故,他们就可以从地租或利息中扣下一部分。\fontbox{\}~}

\tsubsectionnonum{[(b)工人向资本家购买商品。不表现再生产的货币回流]}

工业资本家和工人之间流通的那部分资本(即构成可变资本的那部分流动资本),也发生货币流回它的出发点的现象。资本家以货币支付工人的工资;工人用这笔货币向资本家购买商品,于是货币流回资本家手里。(在实践中,是回到资本家的银行家手里。但是,事实上,银行家对单个资本家来说是总资本的代表,因为总资本表现为\textbf{货币}。)这种货币回流按其本身来说根本不表现再生产。资本家用货币向工人购买劳动,工人用这笔货币向资本家购买商品。同一笔货币,最初表现为购买劳动的手段,后来表现为购买商品的手段。这笔货币之所以流回资本家手里,是因为资本家对同样一些人来说,起先作为买者出现,后来又作为卖者出现。他作为买者,货币从他那里出去,他作为卖者,货币又回到他手里。相反,工人起先作为卖者出现,后来又作为买者出现,他起先收进货币,后来支出货币,而资本家与工人相反,起先支出货币,后来收回货币。

在资本家方面,这里发生了 G—W—G 运动。他用货币购买商品(劳动能力)。他用这种劳动能力的产品(商品)购买货币,或者说,他把这个产品卖回给从前作为卖者同他相对立的工人。相反,工人代表 W—G—W 流通。他卖出自己的商品(劳动能力),并用卖得的货币买回自己的产品(他生产的商品)的一部分。当然,有人会说:工人卖出商品(劳动能力)换得货币,用这笔货币购买商品,然后又出卖自己的劳动能力,所以,从工人方面说,也出现 G—W—G 过程;并且,因为货币是在工人和资本家之间不断来回流通的,所以要看从哪一方面来考察这个过程,同样可以说,工人也象资本家那样代表 G—W—G 运动。然而,资本家是买者。过程是从资本家而不是从工人重新开始的;货币必然流回资本家手里,是因为工人必须购买生活资料。这里,正象在一方的流通形式是 G—W—G,另一方的流通形式是 W—G—W 的一切过程中一样,可以看出,交换过程的目的,对一方来说,是交换价值,货币,因而也就是价值的增殖;对另一方来说,则是使用价值,是消费。这种情况也发生在上面提到的第一种场合的货币回流上。在那种场合,租地农场主一方实现 G—W—G 过程,而土地所有者一方实现 W—G—W;这是很明显的,只要我们注意到,土地所有者用来向租地农场主购买商品的 G 是地租的货币形式,因而已经是 W—G 的结果,是以实物形式实际属于土地所有者的那部分产品的转化形式。

这种 G—W—G 过程在工人和资本家的关系中,只是资本家用来支付工资的那笔货币流回资本家手里的表现,它本身不表现再生产过程,只表现买者对同样一些人来说又是卖者这一情况。它也不表现作为资本的货币,就是说,不表现 G—W—G′过程,在那里,第二个 G′是比第一个 G 大的货币额,因而 G 是自行增殖的价值(资本)。相反,这只是表示同一货币额(往往还少一些)在形式上流回出发点。(这里所说的资本家,当然是指资本家阶级。)因此,我在第一部分\endnote{马克思指《政治经济学批判》第一分册第二章第三节《货币》的头两段(见《马克思恩格斯全集》中文版第 13 卷第 112—113 页)。——第 342 页。}中说 G—W—G 的形式必定是 G—W—G′,是不对的。这种形式可以只表现货币回流的形式,我在同一个地方用买者同时又成为卖者的情况来说明货币流回同一出发点的循环运动时,已暗示了这一点。\endnote{马克思指《政治经济学批判》第一分册的下面这段话:“他们[商品所有者]作为买者付出的货币,一到他们重新作为商品的卖者出现时,又回到他们手里。因此,商品流通的不断更新就反映成:货币不仅在资产阶级社会的整个表面上不断地转手,而且同时画出许多不同的小循环,从无数不同地点出发,又回到这些地点,以便重新再作同样的运动。”(见《马克思恩格斯全集》中文版第 13 卷第 90 页)——第 342 页。}

但资本家不是靠\textbf{这种}货币回流发财的。例如,他付出了 10 先令工资。工人用这 10 先令向他购买商品。资本家给工人 10 先令商品,换取他的劳动能力。如果资本家用实物形式给工人以价格 10 先令的生活资料,这里就根本不发生货币流通,因而也就不发生货币回流。可见,这种货币回流现象同资本家发财无关。资本家所以发财,是因为资本家在生产过程中占有的劳动,比他以工资形式支出的劳动多,因此他的产品大于这个产品的生产费用,可是,资本家支付给工人的货币也决不可能少于工人用来向资本家购买商品的货币。这里,这种形式上的货币回流同发财毫无关系,因而并没有表现[428]作为资本的 G,正如当用来支付地租、利息和税收的货币流回支付地租、利息和税收的人手里的时候,这种回流本身不包含价值的增殖或者[哪怕是]价值的补偿一样。

只要 G—W—G 表示货币流回资本家手里的形式上的回流,那末它只是表明,资本家开出的货币形式的支票由他自己的商品来兑现。

对这种货币回流(即货币回到它的出发点)作错误解释的例子,见前面论德斯杜特·德·特拉西那一节\endnote{见本册第 277—291 页。并见马克思《资本论》第 2 卷第 20 章第 13 节:《德斯杜特·德·特拉西的再生产理论》。——第 343、364 页。}。作为第二个例子——特别适用于工人和资本家之间的货币流通——后面要引用布雷的一段话\endnote{论布雷一节在手稿第 X 本第 441—444 页。这一节没有写完,也没有接触到布雷对工人和资本家之间的货币流通问题的观点。关于布雷对货币的本质和作用的观点,见马克思 1847 年手稿《工资》(《马克思恩格斯全集》中文版第 6 卷第 641 页);《政治经济学批判大纲》1939 年莫斯科德文版第 55、690、754 页;马克思 1858 年 4 月 2 日给恩格斯的信;《政治经济学批判》(《马克思恩格斯全集》中文版第 13 卷第 76 页)。——第 343 页。}。最后关于贷放货币的资本家,要引用\textbf{蒲鲁东}\endnote{后面,在手稿第 X 本第 428 和 437 页,马克思对蒲鲁东在这个问题上的观点作了简短的说明(见本册第 344—345 页)。——第 343 页。}。

G—W—G 这种回流形式,凡是在买者又成为卖者的地方都存在;因而它在整个商业资本中都存在。在商业资本中,一切商人彼此之间都是为卖而买和为买而卖。可能,买者——G——不能把商品(例如米)卖得比买来时贵;他可能甚至不得不低于它的价格把它出卖。在这种情况下,就只发生简单的货币回流,因为这里买变为卖并没有使货币表现为自行增殖的价值即资本。

例如,在不变资本交换时,也发生这种情况。机器制造业者向铁生产者买铁,又把机器卖给铁生产者。在这种情况下,货币流回机器制造业者手里。货币作为买铁的购买手段被付出。后来,它成为铁生产者买机器的购买手段,因此流回机器制造业者手里。机器制造业者付出货币换进铁,又收进货币付出了机器。这里,同一数量的货币可以使两倍的价值流通。例如,机器制造业者用 1000 镑买铁;铁生产者用这 1000 镑买机器。铁和机器的价值加在一起等于 2000 镑。但是,这样一来,就必须有 3000 镑在运动:1000 镑货币,1000 镑机器和 1000 镑铁。如果资本家进行实物交换,商品转手就不必有一个法寻流通。

如果资本家彼此实行结算,货币对他们起支付手段的作用,那末情况也是如此。如果流通的是纸币或信用货币(银行券),那末,情况有一点变化。在这时,还有 1000 镑银行券,但它没有“内在价值”。不管怎样,这里也有 3000 镑:1000 镑铁,1000 镑机器和 1000 镑银行券。但这 3000 镑之所以存在,如同第一种情况一样,只是因为机器制造业者手里有 2000 镑:1000 镑机器和 1000 镑货币(金银或银行券)。在这两种情况下,铁生产者还给机器制造业者的,都只是后者(即货币),而铁生产者之所以得到货币,只是因为作为买者的机器制造业者并不直接又是卖者,他不是用商品而是用货币来支付第一种商品铁。当他用商品来支付,即把自己的商品卖给铁生产者的时候,后者就把货币还给他。因为支付并不是双重的——一次用货币,再一次用商品。

在这两种情况下,金或银行券都代表由机器制造业者买去的商品的转化形式,或者代表后来由别人买去的商品的转化形式;或者代表虽然没有被购买但已转化为货币的那种商品,就象土地所有者(他的祖先等等)\endnote{括号里的话是马克思打算以后加以发挥的思想。马克思很可能指魁奈在土地私有制问题上的辩护论观点。按照这种观点,土地所有者所以对土地持有权利,是因为他们的祖先使处女地变成了适于耕种的土地。在马克思写的《反杜林论》第二编第十章中,对重农学派的这一观点作了如下说明:“魁奈认为,……按照‘自然法’说来,土地所有者的真正职能正是在于‘关心良好的管理和关心维持他们世袭财产所必需的费用’,或者……在于 avancesfoncières,即用来准备土地并供给农场以一切必需东西的费用,这些费用,使租地农场主可以把其全部资本只用在真正的耕种事业上。”(《马克思恩格斯全集》中文版第 20 卷第 274 页)——第 344 页。}获得收入时的情况那样。因而这里货币的回流只是表示:原来为换得商品而把货币付出即投入流通的人,由于卖出他投入流通的另一种商品,又把货币从流通中抽回来。

刚才谈的这 1000 镑,在资本家之间流通,可以在一天之内经过四五十只手,这只不过是资本的转手而已。机器转到铁生产者手里,铁转到农民手里,谷物转到淀粉厂主或酒精厂主手里,等等。最后,这 1000 镑可能又落到机器制造业者手里,再经过后者转到铁生产者手里等等。由此可见,用这 1000 镑就可以使 40000 镑以上的资本流通,而且货币可能不断回到第一个把它投入流通的人手里。用这 40000 镑赚得的利润,有一部分会转化为利息,由各个资本家支付,例如机器制造业者向贷给他 1000 镑的人付利息,铁生产者向借给他 1000 镑(这笔钱他早已用在煤炭等等或工资等等上面了)的人付利息,等等。蒲鲁东先生由此得出结论说,这 1000 镑带来了由 40000 镑赚得的\textbf{全部利息}。这样一来,如果利率等于 5\%,利息就等于 2000 镑。根据这一假设,他正确地算出 1000 镑生出 200\%的利息。这位鼎鼎大名的政治经济学批评家原来如此!\authornote{[437}上面提到的蒲鲁东的那段话,是这样说的:

“抵押债务总额,据最了解情况的作者说,达到 120 亿,据其他一些人说,达 160 亿;期票债务总额至少有 60 亿,股票大约 20 亿,国债 80 亿;总计 280 亿。必须指出,所有这些债务,是按 4\%、5\%、6\%、8\%、12\%,甚至 15\%的利息借来的或视同借来的货币。我假定前三类债务的平均利息是 6\%;200 亿就有利息 12 亿。此外还加上国债利息约 4 亿,总计:10 亿资本的年息为 16 亿。”(第 152 页)因而是 160\%。因为“在法国,现金总额,——我不想说一般存在的,我说的是在流通中的货币总额,包括银行的库存现金在内,——按照最流行的估计,不超过 10 亿”。(第 151 页)“当交换结束时,货币又空出来了,因而又可以重新贷出……货币资本从一次交换到另一次交换,总是不断回到它的出发点。由此可见,每次由同一个人的手重新把这些货币贷出,总能给同一个人带来利润。”(第 153—154 页)(《无息信贷。弗·巴师夏先生和蒲鲁东先生的辩论》1850 年巴黎版\endnote{马克思在《剩余价值理论》第三册补充部分《收入及其源泉。庸俗政治经济学》(手稿第 935—937 页)中,批判了蒲鲁东在《无息信贷》一书中发挥的关于货币资本的作用和利息本质的庸俗观点。并见马克思《资本论》第 3 卷第 21 章。——第 345 页。})[437]]

但是,虽然 G—W—G 过程在代表资本家和工人之间的货币流通时本身不表现任何再生产行为,这个过程的经常重复,货币的不断回流却表示再生产。任何一个买者,如果不把他卖出的商品再生产出来,他就根本不可能经常作为卖者出现。诚然,所有不靠地租、利息或税收生活的人都是这样。但是就一部分人来说,在过程结束时总发生货币的回流 G—W—G,例如在资本家对工人,或对土地所有者,或对食利者的关系上就是如此(从后两者来说是单纯的货币回流)。就另一部分人来说,过程结束则购得了商品,即发生 W—G—W 过程,例如在工人方面就是这样。工人使这个过程不断周而复始。工人总是作为卖者,而不是作为买者开始行动。[429]仅仅表明花费收入的整个货币流通,也是这种情况。例如,资本家自己每年就消费一定数量的产品。他把自己的商品变为货币,以便用这些货币来购买他要想最终消费的商品。这里是 W—G—W,并不发生货币流回资本家手里的现象,但是货币流回卖者(例如店主)手里,而卖者的资本靠资本家花费其收入来补偿。

但是,我们还看到收入同收入交换,即收入之间的流通。屠宰业者向面包业者买面包,面包业者向屠宰业者买肉;他们两者都消费自己的收入。屠宰业者自己吃肉是不付钱的,同样,面包业者自己吃面包也不付钱。他们都是以实物形式消费收入的这一部分。但可能有这种情况:面包业者向屠宰业者买的肉,对屠宰业者来说不是补偿资本,而是补偿收入,即补偿他出卖的肉中不是纯粹代表他的利润,而是代表他的利润中他自己想当作收入消费掉的那一部分。屠宰业者向面包业者买面包,对屠宰业者来说,也是花费他的收入。双方结算的时候,只要有一个人支付差额就行了。他们相互买卖彼此抵销的部分不加入货币流通。但是,假定面包业者要支付差额,而且这个差额对屠宰业者来说代表收入。那末屠宰业者就要用面包业者的货币去买其他消费品。假定这是 10 镑,他支付给裁缝。如果这 10 镑对裁缝来说代表收入,那末裁缝也以类似的方式花费这 10 镑。他又用这 10 镑去买面包等等。这样,货币就流回面包业者手里,但对面包业者来说,这笔货币所补偿的已经不是收入,而是资本了。

还可以提出这样一个问题:在由资本家进行的、代表自行增殖的价值的 G—W—G 过程中,资本家从流通中抽出的货币比他投入流通的货币多。(这曾经是货币贮藏者的真正愿望,但没有实现。因为他以金银形式从流通领域抽出的价值,并不比他以商品形式投入流通的价值多。他现在有更多的货币形式的价值,而过去他有更多的商品形式的价值。)假定资本家的商品的全部生产费用等于 1000 镑。他按 1200 镑把自己的商品卖出去,因为现在他的商品中包含 20\%即 1/5 的无酬劳动,这种劳动是资本家虽然没有支付过代价却拿去出卖的。全体资本家即工业资本家阶级不断从流通中抽出的货币,怎么可能比他们投入流通的货币多呢\fontbox{?}从另一方面可以说,资本家不断投入流通的东西比他从流通中抽出的东西多。他要支付他的固定资本。但是,他出卖固定资本是随着固定资本消费的程度,一部分一部分地进行的。固定资本虽然完全进入商品的生产过程,但它始终是以小得多的部分加到商品的\textbf{价值}中去。假定固定资本的流通期间为 10 年,那末每年加到商品中去的只是它的 1/10,而其余 9/10 不加入货币流通,因为这 9/10 根本没有以商品形式进入流通。这是一个问题。

这个问题我们以后再来考察,\endnote{马克思在《资本论》第二卷第十七章、第二十章第五节和第十二节、第二十一章(特别是第一节中的第一小节《贮藏货币的形成》)中全面考察了这个问题。——第 347、349、365 页。}现在回过头来谈魁奈。

但是,还有一个问题,先得说一说。银行券流回到办理期票贴现和用银行券贷款的银行,是同上面考察的货币回流完全不同的现象。在这种情况下,商品转化为货币是预先实现的。商品取得货币形式还在它出卖以前,甚至在它生产出来以前。但也有可能,它已经卖出了(凭期票)。\textbf{无论如何},它还没有\textbf{被支付},还没有再转化为货币。因此,这种向货币的转化,无论在哪一种情况下,都是预先实现的。一旦商品被卖掉(或\textbf{被认为}已卖掉),货币就流回银行:或者以本银行的银行券的形式流回,于是这种银行券便退出流通;或者以别家银行的银行券的形式流回,这种银行券便同本银行的银行券相交换(在银行家之间),这样一来,两种银行券都退出流通,回到它们的出发点;或者以金银的形式流回。如果这些金银被用来兑换第三者手中的银行券,那末银行券就回到银行。如果银行券不要求兑换,那末流通中的金银就会减少,减少的正是代替银行券存放在银行库里的金银的数目。

在所有这些情况下,过程是这样的:

货币的现有存在(商品向货币转化)已经预先实现。当商品真正转化为货币时,它是第二次向货币转化。但商品的这种第二次货币存在会回到出发点,抵偿、替代商品的第一次货币存在,离开流通,回到银行。很可能,表现商品的这种第二次货币存在的,就是表现过商品的第一次货币存在的\textbf{同一}批银行券。例如,给纺纱厂主贴现了一张期票。这张期票是他从织布业者那里得到的。纺纱厂主用贴现得来的 1000 镑支付煤炭、棉花等等。这些银行券被用来支付商品,经过各种不同的人的手,最后被用于购买麻布,于是银行券落到织布业者手里。期票到期,织布业者就把这些银行券付给纺纱业者,后者则把它们归还银行。在商品预先实现向货币转化之后发生的第二次的(后遗的)向货币转化,除了第一次的货币以外,[430]完全不需要另外的货币。这样,似乎纺纱业者实际上什么也没有得到,因为开始他借了银行券,而在过程结束时,他收回银行券,把它们归还银行。但是,实际上,同一批银行券在这个时期内起了流通手段和支付手段的作用,而纺纱业者把其中一部分用来偿付债务,一部分用来购买再生产纱所必需的商品,从而也实现了通过剥削工人而创造出来的剩余价值,他现在可以把剩余价值的一部分付给银行。而且也是用货币,因为流回他手里的货币要比他支出的、预付的和花费的货币多。怎么发生的呢\fontbox{?}这又属于我们留到后面再考察的那个问题。\endnote{马克思在《资本论》第二卷第十七章、第二十章第五节和第十二节、第二十一章(特别是第一节中的第一小节《贮藏货币的形成》)中全面考察了这个问题。——第 347、349、365 页。}

\tsectionnonum{[(4)《经济表》上租地农场主和工业家之间的流通]}

现在我们回过来讲魁奈。我们要考察第三和第四个流通行为。

P(土地所有者)向 S\endnote{马克思在这里用如下符号代表魁奈著作中出现的三个阶级:P——classedesPropriétaires(土地所有者阶级),S——classeStérile(不生产阶级——工业家),F——Fermiers,classeproductive(租地农场主,生产阶级)。——第 349 页。}(从事工业的“不生产”阶级)购买 10 亿工业品(在表上是 a—c 线\endnote{马克思在这里使用的字母符号(和标记)使《经济表》一目了然,无论在施马尔茨的著作中还是在魁奈的著作中都没有这样清楚。用两个字母(a—b,a—c,c—d 等等)来标明每一条线,使人能确定线的方向,即这条线是从哪个阶级到哪个阶级(方向按字母表上字母的顺序确定,a—b,a—c,c—d 等等)。例如,a—b 线表示土地所有者阶级和生产阶级(租地农场主)之间的流通以土地所有者阶级为出发点(后者向租地农场主购买食物)。用两个字母来标明每一条线,同时表明了货币的运动和商品的运动。例如,a—b 线表示货币的运动(土地所有者阶级向生产阶级支付 10 亿货币);但是这条线从相反的方向(b—a)来看,就表明商品的运动(生产阶级交给土地所有者阶级 10 亿食物)。虚线 a—b—c—d 由以下几个环节组成:(1)a—b 段表示土地所有者和生产阶级之间的流通(土地所有者向租地农场主购买 10 亿食物);(2)a—c 段表示土地所有者和不生产阶级——工业家之间的流通(土地所有者向工业家购买 10 亿工业品);(3)c—d 段表示不生产阶级和生产阶级之间的流通(工业家向租地农场主购买 10 亿食物)。a′—b′线表示生产阶级和不生产阶级之间的流通(租地农场主向工业家购买 10 亿工业品)。a″—b″线表示不生产阶级和生产阶级之间的最后的流通(工业家向租地农场主购买工业生产所必需的 10 亿原料)。——第 324、349、352 页。})。这里,10 亿货币使同额商品进入流通。\fontbox{~\{}因为在这种情况下发生的是一次交换。如果 P 是分次向 S 购买商品,而 P 同样是分次从 F(租地农场主)那里收到地租,那末,这 10 亿工业品就可能,比方说,用 1 亿购买。因为 P 向 S 购买 1 亿工业品,S 向 F 购买 1 亿食物,F 向 P 支付 1 亿地租;如果这样重复 10 次,那末就有 10×1 亿的商品从 S 转到 P 和从 F 转到 S,而有 10×1 亿的地租从 F 转到 P。于是,整个流通用 1 亿就完成了。但是,如果 F 一次支付全部地租,那末,S 手里的 10 亿和流回到 F 手里的 10 亿,就可能有一部分存放着,一部分在流通。\fontbox{\}~}现在有 10 亿商品从 S 转到 P,相反,有 10 亿货币从 P 转到 S。这是简单流通。货币和商品只是按相反方向转手。但是,除了租地农场主已经卖给 P 因而进入消费的 10 亿食物以外,还有 10 亿工业品由 S 卖给 P 而进入消费。必须指出,这些商品在新的收获以前就存在了(否则,P 就不能用新收获的产品购买它们)。

S 再用 10 亿向 F 购买食物。于是\textbf{总产品的第二个}1/5 离开流通,进入消费。在 S 和 F 之间,这 10 亿执行了流通手段的职能。但同时,这里发生了两种现象,这两种现象是在 S 和 P 之间的过程中没有发生的。在后面这个过程中,S 把他的产品的一部分,即 10 亿工业品,再转化为货币。但是,在同 F 交换中,他把货币再转化为食物(在魁奈那里,就是转化为工资),从而补偿他投在工资上的、已消费的资本。10 亿变为生活资料的这种再转化,在 P 那里表示单纯的消费,而在 S 那里则表示生产的消费,表示再生产,因为 S 把他的商品的一部分再转化为商品的生产要素之一,即生活资料。因此,商品的这一形态变化,它从货币到商品的再转化,在这里同时表示商品的\textbf{实际的}(而不仅是\textbf{形式上的})形态变化的开始,表示商品的再生产的开始,商品变为它自己的生产要素的再转化的开始。这里同时也发生资本的形态变化。相反,从 P 这方面来说,只是收入从货币形式转化为商品形式。这只是表示消费。

第二,当 S 向 F 购买 10 亿食物的时候,F 作为货币地租付给 P 的第二个 10 亿,就回到 F 手里。不过,它们之所以回到 F 手里,只是因为 F 用价值 10 亿的商品等价物把它们从流通中再抽回,赎回。这就好比土地所有者向他(除了第一个 10 亿以外)买了 10 亿食物一样,就是说,好比土地所有者以商品形式从租地农场主那里获得了他的货币地租的第二部分,然后又用这些商品换了 S 的商品一样。S 不过是代替 P 以商品形式提取 F 已经用货币付给 P 的那 20 亿的第二部分。如果用实物支付,那就是 F 给 P20 亿食物,P 自己消费其中 10 亿,用另外 10 亿向 S 交换工业品。在这种情况下,只会是:(1)20 亿食物从 F 转到 P;(2)P 和 S 之间进行物物交换,前者用 10 亿食物去换 10 亿工业品,后者则相反。

实际上不是这样,而是发生了四个行为:[431](1)20 亿货币从 F 转到 P;(2)P 向 F 购买 10 亿食物;货币流回 F 手里,执行流通手段的职能;(3)P 用 10 亿货币向 S 购买工业品;货币执行流通手段的职能,按与商品运动相反的方向转手;(4)S 用这 10 亿货币向 F 购买食物;货币执行流通手段的职能。对于 S 来说,货币同时作为资本流通。它流回 F 手里,因为现在那第二个 10 亿的食物——土地所有者从 F 那里得到过这 10 亿支票——被提取了。但是,货币不是直接从土地所有者那里流回 F 手里的,货币先在 P 和 S 之间起了流通手段的作用,货币在提取 10 亿食物之前,中途先提取了 10 亿工业品,并把它们从工业家手里转给土地所有者,在这以后,货币才流回 F 手里。由工业家的商品转化为货币(在同土地所有者交换中),和接着而来的由货币转化为食物(在同租地农场主交换中),在 S 方面,都是他的资本的形态变化,先是变成货币的形式,然后变成资本再生产所必需的构成要素的形式。

因此,以上四个流通行为的结果是:土地所有者花完了他的收入,一半花在食物上,一半花在工业品上。这样一来,他以货币地租形式得到的 20 亿就花完了。其中一半从他那里直接流回租地农场主手里,另一半通过 S 间接流回租地农场主手里。而 S 把他的成品的一部分脱了手,用食物,也就是用再生产的一个要素来补偿。通过这些过程,结束了有土地所有者出现的流通。离开流通进入消费(一部分是非生产消费,一部分是生产消费,因为土地所有者已经用他的收入部分地补偿了 S 的资本)的是:(1)10 亿食物(新收获的产品);(2)10 亿工业品(上年收获的产品);(3)10 亿食物,这个 10 亿是加入再生产的,就是用来生产 S 在次年拿去同土地所有者的一半地租相交换的那些商品的。

20 亿货币又在租地农场主手里了。租地农场主现在为了补偿他的“年预付和原预付”(因为它们一部分由劳动工具等构成,一部分由生产中所消费的其他工业品构成),向 S 购买 10 亿商品。这是简单的流通过程。于是 10 亿转到 S 手里,而 S 的以商品形式存在的产品的第二部分转化为货币。双方都发生资本的形态变化。租地农场主的 10 亿再转化为再生产过程所必需的生产要素。S 的成品再转化为货币,完成了从商品到货币的\textbf{形式上的}形态变化,没有这种形态变化,资本就不能再转化为自己的生产要素,因而也就不能进行再生产。这是第五个流通过程。有\textbf{10 亿工业品}(上年收获的产品)离开流通,进入再生产消费(a′—b′)\endnote{马克思在这里使用的字母符号(和标记)使《经济表》一目了然,无论在施马尔茨的著作中还是在魁奈的著作中都没有这样清楚。用两个字母(a—b,a—c,c—d 等等)来标明每一条线,使人能确定线的方向,即这条线是从哪个阶级到哪个阶级(方向按字母表上字母的顺序确定,a—b,a—c,c—d 等等)。例如,a—b 线表示土地所有者阶级和生产阶级(租地农场主)之间的流通以土地所有者阶级为出发点(后者向租地农场主购买食物)。用两个字母来标明每一条线,同时表明了货币的运动和商品的运动。例如,a—b 线表示货币的运动(土地所有者阶级向生产阶级支付 10 亿货币);但是这条线从相反的方向(b—a)来看,就表明商品的运动(生产阶级交给土地所有者阶级 10 亿食物)。虚线 a—b—c—d 由以下几个环节组成:(1)a—b 段表示土地所有者和生产阶级之间的流通(土地所有者向租地农场主购买 10 亿食物);(2)a—c 段表示土地所有者和不生产阶级——工业家之间的流通(土地所有者向工业家购买 10 亿工业品);(3)c—d 段表示不生产阶级和生产阶级之间的流通(工业家向租地农场主购买 10 亿食物)。a′—b′线表示生产阶级和不生产阶级之间的流通(租地农场主向工业家购买 10 亿工业品)。a″—b″线表示不生产阶级和生产阶级之间的最后的流通(工业家向租地农场主购买工业生产所必需的 10 亿原料)。——第 324、349、352 页。}。

最后,S 把这 10 亿货币——他的一半商品现在以这 10 亿货币的形式存在——再转化为商品的生产条件的另一半,即原料等等(a″—b″)。这是简单流通。这对 S 来说,同时也是他的资本变为适于再生产的形式的形态变化,而对 F 来说,是他的产品变为货币的再转化,现在,“总产品”的最后 1/5 离开流通,进入消费。

总之:1/5 加入租地农场主的再生产过程,不进入流通,1/5 被土地所有者消费掉;合计 2/5;2/5 由 S 取得;共计 4/5。\endnote{马克思在这里和在后面都采用魁奈的说法:只有 1/5 的农业总产品不进入流通,而由生产阶级以实物形式享用。马克思在手稿第 XXIII 本第 1433—1434 页(见本册第 405—406 页)和他写的《反杜林论》第二编第十章中又回过头来谈这个问题。他在这一章中对魁奈关于农业中流动资本的补偿的观点作了如下详细说明:“价值五十亿的全部总产品因而掌握在生产阶级的手中,也就是说,首先是掌握在租地农场主的手中,这些租地农场主每年花费二十亿经营资本(与一百亿基本投资相适应)来生产全部总产品。为了补偿经营资本,因而也为了维持一切直接从事农业的人所需要的农产品、生活资料、原料等等,是以实物形式从总收成中拿出来的,并且花费在新的农业生产上。因为,正如前面所说,是以一次规定了的标准的固定价格和简单再生产为前提,所以总产品中预先拿出去的部分的货币价值,等于二十亿利弗尔。因此,这一部分没有进入一般的流通,因为正如已经指出的,任何发生于每一个别阶级的范围之内而不是发生于各阶级相互之间的流通,都没有列入表内。”(《马克思恩格斯全集》中文版第 20 卷第 270—271 页)因此,按照魁奈的说法,应当说租地农场主以实物形式补偿他们的流动资本的那部分产品,占他们的全部总产品的 2/5。——第 352、406 页。}

在这里,这笔账显然有缺陷。看来,魁奈是这样计算的:F 给 P10 亿食物(a—b 线)。F 用 10 亿原料补偿 S 的资本(a″—b″)。10 亿食物构成 S 的工资,这笔工资的价值就是 S 加在商品上并在加的过程中耗费在食物上的价值(c—d)。10 亿留在再生产过程中(a′),不进入流通。最后,10 亿产品补偿“预付”(a′—b′)。可是,魁奈没有看到:S 既不是用这价值 10 亿的工业品向租地农场主购买食物,也不是用它向租地农场主购买原料;S 在同租地农场主的货币结算中,倒是用租地农场主自己的货币偿还租地农场主。要知道,魁奈一开始就是从下面这个假设出发的,即租地农场主除了他的总产品以外还有 20 亿货币,这 20 亿,总的说来,是一个基金,流通的货币是从这里汲取的。

此外,魁奈忘记了,除了这 50 亿总产品以外,还有 20 亿总产品,即在新收获前就已制造出来了的工业品。因为 50 亿只代表租地农场主的全部年产品,[432]租地农场主得到的全部收成,而决不代表要由这个收成补偿自己再生产要素的工业总产品。

因此,现有:(1)租地农场主方面——20 亿货币;(2)50 亿土地总产品;(3)价值 20 亿的工业品。就是说,有 20 亿货币和 70 亿产品(农产品和工业品)。流通过程可以概括如下(F——租地农场主,P——土地所有者,S——工业家,不生产阶级):

F 付给 P20 亿货币地租,P 向 F 购买 10 亿食物。这样就实现了租地农场主的总产品的 1/5。同时有 10 亿货币流回他的手里。其次,P 向 S 购买 10 亿商品。这样就实现了 S 的总产品的 1/2。S 卖得 10 亿货币。他用这笔货币向 F 购买价值 10 亿的食物。从而 S 补偿了他的资本的再生产要素的 1/2。这样就实现了租地农场主的总产品的又一个 1/5。同时,租地农场主又有了 20 亿货币,这是他卖给 P 和 S 的 20 亿食物的价格。然后,F 向 S 购买 10 亿商品,以补偿自己的“预付”的一半。这样就实现了工业家的总产品的另一半。最后,S 用最后这 10 亿货币向租地农场主购买原料。这样就实现了租地农场主的总产品的第三个 1/5,S 的资本的再生产要素的另一半得到补偿,而 10 亿又流回租地农场主手里。租地农场主又有了 20 亿,这是合乎情理的,因为魁奈把租地农场主看作资本家,在同租地农场主的关系上,P 只是收入所得者,S 只是工资所得者。如果租地农场主直接用他的产品支付 P 和 S,他就不付出任何货币。但是,因为他支付了货币,P 和 S 就用这些货币来买他的产品,货币就流回他手里。这是这样一种形式上的货币回流,即货币流回到以买者资格开始全部业务并将它完成的工业资本家手里。其次,租地农场主的总产品中补偿他的“预付”的那 1/5 属于再生产。还剩下 1/5 食物要实现,这是完全不进入流通的。

\tsectionnonum{[(5)《经济表》上的商品流通和货币流通。货币流回出发点的各种情况]}

S 向租地农场主购买 10 亿食物和 10 亿原料,相反,F 只向 S 购买 10 亿商品以补偿他的“预付”。因此,S 要支付 10 亿的差额,而这个差额最终要用 S 从 P 那里得到的 10 亿来支付。看来,魁奈把向 F\textbf{支付}这 10 亿,同向 F\textbf{购买}10 亿产品混淆起来了。至于魁奈怎样考虑这一点,应该参阅勃多的解说\endnote{马克思指勃多的注释:《经济表说明》(载于《重农学派》,附欧·德尔的绪论和评注,1846 年巴黎版第 2 部第 822—867 页)。——第 354 页。}。

实际上(按我们的计算),20 亿只起下列作用:(1)以货币支付 20 亿的地租;(2)使租地农场主的 30 亿总产品流通(其中 10 亿食物给 P,20 亿食物和原料给 S),并使 S 的 20 亿总产品流通(其中 10 亿给 P 用于消费,10 亿给 F 用于再生产的消费)。

最后一次购买(a″—b″)是 S 向 F 购买原料,S 以货币支付 F。

[433]再说一次吧:

S 从 P 那里取得了 10 亿货币。他用这 10 亿货币向 F 购买价值 10 亿的食物。F 用这 10 亿货币向 S 购买商品。S 又用这 10 亿货币向 F 购买原料。

或者:S 向 F 购买 10 亿货币的原料和 10 亿货币的食物。F 向 S 购买 10 亿货币的商品。在这种情况下,10 亿流回 S 手里,但是,所以如此,只是因为假定 S 除了从土地所有者那里取得的 10 亿货币和他需要出卖的 10 亿商品之外,还有他自己投入流通的 10 亿货币。按照这个假定,为了使商品在 S 和租地农场主之间流通,就需要 20 亿货币,而不是 10 亿货币。结果有 10 亿回到 S 手里。这是因为 S 用 20 亿货币向租地农场主购买。而租地农场主向 S 购买 10 亿,把从 S 那里得到的货币的半数付还给 S。

在第一种情况下,S 分两次购买。第一次他付出 10 亿;这 10 亿从 F 流回他的手里;然后他再次把这 10 亿最后付给 F,这样就不再流回了。

相反,在第二种情况下,S 一次就购买 20 亿,当 F 再向 S 购买 10 亿的时候,这 10 亿就留在 S 手里。在这种情况下,流通就需要 20 亿,而不是 10 亿。在第一种情况下,10 亿货币经过两次流通,实现了 20 亿商品。在第二种情况下,20 亿货币经过一次流通,也实现了 20 亿商品。当租地农场主现在支付 10 亿给工业家 S 时,S 由此得到的货币并不比第一种情况下多。因为 S 除了把 10 亿商品投入流通之外,还从他自己的在流通过程开始前就存在的基金中,拿出 10 亿货币投入流通。他为流通投放了货币,因而货币流回他手里。

在第一种情况下:S 用 10 亿货币向 F 购买 10 亿商品。F 用 10 亿货币向 S 购买 10 亿商品。S 又用 10 亿货币向 F 购买 10 亿商品,于是 10 亿货币留在 F 手里。

在第二种情况下:S 用 20 亿货币向 F 购买 20 亿商品。F 用 10 亿货币向 S 购买 10 亿商品。同第一种情况一样,租地农场主手里留下 10 亿货币。但 S 收回 10 亿,这 10 亿是以前从他这方面预付到流通中的资本,现在从流通中回到他手里。S 向 F 购买 20 亿商品。F 向 S 购买 10 亿商品。因此,S 在任何情况下都要支付 10 亿差额,但是不会更多。既然 S 由于这种流通方式的特点,为上述差额付给了 F20 亿,那末 F 就付还给 S10 亿,而在第一种情况下,F 则不付还给 S 任何货币。

就是说,在第一种情况下,S 向 F 购买 20 亿,F 向 S 购买 10 亿。因此,差额仍然是应付给 F10 亿。但是,这个差额是这样支付给 F 的,就是 F 自己的货币又流回 F 手里,因为 S 先向 F 购买 10 亿,然后 F 向 S 购买 10 亿,最后 S 向 F 购买 10 亿。10 亿在这里使 30 亿流通。但是总的说来,流通中存在过的价值(如果货币是实在货币)等于 40 亿:30 亿是商品,10 亿是货币。流通的和最初(支付给租地农场主)投入流通的货币额,从来不超过 10 亿,就是说,不超过 S 应付给 F 的差额。由于在 S 第二次向 F 购买 10 亿以前,F 已向 S 购买 10 亿,S 就可以用这 10 亿支付他应付的差额。

在第二种情况下,S 把 20 亿投入流通。诚然,S 用这 20 亿向 F 购买了价值 20 亿的商品。这 20 亿在这里用作流通手段,它被支出是要换得商品形式的等价物。但是,F 又向 S 购买 10 亿。这样就有 10 亿回到 S 手里,因为 S 应付给 F 的差额只是 10 亿,而不是 20 亿。S 现在已经用商品给 F 补偿了 10 亿,因此 F 必须付还给 S10 亿,这 10 亿\textbf{现在}看来是 S 以货币形式多付给 F 的。这个情况很值得注意,要稍微费点时间谈一谈。

前面所假定的 30 亿商品(其中 20 亿是食物[和原料],10 亿是工业品)的流通,可以有几种不同情况;但是,这里要注意:\textbf{第一},按照魁奈的假定,当 S 和 F 之间的流通过程开始的时候,有 10 亿货币在 S 手里,10 亿货币在 F 手里;\textbf{第二},为了举例说明,我们假定 S 除了从 P 那里得到 10 亿以外,在钱柜里还有 10 亿货币。

[434](I)\textbf{第一},情况就象魁奈假设的那样。S 用 10 亿货币向 F 购买 10 亿商品;F 用从 S 那里得来的 10 亿货币,向 S 购买 10 亿商品;最后,S 用这样收回的 10 亿货币,向 F 购买 10 亿商品。因此,有 10 亿货币留在 F 手里,这笔货币对 F 说来,代表资本(实际上,这 10 亿货币同 F 从 P 那里收回的另外 10 亿货币一起,形成他下年度用来重新支付货币地租的收入,即 20 亿货币)。在这里,10 亿货币流通三次(从 S 到 F,从 F 到 S,再从 S 到 F),每次偿付 10 亿商品,因而总共偿付 30 亿商品。如果货币本身具有价值,在流通中就有 40 亿价值。货币在这里只执行流通手段的职能,但是对于 F(货币最后留在他的手里)说来,却转化为货币,而且可能转化为资本。

(II)\textbf{第二},货币只执行支付手段的职能。在这种情况下,S 向 F 购买 20 亿商品,F 向 S 购买 10 亿商品,他们彼此进行结算。在交易结束时,S 要用货币支付 10 亿差额。同前面情况一样,10 亿货币落入 F 的钱柜,但它一直没有起过流通手段的作用。这笔货币对 F 说来是资本的转移,因为它只给 F 补偿一笔 10 亿商品的资本。这样一来,同第一种情况一样,有 40 亿价值加入流通。但是,10 亿货币只发生一次运动,而不是三次运动,货币只支付同额的商品价值。而在第一种情况下,货币则支付了三倍于它本身的价值。同第一种情况相比,省去了两次多余的流通行为。

(III)\textbf{第三},F 首先作为买者出现,用 10 亿货币(他从 P 那里得来)向 S 购买 10 亿商品。这 10 亿货币不是当作贮藏货币闲放在 F 身边到下年度支付地租,而是现在就进入流通。于是,S 有了 20 亿货币(10 亿货币从 P 那里得来,10 亿货币从 F 那里得来)。他用这 20 亿货币向 F 购买了价值 20 亿的商品。现在流通中有 50 亿价值(30 亿商品和 20 亿货币)。发生了 10 亿货币和 10 亿商品的流通以及 20 亿货币和 20 亿商品的流通。在这 20 亿货币中,来自 F 的 10 亿流通两次,来自 S 的 10 亿只流通一次。现在 20 亿货币回到 F 手里,但是其中只有 10 亿货币是向他结算差额的,另外 10 亿货币,即他从前因首先作为买者出现而投入流通的那 10 亿,则通过流通过程流回他的手里。

(IV)\textbf{第四},S 用 20 亿货币(10 亿货币从 P 那里得来,10 亿从他自己的钱柜中取出投入流通)一次就向 F 购买价值 20 亿的商品。F 又向 S 购买 10 亿商品,因而把 10 亿货币还给 S,同前面情况一样,F 还有 10 亿货币留在手上,用来结算他同 S 之间的差额。这里投入流通的有 50 亿价值。流通行为是两次。

在 III 的情况下,在 S 还给 F 的 20 亿货币中,10 亿代表 F 自己投入流通的货币,只有 10 亿代表 S 投入流通的货币。这里回到 F 手里的是 20 亿货币,而不是 10 亿货币,但实际上他得到的只是 10 亿,因为另外 10 亿是他自己投入流通的。在 IV 的情况下,有 10 亿货币回到 S 手里,但是这 10 亿货币,是他自己从钱柜中取出投入流通的,而不是向租地农场主出卖自己的商品得来的。

如果说,在 I 的情况下和在 II 的情况下,流通中的货币都是从来不超过 10 亿,可是在 I 的情况下货币流通三次,转手三次,而在 II 的情况下只流通一次,转手一次,那末,这不过是由于在 II 的情况下假定有发达的信用,因此节约了支付的次数,而在 I 的情况下则发生急速的运动,货币每次都作为流通手段出现,因此价值每一次都要以二重形式在两极出现,一极以货币形式,一极以商品形式。如果说,在 III、IV 的情况下有 20 亿货币流通,不象在 I、II 的情况下是 10 亿货币流通,那末,这是因为在 III、IV 的情况下,都有 20 亿的商品价值一次就进入流通过程(在 III 的情况下是 S 作为买者结束流通过程,在 H 的情况下是 S 作为买者开始流通过程);总之,20 亿商品一次就进入流通,并且假定它们立即被购买,而不是结算后才支付。

但不管怎样,在这个运动中最有意思的是,在 III 的情况下 10 亿货币留在租地农场主手里,而在 IV 的情况下 10 亿货币则留在工业家手里,虽然在两种情况下 10 亿货币的差额都是付给租地农场主,而租地农场主在 III 的情况下没有多得分文,在 H 的情况下也没有少得分文。自然,这里总是等价物交换,如果我们谈到差额,它所指的不过是用货币而不是用商品支付的价值等价物。

在 III 的情况下,F 把 10 亿货币投入流通,从 S 那里换得商品等价物,即得到 10 亿商品。但是,后来 S 用 20 亿货币向 F 购买商品。这样,F 投入流通的第一个 10 亿货币回到 F 手里,然而有 10 亿商品离开 F。这 10 亿商品是用 F 以前支出的货币付给 F 的。从对第二个 10 亿商品的支付中 F 得到第二个 10 亿货币。这一货币差额由 F 收进,因为 F 总共只买进 10 亿货币的商品,而从 F 那里买去的却是价值 20 亿的商品。

[435]在 IV 的情况下,S 把 20 亿货币一次投入流通,从 F 那里换得 20 亿商品。F 又用 S 自己支出的这笔货币向 S 购买 10 亿商品,于是 10 亿货币回到 S 手里。

在 IV 的情况下:S 实际上以商品形式给 F10 亿商品(等于 10 亿货币),以货币形式给 F20 亿货币,因此,总共是 30 亿货币。S 从 F 那里得到的只是 20 亿商品。因此,F 应还给 S10 亿货币。

在 III 的情况下:F 以商品形式给 S20 亿商品(等于 20 亿货币),以货币形式给 S10 亿货币,因此,总共是 30 亿货币,而 F 从 S 那里得到的,只是 10 亿商品,等于 10 亿货币。因此,S 应还给 F20 亿货币,其中 10 亿,S 用 F 自己投入流通的货币来付还,另外 10 亿是 S 自己投入流通。F 留下 10 亿货币的差额,但不可能留下 20 亿货币。

在这两种情况下,S 都是得到 20 亿商品,F 都是得到 10 亿商品加 10 亿货币即货币差额。如果说在 G 的情况下,另外还有 10 亿货币流到 F 手里,那末,这只不过是他投入流通的超过他因出卖商品而从流通中抽出的数额的货币。在 H 的情况下,S 的情况也是一样。

在这两种情况下,S 都要用货币支付 10 亿货币的差额,因为他从流通中抽出 20 亿商品,而投入流通的只有 10 亿商品。在这两种情况下,F 都要以货币形式收进 10 亿货币的差额,因为他把 20 亿商品投入流通,而从流通中只抽出 10 亿商品,因此,对第二个 10 亿商品,必须用货币向他结算。在这两种情况下,最后能够转手的只有这 10 亿货币。但是,因为流通中有 20 亿货币,所以这 10 亿货币就不得不流回原来把它投入流通的人手里,不管他是 F(他从流通中收进 10 亿货币的差额,此外还把另外 10 亿货币投进了流通)还是 S(他应支付的只有 10 亿货币的差额,此外还把 10 亿货币投进了流通)。

在 III 的情况下,进入流通的货币,比其他情况下使这个商品量流通所必需的货币量多 10 亿,因为 F 首先作为买者出现,不管最后结算情况如何,他都必须把货币投入流通。在 IV 的情况下,同样有 20 亿货币进入流通,不象在 II 的情况下只有 10 亿,因为在 IV 的情况下,第一,一开始 S 就作为买者出现;第二,他一次就买进 20 亿商品。在 III、IV 的情况下,买者和卖者之间\textbf{流通}的货币,最后只能等于其中一方应付的差额。因为 S 或 F 付出的超过这个数额的货币,都要付还给 S 或 F 本人。

假定 F 向 S 购买 20 亿商品。那末,情况就会变成这样,F 给 S10 亿货币以交换商品。S 向 F 购买 20 亿货币的商品,因此,第一个 10 亿货币就回到 F 手里,还外加 10 亿货币。F 再用 10 亿货币向 S 购买商品,于是这笔货币又回到 S 手里。在过程结束时,F 有 20 亿货币的商品和流通过程开始之前他原来就有的 10 亿货币;而 S 有 20 亿商品和他同样是原来就有的 10 亿货币。F 的 10 亿货币和 S 的 10 亿货币只起了流通手段的作用,后来则作为货币或者在这种情况下也作为资本,流回到把它们投入流通的双方手里。如果双方都把货币用作支付手段的话,他们进行结算,就是 20 亿商品对 20 亿商品;他们彼此销账;双方之间连一分钱也不流通。

因此,在互为买者和卖者两次对立的双方之间作为流通手段流通的货币,都是流回的;这些货币的流通可能有三种情况。

[\textbf{第一},]投入流通的商品价值彼此相等。在这种情况下,货币流回那个把它预付到流通过程中去,并且这样以自己资本开支流通费用的人手里。例如,如果 F 和 S 各向对方购买 20 亿商品,S 先开头,那末,S 就一次向 F 购买 20 亿货币的商品。然后,F 向 S 购买 20 亿商品,把这 20 亿货币还给 S。这样,S 在交易前和交易后,都有 20 亿商品和 20 亿货币。或者,如果象前面提到的情况那样,双方都预付等量流通手段,那末,双方预付到流通过程中的流通手段都回到各自的手里;就象前面 10 亿货币回到 F 手里,10 亿货币回到 S 手里那样。

\textbf{第二},双方交换的商品价值不等。出现一个要用货币支付的差额。如果商品流通象前面 I 的情况那样,进入\textbf{流通}的货币量没有超过为支付这个差额所必需的货币额,——因为始终只有这个货币额在双方之间来回,——那末,这笔货币最终落入收进这个差额的最后卖者手里。

\textbf{第三},双方交换的商品价值不等;有一个差额要支付;但是,商品流通进行时,流通的货币多于为支付这个差额所必需的货币;在这种情况下,超出这个差额的货币流回预付货币的一方。在 III 的情况下,流回收进差额的人手里,在 IV 的情况下,流回应付差额的人手里。

在上面“第二”所说的情况下,只有当收进差额的人是第一个买者的时候,象在工人和资本家的例子中那样,货币才\textbf{流回}收进差额的人手里。如果象在 II 的情况下那样,另一方首先作为买者出现,那末,货币就离开他而到他的对方手里去了。

[436]\fontbox{~\{}自然,这一切只限于这样的假设:一定的商品量在同一些人之间买卖,其中每一个人交替地在对方面前作为买者和卖者出现。相反,我们假定,有 3000 商品平分给商品所有者-卖者 A、A′、A″,同他们相对立的是买者 B、B′、B″。如果这里三次购买行为同时发生,因而在空间上并行地发生,那末,就要有 3000 货币流通,才能使每一个 A 都有 1000 货币,而每一个 B 都有 1000 商品。如果几次购买行为一个接着一个,在时间上连续发生,那末,只有商品形态变化串连在一起,也就是说,只有一些人既作为买者,又作为卖者出现,虽然不是象上面说的那样,对同一些人既作为买者又作为卖者出现,而是对一些人作为买者出现,对另一些人又作为卖者出现,只有这样,同一个 1000 货币的流通才能完成这几次购买行为。例如:(1)A 卖给 B1000 货币的商品;(2)A 用这 1000 货币向 B′购买;(3)B′用这 1000 货币向 A′购买;(4)A′用这 1000 货币向 B″购买;(5)B″用这 1000 货币向 A″购买。货币在 6 个人之间转手 5 次,但是也使 5000 货币的商品得以流通。如果只使价值 3000 货币的商品流通,那就是:(1)A 用 1000 货币向 B 购买;(2)B 用 1000 货币向 A′购买;(3)A′用 1000 货币向 B′购买。在 4 个人之间转手 3 次。这是 G—W 过程。\fontbox{\}~}

上面阐明的各种情况,并不违背以前所阐明的规律,即:

\begin{quote}“已知货币流通速度,已知商品价格总额,流通手段量就已决定。”(第 1 分册第 85 页)\endnote{指《政治经济学批判》第一分册。见《马克思恩格斯全集》中文版第 13 卷第 96 页。——第 363 页。}\end{quote}

在前面 I 的情况的例子中,1000 货币\endnote{在魁奈的《经济表》中是 1000 个百万(即 10 亿)图尔利弗尔,马克思在这里用 1000 货币单位来代替,这丝毫不改变问题的本质。——第 363 页。}流通了 3 次,使 3000 货币的商品进入运动。因此,流通的货币量

\centerbox{=[3000(价格总额)/3(流通速度)]或[3000(价格总额)/3 次流通]=1000 货币}

在 III 或 IV 的情况下,流通的商品的价格总额固然是同一数额(3000 货币),但流通速度不同。2000 货币流通 1 次,即 1000 货币加 1000 货币。但是,这 2000 货币中有 1000 再流通了 1 次。2000 货币使价值 3000 的商品的 2/3 流通,而 2000 货币的半数使价值 1000 的商品即余下的 1/3 流通;一个 1000 货币流通 2 次,但另一个 1000 货币仅仅流通 1 次。1000 货币的 2 次流通实现了等于 2000 货币的商品价格,1000 货币的 1 次流通实现了等于 1000 货币的商品价格,加在一起等于 3000 商品。那末,同这些商品——货币使之流通的商品——有关的货币流通速度怎样呢\fontbox{?}这 2000 货币流通 1+(1/2)次(这就是说\textbf{首先}全额流通 1 次,然后其中半数又流通 1 次),等于 3/2 次。实际是:

\centerbox{[3000(价格总额)/3/2 次流通]=2000 货币}

但是,这里货币流通的\textbf{不同速度}是由什么决定的呢\fontbox{?}

不论 III 或 IV 的情况,它们与 I 的情况的不同是这样引起的。在 I 的情况下,每次流通的商品的价格总额,始终不大于也不小于一般进入流通的商品总量的价格总额的 1/3。始终只有价值 1000 货币的商品在流通。在 III 和 IV 的情况下则相反,一次是 2000 货币的商品流通,一次是 1000 货币的商品流通;因此,一次是现有商品总量的 2/3 流通,一次是它的 1/3 流通。由于同一个理由,批发商业中流通的铸币,必定大于零售商业中流通的铸币。

我已经指出过(第 1 分册,《货币的流通》),货币的回流首先表明,\textbf{买者又变成了卖者},\endnote{见《马克思恩格斯全集》中文版第 13 卷第 89—90 页。——第 364 页。}至于他是否卖给他曾经向之买过东西的人,实际上是无关紧要的。可是,如果事情发生在同一些人之间,那就会出现一些现象,这些现象曾经引起许多混乱(德斯杜特·德·特拉西)\endnote{见本册第 277—291 页。并见马克思《资本论》第 2 卷第 20 章第 13 节:《德斯杜特·德·特拉西的再生产理论》。——第 343、364 页。}。买者变成卖者表明,要出卖的是新的商品。商品流通的继续,就是商品流通的不断更新(第 1 分册第 78 页)\endnote{马克思引的是《政治经济学批判》第一分册(见《马克思恩格斯全集》中文版第 13 卷第 88—89 页)。参看注 100。——第 364 页。},——因而,这里有一个再生产过程。买者可以再变成卖者(如厂主对工人)而不表示任何再生产行为。只有这种货币回流的继续、重复,才表示再生产过程。

当货币的回流代表资本再转化为资本的货币形式的时候,如果资本继续作为资本运动,这种货币回流必然表示一个周转的结束和再生产过程的重新开始。在这里,同在其他一切场合一样,资本家是卖者,W—G,然后变成买者,G—W,但是,他的资本只有变为 G,才重新具有能够同它的再生产要素相交换的形式,而 W 在这里代表这些再生产要素。G—W 在这里代表货币资本转化为生产资本,或产业资本。

其次,我们说过,货币流回它的出发点的这种回流可能表示,在一系列的买卖之后,货币差额由首先开始这一系列过程的买者收进。F 向 S 购买 1000 货币的商品。S 向 F 购买 2000 货币的商品。这里,有 1000 货币流回 F 手里。至于另一个 1000,那不过是货币在 S 和 F 之间简单变换位置。

[437]最后,货币流回出发点的回流可能不表示支付差额,这在下述两种场合都可能发生:(1)双方支付平衡,因此没有任何差额要用货币支付,(2)双方支付\textbf{不}平衡,因此需要支付一个货币差额。请参看上面分析的各种情况。在所有这些情况下,譬如说,同 F 相对的 S 是不是同一个人,那是无关紧要的;这里,S 对于 F 和 F 对于 S,都是代表向他出卖和向他购买的人的总数(完全象在货币回流表现支付差额的例子中一样)。在所有这些情况下,货币都流回到把货币可以说预付到流通过程中去的人手里。货币同银行券一样,在流通中完事之后,就回到把它投入流通的人手里。\textbf{在这里货币只是流通手段。最后出现的资本家彼此支付,这样,货币就回到首先把它投入流通的人手里}。

还有一个问题,即资本家从流通中抽出的货币多于他投入流通的货币的问题,留待以后解决。\endnote{马克思在《资本论》第二卷第十七章、第二十章第五节和第十二节、第二十一章(特别是第一节中的第一小节《贮藏货币的形成》)中全面考察了这个问题。——第 347、349、365 页。}

\tsectionnonum{[(6)《经济表》在政治经济学史上的意义]}

现在回过头来讲魁奈。

亚·斯密带着几分讽刺意味引用了米拉波侯爵的夸张说法:

\begin{quote}“自从世界形成以来,有三大发明……第一是\textbf{文字}的发明……第二是\textbf{货币的发明}〈!〉……第三是《\textbf{经济表}》,这个表是前两者的结果和完成。”(\textbf{加尔涅}的译本,第 3 卷第 4 篇第 9 章第 540 页)\end{quote}

但是,实际上,这是一种尝试:把资本的整个生产过程表现为\textbf{再生产过程},把流通表现为仅仅是这个再生产过程的形式;把货币流通表现为仅仅是资本流通的一个要素;同时,把收入的起源、资本和收入之间的交换、再生产消费对最终消费的关系都包括到这个再生产过程中,把生产者和消费者之间(实际上是资本和收入之间)的流通包括到资本流通中;最后,把生产劳动的两大部门——原料生产和工业——之间的流通表现为这个再生产过程的要素,而且把这一切总结在一张《\textbf{表}》上,这张表实际上只有五条线,连结着六个出发点或归宿点。这个尝试是在十八世纪三十至六十年代政治经济学幼年时期做出的,这是一个极有天才的思想,毫无疑问是政治经济学至今所提出的一切思想中最有天才的思想。

至于资本流通、资本的再生产过程、资本在这个再生产过程中采取的各种不同的形式、资本流通同一般流通的联系,也就是说,不仅资本同资本的交换,而且资本同收入的交换,那末,斯密实际上只是接受了重农学派的遗产,对财产目录的各个项目作了更严格的分类和更详细的描述,但是对于过程的整体未必叙述和说明得象《经济表》大体上描绘的那样正确,尽管魁奈的前提是错误的。

此外,斯密评论重农学派说:

\begin{quote}“他们的著作肯定对他们的国家有些贡献。”(同上,第 538 页)\end{quote}

这样的评价,对于一个比如象杜尔哥(法国革命的直接先导之一)这样的人所起的作用来说,未免失之过稳罢。[437]

\tchapternonum{[第七章]兰盖}

\vicetitle{[对关于工人“自由”的资产阶级自由主义观点的最初批判]}

[438]\textbf{兰盖}《民法论》1767 年伦敦版。

按照我的写作计划,社会主义的和共产主义的著作家都不包括在历史的评论之内。这种历史的评论不过是要指出,一方面,政治经济学家们以怎样的形式自行批判,另一方面,政治经济学规律最先以怎样的历史路标的形式被揭示出来并得到进一步发展。因此,在考察剩余价值时,我把布里索、葛德文等等这样的十八世纪著作家,以及十九世纪的社会主义者和共产主义者,都放在一边了。至于我在这个评论中以后要说到的少数几个社会主义著作家\endnote{在《剩余价值理论》第三册(手稿第 XIV 本和 XV 本,第 852—890 页)有一章:《以李嘉图理论为依据反对政治经济学家的无产阶级反对派》。第 X 本(第 441—444 页)中未完成的论布雷一节和第 XVIII 本(第 1084—1086 页)中论霍吉斯金一节的结尾部分也属于这一章。——第 367 页。},他们不是本身站在资产阶级政治经济学的立场上,便是从资产阶级政治经济学的观点出发去同资产阶级政治经济学作斗争。

然而兰盖并不是社会主义者。他反对他同时代的启蒙运动者的资产阶级自由主义理想,反对资产阶级刚刚开始的统治,他的抨击半是认真半是嘲弄地采取反动的外观。他维护亚洲的专制主义,反对文明的欧洲形式的专制主义,他捍卫奴隶制,反对雇佣劳动。

第一卷。他的一句反对孟德斯鸠的话:

\begin{quote}“法律的精神就是所有权”\endnote{[兰盖,尼]《民法论,或社会的基本原理》1767 年伦敦版第 1 卷第 236 页。——第 368 页。}\end{quote}

就表明了他的见解的深刻。

兰盖碰到的和他对立的唯一的一批政治经济学家,是重农学派。

兰盖证明,富人占有一切生产条件;这是\textbf{生产条件的异化},而最简单形式的生产条件是自然要素本身。

\begin{quote}“在我们的各个文明国家里,一切自然要素都成了奴隶。”(第 188 页)\end{quote}

要取得这些被富人占有的财宝的一部分,必须用增加富人财富的繁重劳动来购买这一部分。

\begin{quote}“这样,整个被俘虏的自然,就不再向自己的儿女提供容易得到的维持生命的源泉了。自然的恩赐必须以辛苦的努力为代价,自然的赐予必须以顽强的劳动来取得。”\end{quote}

(这里,在“自然的赐予”这个词上,露出了重农学派的见解。)

\begin{quote}“\textbf{独占这些财宝的}富人,只有取得这种代价,才同意把财宝的极小部分还给大家使用。\textbf{为了得到分享他的财宝的许可,必须努力劳动来增加财宝}。”(第 189 页)“这样,就必须放弃自由的幻想。”(第 190 页)法律的存在是为了“批准〈对私有财产〉最初的夺取”,并“防止以后的夺取”。(第 192 页)“法律可以说是一种反对人类最大多数〈即无产者〉的阴谋。”(同上[第 195 页])“是社会创造了法律,而不是法律创造了社会。”(第 230 页)“所有权先于法律。”(第 236 页)\end{quote}

“社会”本身——人生活在社会中,而不是作为独立自主的个人——是所有权、建立在所有权基础上的法律以及由所有权必然产生的奴隶制的根源。

\begin{quote}一方面是土地耕种者和牧人过着和平的、孤立的生活。另一方面,还有“猎人,他们习惯于靠屠杀取得生活资料,习惯于成群结队,以便于围猎他们吃的野兽和瓜分猎物”。(第 279 页)“正是在猎人当中出现了最初的社会标志。”(第 278 页)“\textbf{真正的社会是牺牲牧人和土地耕种者的利益而形成的,是以}”联合起来的一伙猎人“\textbf{对他们进行奴役为基础的}”。(第 289 页)社会上的一切义务可归结为命令和服从。“人类一部分的地位降低,先是产生了社会,然后再产生法律。”(第 294 页)\end{quote}

贫困迫使丧失生产条件的工人为生活而劳动,去增加别人的财富。

\begin{quote}“只是因为没有别的活路,我们的短工才不得不耕种土地而自己享受不到它的果实,我们的石匠才不得不修建房屋而自己不能居住。贫穷把他们赶到市场上,等待主人开恩购买他们。\textbf{贫穷迫使他们跪在富人面前,央求富人准许他们使他发财}。”(第 274 页)“可见,奴役是产生社会的第一个原因,暴力是社会的第一个纽带。”(第 302 页)“他们〈人们〉关心的第一件事,无疑是获得自己的食物……关心的第二件事,就是想方设法\textbf{不劳动而获得自己的食物}。”(第 307—308 页)“他们只有\textbf{占有别人劳动的果实},才能做到这一点。”(第 308 页)“最初的征服者们,只是为了不受惩罚地过游手好闲的生活,才实行统治;他们成为国王,只是为了拥有生存资料。这就使统治的观念……大大缩小和简化了。”(第 309 页)“社会由暴力产生,所有权由夺取产生。”(第 347 页)“主人和奴隶一出现,社会就形成了。”(第 343 页)“市民社会一开始就有两个[439]柱石,一方面是大部分男子的奴隶地位,另一方面是全部女子的奴隶地位……社会靠四分之三的人口来保证少数有产者的幸福、财产、闲暇,社会关心的只是这少数人。”(第 365 页)\end{quote}

第二卷。

\begin{quote}“因此,问题不是要弄清奴隶制是否同自然本身有矛盾,而是要弄清奴隶制是否同社会的本性有矛盾……奴隶制是同社会的存在分不开的。”(第 256 页)“社会和市民的受奴役同时产生。”(第 257 页)“终身奴隶制是社会的不可毁灭的基础。”(第 347 页)“被迫靠某个人的施舍才获得生存资料的人们,只是在\textbf{这个人由于从他们手里夺得财物而大大富裕起来},以致有可能把其中的一小部分\textbf{归还}给他们的时候才出现的。此人虚伪的慷慨,不过是\textbf{把他占有的别人劳动果实的一部分归还给别人而已}。”(第 242 页)“人们被迫耕种而自己得不到收获物,为了别人的幸福而牺牲自己的幸福,被迫进行无希望的劳动,这不\textbf{就是奴隶制}吗\fontbox{?}人们被迫在鞭打下劳动,而回到畜栏只得到一点燕麦,奴隶制的真正历史不就是从这个时候开始的吗\fontbox{?}只有在发达的社会中,生存资料对\textbf{饥饿的}贫民来说才是他们的自由的充分\textbf{等价物};在发展初期的社会里,这样不平等的交换,在自由人看来是骇人听闻的事情。只有对\textbf{战俘}才能这样做。只有剥夺了他们享有任何财产的权利之后,才能使这样的交换对他们说来是必然的。”(第 244—245 页)“\textbf{社会的本质……就是使富人免除劳动},使富人获得新的器官、获得不会疲倦的肢体来担负一切繁重劳动,\textbf{而劳动果实则由富人据为己有}。这就是奴隶制使富人轻而易举地达到的目的。他购买了那些必须为他服务的人们。”(第 461 页)“奴隶制废除了,但人们决不会废弃财富和财富的好处……因此,除了名称改变以外,一切都必须照旧。最大多数人总是必须靠工资生活,依赖于\textbf{把全部财物据为己有}的极少数人。这样,奴隶制就在世上永存下来,不过名称更加动听。它现在以仆人的美名出现于我们中间。”(第 462 页)\end{quote}

兰盖说,这里所说的“仆人”不是指仆役等等。

\begin{quote}“城市和乡村住满了另一种仆人,他们人数更多、更有用、更勤劳,他们被称为《journaliers》(短工),‘\textbf{手工工人}’等等。他们没有用奢侈的虚饰来玷污自己;他们穿着令人厌恶的破烂衣衫,穿着这种贫穷的\textbf{制服}在呻吟。\textbf{在他们的劳动所创造的丰裕财富中,他们从来分不到一丝一毫}。当财富竟肯接受\textbf{他们赠送的礼物}时,财富就象是对他们开恩一样。他们还必须\textbf{为他们能够向财富服务}表示感激。当他们抱着财富的双膝,请求\textbf{允许他们对财富做点有用的事时},财富用最侮辱人的轻蔑态度对待他们。财富迫使人们去央求得到这种允许,并且\textbf{在真慷慨同假恩惠之间进行的这种独特的交换中,受惠者方面}是傲慢的和轻蔑的,\textbf{给予者方面}则是驯服的、焦虑的和勤恳的。事实上正是这样的仆人在我们这里取代了奴隶。”(第 463—464 页)“必须弄清\textbf{奴隶制的废除}实际上给他们带来了什么利益。我要沉痛而直率地说:这全部利益就是他们永远经受着饿死的恐怖,这种不幸,至少他们的处在人类社会这一底层的先辈是没有遭受到的。”(第 464 页)“你们说,他[工人]是自由的。唉!他的不幸也正是在这里。他同任何人无关,任何人也同他无关。当需要他的时候,人们就\textbf{用最低的价钱雇用}他。人们答应给他的微不足道的\textbf{工资},只够\textbf{他交换出去的一个工作日所必需的生存资料的价格}。人们叫\textbf{监工督促他尽快劳动};人们催他,赶他,唯恐他想出一种偷懒的法子来少花一半力气,人们生怕他想要\textbf{拉长劳动时间}就会使他的手变得不灵巧,会把他的工具弄钝。\textbf{贪婪的吝啬鬼不放心地监视着他,只要他稍一中断工作,就大加叱责}。只要他休息一下,\textbf{就硬说是偷窃了他}。工作一完,他就被解雇,人们解雇他时,象雇用他时一样冷淡,丝毫也不考虑,[440]\textbf{如果他明天找不到工作},他劳苦了一天所得到的 20 或 30 苏够不够维持生活。”(第 466—467 页)“他是自由了!正因为如此,我才怜惜他。正因为如此,人们雇用他来干活时才极端不爱惜他。正因为如此,人们才更加肆无忌惮地浪费他的生命。奴隶对于自己的主人来说是一种有价值的东西,因为主人为他花了钱。而工人并没有使雇用他的富裕的享乐者花费什么。在奴隶制时期,人的血是有一定价格的。人的价值至少等于他们在市场上被卖的那个数额。自从停止贩卖人口以来,人实际上也就没有任何内在价值了。在军队中,对工兵的估价,比对辎重马的估价低得多,因为马的价钱很贵,而工兵不用花钱就能弄来。奴隶制的废除,使这样的估价方法从军队生活被搬用到市民生活中来;\textbf{从此以后,没有一个富裕的市民不是象威武的勇士那样来评价人了}。”(第 467 页)“短工为了替财富服务而出生、成长、受教育,这就象财富在自己的领地内打死的野兽一样,不用花财富分文。好象财富真的掌握了倒霉的庞培瞎吹嘘的那个秘密似的。只要财富往地上一跺脚,就会从地里钻出一大群勤劳的人,争先恐后地要为他服务。在这一大群为他造房子或管花园的雇佣者当中,如果少了一个人,空缺是看不出来的,它会马上被填补起来,不用任何人过问。大河里失掉一滴水是没有什么可惜的,因为新水流不断流进来。工人的情形也是这样;要找代替的人很容易,所以\textbf{富人}对待他们是冷酷无情的。”\end{quote}

(这是兰盖的说法,他提的还不是资本家)(第 468 页)

\begin{quote}“人们说,他们没有主人……但这是明显的滥用词句。他们没有主人,是什么意思呢\fontbox{?}他们有一个主人,而且是一切主人中最可怕、最专制的主人,这就是\textbf{贫困}。贫困使他们陷入最残酷的奴隶地位。\textbf{他们不是听命于某一个个别的人,而是听命于所有一切人}。他们非去讨好和巴结不可的,不只是某一个统治他们的暴君,否则他们的奴隶地位就有一定的界限了,也比较好忍受了。\textbf{他们成了每一个有钱人的仆人},因此,他们的奴隶地位就是没有界限的,极端严酷的了。有人说:如果他们在一个主人那里过得不好,那末他们至少有一点是可以告慰的,那就是可以向主人申述,并且另找一个主人;而奴隶就不能这样。可见奴隶是更不幸的。什么样的诡辩啊!请想一想,\textbf{迫使别人劳动的人}是很少很少的,而劳动者很多。”(第 470—471 页)“你们授予他们的那个虚幻的自由,对他们会有什么结果呢\fontbox{?}\textbf{他们只能靠出租自己的双手来生活。可见,他们必须找到一个雇用他们的人,要不就饿死。难道这就是自由了吗}\fontbox{?}”(第 472 页)“最可怕的是,他们工资的菲薄竟成了工资进一步下降的原因。短工愈穷,他就愈便宜地出卖自己。他穷得愈厉害,他的劳动的报酬就愈低。当他含泪哀求他面前象暴君一样的人接受他的服务时,这些人一点也不脸红,好象在摸摸他的脉搏,判断他剩下的力气还有多少。他们根据他衰弱的程度,来确定给他多少价钱。在他们看来,他愈是虚弱得濒于死亡,他们就愈是削减还能救他命的那些东西。这些野蛮人给他的东西,与其说用来延长他的生命,不如说用来推迟他的死期。”(第 482—483 页)“〈短工的〉独立……是我们时代的精巧性所造成的最有害的灾祸之一。这种独立使富人愈富,穷人愈穷。富人所积蓄的,都是穷人为维持生活所花费的;穷人不是从剩余中节省,而是不得不从最需要的东西中节省。”(第 483 页)“今天,竟如此容易维持庞大的军队,这些军队连同奢侈一起导致人类的毁灭。这只能归功于奴隶制的废除……自从不再有奴隶以来,放荡和赤贫才造成一日得 5 苏的勇士。”(第 484—485 页)“我认为,它〈亚洲的奴隶制〉对于被迫用每天的劳动去谋生的人们来说,要比所有别的生存方式强百倍。”(第 496 页)“他们〈奴隶和雇佣工人〉的锁链是用同样的材料制成的,只不过颜色不同。一种人的锁链是黑色的,看起来比较重;另一种人的锁链不那么黑,看起来比较轻;但是如果不偏不倚地把它们衡量一下,就会发现它们之间没有丝毫差别,两者都同样是由贫困制成的。它们的重量完全一样,而且,如果说有一种更重一些,那恰好就是从表面看起来比较轻的那一种。”(第 510 页)\end{quote}

兰盖就工人问题向法国启蒙运动者们大声疾呼:

\begin{quote}“难道你们没有看见,这一大群羊的驯服,直率地说,绝对顺从,创造了牧人的财富吗\fontbox{?}……请相信我,为了他〈牧人〉的利益,为了你们的利益,甚至为了它们〈羊〉自己的利益,还是让它们抱定它们历来的信念:相信一只向它们吠叫的狗要比所有的羊加在一起都强大吧。让它们一看见狗的影子就不知所以地逃跑吧。大家都可以由此得到好处……你们可以更容易地把它们赶去剪毛。它们可以更容易地避免被狼吃掉的危险。[441]诚然,它们避免这种危险,只是为了给人当食物。但是,自从它们一进入畜栏,它们的命运就已注定如此了。在谈论把它们从畜栏引出去以前,你们应当先把畜栏即社会砸毁。”(第 512—513 页)[X—441]\end{quote}

\tchapternonum{附录}

\tsectionnonum{[(1)霍布斯论劳动,论价值,论科学的经济作用]}

[XX—1291a]霍布斯认为技艺之母是\textbf{科学},而不是\textbf{实行者的劳动}:

\begin{quote}“对社会有意义的技艺,如修筑要塞、制造兵器和其他战争工具,是一种力量,因为它们有助于防卫和胜利;虽然它们的真正母亲是\textbf{科学,即数学},但由于它们是在工匠手里产生出来的,它们就被看成是工匠的产物,就象老百姓把助产婆叫做母亲一样。”(《利维坦》,载于《托马斯·霍布斯英文著作选》,摩耳斯沃思出版,1839—1844 年伦敦版第 3 卷第 75 页)\end{quote}

对脑力劳动的产物——科学——的估价,总是比它的价值低得多,因为再生产科学所必要的劳动时间,同最初生产科学所需要的劳动时间是无法相比的,例如学生在一小时内就能学会二项式定理。

\textbf{劳动能力}:

\begin{quote}“\textbf{人的价值},和其他一切物的价值一样,等于他的价格,就是说,等于\textbf{对他的能力的使用}所付的报酬。”(同上,第 76 页)“\textbf{人的劳动}〈因而人的劳动力的使用〉也是\textbf{商品},人们可以有利地交换它,就象交换其他任何\textbf{物品}一样。”(同上,第 233 页)\end{quote}

\textbf{生产劳动和非生产劳动}:

\begin{quote}“人仅仅为了维持自己的生活而\textbf{劳动}是不够的。他还应当在必要时为\textbf{保卫自己的劳动}而\textbf{战斗}。人们或者必须象犹太人被俘归来后重建神殿那样,一手建设,一手拿剑;或者要雇用别人来为他们战斗。”(同上,第 333 页)[XX—1291a]\end{quote}

\tsectionnonum{[(2)]历史方面:配第}

\vicetitle{[对于非生产职业的否定态度。劳动价值论的萌芽。在价值论的基础上说明工资、地租、土地价格和利息的尝试]}

[XXII—1346]\textbf{配第}《赋税论》1667 年伦敦版。

我们的朋友配第\endnote{关于配第的某些观点,马克思已经在前面《关于生产劳动和非生产劳动的理论》一章中,即在该章论及区分生产劳动和非生产劳动的最初尝试的那部分谈到过(见本册第 173—176 页)。——第 378 页。}的“人口论”,同马尔萨斯的完全不同。按照配第的意见,应该制止牧师的“繁殖”能力,让他们恢复独身生活。

这一切属于\textbf{生产劳动和非生产劳动}那一节\endnote{指马克思在手稿第 XVIII 本第 1140 页起草的计划所拟定的《资本论》第一部分最后一节即第九节(见本册第 446 页《资本论》第一部分的计划)。——第 378 页。}。

(a)\textbf{牧师}:

\begin{quote}“由于在英国男人比女人多……所以让牧师\textbf{恢复独身生活},换句话说,不让结了婚的人当牧师,是有好处的……这样一来,我们的\textbf{不结婚的牧师}就能够以他们现有俸禄的一半,来维持他们现在用全部俸禄所过的生活。”(第 7—8 页)\end{quote}

(b)\textbf{批发商和零售商}:

\begin{quote}“这些人很大一部分也可以削减。他们\textbf{本来就不配从社会得到什么},因为他们不过是一种\textbf{互相}[1347]\textbf{以贫民劳动为赌注的赌徒},他们自己什么也不生产,只是象静脉和动脉那样,把社会机体的血液和营养液,即工农业产品,\textbf{分配}到各方。”(第 10 页)\end{quote}

(c)\textbf{律师、医生、官吏等等}:

\begin{quote}“如果将有关\textbf{行政、司法}和\textbf{教会}方面的许多职务和费用削减,并且将那些为社会\textbf{工作极少}而所得\textbf{报酬极高}的牧师、律师、医生、批发商、零售商的人数削减,那末公共经费就会很容易地得到抵补。”(第 11 页)\end{quote}

(d)\textbf{贫民:“多余的人”}[supernumeraries]:

\begin{quote}“谁来供养这些人呢\fontbox{?}我的回答是:每一个人……在我看来,很明白,既不应该让他们饿死,也不应该将他们绞死,也不应该把他们送出国外”等等。(第 12 页)必须把“多余的东西”给他们,如果没有多余的东西,“如果\textbf{没有剩余}……则可把别人的丰美食物在数量或质量上\textbf{缩减一点}”。(第 12—13 页)什么劳动都可以让这些“多余的人”担负,只要这种劳动“无需耗用外国的商品”。主要的是,“使他们的精神养成守纪律和服从的习惯,使他们的肉体在必要时担当得了更加有利的劳动”。(第 13 页)“最好利用他们去筑路、架桥和开矿等等。”(第 12 页)\end{quote}

\textbf{人口——财富}:

\begin{quote}“\textbf{人口少是真正的贫穷};有 800 万人口的国家,要比领土面积相同而只有 400 万人口的国家富裕 1 倍以上。”(第 16 页)\end{quote}

关于上述(a)(\textbf{牧师})方面。配第对牧师进行了巧妙的讽刺:

\begin{quote}“牧师最守苦行的时候,宗教最繁荣,正如律师最清闲的时候……法律……最昌明一样。”(第 57 页)他在任何情况下都劝告牧师“\textbf{不要生出多于}现有\textbf{牧师俸禄}所能吸收的\textbf{牧师}”。例如,假定在英格兰和威尔士有 12000 份牧师俸禄。那末“生出 24000 个牧师,是不明智的”。因为,这样一来,12000 个无以为生的人就会同受俸牧师竞争,“他们要做到这一点,最容易的方法就是,向人们游说:那 12000 个受俸牧师在毒害人们的灵魂,使这些灵魂饿死〈这是暗指英国宗教战争〉,把他们引入歧途,使他们无法升入天国”。(第 57 页)\end{quote}

\textbf{剩余价值的起源和计算}。这个问题的叙述有些杂乱无章,但是,在苦苦思索寻求适当表达的过程中,分散在各处的中肯的见解就构成某种有联系的整体。

配第区分了“自然价格”、“政治价格”和“真正的市场价格”。(第 67 页)他所说的“\textbf{自然价格}”实际上是指\textbf{价值},这是我们在这里唯一感兴趣的东西,因为[1348]剩余价值的规定\textbf{取决于价值规定}。

配第在这本著作中,实际上用商品中包含的\textbf{劳动}的比较\textbf{量}来确定\textbf{商品的价值}。

\begin{quote}“但是在我们详细地论述\textbf{各种租金}之前,我们试图一方面联系\textbf{货币(它的租金叫做利息)},另一方面联系\textbf{土地和房屋}的租金,来说明租金的神秘性质。”(第 23 页)\end{quote}

首先要问,什么是商品的\textbf{价值},具体地说,什么是谷物的\textbf{价值}\fontbox{?}

\begin{quote}(α)“假定有人从秘鲁地下获得 1 盎斯银并带到伦敦来,他所用的时间和他生产 1 蒲式耳谷物所需要的\textbf{时间相等},那末,前者就是后者的自然价格;假定现在由于开采更富的新矿,获得 2 盎斯银象以前获得 1 盎斯银花费一样多,那末在其他条件相同的情况下,现在 1 蒲式耳谷物值 10 先令的价格,就和它以前值 5 先令的价格一样便宜。”(第 31 页)“假定生产 1 蒲式耳谷物和\textbf{生产 1 盎斯银}要用\textbf{相等的劳动}。”(第 66 页)这首先是“计算商品价格的真实的而不是想象的方法”。(第 66 页)\end{quote}

(β)现在要研究的第二点是\textbf{劳动的价值}。

\begin{quote}“法律……\textbf{应该使工人得到仅仅最必要的生活资料},因为,如果给工人双倍的生活资料,那末,工人做的工作,将只有他本来能做的并且在工资不加倍时实际所做的一半。\textbf{这对社会说来,就损失了同量劳动所创造的产品}。”(第 64 页)\end{quote}

可见,劳动的价值是由必要的生活资料决定的。工人之所以注定要生产剩余产品,提供剩余劳动,不过是因为人们强迫他用尽他全部可以利用的劳动力,以使他本人得到\textbf{仅仅最必要的生活资料}。因此,他的劳动的贵贱决定于两种情况:自然肥力和因气候影响而造成的费用(需要)大小:

\begin{quote}“自然的\textbf{贵贱}取决于\textbf{需要用多少人手来满足自然需要}(所以,谷物在 1 个人能为 10 个人生产的地方,比 1 个人只能为 6 个人生产的地方便宜);此外还取决于气候,因为它使人们或者多花费一些,或者少花费一些。”(第 67 页)\end{quote}

(γ)在配第看来,\textbf{剩余价值}只有两种形式:\textbf{土地的租金}和\textbf{货币的租金}(利息)。他是从前者推出后者的。前者,在他看来,正如后来在重农学派看来一样,是剩余价值的\textbf{真正的形式}。(可是配第同时声明,谷物应该指

\begin{quote}“一切生活必需品,就象主祷文中的‘面包’一词那样”。)\end{quote}

他在叙述中不仅把租金(剩余价值)说成是雇主抽取的超过必要劳动时间的余额,并且把它说成是生产者本人超出他的工资和他自己的资本的补偿额之上的剩余劳动的余额。

\begin{quote}“假定一个人用自己的双手在一块土地上种植谷物,就是说,他干了耕种这块土地所要干的活,如翻地或犁田、耙地、除草、收割、搬运回家、脱粒、簸扬等等,并且假定他有播种这块土地所需的\textbf{种子}。我认为,这个人\textbf{从他的收成中扣除自己的种子}〈就是说,第一,从产品中扣除了不变资本的等价物〉,[1349]并扣除自己食用的部分以及为换取衣服和其他必需品而给别人的部分之后,\textbf{剩下的谷物}就是\textbf{当年自然的和真正的地租};而 7 年的平均数,或者更确切地说,形成歉收和丰收循环周期的若干年的平均数,就是用谷物表示的这块土地的通常的地租。”(第 23—24 页)\end{quote}

可见,在配第看来,因为谷物的价值决定于它所包含的劳动时间,而地租等于总产品减去工资和种子,所以地租等于剩余劳动借以体现的剩余产品。这里,地租包括利润;利润还没有同地租分开。

接着,配第又以同样机智的方式问道:

\begin{quote}“但是,这里可能发生一个尽管是附带的、但需要进一步解决的问题:\textbf{这种谷物或这种地租值多少英国货币呢}\fontbox{?}我的回答是:值多少货币,要看另一个在同一时间内完全从事货币生产的人,除去自己全部费用之外还能剩下多少货币。也就是说,假定这个人前往产银地区,在那里采掘和提炼银,然后把它运到第一个人种植谷物的地方铸成银币,等等;并且假定这个人在他生产银的全部时间内,同时也谋得生活所必需的食物和衣服等等。这样,我认为\textbf{一个人的银和另一个人的谷物在价值上必定相等}。假定银是 20 盎斯,谷物是 20 蒲式耳,那末,1 盎斯银就是 1 蒲式耳谷物的价格。”(第 24 页)\end{quote}

此外,配第明确地指出,劳动种类的差别在这里是毫无意义的——一切只取决于\textbf{劳动时间}。

\begin{quote}“即使生产银比生产谷物可能需要更多的技术和冒更大的风险,但是结局总是一样的。假定让 100 个人\textbf{在 10 年内}生产谷物,又让\textbf{同样数目的人在同一时间内}开采银;我认为,银的\textbf{纯产量}将是\textbf{谷物全部纯收获量的价格},前者的同样部分就是后者的同样部分的价格。”(第 24 页)\end{quote}

配第在这样确定了\textbf{地租}(这里等于包括利润在内的全部\textbf{剩余价值})和找到了地租的货币表现之后,就着手确定\textbf{土地的货币价值},这又是很有天才的。

\begin{quote}“因此,如果我们能够发现可以自由买卖的土地的\textbf{自然价值},即使这种发现不见得比我们发现上述 ususfructus\authornote{ususfructus 指对别人财产(主要是地产)的使用权;这里是指土地的纯收入。——编者注}的价值好多少,我们也会觉得欣慰……在我们发现了\textbf{地租或一年 ususfructus 的价值}之后,产生了一个问题:一块可以自由买卖的土地的自然价值等于(用我们平常的说法)\textbf{多少年的年租}\fontbox{?}如果我们说年数无限,那就是说 1 英亩土地的价值等于 1000 英亩同样土地的价值(因为 1 的无限等于 1000 的无限),这是荒谬的。因此,我们必须选定一个\textbf{有限的}年\textbf{数}。我认为这个年数就是一个 50 岁的人、一个 28 岁的人、一个 7 岁的人可以同时生活的年数,也就是祖、父、子三代可以同时生活的年数。因为很少有人会为再下一代的子孙操心……所以我认为,\textbf{构成任何一块土地的自然价值的年租的年数},等于这三代人通常[1350]可以同时生活的年数。我们估计在英国这三代人可以同时生活 21 年,因而\textbf{土地的价值}也大约等于\textbf{21 年的年租}。”(第 25—26 页)\end{quote}

配第把地租归结为\textbf{剩余劳动},因而归结为\textbf{剩余价值}之后说道,土地不过是资本化的地租,即\textbf{一定年数的年租},或者说,一定年数的地租总额。

实际上,地租是\textbf{这样资本化}的,或者说,是\textbf{这样}作为\textbf{土地的价值}计算的:

假定 1 英亩土地每年带来 10 镑地租。如果利率等于 5\%,10 镑就代表 200 镑资本的利息,又因为利率是 5\%时利息在 20 年内就补偿了资本,所以,1 英亩土地的价值等于 200 镑(20×10 镑)。可见,地租的资本化取决于利率的高低。如果利率等于 10\%,10 镑就代表 100 镑资本或者说 10 年收入总额的利息。

但是,因为配第是从作为包括利润在内的剩余价值一般形式的\textbf{地租}出发的,所以他不能把资本的利息作为既定的东西,反而必须把利息当作地租的\textbf{特殊形式}从地租中推出来(杜尔哥也是这样做的,这从他的观点来看是完全合乎逻辑的)。但是,这里如何确定形成\textbf{土地的价值}的年数即年租的年数呢\fontbox{?}一个人[配第推论说]有兴趣购买的年租的年数,只是他要为自己和自己最近的后代“操心”的年数,就是说,只是一个\textbf{平均人}——祖、父、子三代——生活的年数。这个年数按“英国的”估计是 21 年。因此,21 年《ususfructus》以外的东西,对他毫无价值。因此他支付 21 年《ususfructus》的代价,而这也就形成\textbf{土地的价值}。

配第用这种巧妙的方式使自己摆脱了困难;但是,这里重要的是:

第一,\textbf{地租},作为全部\textbf{农业剩余价值}的表现,不是从土地,而是从劳动中引出来的,并且被说成劳动所创造的、超过劳动者维持生活所必需的东西的余额;

第二,\textbf{土地的价值}不外是预购的一定年数的地租,是地租本身的\textbf{转化}形式,在这种形式中,若干年(例如 21 年)的剩余价值(或剩余劳动)表现为\textbf{土地的价值};总之,\textbf{土地的价值}无非是\textbf{资本化的地租}。

配第如此深刻地看到问题的实质。因此,从地租购买者(即土地购买者)的观点来看,地租只表现为他用来购买地租的他的资本的利息,而在这种形式中,地租已经变得完全无法辨认,并且表现为资本利息了。

配第这样确定了\textbf{土地的价值}和\textbf{年租的价值}之后,就能够把“货币的租金”即利息当作派生的形式引出来了。

\begin{quote}“至于\textbf{利息},在保证没有问题的地方,它至少要同\textbf{贷出的货币所能买到的那么多土地的租金}相等。”(第 28 页)\end{quote}

这里,利息由\textbf{地租的价格}决定,而实际上相反,\textbf{地租的价格}即\textbf{土地的购买价值}是由利息决定的。但这样说是完全合乎逻辑的,因为配第把\textbf{地租}说成是剩余价值的一般形式,所以必然把\textbf{货币的利息}当作派生的形式从地租引出来。

\textbf{级差地租}。在配第的著作中,我们也看到关于级差地租的最初概念。他不是从同样大小地段的\textbf{不同}肥力引出级差地租,而是从\textbf{同等肥力的地段的不同位置}、从它们对市场的不同距离引出级差地租,大家知道,后者是级差地租的一个要素。他说:

\begin{quote}[1351]“正如对货币的需求大就会提高货币行市一样,对谷物的需求大也会提高\textbf{谷物的价格,从而提高种植谷物的土地的租金}\end{quote}

(可见,这里直接说出了谷物的\textbf{价格}决定地租,正如在前面的阐述中已经包含着地租不决定谷物\textbf{价值}的意思一样),

\begin{quote}\textbf{最后还提高土地本身的价格}。例如,如果供应伦敦或某一支军队的谷物必须从 40 英里远的地方运来,那末,在\textbf{离伦敦}或这支军队驻地\textbf{只有 1 英里的地方种植的谷物,它的自然价格还要加上}把谷物运输 39 英里的费用……由此产生的结果是,在靠近需要由广大地区供应粮食的人口稠密地方的土地,由于这个原因,比距离远而\textbf{土质相同的土地},不仅提供\textbf{更多的地租},并且所值的年租总额也更多”等等。(第 29 页)\end{quote}

配第也提到级差地租的第二个原因,即土地的\textbf{不同肥力},以及由此而来的同等面积的土地上劳动的\textbf{不同生产率}:

\begin{quote}“土地的好坏,或土地的价值,取决于人们\textbf{为利用土地而支付的产品的或大或小的部分对\CJKunderdot{生产上述产品所花费的简单劳动的比例}}。”(第 67 页)\end{quote}

由此可见,配第\textbf{比亚当·斯密更好地}阐明了级差地租。[XXII—1351)

\centerbox{※     ※     ※}

[XXII—1397]《\textbf{赋税论}》(1667 年版)\textbf{补录}。

(1)\textbf{关于}一国所必需的\textbf{流通货币}量,第 16—17 页。

配第对于\textbf{总生产}的看法,可以从下面的话里看出来:

\begin{quote}“如果某地有 1000 人,其中 100 人能够生产全体 1000 人所必需的食物和衣服;另外 200 人生产的商品和别国用来交换的商品或货币一样多,另外 400 人为全体居民的装饰、娱乐和华丽服务,如果还有 200 人是行政官吏、牧师、法官、医生、批发商和零售商,共计 900 人,那末就有一个问题”等等,接着讲了关于贫民(“多余的人”)的事。(第 12 页)\end{quote}

配第在阐述地租和地租的货币表现时——这里他以“\textbf{相等的劳动}”\authornote{见本册第 380 页。——编者注}(相等的劳动量)为基础,——说道:

\begin{quote}“我断定,这一点是\textbf{平衡和衡量各个价值的基础};但是在它的上层建筑和实际应用中,我承认情况是多种多样的和错综复杂的。”(第 25 页)\end{quote}

[1398](2)配第十分注意“\textbf{土地和劳动之间的自然的等同关系}”。

\begin{quote}“我们的银币和金币有各种不同的名称,例如,在英国叫做镑、先令和便士;所有这些铸币都可以用这三种名称中的任何一种来称呼,来理解。但是,关于这个问题,我要指出的是:一切东西都应由\textbf{两个自然单位——土地和劳动来评定价值};换句话,我们应该说:一条船或者一件衣服的价值等于若干土地的价值加上若干数量的劳动,因为船和衣服都是\textbf{土地和人类劳动的产物}。既然这样,我们就很想找出\textbf{土地和劳动之间的自然的等同关系},使我们单用土地或单用劳动来表现价值能够和同时用二者来表现价值一样好(甚至更好),而且能够象把便士折合为镑那样容易和可靠地将一个单位折合成另一个单位。”\end{quote}

因此,配第在找到地租的货币表现之后,又去找“可以自由买卖的土地的\textbf{自然价值}”(第 25 页)。

在配第那里有三种规定混在一起:

(a)由等量劳动时间决定的\textbf{价值量},在这里,\textbf{劳动}被看作\textbf{价值的源泉}。

(b)作为社会劳动的形式的\textbf{价值}。因此,货币表现为\textbf{价值的真正形式},虽然配第在其他地方抛弃了货币主义的一切幻想。所以在他的著作里形成\textbf{定义}。

(c)把作为交换价值的源泉的劳动和作为以自然物质(土地)为前提的使用价值的源泉的劳动混为一谈。实际上,当他建立劳动和土地之间的“等同关系”的时候,他把可以自由买卖的土地说成是\textbf{资本化的地租},因而这里他谈的,不是作为同实在劳动有关的自然物质的土地。

(3)关于\textbf{利率},配第说:

\begin{quote}“我在别处已经说到,制定违反\textbf{自然法}〈就是由资产阶级生产本性产生的法律〉的\textbf{成文民法}是徒劳无益的。”(第 29 页)\end{quote}

(4)关于\textbf{地租}:由于\textbf{劳动生产率较高}而产生的\textbf{剩余价值}:

\begin{quote}“如果上述那些郡,用比现在更多的劳动(如用翻地代替犁田,用点种代替散播,用选种代替任意取种,用浸种代替事先不作准备,用盐类代替腐草施肥等等)能够获得更大的丰产,那末,\textbf{增加的收入超过增加的劳动}越多,\textbf{地租}也上涨得越多。”(第 32 页)\end{quote}

(这里说的增加的劳动是指上涨了的“\textbf{劳动价格}”或\textbf{工资}。)

(5)\textbf{由国家提高货币的价值}(第 14 章)。

(6)前面引过的\authornote{见本册第 380 页。——编者注}一句话“如果给工人双倍的生活资料,那末,工人做的工作,将只有……一半”,应该这样理解:如果工人劳动 6 小时,得到他在这 6 小时内创造的价值,那他就得到他现在所得的二倍,——现在,他劳动 12 小时,只得到他在 6 小时内创造的价值。这样一来,他就会只劳动 6 小时了,“这对社会说来,就损失了”等等。

\centerbox{※     ※     ※}

\textbf{配第}《论人类的增殖》(1682 年)。《分工》(第 35—36 页)。

\centerbox{※     ※     ※}

\textbf{配第}《爱尔兰政治剖视》(1672 年)和《献给英明人士》(1691 年伦敦版)。

\begin{quote}(1)“这使我遇到\textbf{政治经济学}上最重要的一个问题,就是,怎样建立土地和劳动之间的\textbf{等同关系和等式},以便用这两个因素之一来表示任何东西的价值。”(第 63—64 页)\end{quote}

实际上,作为提出这一问题的基础的只是把\textbf{土地}本身的\textbf{价值}归结为\textbf{劳动}。

[1399](2)这部著作比前面考察过的著作\endnote{指 1662 年第一次发表的《赋税论》。——第 388 页。}写得晚。

\begin{quote}“\textbf{价值的一般尺度},是平均\textbf{一个成年男人的一天食物},而\textbf{不是他的一天劳动};这个尺度同纯银的价值一样有规则,一样稳定……因此,我认为一所爱尔兰茅屋的\textbf{价值,是用建筑茅屋的人在建筑时消费了多少天的食物来确定的}。”(第 65 页)\end{quote}

最后这一段,完全是重农主义的调子。

\begin{quote}“有些人吃得比别人多,这是无关紧要的,因为我们所说的一天食物,是指 100 个各种各样的、体格不同的人为生活、劳动和繁殖所吃的食物的 1\%。”(第 64 页)\end{quote}

但是,这里配第在爱尔兰\textbf{统计}中所找的,不是价值的“\textbf{一般}尺度”,而是\textbf{货币}是价值尺度这个意义上的\textbf{价值}尺度。

(3)\textbf{货币数量和国家财富}(《献给英明人士》第 13 页)。

(4)\textbf{资本}:

\begin{quote}“我们称为\textbf{一个国家的财富、资本或储备}并且是\textbf{以前或过去劳动}的成果的东西,不应看成\textbf{同现在发挥作用的力量有区别的}东西。”(第 9 页)\end{quote}

(5)\textbf{劳动的生产力}:

\begin{quote}“我们说过,人口的一半,花费不多的劳动就可以使王国大大富足……这些人应该把劳动用于生产什么呢\fontbox{?}对于这个问题,我的一般回答是:\textbf{由少数人}为全国居民生产食物和必需品;或者靠\textbf{更紧张的劳动},或者靠\textbf{采用节省和减轻劳动的手段};这样得到的结果,和人们徒然希望从\textbf{一夫多妻制}得到的结果相同。因为一个人能做五个人的工作,得到的结果就好比他生了四个成年劳动者。”(第 22 页)“当\textbf{生产}食物\textbf{使用的人手比任何别的地方都少}的时候……\textbf{食物将最便宜}。”(第 23 页)\end{quote}

(6)人的目的和目标(第 24 页)。

(7)关于\textbf{货币},也可参考《货币略论》(1682 年)。[XXII—1399]

\tsectionnonum{[(3)]配第、达德利·诺思爵士、洛克}

[XXII—1397]把诺思和洛克的著作同配第的《货币略论》(1682 年)、《赋税论》(1662 年)以及《爱尔兰政治剖视》(1672 年)比较一下,就可以看出,在(1)关于\textbf{利率降低}的问题,(2)关于国家提高和降低货币价值的问题,(3)\textbf{诺思}的把利息称为“货币的租金”等等问题上,诺思和洛克两人都是追随配第的。

\textbf{诺思}和\textbf{洛克}就同一个问题即\textbf{利率降低}和\textbf{国家提高货币价值}的问题,同时写了他们的著作\endnote{指诺思的《贸易论》和洛克的《论降低利息和提高货币价值的后果》。这两本著作都写于 1691 年,在伦敦出版:前者于 1691 年出版,后者于 1692 年出版。——第 389 页。}。但是他们阐明的观点是完全对立的。洛克认为\textbf{缺乏货币}是高利率的原因,一般说来是货物不能按它们的实际价格出卖并带来应有的收入的原因。相反,诺思指出,原因不是流通中缺乏货币,而是缺乏资本或收入。在他的著作中,第一次出现关于 stock\authornote{储备、基金、资金。——编者注}即\textbf{资本}的明确概念,或者更确切地说,关于不作为流通手段只作为\textbf{资本的形式的货币}的明确概念。在\textbf{达德利·诺思}爵士那里,我们看见同洛克的观点对立的关于利息的第一个正确的概念。[XXII—1397]

\tsectionnonum{[(4)]洛克}

\vicetitle{[从资产阶级自然法理论观点来解释地租和利息]}

[XX—1291a]如果我们把洛克关于劳动的一般观点同他关于\textbf{利息}和\textbf{地租的起源}的观点(因为在洛克那里,剩余价值只表现为利息和地租这两种特定形式)对照一下,那末,剩余价值无非是土地和资本这些劳动条件使它们的所有者能够去占有的\textbf{别人劳动},剩余劳动。在洛克看来,如果劳动条件的数量大于一个人用自己的劳动所能利用的数量,那末,对这些劳动条件的所有权,就是一种同私有制的自然法基础相矛盾的[1292a]\textbf{政治}发明。

\fontbox{~\{}在\textbf{霍布斯}那里,除了处于直接可供消费状态的自然赐予之外,劳动也是一切财富的唯一源泉。上帝(自然)

\begin{quote}“或者把必要的东西\textbf{无代价地赐给}人类,或者\textbf{要劳动作交换卖给}人类”。(《利维坦》[第 232 页])\end{quote}

但是,在霍布斯那里,土地所有权由君主随意分配。\fontbox{\}~}

下面是洛克著作中与此有关的几段话:

\begin{quote}“虽然\textbf{土地}和一切低等生物是一切人所共有的,但是每一个人仍然有一个所有物,就是他自己的人身,对于这个所有物,除了他自己以外,别人是没有任何权利的。他的身体的劳动和他的双手的创作,我们可以说,是理应属于他的。他把他从自然创造并提供给他的东西中取得的一切,同自己的劳动溶合起来,同一种属于他的东西溶合起来;他以这种方式使这一切成为自己所有。”(《论政府的两篇论文》第 2 篇第 5 章;《约翰·洛克著作集》1768 年第 7 版第 2 卷第 229 页)“这一切在自然手里,都是公共所有,都一视同仁地属于自然的全体子女;人的劳动把这一切从自然手里拿过来,从而把它们据为己有。”(同上,第 230 页)“以这种方式给予我们所有权的这一自然法,同时也限制了这个所有权的范围……一个人在对他的生活有某种用处的东西损坏之前能够使用它多少,他用自己的劳动可以使它变为自己所有的也就多少;超出这个限度的,就是超过他的份额而属于别人的东西。”(同上)“但是,现在所有权的主要对象不是\textbf{土地的果实}等等,而是\textbf{土地}本身……一个人能够耕作、播种、施肥和种植多大的土地,能够享用多大土地的产品,多大的土地就是他的所有物。人就好比是用自己的劳动把它从公共财产中圈出来。”(同上)“我们看到,开垦或耕作土地和占有土地,是互相连结在一起的。前者为后者提供了权利。”(同上,第 231 页)“自然已经\textbf{按照人的劳动}以及人的生活方便\textbf{所能达到的程度},正确地确定了所有权的尺度:谁都不可能用自己的劳动征服或占有一切;谁都不可能为了满足自己的需要而消费比这一小部分更多的东西;因此谁都不可能用这种形式侵犯别人的权利,或者为自己取得所有权而损害邻人的利益……早先,这个尺度使每个人的占有限于非常小的一份,限于他自己能够占有而不损害别人利益的范围……\textbf{就是现在},尽管全世界似乎挤满了人,\textbf{仍然可以承认}同一尺度而不损害任何人的利益。”(第 231—232 页)\end{quote}

劳动几乎提供了一切东西的全部价值\fontbox{~\{}在洛克那里,价值等于使用价值,劳动是指具体劳动,不是指劳动的量;但是,交换价值以劳动为尺度,实际上是以劳动者创造使用价值为基础的\fontbox{\}~}。不能归结为劳动的使用价值余额,在洛克看来,是自然的赐予,因而,就它本身来说是\textbf{公共所有物}。因此,洛克想要证明的,不是除劳动之外还可以通过其他办法获得所有权这个同他原来的观点相矛盾的论题,而是怎样才能通过个人劳动创造个人所有权,尽管自然是公共所有物。

\begin{quote}“实际上正是\textbf{劳动决定一切东西的价值的差别}……对人的生活有用的土地产品……有 99%完全要记在劳动的账上。”(第 234 页)“因此,劳动决定土地价值的最大部分。”(第 235 页)“虽然自然的一切东西是给予一切人共同所有的,可是,人是\textbf{他自己的主人},是\textbf{他自身}及其活动或劳动的\textbf{占有者},作为这样的一个人,他本身已经包含着所有权的重大基础。”(第 235 页)\end{quote}

所以,所有权的一个界限是\textbf{个人劳动的界限};另一个界限是,一个人储存的东西不多于他能够使用的东西。后一个界限由于把容易损坏的产品同\textbf{货币}交换(撇开别种交换不说)而扩大了:

\begin{quote}“这种\textbf{耐久的}东西,一个人愿意储存多少,就可以储存多少;所谓超出他的正当所有权的界限\fontbox{~\{}撇开他个人劳动的\textbf{界限}不谈\fontbox{\}~},不是指他有很多东西,而是指其中一部分东西损坏了,对他没有用了。于是货币这种耐久的东西就被采用,人们可以把它储存起来而不致于损坏,并根据相互协议,把它用于[1293a]交换真正有用但容易损坏的生存资料。”(第 236 页)\end{quote}

这样就产生了个人所有权的不均等,但是\textbf{个人劳动}这一\textbf{尺度}仍然有效。

\begin{quote}“人们之所以可能超过社会确定的界限,不经协议,把财物分成不均等的私人财产,只是因为他们使金银具有了价值,默认货币的使用。”(第 237 页)\end{quote}

应该把这段话同洛克关于利息的著作\endnote{这一著作是《论降低利息和提高货币价值的后果》(1691 年)。——第 392 页。}中的下面一段话加以对比,不要忘记,照他看来,自然法使\textbf{个人劳动}成为所有权的界限:

\begin{quote}“现在我们就来考察一下,货币怎么会具有同土地一样的性质,提供我们称作利钱或利息的一定年收入。因为土地自然地生产某种新的、有用的和对人类有价值的东西;相反,货币是不结果实的,它不会生产任何东西,但是,它通过相互协议,\textbf{把作为一个人的劳动报酬的利润转入另一个人的口袋}。这种情况是由货币分配的不均等引起的;这种不均等对土地产生的影响,同它对货币产生的影响一样……土地分配的这种不均等(你的土地多于你能够耕种或愿意耕种的,而另一个人的土地却少于他能够耕种或愿意耕种的)会为你招来一个租种你的土地的佃户;而货币分配的这种不均等……会为我招来一个借用我的货币的债户;这样一来,\textbf{我的货币靠债务人的勤劳,能够}在他的营业中为他带来多于 6%的收入,正如你的土地\textbf{靠佃户的劳动}能够生产一个大于他的地租的收益。”(《约翰·洛克著作集》1740 年对开本版第 2 卷[第 19 页])\endnote{马克思这里引用的洛克的话,取自马西的著作《论决定自然利息率的原因》第 10—11 页的引文。在 1768 年出版的洛克著作集中,这段话在第 2 卷第 24 页。——第 393 页。}\end{quote}

在洛克的这段话里,部分含有论战的意思,他想向土地所有者指出,他们的地租同高利贷者所取得的利息完全没有区别。地租和利息都由于生产条件的不均等的分配而“把作为一个人的劳动报酬的利润转入另一个人的口袋”。

因为洛克是同封建社会相对立的资产阶级社会的法权观念的经典表达者;此外,洛克哲学成了以后整个英国政治经济学的一切观念的基础,所以他的观点就更加重要。[XX—1293a]

\tsectionnonum{[(5)]诺思}

\vicetitle{[作为资本的货币。商业的发展是利率下降的原因]}

[XXIII—1418]\textbf{达德利·诺思爵士}《贸易论》1691 年伦敦版(补充本 C)\endnote{马克思指他的 1861—1863 年手稿的“补充本”(Beihefte)之一,他在 1863 年 5 月 29 日曾写信给恩格斯说,1863 年春他在这些补充本中“摘录了同我已写好的部分有关的政治经济学史方面的各种材料”。现在有补充本 A、B、C、D、E、F、G、H。诺思著作的摘要在补充本 C 第 12—14 页。——第 394 页。}。

这部著作同洛克的经济学著作完全一样,也和配第的著作直接有关,并直接以配第的著作为依据。

诺思的著作主要是研究\textbf{商业资本},就这点来说,和这里要讨论的东西无关。诺思在他所研究的问题的范围内,表现了行家的谙练。

非常值得注意的是,从查理二世复辟时期到十八世纪中叶,地主对地租下降不断发出怨言(因为小麦价格特别从\fontbox{?}年\endnote{手稿这里本来写了“从 1688 年起”,但后来 1688 这个数字被勾掉,而代之以问号。马克思在他的 1861—1863 年手稿第 XI 本第 507—508 页引用了 1641 年起小麦价格变动的资料。1641—1649 年小麦的平均价格是每夸特 60 先令 5+(2/3)便士,十七世纪下半叶平均价格降到每夸特 44 先令 2+(1/5)便士,十八世纪上半叶降到每夸特 35 先令 9+(29/50)便士。——第 394 页。}起,不断下降)。虽然(从卡耳佩珀和约瑟亚·柴尔德爵士以来)工业资本家阶级积极参与强制压低利率,但是这个措施的真正鼓吹者是\textbf{土地所有者}。“\textbf{土地的价值}”和“提高这个价值的方法”作为国民利益被提到首位。(正如,相反的,大约从 1760 年起,地租、土地价值、谷物和其他食物的价格的上涨以及工业资本家对此发出的怨言,成为经济学上对这个对象进行研究的基础一样。)

从 1650 年到 1750 年的整个世纪,除了少数例外,不断发生货币所有者和土地所有者之间的斗争,因为生活阔绰的贵族,看到高利贷者把他们抓在手里,又看到自从十七世纪末建立了现代信用制度和国债制度以后,高利贷者在立法等方面占了他们的上风,心中十分不快。

\textbf{配第}已经谈到地主对地租下降发出怨言和他们反对农业改良的事情(参看有关的段落)。\endnote{马克思显然是指他在《剩余价值理论》第二册论洛贝尔图斯一章(手稿第 494 页)所引用的配第《政治算术》(1676 年)第四章的一段话。参看《资本论》第 3 卷第 39 章:“在配第和戴韦南特时期,农民和土地所有者对改良和开发发出怨言;较好土地的地租下降……”——第 395 页。}他维护高利贷者而反对地主,他把货币的租金同土地的租金相提并论。

\textbf{洛克}把这两者都归结为对劳动的剥削。他同配第站在同一立场。他们两人都反对对利息作强制性的调整。土地所有者注意到,如果利息下降,\textbf{土地的价值}就上升。在地租量已定时,地租的\textbf{资本化的表现}即土地的价值,其升降就同利率的高低成反比。

\textbf{达德利·诺思爵士}在上述著作中是配第路线的第三个代表。

这是\textbf{资本}起来反抗\textbf{土地所有权}的最初形式。因为实际上,\textbf{高利贷},即土地所有者的一部分收入转入高利贷者手中,是资本积累的主要手段之一。但是,工业资本和商业资本或多或少同土地所有者携手共同反对资本的这种古旧的形式。

\begin{quote}“正如土地所有者出租他的土地一样,这些人(他们有\textbf{资本}[stock]可用于商业,可是由于没有必要的才干或由于怕辛苦而没有用于商业)就出借他们的\textbf{资本}。他们从中得到的东西叫做\textbf{利息},但是,利息不过是资本的\textbf{租金}\end{quote}

\fontbox{~\{}在这里如同在配第那里一样,我们看到,对于刚从中世纪走出来的人,租金[1419]表现为剩余价值的原始形式\fontbox{\}~},

\begin{quote}就象土地所有者的收入是土地的租金一样。在几种语言中,借货币和租土地是通用的说法:英国有几个郡的情况也是如此。因此,当一个\textbf{地主}[landlord]或当一个\textbf{财主}[stocklord]是一回事;地主有利的地方,只在于他的佃户不能把土地带走,而资本的债户却很容易把资本带走;因此,土地提供的\textbf{利润}应当比冒更大风险借出的资本提供的利润\textbf{少}。”(第 4 页)\end{quote}

\textbf{利息}。诺思看来是第一个正确理解利息的人,因为,从后面引用的几段话可以看到,stock 这个词他不仅指货币,并且指资本(同样,配第也把 stock 和\textbf{货币}区别开来。在洛克那里,利息完全决定于流通中的货币量,在配第那里也是这样。\textbf{参看马西关于这个问题的论述})。

\begin{quote}“如果放债人多于借债人,利息将下降……不是低利息使商业活跃,而是\textbf{在商业发展时国民资本}使利息下降。”(第 4 页)“金、银和用金银铸造的货币无非是重量和尺度,有它们比没有它们更便于进行交易;此外,它们又是适于\textbf{存放多余资本}的基金。”(第 16 页)\end{quote}

\textbf{价格和货币}。因为价格无非是用\textbf{货币}表示并且由货币实现的(在我们说的是\textbf{卖}的情况下)商品的\textbf{等价物},也就是说,因为\textbf{商品}先在价格中表现为\textbf{交换价值},以便以后再转化为使用价值,所以,在经济思想方面迈出的最初的步伐之一,就是认为金银在这里只作为商品本身的\textbf{交换价值的存在形式},作为\textbf{商品形态变化的一个因素}出现,而不作为金银本身出现。就诺思那个时代来说,诺思把这一点说得很巧妙:

\begin{quote}“因为货币……是买和卖的普遍的尺度,所以每一个要卖东西而找不到买者的人,总以为他的商品卖不出去是因为\textbf{国内缺乏货币};因此到处都叫嚷缺乏货币。然而这是一个大错误……那些叫嚷缺乏货币的人究竟要什么呢\fontbox{?}我从\textbf{乞丐}说起……他要的不是货币,而是面包和其他生活必需品……租地农民抱怨缺乏货币……他以为,如果国内有较多的货币,他的货物就可以卖到好价钱。看来,他要的不是货币,而是他想卖但又卖不出去的谷物和牲畜的好价钱……为什么他卖不到好价钱呢\fontbox{?}……(1)或者是因为国内谷物和牲畜太多,到市场上来的人大多数都象他那样要卖,但只有少数人要买。(2)或者是因为通常的出口停滞,例如在战时,贸易不安全或不准进行。(3)或者是因为消费缩减,例如,人们由于贫困,不能再花费过去那样多的生活费用。可见,有助于租地农民出售货物的,不是增加货币,而是消除这三个真正造成市场缩减的原因中的任何一个原因。批发商和零售商也同样要货币,就是说,因为市场停滞,他们要把他们经营的货物销售出去。”(第 11—12 页)\endnote{这段引文(从“我从乞丐说起”开始),据马克思在 1861—1863 年手稿第 XXIII 本第 1419 页注明,系取自补充本 C 第 12—13 页。——第 397 页。}\end{quote}

其次,\textbf{资本}是\textbf{自行增殖的价值},而\textbf{货币贮藏}却以\textbf{交换价值的结晶形式}本身为目的。因此,古典政治经济学最早的发现之一,是它认识了\textbf{货币贮藏}和\textbf{货币自行增殖}之间的对立,也就是说,它论述了\textbf{作为资本的货币}。

\begin{quote}“谁也不会因为用货币、金银器等形式把自己的全部财产留在身边而变富,相反,倒会因此而变穷。只有财产\textbf{正在增长}的人才是最富的人,不管他的财产是租出去的土地,还是放出去生息的货币,还是投入商业的货物。”(第 11 页)\end{quote}

\fontbox{~\{}约翰·贝勒斯在他所著《论贫民、工业、贸易、殖民地和道德堕落》一书(1699 年伦敦版)中谈到这个问题时说道:

\begin{quote}“货币只有放出去才能\textbf{增殖},才有用处;正如一个私人除非用货币去换取某种更有价值的东西,否则货币对他就无利可图,同样,超过国内贸易绝对需要的全部货币量,对于一个王国或一个民族来说,是死资本,它不会给让货币停滞不动的国家带来任何利润。”(第 13 页)\fontbox{\}~}“虽然每一个人都愿意有它〈货币〉,可是没有一个人或者只有很少的人愿意把它保存起来,大家都力求把它立刻花出去;因为大家都知道,从一切放着不用的货币中,不能得到任何利润,只会受到损失。”(\textbf{诺思},同上第 21 页)\end{quote}

[1420]\textbf{作为世界货币的货币}。

\begin{quote}“从商业来说,一个国家在世界上的地位,无论哪一方面都同一个城市在一个王国中的地位,或者一个家庭在一个城市中的地位一样。”(第 14 页)“在这种商业交往中,金银同其他商品毫无区别,人们从金银过多的人手里把金银拿来,转交给缺少金银或需要金银的人。”(第 13 页)\end{quote}

\textbf{能够流通的货币量决定于商品交换}。

\begin{quote}“不论从国外带进多少货币,或者在国内铸造多少货币,凡是超过一国商业的需要的,\textbf{都只是金银条块},并且只有拿它当作金银条块对待;而且铸造的货币就象旧金银器一样,只有按它所包含的金属成色出卖。”(第 17—18 页)\end{quote}

货币变成金银条块和相反的情况(第 18 页)(补充本 C,第 13 页)。货币的\textbf{估价}和\textbf{衡量}。上下波动(补充本 C,第 14 页)\endnote{在补充本 C 第 14 页有诺思著作的摘录,诺思谈到一国货币流通中的“涨落”。马克思在《资本论》第一卷第三章注 95 中引用了其中一部分摘要。——第 398 页。}。

\textbf{高利贷、土地所有者和商业}:

\begin{quote}“在我国,取息的货币,\textbf{放给商人}去经营业务的还不到 1/10;大部分是借给这样一些人去维持奢侈生活和其他开销的,这些人虽然是大地产的所有者,但是,他们花费收入比他们的地产带来收入快,他们不愿出卖自己的财产,宁愿拿财产去抵押。”(\textbf{诺思},同上第 6—7 页)[XXIII—1420]\end{quote}

\tsectionnonum{[(6)贝克莱论勤劳是财富的源泉]}

[XIII—670a]“难道认为\textbf{土地本身}就是\textbf{财富}不是错误的吗\fontbox{?}难道我们不应当首先把人民的勤劳看成这样一种东西,它形成财富,甚至使那些除了作为勤劳的\textbf{手段和刺激}以外便毫无价值的土地和白银变成财富\fontbox{?}”(\textbf{乔·贝克莱}博士《提问者》1750 年伦敦版。第 38 个问题)[XIII—670a]

\tsectionnonum{[(7)]休谟和马西}

\tsubsectionnonum{[(a)马西和休谟著作中的利息问题]}

[XX—1293a]马西的匿名著作《论决定自然利息率的原因》于 1750 年出版;休谟的《论丛》第二卷,其中有《论利息》,于 1752 年出版,比前书迟了两年。因此马西在先。休谟反对洛克,而马西反对配第和洛克,配第和洛克两人还抱着这样的观点,即认为利息率的高低取决于流通中的货币量,认为真正被拿来贷放的东西实际上是货币(而不是资本)。

马西比休谟更坚决地宣称,\textbf{利息}只不过是利润的一部分。休谟主要证明货币的价值对利息率的高低没有意义,因为在利息和货币资本之间的比率已知(譬如说 6\%)的情况下,6 镑的价值同 100 镑(也可以说 1 镑)的价值一起升降,但并不影响用 6 这个数字表示的比率。

\tsubsectionnonum{[(b)休谟。由于商业和工业增长而引起的利润和利息的降低]}

我们从休谟谈起。

\begin{quote}“世上一切都是用劳动购买的。”(《论丛》第 1 卷第 2 部分,1764 年伦敦版[《论商业》]第 289 页)\end{quote}

在休谟看来,利息率的高低取决于借债人的需求和放债人的供给,即取决于供求。但是后来,它本质上取决于

\begin{quote}“从商业中产生出来的利润”的高低。(同上[《论利息》],第 329 页)“劳动储备和商品储备的多少,对于利息必定有重大影响,因为我们出利息借货币,借的实际上就是劳动和商品。”(同上,第 337 页)“在可以得到高利息的地方,没有人会以低利润为满足,而在可以得到高利润的地方,也没有人会以低利息为满足。”(同上,第 335 页)\end{quote}

高利息和高利润这两者是

\begin{quote}“商业和工业不够发达”的表现,“而不是缺乏金银”的表现,“低利息则表明相反的情况”。(同上,第 329 页)[1294a]“因此,在一个只有土地所有者〈或者象休谟后来说的,“地主和农民”〉的国家,借债人必定多,利息必定高”(第 330 页),\end{quote}

因为代表只供享用的财富的人出于无聊,追求享乐,而另一方面,除了农业以外,生产非常有限。一旦商业发展起来,情况就相反。商人完全被获利的欲望支配。他

\begin{quote}“\textbf{除了看到他的财产一天天增加以外,不知道还有什么更大的享乐}”。\end{quote}

(在这里,对交换价值、对抽象财富的追求大大超过对使用价值的追求。)

\begin{quote}“这就是为什么商业扩大节约,为什么在商人中守财奴大大超过挥霍者,而在土地所有者中情况则相反的原因。”(第 333 页)\end{quote}

\fontbox{~\{}\textbf{非生产劳动}:

\begin{quote}“律师和医生不产生任何生产活动;而且他们的财富是靠牺牲别人得来的;这样,他们使自己的财富增加多少,就一定使某些同胞的财富减少多少。”(第 333—334 页)\fontbox{\}~}“因而,商业的增长造成放债人数目的增加,因此\textbf{引起利息率的降低}。”(第 334 页)“\textbf{低利息}和\textbf{商业}中的\textbf{低利润},是彼此互相促进的两件事,\textbf{两者都来源于}商业的扩展,商业的扩展产生富商,使货币所有者增加。商人有了大笔的资本,不管这些资本是由少量的铸币还是由大量的铸币代表,都必然要常常发生这种情况:当他们倦于经商,或者他们的后代不喜欢或没有才干经商的时候,有很大一部分资本就自然地寻求一个常年的可靠的收入。供应多了就使价格降低,使放债人接受低利息。这种考虑迫使许多人宁愿把他们的资本留在商业中,满足于低利润,而不愿把他们的货币按更低的利息贷放出去。另一方面,当商业有了很大的扩展并且运用大量资本的时候,必然\textbf{产生商人之间的竞争},这种竞争使\textbf{商业利润减少},同时也使商业本身规模扩大。商业中的利润降低,使商人宁肯在离开商业,开始过清闲日子时接受低利息。因此,研究\textbf{低利息}和\textbf{低利润}这两种情况中,究竟哪一个是\textbf{原因},哪一个是\textbf{结果},是\textbf{没有用处}的。两者都是从大大扩展了的商业中产生的,并且彼此促进……大大扩展了的商业产生大量资本,因此,它既降低利息又降低利润;每当它降低利息的时候,总有利润的相应降低来促进它,反之也是一样。我可以补充说一句,正如\textbf{商业和工业的增长}引起低利润一样,低利润反过来又促使商业进一步增长,因为低利润使商品便宜,鼓励消费,促进工业的发展。由此可见……\textbf{利息}是\textbf{国家状况的真正的晴雨表,低利息率}是人民兴旺的几乎屡试不爽的标志。”(同上,第 334—336 页)\end{quote}

\tsubsectionnonum{[(c)马西。利息是利润的一部分。用利润率说明利息的高低]}

[\textbf{约·马西}]《论决定自然利息率的原因。对威廉·配第爵士和洛克先生关于这个问题的见解的考察》1750 年伦敦版。

\begin{quote}“从这些引文\endnote{在这段话前面,马西引用了配第的《政治算术》和洛克的《论降低利息和提高货币价值的后果》两书的摘要。——第 402 页。}中可以看到,洛克先生认为,自然\textbf{利息率}一方面决定于一国货币量同一国居民相互间的债务之比,另一方面决定于一国货币量同一国商业之比,威廉·配第爵士则认为,自然利息率只决定于一国货币量,因此,他们只在债务这一点上有不同意见。”(第 14—15 页)[XX—1294a][XXI—1300]富人“不是自己使用自己的货币,而是把自己的货币借给别人去营利,让别人把这样\textbf{得来的利润拿出一部分}交给货币所有者。但是,如果一国的财富平均分配给许多人,以致国内很少有人能够靠把货币投入商业的办法来供养两个家庭,那末,就\textbf{只能有很少的货币借贷了}:如果 2000 镑属于一个人,它就会被贷出,因为它带来的利息足以供养一个家庭;如果 2000 镑属于 10 个人,它就不会被贷出,因为它的利息不能供养 10 个家庭”。(第 23—24 页)“\textbf{根据政府为所借货币支付的利息率}来推断自然利息率的任何尝试,都是必然要失败的。经验表明,这两种利息率彼此既不一致,又不保持一定的关系;理性告诉我们,它们决不可能是这样,因为\textbf{自然利息率是以利润为基础,而国债的利息率是以需要为基础},利润有界限,而需要没有界限。借货币去改良自己土地的贵族,借货币去经营企业的商人或工业家,都有他们不能超越的一定界限:如果他们用借来的货币能赚得 10\%的利润,他们可以为所借货币付给放债人 5\%;但是他们不会付给 10\%;相反,如果谁由于有迫切需要而借债,那就一切只取决于他的需要的程度,而需要是不承认任何戒律的。”(第 31—32 页)“收取利息的合理性,不取决于借债人是否赚得\textbf{利润},而取决于这些货币如果正确地加以使用,能够带来利润。”(第 49 页)“既然\textbf{借债人}为所借货币支付的利息,是\CJKunderdot{\textbf{所借货币能够带来的利润的一部分}},那末,这个\textbf{利息}总是要由这个\textbf{利润}决定。”(第 49 页)“在这个利润中,多大一部分归借债人,多大一部分归放债人才算合理呢\fontbox{?}这一般地只有根据借贷双方的意见来决定。因为在这方面合理不合理,仅仅是大家同意的结果。”(第 49 页)“可是,这一条\textbf{利润分配}规则,并不是对每一个放债人和借债人都适用,而只是对放债人和借债人总的来说适用……特大的利润和特小的利润是对业务熟练和业务不熟练的报酬,这是\textbf{同放债人绝无关系的};因为他们既不会因业务不熟练而吃亏,也不会因业务熟练而得利。在这里,适用于\textbf{同一工商业部门各个人}的话,也适用于\textbf{各个不同的工商业部门}。”(第 50 页)“\textbf{自然利息率}是由\textbf{工商业企业的利润}决定的。”(第 51 页)\end{quote}

为什么英国现在的利息率是 4\%,而过去是 8\%\fontbox{?}因为那时候英国商人

\begin{quote}“赚得的利润比现在多一倍”。\end{quote}

为什么利息率在荷兰是 3\%,在法国、德国和葡萄牙是 5—6\%,在西印度和东印度是 9\%,在土耳其是 12\%\fontbox{?}

\begin{quote}“对于所有这些情况,只要总的答复一下就够了,就是说,这些国家的商业利润和我国的商业利润不同,并且如此不同,以致产生了上述各种不同的利息率。”(第 51 页)\end{quote}

但是,为什么利润会下降呢\fontbox{?}那是由于国外和国内的竞争:

\begin{quote}“由于对外贸易〈因国外竞争〉减少,或者由于\textbf{商人彼此竞相压低自己商品的价格}……因为他们有必要把东西卖掉,或者因为他们利欲熏心想尽量多卖一些”。(第 52—53 页)“商业利润一般决定于\textbf{商人数目}同\textbf{商业规模}之比。”(第 55 页)在荷兰,“从事商业的人数在人口总数中占的比例最大……\textbf{利息最低}”;在土耳其,这种比例最小,利息最高。(第 55—56 页)[1301]“\textbf{商业规模同商人数目之比}是由什么决定的呢\fontbox{?}”(第 57 页)“由商业的动机决定”:由自然的必要性、自由、私人权利的保护、社会安全来决定。(第 58 页)“没有两个国家能够\textbf{以等量的劳动耗费},同样丰富地提供数目相等的必要生活资料。人的需要的增减取决于人生活在其中的气候的严寒或温暖;所以不同国家的居民必须经营的\textbf{商业的规模}不能不有所差别,只有根据冷热的程度才能知道这种差别的程度。由此可以得出一个一般的结论:维持一定人口生活所需要的\textbf{劳动量},在气候寒冷的地方最大,在气候炎热的地方最小,因为在寒冷的地方,不仅人需要较多的衣服,而且土地也必须耕作得更好。”(第 59 页)“荷兰具有发展商业的特殊必要性……这种必要性是由国内人口过剩引起的;这种情况,再加上\textbf{必须花费很多劳动去筑堤和排水},就使荷兰经营商业的必要性比世界上其他任何可以居住的地方都大。”(第 60 页)\end{quote}

\tsubsectionnonum{[(d)结束语]}

马西比休谟更加明确地说明利息只不过是\textbf{利润的一部分};他们两人都用资本积累(马西特别讲到竞争)和由此产生的利润下降,来说明利息的下降。两人同样很少谈到“\textbf{商业利润}”本身的\textbf{源泉}问题。[XXI—1301]

\tsectionnonum{[(8)对论重农学派的各章的补充]}

\tsubsectionnonum{[(a)对《经济表》的补充意见。魁奈的错误前提]}

[XXIII—1433]

生产阶级

这是《经济表》的最简单的形式。\endnote{马克思在这虽引用(并略加简化)的《经济表》是魁奈在《经济表的分析》(德尔出版的《重农学派》第 1 部第 65 页)中用的那种《经济表》图式。——第 405 页。}

(1)\textbf{货币流通}(假定每年只支付一次)。货币流通的出发点是花钱的阶级,即土地所有者阶级,他们没有任何\textbf{商品}要卖,他们只买不卖。

土地所有者用 10 亿向生产阶级购买,把生产阶级用来付地租的 10 亿货币还给生产阶级。(从而实现了农产品的 1/5。)他们用 10 亿向不生产阶级购买,于是 10 亿货币流到不生产阶级手里。(同时实现了工业品的 1/2。)不生产阶级用这 10 亿向生产阶级购买食物,于是又有 10 亿货币流回生产阶级手里。(从而实现了农产品的另一个 1/5。)生产阶级用这同一个 10 亿货币购买价值 10 亿的工业品,以此补偿他们的“预付”的半数。(同时实现了工业品的另一个 1/2。)不生产阶级[1434]用同一个 10 亿货币购买原料。(从而实现了农产品的又一个 1/5。)这样一来,20 亿货币流回生产阶级手里。

因而还剩下农产品的 2/5。1/5 以实物形式消费,但是第二个 1/5 以什么形式积累起来呢\fontbox{?}这个问题到后面再研究。\endnote{马克思在这里和在后面都采用魁奈的说法:只有 1/5 的农业总产品不进入流通,而由生产阶级以实物形式享用。马克思在手稿第 XXIII 本第 1433—1434 页(见本册第 405—406 页)和他写的《反杜林论》第二编第十章中又回过头来谈这个问题。他在这一章中对魁奈关于农业中流动资本的补偿的观点作了如下详细说明:“价值五十亿的全部总产品因而掌握在生产阶级的手中,也就是说,首先是掌握在租地农场主的手中,这些租地农场主每年花费二十亿经营资本(与一百亿基本投资相适应)来生产全部总产品。为了补偿经营资本,因而也为了维持一切直接从事农业的人所需要的农产品、生活资料、原料等等,是以实物形式从总收成中拿出来的,并且花费在新的农业生产上。因为,正如前面所说,是以一次规定了的标准的固定价格和简单再生产为前提,所以总产品中预先拿出去的部分的货币价值,等于二十亿利弗尔。因此,这一部分没有进入一般的流通,因为正如已经指出的,任何发生于每一个别阶级的范围之内而不是发生于各阶级相互之间的流通,都没有列入表内。”(《马克思恩格斯全集》中文版第 20 卷第 270—271 页)因此,按照魁奈的说法,应当说租地农场主以实物形式补偿他们的流动资本的那部分产品,占他们的全部总产品的 2/5。——第 352、406 页。}

(2)即使从魁奈本人的观点出发(按照他的观点,整个不生产阶级实际上只不过是雇佣劳动者),也已经可以看出,《经济表》的前提是错误的。

这里假定在生产阶级那里,“原预付”(固定资本)是“年预付”数额的 5 倍。在不生产阶级那里,这一项根本没有提及,这当然并不妨碍它的存在。

此外,说再生产等于 50 亿,是错误的。从《经济表》本身来看,再生产等于 70 亿:生产阶级方面 50 亿,不生产阶级方面 20 亿。

\tsubsectionnonum{[(b)个别重农主义者局部地回到重商主义的观点。重农主义者要求竞争自由]}

不生产阶级的产品等于 20 亿。这个产品是由 10 亿原料(这些原料一部分加入产品,一部分补偿加入产品价值的机器的损耗)和在原料加工期间被消费了的 10 亿食物组成的。

不生产阶级把这全部产品卖给土地所有者阶级和生产阶级,以便\textbf{第一},补偿“预付”(以原料形式),\textbf{第二},取得农产生活资料。这样,不生产阶级就\textbf{丝毫没有}留下\textbf{一点工业品}供他们自己消费,更不用说利息和利润了。勃多(或列特隆)看到了这一点,他这样来解释:不生产阶级\textbf{高于}产品的\textbf{价值}出卖他们的产品,因而他们卖 20 亿的东西等于 20 亿减去 x。因此,利润,甚至这个阶级\textbf{本身}所消费的、属于它所必需的生活资料的工业品,按照上面的解释,就只被归结为这个阶级\textbf{把自己商品的价格抬得高于它们的价值}\endnote{重农主义者勃多在他的《经济表说明》第三章第十二节(德尔出版的《重农学派》第 2 部第 852—854 页)发挥了这一观点。——第 407 页。}。可见,重农学派在这里必然回到重商主义体系,回到“\textbf{让渡利润}”的概念。

因此,他们也认为,工业家之间的自由竞争是完全必要的,这样可以使工业家不致过分欺骗生产阶级即农业家。另一方面,这种自由竞争之所以必要,是因为这样就可以使农产品卖得一个“\textbf{好}价钱”,就是说,通过输出国外把农产品的价格抬得\textbf{高于}它原来的本国价格,因为这里假定的是一个出口小麦等等的国家。

\tsubsectionnonum{[(c)关于价值不可能在交换中增殖的最初提法]}

\begin{quote}“每次买都是卖,每次卖都是买。”(\textbf{魁奈}《关于商业和手工业者劳动的问答》,德尔出版,\endnote{德尔出版的《重农学派》第一部在这个标题下把魁奈的两篇问答《关于商业。H 先生和 N 先生的第一次问答》和《关于手工业者劳动。第二次问答》合在一起。马克思的引文取自第一次问答。——第 407 页。}第 170 页)“买就是卖,卖就是买。”(\textbf{魁奈},见\textbf{杜邦·德·奈穆尔}《论近代科学的起源和进步》第 392 页)\endnote{马克思所引的这句话不在杜邦·德·奈穆尔的著作《论近代科学的起源和进步》本文内,而在内容同该著作衔接的《魁奈医生的学说,或他的社会经济学原理概述》一文内。——第 407 页。}“\textbf{价格总是先于买卖}。如果卖者和买者的竞争没有引起任何变化,价格就仍然是由同商业\textbf{无关的}其他原因所确定的那个价格。”(第 148 页)\endnote{引文取自魁奈的《关于商业的问答》。——第 407 页。}“始终可以假定,它〈交换〉对于双方〈当事人〉都是有利的,因为双方都保证自己有可能享受他们只有通过交换才能得到的财富。但是这里所讲的,始终只是具有\textbf{一定价值}的财富同具有\textbf{同一价值}的另一财富交换,因而,\textbf{不可能有财富的任何实际的增加}〈应该说:不可能有价值的任何实际的增加〉。”(同上,第 197 页)\endnote{取自《关于手工业者劳动的问答》。——第 407 页。}\end{quote}

明确地把“\textbf{预付}”和“\textbf{资本}”等同起来。把\textbf{资本积累}作为主要条件。

\begin{quote}“因此,\textbf{增加资本是增加劳动的主要手段},这会给\textbf{社会}带来\textbf{最大的好处}”等等。(\textbf{魁奈},见\textbf{杜邦·德·奈穆尔},同上第 391 页)\endnote{取自《魁奈医生的学说》。——第 407 页。}[XXIII—1434]\end{quote}

\tsectionnonum{[(9)重农学派的追随者毕阿伯爵对土地贵族的赞美]}

[XXII—1399]\textbf{毕阿(伯爵)}《政治要素,或社会经济真正原则的研究》(六卷集)1773 年伦敦版。

这个低能的废话连篇的著作家,把重农主义的外观看成重农主义的实质,竭力赞扬土地贵族,事实上,只有当重农主义符合这个目的时,他才接受重农主义。要不是他的著作中有象后来李嘉图的著作中那样露骨地表现出来的粗俗的资产阶级性质,根本就不会提到他。认为“纯产品”只限于地租的错误看法,不会使问题有丝毫改变。

毕阿伯爵所说的东西,就是李嘉图后来对一般“纯产品”所重复提到的东西\endnote{马克思指李嘉图的《政治经济学和赋税原理》第二十六章(《论总收入与纯收入》)。——第 408、438 页。}。工人属于非生产费用\authornote{见第 159 页脚注。——编者注},他们之所以存在,只是为了使“纯产品”所有者得以“组成社会”(见有关的地方)\endnote{马克思指他在补充本 A(见注 122)第 27—32 页所作的毕阿伯爵著作摘录。在后面正文所用的引文中,马克思注明的不是补充本的页码,而是毕阿伯爵著作的页码。——第 408 页。}。自由工人的地位被他看成只不过是奴隶制的改变了的形式,然而在他看来,这种改变了的形式对于上层组成“社会”来说是必要的。\fontbox{~\{}连\textbf{阿瑟·杨格}也把“纯产品”,即剩余价值,说成生产的目的。\endnote{关于“剩余产品的狂热的崇拜者”杨格,见《资本论》第 1 卷第 7 章注 34。——第 408 页。}\fontbox{\}~}

[1400]由此可以使人想起李嘉图同斯密争论的一段话,\endnote{马克思指李嘉图的《政治经济学和赋税原理》第二十六章(《论总收入与纯收入》)。——第 408、438 页。}他不同意斯密把使用工人最多的资本看成生产能力最大的资本。参看毕阿的著作第 6 卷第 51—52、68—70 页;其次,关于工人阶级和奴隶制,参看第 2 卷第 288、297、309 页;第 3 卷第 74、95—96、103 页;第 6 卷第 43、51 页;关于这些工人被迫进行剩余劳动,以及什么叫做“最必要的生存资料”,参看第 6 卷第 52—53 页。

我们在这里只引一段话,因为这段话对于所谓资本家总是冒\textbf{风险}的空谈,作了很好的反驳:

\begin{quote}“据说他们〈商人〉为了多赚钱而冒很多风险。不过,他们或者拿人去冒险,或者拿商品和货币去冒险。如果他们为了发财而让别人陷于明显的危险境地,那他们就是干了极坏的事情。至于谈到商品,一个人把商品生产出来,是有功绩的;但是,为了一个人的发财致富而拿这些商品去冒险,就不可能是什么功绩了”等等。(第 2 卷第 297 页)[XXII—1400]\end{quote}

\tsectionnonum{[(10)从重农学派的观点出发反驳土地贵族(英国的一个匿名作者)]}

[XXIII—1449]《国民财富基本原理的说明。驳亚当·斯密博士等人的某些错误论点》1797 年伦敦版。\endnote{后来查明,马克思在这里分析的匿名著作的作者是一个叫约翰·格雷(JohnGray)的人,此人生卒年月不详。1802 年这位作者在伦敦还发表了一部关于所得税的著作。——第 410 页。}

这本书的作者知道安德森的著作,因为他在该书的附录中,转载了安德森关于阿贝丁郡的农业报告的片段。

这是英国的一本可直接算在重农主义学说内的\textbf{唯一重要}著作。\textbf{威廉·斯宾斯}的《不列颠不依靠商业》一书(1807 年版)只不过是一幅讽刺画。这个斯宾斯在 1814—1815 年间,是土地所有者的最狂热的维护者之一,他根据主张……贸易自由的重农主义学说来维护土地所有者的利益。不要把这个家伙同\textbf{土地私有制}的死敌\textbf{托马斯·斯宾斯}混淆起来。

《基本原理》一书首先包含着对重农主义学说的卓越而简洁的概括。

作者正确地指出重农主义的观点来源于\textbf{洛克}和\textbf{范德林特}的观点。他把重农学派说成是这样的著作家,他们

\begin{quote}“\textbf{虽不是完全正确地}但很有系统地阐明了”自己的学说。(第 4 页)这点还可参看第 6 页(摘录在\textbf{稿本}H 第 32—33 页\endnote{马克思指他的补充本 H(见注 122)。后面正文引用的是补充本 H 第 32—33 页上对匿名著作第 6 页所作的几乎全部摘录。——第 410 页。})。\end{quote}

从匿名作者对重农主义学说的概括中,可以十分明显地看出,被后来的辩护论者——斯密就已经部分地这样做了——当作资本形成的基础的\textbf{节欲论},是直接从重农学派的这样一个见解产生的:工业等等不创造\textbf{任何剩余价值}。

\begin{quote}“用于使用和维持手工业者、制造业者\endnote{这位英国匿名作者所说的“制造业者”是指制造业工人(他有时把他们叫做“劳动的制造业者”)和工业家-企业主(有时他称他们为“企业老板”)。而“手工业者”一词,这位作者是指雇佣工人和本来意义上的手工业者。——第 411 页。}和商人的费用,结果只能\textbf{保持支出的数额的价值},因而是非生产的〈因为它不生产剩余价值〉。除非手工业者、制造业者和商人\textbf{从本来供他们维持每日生活的东西中节约和积累下来一部分},否则社会的财富靠他们不可能得到丝毫\textbf{增加}。可见,他们\textbf{只有通过节欲和节约}〈西尼耳的节欲论和亚当·斯密的节约论〉才能使总资本有所增加。相反,土地耕种者能够消费自己的全部收入,同时又使国家致富;因为他们的活动会提供叫做地租的剩余产品。”(第 6 页)“有一个阶级,他们的劳动虽然也生产一些东西,但所生产的并不比维持他们的劳动所花费的多,理所当然可以把他们叫做\textbf{非生产阶级}。”(第 10 页)\end{quote}

\textbf{应当把剩余价值的生产同剩余价值的“转手”严格区分开来}。

\begin{quote}“收入的\textbf{增加}〈即\textbf{积累}〉只间接地是经济学家\endnote{“经济学家”是十八世纪下半叶和十九世纪上半叶在法国对重农学派的称呼。——第 38、139、223、411 页。}的研究对象……他们的研究对象是\textbf{收入的生产和再生产}。”(第 18 页)\end{quote}

这正是重农主义的巨大功绩。重农学派给自己提出的问题是,\textbf{剩余价值}(匿名作者把剩余价值叫做“收入”)是怎样生产和再生产出来的。关于剩余价值怎样\textbf{以更大的规模再生产出来},即剩余价值怎样增加的问题,只是属于第二位的问题。首先必须揭示剩余价值的\textbf{范畴},[1450]揭示剩余价值生产的秘密。

\textbf{剩余价值和商业资本}:

\begin{quote}“在谈到收入的\textbf{生产}时,用\textbf{收入的转手}这个问题来替换,是完全不合逻辑的,只有\textbf{一切商业交易}才归结为收入的转手。”(第 22 页)“\textbf{商业}这个词的意思不过是指\textbf{商品的交换}……有时,这种交换对一方比对另一方更有利;但一个人的赢利,总是另一个人的亏损,所以他们之间的商业交易实际上\textbf{不会造成财富的任何增加}。”(第 23 页)“如果一个犹太人把 1 克朗卖了 10 先令,或者把安女王时代的 1 法寻卖了 1 基尼,\endnote{1 克朗是 5 先令的铸币,1 法寻是 1/4 便士,1 基尼等于 21 先令。——第 411 页。}那他毫无疑问会增加自己的收入,但他并不会因此而增加\textbf{现有的贵金属量};而且,无论喜爱古玩的买者是同旧币的卖者住在一条街上,还是住在法国或中国,这种商业交易的性质始终是一样的。”(第 23 页)\end{quote}

\textbf{在重农学派的著作中,工业利润被看成“让渡利润”}(即按重商主义来解释)。\textbf{因此,这个英国人作出正确的结论说,只有当工业品在国外出卖时,这种利润才是真正的利润。他从重商主义的前提出发作出正确的重商主义的结论。}

\begin{quote}“任何一个制造业者,如果他的商品是在国内出售和消费,那末,无论他自己获得多少赢利,也不会使国民收入增加分毫;因为\textbf{买者的亏损……同制造业者的赢利正好一样多}……这里是卖者和买者之间的\textbf{交换},而不是财富的增加。”(第 26 页)“为了\textbf{弥补盈余的缺乏}……企业主从自己支付的工资中提取 50%的利润,或者说,从他们支付给制造业工人的每 1 先令中提取 6 便士……如果商品在国外出卖”,那末这就是若干数量的“手艺人”所提供的“\textbf{国民利润}”。(第 27 页)\end{quote}

\textbf{作者很好地说明了荷兰财富的原因}。渔业(还应当指出畜牧业)。对东方香料的垄断。海运业。向外国人贷款(补充本 H 第 36—37 页)\endnote{在补充本 H 第 36—37 页是对匿名著作第 31—33 页所作的摘录。——第 412 页。}。

这位作者写道,制造业者“是一个\textbf{必要的}阶级”,但他们不是“\textbf{生产阶级}”。(同上,第 35 页)他们“只是使土地耕种者早已取得的收入\textbf{替换}或\textbf{转手},而他们采取的办法是,使这种收入在一种新形式上具有\textbf{耐久性}”。(第 38 页)

只有四个必要的阶级:(1)生产阶级或土地耕种者;(2)制造业者;(3)国家保卫者;(4)“教师阶级”,他用教师来代替重农学派所说的“什一税所得者”即牧师。

\begin{quote}“因为任何市民社会都需要吃饭、穿衣、保卫和教育。”(第 50—51 页)“经济学家”的错误在于,“他们把\textbf{作为单纯的租金所得者的地租所得者}看成社会的\textbf{生产阶级}……他们暗示,教会和国王必定要靠土地所有者获得的地租来维持生活,因而在某种程度上改正了自己的错误。斯密博士……让它〈“经济学家”的上述错误〉贯穿\textbf{他的全部著作}〈这是对的〉,他的批判正好针对着经济学家体系的正确部分”。(第 8 页)\end{quote}

[1451]土地所有者本身不仅不是\textbf{生产}阶级,甚至不是社会的\textbf{必要阶级}:

\begin{quote}“\textbf{土地所有者}作为单纯的地租所得者,\textbf{并不是社会的必要阶级}……\textbf{只要地租脱离宪法所规定的目的——为保卫国家服务},这种地租的所得者就不再是必要阶级,而成为社会上最不需要的、最麻烦的阶级之一。”(第 51 页)关于这一点的进一步内容,见补充本 H 第 38—39 页\endnote{在补充本 H 第 38—39 页是对匿名著作第 51—54 页所作的摘录。在后面正文所用的引文中,马克思注明的不是补充本 H 的页码,而是匿名著作的页码。——第 413 页。}。\end{quote}

所有这些都很好,这种从重农学派观点出发对地租所得者的反驳,\textbf{作为重农学派学说的完成是很重要的}。

作者指出,真正的\textbf{土地税}是土耳其人所特有的。(同上,第 59 页)

\textbf{土地所有者}不仅对现有的“土地改良”\textbf{收税},而且往往对“推测中的将来的改良”也收税。(第 63—64 页)地租税。(第 65 页)

在税收方面,重农主义理论在英格兰、爱尔兰、封建的欧洲、莫卧儿帝国\textbf{早就}实现了。(第 93—94 页)

土地所有者是收税人。(第 118 页)

\textbf{重农主义的局限性表现在下述看法上}(对\textbf{分工}缺乏\textbf{理解}):

\begin{quote}假设一个钟表业者或棉布厂主不能把他的钟表或棉布卖掉;他就陷入困难的境地。这表明,“制造业者只有成为\textbf{卖者}才能发财致富\end{quote}

(实际上,这只是表明,他把自己的产品作为\textbf{商品}生产出来),

\begin{quote}一旦他不再成为\textbf{卖者},他的\textbf{利润}也就立即终止\end{quote}

(而本身不是\textbf{卖者}的租地农场主的利润又是怎么一回事呢\fontbox{?}),

\begin{quote}因为这些利润不是自然的而是人为的利润。土地耕种者……不\textbf{出卖}任何东西就\textbf{能生存}、兴旺和增加自己的财富”(第 38—39 页)\end{quote}

(但在这种场合,他必须同时又是制造业者)。

为什么作者只谈钟表业者或棉布厂主呢\fontbox{?}同样,也可以假设煤炭、铁、亚麻、靛蓝等等的生产者不能把这些产品卖掉,或者连小麦的生产者也不能把自己的小麦卖掉。关于这一点,前面提到过的贝阿尔岱·德·拉贝伊讲得很好。\endnote{马克思在他的 1861—1863 年手稿第 1446 页(第 XXIII 本)提到了贝阿尔岱·德·拉贝伊的旨在反对重农学派的著作《关于取消税收办法的研究》1770 年阿姆斯特丹版。这一著作的摘录在补充本 H 第 10—11 页。马克思指的贝阿尔岱·德·拉贝伊的那段话在该书第 43 页。——第 414 页。}匿名作者不得不提出以\textbf{直接}消费为目的的生产来反对\textbf{商品生产},这是同下面的情况非常矛盾的:对于重农学派来说,最主要的问题倒是\textbf{交换价值}。然而后面这点也贯穿在我们所说的这个人的著作中,这是囿于资产阶级前的\textbf{思考方式}的一种对事物的\textbf{资产阶级}见解。\endnote{在最后几段论述匿名作者的“重农主义局限性”的俄译文中,对马克思在引用所分析的著作的文字(引自该著作第 38—39 页)中加入的某些插话在编排上略加改动。马克思对引文作了删节。本版按原作恢复了删去的文字。——第 414 页。}

这位匿名作者反对阿瑟·杨格认为\textbf{高价格对农业繁荣很重要}的看法;\textbf{但是这样反对杨格同时也就是反驳重农主义}。(同上,第 65—78 页和第 118 页)

\textbf{由卖者在名义上提高价格不能得出剩余价值}:

\begin{quote}“靠提高\textbf{产品的名义价值……卖者不会致富}……因为他们作为卖者所得的利益,在他们作为买者时又如数付出。”(第 66 页)\end{quote}

下面这段话是按\textbf{范德林特}的精神写的:

\begin{quote}“只要能为每个失业者找到一块可耕的土地,任何一个失业者就都不会没有土地了。劳动的房屋是好东西;但劳动的田地更好得多。”(第 47 页)\end{quote}

匿名作者反对一切租佃制,不过他认为长期租佃比短期租佃好,因为如果实行短期租佃,土地所有权只会妨碍生产和阻碍土地改良。(第 118—123 页)(\textbf{爱尔兰的租佃权}。)\endnote{关于“爱尔兰的租佃权”,见马克思发表在 1853 年 7 月 11 日《纽约每日论坛报》上的文章(《马克思恩格斯全集》中文版第 9 卷第 177—183 页)。——第 414 页。}[XXIII—1451]

\tsectionnonum{[(11)关于一切职业都具有生产性的辩护论见解]}

[V—182]哲学家生产观念,诗人生产诗,牧师生产说教,教授生产讲授提纲,等等。罪犯生产罪行。如果我们仔细考察一下最后这个生产部门同整个社会的联系,那就可以摆脱许多偏见。罪犯不仅生产罪行,而且还生产刑法,因而还生产讲授刑法的教授,以及这个教授用来把自己的讲课作为“商品”投到一般商品市场上去的必不可少的讲授提纲。据说这就会使国民财富增加,更不用说象权威证人罗雪尔教授先生所说的,这种讲授提纲的手稿给作者本人带来的个人快乐了。

其次,罪犯生产全体警察和全部刑事司法、侦探、法官、刽子手、陪审官等等,而在所有这些不同职业中,每一种职业都是社会分工中的一定部门,这些不同职业发展着不同的人类精神能力,创造新的需要和满足新需要的新方式。单是刑讯一项就推动了最巧妙的机械的发明,并保证使大量从事刑具生产的可敬的手工业者有工可做。

罪犯生产印象,有时是道德上有教益的印象,有时是悲惨的印象,看情况而定;而且在唤起公众的道德感和审美感这个意义上说也提供一种“服务”。他不仅生产刑法讲授提纲,不仅生产刑法典,因而不仅生产这方面的立法者,而且还生产艺术、文艺——小说,甚至悲剧;不仅缪尔纳的《罪》和席勒的《强盗》,而且《奥狄浦斯王》和《理查三世》都证明了这一点。罪犯打破了资产阶级生活的单调和日常的太平景况。这样,他就防止了资产阶级生活的停滞,造成了令人不安的紧张和动荡,而没有这些东西,连竞争的刺激都会减弱。因此,他就推动了生产力。一方面,犯罪使劳动市场去掉了一部分过剩人口,从而减少了工人之间的竞争,在一定程度上阻止工资降到某种最低额以下;另一方面,反对犯罪的斗争又会吸收另一部分过剩人口。这样一来,罪犯成了一种自然“平衡器”,它可以建立适当的水平并为一系列“有用”职业开辟场所。

罪犯对生产力的发展的影响,可以研究得很细致。如果没有小偷,锁是否能达到今天的完善程度\fontbox{?}如果没有[183]伪造钞票的人,银行券的印制是否能象现在这样完善\fontbox{?}如果商业中没有欺骗,显微镜是否会应用于通常的商业领域(见拜比吉的书)\fontbox{?}应用化学不是也应当把自己取得的成就,象归功于诚实生产者的热情那样,归功于商品的伪造和为发现这种伪造所作的努力吗\fontbox{?}犯罪使侵夺财产的手段不断翻新,从而也使保护财产的手段日益更新,这就象罢工推动机器的发明一样,促进了生产。而且,离开私人犯罪的领域来说,如果没有国家的犯罪,能不能产生世界市场\fontbox{?}如果没有国家的犯罪,能不能产生民族本身\fontbox{?}难道从亚当的时候起,罪恶树不同时就是知善恶树吗\fontbox{?}

孟德维尔在他的《蜜蜂的寓言》(1705 年版)中,已经证明任何一种职业都具有生产性等等,在他的书中,已经可以看到这全部议论的一般倾向:

\begin{quote}“我们在这个世界上称之为恶的东西,不论道德上的恶,还是身体上的恶,都是使我们成为社会生物的伟大原则,是毫无例外的\textbf{一切职业和事业}的牢固基础、\textbf{生命力和支柱};我们应当在这里寻找一切艺术和科学的真正源泉;一旦不再有恶,社会即使不完全毁灭,也一定要衰落。”\endnote{[贝·孟德维尔]《蜜蜂的寓言,或个人劣行即公共利益》1728 年伦敦第 5 版第 428 页。该书第一版于 1705 年出版。——第 417 页。}\end{quote}

当然,只有孟德维尔才比充满庸人精神的资产阶级社会的辩护论者勇敢得多、诚实得多。[V—183]

\tsectionnonum{[(12)]资本的生产性。生产劳动和非生产劳动}

\tsubsectionnonum{[(a)资本的生产力是社会劳动生产力的资本主义表现]}

[XXI—1317]我们不仅看到了资本是怎样进行生产的,而且看到了资本本身是怎样被生产出来的,资本作为一种发生了本质变化的关系,是怎样从生产过程中产生并在生产过程中发展起来的。\endnote{马克思指同《资本的生产性。生产劳动和非生产劳动》这一节紧接的前一节《劳动对资本的形式上的隶属和实际上的隶属。过渡形式》(手稿第 XXI 本,第 1306—1316 页)。关于劳动对资本的形式上的隶属和实际上的隶属的问题,见马克思《资本论》第 1 卷第 14 章和第 24 章第 3 节。——第 418 页。}一方面,资本改变着生产方式的形态,另一方面,生产方式的这种被改变了的形态和物质生产力的这种特殊发展阶段,是资本本身的基础和条件,是资本本身形成的前提。

因为活劳动——由于资本同工人之间的交换——被并入资本,从劳动过程一开始就作为属于资本的活动出现,所以社会劳动的一切生产力都表现为资本的生产力,就和劳动的一般社会形式在货币上表现为一种物的属性的情况完全一样。同样,现在社会劳动的生产力和社会劳动的特殊形式,表现为资本的生产力和形式,即\textbf{物化}劳动的,劳动的物的条件(它们作为这种独立的要素,人格化为资本家,同活劳动相对立)的生产力和形式。这里,我们又遇到关系的颠倒,我们在考察货币时,已经把这种关系颠倒的表现称为\textbf{拜物教}。\endnote{马克思在《政治经济学批判》第一分册(1859 年)中就已指出,在资产阶级社会中,社会关系的神秘化在货币上表现得特别显著,财富结晶为贵金属形式的拜物教是资产阶级生产所固有的(见《马克思恩格斯全集》中文版第 13 卷第 37—39 和 144—146 页)。马克思在《剩余价值理论》第三册补充部分《收入及其源泉。庸俗政治经济学》(手稿第 891—899 和 910—919 页)中对资产阶级关系的拜物教化过程作了分析。——第 418 页。}

资本家本身只有作为\textbf{资本的人格化}才是统治者。(在意大利式簿记中,他作为\textbf{资本家},作为人格化资本的这一作用,总是同他作为单纯的个人相对立,而他作为单纯的个人就是仅仅作为私人消费者,作为他自己的资本的债务人出现。)

资本的\textbf{生产性}(即使仅仅考察劳动对资本的\textbf{形式上的}隶属),首先在于\textbf{强迫进行剩余劳动},强迫进行超过直接需要的劳动。这种强迫,是资本主义生产方式和以前的生产方式所共有的,但是,资本主义生产方式是以更加有利于生产的方式实行并采用这种强迫的。

即使考察这种纯粹形式上的关系,考察资本主义生产的较不发达阶段和较为发达阶段所共有的\textbf{一般}形式,\textbf{生产资料},劳动的物的条件——劳动材料、劳动资料(以及生活资料)——也不是从属于工人,相反,是工人从属于它们。不是工人使用它们,而是它们使用工人。正因为这样,它们才是资本。“资本\textbf{使用}劳动。”对工人来说,它们不是生产产品的手段,不论这些产品采取直接生存资料的形式,还是采取交换手段,商品的形式。相反,工人对它们来说倒是一个手段,它们依靠这个手段,一方面保存自己的价值,另方面使自己的价值转化为资本,也就是说,吸收剩余劳动,使自己的价值增殖。

这种关系在它的简单形式中就已经是一种颠倒,是物的人格化和人的物化;因为这个形式和以前一切形式不同的地方就在于,资本家不是作为这种或那种个人属性的体现者来统治工人,他只在他是“资本”的范围内统治工人;他的统治只不过是物化劳动对活劳动的统治,工人制造的产品对工人本身的统治。

但是,这种关系所以变得更加复杂,显得更加神秘,是因为随着特殊的资本主义生产方式的发展,不仅这些直接物质的东西\fontbox{~\{}它们都是劳动产品;从使用价值来看,它们是劳动产品,又是劳动的物的条件;从交换价值来看,它们是物化的一般劳动时间或货币\fontbox{\}~}起来反对工人,作为“资本”同工人相对立,就连社会地发展了的劳动的形式——协作、工场手工业(作为分工的形式)、工厂(作为以机器体系为自己的物质基础的社会劳动形式)——都表现为\textbf{资本的发展形式},因此,从这些社会劳动形式发展起来的劳动生产力,从而还有科学和自然力,也表现为\textbf{资本的生产力}。事实上,协作中同种劳动的统一,分工中异种劳动的结合,机器工业中自然力、科学和劳动产品的用于生产,所有这一切,都作为某种\textbf{异己的、物的}东西,纯粹作为不依赖于工人而支配着工人的劳动资料的存在形式,同单个工人相对立,正如劳动资料本身在它们作为材料、工具等简单可见的形式上,作为\textbf{资本}的职能,因而作为\textbf{资本家}的职能,同单个工人相对立一样。

工人自己的劳动的社会形式,或者说,工人自己的[1318]社会劳动的形式,是完全不以单个工人为转移而形成的关系;工人从属于资本,变成这些社会构成的要素,但是这些社会构成并不属于工人。因而,这些社会构成,作为资本本身的\textbf{形态},作为不同于每个工人的单个劳动能力的、属于资本的、从资本中产生并被并入资本的结合,同工人相对立。并且这一点随着下述情况的发展越来越具有实在的形式。这些情况是:一方面,工人的劳动能力本身由于上述社会形式而发生了形态变化,以致它在独立存在时,也就是说,\textbf{处在}这种资本主义联系\textbf{之外}时,就变得无能为力,它的独立的生产能力被破坏了;另一方面,随着机器生产的发展,劳动条件在工艺方面也表现为统治劳动的力量,同时又代替劳动,压迫劳动,使独立形式的劳动成为多余的东西。

工人的劳动的\textbf{社会}性质作为从某种意义上说\textbf{资本化的}东西同工人相对立(例如,在机器生产部门,劳动的可见产品表现为劳动的统治者),在这个过程中,各种自然力和科学——历史发展总过程的产物,它抽象地表现了这一发展总过程的精华——自然也发生同样的情况:它们作为资本的\textbf{力量}同工人相对立。科学及其应用,事实上同单个工人的技能和知识分离了,虽然它们——从它们的源泉来看——又是劳动的产品,然而在它们进入劳动过程的一切地方,它们都表现为\textbf{被并入资本的东西}。使用机器的资本家不必懂得机器(见尤尔的著作)。\endnote{马克思在《资本论》第一卷第十三章注 108 中写道:“科学不费资本家‘分文’,但这丝毫不妨碍他们去利用科学。资本象吞并别人的劳动一样,吞并‘别人的’科学。但是,科学或物质财富的‘资本主义的’占有和‘个人的’占有,是截然不同的两件事。尤尔博士本人曾哀叹他的亲爱的、使用机器的工厂主对力学一窍不通……”——第 421 页。}但是,\textbf{在机器上}实现了的科学,作为\textbf{资本}同工人相对立。而事实上,以\textbf{社会劳动}为基础的所有这些对科学、自然力和大量劳动产品的应用本身,只表现为\textbf{剥削}劳动的\textbf{手段},表现为占有剩余劳动的手段,因而,表现为属于资本而同劳动对立的\textbf{力量}。资本使用这一切手段,当然只是为了剥削劳动,但是为了剥削劳动,资本必然要在生产过程中使用这些手段。所以,劳动的\textbf{社会}生产力的发展和这个发展的条件就表现为\textbf{资本的行为},这种行为不仅是不管单个工人的意志如何而完成的,而且是直接反对单个工人的。

因为资本是由商品组成的,所以资本本身具有二重性:

(1)\textbf{交换价值}(货币);但是,它是\textbf{自行增殖的价值},是——因为它是\textbf{价值}——创造价值、\textbf{作为价值而增殖}、取得一个增殖额的价值。这种价值增殖归结为一定量物化劳动同较大量活劳动的交换。

(2)\textbf{使用价值};这里,资本是按照它在劳动过程中所具有的一定关系出现的。但是,正是在这里,资本不仅仅是\textbf{劳动}所归属的、把劳动并入自身的劳动材料和劳动资料:资本还把劳动的\textbf{社会结合}以及与这些社会结合相适应的劳动资料的发展程度,连同劳动一起并入它自身。资本主义生产第一次大规模地发展了劳动过程的物的条件和主观条件,把这些条件同单个的独立的劳动者分割开来,但是资本是把这些条件作为统治\textbf{单个工人}的、对单个工人来说是\textbf{异己的}力量来发展的。

这一切使资本变成一种非常神秘的存在。[1318]\endnote{马克思把 1861—1863 年手稿第 1318 页(除了最后 9 行)从第 XXI 本剪下来贴到《资本论》第一卷倒数第二稿第 490 页(这个倒数第二稿的第六章载于《马克思恩格斯文库》1933 年版第 2(7)卷)。后面第 1318 页、第 1319 页和第 1320 页前半页的正文,马克思在手稿页边(第 1318 页末尾和第 1320 页开头)曾两次注上“利润”字样,显然打算把它用在论利润的一节。——第 422 页。}

\centerbox{※     ※     ※}

[1320]因此,资本(1)作为\textbf{强迫}进行剩余劳动的力量,(2)作为吸收和占有社会劳动生产力和一般社会生产力(如科学)的力量(作为这些生产力的人格化),它是生产的。

试问:既然劳动的生产力已经转给了资本,而同一生产力不能计算两次,一次作为劳动的生产力,另一次作为资本的生产力,那末,同资本相对立的劳动,怎样或者说为什么表现为生产的,表现为\textbf{生产劳动}呢\fontbox{?}\fontbox{~\{}劳动的生产力就是资本的生产力。而\textbf{劳动能力}所以是生产的,是由于它的\textbf{价值}和\textbf{它创造的价值}之间有\textbf{差别}。\fontbox{\}~}

\tsubsectionnonum{[(b)资本主义生产体系中的生产劳动]}

只有把生产的资本主义形式当作生产的绝对形式、因而当作生产的永恒的自然形式的资产阶级狭隘眼界,才会把从资本的观点来看什么是生产劳动的问题,同一般说来哪一种劳动是生产的或什么是\textbf{生产劳动}的问题混为一谈,并且因此自作聪明地回答说,凡是生产某种东西、取得某种结果的劳动,都是生产劳动。

只\textbf{有直接转化为资本的}劳动,也就是说,只有使可变资本成为可变的量,因而使整个资本 C 等于 C+△\endnote{马克思在这里用数学上表示增量的希腊字母Δ代表剩余价值。马克思在后面正文中用拉丁字母 h 表示同一意义。——第 422 页。}的劳动,才是\textbf{生产的}。假定可变资本在同劳动交换之前等于 x,这样,我们得到等式 y=x,那末,把 x 变为 x+h、把等式 y=x 变为等式 y′=x+h 的那种劳动,是生产劳动。这是需要说明的\textbf{第一}点。这里谈的是创造剩余价值的劳动,或者说,是作为使资本能够形成剩余价值,因而能够表现为资本,表现为自行增殖的价值的因素来为资本服务的劳动。

\textbf{第二},劳动的社会的和一般的生产力,是资本的生产力;但是这种生产力只同劳动过程有关,或者说,只涉及使用价值。它表现为作为物的资本所固有的属性,表现为资本的使用价值。它不直接涉及\textbf{交换价值}。无论是 100 个工人一起劳动,还是他们各自单独劳动,他们所生产的产品的价值都等于 100 个工作日,不管这些工作日表现为许多产品或很少产品;换句话说,这些产品的价值不取决于劳动生产率。

[1321]劳动生产率的差别只在一个方面涉及交换价值。

举例来说,如果劳动生产率在某一个生产部门有了发展,例如用机器织机代替手工织机来生产布,已经不是例外的情况,用机器织机织 1 码布所需的劳动时间,只是用手工织机织 1 码布所需的劳动时间的一半,那末,一个手工织工的 12 小时就不再表现为 12 小时的价值,而只是表现为 6 小时的价值,因为\textbf{必要}劳动时间现在缩短为 6 小时了。手工织工虽然同以前一样劳动 12 小时,但他的 12 小时现在只等于 6 小时的社会劳动时间。

但是,这里谈的不是这一点。相反,如果我们拿另一个生产部门例如排字来看,在这里还没有使用机器,那末这个部门中的 12 小时创造的价值,同机器等等最发达的生产部门中的 12 小时创造的\textbf{价值}完全一样多。因此,作为\textbf{价值}的创造者,劳动总是\textbf{单个工人}的劳动,不过表现为\textbf{一般劳动}。因此,生产劳动,作为生产价值的劳动,总是作为单个劳动能力的劳动、\textbf{单个工人}的劳动同资本相对立,而不管这些工人在生产过程中参加什么样的社会结合。所以,同工人相对立的资本,代表劳动的社会生产力,而同资本相对立的工人的生产劳动,始终只代表\textbf{单个工人}的劳动。

\textbf{第}三,如果说榨取工人的剩余劳动和占有劳动的社会生产力,看来是资本的自然属性,因而看来是从资本的使用价值中产生的属性,那末,反过来说,把劳动自己的社会生产力表现为资本的生产力,把劳动生产的剩余产品表现为资本生产的剩余价值、资本的自行增殖,看来就是劳动的自然属性。

这三点现在要详细探讨一下,并从中得出生产劳动和非生产劳动的差别。

\textbf{关于第一点}。资本的生产性在于资本同作为雇佣劳动的劳动相对立,而劳动的生产性在于劳动同作为资本的劳动资料相对立。

我们已经看到,货币转化为资本,就是说,一定的交换价值转化为自行增殖的交换价值,转化为价值加剩余价值,是由于这个交换价值有一部分转化为在劳动过程中用作劳动资料(原料、工具,总之,劳动的物的条件)的商品,而另一部分则用于购买劳动能力。但是使货币转化为资本的,不是货币和劳动能力的最初交换,不是购买劳动能力这一事实本身。这种购买把被使用的劳动能力在一定时间内并入资本;换句话说,使一定量的活劳动成为资本本身的存在形式之一,可以说,成为资本本身的隐德来希\authornote{希腊文?ντελ?χεια的音译,古希腊哲学家亚里士多德的用语;他认为每一事物所要完成或达到的目的即其潜能的实现,就是隐德来希。在这里有活动、现实、效能的意思。——译者注}。

在实际生产过程中,活劳动转化为资本,是由于活劳动一方面把工资再生产出来,也就是把可变资本的价值再生产出来,另一方面又创造一个剩余价值;由于这个转化过程,整个[预付的]货币额就都转化为资本,虽然这个货币额中直接发生变化的部分,只是用于工资的那一部分。如果原先价值等于 C+v,它现在就等于 C+(v+X),或者同样可以说,(C+v)+X\endnote{马克思在这里以及在后面用拉丁字母 x 代表剩余价值。——第 425 页。};换句话说,原来的货币额,原来的价值量,在劳动过程中已经增殖,表现为既保存自己同时又增大自己的价值。

\fontbox{~\{}必须指出:只有资本的\textbf{可变部分}才创造资本的增殖额,这种情况丝毫也不改变以下事实,即通过这个过程,全部原有价值增大了,增加了一个剩余价值量;因此,仍然是全部原有的货币额都转化为资本。因为原有价值等于 C+v(不变资本和可变资本)。在上述过程中,这个价值转化为 C+(v+X);v+X 是再生产出来的部分,是通过活劳动转化为物化劳动产生的,而这个转化是由 v 同劳动能力的交换,由可变资本转化为工资所决定和引起的。但是,C+(v+X)=(C+v)(原有资本)+X。此外,v 所以能转化为 v+X,也就是说,(C+v)所以能转化为(C+v)+X,只是由于货币的一部分已转化为 C。一部分货币所以能转化为\textbf{可变}资本,只是由于另一部分货币转化为不变资本。\fontbox{\}~}

劳动在实际生产过程中\textbf{实际上}转化为资本,但是,这个转化是由货币同劳动能力的最初交换决定的。只是由于劳动\textbf{直接}转化为不属于工人而属于资本家的\textbf{物化}劳动,货币才转化为资本,就连已经取得生产资料即劳动条件的形式的那一部分货币也是这样。在此以前,货币不论以它本身的形式存在,或者以那种在实物形式上可以充当生产新商品所必需的生产资料的商品(产品)的形式存在,都只不过\textbf{从可能性来说}是资本。

[1322]只有这种对劳动的一定\textbf{关系}才使货币或商品转化为资本,只有由于自己对生产条件的上述关系(在实际生产过程中有一定的关系同这个关系相适应)使货币或商品转化为资本的\textbf{劳动},才是\textbf{生产劳动};换句话说,只有使那种同劳动能力相对立的、独立化了的\textbf{物化}劳动的价值保存并增殖的劳动,才是生产劳动。生产劳动不过是对劳动能力出现在资本主义生产过程中所具有的整个关系和方式的简称。但是,把生产劳动同\textbf{其他}种类的劳动区分开来是十分重要的,因为这种区分恰恰表现了那种作为整个资本主义生产方式以及资本本身的基础的劳动的形式规定性。

由此可见,在资本主义生产体系中,\textbf{生产劳动}是给使用劳动的人生产\textbf{剩余价值}的劳动,或者说,是把客观劳动条件转化为资本、把客观劳动条件的所有者转化为资本家的劳动,所以,这是把自己的产品作为资本生产出来的劳动。

因此,我们所说的\textbf{生产劳动},是指\textbf{社会地规定了的}劳动,这种劳动包含着劳动的买者和卖者之间的一个十分确定的关系。

虽然劳动能力的买者手中的货币或商品(生产资料和工人的生活资料),只有经过上述过程,只有在上述过程中才转化为资本(这些东西在进入过程之前并不是资本,而只是必将变成资本),但是,它们\textbf{从可能性来说}是资本。它们所以是资本,是由于它们作为某种独立的东西同劳动能力相对立,而劳动能力也作为某种独立的东西同它们相对立,在这里有一种关系,它决定着并保证着它们同劳动能力的交换以及随后发生的劳动实际上转化为资本的过程。在这里,生产资料和生活资料在它们同工人的关系中,从一开始就具有\textbf{一种社会规定性},这种社会规定性使它们变成资本,给它们以支配劳动的权力。因此,它们在作为资本同劳动相对立的情况下,是劳动的\textbf{前提}。

因此,\textbf{生产劳动}可以说是直接同\textbf{作为资本的货币}交换的劳动,或者说,是直接同\textbf{资本}交换的劳动(这不过是前一说法的简化),也就是直接同这样的货币交换的劳动,这种货币从可能性来说就是资本,预定要执行资本的职能,换句话说,作为\textbf{资本}同劳动能力相对立。“\textbf{直接}同\textbf{资本}交换的劳动”,这句话的意思是指劳动同作为\textbf{资本}的货币交换,并使这些货币在实际上转化为资本。从“\textbf{直接}”一词产生什么后果,现在就要作更详细的说明。

因此,生产劳动是这样的劳动,它为工人仅仅再生产出事先已经确定了的他的劳动能力的价值,可是同时,它作为创造价值的活动却增大资本的价值,换句话说,它把它所创造的价值作为资本同工人本身相对立。

\tsubsectionnonum{[(c)在资本同劳动的交换中两个本质上不同的环节]}

我们在考察生产过程时\endnote{马克思指《资本和劳动之间的交换。劳动过程。价值增殖过程》一节(手稿第 I 本第 15—53 页),其中有一小节:《劳动过程和价值增殖过程的统一(资本主义生产过程)》(第 49—53 页)。——第 427 页。}已经看到,在资本同劳动的交换中,应该区别两个互相制约但本质上不同的环节。

\textbf{第一},劳动同资本的最初交换是一个\textbf{形式上的过程},其中资本作为货币出现,劳动能力作为\textbf{商品}出现。在这第一个过程中,劳动能力的出卖是观念上或法律上的出卖,尽管劳动要等到完成之后,也就是要在一日、一周等等末了才\textbf{支付报酬}。这种情况对于\textbf{出卖}劳动能力的交易并无影响。这里\textbf{直接}被出卖的,不是包含已经物化了的劳动的商品,而是\textbf{劳动能力本身的使用},因此,实际上是\textbf{劳动本身},因为劳动能力的使用表现在它的动作——劳动上。也就是说,这里不是通过商品同商品的交换而完成的劳动同劳动的交换。如果 A 把靴子卖给 B,那末他们两人交换的是劳动,一个换出的是物化在靴子中的劳动,另一个换出的是物化在货币中的劳动。但这里拿来交换的,在一方,是一般社会形式的,即作为\textbf{货币}的\textbf{物化劳动},另一方,是\textbf{还只作为劳动能力存在着的劳动};在这里,虽然被出卖的商品的\textbf{价值}不是劳动的价值(一个不合理的用语),而是劳动能力的\textbf{价值},但是,被买卖的对象却是这个劳动能力的使用,即劳动本身。因此,这里发生的是\textbf{物化}劳动同实际上化为活劳动的\textbf{劳动能力}的直接交换,也就是物化劳动同活劳动的交换。因此,工资——劳动能力的价值——如前所说,就表现为\textbf{劳动的价格},\endnote{指以下两小节:《劳动能力的价值。最低限度的工资,或平均工资》(手稿第 I 本第 21—25 页)和《货币和劳动能力之间的交换》(同上,第 25—34 页)。马克思在第 XXI 本第 1312—1314 页又回过头来谈“劳动的价格”问题。——第 428 页。}表现为劳动的直接的购买价格。

在这第一个环节中,工人和资本家的关系是商品的卖者和买者的关系。资本家支付劳动能力的\textbf{价值},即他所购买的商品的\textbf{价值}。

但是,同时,劳动能力所以被购买,只是因为这个劳动能力能够完成和有义务完成的劳动量比再生产劳动能力所需要的劳动量大;因此,这个劳动能力所完成的劳动,表现为一个比劳动能力的价值大的价值。

[1323]\textbf{第二},资本同劳动的\textbf{交换}的第二个环节,实际上同第一个环节毫无关系,严格地说,这个环节根本不是\textbf{交换}。

第一个环节的特点是货币同商品的交换——等价物的交换;在这里,工人和资本家仅仅作为商品所有者彼此对立。交换的是等价物(就是说,交换实际上\textbf{在什么时候}实现,并不会使这个关系有丝毫变化;劳动的价格究竟\textbf{高于}或\textbf{低于}劳动能力的\textbf{价值},还是\textbf{等于}后者的\textbf{价值},并不会使这个交易的性质有丝毫变化。因此,这个交易\textbf{可以}按照商品交换的一般规律来进行)。

第二个环节的特点是根本不发生任何交换。货币所有者不再是商品的买者,而工人也不再是商品的卖者。货币所有者现在执行资本家的职能。他消费他所购买的商品;工人则提供这个商品,因为他的劳动能力的使用就是他的\textbf{劳动}本身。通过前一个交易,劳动本身变成了物质财富的一部分。工人完成这个劳动,但是他的这个劳动是\textbf{属于}资本的,从此以后,只是资本的一种职能而已。因此,这个劳动是在资本的直接监督和管理之下完成的;而这个劳动借以物化的产品,是资本借以表现的新形式,或者更确切地说,是资本实际上借以\textbf{实现}为资本的新形式。因此,劳动通过第一个交易已经\textbf{在形式上}被并入资本之后,在这个过程中,就直接\textbf{物化}为资本,\textbf{直接}转化为资本。在这里,转化为资本的劳动量,比以前用于购买劳动能力的资本量大。在这个过程中,一定量的无酬劳动被占有了,只是因为这个缘故,货币才转化为资本。

虽然这里事实上没有发生交换,可是,如果撇开中介不谈,我们看到,结果是,在这个过程中——把两个环节结合在一起——一定量的物化劳动同较大量的活劳动相交换。整个过程的结果表现为:物化在自己产品中的劳动,大于物化在劳动能力中的劳动,因而大于作为工资支付给工人的物化劳动;换句话说:过程的实际结果在于,资本家不仅收回了他花在工资上的那部分资本,而且得到了一个完全是无偿占有的剩余价值。劳动同资本的\textbf{直接}交换在这里的意思是:(1)劳动直接转化为资本,变成资本的物质组成部分,这个转化是在生产过程中完成的;(2)一定量的物化劳动与等量活劳动加一个\textbf{不经过交换}而占有的活劳动的追加量相交换。

“\textbf{生产劳动}是\textbf{直接}同\textbf{资本}交换的劳动”这个说法,包括上述所有环节,它不过是从下面这个论点派生出来的提法:生产劳动是这样的\textbf{劳动},它把货币转化为资本,它同作为\textbf{资本}的生产条件相交换;因而,它也决不是简单地作为不带特殊社会规定性的\textbf{劳动}同这些生产条件——在这里不是简单地只作为生产条件出现——发生关系。

这包括:(1)货币和劳动能力作为商品彼此对立的关系,货币所有者和劳动能力所有者之间的买和卖;(2)劳动直接隶属于资本;(3)劳动在生产过程中实际转化为资本,或者同样可以说,为资本创造剩余价值。这里发生了\textbf{劳动和资本之间的双重的交换}。第一种交换只表示对劳动能力的购买,所以,从实际结果来看,就是对劳动的购买,因而也是对劳动产品的购买。第二种交换是活劳动直接转化为资本,或者说,作为资本的实现的活劳动的物化。

\tsubsectionnonum{[(d)生产劳动对资本的特殊使用价值]}

资本主义生产过程的结果,既不是单纯的产品(使用价值),也不是\textbf{商品},即具有一定交换价值的使用价值。它的结果,它的产品,是为资本创造\textbf{剩余价值},因而,是货币或商品实际\textbf{转化}为资本;而在生产过程之前,货币或商品仅仅从自己的目的来说,从可能性来说,从自己的使命来说,才是资本。生产过程吸收的劳动量,比购买的劳动量大。在生产过程中完成的这种对别人无酬劳动的吸收、[1324]\textbf{占有},是资本主义生产过程的\textbf{直接目的};因为资本本身(因而资本家本身)的任务,既不是生产直接供自己消费的使用价值,也不是生产用来转化为货币再转化为使用价值的商品。资本主义生产的目的是\textbf{发财致富},是\textbf{价值的增殖},是价值的\textbf{增大},因而是保存原有价值并创造剩余价值。资本只有在同劳动交换(这种劳动因而被称为\textbf{生产劳动})之后,才能得到在资本主义生产过程中\textbf{生产出来的}这种\textbf{特殊的产品}。

用来生产\textbf{商品}的劳动必须是有用劳动,必须生产某种\textbf{使用价值},必须表现为某种\textbf{使用价值}。所以,只有表现为\textbf{商品}、也就是表现为使用价值的劳动,才是同资本交换的劳动。这是不言而喻的前提。但是,不是劳动的这种具体性质,不是劳动的使用价值本身,因而,不是由于劳动是例如裁缝的劳动、鞋匠的劳动、纺工的劳动、织工的劳动等等——不是这一点构成劳动对资本的特殊使用价值,不是这一点使劳动在资本主义生产体系中打上\textbf{生产劳动}的印记。构成劳动对资本的\textbf{特殊使用价值}的,不是劳动的一定的有用性质,也不是劳动借以物化的产品的特殊有用性质。劳动对资本的使用价值,是由这种劳动作为创造交换价值的因素的性质决定的,是由这种劳动固有的抽象劳动的性质决定的;但是,问题不在于劳动一般地代表着这种一般劳动的一定量,而在于劳动代表着一个比劳动价格即\textbf{劳动能力的价值所包含的}抽象劳动\textbf{量大的}抽象劳动量。

对资本来说,劳动能力的使用价值,在于劳动能力提供的劳动量超过物化在劳动能力本身因而为再生产劳动能力所需要的劳动量的余额。劳动当然是以它作为特殊的有用劳动(如纺纱劳动、织布劳动等等)所固有的\textbf{一定形式}被提供的。但是,劳动的这种使自己能表现为商品的具体性质,不是劳动对资本的\textbf{特殊使用价值}。对资本来说,劳动的这种特殊使用价值,在于劳动作为一般劳动所提供的劳动量,并且在于所完成的劳动量\textbf{超过}构成劳动报酬的劳动量的余额。

一定的货币额 x 变成资本,是由于它在它的产品中表现为 x+h,也就是由于作为产品的货币额所包含的劳动量,大于这个货币额原来包含的劳动量。而这是货币同生产劳动相交换的结果,换句话说,只有那种在同物化劳动交换时能使物化劳动表现为一个增大了的物化劳动量的劳动,才是\textbf{生产劳动}。

因此,资本主义生产过程并不单纯是商品生产。它是一个吸收无酬劳动的过程,是一个使生产资料(材料和劳动资料)变为吸收无酬劳动的手段的过程。

从上述一切可以看出,“生产劳动”是对劳动所下的同劳动的\textbf{一定内容},同劳动的特殊效用或劳动所借以表现的特殊使用价值绝对没有任何直接关系的定义。

\textbf{同一}种劳动可以是\textbf{生产劳动},也可以是\textbf{非生产劳动}。

例如,密尔顿创作《失乐园》得到 5 镑,他是\textbf{非生产劳动者}。相反,为书商提供工厂式劳动的作家,则是\textbf{生产劳动者}。密尔顿出于同春蚕吐丝一样的必要而创作《失乐园》。那是\textbf{他的}天性的能动表现。后来,他把作品卖了 5 镑。但是,在书商指示下编写书籍(例如政治经济学大纲)的莱比锡的一位无产者作家却是\textbf{生产劳动者},因为他的产品从一开始就从属于资本,只是为了增加资本的价值才完成的。一个自行卖唱的歌女是\textbf{非生产劳动者}。但是,同一个歌女,被剧院老板雇用,老板为了赚钱而让她去唱歌,她就是\textbf{生产劳动者},因为她生产资本。

\tsubsectionnonum{[(e)非生产劳动是提供服务的劳动。资本主义条件下对服务的购买。把资本和劳动的关系看成服务的交换的庸俗观点]}

[1325]这里产生了不同的问题,不能混为一谈。

我买一条现成的裤子呢,还是买布请一个裁缝到家里来做一条裤子,我对他的\textbf{服务}(即他的缝纫劳动)支付报酬,——这对我是完全无关紧要的,因为对我来说,重要的是裤子本身。我不请裁缝到家里来,而是到服装商人那里去买裤子,是因为前一种方式花费大,而缝纫业资本家生产的裤子,比裁缝在我家做的裤子,花费的劳动少,也就便宜。但是在这两种情况下,我都不是把我用来买裤子的货币变成资本,而是变成裤子;在这两种情况下,对我来说,都是把货币单纯用作流通手段,即把货币转化为一定的使用价值。因此,虽然在一种情况下,货币同\textbf{商品}交换,在另一种情况下,货币购买作为\textbf{商品}的\textbf{劳动}本身,但是,货币在这里都不是执行资本的职能。它只是执行货币的职能,确切些说,执行流通手段的职能。

另一方面,那个在我家里劳动的裁缝不是\textbf{生产劳动者},虽然他的劳动给我提供产品——裤子,而给他自己提供他的劳动的价格——货币。可能,这个裁缝提供的劳动量,大于他从我这里得到的报酬中包含的劳动量;这甚至是完全可能的,因为他的劳动的价格,是由作为\textbf{生产劳动者}的裁缝所取得的价格决定的。但是,这对我是完全无关紧要的。价格一经确定之后,他劳动 8 小时还是劳动 10 小时,对我都一样。对我来说,有意义的只是\textbf{使用价值}——裤子,并且,不论我用前一种方式或后一种方式购买裤子,我所关心的当然是尽量少支付;在第一种情况下和第二种情况下,我同样关心的是:\textbf{我支付的价格在两种情况下,都不应该超过正常价格}。这是用于我的消费的一笔\textbf{支出},这不是我的货币的增加,倒是我的货币的减少。这决不是发财致富的手段,正如用于我\textbf{个人消费}的任何一笔货币支出,都不是发财致富的手段一样。

保尔·德·科克小说中的一位“学者”会对我说,如果没有这种购买,就象不购买面包一样,我就不能生活,因而也就不能\textbf{发财致富}了;因此,这种购买是我发财致富的一个间接手段,或者说,至少是一个条件。根据同样的理由,可以认为我的血液循环、我的呼吸过程也是我发财致富的条件。但是,无论我的血液循环,还是我的呼吸过程,就其本身而论,都决不能使我发财致富,相反,两者都是以代价昂贵的新陈代谢为前提的,如果完全不需要这种新陈代谢,世界上也就没有穷人了。因此,货币和劳动之间的单纯的、\textbf{直接的}交换,既不会使货币转化为资本,也不会使劳动转化为生产劳动。

什么是这种交换的最大特点呢\fontbox{?}这种交换与货币和生产劳动之间的交换有什么不同呢\fontbox{?}不同之处在于,一方面,这里\textbf{货币}是\textbf{作为货币},作为交换价值的独立形式支出的,这个交换价值应该转化为某种\textbf{使用价值},生活资料,个人消费品。在这里,货币不变成资本,相反,为了作为使用价值来消费,它不再作为交换价值而存在。另一方面,在这里,劳动只是作为使用价值,作为把布做成裤子的\textbf{服务},作为依靠它的一定有用性质给我提供的服务,才使我感到兴趣。

相反,同一个裁缝向雇用他的缝纫业资本家提供的服务,决不在于他把布做成裤子,而在于物化在裤子中的必要劳动时间等于 12 小时,而裁缝所得的工资只等于 6 小时。因此,裁缝向资本家提供的服务,在于他无偿地劳动了 6 小时。这件事以缝制裤子的形式出现,只是\textbf{掩盖}了实际的关系。因此,缝纫业资本家一有可能就设法把裤子再转化为货币,就是说,转化为这样一种形式,在这种形式中,缝纫劳动的一定性质完全消失,而已经提供的服务就不是表现为[1326]由一定货币额代表的 6 小时劳动时间,而是表现为由加倍的货币额代表的 12 小时劳动时间。

我购买缝纫劳动,是为了取得它作为\textbf{缝纫劳动}所提供的服务,来满足我穿衣的需要,也就是为我的一种\textbf{需要}服务。缝纫业资本家购买缝纫劳动,是把它当作使一个塔勒变成两个塔勒的手段。我购买缝纫劳动,是因为它生产一定的使用价值,提供一定的服务。资本家购买缝纫劳动,是因为它提供的交换价值额大于花在它上面的费用,也就是说,是因为它对资本家说来,纯粹是一个用较少劳动交换较多劳动的手段。

凡是货币直接同不生产资本的劳动即\textbf{非生产}劳动相交换的地方,这种劳动都是作为\textbf{服务}被购买的。服务这个名词,一般地说,不过是指这种劳动所提供的特殊使用价值,就象其他一切商品也提供自己的特殊使用价值一样;但是,这种劳动的特殊使用价值在这里取得了“服务”这个特殊名称,是因为劳动不是作为\textbf{物},而是作为\textbf{活动}提供服务的,可是,这一点并不使它例如同某种机器(如钟表)有什么区别。我给为了你做,我做为了你做,我做为了你给,我给为了你给,\endnote{这是罗马法上的契约关系的四种公式。原文是:Doutfacias,facioutfacias,facioutdes,doutdes。参看马克思《资本论》第 1 卷第 17 章。——第 435 页。}在这里是同一关系的、意义完全相同的几种形式,而在资本主义生产中,我给为了你做这个形式所表示的,是被付出的具有物的形式的价值同被占有的活的活动之间的极为特殊的关系。因此,既然对\textbf{服务}的购买中完全不包含劳动和资本的特殊关系(在这里,这个关系或者完全消失了,或者根本不存在),那末,对服务的购买,自然成为萨伊和巴师夏之流最喜欢用来表现\textbf{资本和劳动之间的关系}的形式。

这些服务的\textbf{价值}如何确定,这个\textbf{价值}本身如何由工资规律决定,这是同我们这里研究的关系完全无关的问题,这个问题要在工资那一章考察。

由此可见,单是货币同劳动的交换,还不能使劳动转化为\textbf{生产劳动},另一方面,这种劳动的\textbf{内容}最初是无关紧要的。

工人自己可以购买劳动,就是购买以服务形式提供的商品,他的工资花在这些服务上,同他的工资花在购买其他任何商品上,是没有什么不同的。他购买的服务,可以是相当必要的,也可以是不太必要的:例如,他可以购买医生的服务,也可以购买牧师的服务,就象他可以购买面包,也可以购买烧酒一样。工人作为买者,即作为同商品对立的货币的代表,同仅仅作为买者出现,即仅仅把货币换成商品形式的资本家,完全属于同一个范畴。这些服务的价格怎样决定,这种价格同真正的工资有什么关系,它在什么程度上受工资规律的调节,在什么程度上不受工资规律的调节,这些问题,应当在研究工资时加以考察,同当前的研究完全无关。

可见,如果说单是货币同劳动的交换还不能使劳动转化为\textbf{生产劳动},或者同样可以说,还不能使货币转化为资本,那末,劳动的\textbf{内容}、它的具体性质、它的特殊效用,看来最初也是无关紧要的:我们前面已经看到,同一个裁缝的同样的劳动,在一种情况下表现为生产劳动,在另一种情况下却表现为非生产劳动。

某些\textbf{服务},或者说,作为某些活动或劳动的结果的\textbf{使用价值},体现为\textbf{商品},相反,其他一些服务却不留下任何可以捉摸的、同提供这些服务的人\textbf{分开存在的}结果,或者说,其他一些服务的结果不是\textbf{可以出卖的商品}。例如,一个歌唱家为我提供的服务,满足了我的审美的需要;但是,我所享受的,只是同歌唱家本身分不开的活动,他的劳动即歌唱一停止,我的享受也就结束;我所享受的是活动本身,是它引起的我的听觉的反应。这些服务本身,同我买的商品一样,可以是确实必要的,或者仅仅看来是必要的,例如士兵、医生和律师的服务,——或者它们可以是给我提供享受的服务。但是,这丝毫不改变它们的经济性质。如果我身体健康,用不着医生,或者我有幸不必去打官司,那我就会象避开瘟疫一样,避免把货币花在医生或律师的服务上。

[1328]\endnote{这里马克思把手稿页码“1327”误写为“1328”。——第 437 页。}有些\textbf{服务}也可以是强加于人的,例如\textbf{官吏的服务}等等。

如果我自己购买,或者别人为我购买一个教师的服务,其目的不是发展我的才智,而是让我学会赚钱的本领,而我又真的学到了一些东西(这件事就它本身来说,完全同对于教师的服务支付报酬无关),那末,这笔学费同我的生活费完全一样,应归入我的劳动能力的生产费用。但是,这种服务的特殊效用\textbf{丝毫不改变现有的经济关系};在这里,货币没有转化为资本,换句话说,我对这个提供服务的人即教师来说,并没有成为\textbf{资本家},没有成为他的主人。因此,医生是否把我的病治好了,教师的教导是否有成效,律师是否使我打赢了官司,对于这种关系的\textbf{经济性质}来说,也完全是无关紧要的。在这里,被支付报酬的是服务本身,而就服务的性质来说,其结果是不能由提供服务的人保证的。很大一部分\textbf{服务}的报酬,属于同商品的\textbf{消费}有关的费用,如女厨师、女佣人等等的服务。

一切\textbf{非生产劳动}的特点是,支配多少非生产劳动——象购买其他一切供消费的商品的情况一样——是同剥削多少\textbf{生产工人}成比例的。因此,\textbf{生产工人}支配非生产劳动者的\textbf{服务}的可能性,比一切人都要少,虽然他们对\textbf{强加于}他们的服务(国家、赋税)支付报酬最多。相反,我使用\textbf{生产工人}的劳动的可能性,同我使用\textbf{非生产劳动者}的劳动决不是成比例地增长;相反,这里是成反比例。

对我来说,甚至\textbf{生产工人}也可以是\textbf{非生产劳动者}。例如,如果我请人来把我的房子裱糊一下,而这些裱糊工人是完成我的这项定货的老板的雇佣工人,那末,这个情况,对我来说,就好比是我买了一所裱糊好的房子,也就是说,好比是我把货币花费在一个供我消费的商品上。可是,对于叫这些工人来裱糊的那位老板来说,他们是生产工人,因为他们为他生产剩余价值。[1328]

\centerbox{※     ※     ※}

[1333]如果一个工人虽然生产了可以出卖的商品,但是,他生产的数额仅仅相当于他自己的劳动能力的价值,因而没有为资本生产出剩余价值,那末,从资本主义生产的观点看来,这种工人\textbf{不是生产的},这一点,从李嘉图的话里已经可以看出来,他说,这种人的存在本身就是一个累赘。\endnote{马克思指李嘉图的《政治经济学和赋税原理》第二十六章(《论总收入与纯收入》)。——第 408、438 页。}这就是资本的理论和实践:

\begin{quote}“关于资本的理论,以及\textbf{使劳动停在}除工人生活费用之外还能为资本家生产\textbf{利润的那个点上的实践},看来,都是同调节生产的自然法相违背的。”(\textbf{托·霍吉斯金}《通俗政治经济学》1827 年伦敦版第 238 页)[1333]\end{quote}

\centerbox{※     ※     ※}

[1336]我们已经看到:资本主义生产过程不仅是\textbf{商品}的生产过程,而且是\textbf{剩余价值}的生产过程,是剩余劳动的吸收,因而是资本的生产过程。货币和劳动之间或者说资本和劳动之间的最初的、形式上的交换行为,仅仅\textbf{从可能性来说}是通过物化劳动对别人的活劳动的占有。实际的占有过程只是在实际的生产过程中才完成的,对这个生产过程说来,资本家和工人\textbf{单纯作为商品所有者}相互对立、作为买者和卖者彼此发生关系的那种最初的、形式上的交易,已经是过去的阶段了。因此,一切庸俗经济学家,例如巴师夏,都只停留在这种最初的、形式上的交易上,其目的正是要用欺骗手法摆脱特殊的资本主义关系。在货币同非生产劳动的交换中,这个区别就十分清楚地表现出来了。这里,货币和劳动\textbf{只}作为商品相互交换。因此,这种交换不形成资本,这种交换是\textbf{收入的支出}。[1336]

\tsubsectionnonum{[(f)资本主义社会中手工业者和农民的劳动]}

[1328]那些不雇用工人因而不是作为资本家来进行生产的独立的手工业者或农民的情况又怎样呢\fontbox{?}他们——这是农民的典型情况\fontbox{~\{}但是,比方说,我在家里雇用的园丁却不是这样\fontbox{\}~}——可以是\textbf{商品生产者},而我向他们购买\textbf{商品},至于手工业者按定货供应商品,农民按自己资金的多少供应商品,这些情况并不会使问题有丝毫改变。在这种场合,他们是作为商品的卖者,而不是作为劳动的卖者同我发生一定的关系,所以,这种关系与资本和劳动之间的交换毫无共同之处,因此,在这里也就用不上\textbf{生产劳动}和\textbf{非生产劳动}的区分——这种区分的基础在于,劳动是同作为货币的货币相交换,还是同作为资本的货币相交换。因此,农民和手工业者虽然也是商品生产者,却既不属于\textbf{生产劳动者}的范畴,又不属于\textbf{非生产劳动者}的范畴。但是,他们是自己的生产不从属于资本主义生产方式的商品生产者。

可能,这些用自己的生产资料进行劳动的生产者,不仅再生产了自己的劳动能力,而且创造了剩余价值,并且,他们的地位容许他们占有自己的剩余劳动或剩余劳动的一部分(因为其余部分以税收等形式从他们那里拿走了)。这里我们遇到的是这样一个社会所具有的特点,在这个社会中一定的生产方式占支配地位,但是还不是这个社会的一切生产关系都从属于它。例如,在封建社会中,连那些同封建主义的实质相距很远的关系,也具有封建的外貌(这一点我们在英国可以看得最清楚,因为封建制度是现成地从诺曼底移进英国的,而它的形式给英国固有的、许多方面都和它不同的社会制度打上了印记)。例如,同领主和陪臣相互间的亲身服务毫无关系的单纯货币关系,也具有了封建的外貌。关于小农凭封地权占有自己土地的虚构,也是一个例证。

在资本主义生产方式下,情况也完全一样。独立农民或手工业者分裂为两重身分。[注:“在小企业中……\textbf{企业主}常常是他\textbf{自己的工人}。”(\textbf{施托尔希}[《政治经济学教程》]彼得堡版第 1 卷第 242 页)]作为生产资料的所有者,他是资本家;作为劳动者,他是他自己的雇佣工人。因此,他作为资本家,自己给自己支付工资,从自己的资本中取得利润,就是说,剥削他自己这个雇佣工人,他以剩余价值的形式向自己支付那应由劳动向资本交付的贡物。同样,也许他还向作为土地所有者的自己支付第三个部分(地租),就象我们后面要看到的\endnote{见马克思《资本论》第 3 卷第 23 章。——第 440 页。}工业资本家那样,工业资本家在自己企业中使用自己的[1329]资本,向自己支付利息,并且把这种利息看成他不是作为工业资本家而是单纯作为资本家所应得的东西。

在资本主义生产中,生产资料(它们表现一定的\textbf{生产关系})所具有的\textbf{社会规定性}同生产资料本身的物质存在是这样地结合在一起,而在资产阶级社会的观念中,这种社会规定性同这种物质存在是这样地不可分离,以致这种社会规定性(即范畴的规定性)甚至也被用到同它直接矛盾的那些关系上去了。生产资料只有当它独立化,作为独立的力量来反对劳动的时候,才成为资本。而在我们所考察的场合,生产者——劳动者——是自己的生产资料的占有者、所有者。因此,这些生产资料不是资本,而劳动者也不是作为雇佣工人同这些生产资料相对立。然而,这些生产资料被看作资本,而劳动者自己分裂为两重身分,结果就是\textbf{他}作为资本家来雇用他自己这个工人。

这种表现方式,初看起来虽然很不合理,可是从下述意义来看,实际上还是表现了某种正确的东西。在我们考察的场合,生产者的确创造他自己的剩余价值\fontbox{~\{}假定生产者按商品的价值出卖他的商品\fontbox{\}~},或者说,物化在他的全部产品中的,只是他自己的劳动。但是,他能够\textbf{自己}占有他自己劳动的全部产品,他的产品价值超过他的譬如一天劳动的平均价格的余额没有被第三者即\textbf{老板}占有,这并不是靠他的劳动(就这方面来说,他同其他工人毫无区别),而是仅仅靠他占有生产资料。因此,仅仅由于他是生产资料所有者,他自己的剩余劳动才归他所有,从这个意义上说,他作为他自己的资本家同他自己这个雇佣工人发生关系。

在现在这个社会中,\textbf{分离}表现为正常的关系。因此,在实际上没有分离的地方,也假定有分离,并且象刚才已经指出的,这在一定意义上是正确的;因为(和例如古罗马、挪威以及美国西北部的社会关系不同)在这里,\textbf{结合}表现为某种偶然的东西,而\textbf{分离}却表现为某种正常的东西,因此,即使在各种不同的职能结合在一个人身上的地方,分离还是被作为一定的关系来坚持。这里表现得非常明显:资本家本身不过是资本的职能,工人本身不过是劳动能力的职能。并且这是一条规律:在经济发展过程中,这些职能分配在不同的人身上,而且用自己的生产资料进行生产的手工业者或农民,不是逐渐变成剥削别人劳动的小资本家,就是丧失自己的生产资料\fontbox{~\{}最常见的是后一种情况,即使他仍然是生产资料的\textbf{名义上的}所有者,例如农民在抵押借款的时候就是这样\fontbox{\}~},变成雇佣工人。这是资本主义生产方式占支配地位的社会形式中的发展趋势。

\tsubsectionnonum{[(g)关于生产劳动的补充定义:生产劳动是物化在物质财富中的劳动]}

因此,在考察资本主义生产的本质关系时,可以假定\fontbox{~\{}因为资本主义生产越来越接近这个情况;因为这是过程的基本方向,而且只有在这种情况下,劳动生产力的发展才达到最高峰\fontbox{\}~},整个商品世界,物质生产即物质财富生产的一切领域,都(在形式上或者实际上)从属于资本主义生产方式。这个假定表示上述过程的极限,并且越来越接近于现实情况的正确表述。按照这个假定,一切从事商品生产的工人都是雇佣工人,而生产资料在所有物质生产领域中,都作为资本同他们相对立。在这种情况下,可以认为,\textbf{生产工人}即生产资本的工人的特点,是他们的劳动物化在\textbf{商品}中,物化在物质财富中。这样一来,\textbf{生产劳动},除了它那个与\textbf{劳动内容}完全无关、不以劳动内容为转移的具有决定意义的特征之外,又得到了与这个特征不同的第二个定义,补充的定义。

\tsubsectionnonum{[(h)非物质生产领域中的资本主义表现]}

在非物质生产中,甚至当这种生产纯粹为交换而进行,因而纯粹生产\textbf{商品}的时候,也可能有两种情况:

(1)生产的结果是\textbf{商品},是使用价值,它们具有离开生产者和消费者而独立的形式,因而能在生产和消费之间的一段时间内存在,并能在这段时间内作为\textbf{可以出卖的商品}而流通,如书、画以及一切脱离艺术家的艺术活动而单独存在的艺术作品。在这里,资本主义生产只是在很有限的规模上被应用,例如,一个作家在编一部集体著作百科全书时,把其他许多作家当作助手来剥削。[1330]这里的大多数情况,都还只局限于\textbf{向资本主义生产过渡的形式},就是说,从事各种科学或艺术的生产的人,工匠或行家,为书商的总的商业资本而劳动,这种关系同真正的资本主义生产方式无关,甚至在形式上也还没有从属于它。在这些过渡形式中,恰恰对劳动的剥削最厉害,但这一点并不改变事情的本质。

(2)产品同生产行为不能分离,如一切表演艺术家、演说家、演员、教员、医生、牧师等等的情况。在这里,资本主义生产方式也只是在很小的范围内能够应用,并且就事物的本性来说,只能在某些领域中应用。例如,在学校中,教师对于学校老板,可以是纯粹的雇佣劳动者,这种教育工厂在英国多得很。这些教师对学生来说虽然不是\textbf{生产工人},但是对雇佣他们的老板来说却是生产工人。老板用他的资本交换教师的劳动能力,通过这个过程使自己发财。戏院、娱乐场所等等的老板也是用这种办法发财致富。在这里,演员对观众说来,是艺术家,但是对自己的企业主说来,是\textbf{生产工人}。资本主义生产在这个领域中的所有这些表现,同整个生产比起来是微不足道的,因此可以完全置之不理。

\tsubsectionnonum{[(i)从物质生产总过程的角度看生产劳动问题]}

在特殊的资本主义生产方式中,许多工人共同生产同一个商品;随着这种生产方式的发展,这些或那些工人的劳动同生产对象之间直接存在的关系,自然是各种各样的。例如,前面提到过的那些工厂小工\endnote{马克思在同一稿本(第 XXI 本)第 1308 页写了关于工厂小工的劳动。——第 443 页。},同原料的加工毫无直接关系;监督直接进行原料加工的工人的那些监工,就更远一步;工程师又有另一种关系,他主要只是从事脑力劳动,如此等等。但是,\textbf{所有这些}具有不同价值的劳动能力(虽然使用的劳动量大致是在同一水平上)的\textbf{劳动者的总体}进行生产的结果——从单纯的劳动过程的\textbf{结果}来看——表现为\textbf{商品}或一个\textbf{物质产品}。所有这些劳动者合在一起,作为一个生产集体,是生产这种\textbf{产品}的活机器,就象从整个生产过程来看,他们用自己的劳动同资本交换,把资本家的货币作为资本再生产出来,就是说,作为自行增殖的价值,自行增大的价值再生产出来。

资本主义生产方式的特点,恰恰在于它把各种不同的劳动,因而也把脑力劳动和体力劳动,或者说,把以脑力劳动为主或者以体力劳动为主的各种劳动分离开来,分配给不同的人。但是,这一点并不妨碍物质产品是所有这些人的\textbf{共同劳动的产品},或者说,并不妨碍他们的共同劳动的产品体现在物质财富中;另一方面,这一分离也丝毫不妨碍:这些人中的每一个人对资本的关系是雇佣劳动者的关系,是在这个特定意义上的\textbf{生产工人}的关系。所有这些人不仅\textbf{直接}从事物质财富的生产,并且用自己的劳动\textbf{直接}同作为资本的货币交换,因而不仅把自己的工资再生产出来,并且还直接为资本家创造剩余价值。他们的劳动是由有酬劳动加无酬的剩余劳动组成的。

\tsubsectionnonum{[(k)运输业是一个物质生产领域。运输业中的生产劳动]}

除了采掘工业、农业和加工工业以外,还存在着第四个物质生产领域,这个领域在自己的发展中,也经历了几个不同的生产阶段:手工业生产阶段、工场手工业生产阶段、机器生产阶段。这就是\textbf{运输业},不论它是客运还是货运。在这里,\textbf{生产劳动}对资本家的关系,也就是说,雇佣工人对资本家的关系,同其他物质生产领域是完全一样的。其次,在这里,劳动对象发生某种物质变化——\textbf{空间的}、位置的变化。至于客运,这种位置变化只不过是企业主向乘客提供的\textbf{服务}。但是,这种\textbf{服务}的买者和卖者的关系,就象纱的卖者和买者的关系一样,同生产工人对资本的关系是毫无共同之处的。

如果我们就商品来考察这个过程,那末[1331]这里在劳动过程中,劳动对象,\textbf{商品},确实发生了某种变化。它的位置改变了,从而它的使用价值也起了变化,因为这个使用价值的位置改变了。商品的交换价值增加了,增加的数量等于使商品的使用价值发生这种变化所需要的劳动量。这个劳动量,一部分决定于不变资本的消耗,即加入商品的物化劳动量,另一部分决定于活劳动量,这同其他一切商品的价值增殖过程的情况是一样的。

商品一到达目的地,它的使用价值所发生的这个变化也就消失,这个变化只表现为商品的交换价值提高了,商品变贵了。虽然在这里,实在劳动在使用价值上没有留下一点痕迹,可是这个劳动已经实现在这个物质产品的交换价值中。可见,凡是适用于其他一切物质生产领域的,同样适用于运输业:在这个领域里,劳动也体现在\textbf{商品}中,虽然它在商品的使用价值上并不留下任何可见的痕迹。

\centerbox{※     ※     ※}

我们在这里研究的还只是\textbf{生产资本},就是说,还只是用于\textbf{直接生产过程}中的资本。后面我们还要谈到\textbf{流通过程}中的资本。只有到后面研究资本作为\textbf{商业资本}所采取的特殊形式时,才能答复这样的问题:商业资本所雇用的工人在什么范围内是生产的,在什么范围内是非生产的。\endnote{见马克思《资本论》第 2 卷第 6 章和第 3 卷第 17 章。——第 445 页。}[XXI—1331]

\tsectionnonum{[(13)《资本论》第一部分和第三部分的计划草稿]}

\tsubsectionnonum{[(a)《资本论》第一部分或第一篇的计划]}

[XVIII—1140]第一篇\endnote{马克思把《资本论》的三个理论部分最初称为“章”,然后称为“篇”,最后称为“册”。参看注 13。——第 446 页。}——《\textbf{资本的生产过程}》——分为:\endnote{这些计划草稿马克思写于 1863 年 1 月。它们在 1861—1863 年手稿第 XVIII 本论舍尔比利埃和理查·琼斯两章的正文之间(手稿上用粗体的方括号把它们同这两章的正文隔开)。——第 446 页。}

(1)导言:商品,货币。

(2)货币转化为资本。

(3)绝对剩余价值:(a)劳动过程和价值增殖过程;(b)不变资本和可变资本;(c)绝对剩余价值;(d)争取正常工作日的斗争;(e)同一时间的工作日(同时雇用的工人人数)。剩余价值额和剩余价值率(大小和高低\fontbox{?})。

(4)相对剩余价值:(a 简单协作;(b)分工;(c)机器等等。

(5)绝对剩余价值和相对剩余价值的结合。雇佣劳动和剩余价值的比例。劳动对资本的形式上的隶属和实际上的隶属。资本的生产性。生产劳动和非生产劳动。

(6)剩余价值再转化为资本。原始积累。威克菲尔德的殖民学说。

(7)生产过程的结果。

(占有规律的表现中的变革可以在第 6 点或第 7 点中考察。)

(8)剩余价值理论。

(9)关于生产劳动和非生产劳动的理论。[XVIII—1140]

\tsubsectionnonum{[(b)《资本论》第三部分或第三篇的计划]}

[XVIII—1139]第三篇——《\textbf{资本和利润}》——分为:

(1)剩余价值转化为利润。不同于剩余价值率的利润率。

(2)利润转化为平均利润。一般利润率的形成。价值转化为生产价格。

(3)亚·斯密和李嘉图关于利润和生产价格的理论。

(4)地租(价值和生产价格的区别的例解)。

(5)所谓李嘉图地租规律的历史。

(6)利润率下降的规律。亚·斯密、李嘉图、凯里。

(7)利润理论。

(问题:是不是还应该把西斯蒙第和马尔萨斯包括在《剩余价值理论》里\fontbox{?})

(8)利润分为产业利润和利息。商业资本。货币资本。

(9)收入及其源泉。这里也包括生产过程和分配过程之间的关系问题。

(10)资本主义生产总过程中货币的回流运动。

(11)庸俗政治经济学。

(12)结论。资本和雇佣劳动。[XVIII—1139]

\tsubsectionnonum{[(c)《资本论》第三部分第二章的计划]}

[XVIII—1109]研究《\textbf{资本和利润}》的第三部分第二章论述一般利润率的形成。这里要研究以下几个问题:

(1)资本的不同的有机构成。它部分是由可变资本和不变资本之间的[比例的]差别决定的,因为这个差别是从一定的\textbf{生产发展阶段}产生的,是从机器和原料同推动它们的劳动量之间的绝对的\textbf{数量上}的比例产生的。这些差别同劳动过程有关。同样,在这里还必须考察从流通过程产生的固定资本和流动资本的差别,考察它们如何使一定时期内不同领域中的资本的价值增殖发生变化。

(2)不同资本的组成部分的价值比例的差别,这些差别不是由不同资本的有机构成产生的。这主要是由原料价值的差别产生的,即使假定在两个不同的领域中,原料吸收的劳动量相等。

(3)在资本主义生产的不同领域中,由于上述差别而产生的利润率的差异。不同领域中的利润率相等,以及利润量同所使用的资本量成正比,这只有对同一构成的资本来说才是正确的。

(4)第一章论述的一切,适用于总资本。在资本主义生产中,每一个资本都作为总资本的一部分,作为它的某个份额出现。一般利润率的形成(竞争)。

(5)价值转化为生产价格。价值、成本价格和生产价格之间的差别。

(6)为了还要包括对李嘉图关于这个论题的观点的分析,补充以下一点:工资的一般变动对一般利润率的影响,从而对生产价格的影响。[XVIII—1109]

\tchapternonum{[第八章]洛贝尔图斯先生。新的地租理论(插入部分)}

\tsectionnonum{[(1)农业中的超额剩余价值。在资本主义条件下,农业的发展比工业慢]}

[\endnote{马克思结束了属于斯密部分的篇幅很大的一章《关于生产劳动和非生产劳动的理论》,并写完了对重农学派部分具有补充性质的三章(论奈克尔,论魁奈的《经济表》和论兰盖)之后,按照自己的计划,应该着手写李嘉图部分。但是,马克思没有立即着手进行这一工作。在论兰盖这一章之后,他开始写论布雷的一章,——显然,这与马克思在《兰盖》那一章提到他打算“在以后”论述“少数几个社会主义著作家”(见本卷第1册第367页)有关。按照这个意图,马克思在手稿第X本的目录计划中,把原来写的《f》这一章的标题(这个标题紧接《(e)兰盖》一章的标题)《李嘉图》删去,改为《布雷》(见同上,第4页)。但是,论布雷的一章没有写完。后来,马克思决定把对布雷观点的分析移到论政治经济学家的无产阶级反对派这一章中去(见本卷第3册)。当马克思着手写论布雷一章时,他想从下一章《g》起开始写论李嘉图部分。但是,这一次李嘉图的名字又从目录中删掉了,作为《g》章出现的是《插入部分》,标题是《洛贝尔图斯先生。新的地租理论》。马克思是在1862年6月写作论洛贝尔图斯一章的。拉萨尔在1862年6月9日给马克思的信里,请马克思在最近把洛贝尔图斯论地租的书还给他。这显然成了马克思立即写作论洛贝尔图斯这一章的外因。但是,还有必须首先分析批判洛贝尔图斯的地租理论的重要内因。从马克思的书信中可以看出,马克思在这个时候已经把李嘉图的地租理论究竟错在哪里的问题完全弄清楚了。马克思认为,李嘉图地租理论的根本缺点之一是其中没有绝对地租这个概念。洛贝尔图斯在他给冯·基尔希曼的第三封《社会问题书简》中企图阐明这个概念。马克思在着手专门分析李嘉图的地租理论之前,在这篇《插入部分》中对洛贝尔图斯的这个企图作了详尽的分析批判。——第3页。}X—445]洛贝尔图斯先生。洛贝尔图斯《给冯·基尔希曼的第三封信:驳李嘉图的地租学说,对新的地租理论的论证》1851年柏林版。

事先应该作以下说明。如果我们说,必要工资等于10小时,那末,这句话最简单的解释就是这样:如果,平均地说,10小时劳动(也就是等于10小时的货币额)使农业短工能够购买他所必需的一切生活资料——农产品、工业品等等,那末,10小时劳动也就是非熟练劳动的平均工资。因而,这里所说的是工人一日产品中必须归他的那一部分产品的价值。这个价值最初以他所生产的商品形式,也就是作为这种商品的一定量存在,他可以用这个量——在扣除了他自己消费的那一部分(如果他消费这个商品的话)之后——换得他所需要的生活资料。因此,在这里,对于他的必要“收入”说来,工业、农业等等也具有意义,而不只是他自己生产的使用价值才具有意义。但是,商品的概念本身就包含了这一点。工人生产的是商品,不简单是产品。所以,关于这一点就不用多说了。

洛贝尔图斯先生首先研究在土地占有和资本占有还没有分离的国家中是什么情况,并且在这里得出重要的结论说:租(他所谓租,是指全部剩余价值)只等于无酬劳动,或无酬劳动借以表现的产品量。

首先要注意,洛贝尔图斯所指的只是相对剩余价值的增长,就是说,只是由于劳动生产率提高而产生的剩余价值的增长,而不是由于工作日本身延长而产生的剩余价值的增长。自然,任何绝对剩余价值,从一定意义上来说,都是相对的。劳动必须有足够的生产率,使工人不致为了维持自己的生活而用去全部时间。但是,区别也就从这里开始了。并且,如果说,最初劳动生产率很低,那末,需要也非常简单(如奴隶的情况),主人自己的生活比仆人的生活好得不太多。一个食利的寄生者出现所必需的相对劳动生产率是很低的。如果我们在劳动生产率还很低、机器和分工等等还没有采用的地方,看到有高的利润率,那末,这只有用以下几种情况来解释:或者,象在印度那样,工人的需要极低,而工人本人甚至还被压到这个极低的需要水平以下,另一方面,劳动生产率低,也就是固定资本对花费在工资上的那部分资本的比例小,换句话说,花费在劳动上的那部分资本对总资本的比例大;或者,劳动时间极度延长。后面这种情况,则发生在那些已经存在着资本主义生产方式,但是要同发达得多的国家竞争的国家(例如,奥地利等)。在这里,工资可能很低——部分是因为工人的需要比较不发展,部分是因为农产品按比较便宜的价格出卖,或者对于资本家同样可以说农产品的货币价值比较小。在劳动生产率低的情况下,在例如10小时必要劳动时间中生产出来的、用于支付工人工资的产品量也是小的。但是,如果工人不是工作12小时而是工作17小时,这就可以[为资本家]弥补低的劳动生产率。总之,不应该因为在某个国家中劳动的相对价值随该国劳动生产率的增长而下降,就认为在不同国家中工资与劳动生产率成反比。情况恰恰相反。在世界市场上一个国家同其他国家相比,生产率越高,它的工资也就越高。在英国,不仅名义工资比大陆高,实际工资也比大陆高。工人吃较多的肉,满足较多的需要。可是,这只适用于工业工人,不适用于农业工人。不过,英国的工资高的程度,没有达到英国工人的生产率超过其他国家工人的生产率的程度。

由于农业工人的平均工资低于工业工人的平均工资,地租(也就是说,土地所有权的现代形式)已经成为可能,这是撇开由土地肥力不同引起的地租差别而单单就地租的存在本身说的。因为,在这里,资本家起初按照传统(因为旧时代的租地农民变成资本家早于资本家变成租地农场主),一开始就从他的收入中拿出一部分来交给土地所有者,所以他就把工资压到水平以下,来弥补自己的损失。随着工人从农村外逃,工资必然上涨,实际上也上涨了。但是,当这种压力几乎还没有感觉出来的时候,机器等等就被采用了,农村中又形成了(相对的)人口过剩(请看英国)。尽管劳动时间没有延长,劳动生产力也没有发展,剩余价值可以由于工资压到传统水平以下而增加。凡是以资本主义方式经营农业生产的地方,实际上都是这种情况。在这一点不能靠机器做到的地方,就靠把耕地变为牧羊场来做到。因此,这里已经有了地租的可能性,[446]因为实际上农业工人的工资不等于平均工资。地租存在的这种可能性,完全不取决于产品价格——假定它等于产品价值。

地租的第二种增加,即价格不变,地租由于产品增加而增加,李嘉图也知道,但是没有加以考虑,因为他是按每夸特,而不是按每英亩计算地租的。他不会因为每夸特2先令的20夸特比每夸特2先令的10夸特多,或者比每夸特3先令的10夸特多,就说地租增加了(按这种方式,即使价格下降,地租也可以增加)。

此外,不论怎样解释地租本身,农业同工业比起来,仍然存在着重大差别:超额剩余价值的产生,在工业中是由于产品的生产较便宜,而在农业中是由于产品的生产较贵。如果1磅棉纱的平均价格等于2先令,而我能够用1先令把它生产出来,那末,为了争取销路,我一定会把它按1+(1/2)先令,或者至少低于2先令出卖。这样做甚至是绝对必需的。因为较便宜的生产是以较大规模的生产为前提的。这样一来,我就使市场商品充斥(同以前相比)。我必须出卖比以前多的东西。如果1磅棉纱只花费我1先令,那末,这是由于我现在生产比如说10000磅,而以前只生产8000磅。其所以便宜,只是因为固定资本分摊到10000磅上去了。如果我只出卖8000磅,那末,机器的损耗就会使每1磅的价格提高1/5,即20%。因此,我为了能够出卖10000磅,就以低于2先令的价格[比如说,按1+(1/2)先令]出卖。这样我仍然得到1/2先令的超额利润,也就是我的产品价值1先令(已经包括普通利润在内)的50%。无论如何,这样一来,我迫使市场价格下降,结果消费者一般按较便宜的价格得到产品。而在农业中,在类似的情况下,我按2先令出卖,因为,假定我的肥沃土地够用了,比较不肥沃的土地就不会去耕种。如果肥沃土地的数量增加了,或者较坏土地的肥力提高了,使我能够满足需求,那末,问题就不存在了。李嘉图不仅不否认这一点,并且十分明确地强调这一点。

因此,即使我们承认,土地肥力的不同不能解释地租本身,而只能解释地租的差别,下面这一规律仍然存在:在工业中,超额利润的获得通常是由于产品变得便宜,在农业中,地租的相对量的产生则不仅由于产品相对变贵(肥沃土地的产品的价格提到它的价值之上),并且由于便宜产品按较贵产品的生产费用出卖。但是,我曾经指出过(蒲鲁东)\endnote{马克思指他的反对蒲鲁东的著作《哲学的贫困》(第2章第4节,标题是《土地所有权或地租》)。见《马克思恩格斯全集》中文版第4卷第180—191页,特别是第183—186和190页。——第7页。},这仅仅是竞争的规律,它不是从“土地”产生,而是从“资本主义生产”本身产生的。

其次,李嘉图在另外一点上也是正确的,不过他按照经济学家的习惯,把历史现象变成永恒的规律。这个历史现象就是工业(真正资产阶级的生产部门)比农业发展快。农业生产率提高了,但是比不上工业生产率提高的程度。在工业生产率提高到10倍的地方,农业生产率或许提高到2倍。因此,农业生产率,尽管绝对地说提高了,相对地说却降低了。这一点仅仅证明资产阶级生产的极其古怪的发展和它所固有的矛盾,但是并不妨碍下述论点的正确性:农业生产率在相对地降低,因而同工业品相比,农产品的价值以及地租都在提高。随着资本主义生产的发展,农业劳动同工业劳动相比生产率相对地降低,这只是意味着,农业生产率不是以同样速度和同样程度发展。

假定生产部门A与生产部门B之比是1∶1。最初,农业生产率较高,因为在农业中,参加生产的不仅有自然力,而且有自然本身创造的机器;单个劳动者一开始就用这种机器进行劳动。因此,在古代和中世纪,农产品比工业品相对地说便宜得多,这一点从两种产品在平均工资中所占比例已经可以看出来(见威德的著作)\endnote{指约·威德《中等阶级和工人阶级的历史》1833年伦敦版。——第7页。}。

假定1∶1还表示两个生产部门的生产率。如果现在生产部门A=10,也就是说,它的生产率增大到10倍,而生产部门B=3,也就是说,只增大到3倍,那末,两个生产部门之比,以前是1∶1,现在是10∶3或1∶(3/10)。相对地说,生产部门B的生产率降低了7/10,虽然绝对地说它增加到3倍。对于最高的地租来说,这——同工业对比——就好比它由于最坏土地的肥力减低了7/10而提高一样。

诚然,从这里决不能象李嘉图所想的那样得出结论说,利润率下降是因为工资由于农产品相对变贵[447]而提高了,——要知道,平均工资不决定于加入该工资的产品的相对价值,而决定于这些产品的绝对价值。但是从这里确实可以得出结论说,利润率(其实,是剩余价值率)不是按加工工业生产力提高的比例提高的,并且这是由于农业(不是土地)的生产率比较低的缘故。这是无庸置疑的。必要劳动时间的减少,同工业的进步相比,是微不足道的。这从俄国等这样的国家竟能在农产品市场上打击英国这一点就表现出来了。较富国家的货币价值较小(就是说,对于较富国家说来,货币的生产费用相对地小),在这里不起任何作用。因为问题恰恰在于,为什么这种情况在较富国家同较穷国家的竞争中,不影响它们的工业品,而只影响它们的农产品。(可是,这并不证明穷国生产比较便宜,它们的农业劳动生产率高。即使是美国,不久以前统计材料证明,按既定价格出卖的小麦总量的确增加了,但这不是因为每一英亩出产小麦多了,而是因为种小麦的亩数多了。有些国家拥有大量土地,在大地段上进行粗放耕作,用同量劳动提供的产品,就绝对量来说,大于比较发达的国家在小得多的地段上提供的产品,但是,不能说,前者的土地的生产率高于后者。)

转而耕种生产率较低的土地,不一定证明农业生产率下降。相反,它可以证明农业生产率提高;耕种贫瘠土地,不仅因为农产品的价格已经提高到足以补偿投入土地的资本的程度,而且因为生产资料的发展已经达到使不生产的土地变成“生产的”土地,使它能够不仅支付普通利润,并且还支付地租的地步。对于生产力的一定发展阶段说来是肥沃的土地,对于生产力的较低发展阶段说来,就是贫瘠的土地。

在农业中,绝对延长劳动时间,也就是说,增加绝对剩余价值,只有很小的可能。在农业中,劳动不能借瓦斯照明等等。当然,在夏季和春季可以早起。但是,这一点,由于冬季昼短,一般干活较少,就被抵销了。因此,就这方面说,工业中绝对剩余价值比较大,除非法律上强制规定正常工作日。农业中创造的剩余价值量比较小的第二个原因是,农产品长时间滞留在生产过程中而没有新的劳动加在它上面。但是,从另一方面来说,除了如畜牧业、养羊业等绝对排挤人口的一些农业部门以外,甚至在最先进的大农业中,使用的人数对使用的不变资本的比例,总是比工业,至少比主要工业部门大得多。因此,从这一方面说,即使由于上述原因,农业中的剩余价值量小于工业中使用相同人数时得到的剩余价值量(这种情况又部分地由于农业工人的工资降到平均水平之下而被抵销),农业的利润率仍可能大于工业的利润率。但是,如果说在农业中存在着提高利润率(不是暂时提高,而是同工业相比,平均地提高)的某些原因(上述这些,我们只是大略谈了一下),那末,单单土地所有者的存在这一事实本身,就使这种超额利润不是进入一般利润率的平均化过程,而是固定下来,落到土地所有者手中。

\tsectionnonum{[(2)利润率和剩余价值率的关系。作为农业中的不变资本要素的农业原料价值]}

考察洛贝尔图斯的理论时要回答的问题,总的说来,归结如下。

预付资本的一般形式是:

\todo{}

不变资本的两个要素,一般地说,就是劳动资料和劳动对象。劳动对象不一定是商品,不一定是劳动产品。因此,劳动对象作为劳动过程的要素虽然永远存在,但作为资本的要素可能不存在。土地是土地耕种者的劳动对象\endnote{在手稿中这里是“Rohmaterial”(“原料”)一词。马克思在他的1861—1863年手稿的所有其他地方都是在较狭义上应用“原料”这个术语,象《资本论》第一卷第五章所表述的那样,是指“已经通过劳动而发生变化”的劳动对象(见《马克思恩格斯全集》中文版第23卷第203页)。而在这里,从前后文看来,谈的是最广义的劳动对象,即首先指的不是劳动产品而是自然给予的劳动对象,所以可以设想,“原料”一词是笔误,应当用“Arbeitsgegenstand”(“劳动对象”)代替它。因此,这里就改译为“劳动对象”。——第12页。},煤矿是煤炭业者的劳动对象,水域是渔夫的劳动对象,森林是猎人的劳动对象。但是,当上述劳动过程三要素也作为资本三要素出现,就是说,它们三者都是商品,都是一种具有交换价值并且表现为劳动产品的使用价值的时候,资本具有最完整的形式。在这种情况下,这三个要素也就都进入价值形成过程,虽然机器不是按它进入劳动过程多少,而只是按它被劳动过程消耗多少进入价值形成过程。

现在的问题是:缺少其中一个要素,能否使缺少这个要素的生产部门的利润率(不是剩余价值率)提高呢?一般地说,这个问题可由下面的公式本身来回答:

利润率等于剩余价值和预付资本总额之比。

全部研究是在这种假定下进行的:剩余价值率不变,就是说,产品价值在资本家和雇佣工人之间的分配不变。

[448]剩余价值率=m/v;利润率=m/(c+v)。因为m’即剩余价值率是既定的,v也就既定,而且m/v被假定为常量。所以,m/(c+v)的量只有在c+v变化时才变化,又因为v是既定的,所以m/(c+v)只有在c减小时增大,或者在c增大时减小。并且,m/(c+v)的增大或减小,不是同c∶v成比例,而是同c∶(c+v)之和成比例。假设c=零,那末m/(c+v)=m/v。换句话说,在这种情况下,利润率等于剩余价值率,而这就是利润率不能超越的极限,因为任何计算方法都不能改变m和v的量。如果v=100,m=50,那末m/v=50/100=1/2=50%。现在如果加上不变资本100,那末利润率=50/100+100=50/200=1/4=25%。利润率减了一半。如果把150加到100上,那末利润率=50/(150+100)=50/250=1/5=20%。在第一种情况下,总资本=v=可变资本,因而利润率=剩余价值率。在第二种情况下,总资本=2×v,因而利润率只有剩余价值率的一半。在第三种情况下,总资本=[2+(1/2)]×100=[2+(1/2)]×v=(5/2)×v。在这里,v只是总资本的2/5。剩余价值=v的1/2,100的1/2,因此只是总资本的2/5的1/2,也就是说,只是总资本的2/10。(250/10=25,而250的2/10=50。)而2/10就是20%[也就是说,利润率是剩余价值率的2/5]。

因此,这些是一开始就确定了的。如果v和m/v不变,那末c这个量究竟由哪些部分构成,是完全没有关系的。如果c是一定量,例如等于100,那末,不论c分成50是原料和50是机器,或者10是原料和90是机器,或者0是原料和100是机器,或者反过来,都完全没有关系,因为决定利润率的是m/(c+v)这个比例;构成c的各个生产要素,作为价值部分,同整个c之比究竟如何,在这里是没有关系的。例如,在煤的生产中,可以把原料(本身又用作辅助材料的煤除外)看作零,而假定全部不变资本都是由机器(包括建筑物、劳动工具在内)构成。另一方面,在缝纫业者那里,可以假定机器等于零(就是说,在大缝纫业者还没有应用缝纫机的地方,另一方面,象目前伦敦有一部分做法那样,甚至把建筑物都节省掉,让自己的工人在家里劳动;这是件新鲜事:第二种分工又以第一种分工形式出现\endnote{马克思在1861—1863年手稿第IV本中(第149页及以下各页),把社会上彼此独立的商品生产者之间的分工称为“第一种分工”,把资本主义企业内部,特别是手工业工场内部的分工称为“第二种分工”。参看《马克思恩格斯全集》中文版第23卷第389—398页。——第14页。}),于是,在这个缝纫业者那里,全部不变资本都归结为原料。如果煤炭业者把1000花费在机器上,把1000花费在雇佣劳动上,缝纫业者则把1000花费在原料上,把1000花费在雇佣劳动上,那末,在剩余价值率相等时,这两种情况下的利润率也相等。我们假定,剩余价值=20%,在这两种情况下,利润率就都=10%,即200/2000=2/20=1/10=10%。因此,如果说c的组成部分即原料和机器之间的比例,对利润率有影响,那只有在下列两种情况下才可能:第一,如果c的绝对量由于这个比例发生变化而有了变化;第二,如果v的量由于c的组成部分之间的这个比例而有了变化。这里,必定是生产本身发生了有机变化,而不能归结为这样一个简单的同义反复:如果c的一定部分在总数中占较小的部分,那末c的另一部分在总数中一定占较大的部分。

在一个英国租地农场主的实际开支中,工资=1690镑,肥料=686镑,种子=150镑,牛饲料=100镑。因而用于“原料”的是936镑,比工资的一半还多。(见弗·威·纽曼《政治经济学讲演集》1851年伦敦版第166页)

\begin{quote}{“在弗兰德〈比利时〉,这一带从荷兰进口肥料和干草〈用于种植亚麻等。作为交换,这一带出口亚麻和亚麻籽等〉……荷兰各城市的垃圾成了交易品,经常以高价卖给比利时……从安特卫普溯些耳德河而上约20英里,就可以看到从荷兰运来的肥料的堆栈。肥料贸易由一帮资本家用荷兰船只经营”等等。\fnote{本卷引文中凡是尖括号〈〉和花括号{}内的话都是马克思加的。——译者注}(班菲尔德的著作)\endnote{马克思在这里引用的书是:托·查·班菲尔德的《产业组织》1848年伦敦第2版第40、42页。——第15页。}}\end{quote}

既然连普通粪便这样的肥料都成了交易品,骨粉、鸟粪、炭酸钾等就更不用说了。这里,生产要素用货币来估价,不只是生产中的形式上的变化。为了提高生产率,把新的物质送到地里,而把地里旧的物质卖出去。这也不单纯是资本主义生产方式和它以前的生产方式之间的形式上的差别。随着人们认识到换种的重要性,连种子交易也越来越重要了。因此,就真正的农业来说,如果说没有“原料”——并且是作为商品的原料——加入农业(不论是农业自己把它再生产出来,还是把它作为商品买进、从外面取得,都一样),那是可笑的。如果说机器制造业者[449]自己使用的机器不作为价值要素加入他的资本,那是同样可笑的。

一个年年自己生产自己的生产要素(种子、肥料等等),并且自己全家吃掉自己的一部分谷物的德国农民,只是为了购置少数农具和支付工资才(为生产本身)支出货币。假定他的全部支出的价值等于100[其中50用货币支付]。他以实物形式消费产品的一半([这里也包括实物形式的]生产费用)。他把另一半出卖,比如说得到100。在这种情况下,他的总的[货币]收入等于100。如果他按资本50来计算[他的货币形式的纯收入],那就是100%[利润]。如果现在[作为利润得到的]50中有1/3交地租,1/3交税款(合计33+(1/3)),他自己留下16+(2/3),按50计算,就是[33+(1/3)]%。但是,实际上他只得到[所支出的100的][16+(2/3)]%。这个农民完全算错了,自己骗了自己。在资本主义租地农场主那里是不会有这种错误计算的。

马蒂约·德·东巴尔《农业年鉴》1828年巴黎版第四分册说到,按照对分制租佃契约(例如贝里省),

\begin{quote}{“土地所有者提供土地、建筑物,通常还提供全部或一部分牲畜和生产所必需的农具;租地农民方面提供自己的劳动,此外不提供或几乎不提供什么。土地的产品拿来对分。”(第301页)“对分制租地农民通常是贫困不堪的人。”(第302页)“如果对分制租地农民预付1000法郎,增加总产品1500法郎〈即总利润500法郎〉,他必须同土地所有者对分,因而只得到750法郎,也就是说,自己的预付资本损失250法郎。”(第304页)“在以前的耕作制下,生产支出即生产费用几乎完全以实物形式从产品本身取得,以供饲养牲畜并供土地耕种者和他的家庭消费;几乎完全没有现金支出。只有这种情况才会使人相信,土地所有者和租地农民可以分享没有在生产中消费掉的全部收成;但是,这种做法只适用于这种农业,即处于低水平的农业;人们一旦想要在农业中实行某种改良,就会立即发觉,只有预先付出一笔款项才可能做到,而这笔款项必须从总产品中扣出,供下年生产之用。因此,土地所有者和租地农民对总产品的任何分成,对任何改良都是不可克服的障碍。”(第307页)}\end{quote}

\tsectionnonum{[(3)农业中的价值和平均价格。绝对地租]}

\tsubsectionnonum{[(a)工业中利润率的平均化]}

洛\endnote{“平均价格”(Durchschnittspreis)这一术语,马克思这里是指“生产价格”,即生产费用(c+v)加平均利润。“平均价格”这一术语本身说明,这里所指的,正如马克思后来在本册第359页上所解释的那样,是“一个相当长的时期内的平均市场价格,或者说,市场价格所趋向的中心”。马克思用的这个术语最初见于本卷第一册第76页。——第16页。}贝尔图斯先生对于竞争调节正常利润,或平均利润,或一般利润率,总的说来,似乎是这样想的:竞争使商品还原为它们的实际价值,就是说,竞争调节商品价格之间的比例,使物化在各种商品中的劳动时间的相当量,以货币或其他某种价值尺度表现出来。当然,这种调节,不是使这种或者那种商品的价格在任何时候、任何一定时刻都等于或都必定等于它的价值。[照洛贝尔图斯的想法,这种调节是这样进行的。]例如,商品A的价格提高到它的价值以上,并且,这种价格在一定时间内保持这个高度或者甚至继续提高。资本家A的利润因而提高到平均利润以上,因为他不仅占有他自己的“无酬”劳动时间,而且占有其他资本家“生产”的无酬劳动时间的一部分。与此相应——在其他商品的货币价格不变的情况下——必然有这个或那个生产领域的利润下降。如果该商品作为一般生活资料加入工人的消费,这就会使其他一切部门的利润率下降;如果该商品成为不变资本的组成部分,这就会使那些以该商品作为不变资本要素的生产部门的利润率下降。

最后,可能还有一种情况,即这种商品既不作为要素加入任何不变资本,也不是工人的必要生活资料(因为,工人随自己的意可买可不买的那些商品,工人是作为一般消费者而不是作为工人去消费的),而是消费品,一般个人消费品。如果这种商品作为消费品加入工业资本家本人的消费,那末,它的价格提高决不影响剩余价值量或剩余价值率。但是,如果资本家想要保持他原来的消费水平,那末,利润(剩余价值)中被他用于个人消费的部分,同被他用于工业再生产的部分相比就会增加。这样,用于再生产的部分就会减少。因此,由于A的价格提高,或者说,A的利润提高到平均利润率以上,经过一定时期(这个时间也是由再生产决定的),B、C等的利润量就会降低。如果商品A完全加入非工业资本家的消费,那末,同以前相比,这些非工业资本家消费商品A多了,而消费商品B、C等少了。对商品B、C等的需求会减少;它们的价格将下降,而在这种情况下,A的价格的提高,或者说,A的利润提高到平均利润率以上,会通过压低B、C等的货币价格,使B、C等的利润降到平均利润率以下(这同前面所举的情况不同,在那里,B、C等的货币价格是[450]保持不变的)。利润率降到普通水平以下的B、C等领域的资本,将离开它们自己的生产领域,转入A生产领域;在市场上不断重新出现的一部分资本,尤其是这样,这种资本当然会力求挤进更加有利可图的A生产领域。由于这个原因,商品A的价格,在若干时间以后,将会降到它的价值以下,并且在或长或短的一段时间内继续下降,直到相反的运动重新开始为止。在B、C等领域中,将发生相反的现象,部分由于商品B、C等的供给因资本流出而减少,就是说,由于这些领域本身发生了有机变化,部分则由于A领域中过去发生的变化现在以相反方向作用于B、C等领域。

顺便指出:在刚才描述的运动中,虽然商品B、C等的货币价格(假定货币的价值不变)提高到商品B、C等的价值以上,因而B、C等的利润率也提高到一般利润率以上,但是,商品B、C等的货币价格,有可能再也达不到它们原来的水平。改良、发明、生产资料的更大节约等等,不是在价格提高到自己的平均水平以上的时候运用,而是在价格降到这个水平以下、因而利润降到普通利润率以下的时候运用。因此,在商品B、C等的价格下降的时期,它们的实际价值可能下降,换句话说,为生产这些商品所需的最低限度的劳动时间可能下降。在这种情况下,只有当商品的价格超过它的价值的程度,等于表现它的新价值的价格和表现它的较高的原有价值的价格之间的差额的时候,商品才能恢复它以前的货币价格。在这种情况下,商品价格将会通过影响供给、影响生产费用来改变商品的价值。

上述运动的结果就是这样:如果就商品价格在商品价值上下波动的平均数来看,或者说,如果就上下波动平均化的时期——不断反复出现的时期——来看,那末平均价格等于价值,因而一定生产领域的平均利润也等于一般利润率;因为,在这个领域中,虽然随着价格的涨落,或者,在价格不变的情况下,随着生产费用的增减,利润提高到原来的利润率以上或降到原来的利润率以下,但是就一个时期平均起来,商品是按自己的价值出卖的,因而,赚到的利润等于一般利润率。这就是亚·斯密的观点,尤其是李嘉图的观点,因为后者更明确地坚持真正的价值概念。洛贝尔图斯先生也从他们那里接受了这个观点。可是,这个观点是错误的。

资本的竞争究竟产生什么结果呢?在任何一个平均化的时期中,商品的平均价格是这样的:这种价格向每个领域的商品生产者提供同样的利润率,譬如10%。这又是什么意思呢?这就是说,每种商品的价格,比这种商品使资本家花费的、资本家为生产它而支出的生产费用,高出十分之一。一般说来,这不过是说,等量资本提供等量利润,每种商品的价格,比在这种商品上预付、消费或者体现的资本的价格,高出十分之一。但是,如果以为资本按照自己的大小,在不同的领域中生产相同的剩余价值,那是完全错误的。{这里我们完全不考虑,一个资本家是否比另一个资本家强迫工人劳动更长的时间;我们在这里假定,在一切领域中,绝对工作日是一样的。绝对工作日的差别,一部分由不同长度的工作日的劳动强度等抵销了,一部分不过表现为强求的超额利润、例外等。}即使假定绝对工作日在一切领域中是一样的,就是说,假定剩余价值率是既定的,这种说法也是错误的。

在资本量相等的情况下,——并且在上述假定的条件下,——这些资本所生产的剩余价值量依下述情况不同而不同:第一,资本的有机组成部分即可变资本和不变资本之间的比例;第二,资本的周转时间,因为这个时间取决于固定资本和流动资本之间的比例,以及不同种类固定资本的不同的再生产期间;第三,和劳动时间本身长度不同的、真正生产期间的长度,\endnote{马克思在他的1857—1858年手稿中谈到关于特别是在农业中存在的生产时间和劳动时间的区别,以及与此有关的资本主义在农业中发展的特点(见卡·马克思《政治经济学批判大纲》1939年莫斯科版第560—562页)。生产期间即生产时间(除了劳动时间以外,还包括劳动对象仅仅接受自然界的自然过程的作用的时间),这个概念马克思在《资本论》第二卷第二篇第十三章作了详细的阐述(见《马克思恩格斯全集》中文版第24卷第2篇第13章)。——第19页。}这个长度也决定生产期间和流通期间的比例的重大差别。(上述第一个比例,即不变资本和可变资本之间的比例本身,可以由非常不同的原因产生。例如,它可以仅仅是形式上的,——当一个生产领域加工的原料比另一个生产领域加工的原料贵的时候,就是这样,——或者,它可以由不同的劳动生产率产生,等等。)

因此,如果商品按其价值出卖,或者说,如果商品的平均价格等于其价值,那末,利润率在不同的生产领域中必定是完全不同的;在一种情况下,它会是50%,在其他情况下,它会是40%、30%、20%、10%等。例如,拿A领域一年的商品总量来看,它的价值等于预付在它上面的资本加上它所包含的无酬劳动。在B、C领域中也是一样。但是因为A、B、C包含的无酬劳动量不同,例如,A包含的大于B包含的,B包含的大于C包含的,商品A给自己的生产者提供比方说3M(M是剩余价值),商品B提供2M,商品C提供M。因为利润率决定于剩余价值和预付资本之比,而预付资本,根据假定,在A、B、C等领域中是一样的,所以,[451]如果C代表预付资本,那末,这些领域的不同的利润率就等于(3M)/C、(2M)/C、M/C。因此,资本的竞争要使利润率平均化,在上述例子中,只有使A、B、C领域的利润率等于(2M)/C、(2M)/C、(2M)/C。这样,A将会把它的商品卖得比它的价值便宜1M,而C把它的商品卖得比它的价值贵1M。A领域的平均价格将低于商品A的价值,C领域的平均价格将高于商品C的价值。

B的情况说明,商品的平均价格同价值一致,确实可能发生。这发生在B领域本身生产的剩余价值等于平均利润的时候,也就是说,这时候,在这个领域中,资本的不同部分的相互比例,等于(如果把资本的总额,资本家阶级的全部资本,当作一个量,按这个量来计算全部剩余价值,不问这些剩余价值由总资本的哪个领域生产出来)总资本不同部分的相互比例。在这个总资本中,周转时间等等也平均化了;这整个资本按例如一年周转一次计算,等等。于是,这个总资本的每个部分,实际上就会根据自己的大小,按比例来瓜分全部剩余价值,各自取得全部剩余价值的相应部分。既然每一单个资本被看作这个总资本的股东,那末由此可以得出结论:第一,单个资本的利润率同其他任何资本的利润率是一样的,等量资本提供等量利润;第二,这是从第一点自然得出的,就是,利润量取决于资本的大小,取决于资本家在这个总资本中拥有的股数。资本的竞争力图把每个资本作为总资本的一部分来对待,并且根据这一点来调节每个资本取得剩余价值的份额,也就是说,调节利润。竞争通过它的平均化作用或多或少达到了这个目的。(竞争在个别领域中遇到特殊障碍的原因不应在这里研究。)直截了当地说,这无非是资本家们努力(而这种努力就是竞争)把他们从工人阶级身上榨取的全部无酬劳动量(或这个劳动量的产品)在他们相互之间进行分配,而且这种分配不是根据每一个特殊资本直接生产多少剩余劳动,而是根据:第一,这个特殊资本在总资本中占多大部分;第二,总资本本身生产的剩余劳动总量。资本家们既作为同伙又作为敌手来瓜分赃物——他们所占有的别人劳动,于是他们每个人占有的无酬劳动,平均说来,同其他任何一个资本家占有的一样多。\endnote{马克思在《资本论》第三卷中论证了资本家们既作为竞争的敌手又作为“同伙”这个特点。在利润率平均化的过程中,“每一单个资本家,同每一个特殊生产部门的所有资本家总体一样,参与总资本对全体工人阶级的剥削,并参与决定这个剥削的程度”。(见马克思《资本论》第3卷第10章)马克思在研究了这个过程后写道:“……我们在这里得到一个象数学一样精确的证明:为什么资本家在他们的竞争中表现出彼此都是虚伪的兄弟,但面对着整个工人阶级却结成真正的共济会团体。”(同上)——第21页。}

竞争是通过调节平均价格来实现这种平均化的。但是,这种平均价格本身,使商品高于或低于它的价值,以致该商品不能比其他任何商品提供较大的利润率。因此,认为资本竞争是通过使商品价格等于价值来确立一般利润率的说法,是错误的。相反,竞争正是通过以下途径来确立一般利润率的:它把商品的价值转化为平均价格,在平均价格中,一种商品的剩余价值的一部分转到另一种商品上,等等。商品的价值等于商品包含的有酬劳动和无酬劳动的量。商品的平均价格等于商品包含的有酬劳动(物化劳动或活劳动)量加无酬劳动的平均份额,这个平均份额不取决于它原来是否如数包含在这种商品本身,换句话说,不取决于原来包含在该商品的价值中的无酬劳动是大还是小。

\tsubsectionnonum{[(b)地租问题的提法]}

可能,——这一点我留到以后研究,不属于本册\endnote{“本册”,马克思是指《论资本》那一册。他在1858—1862年间想把他的全部经济著作分为六册,《论资本》是其中第一册也是最基本的一册(见《马克思恩格斯全集》中文版第13卷第7页)。——第22页。}研究范围,——某些生产领域是在这样的环境下工作的,这种环境阻碍它们的价值转化为上述意义的平均价格,也就是说,不让竞争取得这种胜利。如果,比如说,农业地租或矿山地租就是这种情况(有一些租,完全只能用垄断来说明,例如伦巴第和亚洲某些地区的水租;又如实际是地产租的房租),那末,从这里得出的结论是,当所有工业资本的产品的价格提高或者降低到平均价格的水平的时候,农产品的价格却始终等于自己的价值,而这个价值将高于平均价格。这里是否存在着一种障碍,使这个生产领域生产的剩余价值中被当作本领域财产来占有的部分,大于按照竞争规律应得的部分,大于按照投在这个生产部门的资本的份额应得的部分?

我们假定有这样一些工业资本,它们不是暂时地,而是由于它们的生产领域的性质,[452]比其他生产领域中同量工业资本多生产10%,或20%,或30%的剩余价值。我说,如果这些资本能够在竞争面前保住这种超额剩余价值,不让它参加决定一般利润率的总计算(分配),那末,在这种情况下,在这些资本发挥作用的各个生产领域中,就会有两个不同的获利者,一个取得一般利润率,另一个取得该领域所独有的超额部分。每一个资本家,为了有可能把他的资本投入该领域,就要对这个享受特权的人支付、交付这个超额部分,而他自己同其他任何一个在相同条件下经营的资本家一样,为自己保住一般利润率。既然农业中的情况是这样,那末,这里,剩余价值分解为利润和地租,完全不是表明劳动本身在这里比在加工工业中“具有更高的生产率”(从生产剩余价值的意义上来说);因此,把任何创造奇迹的力量归于土地是毫无理由的,并且,这本身就是可笑的,因为价值等于劳动,从而,剩余价值决不能等于土地。(诚然,相对剩余价值可能取决于土地的自然肥力,但是,无论如何不能由此得出土地产品价格较高的结论。倒是恰恰相反。)也不必找李嘉图的理论帮忙,这个理论本身令人讨厌地同马尔萨斯废话联结在一起,得出可鄙的结论,特别是,这个理论同我的相对剩余价值学说,即使在理论上不是对立的,在实践上也把它的意义抹去了一大部分。

在李嘉图那里,问题的全部要点如下:

地租(例如在农业中)——照他的假定——在农业以资本主义方式经营、有租地农场主存在的地方,只能是超过一般利润的余额。土地所有者取得的地租是否真正是这种资产阶级经济学意义上的地租,是完全没有关系的。它可能纯粹是工资的扣除部分(参看爱尔兰的情况),也可能部分地靠租地农场主的利润被压到利润的平均水平以下而得到。这一切可能的情况在这里是绝对无关紧要的。地租之所以在资本主义制度下成为剩余价值的一种特殊的、具有特征的形式,只在于它是超过(一般)利润的余额。

但是,这怎么可能呢?商品小麦,同其他任何商品一样,按它的价值出卖,就是说,按照它所包含的劳动时间同其他商品交换。{这是第一个错误的前提,它人为地使问题变得更加困难了。商品按其价值交换只是例外。商品的平均价格是按另外的方式决定的。见上述。}种植小麦的租地农场主同其他所有资本家一样,赚得同样的利润。这证明,他同其他所有资本家一样,占有自己工人的无酬劳动时间。在这种情况下,究竟还从哪里产生地租呢?地租无非代表劳动时间。为什么剩余劳动在工业中只等于利润,而在农业中却应该分解为利润和地租呢?如果农业中的利润等于其他各个生产领域的利润,这怎么可能呢?{李嘉图的不正确的利润观点,以及他把利润和剩余价值直接混淆起来,在这里也是有害的。这些使他考察问题更困难了。}

李嘉图解决这个困难的办法是:假定困难在原则上是不存在的。{确实,这是在原则上解决困难的唯一方式。不过,这可以有两种办法。或者证明,与一定原则矛盾的现象只是某种表面的东西,只是从事物本身发展中产生出来的假象。或者象李嘉图所作的那样,在某一点上抛开困难,然后把这一点作为出发点,从这里出发,可以说明造成困难的现象在另一点上存在。}

李嘉图假定这样一种情况,那就是,租地农场主的资本{不论是指个别农场的不付地租的那部分资本,或者是指农场的不付地租的那部分土地;总之,这里是指投入农业而不付地租的资本}同其他任何一个资本家的资本一样,只提供利润。这个假定甚至是李嘉图的出发点,它也可以这样表达:

最初,租地农场主的资本只提供利润{但是,这个伪历史形式是无关重要的,它是所有资产阶级经济学家在编造其他类似“规律”时所共有的},这笔资本不支付地租。租地农场主的资本同其他任何产业资本没有区别。只因为对于谷物的需求增加了,结果,和其他生产部门不同,不得不向“比较不”肥沃的土地找出路,这才产生地租。由于生活资料涨价,租地农场主(假定的最初的租地农场主)同其他任何产业资本家一样受损失,因为租地农场主也不得不给自己的工人多支付报酬。但是,租地农场主由于自己的商品的价格提高到它的价值以上,占了便宜。他所以占便宜,第一,因为加入他的不变资本的其他商品,同他的商品比起来,相对价值下降了,于是他按比较便宜的价格购买这些商品;第二,因为他以较贵的商品形式占有他的剩余价值。这样一来,这个租地农场主的利润就提高到已经降低的平均利润率以上。于是,另一个资本家去经营较坏的II等地,这块土地,在这个利润率较低的情况下,能够按I的产品的价格提供产品,甚至还稍便宜一些。不管怎样,我们现在[453]在II等地上又有了使剩余价值仅仅归结为利润的正常关系,然而因此我们已经把I的地租解释了,也就是说,因为存在着两种生产价格,而II的生产价格同时就是I的市场价格。这就完全象在比较有利的条件下生产出来的工业品提供暂时的超额利润一样。除利润外还包含地租的小麦价格,虽然也是仅仅由物化劳动构成,虽然也等于小麦的价值,但是,它不等于小麦本身包含的价值,而等于II上种植的小麦的价值。两种市场价格并存是不可能的。{李嘉图因为利润率下降而引进租地农场主II,斯特林则由于谷物价格使工资下降而不是提高,让租地农场主II登场。这种下降的工资使租地农场主II能够以原来的利润率经营II等地,虽然这块土地比较不肥沃。\endnote{帕·詹·斯特林《贸易的哲学,或利润和价格理论概要》1846年爱丁堡版第209—210页。——第25、525页。}}地租的存在既然这样来解释,其余也就不难推论了。地租的差别同肥力的差别相适应等等,自然还是正确的。但是,肥力的差别本身并不证明必须去耕种越来越坏的土地。

因此,李嘉图的理论就是这样。因为给租地农场主I提供超额利润的上涨了的小麦价格,给租地农场主II提供的甚至不是原来的利润率,而是较低的利润率,所以,很清楚,II的产品包含的价值大于I的产品,或者说,II的产品是较多劳动时间的产品,它包含较多的劳动量;因此,为了生产同样多的产品,例如一夸特小麦,就要花费较多的劳动时间。地租的增长,将同土地肥力的这种不断降低的情况相适应,或者说,将同生产例如一夸特小麦所必需花费的劳动量的增加相适应。当然,如果增加的只是支付地租的夸特数,李嘉图是不会说地租“增长”的,在李嘉图看来,只有同样一夸特的价格增长,例如从30先令涨到60先令,地租才是增长了。诚然,李嘉图有时忘记了,地租的绝对量在地租率下降的情况下可能增长,正如利润的绝对量在利润率下降的情况下可能增长一样。

另外一些人(例如凯里)想绕过这个困难,他们干脆用另一种方式否认这个困难的存在。据说,地租只是以前投入土地的资本的利息。\endnote{马克思在后面(见本册第152—153、157和179页)以及《资本论》第三卷(第37章和第46章)中,都谈到凯里把地租看成投入农业的资本的利息这种庸俗见解。——第26页。}所以,地租也只是利润的一种形式。因此,这里,地租的存在被否定了,从而地租实际上就被解释掉了。

另外一些人,例如布坎南,把地租看成纯粹是垄断的后果。再看霍普金斯的著作。\endnote{马克思在后面,在本册第178、377、439—440页谈到布坎南关于农产品垄断价格的见解。马克思在本册第147—153页分析了霍普金斯的地租观点。——第26页。}这里,地租完全归结为超过价值的附加部分。

在奥普戴克先生那里,土地所有权或地租是“资本价值的合法反映”\endnote{马克思引用的著作是:乔·奥普戴克《论政治经济学》1851年纽约版第60页。——第26页。},这是美国佬所特有的。\fnote{[486}{奥普戴克把土地所有权称为“资本价值的合法反映”,那末,资本同样是“别人劳动的合法反映”。}[486]]

在李嘉图那里,由于两个错误的假定,增加了研究的困难。{确实,李嘉图不是地租理论的发明者。威斯特和马尔萨斯在李嘉图之前已经出版了自己关于地租理论的著作。然而,来源是安德森。但是,李嘉图的特点是他的地租理论和他的价值理论的相互联系(虽然在威斯特的著作中也不是完全没有真实联系)。马尔萨斯后来同李嘉图在地租问题上的论战证明,马尔萨斯甚至并不理解他从安德森那里借用的理论。}如果从商品价值决定于生产商品所必需的劳动时间(以及价值无非是物化了的社会劳动时间)这个正确的原则出发,那末,自然得出结论说,商品的平均价格决定于生产商品所必需的劳动时间。如果平均价格等于价值这一点得到证明,这个结论就会是正确的。可是,我证明情况恰好相反:正因为商品价值决定于劳动时间,商品平均价格决不能等于商品价值(只有一个情况除外,就是某一个生产领域的所谓个别利润率,即由这个生产领域本身生产出来的剩余价值决定的利润,等于总资本的平均利润率),虽然平均价格这个规定只是从由劳动时间决定的价值引伸出来的。

由此首先得出一个结论:即使有些商品的平均价格(如果撇开不变资本的价值不说)只分解为工资和利润,而工资和利润又处于正常水平,是平均工资和平均利润,这种商品,也可能高于或者低于它们自己的价值出卖。因此,一种商品的剩余价值只表现为正常利润的项目这个情况并不足以证明,这种商品就是按它的价值出卖,同样,商品除利润外[454]还提供地租这个情况也不足以证明,这种商品是高于它的内在价值出卖的。既然确定,一种商品所实现的资本的平均利润率即一般利润率,可能低于商品自己的、由商品中实际包含的剩余价值决定的利润率,那就可以由此得出结论:如果一个特殊生产领域的商品,除了提供这种平均利润率以外,还提供第二个剩余价值量,这种剩余价值量具有特殊的名称,比如叫作地租,那末,这并不使利润加地租,即利润与地租之和,一定要大于这个商品本身所包含的剩余价值。既然[资本家所得的]利润可能小于该商品的内在剩余价值,也就是说,小于该商品所包含的无酬劳动量,那末,利润加地租也就不一定要大于商品的内在剩余价值。

的确,剩下还要说明的是,为什么这类现象发生在一个不同于其他生产领域的特殊生产领域。但是,解决这个问题已经非常容易了。提供地租的这种商品和其他一切商品的不同之处在于一部分其他商品的平均价格高于它们的内在价值,但其程度只是使它们的利润率提到一般利润率水平;而另一部分其他商品的平均价格低于它们的内在价值,但其程度只是使它们的利润率降到一般利润率水平;最后,第三部分其他商品的平均价格等于它们的内在价值,但这只是因为它们在按它们的内在价值出卖时提供一般利润率。提供地租的商品同所有这三种情况都不相同。在任何情况下,这种商品出卖的价格都是这样的:这种商品所提供的利润,大于由资本的一般利润率决定的平均利润。

现在产生的问题是:在这里,上述三种情况中哪一种情况或者其中哪几种情况可能发生?提供地租的商品所包含的全部剩余价值在该商品的价格中是否得到实现?如果是这样,上述第三种情况就被排除了,在第三种情况下,商品的全部剩余价值之所以在它们的平均价格中得到实现,是因为它们只有这样才提供普通利润。因此这种情况不属于考察的范围。同样,按照这个假定,第一种情况,就是在商品的价格中实现的剩余价值高于它的内在剩余价值的情况,也不属于考察的范围。因为我们恰恰假定,在提供地租的商品的价格中“实现了它所包含的剩余价值”。因此,这种情况同第二种情况相类似,在第二种情况下,商品的内在剩余价值高于在它们的平均价格中实现的剩余价值。同这第二种情况下的商品一样,特殊生产领域的商品的内在剩余价值——以利润的形式出现,并降低到一般利润率的水平,——在这里形成所花费的资本的利润。但是,和第二种情况下的商品不同,在我们所考察的这些例外的商品的价格中,也实现了商品的内在剩余价值超过这个利润的余额,但是这个余额不是落到资本所有者手里,而是落到别的所有者手里,就是说,落到土地、自然因素、矿山等等的所有者手里。

也许这些商品的价格被哄抬到足以提供多于平均利润率的东西吧?例如,在(真正的)垄断价格的情况下就是这样。这个假定——对于每一个可以自由使用资本和劳动,而生产就使用的资本量来说已经服从于一般规律的生产领域——不仅是petitioprincipii〔本身尚待证明的论据〕,并且是同科学和资本主义生产(前者仅仅是后者的理论表现)的基础直接矛盾的。因为,这种假设的前提恰恰是需要加以说明的东西,即在一个特殊生产领域中,商品的价格所提供的必然要比一般利润率,比平均利润多,为此,商品必然要高于它的价值出卖。因而,它的前提是,农产品不受商品价值和资本主义生产的一般规律影响。并且,所以以此为前提,是因为初看起来,利润之外还特别存在地租,造成了这种假象。所以,上述假设是荒谬的。

因此,唯一的办法就是,假设在这个特殊生产领域存在着特殊的条件,存在着某种影响,使商品的价格实现了商品的[全部]内在剩余价值,而不是象第二种情况下的商品,其价格只在一般利润率所提供的利润的限度内实现其剩余价值。在所考察的特殊生产领域中,商品的平均价格并没有降到商品的剩余价值以下,以致它们只提供一般利润率,或者说,以致它们的平均利润不大于其他一切使用资本的生产领域。

这样一来,问题已经大大简化了。问题已经不是要说明,一种商品的价格,怎么除了提供利润之外还提供地租,——因而,它表面上看来,违背了一般价值规律,并且通过把它的价格提到高于它的内在剩余价值,而给一定量资本提供了大于一般利润率所能提供的东西。相反,问题是要说明:这种商品在商品价格平均化而导致平均价格的过程中,怎么不把它的内在剩余价值让一些给其他商品,使它只留下平均利润;这种商品怎样把自己剩余价值中构成超过平均利润的余额的那部分也加以实现。因此,问题在于一个在该生产领域投资的租地农场主,他出卖商品的价格,怎么会使这种商品除了给他提供普通利润外,同时还使他能够把实现的商品剩余价值超过这个利润的余额,付给第三者即土地所有者。

[455]这样提出问题的提法本身,就已经包含问题的解答。

\tsubsectionnonum{[(c)土地私有权是绝对地租存在的必要条件。农业中剩余价值分解为利润和地租]}

十分简单:一定的人们对土地、矿山和水域等的私有权,使他们能够攫取、拦截和扣留在这个特殊生产领域即这个特殊投资领域的商品中包含的剩余价值超过利润(平均利润,由一般利润率决定的利润)的余额,并且阻止这个余额进入形成一般利润率的总过程。这部分剩余价值,甚至在一切工业企业中也被拦截,因为不论什么地方,都要为使用地皮(工厂建筑物、作坊等所占的地皮)付地租,因为即使在可以完全自由占用土地的地方,也只有在多少是人口稠密和交通发达的地点才建立工厂。

如果在最坏的土地上得到的商品,属于平均价格等于价值的第三类商品,就是说,属于这样一类商品,它们的全部内在剩余价值在它们的价格中得到实现,因为它们只有这样才提供普通利润,——那末,这块土地就不付任何地租,土地所有权在这里就只是名义上的。假如这里付一笔租金,那末,这不过证明小资本家们满足于赚取低于平均利润的利润,在英国有一部分就是这种情况(见纽曼的著作)\endnote{指弗·威·纽曼《政治经济学讲演集》1851年伦敦版第155页。——第31页。}。当地租率大于商品的内在剩余价值和平均利润的差额的时候,总是这种情况。甚至有的土地,耕种它至多只够补偿工资,因为,虽然劳动者在这里用他的整个工作日为自己劳动,但是他的劳动时间超过社会必要劳动时间。他的劳动生产率低于这个劳动部门中占统治地位的生产率,虽然他用12小时为自己劳动,他生产的产品几乎没有工人在比较有利的生产条件下用8小时生产的多。这就好比与机器织机竞争的手工织工的情况一样。这个手工织工的产品,的确包含12劳动小时,但是它只等于8小时或者还不到8小时的社会必要劳动,因此,只有8个必要劳动小时的价值。如果在类似情况下,一个茅舍贫农支付租金,那末这笔租金纯粹是他的必要工资的扣除部分,不代表任何剩余价值,更不代表任何超过平均利润的余额。

假定在某一国家,例如美国,进行竞争的租地农场主的人数还很少,土地占有还不过是形式,每一个人都可以找到空闲的土地来投资耕种,而不必经过在他以前已经经营土地的所有者或租地农场主的许可。在这种情况下,除了因位于人口稠密的地带而被垄断的土地以外,在一个较长的时期内,租地农场主生产的超过平均利润的剩余价值,在他的产品的价格中可能得不到实现;他会被迫把他所得到的剩余价值与资本家同伙瓜分,这正象有些商品的剩余价值一样,它们包含的全部剩余价值如果在商品的价格中得到实现,就会提供超额利润,也就是提供超过一般利润率的利润。在这种情况下,一般利润率就会提高,因为小麦等,将同其他工业品一样,低于它的价值出卖。这种低于价值出卖的情况不会成为例外,相反,倒会阻止小麦成为其他同类商品中的例外。

第二,假定某一国家的全部土地都是一种质量,但是属于这样一种质量:如果商品包含的全部剩余价值都在商品的价格中得到实现,商品就会给资本提供普通利润。在这种情况下,不支付任何地租。地租的消失,丝毫不影响一般利润率,既不会使它提高也不会使它降低,正如其他非农产品属于这一类并不影响利润率一样。这些商品之所以属于这一类,正是因为它们的内在剩余价值等于平均利润;因此,它们不能改变这种利润的高度,相反,它们适应于这种利润而完全不影响这种利润,尽管这种利润影响它们。

第三,假定全国土地都是一类,而且这些土地如此贫瘠,投在它上面的资本的生产率如此低,以致它的产品属于剩余价值低于平均利润的一类商品。自然,在这种情况下(因为工资由于农业生产率低而普遍提高),只有在绝对劳动时间可以延长,原料(如铁等)不是农产品,或者原料(如棉花、丝等)是进口物和比较肥沃土地的产品的地方,剩余价值才处于较高的水平。在这种情况下,[农业]商品的价格包含的剩余价值必须高于它们的内在剩余价值,才能提供普通利润。一般利润率将因此降低,虽然地租并不存在。

或者,我们假定在第二种情况下土地的生产率非常低。那末,这种农产品的剩余价值等于平均利润,说明这里的平均利润本来就低,因为在农业中,12劳动小时里面,单单用来生产工资,或许就要11劳动小时,而剩余价值只有1小时或者更少。

[456]这几种不同的情况说明:

在第一种情况下,地租的消失或不存在,是同一个与地租已经发展的其他国家相比提高了的利润率联系着、并存着的。

在第二种情况下,地租的消失或不存在丝毫不影响利润率。

在第三种情况下,地租的消失或不存在——与有地租存在的其他国家相比——是同一个低的、较低的一般利润率联系着的,并且是一般利润率水平低的标志。

由此可见,一个特殊的地租的发展,就其本身来说,同农业劳动的生产率是绝对无关的,因为地租的不存在或者消失既可以同一个提高的利润率联系着,也可以同一个保持不变的利润率联系着,也可以同一个下降的利润率联系着。

这里的问题不在于为什么在农业等部门剩余价值超过平均利润的余额被扣留下来;相反,问题倒在于:由于什么原因这里竟要发生相反的现象?

剩余价值无非是无酬劳动;平均利润,或者说正常利润,无非是假定由每一个一定量的资本实现的无酬劳动量。如果说平均利润是10%,那末这不过是说,一个100单位的资本摊到10单位无酬劳动;或者说,等于100的物化劳动支配相当于本身数额的1/10的无酬劳动。因此,剩余价值超过平均利润的余额是指:商品中(商品的价格中,或者说,由剩余价值构成的那部分商品价格中)包含的无酬劳动量,大于形成平均利润的无酬劳动量,因而大于商品的平均价格中构成商品价格超过商品生产费用的余额的无酬劳动量。在单个商品中,生产费用代表预付资本,超过这个生产费用的余额代表预付资本所支配的无酬劳动;因此,这个价格余额与生产费用之比,代表用于商品生产过程的一定量资本支配无酬劳动的比率,而不管该特殊生产领域的商品所包含的无酬劳动是否等于这个比率。

那究竟是什么东西迫使单个资本家例如按照平均价格出卖他的商品?(这个平均价格作为某种已经形成的东西强加于资本家,这决不是他的自由行动。他是更愿意高于商品价值出卖商品的。)究竟是什么东西迫使资本家按照这种只向他提供平均利润,使他实现的无酬劳动小于他商品中实际包含的无酬劳动的价格出卖他的商品呢?迫使他这样做的,是其他资本通过竞争所施加的压力。如果A生产部门的无酬劳动对预付资本(例如100镑)之比大于B、C等生产领域(B、C等生产领域的产品,完全同A生产领域的商品一样,以其使用价值满足某种社会需要),任何同量资本也就会涌向A生产部门。

因此,如果存在这样一些生产领域,那里的某些自然生产条件,如耕地、煤层、铁矿、瀑布等,——没有这些条件,生产过程就无从进行,这些领域的商品就不能生产,——不是掌握在物化劳动的所有者或占有者资本家的手里,而是掌握在其他人的手里,那末这第二类的生产条件所有者就对资本家说:

如果我让你使用这些生产条件,那你将赚你的平均利润,占有正常的无酬劳动量。但是你的生产提供一个超过利润率的剩余价值余额,即无酬劳动余额。这个余额,你不应象你们资本家们通常做的那样,投进总库。这个余额我来占有,它是属于我的。这种交易会使你完全满意,因为你的资本在这个生产领域给你提供的,同在其他任何领域一样多,并且,这是一个十分稳定的生产部门。你的资本在这里除了给你提供构成平均利润的那10%的无酬劳动以外,还给你提供20%的超额无酬劳动。你要把这个付给我,为了能够这样做,你要把这20%的无酬劳动加在商品的价格上,但是不要把它算入你和其他资本家的总账。你对一种劳动条件——资本,物化劳动——的所有权,使你能够占有工人的一定数量的无酬劳动,同样,我对另一种生产条件——土地等等——的所有权,使我能够从你和整个资本家阶级那里扣下无酬劳动中超过你的平均利润的那个余额。你们的规律要求在正常情况下等量资本占有等量无酬劳动,你们资本家可以[457]通过竞争彼此强制做到这一点。好吧!我正要把这个规律应用到你的身上。你从你的工人的无酬劳动中占有的,不要多于你用同一笔资本在其他任何生产领域所能占有的。但是,这个规律同你“生产”的那个超过无酬劳动正常量的余额是毫无关系的。谁能阻止我占有这个“余额”呢?我为什么要象你们那样,把它投入资本的大锅,供资本家阶级内部分配,使每个人按他在总资本中拥有的股份取得这个余额的一定部分呢?我不是资本家。我让你使用的生产条件不是物化劳动,而是自然的赐予。你们能制造土地、水、矿山或者煤层么?不能。因此,可以用到你身上、使你把你自己侵吞的剩余劳动吐出一部分来的那种强制手段,对我来说是不存在的!所以,拿来吧!你的资本家同伙能做的唯一事情,不是同我竞争,而是同你竞争。如果你付给我的超额利润,小于你占有的剩余劳动时间与依照资本的规律你应得的那份剩余劳动之间的差额,你的资本家同伙就会出面,通过竞争,逼你把我能从你那里挤出的全部数额老老实实支付给我。

现在本来应该研究:(1)从封建土地所有制到另一种由资本主义生产调节的商业地租的过渡;或者,另一方面,从这种封建土地所有制到自由的农民土地所有制的过渡;(2)在土地最初不是私有财产而资产阶级生产方式至少在形式上一开始就占统治地位的一些国家,如美国,地租是怎样产生的;(3)仍然存在着的土地所有制的亚洲形式。但是这一切都不是这里要谈的。

这样,按照我们所谈的理论,对于自然对象如土地、水、矿山等的私有权,对于这些生产条件,对于自然所提供的这种或那种生产条件的所有权,不是价值的源泉,因为价值只等于物化劳动时间;这种所有权也不是超额剩余价值即无酬劳动中超过利润所包含的无酬劳动的余额的源泉。但是,这种所有权是收入的一个源泉。它是一种权利,一种手段,使这一生产条件的所有者能够在他的所有物作为生产条件加入的生产领域中占有被资本家榨取的无酬劳动的一部分,否则这一部分会作为超过普通利润的余额被投进资本总库中去。这种所有权是一种手段,它能阻止在其余资本主义生产领域发生的上述过程发生,并且把这个特殊生产领域所生产的剩余价值扣留在这个领域中,于是剩余价值现在就在资本家和土地所有者之间进行分配。因此,土地所有权,就象资本一样,变成了支取无酬劳动、无代价劳动的凭证。在资本上,工人的物化劳动表现为统治工人的权力,同样,在土地所有权上,土地所有权使土地所有者能从资本家那里扣下一部分无酬劳动的这种情况,表现为土地所有权似乎是价值的一个源泉。

这就说明了现代地租,说明了它的存在。在投资相等的条件下,地租量不等,只能用土地的肥力不同来说明。在肥力相等的条件下,地租量不等,只能用投资量不等来说明。在前一种情况下,地租增加是因为地租对所投资本(也对土地面积)的比率提高了。在后一种情况下,地租增加是因为在同一比率下,甚至在不同比率下(如果第二笔投入土地的资本的生产率较低的话)地租量增加。

按照这个理论,最坏的土地无论完全不提供地租,或者提供地租,都不是必然的。其次,完全没有必要假定农业生产率减低,虽然,生产率的差异,如果不是人为地加以排除(这是可能的),在农业中比在同一工业生产领域内要大得多。我们谈生产率的高低,指的总是同种产品。至于不同产品之间的关系,那是另外一个问题。

按土地本身计算的地租是地租总额,地租量。地租率不提高,地租也可能增加。如果货币价值不变,农产品的相对价值可能提高,但不是因为农业生产率降低,而是因为农业生产率虽然提高,但是提高的程度不如工业。相反,如果货币价值不变,农产品货币价格的提高,只有在农产品价值本身提高的时候,也就是说,只有在农业生产率降低的时候,才有可能(这里不谈需求对于供给的暂时压力,象其他商品经常发生的情况那样)。

在棉纺织工业中,原料价格随着工业本身的发展不断下降;在制铁、煤炭等工业中情况也是一样。这里,地租的增加只可能由于使用了更多的资本,而不是由于地租率提高。

李嘉图认为:空气、光、电、蒸汽、水这些自然力是白白取得的,土地就不是这样,因为土地是有限的。因此,照李嘉图看来,仅仅由于这一点,农业的生产率已经不如其他生产部门。如果土地象其他自然要素和自然力一样,属于大家而不被占有,要多少有多少,那末,照李嘉图看来,生产率就会高得多。

[458]首先必须指出,假如土地作为自然要素供每个人自由支配,那末,资本的形成就缺少一个主要要素。一个最重要的生产条件,而且是——如果不算人本身和人的劳动——唯一原始的生产条件就不能转让、占有,因而不能作为别人的财产同劳动者对立并因此把他变成雇佣工人。这样一来,李嘉图意义上的即资本主义意义上的劳动生产率,无酬的别人劳动的“生产”,就不可能了。这样一来,资本主义生产就根本完结了。

至于李嘉图列举的那些自然力,一部分的确可以白白取得,它们不要资本家花费什么。煤使资本家花了费用,但是如果资本家白白取得水,蒸汽就不要他花费什么。但是现在我们以蒸汽为例。蒸汽的属性是永远存在的。生产上利用蒸汽,是一个已被资本家据为己有的新的科学发现。由于这个发现,劳动生产率提高了,从而相对剩余价值也提高了。这就是说,资本家从一个工作日中占有的无酬劳动量由于利用蒸汽而增加了。因此,蒸汽的生产力同土地的生产力之间的差别,仅仅在于前者给资本家带来无酬劳动,后者则给土地所有者带来无酬劳动,这种无酬劳动,土地所有者不是[直接]从工人手上而是从资本家手上取去的。因此,资本家也就热中于“废除”这个自然要素的“所有权”。

李嘉图对问题的提法中只有下面一点是正确的:

在资本主义生产方式下,资本家不仅是一个必要的生产当事人,而且是占统治地位的生产当事人。相反,土地所有者在这种生产方式下却完全是多余的。资本主义生产方式所需要的只是:土地不是公共所有,土地作为不属于工人阶级的生产条件同工人阶级对立。如果土地国有,因而国家收地租,这个目的就完全达到了。土地所有者,在古代世界和中世纪世界是那么重要的生产当事人,在工业世界中却是无用的赘疣。因此,激进的资产者在理论上发展到否定土地私有权(而且还打算废止其他一切租税),想把土地私有权以国有的形式变成资产阶级的、资本的公共所有。然而,他们在实践上却缺乏勇气,因为对一种所有制形式——一种劳动条件私有制形式——的攻击,对于另一种私有制形式也是十分危险的。况且,资产者自己已经弄到土地了。

\tsectionnonum{[(4)洛贝尔图斯关于农业中不存在原料价值的论点是站不住脚的]}

现在谈谈洛贝尔图斯先生。

按照洛贝尔图斯的意见,在农业中是根本不计算原料的,因为,洛贝尔图斯肯定说,德国农民不把种子、饲料等算作自己的支出,不计算这些生产费用,也就是说,计算错误。这样说来,在租地农场主进行正确计算已有150年以上的英国,就不应该存在任何地租。因此,洛贝尔图斯从这里得出的不应该是这个结论:租地农场主支付地租是因为他的利润率比工业中的利润率高;而应该完全是另一个结论:租地农场主支付地租是因为他由于计算错误而满足于较低的利润率。本人是租地农场主的儿子,并且十分熟悉法国租佃关系的魁奈医生,是不会欣然同意洛贝尔图斯的。魁奈在“预付”项目下,在“年预付”中,把租地农场主所使用的“原料”价值计算为10亿,尽管租地农场主会把这个原料以实物形式再生产出来。

在一部分工业中几乎完全不存在固定资本,或者说,机器设备,而在另一部分工业中,在整个运输业中,即在(用马车、铁道、船舶等)使位置发生变化的工业中,则根本不使用原料而只使用生产工具。这些工业部门除了利润之外是否还提供地租呢?这种工业部门同例如采矿工业有什么区别呢?在这两种场合,都只有机器设备和辅助材料,例如轮船、火车头和矿山所用的煤,马的饲料等。为什么利润率的计算在一种生产部门中要不同于另一种生产部门呢?假定农民用在生产上的实物形式的预付占他的全部预付资本的1/5,另外,用于购买机器和支付工资的预付占4/5,而全部支出[按价值]等于150夸特。其次,如果农民得到10%的利润,那末利润就是15夸特。因此,总产品等于165夸特。如果农民从他的预付资本中扣除1/5,即30夸特,而15夸特只用120夸特来除,他的利润就等于[12+(1/2)]%。

我们还可以这样来说明。农民的产品价值或他的产品等于165夸特(330镑)。他把自己的预付计算为120夸特(240镑)。这笔预付的10%就是12夸特(24镑)。但是他的总产品等于165夸特,因此,其中总共扣去132夸特[补偿货币支出和它的10%的利润],余下33夸特。但是在33夸特中30夸特是以实物形式支出的。于是剩下3夸特(=6镑)作为超额利润。这个农民的总利润等于15夸特(30镑)而不是12夸特(24镑)。因此,他能够支付3夸特或6镑的地租,并且可以认为同其他任何资本家一样得到了10%的利润。但是这个10%只存在于想象中。实际上他预付的不是120夸特,而是150夸特,它的10%就是15夸特或30镑。实际上他少得了3夸特,即他已经得到的12夸特的1/4,[459]换句话说,他少得了他应该得到的全部利润的1/5,——这是因为,他没有把自己预付的1/5当作支出计算进去。因此,只要农民学会按资本家的方法计算,他就会立刻停止支付地租,因为地租正等于他的利润率同普通利润率之间的差额。

换句话说,包含在165夸特中的无酬劳动产品等于15夸特,或30镑,或30劳动周。如果这30劳动周或15夸特或30镑用150夸特的总预付来除,那末结果只是10%;如果只用120夸特来除,那就会得出较高的利润率。因为用120夸特除12夸特的结果是10%,而用120夸特除15夸特,则不是10%,而是[12+(1/2)]%。这就是说:虽然农民作了上述实物预付,但是因为他没有按照资本家的方法把它们计算进去,所以他不是用自己的全部预付额来除他所积攒的剩余劳动。这样一来,这个剩余劳动就会代表比其他生产部门高的利润率,就能提供地租,因此,这个地租完全是基于计算错误。如果农民知道,为了用货币来计算他的预付,并因此把这种预付看作商品,他完全没有必要预先把它变成实在货币,即把它出卖,那末整个故事也就完了。

没有这种计算错误(许多德国农民会犯这种错误,但是没有一个资本主义租地农场主会犯这种错误),洛贝尔图斯的地租就不可能存在。这种地租只有在原料加入生产费用的地方才有可能存在,而在原料不加入生产费用的地方就不可能存在。它只有在原料加入生产而不被生产者计算的地方才有可能存在。但是它在原料不加入生产的地方就不可能存在,——尽管洛贝尔图斯先生不是想从计算错误得出地租,而是想从预付中缺少一个实际项目得出地租。

我们以采矿工业或渔业为例。原料在这里只是作为辅助材料加入生产,但这一点我们可以撇开不谈,因为机器的采用也总是(除了极少数例外)以辅助材料即机器的生活资料的消费为前提。假定一般利润率为10%。100镑用在机器和工资上。为什么因为这100不是用在原料、机器和工资上[而仅仅用在机器和工资上],它的利润就要大于10呢?或者说,为什么因为这100仅仅用在原料和工资上,它的利润就要大于10呢?如果说这里存在某种差别,那末引起这种差别的原因只能是:在不同情况下不变资本和可变资本的价值之比一般是不同的。这种不同的比例即使在剩余价值率假定不变的情况下也会提供不同的剩余价值。而不同的剩余价值对等量资本之比,必然得出不同的利润。但是,从另一方面说,一般利润率正是意味着把这些差别拉平,把资本的有机组成部分抽象化,把剩余价值分配得使等量资本提供等量利润。

剩余价值量取决于所支出的资本量,这种情况——按照剩余价值的一般规律——绝对不适用于不同生产领域的各个资本,而适用于同一生产领域(在这里,假定资本的有机组成部分之间的比例是相同的)的各个不同资本。如果我举例说,假定在纺纱业中,利润量同所支出的资本量相适应(此外还假定生产率不变,否则这种说法也不完全正确),那末我实际上只是说,在对纺纱工人的剥削率既定的情况下,剥削量取决于被剥削的纺纱工人的人数。相反,如果我说,各个不同生产部门的利润量同所支出的资本量相适应,那末,这就是说,各个一定量资本的利润率都相同,即利润量只能随这个资本量的变化而变化,换句话说,这又意味着利润率不取决于某一单个生产领域中资本组成部分之间的有机比例,它完全不取决于这些单个生产领域中所创造的剩余价值量。

采矿业一开始就应该属于工业,而不属于农业。是什么理由呢?理由就是:没有一种矿产品以实物形式,即以从矿山开采出来时的形式,作为生产要素重新加入矿山所使用的不变资本(渔业和狩猎业的情况也是这样,在这里,支出在更大程度上只限于劳动资料,工资或者说劳动本身)。[460]换句话说,这是由于这里的每一个生产要素,即使它的原料是从矿山开采出来的,在它重新作为要素加入矿业生产之前,不仅先要改变自己的形式,而且要变成商品,即必须被买进来。唯一的例外是煤。但是煤作为生产资料出现只是在这样一个发展阶段,那时矿业主已经成了训练有素的资本家,他用复式簿记记帐,按照这种簿记,不仅他把自己的预付记成对自己的负债,不仅他对自己的基金来说是债务人,而且他的基金对于基金本身也成了债务人。由此可见,恰恰在实际上没有原料加入支出的地方,必然一开始就普遍采用资本主义会计,因而不可能犯农民会犯的错误。

我们现在来看加工工业本身,特别是其中这样一个部分,在这里,劳动过程的一切要素同时作为价值形成过程的要素出现,因此,一切生产要素同时作为支出,作为具有价值的使用价值,即作为商品加入新商品的生产。这里,在生产最初的半成品的制造业者同第二个以及所有以后的(按照生产阶段的序列)制造业者之间有着根本的区别,在后者那里,原料不仅作为商品加入生产,而且已经是二次方的商品,也就是说,这个商品已经取得了不同于最初商品即原产品的自然形式的形式,已经经历了生产过程的第二阶段。以纺纱业者为例。他的原料是棉花,但棉花是已经作为商品的原产品。而织布业者的原料是纺纱业者的产品棉纱,印染业者的原料是织布业者的产品布,而所有这些在生产过程的下一阶段重新作为原料出现的产品同时又是商品。\endnote{马克思接着在手稿中草拟了一个棉花种植业者、纺纱业者和织布业者的例子。他从这三人中间每一个人单独获得的利润问题转而考察在假定织布业者同时也是棉花种植业者和纺纱业者的情况下获得多少利润的问题。但是马克思不满意已写好的东西,他把已经开始起草的东西停下来并且把它全划掉了,随后如正文中所作的那样精确地表述了自己的思想。——第43页。}[460]

[461]这里,我们显然又遇到了已经两次涉及的问题,一次是在考察约翰·斯图亚特·穆勒的观点的时候\endnote{马克思指的是第VII和第VIII本(手稿第319—345页)中篇幅很长的关于约翰·斯图亚特·穆勒的补充部分。按照马克思所编的《剩余价值理论》目录以及他在手稿第VII本(第319页)正文中所作的指示,本版把关于约翰·斯图亚特·穆勒这一节移至《理论》第三册中关于李嘉图学派的解体那一章。在马克思手稿第332—334页即该章专论约翰·斯图亚特·穆勒的第七节中,阐明了这样一个问题:成品的生产和生产这个成品的不变资本的生产结合在一个资本家手里,会不会影响利润率。——第43页。},后来一次是在一般考察不变资本和收入的相互关系的时候\fnote{见本卷第1册第128—129和221页。——编者注}。这个问题一再出现,就说明事情还有些棘手。这个问题本来属于论述利润的第三章\endnote{马克思指他的研究中后来发展成《资本论》第三卷的那一部分。——第44、187页。}。不过在这里谈一谈比较好。

我们举一个例子:

4000磅棉花=100镑;

4000磅棉纱=200镑;

4000码棉布=400镑。

根据这个假定,1磅棉花=6便士,1磅棉纱=1先令,1码棉布=2先令。

假定利润率等于10%,那末,

100镑(A)中——支出=90+(10/11),利润=9+(1/11)。

200镑(B)中——支出=181+(9/11),利润=18+(2/11)。

400镑(C)中——支出=363+(7/11),利润=36+(4/11)。

A是农民(I)的产品棉花;B是纺纱业者(II)的产品棉纱;C是织布业者(III)的产品布。

在这个假定中,产品A的90+(10/11)镑本身是否包含利润,这是完全无关紧要的。如果这个90+(10/11)镑是自行补偿的不变资本,它就不包含利润。同样,[代表产品A的价值的]100镑是否包含利润,对B来说也是无关紧要的。至于产品B,对C来说也是如此。

棉花种植业者(I)、纺纱业者(II)和织布业者(III)的情况如下:

(I)支出——90+(10/11),利润——9+(1/11)。总额——100。

(II)支出——{100(I)+[81+(9/11)]},利润——18+(2/11)。总额——200。

(III)支出——{200(II)+[163+(7/11)]},利润——36+(4/11)。总额——400。

全部总额等于700。

利润等于9+(1/11)+[18+(2/11)]+[36+(4/11)]=63(7/11)。

三个部门的预付资本等于90+(10/11)+[181+(9/11)]+[363+(7/11)]=636+(4/11)。

700超过636+(4/11)的余额等于63+(7/11)。而[63+(7/11)]∶[636+(4/11)]=10∶100。

我们继续分析这个荒唐的例子,就会得出:

(I)支出——90+(10/11),利润——9+(1/11)。总额——100。

(II)支出——{100(I)+[81+(9/11)]},利润——{10+[8+(2/11)]}。总额——200。

(III)支出——{200(II)+[163+(7/11)]},利润——{20+[16+(4/11)]}。总额——400。

棉花种植业者(I)对谁也不必支付利润,因为假定他的90+(10/11)镑不变资本不包含利润,而只代表不变资本。棉花种植业者(I)的全部产品作为不变资本加入纺纱业者(II)的支出。[II的产品中代表]100镑不变资本的部分补偿给棉花种植业者9+(1/11)镑利润。纺纱业者(II)的等于200镑的全部产品加入织布业者(III)的支出;因此,[织布业者的不变资本]补偿18+(2/11)镑利润。但是,这并不妨碍棉花种植业者的利润丝毫也不比纺纱业者和织布业者的利润多,因为按同一比例,应该由棉花种植业者补偿的资本小了,而利润是同资本量相适应的,同这个资本由哪些部分组成完全没有关系。

现在假定织布业者(III)自己生产这一切。这时从表面上看,事情有了变化。[但是实际上利润率在这种情况下不变。]织布业者的支出现在采取了如下形式:90+(10/11)投入棉花生产,181+(9/11)投入棉纱生产,363+(7/11)投入棉布生产。他把这三个生产部门都买下来,因而他必须在每一个生产部门中都投入一定的不变资本。我们把这几笔资本加起来的总额是:90+(10/11)+[181+(9/11)]+[363+(7/11)]=636+(4/11)。这个总额的10%恰恰是63+(7/11),这同上面一样,——不过现在是全部由一个人放进自己的腰包里,而以前这63+(7/11)是在I、II和III之间进行分配的。

[462]这种迷惑人的假象[似乎利润率在这种情况下发生了变化]是从什么地方产生的呢?

但是,还有一点先要说一说。

如果我们从400中扣除织布业者的利润36+(4/11),余下363+(7/11),这是织布业者的支出。在这笔支出中,200是支付棉纱的。这200中有18+(2/11)是纺纱业者的利润。如果我们从363+(7/11)的支出中扣除这18+(2/11),余下345+(5/11)。但是,除此以外,在补偿给纺纱业者200的中还包含棉花种植业者的利润9+(1/11)。如果我们从345+(5/11)中扣除9+(1/11),余下336+(4/11)。如果我们从布的总价值400中扣除这336+(4/11),那就可以看出,其中包含等于63+(7/11)的利润。

但是,63+(7/11)的利润除以336+(4/11),等于[18+(34/37)]%。

以前,这63+(7/11)是除以636+(4/11),利润为10%。总价值700超过636+(4/11的余额恰好是63+(7/11)。

这样,同一笔资本100的利润,照我们新的计算是[18+(34/37)]%,而照以前的计算——只有10%。

这两者怎么一致呢?

我们假定I、II、III是同一个人,但是他不是同时使用自己的三笔资本(一笔用于种植棉花,另一笔用于纺纱,第三笔用于织布),而是这样使用:他只是在完成棉花种植工作以后才开始纺纱,并且只是在完成纺纱以后才着手织布。

于是计算如下:

这个资本家支出90+(10/11)镑用于种植棉花,得到4000磅棉花。为了把所有这些棉花纺成纱,他必须在机器、辅助材料和工资上再支出81+(9/11)镑。他用这些纺出4000磅棉纱。最后,他把这些棉纱变成4000码布,这又需要他支出163+(7/11)镑。现在把他的全部支出加起来,他的预付资本就是90+(10/11)镑+[81+(9/11)]镑+[163+(7/11)]镑即336+(4/11)镑。这笔总额的10%是33+(7/11),因为[336+(4/11)]∶[33+(7/11)]=100∶10。但是336+(4/11)镑+[33+(7/11)]镑=370镑。因而,他是按370镑而不是按400镑出卖4000码布,便宜了30镑,也就是说比以前便宜[7+(1/2)]%。如果布的价值的确等于400镑,那末他能够按普通利润10%出卖商品,另外还能支付30镑地租,因为他的利润率不等于33+(7/11)对预付336+(4/11)之比,而等于63+(7/11)对336+(4/11)之比,——这就是说,同我们上面看到的一样,利润率为[18+(34/37)]%。

看来,这实际上就是洛贝尔图斯先生计算地租的方法。

错误究竟在哪里?首先可以看到,如果纺纱和织布互相结合在一起,它们[照洛贝尔图斯的看法]就必然象纺纱和农业结合在一起或者农业单独经营一样提供地租。

这里显然是两件不同的事情。

第一,我们只用336+(4/11)镑的一笔资本来除63+(7/11)镑,而我们本来应该用总价值636+(4/11)镑的三笔资本来除63+(7/11)镑。

第二,我们把最后一笔资本(III)的支出算作336+(4/11)镑而不是363+(7/11)镑。

这两点需要分别加以分析。

第一,如果一个资本家兼有棉花种植业者、纺纱业者和织布业者的三种身分,他把所收获的全部产品棉花纺成纱,那末,他就绝对没有把这种收获的任何一部分用于补偿自己的农业资本。他不是[同时]把他的资本的一部分用于[463]种植棉花,——用于种植棉花所需的各种费用,用于种子、工资、机器,——把另一部分用于纺纱,而是先把他的资本的一部分投入种植棉花,以后把这部分加上第二部分投入纺纱,然后把已经包含在棉纱中的前两部分再加上第三部分投入织布。最后织成了4000码布,这时他怎样补偿这些布的生产要素呢?当他织布时他并不纺纱,并且也没有纺纱所必需的材料,而当他纺纱时,他不种植棉花。因此,他的生产要素不可能由他来补偿。如果我们自行解脱地说:是的,这个家伙把这4000码卖掉,然后从卖得的400镑中拿出一部分来“购买”棉纱以及棉花的要素。但是这样做会得出什么结论呢?结论只能是,我们实际上承认有三笔资本,它们同时被使用,被投入经营,被预付到生产上。要能买到棉纱就必须有棉纱,要能买到棉花也就必须有棉花,而要使市场上有棉花和棉纱,因而能代替已经织掉的棉纱和纺掉的棉花,生产它们的资本就必须和投入织布的资本同时投入经营,必须在棉纱变成布的同时变成棉花和棉纱。

因此,无论是III把所有三个生产部门结合在一起,或者是这三个生产部门由三个生产者分担,都必须有三笔资本同时存在。如果一个资本家想以同一规模进行生产,他就不可能把他用来织布的同一笔资本用来纺纱和种植棉花。这些资本中的每一笔资本都已投入生产,而它们之间互相补偿这一点同我们要研究的问题毫无关系。互相补偿的资本是不变资本,它们必须同时投入三个部门中的每一个部门,并且同时发挥作用。如果说400镑中包含利润63+(7/11)镑,这只是因为III除了自己的利润36+(4/11)镑以外还得到——根据我们的计算——他应该付给生产者II和I的利润,而这笔利润根据假定是在他的商品中实现的。但是,I和II不是从III的363+(7/11)镑得到利润,而是土地耕种者单独从自己的90+(10/11)镑得到利润,纺纱业者从自己的181+(9/11)镑得到利润。如果III拿到全部利润,那末他仍然不是从他投入织布的363+(7/11)镑得到的,而是从这笔资本加上他投入纺纱和种植棉花的另外两笔资本得到的。

第二,如果我们把III的支出算作336+(4/11)镑而不是363+(7/11)镑,那是由于:

我们把织布业者用于种植棉花的支出仅仅计算为90+(10/11)镑而不是100镑。但是他需要棉花种植业的全部产品,这全部产品是100镑而不是90+(10/11)镑。9+(1/11)镑的利润已经包含在这全部产品中了。否则,他就是使用了一笔90+(10/11)镑没有给他提供任何利润的资本。种植棉花就没有给他带来利润,而仅仅补偿90+(10/11)镑的支出。同样,纺纱也没有给他带来利润,纺纱的全部产品仅仅补偿支出。

在这种情况下,他的支出实际上是90+(10/11)+[81+(9/11)]+[163+(7/11)]=336+(4/11)。这就是他的预付资本。这笔资本的10%是33+(7/11)镑。这时产品价值就等于370镑。这个产品价值决不会更高,因为,根据假定,前面两个部分I和II没有带来任何利润。因此,如果III不插手I和II的部门而保持原来的生产方法,他的情况就会好得多。因为现在III自己只有33+(7/11)镑,而不是象以前那样由I、II、III共同消费63+(7/11)镑,以前他的同伙同他一起分享利润,他倒还得到36+(4/11)镑。他真的成了一个很不中用的生意人了。他在II部门里能节约9+(1/11)镑的支出,只是因为他在I部门里没有得到利润,他在III部门里能节约18+(2/11)镑的支出,只是因为他在II部门里没有得到利润。他种植棉花所得到的90+(10/11)镑以及他纺纱所得到的81+(9/11)+[90+(10/11)]镑,都只会自己补偿自己。只有投入织布的第三笔资本90+(10/11)+[81+(9/11)]+[163+(7/11)],才带来10%的利润。这也就是说,100镑在织布时提供10%的利润,但是在纺纱和种植棉花时不提供丝毫利润。这对III来说,只要I或II不是自己而是别人,确实是非常惬意的,但是当他打算把三个生产部门结合于他尊贵的一身而把这点节约下来的宝贝利润占为己有时,那就一点也不惬意了。因此,用于预付利润(或者说,一个部门的不变资本中[464]对其他两个部门来说是利润的那个组成部分)的支出所以会节约下来,是因为I和II两个部门的产品实际上不包含任何利润,在这些部门中没有完成任何剩余劳动;这些部门仅仅把自己看作雇佣工人,只给自己补偿自己的生产费用即不变资本和工资的支出。但是在这些情况下——只要I和II不愿意例如为III劳动,但是利润就会因此进入后者账内——所完成的劳动因此总是会减少,而且III必须支付代价的劳动是光用在工资上还是用在工资和利润上,对他来说是完全一样的。这对他来说是一回事,只要他所购买和支付代价的是产品,是商品。

不变资本是全部还是部分以实物形式得到补偿,也就是说,不变资本是否由把它作为不变资本的那种商品的生产者来补偿,是完全无关紧要的。首先,任何不变资本最终都必须以实物形式得到补偿:机器由机器补偿,原料由原料补偿,辅助材料由辅助材料补偿。在农业中,不变资本也可以作为商品加入生产,也就是可以直接通过买卖加入生产。当然,只要加入再生产的是有机物,不变资本就必须用本生产领域的产品来补偿。但是不一定要由这个生产领域内的同一个生产者自己来补偿。农业越是发达,它的一切要素也就越是不仅形式上,而且实际上作为商品加入农业,也就是说,这些要素来自外部,是另外一些生产者的产品(种子、肥料、牲畜、畜产品等)。在工业中,例如铁不断地转移到机器制造厂,而机器不断地转移到铁矿,这种情形同小麦从谷仓转移到土地又从土地转移到租地农场主的谷仓一样,是经常发生的。在农业中,产品直接补偿自己。铁不能补偿机器。但是,同机器价值相等的一定量铁[在制铁业者和机器制造业者进行交换时]给前者补偿机器而给后者补偿铁,因为[机器制造业者卖给制铁业者的]机器本身按价值来说由铁补偿。

根本不能想象,如果土地耕种者把他用在100镑产品上的90+(10/11)镑,比如说,这样来计算:20镑用于种子等,20镑用于机器等,50+(10/11)镑用于工资,那末利润率会有什么差别。因为他对这笔总额要求10%的利润。被他当作种子的20镑产品不包含利润。但是,它们是同包含例如10%利润的以机器为形式的20镑完全一样的20镑。诚然,这可能仅仅在形式上如此。以机器为形式的20镑,实际上可能同以种子为形式的20镑一样,也不代表任何利润。例如,当上述20镑仅仅补偿机器制造业者的不变资本中那些取自比如农业的组成部分的时候,情形就是这样。

认为一切机器都作为农业的不变资本加入农业,是错误的,同样,认为一切原料都加入加工工业,也是错误的。相当大一部分原料留在农业中,它仅仅是不变资本的再生产。另一部分作为生活资料直接加入收入中,而且其中一部分如水果、鱼、牲畜等,不通过任何“制造过程”。因此,要工业替农业所“制造”的全部原料付款,是不正确的。当然,那些除了工资、机器之外还有原料作为预付加入生产的加工工业部门,同提供这种作为预付加入生产的原料的那些农业部门比起来,预付资本必然较大。也可以假定,如果在这些加工工业部门中存在自己的(不同于一般利润率的)利润率,那末在这里这种利润率就要小于农业中的利润率,其原因正是由于在这里使用的劳动较少。因此,在剩余价值率相同的情况下,较大的不变资本和较小的可变资本,必然提供较小的利润率。但是,这一点也适用于加工工业的一定部门同加工工业的另一些部门的关系,以及农业(在经济学的意义上)的一定部门同农业的另一些部门的关系。至少在真正的农业中恰恰存在这种情况,因为农业虽然为工业提供原料,但是在本身的领域中仍然有原料、机器和工资作为自己的各项支出,而工业对于这种原料,即农业从自身来补偿而不是通过同工业品交换来补偿的那部分不变资本,是决不向农业支付代价的。

\tsectionnonum{[(5)洛贝尔图斯的地租理论的错误前提]}

[465]现在把洛贝尔图斯先生的思路作一概括。

首先,他照自己的想象描写了(独立经营的)土地所有者既是资本家又是奴隶主的情况。后来分离了。从工人那里剥夺来的那部分“劳动产品”——“一种实物租”——现在分成“地租和资本利润”。(第81—82页)(霍普金斯先生——见札记本\endnote{马克思指他的关于政治经济学的第XII本札记本。在这个札记本的封面上马克思亲笔写着:“1851年7月于伦敦”。马克思在这个札记本的第14页上摘录了霍普金斯的著作《关于调节地租、利润、工资和货币价值的规律的经济研究》(1822年伦敦版)中的一段话,这里所指的就是这段话,后来马克思在《剩余价值理论》手稿第XIII本(第669b页)的封面上又引用了这段话。本版在第二册的《附录》中发表了这段引文(见本册第672页)。——第52页。}——对这一点的说明还要简单粗糙得多。)然后,洛贝尔图斯先生把“原产品”和“工业品”(第89页)在土地所有者和资本家之间进行分配——一种petitioprincipii〔本身尚待证明的论据〕。[事实上是]一个资本家生产原产品,另一个资本家生产工业品。相反,土地所有者什么也不生产,他甚至也不是“原产品的所有者”。[土地所有者就是“原产品的所有者”]这个观念是洛贝尔图斯先生这种德国“地主”所特有的。在英国,资本主义生产是在工业和农业中同时开始的。

关于“资本盈利率”(利润率)形成的方法,洛贝尔图斯先生只是用下面一点来说明:现在有了以货币为形式的“表示盈利对资本之比”的盈利“标准”,从而“为资本盈利平均化提供了一个适当的尺度”。(第94页)洛贝尔图斯根本不知道,这种利润的均等同每个生产部门中“租”和无酬劳动之间的相等,是矛盾的,因此,商品的价值同它们的平均价格必然是不一致的。这个利润率对农业也是一个正常标准,因为“财产的收入只能按资本计算”(第95页),在工业中“使用着国民资本的极大部分”。(第95页)他一点也没有提到,随着资本主义生产的发展,农业本身不仅在形式上而且在实质上也发生了变革;土地所有者成为纯粹的钱袋,在生产中不再执行任何职能。在洛贝尔图斯看来:

\begin{quote}{“在工业中,还要把农业的全部产品的价值——作为材料——包括在资本内,而在原产品生产中就不会有这种情况。”(第95页)}\end{quote}

说全部产品,那是错误的。

接着洛贝尔图斯问道,扣除了工业利润即资本利润之后,是否还剩下“归原产品的租部分”,“如果有,那是由于什么原因”。(第96页)

洛贝尔图斯认为:

\begin{quote}{“原产品同工业品一样,是按耗费的劳动交换的,原产品的价值只等于它所耗费的劳动。”(第96页)}\end{quote}

的确,正如洛贝尔图斯所说的,李嘉图也认为是这样。但是这至少初看起来是错误的,因为商品不是按它们的价值,而是按不同于这些价值的平均价格交换的,并且这是由商品价值决定于“劳动时间”引起的,是由这个表面上看来同该现象相矛盾的规律引起的。如果原产品除了提供平均利润之外还提供一个不同于平均利润的地租,那末,这只有在原产品不是按照平均价格出卖的时候才有可能,而为什么会这样,这正是需要说明的。但是,我们来看看洛贝尔图斯是怎样推论的。

\begin{quote}{“我已经假定,租〈剩余价值,无酬劳动时间〉是按原产品和工业品的价值分配的,而这个价值是由耗费的劳动〈劳动时间〉决定的。”(第96—97页)}\end{quote}

我们首先来验证这第一个假定。这个假定的意思,换句话说,不过是各商品包含的剩余价值之比等于这些商品的价值之比,再换句话说,各商品中包含的无酬劳动量之比等于这些商品中包含的全部劳动量之比。如果商品A和商品B包含的劳动量之比是3∶1,那末它们包含的无酬劳动——剩余价值——之比也是3∶1。这是再错误不过的了。假定,必要劳动时间既定,等于10小时,一个商品(A)是30个工人的产品,另一个商品(B)是10个工人的产品。如果30个工人每天只劳动12小时,那末,他们所创造的剩余价值等于60小时,或等于5天(5×12),如果10个工人每天劳动16小时,那末,他们所创造的剩余价值也是等于60小时。这样,商品A的价值就等于30×12,即360劳动小时,或30个工作日{12小时=1工作日},而商品B的价值等于160劳动小时,或13+(1/3)工作日。商品A和商品B的价值之比是360∶160,即9∶4。两个商品包含的剩余价值之比是60∶60,即1∶1。在这种情况下,虽然价值之比是9∶4,剩余价值却相等。

[466]因此,首先,在绝对剩余价值不同,也就是超出必要劳动之外的劳动时间延长程度不同的时候,因而在剩余价值率不同的时候,各商品的剩余价值之比不等于这些商品的价值之比。

第二,假定剩余价值率相同,剩余价值——且不谈与流通和再生产过程有关的其他情况,——不取决于两个商品中包含的劳动的相对量,而取决于资本中用于工资的部分对用于原料和机器等不变资本的部分之比;而这个比例在价值相同的商品中可能完全不同,不论这些商品是“农产品”还是“工业品”,——这同问题根本没有关系,至少初看起来是如此。

因此,洛贝尔图斯先生的第一个假定,——如果商品价值决定于劳动时间,那末,不同商品中包含的无酬劳动量(或它们的剩余价值)就与价值成正比,——是根本错误的。从而下面的说法也是错误的:

\begin{quote}{“租是按原产品和工业品的价值分配的”,如果“这个价值是由耗费的劳动决定的”。(第96—97页)“当然这也就是说,这些租部分的量,不决定于据以计算盈利的资本的量,而决定于直接耗费的劳动——不论是农业劳动或工业劳动——加上由于工具和机器的损耗应当予以计算的劳动。”(第97页)}\end{quote}

这又错了。剩余价值量(这就是所谓“租部分”,因为洛贝尔图斯把租理解为与利润和地租不同的一般东西)只取决于直接耗费的劳动,不取决于固定资本的损耗,也不取决于原料的价值,总之,不取决于不变资本的任何部分。

当然,这种损耗决定固定资本必须依什么比例进行再生产(固定资本的生产同时取决于资本的新形成,资本的积累)。但是,在固定资本的生产中实现的剩余劳动,同这个固定资本作为固定资本加入的生产领域是没有关系的,就象例如加入原料生产的剩余劳动同上述这个生产领域没有关系一样。相反,在一切生产部门中,如果剩余价值率是既定的,剩余价值就只决定于所使用的劳动量;如果使用的劳动量是既定的,剩余价值就只决定于剩余价值率,这对于一切生产部门——对于农业、机器制造业和加工工业,都同样适用。洛贝尔图斯先生想把损耗“塞进来”,是为了把“原料”推出去。

\begin{quote}{洛贝尔图斯先生认为,相反,“包含在材料价值中的那部分资本”决不能对租部分的量有什么影响,因为“比如说,耗费在作为原产品的羊毛上的劳动,不能加入纱或布这种特殊产品所耗费的劳动”。(第97页)}\end{quote}

纺或织所需的劳动时间,取决于生产机器所必要的劳动时间即机器的价值,同取决于原料所耗费的劳动时间完全一样,或者,更确切地说,不取决于前者,同不取决于后者完全一样。机器和原料两者都加入劳动过程,但两者都不加入价值增殖过程。

\begin{quote}{“相反,原产品的价值即材料价值仍然作为资本支出包括在资本总额中,所有者就是按这个资本总额来计算作为盈利归工业品的租部分。而在农业资本中没有这一部分资本。农业不需要先于它生产的产品作为材料,生产一般是从农业开始的;在农业中同材料相似的财产部分可以说是土地本身,但土地是假定不要任何费用的。”(第97—98页)}\end{quote}

这是德国农民的观念。在农业中(除矿山、渔业、狩猎业以外,但是畜牧业决不除外),种子、饲料、牲畜、矿肥等是[467]用来生产产品的材料,而这种材料是劳动的产品。随着企业化农业的发展,这些“支出”也发展了。任何生产——只要不是指单纯的攫取和占有——都是再生产,因而都需要“先于它生产的产品作为材料”。在生产中成为结果的一切同时也是前提。大规模农业越发达,它购买“先于它生产的”产品和卖出自己的产品就越多。一旦租地农场主一般依存于出卖自己的产品,各种农产品(如干草)的价格由于农业中也有生产领域的划分而开始确定下来,这些支出也就以商品的形式——通过计算货币转化为商品——加入农业。如果农民把他出卖的一夸特小麦算作收入,却不把他下到地里的一夸特小麦算作“支出”,那末,就连他也一定会弄得晕头转向。此外,让洛贝尔图斯先生去试试在某一个没有“先于它生产的产品”的地方“开始生产”比如说麻或丝吧。这完全是荒谬之谈。

因此,洛贝尔图斯进一步得出的全部结论也是荒谬的:

\begin{quote}{“因此,对决定租部分的量有影响的两部分资本,是农业和工业共有的;但是,下面那部分资本不是它们共有的,这部分资本对决定租部分的量没有影响,却同其他部分资本加在一起来计算由上述两部分资本决定的作为盈利的租部分;这第三部分资本只有在工业资本中存在。我们曾经假定,无论原产品的价值还是工业品的价值都决定于所耗费的劳动,而租是按照这个价值在原产品和工业品的所有者之间进行分配。因此,如果说在原产品和工业品的生产中得到的租部分同有关的产品所耗费的劳动量成比例,那末,用在农业和工业中的、把这些租部分当作盈利来分配时所依据的资本(在工业中完全是依据资本,在农业中则依据工业中确定的盈利率)之间的比例,仍然不同于上述劳动量之间的比例以及由上述劳动量决定的租部分之间的比例。相反,在归原产品和工业品的租部分的量相等的情况下,工业资本大于农业资本,所大之数相当于包含在工业资本中的材料价值的总额。因为这个材料价值增大了把租部分作为盈利计算时所依据的工业资本,但是不增大盈利本身,从而引起资本盈利率(它也调节农业盈利)的下降,所以,在农业的租部分中,必然剩下一个依照这个盈利率计算盈利时所吸收不了的部分。”(第98—99页)}\end{quote}

第一个错误的前提:如果工业品和农产品按它们的价值(也就是按照生产它们所需要的劳动时间)交换,那末,它们就给自己的所有者提供等量剩余价值,或者说,等量无酬劳动。两种剩余价值之比不等于两种商品的价值之比。

第二个错误的前提:因为洛贝尔图斯已经以利润率(他把利润率称为“资本盈利率”)作为前提,所以,他那个商品按它们的价值交换的前提是错误的。一个前提排斥另一个前提。为了使(一般)利润率能够存在,商品的价值就必须已经发生形态变化而成为平均价格,或者处于不断的形态变化过程中。在这个一般利润率中,每个生产领域的由剩余价值对预付资本之比决定的特殊利润率平均化了。那末,为什么在农业中就不是这样呢?这正是问题所在。但是洛贝尔图斯先生甚至对问题的提法也不对,因为他第一,假定已经有一个一般利润率存在;第二,假定特殊利润率(以及它们之间的差异)没有平均化,也就是说商品按它们的价值交换。

第三个错误的前提:原料的价值不加入农业。实际上,这里的种子等等的预付是不变资本的组成部分,租地农场主也是把它们作为不变资本的组成部分来计算的。随着农业变成一个纯粹的企业部门以及资本主义生产在农村中确立,[468]随着农业为市场而生产,生产商品,生产为出卖而不是为自己消费的物品,农业也就计算它的支出,把支出的每个项目都看成商品,不管该物品是农业从本身(即从自己生产中)购买的还是向第三者购买的。随着产品变成商品,生产要素当然也变成商品,因为这些生产要素完完全全就是这些产品。因此,既然小麦、干草、牲畜、各种种子等等作为商品出卖,而且具有重要意义的正是出卖这些产品而不是用它们来直接消费,那末,它们也就作为商品加入生产;租地农场主如果不会把货币当作计算货币来用,他就是十足的傻瓜。最初,这只是计算的形式方面。但是,以下的情况也同样发展起来:某个租地农场主购买他在生产中支出的产品,即种子、别人的牲畜、肥料、矿物质等,同时又出卖他的收入;因此对于单个租地农场主来说,这些预付[不仅在实际上,而且]在形式上也是作为预付加入生产的,因为它们是买来的商品。(现在这些东西对于租地农场主来说已经始终是商品,是他的资本的组成部分,而当租地农场主把它们以实物形式重新投入生产的时候,他是把它们当作卖给自己这个生产者的东西看待的。)并且,随着农业的发展,随着最后的产品越来越以工厂的方式、按资本主义的生产方式生产出来,情况也就越来越是这样了。

因此,说这里有一部分资本加入工业而不加入农业,是错误的。

可见,如果照洛贝尔图斯的(错误的)前提,农产品和工业品提供的“租部分”(即剩余价值的份额)是同这些产品的价值成比例的,换句话说,如果具有等量价值的工业品和农产品为它们的所有者提供等量剩余价值,也就是包含等量无酬劳动,那末[即使在这种前提下],也绝对不会因为有一部分资本只加入工业(用于原料)而不加入农业,就发生不成比例的情况,以致比如同一剩余价值在工业中据说要按增加了这个组成部分的资本来计算,从而提供较小的利润率。要知道资本的这样一个组成部分也是加入农业的。因此,剩下要解决的问题只是:这个组成部分是否依同一比例加入农业?但是,在这里,我们遇到的是纯粹量的差别,而洛贝尔图斯先生想找的却是“质的”差别。这种量的差别在不同的工业生产领域也存在。这些差别在一般利润率中被拉平了。为什么工业和农业之间的差别(如果这种差别存在的话)不会被拉平呢?既然洛贝尔图斯先生让农业参与获得一般利润率,为什么又不让它参与形成这个一般利润率呢?当然,如果这样,他的全部理论就完了。

第四个错误的前提:洛贝尔图斯把机器等的损耗这一部分不变资本归入可变资本,即归入创造剩余价值、特别是决定剩余价值率的那部分资本,却不把原料归入其中,这是一个任意作出的错误前提。作出这个错误计算,是为了能够得到一开头就希望得到的计算结果。

第五个错误的前提:如果洛贝尔图斯先生要区别农业和工业,那末,就应当知道,由机器、工具,即由固定资本构成的资本要素是完全属于工业的。只要这个资本要素作为要素加入某个资本,它总是仅仅加入不变资本,丝毫不能提高剩余价值。另一方面,它作为工业品,是一定生产领域的结果。因此,它的价格,或者说,它在全部社会资本中所占的价值部分,同时代表一定量剩余价值(同原料的情况完全一样)。这个要素诚然加入农产品,但是它是来自工业的。洛贝尔图斯先生在计算中既然把原料看作从外面加入工业的资本要素,那末他就应该把机器、工具、容器、建筑物等看作从外面加入农业的资本要素。他就应该说,加入工业的只有工资和原料(因为固定资本只要不是原料,就是工业品,是工业自己的产品);而加入农业的只有工资[469]和机器等固定资本,因为原料只要不包含在工具等等之中,就是农产品。既然工业中少了一个生产费用“项目”,就应当研究在工业中是怎样计算的。

第六:一点不错,在采矿工业、渔业、狩猎业、林业(只指自然生长的林木)等部门,一句话,在采掘工业(对于不进行实物再生产的原产品的采掘)中,除了辅助材料之外,没有原料加入生产。这一点对于农业是不适用的。

但是,同样不错,在工业的一个很大部分即运输业中存在着同样的情况。这里的支出只用于机器、辅助材料和工资。

最后,无庸置疑,在其他一些工业部门中,相对地说,只有原料和工资加入生产,而没有机器即固定资本等加入生产,例如裁缝业等就是这样。

在所有这些情况下,利润量,即剩余价值对预付资本之比,不取决于预付资本——在扣除了可变资本即用于工资的资本部分以后——是由机器构成还是由原料构成,还是由两者一起构成,而是取决于这部分资本同用于工资的那部分资本相比有多大。因此,在不同的生产领域就必然存在着(撇开由流通引起的变化不说)不同的利润率,这些不同的利润率的平均化,恰好形成一般利润率。

洛贝尔图斯先生模糊地猜到的,是剩余价值同它的特殊形式,特别是同利润的区别。但是他不得要领,因为在他那里,问题一开始就只是要说明一定的现象(地租),而不是要揭示普遍规律。

在所有生产部门中都有再生产;但是这种同生产联系的再生产只有在农业中才是同自然的再生产一致的,在采掘工业中就不是这样。因而,在采掘工业中,实物形式的产品不再成为它本身再生产的要素{以辅助材料形式出现的场合除外}。

农业、畜牧业等和其他生产部门所不同的,第一,不是产品在这里成为生产资料,因为一切不具有个人生活资料的最后形式的工业品都是这样;即使具有这种最后形式的工业品也是这样,因为它们是生产者本身的生产资料,生产者靠消费它们把自己再生产出来,并保持自己的劳动能力。

第二,也不是农产品作为商品,即作为资本的组成部分加入生产;它们是以从生产中出来的形式加入生产的:它们作为商品从生产中出来,又作为商品再加入生产,——商品是资本主义生产的前提,又是它的结果。

因此,剩下来只是第三,产品作为本身的生产资料加入生产过程,而这个生产过程的产物就是这些产品。在机器方面也有这种情况。机器生产机器。煤帮助把煤提出矿井,煤运输煤等。在农业中,这种情况表现为自然的过程,这个过程是由人引导的,虽然它也“略微”创造人本身。而在其他生产部门,这种情况直接表现为生产的作用。

但是,如果洛贝尔图斯先生因此便认为,农产品由于作为“使用价值”(在工艺上)加入再生产时所具有的特殊形式,就不能作为“商品”加入再生产,那末,他就完全走入歧途了,他显然是根据对过去的回忆,那时,农业还不是资本主义企业,只有超过生产者本身消费的农产品的余额才变成商品,而这些产品,只要它们加入生产,对农业来说就不是商品。这是根本不了解资本主义生产方式对整个生产过程产生的影响。对资本主义生产来说,一切具有价值——因而从可能性来说是商品——的产品,也都作为商品来计算。

\tsectionnonum{[(6)洛贝尔图斯不理解工业和农业中平均价格和价值之间的关系。平均价格规律]}

假定在采矿工业中,仅仅由机器构成的不变资本等于500镑,用于工资的资本也等于500镑,那末,如果剩余价值等于40%,即200镑,利润就等于20%。

因此,现在是:

\todo{}

如果在有原料加入的加工工业部门(以及农业部门)支出同样多的可变资本,并且,如果使用这笔可变资本(即使用这一定数目的工人)需要机器等500镑,那末,实际上这里就会加上作为第三个要素的材料价值,假定这也是500镑。于是,现在是:

\todo{}

这个200镑剩余价值现在要以1500镑来除,结果只等于[13+(1/3)]%。如果在第一种情况下是运输业的话,上述例子还是适合的。如果在第二种情况下,比例是:机器100,原料400,那末利润率仍旧一样。

[470]因此,洛贝尔图斯先生所想象的是,如果在农业中在工资上支出100,加上在机器上支出100,那末,在工业中就是在机器上支出100,在工资上支出100,在原料上支出x。列成图表就是:

\todo{}

因而,利润率无论如何小于1/4。由此在I中也就产生了地租。

第一,农业和加工工业之间的这种差别是想象出来的,并不存在;因此,它对于那个决定其他一切地租形式的地租形式来说,是没有任何意义的。

第二,洛贝尔图斯先生在任何两个工业部门的利润率之间都可以找到这种差别;这种差别取决于不变资本量同可变资本量之比,而这个比例本身,又是既可以决定于原料的加入,也可以不决定于原料的加入。在既有原料又有机器加入的工业部门,原料的价值,也就是原料在总资本中所占的相对量,自然,如我在前面指出的,有非常重要的意义。\endnote{在注17提到的篇幅很长的关于约翰·斯图亚特·穆勒的补充部分(手稿第335—339页),马克思谈到“原料的低廉或昂贵对原料加工工业的重要性”。按照马克思的指示,本版把这个补充部分移至《剩余价值理论》第三册中关于李嘉图学派的解体那一章。——第64、496页。}这同地租毫无关系。

\begin{quote}{“只有在原产品的价值降到所耗费的劳动以下的时候,在农业中归原产品的整个租部分才有可能被按资本计算的盈利吸收;因为那时,这个租部分会大大减少,以至它和农业资本(虽然其中不包含材料价值)之间的百分比,跟归工业品的租部分和工业资本(虽然其中包含材料价值)之间的百分比相同;因此,只有在这种情况下,在农业中才有可能除了资本盈利以外不再剩下任何地租。但是,既然实际交换照例至少趋向于价值等于耗费的劳动这个规律,那末地租也照例存在;如果没有地租而只有资本盈利,那末,这不是象李嘉图设想的原始状态,而只是一种反常现象。”(第100页)}\end{quote}

因此,如果仍旧用前面的例子,那末情况如下(为了更清楚起见,我们只假定原料等于100镑):

\todo{}

这里,因为农产品比它的价值低16+(2/3)镑出卖,所以农业中和工业中的利润率相等,也就没有什么剩下作为地租。这个对农业来说是错误的例子即使是正确的,那末,原产品的价值降到“所耗费的劳动以下”这种情况,也只是完全符合于平均价格规律而已。其实需要说明的倒是为什么“例外地”在农业中有一部分不是这样,为什么在农业中全部剩余价值(或者至少是超过其他生产部门的剩余价值额;超过平均利润率的余额)都留在这个特殊生产部门的产品价格中,而不加入形成一般利润率的总结算。由此可见,洛贝尔图斯不知道什么是(一般)利润率和什么是平均价格。

为了把这个平均价格规律说清楚,——这比分析洛贝尔图斯的观点重要得多,——我们举五个例子。假定剩余价值率全都一样。

完全没有必要拿具有等量价值的商品来比较;商品应当只按照它们的价值来比较。为了简便起见,这里拿来比较的是等量资本所生产的商品。

[471]

这里,在I、II、III、IV、V各类(这是五个不同的生产领域)中,商品的价值分别是1100、1200、1300、1150和1250镑。如果这些商品按它们的价值进行交换,这些数目也就是它们交换时的货币价格。在所有这些生产领域,预付资本量相同,都等于1000镑。如果这些商品按它们的价值交换,那末,利润率在I只有10%,在II大一倍,即20%,在III是30%,在IV是15%,在V是25%。如果把各种利润率加起来,那末它们的和等于10%+20%+30%+15%+25%=100%。

如果拿所有五个生产领域的全部预付资本来考察,那末,它的一部分(I)提供10%,另一部分(II)提供20%,等等。全部资本平均提供的利润等于这五部分提供的平均量。这就是:

\todo{}

即20%。实际上,我们看到五个生产领域预付的5000镑资本提供的利润等于100+200+300+150+250=1000,也就是1000除以5000等于1/5,即20%。同样,如果我们计算总产品的价值,它是6000镑;超过5000镑预付资本的余额是1000镑,等于预付资本的20%,等于全部产品的1/6或[16+(2/3)]%。(这又是另一种计算法。)

因此,要使每一笔预付资本(I、II、III等等),或者同样可以说,等量资本,或者说,只按量的大小比例,也就是只按在预付总资本中所占的比例来考察的资本,从归总资本的剩余价值中确实获得自己的一份,那末,归每一笔资本的只能是20%的利润,不过必须归它的也正是这么多。[472]而要使这种情况成为可能,不同领域的产品就必须有时高于自己的价值出卖,有时多少低于自己的价值出卖。换句话说,全部剩余价值必须不是按各个生产领域生产多少剩余价值的比例,而是按预付资本的大小的比例在它们之间进行分配。所有生产领域都要按1200镑出卖自己的产品,以使产品价值超过预付资本的余额等于预付资本的1/5,即等于20%。

这种分配的结果如下:

这里我们看到,只有在一种情况下(II)平均价格等于商品的价值,因为这里的剩余价值恰好等于正常平均利润200。在所有其他情况下,都是把剩余价值从一种商品上拿走而加到另一种商品上去,有时多一点,有时少一点等等。

洛贝尔图斯先生本来应该加以说明的是,为什么在农业中不是这种情况,为什么在农业中商品必定按它们的价值,而不是按平均价格出卖。

竞争的作用是把利润平均化,也就是使商品的价值转化为平均价格。照马尔萨斯先生的说法,单个资本家希望从他的资本的每一部分都得到同样大小的一份利润\endnote{托·罗·马尔萨斯《政治经济学原理》1836年伦敦第2版第268页。马克思在《剩余价值理论》第三册《托·罗·马尔萨斯》一章中,引用和分析了马尔萨斯的这句话(手稿第765—766页)。——第68页。},——换句话说,这不过意味着资本家把资本的每一部分(不论它的有机的职能如何)都看成利润的独立源泉,资本的每一部分在他看来都是这样的源泉,——同样,对于资本家阶级来说,每一个资本家都把自己的资本看成同其他任何等量资本一样,是提供同量利润的源泉;就是说,把每笔投在单个生产领域的资本,只看成预付在总生产上的总资本的一部分;每笔资本,都按自己的量,自己的股份,按自己在总资本中所占的份额,在总剩余价值中,在无酬劳动或无酬劳动产品总量中要求自己的一份。这个假象使资本家(总之,对资本家来说,在竞争中,一切都以颠倒的形式出现),不仅使资本家,并且使某些最忠实于资本家的伪善者和文人确认,资本是一个与劳动无关的收入源泉,因为实际上各个生产领域的资本的利润,决不是单独由它自己“生产”的无酬劳动量决定的;这个利润落进盈利总额的大锅里,各个资本家都从那里按他参加总资本的份额获得自己的一份。

可见,洛贝尔图斯的阐述是无稽之谈。附带还要指出,在某些农业部门,例如在独立的畜牧业中,可变资本,即用于工资的资本同资本的不变部分比起来是极小的。

[洛贝尔图斯说:]

\begin{quote}{“租金就其本质来说总是地租。”(同上,第113页)}\end{quote}

不对。租金总是付给土地所有者的;如此而已。但是,如果象实践中常有的情况那样,租金,部分地或者全部地,是正常利润或者正常工资的扣除部分{实际的剩余价值即利润加地租,决不是工资的扣除部分,而是工人的劳动产品扣除了工资以后剩下来的部分},那末这个租金从经济学的观点来看就不是地租,并且一旦[473]竞争的条件恢复了正常工资和正常利润,这一点就立刻为实践证明了。

竞争不断使商品的价值转化为平均价格,在平均价格中,除了上表II的情况以外,一个生产领域的产品经常出现价值的追加部分,而另一个生产领域的产品则经常出现价值的扣除部分,只有这样才得出一般利润率。在可变资本对预付资本总额之比{假定剩余劳动率是既定的,相同的}符合于社会资本的平均比例的生产领域,商品的价值就等于平均价格;因此这里既没有价值的追加部分,也没有价值的扣除部分。如果由于特殊条件(这些条件这里不需要细说),在一定的生产领域,商品的价值——虽然超过平均价格——没有任何扣除(不是暂时地而是平均地),那末,全部剩余价值保持在某一个特殊生产领域,——虽然这种情况使商品的[实现的]价值提高到平均价格以上,因而提供一个大于平均利润率的利润率,——应看成是这些生产领域的特权。这里要作为特殊性、作为例外来研究和说明的,不是商品的平均价格降到它们的价值以下,——这是一般现象,是平均化的必要前提,——而是该商品为什么不同于其他商品,恰恰按它们的高于平均价格的价值来出卖。

商品的平均价格等于它的生产费用(商品中的预付资本,不论是工资、原料、机器还是其他)加平均利润。因此,如果[平均利润率是既定的]象在上述例子中那样,平均利润等于20%,即1/5,那末每个商品的平均价格等于C(预付资本)+P/C(平均利润率)。如果C+P/C等于这个商品的价值,也就是说,如果这个生产领域生产出来的剩余价值M=P,那末,商品的价值就等于它的平均价格。如果C+P/C小于商品的价值,因而这个生产领域生产出来的剩余价值M大于P,那末,商品的[实现的]价值就降低到它的平均价格水平,它的剩余价值的一部分就加到其他商品的价值上去。最后,如果C+P/C大于商品的价值,也就是M小于P,那末,商品的[实现的]价值就提高到它的平均价格水平,其他生产领域生产出来的剩余价值就加到这个商品的[内在]价值上来。

最后,如果有些商品,虽然它们的价值大于C+P/C,还是按它们的价值出卖,或者它们的[实现的]价值至少没有降到它们的正常平均价格C+P/C,那末这里一定有一些使这些商品成为例外的特殊条件在发生作用。在这种情况下,这些生产领域实现的利润就高于一般利润率。如果资本家在这里得到一般利润率,那末土地所有者就能够以地租形式取得超额利润。

\tsectionnonum{[(7)洛贝尔图斯在决定利润率和地租率的因素问题上的错误]}

我称为利润率、利率和地租率的,洛贝尔图斯称为“资本盈利的高度和利息的高度”(第113页)[和“地租的高度”]。

\begin{quote}{“资本盈利的高度和利息的高度决定于它们对资本之比……一切文明民族都把资本额100作为计算单位,也作为计算高度的标准。因此,资本额100所得的盈利量或利息量的比例数越大,换句话说,资本提供的‘百分率越大’,盈利和利息就越高。”(第113—114页)“地租和租金的高度决定于它们对一定地段之比。”(第114页)}\end{quote}

这种说法不合适。地租率首先应当按资本计算,因而应当作为商品价格超过商品生产费用和超过价格中构成利润的部分的余额来计算。洛贝尔图斯先生按英亩和摩尔根\fnote{土地面积单位,合25.53英亩。——译者注}计算,这样计算,内部联系就没有了,他抓住了[474]事物的表面形式,因为表面形式给他说明某些现象。一英亩提供的地租是地租额,是地租的绝对量。地租额在地租率不变甚至下降的情况下也可以增加。

\begin{quote}{“土地价值的高度决定于一定地段的地租的资本化。一定面积的地段的地租的资本化所提供的资本额越大,土地价值就越高。”(第114页)}\end{quote}

“高度”一词在这里是荒谬的。这个词实际上表示对什么的比例呢?资本化在利率为10%时比在20%时提供的资本额大,这是清楚的,但是这里的计算单位是100。“土地价值的高度”这整个说法,同商品价格的高或低一样,是一般化的说法。

洛贝尔图斯先生现在想研究:

\begin{quote}{“决定资本盈利的高度和地租的高度的是什么?”(第115页)}\end{quote}

\tsubsectionnonum{[(a)洛贝尔图斯的第一个论题]}

首先他研究:决定“一般租的高度”的是什么,也就是说,决定剩余价值率的是什么?

\begin{quote}{“(I)就一定的产品价值,或者一定量劳动的产品来说,或者同样可以说,就一定的国民产品来说,一般租的高度和工资的高度成反比,和一般劳动生产率的高度成正比。工资越低,租就越高;一般劳动生产率越高,工资就越低,租就越高。”(第115—116页)洛贝尔图斯说,租的“高度”——剩余价值率——取决于“这个剩下作为租的部分的大小”,也就是说,取决于从总产品中扣除了工资之后剩下来的部分的大小,在这里“产品价值中用于补偿资本的那一部分可以撇开不谈”。(第117页)}\end{quote}

这是对的(我指的是考察剩余价值时把资本的不变部分“撇开不谈”)。

洛贝尔图斯的一个有些奇怪的观点:

\begin{quote}{“如果工资降低,也就是说,如果工资今后在全部产品价值中占较小的一份,那末另一部分租{即工业利润}作为盈利计算时所依据的总资本也变小。但是,规定盈利和地租的高度的,只是转化为资本盈利或地租的价值同这个价值作为盈利或地租计算时所依据的资本或土地面积之比。因此,如果工资留下一个较大的价值作为租,那末,就要依据变小了的资本和保持不变的土地面积,来计算这个较大的作为资本盈利和地租的价值;由此得出的盈利和地租的相对量就大,因此两者合在一起,或者说一般租,就较高……假定产品价值不变……这只是因为,花费在劳动上的工资减少了,花费在产品上的劳动还没有减少。”(第117—118页)}\end{quote}

最后这句话是对的。但是,说可变资本即用于工资的资本减少,不变资本就必定减少,那就错了;换句话,说利润率{这里把剩余价值同土地面积之比等等根本不恰当的提法撇开不谈}由于剩余价值率提高就必然提高,那就错了。工资降低,比方说,是由于劳动生产率提高,而生产率的这种提高,在一切场合都表现为同一个工人在同一时间内加工更多的原料,——因此,这一部分不变资本增加了,机器和机器的价值也是这样。因此,利润率在工资减少的情况下也可能降低。利润率取决于剩余价值量,而剩余价值量不仅取决于剩余价值率,而且取决于被雇用的工人人数。

洛贝尔图斯正确地规定必要工资等于

\begin{quote}{“工人的必要生活费的总额,即对于一定国家和一定时期来说大致相同的一定实际产品量”。(第118页)}\end{quote}

[475]但是洛贝尔图斯先生把李嘉图提出来的利润和工资成反比以及这个比例决定于劳动生产率的原理叙述得非常混乱,极其笨拙。这种混乱一部分是由于他不把劳动时间作为尺度,却愚蠢地把产品量作为尺度,并且荒唐地去区别“产品价值的高度”和“产品价值的大小”。

这个好汉所谓的“产品价值的高度”,不过是指产品对劳动时间之比。如果在同一劳动时间内生产许多产品,那末产品价值即每一部分产品的价值就低,在相反的场合,结果也相反。如果1工作日以前提供100磅棉纱,后来提供200磅棉纱,那末棉纱的价值在后一种情况下比前一种情况下小了一半。在前一种情况下,1磅棉纱的价值等于1/100工作日;在后一种情况下,1磅棉纱的价值等于1/200工作日。因为工人得到的是同量产品,不管产品的价值是高还是低,也就是说,不管产品包含的是较多的劳动还是较少的劳动,所以工资和利润成反比,并且工资根据劳动生产率的不同而在总产品中占较大的部分或者较小的部分。这一点洛贝尔图斯用混乱的论点叙述如下:

\begin{quote}{“……如果工资作为工人的必要生活费是一定的实际产品量,那末工资在产品价值高的情况下必然是一个大的价值,在产品价值低的情况下必然是一个小的价值,因此,既然假定加入分配的是同一产品价值,工资在产品价值高的情况下必然吸收产品价值的相当大的部分,而在产品价值低的情况下必然吸收产品价值的小部分,其结果也必然把产品价值的一个相当大的份额或一个小的份额留下作租。但是,如果产品价值等于产品所耗费的劳动量这个定律有效,那末决定产品价值的高度的仍然只是劳动生产率,或者说,产品量对生产这些产品所花费的劳动量之比……如果同量劳动生产的产品较多,换句话说,如果生产率提高了,那末同量产品中包含的劳动就较少;反之,如果同量劳动生产的产品较少,换句话说,如果生产率降低了,那末同量产品中包含的劳动就较多。但是,劳动量决定产品的价值,而一定量产品的相对价值决定产品价值的高度……因而一般劳动生产率越高,一般租就必然……越高。”(第119—120页)}\end{quote}

可是,这种说法只有在工人生产的产品属于作为生活资料——根据传统或者由于必要——加入工人消费的那一类产品的时候,才是正确的。如果工人生产的产品不属于这一类产品,那末工人的劳动生产率对于工资和利润的相对高度,就象对于一般剩余价值量一样,是完全无关紧要的。[在这种情况下]全部产品中作为工资归工人所得的是同样大的价值部分,不管表现这个价值部分的产品数或产品量是大还是小。在这种场合,不管劳动生产率发生什么变化,产品价值的分配不会发生任何变化。

\tsubsectionnonum{[(b)洛贝尔图斯的第二个论题]}

\begin{quote}{“(II)如果在产品价值既定的情况下一般租的高度既定,那末,地租的高度和资本盈利的高度既互成反比,又分别与原产品生产和工业品生产中的劳动生产率成反比。地租越高,资本盈利就越低;地租越低,资本盈利就越高;反过来也是一样。原产品生产的劳动生产率越高,地租就越低,资本盈利就越高;原产品生产的劳动生产率越低,地租越高,资本盈利就越低;工业品生产的劳动生产率越高,资本盈利就越低,地租就越高;工业品生产的劳动生产率越低,资本盈利就越高,地租就越低。”(第116页)}\end{quote}

一开始(在第一个论题中)我们已看到李嘉图关于工资和利润成反比的规律。

现在我们看到李嘉图的第二个规律:利润和地租成反比;这个规律被洛贝尔图斯用另一种方式发挥了,或者不如说,弄乱了。

十分清楚,如果一定的剩余价值在资本家和土地所有者之间分配,前者所得越大,后者所得就越小,反过来也是一样。但是,洛贝尔图斯先生在这里还加进了他自己的一些东西,这些东西应该比较详细地加以研究。

洛贝尔图斯先生首先把下述论点当作一个新的发现:一般剩余价值{“作为一般租供分配的劳动产品价值”},也就是由资本家榨取的全部剩余价值,“是由原产品价值加工业品价值构成的”。(第120页)

一开头,洛贝尔图斯先生又向我们重述了他关于在[476]农业中不存在“材料价值”的“发现”。这次用的是以下的说法:

\begin{quote}{“归工业品的、决定资本盈利率的租部分,作为盈利计算,不仅要依据实际用于制造这种产品的资本,也要依据作为材料价值列入工厂主企业基金的全部原产品价值;但是,这种材料价值,对于归原产品的租部分——扣除了按照工业中既定盈利率〈当然!按照既定盈利率〉计算的用于原产品生产的资本的盈利,其余额就形成地租——来说,是不存在的。”(第121页)}\end{quote}

我们再说一遍:不是不存在!

我们假定地租存在,因而原产品的剩余价值的一定部分归土地所有者——这是洛贝尔图斯先生没有证明过的,按照他的思路也是不能证明的。

其次假定:

\begin{quote}{“在产品价值既定的情况下,一般租的高度〈剩余价值率〉也是既定的。”(第121页)}\end{quote}

这就是说,例如,在价值100镑的商品中,有一半,即50镑,是无酬劳动;因而,这一半形成一个用来支付剩余价值的一切项目——地租、利润等——的基金。在这种情况下,十分明白,分享这50镑的人中,一个人得的越多,另一个人得的就越少,反过来也是一样,或者说,利润和地租成反比。现在要问:剩余价值分为这两部分,是由什么决定的?

无论如何,下述一点仍然是正确的:企业主(不管他是农业主还是工厂主)的收入等于他从出卖产品中取得的剩余价值(他从他的生产领域的工人身上榨取了这个剩余价值),而地租(在它不象卖给工业家的瀑布那样直接从工业品取得的地方;房租等的情况则同瀑布的情况一样,因为住房不是原产品)则仅仅从包含在原产品中、由租地农场主支付给土地所有者的超额利润(不加入一般利润率的那部分剩余价值)产生。

一点不错,如果原产品的价值提高[或降低],在使用原料的工业部门中,利润率将同原产品价值成反比地降低或提高。如果棉花的价值增加了一倍,那末,在工资和剩余价值率既定的条件下,利润率将下降,就象我在前面一个具体例子中指出的那样\endnote{马克思是指他在手稿第335—336页,即注17和注20中谈到的篇幅很长的关于约翰·斯图亚特·穆勒的补充部分中所引用的例子。——第76页。}。但是,这种情况在农业中也是存在的。如果收成不好,而生产要按原来的规模继续进行(这里我们假定商品按它们的价值出卖),那末,总产品,或者说,总产品价值中就有较大的一部分必须投回到土地中去,而且,在工资不变的条件下,在扣除工资以后,租地农场主的剩余价值将是产品的较小部分;因而剩下供租地农场主和土地所有者分配的将是较小量的价值。虽然单位产品的价值会比以前高,可是,不仅余下的产品量,而且余下的价值部分都会比以前小。如果产品由于需求[增加]而高于它的价值出卖,以致现在较小量产品的价格比以前较大量产品的价格高,情况就不同了。但是,这同我们假定产品按照它们的价值出卖这一点是矛盾的。

假定情况相反:棉花收成加倍,直接投回到土地中去的部分如肥料和种子的价值比以前小。在这种情况下,扣除工资后剩下给棉花种植业者的那部分价值比以前大。在棉花种植业中,如同在棉纺织业中一样,利润率将提高。自然,一点不错,现在一码棉布中,归原产品的价值部分将比以前小,归原料加工的价值部分将比以前大。假定一码棉布包含的棉花价值等于1先令,一码棉布值2先令。如果现在棉花价格从1先令降到6便士(在棉花价值等于棉花价格的前提下,棉花价格之所以可能下降,只是因为棉花种植业的生产率提高了),那末一码棉布的价值等于18便士。它降了1/4,即25%。但是,在棉花种植业者以前按1先令的价格卖出100磅的地方,现在按6便士要卖出200磅。以前全部棉花的价值是100先令,现在也是100先令。虽然棉花以前占产品价值的一个较大部分,棉花生产者以前用每磅按1先令计价的100先令棉花只换到50码棉布;现在他(即使棉花种植业中的剩余价值率同时降低了)用每磅棉花按6便士出卖的100先令,却换到66+(2/3)码棉布。

假定商品按它们的价值出卖,那末,说参与产品生产的生产者的收入,必然取决于他们的产品在产品总价值中形成多大的价值组成部分,[477]这种说法是不正确的。

假定在一切工业品中,包括机器在内,一个生产部门的总产品价值是300镑,另一个是900镑,第三个是1800镑。

如果说,全部产品价值分为原产品价值和工业品价值的那种比例决定剩余价值——按洛贝尔图斯的说法是租——分为利润和地租的比例这一点是正确的,那末,这一点对于有原料和工业品以各种比例参加的各种生产领域的各种产品,也一定是正确的。

如果在900镑价值中,工业品是300镑,原产品是600镑;如果1镑等于1工作日;其次,如果剩余价值率是既定的,例如,在正常工作日是12小时的时候,是2小时比10小时,那末,在300镑[工业品]中包含300工作日,在600镑[原产品]中则多一倍(2×300)。在前一种产品中剩余价值额等于600小时,在后一种产品中是1200小时。这无非是说,在剩余价值率既定的时候,剩余价值量取决于工人人数,即取决于同时使用的工人人数。其次,既然已经假定(但不是已经证明)在加入农产品价值的剩余价值中,一部分作为地租归土地所有者,那末,从这里必然进一步得出结论:地租量实际上是按农产品价值增加(同“工业品”价值相比)的比例增加的。

在上例中,农产品对工业品之比是2∶1,即600∶300。现在假定,这个比是300∶600。既然地租取决于农产品中所包含的剩余价值,显然,如果剩余价值在前一种情况下是1200小时,而在后一种情况下只有600小时,那末地租既然是这个剩余价值的一定部分,在前一种情况下就必然比在后一种情况下大。换句话说:农产品在全部产品价值中占的价值部分越大,全部产品的剩余价值中归农产品的部分就越大,因为产品的每一价值部分都包含一定的剩余价值;而全部产品的剩余价值中归农产品的部分越大,地租也就越大,因为农产品剩余价值的一定比例部分表现为地租。

假定地租等于农业中生产的剩余价值的1/10,如果农产品价值(在900镑总价值中)占600镑,地租就等于120小时,如果农产品价值占300镑,地租只等于60小时。这样一来,地租量的确同农产品价值量一起变化,因而也同农产品价值对工业品价值的相对量一起变化。但是,地租和利润的“高度”,即它们的比率,与此绝对没有关系。在前一种情况下,产品价值等于900镑,其中300镑是工业品,600镑是农产品。这个总数中,有600小时剩余价值归工业品,1200小时归农产品。合计1800小时。其中120小时归地租,1680小时归利润。在后一种情况下,产品价值等于900镑;其中600镑是工业品,300镑是农产品。因而归工业的剩余价值是1200小时,归农业的是600小时。合计1800小时。这个总数中,归地租的是60小时;归利润的部分中,1200小时归工业,540小时归农业,合计1740小时。在后一种情况下,工业品(按价值)两倍于农产品,在前一种情况下相反。在后一种情况下地租等于60小时,在前一种情况下等于120小时。地租纯粹同农产品价值成比例地增加。农产品价值量增加多少倍,地租量也增长多少倍。就全部剩余价值(等于1800小时)来看,地租在前一种情况下占1/15,在后一种情况下占1/30。

如果这里地租量随同归农产品的价值部分的量一起增加,而地租在全部剩余价值中所占的比例部分也随同地租量一起增加;因而,如果剩余价值归地租的部分同它归利润的部分比较起来也有了增加,——那末,这只是因为洛贝尔图斯假定地租按一定比例参与农产品剩余价值的分配。既然这个事实是既定的,或者说,是已经假定的,情况也必然是这样。但是这个事实本身,决不能从洛贝尔图斯再次给我们讲的、我在前面第476页一开头\fnote{见本册第75页。——编者注}就引过的关于“材料价值”的废话中得出来。

至于地租的高度,那它也不会同地租所参与分配的产品[中包含的剩余价值]成比例增加,因为这个比例仍然是1/10。地租量增加,是因为这个产品量增加了;既然在地租的“高度”没有增加的情况下地租量还是增加了,那末,同[总]利润量相比,或者说,同这个利润在总产品价值中所占的份额[478]相比,地租的“高度”也增加了。因为假定,现在是总产品价值的一个较大的部分提供地租,剩余价值的一个较大的部分能够转化为地租,所以,剩余价值中转化为地租的部分自然就增加了。这一切同“材料价值”绝对没有关系。而洛贝尔图斯却说:

\begin{quote}{“较多的地租”同时也表现为“较高的地租”,“因为这个地租据以计算的土地面积或摩尔根数仍然不变,从而,每一摩尔根分摊到一个较大的价值额”。(第122页)}\end{quote}

这种说法是荒谬的。这是用一种回避问题本身困难的“尺度”来衡量地租的“高度”。

我们应该把上面举的例子略为改变一下——因为,我们还不知道什么是地租。如果我们使农产品的利润率同工业品的一样,只是另加1/10作地租,那末,情况就不同了,并且可以清楚地看出:[为了解决问题,]本来就应该这样做,因为同一利润率的存在是作为前提的。

在II的情况下,地租比在I的情况下大一倍,因为[总]产品价值中被地租象虱子一样叮着的那一部分价值,即农产品价值,在这里比工业品大。利润量在两种情况下都是一样——1800小时。在I的情况下,[地租]占全部剩余价值的1/31,在II的情况下占1/16。

如果洛贝尔图斯无论如何要把“材料价值”单单算在工业上,那末他首先就应该把由机器等组成的那部分不变资本单单归到农业上。这部分资本是作为工业提供给农业的产品,作为充当生产“原产品”的生产资料的“工业品”加入农业的。

至于工业,机器中由“原料”构成的那部分价值,已经在“原料”或者说“材料价值”的项下记入工业的借方,因为这里所谈的是两家公司之间的结算。因此,这一部分是不能重复入账的。工业中使用的机器的另一部分价值,是由加进去的“工业劳动”(过去的和现在的)构成的,而这种劳动分解为工资和利润(有酬劳动和无酬劳动)。因此,在这里预付的那部分资本(除了机器的原料所包含的以外),[可以看出]仅仅由工资构成,因而,它不仅使预付资本量增加,并且使应该依据这个预付资本来计算的剩余价值量增加,从而使利润增加。

(在这样的计算中,错误始终在于[不理解]:例如,机器本身即其价值中所包含的机器或工具的损耗,虽然归根到底也可以归结为劳动,——不论是原料中包含的劳动,还是把原料变成机器等的劳动,——可是,这个过去劳动既不再加入利润,也不再加入工资,只要再生产所必需的劳动时间不变,它只能作为已经生产出来的生产条件发挥作用;不论这种生产条件在劳动过程中的使用价值如何,它在价值形成过程中仅仅作为不变资本的价值出现。这一点非常重要,我在研究不变资本和收入的交换问题时已经阐明\fnote{见本卷第1册第248—255页。——编者注}。但是,除此以外,这一点还要在论资本积累那一节中较详细地加以阐述。)

至于农业,即单纯的原产品生产或者所谓初级生产,相反,在“初级生产”公司和“加工工业”公司之间相互结算时,加入农业的代表机器、工具等的资本价值部分,即一部分不变资本,只能看作是加入农业资本但不增加其剩余价值的项目,决不能作任何别的理解。如果农业劳动由于使用机器等等而提高了生产率,那末,机器等的价格越高,农业劳动生产率的增长就越慢。使农业劳动生产率以及任何其他劳动生产率增长的,是机器的使用价值,而不是机器的价值。此外,同样可以说,工业劳动生产率首先取决于原料的存在和它的特性。但是,这里成为工业的生产条件的,又是原料的使用价值,而不是它的价值。价值倒不如说是一个障碍。因此,[479]对洛贝尔图斯先生在工业资本方面关于“材料价值”所说的话,作一些相应的改动,就完全适用于[农业中使用的]机器等等:

\begin{quote}{“比如说,花费在作为机器的犁或轧棉机上的劳动〈以及耗费在排水渠或马厩上的劳动〉不能加入小麦或棉花这种特殊产品所耗费的劳动。”“相反,机器的价值,或者说机器价值,总是算在资本总额内,所有者就是根据这个资本总额来计算作为盈利归原产品的租部分的。”(参看洛贝尔图斯著作第97页)\endnote{马克思对洛贝尔图斯的这段话作了“相应的改动”,这些改动是根据洛贝尔图斯所忽视的下述情况而作的:机器和其他生产资料的价值必须加入农产品,正如农业原料的价值必须加入农业原料加工工业的产品一样。马克思按照洛贝尔图斯的表述形式,在前面引用过这段话(见本册第56页)。马克思仿照洛贝尔图斯的术语“材料价值”(《Materialwert》),不无讽刺地造出“机器价值”(《Maschinenwert》)这一术语。凡是马克思用的词,在引文中都以黑体加着重号刊印。——第82页。}}\end{quote}

换句话说:小麦和棉花的价值中代表磨损了的犁或轧棉机的价值的那一部分,不是犁地劳动或轧棉劳动的结果,而是造犁劳动或造轧棉机劳动的结果。这个价值组成部分加入农产品,尽管它不是在农业中生产的。它仅仅经过农业之手,因为农业只是用它从机器制造业者那里购买新犁或新轧棉机来补偿磨损了的犁和轧棉机。

农业用的这些机器、工具、建筑物和其他工业品由两个部分构成:(1)这些工业品的原料[和(2)加在原料上的劳动]。

虽然这种原料是农产品,但是它是农产品中从来既不加入工资也不加入利润的部分。即使农业中根本不存在资本家,土地耕种者仍然不能把他的产品的这一部分作为工资记在他的账上。事实上他必须把这一部分无代价地交给机器厂主,让机器厂主用来为他制造机器,此外,他还必须对加在这种原料上的劳动支付报酬(=工资+利润)。实际上情况也是这样。机器制造业者购买原料,但是农业主在购买机器时必然把这种原料买回。因此,这就好比农业主什么也没有卖给机器制造业者,而只是把原料借给他,让他赋予原料以机器形式。因此,农业用的机器的价值中归结为原料的那部分,虽然是农业劳动的产品,是农业劳动所创造的价值的一部分,却仍然属于生产,而不属于生产者,因而同种子一样算在生产者的费用中。另一部分价值,即代表投入机器的工业劳动的价值,是“工业品”,它作为生产资料加入农业,同原料作为生产资料加入加工工业完全一样。

因此,如果说“原产品生产”公司把“材料价值”提供给“加工工业”公司,这个材料价值作为一个项目列入工厂主的资本总额,这种说法是正确的,那末,下面的说法同样正确:“加工工业”公司把机器价值提供给“原产品生产”公司,这个机器价值全部(包括由原料组成的部分在内)加入租地农场主的资本总额,尽管这个“价值组成部分”并不给租地农场主提供剩余价值。正是由于这个原因,在英国人所谓的高级农业中,同原始农业比较,虽然剩余价值率较高,利润率却显得较低。

同时,这里向洛贝尔图斯先生提供了一个鲜明的证据,说明对于资本预付的本质来说,产品价值中投在不变资本的部分究竟是用实物补偿,因而仅仅被作为商品——作为货币价值——计算,还是确实被让渡,并经过买卖的过程,那是无关紧要的。例如,如果原产品生产者把他购买的机器中所包含的铁、铜、木材等无代价地交给机器制造业者,因而机器制造业者在向他出卖机器时只算加进去的劳动和自己的机器的磨损这笔账,那末,所购买的机器要农业主花的费用,就会恰好同它现在要他花的费用一样多,而且在他的生产中作为不变资本、作为预付出现的,就会是同一个价值组成部分;就象农民是把他的收获物全部卖掉,用收获物价值中代表种子(原料)的那部分价值买进别人的种子,——比如说,为了进行有利的品种更换,从而避免经常的同种繁殖引起的植物退化,——还是直接从他的产品中扣出这个价值组成部分投回到土地中去,这完全是无关紧要的。

但是,洛贝尔图斯先生为了作出他的计算,把不变资本中由机器构成的部分理解错了。

在分析洛贝尔图斯先生的第二个论题时应该考察的第二点是:

洛贝尔图斯先生所说的构成收入的工业品和农产品,同构成全部年产品的工业品和农产品,完全不是一回事。即使就全部年产品来看,下述说法是正确的:在把农业资本中由机器等构成的整个部分,[480]以及农产品中直接投回农业生产的部分扣除之后,剩余价值在租地农场主和工业家之间的分配(因而,归租地农场主的剩余价值也在租地农场主自己和土地所有者之间的分配),必然决定于工业和农业在产品总价值中所占份额的量,——那末,当我们说的是构成共同收入基金的产品时,这种说法是否正确就大成问题。收入(这里把再转化为新资本的部分除外)是由加入个人消费的产品构成的,这里的问题是工业家、租地农场主和土地所有者从这个大锅里究竟各取多少。取出的各部分,是不是由工业和原产品生产在构成收入的产品价值中所占的份额决定的?换句话说,是不是由构成收入的全部产品的价值所分成的农业劳动和工业劳动的份额决定的?

前面我已经指出\fnote{见本卷第1册第248—255页。——编者注},构成收入的产品量,不包括作为劳动工具(机器)、辅助材料、半成品和半成品的原料加入生产并构成劳动年产品的一部分的一切产品。它不仅不包括原产品生产中使用的不变资本,而且不包括机器制造业者的不变资本,以及租地农场主和[工业]资本家的虽然加入劳动过程但不加入价值形成过程的全部不变资本。其次,它不仅不包括不变资本,而且不包括那些不加入个人消费的产品——这些产品代表其生产者的收入,并加入作为收入来消费的那些产品的生产者的资本,以补偿用掉的不变资本。

作为收入来消费的,并且事实上无论按使用价值还是按交换价值都代表财富中构成收入的部分的产品量,如我前面所指出的\fnote{见本卷第1册第238—248页。——编者注},可以看作仅仅由(一年内)新加劳动构成,因而也仅仅归结为收入,即工资和利润(利润又分解为[留在资本家手里的]利润、地租、税收等),其中既丝毫不包含加入生产的原料的价值,也丝毫不包含加入生产的机器或者一般说劳动资料损耗的价值。因此,如果我们考察这种收入(把收入的各种派生形式完全撇开不谈,因为这些形式只是表示收入所有者把他在上述产品量中得到的份额转让给别人,或者是为了支付服务报酬等等,或者是为了还债等等),并假定其中工资占1/3,利润占1/3,地租占1/3,而产品按价值等于90镑,那末,每项收入的所得者都可以从总量中取得等于30镑的产品。

既然形成收入的产品量仅仅由新加(一年内加入的)劳动构成,那末,看来似乎很简单:如果农业劳动在这个产品量中占2/3,工业劳动占1/3,那末,工业家和农业主彼此就按照这个比例来分配价值。上述产品量价值的三分之一归工业家,三分之二归农业主,而且工业中和农业中实现的剩余价值的比例量(假定两个部门的剩余价值率一样),将同工业和农业在总产品价值中所占的份额相适应;地租将同租地农场主的利润量成比例增长,因为地租是象虱子一样叮着利润的。但是,这样来论述问题终究是错误的。问题在于,由农业劳动[产品]构成的一部分价值,将形成那些生产固定资本以补偿农业中损耗的固定资本的工厂主的收入。因此,形成收入的产品的价值组成部分之间的比例即农业劳动和工业劳动之间的比例,决不表示形成收入的产品量的价值或这个产品量本身在工业家和租地农场主之间分配的比例,也不表示工业和农业参与总生产的比例。

洛贝尔图斯接着说:

\begin{quote}{“但是,原产品价值和工业品价值的相应高度,或者说,两者在全部产品价值中所占的份额,又只是分别由原产品生产的劳动生产率和工业的劳动生产率决定的。原产品生产的劳动生产率越低,原产品的价值越高,反过来也是一样。同样,工业劳动的生产率越低,工业品的价值越高,反过来也是一样。因此,如果一般租的高度是既定的,那末,由于高的原产品价值造成高的地租和低的资本盈利,而高的工业品价值造成高的资本盈利和低的地租,——地租的高度和资本盈利的高度不仅应当彼此成反比,而且应当同相应的劳动的生产率,即原产品生产劳动和工业劳动的生产率成反比。”(第123页)}\end{quote}

把两个不同生产领域的生产率拿来比较,这只能是相对的。就是说,取任何一点,比如说,大麻的价值和麻布的价值即它们所包含的劳动时间的相对量之比1∶3作为出发点。如果这个比例改变了,那末,说这两种不同劳动的生产率有了改变,是正确的。但是,如果因为生产一盎斯金[481]所需要的劳动时间等于3,生产一吨铁所需要的劳动时间也等于3,就说金的“生产率低于”铁,那就错了。

两种商品的价值比例,表示生产一种商品比生产另一种商品花费较多的劳动时间;但是,不能因此就说,一个生产部门比另一个“生产率高”。这样说,只有在两者都把劳动时间用于生产同一使用价值的时候,才是正确的。

因此,如果原产品价值和工业品价值之比是3∶1,由此决不能得出结论说,工业生产率三倍于农业。只有这个比例改变了,例如变成4∶1或3∶2或2∶1等等,才可以说,两个部门的相对生产率有了改变。因而,只有在生产率提高或降低的情况下才可以这样说。

\tsubsectionnonum{[(c)洛贝尔图斯的第三个论题]}

\begin{quote}{“(III)资本盈利的高度,一般地说,仅仅决定于产品价值的高度,具体地说,仅仅决定于原产品价值和工业品价值的高度,或者,一般地说,仅仅决定于劳动生产率的程度,具体地说,仅仅决定于原产品生产劳动生产率和工业劳动生产率的程度;地租的高度,除此以外,还取决于产品价值的量,就是说,取决于在既定的生产率程度下用于生产的劳动或生产力的量。”(第116—117页)}\end{quote}

换句话说:利润率仅仅取决于剩余价值率,而剩余价值率仅仅取决于劳动生产率;可是,地租率还取决于在既定的劳动生产率下使用的劳动量(工人的数量)。

在这个论断中,几乎每一句话都是错的。

第一,利润率决不是仅仅取决于剩余价值率,——关于这一点我们马上就要谈到。但是,首先,说剩余价值率仅仅取决于劳动生产率,是错误的。在既定的劳动生产率下,剩余价值率随剩余劳动时间的长短而变动。因此,剩余价值率不仅取决于劳动生产率,也取决于使用的劳动量,因为(在生产率不变的情况下),即使有酬劳动量不增加,即花费在工资上的资本部分不增加,无酬劳动量也可能增加。如果劳动没有起码达到这样程度的生产率,也就是在一个工作日中,除了工人自己再生产所必需的时间以外,还剩下剩余劳动时间,那末,无论绝对剩余价值还是相对剩余价值(洛贝尔图斯追随李嘉图,只知道有相对剩余价值),都是不可能的。但是,既然假定存在剩余劳动时间,那末,在既定的最低限度的生产率下,剩余价值率就随剩余劳动时间的长短而变动。

因此,第一,说利润率,或者说,“资本盈利的高度”仅仅决定于被资本剥削的劳动的生产率(因为据说剩余价值率也决定于它),是错误的。第二,假定在既定的劳动生产率下随工作日的长短而变动、在既定的正常工作日下随劳动生产率而变动的剩余价值率是既定的。这样,剩余价值本身就随工人(他们的每一个工作日被榨取一定量的剩余价值)的人数而不同,换句话说,随花费在工资上的可变资本量而不同。但是,利润率取决于这个剩余价值对可变资本加不变资本之比。在剩余价值率既定时,剩余价值量当然取决于可变资本量,但是,利润的高度,即利润率,取决于这个剩余价值对预付总资本之比。因此,这里利润率当然决定于原料的价格(如果该工业部门使用原料的话)和具有一定效率的机器的价值。

因此,洛贝尔图斯的下面一段话是根本错误的:

\begin{quote}{“因此,据以计算盈利的资本价值总额,同资本盈利额——随产品价值增加而增加——是按同一比例增加的,盈利和资本之间的原来比例不因资本盈利的增加而有丝毫改变。”(第125页)}\end{quote}

这句话如果说正确,除非它是这样一个同义反复:在利润率既定时{而利润率跟剩余价值率和剩余价值本身是极不相同的},使用的资本的大小之所以无关紧要,正是因为利润率假定不变。一般说来,虽然劳动生产率不变,利润率还是可能提高,或者,虽然劳动生产率提高,并且每个领域都提高,利润率还是可能下降。

接着,又出现了关于地租的笨拙的笑话(第125—126页),说什么地租的单纯增加就可以使地租率提高,因为地租在每个国家都是按“不变的摩尔根数”(第126页)计算的。既然(在利润率既定时)利润量增加,那末提供利润的资本量也增加;可是,如果地租增加,据说发生变化的只有一个因素,即地租本身,至于它的尺度“摩尔根数”,仍然固定不变。

\begin{quote}{[482]“因此,地租的提高可以由于社会经济发展中到处发生的一个原因,即由于生产中使用的劳动增加,换句话说,由于人口增加,而不必同时由于原产品价值提高,因为从更多的原产品取得地租,就必然产生这样的结果。”(第127页)}\end{quote}

洛贝尔图斯在第128页上有一个奇怪的发现:即使地租由于原产品的价格降到它的正常价值之下而完全消失,也不可能

\begin{quote}{“使资本盈利在什么时候会达到100%〈即在商品按其价值出卖的情况下〉;盈利不论多高,它始终要低得多”。(第128页)}\end{quote}

为什么呢?

\begin{quote}{“因为它”〈资本盈利〉“仅仅是产品价值分配的结果。因此,它在任何时候都只能是这个单位的一个分数。”(第127—128页)}\end{quote}

洛贝尔图斯先生,这全看您是怎么计算的。

假定,预付不变资本等于100,预付工资等于50,而劳动的产品超过这个50的余额等于150,那末,我们就可以这样计算:

\todo{}

要使这种情况出现,只须假定工人以3/4的工作日为他的老板劳动,就是说,四分之一的劳动时间就够工人自身的再生产。当然,如果洛贝尔图斯先生把等于300的全部产品价值作为一个整体,如果他不是从其中超过生产费用的余额来考察,而是说这种产品要在资本家和工人之间分配,那末,资本家的份额的确只占这种产品的一部分,哪怕这一部分等于全部产品的999/1000。但是,这种计算是错误的,至少几乎从各方面说都是无用的。如果有人花费150,得到300,那他通常不会说,用300(而不是用150)来除150,他获得50%的利润。

在上面的例子中,假定工人劳动12小时:3小时为自己,9小时为资本家。现在假定他劳动15小时,即3小时为自己,12小时为资本家。按照生产中[生产资料和活劳动量的]原来的比例,不变资本应该追加25(实际上没有这么多,因为机器的花费,不会和劳动量按同一比例增加)。于是:

\todo{}

接着,洛贝尔图斯又向我们谈论“地租的无限”增长,因为第一,他把地租量的单纯的增加理解为地租的提高,就是说,当相同的地租率用较大的产品量来支付时,他也说是地租的提高。其次,因为他计算地租,是把“摩尔根”当作尺度,——这是毫不相干的两回事。

\centerbox{※     ※     ※}

下面各点可以简单提一下,因为它们同我的目的毫无关系。

“土地价值”是“资本化的地租”。因此,土地价值的这种货币表现要看通行的利率高低而定。按4%资本化,地租应当乘以25(因为4%=100的1/25);按5%资本化,应当用20去乘(因为5%=100的1/20)。在土地价值上,这里的差额是20%。(第131页)甚至由于货币价值下降,地租,从而土地价值,也会在名义上提高,因为,这里据说同资本的情况不一样。如果借贷利息或利润表现为较多的货币,那末资本在其货币表现上也同时以同一程度增多。相反,表现为较多货币的地租,据说应分配在“地段的不变的摩尔根数上。”(第132页)

洛贝尔图斯先生把他的智慧应用到欧洲的经济发展时,作了如下的概述:

\begin{quote}{(1)“……在欧洲各国,一般劳动——原产品生产劳动和工业劳动——的生产率提高了……其结果,国民产品中归工资的份额减少了,剩下作为租的份额增加了……因而一般租提高了。”(第138—139页)(2)“……工业生产率增长的比例大于原产品生产的生产率……因此,今天在同量国民产品价值中,归原产品的租的份额,大于归工业品的租的份额;因此,尽管一般租提高了,可是提高的只有地租,资本盈利反而下降了。”(第139页)}\end{quote}

由此可见,洛贝尔图斯先生在这里完全同李嘉图一样,让地租提高和利润率降低互相说明;一个降低等于另一个提高,而地租的提高用农业生产率相对低[483]来说明。李嘉图在有的地方甚至明确地说,问题不在于生产率绝对低,而在于生产率“相对”低。\fnote{见本册第381—382页。——编者注}但是即使他说到相反的情况,这也不是从他提出的原则得出的,因为李嘉图观点的原作者安德森明确地讲过,任何土地都有改良的绝对可能性。

如果一般“剩余价值”(利润和地租)提高了,全部租对不变资本的比率不仅可能下降,而且一定下降,因为劳动生产率提高了。虽然使用的工人人数增加了,他们受剥削的程度增加了,可是花费在工资上的资本,尽管绝对地说增加了,相对地说却下降了;因为由这些工人推动的、作为过去劳动的预付产品、作为生产的前提加入生产的[不变]资本,构成总资本中不断增长的部分。因此,利润加地租的比率下降了,虽然不仅它们的数额(它们的绝对量)提高了,而且对劳动的剥削率也提高了。这一点洛贝尔图斯先生是无法看到的,因为在他看来,不变资本是工业的发明,而农业对此则毫无所知。

至于利润和地租的相对量,那末,决不能因为农业生产率相对地低于加工工业生产率,就得出结论说,利润率因此要绝对地下降。如果说,利润对地租之比以前是2∶3,现在只是1∶3,那末,利润以前是地租的2/3,现在只等于地租的1/3;换句话说,如果以前利润占总剩余价值的2/5(8/20),现在只占1/4(5/20),那末,利润下降了3/20,就是说,从40%降到25%。

假定,一磅棉花的价值以前是2先令。现在降到1先令。100工人以前一天纺100磅棉花,现在纺300磅。花费在300磅棉花上的资本,以前是600先令,现在只有300先令。再假定,在两种情况下机器的价值都等于[第一种情况下花在棉花上的数额的]1/10,即60先令。最后,为了把300磅棉花纺成棉纱,以前对300工人支付300先令工资,现在只对100工人支付100先令。因为工人的劳动生产率“增长了”,并且我们必须假定,这里是用工人自己劳动的产品支付工人报酬的,所以假定剩余价值以前等于工资的20%,现在等于工资的40%。

这样,300磅棉纱值:

\begin{quote}{(I)在第一种情况下——原料600,机器60,工资300,剩余价值60,合计1020先令;(II)在第二种情况下——原料300,机器60,工资100,剩余价值40,合计500先令。}\end{quote}

在第一种情况下,生产费用960先令,利润60先令,利润率[6+(1/4)]%。

在第二种情况下,生产费用460先令,利润40先令,利润率[8+(16/23)]%。

假定地租是每磅棉花的1/3,那末,在第一种情况下等于200先令,或10镑,在第二种情况下等于100先令,或5镑。这里,地租下降了,因为原产品便宜了50%。但是整个产品便宜了50%以上。新加的工业劳动对原料价值之比,在第一种情况下是360∶600,或6∶10,或1∶[1+(2/3)];在第二种情况下是140∶300,或1∶[2+(1/7)]。工业劳动生产率增长的比例大于农业劳动生产率;可是在第一种情况下,利润率比第二种情况下低,地租比第二种情况下高。在两种情况下,地租都是原料的1/3。

假定原料量在第二种情况下加了一倍,因而纺了600磅棉花,那末情况就变成:

\begin{quote}{(II)600磅棉花(原料)600先令,机器120先令,工资200先令,剩余价值80先令,合计生产费用920先令,利润80先令,利润率[8+(16/23)]%。}\end{quote}

同第一种情况对比,利润率提高了。地租却同第一种情况下一样多。600磅棉纱只值1000先令,而以前却值2040先令。

[484]决不能从农产品的相对昂贵得出结论说,农产品提供地租。但是,只要假定地租作为一定百分比加到农产品的每一个价值部分上,——洛贝尔图斯是这样假定的,因为他的所谓证明是荒谬的,——那当然就会得出结论说,地租随农产品的不断昂贵而提高。

\begin{quote}{“……由于人口增加,国民产品的价值总额也极度增加了……因此,现在国内得到更多的工资,更多的盈利,更多的地租……地租总额的这种增加还引起地租水平的提高,可是工资和盈利的总额的增加却不会发生这种作用。”(第139页)}\end{quote}

\tsectionnonum{[(8)洛贝尔图斯所歪曲的规律的真实含义]}

现在让我们抛开洛贝尔图斯先生的一切谬论(更不用说我在上面已经详细指出的那些有缺陷的见解了,例如,说剩余价值率(“租的高度”)只有在劳动生产率增长时才能提高,——这就是不懂得绝对剩余价值,等等)。

就是说,让我们抛开:

[第一个]谬论:在真正的农业中(在资本主义的农业中),“材料价值”完全不加入预付;

第二个谬论:洛贝尔图斯不是把加入农业和工业的第二部分不变资本(机器等)看作这样一个“价值组成部分”:它同“材料价值”完全一样,不是它以机器形式加入的那个生产领域的劳动的结果,因而在计算每个生产领域所取得的盈利时也要以它为依据,虽然机器的价值同材料的“价值”一样,一点也没有加到这个盈利中,——尽管两者都是生产资料,并作为生产资料进入劳动过程;

第三个谬论:洛贝尔图斯不把加入农业的“机器”等所构成的整个“价值组成部分”作为预付,记在农业的账上,而且他不把这个“价值组成部分”中不归结为原料的部分看作农业对工业的负债,而为了偿付这笔债务,农业必须把一部分原料无代价地提供给工业,因此,这一部分原料不属于作为整体的工业预付;

第四个谬论:他认为一切工业部门,除了有机器和机器所必需的辅助材料加入外,还有“材料价值”加入,可是无论运输业或采掘工业,都完全没有这种情况;

第五个谬论:他不懂得,在许多加工工业部门(它们越是提供用于消费的成品,就越是如此),虽然除了可变资本之外还有“原料”加入,但是不变资本的其他组成部分几乎完全消失,或者数量极小,小得同大工业和农业无法比较;

第六个谬论:他把商品的平均价格同商品的价值混为一谈。

这一切谬论使洛贝尔图斯从租地农场主和他自己的错误计算中得出他对地租的解释,以致地租应该随着租地农场主开始确实计算他的支出而消失。如果把这一切谬论抛开,那末,作为隐藏在这些谬论后面的核心剩下来的,只是下面的论断:

当原产品按照它们的价值出卖的时候,它们的价值高于其他商品的平均价格,或者说,高于它们自己的平均价格,就是说,超过生产费用加平均利润,因而提供一个超额利润,这个超额利润形成地租。这就是说,可变资本(假定剩余价值率相等)对不变资本之比,在原产品生产中比工业各生产领域的平均数大(这并不妨碍可变资本在某些工业部门比农业中高)。或者,更一般地说:农业同这样一些工业生产领域属于一类,这些工业生产领域的可变资本对不变资本之比,高于各工业生产领域的平均数。因此,农业中的剩余价值,按它的生产费用计算,必然高于各工业生产领域的平均数。这也就是说,农业中的特殊利润率高于平均利润率即一般利润率。这也就是说,如果剩余价值率相等,而剩余价值本身又是既定的,那末各个生产领域的特殊利润率,取决于各个领域中可变资本对不变资本之比。

由此可见,这只是把我从其一般形式上论述的规律\fnote{见本册第64—70页。——编者注}应用到一个特殊生产部门而已。

[485]其次:

(1)要证明农业属于这样一些特殊的生产领域,它们的商品价值高于它们的平均价格,因此,它们的利润,——如果它们把利润据为己有,而不把它交出去参与一般利润率的平均化,——就高于平均利润,这样,在扣除平均利润之后,这些生产领域还剩下超额利润。看来,这一点对于农业平均说来是正确的,因为在农业中,手工劳动相对地说还占优势,而使工业的发展快于农业则是资产阶级生产方式所固有的。不过,这是历史的差别,是会消失的。这也造成了下面这种现象:工业供给农业的生产资料,总的说来,价值下降,而农业供给工业的原料,总的说来,价值提高;因此,在很大一部分加工工业中,不变资本的价值相对地高于农业。对于采掘工业,这一点在大多数情况下是不适用的。

(2)不能象洛贝尔图斯那样,说什么如果农产品按照一般规律,平均按它的价值出卖,就一定提供超额利润,或者说,地租,——好象商品按价值但高于平均价格出卖,成了资本主义生产的一般规律。相反,应该证明,为什么在原产品生产中,例外地,与价值也高于平均价格的那一类工业品不同,价值不降到平均价格,因而提供超额利润,或者说,地租。这一点只能用土地所有权的存在来解释。平均化只有通过资本同资本互相作用才会发生,因为只有互相起作用的各个资本才有力量实现资本的内在规律。就这一点来说,那些用垄断来解释地租的人是正确的;正如唯有资本的垄断使资本家能从工人身上榨取剩余劳动一样,土地所有权的垄断也使土地所有者能从资本家那里榨取那部分能够形成经常的超额利润的剩余劳动。那些用垄断来解释地租的人的错误,在于他们认为垄断使土地所有者能够把商品价格抬得高于商品价值。相反,垄断在这里的作用,是把商品价值保持在高于商品的平均价格的水平,是使商品能够按照它的价值而不是高于它的价值出卖。

如果洛贝尔图斯的观点经过这样的修正,那对问题的理解就正确了。这样一来,地租的存在也就得到了解释,而李嘉图只解释了差别地租的存在,实际上否认土地所有权有任何经济影响。其次,这样来理解问题,就排除了在李嘉图自己的著作中只是任意加上的、对他的思路来说没有必要的加筑部分,即所谓农业生产率递减的论断,而相反地,承认农业生产率的增长。问题仅仅在于,在资产阶级的基础上,农业生产率相对地低于工业,换句话说,农业劳动生产力的发展比工业慢。李嘉图不是从较高的生产率,而是从较低的生产率得出农业中的“超额剩余价值”,这是正确的。

\tsectionnonum{[(9)级差地租和绝对地租的关系。地租的历史性。斯密和李嘉图的研究方法问题]}

讲到地租的差别,在土地面积相同、投资额相等的情况下,它是由自然肥力的差别造成的,特别是,首先是对于那些提供面包这种主要食品的产品来说,是如此;在土地面积相同、肥力相等的情况下,地租的差别是由投资额不等造成的。第一种差别,即自然的差别,不仅引起地租量的差别,而且引起地租的高度,或者说,地租率(同支出的资本相比)的差别;第二种差别,即工业的差别,则仅仅引起地租与支出的资本量成比例的增加。在同一地段上连续投资,也可能产生不同的结果。在肥力不同的地段上存在不同的超额利润,或者说,存在差别地租,还不能使农业同工业区别开来。使农业同工业区别开来的,是这些超额利润的固定化,因为在农业中,这些超额利润是建立在自然所提供的基础上的(诚然,这个基础可能或多或少地趋于平衡),而在工业中,在平均利润相同的情况下,超额利润总是具有转瞬即逝的性质,超额利润的产生,总是仅仅因为采用了生产率更高的机器和更有效的劳动组合。在工业中,由于平均价格降低而得到的超额利润,总是由最后出现的、生产率最高的资本提供的。在农业中,可能成为而且往往必然成为超额利润的原因的,并不是较好地段的肥力的绝对增长,而是它的肥力的相对增长,这种增长是由比较不肥沃的土地投入耕种引起的。在工业中,较高的相对生产率,超额利润(它很快消失),总是新投资本的生产率比原有资本的生产率绝对增长的结果。在工业中,任何资本(这里不谈需求的暂时增长)都不可能因生产率较低的资本新加入该工业部门而提供超额利润。

[486]然而就是在农业中(正如李嘉图所设想的),比较肥沃——或者天然比较肥沃,或者由于最新技术成就变得比原有技术条件下的老地肥沃——的土地,也可能在较晚的阶段投入耕种,甚至可能使一部分老地停止耕种(在采矿工业和殖民地产品生产中就有这种情况),或者使老地进行另一种农业生产,提供另一种产品。

地租(超额利润)的差别比较固定,这是农业和工业不同的地方。然而,使平均生产条件决定市场价格,从而把低于这种平均水平的产品价格提到高于该产品的价格,甚至高于它的价值的原因,决不是土地,而是竞争,是资本主义生产;因此,这不是自然规律,而是社会规律。

按照这个理论,最坏的土地无论支付地租,或者不支付任何地租,都不是必然的。同样可能的是,在不提供地租、只提供普通利润的地方,或者甚至连普通利润也没有的地方,仍然支付租金,也就是说,土地所有者得到地租,虽然从经济学观点来看,这里没有任何地租存在。

例如第一种情况:只有较好(比较肥沃)的土地支付地租(超额利润)。“作为地租”的地租在这里并不存在。在这种情况下,超额利润很少作为地租固定下来,正如工业中的超额利润并不固定下来一样(例如北美合众国西部就是这样)\endnote{手稿中接下去有一段关于资本是“别人劳动的合法反映”的简短插话,马克思把这段话放在花括号内,并且指出,它对叙述的直接联系有妨碍,应该移到别处去。本版用脚注形式把这段话放在前面,即放在本册第26页。——第99页。}。[486]

[486]这种情况常常发生在这样的地方:一方面,比较大量可供自由使用的土地还没有变为私有,另一方面,自然肥力相当大,虽然资本主义生产不太发达,因而,虽然可变资本对不变资本的比例很大,农产品的价值还是等于(有时甚至低于)它们的平均价格。如果价值高于平均价格,竞争会使它降低到这个水平。但是,象洛贝尔图斯那样的说法则是荒谬的:国家让每一英亩缴纳一个微小的、几乎是名义上的价格,比如说,一美元[就是地租]。\fnote{见本册第171页。——编者注}人们同样可以引用国家让每一个工业部门缴纳“营业税”为理由[并得出结论:工业提供“营业租”]。在我们考察的这种情况下,李嘉图[关于只有较好的土地支付地租]的规律是有效的。这里,只有相对地说更肥沃的土地才有地租,而且大部分还是不固定的,是流动的,象工业中的超额利润一样。不支付地租的土地所以不支付地租,不是因为它贫瘠,而是相反,因为它肥沃。较好的土地支付地租,是因为它相对地更肥沃,它的肥力高于平均水平。

但是在存在土地所有权的国家,同样的情况,即最后进入耕种的土地不支付地租,可能是由于相反的原因[即由于这种土地相对的贫瘠]。如果,比方说,谷物的价值很低(而这种低水平同支付地租根本没有关系),以致最后进入耕种的土地所提供的这个价值仅仅等于平均价格,那末,这种情况就会发生。这是由于这种土地相对的贫瘠,因而在这种土地上虽然和在提供地租的土地上花费同样多的劳动,但是,收获的夸特数量(按支出的资本计算)是这样少,以致在实现谷物产品的平均价值时,只得到比方说小麦的平均价格。

[487]举例来说,假定提供地租的最后的土地(提供最少地租的土地提供纯地租,其他土地提供级差地租),支出100镑资本,生产120镑或360夸特小麦的产品,每夸特值1/3镑。在这种情况下,3夸特小麦值1镑。假定1镑等于1周劳动,100镑=100劳动周,120镑=120劳动周。那末,1夸特小麦等于1/3劳动周即2工作日。假定在2日或24小时中(如果正常工作日是12小时),1/5日即4+(4/5)小时是无酬劳动,也就是一夸特包含的剩余价值。1夸特=1/3镑=6+(2/3)先令=6+(6/9)先令。

如果平均利润等于10%,360夸特的平均价格就等于110镑,1夸特的平均价格=6+(1/9)先令。这样,如果一夸特小麦按它的价值出卖,全部产品的价值就比平均价格高10镑。因为平均利润等于10%,地租就等于剩余价值的一半,即10镑或每夸特5/9先令。等级较高的土地,同样花费120劳动周(不过其中只有100是有酬劳动,不管是物化劳动还是活劳动),却能提供较多夸特,按每夸特6+(6/9)先令的价格,将提供较高的地租。而最坏的耕地花费100镑资本将提供10镑地租,或者说,一夸特小麦提供5/9先令地租。

假定有一块新地投入耕种,花费120劳动周,只提供330夸特。如果[以前]3夸特的价值等于1镑,330夸特就等于110镑。现在一夸特等于2日2+(2/11)小时,以前只等于2日。以前一夸特的价值是6+(6/9)先令即6先令8便士,现在(因为一镑等于6日)是7先令3便士1+(1/11)法寻。现在一夸特要按它的价值出卖,并在实现价值时也提供5/9先令地租,它就必须比以前贵7便士1+(1/11)法寻出卖。较好土地上生产的小麦价值,在这里低于最坏土地上生产的小麦价值;如果这种最坏土地按照略好一点的土地,即提供地租的土地所生产的一夸特的价格出卖它的产品,那末它就低于产品的价值而按照产品的平均价格,也就是按照最坏土地能提供10%普通利润的那种价格出卖产品。因此,这种土地可以耕种,并给资本家提供普通平均利润。

在两种情况下,最坏土地在这里除了提供利润外,还会提供地租。

第一,如果一夸特小麦的价值高于6+(6/9)先令(由于需求增长,一夸特的价格可能高于6+(6/9)先令,即高于它的价值,但我们不研究这种情况;6+(6/9)先令,也就是早先最坏的耕地得以提供10镑地租的一夸特小麦价格,等于这块提供非级差地租的土地上生产的小麦的价值),就是说,如果早先最坏的耕地和其他一切土地,为了提供同一地租,相对地说比较不肥沃,以致它们的产品价值更加高于它们的平均价格和其他商品的平均价格。因此,新的最坏土地不提供地租,不是由于它贫瘠,而是由于其他地段相对的肥沃。提供地租的最坏的耕地,同投入新资本的新等级土地相比,则提供一般地租,即非级差地租。从提供地租的最坏土地上得到的地租不高于它现有的水平,正是因为这种提供地租的土地肥沃。

假定除了最后的提供地租的土地以外,还有三个等级的土地。等级II(在I即最后的提供地租的土地之上)提供的地租比等级多1/5,因为这种土地比等级I肥力大1/5,等级III提供的地租又比等级II多1/5,因为它比等级II肥力大1/5;等级IV也是这样,因为它比等级III肥力大1/5。由于等级I的地租等于10镑,等级II的地租就等于10+10的1/5=12镑,等级III等于12+12的1/5=14+(2/5)镑,等级IV等于14+(2/5)+[14+(2/5)]的1/5=17+(7/25)镑。\endnote{在这一段中,马克思还只是开始研究地租额(绝对地租额和级差地租额)对土地相对肥力的依存关系,他预先假定地租额和土地肥力成正比(就是说,如果某一级土地比另一级肥力大1/5,那末这一级土地的地租额也比比较不肥沃的那一级土地的地租额大1/5)。马克思在后来的研究中已经不用这个假定,而是就地租额对土地相对肥力的依存关系作了精确的表述。如果按照马克思后来的解释,根据等级II、III、IV提供的、并且按各级的共同价格即一夸特1/3镑出卖的夸特量来计算这三级土地的地租额,那末,II的地租额是34镑,III是62+(4/5)镑,IV是97+(9/25)镑。计算方法是这样的:因为II比I肥力大1/5,所以它提供360+72,即432夸特,卖得432/3镑即144镑;其中110镑是生产费用加平均利润;剩下34镑是地租(绝对地租和级差地租)。III、IV的计算方法也是这样。马克思把计算地租额的精确方法广泛应用于第十二章(《级差地租表及其说明》),但这个方法在第八章就已经拟定了。例如,在第109—110页,马克思起初重复引用了第102页的数字,即IV的地租额17+(7/25)镑,得出这一级土地的级差地租7+(7/25)镑,然后他拟定了确定IV的级差地租的正确方法,这就是:207+(9/25)镑减120镑,得87+(9/25)镑。如果再加上10镑绝对地租,那末IV的地租总额是97+(9/25)镑,完全符合马克思后来的结论。——第102页。}

如果IV的肥力减低,从III到I的地租就会[488]增多,IV的地租也会绝对增多(但它们之间的比例是否仍然不变?)。这一点可以从两方面来理解。如果I比较肥沃了,II、III、IV的地租就会相应减少。另一方面,I对II、II对III、III对IV的关系,同新加入的、不提供地租的那一级土地对I的关系一样。新等级的土地不提供地租,是因为I的小麦价值并不高于新土地产品的平均价格。如果I比较不肥沃了,I的小麦价值就会高于新土地产品的平均价格。这时,新土地也会提供地租。就I来说,情形也是这样。如果II比较肥沃了,I就不提供地租或提供较少的地租;II对III的关系,III对IV的关系,也是这样。因此,归根到底,这里是一种相反的顺序:IV的绝对肥力决定III的地租;如果IV还要肥沃一些,III、II、I就会提供较少的地租,或完全不提供地租。由此可见,I提供的地租,即非级差地租,是由IV的肥力决定的,正如新地完全不提供地租,是由I的肥力决定的一样。因此,在这里,施托尔希的规律是适用的:最肥沃的土地的地租决定最后的、一般提供地租的土地的地租,\endnote{施托尔希在他的著作《政治经济学教程》1815年圣彼得堡版第二卷第78—79页上写道:“最肥沃的土地的地租决定与最肥沃土地竞争的其他一切土地的地租量。只要最肥沃的土地的产品还足以满足需求,与最肥沃的土地竞争的比较不肥沃的土地就不可能被耕种,或者说,至少不提供地租。但是,只要需求开始超过肥沃土地所能提供的产品量,产品价格就会提高,比较不肥沃的土地就可能被耕种,就可能从这个土地上获得地租。”施托尔希的这个观点,马克思在《资本论》第三卷(第十、三十九章)中也谈到过。——第103、328页。}就是说,它也决定提供非级差地租的土地和完全不提供地租的土地之间的差额。

因此,第五个等级即新耕地I′(和I相区别)不提供地租,不是由于它本身贫瘠,而是由于它同I比较相对的贫瘠,也就是说,由于I比I′相对的肥沃。

[第二,]提供地租的I、II、III、IV级地的产品一夸特(为了更符合实际,可以用蒲式耳代替夸特)的价值6+(6/9)先令即6先令8便士,等于I′的平均价格而低于这一级本身的价值。但这里可能有许多中间阶段。如果I′花费100镑资本提供的蒲式耳量,是在它的实际收成330蒲式耳和I的收成360蒲式耳之间,例如333、340、350直到(360—x)蒲式耳,那末,一蒲式耳的价值,即6先令8便士,就会高于I′(一蒲式耳)的平均价格,这种最后的耕地也就会提供地租。这种土地一般提供平均利润,是因为I,也就是I—IV相对的贫瘠。这种土地不提供地租,是因为I相对的肥沃和它本身相对的贫瘠。如果一蒲式耳的价值高于6先令8便士,就是说,如果I、II、III、IV比较不肥沃了,最后的耕地I′就能提供地租,因为在这种情况下,小麦的价值提高了。但是,如果一蒲式耳的价值等于6先令8便士,就是说,I、II、III、IV的肥力照旧不变,而I′本身比较肥沃了,提供330蒲式耳以上,因而一蒲式耳6先令8便士的价值高于I′提供的一蒲式耳的平均价格,换句话说,如果它的平均价格现在低于6先令8便士,即低于I、II、III、IV种植的小麦价值,那末,在这种情况下,最后的耕地也能提供地租。既然价值高于平均价格,就会有一个高于平均利润的超额利润,从而就可能有地租。

由此可见:

在拿不同的生产领域(例如,工业和农业)作比较时,价值超过平均价格这一事实,证明提供超额利润即价值超过平均价格的余额的那个生产领域生产率较低。相反,在同一领域内,这一事实却证明这个资本比同一生产领域的其他资本生产率高。在上例中,I提供地租,一般说来,是因为农业中可变资本对不变资本之比大于工业,也就是说,农业中必须有更多的新劳动加到物化劳动上去,还因为土地所有权的存在使价值超过平均价格的余额不因资本竞争而平均化。但是[具体说来],I一般提供地租,还因为一蒲式耳价值6先令8便士不低于这一级土地产品的平均价格;就是说,因为这一级土地还不是那么贫瘠,以致它本身产品的价值高于一蒲式耳6先令8便士,并且决定它的产品的价格的,不是这一级土地本身产品的价值,而是II、III、IV(或者更确切地说,是II)上种植的小麦价值。这个市场价格现在是仅仅等于这一级土地本身产品的平均价格,还是高于这个平均价格,换句话说,I的产品的价值是否高于它的平均价格,这取决于这一级土地本身的生产率。

正因为如此,洛贝尔图斯所谓农业中任何提供平均利润的资本,都必定提供地租的观点是错误的。这个错误的结论,洛贝尔图斯是从[489]他的错误的理论基础得出的。洛贝尔图斯这样说:在农业中资本提供,比方说,10镑,但是,因为农业和工业不同,原料不算在内,所以10镑要用一个较小的总数来除,因而结果就大于百分之十。但是,问题的症结在于:不是所谓农业中原料不加入生产(相反,原料是加入本来意义的农业的;即使它不加入农业,也毫无关系,如果农业中机器等等相对增加的话)以致农产品价值高于平均价格(它们本身的平均价格和其他商品的平均价格),而是农业中可变资本对不变资本之比高于工业(不是工业中某些特殊的生产领域,而是全部工业的平均数)。这种总差别的大小决定I的地租即绝对的、非级差的、因而是最小的地租的大小和存在。而完全不提供地租的新耕地即I′的小麦价格,不是由这一级土地本身产品的价值决定,而是由I的产品的价值,因而是由I、II、III、IV所提供的小麦的平均市场价格决定的。

农业的特权(由土地所有权造成的)表现为:当农产品的价值高于平均价格时,农业不是按平均价格,而是按产品价值出卖自己的产品。这种特权,对于不同土地所生产的各种产品的相互关系,对于同一生产领域内生产的各种价值不同的产品,是完全不适用的。对于工业品来说,农产品只要求按自己的价值出卖。对于同一领域的其他产品来说,农产品要由市场价格来决定;价值(在这里,等于平均市场价格)是高还是低,即I的肥力是大还是小,以致I′在按照这个价值出卖它的产品时,在小麦价值和平均价格的总差额中所占份额是多是少还是完全没有,这取决于I的肥力。但是,既然洛贝尔图斯先生根本不去区分价值和平均价格,既然他认为商品按它们的价值出卖是一切商品的普遍规律(他不理解,这是农产品的特权),他就必然认为,最坏土地的产品也一定按它的个别价值出卖。但是,最坏土地的产品在和同类产品竞争时,是会丧失这种特权的。

但是,I′的产品的平均价格也可能高于I的产品的价值——一蒲式耳6先令8便士。可以设想(虽然这不完全正确),要使I′级地一般能投入耕种,需求必须增长。因此,I的小麦价格一定会提高到它的价值以上,即高于6先令8便士,并且这种提高是稳定的。在这种情况下,I′级地会投入耕种。如果它在产品价格为6先令8便士的情况下能够提供平均利润(虽然它的产品价值高于这个价格),并且满足需求,那末小麦的价格就会回到6先令8便士,因为现在需求又同供给相适应了;于是,I又必须按6先令8便士出卖它的产品,II、III、IV也是这样,从而I′也是如此。但是,如果I′的产品的平均价格是7先令8便士,从而这一等级只有按照这个价格(这个价格会大大低于它的个别价值)才能提供普通利润,那末,在不可能用其他方法满足需求时,一蒲式耳的价值就会固定在7先令8便士,而I的一蒲式耳符合于需求的价格就会提高到它的[个别]价值以上。II、III、IV的一蒲式耳的价格,已经高于它们的个别价值。它们的价格还会更加提高。但是,如果预期会有谷物输入,这在任何情况下都不会容许一蒲式耳的价格固定在7先令8便士,那末,如果有小租地农场主满足于平均利润以下的利润,I′仍然会投入耕种。这种情况,在农业中以及在工业中都经常发生。在这种情况下,也象在I′提供平均利润时一样,也可能支付地租,不过这种地租只是租地农场主利润的扣除部分。假如连这一点也做不到,土地所有者就会把土地租给茅舍贫农。茅舍贫农和手工织工一样,关心的主要是怎样挣得自己的工资,而把剩下的或大或小的余额以地租形式支付给土地所有者。象手工织工的情形一样,这种余额甚至不只是劳动产品的扣除部分,而且是劳动报酬的扣除部分。在所有这些情况下,都可能支付地租。在一种场合,地租是资本家利润的扣除部分。在另一种场合,土地所有者把通常由资本家占有的工人的剩余劳动据为己有。在最后一种场合,土地所有者也象资本家常做的那样,靠削减工人的工资来生活。但是,只有在最后的耕地至少能提供平均利润的地方,就是说,只有在I的产品的价值至少能给I′保证平均价格的地方,大规模的资本主义生产才是可能的。

我们在这里看到,价值和平均价格的区分使问题迎刃而解,并且证明李嘉图和他的反对者都是正确的\endnote{马克思在《资本论》第三卷第十章注30中写道,在农产品的市场价值问题上,李嘉图和施托尔希都是正确的,同时他们又都是不正确的,因为“完全忽视了中等情况”。——第107页。}。

[XI—490]如果提供绝对地租的I是唯一的一个等级的耕地,那末它的一蒲式耳小麦就会按它的价值,也就是按6先令8便士或6+(6/9)先令出卖,而不会把它的价格降低到平均价格6+(1/9)先令或6先令1+(1/3)便士。如果需求增长了,如果国内全部土地属于同一等级,如果耕地面积增加到十倍,那末,假定I花费100镑提供10镑地租,虽然只有唯一的一个等级的土地,地租也会增加到100镑。然而,地租无论就其对预付资本还是对耕地面积的比率或者说高度来看,并没有增加。耕种的英亩数大了十倍,预付资本大了十倍。因此,我们在这里看到的只是地租总额、地租量的增加,不是地租高度的增长。利润率不会下降,因为农产品的价值和价格保持不变。一个十倍的资本,当然能提供十倍的地租。另一方面,如果在同一土地面积上使用十倍的资本并产生同样的结果,那末,同花费的资本相比,地租率保持不变;同土地面积相比,地租率提高了,但它也不会引起利润率的任何变化。

但是,现在假定,耕地I所以变得比较肥沃,不是因为土质变化了,而是因为花费的不变资本增加了,可变资本减少了,机器、马匹、矿肥等形式的资本增加了,工资形式的资本减少了;这时,小麦的价值就会接近于它的平均价格和工业品的平均价格,因为同不变资本相比,可变资本份额减少了。在这种情况下,地租会下降,利润率则保持不变。如果这里生产方式发生变动,以致可变资本对不变资本的比例等于它们在工业中的平均比例,那末,小麦价值高于小麦平均价格的余额就会消失,从而地租,即超额利润也就会消失。I将不再支付地租,土地所有权将变得有名无实(假如生产方式的变动没有引起对土地的追加投资,结果土地所有者在租佃期满后就会从不是由他预付的资本得到利息;这正是土地所有者发财致富的主要手段之一,而在爱尔兰,围绕租佃权进行的斗争也是由此引起的)。如果除了I以外,还存在II、III、IV,这些土地也普遍采用了这种新的生产方式,那末由于它们的自然肥力比I大,它们还是会提供地租,并与这种较大的肥力成比例地提供地租。在这种情况下,I不再提供地租,II、III、IV的地租则相应下降,因为农业中生产率的一般比例,已经和工业中生产率的一般比例相等了。II、III、IV的地租是符合李嘉图的规律的;它只等于比较肥沃的土地超过比较不肥沃的土地所提供的超额利润,并且只是作为这种超额利润存在,正如工业中也存在这种超额利润一样,不同的只是:在工业中,这种超额利润没有自然赋予的固定化基础。

在完全不存在土地所有权的情况下,李嘉图的规律也同样起支配作用。如果土地所有权被废除而资本主义生产保存下来,这种由肥力不同引起的超额利润也不会消失。如果国家把土地所有权据为己有,而资本主义生产继续存在,II、III、IV的地租就会支付给国家,但地租本身还是存在。如果土地所有权归人民所有,资本主义生产的整个基础,使劳动条件变成一种独立于工人之外并同工人相对立的力量的基础,就不再存在了。

以后在考察地租时要分析的一个问题是:在耕作比较集约化的情况下,尽管地租对预付资本的比率下降,地租怎么可能在价值和总量上增加起来。这一点所以可能,显然只是因为预付资本量增加了。如果地租是1/5,后来变为1/10,那末20×(1/5)=4,而50×(1/10)=5。这就是全部问题之所在。但是,如果在耕作比较集约化的情况下,在农业中确立的各生产要素之间的比例,就是工业中的平均比例,而不只是接近于这个比例,那末,最贫瘠的土地的地租就会完全消失,比较肥沃的土地的地租,也会纯粹归结为土地的级差。绝对地租就会消失。

现在假定,由于需求增长,从I向II推移。I支付绝对地租,II支付级差地租,但小麦的价格{对于I是价值,对于II是超额价值}保持不变。利润率也保持不变。如此类推一直到IV。因而,如果我们把用于I、II、III、IV的全部资本加在一起计算,地租高度,地租率就会增加。但是II、III、IV的平均利润率仍然和I的利润率相等,后者又和工业的利润率即一般利润率相等。可见,[491]在向比较肥沃的土地推移时,虽然利润率和小麦价格保持不变,地租量和地租率却可能增长。地租高度和地租量的增长,在这里是由II、III、IV的资本生产率提高引起的,而不是由I的资本生产率减低引起的。不过,提高了的生产率不象工业中那样必定使利润提高,并使商品价格和工资降低。

但是,如果向相反的方向——从IV向III、II、I推移,那末一蒲式耳的价格就要提高到6先令8便士,按照这个价格,I的小麦每100镑[花费的资本]还会提供10镑地租。也就是IV的小麦每100镑[花费的资本]提供的地租是17+(7/25)镑,不过,其中7+(7/25)镑是[IV的全部产品的市场]价格超过I的[全部]产品价值的余额。I(在地租为10镑、一蒲式耳的价值为6先令8便士时)花费100镑资本提供360蒲式耳小麦,II提供432蒲式耳,III提供518+(2/5)蒲式耳,IV提供622+(2/25)蒲式耳。但是,一蒲式耳6先7令8便士的价格使IV每花费100镑提供7+(7/25)镑超额地租,IV的3蒲式耳卖1镑,622+(2/25)蒲式耳卖207+(9/25)镑。但是IV的产品的[个别]价值和I一样,只有120镑;超过这个数字的,就是它的[市场]价格超过它的[个别]价值的余额\endnote{这个例子中的两种计算方法,见注25。——第110页。}。IV要是按3先令10+(8/27)便士出卖一蒲式耳小麦,那就是按价值出卖,按照这个价格,它也是每100镑[花费的资本]有10镑地租。如果现在从IV推移到III,从III推移到II,从II推移到I,那末,一蒲式耳的价格(以及地租)就会提高,直到最后达到I的6先令8便士的水平,在I那里,这个价格现在提供的地租,和以前IV提供的地租一样。随着[农产品]价格的提高,利润率却会下降,这部分地因为生活资料和原料的价值会提高。从IV向III的推移可能如下。由于需求增长,IV的价格提高到它的价值以上,因而这一等级不仅提供地租,而且提供超额地租。结果,III投入耕种,它按照这个价格出卖产品,在提供普通平均利润的情况下就不应提供任何地租。如果由于IV的产品价格提高,不是利润率下降,而是工资下降了,那末,III将提供[以前的]平均利润。但是因为有III的追加供给,工资势必又提高到正常水平;于是III的利润率就下降,等等。

因此,在这个下降运动中,利润率下降要有下列假设的前提:III按照IV的价格不能提供地租;它所以能够耕种并提供以前的利润率,只是因为工资暂时下降到自己的水平以下。

在这些前提下,李嘉图的规律又是可能的了。但即使按照李嘉图的观点,这个规律也不是必然的。它只在各种条件的一定结合下才是可能的。实际上各种运动是互相交错的。

\centerbox{※     ※     ※}

综上所述,地租理论实质上已经讲清了。

洛贝尔图斯先生由于他的“材料价值”观念,认为地租存在于事物的永恒本性中,至少存在于资本主义生产的永恒本性中。在我们看来,地租是资本有机组成部分的比例的历史性差别造成的,这种差别一部分会趋于平衡,甚至随着农业的发展会完全消失。诚然,即使绝对地租消失了,仅仅由土地自然肥力不同而引起的差别仍会存在。但是——把自然差别可能拉平这一点完全撇开——这种级差地租是同市场价格的调节作用联系在一起的,因而会随着价格和资本主义生产一起消失。保持不变的只是这种情况:社会劳动耕种肥力不同的土地,而且,尽管使用的劳动量不同,这种社会劳动在各种土地上的生产率都会提高。但是较坏土地产品所耗费的较大的劳动量,决不会产生资产阶级制度下的那种后果,也就是对较好土地的产品也必须以较大的劳动量来支付。相反,在IV上节省下来的劳动,会用来改良III,在III上节省下来的劳动,会用来改良II,在II上节省下来的劳动,会用来改良I;因此,现在由土地所有者吞食的全部资本,那时将被用来使不同土地上的劳动相等,并使农业上花费的总劳动量减少。

\centerbox{※     ※     ※}

[492]{如我们在前面看到的\fnote{见本卷第1册第46—64、73—78页。——编者注},

亚·斯密起初对价值以及对价值组成部分即利润、工资等的关系发表过正确的观点,后来走上了相反的道路,把工资价格、利润价格、地租价格假定为某种既定的东西,试图把它们规定为独立的量,并把它们加起来得出商品的价格。这种向相反观点的转变意味着:斯密起初是从事物的内部联系考察事物,后来却从它们在竞争中表现出来的颠倒了的形式去考察事物。他天真地把这两种考察方法交织在一起,而且没有觉察到它们之间的矛盾。相反,李嘉图有意识地把竞争形式,把竞争造成的表面现象抽象化,以便考察规律本身。应该指责李嘉图的是,一方面,他的抽象还不够深刻,不够完全,因而当他,比如说,考察商品价值时,一开始就同样受到各种具体关系的限制;另一方面是,他把表现形式理解为普遍规律的直接的、真正的证实或表现;他根本没有揭示这种形式的发展。就第一点来说,他的抽象是极不完全的,就第二点来说,他的抽象是形式的,本身是虚假的。}

\tsectionnonum{[(10)地租率和利润率。不同历史发展阶段上农业生产率和工业生产率的关系]}

现在回过头来简略地分析一下洛贝尔图斯的其他观点。

\begin{quote}{“由国民产品价值的增加引起的工资、资本盈利和地租的增加,在国内既不能提高工资,也不能提高资本盈利,因为现在更多的工资要在更多的工人中分配,增加的盈利要分摊到按同一比例增加的资本上;而地租不管怎样一定会提高,因为地租始终分摊到面积不变的地段上。因此,我所发挥的理论能够令人满意地解释土地价值的巨大增长,——土地价值不外是按普通利息率资本化的地租,——而不必求助于假定农业劳动生产率日益减退,这种假定,同人类社会能够日益完善的思想,以及同一切农业的和统计的事实,是正相矛盾的。”(第160—161页)}\end{quote}

首先要指出,李嘉图[洛贝尔图斯的这个论断是针对他的]在任何地方也没有力求解释“土地价值的巨大增长”。对他来说,这是全然不成问题的。其次,李嘉图(见后面对李嘉图观点的分析)自己明确地说,在谷物或[一般]农产品的价值不变时,在地租率既定时,地租可能增加。\fnote{见本册第357—358页。——编者注}这种增加对李嘉图来说,也是不成问题的。对他来说,成问题的,不是地租总额在地租率保持不变时的增加,而是地租率即地租对预付农业资本之比的提高;因此,问题也不是农产品总量的价值的提高,而是同一数量的农产品例如一夸特小麦价值的提高,——随着这种提高,产品价值超过产品平均价格的余额,从而超过利润率的地租余额,也会增加起来。洛贝尔图斯先生在这里把李嘉图的问题抛开了(更不用说洛贝尔图斯的错误的“材料价值”观念了)。

当然,尽管农业生产率越来越高,地租对预付资本的比率,也就是说,农产品同工业品相比的相对价值,也能够提高。这种情况之所以可能发生有以下两个原因:

第一,拿上例来看,从I推移到II、III、IV,也就是推移到越来越肥沃的土地(但是,比较肥沃的土地所提供的产品量,还没有大到足以使I停止耕种,或者说,足以使价值和平均价格之间的差额缩小到这种程度,以致IV、III、II提供按比例减少的地租,I完全不提供地租)。假如I提供10镑地租,II提供20镑,III提供30镑,IV提供40镑,假如在这四级土地上各投下100镑,那末,I的地租是预付资本的1/10或10%,II是2/10或20%,III是3/10或30%,IV是4/10或40%。总共是100镑地租对400镑预付资本,平均地租率是100/4=25%。如果以投入农业的全部资本来看,现在地租是25%。如果只有I级地(最贫瘠的土地)继续耕种,地租是40镑对400镑,即仍然是10%,不会增加到25%。但是在第一种情况下(如果在I上花费100镑,生产330蒲式耳)只会生产1320蒲式耳,每蒲式耳价格6先令8便士;在第二种情况下(全部四级土地都耕种)却生产1500蒲式耳,而价格相同。在这两种情况下,预付资本是相同的。\endnote{马克思在这里撇开了预付在等级I、II、III、IV上的农业资本的利润。如果I的100镑资本生产330蒲式耳,一蒲式耳为6先令8便士(或1/3镑),那末这一等级的总产品价值是110镑,其中10镑归地租,就没有什么留作利润了。同样,所有四个等级如果各自花费100镑,它们的总产品价值是500镑,这个数额就仅仅是补偿所花费的资本400镑和I、II、III、IV的地租额100镑(10+20+30+40)。——第114页。}

但是,在这里,地租高度的增加不过是表面上的。\endnote{由于在比较肥沃的土地上使用的资本支付级差地租而造成的地租高度(率)的增长“不过是表面上的”,意思是说,这种增长是以“虚假的社会价值”为基础的,关于“虚假的社会价值”,马克思在《资本论》第三卷第三十九章比较详细地谈到。如马克思接着在文中说明的,承租比较肥沃的土地的资本家按比较不肥沃的土地的产品价格出卖他的产品,“就好象他生产同量产品所需要的资本”和比较不肥沃的土地“一样多”。——第114页。}的确,如果我们就产品来计算资本的支出,那就可以看到,在I上,要生产330蒲式耳,必须花费100镑,要生产1320蒲式耳,必须花费400镑。而现在要生产1320蒲式耳,只须花费100+90+80+70,即总共340镑。\endnote{数字90、80、70,马克思大概是用来表示投入等级II、III、IV的资本和这些资本所提供的级差地租(II的100镑资本提供10镑级差地租,III提供20镑,IV提供30镑)之间的差额。如果根据II的产品是360蒲式耳、III的产品是390蒲式耳、IV的产品是420蒲式耳进行确切的计算,那末得出的数字就是:91+(2/3)镑,84+(8/13)镑和78+(4/7)镑。——第114页。}II的90镑和I的100镑生产的同样多,III的80镑和II的90镑生产的同样多,IV的70镑和III的80镑生产的同样多。II、III、IV的地租率,同I比较是提高了。

如果就整个社会来考察,现在要生产同样多的产品,不是使用400镑资本,而是使用340镑,就是说,只使用[原来]资本的85%。

[493]1320蒲式耳只是按不同于第一种情况的方式进行分配。租地农场主现在花费90镑[预付资本]必须交付的[地租],和以前花费100镑交付的一样多,现在花费80镑必须交付的,和以前花费90镑交付的一样多,现在花费70镑必须交付的,和以前花费80镑交付的一样多。但是,现在花费资本90镑、80镑、70镑给他提供的产品,和以前花费100镑提供的正好一样多。他交付的多了,不是因为他为了获得同量产品必须花费的资本多了,而是因为他花费的资本少了;不是因为他的资本的生产率减低了,而是因为这个资本的生产率提高了,但租地农场主照旧按I的产品价格出卖他的产品,就好象他生产同量产品所需要的资本和以前一样多。

[第二,]除了地租率的这种提高(这种提高和各个工业部门中超额利润按不同程度提高是一致的,虽然在工业中这种提高是不固定的),只可能有第二种情况:虽然农产品价值保持不变,就是说,虽然劳动生产率没有减低,地租率却可能提高。这种情况或者是发生在这样的时候:农业生产率和以前一样,但工业生产率提高了,并且这种提高表现在利润率下降;就是说,工业中可变资本对不变资本之比降低了。或者发生在这样的时候:农业生产率也提高,但提高的比例和工业不一样,比工业小。如果农业生产率按1∶2提高,工业生产率按1∶4提高,那末,相对地说,这就同农业生产率保持不变,而工业生产率提高一倍一样。在这种情况下,可变资本对不变资本之比降低,在工业中比在农业中快一倍。

在这两种情况下,工业的利润率都会下降,由于利润率下降,地租率就会提高。在其他情况下,利润率下降不是绝对的(不如说保持不变),它不过对地租来说下降了,这不是因为它本身下降了,而是因为地租提高了,地租对预付资本的比率提高了。

李嘉图没有把上述这些情况加以区别。除开这些情况(即利润率——虽然它是固定的——由于用在比较肥沃土地上的资本有级差地租而相对下降,或者,由于工业生产率的增长,不变资本和可变资本的一般比例发生变化,因而农产品价值超过其平均价格的余额增加),地租率就只有在工业生产率不增长而利润率下降的条件下才能提高。而这又只有在由于农业生产率减低,工资提高或原料价值增加的情况下才能发生。在这种情况下,利润率的下降和地租高度的增长,是由于同一个原因——农业生产率减低,农业中使用的资本的生产率减低。这就是李嘉图的见解。在货币价值不变时,这必然表现为原产品价格的提高。如果价格的提高,象前面考察的那样,是相对的,那末货币价格的任何变动,都不会使农产品的货币价格,与工业品相对而言,绝对地提高。在货币价值降低50%的情况下,值3镑的一夸特小麦就会值6镑,但是值1先令的一磅棉纱也会值2先令。因此,与工业品相对而言,农产品货币价格的绝对提高,决不能用货币价值的变动来解释。

一般说来,应该承认,在原始的、资本主义前的生产方式下,农业生产率高于工业,因为自然在农业中是作为机器和有机体参与人的劳动的,而在工业中,自然力几乎还完全由人力代替(例如手工业等等)。在资本主义生产蓬勃发展的时期,同农业比较,工业生产率发展很快,虽然工业的发展以农业中可变资本和不变资本之比已经发生巨大变化为前提,就是说,以大批人从土地上被赶走为前提。以后,生产率无论在工业中或农业中都增长起来,虽然速度不同。但是工业发展到一定阶段,这种不平衡必定开始缩小,就是说,农业生产率必定比工业生产率相对地增长得快。这里包括:(1)懒惰的农场主被实业家,农业资本家所取代,土地耕种者变为纯粹的雇佣工人,农业大规模经营,即以积聚的资本经营;(2)特别是:大工业的真正科学的基础——力学,在十八世纪已经在一定程度上臻于完善;那些更直接地(与工业相比)成为农业的专门基础的科学[494]——化学、地质学和生理学,只是在十九世纪,特别是在十九世纪的近几十年\fnote{即四十年代和五十年代。——编者注},才发展起来。

根据对两个不同生产部门的商品价值的简单比较,来谈两个部门生产率的大小,是荒谬的。如果1800年1磅棉花值2先令,1磅棉纱值4先令,而1830年棉花的价值是2先令,或者比如说,是18便士,棉纱的价值是3先令或1先令8便士,那就可以比较这两个部门生产率增长的比例。但是,所以可能这样做,只是因为拿1800年的水平作为出发点。可是,如果根据1磅棉花值2先令,1磅棉纱值3先令,就是说,如果根据生产棉花的劳动比纺纱工人的(新加)劳动多一倍,就得出结论说,一种劳动的生产率比另一种劳动的生产率高一倍,这是荒谬的,——这就象因为织造画布比画家在布上绘成的画便宜,就说画家的劳动比织造画布的劳动生产率低一样荒谬。

这里,只有下面的说法才是正确的(如果“生产率”这个概念,也是从资本主义意义即生产剩余价值这个意义上理解,同时考虑到产品的相对量):

如果平均说来,为了在棉纺织工业中使用100个工人花费100镑,按照生产条件,必须在原料、机器等等{价值既定}上花费500镑,另一方面,如果为了在小麦生产中使用100个工人,同样花费100镑,在原料和机器上必须花费150镑,那末,在前一种情况下,可变资本占总资本600镑的1/6,占不变资本的1/5;在后一种情况下,可变资本占总资本250镑的2/5,占不变资本的2/3。这样,投入棉纺织工业的每100镑,只能包含16+(2/3)镑可变资本,而必须包含83+(1/3)镑不变资本;在后一种情况下,却包含40镑可变资本和60镑不变资本。在前一种情况下,可变资本占总资本的1/6即[16+(2/3)]%,在后一种情况下占40%。现在价格史方面的著作少得可怜,这是很明显的。当理论还没有指明必须研究的究竟是什么时,这方面的著作也不能不少得可怜。在剩余价值率既定,比方说,等于20%时,剩余价值在前一种情况下是3+(1/3)镑(因而,利润等于[3+(1/3)]%);在后一种情况下是8镑(因而,利润等于8%)。棉纺织工业中劳动的生产率不如谷物生产中劳动的生产率高,却正是因为棉纺织工业中劳动的生产率较高(就是说,棉纺织工业中的劳动在生产剩余价值的意义上生产率不那么高,是因为它在生产产品的意义上生产率较高)。顺便指出,很明显,例如在棉纺织工业中,[总资本和可变资本之间]1∶(1/6)这个比例只有在不变资本(取决于机器等)花费比如说10000镑,工资花费2000镑,因而总资本是12000镑的情况下才有可能。如果只花费6000,其中工资是1000,那末机器的生产率就会比较低等等。如果只花费100镑,那就根本不能经营。另一方面,可能在花费23000镑的情况下,机器效率增大,其他方面有节约等等,以致也许用不着把全部19166+(2/3)镑用于不变资本,较多的原料和同量的劳动所需要的机器等等少了(按价值),而且在这上面将节省1000镑。于是,可变资本对不变资本之比又会增大,但这不过是因为资本绝对增加了。这是阻碍利润率下降的一个因素。两笔12000镑的资本,会和一笔23000镑的资本生产同量的商品,但是,第一,商品会贵些,因为它们多花费了1000镑;第二,利润率会低一些,因为在23000镑资本中,可变资本占的份额大于总资本的1/6,因而大于两笔12000镑资本总额内可变资本所占的份额。[494]

[494](一方面,随着工业的进步,机器效率更高,也更便宜,因而,如果农业中使用的机器和过去数量相同的话,农业中这部分不变资本会减少;但是机器数量的增加快于它的价格的降低,因为机器这个要素在农业中的发展还是薄弱的。另一方面,随着农业生产率的增长,原料,比如说棉花,价格下降,因而,原料作为价值形成过程的组成部分,和原料作为劳动过程的组成部分,不是按照同一比例增加的。)\endnote{马克思放在括号内的这一段,在手稿中原来是在下两段之后(也在第494页上),即插在篇幅不大的关于配第和戴韦南特的地租量变动观点的历史补充部分中间。从这一段的内容来看,是同前面马克思关于农业生产率和工业生产率的关系的论述相衔接的。——第118页。}

\centerbox{※     ※     ※}

[494]配第已经说过,当时地主害怕农业改良,因为改良的结果会使农产品价格和地租(就其高度来说)下降;由于同样的原因,他们也害怕土地面积增加,并且害怕以前还没有利用的土地投入耕种,因为这等于土地面积增加。(在荷兰,土地面积的增加应该从更直接的意义上理解。)配第说:

\begin{quote}{“地主对排干沼泽、垦伐森林、圈围公有地、栽种驴喜豆和三叶草常出怨言,因为这些做法引起食品价格的下降。”(《政治算术》1699年伦敦版第230页)“整个英格兰和威尔士以及苏格兰低地,一年的地租约为900万镑。”(同上,第231页)}\end{quote}

配第反对地主的这些观点,而戴韦南特进一步发挥[495]这样一种主张:地租的高度可能降低,同时地租量即地租总额却可能增加。戴韦南特说:

\begin{quote}{“地租可能在一些地区和一些郡内下降,但整个说来国内的土地〈他是指土地的价值〉仍然可以不断改良;比方说,如果公园和森林被垦伐,公有地被圈围,如果沼泽被排干,许多地段由于耕种和施肥而改良,那末,自然,这一定会使那些过去已经充分改良、现在已无法再改良的土地的价值减低;虽然某些私人的地租收入因此降低,但与此同时,王国的总地租却由于这些改良而提高。”(戴韦南特《论公共收入和英国贸易》1698年伦敦版第2部分第26—27页)“1666年至1688年期间,私人地租下降了,但王国的地租总额,在这期间比前几年有更大的增加,因为这段时间内土地的改良比以往任何时候都大,都普遍。”(同上,第28页)}\end{quote}

这里我们也看到,英国人说到地租的高度,总是指地租对资本之比,而决不是指地租对王国土地总面积之比(或者笼统地对英亩之比,象洛贝尔图斯先生所说的那样)。

\tchapternonum{[第九章]对所谓李嘉图地租规律的发现史的评论[对洛贝尔图斯的补充评论](插入部分)}

\tsectionnonum{[(1)安德森发现级差地租规律。安德森理论的剽窃者马尔萨斯为了土地所有者的利益歪曲安德森的观点]}

安德森是个实践的租地农场主。他的第一部顺便考察地租性质的著作,出版于1777年。\endnote{马克思指安德森的著作:《谷物法本质的研究;关于为苏格兰提出的新谷物法案》1777年爱丁堡版。——第120页。}当时,詹姆斯·斯图亚特爵士对于很大一部分公众来说还是最有威望的经济学家,但同时,普遍的注意力已经转向一年以前出版的《国富论》了\endnote{指亚当·斯密的著作:《国民财富的性质和原因的研究》1776年伦敦版。——第120页。}。相反,这个苏格兰租地农场主就当时争论的一个直接的实际问题而写的著作(作者在这部著作中不是专门谈地租的,只是顺便说明了地租的性质),却没有能够引起人们的注意。安德森在这部著作中只是偶然地而不是专门地考察地租的。在他自己出版的三卷文集《论农业和农村事务》(三卷集,1775—1796年爱丁堡版)中,有一两篇文章也顺便谈到他的这个理论。1799—1802年伦敦出版的《关于农业、博物学、技艺及其他各种问题的通俗讲座》(见英国博物馆\endnote{英国博物馆是英国国立博物馆(建立于1753年),位于伦敦。博物馆的最重要部分是图书馆,它是世界上最大的图书馆之一。马克思和恩格斯都在英国博物馆的图书馆里从事过研究工作。1908年5月至6月,弗·伊·列宁在图书馆中从事过研究。描绘十九世纪中叶英国博物馆阅览室情况的英国版画的复制品,见本卷第1册第406—407页之间的插图。——第120页。})也是这样的情形。所有这些都是直接为租地农场主和农业家写的著作。如果安德森意识到他的发现的重要性,并且把它作为《地租性质的研究》单独地献给公众,或者,如果他有一点靠贩卖自己的思想为生的才能,就象他的同乡麦克库洛赫靠贩卖别人的思想为生那样,那情况就不同了。

安德森的理论的复制品出现于1815年,一开始就是作为单独的对地租性质的理论研究出现的。这可以从威斯特和马尔萨斯的有关著作的标题看出来:

马尔萨斯:《关于地租的本质和增长的研究》。

威斯特:《论资本用于土地》。

其次,马尔萨斯利用安德森的地租理论,为的是使自己的人口规律第一次同时有政治经济学的和现实的(博物学的)论据,而他从以前的著作家那里借用的关于几何级数和算术级数的荒谬说法,则是一种纯粹空想的假设。马尔萨斯先生立即“抓住了”这个机会。可是李嘉图,正象他自己在序言\endnote{指李嘉图为他的《政治经济学和赋税原理》第一版写的序言。见大卫·李嘉图《政治经济学和赋税原理》1821年伦敦第3版第VI—VII页。——第121页。}中说的那样,是把这个地租学说当作整个政治经济学体系最重要的环节之一,并且赋予它以崭新的理论上的重要性,而在实践方面就更不用说了。

李嘉图显然不知道安德森,因为他在他的政治经济学序言中,把威斯特和马尔萨斯叫做地租理论的创始人。从威斯特叙述这个规律的独特方式判断,他大概也不知道安德森,就象图克不知道斯图亚特一样。马尔萨斯先生的情况就不同了。把马尔萨斯的著作同安德森的著作仔细比较一下,就可以看出马尔萨斯知道安德森,并且利用安德森。马尔萨斯本来就是一个职业剽窃者。[496]只要把他论人口的著作第一版\endnote{[托·罗·马尔萨斯]《人口原理》1798年伦敦版。——第121、125页。}同我以前引用过的唐森牧师的著作\endnote{马克思指唐森的书《论济贫法》(1786年伦敦版)。马克思在1861—1863年手稿第III本(第112—113页)《绝对剩余价值》这一节中引用了这本书。马克思在那里引用的三段引文也在《资本论》第一卷第二十三章引用过(见《马克思恩格斯全集》中文版第23卷第709页)。——第121页。}对比一下,就会相信,马尔萨斯不是作为具有自由创作思想的人来加工唐森的著作,而是作为盲从的剽窃者照抄和转述唐森的著作,同时没有一个地方提到唐森,隐匿了唐森的存在。

马尔萨斯利用安德森的观点的方式,是很有特色的。安德森维护鼓励谷物输出的出口奖励和限制谷物输入的进口税,决不是从地主的利益出发,而是认为这样的立法会“降低谷物的平均价格”,保证农业生产力的均衡发展。马尔萨斯采用安德森的这个实际结论,则因为马尔萨斯作为英国国教会的真诚教徒,是土地贵族的职业献媚者,他从经济学上替土地贵族的地租、领干薪、挥霍、残忍等等辩护。只是在工业资产阶级的利益同土地所有权的利益,同贵族的利益一致时,马尔萨斯才拥护工业资产阶级的利益,即拥护他们反对人民群众,反对无产阶级;但是,凡是土地贵族同工业资产阶级的利益发生分歧并且互相敌对时,马尔萨斯就站在贵族一边,反对资产阶级。因此,他为“非生产劳动者”、消费过度等等辩护。

相反,安德森认为,支付地租的土地和不支付地租的土地之间,或支付不同地租的土地之间所以会产生差别,是因为不提供地租或提供较少地租的土地,同提供地租或提供较多地租的土地比较起来,相对的不肥沃。但是他明确地说,不同等级土地的这种相对的肥沃程度,从而较坏等级土地同较好等级土地比较起来相对的不肥沃,同农业的绝对生产率绝对没有任何关系。相反,他曾着重指出,各种等级土地的绝对肥力不但能够不断提高,而且随着人口的增长也必定会提高,他还进一步断言,不同等级土地的肥力的不平衡,能够日益趋于平衡。他说,英国农业目前的发展程度,丝毫不能说明它的可能的发展。因而他说,在一国可能是谷物价格高而地租低,在另一国可能是谷物价格低而地租高;这是从他的基本原理出发的,因为在这两个国家,地租的高低和地租本身的存在,决定于肥沃土地和贫瘠土地之间的差别,但其中任何一个国家,地租都不决定于绝对肥力;其中每一个国家,地租只决定于现有各种等级土地的肥沃程度的差别,其中任何一个国家,地租都不决定于各种等级土地的平均肥力。他由此得出结论说,农业的绝对生产率同地租绝对没有任何关系。因此他后来声明——我们在后面会看到\fnote{见本册第158页。——编者注}——他是马尔萨斯人口论的死敌,可是他没有料到他自己的地租理论会成为这种奇谈怪论的根据。安德森说明,在英国,1750至1801年的谷物价格高于1700至1750年,决不是由于越来越不肥沃的土地投入耕种,而是由于这两个时期立法对农业的影响。

马尔萨斯干了什么呢?

他利用安德森的理论,代替自己的(也是剽窃来的)几何级数和算术级数的怪诞幻想——他把这种怪诞幻想当作“漂亮辞句”保留着——来证明自己的人口论。在安德森理论的实际结论符合地主利益的限度内,他保留安德森理论的实际结论,——仅仅这一事实就证明,马尔萨斯同安德森本人一样,不了解这个理论同资产阶级社会的政治经济学体系的联系;——他不去考察这个理论的创始人的反证,就利用这个理论去反对无产阶级。从这个理论出发,在理论上和实践上向前迈进一步的使命就落到了李嘉图身上,这就是:在理论上,作出商品的价值规定等等,并阐明土地所有权的性质;在实践上,反对资产阶级生产基础上的土地私有权的必要性,并且更直接地反对国家促进这种土地所有权发展的一切措施,如谷物法。马尔萨斯得出的唯一的实际结论在于:为地主在1815年要求的保护关税辩护——这是巴结贵族,——并且对财富生产者的贫困进行新的辩解,为劳动剥削者进行新的辩护。从这一方面来说,是巴结工业资本家。

马尔萨斯的特点,是思想极端卑鄙,——只有牧师才可能这样卑鄙,[497]他把人间的贫困看作对罪恶的惩罚,而且在他看来,非有一个“悲惨的尘世”不行,但是同时,他考虑到他所领取的牧师俸禄,借助于关于命运的教义,认为使统治阶级在这个悲惨的尘世上“愉快起来”,对他是极为有利的。这种思想的卑鄙还表现在马尔萨斯的科学工作上。第一,表现在他无耻的熟练的剽窃手艺上;第二,表现在他从科学的前提做出的那些看人眼色的而不是毫无顾忌的结论上。

\tsectionnonum{[(2)发展生产力的要求是李嘉图评价经济现象的基本原则。马尔萨斯为统治阶级最反动的分子辩护。达尔文实际上推翻了马尔萨斯的人口论]}

李嘉图把资本主义生产方式看作最有利于生产、最有利于创造财富的生产方式,对于他那个时代来说,李嘉图是完全正确的。他希望为生产而生产,这是正确的。如果象李嘉图的感伤主义的反对者们那样,断言生产本身不是目的本身,那就是忘记了,为生产而生产无非就是发展人类的生产力,也就是发展人类天性的财富这种目的本身。如果象西斯蒙第那样,把个人的福利同这个目的对立起来,那就是主张,为了保证个人的福利,全人类的发展应该受到阻碍,因而,举例来说,就不能进行任何战争,因为战争无论如何会造成个人的死亡。(西斯蒙第只是与那些掩盖这种对立、否认这种对立的经济学家相比较而言,才是正确的。)这种议论,就是不理解:“人”类的才能的这种发展,虽然在开始时要靠牺牲多数的个人,甚至靠牺牲整个阶级,但最终会克服这种对抗,而同每个个人的发展相一致;因此,个性的比较高度的发展,只有以牺牲个人的历史过程为代价。至于这种感化议论的徒劳,那就不用说了,因为在人类,也象在动植物界一样,种族的利益总是要靠牺牲个体的利益来为自己开辟道路的,其所以会如此,是因为种族的利益同特殊个体的利益相一致,这些特殊个体的力量,他们的优越性,也就在这里。

由此可见,李嘉图的毫无顾忌不仅是科学上的诚实,而且从他的立场来说也是科学上的必要。因此对李嘉图来说,生产力的进一步发展究竟是毁灭土地所有权还是毁灭工人,这是无关紧要的。如果这种进步使工业资产阶级的资本贬值,李嘉图也是欢迎的。如果劳动生产力的发展使现有的固定资本贬值一半,那将怎样呢?——李嘉图说,——要知道人类劳动生产率却因此提高了一倍。这就是科学上的诚实。如果说李嘉图的观点整个说来符合工业资产阶级的利益,这只是因为工业资产阶级的利益符合生产的利益,或者说,符合人类劳动生产率发展的利益,并且以此为限。凡是资产阶级同这种发展发生矛盾的场合,李嘉图就毫无顾忌地反对资产阶级,就象他在别的场合反对无产阶级和贵族一样。

而马尔萨斯呢!这个无赖,从已经由科学得出的(而且总是他剽窃来的)前提,只做出对于贵族反对资产阶级以及对于贵族和资产阶级两者反对无产阶级来说,是“合乎心意的”(有用的)结论。因此,他不希望为生产而生产,他所希望的只是在维持或加强现有制度并且为统治阶级利益服务的那种限度内的生产。

他的第一部著作\endnote{[托·罗·马尔萨斯]《人口原理》1798年伦敦版。——第121、125页。},就已经是靠牺牲原著而剽窃成功的最明显的写作例子之一。这部著作的实际目的,是为了英国现政府和土地贵族的利益,“从经济学上”证明法国革命及其英国的支持者追求改革的意图是空想。一句话,这是一本歌功颂德的小册子,它维护现有制度,反对历史的发展;而且它还为反对革命法国的战争辩护。

他1815年关于保护关税和地租的著作\endnote{指马尔萨斯的两本小册子《关于限制外国谷物进口政策的意见的理由》和《关于地租的本质和增长的研究》。——第126页。},部分地是要证明他以前为生产者的贫困所作的辩解,但首先是为了维护反动的土地所有权,反对“开明的”、“自由的”和“进步的”资本,特别是要证明英国立法当时为了保护贵族利益反对工业资产阶级而采取的倒退措施是正确的。\endnote{指1815年的谷物法。该法禁止向英国输入谷物,直到英国国内谷物价格每夸特不低于80先令时为止。——第126页。}最后,[498]他的《政治经济学原理》是反对李嘉图的,这本书的根本目的,就是要把“工业资本”及其生产率依以发展的那些规律的绝对要求,纳入从土地贵族、国教会(马尔萨斯所属的教会)、政府养老金领取者和食税者的现有利益看来是“有利的”和“适宜的”“范围”。但是,一个人如果力求使科学去适应不是从科学本身(不管这种科学如何错误),而是从外部引出的、与科学无关的、由外在利益支配的观点,我就说这种人“卑鄙”。

从李嘉图来说,他把无产者看成同机器、驮畜或商品一样,却没有任何卑鄙之处,因为无产者只有当作机器或驮畜,才促进“生产”(从李嘉图的观点看),或者说,因为无产者在资产阶级生产中实际上只是商品。这是斯多葛精神,这是客观的,这是科学的。只要有可能不对他的科学犯罪,李嘉图总是一个博爱主义者,而且他在实际生活中也确是一个博爱主义者。

马尔萨斯牧师就完全不同了。他[也]为了生产而把工人贬低到驮畜的地位,甚至使工人陷于饿死和当光棍的境地。[但是]在同样的生产的要求减少地主的“地租”时,在同样的生产的要求威胁国教会的“什一税”或“食税者”的利益时,或者,在为了一部分代表生产进步的工业资产阶级而去牺牲另一部分本身利益阻碍生产进步的工业资产阶级时,——总之,在贵族的某种利益同资产阶级的利益对立时,或者,在资产阶级中保守和停滞的阶层的某种利益同进步的资产阶级的利益对立时,——在所有这些场合,马尔萨斯“牧师”都不是为了生产而牺牲特殊利益,而是竭尽全力企图为了现有社会统治阶级或统治阶级集团的特殊利益而牺牲生产的要求。为了这个目的,他在科学领域内伪造自己的结论。这就是他在科学上的卑鄙,他对科学的犯罪,更不用说他那无耻的熟练的剽窃手艺了。马尔萨斯在科学上的结论,是看着统治阶级特别是统治阶级的反动分子的“眼色”捏造出来的;这就是说,马尔萨斯为了这些阶级的利益而伪造科学。相反,对于被压迫阶级,他的结论却是毫无顾忌的,残酷无情的。他不单单是残酷无情,而且宣扬他的残酷无情,厚颜无耻地以此自夸,并且在用他的结论反对“无权者”时,把他的结论夸大到极端,甚至超过了从他的观点看来还可以在科学上说得过去的程度。\fnote{[499}例如李嘉图(见前),当他的理论使他得出结论说,把工资提高到工资最低限度以上并不会增加商品的价值时,他就直接说出这一点。而马尔萨斯坚持工资保持低水平,目的是要资产者借此来发财。[499]]

因此,英国工人阶级憎恨马尔萨斯——科贝特粗鲁地称他为“江湖牧师”(科贝特虽然是当代英国最大的政论家;但他缺少莱比锡大学教授\endnote{暗指德国庸俗经济学家、莱比锡大学教授罗雪尔。——第127页。}的教养,并且公开反对“学者的语言”),——对马尔萨斯的这种憎恨是完全正当的;人民凭着真实的本能感觉到,在这里反对他们的不是一个科学的人,而是一个被他们的敌人收买的统治阶级的辩护士,是统治阶级的无耻的献媚者。

一个最初发现某种思想的人,可能由于善意的误解,把这种思想夸大到极端;而一个把这种思想夸大到极端的剽窃者,却总是把这种夸大当作“有利可图的生意”。

马尔萨斯的“人口论”这部著作第一版没有包含一个新的科学词汇;这本书只应看作卡普勤教士喋喋不休的说教,只应看作是用阿伯拉罕·圣克拉\endnote{阿伯拉罕·圣克拉是奥地利传教士和著作家乌尔利希·梅格尔勒(1644—1709年)的笔名,他力图用公众易懂的形式宣传天主教,并用所谓民间文体来进行“救人”的说教和写劝善的作品。——第128页。}文体对唐森、斯图亚特、华莱士、埃尔伯等人的论断的改写。因为这本书实际上只是指望以它的大众化的形式来引人注意,所以它理所当然地要引起大众的憎恨。

同资产阶级政治经济学界那些可怜的和谐论者比较起来,马尔萨斯的唯一功绩,就是特别强调不和谐。的确,他决没有发现不和谐,但他毕竟以牧师所固有的扬扬得意的厚颜无耻肯定、描绘并宣扬了这种不和谐。

\centerbox{※     ※     ※}

[499]查理·达尔文在他的著作《根据自然选择即在生存斗争中适者保存的物种起源》1860年伦敦版(第五次印刷,一千册)[第4—5页]的绪论中说:

\begin{quote}{“下一章将考察全世界整个生物界中的生存斗争,那是依照几何级数高度繁殖的不可避免的结果。这是马尔萨斯学说对于整个动物界和整个植物界的应用。”}\end{quote}

达尔文在他的卓越的著作中没有看到,他在动物界和植物界发现了“几何”级数,就是把马尔萨斯的理论驳倒了。马尔萨斯的理论正好建立在他用华莱士关于人类繁殖的几何级数同幻想的动植物的“算术”级数相对立上面。在达尔文的著作中,例如在谈到物种消灭的地方,也有在细节上(更不用说达尔文的基本原则了)从博物学方面对马尔萨斯理论的反驳。而当马尔萨斯的理论以安德森的地租理论为依据时,他的理论又被安德森本人驳倒了。\endnote{在手稿中,紧接着插入了一小段话,在这段话中,马克思用李嘉图关于工资水平的观点同马尔萨斯关于这个问题的观点相对比。这段话以脚注的形式移至前面,即本册第127页。——第128页。}

[499]

\tsectionnonum{[(3)罗雪尔歪曲地租观点的历史。李嘉图在科学上公正的例子。投资于土地时的地租和利用其他自然要素时的地租。竞争的双重作用]}

[499]安德森顺便发挥了地租理论的第一部著作,是具有实际意义的论战性著作,它谈的不是地租,而是保护关税政策。这部著作出版于1777年,它的标题已经表明:第一,它追求实际的目的;第二,它涉及当时一个直接的立法行为,在这个问题上工业家和地主的利益是对立的。这部著作就是:《谷物法本质的研究;关于为苏格兰提出的新谷物法案》(1777年爱丁堡版)。

1773年的法律(英格兰的法律;关于这个问题参看麦克库洛赫的书目\endnote{马克思指麦克库洛赫的著作《政治经济学文献。各科分类书目。附史评、论述和作者介绍》1845年伦敦版。——第129页。})在1777年(好象)应该在苏格兰施行(见英国博物馆)。

\begin{quote}{安德森说:“1773年的法律是由一种公开宣布的意图引起的,这种意图就是要为我国的工业家降低谷物价格,以便通过进口奖励,保证本国人民得到更便宜的食物。”(《关于导致不列颠目前粮荒的情况的冷静考察》1801年伦敦版第50页)}\end{quote}

由此可见,安德森的著作是一部维护包括地主在内的农业主利益(保护关税政策)、反对工业家利益的论战性著作。安德森是把自己的书当作“公开”维护一定党派的利益的著作出版的。地租理论在这里只是顺便谈到。而且在他后来一些总是或多或少涉及上述利益斗争的著作中,他也只是顺便把地租理论重新提了一两次,从来没有想到它的科学意义,或者说,甚至没有想到把它作为一个独立的论题。了解了这一点,我们就可以判断显然不知道安德森著作的威廉·修昔的底斯·罗雪尔\endnote{马克思用古希腊大历史学家修昔的底斯的名字来称呼罗雪尔,这是因为,如马克思在《剩余价值理论》第三册(马克思手稿第922页)中所说,“罗雪尔教授先生谦虚地宣称自己是政治经济学的修昔的底斯”。罗雪尔在他的《国民经济学原理》的序言中不知羞耻地引证了修昔的底斯。“修昔的底斯·罗雪尔”这个称呼具有辛辣的讽刺性:正象马克思在本章和其他许多地方指出的那样,罗雪尔既严重歪曲了经济关系的历史,又严重歪曲了经济理论的历史。——第130页。}的下述意见是否正确:

\begin{quote}{“值得惊奇的是,在1777年几乎无人注意的学说,到1815年以后却突然引起人们极大的关心,一些人拥护,另一些人反驳,因为这个学说涉及货币所有者和土地所有者之间在这个时期如此尖锐起来的对立。”(《国民经济学原理》1858年第3版第297—298页)}\end{quote}

在这段话中,错误和字数一样多。第一,安德森并没有象威斯特、马尔萨斯和李嘉图那样,把自己的见解当作“学说”提出来。第二,这个见解不是“几乎”而是“完全”无人注意。第三,这个见解最初在一部著作中顺便被提到,这部著作专门只以1777年在工业家和地主之间大大发展起来的利益对立为中心,只“涉及”这个实际的利益斗争,但根本“没有涉及”一般的[500]国民经济学理论。第四,在1815年,这个理论被它的复制者之一马尔萨斯大肆宣扬,完全是为了维护谷物法,就象安德森所做的一样。同一个学说,它的创始人和马尔萨斯都是用来维护土地所有权,而大卫·李嘉图却用来反对土地所有权。因此,至多可以说,这个理论的一些拥护者维护土地所有权的利益,而另一些拥护者反对这种利益,但是不能说,在1815年拥护土地所有权的人反对这个理论(因为马尔萨斯在李嘉图之前拥护这个理论),也不能说,反对土地所有权的人拥护这个理论(因为大卫·李嘉图无须“拥护”这个理论去反对马尔萨斯,因为他自己把马尔萨斯当作这个理论的创始人之一,并且把马尔萨斯当作自己的先驱;他只是不得不“反驳”马尔萨斯从这个理论得出的实际结论)。第五,威廉·修昔的底斯·罗雪尔所“涉及”的“货币所有者”和“土地所有者”之间的对立,直到现在同安德森的地租理论、同复制这个理论、同拥护这个理论、同反对这个理论都绝对没有任何关系。威廉·修昔的底斯可以从约翰·斯图亚特·穆勒的书(《略论政治经济学的某些有待解决的问题》1844年伦敦版第109—110页)中知道:(1)英国人所说的“金融阶级”是指货币贷放者;(2)这些货币贷放者或者一般是靠利息生活的人,或者是职业的放债人,如银行家、票据经纪人等等。照穆勒所说,所有这些人都以“金融阶级”的身分同“生产阶级”(穆勒所说的生产阶级是指“工业资本家”,而把工人撇在一边)相对立,或者至少同生产阶级有区别。因此,威廉·修昔的底斯应当看到,“生产阶级”即工业家,工业资本家的利益,同金融阶级的利益是极不相同的两回事;这些阶级也是不同的阶级。其次,威廉·修昔的底斯应当看到,工业资本家和地主之间的斗争决不是“货币所有者”和“土地所有者”之间的斗争。如果威廉·修昔的底斯了解1815年谷物法的历史和围绕谷物法进行的斗争,他就会从科贝特的著作中知道,地主(土地所有者)和放债人(货币所有者)曾经一道反对工业资本家。不过科贝特是“粗鲁的”。最后,威廉·修昔的底斯从1815至1847年的历史应当知道,大部分货币所有者,甚至一部分商人(例如在利物浦),在围绕谷物法的斗争中,都成了土地所有者反对工业资本家的同盟者。[500]

[502](可以使罗雪尔先生感到惊奇的至多是这件事:同一个“学说”,在1777年用来维护“土地所有者”,而在1815年却成了反对“土地所有者”的武器,并且只是到这时才开始引起人们的注意\endnote{指1815年伦敦出版的威斯特的著作《论资本用于土地,对谷物进口严加限制的失策》和李嘉图的著作《论谷物的低价对资本利润的影响;证明限制进口的不当》。——第131页。}。)[502]

[500]如果我要把威廉·修昔的底斯在他的历史文献评介中犯的所有这类粗暴歪曲历史的错误,都如此详细地加以说明,那我就得写一部象他的《国民经济学原理》一样厚的书,而这样的书确实“不值得花费那么多纸张来写”。这么一个威廉·修昔的底斯在学术上的无知,对于其他科学领域的研究者会造成多么有害的影响,从阿·巴斯提安先生的例子可以看出来。巴斯提安在自己的著作《历史上的人》1860年版第一卷第374页的注中,就引用了威廉·修昔的底斯上述的这段话来证明一个“心理学的”原理。顺便说一下,关于巴斯提安,不能说“技巧驾驭了材料”\fnote{见奥维狄乌斯《变形记》。——编者注}。在这里倒是技巧应付不了自己的原材料。而且,通过我所“知道”的不多的几门科学,我发现,知道“一切”科学的巴斯提安先生经常信赖威廉·修昔的底斯之流的权威,这对于“万能学者”一般是不可避免的。

[501]我希望人们不要责备我对威廉·修昔的底斯“残酷无情”。这个书呆子自己对待科学是多么“残酷无情”啊!既然他胆敢以高傲自大的口气谈论李嘉图的“半真理”\endnote{威·罗雪尔《国民经济体系》,第1卷《国民经济学原理》1858年斯图加特和奥格斯堡增订第3版第191页。——第132页。},那我无论如何也同样可以谈他的“全无真理”。而且,威廉·修昔的底斯在列举书目方面一点也不“公正”。在他看来,谁不“值得尊敬”,谁就在历史上不存在;例如,在他看来,洛贝尔图斯作为地租理论家是不存在的,因为洛贝尔图斯是“共产主义者”。而且,威廉·修昔的底斯对于“值得尊敬的著作家”的看法也不准确。例如,在麦克库洛赫看来,贝利是存在的,甚至被看作是划时代的人。在威廉·修昔的底斯看来,贝利是不存在的。要在德国发展并普及政治经济科学[502],本应该由洛贝尔图斯这样一些人来创办一种杂志,向一切研究者开放(但不向书呆子和庸俗化者开放),并且杂志的主要目的是揭露在这门科学本身及其历史方面的职业学者的不学无术。[502]

\centerbox{※     ※     ※}

[501]安德森根本没有研究他的地租理论同政治经济学体系的关系,这一点是不足为奇的,因为他的第一本书是在亚·斯密《国富论》出版一年以后出版的,当时“政治经济学体系”一般说来还刚刚在形成,因为斯图亚特的体系也只是在几年以前才出现。不过,如果谈到安德森在他考察的特殊对象范围内所拥有的材料,那末这种材料无疑比李嘉图的更广泛。正如李嘉图在他根据休谟的理论复制而成的货币理论中主要只看到1797至1809年间的事件一样,李嘉图在他根据安德森的理论复制而成的地租理论中只看到1800至1815年间谷物价格上涨的经济现象。

\centerbox{※     ※     ※}

下面这几段话,很能说明李嘉图的特点:

\begin{quote}{“如果因考虑到某一个阶级的利益而使国家财富和人口的增长受到阻碍,我将感到非常遗憾。”(李嘉图《论谷物的低价对资本利润的影响》1815年伦敦第2版第49页)}\end{quote}

在谷物自由输入的情况下,“土地停止耕种”。(同上,第46页)因此,土地所有权为发展生产而被牺牲。

可是,在谷物自由输入的同样情况下:

\begin{quote}{“不可否认会有一定数量的资本损失掉。但是,拥有资本或保持资本是目的呢,还是手段?毫无疑问是手段。我们所需要的是商品的富足〈一般财富〉,如果能证明,牺牲我们的资本的一部分,我们就可以增加用于使我们享乐和幸福的那些物品的年生产,那我们就不应当为我们的资本的一部分遭受损失而发牢骚。”(《论农业的保护关税》1822年伦敦第4版第60页)}\end{quote}

李嘉图把不属于我们或他的、而是由资本家投入土地的资本,叫作“我们的资本”。但是这个我们是谁呢?是指国民的平均数。“我们的”财富的增加就是社会财富的增加,这个社会财富本身就是目的,而不管这个财富由谁分享!

\begin{quote}{“对于一个拥有2万镑资本,每年获得利润2000镑的人来说,只要他的利润不低于2000镑,不管他的资本是雇100个工人还是雇1000个工人,不管生产的商品是卖1万镑还是卖2万镑,都是一样的。一个国家的实际利益不也是这样吗?只要这个国家的实际纯收入,它的地租和利润不变,这个国家的人口有1000万还是有1200万,都是无关紧要的。”(《政治经济学和赋税原理》第3版第416页)}\end{quote}

在这里,“无产阶级”为财富而被牺牲。在无产阶级对于财富的存在是无关紧要的时候,财富对于无产阶级的存在也是无关紧要的。在这里群众本身——人类大众——是“不值什么的”。

这三个例子[502]表明了李嘉图科学上的公正。

\centerbox{※     ※     ※}

{土地(自然)等等,是农业中使用的资本借以投入的要素。因此在这里,地租等于这个要素中所创造的劳动产品的价值超过这个产品平均价格的余额。如果有一种属于个人私有财产的自然要素(或物质)加入别一种生产,而又不构成这种生产的(物质)基础,那末地租——如果这种地租只是由于该要素单纯加入生产而产生的话——就不可能是这个产品的价值超过平均价格的余额,而只是这个产品的一般平均价格超过它自身平均价格的余额。例如,瀑布可以为工厂主代替蒸气机,并使工厂主能够节省煤炭。工厂主拥有这个瀑布,比方说,就可以经常把纱卖得比它的[个别]平均价格贵,并得到超额利润。如果这个瀑布为土地所有者所有,这种超额利润就会以地租的形式归土地所有者所得。霍普金斯先生在他论“地租”的书中也指出,在朗卡郡,瀑布不只提供地租,而且还因瀑布落差的天然能量不同,而提供级差地租。\endnote{马克思指霍普金斯的书《关于调节地租、利润、工资和货币价值的规律的经济研究》1822年伦敦版。这本书的有关段落马克思在后面(见本册第153页)引用了。——第135页。}这里的地租无非就是产品的市场平均价格超过它的个别平均价格的余额。}[502]

\centerbox{※     ※     ※}

[502]{在竞争中,应当区分两种平均化运动。在同一生产领域内部,资本把这个领域内部生产的商品的价格平均化为同一市场价格,而不管这些商品的[个别]价值同这个市场价格的关系怎样。如果没有不同生产领域之间的平均化,平均市场价格就应当等于商品的[市场]价值。这些不同领域之间的竞争,在资本的相互作用不被第三种力量——土地所有权等等——阻碍、破坏的情况下,把[市场]价值平均化为平均价格。}

\tsectionnonum{[(4)洛贝尔图斯关于产品变贵时价值和剩余价值的关系问题的错误]}

洛贝尔图斯认为,如果一种商品比另一种商品贵(因而,如果前者比后者物化的劳动时间多),那末,在不同领域的剩余价值率相等,或者说工人受的剥削程度相同的情况下,这种商品也必定会因此包含更多的无酬劳动时间,即剩余劳动时间。他的这个观点是完全错误的。

如果同一劳动在贫瘠土地上(或在歉收年)生产1夸特小麦,而在肥沃土地上(或在丰收年)生产3夸特小麦;如果同一劳动在很富的金矿生产1盎斯金,而在较贫的或已经枯竭的金矿只生产1/3盎斯金;如果生产1磅羊毛需要的劳动时间同把3磅羊毛纺成纱需要的劳动时间一样多,那末,首先,1夸特小麦和3夸特小麦1的价值,1盎斯金和1/3盎斯金的价值,1磅羊毛和3磅毛纱的价值(扣除其中包含的羊毛价值)一样大。它们包含着同量的劳动时间,因而,按照[剩余价值率相等]的假定,也包含着同量的剩余劳动时间。当然,1夸特[贫瘠土地上长的小麦]包含的剩余劳动量[比肥沃土地上长的1夸特小麦]多,然而,在前一场合我们只有1夸特小麦,在后一场合却有3夸特小麦;或者,在前一场合我们有1磅羊毛,而在后一场合有3磅纱(减去材料的价值)。因而,在两种场合[剩余价值]量是相等的。如果拿一个单位商品同另一个单位商品相比较,那末剩余价值的比例量也是相等的。按照假定,1夸特[坏地上长的小麦]或1磅羊毛与3夸特[好地上长的小麦]或3磅纱包含着同样多的劳动。因此,在两种场合,花费在工资上的资本同剩余价值之比是相同的。1磅羊毛包含的劳动时间比1磅纱包含的大两倍。如果剩余价值大两倍,那末与它有关的花费在工资上的资本也大两倍。因而,它们二者之比仍然一样。

在这里,洛贝尔图斯的计算是完全错误的,他完全错误地把花费在工资上的资本同[503]物化着用工资换来的劳动的那个较多或较少的商品量相比。只要象洛贝尔图斯假定的那样,工资(或剩余价值率)是既定的,那末,这种计算就是完全错误的。同一劳动量,例如12小时,可以表现为x商品或3x商品。一种场合的1x商品和另一种场合的3x商品包含着同样多的劳动和剩余劳动;但是,不论在哪一种场合,所花费的都不多于一个工作日,而且不论在哪一种场合,剩余价值率都不大于例如1/5。第一种场合1x的1/5比x,同第二种场合3x的1/5比3x是一样的。如果我们把三个x分别叫做x′、x″、,那末x′、x″、各包含4/5的有酬劳动和1/5的无酬劳动。另一方面,如果在生产率较低的条件下要生产出同在生产率较高的条件下一样多的商品,那末商品中就会包含更多的劳动,因而也包含更多的剩余劳动。这是完全正确的。不过这时就要相应地花费更多的资本。为了生产3x,就应当比生产1x(在工资上)多花费两倍的资本。

当然,毫无疑问,工业加工的原料不可能多于农业所提供的原料,因而,例如,用来纺纱的羊毛磅数不可能超过生产出来的羊毛磅数。所以,如果纺毛的生产率提高两倍,那末,在假定羊毛生产条件不变的情况下,为生产羊毛花费的时间就要比过去增加两倍,为生产羊毛使用的资本也要增加两倍,可是把这些数量增加两倍的羊毛纺成纱,只需要和过去一样多的纺纱工人的劳动时间。但是,[剩余价值]率不变。同量纺纱工人的劳动创造的价值和过去一样多,包含的剩余价值也一样多。生产羊毛的劳动所包含的剩余价值增加了两倍,但是与此相适应,包含在羊毛中的全部劳动或预付在工资上的资本也增加了两倍。因而,增大了两倍的剩余价值要按增大了两倍的资本来计算。所以不能根据这一点说,纺纱生产中的剩余价值率比羊毛生产中的低。只能说后一部门花费在工资上的资本比前一部门多两倍(因为这里假定,纺纱和羊毛生产中的变化,都不是由于它们的不变资本的变化引起的)。

这里必须把下述情况加以区别。同一劳动加不变资本提供的产品,歉收年少于丰收年,贫瘠土地少于肥沃土地,较贫的金属矿少于较富的金属矿。因而,前者的产品比较贵,也就是说,同量产品在这里包含着较多的劳动和较多的剩余劳动;可是在第二种场合产品的数量比较多。其次,两类产品的每一单位产品中有酬劳动和无酬劳动之比,并不因此而发生任何变化;因为,如果单位产品中包含较少的无酬劳动,那末根据假定,它也相应地包含较少的有酬劳动。因为这里假定,资本各有机组成部分的比例,即可变资本和不变资本的比例,没有任何变化。我们已经假定,同数额的可变资本和不变资本,在不同的条件下会提供不同的,即较多或较少的产品量。

看来洛贝尔图斯先生经常把这些东西混淆起来,并且好象理所当然地从产品的单纯变贵得出了关于剩余价值更多的结论。至于剩余价值率,那末根据所作的假定,这种结论已经是不正确的了,至于剩余价值量,那也只有在下述条件下才正确:在一种场合比另一种场合预付更多的资本,也就是说,原来生产多少较便宜的产品,现在也生产多少较贵的产品,或者(象上述纺纱的例子那样),较便宜的产品数量的增加,先要有较贵的产品数量相应的增加。

\tsectionnonum{[(5)李嘉图否认绝对地租——他的价值理论中的错误的后果]}

[504]尽管地租率不变或者甚至下降,地租,从而土地价值,可能增长,也就是说,农业生产率也增长,——这一点李嘉图有时忘记,但他是知道的。无论如何安德森是知道这一点的,而且配第和戴韦南特就已经知道了。问题不在这里。

李嘉图撇开了绝对地租问题,他为了理论而否认绝对地租,因为他从错误的前提出发:如果商品的价值决定于劳动时间,商品的平均价格就必定等于商品的价值(因此他又做出一个与实际相矛盾的结论:比较肥沃的土地的竞争必然使比较不肥沃的土地停止耕种,即使后者过去是提供地租的)。如果商品的价值和它们的平均价格等同,那末绝对地租——即最坏的耕地上的地租或最初的耕地上的地租——在这两种情况下都是不可能的。什么是商品的平均价格?就是生产商品花费的全部资本(不变资本加可变资本)加上包含在平均利润(例如10%)中的劳动时间。因此,如果资本在某种要素中——仅仅因为这是一种特殊的自然要素例如土地,——生产出高于平均价格的价值,那末这种商品的价值就会超过它的价值,而这个超额价值就同价值等于一定量劳动时间的价值概念发生矛盾。结果就成了:一种自然要素,即某种不同于社会劳动时间的东西,创造了价值。但这是不可能的。因此,投在仅仅作为土地的土地上的资本不可能提供任何地租。最坏的土地仅仅是土地。如果说较好土地提供地租,这只不过证明,社会必要劳动和个别必要劳动之间的差别在农业中固定下来了,因为这种差别在这里有一个自然基础,而这种差别在工业中却是不断消失的。

因而,不应该有绝对地租存在,只可能有级差地租存在。因为承认绝对地租存在,就是承认同量劳动(投入不变资本的物化劳动和用工资购买的劳动)由于投入不同的要素,或加工不同的材料,会创造不同的价值。但是如果承认,虽然在每一个生产领域的产品中物化着同一劳动时间,却依然存在这种价值差别,那就是承认,不是劳动时间决定价值,而是某种不同于劳动时间的东西决定价值。这种价值量的差别会取消价值概念,推翻下述原理:价值实体是社会劳动时间,因而价值的差别只能是量的差别,而这个量的差别只能等于所耗费的社会劳动时间量的差别。

由此可见,从这个观点出发,为了保持价值范畴——不仅价值量决定于不同的劳动时间量,而且价值实体决定于社会劳动,——就要否认绝对地租。而否认绝对地租可以表现为两种说法。

第一,最坏的土地不能提供地租。较好等级土地的地租可以这样来解释:比较肥沃的土地和比较不肥沃的土地的产品市场价格是一样的。但是最坏的土地仅仅是土地,它本身没有等级差别。它只是不同于工业投资的特殊投资领域。如果它提供地租,那末地租的产生就是由于:同一劳动量用在不同的生产领域表现为不同的价值,从而不是劳动量本身决定价值,包含等量劳动的产品[在价值上]彼此也就不等。

[505]或者[第二],最初的耕地不能提供地租。因为,什么是最初的耕地呢?“最初的”耕地,既不是较好的土地,也不是较坏的土地。这仅仅是土地,是没有等级差别的土地。最初,农业投资与工业投资的区别仅仅在于资本所投入的领域不同。但是,既然等量劳动表现为等量价值,那就绝对没有理由使投入土地的资本除了利润之外还提供地租,除非投入这个领域的同一劳动量生产出一个较大的价值,以致这个价值超过工业中生产的价值的余额构成等于地租的超额利润。但是,这就意味着确认土地本身创造价值,也就是意味着取消价值概念本身。

因此,最初的耕地最初不能提供任何地租,否则整个价值理论就要被推翻。而且这一点很容易(虽然不一定,这可以从安德森那里看到)同这样一种观点联系起来,即认为人们最初自然不是选择最坏的土地,而是选择最好的土地耕种,因而,最初不提供地租的土地,后来由于人们不得不进而耕种越来越坏的土地,也开始提供地租了,这样,在向地狱下降的过程中,即向越来越坏的土地推移的过程中,随着文明的进步和人口的增长,地租必然在最初耕种的最肥沃的土地上,而后逐步地在越来越坏的土地上产生,然而始终代表仅仅是土地——特殊的投资领域——的最坏土地任何时候都不提供地租。所有这些论点在逻辑上多少是互相联系的。

相反,如果知道,平均价格和价值并不是等同的,商品的平均价格可能等于商品价值,也可能大于或小于商品价值,那末问题就不存在了,问题本身不存在了,为解决问题而提出的假设也就不存在了。剩下的只是这样一个问题:为什么农业中商品的价值(或者,无论如何,它的价格)不是超过商品的价值,而是超过商品的平均价格呢?但是这个问题已经完全不涉及理论的基础即价值规定本身了。

李嘉图当然知道,商品的“相对价值”随着加入商品生产的固定资本和花费在工资上的资本之间的比例不同而发生变化。{但是这两种资本并不构成对立面;彼此对立的是固定资本和流动资本,流动资本不仅包括工资,而且包括原料和辅助材料。例如,在采矿工业和渔业中花费在工资上的资本和固定资本之比,可能和裁缝业中花费在工资上的资本和花费在原料上的资本之比相同。}但李嘉图同时知道,这些相对价值由于竞争而平均化。而且,他承认上述变化只是为了在这些不同的投资中得出同一的平均利润。这就是说,他所说的相对价值无非就是平均价格。他甚至没有想到过价值和平均价格是不同的。他只以它们的等同作出发点。但是,因为在资本各有机组成部分之间的比例不同的情况下不存在这种等同,他就把这种等同假定为还没有得到解释的、由竞争引起的事实。因此他也没有想到这样一个问题:为什么农产品的价值不平均化为平均[506]价格?相反,他假定农产品的价值会平均化,而且他正是从这个观点出发提出问题的。

真不懂威廉·修昔的底斯\fnote{指罗雪尔。——编者注}之流的好汉们为什么拥护李嘉图的地租理论。从他们的观点看来,李嘉图的“半真理”——照修昔的底斯的宽容的说法——丧失了它的全部价值。

在李嘉图看来,问题所以存在,只是因为价值是由劳动时间决定的。在这些好汉们看来,就不是这么回事。照罗雪尔的看法,自然本身就具有价值。关于这一点请参看后面。\endnote{马克思在《剩余价值理论》中以后没有再回过头来分析罗雪尔的这些观点。但是在《理论》的第三册《李嘉图学派的解体》一章中,马克思详细地批判了麦克库洛赫的类似的庸俗观点,这些观点的形成同罗雪尔的观点一样,受了让·巴·萨伊提出的“生产性服务”的辩护论见解的很大影响,马克思在下一段谈到这种辩护论见解。马克思在《资本论》第一卷第六章注22中提到过罗雪尔把自然看成价值源泉之一的观点(《马克思恩格斯全集》中文版第23卷第232页)。并见《资本论》第3卷第48章。——第142页。}这就是说,罗雪尔根本不知道什么是价值。既然如此,那还有什么能妨碍他让土地价值从一开始就加入生产费用并形成地租,把土地价值即地租作为解释地租的前提呢?

在这些好汉们那里,“生产费用”一词是毫无意义的。我们从萨伊那里看到这种情况。在他那里,商品的价值决定于生产费用——资本、土地、劳动,而这些费用又决定于供求。这就是说,根本不存在什么价值规定。既然土地提供“生产性服务”,那末这些“服务”的价格为什么不应该象劳动和资本提供的服务的价格那样决定于供求呢?既然“土地的服务”为一定的卖主占有,那末为什么他们的商品不应该有市场价格,从而,为什么地租不应该作为价格要素存在呢?

我们看到,威廉·修昔的底斯是没有丝毫理由如此热心“拥护”李嘉图的理论的。

\tsectionnonum{[(6)李嘉图关于谷物价格不断上涨的论点。1641—1859年谷物的年平均价格表]}

如果撇开绝对地租不谈,在李嘉图那里还有这样一个问题:

人口增加,对农产品的需求也增加,因而农产品的价格上涨,就象在类似条件下工业中发生的情况一样。但是,在工业中,一旦需求起了作用,引起了商品供给的增加,这种价格上涨就停止了。这时产品就降到它原来的价值,甚至降到原来的价值以下。但是,在农业中,这种追加产品不是按原来的价格,也不是按更低的价格投入市场。它的价值更大,从而引起市场价格不断上涨,同时也引起地租提高。如果这不是由于不得不耕种越来越不肥沃的土地,为生产同样的产品需要花费越来越多的劳动,农业生产率越来越降低,那又怎样解释呢?撇开货币贬值的影响不说,为什么在英国从1797年到1815年农产品价格随着人口的迅速增加而上涨呢?农产品价格后来又下降的事实不能证明什么。国外市场的供给被切断也不能证明什么。恰恰相反。这只是创造了使地租规律能以纯粹形式表现出来的适当条件。因为恰恰是同外国失去联系迫使国内去耕种越来越不肥沃的土地。农产品价格上涨不能用地租的绝对增加来解释,因为不仅地租总额增加了,而且地租率也提高了。一夸特小麦等等的价格提高了。也不能用货币贬值来解释,因为货币贬值只能解释为什么在工业生产率高度发展情况下工业品价格跌落,也就是说农产品价格相对上涨。货币贬值不能解释为什么农产品价格除了这种相对上涨之外,还不断地绝对上涨。

同样也不能认为这是利润率下降的后果。利润率下降决不能说明价格的变动,而只能说明价值或价格在地主、工业家和工人之间的分配的变动。

关于货币贬值,我们假定以前1镑等于现在2镑。一夸特小麦以前值2镑,现在值4镑。假定工业品跌到1/10,过去值20先令,现在值2先令。但是这2先令现在等于4先令。货币贬值——正如歉收一样——在这里当然会有影响。

[507]但是,撇开这一切不谈,可以假定,就当时的农业(小麦)情况来看,贫瘠土地投入了耕种。这种土地后来就变成肥沃的了,因为级差地租(就地租率看)下降了,正如小麦价格这个最好的晴雨表所表明的那样。

最高价格发生在1800和1801年以及1811和1812年。其中前两年是歉收年份,后两年是货币贬值最严重的年份。同样,1817和1818年也是货币贬值的年份。但是,如果把这些年份除掉,那末剩下的就应该看作是(请看后面)这一时期或那一时期的平均价格。

在比较不同时期小麦等等的价格时,必须同时把产品生产量和每夸特价格相对照,因为这样才看得出追加谷物的产量对价格的影响。

\todo{}

可见,从1650到1699年50年的平均价格是44先令2+(1/5)便士。

从1641到1649年(9年)期间,最高年平均价格是革命的一年即1645年的75先令6便士,其次是1649年的71先令1便士,1647年的65先令5便士,最低价格是1646年的42先令8便士。

\todo{}

从1700到1749年50年的年平均价格:35先令9+(29/50)便士。

\todo{}

从1750到1799年50年的年平均价格:45先信纸3+(13/50)便士。

\todo{}

从1800到1849年50年的年平均价格:69先令6+(9/50)便士。

从1800到1859年60年的年平均价格:66先令9+(14/15)便士。

因此,年平均价格是:

\todo{}

\centerbox{※     ※     ※}

连威斯特也说:

\begin{quote}{“在农业改良了的情况下,用在旧制度下最好土地上用的那样少的费用,就能够在二等或三等质量的土地上进行生产。”(爱德华·威斯特爵士《谷物价格和工资》1826年伦敦版第98页)}\end{quote}

\tsectionnonum{[(7)霍普金斯关于绝对地租和级差地租之间的区别的猜测;用土地私有权解释地租]}

霍普金斯正确地捉摸到了绝对地租和级差地租之间的区别:

\begin{quote}{“竞争原则使同一国家不可能有两种利润率;但是这一点决定相对地租,而不决定地租的总平均数。”(托·霍普金斯《论地租及其对生存资料和人口的影响》1828年伦敦版第30页)}\end{quote}

[508a]霍普金斯在生产劳动和非生产劳动——或者,照他的说法,首要劳动和次要劳动——之间作了下述区别:

\begin{quote}{“如果所有劳动者都被用来达到象钻石匠和歌剧演员被用来达到的同一目的,那末不要很久就将没有财富来养活这些人了,因为那时生产的财富中就没有什么可以再变成资本了。如果相当大一部分劳动者从事这类劳动,工资就会低,因为生产出来的东西只有比较小的一部分用作资本;但是,如果只有少数劳动者从事这类劳动,因而,几乎所有劳动者都是农夫、鞋匠、织工等等,那末,就会生产出许多资本,工资也会相应地高。”(同上,第84—85页)“所有为地主或食利者劳动并以工资形式得到他们的一部分收入的人,即所有实际上把自己的劳动限于生产供地主和食利者享乐的东西,并以自己的劳动换得地主的一部分地租或食利者的一部分收入的人,——所有这些人都必须列入钻石匠和歌剧演员一类。这些人都是生产劳动者,但是,他们的全部劳动都是为了把那种以地租和货币资本的收入形式存在的财富变为更能满足地主和食利者需要的其他某种形式,因此,这些劳动者是次要生产者。其他一切劳动者是首要生产者。”(同上,第85页)}\end{quote}

钻石和歌唱——这两者在这里被看作物化劳动——可以象一切商品一样转化为货币,并作为货币转化为资本。但是,在货币向资本的这种转化中要区别两种情况。一切商品都可以转化为货币,并作为货币转化为资本,因为在它们所采取的货币形式中,它们的使用价值和使用价值的特殊自然形式消失了。它们是具有社会形式的物化劳动,在这种社会形式中,这一劳动本身可以同任何实在劳动交换,因而可以转化为任何形式的实在劳动。相反,作为劳动产品的商品本身是否能够重新作为要素进入生产资本,这要看它们的使用价值的性质是否容许它们或者作为劳动的客观条件(生产工具和材料),或者作为劳动的主观条件(工人的生活资料),也就是作为不变资本或可变资本的要素重新进入生产过程。

\begin{quote}{“在爱尔兰,照适中的计算和1821年人口调查,交给地主、政府和什一税所得者的全部纯产品达2075万镑,而全部工资却只有14114000镑。”(霍普金斯,同上第94页)在意大利,“土地耕种者在耕作水平不高、固定资本极少的情况下,一般把产品的半数甚至半数以上作为地租交给地主。人口的较大部分是次要生产者和土地所有者,而首要生产者一般地说是一个穷困的、受侮辱的阶级”。(第101—102页)“法国在路易十四[以及他的后继者路易十五和路易十六]的时代有同样情况。照[阿瑟·]杨格的计算,地租、什一税和税收总计140905304镑。同时农业处于悲惨的状况。法国人口当时是26363074人。如果劳动人口就算有600万户——显然过高了,——那末每个劳动者家庭每年直接或间接必须交给地主、教会和政府的纯财富平均约23镑。根据杨格的资料,并考虑到其他一切因素,一个劳动者家庭每年所得的产品合42镑10先令,其中23镑交给别人,19镑10先令留下维持自己的生活。”(同上,第102—104页)}\end{quote}

人口对资本的依赖:

\begin{quote}{“应当看到,马尔萨斯先生和他的信徒的错误在于假定工人人口的减少不会引起资本的相应减少。”(同上,第118页)“马尔萨斯先生忘记了,[对工人的]需求受到用作工资的资金的限制,这些资金不是自然而然产生的,而总是事先由劳动创造的。”(同上,第122页)}\end{quote}

\todo{}

这是关于资本积累的正确见解。但是,[霍普金斯没有看到]在劳动量不以同一程度增加的情况下,劳动创造的资金是可以增加的,也就是说,剩余产品或剩余劳动的量是可以增加的。

\begin{quote}{“奇怪的是,有一种强烈的倾向把纯财富说成是对工人阶级有利的东西,因为据说它提供就业机会。其实,很明显,[509]即使它能够提供就业机会,也不是因为它是纯的,而是因为它是财富,是劳动创造的东西。而另一方面,又把追加的工人人口说成是对工人阶级有害的东西,虽然这些工人生产的东西比他们所消费的多两倍。”(同上,第126页)“如果由于使用较好的机器可以把全部首要产品从200增加到250或300,而纯财富和利润仍旧只有140,那末,显然,留作首要生产者的工资基金的就不是60,而是110或160。”(同上,第128页)“工人状况的恶化,或者是由于他们的生产力被摧残,或者是由于他们生产的东西被掠夺。”(第129页)“不,马尔萨斯先生说,‘你们的贫困决不能用你们的负担沉重来解释;你们的贫困完全是由有这种负担的人太多引起的’。”(同上,第134页)“原始材料[土地、水、矿物]是不包括在生产费用调节一切商品的交换价值这个一般原理之内的;但是这些原始材料的所有者要求[由于这些原始材料被利用而得到]产品的权利,使地租加入价值。”(托·霍普金斯《关于调节地租、利润、工资和货币价值的规律的经济研究》1822年伦敦版第11页)“地租,或者说,[土地]使用费,自然是从[土地]所有权产生的,或者说,是从财产权的确立产生的。”(同上,第13页)“凡具有以下特性的东西都能够提供地租:第一,它必须在某种程度上是稀少的;第二,它必须在生产这种大事业中有协助劳动的能力。”(同上,第14页)“当然,不应该假定有这样一种情况,即土地同投到土地上的劳动和资本相比{土地的多或少自然是相对的,是同可以使用的劳动和资本的量有关系的},是如此之多,以致不可能提供地租形式的费用,因为土地并不稀少。”(第21页)“在某些国家,土地所有者能榨取50%,在另一些国家连10%也榨取不到。在东方富饶地区,一个人有了他投入土地的劳动的产品的1/3就可以过活;但是在瑞士和挪威的某些地区,征收10%就可以使土地荒芜……除了交租人的有限的支付能力之外,”(第31页)以及“在有较坏土地存在的地方,除了较坏土地同较好土地的竞争之外,”(第33—34页)“我们看不见可以榨取的地租的任何其他自然界限”。(第31页)“在英国有许多荒地,它的自然肥力同相当大一部分现在的已耕地在未耕种前一样;可是,把这种荒地变成耕地所需的费用太大,以致它不能为花费在改良土地上的货币支付普通利息,更不能留下什么为土地的自然肥力支付地租,尽管这里有一切便利条件,可以在巧妙使用的资本帮助下,在便宜的工业品配合下,立即使用劳动,并且附近已经开辟了很好的道路等等。现在的土地所有者可以被看作几百年来为使土地达到现有生产率状态而耗费的全部积累劳动的所有者。”(同上,第35页)}\end{quote}

这一情况对于地租有非常重要的意义,特别是在人口突然大大增加(就象1780—1815年由于工业发展而发生的情况),因而大量过去没有耕种过的土地突然投入耕种的时候。新耕地的肥力可能等于,甚至高于老地尚未经过几百年耕种以前的肥力。但是,要使新耕地不是按较贵的价格出卖产品,就要求新地的肥力等于:第一,已耕[510]地的自然肥力,加上第二,已耕地由于耕种而形成的人工的、现已变成自然的肥力。因此,新耕地一定要比老地被耕种之前肥沃得多。

但是有人会说:

已耕地的肥力,首先取决于它的自然肥力。因而,新耕地是否具备这种由自然产生并受到自然因素制约的肥力,就取决于新耕地的自然性质。在两种情况下,自然肥力都是什么也不花费的。已耕地的肥力的另一部分则是依靠耕种、依靠投资的人工产物。但是,这部分生产率是花费了生产费用的;这笔生产费用是以投入土地的固定资本的利息的形式支付的。这部分地租只是同土地结合在一起的固定资本的利息。因此它加入早先的已耕地的产品的生产费用。由此可见,只要把同量资本投到新耕地上,它也将具有肥力的这个第二部分;如同早先的已耕地的情况一样,为创造这种肥力而使用的资本的利息将加入产品价格。在这种情况下,为什么新地——如果不是肥沃得多的土地——没有产品价格的提高就不能投入耕种呢?如果自然肥力是相同的,那末差别就只是由投资造成的,而在两种情况下,这笔资本的利息都是以同等程度加入生产费用的。

可是,这种推理是错误的。一部分开垦等费用不用再支付了,因为由此造成的肥力,正如李嘉图已经指出,部分地同土地的自然性质长在一起了(例如挖树根、改良土壤、排水、平整、通过多次反复的化学处理改变土壤化学成分等费用)。因此,新耕地为了能够与最后的已耕地按照同一价格出卖产品,它的肥力就必须足以使这个价格为它补偿那部分开垦费用,这部分费用加入它自己的生产费用,但已不再加入早先的已耕地的生产费用,而且在这里已经同土地的自然肥力长在一起了。

\begin{quote}{“位置有利的瀑布,给我们提供了一个为那种变成私有的、可以设想为最带独特性的自然赐予而支付地租的例子。这种情况在工业区域是大家都了解的,在那里,对小瀑布,特别是对落差大的瀑布支付相当高的地租。从这样的瀑布取得的力,等于大蒸汽机所供应的力;因此,利用这些瀑布即使要支付巨额地租,也同花费大笔资金去建设和运转蒸汽机一样合算。瀑布有大有小。离工业企业所在地近也是一个取得较高地租的有利条件。在约克郡和郎卡斯特郡,最小瀑布和最大瀑布的地租之间的差额,大概比用作普通耕地的50英亩最贫瘠土地和50英亩最肥沃土地的地租之间的差额还要大得多。”(霍普金斯,同上第37—38页)}\end{quote}

\tsectionnonum{[(8)开垦费用。谷物价格上涨时期和谷物价格下降时期(1641—1859年)]}

如果我们把前面引用的谷物年平均价格\fnote{见本册第144—146页。——编者注}拿来比较,并且,第一,把受货币贬值影响的情况(1809—1813年)除外,第二,把特别歉收的年份如1800和1801年造成的情况除外,我们就会看到,一定时候或一定时期耕种多少新地具有多么重要的意义。这里,已耕地上的价格上涨表明人口增长和由此引起的谷物价格超过[价值];另一方面,这种需求增长本身使新地投入耕种。如果新耕地数量相对地说增加很多,那末,价格上涨,价格比前一时期高,只不过证明有相当大一部分开垦费用加入了追加食物的价格。如果谷物价格不涨,那就是生产没有增加。这种增加的后果即价格下降要到以后才能表现出来,因为新近投资生产的食物的价格包含着生产费用或价格的这样一个要素,这个要素在早先的土地投资中,或者说在早先的那部分耕地上早已消失了。如果耕种新地的费用不是因劳动生产率提高而比过去时期已经大大降低,这个差额甚至还要大。

[511]不论新地的肥力高于、等于或低于老地,为了把新地改造成适于使用——象在已耕地上平均使用资本和劳动那样的条件下使用——资本和劳动的状态(这种状态由现有已耕地上通行的一般开垦标准来决定),必须支付变未耕地为耕地的费用。生产费用的这个差额必须由新耕地补偿。如果这个差额不加入新耕地的产品价格,那末,这种结果只有在两种情况下可能出现。或者,新耕地的产品不按它的实际价值出卖。它的价格低于它的价值,大部分不提供地租的土地实际上就是这种情况,因为这种土地的产品价格不是由自己的价值,而是由较肥沃土地的产品的价值决定的。或者,新耕地必须相当肥沃,以致它的产品如果按自己的内在的价值,按物化在产品中的劳动量出卖,其价格就会低于早先的已耕地产品的价格。

如果由已耕地的产品价值调节的市场价格和新耕地产品的内在价值之间的差额比方说是5%,另一方面,如果加入新耕地产品的生产费用的、为使新地提高到老地通常具有的生产能力水平而花费的资本的利息也等于5%,那末新耕地的产品按原来的市场价格将能支付通常的工资、利润和地租。如果支付所花费的资本的利息只需4%,而新地的肥力超过老地的肥力4%以上,那末市场价格在扣除了为使新地变成“适于耕种的”状态而花费的资本的利息4%之后,将有一个余额,或者说,产品可以卖得低于由最贫瘠土地的产品价值调节的市场价格。结果,一切地租将同产品的市场价格一起降低。

绝对地租是原产品价值超过平均价格的余额。级差地租是比较肥沃的土地上生产的产品的市场价格超过这种比较肥沃的土地自己产品的价值的余额。

因此,如果在一段时间内,人口增加所需要的追加食物有比较大的一部分是初耕地生产的,同时原产品的价格上涨或者不变,这还不能证明国内土地的肥力已经下降。这只是证明,土地的肥力还没有提高到足以补偿生产费用的新要素的程度,这个新要素就是为使未耕地提高到老地在当时发展阶段所具备的通常生产条件的水平而花费的资本的利息。

由此可见,如果在不同时期新耕地的相对量不同,即使价格不变或上涨,也不能证明新地贫瘠或提供的产品较少,而只是证明,有一个费用要素加入它的产品价值,这个费用要素在早先的已耕地上已经消失;只是证明,这个新的费用要素仍然在起作用,尽管在新的生产条件下,开垦费用与过去为了把老地从肥力的原始的自然状态改变成现在的状态所必需的费用比较起来,已经大大降低。因此,要[512]查明不同时期圈围[公有地和开垦公有地]的相对比例。\endnote{关于英国“圈围”公有地,马克思在《资本论》第一卷第二十四章中谈得比较详细(见《马克思恩格斯全集》中文版第23卷第793—797页)。——第156页。}

此外,从前面所引的表(第507—508页)可以看出:

如果把每十年作一个时期来考察,那末,1641—1649年时期比1860年前的任何一个十年时期都高,只有1800—1809年和1810—1819年除外。

如果把每五十年作一个时期来考察,那末,1650—1699年时期高于1700—1749年,而1750—1799年时期高于1700—1749年,但低于1800—1849年(或1800—1859年)。

在1810—1859年期间,发生了有规律的价格下降,而在1750—1799年期间,虽然这五十年平均价格较低,却是上升的运动,这是有规律地上涨,就象1810—1859年期间有规律地下降一样。

事实上,同1641—1649年时期比较起来,总的说来,十年的平均价格不断下降,这种下降直到十八世纪上半叶最后两个十年达到它的极限(最低点)为止。

从十八世纪中叶开始上涨,这次上涨以1750—1759年价格[36先令4+(5/10)便士]为出发点,这个价格低于十七世纪下半叶的平均价格,几乎相当(略高)于1700—1749年(十八世纪上半叶)的平均价格35先令9+(29/50)便士。这个上升运动在1800—1809年和1810—1819年两个十年时期一直在继续。在后一个十年达到它的最高点。从这时又开始了有规律的下降运动。如果我们把1750—1819年这个上涨时期平均一下,它的平均价格(每夸特57先令多)[几乎]等于1820年开始的下降时期的出发点(即1820—1829年这个十年时期的58先令多);正如十八世纪下半叶的出发点[几乎]等于十八世纪上半叶的平均价格一样。

歉收、货币贬值等个别情况对平均数字能够发生多么大的影响,可以任意举一个算术例子来说明。例如,30+20+5+5+5=65。尽管这里最后三个数都只是5,平均数却等于13。相反,12+11+10+9+8[=50],平均数等于10,尽管把第一式中例外的数30和20划去时第二式中任何三年的平均数都要大些。

如果把付给资本——陆续用于开垦、在一定时期作为特殊项目加入生产费用的资本——的差额费用除去,那末,1820—1859年的价格或许比过去所有的价格都低。应该认为,那些用投入土地的固定资本的利息来解释地租的糊涂人,也多少看到了这种情况。

\tsectionnonum{[(9)安德森反对马尔萨斯。安德森对地租的理解。安德森关于农业生产率提高和它对级差地租的影响的论点]}

安德森在《关于导致不列颠目前粮荒的情况的冷静考察》(1801年伦敦版)中说:

\begin{quote}{“从1700到1750年,小麦的价格不断下降,从每夸特2镑18先令1便士降到1镑12先令6便士;从1750到1800年,小麦的价格不断上涨,从每夸特1镑12先令6便士涨到5镑10先令。”(第11页)}\end{quote}

可见,安德森不象威斯特、马尔萨斯和李嘉图那样,只看到谷物价格不断上涨(从1750到1813年)一个方面的现象,相反,他看到两方面的现象:整个世纪,上半世纪谷物价格不断下降,下半世纪谷物价格不断上涨。同时安德森明确指出:

\begin{quote}{“人口不论在十八世纪上半叶和下半叶同样都在增长。”(同上,第12页)}\end{quote}

安德森是人口论\endnote{指马尔萨斯的人口论。——第158页。}的死敌,他非常明确地强调指出,土地有不断增长的持久的改良能力:

\begin{quote}{“通过化学作用和耕种,土地可以越来越得到改良。”(同上,第38页)\endnote{这里引用的是意大利人卡米洛·塔雷洛·德·列奥纳托(十六世纪)的话,安德森在这个问题上完全同意他的意见。——第158页。}[513]“在合理的经营制度下,土地的生产率可以无限期地逐年提高,最后一直达到我们现在还难于设想的程度。”(第35—36页)“可以有把握地说,现在的人口同这个岛能够供养的人口比较起来是很少的,远没有达到引起严重忧虑的程度。”(第37页)“凡人口增加的地方,国家的生产也必定一起增加,除非人们允许某种精神的影响破坏自然的经济。”(第41页)}\end{quote}

“人口论”是“最危险的偏见”。(第54页)安德森力求用历史的例子证明,“农业生产率”随着人口的增长而提高,随着人口的减少而下降。(第55、56、60、61页及以下各页)

如果对地租有正确的理解,自然首先会认识到,地租不是来自土地,而是来自农产品,也就是来自劳动,来自劳动产品(例如小麦)的价格,即来自农产品的价值,来自投入土地的劳动,而不是来自土地本身。关于这一点,安德森正确地着重指出:

\begin{quote}{“不是地租决定土地产品的价格,而是土地产品的价格决定地租,虽然土地产品的价格在地租最低的国家往往最高。”}\end{quote}

{因此,地租同农业的绝对生产率毫无关系。}

\begin{quote}{“这似乎是一个奇论,需要解释一下。每一个国家有各种土地,它们的肥力彼此大不相同。我们把这些土地分成不同的等级,用A、B、C、D、E、F等字母表示。等级A包括肥力最大的土地,以下字母表示不同等级的土地,它们的肥力依次递减。既然耕种最贫瘠的土地的费用同耕种最肥沃的土地的费用一样大或者甚至更大,那末,由此必然得出一个结论,如果等量谷物,不论它来自哪一个等级的土地,可以按照同一价格出卖,耕种最肥沃的土地的利润一定比耕种其他土地的利润大得多}\end{quote}

{指[产品]价格超过费用,或者说,超过预付资本价格的余额},

\begin{quote}{而且,由于肥力越低这种利润越少,最后必然达到这种情况,就是在某些等级低的土地上,耕种费用同全部产品价值相等。”(第45—48页)}\end{quote}

最后的土地不支付任何地租。(引文来自麦克库洛赫的《政治经济学文献》1845年伦敦版。麦克库洛赫又引自《谷物法本质的研究》或《关于农业、自然史、技艺及其他各种问题的通俗讲座》1799—1802年伦敦版。这要到英国博物馆去核对。\endnote{马克思所引的安德森的这段话,是麦克库洛赫从安德森的《谷物法本质的研究》(1777年爱丁堡版)一书中引用的一大段中的一部分,这整个一大段话在该书第45—48页。——第159页。})

安德森在这里叫做“全部产品价值”的东西,在他的观念中显然就是市场价格,不论较好或较坏土地出产的产品都要按照它出卖。这个“价格”(价值)使比较肥沃的各个等级的土地有一个或大或小的超过费用的余额。最后的产品没有这种余额。对于这种产品,平均价格,即由生产费用加平均利润决定的价格,同产品的市场价格一致,因此这里没有任何超额利润,按照安德森的见解,只有超额利润能够形成地租。在安德森看来,地租等于产品的市场价格超过产品的平均价格的余额。(价值理论还完全没有引起安德森不安。)因此,如果由于土地特别贫瘠,这种土地的产品的平均价格同产品的市场价格一致,那末,这个余额就没有了,就是说,形成地租的基金就不存在了。安德森不说最后的耕地不可能提供任何地租。他只说,当费用(生产费用加平均利润)大到使产品市场价格和产品平均价格之间的差额消失时,地租也就消失,并且说,如果耕种的土地的等级越来越低,这种情况必然发生。安德森明确地说,在不同程度的有利生产条件下生产出来的等量产品具有一定的、同一的市场价格,是形成地租的前提。他说,“如果等量谷物,不论它来自哪一个等级的土地,可以按照同一价格出卖”,因此,如果假定有一个一般市场价格的话,那末,在较好等级土地上必然有超过较坏等级土地的超额利润,或者说,利润余额。

[514]从前面引的一段话可以看出,安德森决不认为不同的肥沃程度仅仅是自然的产物。相反,他认为:

\begin{quote}{“土地的无限多样性”部分地是因为这些“土地由于它们所经历的耕作方式不同,由于肥料等等,可以从它们的原始状态改变成完全不同的状态”。(《关于至今阻碍欧洲农业进步的原因的研究》1779年爱丁堡版第5页)}\end{quote}

一方面,社会劳动生产率的发展使开垦新地比较容易;但是,另一方面,耕种又使土地之间的差别扩大,因为已耕的A地和未耕的B地的原有肥力完全可能是一样的,如果我们从A地的肥力中扣除对这块土地来说现在固然是自然的、但从前是人工赋予的那一部分的话。因此,耕种本身使已耕地和未耕地的自然肥力之间的差别扩大。

安德森明确地说,一块土地,如果它的产品的平均价格同市场价格一致,就不能支付任何地租:

\begin{quote}{“有两块土地,它们的产量同上面说过的例子大致相符,就是说,一块收12蒲式耳,正够补偿费用,另一块收20蒲式耳;如果这两块土地都不需要立刻支出任何改良土壤的费用,那末租地农场主可以为后一块土地,比方说,支付甚至6蒲式耳以上的地租,而不能为前一块土地支付任何地租。如果12蒲式耳刚够补偿耕种费用,那末仅仅生产12蒲式耳的已耕地就不能提供任何地租。”(《论农业和农村事务》1775—1796年爱丁堡和伦敦版第3卷第107—109页)}\end{quote}

紧接着他又说:

\begin{quote}{“可是,如果租地农场主直接靠他花费的资本和他的努力得到较大量的产品,那就不能指望他能够把产品中几乎同样大小的份额当作地租来支付;但是,如果土地肥力在一定时间内稳定在同样高的水平上,尽管这块土地本来是靠他自己的努力才提高了肥力,他将乐于支付上述数量的地租。”(同上,第109—110页)}\end{quote}

这样,举例来说,最好耕地的产品每英亩为20蒲式耳;依照假定,其中12蒲式耳支付费用(预付资本加平均利润)。在这种情况下,可以有8蒲式耳支付地租。假定,一蒲式耳值5先令,那末,8蒲式耳或1夸特值40先令或2镑,20蒲式耳[2+(1/2)夸特]值5镑。这5镑中,扣掉作为费用的12蒲式耳或60先令即3镑。那末就剩下2镑或8蒲式耳支付地租。在3镑费用中,如果利润率等于10%,那末支出等于54+(6/11)先令,利润等于5+(5/11)先令{[54+(6/11)]∶[5+(5/11)]=100∶10}。现在假定,租地农场主必须在一块肥力同生产20蒲式耳的土地的原有肥力一样的未耕地上进行各种改良,以便使它达到相当于农业耕作一般水平的状态。这使他除了54+(6/11)先令的支出外,或者说,我们把利润也包括在费用内,除了60先令外,还要支出36+(4/11)先令;这笔支出的10%等于3+(7/11)先令;如果租地农场主总是按照每蒲式耳5先令的价格出卖20蒲式耳,那就只有经过10年,只有在他的资本再生产出来之后,他才能够支付地租。从那个时候起,人工的土地肥力就被算作原有肥力,它的利益将落到地主手里。

虽然新耕地的肥力和最好的已耕地的原有肥力相同,可是,对新耕地的产品来说,市场价格和平均价格现在是一致的,因为在平均价格中加入了一项费用,这项费用,在人工的和自然的肥力在一定程度上溶合在一起的最好土地上,已经消失了。而在新耕地上,人工的、由投资造成的那部分肥力,还与土地的自然肥力完全不同。因此,新耕地虽然具有与最好的已耕地同样的原有肥力,却不能支付任何地租。可是十年以后,它不仅能够一般地支付地租,而且能够与早已耕种的最好的土地支付同样多的地租。

可见,安德森在这里看到了两种现象:

(1)地主得到的级差地租,一部分是租地农场主人工地赋予土地肥力的结果;

(2)这种人工肥力经过一定期间开始表现为土地本身的原有生产率,因为土地本身已被改造了,而实现这种改造的过程却消失了,看不出来了。

[515]如果我今天建立一个棉纺厂,花费10万镑,那末,我的棉纺厂的生产率,比十年前我的前辈所建立的棉纺厂要高。对于今天的机器制造业、一般建筑业等等的生产率和十年前的生产率之间的差别,我是不付代价的。相反,这种差别却使我能支付较少代价而得到具有同样生产率的工厂,或者仅仅支付同一代价就得到生产率较高的工厂。农业中情况不是这样。土地的原有肥力之间的差别由于增加一部分所谓土地自然肥力而扩大了,这部分肥力,事实上是以前由人们创造的,现在却同土地本身有机地溶合在一起,同土地的原有肥力已不能再区别开。由于社会劳动生产力的发展,为了使具有同样自然肥力的未耕地达到这种已增大的肥力水平所必需的费用,已不象把已耕地的原有肥力提到它现在看来是原有的肥力所需要的那样多了;但是为了达到这同一水平,现在还是需要或多或少的费用。因此,新产品的平均价格高于老产品的平均价格,而市场价格和平均价格之间的差额会缩小,甚至可能完全消失。但是,假定在上述情况下新耕地很肥沃,在花上40先令追加费用(包括利润)之后,它不是提供20蒲式耳,而是提供28蒲式耳。在这种情况下,租地农场主就可以支付8蒲式耳或2镑的地租。为什么呢?因为新耕地比老地多提供8蒲式耳,所以,尽管平均价格较高,新耕地在同一市场价格下仍然提供同老地一样多的价格余额。如果新耕地不需要任何追加费用的话,新耕地的肥力就会两倍于老地。\endnote{马克思这里所说的“土地肥力”,是指从这块土地上得到的地租总额。——第163页。}就因为有了这种费用,才可以说新耕地的肥力同老地一样高。

\tsectionnonum{[(10)洛贝尔图斯对李嘉图地租理论的批判不能成立。洛贝尔图斯不懂资本主义农业的特点]}

现在最后回过头来谈也是最后一次谈洛贝尔图斯。

\begin{quote}{“它〈洛贝尔图斯的地租理论〉从劳动产品的分配出发来说明……工资、地租等一切现象,而这种分配,只要具备两个先决条件即足够的劳动生产率以及土地和资本的所有权,就必然要出现。它说明,只有足够的劳动生产率才造成这种分配的经济上的可能性,因为这种生产率赋予产品价值这么多实际内容,以致其他不劳动的人也可以靠它生活;它也说明,只有土地和资本的所有权才造成这种分配的法律上的现实性,强迫工人把自己的产品同不劳动的土地所有者和资本所有者按照这样一种比例分配,以致他们工人从中分得的刚够活命。”(洛贝尔图斯《给冯·基尔希曼的社会问题书简。第三封信:驳李嘉图的地租学说,对新的地租理论的论证》1851年柏林版第156—157页)}\end{quote}

亚·斯密对问题的解释是双重的。[第一个解释:]劳动产品的分配,这里把劳动产品看成既定的,并且实际上说的是使用价值的份额。洛贝尔图斯先生也是这个看法。在李嘉图著作中也可遇到这个看法,而且李嘉图更应该因此受到责备,因为他对价值决定于劳动时间这个规定不限于泛泛议论,而是认真对待。这个看法,加以相应的修改之后,或多或少适用于使劳动者和劳动客观条件所有者成为不同阶级的一切生产方式。

相反,斯密的第二个解释表现了资本主义生产方式的特征。因此,只有它才是理论上有成果的公式。那就是,斯密在这里认为利润和地租来源于剩余劳动,来源于工人除了用来仅仅再生产他自己工资的那部分劳动以外加到劳动对象上的剩余劳动。在生产完全以交换价值为基础的地方,这是唯一正确的观点。这个观点奠定了理解发展过程的基础,而在第一个解释里,劳动时间被假定为不变的。

李嘉图所以有片面性,是因为他总想证明不同的经济范畴或关系同价值理论并不矛盾,而不是相反地从这个基础出发,去阐明这些范畴以及它们的表面上的矛盾,换句话说,去揭示这个基础本身的发展。

\begin{quote}{[516]“您\fnote{指冯·基尔希曼。——编者注}知道,所有经济学家从亚·斯密那时候起已经把产品的价值分解为工资、地租和资本盈利,因而,把各阶级的收入,特别是地租部分,建立在产品的分配上这种观念,不是新的〈的确不是!〉。可是经济学家们立刻走入歧途。所有的人,连李嘉图学派也不例外,首先犯了这样一个错误,他们不是把全部产品、完成的财富、全部国民产品看作工人、土地所有者和资本家参与分配的一个整体,而是把原产品的分配看作一种有三者参与的特殊分配,把工业品的分配看作又一种只有两者参与的特殊分配。这样,这些体系已经把原产品本身和工业品本身分别当作一种构成收入的特殊财富看待了。”(第162页)}\end{quote}

首先,亚·斯密把“产品的全部价值分解为工资、地租和资本盈利”,从而忘掉了也构成价值一部分的不变资本;这样,他的确把后来的所有经济学家,包括李嘉图,也包括洛贝尔图斯先生在内,都引入了“歧途”。没有[劳动总产品和新加劳动产品之间的]这种区分,要对问题做出任何科学的解释简直是不可能的,这一点在我对这个问题的分析中已经证明了\fnote{见本卷第1册第78—140页。——编者注}。重农学派在这方面更接近于真理。他们的“原预付和年预付”是作为年产品价值或年产品本身的一部分区分出来的,这个部分无论对国家或个人都不再分解为工资、利润或地租了。重农学派认为,农业主用原料补偿不生产阶级的预付(把这种原料变成机器的事落到“不生产”阶级身上),另一方面,农业主用自己的产品补偿自己的一部分预付(种子、种畜、役畜、肥料等),而另一部分预付(机器等)则通过用原料同“不生产”阶级交换得到补偿。

第二,洛贝尔图斯先生的错误是把价值的分配和产品的分配等同起来。“构成收入的财富”同这种产品价值的分配没有任何直接关系。比方说,棉纱生产者得到的并表现为一定金量的价值部分在各种产品——农产品或工业品——中得到实现,对于这一点,经济学家们同洛贝尔图斯一样,知道得很清楚。这一点是事先假定的,因为这些生产者是生产商品,而不是生产供自己直接消费的产品。既然供分配的价值,即一般说来归结为收入的那个价值组成部分,是在各个生产领域内部,在不依赖其他领域的情况下(虽然由于分工每个生产领域都以其他领域为前提)创造出来的,所以,洛贝尔图斯不去考察这个价值创造的纯粹形式,而一开头就把事情搞乱,提出这些价值组成部分能保证自己的所有者取得一国现有总产品的什么份额的问题,那他就倒退了一步并且造成了混乱。在他那里,产品价值的分配立即变成使用价值的分配。既然他把这种混乱转嫁给其他经济学家,所以他提出的矫正方法,即把工业品和原产品放在一起考察的方法,就成为必要了,而这种考察方法同价值的创造无关,因此,如果用它来说明价值的创造,是错误的。

工业品的价值,只要它归结为收入,只要工厂主不付地租(不论为建筑物的地皮或者为瀑布等),就只有资本家和雇佣工人参加分配。农产品的价值在大多数情况下有三方面参加分配。这是洛贝尔图斯先生也承认的。他对这种现象所作的解释丝毫不能改变事实本身。但是,如果其他经济学家,特别是李嘉图,从资本家和雇佣工人两者分配出发,到后来才把地租所得者作为一种特殊赘疣引进来,那末,这是完全符合资本主义生产的实质的。物化劳动和活劳动,这是两个[517]因素,资本主义生产正是建立在这两个因素的对立之上。资本家和雇佣工人是生产职能的唯一承担者和当事人,他们之间的相互关系和对立是从资本主义生产方式的本质产生的。

资本家不得不把他所侵占的一部分剩余劳动或剩余价值再同不劳动的第三者分配的情况,只是后来才出现。扣除作为工资支付出去的产品价值部分和等于不变资本的价值部分之后,全部剩余价值直接从工人手里转到资本家手里,这也是生产的事实。对于工人来说,资本家是全部剩余价值的直接占有者,不管他后来怎样同借贷资本家、土地所有者等分配剩余价值。因此,正如詹姆斯·穆勒指出的那样\endnote{詹·穆勒《政治经济学原理》1821年伦敦版第198页。——第166页。},如果地租所得者消失,由国家来代替他的地位,生产可以继续进行而不受影响。他——土地私有者——决不是资本主义生产方式所必要的生产当事人,虽然对于资本主义生产方式来说,必须使土地所有权属于什么人,只要不是属于工人,而是例如属于国家。根据资本主义生产方式——不同于封建、古代等生产方式——的本质,把直接参与生产,因而也是直接参与分配所生产的价值以及这个价值所借以实现的产品的阶级,归结为资本家和雇佣工人,而把土地所有者排除在外(由于那种不是从资本主义生产方式生长出来,而是被这种生产方式继承下来的对自然力的所有权关系,土地所有者只是事后才参加进来),这丝毫不是李嘉图等人的错误,它倒是资本主义生产方式的恰当的理论表现,表现了这种生产方式的特点。洛贝尔图斯先生还是一个十足的老普鲁士式的“地主”,理解不了这一点。只有当资本家控制了农业,并且到处象英国大多数地方那样,成为农业的领导者(完全同成为工业的领导者一样),排斥土地所有者以任何形式直接参加生产过程的时候,上述情况才变得可以理解和引人注意。因此,洛贝尔图斯先生在这里认为是“歧途”的,恰好是他所不理解的正道;整个问题在于,洛贝尔图斯还陷在资本主义前的生产方式所产生的种种观点之中。

\begin{quote}{“他〈李嘉图〉也不是让成品在有关参加者之间分配,而是象其他经济学家一样,把农产品和工业品分别当作一种特殊的有待分配的产品。”(同上,第167页)}\end{quote}

洛贝尔图斯先生!李嘉图在这里考察的不是产品,而是产品的价值,这是完全正确的。您的“成”品及其分配同这种价值分配毫无共同之处。

\begin{quote}{“在他〈李嘉图〉看来,资本所有权是既定的,并且还早于土地所有权……因此,他不是从产品分配的根据,而是从产品分配的事实开始,而他的全部理论限于研究那些决定和改变产品分配关系的原因……产品只分为工资和资本盈利,在他看来,是最初的分配,而且是最初唯一的分配。”(第167页)}\end{quote}

这个您又不懂了,洛贝尔图斯先生!从资本主义生产的观点看来,资本所有权的确是作为“最初的”所有权出现的,因为它是一种作为资本主义生产的基础,并在这种生产制度中表现为生产的当事人和生产职能的承担者(对土地所有权就不能这样说)的所有权。土地所有权在这里表现为派生的东西,因为,现代土地所有权,实际上是封建的,但是由于资本对它的作用,发生了形态变化,因而它作为现代土地所有权所特有的形式是派生的,是资本主义生产方式的结果。李嘉图把现代社会中存在和表现出来的这个事实也看成历史上最初的东西(而您呢,洛贝尔图斯先生,不是去研究现代形式,而是摆脱不了地主的回忆),这是一种误解,资产阶级经济学家们在考察资产阶级社会的一切经济规律时都陷入这种误解,在他们看来,这些规律是“自然规律”,因而也表现为历史上最初的东西。

[518]但是,李嘉图在谈到不是产品的价值而是产品本身的地方,是指全部“成”品的分配,洛贝尔图斯先生从李嘉图序言的头一句话就可以看到这一点:

\begin{quote}{“土地产品——通过劳动、机器和资本联合运用而从地面上得到的一切产品——在社会的三个阶级之间,也就是在土地所有者、耕种土地所必需的基金或资本的所有者和以自己的劳动耕种土地的工人之间进行分配。”(《政治经济学和赋税原理》,序言,1821年伦敦第3版)}\end{quote}

李嘉图紧接着说:

\begin{quote}{“但在不同的社会发展阶段,这些阶级中的每一个阶级在地租、利润和工资的名义下分到的全部土地产品的份额是极不相同的”。(同上)}\end{quote}

这里说的是“全部产品”的分配,而不是工业品或原产品的分配。如果“全部产品”是既定的,“全部产品”的这些份额就完全决定于每一生产领域内部每个参与分配者在自己产品的“价值”中拥有的份额。这个“价值”可以转化并表现为“全部产品”的一定的相应份额。李嘉图在这里的错误,只是他步亚·斯密的后尘,忘记了不是“全部产品”分解为地租、利润和工资,因为“全部产品”中有一部分作为资本“分给”这三个阶级中的一个或几个阶级。

\begin{quote}{“可能您想断言,最初资本盈利均等规律必定压低原产品价格,直到地租消失,后来由于价格上涨,地租又从比较肥沃和比较不肥沃的土地的收成的差额中产生出来,同样,现在,除普通的资本盈利之外还取得地租这样一种好处,也必定推动资本家把资本用于开垦新地和改良老地,直到由此引起的市场商品充斥使价格又降低,以致在最不利的投资情况下地租消失。换句话说,这等于断言:就原产品说,资本盈利均等这条规律,把另一条规律,即产品价值决定于所耗费的劳动的规律取消了,可是,李嘉图在他的著作的第一章中恰恰是用前一条规律来证明后一条规律的。”(洛贝尔图斯,同上,第174页)}\end{quote}

当然,洛贝尔图斯先生!“资本盈利均等”规律并不取消产品“价值”决定于“所耗费的劳动”的规律;但是,它的确取消李嘉图关于产品的平均价格等于产品“价值”这个前提。然而问题也不是“原产品”的价值降低到平均价格。正好相反,原产品的特点是,由于土地所有权的存在,它有这样一个特权,即它的价值不降低到平均价格。如果它的价值真的降到商品的平均价格的水平(这是可能的,尽管其中存在着您所说的“材料价值”),地租也就消失了。有一些等级的土地,现在也许不提供任何地租,它们之所以如此,是因为原产品的市场价格等于它们的产品本身的平均价格,使它们因比较肥沃土地的竞争而失去按自己产品的“价值”出卖产品的特权。

\begin{quote}{“难道在人们还根本没有从事农业之前,就已经有获得盈利,并且按照盈利均等规律投放资本的资本家存在了吗?〈多么愚蠢!〉……我认为,如果现在从文明国家[519]派遣一个开发队到一个新的未开垦的国家去,较富的成员带着已经发达的农业的储备和工具——即资本,较贫的成员也跟着一块去,希望通过为较富的成员服务得到高的工资,在这种情况下,资本家将把支付工人工资之后剩下的余额看成自己的盈利,因为他们从宗主国带来了早已存在的事物和概念。”(第174—175页)}\end{quote}

这下子您说对了,洛贝尔图斯先生!李嘉图的全部观点只有在资本主义生产方式占支配地位的前提下才有意义。至于他用什么形式表达这个前提,他在这上面是否采用了hysteronproteron〔颠倒历史顺序的逆序法〕,这同问题的实质无关。必须有这个前提,因此,不能象您做的那样,把那种不懂资本主义簿记的、因而不把种子等等算进预付资本的农民经济引进来!“荒谬”的不是李嘉图,而是洛贝尔图斯,他把资本家和工人存在于“土地耕种之前”这一观点强加于李嘉图了。(第176页)

\begin{quote}{“照李嘉图的观点,只有在……社会中产生了资本,知道有资本盈利并支付这种盈利的时候,土地的耕种才开始。”(第178页)}\end{quote}

真是胡扯!只有当资本家以租地农场主的身分插在土地耕种者和土地所有者之间的时候(不论是以前的臣仆靠欺骗手法成了资本主义租地农场主,还是工业家把他的资本不投于工业而投于农业),才开始有——当然不是一般的“土地耕种”,而是——“资本主义的”土地耕种,它在形式上和内容上都同以前的耕种方式大不相同。

\begin{quote}{“在每一个国家,土地的大部分变成私有财产比土地被耕种早得多,无论如何比工业中形成资本盈利率早得多。”(第179页)}\end{quote}

洛贝尔图斯要在这个问题上懂得李嘉图的观点,他就必须是一个英国人,而不是一个波美拉尼亚的地主,而且必须懂得圈围公有地和荒地的历史。洛贝尔图斯先生举出美国作例子。这里,国家把土地

\begin{quote}{“一小块一小块地卖给移民,的确,价格很便宜,可是这个价格无论如何一定已经代表着地租”。(第179—180页)}\end{quote}

绝对不是。这个价格并不构成地租,正象一般的营业税不能构成营业租,或者一般的任何税不能构成任何“租”一样。

\begin{quote}{“我断定,b点所说的地租提高的原因{由于人口增加或使用的劳动量增加}构成地租对资本盈利的优越性。这个原因任何时候也不能提高资本盈利。的确,在生产率不变但生产力增加(人口增加)的条件下,由于国民总产品价值的增加,国民得到更多的资本盈利,但是这种增加了的资本盈利总是摊到按同一比例增加了的资本身上,所以盈利率还是同过去一样高。”(第184—185页)}\end{quote}

这是错误的。如果剩余劳动时间例如不是2小时,而是3、4、5小时,无酬剩余劳动量就会增加。预付资本量并不随着这个无酬剩余劳动量[按同一比例]增加,第一,因为对这个剩余劳动的新的余额是不付报酬的,也就是说不引起任何资本支出;第二,因为用于固定资本的支出不是同这里的固定资本的使用以同一比例增加的。纱锭等的数量并不增加。当然,纱锭磨损加速,但不是与它们的使用增加成同一比例。由此可见,在生产率不变的条件下,这里利润是增加了,因为不仅剩余价值增加,剩余价值率也增加了。在农业中,由于自然条件的关系,这是办不到的。另一方面,生产率随着投资增加而迅速发生变化。撇开分工和机器不说,虽然支出的资本绝对地说很大,但是由于生产条件的节约,相对地说就不是那么大了。因此,即使剩余价值(不仅剩余价值率)不变,利润率也可能提高。

[520]洛贝尔图斯的下述论点是完全错误的,是带有波美拉尼亚地主气味的:

\begin{quote}{“可能,在这三十年的过程中〈从1800到1830年〉通过地产析分,甚至开垦新地,出现了许多新的土地占有,因而增加了的地租也在更多的所有者之间分配,但是地租在1830年并没有比1800年分摊在更多的摩尔根上;那些新划分或新耕种的地段的全部面积原先就包括在早已存在的地段中了,因此,1800年的较低的地租,象1830年的较高的地租一样,也是由这些地段分摊,也影响英国整个地租的高度。”(第186页)}\end{quote}

亲爱的波美拉尼亚人呀!为什么老是自以为是地把您的普鲁士关系搬到英国去呢?英国人完全不认为,如果从1800年到1830年“圈围”了300万到400万英亩\endnote{马克思在《资本论》第一卷第二十四章中写道:1801年到1831年英国农村居民被夺去3511770英亩公有地,并“由地主通过议会赠送给地主”(见《马克思恩格斯全集》中文版第23卷第796页)。——第172页。}——这是事实(要核实一下),——那末在1830年以前,在1800年,地租也是分摊在这400万英亩上。相反,当时这400万英亩都是荒地或公有地,是不提供任何地租、也不属于任何人的。

如果洛贝尔图斯同凯里一样(不过方式不同)想向李嘉图证明,由于自然原因和其他原因,“最肥沃的”土地大部分并非首先被耕种,那末,这同李嘉图是毫无关系的。所谓“最肥沃的”土地,每一次都是指一定生产条件下的“最肥沃的”土地。

洛贝尔图斯对李嘉图的反驳有很大一部分是由于他把“波美拉尼亚”生产关系和“英国”生产关系天真地等同起来。李嘉图是以资本主义生产为前提的,在象英国这样发达的资本主义生产中,资本主义租地农场主和土地所有者是分离的。洛贝尔图斯引用的却是本身与资本主义生产方式无关的关系,资本主义生产方式只是加筑在这些关系之上。例如,洛贝尔图斯先生关于经济中心在经济复合体中的地位所说的话,完全适用于波美拉尼亚,却不适用于英国,在英国,资本主义生产方式自从十六世纪末叶以来越来越占优势,它把一切条件同化,在各个不同时期把历史造成的各种前提——村落、建筑物和人——一个一个地彻底铲除,以保证“最有效的”投资。

洛贝尔图斯关于“投资”所说的话同样是错误的。

\begin{quote}{“李嘉图把地租限于为使用土地原有的、自然的和不可摧毁的力而支付给土地所有者的数额。从而,他想把已耕地上应归于资本的一切从地租中扣除。但是,很明白,李嘉图从一块土地上的收入中划归资本的决不能多于十足的国内普通利息。因为不然的话,他就得假定在一国的经济发展中有两种不同的盈利率,一种是农业的盈利率,它的盈利大于普通的工业盈利;一种是工业的盈利率。可是这个假定就会推翻他的正是以盈利率的均等为基础的整个体系。”(第215—216页)}\end{quote}

这又是波美拉尼亚地主的观念,这种地主贷进资本,是为了使自己的地产更加有利可图,因此,他出于理论的和实际的考虑,想向贷出资本的人仅仅支付“国内普通利息”。可是在英国事情却不是这样。那里用于改良土地的资本,是租地农场主即资本主义农场主支出的。他对这种资本所要求的,完全同对他直接投入生产的资本所要求的一样,不是国内普通利息,而是国内普通利润。他不会把资本贷给地主,让地主对这种资本支付“国内普通”利息。他可能自己贷进资本,或者使用他自己的追加资本,以便这种资本给他提供“国内普通”工业利润,这种工业利润至少是国内普通利息的两倍。

此外,安德森已经知道的,李嘉图也知道。而且,李嘉图还明确地说过,[521]这样由资本造成的土地生产力,后来同土地的“自然”生产力溶合在一起,从而提高了地租。洛贝尔图斯对这一点毫无所知,因而胡说八道。

我已经完全正确地说明过现代的土地所有权:

\begin{quote}{“李嘉图所说的地租就是资产阶级状态的土地所有权,也就是从属于资产阶级生产条件的封建所有权。”(《哲学的贫困》1847年巴黎版第156页)\endnote{见《马克思恩格斯全集》中文版第4卷第183页。——第174页。}}\end{quote}

在那里我已经正确地指出:

\begin{quote}{“尽管李嘉图已经假定资产阶级的生产是地租存在的必要条件,但是他仍然把他的地租概念用于一切时代和一切国家的土地所有权。这就是把资产阶级的生产关系当作永恒范畴的一切经济学家的通病。”(同上,第160页)\endnote{同上,第186页。——第174页。}}\end{quote}

我同样正确地指出,正如所有其他资本一样,“土地资本”是可以增多的:

\begin{quote}{“正如所有其他生产工具一样,土地资本是可以增多的。我们不能在它的物质成分上(用蒲鲁东先生的说法)添加任何东西,但是我们可以增加作为生产工具的土地。人们只要对已经变成生产资料的土地进行新的投资,也就是在不增加土地的物质即土地面积的情况下增加土地资本。”(同上,第165页)\endnote{同上,第189页。——第174页。}}\end{quote}

我那时着重指出的工业和农业之间的差别仍然正确:

\begin{quote}{“首先,这里不能象工业生产中那样随意增加效率相同的生产工具的数量,即肥沃程度相同的土地数量。其次,由于人口逐渐增加,人们就开始经营劣等地,或者在原有土地上进行新的投资,这新的投资的生产率比最初投资的生产率就相应地降低。”(同上,第157页)\endnote{同上,第183页。——第174页。}}\end{quote}

洛贝尔图斯说:

\begin{quote}{“但是,我还要注意到使农业机器\endnote{洛贝尔图斯这里说的“农业机器”,指肥力不同的各级土地。洛贝尔图斯把土地同效率不等的机器相比,是从马尔萨斯那里借用来的。——第174页。}从坏变好的另一种情况,这种情况的发生固然缓慢得多,但是要普遍得多。这就是对一块土地不断耕种,只要依照合理的制度,即使没有一点额外投资,这种耕种本身也能改良土地。”(《给冯·基尔希曼的社会问题书简。第三封信》第222页)}\end{quote}

这一点安德森已经说过了。耕种会改良土地。

[洛贝尔图斯接着说:]

\begin{quote}{“您应当证明,从事农业的劳动人口,随着时间的推移,同食物的生产比较起来,或者至少是同一国人口的其余部分比较起来,是以更大的比例增长着。只有根据这一点才能得出驳不倒的结论:随着农业生产的扩大,必须把越来越多的劳动用在农业上。但是统计恰恰在这一点上同您矛盾。”(第274页)“是呵,您甚至可以确信,到处占统治地位的是这样一条规则:一国的人口越密,从事农业的人的比例越小……这种现象在同一个国家的人口增长中也表现出来:不从事农业的那部分人口几乎到处都以较大的比例增长。”(第275页)}\end{quote}

但是,这种情况一部分是因为有更多的耕地变成放牧牛羊的牧场,一部分是因为在较大规模的生产中——大农业中——劳动的生产率提高了。但是也因为——洛贝尔图斯先生完全没有注意到这个情况——非农业人口中有相当大的一部分人从事为农业服务的劳动,他们提供不变资本——这种不变资本随农业技术进步而不断增长——如矿肥、外国种子、各种机器。

照洛贝尔图斯先生的说法,

\begin{quote}{“今天〈在波美拉尼亚〉农业主不把自己农场生产的耕畜饲料看成资本”。(第78页)[522]“资本就其本身来说,或者从国民经济的意义上说,是进一步用于生产的产品……但是,就它所提供的特殊盈利来说,或者,就现在的企业主对资本所理解的意义来说,它要成为资本,就必须表现为‘支出’。”(第77页)}\end{quote}

不过,“支出”这个概念并不象洛贝尔图斯所认为的那样,要求把产品作为商品买进来。如果某一部分产品不是作为商品卖出,而是再加入生产,那末这部分产品就是作为商品加入生产。这部分产品一开始就是作为“货币”来估价的,而这一点由于所有这些“支出”——其中包括农业中的牲畜、饲料、肥料、谷种、各类种子——同时也都作为“商品”出现在市场上,就看得更加清楚了。但是,看来在“波美拉尼亚”,人们是不把所有这些算到“支出”项下的。

\begin{quote}{“这些不同劳动〈在工业和原产品生产中〉的特殊成果的价值,还不是它们的所有者的收入本身,而只是计算这种收入的尺度。这种各自得到的收入本身,都是社会收入的一部分,社会收入只有农业和工业的共同劳动才能创造出来,因此,它的各部分也只有这种共同的劳动才能创造出来。”(第36页)}\end{quote}

这有什么相干呢?这个价值的实现只能是它在使用价值中的实现。但是所谈的完全不是这一点。而且,必要工资这个概念已经包含着:有多少价值表现为维持工人生活的必要生活资料(农产品和工业品)。

到此结束。

\tchapternonum{[第十章]李嘉图和亚当·斯密的费用价格\endnote{在《剩余价值理论》第二册中,“费用价格”(《Kostenpreis》或《Kostpreis》,《costprice》)这一术语,马克思用在“生产价格”即“平均价格”(c+v+平均利润)的意义上。关于“平均价格”这一术语见注7。在马克思的著作中首次见到《Kostenpreis》这一术语是在本卷第一册第77页,不过在那里它是用在商品“内在的生产费用”(c+v+m)的意义上,商品“内在的生产费用”是和商品的价值一致的。在《剩余价值理论》第三册中,《Kostenpreis》这一术语马克思有时用在生产价格的意义上,有时用在资本家的生产费用的意义上,也就是指c+v。《Kostenpreis》这一术语所以有三种用法,是由于《Kosten》(“费用”,“生产费用”)这个词在经济科学中被用在三种不同的意义上,正如马克思在《剩余价值理论》第三册(1861—1863年手稿第788—790页和第928页)特别指出的,这三种意义是:(1)资本家预付的东西,(2)预付资本的价格加平均利润,(3)商品本身的实在的(或内在的)生产费用。除了资产阶级政治经济学古典作家使用的这三种意义以外,“生产费用”这一术语还有第四种庸俗的意义,即让·巴·萨伊给“生产费用”下的定义:“生产费用是为劳动、资本和土地的生产性服务支付的东西。”(让·巴·萨伊《论政治经济学》1814年巴黎第2版第2卷第453页)马克思坚决否定了对“生产费用”的这种庸俗的理解(例如见本册第142、239和535—536页)。——第177页。}理论(批驳部分)}

\tsectionnonum{[A.李嘉图的费用价格理论]}

\tsubsectionnonum{[(1)重农学派理论的破产和地租观点的进一步发展]}

安德森关于“不是地租决定土地产品的价格,而是土地产品的价格决定地租”\fnote{见本册第158页。——编者注}的论点(在亚·斯密那里部分地也有这种论点),完全推翻了重农学派的学说。这样,地租的源泉就是农产品的价格,而不是农产品本身,也不是土地。因此,认为地租是农业的特殊生产率的产物,而这种生产率又是土地特殊肥力的产物的观点也就站不住脚了。因为,如果同量劳动用在特别肥沃的要素中,因而劳动本身的生产率也特别高,那末,结果只能是,这种劳动表现为较大的产品量,因而单位产品的价格较低,而决不会相反,即这种劳动的产品的价格高于物化了同量劳动的其他产品的价格,因而它的价格和其他商品不同,除了利润和工资以外,还能提供地租。(亚·斯密在考察地租的时候,起先用他原来的关于地租是剩余劳动的一部分的观点,反驳了,或者至少是否定了重农学派的观点,后来,部分地又回到重农学派的观点上去。)

布坎南用下面的话概述了重农学派观点被摈弃的情况:

\begin{quote}{“有人认为农业提供产品并从而提供地租,是因为自然在耕种土地的过程中和人类劳动一起发挥作用,这种观点纯粹是幻想。地租不是来源于产品,而是来源于产品出卖的价格;而这个价格的获得,不是因为自然协助了生产,而是因为这个价格能使消费适应于供给。”\endnote{引自布坎南在他出版的亚·斯密《国富论》中加的一个脚注(亚当·斯密《国民财富的性质和原因的研究》,附大卫·布坎南的注释和增补,三卷集,1814年爱丁堡版第2卷第55页)。李嘉图的《原理》第2章(脚注中)引用了布坎南的这段话。——第178页。}}\end{quote}

重农学派的这个观点被摈弃了,——但是这个观点就其更深刻的意义来说是完全合理的,因为重农学派把地租看作剩余价值的唯一形式,而把资本家和工人一齐都只看作地主的雇佣劳动者,——剩下可能存在的就只有下述几种观点:

[523][第一,]认为地租来自农产品的垄断价格,而垄断价格又来自土地所有者对土地的垄断。\fnote{见本册第26页。——编者注}按照这一观点,农产品的价格总是高于其价值。这里有一个价格的附加额,商品的价值规律为土地所有权的垄断所破坏。

按照这一观点,地租所以来自农产品的垄断价格,是因为农产品的供给总是低于需求的水平,或者说,需求总是高于供给的水平。可是,为什么供给不会提高到需求的水平呢?为什么追加的供给不会使这种关系达到平衡,从而——按照这一理论——把一切地租取消呢?为了解释这一点,马尔萨斯一方面求助于臆造,说什么农产品直接为自己创造了消费者(关于这一点,以后评论他和李嘉图的论战时再谈),另一方面又求助于安德森的理论,说什么因为追加的供给耗费更多的劳动,所以农业的生产率降低。因此,这个观点就其不是根据纯粹臆造这一方面来说,是同李嘉图的理论一致的。这里也是价格高于价值,有一个附加额。

[第二,]李嘉图的理论:没有绝对地租,只有级差地租。这里也是提供地租的农产品的价格高于其个别价值,只要有地租存在,那就是由于有农产品的价格超过其价值的余额。不过在这里,这种价格超过价值的余额和一般的价值理论并不矛盾(虽然事实还是事实),因为在每一个生产领域内部,属于这个领域的商品的价值不是决定于商品的个别价值,而是决定于商品在该领域一般生产条件下所具有的价值。这里提供地租的产品的价格也是垄断价格,不过这种垄断在一切生产领域都有,它只是在这个生产领域才固定下来,因而采取了不同于超额利润的地租形式。这里,也是需求超过供给,或者也可以说,在价格由于需求超过供给而上涨以前,追加的需求不可能按原来供给状况下的价格,由追加的供给来满足。这里,地租(级差地租)的产生也是由于有价格超过价值的余额,由于较好土地的产品的价格上涨到高于其价值,从而引起追加的供给。

[第三,]地租只不过是投入土地的资本的利息。\fnote{见本册第26、152—153和157页。——编者注}这种观点和李嘉图的观点有一个共同的地方,就是否认绝对地租。在投入同量资本的不同地段提供数量不等的地租的情况下,它不得不承认级差地租。因此,实际上它可归结为李嘉图的观点,即某种土地不提供地租,凡是提供本来意义的地租的地方,提供的都是级差地租。但是这种观点绝对不能解释没有投入任何资本的土地的地租,瀑布、矿山等的地租。实际上,这种观点不过是从资本主义的立场出发,以利息为名,把地租从李嘉图的抨击下拯救出来的一种尝试。

最后[第四],李嘉图认为,在不提供地租的土地上,产品的价格等于产品的价值,因为价值等于平均价格,即预付资本加平均利润。所以,李嘉图错误地认为,商品的价值等于商品的平均价格。如果这种错误的前提不能成立的话,那末绝对地租就是可能的,因为农产品的价值,如同其他所有商品中的一大类商品的价值一样,是高于它们的平均价格的,但是,由于土地所有权的存在,农产品的价值不会象其他这些商品那样平均化为平均价格。所以,这种观点同垄断论一起承认土地所有权本身和地租有直接的关系;它同李嘉图一起承认有级差地租,最后,它认为绝对地租的存在绝不违反价值规律。

\tsubsectionnonum{[(2)价值决定于劳动时间是李嘉图理论的基本论点。作为经济科学发展的必然阶段的李嘉图研究方法及其缺点。李嘉图著作的错误结构]}

李嘉图是从商品的相对价值(或交换价值)决定于“劳动量”这一论点出发的。(后面我们将要研究李嘉图使用“价值”一词的不同含义。贝利对李嘉图理论的批评就是以此为根据的,同时,李嘉图的价值论的缺点也就在这里。)决定价值的这种“劳动”的性质,李嘉图并没有进一步研究。如果两种商品是等价物,或者说,它们在一定的比例上是等价物,或者也可以说,如果它们的量按[524]它们各自包含的“劳动”量来说是不相同的,那也很明显,在它们是交换价值的情况下,它们按其实体来说是相同的。它们的实体是劳动。所以它们是“价值”。根据它们各自包含的这种实体是多还是少,它们的量是不相同的。而这种劳动的形式——作为创造交换价值或表现为交换价值的劳动的特殊规定,——这种劳动的性质,李嘉图并没有研究。因此,李嘉图不了解这种劳动同货币的关系,也就是说,不了解这种劳动必定要表现为货币。所以,他完全不了解商品的交换价值决定于劳动时间和商品必然要发展到形成货币这两者之间的联系。他的错误的货币理论就是由此而来的。他一开始就只谈价值量,就是说,只谈各个商品价值量之比等于生产这些商品所必需的劳动量之比。李嘉图是从这一点出发的。他明确指出,亚·斯密是他的出发点(第一章第一节)\endnote{大·李嘉图《政治经济学和赋税原理》1821年伦敦第3版第1—12页。——第181页。}。

李嘉图的方法是这样的:李嘉图从商品的价值量决定于劳动时间这个规定出发,然后研究其他经济关系(其他经济范畴)是否同这个价值规定相矛盾,或者说,它们在多大的程度上改变着这个价值规定。人们一眼就可以看出这种方法的历史合理性,它在政治经济学史上的科学必然性,同时也可以看出它在科学上的不完备性,这种不完备性不仅表现在叙述的方式上(形式方面),而且导致错误的结论,因为这种方法跳过必要的中介环节,企图直接证明各种经济范畴相互一致。

这种研究方法从历史上看是合理的和必然的。在亚·斯密那里,政治经济学已发展为某种整体,它所包括的范围在一定程度上已经形成,因此,萨伊能够肤浅而系统地把它概述在一本教科书里。在斯密和李嘉图之间的这段时期,仅仅对生产劳动和非生产劳动、货币、人口论、土地所有权以及税收等个别问题作了一些研究。斯密本人非常天真地活动于不断的矛盾之中。一方面,他探索各种经济范畴的内在联系,或者说,资产阶级经济制度的隐蔽结构。另一方面,他同时又按照联系在竞争现象中表面上所表现的那个样子,也就是按照它在非科学的观察者眼中,同样在那些被实际卷入资产阶级生产过程并同这一过程有实际利害关系的人们眼中所表现的那个样子,把联系提出来。这是两种理解方法,一种是深入研究资产阶级制度的内在联系,可以说是深入研究资产阶级制度的生理学,另一种则只是把生活过程中外部表现出来的东西,按照它表现出来的样子加以描写、分类、叙述并归入简单概括的概念规定之中。这两种理解方法在斯密的著作中不仅安然并存,而且相互交错,不断自相矛盾。在斯密那里,这样做是有理由的(个别的专门的研究,如关于货币的研究除外),因为他的任务实际上是双重的。一方面,他试图深入研究资产阶级社会的内部生理学,另一方面,他试图既要部分地第一次描写这个社会外部表现出来的生活形式,描述它外部表现出来的联系,又要部分地为这些现象寻找术语和相应的理性概念,也就是说,部分地第一次在语言和思维过程中把它们再现出来。前一任务,同后一任务一样使他感到兴趣,因为两个任务是各自独立进行的,所以这里就出现了完全矛盾的表述方法:一种方法或多或少正确地表达了内在联系,另一种方法同样合理地,并且缺乏任何内在关系地,——和前一种理解方法没有任何联系地——表达了外部表现出来的联系。

斯密的后继者们,只要他们的观点不是从比较陈旧的、已被推翻的理解方法出发对斯密的反动,都能够在自己的专门研究和考察中毫无阻挡地前进,而且始终把亚·斯密作为自己的基础,不管他们是和斯密著作中的内在部分还是外在部分连结在一起,或者几乎总是把这两部分混在一起。但是,李嘉图终于在这些人中间出现了,他向科学大喝一声:“站住!”资产阶级制度的生理学——对这个制度的内在有机联系和生活过程的理解——的基础、出发点,是价值决定于劳动时间这一规定。李嘉图从这一点出发,迫使科学抛弃原来的陈规旧套,要科学讲清楚:它所阐明和提出的其余范畴——生产关系和交往关系——同这个基础、这个出发点适合或矛盾到什么程度;一般说来,只是反映、再现过程的表现形式的科学以及这些表现本身,同资产阶级社会的内在联系即现实生理学所依据的,或者说成为它的出发点的那个基础适合到什么程度;一般说来,这个制度的表面运动和它的实际运动之间的矛盾是怎么回事。李嘉图在科学上的巨大[525]历史意义也就在这里,因此,被李嘉图抽掉了立足点的庸俗的萨伊怒气冲冲地说:

\begin{quote}{“有人借口扩充它〈科学〉,把它推到真空里去了。”\endnote{让·巴·萨伊《论政治经济学》1826年巴黎第5版第1卷第83—84页,或让·巴·萨伊《论政治经济学》1841年巴黎第6版第41页。——第183页。}}\end{quote}

同这个科学功绩紧密联系着的是,李嘉图揭示并说明了阶级之间的经济对立——正如内在联系所表明的那样,——这样一来,在政治经济学中,历史斗争和历史发展过程的根源被抓住了,并且被揭示出来了。所以,凯里——参看后面有关段落——给李嘉图加上了共产主义之父的罪名:

\begin{quote}{“李嘉图先生的体系是一个制造纷争的体系……整个体系具有挑动阶级之间和民族之间的仇恨的倾向……他的著作是那些企图用平分土地、战争和掠夺的手段来攫取政权的蛊惑者们的真正手册。”(亨·凯里《过去、现在和将来》1848年费拉得尔菲亚版第74—75页)}\end{quote}

可见,李嘉图的研究方法,一方面具有科学的合理性和巨大的历史价值,另一方面,它在科学上的缺陷也是很明显的,这一点将在后面详细说明。

李嘉图著作的非常奇特的、必然谬误的结构,也是由此而来。全书(第三版)共分三十二章。其中有十四章论述赋税,因而只是理论原则的运用。\endnote{马克思除了把李嘉图著作中有关本来意义上的赋税的十二章(第8—18章和第29章)列为论述赋税的各章以外,还把涉及赋税问题的另外两章——第22章和第23章(《出口补贴和进口禁令》和《论生产补贴》)也包括进去。按照李嘉图的理论,补贴是由居民交纳的各种赋税所组成的基金来支付的。——第184页。}第二十章《价值和财富,它们的特性》,无非是研究使用价值和交换价值的区别,因而是第一章《论价值》的补充。第二十四章《亚·斯密的地租学说》,以及第二十八章《论……黄金、谷物和劳动的比较价值》和第三十二章《马尔萨斯先生的地租观点》,不过是李嘉图地租理论的补充,部分地是对这个理论的辩护,因而仅仅是论述地租的第二章和第三章的附录。第三十章《论需求和供给对价格的影响》不过是第四章《论自然价格和市场价格》的附录。而第十九章《论商业途径的突然变化》则是这一章的第二个附录。第三十一章《论机器》不过是第五章《论工资》和第六章《论利润》的附录。第七章《论对外贸易》和第二十五章《论殖民地贸易》,同论赋税各章一样,仅仅是前面提出的原则的运用。第二十一章《积累对于利润和利息的影响》是论地租、利润和工资各章的附录。第二十六章《论总收入和纯收入》是论工资、利润和地租各章的附录。最后,第二十七章《论货币流通和银行》在这本书中完全是孤立的,它只是李嘉图在他较早的论货币的著作中提出的观点的进一步发挥,部分地是这些观点的变态。

可见,李嘉图的理论完全包括在他这部著作的前六章中。我说的这部著作的错误结构,就是指这一部分。另一部分(论货币的那部分除外)是实际运用、解释和补充,按其内容的性质来说是杂乱地放在那里的,根本不要求有什么结构。但是理论部分(前六章)的错误结构并不是偶然的,而是由李嘉图的研究方法本身和他给自己的研究提出的特定任务决定的。这种结构表现了这种研究方法本身在科学上的缺陷。

第一章是《论价值》。它又分为七节。第一节研究的其实是:工资是否同商品价值决定于商品所包含的劳动时间这一规定相矛盾?第三节是要证明:我称为不变资本的东西加入商品价值,是和价值规定不矛盾的,工资的提高或降低同样不会影响商品的价值。第四节研究的是:在机器和其他固定的、耐久的资本在不同生产领域以不同的比例加入总资本的情况下,它们的运用能在多大程度上改变交换价值决定于劳动时间这个规定。第五节研究的是:如果不同生产领域所使用的资本的耐久程度不等、周转时间不同,工资的提高或降低能在多大程度上改变价值决定于劳动时间这个规定。由此可见,在这第一章里不仅假定了商品的存在,——而在考察价值本身的时候是不应该作进一步的假定的,——而且假定了工资、资本、利润,甚至,如我们将会看到的,还假定了一般利润率、由流通过程产生的资本的各种形式,以及“自然价格和市场价格”的区别,这种区别在后面两章(《论地租》和《论矿山地租》)中甚至起着决定性的作用。

第二章(《论地租》),[526]——第三章(《论矿山地租》)只是第二章的补充,——完全依照李嘉图的研究进程,一开始又提出这样的问题:土地所有权和地租是否同商品价值决定于劳动时间这一规定相矛盾?

\begin{quote}{李嘉图在第二章(《论地租》)一开头就说:“但尚待考察的是,对土地的占有以及由此而来的地租的产生,是否会引起商品相对价值的变动而不管生产商品所必需的劳动量如何。”(《政治经济学和赋税原理》1821年伦敦第3版第53页)}\end{quote}

李嘉图为了进行这一研究,不仅顺便把“市场价格”和“实际价格”(价值的货币表现)的关系引进来,而且把整个资本主义生产以及他对工资和利润之间的关系的全部见解作为前提。因此,第四章(《论自然价格和市场价格》)、第五章(《论工资》)和第六章(《论利润》)所谈的东西,在头两章(《论价值》和《论地租》)以及作为第二章附录的第三章中不仅已经作了假定,而且有了充分的发挥。在后面三章中,就它们所提出的理论上的新东西来说,只是在这里或那里堵塞漏洞,补充一些更确切的规定,其中大部分按理在第一章、第二章中本来就应当谈到。

可见,李嘉图的全部著作已经包括在它头两章里了。在这两章中,把发展了的资产阶级生产关系,因而也把被阐明的政治经济学范畴,同它们的原则即价值规定对质,查清它们同这个原则直接适合到什么程度,或者说,查清它们给商品的价值关系造成的表面偏差究竟是什么情况。李嘉图著作的这两章包含着他对以往政治经济学的全部批判,他在这里同亚·斯密的贯串其全部著作的内在观察法和外在观察法之间的矛盾断然决裂,而且通过这种批判得出了一些崭新的惊人结果。因此,这头两章给人以高度的理论享受,因为它们简明扼要地批判了那些连篇累牍、把人引入歧途的老观念,从分散的各种各样的现象中吸取并集中了最本质的东西,使整个资产阶级经济体系都从属于一个基本规律。这头两章由于其独创性、基本观点一致、简单、集中、深刻、新颖和洗炼而给人以理论上的满足,但是再往下读这本著作时这种理论上的满足就必然会消失。在那里,有的地方也会有个别独到的见解吸引住我们。但总的说来令人感到疲倦和乏味。进一步的阐述已经不再是思想的进一步发展了。这种阐述不是单调地、形式地把同一些原则运用于各种各样凭外表拿来的材料或者为这些原则进行辩护,就是单纯地重复或者补充;最多是在该书的最后部分有些地方作出某种引人注意的结论。

我们在批判李嘉图的时候,应该把他自己没有加以区别的东西区别开来。[第一,]是他的剩余价值理论,这个理论在他那里当然是存在的,虽然他没有把剩余价值确定下来,使之有别于它的特殊形式利润、地租、利息。第二是他的利润理论。我们将从分析李嘉图的利润理论开始,虽然它不属于这一篇,而属于第三篇[18]的历史附录。

\tsubsectionnonum{[(3)李嘉图在绝对价值和相对价值问题上的混乱。他不懂价值形式]}

首先还要稍微说明一下,李嘉图怎样把各种[不同的]“价值”规定混淆起来了。贝利反驳李嘉图,就是根据这一点。不过,这一点对我们来说也是重要的。

李嘉图起先把价值称为“交换价值”,他和亚·斯密一起把价值规定为“购买其他货物的能力”。(《原理》第1页)这是作为最初的表现形式的交换价值。但是,接着他就谈到真正的价值规定:

\begin{quote}{“各种商品的现在的或过去的相对价值,决定于劳动所生产的各种商品的相对量”。(同上,第9页)}\end{quote}

这里所说的“相对价值”无非是由劳动时间决定的交换价值。但是相对价值也可能有另一种意义,就是说,我用另一种商品的使用价值来表现一种商品的交换价值,比如说,用咖啡的使用价值来表现糖的交换价值。

\begin{quote}{“两种商品的相对价值发生变动,我们想知道是哪一种发生了变动。”(同上,第9页)}\end{quote}

什么样的变动?这种“相对价值”李嘉图在后面也称为“比较价值”。(同上,第488页及以下各页)我们想知道是哪一种商品发生了“变动”,就是说,我们想确定前面称为相对价值的那种“价值”的变动。例如,1磅糖=2磅咖啡。后来1磅糖=4磅咖啡。我们想知道的“变动”在于:是糖的“必要劳动时间”变了呢,还是咖啡的“必要劳动时间”变了,是糖耗费的劳动时间比过去多一倍呢,还是咖啡耗费的劳动时间比过去少一半,生产这两种商品各自所必要的劳动时间的这两种“变动”中,是哪一种变动引起了它们的交换比例的变动。可见,糖或咖啡的这种“相对价值,或者说,比较价值”——它们交换的比例——不同于第一种意义的相对价值。在第一种意义上,糖的相对价值决定于[527]一定劳动时间内能够生产出来的糖的量。在第二种场合,糖[和咖啡]的相对价值表示它们相互交换的比例,而这个比例的变动可能是咖啡或者糖的第一种意义的“相对价值”变动的结果。虽然它们的第一种意义的“相对价值”发生了变动,它们相互交换的比例可能不变。虽然生产糖和咖啡的劳动时间增加了一倍,或者减少了一半,1磅糖可能仍旧等于2磅咖啡。它们的比较价值(就是说,糖的交换价值用咖啡来表现,咖啡的交换价值用糖来表现)的变动,只有在它们的第一种意义的相对价值,即由劳动量决定的价值,按不同的程度变动,因而它们的比例发生了变动的时候,才表现出来。绝对变动如果不改变原来的比例,就是说,如果变动的幅度一样,方向一致,就不会引起这些商品的比较价值的任何变动,也不会引起它们的货币价格的任何变动,因为货币的价值即使发生变动,对它们两者也是按相同的程度变动的。因此,不论两种商品中每一种的价值我是用其中另一种的使用价值来表现,还是用它们的货币价格来表现,即这两者的价值用第三种商品的使用价值来表示,这些相对价值,或者说,比较价值,或者说,价格,仍旧不变,应该把这种相对价值的变动同商品的第一种意义的相对价值的变动区别开来,因为后者所表示的仅仅是生产商品本身所必需的,即物化在商品本身中的劳动时间量的变动。因此,同第二种意义的相对价值(即一个商品的交换价值用另一个商品的使用价值或者用货币来实际表现)相比,第一种意义的相对价值就表现为“绝对价值”。所以,在李嘉图的著作里,也可以看到用“绝对价值”这一术语来表示第一种意义的相对价值。

如果在上述例子中1磅糖耗费的劳动时间仍然和过去一样多,那末它的第一种意义的“相对价值”就没有变动。如果咖啡耗费的劳动量减少一半,那末用咖啡来表现的糖的价值就发生变动,因为咖啡的第一种意义的“相对价值”变动了。可见,糖和咖啡的相对价值与它们的“绝对价值”是表现得不同的,而这种差别所以会表现出来,是因为比如说糖的比较价值同那些绝对价值保持不变的商品相比并没有变动。

\begin{quote}{“我希望引起读者注意的这个研究,涉及的是商品相对价值的变动的影响,而不是商品绝对价值的变动的影响。”(同上,第15页)}\end{quote}

这种“绝对”价值,李嘉图在其他场合也称为“实际价值”,或直接称为“价值”。(例如第16页)

请看贝利在下面这本书中对李嘉图的反驳:《对价值的本质、尺度和原因的批判研究,主要是论李嘉图先生及其信徒的著作》,《略论意见的形成和发表》一书的作者著,1825年伦敦版(并见同一作者所著:《为〈韦斯明斯特评论〉杂志上一篇关于价值的论文给一位政治经济学家的信》1826年伦敦版)。贝利的整个反驳部分地是围绕价值概念规定中这些不同方面的,这些不同方面在李嘉图的著作中并没有发挥,只是实际存在着,彼此交错着,而在其中贝利看到的只是“矛盾”。第二,他的反驳是针对不同于比较价值(即第二种意义的相对价值)的“绝对价值”,或者说,“实际价值”的。

\begin{quote}{贝利在上述第一部著作中说:“他们〈李嘉图及其信徒〉不是把价值看成两个物之间的比例,而是把价值看成由一定量劳动生产出来的有用的成果。”(第30页)他们认为,“价值是某种内在的和绝对的东西”。(同上,第8页)}\end{quote}

最后这个指责是由李嘉图说明问题的缺陷引起的,因为他完全不是从形式方面,从劳动作为价值实体所采取的一定形式方面来研究价值,而只是研究价值量,就是说,研究造成商品价值量差别的这种抽象一般的、并在这种形式上是社会的劳动的量。否则贝利就会看到,决不因为一切商品就它们是交换价值来说都只是社会劳动、社会劳动时间的相对表现,价值概念的相对性就取消了;贝利也就会明白,商品的相对性决不仅仅在于商品彼此交换的比例,而且在于一切交换价值同作为它们的实体的这种社会劳动的比例。

相反,后面我们将会看到,应该责备李嘉图的,倒是他经常忘记了这种“实际价值”,或者说,“绝对价值”,而只是念念不忘“相对价值”,或者说,“比较价值”。

[528]因而:

\tsubsectionnonum{[(4)]李嘉图对利润、利润率和平均价格等的解释}

\tsubsubsectionnonum{[(a)李嘉图把不变资本同固定资本,可变资本同流动资本混淆起来。关于“相对价值”的变动及其因素问题的错误提法]}

在第一章第三节中,李嘉图阐述了下面这种思想:如果我们说,商品价值决定于劳动时间,那末,这既包括最后的劳动过程中直接花费在这种商品上的劳动,也包括花费在为生产这种商品所必需的原料和劳动资料上的劳动时间;因此,不仅包括新加的、用工资支付的、买进的劳动所包含的劳动时间,并且包括我称为不变资本的那部分商品所包含的劳动时间。李嘉图对这个问题的解释的缺点在这第一章第三节的标题上就已经表现出来了。这个标题是:

\begin{quote}{“影响商品价值的,不仅是直接花费在商品上的劳动,而且还有花费在协助这种劳动的器具、工具和建筑物上的劳动。”(第16页)}\end{quote}

这里漏掉了原料,而花费在原料上的劳动,象花费在劳动资料“器具、工具和建筑物”上的劳动一样,是不同于“直接花费在商品上的劳动”的。但是李嘉图头脑里考虑的已经是下一节了。在第三节,他假定用掉的劳动资料以同样的价值组成部分加入不同商品的生产。而在下一节,考察的是由于固定资本以不同的比例加入生产而产生的差别。因此,李嘉图没有得出不变资本的概念,不变资本一部分由固定资本组成,另一部分即原料和辅助材料则由流动资本组成,正如流动资本不仅包括可变资本,而且包括原料等以及一切加入一般消费(不只是加入工人的消费)的生活资料\endnote{马克思这里说的“加入一般消费的生活资料”,一方面是指所有个人消费的资料,另一方面是指用于机器的生产消费资料,即辅助材料(煤、润滑油等)。——第192页。}。

不变资本加入商品的比例,并不影响商品的价值,并不影响商品包含的相对劳动量,但是,这种比例直接影响包含等量劳动时间的商品所包含的不同的剩余价值量,或者说,剩余劳动量。因此,这种不同的比例就造成不同于价值的平均价格。

关于第一章第四、五两节,首先要指出,李嘉图不去研究不同生产领域中同一资本量的组成部分由不变资本和可变资本构成的比例这种极为重要的、影响剩余价值直接生产的差别,却专门去研究资本形式的差别和同量资本采取这些不同形式的不同比例,研究从资本的流通过程产生的形式差别,即固定资本和流动资本、固定程度较大或较小的资本(即具有不同耐久程度的固定资本)和资本的不等的流通速度或周转速度。并且,李嘉图研究的方法是这样的:他为等量的各种投资,或者说,为使用等量资本的不同生产领域,假定一个一般利润率,或者说,一个等量的平均利润,或者也可以说,他先假定利润和不同生产领域使用的资本的量成比例。其实,李嘉图不应该先假定这种一般利润率,相反,他倒是应该研究一般利润率的存在究竟同价值决定于劳动时间这一规定符合到什么程度,这样,他就会发现,一般利润率同这一规定不是符合的,乍看起来倒是矛盾的,所以一般利润率的存在还须要通过许多中介环节来阐明,而这样做与简单地把它归到价值规律下是大不相同的。这样,李嘉图就会得到一个关于利润本质的完全不同的概念,而不会把利润直接同剩余价值等同起来。

李嘉图先作了这个假定,接着就给自己提出一个问题:如果固定资本和流动资本以不同的比例加入生产,工资的提高或降低对“相对价值”会发生什么影响?或者确切些说,他自以为正是这样来考察问题的。其实,他根本不是这样考察问题的。他考察问题的方法是:他问自己,在几笔资本的流通时间不同、其中包含的不同资本形式所占的比例也不同的情况下,工资的提高或降低对这些资本各自的利润将发生什么影响?这里,他自然发现,根据加入的固定资本等等的多少不同,根据资本中由可变资本即由直接花在工资上的资本组成的部分的大小不同,工资的提高或降低对资本的影响必然大不相同。因此,为了使不同[529]生产领域的利润重新平均化,换句话说,为了恢复一般利润率,商品的价格——不同于商品的价值——就必须按另外的方式来决定。就是说,——他接着得出结论说,——在工资提高或降低的情况下,这些差别会影响“相对价值”。他本应反过来说:这些差别虽然同价值本身毫无关系,但是由于它们对不同生产领域的利润发生不同影响,就造成不同于价值本身的平均价格,即我们后面所说的费用价格,这种费用价格不直接决定于商品的价值,而决定于预付在这些商品上的资本加平均利润。因此,李嘉图本应说:这种平均的费用价格不同于商品的价值。可他不是这样,却得出结论说,它们是等同的,并且带着这个错误的前提去考察地租。

李嘉图认为,只是由于他所研究的三种情况,他才考虑到同商品所包含的劳动时间无关的“相对价值的变动”,就是说,实际上才考虑到商品的费用价格和价值的差别,这种看法也是错误的。他已经假定了这个差别,因为他假定有一个一般利润率,就是说,假定尽管资本的有机组成部分的比例不同,资本提供的利润总是同资本的量成比例,可是资本所提供的剩余价值,却完全决定于资本所吸收的无酬劳动时间的量,这个量,在工资既定时,完全取决于花在工资上的那部分资本的量,而不取决于资本的绝对量。

实际上我们在李嘉图那里看到的是:他先假定有不同于商品价值的费用价格,——既然假定有一般利润率,也就假定了这个差别,——再研究这些费用价格(为了换个花样,现在叫做“相对价值”)本身又怎样由于工资的提高或降低以及在资本的有机组成部分的比例不同的情况下,在相互之间,彼此相对地发生变动。如果把问题钻得更深一些,李嘉图就会发现,在资本的有机组成部分不同(这种不同最初在直接生产过程中表现为可变资本和不变资本的差别,后来由于从流通过程中产生的差别而进一步扩大)的情况下,即使假定工资不变,单单一般利润率的存在,就已经决定了有一种不同于价值的费用价格。换句话说,李嘉图就会发现,单单一般利润率的存在,就决定了有一个同工资的提高或降低完全无关的差别和新的形式规定。李嘉图也会看到,理解这个差别,同他对工资的提高或降低所引起的商品费用价格变动的分析相比,对于整个理论具有无比重要的、决定性的意义。他所满足的结论(而满足于这个结论是符合他的整个研究方法的)是:如果承认并考虑到商品费用价格(或者照他的说法,“相对价值”)的变动,只要这种变动是在投入不同领域的资本的有机构成不同的条件下由工资的变动,由工资的提高或降低引起的,那末,规律仍然是正确的,这种情况同商品“相对价值”决定于劳动时间的规律并不矛盾,因为商品费用价格的其他一切不只是短暂的变动,都只能用生产这些商品各自所必需的劳动时间量的变动来解释。

相反,李嘉图把固定资本和流动资本的差别与不同的资本周转时间相对比,并从不同的流通时间,实际上也就是从资本的流通时间或再生产时间引出这一切差别,却应该看成是一个重大的功绩。

我们首先考察一下李嘉图最初在(第一章)第四节中所叙述的这些差别本身,然后考察在李嘉图的叙述中这些差别以什么方式影响或引起“相对价值”的变动。

\begin{quote}{(1)“在每一种社会状态中,不同行业所使用的工具、器具、建筑物和机器的耐久程度可能彼此不同,生产它们所需要的劳动量也可能各不相同”。(同上,第25页)}\end{quote}

谈到“生产它们所需要的劳动量各不相同”,它的意思可能是指:——看来李嘉图在这里指的只是这一点,——耐久性较差的工具,部分地为了它们的修理,部分地为了它们的再生产,需要较多的劳动(重新进行的直接劳动);也可能是指:耐久程度相同的机器等可能有贵有贱,可能是较多劳动或较少劳动的产品。后面这个观点对于理解可变资本和不变资本的比例很重要,但它同李嘉图所考察的问题毫无关系,因此,李嘉图在任何地方都没有把它当作一个独立的观点。

\begin{quote}{[530](2)“维持劳动的资本〈可变资本〉和投在工具、机器和建筑物上的资本〈固定资本〉可能结合的比例也是多种多样的。”这样,我们就有了“固定资本耐久程度的这种差别,和这两种资本可能结合的比例的这种多样性”。(第25页)}\end{quote}

一眼就可以看出,李嘉图为什么对作为原料存在的那部分不变资本不感兴趣。这部分不变资本属于流动资本。如果工资提高,由机器组成的、无需更换而继续存在的那部分资本并不因此增加支出,可是由原料组成的那部分资本却因此增加支出,因为原料要不断补充,也就是说,要不断地再生产出来。

\begin{quote}{“工人消费的食物和衣服,他在其中从事劳动的建筑物,他劳动时使用的工具,都是会损坏的。但是,这些不同资本的耐用时间却大有差别……有的资本损耗得快,必须经常再生产,有的资本消费得慢,根据这种情况,就有流动资本和固定资本之分。”(第26页)}\end{quote}

可见,这里把固定资本和流动资本的差别归结为再生产时间(同流通时间一致的再生产时间)的差别。

\begin{quote}{(3)“还必须指出,流动资本流通或流回到它的使用者手中的时间可以极不相等。租地农场主买来作种子的小麦\fnote{这里洛贝尔图斯先生可以看到,在英国,种子是“买来”的。},和面包业主买来做面包的小麦相比,是固定资本。前者把小麦播在地里,要等一年以后才能收回;后者把小麦磨成面粉,制成面包卖给顾客,一周之内就能自由地用他的资本重新开始同一事业或开始任何别的事业。”(第26—27页)}\end{quote}

为什么会产生不同流动资本的流通时间的这种差别呢?因为同一资本在一种情况下停留在本来意义的生产领域内的时间较长,虽然在这段时间内劳动过程并没有继续。葡萄酒放在窖里变陈,以及制革、染色等化学过程,就是这种情况。

\begin{quote}{“因此,两种行业可能使用同量的资本,但其固定部分和流动部分的划分却大不相同。”(第27页)(4)“此外,两个工厂主可能使用同量的固定资本和同量的流动资本,但是他们的固定资本的耐久程度〈因而它们的再生产时间〉可能大不相同。一个可能有价值10000镑的蒸汽机,另一个则有价值相等的船舶。”(第27—28页)“……资本的耐久程度不同,或者也可以说……一批商品在能够进入市场以前必须经历的时间不同”。(第30页)(5)“无需说,花费了同量劳动生产出来的商品,如果不能在同样长的时间内进入市场,它们的交换价值就会不相等。”(第34页)}\end{quote}

这样,我们就有:(1)固定资本和流动资本的比例的差别;(2)生产过程继续进行时由于劳动过程中断而引起的流动资本周转的差别;(3)固定资本耐久程度的差别;(4)商品在能够进入本来意义的流通过程以前经受劳动过程的时间(不算劳动时间的中断,不算生产时间和劳动时间的差别\endnote{关于“生产时间”和“劳动时间”,见注8。——第197页。})的差别。关于最后一点,李嘉图做了这样的描述:

\begin{quote}{“假定我花1000镑雇用20个工人在一年内生产一种商品,年终,再花1000镑雇用20个工人在下一年内完成或改进这种商品,在两年末了,我把商品运到市场上去。如果利润为10%,我的商品就必须卖2310镑,因为我在第一年用了1000镑资本,第二年用了2100镑资本。另一个人使用完全相同的劳动量,但是全部用在第一年;他花2000镑雇用40个工人,在第一年末,就把他的商品按10%的利润出卖,也就是卖2200镑。于是,这里就有两种商品,所耗费的劳动量完全相同,其中一种卖2310镑,另一种卖2200镑。”(第34页)}\end{quote}

[531]但是,这种差别——不论是固定资本耐久程度的差别,还是流动资本流通时间的差别,或者是两种资本结合比例的差别,最后,或者是花费了同量劳动生产出来的不同商品[在进入市场以前]所需要的时间的差别,——究竟怎样引起这些商品的相对价值的变动呢?李嘉图起初说,这是因为:

\begin{quote}{“……这种差别,和……比例的这种多样性,在生产商品所必需的劳动量增减之外,又给商品相对价值的变动提供了另一个原因,这个原因就是劳动价值的提高或降低。”(第25—26页)}\end{quote}

怎样证明这一点呢?

\begin{quote}{“工资的提高,对于在如此不同的情况下生产出来的商品,不能不产生不同的影响”,(第27页)}\end{quote}

这里指的就是这样一种情况:在不同行业中使用同样大小的资本,一笔资本主要由固定资本组成,只有很小一部分由“用于维持劳动的”资本组成,而另一笔资本情况恰好相反。首先,说对“商品”产生影响是荒谬的。李嘉图指的是商品的价值。但是,价值在多大程度上受这种情况的影响呢?根本不受影响。在两种情况下,受影响的是利润。例如,一个人只把1/5的资本用作可变资本,在工资相等和剩余劳动率相等的条件下,如果剩余价值率等于20%,用100只能生产剩余价值4;而另一个人把4/5的资本用作可变资本,就会生产剩余价值16。因为在第一种场合,花在工资上的资本等于100/5,也就是20,20的5/1(或20%)等于4。在第二种场合,花在工资上的资本等于4/5×100,也就是80,80的1/5(或20%)等于16。在第一种场合利润等于4,在第二种场合利润等于16。两笔资本的平均利润是(16+4)/2,即20/2,也就是10%。李嘉图所说的那种情况其实就是这样。所以,如果两个企业主都按费用价格出卖商品,——李嘉图就是这样假定的,——他们每人的商品就都卖110。现在假定,工资比以前提高比方说20%。以前一个工人花费1镑,现在花费1镑4先令,或24先令。第一个企业主和以前一样把80镑用作不变资本(因为李嘉图在这里把劳动材料撇开,所以我们也可以这样做),对于他所使用的20个工人,除20镑以外,他还要多付80先令即4镑。因此,他的资本现在应为104镑。在110镑中留给他的利润只有6镑,因为工人提供的剩余价值不是多了,而是少了。6镑比104,得[5+(10/13)]%。相反,另一个企业主使用80个工人,就要多付320先令即16镑。因此他必须花费116镑。如果他不得不按110镑出卖商品,他就不但得不到盈利,反而要亏损6镑。但是,这种情况之所以发生,只是因为平均利润已经改变了这个企业主花在劳动上的费用和他自己的企业所提供的剩余价值之间的比例。

这样一来,李嘉图没有去研究重要问题,就是究竟发生了什么变动,才使得一个把100镑中的80镑花在工资上的企业主没有得到四倍于另一个把100镑中的20镑花在工资上的企业主的利润;他却去研究一个次要问题,即下述情况是怎样产生的:在这个大差别拉平之后,也就是说在利润率既定时,这种利润率的任何变动,比如由于工资的提高,对于用100镑资本使用许多工人的企业主的影响,比对于用100镑资本使用很少工人的企业主的影响要大得多,因此,——在利润率相等的条件下,——前者的商品价格或费用价格必须上涨,后者的商品价格或费用价格必须下降,才能使利润率继续保持相等。

虽然李嘉图最初就告诉我们,“相对价值”的全部变动必定起因于“劳动价值的提高”,但是他提出的第一个例证,却同这个原因绝对没有关系。这个例证如下:

\begin{quote}{“假定有两个人各自雇用100个工人在一年内制造两台机器,另外有一个人雇用同样数目的工人种植谷物,年终,每台机器的价值将和谷物的价值相等,因为所有这三种商品都是由同量劳动生产出来的。假定一台机器的所有者在下一年雇100个工人用这台机器织造呢绒;另一台机器的所有者也雇100个工人用他的机器织造棉布,而租地农场主和以前一样继续雇用100个工人种植谷物。第二年他们都使用同量劳动}\end{quote}

{就是说,他们在工资上花费同量资本,但决不是使用同量劳动},

\begin{quote}{但是,毛织厂主的产品加上他的机器[532]和棉织厂主的产品加上他的机器,同样都是200个工人劳动一年的结果,或者更确切地说,是100个工人劳动两年的结果,而谷物则是100个工人劳动一年的结果。所以,如果谷物的价值是500镑,那末,毛织厂主的机器和呢绒加在一起的价值就应该是1000镑,棉织厂主的机器和棉布的价值也应该是谷物价值的两倍。但是,它们的价值将不止是谷物价值的两倍,因为毛织厂主和棉织厂主的资本已经加上了这些资本在第一年的利润,而租地农场主却把利润花费和享受掉了。因此,由于他们的资本的耐久程度不同,或者也可以说,由于一批商品在能够进入市场以前必须经历的时间不同,这些商品的价值同花费在它们上面的劳动量不会恰好成比例。在这种情况下,商品价值的比例不是2∶1,而是稍大一些,以便补偿价值较大的一种商品在能够进入市场以前必须经历的较长时间。假定为了购买每个工人的劳动每年付出50镑,或者说,使用资本5000镑,利润是10%,那末,在第一年末,每台机器的价值同谷物的价值一样都是5500镑。第二年,工厂主和租地农场主各自再花5000镑来维持劳动,因而他们的产品将仍然卖5500镑。但是,使用机器的两个工厂主要与租地农场主处于平等地位,就不仅必须为花费在劳动上的等量资本5000镑取得5500镑,而且必须取得一个追加额550镑作为他们投在机器上的5500镑的利润,因此〈就是因为相等的10%的年利润率已经被假定为一种必然和规律〉,他们的产品必须卖6050镑。”}\end{quote}

{这样一来,由于平均利润——由于李嘉图预先假定的一般利润率,——就产生了不同于商品价值的平均价格,或者说,费用价格。}

\begin{quote}{“因此,这里我们看到,虽然资本家们每年正好使用同量的劳动来生产他们的商品,但是,由于他们各自使用的固定资本,或者说,积累劳动的量不同,他们生产出来的商品的价值是各不相同的。”}\end{quote}

{不是由于这一点,而是由于这两个无赖都有一个固定观念,认为他们两个由于“维持了劳动”都应该获得同量的赃物,或者说,不管他们的商品各自具有多少价值,都必须按照平均价格出卖,使他们每个人都得到同样的利润率。}

\begin{quote}{“呢绒和棉布具有相同的价值,因为它们是同量劳动和同量固定资本的产品。但是谷物和这些商品不具有相同的价值{应该说:费用价格},因为就固定资本来说,谷物是在不同的条件下生产出来的。”(第29—31页)}\end{quote}

李嘉图举出这个极其笨拙而难懂的例子来说明极其简单的事情,就是不想简单地说:因为等量的资本,不管其有机部分的比例如何,或者不管其流通时间如何,都提供等量的利润,——如果商品按其价值出卖,就不可能如此,——所以,有一种不同于这些价值的商品费用价格存在。而且这一点已经包含在一般利润率的概念中了。

我们就来分析一下这个复杂的例子,并且把它还原为它本来的毫不“复杂”的样子。为了这个目的,让我们从结尾开始,同时为了把问题理解得更清楚,让我们事先指出,根据李嘉图的“假定”,租地农场主和经营棉布业的家伙并不为原料花费什么,其次,租地农场主在劳动工具上也不花费任何资本,最后,经营棉布业的畜生投入的固定资本的任何部分都不作为损耗加入他的产品。所有这些假定虽然是荒谬的,但是它们本身丝毫无损于这个例证。

在所有这些假定之下,李嘉图的例子如果从结尾开始,就是这样:租地农场主在工资上花费5000镑;经营棉布业的坏蛋在工资上花费5000镑,在机器上花费5500镑。所以前者花费5000镑,而后者花费10500镑,也就是[533]比前者多一倍。假定两人都要赚10%的利润,那末,租地农场主的商品就必须卖5500镑,经营棉布业的家伙的商品就必须卖6050镑(因为已经假定,投在机器上的5500镑中任何部分都不作为机器的损耗构成产品价值的组成部分)。这只能说明,商品的费用价格只要决定于商品所包含的预付的价值加上同一个年利润率,它就不同于商品的价值,而这个差别的产生是由于商品按照给预付资本提供同一利润率的价格出卖;简单地说,费用价格和价值之间的这个差别,同一般利润率是一回事。除此以外,李嘉图究竟还要说明什么,实在无法理解。连他这里硬加进来的固定资本和流动资本的差别,在这个例子里也纯粹是胡扯。因为,比如说,如果棉织厂主多花的5500镑由原料组成,而租地农场主则不需要种子等等,得出来的结果还是完全一样。

这个例子也没有证明李嘉图所说的:

\begin{quote}{“由于他们〈棉织厂主和租地农场主〉各自使用的固定资本,或者说,积累劳动的量不同,他们生产出来的商品的价值是各不相同的。”(第31页)}\end{quote}

因为根据李嘉图的假定,棉织厂主的固定资本等于5500镑,租地农场主的固定资本等于零;前者使用固定资本,后者不使用固定资本。因此,决不是他们使用的固定资本的“量不同”,正如一个人吃肉,一个人不吃肉,我们不能说他们吃的肉的“量不同”一样。相反,说棉织厂主和租地农场主使用的“积累劳动”即物化劳动的“量不同”,就是说,前者使用10500镑,后者只使用5000镑,倒是正确的(虽然用一个“或者说”把“积累劳动”一词偷偷塞进来是完全错误的)。但是,说他们使用的“积累劳动的量不同”,无非是说:他们投入自己企业的“资本的量不同”;利润量取决于他们所使用的资本量的差别,因为已经假定利润率是相同的;最后,和资本量成比例的利润量的这种差别,表现在商品各自的费用价格上。

但是,李嘉图例证的笨拙是从哪里来的呢?

\begin{quote}{“因此,这里我们看到,虽然资本家们每年正好使用同量的劳动来生产他们的商品,但是……他们生产出来的商品的价值是各不相同的。”(第30—31页)}\end{quote}

就是说,他们不是一般地使用同量的劳动——直接劳动和积累劳动加在一起,而是使用同量的可变资本,花费在工资上的资本,使用同量的活劳动。而且,因为货币同积累劳动也就是同以机器等形式存在的商品只能按照商品交换的规律相交换,因为剩余价值只是由于一部分被使用的活劳动被无偿占有才会产生,所以,很明显(因为根据假定,机器的任何部分都不作为损耗加入商品),只有在利润和剩余价值是等同的时候,两个资本家才能获得同量的利润。棉织厂主虽然花费两倍以上的资本,但是他必须同租地农场主一样,按5500镑出卖他的商品。即使机器全部加入商品价值,他的商品也只能卖11000镑,就是说,他获得的利润还不到5%,而租地农场主获得的却是10%。但是,假定租地农场主获得的10%代表他的商品中包含的实际无酬劳动,那末,尽管租地农场主和工厂主获得的利润不等,他们的商品倒是按其价值出卖的。这就是说,如果他们出卖商品获得同量利润,那末,就是下述情况二者必居其一:或者是工厂主随意在他的商品上附加了5%,在这种情况下,工厂主和租地农场主的商品总起来看就是高于其价值出卖;或者是租地农场主赚得的实际剩余价值大约为15%,他们两人用自己的商品获得10%的平均利润。在这后一种情况下,虽然他们各自的商品的费用价格每一次不是高于商品的价值就是低于商品的价值,但商品总额则按其价值出卖,而利润的平均化本身决定于商品中包含的剩余价值总额。这里,上面所引的李嘉图那句话,如果适当地修改一下,就包含下面这个正确的思想:在花费同量资本的情况下,可变资本和不变资本的比例不同,生产出来的商品就必然具有不同的价值,因而必然提供不同的利润;因此,这些利润的平均化必然产生不同于商品价值的费用价格。“因此,这里我们看到,虽然资本家们每年正好使用同量的〈直接的、活的〉劳动来生产他们的商品,但是,由于他们各自使用的……积累劳动的量不同,他们生产出来的商品的价值是各不相同的〈这就是说,商品具有不同于其价值的费用价格〉。”但是这种猜想在李嘉图那里是没有说出来的。它只不过说明他举的这个例证是自相矛盾和显然错误的,这个例证直到现在同“资本家使用的固定资本的量不同”毫无关系。

现在我们继续往下分析。工厂主第一年雇用100个工人制造一台机器,在这个时间内租地农场主也雇用100个工人生产谷物。第二年工厂主用机器来织布,为此又雇用100个工人。而租地农场主又雇用100个工人来种植谷物。李嘉图说,假定谷物的价值每年为500镑。我们假定,其中无酬劳动为有酬劳动的25%,也就是每400单位有酬劳动提供100单位无酬劳动。那末,机器在第一年末的价值也是500镑,其中400镑为有酬劳动,100镑为无酬劳动。我们还要假定,[534]机器在第二年末全部损耗完,加入棉布的价值。实际上李嘉图就是这样假定的,因为在第二年末他拿来同“谷物的价值”比较的不只是棉布的价值,而是“棉布和机器的价值”。

好极了!在这种情况下,第二年末棉布的价值必定等于1000镑,也就是500镑为机器的价值,500镑为劳动新加的价值。相反,谷物的价值等于500镑,也就是400镑为工资,100镑为无酬劳动。到此为止,这个例子中还没有什么同价值规律矛盾的东西。棉织厂主同谷物种植业者完全一样,赚得25%的利润,但是前者的商品等于1000镑,后者的商品等于500镑,因为前者的商品包含200个工人的劳动,而后者的商品每年只包含100个工人的劳动。其次,棉织厂主第一年由于制造机器(机器无偿地吸收了制造机器的工人的1/5的劳动时间)而赚得的100镑利润(剩余价值),要到第二年才得以实现,因为只有到那时他才在棉布的价值中同时实现机器的价值。但是难题也就发生在这里。棉织厂主出卖商品,价格高于1000镑,也就是高于他的商品所包含的价值,而租地农场主按500镑出卖谷物,也就是根据假定按谷物的价值出卖。因此,如果只有这两个人进行交换,工厂主从租地农场主那里换得谷物,租地农场主从工厂主那里换得棉布,那就好比租地农场主低于商品的价值出卖商品,赚得的利润少于25%,工厂主则高于棉布的价值出卖棉布。让我们把李嘉图多余地塞进他的例子的两个资本家(毛织厂主和棉织厂主)撇开不谈,把他的例子改变一下,只讲棉织厂主。对我们所考察的例证来说,到现在为止这种双重计算是毫无用处的。所以:

\begin{quote}{“但是,它们〈棉布〉的价值将不止是谷物价值的两倍,因为……棉织厂主的资本已经加上了这个资本在第一年的利润,而租地农场主却把利润花费和享受掉了。”}\end{quote}

(最后这一句资产阶级的粉饰话这里在理论上是毫无意义的。道德方面的考虑同这里所考察的问题没有任何关系。)

\begin{quote}{“因此,由于他们的资本的耐久程度不同,或者也可以说,由于一批商品在能够进入市场以前必须经历的时间不同,这些商品的价值同花费在它们上面的劳动量不会恰好成比例。在这种情况下,商品价值的比例不是2∶1,而是稍大一些,以便补偿价值较大的一种商品在能够进入市场以前必须经历的较长时间。”(第30页)}\end{quote}

如果工厂主按商品的价值出卖商品,他就会把它卖1000镑,比谷物贵一倍,因为这个商品包含的劳动多一倍:500镑是机器形式的积累劳动(其中有100镑工厂主没有支付过代价),另外500镑是织布劳动,其中又有100镑工厂主没有支付过代价。但工厂主是这样计算的:第一年我花费400镑,从而通过剥削工人,我制成了一台价值500镑的机器。因此,我赚得了25%的利润。第二年我花费900镑,即500镑为上述机器,又有400镑为劳动。为了再赚得25%的利润,我就必须按1125镑出卖棉布,也就是高于它的价值125镑。因为这125镑并不代表棉布中包含的劳动,既不代表第一年的积累劳动,也不代表第二年的新加劳动。棉布中包含的全部劳动量只等于1000镑。另一方面,假定他们两人互相交换自己的商品,或者说,资本家中有半数处于棉织厂主的地位,另有半数处于租地农场主的地位。前一半资本家从哪里获得必须支付给他们的这125镑呢?从什么基金呢?显然,只有从后一半资本家那里获得。但是这样一来,后一半资本家显然就会得不到25%的利润。因而,前一半资本家以一般利润率为借口欺骗了后一半资本家,而实际上,工厂主的利润率是25%,租地农场主的利润率却低于25%。所以,发生的必定是另一种情况。

为了使这个例证更加正确和更加明显起见,我们假定租地农场主在第二年花费900镑。这样,在利润率为25%的情况下,他在第一年从他所花费的400镑赚得100镑利润,第二年赚得225镑,合计325镑。相反,工厂主在第一年从400镑赚得25%的利润,但在第二年从900镑只赚得100镑(因为投在机器上的500镑是不提供剩余价值的,只有花在工资上的那400镑才提供剩余价值),只占[11+(1/9)]%。或者,让租地农场主再花费400镑;这样他在第一年和第二年都赚得25%;两年合计,从花费的800镑赚得200镑,就是说,也是25%。相反,工厂主在第一年赚得25%,第二年赚得[11+(1/9)]%,两年合计,从花费的1300镑赚得200镑,也就是[15+(5/13)]%。所以,在利润平均化的时候,工厂主将赚得利润[20+(5/26)]%,租地农场主赚得的也将这么多。\endnote{在租地农场主和工厂主所花资本相等的情况下,平均利润是[20+(5/26)]%。如果考虑到所花资本的量的不同——租地农场主800镑,工厂主1300镑(总共2100镑),那末,在两者的总利润等于400镑的情况下,平均利润是(400×100)/2100=[19+(1/21)]%。——第206页。}这就是平均利润。这样一来,[在第二年]租地农场主的商品的费用价格就会低于500镑,而工厂主的商品的费用价格却高于1000镑。

[535]无论如何,工厂主在这里第一年花费400镑,第二年花费900镑,而租地农场主每次都只花费400镑。假如工厂主不是生产棉布而是建造房屋(如果他是个建筑业者),那末,第一年末在未建成的房子中包含500镑,他还要在劳动上再花400镑,才能把房子建成。租地农场主的资本一年周转一次,他可以从100镑利润中提取一部分,比如说,50镑,再作为资本,重新花在劳动上,这一点工厂主在上述假定的情况下是做不到的。为了使利润率在两种情况下都一样,一个人的商品就必须高于其价值出卖,而另一个人的商品则必须低于其价值出卖。因为竞争力求把价值平均化为费用价格,所以,实际上发生的也就是这种情况。

但是,李嘉图说,相对价值在这里所以发生变动,是“由于资本的耐久程度不同”,或者说,“由于一批商品在能够进入市场以前必须经历的时间不同”,那是错误的。相反,正是预先假定的一般利润率,不顾流通过程所决定的价值差别,造成相等的、和这些仅仅由劳动时间决定的价值不同的费用价格。

李嘉图的例证分为两个例子。在后一个例子中,资本的耐久程度,或者说,资本作为固定资本的性质,完全没有包括进来。谈的只是几笔资本大小不等,但花费在工资上的资本量相等,花费的可变资本相等,而利润应该相等,虽然剩余价值和价值必然不等。

在第一个例子中,资本的耐久程度也没有包括进来。谈的是较长的劳动过程,是商品在能够进入流通之前,直到它制成为止,有较长时间停留在生产领域。这里,在李嘉图那里,工厂主第二年使用的资本也比租地农场主使用的大,虽然他在两年内花费的可变资本和租地农场主一样。但是,租地农场主由于他的商品停留在劳动过程中的时间较短,由于商品转化为货币较早,第二年可以使用较大的可变资本。此外,利润中作为收入来消费的部分,在租地农场主那里,第一年末就可以消费,在工厂主那里,要到第二年末才可以消费。因此,工厂主必须支出额外的资本来维持自己的生活,为自己预付生活费用。其实,这里这一切都取决于在一年内周转的资本在多大程度上把自己的利润再资本化,因而,取决于生产出来的利润的实际量,以便第二种情况可以得到弥补,利润可以平均化。在什么都没有的地方,也就没有什么可以平均化。这里,各个资本生产的价值,因而它们生产的剩余价值,因而它们生产的利润,又不是同资本量成比例的。要使它们成比例,就必须有不同于价值的费用价格存在。

李嘉图还提出了第三个例证,但这个例证同第一个例证中的第一个例子完全一致,连一个新鲜的字都没有。

\begin{quote}{“假定我花1000镑雇用20个工人在一年内生产一种商品,年终,再花1000镑雇用20个工人在下一年内完成或改进这种商品,在两年末了,我把商品运到市场上去。如果利润为10%,我的商品就必须卖2310镑,因为我在第一年用了1000镑资本,第二年用了2100镑资本。另一个人使用完全相同的劳动量,但是全部用在第一年;他花2000镑雇用40个工人,在第一年末,就把他的商品按10%的利润出卖,也就是卖2200镑。于是,这里就有两种商品,所耗费的劳动量完全相同,其中一种卖2310镑,另一种卖2200镑。这个例子表面上和前一个不同,但实际上是相同的。”(第34—35页)}\end{quote}

这个例子和前一个不仅“实际上”相同,而且“表面上”也相同,只有一点不同,就是在前一个例子中,商品叫做“机器”,在这里,直接就叫“商品”。在第一个例子中,工厂主第一年花费400镑,第二年花费900镑,这一次,第一年花费1000镑,第二年花费2100镑。在前一个例子中,租地农场主第一年花费400镑,第二年也花费400镑。这一次,第二个企业主第一年花费2000镑,第二年什么也不花费。这就是全部差别。但是两个例子的《fabuladocet》\fnote{直译是:“寓言教导说”;转意是:由某事物得出的(劝谕性的)结论,“寓意”。——编者注}都在于:企业主中有一个人第二年花费了第一年的全部产品(包括剩余价值在内)加上一个追加额。

这些例子的笨拙,表明李嘉图正在攻一个难关,这个难关,他自己不清楚,更说不上把它攻破了。笨拙之处在于:第一个例证的第一个例子应该把资本的耐久程度包括进来;但结果完全不是这样;李嘉图自己使这样做成为不可能,因为他不让固定资本的任何部分作为损耗加入商品的价值,也就是恰恰漏掉了表现固定资本所特有的流通方式的因素。李嘉图表明的只是,由于劳动过程经历的时间较长,同劳动过程经历的时间较短的地方相比,要使用更大的资本。第三个例子应该说明与此不同的情况,但是实际上说明的是一回事。第一个[536]例证的第二个例子应该表明,由于固定资本的比例不同会造成什么样的差别。它不是这样,而是仅仅表明,大小不等但花在工资上的部分相等的两笔资本会有什么样的差别。并且在这里,工厂主没有棉花和棉纱,租地农场主没有种子和工具就能进行活动!这个例证之所以必然站不住脚,甚至荒诞无稽,就是因为它内在地是含糊不清的。

\tsubsubsectionnonum{[(b)李嘉图把费用价格同价值混淆起来,由此产生了他的价值理论中的矛盾。他不懂利润率平均化和价值转化为费用价格的过程]}

最后,李嘉图说出了所有这些例证的实际结论:

\begin{quote}{“在这两种情况下,价值的差额都是由于利润作为资本积累起来而造成的,这个差额只不过是对利润被扣留的那段时间的一种公正的补偿〈好象这里的问题在于公正〉。”(第35页)}\end{quote}

这无非是说,一笔资本,不管它的特殊流通时间如何,也完全不管等量资本在不同生产部门中由于资本的有机组成部分的比例不同(撇开流通过程不说)必然生产出不同的剩余价值,在一定的流通时间内,比如说在一年内,必定提供10%。

李嘉图本应从自己的例证中作出如下的结论:

[第一,]等量资本生产的商品价值不等,从而提供的剩余价值或利润也不等,因为价值决定于劳动时间,而一笔资本所实现的劳动时间量,不取决于资本的绝对量,而取决于可变资本量即花费在工资上的资本量。第二,即使假定等量资本生产的价值相等(虽然在大多数情况下,生产领域中的不等是同流通领域中的不等相一致的),等量资本占有同量无酬劳动并把它转化为货币所需要的那段时间,也还是由于资本的流通过程不同而有所不同。这就使等量资本在不同生产部门中在一定时间内必须提供的价值、剩余价值和利润产生了第二个差别。

因此,如果利润按其比如说在一年内对资本的百分率计算必须相等,从而等量资本在同一时间内提供的利润必须相等,那末,商品的价格必然不同于商品的价值。一切商品的这些费用价格加在一起,其总和将等于这一切商品的价值。同样,总利润将等于这些资本加在一起比如说在一年内提供的总剩余价值。如果我们不以价值规定为基础,那末,平均利润,从而费用价格,就都成了纯粹想象的、没有依据的东西。各个不同生产部门的剩余价值的平均化丝毫不改变这个总剩余价值的绝对量,它所改变的只是剩余价值在不同生产部门中的分配。但是,这个剩余价值本身的规定,只有来自价值决定于劳动时间这一规定。没有这一规定,平均利润就是无中生有的平均,就是纯粹的幻想。那样的话,平均利润就既可以是10%,也可以是1000%。

李嘉图的一切例证只有一个用处,就是帮助他偷偷地把一般利润率作为前提引进来。这在第一章(《论价值》)就发生了,而表面上,李嘉图是在第五章才考察工资,在第六章才考察利润。怎样单纯从商品的“价值”规定得出商品所包含的剩余价值、利润、甚至一般利润率,——这一点对李嘉图来说仍然是一个秘密。他在上述例证中实际证明的唯一东西是:商品的价格只要决定于一般利润率,它就根本不同于商品的价值。他所以会看出这个差别,是因为他预先就把利润率当作规律来假定。我们看到,如果说人们责备李嘉图过于抽象,那末相反的责备倒是公正的,这就是:他缺乏抽象力,他在考察商品价值时无法忘掉利润这个从竞争领域来到他面前的事实。

因为李嘉图不是从价值规定本身出发来阐述费用价格和价值的差别,而是承认那些与劳动时间无关的影响决定“价值”本身(这里他如果坚持“绝对价值”,或者说,“实际价值”,或者直接说,“价值”这样的概念,倒是合适的)并且有时使价值规律失效,所以他的反对者如马尔萨斯之流就抓住这一点来攻击他的全部[537]价值理论。在这里马尔萨斯正确地指出,不同部门中资本有机组成部分之间的差别和资本周转时间的差别是随着生产的进步而发展的,结果就必然要得出亚·斯密的观点,认为价值决定于劳动时间这一规定不再适用于“文明”时代了。(并见托伦斯。)另一方面,李嘉图的门徒为了使这些现象符合于基本原则,就求助于最可怜的烦琐哲学的臆造(见[詹姆斯·]穆勒和可怜的饶舌家麦克库洛赫)\endnote{关于马尔萨斯、托伦斯、詹姆斯·穆勒和麦克库洛赫的观点,见本卷第3册有关章节。——第211页。}。

李嘉图不去研究从他自己的例证中得出的结论,——就是完全不管工资提高还是降低,假定工资不变,商品的费用价格如果由同一个利润百分率决定,就必然不同于商品的价值,——却在这一节里转而考察工资的提高或降低对那已经由价值平均化而成的费用价格所产生的影响。

问题的实质本身是非常简单的。

租地农场主花费5000镑,获得利润10%,他的商品的货币价格为5500镑。如果由于工资提高并引起利润下降,利润下降1%,即从10%下降到9%,那末,租地农场主仍然会按5500镑出卖自己的商品(因为假定他已把自己的全部资本花在工资上)。但是,在这5500镑中,他的利润已经不是500,而只是454+(14/109)。工厂主的资本包括用于机器的5500镑和用于劳动的5000镑。后面这5000镑仍旧表现为5500镑,不过他现在花费的不是5000镑,而是5045+(95/109)镑,他和租地农场主一样,从这笔资本只赚得利润454+(14/109)镑。而且这5500镑固定资本,他不能再按10%来计算利润,即550镑,只能按9%来计算利润,即495镑。因此,他的商品将不是卖6050镑,而是卖5995镑。这样一来,工资提高的结果,租地农场主的商品的货币价格仍旧不变,而工厂主的商品的货币价格却下跌了;因此,同工厂主的商品价值相比,租地农场主的商品价值提高了。全部关键在于:如果工厂主按以前的价值出卖自己的商品,他赚得的利润就会高于平均利润,因为工资的提高只是直接影响到花在工资上的那部分资本。在这个例证中,已经预先假定了由10%的平均利润调节的、不同于商品价值的费用价格。李嘉图的问题是:利润的提高或降低如何根据固定资本和流动资本在全部资本中的不同比例来影响费用价格。这个例证(李嘉图的著作第31—32页)同价值转化为费用价格这一根本问题毫无关系。但这个例证还是不错的,因为李嘉图在这里一般指出了:在资本构成相同的条件下,工资的提高只会引起利润的降低,不会影响商品的价值(这和庸俗观点相反),在资本构成不同的条件下,它只会引起某些商品价格的下跌,而不象庸俗见解所认为的那样,会引起一切商品价格的上涨。在李嘉图的例子中,商品价格的下跌是利润率降低的结果,或者[在李嘉图看来]也可以说,是工资提高的结果。在工厂主的例子中,商品的费用价格的很大一部分决定于他按固定资本计算的平均利润。因此,如果由于工资的提高或降低,这种利润率会降低或提高,那末,这些商品的价格就会相应地(同那部分由于按固定资本计算的利润而产生的价格相应地)下跌或上涨。这一点也适用于“要经过较长时期才流回的流动资本以及相反的情况”。(麦克库洛赫)[《政治经济学原理》1825年爱丁堡版第300页]如果使用较少可变资本的资本家继续按原来的利润率把自己的固定资本计算到商品的价格中去,那末他们的利润率就会提高,而且,同那些有较大部分资本由可变资本构成的资本家相比,他们使用的固定资本越多,他们的利润率就提得越高。竞争会把这种情况拉平。

\begin{quote}{饶舌家麦克说:“李嘉图是第一个研究在生产商品所使用的资本具有不同耐久程度时工资的波动对商品价值的影响的人。李嘉图不仅指出,工资的提高不可能使一切商品的价格都上涨,而且指出,在许多情况下,工资的提高必然引起价格的下跌,而工资的降低必然引起价格的上涨。”(麦克库洛赫《政治经济学原理》1825年爱丁堡版第298—299页)}\end{quote}

李嘉图证明其论点的办法是,第一,假定有一种由一般利润率调节的费用价格。

第二,指出:“劳动的价值提高,利润就不能不降低。”(第31页)

可见,在第一章(《论价值》)中已经假定了第五章(《论工资》)和第六章(《论利润》)中应该从《论价值》那一章引伸出来的那些规律。顺便指出,[538]李嘉图作出了完全错误的结论,他说,因为“劳动的价值提高,利润就不能不降低”,所以,利润提高,劳动的价值就不能不降低。第一个规律与剩余价值有关。但是,因为利润是剩余价值同全部预付资本之比,所以,在劳动价值不变的条件下,如果不变资本的价值降低,利润就可能上涨。李嘉图根本混淆了剩余价值和利润。由此他就得出了关于利润和利润率的错误规律。

最后这个例证的一般结论是这样的:

\begin{quote}{“由劳动价值的提高或降低〈或者也可以说,由利润率的降低或提高〉引起的商品相对价值的变动的幅度,将取决于固定资本在已花费的全部资本中所占的比例。一切用很贵的机器或在很贵的建筑物里生产的,或者在能够进入市场以前必须经历长时间的商品的相对价值会降低,而一切主要由劳动生产的,或能迅速进入市场的商品的相对价值则会提高。”(第32页)}\end{quote}

李嘉图又回到这一研究中真正唯一使他关心的问题上来了。他说,由工资的提高或降低引起的商品费用价格的这些变动,同由商品价值的变动{李嘉图远远不会用这些恰当的术语来表达这一事实真相}即由生产商品所使用的劳动量的变动引起的商品费用价格的变动相比,是微不足道的。因此,他说,完全可以把这一点“撇开不谈”,这样,价值规律就是在实际上也仍然是正确的。(李嘉图本应补充一句:没有决定于劳动时间的价值,费用价格本身就仍然是无法解释的。)这就是李嘉图的研究的真正进程。事实上,很明显,尽管商品的价值转化为费用价格,——既然我们已经假定费用价格存在——{应当把这种费用价格同市场价格区别开来;费用价格是不同部门的商品的平均市场价格。因为同一生产领域的商品的市场价格决定于这一领域中等的、平均的生产条件下生产出来的商品的价格,所以市场价格本身就包含着一个平均数。市场价格决不是象李嘉图在考察地租时假定的那样,决定于最坏条件下生产出来的商品的价格。因为平均需求取决于一定的价格,甚至在谷物上也是这样。因此,进入市场的一定量商品不会高于这个价格出卖。否则需求就会下降。所以,不是在平均条件下而是在低于平均条件的情况下生产商品的人,往往只得不仅低于商品的价值,而且低于商品的费用价格出卖自己的商品},只要费用价格的变动不是由利润率的持续降低或提高,不是由经过多年才能确定的利润率的持续变动引起,这种变动就只能仅仅归因于这些商品的价值的变动,归因于生产这些商品所必需的劳动时间量的变动。

\begin{quote}{“但是,读者应当注意,商品的变动〈即商品费用价格的变动,或者照李嘉图的说法,商品相对价值的变动〉的这一原因所产生的影响是比较小的……商品价值变动的另一重要原因,即生产商品所必需的劳动量的增减,情况却不是这样……持久的利润率的任何大变动,都是经过多年才发生影响的那些原因所造成的结果,而生产商品所必需的劳动量的变动却是天天都发生的。在机器、工具、建筑物以及开采或种植原料方面的每一改良都可以节省劳动,使我们在利用这种改良来生产商品时,能够更加容易地把商品生产出来,结果商品的价值就会发生变动。可见,在研究商品价值变动的原因时,虽然完全不考虑劳动价值的提高或降低所产生的影响是错误的,但认为这种影响具有很大意义也是不正确的。”(第32—33页)}\end{quote}

所以,李嘉图就把劳动价值的变动撇开不谈。

第一章《论价值》第四节全节都非常混乱,因此,虽然李嘉图在开头一段就声称,他要考察由于资本构成不同工资的提高或降低引起的商品价值变动的影响,其实他只是顺便举例证明这一点,相反,第四节的主要部分实际上是罗列了许多例证来证明:完全不管工资提高或降低——在他自己假定的工资不变的条件下——而且甚至不管固定资本和流动资本的比例不同,既然假定[539]一般利润率存在,就必然得出不同于商品价值的费用价格。他在这一节的结尾又把这一点忘记了。

他在第四节用这样的话预告他要研究的问题:

\begin{quote}{“固定资本耐久程度的这种差别,和这两种资本可能结合的比例的这种多样性,在生产商品所必需的劳动量增减之外,又给商品相对价值的变动提供了另一个原因,这个原因就是劳动价值的提高或降低。”(第25—26页)}\end{quote}

实际上,他用自己的例证首先证明的是:只有一般利润率才能使两种资本(即可变资本和不变资本)的不同结合产生这种使商品价格不同于商品价值的影响;因此,这些变动的原因,正是一般利润率,而不是劳动价值,劳动价值在这里假定是不变的。然后,第二步,他才假定有一种因一般利润率的存在而已经不同于价值的费用价格,并研究劳动价值的变动怎样影响费用价格。第一个主要问题,他不研究,他忘得一干二净,他在这一节结尾说的还是这一节开头所说的话:

\begin{quote}{“本节已经证明,在劳动量没有任何变动的条件下,单是劳动价值提高,就会使那些在生产时使用固定资本的商品的交换价值降低;固定资本量越大,降低的幅度也越大。”(第35页)}\end{quote}

在下一节即第五节(第一章),他继续沿着这条线走下去,就是说,他仅仅研究:如果在两个不同生产部门中有两笔等量资本,不是它们的固定资本和流动资本的比例不同,而是“固定资本耐久程度不同”,或者说,“资本流回所有者手里的速度不同”,那末,由于劳动价值或者说工资的变动,商品的费用价格会发生什么变动。在第四节,对于由于一般利润率而产生的费用价格和价值之间的差别,还有一点正确的猜想,在第五节,却再也看不到了。考察的只是关于费用价格本身的变动的次要问题。因此,这一节除了偶然触及的从流通过程产生的资本形式差别的问题以外,实际上几乎引不起什么理论兴趣。

\begin{quote}{“固定资本的耐久程度越低,它在性质上就越接近于流动资本。它将在较短时间内被消费掉,它的价值也将在较短时间内被再生产出来,以便保持工厂主的资本。”(第36页)}\end{quote}

可见,李嘉图把资本的较低耐久程度以及一般说来固定资本和流动资本的差别都归结为再生产时间的差别。这无疑是一个极为重要的规定,但决不是唯一的规定。固定资本全部加入劳动过程,但只是陆续地、一部分一部分地加入价值形成过程。这是固定资本和流动资本的流通形式的另一个主要差别。其次,固定资本必然只以其交换价值加入流通过程,而它的使用价值则在劳动过程中消费掉,从来不离开劳动过程。这是流通形式的又一个重要差别。流通形式的这两个差别也同流通时间有关,但它们与[资本的耐久]程度和流通时间的差别决不是等同的。

耐久程度较低的资本需要较多的经常劳动,

\begin{quote}{“以保持其原来的有效状态,但这样使用的劳动可以看作是实际花费在成品上的劳动,因此,这种成品必然具有同这种劳动成比例的价值”。(第36—37页)“如果机器的损耗大,为维持其有效状态所需要的劳动量为每年50个工人,那末,我就要求为我的商品提供追加价格,其数额等于雇用50个工人来生产其他商品而完全不使用机器的其他任何工厂主所得的价格。但是,工资的提高,对于用损耗得快的机器生产的商品和用损耗得慢的机器生产的商品,影响是不同的。在生产前一种商品时,有大量劳动会不断转移到所生产的商品上去}\end{quote}

{但是,李嘉图以他的一般利润率为前提,看不到同时也有相对大量的剩余劳动会不断转移到商品上去},

\begin{quote}{而在生产后一种商品时,这样转移的劳动却很少}\end{quote}

{因此,剩余劳动也很少,就是说,如果商品按其价值交换,[剩余]价值会少得多}。

\begin{quote}{因此,只要工资有所提高,或者也可以说,[540]只要利润有所降低,那些用耐久程度较高的资本生产的商品的相对价值就会降低,而那些用损耗较快的资本生产的商品的价值则会相应地提高。工资下降的作用则恰好相反。”(第37—38页)}\end{quote}

换句话说,同使用耐久程度较高的资本的工厂主相比,使用耐久程度较低的固定资本的工厂主,使用的固定资本较少,而花在工资上的资本较多。因此,这个例子同前面所说的那个例子是一致的,前面说的是,如果一笔资本比另一笔资本使用的固定资本相对地即在比例上较多,工资的变动对这两笔资本会发生什么影响。这里没有什么新东西。

李嘉图还谈到机器(第38—40页),关于这个问题,到考察第三十一章(《论机器》)时再作评论\fnote{见本册第628—630页。——编者注}。

值得注意的是,李嘉图在第五节结尾已接近于对事物的正确看法,几乎找到了有关的字句,可是马上就离开了正确的道路,他在接近于正确观点(我们就要引述这方面的话)之后,又回到支配着他的观念上去,即回到劳动价值的变动对费用价格的影响上去,并以对这个次要问题的结论结束了他的研究。

有关段落是这样说的:

\begin{quote}{“因此,我们可以看到,在还没有大量使用机器或耐久资本的社会发展早期阶段,用等量资本生产的商品会具有几乎相等的价值;只是由于生产它们所必需的劳动有了增减,这些商品彼此相对地说才会提高或降低}\end{quote}

{后面半句话说得不好;并且它不说价值,而说商品,这里除非是指商品的价格,否则就没有任何意义;因为说价值和劳动时间成比例地降低,就等于说价值随着自己的提高或降低而提高或降低};

\begin{quote}{但是,在采用了这些昂贵而耐久的工具之后,使用等量资本生产的商品就会具有极不相等的价值,虽然由于生产它们所必需的劳动的增减,它们的价值彼此相对地说仍然会提高或降低,可是由于工资和利润的提高或降低,它们还会发生另一种变动,虽然是较小的变动。因为卖5000镑的商品所用的资本量可能等于生产其他卖10000镑的商品所用的资本量,所以生产这两种商品所赚得的利润也会相等;但是,如果商品的价格不是随着利润率的提高或降低而变动,这些利润就会不相等。”(第40—41页)}\end{quote}

实际上李嘉图在这里是说:

如果等量资本的有机组成部分的比例相同,如果它们花费在工资和劳动条件上的份额相同,它们就会生产价值相等的商品。在这种情况下,它们所生产的商品中体现着等量的劳动,也就是相等的价值{流通过程可能带来的差别撇开不谈}。相反,如果等量资本的有机构成不同,尤其是如果它们的作为固定资本存在的那一部分同花费在工资上的那一部分的比例大不相同,它们就会生产价值大不相等的商品。第一,固定资本只有一部分作为价值组成部分加入商品,因此,根据生产商品时使用的固定资本的多少不同,价值量就已经大不相同。第二,[在固定资本多的情况下]花费在工资上的那一部分——按其在等量资本中所占的百分比计算,——就会少得多,因而体现在商品中的全部[新加]劳动也少得多,形成剩余价值的剩余劳动也少得多{已知工作日长度相同}。所以,既然这些等量资本生产的商品具有不等的价值,在这些不等的价值中包含不等的剩余价值,因而也包含不等的利润,如果这些资本由于数量相等而提供的利润也必定相等,那末,商品的价格(既然这种价格决定于一定费用的一般利润率)就必然和商品的价值大不相同。由此得出的结论不是价值改变了它的本性,而是费用价格不同于价值。李嘉图没有得出这个结论,这是令人奇怪的,特别是因为他的确看到了,即使在假定存在费用价格(它决定于一般利润率)的条件下,利润率(或工资率)的变动也一定要引起这种费用价格的变动,这样才能使[541]各个生产部门中的利润率保持一致。因此,一般利润率的确立,必然会使不等的价值发生更大的变动,因为这种一般利润率无非是等量资本生产的各种不同商品所包含的不同的剩余价值率的平均化。

李嘉图对于商品的费用和价值之间、商品的费用价格和价值之间的差别虽然没有阐述,没有理解,但是无论如何,他自己实际上已经确认了这种差别,在这之后,他结束他的论断说:

\begin{quote}{“马尔萨斯先生似乎认为,把某物的费用和价值等同起来,是我的学说的一部分。如果他说的费用是指包括利润在内的‘生产费用’〈就是指支出加由一般利润率决定的利润〉,那确是如此。”(第46页注)}\end{quote}

后来,李嘉图就带着他自己驳倒了的这种把费用价格和价值混淆起来的错误观点去考察地租。

李嘉图在第一章第六节谈到劳动价值的变动对金的费用价格的影响时说:

\begin{quote}{“难道我们不能把金看成这样一种商品,它在生产时所用的两种资本的比例同大多数商品生产时所用的两种资本的平均比例最接近吗?难道我们不能把这种比例看成同两个极端(一个极端是固定资本用得少,另一个极端是劳动用得少)保持相等距离而成为两者之间的中数吗?”(第44页)}\end{quote}

李嘉图的这些话,不如说适用于这样一些商品,这些商品的价值中各个不同有机组成部分的比例是平均比例,而且这些商品的流通时间和再生产时间也是平均时间。对这些商品来说,费用价格和价值是一致的,因为这些商品的平均利润和它们的实际剩余价值是一致的,但是只有这些商品才是这种情况。

第一章第四、五节,关于劳动价值的变动对“相对价值”的影响这个问题——这同价值因平均利润率而转化为费用价格的问题相比,(在理论上)是一个次要问题——的考察尽管有很大缺陷,但李嘉图由此却得出了十分重要的结论,推翻了自亚·斯密以来一直流传下来的主要错误之一,即认为工资的提高不是使利润降低,而是使商品的价格上涨。诚然,这一点已经包含在价值概念本身了,并且决不会由于价值转化为费用价格而有所改变,因为后者仅仅涉及总资本所赚得的剩余价值在不同部门之间或在不同生产领域的各个资本之间的分配。但是,李嘉图强调指出了这个问题,并且证明情况甚至相反,这仍然是有重要意义的。因此,他在第一章第六节公正地说:

\begin{quote}{“在结束这个题目之前,指出一点可能是适当的,就是亚当·斯密和一切追随他的著作家,据我所知,无一例外地都认为,劳动价格的上涨,必然会引起一切商品价格的上涨。”}\end{quote}

{这是和斯密的第二种价值规定相适应的,按照这个规定,价值等于一个商品能够买到的劳动量。}

\begin{quote}{“我希望,我已成功地证明了这种意见是毫无根据的,当工资提高时,只有比用来计算价格的中介物使用固定资本少的那些商品的价格才会上涨〈这里,“相对价值”等于价值的货币表现〉,而一切使用固定资本较多的商品的价格都必定下跌。反之,当工资降低时,只有比用来计算价格的中介物使用固定资本少的那些商品的价格才会下跌;而一切使用固定资本较多的商品的价格都必定上涨。”(第45页)}\end{quote}

这对于货币价格,看来是错误的。如果金的价值由于随便什么原因提高或降低了,那末这种[提高或]降低会同样地涉及用金计价的一切商品。因此,金尽管本身具有可变性,却表现为商品之间的相对不变的中介物。既然如此,那就绝对不能理解,同商品相比,在生产金时固定资本和流动资本的任何相对结合,能引起什么差别。但是在这里,李嘉图的错误前提也就出现了,他认为,货币只要用作流通手段,就是作为商品来同商品交换。商品在货币使它们流通以前,就以货币来计价了。我们假定中介物不是金,而是小麦。如果,比如说,按可变资本和不变资本的比例来说,加入小麦这种商品的可变资本超过平均水平,因此,由于工资提高,小麦的生产价格相对地上涨,那末,一切商品就都按具有较高“相对价值”的小麦来计价。那些有较多固定资本加入的商品,就会表现为比以前少的小麦,这不是因为这些商品的特殊价格同小麦相比下跌了,而是因为价格普遍下跌了。如果一个商品包含的同积累劳动相对立的[活]劳动,恰好和小麦包含的一样多,那末这一商品价格的上涨就会这样显示出来:它[542]与一个同小麦相比价格已经下跌的商品比较起来,表现为较多的小麦。如果引起小麦价格上涨的那些原因,也引起比如说衣服的价格上涨,那末,虽然衣服不会表现为比以前多的小麦,但是,同小麦相比价格已经下跌的那些商品,比如说棉布,就会表现为较少的小麦。棉布和衣服的价格差额,就会在小麦这个中介物上表现出来。

但是,李嘉图的意思不是这样。他的意思是:由于工资提高,同棉布相比,而不是同衣服相比,小麦的价格会上涨;因此,衣服就会按小麦原有的价格同小麦交换,而棉布则按小麦上涨了的价格同小麦交换。说什么英国工资价格的变动会促使工资没有提高的地方,比如说加利福尼亚的金的费用价格发生变动,这种假定本身就是极端荒谬的。价值通过劳动时间来平均化,尤其是费用价格通过一般利润率来平均化,在不同国家之间不是以这种直接的形式进行的。但就拿小麦这种国内产品来说吧。假如一夸特小麦的价格由40先令上涨到50先令,即上涨25%。如果衣服的价格也上涨25%,那末一件衣服仍旧值1夸特小麦。如果棉布的价格下降25%,那末过去值1夸特小麦的同样数量的棉布,现在只值6蒲式耳小麦\endnote{英国1夸特(散体量等于290.9公升)等于8蒲式耳。——第223页。}。这种用小麦表现出来的数字,准确地代表了棉布价格和衣服价格之比,因为棉布和衣服是用同一个尺度1夸特小麦来计量的。

此外,李嘉图的观点还有更荒谬的一面。用作价值尺度因而用作货币的商品的价格是根本不存在的;因为不然的话,我除了用作货币的商品之外还必须有第二种用作货币的商品——双重的价值尺度。货币的相对价值是以一切商品的无数价格表现出来的;因为,在商品交换价值借以表现为货币的这许多价格的每一个价格中,货币的交换价值都表现为商品的使用价值。因此,谈不上货币价格上涨或下跌的问题。我可以说:货币的小麦价格或货币的衣服价格保持不变,而货币的棉布价格上涨了,这等于说棉布的货币价格下跌了。但我不能说,货币的价格上涨或下跌了。可是,李嘉图实际上认为,比如说,货币的棉布价格所以上涨,或者说,棉布的货币价格所以下跌,正是因为同棉布相比,货币的相对价值提高了,而同衣服或小麦相比,货币却保持其原有价值。这样一来,这两种价值就是用不同的尺度来计量了。

这第六节(《论不变的价值尺度》)论述的是“价值尺度”,但其中没有什么重要的东西。对价值,价值的内在尺度——劳动时间——同商品价值的外在尺度的必要性之间的联系,根本不了解,甚至没有把它当作问题提出来。

第六节一开头就表现了肤浅的论述方法:

\begin{quote}{“当商品的相对价值发生变动时,最好有一个方法能确定,哪种商品的实际价值降低了,哪种商品的实际价值提高了。要做到这一点,只有把它们逐一同某种不变的、本身不会发生其他商品所发生的变动的标准尺度相比较。”但是“没有一种商品本身不发生……同样的变动,就是说,没有一种商品在生产时所需要的劳动能够不有所增减”。(第41—42页)}\end{quote}

但是,即使有这样一种商品,工资提高或降低的影响,以及固定资本和流动资本的不同结合、固定资本的不同耐久程度、商品在能够进入市场以前必须经历的时间不同等等的影响,也都会部分地妨碍它

\begin{quote}{“成为我们能够用来准确地确定一切其他物品的价值变动的一种完美的价值尺度”。“对于在和它本身完全相同的条件下生产出来的一切物品来说,它是完美的价值尺度,但对其他物品来说就不是了。”(第43页)}\end{quote}

换句话说,在这两类“其他物品”中前一类的价格发生变动时,我们可以(如果货币的价值不提高或降低的话)说,这种变动是因为“它们的价值”有了提高或降低,即生产它们所需要的劳动时间有了增减。至于其他物品,我们就无法知道,它们的货币价格发生“变动”是否由于其他原因等等。后面(以后考察货币理论时)还要回过头来谈这些很不恰当的论断。

第一章第七节。除了关于“相对”工资、利润和地租的重要学说(这方面后面还要回过头来谈\fnote{见本册第476—482页。——编者注})以外,这一节只包含这样一个论点:在货币价值降低或提高时,工资、利润和地租的相应的提高或降低,决不会改变它们之间的比例,而只会改变它们的货币表现。如果同一商品的价值表现为两倍的镑数,那末,转化为利润、工资或地租的那一部分价值也增加一倍。但是,这三个部分互相之间的比例和它们所代表的实际价值仍然不变。同样,如果利润表现为两倍的镑数,那末100镑现在也就表现为200镑;因此,利润和资本之间的比例,即利润率也仍然不变。货币表现的变动同时影响利润和资本,就象它同时影响利润、工资和地租一样。这对于地租也是适用的,只要地租不是按英亩计算,而是按预付在耕种土地等等上面的资本计算。总之,在这种场合,变动不是发生在商品上,等等:

\begin{quote}{“由这种原因造成的工资提高,当然不可避免地会引起商品价格的同时上涨;但是,在这种情况下我们会发现,劳动和一切商品之间的比例没有变动,变动的只是货币。”(第47页)}\end{quote}

\tsubsectionnonum{[(5)]平均价格或费用价格和市场价格}

\tsubsubsectionnonum{[(a)引言:个别价值和市场价值;市场价值和市场价格]}

[543]李嘉图为了阐明级差地租理论,在第二章(《论地租》)提出以下论点:

\begin{quote}{“一切商品,不论是工业品、矿产品还是土地产品,它们的交换价值始终不决定于在只是享有特殊生产便利的人才具备的最有利条件下足以把它们生产出来的较小量劳动,而决定于没有这样的便利,也就是在最不利条件下继续进行生产的人所必须花在它们生产上的较大量劳动;这里说的最不利条件,是指为了把需要的产品量生产出来而必须继续进行生产的那种最不利的条件。”(第60—61页)}\end{quote}

最后一句话不完全正确。“需要的产品量”不是一个固定的量。应当说:一定价格界限内需要的一定产品量。如果价格上涨超过了这种界限,“需要的量”就会同需求一起减少。

上述论点可以一般表达如下:商品(它是某个特殊生产领域的产品)的价值,决定于为生产这个生产领域的全部商品量即商品总额所需要的劳动,而不决定于这个生产领域内部单个资本家或企业主所需要的特殊劳动时间。这个特殊生产领域,比如说棉纺织工业的一般生产条件和一般劳动生产率,是这个领域即棉纺织工业的平均生产条件和平均劳动生产率。因此,决定比如一码棉布价值的劳动量,并不是这码棉布中包含的、这个棉织厂主花费在它上面的劳动量,而是出现在市场上的全体棉织厂主生产一码棉布所花费的平均量。单个资本家,比如棉纺织工业的资本家,进行生产的特殊条件必然分为三类。有一类人是在中等条件下进行生产;这就是说,他们进行生产的个别生产条件同这个领域的一般生产条件一致。平均比例就是他们的实际比例。他们的劳动生产率处于平均水平。他们的商品的个别价值同这些商品的一般价值一致。如果他们比如把棉布按2先令一码即按它的平均价值出卖,那末,他们就是按照他们生产的棉布在实物形式上所代表的价值出卖棉布。第二类企业主进行生产的条件比平均条件好。他们的商品的个别价值低于同种商品的一般价值。如果他们按这种一般价值出卖自己的商品,他们就是把自己的商品卖得高于它们的个别价值。最后,第三类企业主是在低于平均条件的生产条件下进行生产。

前面已经说过,这个特殊生产领域的“需要的产品量”不是一个固定的量。如果商品价值超过平均价值的一定界限,“需要的产品量”就会减少,或者说,这个量只有按照某种价格或者至少是在一定价格的界限内才是需要的。因此,最后一类企业主也有可能不得不低于自己商品的个别价值出卖商品,正如条件最好的那一类企业主总是高于自己商品的个别价值出卖商品一样。这几类中究竟由哪一类最后确定平均价值,正是取决于这几类的数量或数量的比例关系\endnote{马克思这里说的各类企业主的“数量或数量的比例关系”,是指每一类企业主运到市场的产品数量。——第227页。}。如果中等的一类在数量上占很大优势,那就由它确定平均价值。如果这一类数量少,而生产条件低于平均条件的那一类数量大,占了优势,那就由这后一类确定这个领域的产品的一般价值,虽然这还决不是说,甚至很少可能,恰好由这一类中条件最不利的个别资本家决定问题(见柯贝特的著作)\endnote{马克思指柯贝特的书《个人致富的原因和方法的研究;或贸易和投机原理的解释》(1841年伦敦版)。柯贝特在书中断言,在工业中,价格是由最好条件下生产出来的商品调节的,按照他的意见,正是这些商品占所有这种商品的绝大多数(第42—44页)。——第227页。}。

但是我们把这一点撇开不谈。一般的结果是:这种产品具有的一般价值,对所有这种产品都是相同的,不管它对每一个别商品的个别价值的比例如何。这种一般价值,就是这些商品的市场价值,就是它们进入市场时具有的价值。这种市场价值用货币表现出来就是市场价格,正如价值用货币表现出来就是价格一样。实际的市场价格,有时高于这种市场价值,有时低于这种市场价值,只是偶然同市场价值一致。但是在一定时期内,波动会互相抵销,因此可以说,实际市场价格的平均数,就是表现市场价值的市场价格。不管实际市场价格在当时按其大小来说,从数量来说是否同这种市场价值一致,市场价格总是同市场价值有一个共同的质的规定,即同一生产领域的所有在市场上的商品(自然假定它们的质是相同的)都具有同一价格,或者说,它们实际上代表这个领域的商品的一般价值。

[544]因此,李嘉图为他的地租理论提出的上述论点,他的门徒作了这样的表述:在一个市场上不可能同时存在两种不同的市场价格,或者说,同时出现在市场上的同一种产品具有同一价格,或者说,——因为这里我们可以把这种价格的偶然性撇开不谈,——具有同一市场价值。

于是,竞争——部分地是资本家之间的竞争,部分地是商品的买者同资本家的竞争以及商品的买者之间的竞争——在这里就导致这样的结果:某一特殊生产领域的每一个别商品的价值决定于这一特殊社会生产领域的商品总量所需要的社会劳动时间总量,而不决定于个别商品的个别价值,换句话说,不决定于个别商品的特殊生产者和卖者为这一个别商品花费的劳动时间。

但是,从这里自然就会得出结论:属于第一类的、生产条件比平均生产条件有利的资本家,在所有情况下都会赚得一种超额利润,就是说,他们的利润会超过这个领域的一般利润率。因此,竞争并不是通过把一个生产领域内部的各种利润平均化的办法来确立市场价值或市场价格。(市场价值和市场价格之间的差别对这里的研究没有意义,因为不管市场价格和市场价值的比例如何,生产条件的差别以及由此产生的不同利润率,对同一领域的各个资本家来说是始终存在的。)相反,竞争在这里正是通过容许有个别利润之间的差别,即各个资本家的利润之间的差别,通过容许有个别利润对该领域平均利润率的偏离,把不同的个别价值平均化为同一的、相等的、没有差别的市场价值。竞争甚至通过为那些在有利程度不同的生产条件下,因而在劳动生产率不同的条件下生产出来,因此代表个别的、不等量的劳动时间的商品确立同一的市场价值,来造成这种偏离。在比较有利的条件下生产出来的商品,同在比较不利的条件下生产出来的商品相比,包含的劳动时间较少,可是却按同一价格出卖,具有同一价值,就好比它包含了它实际上并不包含的同一劳动时间。

\tsubsubsectionnonum{[(b)李嘉图把同一生产领域内的市场价值形成过程同不同生产领域的费用价格形成过程混淆起来]}

李嘉图为了建立他的地租理论,需要两个论点,这两个论点表达的不仅不是竞争的同一种作用,而且恰恰是竞争的相反的作用。第一个论点是,同一领域的产品按同一市场价值出卖,因而竞争以强制的方式造成不同的利润率,即造成对一般利润率的偏离。第二个论点是,对一切投资来说,利润率都必须是相同的,或者说,竞争造成一般利润率。第一个规律适用于投入同一生产领域的不同的独立资本。第二个规律适用于投入不同生产领域的资本。竞争通过它的第一种作用造成市场价值,即为同一生产领域的商品造成同一价值,虽然这同一价值必然要产生不同的利润;因此,竞争不顾不同的利润率,或者不如说,利用不同的利润率,通过它的第一种作用造成同一价值。竞争通过它的第二种作用(不过,第二种作用是以另一种方式实现的;这是不同领域的资本家之间的竞争,它使资本从一个领域转移到另一个领域,而前面所说的那种竞争,只要不是在买者之间进行,则是发生在同一领域的资本之间),造成费用价格,即造成不同生产领域的同一利润率,虽然这同一利润率与价值不等的情况相矛盾,因而只有通过不同于价值的价格才能造成。

既然李嘉图本人为了建立他的地租理论需要这两者,既需要在利润率不等的情况下的相等价值或价格,又需要在价值不等的情况下的相等利润率,那末非常令人奇怪的是,他竟没有觉察到这个双重的规定,甚至在他专门论述市场价格的那一部分即第四章(《论自然价格和市场价格》),也完全没有论述市场价格或市场价值,尽管在前面引用的那段话\fnote{见本册第225页。——编者注}中,他还是把后者作为基础,来说明级差地租就是结晶为地租的超额利润。[545]相反,在第四章,他只是说明不同生产领域的价格归结为费用价格或者说平均价格,也就是说,只是说明不同生产领域的市场价值的相互关系,却没有说明每个特殊领域的市场价值的形成过程,而没有这个形成过程,就根本不存在什么市场价值。

每个特殊生产领域的市场价值,因而,每个特殊领域的市场价格(如果市场价格符合“自然价格”,就是说,它只是用货币把价值表现出来),都会提供极不相同的利润率,因为不同生产领域的等量资本(这些资本的不同流通过程产生的差别完全撇开不谈)使用不变资本和可变资本的比例极不相同,所以它们提供的剩余价值,从而它们提供的利润,也就极不相等。因此,不同市场价值平均化的结果是,在不同领域确立相同的利润率,使等量资本提供相等的平均利润,而不同市场价值的这种平均化,只有通过市场价值转化为不同于实际价值的费用价格才有可能。\fnote{剩余价值率在不同生产领域中可能并不平均化(例如由于劳动时间的长度不等)。并不因为剩余价值本身会平均化,剩余价值率就必然要平均化。}

竞争在同一生产领域所起的作用是:使这一领域生产的商品的价值决定于这个领域中平均需要的劳动时间;从而确立市场价值。竞争在不同生产领域之间所起的作用是:把不同的市场价值平均化为代表不同于实际市场价值的费用价格的市场价格,从而在不同领域确立同一的一般利润率。因此,在这第二种情况下,竞争决不是力求使商品价格去适应商品价值,而是相反,力求使商品价值化为不同于商品价值的费用价格,取消商品价值同费用价格之间的差别。

李嘉图在第四章考察的只是后面这种运动,而且他十分奇怪地把它看成是商品价格——通过竞争——还原为商品价值的运动,看成是“市场价格”(不同于价值的价格)还原为“自然价格”(用货币表现出来的价值)的运动。其实,这个谬误,是由在第一章(《论价值》)已经犯下的把费用价格和价值等同起来的错误\fnote{见本册第220页。——编者注}造成的,而后面这个错误的产生,又是因为李嘉图在他只需要阐明“价值”的地方,就是说,在他面前还只有“商品”的地方,就把一般利润率以及由比较发达的资本主义生产关系产生的一切前提全都拉扯上了。

因此,李嘉图在第四章所遵循的全部思路也是极其肤浅的。他的出发点是由变动的供求关系引起的商品“价格的偶然和暂时的变动”(第80页)。

\begin{quote}{“随着价格的上涨或下跌,利润就提高到它的一般水平之上或下降到它的一般水平之下,于是资本或者被鼓励转入那个发生这种变动的个别投资部门,或者被警告要退出这一部门。”(第80页)}\end{quote}

这里已经假定有一个不同生产领域之间的、“个别投资部门”之间的“利润的一般水平”。然而首先应当考察的是,同一投资部门中价格的一般水平和不同投资部门之间利润的一般水平是如何确立的。这样,李嘉图就会看到,后一种活动已经以资本的不断来回交叉游动为前提,或者说,以由竞争决定的、全部社会资本在不同投资领域之间的分配为前提。既然已经假定,在不同领域中市场价值或者说平均市场价格化为提供同一平均利润率的费用价格{但这种情况只是在没有土地所有权干预的生产领域才会发生,在有土地所有权干预的领域,这些领域内部的竞争会使价格化为价值,使价值化为市场价值,但不会使后者降到费用价格},既然已经假定了这一点,那末某些特殊领域中发生的市场价格对费用价格的经常偏离,即经常高于或低于费用价格的情况,就会引起社会资本的新的转移和新的分配。第一种转移的发生是为了确立不同于价值的费用价格;第二种转移是为了在市场价格高于或低于费用价格的时候使实际市场价格同费用价格趋于一致。第一种转移是价值转化为费用价格。第二种转移是不同领域中实际的[546]偶然的市场价格围绕费用价格旋转,费用价格现在表现为“自然价格”,虽然它不同于价值,它只是社会活动的结果。

李嘉图考察的正是后面这种比较表面的运动,他有时不自觉地把这种运动同另一种运动混淆起来。这两种运动自然是由“同一个原则”引起的,这个原则就是:

\begin{quote}{“每一个人都可以随意把自己的资本投在他所喜欢的地方……他自然要为自己的资本找一个最有利的行业;如果把资本转移一下能够得到15%的利润,他自然不会满足于10%的利润。一切资本家都想放弃利润较低的行业而转入利润较高的行业的这种不会止息的愿望,产生一种强烈的趋势,就是使大家的利润率平均化,或者把大家的利润固定在当事人看来可以抵销一方所享有的或看来享有的超过另一方的利益的那种比例上。”(第81页)}\end{quote}

这种趋势促使社会劳动时间总量按社会需要在不同生产领域之间进行分配。同时,不同领域的价值由此转化为费用价格,另一方面,各特殊领域的实际价格对费用价格的偏离也被拉平了。

这一切都来自亚·斯密。李嘉图自己说:

\begin{quote}{“如果一个投资部门生产的商品不能用自己的价格抵补把它们生产出来并运到市场的全部费用(包括普通利润在内)〈也就是不能补偿费用价格〉,资本就有离开这个部门的趋势,关于这一点,再没有一个著作家比斯密博士说得更令人满意、更精辟的了。”(第342页注)}\end{quote}

李嘉图的错误,一般说来,是由于他在这里不加批判地对待亚·斯密而产生的,而他的功绩则在于更确切地说明了资本从一个领域到另一个领域的这种转移,或者不如说,更确切地说明了这种转移的方式本身。但是,他能做到这一点,只是因为信用制度在他那个时代比在斯密时代更加发达罢了。李嘉图说:

\begin{quote}{“要追溯这种变化借以实现的步骤或许是非常困难的;它可能通过一个工厂主并不完全改变他的行业,而只是减少他在自己企业中的投资的方式来实现。在一切富裕的国家中,都有一定数目的人,形成所谓货币所有者阶级[注:罗雪尔在这里又可以看到,英国人所谓的“货币所有者阶级”是指什么。“货币所有者阶级”和“社会上有企业精神的人”在这里是完全对立的。\endnote{前面,在本册第130页上马克思引了罗雪尔《国民经济学原理》中的一段话,证明罗雪尔对于围绕着安德森的地租理论的斗争和关于“货币所有者和土地所有者之间的对立”的观点十分混乱。——第233页。}];这些人不从事任何行业,而把他们的货币用于期票贴现或者借给社会上更有企业精神的人,他们就靠这种货币的利息生活。银行家也把大量资本用于同样的目的。这样使用的资本形成巨额的流动资本,全国各行业或多或少地都使用它。一个工厂主不论怎样富有,大概也不会把他的营业限制在仅仅他自己的资金所容许的范围以内,他会经常使用这种流动资本的一部分,这部分资本的增减取决于对他的商品的需求的强弱。当对丝绸的需求增加而对呢绒的需求减少的时候,毛织厂主并不会把他的资本转到丝纺织业中去,而是解雇一部分工人,不再向银行家和货币所有者借款;丝织厂主的情况则相反,他会借更多的货币,于是资本就从一个行业转到另一个行业,而工厂主不必中断他通常经营的行业。如果我们观察一个大城市的市场,看到在所有由于嗜好改变或人口数量变动而需求发生变化的情况下,市场上国内外商品都能按需要的数量有规则地得到供应,既不是常常因供给过多而发生市场商品充斥现象,也不是常常因供不应求而造成物价腾贵,我们就必须承认,在一切行业之间恰好按其需要的数量分配资本的原则所起的作用,比一般设想的还大。”(第81—82页)}\end{quote}

由此可见,正是信用促使每个生产领域不是按照这个领域的资本家自有资本的数额,而是按照他们生产的需要,去支配整个资本家阶级的资本,——而在竞争中单个资本对于别的资本来说是独立地出现的。这种信用既是资本主义生产的结果,又是资本主义生产的条件,这样就从资本的竞争巧妙地过渡到作为信用的资本。

\tsubsubsectionnonum{[(c)李嘉图著作中关于“自然价格”的两种不同的规定。费用价格随着劳动生产率的变动而变动]}

李嘉图在第四章开头说,他所谓的“自然价格”,是指商品的“价值”,也就是指由商品的相对劳动时间决定的价格,而他所谓的“市场价格”,是指对这种等于“价值”的“自然价格”的偶然和暂时的偏离。[547]但是,在这一章的以后的全部行文中——甚至说得很明确——他所谓的“自然价格”,是指完全不同的东西,就是说,指不同于价值的费用价格。因此,他不去说明竞争怎样使价值转化为费用价格,从而造成对价值的经常偏离,却按照亚·斯密那样说明,竞争怎样使不同行业的市场价格在它们的相互关系中化为费用价格。

第四章开头这样说:

\begin{quote}{“如果我们把劳动作为商品价值的基础,把生产商品所必需的相对劳动量作为确定商品相互交换时各自必须付出的相应商品量的尺度,不要以为我们否定商品的实际价格或者说市场价格对商品的这种原始自然价格的偶然和暂时的偏离。”(第80页)}\end{quote}

可见,在这里“自然价格”等于价值,而市场价格无非是实际价格对价值的偏离。

相反:

\begin{quote}{“我们假定一切商品都按其自然价格出卖,因而资本的利润率在所有行业完全相同,或者只有这样一点差别,这种差别在当事人看来是与他们所享有或放弃的任何现实的或想象的利益一致的。”(第83页)}\end{quote}

可见,在这里“自然价格”等于费用价格,也就是等于这样的价格,在其中,利润对商品所包含的支出的比率是同一比率,尽管不同行业的资本生产的商品的等量价值包含极不相等的剩余价值,因而包含不相等的利润。因此,价格要提供同一利润,就必须不同于商品的价值。另一方面,由于加入商品的那部分固定资本大小不同,等量资本生产的商品的价值也是极不相等的。但是关于这一点,到考察资本流通时再谈。

所以,李嘉图所谓的竞争的平均化作用,不过是指实际价格,或者说,实际市场价格围绕费用价格,或者说,不同于价值的“自然价格”而波动,是指不同行业中的市场价格平均化为一般费用价格,也就是恰恰平均化为不同于某一行业的实际价值的价格:

\begin{quote}{“所以,正是每一个资本家都想把资金从利润较低的行业转移到利润较高的行业的这种愿望,使商品的市场价格不致长期大大高于或大大低于商品的自然价格。正是这种竞争会这样调节商品的交换价值{也调节不同的实际价值},以致在支付生产商品所必需的劳动的工资和其他一切为维持所使用的资本的原有效率所需要的费用之后剩下来的价值即价值余额,在每个行业中都同使用的资本的价值成比例。”(第84页)}\end{quote}

情况确实如此。竞争会这样调节不同行业的价格,以致剩下来的价值即价值余额,也就是利润,同使用的资本的价值相一致,而不是同商品的实际价值相一致,不是同商品在扣除费用以后所包含的实际的价值余额相一致。要实现这种调节,一种商品的价格就必须上涨到它的实际价值以上,而另一种商品的价格则必须下降到它的实际价值以下。竞争迫使不同行业的市场价格不是围绕商品的价值旋转,而是围绕商品的费用价格,也就是围绕商品中包含的费用加一般利润率旋转。

李嘉图继续说道:

\begin{quote}{“在《国富论》第七章对于同这个问题有关的一切都作了极为出色的论述。”(第84页)}\end{quote}

的确如此。正是由于不加批判地相信斯密的传统,李嘉图在这里走上了歧途。

李嘉图跟平常一样在结束这一章时说,在以后的研究中他将“完全不考虑”(第85页)市场价格对费用价格的偶然偏离,但是他忽略了一点,就是他根本没有注意到市场价格在同费用价格相一致的条件下对商品的实际价值的经常偏离,并且用费用价格代替了价值。

第三十章《论需求和供给对价格的影响》。

李嘉图在这里维护这样一个论点:持久的价格决定于费用价格,而不决定于需求和供给,因此,只是由于商品价值决定费用价格,持久的价格才决定于商品价值。假定商品的价格经过调节,都提供10%的利润,那末,它们的任何持久的变动都将决定于商品价值的变动,决定于生产商品所需要的劳动时间的变动。正如这种价值继续决定一般利润率一样,它的变动也继续决定费用价格的变动,自然,这并不会取消这种费用价格和价值之间的差额。取消的只是超出这一差额的东西,因为价值和实际价格之间的差额不应[548]大于一般利润率造成的费用价格和价值之间的差额。随着商品的价值的变动,商品的费用价格也发生变动。于是便形成“新的自然价格”(第460页)。例如,一个工人过去生产10顶帽子,现在用同样的时间能够生产20顶,如果工资占帽子的生产费用的一半,那末,20顶帽子的费用即生产费用,就其由工资组成的部分来看,是降低了一半。因为现在为生产20顶帽子支付的工资,同过去为生产10顶帽子支付的一样多。因此,每一顶帽子中现在只包含以前工资费用的一半。如果制帽厂主按以前的价格出卖帽子,他的帽子就会卖得高于费用价格。如果过去利润是10%(假定制造一定数量的帽子所必需的支出中,原来有50用于原,50用于劳动),那末现在利润就是[46+(2/3)]%。现在支出中有50用于原料等等,25用于工资。如果商品按以前的价格出卖,那末现在利润就是35/75,即[46+(2/3)]%。因此,由于价值降低,新的“自然价格”就会下跌,直到价格只提供10%的利润为止。价值降低,或者说,生产商品所必需的劳动时间减少,表现为同量商品所耗费的劳动时间减少,也就是耗费的有酬劳动时间减少,花费的工资减少,因而费用,为生产每一单位商品按比例支付的工资(按绝对量来说;这并不以工资率的下降为前提),也就下降。

当价值变动发生在制帽过程本身时,就会出现这种情况。如果价值变动发生在原料或劳动工具的生产上,这种变动在这些领域中同样表现为生产一定量产品所必需的工资费用减少,而对制帽厂主来说,则表现为他在不变资本上花费的钱减少。

费用价格,或者说,“自然价格”(它同“自然”毫无关系)由于商品价值变动——这里是降低——可能发生双重的变动:

第一,如果生产一定量商品所支付的工资由于生产该一定量商品所花费的劳动(包括有酬劳动和无酬劳动)的整个绝对量减少而减少;

第二,如果由于劳动生产率提高或降低(两种情况都可能发生:一种是在可变资本同不变资本相比减少的时候;另一种是在工资由于生活资料涨价而提高的时候),剩余价值和商品价值的比例,或者说,剩余价值和商品中包含的[新加]劳动所创造的价值的比例发生变动,因而利润率提高或降低,整个[新加]劳动量分为有酬劳动和无酬劳动的那种比例发生变动。

在后一种场合,生产价格即费用价格只能根据劳动价值的变动对它们发生影响的程度来变动。在前一种场合,劳动价值保持不变。在后一种场合,变动的不是商品的价值,而只是[必要]劳动和剩余劳动之间的分配。可是在这种场合,[劳动]生产率,因而每一单位商品的价值,仍然会发生变动。同一资本在一种场合生产的商品将比从前多,在另一种场合生产的商品将比从前少。资本借以表现的商品总量仍然具有同样的价值,但是单位商品的价值却和以前不同。虽然工资的价值并不决定商品的价值,但是(加入工人消费的)商品的价值却决定工资的价值。

既然不同行业商品的费用价格是既定的,这些费用价格就随着商品价值的变动而彼此相对地上涨或下跌。如果劳动生产率提高,就是说,生产一定商品所需要的劳动时间减少,因而商品的价值降低——不管生产率的这一变动是发生在最后阶段使用的劳动上,还是发生在为生产该商品所必需的不变资本包含的劳动上,——这种商品的费用价格就必然要相应地下跌。用于这种商品的绝对劳动量减少了,因而这种商品包含的有酬劳动量也减少了,花费在这种商品上的工资量也减少了,即使工资率保持不变。如果商品按其原来的费用价格出卖,它提供的利润就会高于一般利润率,因为以前按较大的支出计算,这个利润是10%。所以现在按减少了的支出计算,利润就会大于10%。相反,如果劳动生产率降低,商品的实际价值就提高。如果利润率是既定的,或者同样可以说,如果费用价格是既定的,那末,费用价格的相对提高或降低,就取决于商品实际价值的提高或降低,取决于商品实际价值的变动。由于这种变动,新的费用价格,或者象李嘉图仿照斯密所说的“新的自然价格”,就代替旧的价格。

在刚才引用过的第三十章里,李嘉图甚至在名称上也把“自然价格”即费用价格和“自然价值”即决定于劳动时间的价值等同起来了:

\begin{quote}{“它们的价格〈垄断商品的价格〉同它们的自然价值并没有必然的联系;但是,受竞争影响……的商品的价格最后都……取决于它们的生产费用。”(第465页)}\end{quote}

可见,这里把费用价格,或者说,“自然价格”和“自然价值”即“价值”直接[549]等同起来了。

这种混乱说明了为什么李嘉图以后的一批家伙,和萨伊本人一样,能把“生产费用”当作价格的最后调节者,而对价值决定于劳动时间这一规定却毫无所知,甚至在坚持“生产费用”的同时直接否定这一规定。

李嘉图的这整个错误和由此而来的对地租等的错误论述,以及关于利润率等的错误规律,都是由于他没有区分剩余价值和利润而造成的,总之,是由于他象其余的政治经济学家那样粗暴地、缺乏理解地对待形式规定而造成的。李嘉图怎样被斯密俘虏,从下文就可以看出。[549]

\centerbox{※     ※     ※}

[XII—636](对前面讲过的还要补充一点意见:

李嘉图不知道价值和自然价格有其他的差别,只知道:自然价格是价值的货币表现,因此,在商品本身的价值没有变动的情况下,由于贵金属的价值发生变动,自然价格也会变动。但是自然价格的这种变动只关系到价值的货币计量或货币表现。例如,李嘉图说:

\begin{quote}{“它〈对外贸易〉只能通过改变自然价格,但不是改变各国能据以生产商品的自然价值来调节,而这是通过改变贵金属的分配来实现的。”(同上,第409页))[XII—636]}\end{quote}

\tsectionnonum{[B.斯密的费用价格理论]}

\tsubsectionnonum{[(1)斯密的费用价格理论的错误前提。李嘉图由于保留了斯密把价值和费用价格等同起来的观点而表现出前后矛盾]}

[XI—549]关于亚·斯密,首先应当指出,他也认为:

\begin{quote}{“总是有……一些商品,它们的价格只分解为两部分,即工资和资本的利润。”(亚·斯密《国民财富的性质和原因的研究》[1802年法文版]第1卷第1篇第6章第103页)}\end{quote}

因此,对于斯密和李嘉图在这个问题上的区别在这里可以完全不去注意。

斯密起先阐述了一个观点,认为交换价值归结为一定量的劳动,交换价值中包含的价值,在扣除原料等之后,分解为付给工人报酬的劳动和不付给工人报酬的劳动,而后面这种不付给报酬的部分又分解为利润和地租(利润又可以分解为利润和利息),——在此以后,他突然来了一个大转变,不是把交换价值分解为工资、利润和地租,而是相反,把工资、利润和地租说成是构成交换价值的因素,硬把它们当作独立的交换价值来构成产品的交换价值,认为商品的交换价值是由不依赖于它而独立决定的工资价值、利润价值和地租价值构成的。价值不是它们的源泉,它们倒成了价值的源泉。

\begin{quote}{“工资、利润和地租,是一切收入的三个原始源泉,也是一切交换价值的三个原始源泉。”(同上,第105页)}\end{quote}

斯密在阐述了他所研究的对象的内在联系之后,突然又被表面现象所迷惑,被竞争中表现出来的事物联系所迷惑,而在竞争中一切总是表现为颠倒的、头足倒置的。

斯密正是从这个颠倒了的出发点来阐明“商品的自然价格”同商品的“市场价格”之间的区别的。李嘉图接受了斯密的这个观点,但是他忘记了,按照斯密的前提,斯密的“自然价格”只不过是由竞争而产生的费用价格,而在斯密本人的著作中,只有当斯密忘记了他自己的比较深刻的观点,仍然保持从表面的外观中得出来的,认为商品的交换价值是由独立决定的工资价值、利润价值和地租价值相加而成的错误观点的时候,费用价格才和商品的价值等同。李嘉图处处都反对这一观点,但是他又接受了亚·斯密在这一观点的基础上产生的,把交换价值同费用价格或“自然价格”混淆起来,或者说,等同起来的看法。这种混淆在斯密那里还可以说得过去,因为他对“自然价格”的全部研究是从他对价值的第二个观点即错误的观点出发的。而在李嘉图那里就毫无道理了,因为他在任何地方也没有接受斯密的这一错误观点,相反,他认为它前后矛盾而专门加以驳斥。但是,斯密又用“自然价格”把李嘉图引入了迷途。

斯密用不依赖于商品价值而独立决定的工资价值、利润价值和地租价值构成商品价值之后,接着就给自己提出了一个问题:这些作为要素的价值又是怎样决定的?这里斯密是从竞争中呈现出来的现象出发的。

第一篇第七章《论商品的自然价格和市场价格》。

\begin{quote}{“在任何社会或任何地方,工资、利润、地租都有一种普通率,或者说,平均率。”这种“平均率对于它所通行的时间和地方来说可以称为工资、利润和地租的自然率”。“如果一种商品的价格恰好足够按自然率支付地租、工资和利润,这种商品就是按照它的自然价格出卖。”(第110—111页)}\end{quote}

这样一来,这种自然价格就是商品的费用价格,而费用价格就和商品的价值等同起来了,因为已经假定,商品的价值是由工资价值、利润价值和地租价值构成的。

\begin{quote}{“商品[550]在这种情况下恰好是按其所值出卖〈这时商品是按其价值出卖〉,或者说〈或者说!〉,按照使该商品进入市场的人的实际花费出卖〈对使商品进入市场的人来说,是按商品的价值,或者说,费用价格出卖〉,因为,虽然照普通的说法,在谈到商品的生产费用时,其中不包括出卖自己生产的商品的人的利润,但是,如果他按照不能给他提供当地普通利润的价格出卖自己的商品,他的营业显然就要受到损失,因为他如果以其他某种方式使用自己的资本,是能够获得这一利润的。”(第111页)}\end{quote}

在这里,我们看到了“自然价格”产生的全部历史以及同它完全相适应的语言和逻辑。因为在斯密看来,商品的价值是由工资、利润和地租的价格构成的,而工资、利润和地租的真正价值也是以同样方式构成的,所以很明显,在它们处于自己的自然水平的情况下,商品的价值和商品的费用价格是等同的,而商品的费用价格又是和商品的“自然价格”等同的。利润水平即利润率,以及工资率,都被假定为事先既定的。对于费用价格的形成来说,它们确实是既定的。它们是费用价格的前提。因此,它们对单个资本家来说也表现为既定的。至于它们怎样产生,在什么地方产生和为什么产生,资本家是不关心的。斯密在这里是站到确定自己商品的费用价格的单个资本家即资本主义生产当事人的立场上去了。工资等等花费多少,一般利润率是多少。因此……在这个资本家看来,确定商品费用价格的程序,或者,在他进一步看来,确定商品价值的程序就是这样,因为他也知道,市场价格有时高于这种费用价格,有时低于这种费用价格;所以在资本家看来,这种费用价格就是商品的理想价格,就是不同于商品价格波动的商品的绝对价格,一句话,就是商品的价值,如果资本家有时间去思考这类事情的话。由于斯密置身于竞争的中心,他立即就开始按照受这个领域局限的资本家所特有的逻辑发议论。他反驳说,在日常生活中,费用不是指卖者所赚得(并且必然是超过他的支出的余额)的利润;你为什么把利润算在费用价格之内呢?亚·斯密同被提出这一问题的深思熟虑的资本家一起,作了如下的回答:

利润一般必须加入费用价格,因为,即使加入费用价格的利润总共只有9%而不是10%\endnote{假定10%是当地的平均利润率。——第243页。},我也是受骗了。

斯密天真地一方面用资本主义生产当事人的眼光来看待事物,完全按照这种当事人所看到和所设想的样子,按照事物决定这种当事人的实践活动的情况,按照事物实际上呈现出来的样子,来描绘事物,另一方面,在有些地方也揭示了现象的更为深刻的联系,——斯密的这种天真使他的著作具有巨大的吸引力。

这里也可以看出,为什么斯密——尽管在这一问题上内心有很大的犹豫——把商品的价值只分解为地租、利润和工资,而略去了不变资本,尽管他自然也承认“单个”资本家的不变资本。因为不然的话,他就必须说商品的价值是由工资、利润、地租以及不由工资、利润、地租构成的那个商品价值部分构成的了。这样一来,就必须离开工资、利润和地租来确定价值了。

如果除了补偿平均工资等等的支出以外,商品的价格还提供平均利润,而在支出数包括地租的情况下,还提供平均地租,那末,商品就是按其“自然价格”,或者说,费用价格出卖,而且商品的费用价格就等于商品的价值,因为在斯密看来,商品的价值无非是工资、利润和地租的自然价值的总和。

[551]此外,斯密既然已经站在竞争的立场上,并且假定了利润率等是既定的,他也就正确地阐述了“自然价格”,或者说,费用价格,也就是不同于市场价格的那种费用价格。

\begin{quote}{“自然价格,或者说,使它〈商品〉进入市场所必须支付的地租、利润和工资的全部价值”。商品的这种费用价格是和它的“实际价格”,或者说,“市场价格”不同的。(第112页)后者取决于需求和供给。}\end{quote}

商品的生产费用,或者说,商品的费用价格,恰好是“使这一商品进入市场所必须支付的地租、工资和利润的全部价值”。如果供求相互适应,“市场价格”就等于“自然价格”。

\begin{quote}{“如果进入市场的数量恰好足够满足实际需求而不超出这一限度,那末,市场价格当然就会和自然价格完全一致……”(第114页)“因此,自然价格可以说是一个中心点,一切商品的价格都不断趋向这个中心点。各种偶然的情况有时会使商品的价格在某一时期内高于自然价格,而有时又会使它略低于自然价格。”(第116页)}\end{quote}

于是斯密由此得出结论说,总的说来

\begin{quote}{“为了使某种商品进入市场而在一年内使用的勤劳总量”,将同社会的需要,或者说,“实际的需求”相适应。(第117页)}\end{quote}

李嘉图所谓的总资本在各行业之间的分配,在这里还是以生产“某种特定商品”所必需的“勤劳”这一比较素朴的形式出现的。同一种商品的卖者之间的价格平均化为市场价格,以及各种不同商品的市场价格平均化为费用价格,这两种情况在这里还是杂乱地相互交错在一起的。

在这里,斯密只是完全偶然地谈到商品实际价值的变动对“自然价格”,或者说,费用价格的影响。

他是这样说的:

\begin{quote}{在农业中“同量劳动在不同的年份会生产出极不相同的商品量,而在另一些行业中,同量劳动总是会生产出同量或差不多同量的商品。在农业中,同一数量的工人在不同的年份会生产出数量极不相同的谷物、酒、植物油、啤酒花等等。但是同一数目的纺纱工人和织布工人每年会生产出同量或差不多同量的麻布或呢绒……在其他行业〈非农业〉中同量劳动的产品总是相同的或者差不多相同的〈就是说,只要生产条件相同〉,产品能更加准确地适应实际的需求”。(第117—118页)}\end{quote}

在这里,斯密看到了,“同量劳动”的生产率的单纯变动,从而,商品的实际价值的变动,会使费用价格发生变动。可是,他由于把整个问题归结为供求关系,又把这一点庸俗化了。根据他自己的论断,他对这一问题的阐述也是不正确的。因为如果在农业中“同量劳动”由于气候等条件而提供不同量的产品,那末,斯密自己就已经说明,由于机器、分工等等,在工业等部门中“同量劳动”提供的产品量也是极不相同的。可见,农业和其他行业之间的区别并不在这一点上。这种区别在于,在一种场合,“生产力”是“在事先决定了的程度上”被使用,而在另一种场合,生产力却取决于自然界的偶然性。但结果仍然是:商品的价值,或者说,根据劳动生产率必须花费在某种商品上的劳动量,会使商品的费用价格发生变动。

在后面所引的亚·斯密的论点中已经包含这样一种思想,就是资本由一个行业向另一个行业转移,会确立不同行业的费用价格。不过,对于这一点,斯密说得不象李嘉图那样明白,因为如果[552]商品的价格降到其“自然价格”以下,那末,根据斯密的说法,这是由这种价格的要素之一降到自然水平即自然率以下造成的。因此,[要消除商品价格的这种下降,]不是靠单单把资本抽出或转移,而是靠把劳动、资本或者土地从一个部门转到另一个部门。在这里,斯密的观点比李嘉图的观点彻底,不过这种观点是错误的。

\begin{quote}{“不管这种价格〈自然价格〉的哪一部分是低于其自然率支付的,那些利益受影响的人,很快就会感到受了损失,并立即把若干土地,或若干劳动,或若干资本从这种行业中抽出,从而使这种商品进入市场的数量很快只够满足实际的需求。因此,这种商品的市场价格很快就会提高到它的自然价格的水平;至少在有完全自由的地方是这样。”(第125页)}\end{quote}

在这里,斯密和李嘉图对于同“自然价格”趋于一致这一点的理解存在着根本的区别。斯密的理解是以他的错误的前提为基础的,即认为上述三个要素独立地决定商品的价值,而李嘉图的理解是以正确的前提为基础的,即只有平均利润率(在工资既定的情况下)才能确立费用价格。

\tsubsectionnonum{[(2)斯密关于工资、利润和地租的“自然率”的理论]}

\begin{quote}{“自然价格本身随着它的每一构成部分即工资、利润和地租的自然率的变动而变动。”(第127页)}\end{quote}

斯密试图在第一篇第八、九、十章和第十一章确定这些“价格的构成部分”即工资、利润和地租的“自然率”,以及这种自然率的变动。

第八章《论工资》。

在论工资这一章一开头,斯密就抛开虚幻的竞争观点,首先分析剩余价值的真正的本质,把利润和地租看作只是剩余价值的形式。

在考察工资的时候,斯密有一个确定工资的“自然率”的牢固的出发点,即劳动能力本身的价值:必要工资。

\begin{quote}{“一个人总要靠自己的劳动来生活,他的工资至少要够维持他的生存。在大多数情况下,他的工资甚至应略高于这个水平,否则,工人就不可能养活一家人,这些工人就不能传宗接代。”(第136页)}\end{quote}

不过,斯密的这一论点从另一方面来说没有任何意义,因为他从来没有向自己提出这样的问题:必要生活资料的价值,也就是说,商品的价值,是由什么决定的?因为斯密离开了他的基本观点,所以他在这里不得不说:工资的价格是由生活资料的价格决定的,而生活资料的价格是由工资的价格决定的。他先假定工资的价值是固定不变的,接着又准确地描绘了工资价值在竞争中表现出来的波动,以及造成这种波动的那些情况。这属于[斯密观点的]外在部分,在这里和我们没有关系。

{斯密特别描绘了资本的“增长”(积累)[对工资]的影响,但是他没有告诉我们,资本的增长是由什么决定的。因为这种“增长”只有在下述两种情况下才能迅速进行:或者是工资率比较低,而劳动生产率高(在这种情况下,工资的提高始终只是整个前一段时间工资水平低的结果);或者是积累率低[即利润率低],但劳动生产率高。在第一种情况下,斯密从他的观点出发,本应从利润率(即从工资率)得出工资率,而在第二种情况下,则从利润量得出工资率。但是,这又有必要去研究商品价值。}

斯密想从作为商品价值的构成要素之一的劳动价值得出商品价值。另一方面,他又从以下事实得出工资的高度。

\begin{quote}{“……工资并不随着食物价格的波动而波动”(第149页),“各地工资的变动比食物价格的变动大”。(第150页)}\end{quote}

事实上,整个这一章除了最低限度的工资,换句话说,劳动能力的价值这一规定以外,有关的问题一点也没有谈到。在这里,斯密本能地重新提到了他的比较深刻的观点,但是接着又把它抛弃,以致上述规定在他那里并没有产生任何结果。实际上,必要生活资料的价值,也就是说,商品的价值,是由什么决定的呢?部分地由“劳动的自然价格”决定。而劳动的自然价格又是由什么决定的呢?由生活资料的价值,或者说,商品的价值决定。这是可怜地在没有出路的圈子里打转转。此外,这一章没有一个字谈到本题,没有一个字谈到“劳动的自然价格”,[553]只是研究了工资怎样提高到“自然率”的水平以上,也就是说,工资的提高同资本积累的速度,同资本的日益增长的积累成比例。然后研究了产生这种情况的各种社会状况,最后,斯密给了商品的价值决定于工资,而工资的价值决定于必要生活资料的价值这种规定以直接的打击,证明英国的情况似乎不是这样。因为工资不仅决定于维持现有人口的生活所必需的生活资料,而且决定于现有人口的再生产所必需的生活资料,所以这里包含有一些类似马尔萨斯人口论的东西。

这就是,亚·斯密试图证明工资在十八世纪,特别是在英国已经提高之后,提出这样一个问题:应当把这看作“对社会有利还是不利”(第159页)。谈到这里,他又顺便回到他的比较深刻的观点,根据这种观点,利润和地租都只是工人劳动产品的一部分。他说,工人

\begin{quote}{“首先占社会的绝大部分。难道我们什么时候能够认为,这个整体的大部分的命运得到改善,是对这个整体不利的吗?如果社会的绝大部分成员都是贫困的和不幸的,毫无疑问,不能认为这个社会是幸福的和繁荣的。此外,单是从公道出发,也要求使那些供给整个国家吃穿住的人,在他们自己的劳动产品中享有这样一个份额,这一份额至少足够使他们自己获得可以过得去的食物、衣服和住房”。(第159—160页)}\end{quote}

谈到这里,斯密又涉及人口论:

\begin{quote}{“虽然贫困无疑会使人不愿结婚,但它并不总是使人不能结婚;贫困似乎还会促进繁殖……在上层社会的妇女中如此常见的不妊症,在地位低下的妇女中是极少见的……不过,贫困虽然不妨碍生孩子,但是会给抚养儿女造成极大的困难。柔弱的植物出世了,但是出生在那样寒冷的土壤里和那样严酷的气候里,它很快就会枯萎和死亡……各种动物都自然地适应它现有的生存资料的数量而繁殖,没有一种动物的繁殖能够超过这个界限。但是在文明社会,只有在人民的下层阶级中,生存资料的缺乏才能限制人类进一步的繁殖……正象其他任何商品一样,对人的需求必然会调节人的生产;当人的生产过慢的时候,这种需求会使之加速,而当人的生产过快的时候,这种需求就使之缓慢……”(第160—163页)}\end{quote}

最低限度的工资和不同社会状况的关系是这样的:

\begin{quote}{“付给各种短工和仆人的工资,必须足以使他们的繁殖总的来说能够同社会〈社会,也就是资本〉对他们的需求的增加、减少或保持不变相适应。”(第164页)}\end{quote}

斯密接着指出,奴隶比自由工人“贵”,因为后者的“损耗”是由他本人照管,而前者的“损耗”却由“不大经心的主人或玩忽职守的监工”监督。(第163页及以下各页)补偿“损耗”的基金,自由工人使用得很“节约”,而在奴隶那里却由于管理混乱而被浪费:

\begin{quote}{“用来补偿和抵补奴隶劳力因长年服务而造成的可以说是损耗的基金,一般都由不大经心的主人或玩忽职守的监工管理。相反,在自由工人那里,用于同一目的的基金,却由工人自己管理得很节约。富人经营管理中常有的混乱,自然在前一种基金的管理上表现出来;穷人的极度节俭和精打细算,同样自然地表现在后一种基金的管理上。”(第164页)}\end{quote}

在最低限度的工资,或者说,“劳动的自然价格”的规定中,还包括自由雇佣工人的“劳动的自然价格”比奴隶的低这样一点。斯密透露了这个思想:

\begin{quote}{“自由人的劳动归根到底比奴隶的劳动便宜。”(第165页)“如果说优厚的劳动报酬是国民财富增长的结果,那末它也是人口增长的原因。抱怨劳动报酬优厚,[554]就是对最大的公共福利的结果和原因不满。”(第165页)}\end{quote}

接着,斯密为高工资辩护说:

\begin{quote}{高工资“不仅会促进人口的增长”,而且会“增进普通人民的勤劳。工资是对勤劳的奖励,而勤劳,也和人的其他各种特性一样,越是受到奖励就越发展。丰富的食物会增强工人的体力,而改善自己状况……的向往会激励他极端卖力。因此我们看到,工资高的地方的工人总是比工资水平低的地方的工人更积极、更勤勉和更敏捷”。(第166页)}\end{quote}

但是,高工资也会使工人过度劳累,过早地毁坏自己的劳动能力:

\begin{quote}{“领取高额计件工资的工人,很容易进行过度劳动,在不几年内就把自己的健康和劳力毁掉。”(第166—167页)“如果雇主始终听从理性和人道的支配,他倒是常常有理由去节制而不是去鼓励他的许多工人的勤奋。”(第168页)接着,斯密驳斥了“增加福利会使工人懒惰”的说法。(第169页)}\end{quote}

然后,斯密研究了工人在丰年比在荒年懒惰的说法是否正确的问题,并且说明了工资和商品价格之间的关系一般是怎样的情况。这里他又表现出前后矛盾。

\begin{quote}{“劳动的货币价格必然决定于两种情况:对劳动的需求以及必需品和舒适品的价格……劳动的货币价格决定于购买一定量的物品〈必需品和舒适品〉所需要的货币额。”(第175页)}\end{quote}

接着,斯密研究了为什么——由于对劳动的需求——在丰年工资会提高,而在荒年工资会降低。(第176页及以下各页)

在好年景和坏年景,[工资提高和降低的]原因会互相抵销:

\begin{quote}{“物价高涨年份的贫乏,由于减少对劳动的需求,有降低劳动价格的趋势,而食物价格的昂贵又有提高劳动价格的趋势。相反,物价低廉年份的丰裕,由于增加对劳动的需求,有提高劳动价格的趋势,而食物价格的低廉,又有降低劳动价格的趋势。在食物价格发生一般波动的情况下,这两种对立的原因看来会互相抵销;这一点也许部分地说明了,为什么工资到处都比食物价格稳定得多。”(第177页)}\end{quote}

最后,在作了所有这些反复曲折的论证之后,斯密又用他原来比较深刻的观点,即商品价值由劳动量决定的观点,来同工资是商品价值的源泉这一观点相对立;如果说在丰年或资本增长的时候工人得到较多的商品,那末他也生产出多得多的商品,也就是说,在这种情况下单位商品包含的劳动量少了。因此,工人可能得到数量较大而价值较小的商品,由此产生的一个合乎逻辑的结论就是:尽管绝对工资提高,利润还可能增加。

\begin{quote}{“工资的提高,由于使商品价格中分解为工资的部分扩大,必然会使许多商品的价格提高,并且相应地使这些商品在国内外的消费有缩减的趋势。但是,引起工资提高的原因,即资本的增长,又有提高劳动生产能力的趋势,使较小量的劳动能够生产出较大量的产品”……分工,使用机器,发明等等……“由于这一切改良,现在有许多商品已经能够用比以前少得多的劳动来生产了。结果,这种劳动价格的提高,会由于劳动量的减少得到补偿而有余。”(第177—178页)}\end{quote}

劳动得到较好的报酬,但单位商品包含的劳动少了,也就是说,必须支付报酬的劳动量少了。这样,斯密就抛弃了他的错误理论,或者更确切地说,斯密在这里用他的正确理论抵销了、补救了错误的理论;按照他的错误理论,工资作为构成价值的一个要素,决定商品的价值,而按照他的正确理论,商品的价值是由商品中包含的劳动量决定的。

[555]第九章《论资本利润》。

因此,这里应当确定那种决定并构成商品的“自然价格”,或者说,商品的价值的第二个要素的“自然率”。斯密关于利润率下降的原因所说的话(第179、189、190、193、196、197等页)以后再考察。\fnote{见本册第497和533—535页。——编者注}

这里,斯密陷入了极其困难的境地。他说,工资的“平均率”这一规定只能归结为:这是“普通的工资水平”(第179页),即实际上既定的工资水平。

\begin{quote}{“但是对资本利润来说,就连这一点也未必能做到。”(第179页)除了企业主的成功或失败,“这种利润还取决于商品价格的每一次变动”。(第180页)}\end{quote}

然而,我们正是应当通过作为构成“价值”的要素之一的利润的“自然率”,来决定这些商品的“自然价格”。在单个行业,对单个资本家来说,要确定平均利润率已经很困难了。

\begin{quote}{“要确定一个大的王国内所有行业的平均利润,必然更加困难。”(第180页)}\end{quote}

但是,关于“资本的平均利润”,可以“根据货币利息的高低”得出一个概念:

\begin{quote}{“可以确定这样一个原则:凡是从投资中能获得大量利润的地方,通常为使用货币而付出的报酬就多,而在只能获得少量利润的地方,通常为使用货币而付出的报酬就少。”(第180—181页)}\end{quote}

斯密不是说,利息率决定利润率。他所说的显然是相反的意思。但是关于不同时期的利息率等等,我们已有记载,而利润率则没有这种记载。因此,利息率是个征兆,根据它可以大体判断利润率的情况。但任务不是去比较既有的各种利润率,而是要确定“利润的自然率”。斯密避开这个任务而去对不同时期的利息率的水平进行无关紧要的研究,这和他所提出的问题毫不相干。他粗略地描绘了英格兰不同时期的情况,然后拿英格兰同苏格兰、法国、荷兰相比较,发现除美洲殖民地外,

\begin{quote}{“高工资和高利润,自然是很少同时出现的东西,只是在某种新殖民地的特定情况下才会同时出现”。(第187页)}\end{quote}

这里,亚·斯密已经试图几乎象李嘉图那样(但在某种程度上更成功)说明高利润:

\begin{quote}{“新殖民地拥有的资本和领土范围的比例,以及人口和资本量的比例,有一个时期总是要比其他大多数国家小。殖民者所拥有的土地多,而用来开发土地的资本量少;所以,他们所拥有的资本只是用来耕种最肥沃和位置最好的土地,也就是沿海和通航河流两岸的地区。而且购买这种土地的价格,往往低于其自然生长的产品的价值。〈可见,实际上这种土地毫无所值。〉用来购买和改良这种土地的资本,必然会提供很高的利润,因而使用资本也有可能付出很高的利息。在这样有利可图的企业中,这种资本的迅速积累,使种植场主有可能迅速增加自己的工人人数,以致在新的居留地无法找到这样多的工人。因此,他所能找到的工人就会得到优厚的报酬。随着殖民地的不断扩大,资本利润也逐渐下降。当最肥沃和位置最好的土地已全被占有的时候,耕种比较不肥沃和位置比较差的土地,只能提供较少的利润,因而对所使用的资本也只能支付较少的利息。正因为如此……利息率,在本世纪中,在我们的大部分殖民地,都大大降低了。”(第187—189页)}\end{quote}

虽然论证的方法不同,但是这成了李嘉图说明利润下降的基础之一。总之,斯密在这里是用资本的竞争来说明一切,资本一增长,利润就下降,资本一减少,利润就提高,而工资则相反,在前一种场合,工资会提高,在后一种场合,工资会降低。

\begin{quote}{[556]“社会的资本,或者说,用于生产的基金减少,一方面使工人的工资降低,另方面使资本利润提高,从而也使利息率提高。由于工资降低,社会上剩下的资本的所有者就能以比从前少的费用使自己的商品进入市场;由于现在是以较少量的资本实现商品对市场的供应,资本家就能够把自己的商品卖得贵些。”(第191—192页)}\end{quote}

其次,斯密谈到尽可能高的和尽可能低的利润率。

\begin{quote}{“最高的利润率”是这样的利润率,“它从大部分商品的价格中吞并了所有应当归入地租份内的部分,而留下的部分仅仅足够支付生产商品并把商品运到市场所需的劳动的报酬,并且是按照某地最低的工资率支付的,就是说,按照只够维持工人生存的工资率支付的”。(第197—198页)“最低的普通利润率,总是除了足够补偿任何投资都可能遇到的意外损失外,还须略有剩余。只有这个余额才是纯利润。”(第196页)}\end{quote}

实际上,斯密自己对他关于“利润的自然率”的看法作了如下说明:

\begin{quote}{“在英国,人们认为,商人称之为正当的、适度的、合理的利润的,就是双倍的利息;我认为,这些说法的意思无非就是通常的、普通的利润。”(第198页)}\end{quote}

确实,斯密并没有把“通常的、普通的利润”叫作适度的或正当的,但他还是把它称为“利润的自然率”;不过他根本没有告诉我们,这是什么样的东西,或者说,这种利润率是怎样确定的,不过按照斯密的说法,我们就应当利用这种“利润的自然率”来决定商品的“自然价格”。

\begin{quote}{“在财富迅速增加的国家里,在许多商品的价格中,高工资可以用低利润率来弥补,这样,这些国家就能够象它的繁荣程度较低、工资也低的邻国那样便宜地出卖自己的商品。”(第199页)}\end{quote}

低利润和高工资,在这里并不是作为互相影响的东西而彼此对立,二者都是由同一个原因,即资本的迅速增长,或者说,迅速积累造成的。利润和工资都加入价格,构成价格。因此,如果一个高而另一个低,价格就保持不变,等等。

在这里,斯密把利润看作纯粹是[价格的]附加额,因为他在这一章的结尾说:

\begin{quote}{“实际上,高利润比高工资能在大得多的程度上促使产品价格提高。”(第199页)例如,如果在麻织厂工作的所有工人的工资一天各增加2便士,那末,“一匹麻布”的价格将要上涨的数额,只是等于生产这匹麻布所用的工人人数乘2便士,再“乘以工人生产麻布所用的日数。商品价格中分解为工资的部分,由于工资的增加,在生产商品的每一个阶段只按工资增加的算术级数增加。但是,如果所有雇用这些工人的各种企业主的利润都增加5%,那末,商品价格中分解为利润的部分,由于利润率的增加,从一个生产阶段到另一个生产阶段将按利润率增加的几何级数增加……工资提高对商品价格的提高所起的作用,就象单利对债务额的增加所起的作用一样。利润提高所起的作用却象复利”。(第200—201页)}\end{quote}

在这一章的结尾,斯密还告诉我们,他这全部观点,即商品的价格,或者说,商品的价值由工资和利润的价值构成,是从哪里来的;那是从“商业之友”\fnote{原文是《amisducommerce》(傅立叶语)。——编者注},从实际的竞争信奉者那里来的。

\begin{quote}{“我国商人和工业家,对于高工资使商品价格提高,从而减少商品在国内外销路的有害作用,常出怨言;但对高利润的有害作用却默不作声;他们对自己的收入所产生的致命后果保持沉默。[557]他们只是对别人的收入愤愤不平。”(第201页)}\end{quote}

第十章《论劳动和资本的不同使用部门的工资和利润》。它只涉及细节,所以是论述竞争的一章,并且独具特色。它具有完全外在的性质。

{生产劳动和非生产劳动:

\begin{quote}{“法律职业的彩票,是十分不公平的;这一行,象其他大多数自由的、荣誉的职业一样,从金钱收入来说,所得的报偿显然太低了。”(第216—217页)}\end{quote}

他同样谈到士兵:

\begin{quote}{“他们的薪饷比普通短工的工资低,而他们在服役期间的劳累程度却大得多。”(第223页)}\end{quote}

关于水兵:

\begin{quote}{“虽然他们的职业所要求的技能和熟练程度,几乎比其他一切行业都高得多,虽然他们的全部生涯充满着无穷无尽的辛苦和危险……他们的工资却不比海港普通工人的工资高,海港普通工人的工资调节着海员的工资率。”(第224页)}\end{quote}

他讽刺地说:

\begin{quote}{“拿副牧师或礼拜堂牧师同短工比较无疑是不礼貌的。但是,我们完全可以认为,副牧师或礼拜堂牧师的薪俸和短工的工资具有同样的性质。”(第271页)}\end{quote}

至于“文人”,斯密明确地认为,他们由于人数太多而报酬过低,而且他提醒说,在印刷术发明以前,“大学生和乞丐”(第276—277页)是一个意思,看来斯密认为,这在一定意义上也适用于文人。}

这一章充满着锐敏的观察和重要的评论。

\begin{quote}{“在同一社会或同一地区,不同投资部门的平均的、普通的利润率,和不同种类劳动的货币工资相比,大大接近于同一水平。”(第228页)“市场广阔,由于容许使用较多的资本,会使表面利润减少;但是由于要求从更远的地方运来商品,又会使成本增加。这种利润的减少和成本的增加,在许多场合,似乎是接近于互相抵销〈指面包、肉类等商品的价格〉。”(第232页)“在小城市和乡村,由于市场狭小,商业并不能总是随着资本的增长而扩大。因此,在这些地方,虽然个人的利润率可能很高,但是利润的总额或总量决不可能很大,从而他的年积累总额也不可能大。相反,在大城市,营业可能随着资本的增长而扩大,一个勤俭而又交财运的人的信用会比他的资本增长得更快。他的营业会随着二者的增长而日益扩大。”(第233页)}\end{quote}

关于工资水平的一些错误统计材料(例如十六、十七世纪等的),斯密很正确地指出,这里的工资只是例如茅舍贫农的工资。这种茅舍贫农不在自己的小屋里干活或者不为自己的主人劳动的时候(他们的主人给他们“一座小屋,一小块菜地,一块够饲养一头母牛的草地,也许还有一两英亩坏的耕地”,主人叫他们干活的时候,也只付给他们很低的工资),他们

\begin{quote}{“情愿向愿意雇用他们的人提供自己的空闲时间,并且挣比其他工人低的工资”。(第241页)“可是有许多收集关于以前各个时代的劳动价格和食品价格的资料的著作家,非常喜欢把这两种价格说得格外低廉,他们把这种偶然的额外收入看成这些工人的全部工资。”(第242页)}\end{quote}

前面,斯密还作了正确的一般性评论:

\begin{quote}{“劳动和资本在不同部门使用的有利与不利在总体上的平衡,只有在那些被人们作为唯一的或主要的职业来从事的部门中才可能发生。”(第240页)}\end{quote}

不过,这一思想,特别是关于“人们开始珍惜时间”\endnote{马克思指的是詹姆斯·斯图亚特的书《政治经济学原理研究》1770年都柏林版第一卷。在这车书里描写了英国农村从主要是自然经济转变为商品经济和资本主义经济的过程,伴随这一过程发生的是农业变为资本主义经营的一个部门,农业劳动强度的增大和对农村居民的剥夺。斯图亚特的用语“人们开始珍惜时间”见该书第1卷第171页。这句话和摘自斯图亚特的其他引文一起,马克思在1857—1858年经济学手稿中也曾经引用过(见卡·马克思《政治经济学批判大纲》1939年莫斯科版第742页)。——第257页。}以来的农业的工资问题,斯图亚特已经很好地阐明了。

[558]关于中世纪城市资本的积累,斯密在这一章中很正确地指出,它主要来源于(商人和手工业者)对农村的剥削。(还有高利贷者,以及金融贵族,一句话,货币经营者。)

\begin{quote}{“城市工商业居民的每一个集团〈在实行行会制度的城市内〉由于实行这种规约,当然不得不付出略高于没有规约时的价格,向城市其他集团的商人和手工业者购买他们需要的商品。但是,为了弥补这一点,他们也可以按同样较高的价格出卖自己的商品。结果是正如一般所说,贵买贵卖,横竖一样。在城市内各个集团之间进行交易时,他们都不会因这种规约而蒙受任何损失。但在与农村进行交易时,他们却都会得到很大的利益,而城市赖以维持和富裕起来的商业,也就是后面这种交易。每一个城市都从农村取得它的全部粮食和全部工业原料。对这些东西,它主要用以下两种办法来支付:第一,把这种原料的一部分加工以后运回农村,在这种场合,原料的价格中增加了工人的工资和他们的主人或者说直接雇用者的利润;第二,从城市把外国进口或由本国遥远地区运来的原产品或工业品运往农村,在这种场合,这些商品的原来价格中同样要增加水陆运输工人的工资和雇用他们的商人的利润。由第一类商业赚到的钱,构成城市从工业得到的全部利益。由第二类商业赚到的钱,构成城市从国内外贸易得到的全部利益。工人的工资和雇主的利润,构成从这两个部门赚到的钱的全部。因此,目的是要把这些工资和利润提高到它们的自然水平以上的一切规约,其作用就是使城市能够以自己较小量的劳动购买农村较大量劳动的产品。”}\end{quote}

{可见,在最后一句话中,斯密又回到正确的价值规定上来了。这句话在第259页。价值由劳动量决定。在考察斯密对剩余价值的解释时应把这作为一个例子举出来。如果城市和农村相互交换的商品的价格是代表等量劳动,那末商品的价格就等于商品的价值。因此,不论哪一方面的利润和工资都不能决定这些价值,倒是这些价值的分配决定利润和工资。因此,斯密也发现,以较小量劳动交换农村较大量劳动的城市,在同农村的交往中会取得超额利润和超额工资。如果城市不是把自己的商品高于其价值卖给农村,这种情况就不会发生。那样的话,“利润和工资”就不会提高到“它们的自然水平以上”。所以,如果利润和工资处于“它们的自然水平”,那就不是由它们决定商品价值,而是它们自己由商品价值决定。那时,利润和工资就只能从既定的、作为它们前提的商品价值的分配中产生;但是这个价值不能由利润和工资决定,不能从作为价值本身的前提的利润和工资得出来。}

\begin{quote}{“这种规约,造成了城市的商人和手工业者对农村的土地所有者、租地农场主和农业工人的优势地位,并且破坏了城乡贸易中没有这种规约时存在的自然平衡。现有社会的全年劳动总产品,每年都是在这两部分不同的居民之间分配的。由于有这种〈城市的〉规约,城市居民就会得到比没有这种规约时较大的一部分产品,农村居民则得到较小的一部分。城市每年为输入的粮食和原料实际支付的价格,也就是城市每年输出的工业品和其他商品的量。后者卖得越贵,前者就买得越便宜。因此,城市的实业活动就变得比较有利,农村的实业活动则变得比较不利。”(第258—260页)}\end{quote}

这样,按照斯密本人对问题的解释,如果城市和农村的商品都按这些商品各自包含的劳动量出卖,那它们就是按照自己的价值出卖,因而,两方面的利润和工资都不能决定这些价值,倒是利润和工资由商品的价值决定。关于因资本有机构成不同而有所不同的利润的平均化,在这里和我们无关;因为它不仅不会造成利润的差别,反而会使利润趋于同一水平。

\begin{quote}{[559]“城市的居民,由于集中在一个地方,彼此间容易交往和达成协议。因此,城市中甚至最无关紧要的行业,也几乎到处都组成了行会……”(第261页)“农村的居民,由于居住分散,彼此距离较远,就不那么容易结合起来。他们不仅从来没有组织过行会,甚至连行会精神也从来没有在他们中间盛行过。人们从未认为,为了使人能够从事农业这种农村的主要行业,有必要建立学徒制度。”(第262页)}\end{quote}

在这里,斯密还谈到了“分工”的不利方面。农民的劳动,比受分工支配的制造业工人的劳动,具有更大程度的脑力性质:

\begin{quote}{“从事那种必需随着气候的每一变化和其他许多情况的变化而变化的工作,比从事那种同一的或者差不多同一的操作,要求更高得多的判断力和预见性。”(第263页)}\end{quote}

分工使劳动的社会生产力,或者说,社会劳动的生产力获得发展,但这是靠牺牲工人的一般生产能力来实现的。所以,社会生产力的提高不是作为工人的劳动的生产力的提高,而是作为支配工人的权力即资本的生产力的提高而同工人相对立。如果说城市工人比农村工人发展,这只是由于他的劳动方式使他生活在社会之中,而土地耕种者的劳动方式则使他直接和自然打交道。

\begin{quote}{“在欧洲,城市实业活动到处都对农村实业活动占优势,这并不完全是由于行会和行会规约。这种优势还依靠许多其他的规定:对外国工业品和外国商人运来的一切商品课以高额关税,也是为了同样的目的。”(第265页)“这些规定保护着它们〈城市〉不受外国人的竞争。”(同上)}\end{quote}

这已经不是个别城市的资产阶级的行动,而是作为民族的主要部分,或者甚至作为国会的第三等级,或者作为下院,在全国范围内实行立法的那个资产阶级的行动了。城市资产阶级为了反对农村而实行的特别措施,就是消费税和入城税,一般说来,是间接税,这种间接税起源于城市(见休耳曼的著作)\endnote{马克思指的是休耳曼的书《中世纪城市》1826—1829年波恩版第1—4集。——第260页。},直接税则起源于农村。看起来,例如,消费税只是城市间接课在自己身上的税。农村居民据说必须预先缴纳这种税,但他让别人在产品的价格内把它交回来。不过在中世纪,情况并不是这样。对于农村居民劳动产品的需求,——在农村居民要把自己的产品变为商品和货币的情况下,——在多数场合,都被强制地局限于城市范围,所以农村没有可能把城市税总额加到自己产品的价格上去。

\begin{quote}{“在英国,城市实业活动对农村实业活动的优势,过去似乎比现在更大。与上世纪〈十七世纪〉和本世纪〈十八世纪〉初期相比,现在农村工人的工资和工业工人的工资更加接近了,而农业资本的利润也和工商业资本的利润更加接近了。这种变化,可以看作是城市实业活动得到特别鼓励的必然结果,尽管这种结果出现得相当晚。城市积累起来的资本,随着时间的推移,变得如此之大,以致把它投入城市固有的实业中去,已经不可能获得以前的利润了。城市固有的实业,和其他一切实业一样,都有自己的界限,而资本的增长,由于使竞争加剧,必然会降低利润。城市中利润的降低,促使资本流入农村,这就造成对农业劳动的新的需求,从而提高农业劳动的报酬。那时资本就可以说是遍布全国,并在农业中找到用途,于是原来在很大程度上是靠农村积累起来的城市资本又部分地回到了农村。”(第266—267页)}\end{quote}

在第十一章,斯密试图确定构成商品价值的第三个要素即“地租的自然率”。我们准备再回过头去谈一谈李嘉图,然后就考察这一点。

由上所述,很清楚:亚·斯密把商品的“自然价格”,或者说,费用价格和商品的价值等同起来,是由于他事先抛弃了他对价值的正确的观点,而代之以由竞争现象所引起的、来源于竞争现象的观点。在竞争中,并不是价值,而是费用价格作为市场价格的调节者,可以说,作为内在价格——商品的价值出现。而这种费用价格本身在竞争中又作为由工资、利润和地租的既定平均率决定的某种既定的东西出现。因此,斯密也就试图离开商品的价值而独立地确定工资、利润和地租,更确切地说,把它们作为“自然价格”的要素来考察。李嘉图的主要任务是推翻斯密的[560]这种谬误说法,可是他也接受了这种说法的必然的,而如果他前后一贯的话,对他说来是不可能有的后果——把价值和费用价格等同起来。

\tchapternonum{[第十一章]李嘉图的地租理论}

\tsectionnonum{[(1)安德森和李嘉图发展地租理论的历史条件]}

主要的方面在考察洛贝尔图斯的理论时已经阐明了。这里不过再作一些补充。

首先要谈的是历史环境:

李嘉图所考察的时期首先是他差不多完全亲身经历过的1770—1815年,这是小麦价格不断上涨的时期;安德森所考察的时期是十八世纪,他是在这个世纪的末叶写作的。从这个世纪初叶到中叶,小麦价格下降,从中叶到末叶,小麦价格上涨。因此,在安德森看来,他所发现的规律同农业生产率的降低或产品正常的{安德森认为是不自然的}涨价毫无联系。而在李嘉图看来,却肯定是有联系的。安德森认为,谷物法(当时是出口奖励)的废除,是引起十八世纪下半叶价格上涨的原因。李嘉图知道,谷物法(1815年)的实行是为了制止价格下降,并且必然会在一定程度上制止价格下降。因此,李嘉图着重指出,自由发生作用的地租规律必定会——在一定疆域之内——使比较不肥沃的土地投入耕种,从而使农产品价格上涨,使地租靠损害工业和广大居民的利益而上涨。李嘉图在这里无论从实际方面或历史方面来说都是对的。相反,安德森则认为,谷物法(他也赞成进口税)必然会在一定疆域内促进农业的均衡发展;农业的均衡发展需要加以保证;因此,这种前进的发展过程本身,由于安德森所发现的地租规律的作用,必然会引起农业生产率的提高,从而引起农产品平均价格的下降。

但是他们两人都是从一种在大陆上看来非常奇怪的观点出发的,这就是:(1)根本不存在妨碍对土地进行任意投资的土地所有权;(2)从较好的土地向较坏的土地推移(在李嘉图看来,如果把由于科学和工业的反作用造成的中断除外,这一点是绝对的;在安德森看来,较坏的土地又会变成较好的土地,所以,这一点是相对的);(3)始终都有资本,都有足够数量的资本用于农业。

说到(1)、(2)两点,大陆上的人们一定会感到非常奇怪:在这样一个他们看来最顽固地保存了封建土地所有权的国家里,经济学家们——安德森也好,李嘉图也好——却从不存在土地所有权的观点出发。这种情况可用以下两点来解释:

第一,英国的“公有地圈围法”有它的特点,同大陆上的瓜分公有地毫无共同之处;

第二,从亨利七世以来,资本主义生产在世界任何地方都不曾这样无情地处置过传统的农业关系,都没有创造出如此适合自己的条件,并使这些条件如此服从自己支配。在这一方面,英国是世界上最革命的国家。从历史上遗留下来的一切关系,不仅村落的位置,而且村落本身,不仅农业人口的住所,而且农业人口本身,不仅原来的经济中心,而且这种经济本身,凡是同农业的资本主义生产条件相矛盾或不相适应的,都被毫不怜惜地一扫而光。举例来说,在德国人那里,经济关系是由各种土地占有的传统关系、经济中心的位置和居民的一定集中点决定的。在英国人那里,农业的历史条件则是从十五世纪末以来由资本逐渐创造出来的。联合王国的常用术语“清扫领地”,在任何一个大陆国家都是听不到的。但是什么叫做“清扫领地”呢?就是毫不考虑定居在那里的居民,把他们赶走,毫不考虑原有的村落,把它们夷平,毫不考虑经济建筑物,把它们拆毁,毫不考虑原来农业的类别,把它们一下子改变,例如把耕地变成牧场,总而言之,一切生产条件都不是按照它们传统的样子接受下来,而是按照它们在每一场合怎样最有利于投资历史地创造出来。因此,就这一点来说,不存在土地所有权;土地所有权让资本——租地农场主——自由经营,因为土地所有权关心的只是货币收入。一个波美拉尼亚的地主\fnote{暗指洛贝尔图斯。——编者注},脑袋里只有祖传的土地占有、经济中心和农业公会等等,因而对李嘉图关于农业关系发展的“非历史”观点[561]就会大惊小怪。而这只说明他天真地混淆了波美拉尼亚关系和英国关系。可是决不能说,这里从英国关系出发的李嘉图会同那个思想局限于波美拉尼亚关系的波美拉尼亚地主一样眼光短浅。因为英国关系是使现代土地所有权——被资本主义生产改变了形式的土地所有权——得到合适发展的唯一关系。在这里,英国的观点对于现代的即资本主义的生产方式来说具有古典意义。相反,波美拉尼亚的观点却是按照历史上处于较低阶段的、还不合适的形式来评论已经发展了的关系。

不仅如此,大陆上批评李嘉图的人中,大多数甚至是从这样一种关系出发的,在这种关系内,资本主义生产方式,不论合适的或不合适的,根本还不存在。这就好比一个行会师傅想要把亚·斯密的以自由竞争为前提的规律完完全全地应用到他的行会经济上一样。

从较好的土地向较坏的土地推移这个前提,对于劳动生产力的每一个发展阶段,都是象安德森所认为的那样是相对的,而不是象李嘉图所认为的那样是绝对的;这个前提只有在象英国这样一个国家才能产生,在那里,资本在一个相对来说很小的疆域内如此残酷无情地实行统治,几百年来毫不怜惜地极力使一切传统的农业关系完全适合于自己。因此,只有在农业的资本主义生产不是象大陆那样从昨天才开始的地方,只有在它已经不用同旧传统作斗争的地方,这个前提才能产生。

第二个情况是,英国人有一种从他们的殖民地得来的观点。我们已经看到\fnote{见本册第253—254页。——编者注},李嘉图整个观点的基础在斯密的著作中——在直接论述殖民地的地方——已经有了。在这些殖民地——特别是在专门生产交易品如烟草、棉花、糖等而不生产普通食物的殖民地,在那里,殖民者一开头就不是谋生,而是建立商业企业,——具有决定意义的,在位置既定的条件下自然是肥力,在肥力既定的条件下自然是土地的位置。殖民者的做法不象日耳曼人,日耳曼人在德国住下来,是为了在那里定居,殖民者则象这样一种人,他们按照资产阶级生产的动机行事,他们想要生产商品,他们的出发点从一开头就不是决定于产品,而是决定于出卖产品。李嘉图和其他英国著作家把这种从殖民地得来的观点,也就是从本身已经是资本主义生产方式的产物的人们那里得来的观点,移到了世界历史的整个进程中来,他们象他们的殖民者一样,一般地把资本主义生产方式看作农业的先决条件,其所以如此,就是因为,他们在这些殖民地,一般说来,只是在更加鲜明的形式上,在没有同传统关系斗争的情况下,因而在没有被弄模糊的形式上,发现了在他们本国到处可以看到的资本主义生产在农业中占统治地位的同样现象。因此,如果一个德国教授或地主(他的国家和其他国家不同之点就是根本没有殖民地)认为这样的观点是“错误的”,那是完全可以理解的。

最后,资本不断从一个生产部门流入另一个生产部门这个前提,这个李嘉图的基本前提,无非就是发达的资本主义生产占统治地位这样一个前提。在资本主义生产的统治还没有建立的地方,这个前提就不存在。例如,一个波美拉尼亚地主,对于李嘉图和其他英国著作家居然没有想到农业会缺乏资本,一定感到奇怪。英国人当然会抱怨土地同资本相比显得缺乏,但是从来不抱怨资本同土地相比显得缺乏。威克菲尔德、查默斯等人想用前一种情况来说明利润率下降。没有一个英国著作家提到后一种情况,在英国,就象柯贝特当作不言而喻的事实指出的那样,资本在所有部门中始终都是绰绰有余的。如果设想一下德国的情况,设想一下土地所有者借钱时的困难,——因为他多半是自己经营农业,而不是由一个完全独立于他的资本家阶级经营农业,——那就可以理解,例如洛贝尔图斯先生为什么会对“李嘉图的虚构——资本储备适应于对投资的渴望”(《给冯·基尔希曼的社会问题书简。第三封信》1851年柏林版第211页)表示惊讶。如果说英国人有什么感到不足,那就是“活动场所”,就是供现有资本储备投放的场所。但是,在英国,对于要投资的唯一阶级即资本家阶级来说,对用于“投放”的“资本的渴望”是不存在的。

[562]这种“对资本的渴望”是波美拉尼亚人的。

英国著作家们拿来反驳李嘉图的,不是资本没有足够的储备以供各种特殊投资之用,而是资本从农业流出会遇到特殊的技术等等方面的困难。

因此,上述用大陆的批判眼光对李嘉图吹毛求疵,只是证明那些“聪明人”是从生产条件较低的阶段出发的。

\tsectionnonum{[(2)李嘉图的地租理论同他对费用价格的解释的联系]}

现在来谈问题本身。

首先,为了在纯粹的形式上理解问题,我们必须把李嘉图那里唯一存在的级差地租完全撇开。我所说的级差地租,是指由于不同等级土地的肥力不同而产生的地租量的差别——较多的或较少的地租。(如果肥力一样,级差地租只能由于投资量不同而产生。就我们研究的问题来说,这种情况不存在,与问题无关。)这种级差地租完全相当于超额利润,就是在每一工业部门,例如在棉纺业中,在市场价格既定时,或者更确切地说,在市场价值既定时,生产条件比这个生产部门的平均条件好的那个资本家赚得的超额利润,因为一定生产领域的商品的价值不是决定于单个商品所耗费的劳动量,而是决定于在该领域的平均条件下生产的那个商品所耗费的劳动量。这里,工业和农业不同之处只是:在工业中超额利润落进资本家自己的腰包,而在农业中落进土地所有者的腰包;其次,超额利润在工业中是流动的、不稳定的,时而由这个资本家赚去,时而由那个资本家赚去,并且又不断地消失,而超额利润在农业中,却由于有土地差别这种稳定的(至少在一段较长的时间内)自然基础而固定下来。

总之,我们要把这种级差地租撇开,但是要指出,不论是从较好的土地向较坏的土地推移,还是从较坏的土地向较好的土地推移,级差地租同样是可能的。在两种情况下只假定,为了满足追加需求,新耕地是必要的,但是它只要够满足追加需求就行了。假如新耕种的较好的土地能够满足的需求大于这个追加需求,那末,按照追加需求的大小,必将有部分或全部坏地停止耕种,至少在这些土地上不再种植成为农业地租的基础的产品,也就是说,在英国不再种植小麦,在印度不再种植水稻。因此,级差地租并不以农业的不断恶化为前提,它也可以从农业的不断改良产生。即使在级差地租以向较坏土地推移为前提的地方,第一,这种按下降序列推移可能是由于农业生产力的改良,因为,在需求所容许的价格之下,只有较高的生产力才使耕种较坏的土地成为可能。第二,较坏的土地可以改良,不过差别仍然会存在,尽管这个差别在很大程度上被抵销了,结果,发生的只是生产率的相对的、比较的降低,可是绝对的生产率提高了。这甚至是第一个提出李嘉图规律的安德森的前提。

其次,这里应当考察的仅仅是真正的农业地租,就是提供主要植物性食物的土地的地租。斯密已经说明,提供其他产品(例如畜产品等等)的土地的地租,是由上述地租决定的,因而已经是派生的地租,它们由地租规律决定,而不是决定地租规律;所以就其本身来考察,它们是不能提供任何材料来理解最初的、纯粹的条件下的地租规律的。其中没有什么第一性的东西。

上述这些解决了之后,问题就归结为:是否存在绝对地租?就是说,是否存在由资本投入农业而不是投入工业产生的、同投入较好土地的资本所提供的级差地租即超额利润完全无关的地租?

很清楚,李嘉图既然从商品价值和商品平均价格等同这个错误前提出发,他理所当然地要对这个问题作否定的答复。如果接受这个前提,那末,下面的说法便是同义反复:如果[563]农产品的固定价格除了提供平均利润外还提供地租,提供一个超过这个平均利润的经常余额,那末农产品的价格就高于它们的费用价格,因为这个费用价格等于预付加平均利润,再无其他。如果农产品的价格高于它们的费用价格,必然提供一个超额利润,那末,按照上述前提,农产品的价格也就会高于它们的价值。这除了承认农产品经常高于它们的价值出卖以外,就再没有别的了,但是这也就等于假定其他一切产品都是低于它们的价值出卖,或者说,一般说来价值同从理论上对它的必然的理解是完全不同的东西。同量劳动(直接劳动和积累劳动)——把各个资本之间由于它们在流通过程中产生的差别而发生的一切平均化现象都考虑进去——在农业中生产的价值会比在工业中生产的价值高。因而商品的价值就不是由商品中包含的劳动量来决定了。这样一来,政治经济学的整个基础就被推翻了。因此,李嘉图理所当然地得出结论说,不存在绝对地租。只可能有级差地租;换句话说,最坏土地所生产的农产品的价值,同其他任何商品的价值一样,等于产品的费用价格。投在最坏土地上的资本,是一种仅仅在投资方式上,仅仅作为特种投资,与投在工业中的资本不同的资本。因此这里表现出价值规律的普遍适用性。级差地租——而这是较好土地上的唯一地租——不过是生产条件比平均条件好的资本由于在每一个生产领域有一个相同的市场价值而提供的超额利润。这种超额利润,由于农业的自然基础,只有在农业中才固定下来;而且,因为这个自然基础的代表是土地所有者,所以这种超额利润不是落入资本家的腰包,而是落入土地所有者的腰包。

李嘉图的费用价格等于价值这个前提不成立,他的所有这些论证也就不成立。那种迫使他否定绝对地租的理论兴趣也就丧失。如果商品的价值不同于商品的费用价格,如果所有商品必然分成三类:一类商品的费用价格等于它们的价值,另一类商品的价值低于它们的费用价格,第三类商品的价值高于它们的费用价格,那末,农产品价格提供地租这种情况,只不过证明农产品属于价值高于费用价格的一类商品。唯一有待解决的问题是:为什么农产品跟其他那些价值同样高于费用价格的商品不同,它们的价值不因资本的竞争而降低到它们的费用价格的水平?答案已经包含在问题里了。因为,按照假定,这种情况只有在资本的竞争能够实现这种平均化的时候才发生,而实现平均化又只有在一切生产条件由资本本身创造出来,或者作为自然要素同样受资本支配的时候才有可能。对土地来说不发生这种情况,因为存在着土地所有权,资本主义生产是在存在土地所有权的前提下开始的,而土地所有权不是从资本主义生产中产生的,它在资本主义生产之前就已经存在。因此,单单土地所有权的存在本身就给问题作了答复。资本所能做的一切,就是使农业服从资本主义生产的条件。但是,资本主义生产不能剥夺土地所有权占有一部分农产品的可能性,这部分农产品资本要据为己有,就不是靠它自己的活动,而只有靠没有土地所有权存在这个前提。在土地所有权存在的条件下,资本就不得不把价值超过费用价格的余额让给土地所有者。但是,这个价值和费用价格之间的差额本身,仅仅是从资本有机组成部分的比例不同产生出来的。因此,凡是按照这种有机构成价值高于费用价格的商品都表明,同价值等于费用价格的商品相比,生产它们的劳动生产率相对地说比较低,而同价值低于费用价格的商品相比,劳动生产率则更低;这是因为,它们需要较大量的直接劳动(同包含在不变资本中的过去劳动相比),需要有较多的劳动去推动一定量资本。这个差别是历史性的,因此是会消失的。正是那个证明绝对地租可能存在的论据也证明,绝对地租的现实性、绝对地租的存在仅仅是一个历史事实,是农业的一定发展阶段所特有的、到了更高阶段就会消失的历史事实。

李嘉图用农业生产率的绝对降低来说明级差地租,而这种降低完全不是级差地租的前提,安德森也没有把它当作前提。李嘉图否定绝对地租,这是因为他[564]以工业和农业的资本有机构成相同为前提,从而他也就否定了农业劳动生产力同工业相比处于只是历史地存在的较低发展阶段。因此他犯了双重历史错误:一方面,把农业和工业中的劳动生产率看成绝对相等,因而否定它们在一定发展阶段上的仅仅是历史性的差别,另一方面,认为农业生产率绝对降低,并把这种降低说成是农业的发展规律。他这样做一方面是为了把较坏土地的费用价格同价值等同起来;另一方面是为了说明较好土地的产品的[费用]价格同价值之间存在差额。全部错误的产生都是由于混淆了费用价格和价值。

这样,李嘉图的理论也就被排除了。其他方面,我们在前面考察洛贝尔图斯的理论时已经说过了。

\tsectionnonum{[(3)李嘉图的地租定义不能令人满意]}

我已经指出\fnote{见本册第185页。——编者注},李嘉图在论地租的那一章一开头就说,应当研究“对土地的占有以及由此而来的地租的产生”(《政治经济学和赋税原理》1821年伦敦第3版第53页)是否同价值决定于劳动时间这一规定相矛盾。接着他又说:

\begin{quote}{“亚当·斯密认为,调节商品交换价值的基本尺度,即生产商品所用的相对劳动量,会由于土地的占有和地租的支付而完全改变,这个看法不能说是正确的。”(第67页)}\end{quote}

李嘉图把地租理论同价值规定直接地、有意识地联系起来,这是他的理论贡献。在其他方面,第二章《论地租》可以说比威斯特的论述还要差。这里有许多值得怀疑的东西,有petitioprincipii〔本身尚待证明的论据〕以及对待问题的不公正态度。

就真正的农业地租——这里,李嘉图把这种地租正确地看作是真正意义上的地租——来说,地租是为了获得许可在土地这个生产要素上投资,以资本主义方式进行生产而支付的东西。土地在这里是生产要素。至于例如建筑物、瀑布等的地租,情况就不同了。这里,为了加以使用而支付地租的自然力,是作为生产条件参加生产的,不论是作为生产力或者是作为不可缺少的条件,但是它们不是这一特定生产领域本身的要素。其次,说到矿山、煤矿等的地租,土地则是可从其中挖掘使用价值的储藏库。这里为土地支付地租,并不是因为土地象在农业中那样作为可以在其上进行生产的要素,也不是因为土地象瀑布和建筑地段那样作为生产条件之一加入生产过程,而是因为土地作为储藏库蕴藏着有待通过生产活动来取得的使用价值。

李嘉图的定义:

\begin{quote}{“地租是为使用土地原有的和不可摧毁的力而付给土地所有者的那一部分土地产品。”(第53页)}\end{quote}

这是不能令人满意的。第一,土地并没有“不可摧毁的力”。(关于这一点在本章末尾要作个注。)第二,土地也不具有“原有的”力,因为土地根本就不是什么“原有的”东西,而是自然历史过程的产物。但是,我们且不管这个。所谓土地的“原有的”力,在这里应该理解为土地不依赖于人的生产活动而具有的力,虽然从另一方面说,通过人的生产活动给它的力,完全同自然过程赋予它的力一样要变成它的原有的力。除此以外,下面这一点还是对的,即地租是为“使用”自然物而支付的,完全不管这里所说的是使用土地的“原有的力”,还是瀑布落差的能量,或者是建筑地段,或者是水中或地下蕴藏的有待利用的宝藏。

为区别于真正的农业地租,亚·斯密(李嘉图指出)谈到为原始森林的木材支付的地租,谈到为煤矿和采石场支付的地租。李嘉图排除这种地租的方法是相当奇怪的。

李嘉图开头说不应该把资本的利息和利润同地租混淆起来(第53页),这种资本是指

\begin{quote}{“原先用于改良土壤以及建造为储存和保管产品所必需的建筑物而支付的资本”。(第54页)}\end{quote}

李嘉图从这里立刻转到上面提到的亚·斯密所举的例子。关于原始森林,李嘉图说:

\begin{quote}{“但是,支付他〈斯密〉所谓的地租的人,是为了当时已经长在地上的有价值的商品而支付这个地租的,而且通过出卖木材实际上已收回自己所付的钱并获得利润,这不是很明显的吗?”(第54页)}\end{quote}

关于采石场和煤矿的情况也是一样:

\begin{quote}{“为[565]煤矿或采石场支付的报酬,是为了可以从那里开采的煤或石料的价值而支付的,它和土地原有的和不可摧毁的力没有任何关系。这种区别在地租和利润的研究中极为重要;因为很清楚,决定地租发展的规律同决定利润发展的规律是大不相同的,并且也很少朝着相同的方向发生作用。”(第54—55页)}\end{quote}

这是非常奇怪的逻辑。李嘉图说,要把为使用“土地原有的和不可摧毁的力”而付给土地所有者的地租,同那为了他在改良土地等方面的投资而付给他的利息和利润区别开来。为了取得“采伐”木材的权利而付给自然森林所有者的“报酬”,或为了取得“开采”石料和煤的权利而付给采石场和煤矿所有者的“报酬”,不是地租,因为它不是为“使用土地原有的和不可摧毁的力”而支付的。很好!可是李嘉图在他的议论中却把这种“报酬”说成好象同那为改良土地而进行的投资的利润和利息是一回事!而这是完全错误的!原始森林所有者向“原始森林”投过“资本”让它生产“木材”吗?或者,采石场和煤矿的所有者向采石场和煤矿投过“资本”让它们蕴藏“石料”和“煤”吗?那末他得到的“报酬”来自何处呢!这种报酬在任何场合都不象李嘉图想偷换的那样是资本的利润或利息。因此,它是“地租”,而不是别的,尽管它不是李嘉图的地租定义所指的那种地租。但是,这不过表明李嘉图的地租定义排除了某些形式,在这些形式中,“报酬”是为了不体现任何人的劳动的单纯自然物而支付的,并且是支付给这些自然物的所有者,而且仅仅因为他是个“所有者”,是土地所有者,不管这块土地是耕地、森林、鱼塘、瀑布、建筑地段等等。但是,李嘉图说,为了取得在原始森林中伐木的权利而支付的人,支付“是为了当时已经长在地上的有价值的商品,而且通过出卖木材实际上已收回自己所付的钱并获得利润。”且慢!如果李嘉图这里把原始森林中“长在地上的”树木称作“有价值的商品”,这不过是说,它就可能性来说是使用价值。这个使用价值在这里用“有价值的”一词表达出来。但是它不是“商品”。因为要成为商品,它就必须同时是交换价值,就是说,它必须是耗费在它上面的一定量劳动的体现。只是由于把它从原始森林分离开来、伐倒、搬动、运走,由树干变成木材,它才变成商品。或者说,它变成商品,仅仅因为被出卖吗?这样的话,耕地岂不是也可以仅仅因为出卖的行为就变成商品了吗?

因而,我们就应该说:地租是为了取得使用自然力或者(通过使用劳动)占有单纯自然产品的权利而付给这些自然力或单纯自然产品的所有者的价格。实际上,这也就是所有地租最初表现的形式。但是这样一来,就还有一个问题要解决:没有价值的东西怎么会有价格,这又怎么同一般价值理论一致。至于为取得在生长树木的土地上采伐木材的权利而支付“报酬”的人抱什么目的,这个问题同实际的问题毫无关系。问题是:他是用什么基金支付的?李嘉图说,“通过出卖木材”,也就是说用木材的价格。而且这个价格,照李嘉图说,使这个人“实际上已收回自己所付的钱并获得利润”。因此,现在我们知道问题究竟在哪里了。木材的价格至少必须等于代表伐木、搬动、运输和把木材送到市场所必需的劳动量的货币额。那末,这个人在“收回自己所付的钱”时获得的利润,是不是这个价值的附加额,这个只是现在由耗费在木材上的劳动赋予木材的交换价值的附加额呢?如果李嘉图这样说的话,他就退到低于他自己的学说水平的最粗俗的观念上去了。决不是的。假定这个人是一个资本家,利润就是他在“木材”生产上使用的劳动中他没有付酬的部分,我们可以说,如果这个人把同量劳动用在棉纺工厂中,他会赚到同量利润。(如果这个人不是资本家,那末利润等于他超出补偿其工资之外的那部分劳动量,这部分劳动量,如果有一个资本家雇用他的话,就会成为资本家的利润,而现在却成为他自己的利润,因为他既是他自己的雇佣工人,又是他自己的资本家,一身兼而有之。)但是这里用了荒谬的说法,说这个木材业者“实际上已收回自己所付的钱并获得利润”。这就使整个事情具有十分平庸的性质,同这个经营木材的资本家自己对他的利润来源所能持有的粗俗观念相吻合。他首先为树木的使用价值向原始森林的所有者支付报酬,但是树木是没有“价值”(交换价值)的,并且,只要它还“长在地上”,它就连使用价值都没有。假定他向原始森林所有者每吨支付5镑。然后他按6镑(他的其他费用不计在内)把这些木材卖给别人,这样实际上收回5镑并获得20%的利润。“实际上已收回自己所付的钱并获得利润。”如果原始森林所有者只要2镑(40先令)“报酬”,木材业者就会按每吨2镑8先令而不是按6镑卖出去。[566]因为他总是按同一利润率来加价的,所以这里木材价格的高低取决于地租的高低。地租是作为构成要素加入价格,而决不是价格的结果。不论支付“地租”(“报酬”)给土地所有者是为了使用土地的“力”,还是为了“使用”土地的“自然产品”,都丝毫不改变经济关系,不改变它是为过去没有花费过人的劳动的“自然物”(土地的力或产品)支付的。这样,李嘉图在他《论地租》一章的第二页上,为了回避困难,就推翻了他的整个理论。看来,亚·斯密在这个问题上的见解要透彻得多。

关于采石场和煤矿,情况也是一样。

\begin{quote}{“为煤矿或采石场支付的报酬,是为了可以从那里开采的煤或石料的价值而支付的,它和土地原有的和不可摧毁的力没有任何关系。”(第54—55页)}\end{quote}

没有任何关系!但是这种报酬和“土地原有的和可以摧毁的产品”有很重要的关系。这里的“价值”一词同前面的“已收回自己所付的钱并获得利润”同样荒谬。

李嘉图从来不用价值这个词来表示效用或有用性或“使用价值”。因此,他是不是想说,把“报酬”付给采石场和煤矿所有者,是为了煤和石料在它们从采石场和煤矿开采出来以前即在它们的原始状态就有的“价值”呢?如果是这样,李嘉图就推翻了他的整个价值学说。或者,就象本来应当说的那样,价值在这里是指煤和石料的可能的使用价值,因此也就是它们的预期的交换价值呢?如果是这样,这就不过是说,把地租付给煤和石料的所有者是为了获得许可使用“土地的原有成分”来开采煤和石料。可是,为什么这不应当象为了获得许可使用土地的“力”来生产小麦时一样也叫作“地租”呢,这就完全不能理解了。不然的话,我们又会看到象前面在木材的例子上分析过的那种推翻整个地租理论的情况了。按照正确的理论,问题完全没有困难。用在“生产”{不是再生产}木材、煤和石料上的劳动(这种劳动的确没有创造这些自然产品,但是它把这些自然产品从它们同土地的原始联系中分离开来,因而把它们作为可用的木材、煤和石料“生产”出来)或资本显然属于这样的生产领域,在这些生产领域中,资本中用于工资的部分大于用于不变资本的部分,直接劳动大于“过去”劳动——其成果用作生产资料。因此,如果商品在这里按照它的价值出卖,这个价值就高于它的费用价格,就是说高于工具的磨损、工资和平均利润。所以,余额可以作为地租付给森林、采石场或煤矿的所有者。

但是,为什么李嘉图要耍这些拙劣的手法,错误地使用“价值”这个词等等呢?为什么他死抓住这样的地租定义即地租是为使用“土地原有的和不可摧毁的力”而支付的呢?我们在后面也许会找到答案。无论如何,李嘉图是想把真正的农业地租区分出来,强调它的特点,同时指出,这些原有的力只有当它们达到不同的发展程度时才能得到报酬,借此为级差地租奠定基础。

\tchapternonum{[第十二章]级差地租表及其说明}

\tsectionnonum{[(1)地租量和地租率的变动]}

对于前面所说的还要补充如下:

假定发现了比较富饶的或位置较好的煤矿和采石场,它们在使用同量劳动的情况下比老的煤矿和采石场能提供更多的产品,并且产量足以满足全部需求。这时,煤炭、石料和木材的价格就会下降,因为它们的价值会下降。老的煤矿和采石场必然因此停闭。它们将不能提供利润,不能提供工资,也不能提供地租。然而新的煤矿和采石场必然会象以前老的那样提供地租,尽管提供的(从地租率上看)比较少些。因为,劳动生产率每提高一步,花费在工资上的资本同不变资本(这里是指用在生产工具上的资本)对比起来就减少。这种说法对吗?如果劳动生产率的变动不是生产方式本身的变动引起,而是煤矿或采石场的自然富饶程度或它们的位置引起,这种说法对吗?在这里我们唯一能够说的就是,同量资本在这里提供吨数更多的煤炭或石料,因此,在每一吨中包含较少的劳动,但是,所有吨数加在一起就包含同样多的或者甚至更多的劳动,——如果新的煤矿或采石场除了满足以前由老的煤矿或采石场满足的原有需求以外,还能满足一个追加的需求,即比新老矿、场富饶程度的差额还要大的需求。可是使用的资本的有机构成并不因此改变。的确,在一吨的价格中,在单独的一吨的价格中,将包含较少的地租,但这只是因为一般说来在其中包含较少的劳动,也就是包含较少的工资和较少的利润。可是,地租率对利润之比并不因此受到影响。因此,我们只能[567]说:

如果需求不变,也就是说,如果要生产同以前一样多的煤炭和石料,那末,为了生产同一商品量,现在在新的比较富饶的煤矿和采石场中使用的资本,就比以前在老的矿、场中使用的少。于是商品总量的总价值就下降,地租、利润、工资和使用的不变资本的总量也因此减少。但是地租和利润之间的比例,就象利润和工资之间的比例或利润和投资之间的比例一样不会改变,因为在使用的资本中没有发生任何有机的变动。改变了的只是使用的资本的量,不是使用的资本的构成,因而也不是生产方式。

如果有追加需求要满足,但是这个追加需求等于新老矿、场富饶程度的差额,那就使用和以前同样大小的资本。每一吨的价值减少了。但是总吨数仍有和以前同样的价值。就每一吨来看,随着其中包含的价值的减少,价值中转化为利润和地租的那部分的量也减少。但是,因为资本的量以及它的产品的总价值没有变,资本构成中也没有发生有机的变动,所以地租和利润的绝对量不变。

如果追加需求很大,在投资照旧的条件下,新老矿、场富饶程度的差额不能满足这一需求,那末在新矿中必须使用追加资本。在这种情况下,——如果在分工和机器使用方面没有随着总投资的增加而发生变动,也就是说,如果资本有机构成没有任何变动,——地租和利润的量就增加,因为总产品的价值、总吨数的价值增加了,尽管每一吨的价值减少了,就是说每一吨价值中转化为地租和利润的那一部分也减少了。

在所有这些情况下,地租率都没有发生任何变动,因为使用的资本的有机构成没有变动(不论资本的量如何变动)。相反,如果变动是由于资本有机构成的变动,是由于花费在工资上的资本同花费在机器等方面的资本相比有所减少,——因而生产方式本身也发生变动——那末地租率就会下降,因为商品价值和费用价格之间的差额缩小了。在上面考察的三种情况中,这个差额并没有缩小。因为,如果价值下降,那末,由于在单位商品上耗费的劳动(有酬劳动和无酬劳动)较少,单位商品的费用价格则同样下降。

由此可见,如果劳动生产率提高——或者说,生产出来的一定量商品的价值减少——仅仅是由自然要素的富饶程度的变动引起的,是由土地、矿山、采石场等的自然富饶程度不同引起的,那末,地租量可以由于在改变了的条件下使用的资本量减少而减少;地租量可以由于有追加需求而保持不变;地租量可以由于追加需求大于原来使用的自然因素和现在使用的自然因素的富饶程度之间的差额而增长。但是,地租率只有在使用的资本的有机构成发生变动的情况下才能增长。

因此,当放弃较坏的土地、较次的采石场、较次的煤矿等的时候,地租量不一定下降。而且,如果这种放弃只是它们的自然富饶程度较低的结果,地租率甚至永远不会下降。

在这种场合,说地租量在一定的需求情况下可能下降,就是说,地租量的变动取决于使用的资本量是减少、不变还是增加,这是正确的看法。但是,说地租率一定下降,那是根本错误的看法,在这种前提下,这是不可能的,因为已经假定,资本有机构成没有发生任何变动,也就是说,没有发生足以使价值和费用价格之间的比例受到影响的变动,而这个比例是决定[绝对]地租率的唯一比例。李嘉图十分荒谬地把正确的看法同根本错误的看法混在一起了。

\tsectionnonum{[(2)级差地租和绝对地租的各种组合。A、B、C、D、E表]}

但是,在上述场合,级差地租是什么情况呢?

假定,开采的煤矿有I、II、III三个等级,其中I提供绝对地租,II提供的地租两倍于I,III提供的地租两倍于II,或四倍于I。在这种场合,I提供绝对地租R,II提供地租2R,III提供地租4R。假定现在开采IV,它比I、II、III更富饶,按其规模来说,可以容纳与投入I的资本同样大小的资本。在这种场合,如果需求不变,以前投入I的资本就投入IV。于是I将停闭。投入II的资本有一部分必然抽出。IV足以代替I并代替II的一部分,但是,如果II的一部分不继续开采,III和IV就不能满足全部需求。为了用具体例子说明这一切,我们假定,IV使用的资本同以前投入I的资本一样多,它能提供I的全部产量和II的一半产量。因此,如果对II投入原来资本的一半,对III投入原来的资本,加上投在IV上的新资本,就足以供给整个市场。

[568]在这种情况下,发生的变化是什么呢,或者说,这些变化对地租总额,对I、II、III、IV的地租有什么影响呢?

从IV得到的绝对地租的量和率,同以前从E得到的完全相同;实际上,以前在I、II、III中,绝对地租的量和率本来就是相同的,如果我们始终假定这些不同的等级使用的是同量资本。IV的产品的价值和以前I的产品的价值完全相等,因为它是大小相同和有机构成相同的资本的产品。因此,价值和费用价格之间的差额必定相同;因而地租率也必定相同。此外,地租量也必定相同,因为——在地租率既定的情况下——使用的是同样大小的资本。但是,因为煤的[市场]价值不决定于IV所生产的煤的[个别]价值,所以IV就提供超额地租,或者说,提供超过它的绝对地租的余额;这种地租,不是来自价值和费用价格之间的差额,而是来自IV的产品的市场价值和个别价值之间的差额。

如果我们说,在投入I、II、III、IV的资本量相同,因而在地租率既定时地租量也相同的条件下,它们的绝对地租,或者说,价值和费用价格之间的差额,是相同的,那末这句话应当理解为:煤的(个别)价值,I高于II,II高于III,因为在I的一吨煤中比II的一吨煤中包含较多的劳动,在II的一吨煤中比III的一吨煤中包含较多的劳动。但是,既然资本的有机构成在三种场合都是一样,这个差别就不影响I、II、III提供的个别绝对地租。因为,I的一吨的价值较大,它的费用价格也较大;大的程度,只是同I生产一吨所用的具有同样有机构成的资本大于II的程度、II大于III的程度成比例。因此,它们的价值的这个差别恰恰等于它们之间的费用价格的差别,就是说,等于在I、II、III中为生产一吨煤所花费的相对资本的差别。因此,三个等级的价值量的差别不影响这些不同等级的价值和费用价格之间的差额。如果价值较大,费用价格也相应地较大,因为价值的增大,只是同资本或劳动耗费的增大成比例;因此价值和费用价格之间的比例仍然不变,也就是说绝对地租仍然不变。

但是,我们进一步看看级差地租是什么情况。

首先,在II、III、IV的煤的全部生产上现在用了较少的资本。因为IV的资本同I的资本一样大,而用在II上的资本抽出一半;因此,II的地租量无论如何减少一半。在投资方面只有II发生了变化,因为IV的投资同以前I的投资一样大。此外,我们曾经假定,对I、II、III投入的是等量资本,例如,都是100镑,合计是300镑;因而现在II、III、IV总共只有250镑,换句话说,有六分之一的资本已经从煤的生产中抽出。

其次,煤的市场价值下降了。我们前面看到,I提供R,II提供2R,III提供4R。假定,I花费100镑生产出来的产品价值等于120镑,其中10镑是地租,10镑是利润,那末,II的市场价值是130镑(10镑利润和20镑地租),III的市场价值是150镑(10镑利润和40镑地租)。如果I的产品等于60吨(每吨等于2镑),那末,II的产品等于65吨,III的产品等于75吨,总产量等于60+65+75=200吨。现在,因为IV的100镑生产出来的产品等于I的产品的全部和II的产品的一半,就是60+[32+(1/2)]=92+(1/2)吨,那末,这92+(1/2)吨照原来的市场价值就值185镑,因为利润等于10镑,所以提供的地租是75镑;因为绝对地租等于10镑,所以IV的地租量就等于[7+(1/2)]R。

同以前一样,II、III、IV生产的还是200吨煤,因为[32+(1/2)]+75+[92+(1/2)]=200吨。但是,现在市场价值和级差地租又是什么情况呢?

要答复这个问题,我们就要看看II的绝对个别地租量多大。我们假定,在这个生产领域中价值和费用价格之间的绝对差额等于10镑,就是说,等于原来最次矿提供的地租,——虽然情况不一定是这样,除非I的价值绝对地决定市场价值。[569]如果实际上发生这种情况,那末I的地租(在I的煤按其价值出卖的情况下)一般说来就代表这个生产领域的价值超过它[I的煤]自己的费用价格和商品的一般费用价格的余额。因此,如果II把它的65吨卖120镑,也就是每吨卖1+(11/13)镑,II就是按照产品的价值出卖自己的产品。过去它的一吨所以不卖1+(11/13)镑,而卖2镑,那只是因为存在一个由I决定的市场价值超过它[II的煤]的个别价值的余额,存在它[II的煤]的市场价值(而不是它的价值)超过它的费用价格的余额。

其次,根据假定,II现在出卖的不是65吨,而只是32+(1/2)吨,因为投入煤矿的资本不是100镑,而只是50镑。

因此,II现在出卖32+(1/2)吨得到的是60镑。10镑对50镑[预付资本]之比是20%。60镑中有5镑是利润,5镑是地租。

这样,II的情况是:每吨产品价值1+(11/13)镑;吨数32+(1/2)吨;总产品价值60镑;地租5镑。地租从20镑降到5镑。如果还是用同量资本,地租就只降到10镑。因而地租率只降了一半。换句话说,地租减少的数目,等于由I决定的市场价值超过II的煤的自身价值的全部差额,或者说,等于II的煤的自身价值和它的费用价格之间的差额之上的余额。它的级差地租以前等于10镑;现在它的全部地租等于10镑,也就是等于它的绝对地租。因此,在II中,随着市场价值降到(II的煤的)价值,级差地租消失了,从而,由于这种级差地租的存在而膨胀和加倍了的地租率也消失了。地租率从20降到10。其次,地租从10降到5,因为在地租率既定时,投入II的资本减少了一半。

既然市场价值现在决定于II的煤的价值,即每吨1+(11/13)镑,那末,III所生产的75吨的市场价值现在就等于138+(6/13)镑,其中地租是28+(6/13)镑。以前地租是40镑;因此,地租减少了11+(7/13)镑。以前地租超过绝对地租30镑,现在只超过18+(6/13)镑(因为18+(6/13)+10=28+(6/13))。以前地租等于4R,现在只等于2R+[8+(6/13)]镑。因为投入III的资本量没有变,所以地租的这种下降完全是由于级差地租率的下降,也就是由于III的煤的市场价值超过它的个别价值的余额的减少。以前,III的地租总额等于较高的市场价值超过费用价格的余额,现在它只等于较低的市场价值超过费用价格的余额。\endnote{关于地租总额(绝对地租和级差地租加在一起)等于市场价值和费用价格之间的差额这个原理,马克思在后面作了更详细的考察(见本册第328—329页)。——第286页。}因此这个差额接近于III的绝对地租。III用100镑资本生产75吨煤,其[个别]价值等于120镑;因而一吨等于1+(3/5)镑。可是III过去是按以前的市场价格出卖,一吨卖2镑,即贵2/5镑。75吨总共贵2/5×75=30镑,这实际上就是III的地租总额中的级差地租;因为它的地租等于40镑(10镑绝对地租,30镑级差地租)。现在III按新的市场价值一吨只卖1+(11/13)镑。III的一吨煤的这个价格超过它的[个别]价值多少呢?3/5=39/65,11/13=55/65。因此,III的每吨煤卖得比它的[个别]价值贵16/65镑。\endnote{16/65镑这个数,是从每吨煤的新的市场价值1+(11/13)镑减去等级III每吨煤的个别价值1+(3/5)镑而得出来的。——第286页。}75吨总共贵18+(6/13)镑,这个数目恰好是现在的级差地租,因此,级差地租总是等于吨数与每吨市场价值超过每吨[个别]价值的余额的乘积。现在还要计算地租怎么减少了11+(7/13)镑。市场价值超过III的煤的价值的余额,从每吨2/5镑(当时每吨按2镑出卖)降到每吨16/65镑(现在每吨按1+(11/13)镑出卖),也就是从26/65降到16/65,即降低10/65了镑。75吨总共降低了750/65=150/13=11+(7/13)镑,这个数目恰好是III的地租减少的数目。

[570]IV的92+(1/2)吨按1+(11/13)镑的价格计算共值170+(10/13)镑。这里,地租是60+(10/13),而级差地租50+(10/13)是镑。如果92+(1/2)吨按自己的价值出卖,即按120镑出卖,则每吨值1+(11/37)镑。可是现在它按1+(11/13)出卖。而11/13=407/481,11/37=143/481。这里得出IV的煤的市场价值超过它的价值的余额是264/481镑。92+(1/2)吨的余额恰恰是50+(10/13)镑,即的级差地租。

我们现在用A表和B表把这两种情况作一对比:

\todo{}

\todo{}

这两个表使我们有理由去做一些非常重要的考察。

首先我们看到,绝对地租的数额,同投入农业\endnote{马克思在上面所举的例子不是指农业,而是指开采富饶程度不同的煤矿。但是关于这些煤矿所谈的一切,也同样适用于在肥力不同的土地上经营的农业。——第287页。}的资本,同投在I、II、III的资本总额成比例地增减。这个绝对地租的比率完全不取决于所投资本的大小,因为它同土地等级的差别完全无关,相反,它是由价值与费用价格之间的差额产生的,而这个差额本身决定于农业资本的有机构成,决定于生产方式,而不决定于土地。在II中,绝对地租的数额现在从10减到5,这是因为资本从100减到50,有半数[571]资本已经抽出。

在我们进一步考察这两个表之前,我们再列出几个表。我们看到,在B中市场价值降到每吨1+(11/13)镑。但是,按这个价值,AI的产品不必从市场上完全消失,BII也不必只使用原来资本的一半。因为在I中,商品总价值为120镑,地租等于10镑,即等于总价值的1/12,所以这对于每一吨价值(等于2镑)也是适用的。但是2/12镑等于1/6镑或3+(1/3)先令(3+(1/3)先令×60=10镑)。因此,I的每吨的费用价格是{2镑-[3+(1/3)]先令,即}1镑16+(2/3)先令。[新的]市场价值是1+(11/13)镑或1镑16+(12/13)先令。但是16+(2/3)先令等于16先令8便士,或16+(26/39)先令。与此相比,16+(12/13)(或16+(36/39))先令多了10/39先令。这个数目是在新的市场价值下每吨的地租,60吨的地租总数是15+(5/13)先令。因此,地租还不到资本100镑的1%。要AI完全不提供地租,市场价值必须降到它的[这个等级的]费用价格的水平,就是降到1镑16+(2/3)先令,或1+(5/6)镑(或1+(10/12)镑)。在这种场合,AI的地租就会消失。但是它仍然可以开采,提供10%的利润。只是在市场价值进一步降到1+(5/6)镑以下的时候,才会停止开采。

至于BII,在B表中假定有一半资本从生产中抽出。但是,因为市场价值1+(11/13)镑还能提供10%的地租,所以这个市场价值不论对100镑资本还是对50镑资本都同样提供这种地租。因此,假定抽出一半资本,那末这只是因为在这种条件下BII还能提供10%的绝对地租。事实上,如果BII继续生产65吨而不是生产32+(1/2)吨的话,市场将会负担过重,在IV的煤支配着市场的情况下,市场价值将会下降,以致必须减少对BII的投资,才能使它提供绝对地租。可是很明白,在全部资本100镑提供9%的地租时,地租总额会比在资本50镑提供10%的地租时大。因此,如果根据市场情况,为了满足现有的需求对II只需投50镑资本,那末地租必定会降到5镑。但是,假定追加的32+(1/2)吨不能找到经常的销路,因而被挤出市场,那末地租实际上会降得更低。市场价值将降到不仅使BII的地租消失,并且使利润也受到影响。这时就会抽出资本以减少供给,直至资本减少到50镑这个恰当的数额为止,这时市场价值将稳定在1+(11/13)镑上,同时市场价值又为BII提供绝对地租,但是只为以前投资的半数提供绝对地租。就是在这种场合,起决定作用的也是支配着市场的IV和III。

但是,如果市场在每吨价格为1+(11/13)镑时只能吸收200吨,这决不能说,当市场价值下降的时候,即由于追加的32+(1/2)吨对市场的压力,232+(1/2)吨的市场价值降低的时候,市场就不能再多吸收32+(1/2)吨了。BII每吨的费用价格是[110∶65,即]1+(9/13)镑或1镑13+(11/13)先令,而市场价值是1+(11/13)镑或1镑16+(12/13)先令。如果市场价值降到AI不再能提供地租,就是说,如果降到AI的费用价格的水平,降到1镑16+(2/3)先令,或1+(5/6)镑,即1+(10/12)镑,那末,为了使BII用上全部资本,需求就必须大大增加,因为AI由于提供普通利润,还可能继续开采。市场可能不是要多吸收32+(1/2)吨,而是要多吸收92+(1/2)吨,不是吸收200吨,而是吸收292+(1/2),因此[几乎]多了一半。这必须以需求已有极大增加为前提。就是说,为了使我们假定的需求的增加不是太大,市场价值应当降到把AI挤出市场。换句话说,市场价格应当降到低于AI的费用价格,即低于1+(10/12)镑,比如说,降到1+(9/12)镑即1镑15先令。在这之后,市场价格仍然大大高于BII的费用价格。

因此,我们在A表和B表之外再加上三个表:C表、D表和E表。在C表中我们假定,需求的增加使A表和B表中的所有等级都能继续生产,但是按照B的市场价值,同时AI还提供地租。在D表中我们假定,需求量足以使AI不再提供地租,但是还提供普通利润。在E表中我们假定,价格降到把AI挤出市场,[572]但是同时,价格的降低能使市场吸收BII的追加的32+(1/2)吨。

A表和B表中所假定的情况是可能的。可能有这样的情况:AI在地租从10镑降到不足16先令时停止对自己的土地的这种利用,而把它出租,另作他用,这样,它可以提供较高的地租。但是,在这种场合,如果市场不是随着新的市场价值的形成而扩大,BII就不得不由于上面描写的过程而抽出它的一半资本。

\todo{}

\todo{}

\todo{}

[573]现在,我们把A、B、C、D、E表排成一个总表,不过排法应当象本来应该有的那样:资本、总价值、总产品、每吨市场价值、个别价值、差额价值\endnote{正如马克思在后面(见本册第298—299页)所解释的,他把市场价值和个别价值之间的差额叫做差额价值(Differentialwert)。差额价值是马克思就单位产品来说的,级差地租是马克思就一定等级所生产的全部产品来说的。如果单位产品的市场价值大于单位产品的个别价值,差额就是正数;如果单位产品的市场价值小于它的个别价值,差额就是负数。所以马克思在手稿第574页的总表上加了+和-的符号(见本册第302—303页)。在手稿第572页C、D、E各表上(见本册第290—291页),马克思在表明级差地租量的数字(镑)前加了+和-的符号,例如在C表“级差地租”一栏中有负数“-[9+(3/13)]镑”。这表明,在这种情况下等级I的肥力不高,以致这个等级的土地在现有的市场价值下不仅不能提供任何级差地租,而且连绝对地租也大大降低到正常量之下。在CI中,绝对地租只等于10/13镑,即比本例中作为绝对地租正常量的10镑少9+(3/13)镑。马克思在手稿第574页的总表里,用负差额价值来表示这种负级差地租现象,而在“级差地租”一栏,遇到这种情况时只记上“0”,表示这里没有正级差地租(在许多情况下,遇有负级差地租时,负级差地租由绝对地租量的相应减少来表示,这反映在“绝对地租”一栏中)。把负数移入“差额价值”一栏,就可以免除手稿第572页C表中在把不同等级土地的级差地租相加时所发生的那种麻烦,这样,在统计级差地租时,只把带+号的正级差地租加到总数中去,而为了避免重复计算起见,就把负数“-[9+(3/13)]镑”当作零。因此,马克思为了计算负级差地租,在总表里另加了“每吨差额价值”一项,把负差额价值也列进去。——第292页。}、费用价格、绝对地租、绝对地租(吨)、级差地租、级差地租(吨)、总地租。然后在每个表下列出所有等级的合计。\endnote{紧接这些话之后,马克思在手稿第573页上按照上面说的各项把A、B、C、D表作了对比。在手稿的下一页即第574页上,马克思再一次把A、B、C、D表的全部材料写成更有次序的格式,并补上E表的有关材料。这就是本册第302—303页上的统一的总表。马克思在手稿第573页上草拟的A、B、C、D表的对比材料已全部列入总表,因此在本书正文中就不再另外列出。——第292页。}[见第302—303页]

\tsubsectionnonum{[575]对表的说明}

假定:花费资本100(不变资本和可变资本),由这笔资本推动的劳动提供等于预付总资本1/5的剩余劳动(无酬劳动),或者说,提供等于100/5的剩余价值。因此,如果预付资本等于100镑,总产品的价值就应该等于120镑。再假定平均利润等于10%;在这种场合,110镑就是总产品(在上例中是煤)的费用价格。100镑的资本,不管开采的是富矿还是贫矿,在剩余价值率或剩余劳动率既定时转化为120镑的价值;总之,劳动的不同生产率,不论它是劳动的不同自然条件的后果,还是劳动的不同社会条件的后果,还是不同技术条件的后果,都丝毫不会改变商品价值等于物化在商品中的劳动量这一论点。

因此,如果说100镑资本生产的产品的价值等于120镑,这只不过是说,在产品中包含着物化在100镑资本中的劳动时间加1/6为资本家占有的无酬劳动时间。不论这100镑资本在一个等级的矿井中生产60吨,在另一个等级生产65或75或92+(1/2)吨,产品的总价值都等于120镑。但是很明显,不论每一单位产品是象这里一样用吨计算,还是用夸特、码等计算,它的价值却随着劳动生产率的不同而完全不同。拿我们的表来说(对于作为资本主义生产结果的任何别的商品量来说同样适用),如果资本的总产品是60吨,那末一吨的价值就是2镑,因而60吨值120镑,换句话说,它们代表物化在120镑中的劳动时间。如果总产品是65吨,每一吨的价值就等于1镑16+(12/13)先令,即1+(11/13)镑;如果总产品是75吨,每一吨的价值就等于1+(3/5)镑,即1镑12先令;最后,如果总产品是92+(1/2)吨,一吨的价值就等于1+(11/37)镑,即1镑5+(35/37)先令。因为100镑资本所生产的商品总量或总吨数总是具有同一价值,等于120镑,因为它们总是代表120镑所包含的同一劳动总量,正因为如此,所以每一吨的价值,随着同一价值表现为60、65、75或92+(1/2)吨而不同,也就是随着劳动生产率的不同而不同。正是这种劳动生产率的不同造成这样的情况:同量劳动有时表现为较小的商品总量,有时表现为较大的商品总量,因而这个商品总量的每一部分包含的已耗费的劳动的绝对量,有时较多,有时较少,也就是说,与此相应,它有时有较大价值,有时有较小价值。这个随100镑资本投在富矿或贫矿而不同的,即随劳动生产率不同而不同的每一吨的价值,就是表上的每吨个别价值。

因此,再没有什么比下面这样一种看法更错误了:如果单位商品的价值在劳动生产率提高时下降,那末一定资本(例如100镑)所生产的产品的总价值就要由于它借以表现的商品量的增加而提高。其实,单位商品的价值之所以下降,只是因为总价值,即已耗费的劳动总量,表现为较大的使用价值量,较大的产品量,因而分摊到单位产品上的是总价值(或者说,已耗费的劳动)的一个较小的比例部分,而且单位产品价值下降的程度,就是单位产品吸收的劳动量减少或分摊到的总价值的份额减少的程度。

最初,我们把单个商品看作一定量劳动的结果和直接产品。现在,当商品表现为资本主义生产的产品时,事情在形式上就发生了如下的变化:

生产出来的使用价值量代表一个劳动时间量,这个劳动时间量等于在生产使用价值量时消耗的资本(不变资本和可变资本)中包含的劳动时间量加资本家占有的无酬劳动时间。如果包含在资本中的劳动时间用货币来表现等于100镑,如果这100镑资本中包含40镑花费在工资上的资本,而剩余劳动时间是可变资本的50%,就是说,剩余价值率等于50%,那末,100镑资本生产的商品总量的价值就等于120镑。我们在这部著作的第一部分\endnote{指《政治经济学批判》第一分册。见《马克思恩格斯全集》中文版第13卷第56页。——第294页。}已经说过,商品要能够流通,商品的交换价值必须先变成价格,就是说,必须表现为货币。因此,[576]如果总产品不是一个代表全部资本的不可分割的东西(例如一座房子),不是一个唯一的商品,而其价格根据假定等于120镑,即等于表现为货币的总价值,那末,资本家在把商品抛到市场上之前,就必定首先计算单位商品的价格。这里,价格等于价值的货币表现。

120镑总价值将依劳动生产率的不同而分配在较多或较少的产品量上,因而单位产品的价值将依此——成比例地——等于120镑的一个较小或较大的相应部分。这里计算很简单。如果全部产品等于例如60吨煤,那末60吨等于120镑,1吨等于120/60镑,也就是2镑;如果产品是65吨,那末一吨的价值等于120/65镑,也就是1+(11/13)镑或1镑16+(12/13)先令(1镑16先令11+(1/13)便士);如果产品是75吨,那末一吨的价值等于120/75镑,也就是1镑12先令;如果产品是92+(1/2)吨,那末一吨的价值等于1+(11/37)镑,或1镑5+(35/37)先令。因此,单位商品的价值(价格)等于产品的总价值除以产品总量,这个总量是用产品作为使用价值所适用的度量单位,如吨(在上述场合)、夸特、码等计算的。

因此,如果单位商品的价格等于100镑资本生产的商品量的总价值除以商品总量,那末总价值就等于单个商品的价格乘以这些商品的总量,或者说,等于作为一个单位的一定量商品的价格乘以用这个单位计算的全部商品量。其次,总价值由预付在生产中的资本的价值加剩余价值组成,由包含在预付资本中的劳动时间加资本占有的剩余劳动时间即无酬劳动时间组成。因此,商品量的每一部分包含的剩余价值,同它包含的价值具有同一比例。随着120镑是分配在60、65、75吨还是分配在92+(1/2)吨上,20镑剩余价值也就分配在那些吨上。如果吨数等于60,因而每吨价值等于120/60,即2镑或40先令,那末这个40先令或2镑的1/6,即6+(2/3)先令,就是分摊到一吨上的剩余价值份额。剩余价值在值2镑的一吨中所占的比例,同它在值120镑的60吨中所占的比例一样。剩余价值对价值之比,在单位商品的价格中同在全部商品量的总价值中一样。在上例中,每一吨包含全部剩余价值的20/60=2/6=1/3镑,或者说上述40先令的1/6。因此,一吨的剩余价值乘60就等于资本生产出来的全部剩余价值。如果由于产品数量较大,也就是由于劳动生产率较高,摊到单位产品上的价值部分即总价值的比例部分较小,那末摊到每一单位产品上的剩余价值部分,即单位产品中包含的全部剩余价值的相应部分也较小。但这并不影响剩余价值即新创造的价值对预付的和只是被再生产出来的价值的比例关系。的确,我们已经看到\fnote{见本卷第1册第215—217页。——编者注},虽然劳动生产率并不影响产品的总价值,但是,如果产品加入工人的消费,如果由于单个商品的价格下降,或者换句话说,由于一定量商品的价格下降,因而正常工资减少,或者换句话说,劳动能力的价值减少,那末劳动生产率就会使剩余价值增大。由于较高的劳动生产率创造相对剩余价值,它就不是使产品的总价值增大,而是使这个总价值中代表剩余价值即无酬劳动的部分增大。因此,如果劳动生产率较高,但由于价值借以表现的商品总量已经增大,摊到单位产品上的价值部分较小,因而单位产品的价格下降,那末,在上述情况下,这个价格中代表剩余价值的部分仍然会增大,也就是说,剩余价值对再生产出来的价值之比会增大{其实,这里首先还是应该谈对可变资本的关系,这里还谈不上利润}。但是,所以会发生这种情况,是因为在产品的总价值中,由于劳动生产率增长,剩余价值增大了。正是这个原因,即劳动生产率的提高,——它的提高使同量劳动表现为一个较大的产品量,从而使这个产品量的任何一部分的价值或单位商品的价格降低,——使劳动能力的价值减少,因而使包含在总产品价值中的,从而包含在单位商品价格中的剩余劳动即无酬劳动增加。因此,虽然单位商品的价格降低,虽然包含在单位商品中的劳动总量减少,从而它的价值也减少,这个价值中由剩余价值组成的比例部分却增大,换句话说,同以前劳动生产率较低,因而单位商品的价格较高,包含在单位商品中的劳动总量较大的时候比较起来,在单位商品包含的较少的[577]劳动总量中,却包含较大的无酬劳动量。虽然在这种场合,一吨包含较少的劳动,因而比较便宜,但是它包含较多的剩余劳动,因而提供较多的剩余价值。

因为在竞争的条件下一切事情都以虚假的、颠倒的形式表现出来,所以单个资本家会以为:(1)由于单位商品价格降低,他从单位商品赚到的利润降低了,但是由于商品量增加,他才赚到较大的利润(这里又同由于使用的资本增大,即使在利润率较低时也可能获得较大的利润量的情况混淆起来了);(2)他确定单位商品的价格,并通过乘法确定产品的总价值,可是,本来的过程却是除法,然后才是乘法,乘法以除法作为自己的前提。庸俗经济学家实际上只不过把陷入竞争中的资本家们的奇怪想法翻译成一种表面上比较理论化的语言,并企图借此来说明这些想法正确而已。

现在回过头来谈我们的表。

用100镑资本创造的产品或商品量的总价值等于120镑;商品量可大可小,全看劳动生产率的不同程度而定。不论总产品的量大小如何,如果平均利润象我们假定的那样是10%,这个总产品的费用价格就总是等于110镑。不论总产品的量大小如何,总产品的价值超过费用价格的余额总是等于10镑,即等于总价值的1/12,或预付资本的1/10。总产品的价值超过费用价格的这个余额,这个10镑,构成地租。很明显,它同煤矿、土地,总之同这100镑资本曾经被用上去的那个自然要素的不同自然富饶程度所引起的不同劳动生产率完全无关,因为由自然因素的不同富饶程度引起的劳动生产率的不同,并不妨碍总产品有120镑的价值,有110镑的费用价格,因而有一个等于10镑的、价值超过费用价格的余额。资本的竞争所能起的作用,只是使一个资本家在煤的生产这个特殊生产领域中用100镑资本创造出来的商品的费用价格等于110镑。但是竞争并不能象它在其他生产领域中起的强制作用那样,即使产品值120镑,它也要迫使这个资本家按照110镑出卖。这是因为有土地所有者插手进来,拿走这10镑。因此,我把这个地租称为绝对地租。所以,不论煤矿的富饶程度如何改变,也不论由此引起的劳动生产率如何改变,在表里,这个地租总是同一的。但是,它在煤矿富饶程度不同因而劳动生产率不同的条件下,不是表现为同一吨数。因为包含在10镑中的劳动量随着劳动生产率的不同而表现为较多或较少的使用价值量,表现为较多或较少的吨数。这个绝对地租在富饶程度不同的条件下是否总是全部得到支付或者[有时只是]部分得到支付,将在表的进一步分析中说明。

其次,存在于市场上的煤却是富饶程度不同的矿井的产品,这些矿井,我从最贫瘠的开始,已标作I、II、III、IV四个等级。例如:第一等级,100镑资本的产品是60吨;第二等级,100镑资本的产品是65吨;等等。因此,由于劳动生产率程度随矿井、土地,总之随自然因素的富饶程度而有所不同,在这里同样大小的、具有同一有机构成的资本100镑,在同一生产领域内却有不同的生产率。但是,竞争为这些具有不同个别价值的产品规定了统一的市场价值。这个市场价值本身决不能大于最贫瘠的等级的产品的个别价值。如果它高一些,这只是证明市场价格高于市场价值。但是市场价值必定表现实际价值。就各个等级的产品来看,当然,它们的[个别]价值可能高于或者低于市场价值。如果它们的个别价值高于市场价值,那末市场价值和它们的费用价格之间的差额就小于它们的个别价值和它们的费用价格之间的差额。但是,因为绝对地租等于它们的个别[578]价值和它们的费用价格之间的差额,所以在这种场合,这些产品的市场价值就不能提供全部绝对地租。如果市场价值降到等于这些产品的费用价格,这些产品的市场价值就完全不提供地租。这些产品的生产者就不能支付任何地租,因为[总]地租只是[市场]价值和费用价格之间的差额,而就这些产品个别地说,在这样的市场价值之下,这个差额就会消失。在这种场合,它们的市场价值和个别价值之间的差额是负数,就是说,市场价值和它们的个别价值相差一个负数。我把市场价值和个别价值之间的差额通称为差额价值。对于这种情况下的商品,我在差额价值前加了一个负号。

相反,如果某一等级的煤矿(土地)的产品的个别价值低于市场价值,那就是说,市场价值高于产品的个别价值。这样,在这些产品的生产领域中占支配地位的价值或市场价值,就提供一个超过它们的个别价值的余额。比方说,如果一吨的市场价值等于2镑,那末,个别价值等于1镑12先令的一吨,它的差额价值就是8先令。因为在一吨的个别价值等于1镑12先令的等级中,100镑资本生产75吨,所以这75吨的全部差额价值就是8先令×75,即30镑。这个等级因土地或矿井相对来说比较富饶而造成的、全部产品的市场价值超过其个别价值的余额,就形成级差地租,因为费用价格对于这笔资本来说仍旧同以前一样。这个级差地租是较大还是较小,就看市场价值超过个别价值的余额是较大还是较小,而这个余额是较大还是较小,又要看生产出这种产品的矿山等级或土地等级,同生产出的产品对市场价值起决定作用的那个比较不富饶的等级比较起来,其富饶程度相对说来是高得多些还是少些。

最后,还必须指出,不同等级的产品的个别费用价格是不同的。例如,100镑资本生产75吨的那个等级,因为总价值等于120镑,总费用价格等于110镑,单位商品的费用价格就等于1镑9+(1/3)先令;如果市场价值等于这个等级的个别价值,就是说等于1镑12先令,那末,按120镑出卖的75吨将提供地租10镑,而110镑就代表它们的费用价格。

但是单独一吨的个别费用价格当然随着100镑资本借以表现的吨数,或随着不同等级的单位产品的个别价值而不同。例如,100镑资本生产60吨,一吨的价值就等于2镑,它的费用价格等于1镑16+(2/3)先令。55吨就等于110镑或总产品的费用价格。如果100镑资本生产75吨,那末一吨的价值就等于1镑12先令,它的费用价格等于1镑9+(1/3)先令,总产品中的68+(3/4)吨值110镑,也就是说正好补偿费用价格。在不同等级中,个别费用价格即每吨费用价格的不同,与个别价值的不同具有同一比例。

五个表都表明,绝对地租总是等于商品[个别]价值超过它自己的费用价格的余额;级差地租等于商品的市场价值超过它的个别价值的余额;总地租(如果除绝对地租外还有级差地租的话)等于市场价值超过个别价值的余额加上个别价值超过费用价格的余额,或者说,等于市场价值超过个别费用价格的余额。

因为这里只是把地租的一般规律作为我的价值理论和费用价格理论的例证来发挥,只有到我专门考察土地所有权时我才详细论述地租,[579]所以我撇开了一切使问题复杂化的情况:矿井或各种土地的位置的影响;用于同一矿井或同一土地的几批资本的不同生产率;同一生产领域的不同部门例如农业的不同部门所提供的地租的相互关系;彼此不同但可以互相转化的各生产领域——例如从农业中抽出土地用于建筑房屋等——所提供的地租的相互关系。这一切都不是这里所要讨论的。

\tsectionnonum{[(3)对表的分析]}

现在我们来考察这些表。这些表说明,一般规律可以解释多种多样的组合,而李嘉图由于不知道地租的一般规律,对级差地租的本质也只有片面的理解,因此他想通过强制的抽象把多种多样的现象只归结为唯一的一种情况。这些表所要说明的,并不是所有可能的组合,只是对我们的专门目的有用的几种最重要的组合。

\tsubsubsectionnonum{[(a)]A表[不同等级的个别价值和市场价值之间的关系]}

在A表中,一吨煤的市场价值由等级I的一吨煤的个别价值决定,这等级I的矿井最贫瘠,因而劳动生产率最低,也就是说100镑投资提供的产品量最少,因此,单位产品的价格(由它的价值决定的价格)最高。

假定市场吸收200吨,既不多,也不少。

市场价值不能高于I的每吨价值,即不能高于在最不利的生产条件下生产出来的商品的价值。II和III的一吨煤高于自己的个别价值出卖,这是因为这里的生产条件比同一生产领域(部门)生产的其他商品所具备的条件优越;因此,这并不违反价值规律。相反,如果市场价值高于I的每吨价值,那末,其所以可能,只是因为I的产品完全不顾市场价值高于自己的价值出卖。一般说来,市场价值和[个别]价值之间有差别,不是因为产品绝对高于自己的价值出卖,只是因为整个生产领域的产品所具有的那个价值,可能和个别产品的价值不同,也就是说,因为生产总产品(这里是200吨)所必要的劳动时间,可能和生产其中部分吨数(这里指II和III出产的那些吨数)所用的劳动时间不同;总之,因为得到的总产品是生产率程度不同的劳动的产品。因此,产品的市场价值和它的个别价值之间的差别,只能同生产率程度的差别有关,在这些不同程度的生产率下,一定量劳动创造出总产品的不同份额。这种差别决不能意味着价值是不依赖于该生产领域一般使用的劳动量而决定的。如果一吨的市场价值高于2镑,那末,其所以可能,只是因为I不管同II和III的关系如何,一般来说都是把自己的产品高于它的价值出卖。在这种场合,由于市场状况,即由于供求关系,市场价格就会高于市场价值。但是,我们这里所谈的市场价值——按照假定在这里市场价格等于市场价值,——是不能高于它自己的。

\todo{}

在这里,市场价值等于I的产品的价值,而且I提供的产品占市场上所有产品的3/10,因为II和III只提供补足全部需求所必要的那么多产品,也就是说,只满足由I的产品满足的需求之外的追加需求。因此,II和III没有任何理由低于2镑出卖自己的产品,因为全部产品都能够按2镑卖掉。它们也不能把自己的产品卖得[580]高于2镑,因为I的一吨卖2镑。

市场价值不能高于在最坏生产条件下生产出来,但是作为必要供给的一部分提供的产品的个别价值,这个规律被李嘉图歪曲为市场价值不能降到低于这种产品的价值,也就是说始终都要由这个价值来决定。下面我们就会看到,这是多么错误。

因为在I中,一吨的市场价值和一吨的个别价值一致,所以它提供的地租就是产品价值超过产品费用价格的绝对余额,即绝对地租,等于10镑。II提供级差地租10镑,III提供级差地租30镑,因为由I决定的市场价值分别为II和III提供一个超过其产品个别价值因而也超过10镑绝对地租(绝对地租代表个别价值超过费用价格的余额)的余额10镑和30镑。II提供总地租20镑,III提供总地租40镑,因为市场价值提供超过它们的费用价格的余额分别为20镑和40镑。

我们假定,由等级I即最贫瘠的矿井推移到比较富饶的等级II,再由II推移到更富饶的等级III。虽然II和III比I富饶,但是它们只满足整个需求的7/10,因而象刚才已经阐明的,能够按2镑出卖自己的产品,尽管产品的价值分别只是1镑16+(12/13)先令和1镑12先令。很明显,如果为满足需求所必要的一定量产品得到供应,而满足这个需求的不同部分的劳动生产率又有不同,那末,不论向这个还是那个方向推移,在两种场合,比较富饶的那些等级的市场价值都会高于它们的个别价值:在一种场合,是因为它们遇到的市场价值由贫瘠的等级决定,而它们提供的追加供给又不足以促使由I决定的市场价值发生变动;在另一种场合,是因为原来由它们决定的市场价值,即由III或II决定的市场价值,现在由I来决定,因为I提供市场所要求的追加供给,而且它只能按照现在决定市场价值的那个较高价值提供这种供给。

\tsubsubsectionnonum{[(b)李嘉图的地租理论同农业生产率递减观点的联系。绝对地租率的变动及其同利润率的变动的关系]}

例如,李嘉图在A表这样的情况下就会说:出发点是III;追加供给先是由II提供;最后,市场要求的最后的追加供给由I提供;因为I只能按120镑提供追加的60吨,每吨2镑,而这个供给又是市场所要求的,所以一吨的市场价值,原来是1镑12先令,然后是1镑16+(12/13)先令,现在提高到2镑。可是,反过来同样正确:如果出发点是I,它按每吨2镑满足60吨需求,然后追加需求由II提供,II的产品将按市场价值2镑出卖,虽然它的个别价值只是1镑16+(12/13)先令;因为市场要求的125吨,象以前一样,只有在I提供60吨而每吨价值为2镑这个条件下,才能提供出来。同样,如果又必需有一个75吨的新的追加供给,而III只提供75吨,仅仅满足这个追加的需求,那就仍然要有I的每吨2镑的60吨。如果II满足全部需求200吨,那末这200吨就卖400镑。这200吨现在也是按这个价格出卖的,因为II和III不是按照它们能够满足140吨追加需求的那个价格出卖自己的产品,[XII—581]而是按照只提供3/10产品的I能够满足需求的那个价格出卖自己的产品。市场要求的产品总量是200吨,在这里每吨按2镑出卖,因为这个数量的3/10只有每吨按2镑出卖才能生产出来,而不管满足需求所必需的追加产品量按什么次序进行生产,是从III开始到II再到I,还是从I开始到II再到III。

李嘉图说:如果III和II的产品是出发点,那末它们的市场价值必定提高到I的产品价值(在李嘉图那里,是费用价格)的水平,因为I提供的3/10是满足需求所必需的;换句话说,因为这里谈的是所要求的产品量,而不是这个量的一些特殊部分的个别价值。但是,下面的情况同样是正确的:如果I是出发点,II和III提供的只是追加供给,由I提供的3/10仍然同样是必需的;由此可见,如果I在下降序列中决定市场价值,那末由于同样的原因,它在上升序列中也决定市场价值。因此,A表向我们表明,李嘉图的观点是错误的,他认为级差地租以比较富饶的矿井或土地推移到比较不富饶的矿井或土地为条件,也就是说,级差地租以劳动生产率不断减低为条件。级差地租即使在运动的方向相反时,也就是说,在劳动生产率不断提高时,也完全可能存在。不管这两种运动发生哪一种,都和级差地租的本质,和级差地租存在的事实没有任何关系,这是一个历史问题。实际上,上升序列和下降序列将互相交错,追加需求的满足,有时靠推移到比较富饶的土地、煤矿等这类自然因素,有时靠推移到比较不富饶的土地、煤矿等这类自然因素。同时始终假定,由不同于其他等级的新等级(不管这个等级是比较富饶还是比较不富饶)的自然因素提供的供给,只等于追加的需求,也就是说不引起供求关系的任何变动,因此,不是在花费较小的费用就能提供供给时,而只是在花费较大的费用方能提供供给时,才会引起市场价值本身的变动。

因此,A表从一开始就向我们揭示了李嘉图的这个基本前提是错误的,从安德森的例子可以看出,甚至在对绝对地租问题有错误看法时,这个前提也不是必要的。

如果发生从III到II,从II到I,即按下降序列的推移,即推移到越来越不富饶的自然因素,那末,先是投资100镑的III,按照商品价值120镑出卖自己的商品。每吨是1镑12先令,因为III生产75吨。如果有65吨的追加需求必须满足,那末,使用资本100镑的II,也会按照产品价值120镑出卖自己的产品。每吨是1镑16+(12/13)先令。最后,如果又必须提供追加供给60吨,而只有I能够提供这些吨数,那末,I也按照产品价值120镑出卖自己的产品,每吨是2镑。在这样的过程中,一旦II的产品出现在市场上,III就提供级差地租18+(6/13)镑,而它原来只提供绝对地租10镑。一旦I出现,II就提供级差地租10镑,III的级差地租就提高到30镑。

李嘉图从III降到I时,在I那里已经找不到地租了,这是因为他在考察III时,是从不存在任何绝对地租出发的。

当然,上升序列和下降序列之间有一定的差别。如果是从I推移到III,因而II和III只提供追加供给,那末市场价值就依然等于I的产品的个别价值,即2镑。并且,如果平均利润象这里假定的那样是10%,那末可以认为,煤的价格(相应地说,小麦的价格;到处都可以用一夸特小麦等等代替一吨煤)加入了平均利润的计算,因为煤既作为生活资料加入工人的消费,又有相当大的份额作为辅助材料加入不变资本。因此,同样可以认为,如果I的生产率较高,或者说,如果一吨煤的价值低于2镑,剩余价值率就较高,同时剩余价值本身也较大,也就是说,利润率高于10%。但是,在III是出发点时,正好就是这样的情况。一吨煤的[市场]价值原来只等于1镑12先令,当[582]II的产品出现在市场上时,它提高到1镑16+(12/13)先令,最后,当I的产品出现时,它提高到2镑。因此,可以认为,在其他一切情况如剩余劳动时间以及其他生产条件等等固定不变时,如果只有III被开发,利润率就更高(剩余价值率就更高,因为工资的一个要素更便宜;由于剩余价值率更高,剩余价值量也更大,就是说利润率也更高;除此之外,在剩余价值发生这种变化的情况下,利润率所以更高,还因为用于不变资本的一个费用要素更便宜),当II的产品出现在市场上时,利润率就降低,最后,当I的产品出现时,利润率就降到10%这个最低水平。因此,在这种场合应当假定,当只有III被开发时,利润率等于例如12%(不考虑原来假定的数字),当II出现时,利润率降到11%,最后,当I出现在市场上时,利润率降到10%。在这种场合,III的绝对地租是8镑,因为费用价格是112镑;在II的产品出现在市场上以后,绝对地租是9镑,因为费用价格现在是111镑,最后,绝对地租提高到10镑,因为费用价格降到110镑。可见,这里绝对地租率本身发生了变动,而且和利润率的变动成反比。地租率逐步提高,因为利润率逐步下降。而利润率所以下降,是由于矿井、农业等等的劳动生产率下降,并且与此相应,生活资料和辅助材料越来越贵。

\tsubsubsectionnonum{[(c)]考察生活资料和原料的价值——以及机器的价值——的变动对资本有机构成的影响}

在上述情况下,地租率因为利润率下降而提高。在这里,利润率是不是因为资本有机构成发生变动而下降呢?如果资本的平均构成是80c+20v,那末这种构成是否保持不变?假定正常工作日保持不变。否则生活资料涨价的影响可能被抵销。这里应当把两种情况分开。第一,生活资料涨价,因而剩余劳动和剩余价值减少。第二,不变资本涨价,因为辅助材料例如煤的价值可能提高;如果说的是小麦,那末不变资本的另一个组成部分——种子——的价值可能提高,或者,由于小麦涨价,另一种原产品(原料)的费用价格可能上涨。最后,如果涨价的产品是铁、铜等等,那末某些工业部门所必需的原料的价值和所有生产部门制造设备(包括各种容器)用的原料的价值就会提高。

一方面假定资本的有机构成没有发生变动,也就是说,假定生产方式没有发生那种使必须使用的活劳动量与所使用的不变资本量相对来说增加或减少的变动。现在和以前一样,仍然需要同样数量的工人(正常工作日的界限不变),来利用同样数量的机器等等加工同样数量的原料,或者在没有原料的场合,来推动同样数量的机器、工具等等。这是在谈到资本有机构成时应当指出的第一个着眼点。但是,还必须补充另一个着眼点,就是:要考察资本各要素的价值的变动,虽然资本各要素作为使用价值使用的数量和以前完全一样。这里又应当区分如下的情况。

第一,价值的变动以同样的程度影响可变资本和不变资本这两个要素。显然,这种情况实际上是从来不会有的。小麦之类的农产品的价值提高会使(必要)工资和原料(例如种子)涨价。煤价上涨会使必要工资和大部分工业部门的辅助材料的价值提高。但是在前一种场合,工资的提高是在所有生产部门发生,而原料的涨价只是在某些部门发生。至于煤,它加入工资的份额比加入生产的份额小。所以,在总资本中煤和小麦的价值变动未必能以同样的程度影响资本的两个要素。不过我们假定可能有这种情况。

假定资本80c+20v生产的产品的价值等于120。就总资本来说,产品的价值和产品的费用价格相符合。价值和费用价格之间的差别对总资本来说刚好拉平。假定煤这样的物品的价值的提高(根据假定,煤按同一比例加入资本的两个组成部分)使两个要素的费用各增加1/10。这样,用80c只能买到以前[大约]用70c就能买到的商品,而用20v只够支付以前[大约]用18v就能支付的那样多工人的报酬。或者,为了按原来的规模继续生产,现在必须花费[大约]90c和22v。产品的价值和以前一样是120,可是支出现在是112(不变资本90和可变资本22)。所以,利润等于8,这占112的1/14,也就是占[7+(1/7)]%。因而支出资本100所生产的产品价值现在等于107+(1/7)。

现在加入这笔新资本的c和v的比例关系是怎样的呢?以前v和c之比是20∶80,即1∶4,现在是22∶90,即11∶45。1/4=45/180;11/45=44/180。可见,可变资本[583]和不变资本相比减少了1/180。因此,根据假定,要承认煤等的涨价按同一比例影响资本的两个部分,我们必须用88c+22v。因为产品的价值等于120;支出88+22=110。剩下10作为利润。22∶88=20∶80。c和v之比仍然象在原来的资本中那样。同以前一样,v∶c=1∶4。而利润10占110的1/11,即[9+(1/11)]%。因此,如果生产要按原来的规模继续进行,那末,以前投入资本100,现在就要投入资本110,而产品的价值和以前一样等于120。\endnote{在马克思所举的例子中,其生产依赖于土地所有权的那种产品按同一比例加入预付资本的两个组成部分。在这里马克思假定,虽然不变资本增加(由于原料涨价,80c变成88c)以及可变资本也增加(由于工人的消费品涨价,20v变成22v),总产品的市场价值仍然和从前一样等于120(在本册第313—323页所考察的另外一个例子中,马克思相反地从市场价值的变动出发)。在总产品市场价值保持不变的情况下,由于不变资本和可变资本的这种涨价,资本家攫取的剩余价值从20减到10,相应地,由于推移到比较不富饶的地段,在比较富饶的地段上的级差地租就增加10单位。因此,新创造的价值仍旧等于40(因为生产方式未变),在这里它重新分配如下:现在10单位构成剩余价值,归资本家,20单位补偿可变资本,10单位用于增加级差地租(由于不变资本价值增加8单位,由于可变资本价值增加2单位)。后来,在手稿第684—686页(本册第518—522页)上,马克思以农业资本为例考察类似的情况,这种农业资本的产品以实物形式加入这一资本的不变部分和可变部分的诸要素的构成之中。——第311页。}对于100单位的资本来说,其有机构成是80c+20v,产品价值是109+(1/11)。

[第二,]如果在上例中80c的价值保持不变,只是v的价值变了,比如说不是20v而是22v,那末以前的20∶80或10∶40现在就变成22∶80或11∶40。如果发生这样的变化,那末资本就是80c+22v,产品价值是120;因而支出是102,利润是18,即占[17+(33/51)]%。22∶18=[21+(29/51)]∶[17+(33/51)]。如果说为了推动价值80的不变资本而必须花费在工资上的资本是22v,那末要推动价值78+(22/51)的不变资本就需要21+(29/51)。按照这样的比例,在100单位的资本中只有78+(22/51)能够用在机器和原料上,而21+(29/51)则必须用在工资上,而以前是80用在原料等等上,只有20用在工资上。产品的价值现在等于117+(33/51),资本构成则是[78+(22/51)]c+[21+(29/51)]v。而21+(29/51)+[17+(33/51)]=39+(11/51)。在以前的资本构成情况下,全部新加劳动等于40;现在等于39+(11/51),或者说,比以前少40/51;发生这一变化的原因是,虽然不变资本的价值未变,但现在不得不使用较少的不变资本,因而100单位的资本能够推动的劳动比以前略少,尽管支付的报酬较高。

因此,如果某一费用要素的变动——这里是涨价,价值提高——只引起(必要)工资的变动,那末就会发生如下的情况:第一,剩余价值率下降;第二,在资本既定的情况下,可以使用较少的不变资本,较少的原料和机器。资本不变部分的绝对量与可变资本相比减少,这在其他条件不变的情况下,总会引起利润率的提高(如果不变资本的价值保持不变)。不变资本的量减少,虽然它的价值仍然不变。但是剩余价值率和剩余价值本身减少了,因为在剩余价值率下降的情况下雇用的工人人数没有增加。剩余价值(剩余劳动)率比不变资本和可变资本的比率下降得更多。现在为了推动同量的不变资本,必需雇用和以前同样多的工人,也就是使用同样的劳动绝对量。不过在这个劳动绝对量中,必要劳动多了,剩余劳动少了。因而对同量的劳动要支付较高的报酬。因此,同一资本(例如100)用于不变资本的部分少了一些,因为它必须多花一些可变资本来推动较少的不变资本。这里剩余价值率的下降不是和一定资本所使用的劳动绝对量的增长相联系,或者说,不是和它所雇用的工人人数的增长相联系。因而在这里,剩余价值本身不可能在剩余价值率下降的情况下增长。

因此,如果从资本的组成部分的物质方面即把它们作为使用价值来考察时,资本的有机构成保持不变;也就是说,如果资本构成的变动不是由于投入这个资本的生产领域的生产方式发生变动,而仅仅是由于劳动能力价值提高,因而必要工资也提高了(这等于说,剩余劳动减少或剩余价值率下降,这种减少和下降在这里不可能由于一定量资本,例如100单位,所雇用的工人人数的增加而全部抵销或部分抵销),那末,利润率的下降就只能是由于剩余价值本身下降。而资本有机构成的变动也是由同一原因引起。这种变动,在生产方式不变以及使用的直接劳动量和积累劳动量之间的比例不变的情况下,只能由使用的直接劳动量和积累劳动量的价值(比例价值)的变动引起。同一资本使用[584]的较少量直接劳动与它使用的较少量不变资本具有同一比例,但它对这较少量劳动支付的报酬却较高。它只能使用较少量的不变资本,因为推动这较少量不变资本的那个较少量劳动,将吸收总资本中较大的份额。为了推动78单位的不变资本,资本家现在必须在可变资本上花费比如22单位,而以前推动80c只花费20v就够了。

总之,其生产依赖于土地所有权的那种产品的涨价只影响工资的情况,就是这样。如果这种产品跌价,就产生相反的结果。

现在我们来看看上面假设的[“第一”种]情况。假定农产品的涨价按同一比例影响不变资本和可变资本。那末在这里,根据假定,资本的有机构成不发生任何变动。第一,生产方式没有变动。同样的直接劳动绝对量推动着和以前一样的积累劳动量。这两种劳动量之间的比例和以前一样。第二,积累劳动和直接劳动的价值比例没有变动。如果其中一个的价值提高或下降,那末另一个的价值就同它成比例地发生相应的变动,结果它们的比例仍然不变。而以前是80c+20v。产品价值=120。现在是88c+22v。产品价值=120。这就得出10比110,即[9+(1/11)]%的利润。因此,对于80c+20v的资本来说,产品价值是109+(1/11)。

以前是:

\todo{}

现在是:

\todo{}

80c在这里代表较少量的原料等;20v相应地代表较少的劳动绝对量。原料等贵了;用80购买的原料等的量也就减少了;因为生产方式没有变,所以它需要较少量的直接劳动。但是这较少量的直接劳动却和以前较大量的直接劳动所值一样多,而且它和原料等恰好按同样程度涨价,因而也按同样程度减少。可见,如果剩余价值不变,利润率就会按照原料等涨价的程度,按照可变资本价值对不变资本价值的比例的变化程度而下降。但是剩余价值率不是原来那样,而是随着可变资本价值的增长发生了变化。

我们举[另外]一个例子。

一磅棉花的价值由1先令上涨到2先令。以前用80镑{这里我们假定机器等等于0}可以买1600磅棉花。现在用80镑只能买800磅棉花。以前把1600磅棉花加工成棉纱需要花费工资20镑,假定这相当于20个工人。加工800磅就只需要10个工人,因为生产方式没有变。10个工人以前值10镑,现在值20镑,就同800磅棉花以前值40镑现在值80镑完全一样。假定以前利润是20%。因此,所假定的就是这样:

实际上,如果20个工人创造的剩余价值等于20镑,那末10个工人创造的剩余价值就等于10镑;但是要生产这笔剩余价值,必须照旧支付20镑,而在以前的条件下只支付10镑。一磅棉纱这个产品[585]的价值在这里无论如何都会提高,因为产品包含着更多的劳动,即(一磅棉纱所含的棉花中的)积累劳动和直接劳动。

如果只是棉花的价值提高,工资依然不变,那末把800磅棉花加工成棉纱照旧只需要10个工人。而这10个工人也只花费10镑。所以剩余价值率照旧是100%。把800磅棉花变成棉纱,需要10个工人,为他们支出资本10镑。所以资本的总支出等于90镑。同时始终假定每80磅棉花需要一个工人。因此800磅就需要10个工人,1600磅就需要20个工人。那末,现在全部资本100镑能把多少磅棉花变成棉纱呢?可以用88+(8/9)镑购买棉花,11+(1/9)镑支付工资。

比例是这样的:

在这种情况下,可变资本的价值没有变动,因而剩余价值率仍然不变。

在I的情况下,可变资本和不变资本之比是20∶80,或者说1∶4。在III的情况下,可变资本和不变资本之比是[11+(1/9)]∶[88+(8/9)],或者说1∶8;可见,这里可变资本相对地减少了一半,因为不变资本的价值增加了一倍。同一数量的工人把同一数量的棉花纺成棉纱,可是100镑资本现在只能雇用11+(1/9)个工人,用余下的88+(8/9)镑只能购买888+(8/9)磅棉花,而不能象I的情况下那样购买1600磅棉花。剩余价值率仍然不变。但是由于不变资本价值发生变动,在资本为100镑时已经不能雇用以前那样多的工人;可变资本和不变资本的比例变了。结果,剩余价值量减少,随之利润也减少,因为这个剩余价值照旧按同样的资本支出计算。在I的情况下,可变资本(20镑)是不变资本的1/4(20∶80),总资本的1/5。现在可变资本(11+(1/9)镑)只是不变资本的1/8{[11+(1/9)]∶[88+(8/9)]},总资本100镑的1/9。但是100/5镑(即20镑)的100%是20镑,而100/9镑(即11+(1/9)镑)的100%只有11+(1/9)镑。在工资不变,或者说,可变资本价值不变的情况下,可变资本的绝对量在这里减少了,因为不变资本的价值提高了。因此,可变资本的百分比下降,随之剩余价值本身,剩余价值的绝对量,因而还有利润率也下降。

在可变资本价值不变和生产方式不变的情况下(也就是说,在使用的劳动、原料和机器的量保持原来比例的情况下),不变资本价值的变动[这里是提高],会使资本构成发生这样一种变化,就好比不变资本价值没有变,可是同花费在劳动上的资本相比,却使用了更大量的价值未变的[不变]资本(也就是说,所使用的不变资本的价值总额更大)。由此产生的必然后果是利润下降。(如果不变资本价值下降,就会产生相反的情况。)

相反,可变资本价值的变动(这里是提高),会使可变资本同不变资本相对来说增加,从而使可变资本的百分比增加,或者说,使它在总资本中所占的比例部分增加。然而利润率在这里不是提高,而是下降。这是因为生产方式保持不变。为了把同量的原料、机器等变成产品,使用的活劳动量和以前一样。这里和上面的情况一样,用同一资本100镑只能[586]推动较少量的直接劳动和积累劳动,但这较少量的直接劳动所值更多了。必要工资提高了。这较少量的直接劳动中有一个较大部分补偿必要劳动,因而它只有一个较小部分形成剩余劳动。剩余价值率下降了,与此同时,同一资本所支配的工人人数或劳动总量也减少了。可变资本同不变资本相对来说增加了,因而同总资本相对来说也增加了,尽管使用的劳动量同不变资本量相对来说减少了。所以剩余价值下降了,随之利润率也下降了。在前一场合,利润率下降是因为在剩余价值率不变的情况下,可变资本同不变资本相对来说,因而同总资本相对来说减少了;换句话说,剩余价值下降是因为在剩余价值率不变的情况下,工人人数减少了,剩余价值的乘数变小了。在后一场合,利润率下降则是因为,可变资本同不变资本相对来说,因而同总资本相对来说增加了,但是可变资本的这种增加伴随着使用的劳动量(同一资本使用的劳动量)的减少;换句话说,这里剩余价值下降,是因为剩余价值率的下降同使用的劳动量的减少联系在一起。有酬劳动同不变资本相对来说增加了,但使用的劳动总量减少了。

由此可见,价值的这些变动总是影响剩余价值本身,剩余价值的绝对量在两种场合都减少了,因为它的两个因素中有一个变小了,或者这两个因素都变小了;在一种场合,剩余价值减少是因为剩余价值率不变而工人人数减少了;在另一种场合,剩余价值减少是因为剩余价值率和100单位资本所雇用的工人人数都减少了。

最后,我们来谈谈II的情况。在这里农产品价值的变动按同一比例影响资本的两个部分,因而这种价值变动不会造成资本有机构成的变动。

在这种情况下(见第584页)\fnote{见本册第314页。——编者注},一磅棉纱的价值从1先令6便士涨到2先令9便士,因为现在它是比以前耗费更多劳动时间的产品。虽然现在一磅棉纱包含的直接劳动和以前一样(不过这时有酬劳动较多,无酬劳动较少),但是它现在包含的积累劳动比以前多。由于棉花价值发生变动,从1先令涨到2先令,现在加入一磅棉纱价值的也就不是1先令而是2先令。

然而第584页上II的例子是不正确的。

我们有:

20个工人的劳动表现为40镑。在这里,这个数目的一半是无酬劳动,因此剩余价值是20镑。依照这一比例,10个工人将生产20镑,其中10镑是工资,10镑是剩余价值。

因此,如果劳动能力的价值和原料价值按同一比例提高,就是说,如果劳动能力的价值提高一倍,那末它就等于10个工人得20镑,就象以前它等于20个工人得20镑一样。在这种情况下就没有任何剩余劳动了。因为,如果20个工人提供的价值用货币表现等于40镑,那末10个工人提供的价值用货币表现就等于20镑[这也就是II的情况下全部可变资本的价值]。这是不可能的。在这种情况下资本主义生产的基础就消失了。

但是,因为根据假定,不变资本和可变资本价值的变动应当是一样的(按比例),所以我们必须用别的方式表达这一情况。比如说,假定棉花价值提高1/3;用80镑现在能够买到1200磅棉花,而以前用这些货币能够买到1600磅。以前1镑等于20磅棉花,或1磅棉花等于1/20镑,即1先令。现在1镑等于15磅棉花,或1磅棉花等于1/15镑,即1+(1/3)先令,或1先令4便士。以前1个工人花费1镑,现在要花费1+(1/3)镑,即1镑6+(2/3)先令,或1镑6先令8便士。15个工人就要花费20镑(15镑+15/3镑)。[587]因为20个工人生产40镑的价值,所以15个工人生产30镑的价值。在这一价值中现在20镑是工人的工资,10镑是剩余价值或无酬劳动。

这样,我们就有:

在这1先令10便士中,棉花是1先令4便士,劳动是6便士。

产品涨价是因为棉花贵了1/3。但是产品没有涨价1/3。以前在I的情况下产品值18便士,因此,产品如果涨价1/3,现在就应该值18+6,即24便士。可是它现在只值22便士。以前在1600磅棉纱中包含40镑的劳动,因而1磅棉纱中包含1/40镑,即20/40或1/2先令,也就是6便士。现在1200磅棉纱中包含30镑的劳动,所以花在1磅棉纱上的劳动也是1/40镑,即1/2先令或6便士。虽然劳动和原料按同样程度涨价,1磅棉纱所包含的直接劳动的量仍然和以前一样;不过在这一劳动量中现在包含的有酬劳动较多,无酬劳动较少。所以工资价值的这种变动丝毫不改变产品1磅棉纱的价值。这里劳动依然只是6便士,而棉花现在是1先令4便士,不是过去的1先令。一般说来,如果商品按照它的价值出卖,工资价值的变动就不会引起产品价格的变动。不过以前在6便士中工资占3便士,剩余价值也占3便士。现在在6便士中工资是4便士,剩余价值是2便士。实际上,1磅棉纱包含工资3便士,1600磅棉纱就是3×1600便士,即20镑,而1磅棉纱包含工资4便士,1200磅棉纱就是4×1200便士,即20镑。但3便士和15便士(1先令棉花加3便士工资)之比,构成前一种情况下的1/5的利润,即20%的利润。2便士和20便士(16便士棉花加4便士工资)之比,则构成1/10,或10%的利润。

如果在上述例子中棉花价格保持不变,我们就会得到如下结果。一个工人把80磅棉花纺成棉纱(因为生产方式在所有的例子中保持不变),而1磅棉花又=1先令。

资本现在分解如下\endnote{在紧接着的下一段,马克思自己就认为这种计算法是“不能成立的”,在作这种计算时,马克思的出发点是:纺纱的新加劳动等于40镑,并且只限于按照必要工资提高1/3的假定,把这40镑分为必要劳动和剩余劳动。计算之所不能成立,是因为在这种情况下(在资本为100镑的情况下),以前的工人人数(20人)和他们新加的劳动的数额(40镑)不可能保持不变。既然必要工资按照假定提高了1/3,100镑资本就不可能雇用20个工人,而必须把工人人数减为18+(3/4)人,就象马克思在后来的计算中所做的那样。而工人人数的变化会引起第II种情况的计算中的一切其它变化。这种计算法如下:如果以前一个工人花费1镑,那末现在他花费1+(1/3)镑,20个工人现在花费26+(2/3)镑。因此,为了在原有的规模上继续进行棉纱的生产,就需有106+(2/3)镑的资本,其构成为80c+[26+(2/3)]v。折算成100镑,资本的构成就是75c+25v。——第320页。}:

\todo{}

这种计算法是不能成立;因为,如果一个工人纺80磅棉花的棉纱,那末20个工人就纺1600磅,而不是1466+(2/3)磅,因为假定生产方式保持不变。工人的不同的报酬丝毫不可能改变这个事实。因而,必须另外举例:

\todo{}

在这6便士中,工资是4便士,利润是2便士。2便士和16便士[1先令棉花加4便士工资]之比,是1/8,即[12+(1/2)]%。

最后,如果可变资本价值保持不变(1个工人=1镑),而不变资本价值变了,结果1磅棉花不是花费1先令,而是花费1先令4便士或16便士,那末情况如下:

\todo{}

[588]利润=3便士。它和19便士[16便士棉花加3便士工资]之比恰好是[15+(15/19)]%。

现在我们从还没有发生价值变动的I开始,把所有这四种情况加以比较。

\todo{}

产品价格在III和IV的情况下发生变动,因为不变资本的价值发生了变动。而可变资本价值的变动却不引起产品价格的任何变动,因为直接劳动的绝对量没有变,只是以不同的方式分为必要劳动和剩余劳动。

但是在IV的情况下,价值的变动按同一比例影响不变资本和可变资本,两者的价值都提高了1/3,这时的情况如何呢?

如果只是工资提高(II),那末利润就会从20%下降到[12+(1/2)]%,即下降7+(1/2)单位。如果只是不变资本的价值提高(III),那末利润就会从20%下降到[15+(15/19)]%,即下降4+(4/19)单位。因为在我们考察的IV的情况下,工资和不变资本的价值按同样程度提高,所以利润从20%下降到10%,即下降10单位。可是为什么不下降7+(1/2)+[4+(4/19)]即11+(27/38)这个II和III的差额之和呢?必须弄清楚这个[1+(27/38)]%,它使利润(IV)应当下降到[8+(11/38)]%,而不是10%。利润量决定于剩余价值量,而在剩余劳动率既定的条件下,剩余价值量又决定于工人人数。在I的情况下有20个工人,他们的一半劳动时间是无酬的。在II的情况下只有总劳动的1/3是无酬劳动;因此剩余价值率下降;此外,使用的工人比I少1+(1/4),因而工人人数或总劳动也减少了。在III的情况下剩余价值率又和I一样,一半工作日是无酬的,但是由于不变资本价值提高,工人人数从20减少到15+(15/19),或者说,减少4+(4/19)。在IV的情况下工人人数减少5人(剩余价值率也下降到II的水平,即下降到1/3工作日),也就是从20人减到15人。和I相比,在IV的情况下工人人数减少了5人,和II相比减少了3+(3/4)人,和III相比减少了15/19人;但是和I相比并不是减少1+(1/4)+[4+(4/19)]即5+(35/76)。否则在IV的情况下雇用的工人人数就会是14+(41/76)。

从上面的论述中可得出如下结论:

加入不变资本或可变资本的商品的价值变动,——在生产方式不变,或者说,资本的物质构成不变的情况下(即在使用的直接劳动和积累劳动之间的比例不变的情况下),——只要按同一比例影响可变资本和不变资本,就象IV的情况那样(例如,这里的棉花和工人消费的小麦按同样程度涨价),就不会引起资本有机构成的变动。这里利润率下降(在不变资本和可变资本价值提高的情况下),第一,是因为工资上涨而造成剩余价值率下降;第二,是因为工人人数减少。

在价值变动只影响不变资本或只影响可变资本时,尽管生产方式保持不变,这种变动所起的作用和资本有机构成变动所起的作用一样,会在资本组成部分的价值比例中引起同样的变动。

如果价值的变动只影响可变资本,那末可变资本同不变资本[589]以及同总资本相对来说就会增加。但在这种情况下,不仅剩余价值率下降,而且雇用的工人人数也减少。所以在这种情况下(II)也使用较少的不变资本(其价值仍然不变)。

如果价值的变动只影响不变资本,那末可变资本同不变资本以及同总资本相对来说就会减少。虽然剩余价值率不变,剩余价值量却会减少,因为雇用的工人人数减少了(III)。

最后,可能有这样的情况:价值的变动既影响不变资本,也影响可变资本,但是影响的程度不同。这种情况应当包括在上述的各种情况中。例如,假定价值的变动影响不变资本和可变资本,使不变资本价值提高10%,可变资本价值提高5%。那末,就它们二者都涨价5%来说,一个上涨5+5,另一个上涨5,这就是IV的情况。然而就不变资本在此之外还有5%的变动来说,这就是III的情况。

上面我们只是假定价值提高。在价值下降的情况下会发生相反的结果。例如,从IV推移到I,就是考察价值下降按同一比例影响资本的两个组成部分的情况。如果要说明只是资本的一个组成部分的价值下降会引起怎样的结果,只须把第II种情况和第III种情况相应地修改一下就行了。[589]

\centerbox{※     ※     ※}

[600]关于上述[原料、机器和生活资料的]价值变动对资本有机构成的影响,我还要补充一点:拿投入不同生产部门的资本来说,在这些资本的物质构成在其他方面相同的情况下,只要它们使用的机器或材料的价值较高,就会造成它们的有机构成的不同。例如,如果棉纺织业、丝纺织业、麻纺织业和毛纺织业资本的物质构成完全相同,那末只要它们所使用的材料的价值不同,就会造成这些资本的有机构成不同。[600]

\tsubsubsectionnonum{[(d)总地租的变动取决于市场价值的变动]}

[589]我们回过头来谈谈A表。我们已经看到\fnote{见本册第307—308页。——编者注},假定由于利润下降而形成了利润率10%(在最初只有III被开发时,利润率比较高,后来,II出现,就比只开发III时低了,但仍然高于10%),这个假定在一定条件下,也就是在确实按下降序列发展的时候,可能是正确的,但它决不是由于地租的差别,决不是仅仅由于级差地租的存在而必然得出来的;相反,在按上升序列发展时,这种地租的差别是以利润率始终不变为前提的。

B表。在这里,正如前面已经说明的\fnote{见本册第282页及以下各页。——编者注},III和IV的竞争迫使II把自己的资本抽出一半。在按下降序列发展时,这相反会表现为:所需要的追加供给只有32+(1/2)吨,所以对II只应当投资50镑。

但是在这个表中最令人感兴趣的是:以前投入的资本是300镑,现在只有250镑,即减少1/6;而产品量仍然和以前一样,是200吨。可见,劳动生产率提高了,单位商品的价值下降了。同样,商品的总价值也从400镑降到369+(3/13)镑。同A相比,一吨的市场价值从2镑降到1镑16+(12/13)先令,因为新的市场价值由II的产品的个别价值决定,而不是象以前那样由I的产品的较高的个别价值决定。尽管投资减少了,在产品量不变的情况下产品总价值减少了,市场价值下降了,开发了比较富饶的等级,——尽管有这些情况,同A相比,B的地租却绝对增加了24+(3/13)镑(从70镑增加到94+(3/13)镑)。如果我们考察一下各个等级使地租总额增大的情况,那就会发现,在等级II中,绝对地租率仍然保持不变,因为5镑是50镑的10%,但是地租量减少了一半,从10镑减少到5镑,因为BII的投资减少了一半,从100镑减少到50镑。在BII中,地租总额不是增加,而是减少5镑。其次,BII的级差地租完全消失,因为市场价值现在等于II的产品的个别价值。这使地租总额又减少10镑。因此,II的地租总共减少15镑。

在III中,绝对地租额保持不变,但是由于产品的市场价值下降,它的差额价值也下降了,从而级差地租也减少了。级差地租以前是30镑,现在只有18+(6/13)镑,即减少11+(7/13)镑。因此,II和III合在一起,地租减少26+(7/13)镑。可见,地租总额的增加并不象初看起来的那样是24+(3/13)镑,而是{24+(3/13)+[26+(7/13)]=}50+(10/13)镑,这一点仍须弄清楚。其次,和A相比,对于B来说,AI的绝对地租随着I本身的消失而消失了。因此,这又使地租总额减少10镑。由此可见,应当说明的总共是60+(10/13)镑。但这正好是新等级BIV的地租总额。因此,B的地租总额的增加仅仅用BIV的地租就可说明。BIV的绝对地租同所有其他各等级一样,等于10镑。而50+(10/13)镑的级差地租是这样得出来的:[590]IV的产品的差额价值一吨是10+(470/481)先令,再把它乘以92+(1/2),因为这是这个等级开采出来的吨数。II和III的富饶程度仍然不变;最贫瘠的等级完全消失,然而地租总额增加了,因为,由于IV的富饶程度相对地较高,单单是IV的级差地租就比A的总级差地租大。级差地租不取决于各个被开发等级的绝对富饶程度,不取决于1/2II、III、IV比I、II、III更富饶;可是1/2II、III、IV的级差地租比A的I、II、III的级差地租大。所以会有这种情况,是因为所提供的产品的最大部分——92+(1/2)吨——是从这样一个等级得来的,这个等级的差额价值比A表上任何一个等级的差额价值都大。如果某一等级的差额价值是既定的,那末它的级差地租的绝对额自然就取决于这个等级的产品量。但是在计算差额价值和考察这个价值的形成过程时,已把这个产品量计算进去了。因为IV用100镑资本生产92+(1/2)吨,不多也不少,所以在每吨市场价值等于1镑16+(12/13)先令的B表中,它的差额价值就是每吨10+(470/481)先令。

在A表中,地租总额是70镑,资本是300镑,即占[23+(1/3)]%。而在B表中,如果3/13略去不计,地租总额是94,资本是250,即占[37+(3/5)]%。

C表。这里假定,在IV加进来并开始由II决定市场价值之后,需求不象在B表中那样保持不变,而是随价格下降而提高,以致IV提供的全部追加量92+(1/2)吨都被市场吸收。在每吨价格为2镑时,只有200吨被吸收;在价格为1+(11/13)镑时,需求增加到292+(1/2)吨。如果假定市场容量在每吨价格为1+(11/13)镑时仍然和每吨价格为2镑时一样,那是错误的。相反,市场总是随着价格下降而扩大到一定程度,甚至拿小麦这样一般的生活资料来说也是如此。

这就是我们对C表首先要指出的唯一的一点。

D表。这里假定,只有当市场价值降到1+(5/6)镑时,292+(1/2)吨才被市场吸收;而1+(5/6)镑是I的一吨的费用价格,因而I不带来任何地租,只提供10%的普通利润。这是李嘉图所认为的正常的情况,因此,应当比较详细地讨论一下。

这里和前面的各表一样,先按上升序列;随后我们还要按下降序列考察这个过程。

如果II、III、IV只提供140吨的追加供给\fnote{对I生产的60吨的追加供给。——编者注},即按照2镑一吨被市场吸收掉的追加供给,那末I就继续决定市场价值。

可是情况并不是这样。市场上还有IV生产的92+(1/2)吨余额。如果这个数量是绝对超过市场需要的一般剩余产品,那就会象B表那样,I被完全排挤出市场,II必须抽出自己的一半资本。这样一来,也会象B表那样,由II决定市场价值。可是现在我们假定,在市场价值进一步降低的情况下,市场能够吸收这92+(1/2)吨。这个过程将怎样进行呢?IV、III和1/2II在市场上占绝对统治地位。这就是说,如果市场绝对地说只能吸收200吨,这些等级就会把I排挤出市场。

但是我们先来看看实际情况。在市场上过去只有200吨,现在有292+(1/2)吨。II将按照商品的个别价值1+(11/13)镑出卖,以便在市场上站住脚,并把个别价值等于2镑的I的产品从市场上排挤出去。但是因为按照这个市场价值容纳不了292+(1/2)吨,所以IV和III将挤压II,直到市场价格降到1+(5/6)镑为止。按照这样的价格,IV、III、II和I都将在市场上为自己的产品找到销路,市场按照这个[591]市场价格将吸收全部产品。由于价格的这种下降,供求趋于平衡。一旦追加供给开始超过由原来的市场价值决定的市场容量,每一个等级自然都会竭力把自己的全部产品塞进市场,而把别的等级的产品排挤出去。这只能通过降低价格的办法,也就是把价格降低到使大家都能找到销路的水平。如果价格降低很多,以致I、II等等不得不低于生产费用\endnote{生产费用(《Produktionskosten》)这一术语,马克思在这里以及有时在后面是用在生产费用加平均利润,即费用价格(生产价格)的意义上。《Produktionskosten》这一术语在《资本论》第三卷的一些地方也有这样用法。见《马克思恩格斯全集》1964年德文版第25卷第665、686、747—749页。——第327页。}出卖商品,那它们自然会被迫从生产中抽出自己的资本。如果为了使产品适应市场状况,价格看来不必降低这么多,那末全部资本就能按照产品的这个新市场价值在这个生产领域继续发挥作用。

但是再往下看,很明显,在这样的条件下,决定产品市场价值的就不是I、II这些较坏的地段,而是III、IV这些较好的地段,因而,正象施托尔希对于这种情况正确理解的那样,[26]是较好地段的地租决定较坏地段的地租。

IV按照那种使它能把自己的全部产品塞进市场,又能消除其他等级的各种反抗的价格出卖产品。这个价格就是1+(5/6)镑。如果IV按照更高的价格出卖,那末市场的容量就会缩小,互相排挤的过程又会重新开始。

只有假定II等等所提供的追加供给,仅仅是市场按照I的产品决定的市场价值吸收掉的那个追加供给,I才决定市场价值。如果追加供给超过这种界限,I会起完全消极的作用,并由于它在市场上占据的地位仅仅迫使II、III、IV作出相应的反应,直到价格减低,使市场足以吸收生产出来的全部产品为止。现在看来,按照这种实际上由IV决定的市场价值,IV除绝对地租外,还支付49+(7/12)镑级差地租,III除绝对地租外,还支付17+(1/2)镑级差地租,II则不支付任何级差地租,只支付绝对地租的一部分9+(1/6)镑,而不是绝对地租的全部10镑。为什么?因为新的市场价值1+(5/6)镑,虽然也高于II的产品的费用价格,却仍低于它的个别价值。如果新的市场价值等于它的个别价值,II就支付10镑绝对地租,即等于个别价值和费用价格之间的差额。但是,因为新的市场价值低于II的产品的个别价值,——它所支付的实际地租等于市场价值和费用价格之间的差额,而这个差额小于它的个别价值和它的费用价格之间的差额,——所以II只支付它的绝对地租的一部分9+(1/6)镑,而不是10镑。

{实际地租等于市场价值和[个别]费用价格之间的差额。}

绝对地租等于个别价值和费用价格之间的差额。

级差地租等于市场价值和个别价值之间的差额。

实际地租,或者说,总地租,等于绝对地租加级差地租;换句话说,等于市场价值超过个别价值的余额加个别价值超过费用价格的余额,即等于市场价值和费用价格之间的差额。

因此,如果市场价值等于个别价值,那末级差地租就等于零,总地租就等于个别价值和费用价格之间的差额。

如果市场价值大于个别价值,那末级差地租就等于市场价值超过个别价值的余额,总地租就等于这个级差地租加绝对地租。

如果市场价值小于个别价值,大于费用价格,那末级差地租就是一个负数;因而总地租就等于绝对地租加这个负级差地租,即减个别价值超过市场价值的余额。

如果市场价值等于费用价格,那末地租就整个等于零。

为了把这一切用方程式来表现,我们用AR表示绝对地租,用DR表示级差地租,用GR表示总地租,用MW表示市场价值,用IW表示个别价值,用KP表示费用价格。这样,我们就得出如下的方程式:

[592](1)AR=IW-KP=+y.

(2)DR=MW-IW=x.

(3)GR=AR+DR=MW-IW+IW―KP=y+x=MW-KP.

如果MW>IW,那末MW-IW=+x,因此DR就是正数,并且GR=y+x。

因此,MW-KP=y+x,或MW-y-x=KP,或MW=y+x+KP。

如果MW<IW,那末MW-IW=-x,因此DR就是负数,并且GR=y-x。

因此,MW-KP=y-x,或MW+x=IW,或MW+x-y=KP,或MW=y-x+KP。

如果MW=IW,那末DR=0,x=0,因为MW-IW=0。

因而在这种情况下,GR=AR+DR=AR+0=MW-IW+IW-KP=0+IW-KP=IW-KP=MW-KP=+y。

如果MW=KP,那末GR(或MW-KP)=0。

在上面假定的[D表]情况下,I不支付任何地租。为什么?因为绝对地租等于个别价值和费用价格之间的差额,级差地租等于市场价值和个别价值之间的差额;而在这里,市场价值等于I的产品的费用价格。I的个别价值等于每吨2镑;市场价值等于1+(5/6)镑。因此,I的级差地租等于1+(5/6)镑减2镑,即等于-1/6镑。I的绝对地租等于2镑减1+(5/6)镑,即等于它的个别价值和它的费用价格之间的差额(+1/6镑)。于是,因为I的实际地租等于绝对地租(+1/6镑)加级差地租(-1/6镑),所以等于零。因此,I的产品既不支付级差地租,也不支付绝对地租,只支付费用价格。这个产品的价值等于2镑,但它按照1+(5/6)镑出卖,即低于它的价值1/12或[8+(1/3)]%出卖。I不能卖得更贵,因为决定市场的不是它,而是和它相对的IV、III、II。I所能做的,只是按照每吨1+(5/6)镑的价格提供追加供给。

I不支付地租这个事实,是由市场价值等于它的费用价格造成的。

但这个事实是下列情况的后果:

第一,I相对不肥沃。它必须按照1+(5/6)镑提供追加的60吨。假定它用100镑资本不只是提供60吨,而是提供64吨,比II少1吨。在这种场合,只要在这一等级投入93+(3/4)镑资本,就足以提供60吨。这样I的每吨个别价值就是1+(7/8)镑,或1镑17+(1/2)先令,而它的费用价格就是1镑14+(3/8)先令。因为市场价值等于1+(5/6)镑,或1镑16+(2/3)先令,所以市场价值和费用价格之间的差额就等于2+(7/24)先令。按60吨计算,[593]地租就是6镑17+(1/2)先令。

由此可见,如果一切条件不变,I只要比现在肥沃1/15(因为60/15=4),它就还会支付一部分绝对地租,因为在市场价值和它的费用价格之间存在一个差额,虽然这个差额比它的个别价值和它的费用价格之间的差额小。因此在这里,最坏的土地如果比我们假定的肥沃一些,也还会提供地租。如果I比现在绝对肥沃一些,那末II、III、IV和它相比,就相对不肥沃一些。I的个别价值和它们的个别价值之间的差额就比较小。因此,I不提供任何地租这一点,是在同样程度上由两个情况造成的:它自己既不绝对肥沃一些,II、III、IV也不相对不肥沃一些。

但是第二,假定I的产量是既定的,100镑资本生产60吨。如果II、III、IV,特别是作为新的竞争者出现在市场上的IV,和I相比不仅相对不肥沃一些,而且绝对不肥沃一些,那末I就会提供地租,虽然这只是一部分绝对地租。事实上,因为市场按照1+(5/6)镑吸收292+(1/2)吨,所以较少的吨数,例如280吨,市场就会按照高于1+(5/6)镑的市场价值来吸收。但是任何高于1+(5/6)镑即高于I的费用价格的市场价值,都会为I提供地租——它等于市场价值减I的产品的费用价格。

因此,也可以说,I不提供地租是由于IV绝对肥沃,因为在市场上只有II和III是竞争者的时候,I还提供地租,甚至当IV出现了,也就是说,即使有追加的供给,但只要IV使用资本100镑生产出来的不是92+(1/2)吨,而只是80吨,I仍会继续提供地租,虽然比原来的少些。

第三,我们曾假定,花费资本100镑,绝对地租是10镑,即资本的10%,或费用价格的1/11,因而在农业上资本100镑生产出来的产品价值等于120镑,其中10镑是利润。

但是不要以为,如果我们说“在农业上花费资本100镑”,并且如果一个工作日等于1镑,那就是在农业上也花费了100工作日。一般说来,如果资本100镑等于100工作日,那末不管这笔资本花费在什么生产部门,都不能说[这笔资本的产品等于100工作日]。假定1镑金等于一个12小时的工作日,并且假定这就是正常的工作日。这里产生的第一个问题就是:对劳动的剥削率怎样?也就是说,在12小时中,工人有几小时为自己,为再生产自己的工资(等价物)劳动?他又有几小时白白地为资本家劳动?也就是说,资本家没有支付报酬却拿来出卖的那个劳动时间,因而是形成剩余价值的源泉、资本增殖的源泉的那个劳动时间有多大?如果这个比率等于50%,那末工人就是为自己劳动8小时,为资本家白白地劳动4小时。产品等于12小时,或1镑(因为根据假定,1镑金包含12小时劳动时间)。在这等于1镑的12小时中,8小时补偿资本家支付给工人的工资,4小时构成资本家的剩余价值。因此,花费工资13+(1/3)先令,会得到剩余价值6+(2/3)先令。换句话说,花费资本1镑,会得到剩余价值10先令,花费100镑,会得到50镑。在这种情况下,用资本100镑生产出来的商品的价值会等于150镑。资本家的利润一般地说在于出卖产品中包含的无酬劳动。正常的利润就是来自出卖不支付报酬的东西。

[594]第二个问题就是:资本的有机构成怎样?资本的价值中由机器等和原料构成的那一部分,只不过在产品中再生产出来,再现出来,它是保持不变的。对于资本的这个组成部分,资本家必须按其价值支付。因此,它是作为既定的、预定的价值加入产品的。只有被资本家使用的劳动仅仅有一部分得到资本家的支付,虽然它全部加入产品的价值并被资本家全部购买。因此,在假定上述的对劳动的剥削率的情况下,等量资本的剩余价值的大小取决于资本的有机构成。如果资本a的构成是80c+20v,产品的价值就等于110,利润就是10%(虽然在产品中包含的无酬劳动是50%)。如果资本b的构成是40c+60v,产品的价值就等于130,利润就是30%,虽然产品中也还是只包含50%的无酬劳动。如果资本c的构成是60c+40v,产品价值就等于120,利润就是20%,虽然这里包含的无酬劳动也是50%。因此,从这等于300的三笔资本得到的总利润,等于10+30+20=60,每100平均是20%。上述每一笔资本如果把它生产的商品卖120镑,就会提供这个平均利润。资本a的构成是80c+20v,它比自己产品的价值高10镑出卖产品,资本b的构成是40c+60v,它比自己产品的价值低10镑出卖产品,资本c的构成是60c+40v,它按照自己产品的价值出卖产品。合计起来,这些商品是按照它们的价值即120+120+120=360镑出卖。而实际上,a+b+c的产品的价值=110+130+120=360镑。但是各笔资本的产品的价格有的高于它们的价值,有的低于它们的价值,有的等于它们的价值,结果这些资本中的每一笔都提供20%的利润。这样改变了形态的商品价值也就是商品的费用价格,竞争不断地使费用价格成为市场价格的引力中心。

就投入农业的100镑来说,我们假定资本的构成是60c+40v(就v来说,也许还太低);这样,产品的价值就等于120。但是这样的价值等于[上面假定的]工业品的费用价格。因此我们假定上例中100镑资本生产出来的产品的平均价格[不是120,而]是110镑。现在我们说:如果农产品按照自己的价值出卖,那末它的价值就比它的费用价格高10镑。它就会提供10%的地租,并且我们认为,农产品和别的产品不同,它不按照自己的费用价格出卖,而按照自己的价值出卖,这是资本主义生产下的正常现象。这是土地所有权造成的后果。如果平均利润等于10%,总资本的构成就会是80c+20v。我们假定农业资本的构成是60c+40v,也就是说,这种资本构成中用于工资的(即用于直接劳动的)份额,比投在其余生产部门的资本总额中的工资份额大。这表明,在这个部门中,劳动生产率的发展相对较低。当然,在农业的某些部门例如畜牧业中,资本的构成也许是90c+10v,即v与c之比也许还小于工业总资本。但决定地租的并不是这种部门,而是真正的农业,即农业中生产主要生活资料如小麦等等的那一部分。在其他农业部门中,地租不决定于投在这些部门本身的资本的构成[595],而决定于生产主要生活资料的那种资本的构成。资本主义生产存在本身是以这样一种情况为前提,即生活资料的最主要成分是植物性食物,而不是动物性食物。农业各部门的地租之间的关系是个次要问题,在这里我们可以不去注意,也可以不去考察。

这样,为使绝对地租是10%,我们假定资本一般的平均构成:

非农业资本是80c+20v,

农业资本是60c+40v。

现在要问,农业资本的另一种构成,例如50c+50v或70c+30v,是否会影响D表中所假定的I不提供任何地租的情况呢?在第一种情况下,产品的价值等于125镑;在第二种情况下,产品的价值等于115镑。在第一种情况下,农业资本构成和非农业资本构成的不同所造成的差额是15镑,在第二种情况下,差额是5镑。换句话说,农产品的价值和它的费用价格之间的差额,在第一种情况下比我们假定的高50%,在第二种情况下低50%。

如果出现第一种情况,就是说,如果100镑资本生产出来的产品价值等于125镑,那末在A表中,I的每吨价值就等于2+(1/12)镑。这也就是A的市场价值,因为这里是I决定市场价值。相反,对于AI来说,费用价格仍然是1+(5/6)镑。而因为根据假定,292+(1/2)吨[在表中]只有按照1+(5/6)镑才能卖掉,所以现在所考察的农业资本有机构成的这种变动,对DI不提供任何地租的情况不发生任何影响;同样,如果农业资本的构成是70c+30v,换句话说,如果农产品的价值和它的费用价格之间的差额只是5镑,即只是我们假定的一半,那末情况也丝毫不会发生变化。由此可见,假定费用价格,因而非农业资本的平均有机构成(80c+20v)不变,那末农业资本的构成较高还是较低,对于这里的情况(对于DI)是没有任何意义的,虽然这种差别对于A表有意义,并且[在各个表中]使绝对地租发生50%的变化。

但是我们现在假定相反的情况:农业资本的构成仍旧是60c+40v,而非农业资本的构成改变了。它不是80c+20v,而是70c+30v,或90c+10v。在第一种情况下,平均利润将是15镑,或者说,比原来假定的高50%;在第二种情况下,平均利润将是5镑,或者说,比原来假定的低50%。在第一种情况下,绝对地租是5镑。因此,这对于DI同样不发生任何影响。在第二种情况下,绝对地租等于15镑。这也不会使DI发生任何变化。因此,对于DI来说,所有这一切都是毫无关系的,虽然对于A、B、C和E表有着重要的意义,也就是说,每当新的等级(不管按上升序列还是按下降序列推移都一样)按照原来的市场价值只提供必要的追加供给时,对于决定绝对地租和级差地租的绝对量有着重要的意义。

\centerbox{※     ※     ※}

现在产生下面一个问题:

D表的情况实际上是可能的吗?而且首先,李嘉图假定的这种情况是正常的吗?这种情况只有在两个条件下才可能是正常的:

或者,农业资本的构成是80c+20v,即象非农业资本的平均构成一样,从而农产品的价值就会等于非农产品的费用价格。这从统计材料来看目前还是不正确的。关于农业劳动生产率相对较低的假定,无论如何比李嘉图关于农业劳动生产率绝对递减的假定符合实际。

[596]李嘉图在第一章(《论价值》)中假定,金银矿中的资本构成是平均构成(固然,他这里谈的只是固定资本和流动资本,不过我们要予以“纠正”)。根据这个假定,这些矿山始终只能有级差地租,决不会有绝对地租。而这个假定本身又建立在另一个假定上,即比较富饶的矿山所提供的追加供给,总是比按照原来的市场价值所要求的追加供给多。但绝对令人不解的是,为什么不能同样存在相反的情况。单是存在级差地租这件事就已经证明,原有市场价值保持不变的追加供给是可能的。因为,IV或III或II这些等级,如果不是按照I的产品的市场价值(不管这个市场价值是怎样决定的),也就是说,不是按照同这些等级的供给的绝对量无关而决定的市场价值出卖自己的产品,它们就不会提供级差地租。

或者,如果D表所假定的各种情况始终是正常的,就是说,如果由于IV、III和II的竞争,特别是IV的竞争,I总是不得不比自己产品价值低一个绝对地租全额即按照费用价格来出卖自己的产品,那末D表的情况就必定总是正常的。IV、III和II都有级差地租这件事本身就证明,它们出卖自己产品所按照的市场价值高于它们的个别价值。如果李嘉图认为I不可能有这种情况,那只是因为他事先假定不可能有绝对地租,而他所以认为不可能有绝对地租,是由于他是以价值和费用价格等同这个前提为出发点。

我们拿C表的情况来看,这里的292+(1/2)吨按照1镑16+(12/13)先令的市场价值都能找到销路。并且,我们也象李嘉图那样从IV出发。当市场只要求92+(1/2)吨时,IV按照每吨1镑5+(35/37)先令出卖,也就是说,它把100镑资本生产出来的商品,按照它的价值120镑出卖,从而提供10镑绝对地租。为什么IV要低于产品价值、按费用价格出卖自己的商品呢?当市场上只有IV的时候,III、II、I不可能同它竞争。连III的产品的费用价格都比使IV提供10镑地租的那个价值高,II和I的费用价格比这个价值就更高了。因此,即使III、II和I只是按照费用价格出卖自己的产品,它们也不可能同IV竞争。

假定总共只有一个等级——是较好还是较坏等级的土地,是IV或I,或III,或II,这对于理论无关紧要;假定这个等级是作为自然要素而存在,也就是说,对一般可以支配的并且在这个生产部门可以被吸收的资本量和劳动量来说作为自然要素而存在,这样,这种土地就不会构成任何界限,而是现有的劳动量和资本量的相对无限的活动场所;因而,假定不存在级差地租,因为被耕种的土地没有自然肥力上的不同,这样,就不存在任何级差地租(或者说,级差地租在这里微不足道);再假定不存在任何土地所有权,那就很明显,不存在绝对地租,因而,也就根本不存在任何地租(因为根据假定,也不存在级差地租)。这是同义反复。因为绝对地租的存在不仅以土地所有权为前提,而且就是被当作前提的土地所有权,即被资本主义生产的作用决定并改变了形态的土地所有权。这种同义反复对于解决问题毫无用处,因为我们正是用农业中的土地所有权对商品价值转变为平均价格的资本主义平均化进行抵抗这一点,来说明绝对地租的形成。如果我们取消土地所有权的这种作用,取消资本竞争在这个投资领域内遇到的这种抵抗,这种特殊的抵抗,那我们当然也就取消了地租存在的前提本身。而且这里还是一个自相矛盾的假定(威克菲尔德先生在他的殖民理论\endnote{关于威克菲尔德的殖民理论,见卡·马克思《资本论》第1卷第25章(见《马克思恩格斯全集》中文版第23卷第25章)。——第338页。}中很好地看到了这一点):一方面是发达的资本主义生产,另一方面又没有土地所有权。在这种情况下雇佣工人从哪里来呢?

在殖民地中有某种近似的情况,即使在法律上存在土地所有权,——这是由政府无偿地分给土地造成的,如当初英国向海外殖民时的情况,——并且,即使[597]政府在实际上培植土地所有权,以非常便宜的价格出卖土地,如美国的情况(1美元或大致这么多的东西可买一英亩土地)。

这里应当把两种类型的殖民地区别开来。

第一,说的是本来意义的殖民地,例如美国、澳大利亚等地的殖民地。这里从事农业的大部分殖民者,虽然也从宗主国带来或多或少的资本,但并不是资本家阶级,他们的生产也不是资本主义生产。这是在或大或小的程度上自己从事劳动的农民,他们主要是为了保证自身的生活,为自己生产生存资料。因此他们的主要产品并不是商品,目的也不是为了做买卖。他们把自己产品中超过他们自己消费的余额卖掉,换取运入殖民地的工业品等等。另一小部分殖民者,住在沿海,住在通航河流附近等地,形成商业城市。这里也还谈不上资本主义生产。但是,即使资本主义生产逐渐开始发展,以致对于自己从事劳动和自己占有土地的农场主来说,开始起决定作用的是出卖自己的产品和由出卖而得的盈利,——即使在这种情况下,也会发现,只要土地对资本和劳动来说还处在自然要素那样的丰富状态,从而,只要土地实际上还是无限的活动场所,也就继续存在第一种殖民形式,因而生产也决不按照市场的需要,即按照某种既定的市场价值来调节。第一类殖民者把他们所生产的超过他们自己直接消费的一切东西投入市场,产品的售价只要高于工资就行。他们是,并且在一个很长时期内依然是那些多少已经按照资本主义方式生产自己产品的农场主的竞争者,因而他们使农产品的市场价格经常低于它们的价值。因此,耕种较坏等级的土地的农场主,只要能得到平均利润,而在出卖自己的农场时能够收回自己的投资,就会感到很满足,因为这在大多数情况下是办不到的。可见,这里有两个重大情况共同起着作用:第一,资本主义生产还没有在农业中占统治地位;第二,土地所有权虽然在法律上存在着,实际上还只是偶然的现象,还只是本来意义上的土地占有。换句话说,虽然土地所有权在法律上存在着,但由于土地对劳动和资本来说作为自然要素而存在的关系,它还不能对资本进行抵抗,还不能把农业变成与非农业生产部门有别的、对投资进行特殊抵抗的活动场所。

在第二种殖民地(种植园)中,一开始就是为了做买卖,为了世界市场而生产,这里存在着资本主义生产,虽然这只是形式上的,因为黑人奴隶制排除了自由雇佣劳动,即排除了资本主义生产的基础本身。但是在这里我们看到的是把自己的经济建立在黑人奴隶劳动上的资本家。他们采用的生产方式不是从奴隶制产生的,而是接种在奴隶制上面的。在这种场合,资本家和土地所有者是同一个人。土地对资本和劳动来说作为自然要素而存在,并不对投资进行任何抵抗,因而也不对资本竞争进行任何抵抗。这里也并没有形成与土地所有者不同的租地农场主阶级。只要维持着这种状况,就没有任何东西妨碍费用价格调节市场价值。

所有这些前提都和绝对地租存在的前提毫无共同之处。绝对地租存在的前提是:一方面有发达的资本主义生产,另一方面有土地所有权,这种土地所有权不仅在法律上存在,而且在实际上对资本进行抵抗,保护这个活动场所不受资本侵占,只有在一定条件下才把地盘让给资本。

在这样的情况下,即使只耕种IV或III或II或I,也会存在绝对地租。资本只有交纳地租,也就是说,只有按照农产品的价值出卖农产品,才能在这唯一存在的土地等级中占领新地盘。也只有在这样的情况下,才能把投入农业(即投入某种自然要素本身,投入初级生产)的资本和投入非农业生产的资本加以比较和区别。

下一个问题是:

在以I为出发点时,很清楚,如果II、III、IV只提供按原来的市场价值所能提供的那些追加供给,那末它们就会按照I所决定的市场价值出卖自己的产品;因此,它们除绝对地租外,还将根据它们各自的相对肥力提供级差地租。相反,当以IV为出发点时,那末这里看来[598]就可能有一些异议。

我们已经看到,[在B表和C表]如果II按照自己产品的价值出卖产品,即按照1+(11/13)镑或1镑16+(12/13)先令出卖产品,II就会得到绝对地租。

在D表,紧接着的下一个等级(按下降序列)即III的费用价格,比能提供10镑地租的IV的产品价值还高。因此,即使III按照自己产品的费用价格出卖产品,这里也谈不上竞争或者按较低价格的供给。但是如果IV已经不能满足整个需求,如果市场要求的比92+(1/2)吨还多,那末产品的价格就会上涨。在上述情况下,产品的价格必定要每吨上涨3+(43/111)先令,III才能作为按照自己产品的费用价格出卖产品的竞争者出现。现在要问:III是否会作为这样的竞争者出现呢?我们现在用另一种方式来说明这种情况。要使IV的产品价格上涨到1镑12先令,即上涨到III的产品的个别价值,需求不一定非增加75吨不可;至少就主要农产品来说情况是这样,主要农产品的供给不足所引起的价格上涨,大大超过与算术上的供给不足相应的程度。但是,如果IV的产品的价格上涨到1镑12先令,那末按照这种等于III的产品个别价值的市场价值,III的产品就会提供绝对地租,而IV就会提供级差地租。一般说来,如果有追加的需求,那末III就能按照自己产品的个别价值出卖产品,因为这时正是III这个等级决定市场价值,而在这种情况下,就不会有任何理由迫使土地所有者放弃地租。

但是,假定IV的产品的市场价格只涨到1镑9+(1/3)先令,即III的产品的费用价格。或者,为使这个例子更明显起见,假定III的产品的费用价格只是1镑5先令,即只比IV的产品的费用价格高1+(8/37)先令。它必定比后者高,因为III的肥力比IV的肥力低。现在III能否被耕种,能否开始同高于III的产品的费用价格也就是按照1镑5+(35/37)先令来出卖自己产品的IV竞争呢?这取决于这里是否有追加需求。如果有追加需求,那末IV的产品的市场价格就会涨到它的价值以上,即涨到1镑5+(35/37)先令以上。这时III在任何情况下都会高于自己产品的费用价格出卖产品,尽管还不能因此而得到它的绝对地租的全额。

或者,没有追加需求。这里又可能有两种情况。III的竞争,只有在耕种III的农场主同时又是这块土地的所有者时才可能发生:对于这种作为资本家的农场主来说,土地所有权不成其为障碍,并不进行抵抗,因为他不是作为资本家而是作为土地所有者来支配这块土地。III的竞争会迫使IV低于原来的价格,即低于1镑5+(35/37)先令,甚至低于III的产品的费用价格1镑5先令出卖自己的产品。这样一来,III就会被击退。而IV总是能够把III击退的,因为它只要把价格降低到本身的费用价格即比III更低的费用价格就行了。但是,如果由于III的产品出现在市场上而造成的价格降低,市场的容量增大了,那又会怎样呢?要就是:市场扩大到如此程度,以致虽有新的75吨出现,IV仍能象以前一样把自己的92+(1/2)吨卖掉;要就是:市场没有扩大到如此程度,以致IV和III都有一部分产品成为过剩的。在这种情况下,IV由于在市场上占统治地位,就会把价格降低,直到III的资本缩小到应有的限度,即投入III的资本数量恰好足以使IV的全部产品被吸收掉。但是按照1镑5先令的价格可以卖掉全部产品,并且,因为III按照这个价格会卖掉这种产品的一部分,所以IV就不能卖得更贵。不过这是唯一可能出现的不是由追加需求引起,但会导致市场容量增大的暂时生产过剩的情况。而这种情况之所以可能发生,只是由于在等级III资本家和土地所有者是同一个人,可见其前提又是:土地所有权不是作为和资本对抗的力量存在,因为资本家自己是土地所有者,并且为了资本家的利益而牺牲土地所有者的利益。如果在等级III土地所有权本身和资本相对抗,那就没有任何理由可以使土地所有者情愿把自己的耕地让人家耕种,而不收取地租,也就是说,使土地所有者在IV的产品价格至少上涨到高于III的产品的费用价格以前,情愿把自己的耕地让人家耕种。如果这种涨价只是[599]微不足道的,那末在任何进行资本主义生产的国家里,III就会仍然被排除于资本活动领域之外,除非这种土地是无论以别的什么形式都不能提供地租的土地。但是,这种土地在提供地租以前,在IV的产品价格上涨到高于III的产品的费用价格以前,也就是在IV除自己原有的地租外还提供级差地租以前,是决不会被耕种的。随着需求的进一步增长,III的产品价格会上涨到它的价值的水平,因为II的产品的费用价格高于III的产品的个别价值。只要III的产品价格超过1镑13+(11/13)先令,就是说,只要这个价格开始为II提供某种地租,II就会被耕种。

但是现在在D表中已经假定I不提供任何地租。然而这个等级不提供地租只是因为:它根据假定已是被耕种的土地,由于IV的出现所引起的市场价值的变动,不得不低于产品价值即按照产品的费用价格来出卖产品。这种土地继续在农业上被利用只有在下述情况下才有可能:即土地所有者自己是农场主,因而在这种个别的场合,同资本相对抗的土地所有权消失了;或者利用土地的农场主是个小资本家,他情愿得到小于10%的利润,或者他是个工人,他想要得到的就是比工资略多一些或者就是工资,他把自己的等于10镑或少一些(例如9镑)的剩余劳动交给土地所有者,而不是交给资本家。虽然在后两种场合都给土地所有者交付租金,但是从经济学来说,这并不是地租,而我们所谈的只是地租。在一种场合,农场主只不过是一个农业工人,在另一种场合,农场主是介于农业工人和资本家之间的一种中间人物。

有种说法认为,土地所有者不能象资本家从某个生产部门抽出自己的资本那样容易地从市场上抽出自己的耕地。再也没有比这样的说法更荒谬的了。对于这种见解的最好的反驳是:在欧洲最发达的国家例如英国,有大量的肥沃土地未被耕种;土地从农业中被抽出,用于修建铁路或房屋,或者为了这些目的留下备用;还有例如在苏格兰高地等处,土地被土地所有者用作射击场或猎场。英国工人为了把未耕种的土地夺到手而进行斗争,但没有成功,这也是最好的反驳。

必须指出:绝对地租额降到自己的正常量以下,这或者是由于市场价值低于该等级的产品的个别价值(如DII的情况),或者是由于好地的竞争迫使一部分资本从坏地中抽出(如BII的情况),或者是由于地租完全消失(如DI的情况),——在所有这些场合,都是以下列各点为前提:

(1)在地租完全消失的地方,土地所有者和资本家是同一个人,就是说,这里土地所有权对资本的抵抗,以及土地所有权对资本活动场所的限制,作为个别的和例外的情况消失了。这里的情形也和殖民地一样,只不过土地所有权这个前提的消失在这里是个别的情况;

(2)比较肥沃的土地的竞争,或者还有比较不肥沃的土地的竞争(在下降序列中),会造成生产过剩并强制地使市场容量增大,会由于强制地降低价格而引起追加的需求。但这里的情况正好是李嘉图所没有假定的,因为他所有的推论都始终从这样一个前提出发:所满足的只是必要的追加需求;

(3)在B、C和D表中II和I完全不提供地租,或者提供的不是绝对地租的全额,因为比较肥沃的土地的竞争,迫使它们低于自己产品的价值出卖产品[或者象BII的情况,从生产中抽出一部分资本]。相反,李嘉图假定,它们是按照产品价值出卖产品,并且总是由最坏的土地决定市场价值,而正是在他认为正常的DI中发生了根本相反的情况。此外,他的推论的前提,总是生产按下降序列进行。

在非农业资本的平均构成是80c+20v,而剩余价值率等于50%时,如果农业资本的构成等于90c+10v,就是说,如果它比工业资本的构成还高——而这[600]对于资本主义生产来说在历史上是不正确的——那就没有绝对地租;如果农业资本的构成是80c+20v——这也是至今没有过的事情——那也不会有绝对地租;如果农业资本的构成比工业资本的构成低,例如等于60c+40v,那就会得到绝对地租。

如果从这个理论出发,那末,根据不同等级的肥力以及它们对市场的关系,即根据这一或那一等级在市场上占支配地位的程度,可以有如下各种情况:

(A)最后一个等级支付绝对地租。它决定市场价值,是因为所有的等级按照这个市场价值只提供必要供给;

(B)最后一个等级决定市场价值;它支付绝对地租,按绝对地租的全率支付,但不按绝对地租原来的全额支付,因为III和IV的竞争迫使它从生产中抽出一部分资本;

(C)I、II、III、IV按照原来的市场价值提供的超额供给,必然引起市场价值的下降,而由较高等级调节的已经下降的市场价值,又造成市场容量的增大。I只支付一部分绝对地租,II只支付绝对地租;

(D)由于较好等级这样调节市场价值,或者说,由于较好等级通过超额供给支配较坏等级,I的地租完全消失,II的地租降到绝对地租的水平以下;最后,

(E)较好等级使市场价值降到I的产品的费用价格以下,从而把I从市场上排挤出去。现在由II来调节市场价值,因为按照这个新的市场价值,所有的三个等级都只提供必要供给。[600]

[600]现在我们回过头来谈李嘉图。

\centerbox{※     ※     ※}

不言而喻,当我们谈农业资本的构成时,其中并不包括土地价值或土地价格。土地价格不过是资本化的地租。

\tchapternonum{[第十三章]李嘉图的地租理论(结尾)}

\vicetitle{[(1)李嘉图关于不存在土地所有权的前提。向新的土地推移取决于土地的位置和肥力]}

现在回过来研究李嘉图著作的第二章《论地租》。首先遇到的是在斯密那里已经熟悉的“殖民理论”\fnote{见本册第253—254页和第265—266页。——编者注}。这里只要简单指出思想上的逻辑联系就够了。

\begin{quote}{“初到一个地方殖民,那里有着大量富饶而肥沃的土地,为维持现有人口的生活只需耕种很小一部分土地,或者,这些人口所能支配的资本实际上只能耕种很小一部分土地,在这样的时候,不存在地租;当大量土地还没有被占有,因此〈因为没有被占有,李嘉图后来把这一点完全忘记了〉谁愿意耕种就归谁支配的时候,没有人会为使用土地付出代价。”(第55页)}\end{quote}

{因此,这里是以不存在土地所有权为前提的。虽然这个过程的描述,对现代民族的殖民来说接近于正确,但是,第一,它不适用于发达的资本主义生产;第二,如果把这个过程设想为旧欧洲的历史发展进程,那就错了。}

\begin{quote}{“按照一般的供求规律,这样的土地是不可能支付地租的,其理由同以上所说的使用空气、水或其他任何数量上无限的自然赐予无须付任何代价一样……使用这些[601]自然力之所以不付代价,是因为它们取之不尽,每个人都可以支配……如果所有土地都具有同一特性,如果它们的数量无限、质量相同,使用土地就不能索取代价〈因为土地根本不能变成私有财产〉,除非它的位置特别有利〈李嘉图本应加上一句:并且归一个所有者支配〉。因此,只是由于土地在数量上并非无限,在质量上并不相同,又因为随着人口的增长,质量较坏或位置比较不利的土地投入耕种,使用土地才支付地租。随着社会的发展,就肥力来说属于二等的土地投入耕种时,在一等地上立即产生地租,这一地租的大小将取决于这两块土地质量上的差别。”(第56—57页)}\end{quote}

正是这一点我们必须加以研究。这里的逻辑联系是这样的:

如果土地,——李嘉图在谈到初到一个地方殖民时(斯密的殖民理论)是这样假定的——如果富饶而肥沃的土地对现有人口和资本来说作为自然要素而存在,实际上是无限的;如果“大量”这种土地“还没有被占有”,因此——因为“还没有被占有”——“谁愿意耕种就归谁支配”,在这种情况下,自然不会为使用土地付任何代价,不会有任何地租。如果土地——不仅对资本和人口来说,而且实际上也是一个无限的要素(象空气和水一样“无限”)——“数量无限”,那末,一个人对土地的占有实际上根本不排斥另一个人对土地的占有。这样,就不可能有任何私人的(也不可能有“公共的”或国家的)土地所有权存在。在这种情况下,如果所有的土地质量相同,那就根本不可能为土地支付地租。至多会向“位置特别有利”的土地的占有者支付地租。

因此,在李嘉图所假定的情况下——即在土地“没有被占有”,“因此”,未被耕种的土地“谁愿意耕种就归谁支配”的情况下——支付地租,那只能是由于“土地在数量上并非无限,在质量上并不相同”,就是说,因为有不同等级的土地存在,而同一等级的土地又是“数量有限”。我们说,在李嘉图的前提下只能支付级差地租。但是,李嘉图不是这样加以限制,而是——撇开他的不存在土地所有权这个前提——立刻匆促作出结论说:使用土地,从来不支付绝对地租,只支付级差地租。

因此,问题的关键在于:如果土地对资本来说作为自然要素而存在,那末,资本在农业方面的活动就会同它在其他任何生产部门的活动完全一样。在这种情况下就不存在土地所有权,不存在地租。至多在一部分土地比另一部分土地肥沃的时候,象在工业中一样,能够有超额利润存在。在农业中,这种超额利润由于有土地的不同肥沃程度为自然基础而作为级差地租固定下来。

相反,如果土地(1)是有限的,(2)是被占有的,如果资本遇到作为前提的土地所有权——在资本主义生产发展的国家,情况正是这样,而在那些不是象旧欧洲那样存在着这种前提的国家,资本主义生产本身就为自己创造这种前提,例如美国就是这样,——那末,土地对资本来说一开始就不是自然要素那样的活动场所。因此,在级差地租之外,还是存在地租的。但是从一个等级的土地推移到另一个等级的土地,不论是按上升序列(I、II、III、IV)还是按下降序列(IV、III、II、I),也都和李嘉图前提下发生的情况不同。因为,不论在I还是在II、III、IV使用资本,都会遭到土地所有权的抵抗,如果倒过来从IV推移到III等等,情况也是一样。从IV推移到III等等的时候,IV的产品价格单是提高到使III使用的资本能够得到平均利润,那是不够的,它必须提高到使III能够支付地租。如果从I推移到II等等,那末,使I能够支付地租的那个价格,不仅能够使II支付这种地租,并且除此之外,还支付级差地租,这是不言而喻的。李嘉图提出的不存在土地所有权的前提,当然排除不了那个受土地所有权的存在制约并与此密切联系的规律的存在。

李嘉图说明了在他的前提下怎样能够产生级差地租之后,接着说:

\begin{quote}{“三等地一投入耕种,二等地立刻产生地租,而且同前面一样,这一地租是由两种土地生产力的差别决定的。同时,一等地的地租也会提高,因为一等地的地租必然总是高于二等地的地租,其差额等于这两种土地使用同量的资本和劳动所获得的产品的差额。每当人口的增长迫使一个国家耕种质量较坏的土地(但这决不是说,人口的每一次增长都会迫使一个国家耕种质量较坏的土地),以增加食物的供应时,[602]一切比较肥沃的土地的地租就会提高。”(第57页)}\end{quote}

这完全正确。

李嘉图接着举了一个例子。但是这个例子(暂且撇开后面要谈的)假定的是下降序列。但是,这不过是假定而已。李嘉图为了把这个假定悄悄地塞进来,他说:

\begin{quote}{“初到一个地方殖民,那里有着大量富饶而肥沃的土地……还没有被占有。”(第55页)}\end{quote}

但是,如果与殖民者的人数相对而言,那里有着“大量贫瘠而不肥沃的土地,还没有被占有”,情况还是一样。土地的富饶或肥沃不是不支付任何地租的前提,而土地的数量无限、没有被占有以及质量相同(不管这个质量在肥沃程度上可能是什么样),才是这种前提。因此,李嘉图在进一步阐述的时候,是这样来表述他的前提的:

\begin{quote}{“如果所有土地都具有同一特性,如果它们的数量无限、质量相同,使用土地就不能索取代价。”(第56页)}\end{quote}

他没有说而且不能说,如果土地“富饶而肥沃”,因为这类条件同这一规律是绝对无关的。如果土地不是富饶而肥沃,而是贫瘠而不肥沃,那末,每一个殖民者都不得不耕种全部土地中的较大部分,因此,随着人口的增长,即使在没有土地所有权存在的情况下,他们也会很快接近于这样的状况:土地同人口和资本相比,实际上不再是绰绰有余,事实上不再是无限的了。

的确,毫无疑问,殖民者自然不会去选择最贫瘠的土地,而是选择最肥沃的土地,就是说,对他们所支配的耕作手段来说是最肥沃的土地。但是这并不是他们进行选择的唯一条件。对他们来说,首先具有决定意义的是位置,是位于沿海、靠近大河等等。美洲西部等地区的土地可以说要多么肥沃就有多么肥沃,但是移民自然地定居在新英格兰、宾夕法尼亚、北卡罗来纳、弗吉尼亚等地,总之,是在东临大西洋的地区。如果说他们选择最肥沃的土地的话,他们只是选择这个地区的最肥沃的土地。这并不妨碍他们后来当人口增加、资本形成、交通工具发达和城市兴建使他们能够到较远地区利用比较肥沃的土地的时候,去耕种西部比较肥沃的土地。他们找的不是最肥沃的地区,倒是位置最好的地区,而在这个地区内,在其他位置条件相同的情况下,自然是找最肥沃的土地。但是,这当然不是要证明,人们是从比较肥沃的地区转到比较不肥沃的地区,而只是证明,在同一地区内,在位置相同的情况下,自然是先耕种比较肥沃的土地,其次才耕种比较不肥沃的土地。

但是,李嘉图在正确地把“大量富饶而肥沃的土地”这个说法改善成具有“同一特性、数量无限、质量相同”的土地这个说法以后,便去举例,接着就跳回到他最初的错误的前提:

\begin{quote}{“最肥沃的和位置最有利的土地首先耕种……”(第60页)}\end{quote}

李嘉图感觉到这个说法的弱点和错误,因而对“最肥沃的土地”又补充了一个新的条件:“位置最有利的”;这个条件是他开头论述时所没有的。显然,他本来应该说“在位置最有利的地区内的最肥沃的土地”,那样,就不致荒谬到把偶然找到的位置最有利于新来移民同宗主国、故乡的老亲友以及同外界保持联系的那些地区,当作殖民者还没有调查清楚也不可能一下子调查清楚的全部土地中“最肥沃的地区”了。

因此,李嘉图的从比较肥沃地区向比较不肥沃地区这个按下降序列推移的假定,完全是偷运进来的。只能这样说:因位置最有利而最早被耕种的地区不支付任何地租,直到在这个地区内从比较肥沃的土地推移到比较不肥沃的土地为止。如果现在转到比第一个地区更肥沃的第二个地区,那末,依照假定,这第二个地区的位置是比较不利的。因此,很可能这一地区的土地的比较肥沃还不足以抵销位置方面的比较不利,在这种情况下,第一个地区的土地将继续支付地租。但是,因为“位置”是一个随着经济发展历史地发生变化的条件,因为它随着交通工具的设置、新城市的兴建、人口的增长等等而必然不断改善,所以很明显,第二个地区生产出来的产品,将逐渐按照一个必然使第一个地区的(同一产品的)地租下降的价格投入市场,而第二个地区,随着它的位置的不利条件的消失,将逐渐作为比较肥沃的土地出现。

[603]因此,很明显:

在李嘉图自己对产生级差地租的必要条件作了正确的和一般的表述(“所有土地都具有同一特性……数量无限、质量相同”)的地方,不包括从比较肥沃的土地推移到比较不肥沃的土地这种情况;

这种情况,从历史上看,就他和亚·斯密所指的美国的殖民过程来说也是错误的,正因为如此,凯里才在这一点上提出了合理的反对意见;

李嘉图自己又用“最肥沃的和位置最有利的土地首先耕种”这个关于“位置”的补充说明,推翻了自己的理论;

李嘉图用一个例子来证明他随意作出的假定,而这个例子又假定了一个尚待证明的情况:即从较好的土地推移到越来越坏的土地;

最后,李嘉图{当然他已经打算用这一点来说明一般利润率下降的趋势}之所以作出这样的假定,是因为他否则就不能解释级差地租,尽管级差地租完全不取决于从I推移到II、III、IV还是从IV推移到III、II、I。

[(2)李嘉图关于地租不可能影响谷物价格的论点。绝对地租是农产品价格提高的原因]

在李嘉图的例子里假定有三个等级的土地,即一等地、二等地、三等地,在投资相等的情况下它们分别提供100夸特、90夸特、80夸特谷物的“纯产品”。“在新地区”一等地最先耕种。

\begin{quote}{“在新地区,肥沃的土地同人口对比起来绰绰有余,因而只需要耕种一等地。”(第57页)}\end{quote}

在这种情况下,“全部纯产品”属于“土地耕种者”,“成为他所预付的资本的利润”。(第57页)这里{我们不是谈种植园}虽然没有以任何资本主义生产为前提,却把这个“纯产品”立刻看作资本的利润,这也是不合适的。但是从“老地区”来的殖民者本人是可以这样看待自己的“纯产品”的。如果现在人口增加到必须耕种二等地的程度,那末一等地就会提供10夸特地租。这里自然要假定二等地和三等地“没有被占有”,而同人口和资本对比起来,它们实际上仍旧是“数量无限”。否则,事情就可能是另外一个样子。因此,在这个前提下,一等地将提供10夸特地租。

\begin{quote}{“因为二者必居其一:或者是,必定有两种农业资本利润率,或者是,必定有10夸特(或10夸特的价值)从一等地的产品中抽出来用于其他目的。不论是土地所有者还是其他任何人耕种一等地,这10夸特都同样形成地租;因为二等地的耕种者,不论他耕种一等地支付10夸特作为地租,还是继续耕种二等地不支付地租,他用他的资本得到的结果是相同的。”(第58页)}\end{quote}

实际上农业资本[在有两个不同等级的土地存在的情况下]有两种利润率,就是说,一等地提供10夸特超额利润(这种超额利润在这种情况下可以固定下来作为地租)。但是在同一生产领域内,对同一种类的资本,因而也对农业资本,不是有两种,而是有许多很不相同的利润率,这不仅是可能的,而且是必然的,——李嘉图自己在两页以后就谈到了这一点:

\begin{quote}{“最肥沃的和位置最有利的土地首先耕种,它的产品的交换价值,象其他一切商品的交换价值一样,是由生产产品并把产品运到市场这一整个过程中所必需的各种不同形式的劳动总量决定的。当质量较坏的土地投入耕种时,原产品的交换价值就会上涨,因为生产产品需要较多的劳动。一切商品,不论是工业品、矿产品还是土地产品,它们的交换价值始终不决定于在只是享有特殊生产便利的人才具备的最有利条件下足以把它们生产出来的较小量劳动,而决定于没有这样的便利,也就是在最不利条件下继续进行生产的人所必须花在它们生产上的较大量劳动;这里说的最不利条件,是指为了把需要的〈在原有价格下〉产品量生产出来而必须继续进行生产的那种最不利的条件。”(第60—61页)}\end{quote}

因此,在每个特殊生产部门不仅有两种利润率,而且有许多利润率,就是说,有许多对一般利润率的偏离。

李嘉图对例子的进一步说明,谈的是在同一土地上不同的[一个接着一个使用的]资本量的效果(第58—59页),这些说明没有必要在这里研究。需要指出的只是下面两个论点:

(1)“地租总是使用两个[604]等量的资本和劳动所取得的产品量之间的差额;”(第59页)

这就是说,只存在级差地租(根据不存在土地所有权的假定)。

(2)“不可能有两种利润率。”(第59页)

\begin{quote}{“不错,在最好的土地上,花费同以前一样多的劳动仍然能得到同以前一样多的产品,但是因为把新的劳动和新的资本用在比较不肥沃的土地上的人得到的产品较少,产品的价值就会提高。因此,尽管肥沃的土地同较坏的土地相比所提供的利益在任何情况下都不会消失,它只是从土地耕种者或消费者手里转移到土地所有者手里,但是,既然耕种较坏土地需要更多的劳动,既然只有耕种这种土地才能获得我们所必需的原产品的追加供给,这种产品的比较价值就会经常高于它过去的水平,并使这种产品能够交换更多的帽子、衣料、鞋子等等,因为生产这些东西不需要这种追加的劳动量。”(第62—63页)\end{quote}

\begin{quote}“因此,原产品的比较价值之所以提高,是因为在生产最后取得的那一部分产品时花费了较多的劳动,而不是因为向土地所有者支付了地租。谷物的价值决定于不支付地租的那一等土地或用不支付地租的那一笔资本生产谷物所花费的劳动量。不是因为支付地租谷物才贵,而是因为谷物贵了才支付地租;有人曾经公正地指出,即使土地所有者放弃全部地租,谷物价格也丝毫不会降低。这只能使某些租地农场主生活得象绅士一样,而不会减少在生产率最低的耕地上生产原产品所必需的劳动量。”(第63页)}\end{quote}

经过我上面的探讨之后,对于“谷物的价值决定于不支付地租的那一等土地……生产谷物所花费的劳动量”这个论点的错误,就没有必要再详细论述了。我指出过,最后一等[按质量]土地是支付地租还是不支付地租,是支付全部绝对地租还是只支付它的一部分,或者除了绝对地租以外还支付级差地租(在上升序列中),——这种情况部分地取决于发展方向是按上升序列还是按下降序列,而在任何情况下都取决于农业资本构成同非农业资本构成之比。我还指出过,如果已经假定绝对地租的存在是由于这种资本构成的差别,那末,上述种种情况就取决于市场情况,但是,正是李嘉图所提出的情况只有在两种条件下才能够出现(那时即使不能支付地租,也还可以支付租金):或者是,不论法律上还是事实上都没有土地所有权存在,或者是,较好的土地提供的追加供给只有当市场价值降低时才能在市场上找到销路。

但是,除此以外,在上面所引的那一段话中还有许多错误的和片面的东西。原产品的比较价值(这里无非是指市场价值)之所以可能上涨,除了李嘉图所指出的原因以外,还有别的情况:[第一,]如果原产品到现在为止都是低于它的价值或者低于它的费用价格出卖,——这种情况,总是发生在原产品的生产主要还是为维持土地耕种者的生活的那种社会状态(还有象在中世纪那样当城市产品保持着垄断价格的时候);第二,如果原产品——不同于其他按照费用价格出卖的商品——按照自己的价值出卖的条件还没有形成。

最后,关于如果土地所有者放弃了地租,租地农场主把地租装进了自己的腰包,谷物价格就将保持不变的说法,对级差地租来说是正确的。对绝对地租来说,那是错误的。说这里土地所有权不提高原产品的价格,是错误的。相反,在这种情况下会提高价格,因为土地所有权的干涉使得原产品按照它的价值出卖,而它的价值高于它的费用价格。假定,象前面那样,平均的非农业资本等于80c+20v,剩余价值是50%,利润率就是10%,产品的价值是110。而农业[605]资本等于60c+40v,产品的价值是120。原产品将按照这个价值出卖。如果象在殖民地那样,由于土地相对地绰绰有余,因而土地所有权不论法律上或事实上都不存在,那末,农产品就会按照115出卖。就是说,前一种资本和后一种资本(共200)的全部利润等于30,因而平均利润等于15。非农产品将按照115而不是按照110出卖;农产品将按照115而不是按照120出卖。因此,农产品同非农产品相比,相对价值下降1/12;但是两笔资本或总资本——农业资本和工业资本加在一起——的平均利润提高了50%,从10提高到15。[605]

※     ※     ※

[636]李嘉图谈到他自己对地租的理解时说:

\begin{quote}{“我始终认为地租是局部垄断的结果,它实际上决不调节价格(因此,决不是作为垄断起作用,也就是说,决不是垄断的结果。在李嘉图看来,只有地租不是落进租地农场主的腰包而是落进较好等级的土地所有者的腰包,才能是垄断的结果);地租倒是价格的结果。我认为,如果土地所有者放弃全部地租,土地上生产的商品也不会变得便宜一些,因为这些商品中总有一部分是在不支付地租或不能支付地租的土地上生产的,因为在那里,剩余产品只够支付资本的利润。”(李嘉图《原理》第332—333页)}\end{quote}

这里,“剩余产品”是产品中超过用于工资的部分的余额。李嘉图的论断只有在假定某一等级的土地不支付任何地租的情况下,在这种不支付地租的土地,或者不如说这种土地的产品调节市场价值的情况下才是正确的。相反,如果这种土地的产品不支付地租,是因为比较肥沃的土地调节市场价值,那末,不支付地租这个事实根本不能成为有利于李嘉图的论断的证据。

实际上,如果“土地所有者放弃”级差地租,那它就会归租地农场主所有。相反,放弃绝对地租却会降低农产品的价格,提高工业品的价格,其提高的程度相当于平均利润由于这一过程而增长的程度。[636]

※     ※     ※

[605]“地租的提高总是一个国家的财富增加以及对这个国家已增加的人口提供食物发生困难的结果。”(第65—66页)

这种论断的后一部分是错误的。

\begin{quote}{“在那些拥有最肥沃的土地,输入限制最少,由于农业改良而无需增加相应的劳动量就可以增加生产,因而地租增长得缓慢的国家里,财富增长得最快。”(第66—67页)}\end{quote}

如果地租率不变,只是投于农业的资本随着人口增长而增长,地租的绝对量也可能增加;如果I不支付地租,II只支付一部分绝对地租,但是由于较好土地比较肥沃而级差地租大大增长等等,地租的绝对量也可能增加。(见表\fnote{见第302—303页。——编者注})

[(3)斯密和李嘉图关于农产品的“自然价格”的见解]

\begin{quote}{“如果昂贵的谷物价格是地租的结果而不是地租的原因,价格就会随着地租的高低而成比例地变动,地租就会成为价格的构成部分。但是花费最多的劳动生产出来的谷物是谷物价格的调节者,地租不是也决不可能是这种谷物的价格的构成部分……原料成为大多数商品的组成部分,但是,这个原料的价值,同谷物价值一样,是由最后投入土地并且不支付任何地租的那一笔资本的生产率调节的;因此,地租不是商品价格的构成部分。”(第67页)}\end{quote}

这里,由于混淆了“自然价格”(因为这里所谈的是这种价格)和价值,引起了许多混乱。李嘉图从斯密那里因袭了这种混乱。在斯密那里,这种混乱相对地说还是可以理解的,因为他放弃了,并且仅仅是因为他放弃了他自己对于价值的正确解释。不论地租、利润或工资,都不是商品价值的构成部分。相反,在商品价值既定的情况下,这个价值所能分解成的各个部分,却或者属于积累劳动(不变资本)的范畴,或者属于工资、利润或地租的范畴。而关于“自然价格”,或者说,费用价格,斯密倒可以把它的构成部分当作既定的前提来谈。仅仅由于混淆了“自然价格”和价值,斯密才把这种看法搬到商品的价值上来。

除了原料和机器(简言之,不变资本——它对每个特殊生产领域的资本家来说,都是外来的既定的量,它在每个资本家那里,都以一定的价格加入生产)的价格之外,资本家在确定他的商品价格时还必须考虑到以下两件事。[第一]必须加上工资的价格,这个价格在他看来也是(在一定限度内)既定的。商品的“自然价格”不是指市场价格,而是指一个相当长的时期内的平均市场价格,或者说,市场价格所趋向的中心。因此,在这里,工资的价格总的说来是由劳动能力的价值决定的。至于[第二]利润率——“自然利润率”,那是由非农业生产中使用的全部资本所创造的全部商品的价值决定的。这就是全部商品的总价值超过商品中包含的不变资本的价值和工资价值的余额。这个全部资本所创造的全部剩余价值形成利润的绝对量。利润的这个绝对量同全部预付资本之比决定一般利润率。因此,这个一般利润率不仅对于单个资本家,而且对于每个特殊生产领域的资本来说,也都表现为外来的既定的东西。因此,他必须在产品所包含的预付于原料等等的价格[606]和工资的“自然价格”之上再加一个一般利润,比如说10%,以便这样——在他看来必然表现为这样——通过把各构成部分相加的办法,或者说,通过把它们结合起来的办法,得出商品的“自然价格”。在出卖商品的时候,它的自然价格是否得到支付,是支付得高些还是低些,这取决于当时的市场价格水平。费用价格不同于价值,加入费用价格的只有工资和利润,而地租只有在它已经加入预付于原料、机器等等的价格的限度内才加入费用价格。因此,在资本家看来,地租不是作为地租加入费用价格,对资本家来说,原料、机器的价格,简言之,不变资本的价格,一般说来是作为前提存在的一个整体。

地租不是作为构成部分加入费用价格。如果在特殊情况下农产品按照它的费用价格出卖,那就根本不存在地租。这时,土地所有权对资本来说在经济上是不存在的,也就是说,在按照费用价格出卖的那一级土地产品调节[根据李嘉图的理论]该领域的产品市场价值的情况下,土地所有权是不存在的。(D表I是另外一种情况\endnote{马克思在上一章中指出,D表内的等级I“会起完全消极的作用”(第328页):“决定市场的不是它,而是和它相对的IV、III、II”(第330页),它们向市场施加压力,使产品的市场价值维持在I的产品的个别费用价格的水平,即大大低于这一产品的个别价值的水平。——第360页。}。)

或者(绝对)地租是存在的。在这种情况下农产品高于它的费用价格出卖。农产品按照高于它的费用价格的价值出卖。这样,地租便加入产品的市场价值,或者说得更确切些,成为市场价值的一部分。但是租地农场主把地租看成是既定的,正如工业家把利润看成是既定的一样。地租决定于农产品的价值超过它的费用价格的余额。但是,租地农场主的计算同资本家完全一样:第一是预付的不变资本,第二是工资,第三是平均利润,最后,是在租地农场主看来同样为既定的地租。这对他来说也就是例如小麦的“自然价格”。他是否能得到这个价格,又取决于当时的市场情况。

如果按照实际情况把握住费用价格和价值的差别,那末地租就决不会作为构成部分加入费用价格,而且只有在我们谈到不同于商品价值的费用价格的时候才可以谈构成部分。(级差地租同超额利润一样决不加入[个别]费用价格,因为它始终只是市场费用价格\endnote{马克思说的市场费用价格(themarketcost-price)是指调节某一生产领域商品的市场价格的一般费用价格。参看本册第134—135页,那里市场费用价格称为“一般平均价格”,“市场平均价格”。——第361页。}超过个别费用价格的余额,或者说,只是市场价值超过个别价值的余额。)

因此,当李嘉图同亚·斯密相反,认为地租决不加入费用价格的时候,他在本质上是正确的。但是,从另一方面说,他又错了,因为他证明这一点的方法,不是把费用价格和价值区别开来,而是象亚·斯密一样把两者等同起来;因为不论地租、利润还是工资,都不是价值的构成部分,尽管价值可以分解为工资、利润和地租,而且有同样理由分解为所有这三者,如果这三者都存在的话。李嘉图的论断是这样的:地租不是农产品“自然价格”的构成部分,因为最坏的土地的产品价格等于这个产品的费用价格,等于这个产品的价值,它决定农产品的市场价值。因此,地租并不构成价值的任何部分,因为它不构成“自然价格”的任何部分,而这个“自然价格”等于价值。但是这恰恰是错误的。最坏的土地上种植出来的产品的价格等于它的费用价格,或者是因为这个产品低于它的价值出卖,就是说,决不象李嘉图所说的那样,是因为它按照它的价值出卖,或者是因为这种农产品属于价值和费用价格例外地完全一致的那样一类商品,那样一等商品。如果在某个特殊生产领域中,用一定资本如100货币单位创造的剩余价值,恰巧等于按平均计算应摊到总资本的同样的相应部分(例如100货币单位)的剩余价值,那就是这种情况。因此,这就造成了李嘉图的混乱。

至于亚·斯密,他既然把费用价格和价值等同起来,他从这个错误的前提出发,便有理由说地租同利润和工资一样是“自然价格的构成部分”。而他的前后矛盾却在于,他在进一步说明时,又认为地租不象工资和利润那样加入“自然价格”。他所以这样前后矛盾,是因为观察和正确的分析又使他承认,在非农产品的“自然价格”和农产品的市场价值的规定中存在着某种差别。关于这一点,在我们谈到斯密的地租理论时还要更详细地谈。

[(4)李嘉图对农业改良的看法。他不懂农业资本有机构成发生变化的经济后果]

[607]“我们已经看到:当把追加资本投入产量较少的土地成为必要时,每投入一笔追加资本,地租就提高一次。

(但是并不是每一笔追加资本都生产出较少量的产品。)

根据同样的原理可以得出结论:社会上的任何条件,如果能使我们无须在土地上使用同量资本,从而使最后使用的一笔资本具有较高的生产率,就都会使地租降低。”(第68页)

也就是说,它们会使绝对地租降低,但不一定使级差地租降低。(见B表)

这样一些条件,可以是由于人口减少而发生的“一个国家的资本的减少”,但是,也可以是农业劳动生产力的更高度的发展。

\begin{quote}{“但是,这样的结果也能在一个国家的财富和人口增加的情况下产生,如果随着这种增加农业也进行显著的改良,因为这种改良能够得到使耕种比较贫瘠的土地的必要性减少,或者在耕种比较肥沃的土地时花费同量资本的必要性减少的同样效果。”(第68—69页)}\end{quote}

(奇怪的是李嘉图在这里忘记了:那些改良也可以使比较贫瘠的土地的质量得到改良,并把比较贫瘠的土地变成比较富饶的土地,——这个观点在安德森那里占主导地位。)

李嘉图的下面这一论点是非常错误的:

\begin{quote}{“如果人口不增加,就不可能有对追加谷物量的需求。”(第69页)}\end{quote}

随着谷物价格下降,对其他原产品如蔬菜、肉类等的追加需求将会产生,而且可以用谷物酿制烧酒等等,这些姑且不论;李嘉图在这里假定,全部人口想消费多少谷物就消费多少谷物。这是错误的。

{“我们的消费量在1848、1849、1850年大大增长,说明我们以前吃不饱,说明价格由于供给不足而维持在高水平上。”(弗·威·纽曼《政治经济学讲演集》1851年伦敦版第158页)

同一个纽曼说:

\begin{quote}{“李嘉图关于地租不能提高价格的论证是根据这样一个假定,就是索取地租的权力在实际生活中决不可能使供给减少。但是为什么不可能呢?有着非常广阔的土地,这些土地,如果不索取地租,立刻就会投入耕种,可是它们人为地荒芜着,这或者是因为土地所有者把它们当作猎场出租可以得到更多的利益,或者是因为土地所有者宁肯让它们成为具有诗情画意的荒野,而不愿让人耕种来取得那一点点徒有其名的地租。”(第159页)}\end{quote}

如果认为,土地所有者从谷物生产中抽出自己的土地,便不能通过把它变成牧场或建筑地段,或者象苏格兰高地某些地区那样把它变成供狩猎用的人造森林,来取得地租,那是完全错误的。

李嘉图区别了农业上的两种改良。一种改良

\begin{quote}“提高土地的生产力……如采用更合理的轮作制或更好地选用肥料。这些改良确实能使我们从较少量的土地得到同量的产品。”(李嘉图《政治经济学和赋税原理》第70页)\end{quote}

照李嘉图的意见,在这种情况下地租一定下降。

\begin{quote}“例如,如果连续投入的各笔资本的产量分别是100、90、80、70夸特;当我在使用这四笔资本时,我的地租是60夸特,或者说只要我使用的还是这四笔资本,即使每一笔资本的产品有等量的增加,地租仍旧不变。”\end{quote}

(如果产品有不等量的增加,那末,尽管肥力提高了,地租也能提高。)

\todo{}

\begin{quote}“如果产量不是100、90、80、70夸特,而是增加到125、115、105和95夸特,那末地租仍旧是60夸特,或者说:\end{quote}

\todo{}

\begin{quote}但是当产品这样增加时,如果需求没有增加,就没有理由把这样多的资本投在土地上;有一笔将被抽出,因此,最后一笔资本将提供105夸特而不是95夸特,地租降到30夸特,或者说\end{quote}

\todo{}

且不说在价格下降时即使人口不增加,需求也可能增加(李嘉图自己在他所举的例子中就假定需求增加了5夸特);李嘉图之所以从不断向比较不肥沃的土地推移这个前提出发,也正是因为人口每年都在增加,就是说,消费谷物、吃面包的那部分人口在增加,而且这部分人口比整个人口增加得快,因为面包是大部分人口的主要食物。因此,就没有必要假定,需求不会随着[农业]资本的生产率一起增长,所以地租会下降。如果农业改良对于各级土地肥沃程度的差别的影响不一样,地租就可能提高。

此外,毫无疑问(B表和E表),在需求不变的条件下,肥力提高不仅可能把最坏的土地从市场排挤出去,甚至还可能迫使投在比较肥沃的土地上的一部分资本从谷物生产中抽出(B表)。在这种情况下,如果各级土地的产品增加的量相同,谷物地租就下降。

接着,李嘉图谈到第二种农业改良。

\todo{}

\begin{quote}“但是有些改良可能降低产品的相对价值而不降低谷物地租,尽管它们会降低货币地租。这种改良并不提高土地的生产力,但是使我们能够用较少的劳动获得土地产品。这些改良与其说是针对土地耕作方法本身,不如说是针对投在土地上的资本的构成。例如犁和脱粒机等农具的改良,在农业上使用马匹方面的节约,兽医知识的增进,都具有这样的性质。因此投到土地上的将是较少的资本,也就是较少的劳动,但是要获得同量产品,耕种的土地就不能减少。可是这种改良是否影响谷物地租,必然取决于使用各笔资本所得到的产品之间的差额是扩大、不变还是缩小。”\end{quote}

{李嘉图在谈到土地的自然肥力的时候也应该坚持这一点。向新的等级的土地推移,究竟是使级差地租减少、不变还是增加,取决于投在这些肥力不同的土地上的资本的产品之间的差额是扩大、不变还是缩小。}

\begin{quote}“如果有四笔资本50、60、70、80投在土地上,每笔都得到同样的结果,如果这种资本构成的某种改良使我能从每笔资本中减去 5,使它们分别成45、55、65和75,那末谷物地租将不变。但是,如果这种改良使我能够在生产率最低的那一笔资本上进行所有这些节约,那末谷物地租马上就会下降,因为生产率最高的资本和[609]生产率最低的资本之间的差额缩小了,而正是这个差额,构成了地租。”(第73—74页)\end{quote}

对于在李嘉图那里唯一存在的级差地租来说,这是正确的。

不过,李嘉图完全没有接触到真正的问题。为了解决这个问题,不在于每一夸特的价值下降,也不在于是否必须耕种和以前同一数量的土地,同一数量的同等级土地,而在于在农业中使用的直接劳动量的减少、增加或保持不变是否与不变资本的降价(按照假定,不变资本现在耗费较少的劳动)有关。简言之,是否在资本中发生有机的变动。

假定我们以A表为例(手稿第XI本第574页)\fnote{见第302—303页。——编者注},用一夸特小麦代替一吨煤。

这里假定,非农业资本的构成等于80c+20v,农业资本的构成等于60c+40v,两种资本的剩余价值率都等于50%。因而,农业资本的[绝对]地租,或者说,农业资本的产品的价值超过它的费用价格的余额等于10镑。那末,我们得到:

\todo{}

为了在纯粹的形式上研究这个问题,必须假定[农业中]不变资本(100镑)的降价对用于I、II、III三个等级的资本量发生同样的影响,因为不同的影响只涉及级差地租,而与我们现在研究的问题毫无关系。因此,我们假定,由于改良,同样的资本量,以前值100镑,现在只值90镑,就是说它的价值减少了1/10,即10%。现在要问,这些改良对农业资本的构成有什么影响?

如果花在工资上的资本[对不变资本]的比例不变,如果100镑分为60c+40v,那末90镑就分为54c+36v,在这种情况下,I级地上生产的60夸特的价值等于108镑。但是,如果降价表现为不变资本以前值60镑,现在只值54镑,而v(即花在工资上的资本)只值32+(2/5)镑而不是值36镑(再减少1/10),那末,支出的就不是100镑而是86+(2/5)镑。这一资本的构成是54c+[32+(2/5)]v。按100计算,资本构成是[62+(1/2)]c+[37+(1/2)]v。在这种情况下,I的60夸特的价值等于102+(3/5)镑。最后我们假定,虽然不变资本的价值减少了,花在工资上的资本在绝对量上仍然不变,因此它同不变资本相对来说增大了,结果支出的资本90镑分为50c+40v,资本构成按100计算,则等于[55+(5/9)]c+[44+(4/9)]v。

现在我们来看,在这三种情况下谷物地租和货币地租的情况怎样。在B的情况下,c和v的价值虽然减少,c和v的比例却保持不变。在C的情况下,[610]c的价值减少,但v的价值相对地减少得更多。在D的情况下,只有c的价值减少,而v的价值不变。

我们首先把前页的原表\fnote{见本册第366页。——编者注}列出[标以字母A,然后把它同说明上述农业资本有机组成部分价值变动的各种情况的B、C、D三个新表加以对比]。\fnote{在手稿中,下面按次序排列了A、B、C、D各表,这些表印在第368—369页上。C表和D表在手稿中有几栏空着。漏写的数字是编者补上的。最后一栏的标题(《资本构成和绝对地租率》)在手稿中原来没有,也是编者根据该栏的内容补上的。——编者注}

\todo{}

\todo{}

[611]从[第368—369页]所列的[总]表我们可以看到:

最初,在A的情况下,[农业资本的各有机组成部分之间的]比例是60c+40v;投入每级土地的资本都是100镑,地租表现为货币是70镑,表现为谷物是35夸特。

在B的情况下,不变资本降价,因而投入每级土地的资本只有90镑,但是可变资本也相应降价,结果比例不变。这里货币地租减少了,谷物地租不变;绝对地租\endnote{在第368—369页所列的总表里最后一栏以及第370—371页正文里的“绝对地租”,马克思是指绝对地租率。——第370页。}也不变。货币地租减少,是因为投入的资本减少。谷物地租不变,是因为在支出货币量较少的情况下每一货币单位生产的谷物多了,而各级之间的比例保持不变。

在C的情况下,不变资本降价;但是v减少得更多,结果不变资本相对地变贵了。绝对地租减少。谷物地租和货币地租都减少。货币地租减少,是因为资本总的说来大大减少了,而谷物地租减少,是因为绝对地租减少而各级间的差额保持不变,结果所有[各级的谷物地租]都同等地减少了。

在D表中,情况却完全相反。只有不变资本减少,而可变资本不变。李嘉图的前提就是这样。在这种情况下,货币地租由于资本减少,在绝对量上只是稍有减少(只减少1/3镑),但同花费的资本相对来说却有很大增加。相反,谷物地租的绝对量增加了。为什么呢?因为绝对地租从10%提高到[12+(2/9)]%,而这是由于v同c相对来说增加了。

于是,得出下表:

\todo{}

李嘉图继续说:

\begin{quote}“凡是使连续投入同一土地或新地的各笔资本所得产品的差额缩小的事物,都有降低地租的趋势;凡是扩大这种差额的,必然产生相反的结果,都有提高地租的趋势。”(第74页)\end{quote}

当资本从农业中抽出的时候,当坏地变得比较肥沃的时候,或者甚至当比较不肥沃的土地被排挤出市场的时候,这种差额就可能扩大。

{地主和资本家。1862年7月15日《晨星报》\endnote{《晨星报》(《The Morning Star》)是英国的一家日报,自由贸易派的机关报,1856年至1869年在伦敦出版。——第371页。},在一篇论谁有义务(自愿地或被迫地)援助由于棉花歉收和美国内战而处于困境的郎卡郡等地棉纺织工业工人的社论中写道:

\todo{}

\begin{quote}“这些人有合法权利要求用主要由他们自己的勤劳创造出来的财产来维持生活……有人说,那些靠棉纺织工业发了大财的人特别有义务慷慨救济。这毫无疑问是正确的……工商业界已经这样做了……但是,难道他们是靠棉纺织工业发了财的唯一阶级吗?当然不是。郎卡郡和柴郡北部的土地所有者们在这样创造出来的财富中分享了很大一份。而且土地所有者是占了特殊的便宜的,他们分享财富,可是对于创造这个财富的工业却毫无帮助,既不动手,也不动脑……为了[612]创立这个目前正在受到严重震荡的大工业,工厂主付出了他的资本和才干,经常提心吊胆,工厂的工人付出了他的技能、时间和体力劳动;但是,郎卡郡的土地所有者们付出了什么呢?什么也没有,真是一点也没有;但是他们从这个工业得到的实际利益却比另外两个阶级的哪个都多……肯定地说,这些大地主单单由于这个原因而增加的年收入是很大的,很可能至少增加两倍。\end{quote}

”资本家是工人的直接剥削者,他不仅是剩余劳动的直接占有者,而且是剩余劳动的直接创造者。但是,因为剩余劳动对产业资本家来说只有通过生产并在生产过程中才能实现,所以产业资本家本身就是这一生产职能的承担者,生产的领导者。相反,地主在土地所有权上(就绝对地租来说)和在土地等级的自然差别上(级差地租)却拥有一种特权,使他能把这种剩余劳动或剩余价值的一部分装进自己的腰包,尽管他在管理和创造这种剩余劳动或这种剩余价值方面毫无贡献。因此,在发生冲突时,资本家把地主看作纯粹是一个多余而有害的赘疣,看作资本主义生产的游手好闲的寄生虫,看作长在资本家身上的虱子。}

第三章《论矿山地租》。

这里又说:



关于绝对地租,它既不是“价值高昂”的结果,也不是“价值高昂”的原因,而是价值超过费用价格的结果。为矿山产品或土地产品而支付这一余额,从而形成绝对地租,这种情况不是这一余额的结果,因为这种余额在一系列生产部门中都存在,它并不加入这些部门的产品的价格;这种情况倒是土地所有权的结果。

至于级差地租,可以说它是“价值高昂”的结果,只要“价值高昂”是指那些比较富饶的等级的土地或矿山的产品市场价值超过它们的实际价值,或者说,个别价值的余额。

李嘉图所谓调节着最贫瘠的土地或矿山的产品价格的“交换价值”,无非是指费用价格,而他所谓的费用价格,无非是指预付加普通利润,他错误地把这个费用价格与实际价值等同起来,这从下面一段话里也可以看到:



可见,这里直截了当地说:地租等于农产品的价格(在这里也就是“交换价值”)超过它的费用价格的余额,也就是超过预付资本的价值加资本的普通(平均)利润的余额。因此,如果农产品的价值高于它的费用价格,那末,它就能够支付地租,而根本不管土地的差别如何,那时,最贫瘠的土地和最贫的矿山就可以同最富饶的一样支付同样的绝对地租。如果农产品的价值不高于它的费用价格,那末,地租只能来自比较肥沃的土地等等的产品的市场价值超过实际价值的余额。



这种适用于黄金和矿山的情况,也适用于谷物和土地。因此,如果继续开垦的总是同级的土地,如果在花费同量劳动的情况下它们总是提供同量产品,[613]那末一磅黄金或一夸特小麦的价值就保持不变,尽管其数量会随着需求而增加。这就是说,它们的地租(指地租额,不是指地租率)在产品价格没有任何变动的情况下也将增加。使用的资本将会更多,但是资本的生产率始终不变。这是地租的绝对量增长的重大原因之一,它同产品价格的提高毫无关系,因此,不同土地和不同矿山的产品所支付的地租不会发生相应的变动。

[(5)李嘉图对斯密的地租观点和马尔萨斯某些论点的批判]

李嘉图著作第二十四章《亚当·斯密的地租学说》。

这一章对于了解李嘉图和亚·斯密之间的差别是非常重要的。对于这个差别的更深入的阐述(关于亚·斯密),我们留待以后再说,因为考察了李嘉图的学说之后要专门考察斯密的学说。

李嘉图一开始就引了亚·斯密的一段话,照李嘉图的看法,斯密在这一段话里正确地确定了农产品的价格在什么时候提供地租,什么时候不提供地租。但是,后来斯密又认为,土地的某些产品,如食物,应当始终提供地租。

关于这个问题,李嘉图作了下面的评论,这个评论对他[李嘉图]是很重要的:



这些原理当然有很大“不同”。在土地所有权——实际上或法律上——不存在的地方,不会有绝对地租存在。土地所有权的恰当表现,是绝对地租,而不是级差地租。如果说,在有土地所有权存在和没有土地所有权存在的地方,都是同一些原理支配着地租,那就等于说,土地所有权的经济形式不取决于是否存在土地所有权。

其次,所谓“都有这样一种质量的土地,它提供的产品的价值只够补偿……资本并支付……普通利润”,这究竟是什么意思呢?如果同量劳动生产4夸特,同这个劳动生产2夸特对比起来,产品并不具有更大的价值,虽然一夸特的价值在一种情况下是另一种情况下的两倍。因此,产品是否提供地租,与产品的这个“价值”的量本身绝对无关。产品只有在它的价值高于它的费用价格时才能提供地租,而这个费用价格,是由其他一切产品的费用价格决定的,或者,换句话说,是由100货币单位的资本在每一生产部门中平均占有的无酬劳动量决定的。但是,产品的价值是否高于它的费用价格,完全不取决于它的价值的绝对量,而取决于用在它的生产上的资本的构成同用在非农业生产上的资本的平均构成的对比。



这里,李嘉图承认最坏的土地也能够提供地租。他怎么解释这一点呢?为了生产满足追加需求所必要的追加供给而投在最坏土地上的第二笔资本,[614]只有在谷物价格提高的情况下才能补偿费用价格。因此,第一笔资本现在将提供一个超过这个费用价格的余额,即提供地租。所以,情况是这样:在投入第二笔资本以前,因为市场价值高于费用价格,最坏土地上的第一笔资本就已提供地租。因此,问题只是在于,市场价值是否还必须高于最坏土地产品的价值,或者相反,是否产品的价值高于它的费用价格,而价格的提高只是使它能够按照它的价值出卖。

其次:为什么价格必须高到等于费用价格即预付资本加平均利润呢?这是由于不同生产部门的资本的竞争,由于资本从一个生产部门转到另一个生产部门,因此,是通过资本对资本的作用。但是资本通过什么作用才能迫使土地所有权让产品的价值降低到费用价格呢?从农业中抽出资本不能产生这种效果,除非同时使农产品的需求减少。抽出资本倒会产生相反的结果,会使农产品的市场价格涨到农产品的价值之上。把新的资本转到农业中去,同样不能产生这样的效果。因为资本之间的相互竞争恰恰使土地所有者能够要求每个资本家满足于“平均利润”,把价值超过提供这一利润的价格的余额付给土地所有者。

但是,可能提出这样的问题:如果土地所有权使人们有权让产品高于它的费用价格而按照它的价值出卖,那末,为什么土地所有权不能同样使人们有权让产品高于它的价值出卖,就是说,按照任何一个垄断价格出卖呢?在一个没有对外谷物贸易的小岛上,谷物、食品,同其他任何产品一样,无疑能够按照垄断价格出卖,就是说,按照这样一个价格出卖,这个价格只受需求情况的限制,就是说,只受有支付能力的需求的限制,而这个有支付能力的需求随着所提供的产品的价格水平而具有极为不同的大小和范围。

我们撇开这种例外情况不谈,——在欧洲各国根本谈不到这种情况;甚至在英国也有相当大一部分肥沃的土地被人为地从农业,总之从市场抽出去,以便提高其余部分的价值,——土地所有权只是在资本的竞争使商品价值规定发生变化的限度内才能影响和麻痹资本的作用即资本的竞争。价值转化为费用价格只是资本主义生产发展的后果和结果。本来(平均地说)商品是按其价值出卖的。在农业中,土地所有权的存在阻碍着对价值的偏离。

李嘉图说,如果一个租地农场主承租了一块土地,为期七年或十四年,他打算投下譬如10000镑资本,谷物价值(平均市场价值)使他能够补偿预付资本加平均利润加租约上规定的地租。因此,既然他“租用”土地,对他来说,平均市场价值即产品的价值是出发点;利润和地租只是这个价值所分解成的部分,而不是这些部分构成这个价值。既定的市场价格对资本家,就象作为前提的产品价值对理论以及对生产的内在联系一样。这就是李嘉图由此得出的结论。如果租地农场主追加1000镑,他所考虑的仅仅是,在市场价格既定的条件下,这1000镑是否能为他提供普通利润。因此,李嘉图大概是这样想的:费用价格是起决定作用的东西,作为调节要素加入这种费用价格的恰恰是利润,而不是地租。

首先,利润也不是作为构成要素加入费用价格的。按照假定,租地农场主把市场价格作为出发点,计算着在这一既定的市场价格下追加的1000镑是否能为他提供普通利润。因此,这一利润不是这一价格的原因,而是它的结果。但是,李嘉图进一步推论,这1000镑的投入本身,是通过计算那一价格是否能提供普通利润来决定的。因此,利润对于这1000镑的投入,对于生产价格,是决定的因素。

其次,李嘉图说,如果资本家发现这1000镑不能提供普通利润,那他就不会投入这笔资本。就不会有追加食物的生产。如果追加食物的生产是满足追加需求所必需的,那末,需求就必须把价格即市场价格提高到它能提供普通利润的水平。因此,利润不同于地租,在这里利润是作为构成要素加入的,这不是由于利润创造产品的价值,而是由于产品价格如果不提高到除补偿预付资本以外还支付普通利润率的高度,产品本身[615]就不会被创造出来。相反,在这种情况下,价格没有必要提高到足以支付地租的地步。因此,地租和利润之间在这里存在着一个本质差别,在某种意义上可以说,利润是价格的构成要素,地租则不是。(这显然也是亚·斯密的内心想法。)

就这种情况说,这是对的。

但是,为什么呢?

因为,在这种情况下,土地所有权不能作为土地所有权同资本对立,因此,按照假定,这里恰恰不存在形成地租,形成绝对地租的那种组合。用第二笔资本1000镑生产的追加谷物,是在市场价值不变的条件下,也就是在只有假定价格不变时才产生的追加需求的条件下生产出来的,它必须低于它的价值而按费用价格出卖。因此,这1000镑追加产品所处的情况,正象一块新的比较不肥沃的土地投入耕种时的情况一样,这种土地不决定市场价值,而只有在按现有的、原来的市场价值即按一个不由这个新的生产决定的价格来提供追加供给的条件下,才能提供自己的追加供给。在这种情况下,这块追加的土地是否提供地租,完全取决于它的相对肥力,而这正是由于它不决定市场价值。在原有土地上追加1000镑的情况完全一样。而李嘉图恰好由此作出了相反的结论:追加的土地或追加的那笔资本决定市场价值,因为它们的产品价格在市场价值既定、不由它们决定的条件下不提供地租,而只提供利润,不抵偿产品的价值,而只抵偿费用价格!这难道不是自相矛盾吗!

但是这里尽管不提供地租,产品还是在生产!的确是这样!在租地农场主已经租用的土地上,在他本人由于租约实际上成了土地所有者的期间,土地所有权对于他资本家来说,就不是作为独立的、起阻碍作用的要素存在了。因此,资本现在是不受阻碍地在这个要素中活动,对资本来说,能得到产品的费用价格也就满足了。同样,在租佃期满后,租地农场主自然将根据土地投资在多大范围内提供能按自己价值出卖的产品,也就是能提供地租的产品来调节地租。在市场价值既定的条件下不能提供超过费用价格的余额的那部分投资,在确定地租额时是不计算在内的,正如那种由于相对贫瘠而使市场价格仅仅支付产品的费用价格的土地,资本不为它支付地租或租约不规定支付地租一样。

实际情况不完全象李嘉图说的那样。如果租地农场主拥有闲置资本,或者在十四年租期的最初几年获得闲置资本,那末,他在这种情况下并不要求普通利润。只有在他借进追加资本的时候,他才要求普通利润。他究竟用这笔闲置资本来做什么呢?租进新的土地吗?在农业生产上,进行比较集约的投资比起以较大资本进行粗放耕作来,要合算得多。或者,如果在老地附近没有可供租种的土地,那末,租地农场主在经营两个分开的农场的情况下,他的监督管理活动,比加工工业中一个工厂主经营六个工厂还要分散得多。或者,租地农场主只好把货币存在银行里生息,投在公债券、铁路股票等等上面吗?这样,他一开始就要至少放弃普通利润的一半或三分之一。因此,如果他能把这些货币作为追加资本投到原来的农场中,收入虽然会低于平均利润,例如当平均利润等于12%的时候得到的利润为10%,但是,在利率为5%时,他仍然多赚100%。因此,把追加的1000镑[616]投在原来的农场上,对于租地农场主来说,仍然是一笔有利可图的生意。

因此,李嘉图把追加资本的投入[原来的土地]同追加资本用在新的土地上等同起来,是完全错误的。在前一种情况下,就是在资本主义生产中,产品也不一定要提供普通利润。它只是必须提供高于普通利率的利润,使租地农场主感到把自己的闲置资本用于生产虽然要操心和担风险,但还是比用作货币资本合算。

但是,正象已经指出的那样,李嘉图从这个论断得出的下述结论,是非常荒谬的:



李嘉图的例子恰恰证明了相反的情况:这最后一笔资本投入土地,是由市场价格调节的,这个市场价格不取决于这笔资本的投入,它在这笔资本投入以前早已存在,因此它只让最后这笔资本得到利润,而不是地租。说利润是资本主义生产的唯一调节者,那是完全正确的。因此,说生产如果完全受资本调节,就不存在绝对地租,那也是正确的。绝对地租恰恰是在生产条件使土地所有者有权限制资本对生产实行完全调节的地方产生的。

第二,李嘉图责备亚·斯密(第391页及以下各页)[仅仅]在煤矿方面发挥了正确的地租原理;李嘉图甚至说:

***亚·斯密觉得,土地所有者在一定情况下有权力对资本进行有效的抵抗,使人感到土地所有权的力量并因而要求绝对地租,而他在其他情况下就没有这种权力;但是,正是食物的生产确定地租规律,而资本在土地上作其他用途时产生的地租是由农业地租决定的。

李嘉图在反驳斯密时,尽可能地接近真正的地租原理。他说:



这里,李嘉图说出了正确的地租原理。如果最坏的土地支付地租,也就是说,如果支付的地租与土地的自然肥力的差别无关,即支付的是绝对地租,那末这种地租必定等于“产品价值超过资本支出加资本的普通利润的余额”,就是说,等于产品价值超过产品费用价格的余额。李嘉图认为这样的余额是不可能存在的,因为他违反他自己的原理,错误地接受了斯密教条,[617]即产品价值等于产品的费用价格。

此外,李嘉图还犯了一个错误。

级差地租自然决定于“相对肥力”。但是绝对地租同“自然肥力”毫无关系。

可是,另一方面,斯密正确地认为,最坏土地支付的实际地租可以取决于其他土地的绝对肥力和最坏土地的相对肥力,或者取决于最坏土地的绝对肥力和其他等级的土地的相对肥力。

问题在于,最坏土地支付的地租的实际数额,不是象李嘉图所想的那样,取决于这种土地自己的产品价值超过产品费用价格的余额,而是取决于产品市场价值超过产品费用价格的余额。但是,这是极不相同的两回事。如果最坏土地的产品本身决定市场价格,市场价值就等于它的实际价值,因而它的市场价值超过它的费用价格的余额就等于它自己的个别价值(它的实际价值)超过它的费用价格的余额。如果市场价格不取决于最坏土地的产品而由其他等级的土地决定,那末情况就不是如此。李嘉图是从下降序列这个假定出发的。他假定,最坏的土地最后耕种,而且(在假定的场合)只有当追加需求使得按照最后耕种的最坏土地的产品价值提供追加供给成为必要的时候,这种土地才会耕种。在这种情况下,最坏土地的产品价值调节市场价值。而在上升序列中,(即使按照李嘉图的看法)只有在较好等级的土地的追加供给按照原来市场价值仅仅等于追加需求的时候,最坏土地的产品价值才调节市场价值。如果追加供给大于这种需求,李嘉图总是假定,老地一定会停止耕种,结果只能是老地将提供比过去低的地租,或者完全不提供地租。在下降序列中,情况也是一样。如果追加供给只有按照原来的市场价值才能提供,那末,较坏的新地是否提供地租以及提供多少地租,就取决于这个市场价值超过这种土地产品的费用价格的程度的大小。在两种情况下[即无论在上升序列还是在下降序列中],它的地租都是由绝对肥力决定,而不是由相对肥力决定。较好土地的产品的市场价值究竟超过新地产品自己的实际个别价值多少,取决于新地的绝对肥力。

亚·斯密在这里正确地区别了土地和矿山,因为他在谈到矿山时,假定决不会向较坏的等级推移,而总是向较好的等级推移,它们提供的产品总是多于必要的追加供给。那时,最坏土地的地租就取决于它的绝对肥力。



亚·斯密的错误在于,他把最富饶的煤矿(或土地)支配市场这种特殊的市场状况当作一般的情况。但是,如果假定是这种情况,那末,斯密的论证(总的说来)就是正确的,而李嘉图的论证却是错误的。斯密假定,由于需求的情况和较高的富饶程度,最好的煤矿只有在把煤卖得低于竞争者的时候,只有在它们的产品价格低于原来的市场价值的时候,才能使它们的全部产品挤进市场。这样一来,对较次的煤矿来说,产品的价格也下降了。市场价格下降了。这种下降在任何情况下都会压低较次煤矿的地租,甚至可能使它完全消失。因为不论市场价值是否等于某一级土地(或煤矿)的产品的个别价值,地租总是等于市场价值超过产品的费用价格的余额。斯密没有注意到,只有在抽出部分资本和缩减产量成为必要时,利润才可能因此减少。如果在一定情况下由较好煤矿的产品调节的市场价格,降低到使最次煤矿的产品不能提供任何超过费用价格的余额,那末最次的煤矿就只能由其所有者自己开采。在这种市场价格条件下,没有一个资本家会向他支付地租。在这种情况下,他的土地所有权并不赋予他任何支配资本的权力,但是,土地所有权为他排除了其他资本家向土地投资时遇到的那种抵抗。对他来说,土地所有权是不存在的,因为他自己就是土地所有者。因此,他可以把自己的土地用于采煤,就象用于其他任何生产部门一样,也就是说,如果那个不是由他决定而是他发现时就已经确定的产品市场价格给他提供平均利润并补偿他的费用价格,他便可以把自己的土地用于采煤。

李嘉图竟由此得出结论说,斯密自相矛盾!根据原来的市场价格决定新矿在什么情况下可以由它的所有者自己开采,——就是说,新矿可以在土地所有权实际上消失的情况下开采,因为按照原来的市场价格,新矿能保证给企业主提供费用价格,——李嘉图就得出结论说,这个费用价格决定市场价格!但是,他又求助于下降序列,并且说,比较不富饶的煤矿只有在产品的市场价格涨到高于较好的煤矿的产品价值时,才会被开采;其实只要市场价格高于费用价格就行了,或者,对于由所有者自己开采的较次的煤矿来说,甚至只要市场价格能够补偿费用价格就行了。

此外,如果说李嘉图认为,“由于新法开采〈煤的〉产量增加了,价格就会下降,有些煤矿就会被放弃”,那就要知道,这仅仅取决于价格下降的程度和需求的情况。如果在价格这样下降的时候市场还能吸收全部产品,那末,只要市场价格的下降仍能使市场价值保持一个超过较贫的煤矿的费用价格的余额,次的煤矿就仍然会提供地租;如果市场价值只能补偿这一费用价格,即与费用价格一致,那末较贫的煤矿将由它们的所有者开采。但是,在这两种情况下,说最次的煤矿的费用价格调节市场价格,那都是荒谬的。当然,最次的煤矿的费用价格将决定它的产品的价格和起调节作用的市场价格之间的比例,因此决定这个煤矿是否[618]可以开采的问题。但是,在市场价格既定的条件下具有一定富饶程度的土地或煤矿是否可以开发的问题,同这块土地或这个煤矿的产品的费用价格是否调节市场价格,显然是没有关系的,它们根本不是一回事。如果在市场价值提高的情况下需要或可能有追加供给,那末,最坏的土地就调节市场价值,但是,这时候它也就提供绝对地租。这种情况恰恰同斯密所假定的情况相反。

第三,李嘉图(第395—396页)责备斯密,因为斯密认为原产品低廉,例如用马铃薯代替谷物,从而使工资下降,生产费用减少,就会使土地所有者从产品中得到更大的份额,同样也得到更多的产品数量。李嘉图的看法相反:



这一点肯定是错误的。地租所占的份额,因而,地租的相对量将会减少。用马铃薯作主要食物,就会降低劳动能力的价值,缩短必要劳动时间,延长剩余劳动时间,因而提高剩余价值率;因此,在其他条件不变的情况下,资本构成会发生变动,虽然使用的活劳动量仍然和以前一样,可变部分的价值同不变部分的价值相比却减少了。利润率将因此提高。在这种情况下,绝对地租下降,级差地租相应下降。(见第610页C表\fnote{见本册第368—369页。——编者注}。)这种原因将同样地影响农业资本和非农业资本。一般利润率将提高,因而地租将下降。

第二十八章(《论富裕国家和贫穷国家中黄金、谷物和劳动的比较价值》)。李嘉图写道:



第三十二章(《马尔萨斯先生的地租观点》)。李嘉图写道:

\begin{quote}“这种地租〈矿山地租〉同土地的地租一样,是它们产品价值高昂的结果,决不是价值高昂的原因。”(第76页)\end{quote}

\begin{quote}“被开采的最贫的矿山所产金属的交换价值,应当至少不仅足以供给开采金属并把它运到市场上的那些人的衣着、食物和其他生活必需品的费用,而且还足以给预付经营企业所必需的资本的人提供普通平均利润。资本从这种最贫的、不支付地租的矿山得到的报酬,将调节其他一切生产率较高的矿山的地租。假定,这种矿山提供资本的普通利润。其他矿山生产的超过这个普通利润的一切东西,必须作为地租支付给矿山所有者。”(第76—77页)\end{quote}

\begin{quote}“如果用等量劳动加等量固定资本总是可以从不支付地租的矿山获得等量的黄金……〈黄金的〉数量确实会随着需求而增加,但是它的价值不变。”(第79页)\end{quote}

\begin{quote}“我相信直到目前为止,在每一个国家,从最不开化的到最文明的,都有这样一种质量的土地,它提供的产品的价值只够补偿它所花费的资本并支付该国的平均普通利润。我们都知道,美国的情况就是这样,可是谁也没有说,决定地租的原理在美国和在欧洲不同。”(第389—390页)\end{quote}

\begin{quote}“但是,如果说英国的农业已发达到目前已经没有不提供地租的土地这一点是事实,那末,那里以前一定有过这样的土地这一点同样是事实;而且那里有没有这样的土地,对于这个问题是无关紧要的,因为如果大不列颠有任何投在土地上的资本只能补偿资本并为它提供普通利润,那末,不论这笔资本是投在老地或新地上,事情完全一样。如果一个租地农场主承租了一块土地,为期七年或十四年,他可能打算在土地上投下10000镑资本,因为他知道,按当时的谷物和原产品的价格,他能够补偿他所必须花费的资本,支付地租并获得普通利润率。他不会投资11000镑,除非投入这最后1000镑能够给他提供普通的资本利润。当他计算是否投入这一笔追加资本时,他所考虑的仅仅是原产品的价格够不够补偿他的费用和利润,因为他知道他无须支付追加地租。即使在租佃期满后,他的地租也不会提高;如果他的土地所有者因他投了1000镑追加资本而要提高地租,他就会把这笔资本抽回;因为,依照假定,他投入这笔资本只得到把资本用在其他任何地方也能得到的普通平均利润;因此,租地农场主不可能同意为这笔资本支付地租,除非原产品价格进一步提高,或者同样可以说,除非普通一般利润率下降。”(第390—391页)\end{quote}

\begin{quote}“如果亚·斯密的敏锐的头脑注意到这个事实,他就不会认为地租是原产品价格的一个构成部分,因为价格到处都是由不支付任何地租的最后一笔资本的收益调节的。”(第391页)\end{quote}

\begin{quote}“整个地租原理在这里得到了精辟而明确的说明,但是其中每一个字,不仅适用于煤矿,而且适用于土地;可是斯密断然认为,‘地面上的地产却是另外一种情况’。”(第392页)\end{quote}

\begin{quote}“‘它们的产品的价值和它们所提供的地租的价值〈亚·斯密说〉,都是同它们的绝对肥力,而不是同它们的相对肥力成比例。’”(第392页)\end{quote}

\begin{quote}“但是,假定没有不提供地租的土地。这样,最坏土地的地租额将同产品价值超过资本支出加资本的普通利润的余额成比例。同一原理将决定质量或位置比较好的土地的地租,因此,这些土地的地租,由于它们有较大的优越性,将高于比它们坏的土地的地租。对于第三种质量更好的土地,一直到最好的土地,都可以这样说。因此,正是土地的相对肥力决定作为地租支付的那部分产品,正象矿山的相对富饶程度决定作为矿山地租支付的那部分产品一样,这一点难道不是很清楚吗?”(第392—393页)\end{quote}

\begin{quote}“亚·斯密说,有一些煤矿只能由其所有者来开采,因为它们只能补偿开采的费用和所用资本的普通利润。在他说了这样的话以后,我们本来希望他会承认,正是这些煤矿调节一切煤矿的产品的价格。如果老矿不能提供煤的全部需要量,那末,煤的价格就会上涨,并且一直上涨到新的较贫的煤矿的所有者发现开采他的煤矿能获得资本的普通利润为止……因此,可以说,永远是最贫瘠的煤矿调节煤的价格。可是,亚·斯密的看法却不同。他认为,‘最富饶的煤矿也调节邻近其他一切煤矿的煤的价格。不论是这些煤矿的所有者还是从事开采煤矿的企业主都会发现,如果煤的卖价比邻近的煤矿低一些,煤矿所有者就能得到更多的地租,企业主就能得到更多的利润。邻近的煤矿很快就会被迫按同一价格出卖自己的煤,虽然他们这样做不那么容易,虽然这样做总会减少他们的地租和利润,有时还会使他们完全失去地租和利润。结果,一些煤矿完全被放弃,另外一些煤矿提供不了地租,而只能由它们的所有者开采’。如果煤的需求[617a]减少了,或者由于新法开采产量增加了,价格就会下降,有些煤矿就会被放弃。但是,在任何情况下,煤的价格都必须足以支付不担负地租的煤矿的开采费用和利润。因此,价格是由最贫的煤矿调节的。确实,亚·斯密自己在另一个地方也承认了这一点,因为他说:‘煤正象其他一切商品一样,在一个较长时间内能够出卖的最低价格,就是仅仅足以补偿使煤进入市场所使用的资本加上它的普通利润的价格。在土地所有者不能得到地租而必须或者亲自开采,或者干脆放弃的煤矿上,煤的价格一般必然接近于这一价格。’”(第393—395页)\end{quote}

\begin{quote}“这个附加额的任何一部分都不会归入地租,它必然全部归入利润……只要被耕种的土地质量相同,它们的相对肥力或其他优越条件又没有变动,地租对总产品的比例总是保持不变。”(第396页)\end{quote}

\begin{quote}“斯密博士贯穿于全书的一个错误,就是假定谷物的价值不变,虽然其他一切物品的价值可能提高,谷物的价值却永远不会提高。在他看来,谷物的价值始终不变,因为它能养活的人数始终相同。同样也可以说,衣料的价值始终不变,因为它能制成的上衣的数量始终相同。价值同物品用作衣食的能力又有什么相干呢?”(第449—450页)\end{quote}

\begin{quote}“……斯密博士……十分巧妙地论证了商品的市场价格归根结底是由商品的自然价格调节的这一理论。”(第451页)\end{quote}

\begin{quote}“……黄金的价值如果用谷物来表现,在两个不同的国家可能极不相同。我曾竭力证明黄金的价值在富裕的国家低,在贫穷的国家高。亚当·斯密的看法却不同:他认为,用谷物表现的黄金的价值在富裕的国家最高。”(第454页)\end{quote}

\begin{quote}“地租是价值的创造,但不是财富的创造。”\endnote{李嘉图把地租叫作“价值的创造”(《acreationofvalue》),是在这样的意义上说的:地租使土地所有者有可能支配整个社会产品的价值增殖额,在李嘉图看来,这种价值增殖额是由于这一或那一部分谷物生产的困难增加造成的,这种价值增殖额李嘉图叫做“名义上的”,因为社会实际财富并不因此而有丝毫增加。李嘉图在他的著作第三十二章中对马尔萨斯把地租看作是“一种纯收益和新创造的财富”的观点进行了批判,并且提出这样的论点:地租根本不会使整个社会的财富有任何增加,它只是“谷物和商品的一部分价值从原来的所有者手里转到土地所有者手里”。后来马克思在本册第627页上更完整地引用了李嘉图所著《原理》中的这一段。它成了马克思论“虚假的社会价值”的学说的出发点(见马克思《资本论》第3卷第39章)。并参看注30。——第387页。}(第485—486页)\end{quote}

\begin{quote}“当马尔萨斯先生谈到谷物的高昂价格时,他所指的显然不是一夸特或一蒲式耳谷物的价格,而是全部产品销售价格超过产品生产费用的余额,而他的‘生产费用’一词总是既包括工资,又包括利润。只要生产费用相同,每夸特值3镑10先令的谷物150夸特,就会比每夸特值4镑的谷物100夸特给土地所有者提供更多的地租。”(第487页)“不论土地的性质如何,高额地租必然取决于产品的高昂价格;但是,如果高昂的价格是既定的,地租的高度就必然同产品的丰富成比例,而不是同产品的匮乏成比例。”(第492页)\end{quote}

\begin{quote}“因为地租是谷物价格高昂的结果,所以地租的消失便是谷物价格低廉的结果。外国进口的谷物决不会同提供地租的国内生产的谷物竞争。价格下跌必然会打击土地所有者,直到他的地租全部被吞没;如果价格继续下降,它就连资本的普通利润也不能提供;在这种情况下,资本就会放弃土地去寻找别的用途,而以前在这一土地上生产的谷物,就会在这个时候,但不会早于这个时候,被进口谷物代替。由于地租消失,价值,用货币表现的价值,也会随之遭受损失,但是财富却会因此增长。原产品和其他产品的总量将增加;但是,由于生产起来比较容易,这些产品的数量虽然增加,它们的价值却会减少。”(第519页)}\end{quote}

\tchapternonum{[第十四章]亚·斯密的地租理论}

\tsectionnonum{[(1)斯密在地租问题提法上的矛盾]}

[619]在这里,我们不去探讨斯密的这种有趣说法:从主要植物性食物得到的地租,决定其余所有严格意义上的农业(畜牧业、林业、经济作物种植业)的地租,因为这些生产部门是可以互相转化的。在以大米为主要植物性食物的地方,斯密把大米除外,因为稻田不能转化为草地、麦田等等,反过来也是一样。

斯密正确地下定义说,地租是“为使用土地而支付的价格”([1802年法文版]第1卷第299页),在这里土地应理解为各种自然力本身,因而也包括水力等等。

同洛贝尔图斯的奇特的观念\endnote{马克思指洛贝尔图斯关于农产品生产费用中不包括原料价值的论点。见本册第8章第4节。——第388页。}相反,斯密在[第十一章]引言中就列举了农业资本的各个项目:“置备种子〈原料〉、支付劳动报酬、购买并维持牲畜和其他农具”。(同上)

但是,什么是这种“为使用土地而支付的价格”呢?

\begin{quote}{“产品或产品价格超过这一部分{即补偿预付资本“和普通利润”的部分}的余额,不论这个余额有多大,土地所有者都力图把它作为自己土地的地租攫为己有。”(同上,第300页)“这个余额始终可以看作自然地租。”(第300页)}\end{quote}

斯密反对把地租和投在土地上的资本的利息混淆起来:

\begin{quote}{“土地所有者甚至对于未经人力改良的土地也要求地租”(第300—301页),}\end{quote}

他补充说,就是这第二种地租形式\fnote{指经过改良的土地的地租。——编者注},也有一个特点,即用于改良土地的资本的利息,并不是土地所有者投下的资本的利息,而是租地农场主投下的资本的利息。

\begin{quote}{“他〈土地所有者〉有时对于完全不适于人们耕种的土地也要求地租。”(第301页)}\end{quote}

斯密非常明确地强调,土地所有权即作为所有者的土地所有者“要求地租”。斯密因此把地租看作土地所有权的单纯结果,认为地租是一种垄断价格,这是完全正确的,因为只是由于土地所有权的干预,产品才按照高于费用价格的价格出卖,按照自己的价值出卖。

\begin{quote}{“被看成是为使用土地而支付的价格的地租,自然是一种垄断价格。”(第302页)}\end{quote}

这确实是一种仅仅由于土地所有权的垄断才不得不支付的、并且在这方面作为垄断价格与工业品价格不同的价格。

从资本——而资本在生产中占统治地位——的观点看来,费用价格只要求产品除支付预付资本之外,还支付平均利润。在这种情况下,产品——不管是土地产品或别的什么产品——就能够

\begin{quote}{“进入市场”。“如果普通价格超过足够价格,它的余额自然会归入地租。如果它恰好是这个足够价格,商品虽然完全能够进入市场,但是不能给土地所有者提供地租。价格是否超过这个足够价格,这取决于需求。”(第1卷第302—303页)}\end{quote}

现在要问:为什么按照斯密的意见,地租以不同于工资和利润的方式加入价格?最初斯密正确地把价值分解为工资、利润和地租(撇开不变资本)。但是他立即走上了相反的道路,把价值和“自然价格”(即由竞争决定的商品的平均价格,或者说,费用价格)等同起来,认为后者是由工资、利润和地租构成的。

\begin{quote}{“这三部分看来直接地或最终地构成……全部价格。”(第1卷第101页)(第1篇第6章)“但是,就是在最发达的社会里,也总是有为数不多的一些商品,它们的价格只分解为两部分,即工资和资本的利润,还有为数更少的商品,它们的价格只由工资构成。例如,海鱼的价格中,就是一部分用于偿付渔人的劳动,另一部分用于支付投在渔业上的资本的利润。地租很少构成这个价格的一部分[620]……在苏格兰的一些地区,贫民以在海滨捡拾各种色彩的通称苏格兰玛瑙的小石子为业。雕石业主付给他们的小石子的价格,完全由他们的劳动报酬构成;地租和利润都不形成这种价格的任何部分。但是任何一个商品的全部价格,最终总是分解为这三部分中的一、两部分或所有三部分。”(第1卷第103—104页)(第1篇第6章)}\end{quote}

在上面的引文中(而且在整个论述“商品价格的构成部分”的第六章),价值分解为工资等等和价格由工资等等构成这类说法混杂在一起。(只是到第七章,才第一次谈到“自然价格”和“市场价格”。)

第一篇的第一、二、三章论述“分工”,第四章论述货币。在这几章以及以后几章,附带地提出了价值规定。第五章论述商品的实际价格和名义价格,论述价值向价格转化。第六章是《论商品价格的构成部分》。第七章论述自然价格和市场价格。然后,第八章论述工资。第九章论述资本利润。第十章论述各个使用劳动和资本的部门的工资和利润。最后,第十一章论述地租。

但是这里我们想首先要注意下面一点:按照刚刚引过的论点,有些商品的价格只由工资构成,另一些商品的价格只由工资和利润构成,最后,还有一些商品的价格由工资、利润和地租构成。因此:

\begin{quote}{“任何一个商品的全部价格……总是分解为这三部分的一、两部分或所有三部分。”}\end{quote}

根据这一点,也就没有理由说,地租是以不同于工资和利润的方式加入价格的;但是应该说,地租和利润是以不同于工资的方式加入价格的,因为后者是始终加入的,而地租和利润却不是始终加入的。这种差别是从哪里来的呢?

其次,斯密应当研究这样一个问题:只有工资加入的少数商品,能不能按照它们的价值出卖?或者说,那些收集苏格兰玛瑙的贫民,是否就不是雕石业主的雇佣工人?这些雕石业主对这种商品只付给他们普通工资,也就是说,对表面看来完全属于他们的整个工作日所付的报酬,只和其他部门(这里工人的工作日的一部分构成不属于他自己而属于资本家的利润)的工人得到的一样多。斯密应当要么承认这一点,要么相反地说明,在这种场合利润只是在表面上表现为同工资没有区别的东西。他自己说:

\begin{quote}{“当这三种不同的收入属于不同的人时,它们是很容易区分的;但是当它们属于同一个人时,它们往往会彼此混淆,至少在日常用语上是这样。”(第1卷第106页)(第1篇第6章)}\end{quote}

然而在斯密那里,问题是这样解决的:

如果一个独立劳动者(和上述苏格兰贫民一样)只使用劳动(而不必同时使用资本),一般说来,只使用自己的劳动和自然要素,价格在分解时就只归结为工资。如果劳动者还使用少量资本,他一个人就既取得工资又取得利润。最后,如果他使用自己的劳动、自己的资本和自己的土地所有权,他一个人就兼有土地所有者、租地农场主和工人这三重身分。

{斯密在问题提法上的全部荒谬之处,在第一篇第六章结尾中暴露出来了:

\begin{quote}{“因为在一个文明国家里,只有极少数商品的全部交换价值仅由劳动产生〈这里把劳动和工资等同起来了〉,绝大多数商品的交换价值中有大量地租和利润加入,所以,这个国家的劳动的年产品〈可见在这里,商品仍然等于劳动产品,尽管不是“这种产品的全部价值仅由劳动产生”〉所能购买和支配的劳动量,比这种产品的生产、加工和运到市场所必须使用的劳动量总要大得多。”(同上,第1卷第108—109页)}\end{quote}

结果,劳动产品不等于这种产品的价值。不如说(可以这样来理解斯密的意思),这个价值由于加上利润和地租而增大了。因此,劳动产品可以支配、购买更大的劳动量,也就是说,它能购买的劳动形式的价值比它本身包含的劳动量所构成的价值要大。这个论点如果这样表达就对了:

[621]斯密说:

\begin{quote}{“因为在一个文明国家里,只有极少数商品的全部交换价值仅由劳动产生,绝大多数商品的交换价值中有大量地租和利润加入,所以,这个国家的劳动的年产品所能购买和支配的劳动量,比这种产品的生产、加工和运到市场所必须使用的劳动量总要大得多。”}\end{quote}

根据他自己的观点,应当说:

\begin{quote}{“因为在一个文明国家里,只有极少数商品的全部交换价值在分解时只归结为工资,绝大多数商品的价值中有很大部分分解为地租和利润,所以,这个国家的劳动的年产品所能购买和支配的劳动量,比这种产品的生产、加工和运到市场所必须支付的(也就是使用的)劳动量总要大得多”}\end{quote}

(斯密在这里又回到了他的第二种价值概念;他在这一章谈到价值时说道:

\begin{quote}{“应当注意到,价格的各个不同构成部分的实际价值,是以每一构成部分所能购买或支配的劳动量来衡量的。劳动〈在这个意义上〉不仅衡量价格中归结为劳动〈应当说:工资〉的部分的价值,而且还衡量归结为地租的部分和归结为利润的部分的价值。”(第1卷第1篇第6章第100页)}\end{quote}

在第六章里,主要还是“价值分解为工资、利润和地租”。只是在论述自然价格和市场价格的第七章里,价格由这些构成要素构成的观点才占了上风。)

总之:劳动的年产品的交换价值,不仅由生产这种产品所使用的劳动的工资构成,而且由利润和地租构成。但是支配或者说购买这种劳动的,只是价值中归结为工资的部分。因此,如果把利润和地租的一部分用于支配或者说购买劳动,也就是,如果把这一部分变为工资,能够推动的劳动量就大得多。这样就得出如下的结果:劳动的年产品的交换价值分解为有酬劳动(工资)和无酬劳动(利润和地租)。如果把归结为无酬劳动的那部分价值的一些份额变为工资,那末,比起单单使用由工资构成的那部分价值来重新购买劳动,就可以买到更大量的劳动。}

现在回到我们的本题。

\begin{quote}{“如果一个独立劳动者拥有小量的资本,足以购买原料并维持生活直到能把他的产品运到市场,他就将同时获得一个给老板干活的帮工的工资以及这个老板从出卖帮工的劳动产品中取得的利润。不过这个劳动者的全部收入通常被称为利润,在这里,工资同利润混淆起来了。一个自己亲手种植自己果园的果园业者,一身兼有土地所有者、租地农场主和工人这三种不同的身分。所以,他的产品应该向他支付土地所有者的地租、租地农场主的利润和工人的工资。但是这一切通常都被看成他的劳动所得。在这里,地租和利润,又同工资混淆起来了。”(第1卷第1篇第6章第108页)}\end{quote}

斯密在这里实际上把所有的概念都混淆起来了。难道“这一切”不是“他的劳动所得”吗?相反,把这个果园业者的劳动产品,或者更确切地说,把这种产品的价值,一部分看成作为他的劳动报酬的工资,一部分看成使用的资本的利润,一部分看成应交给土地,或者更确切地说,应交给土地所有者的地租,这难道不是把资本主义生产关系(在资本主义生产关系下,随着劳动同劳动的客观条件分离,工人、资本家和土地所有者作为三种不同的身分而互相对立)转到这个果园业者身上吗?在资本主义生产范围内,对于上述各要素(实际上)并不相互分离的那种劳动关系来说,把这些要素假定为相互分离的,从而把这个果园业者当作一身兼任自己的[622]帮工和自己的土地所有者,那也是完全正确的。但是这里斯密已经明显地流露出一种庸俗的观念,似乎工资由劳动产生,而利润和地租则不依赖于工人的劳动,由当作独立源泉(不是当作占有别人劳动的源泉,而是当作财富本身的源泉)的资本和土地产生。在斯密那里,最深刻的见解和最荒谬的观念就这样奇怪地交错在一起,而这种荒谬的观念,是由从竞争现象抽象出来的庸俗意识形成的。

斯密首先把价值分解为工资、利润和地租,随后又反过来,用不依赖价值而决定的工资、利润和地租来构成价值。这样他就忘记了他原来正确阐述过的利润和地租的起源,因此他才能说:

\begin{quote}{“工资、利润和地租,是一切收入的三个原始源泉,也是一切交换价值的三个原始源泉。”(第1卷第105页)(第1篇第6章)}\end{quote}

按照他自己的论证,他本来应该说:

\begin{quote}{“商品的价值只由包含在这个商品里的劳动(劳动量)产生。这个价值分解为工资、利润和地租。工资、利润和地租,是雇佣工人、资本家和土地所有者分配由工人劳动创造的价值的原始形式。从这个意义上说,工资、利润和地租是一切收入的三个原始源泉,虽然这些所谓源泉没有一个参与创造价值。”}\end{quote}

从前面的各段引文中可以看到,斯密在论述“商品价格的构成部分”的第六章里,在只有劳动(直接劳动)加入生产时,把价格归结为工资;在不是一个独立劳动者,而是一个帮工受雇于资本家(即有资本存在)时,把价格分解为工资和利润;最后,在除了资本和劳动之外还有“土地”加入生产时,把价格分解为工资、利润和地租;但是在最后这种情况下,又预先假定土地已被占有,也就是说,除了工人和资本家还有土地所有者(虽然斯密指出,所有这三种独特的身分——或者其中两种——可以一人兼而有之)。

而在论述自然价格和市场价格的第七章里,地租完全和工资、利润一样,被说成是自然价格的构成部分(在土地加入生产时)。

下面的引文(第1篇第7章)就是证明:

\begin{quote}{“如果一种商品的价格恰好足够按自然率支付地租、工资和用于生产、加工商品并把它运到市场去的资本的利润,这种商品就是按照可以叫作它的自然价格的价格出卖。商品在这种情况下恰好按其所值出卖。”(第1卷第111页)(同时在这里,自然价格被说成和商品价值是等同的。)“单个商品的市场价格,决定于市场上现有的这种商品的数量,与愿意支付这种商品的自然价格,或者说,使商品进入市场所必须支付的地租、利润和工资的全部价值的人的需求之间的比例。”(第1卷第112页)“如果某种商品进入市场的数量不能满足对这种商品的实际需求,那些愿意支付使这种商品进入市场所必需的地租、工资和利润的全部价值的人,就不可能全都得到他们所需要的这种商品的数量……于是,市场价格就会或多或少地高于自然价格,高多少,取决于这种商品的不足额或竞争者的财富和奢欲所引起的竞争程度。”(第1卷第113页)“如果商品进入市场的数量超过了对它的实际需求,这个数量就不可能全部卖给那些愿意支付使这种商品进入市场所必需的地租、工资和利润的全部价值的人……于是,市场价格就会或多或少地低于自然价格,低落多少,取决于商品的超过额所引起的卖者之间的竞争程度,或者说,取决于卖者急于使商品脱手的程度。”(第1卷第114页)“如果进入市场的数量恰好足够满足实际需求,那末,市场价格当然就会和自然价格完全一致……不同卖者之间的竞争会强迫他们接受这个价格,但是不会强迫他们接受更低的价格。”(第1卷第114—115页)}\end{quote}

[623]斯密认为,如果地租由于市场状况而低于或高于它的自然率,土地所有者就会把自己的土地从生产中抽出,或者从一种商品(例如小麦)的生产转到另一种商品的生产(例如牧场)[或者相反,扩大自己商品的生产]。

\begin{quote}{“如果这个〈进入市场的〉数量在一段时间内超过了实际需求,商品价格的某一构成部分就必然会低于其自然率被支付。如果这是地租,土地所有者受利益的驱使,就会立即把自己的一部分土地从这种生产中抽出。”(第1卷第115页)“反之,如果进入市场的商品量在一段时间内不能满足实际需求,商品价格的某一构成部分就必然会提高到自己的自然率以上。如果这是地租,所有其余的土地所有者受利益的驱使,自然会利用更多的土地来生产这种商品。”(第1卷第116页)“商品市场价格的偶然的和暂时的波动,主要是影响商品价格中分解为工资和利润的部分。对于分解为地租的部分影响较小。”(第1卷第118—119页)“垄断价格是在一切情况下可能得到的最高价格。相反,自然价格,或者说,由自由竞争形成的价格,虽不是在一切场合,但在一段较长的时间内,却是可以接受的最低价格。”(第1卷第124页)“一种商品的市场价格虽然能够长期高于自然价格,却不大可能长期低于自然价格。不管这种价格的哪一部分是低于其自然率支付的,那些利益受影响的人,很快就会感到受了损失,并立即把若干土地,或若干劳动,或若干资本从这种行业中抽出,从而使这种商品进入市场的数量很快只够满足实际的需求。因此,这种商品的市场价格很快就会提高到它的自然价格的水平;至少在有完全自由的地方是这样。”(第1卷第125页)}\end{quote}

在第七章作了这样的论述之后,很难理解,斯密在第十一章(第一篇)《论地租》有什么根据断言,在被占有的土地加入生产的地方,地租却不是始终加入价格的;很难理解,他怎么能把地租加入价格的方式同利润、工资加入价格的方式区别开来,因为他在第六章和第七章已经把地租说成完全同利润、工资一样,是“自然价格”的构成部分。现在我们回过来谈第十一章(第一篇)。

我们看到,在第六章和第七章,斯密下定义说,地租是产品价格在支付资本家(租地农场主)的预付资本和平均利润之后剩下的余额。

在第十一章,斯密却完全颠倒过来。地租已不加入自然价格。或者,更确切地说,亚·斯密在这里求助于通常与自然价格不同的普通价格,虽然在第七章我们曾经听说,普通价格决不会长期低于自然价格,普通价格决不能长期低于自然价格的自然率支付自然价格的某一构成部分,更不能象他现在谈到地租时所说的那样,完全不支付。斯密也没有告诉我们,在产品不支付地租时,它是否低于自己的价值出卖,或者说,在它支付地租时,它是否高于自己的价值出卖。

以前,商品的自然价格是

\begin{quote}{“使商品进入市场所必须支付的地租、利润和工资的全部价值”。(第1卷第112页)}\end{quote}

现在我们听到:

\begin{quote}{“通常能够进入市场的只有那样一些土地产品,其普通价格足够补偿使产品进入市场所使用的资本,并提供普通利润。”(第302—303页)}\end{quote}

因此,普通价格并不是自然价格,而且要使商品进入市场,也无须支付它的自然价格。

[624]以前我们听说,如果普通价格(在第七章叫做市场价格)不够支付全部地租(地租等等的全部价值),土地就会从生产中抽出,直到市场价格提高到自然价格的水平并开始支付全部地租为止。现在,我们却听到:

\begin{quote}{“如果普通价格超过足够〈补偿资本和支付这笔资本的普通利润的〉价格,它的余额自然会归入地租。如果它恰好是这个足够价格,商品虽然完全能够进入市场,但是不能给土地所有者提供地租。价格是否超过这个足够价格,这取决于需求。”(第1卷第303页)(第1篇第11章)}\end{quote}

地租从自然价格的构成部分突然变成了超过足够价格的余额,有没有这个余额,取决于需求的状况。但是足够价格是使商品进入市场,也就是使商品生产出来所必需的价格,即商品的生产价格。因为供给商品所必要的,使商品生产出来并作为商品出现在市场上所必要的价格,当然是商品的生产价格,或者说,费用价格。这是商品存在的必要条件。另一方面,对某些土地产品的需求,必然总是使这些产品的普通价格提供一个超过生产价格的余额,也就是提供地租。而对另外一些土地产品来说,需求可以是这样,也可以不是这样。

\begin{quote}{“对有些土地产品的需求,必然总是使它们的卖价超过足够使它们进入市场的价格。还有一些土地产品,对它们的需求可能使它们的卖价超过足够价格,也可能使它们的卖价不超过这样的价格。前一类产品必然始终向土地所有者提供地租,后一类产品有时提供地租,有时则不提供,这要看情况如何而定。”(第1卷第303页)}\end{quote}

这样,我们在这里看到的不是自然价格,而是足够价格。普通价格又和这个足够价格不同。普通价格包括地租时,就超过足够价格。普通价格不包括地租时,就等于足够价格。而不包括地租,甚至是足够价格的特征。如果普通价格只能补偿资本,而不能支付平均利润,它就低于足够价格。因此,足够价格实际上就是李嘉图从亚·斯密学说中抽象出来的,并且从资本主义生产的观点来看确实出现的生产价格,或者说,费用价格,也就是说,这是一种除了支付资本家预付资本以外还能支付平均利润的价格,这是各个投资领域的资本家相互竞争所造成的平均价格。正是这种对竞争现象的抽象,使斯密把足够价格和他提出的自然价格对立起来,虽然斯密在对自然价格的说明中相反却宣称,只有支付自然价格各构成部分(地租、利润、工资)的普通价格,才是较长时期的足够价格。因为商品生产是由资本家支配的,所以足够价格就是对资本主义生产来说、从资本的观点来说是足够的价格,而这种对资本来说是足够的价格不是包括地租,而是相反,排除地租。

另一方面,这个足够价格对于某些土地产品来说却不是足够的。对于这些产品,普通价格必须高到能提供一个超过“足够价格”的余额,这样才能给土地所有者提供地租。对于另外一些土地产品,据说这又要看情况而定。矛盾在于:足够价格并不足够,足够使产品进入市场的价格并不足够使产品进入市场。而这个矛盾并没有使斯密感到不安。

虽然斯密没有稍微回过去看一看他在第五、六、七章中所发挥的论点,但他毕竟还是意识到他已经用这个“足够价格”推翻了他关于“自然价格”的全部学说(不过他认为这不是矛盾,而是他无意中碰到的新发现)。

\begin{quote}{“因此,应当注意〈斯密用这样一种非常天真的形式从一种主张转到了另一种截然相反的主张〉,地租是以与工资、利润不同的方式加入商品价格的构成。工资和利润的高低,是商品价格[625]高低的原因;地租的高低,是这一价格的结果。由于使商品进入市场所必须支付的工资、利润有高有低,商品的价格也就有高有低。不过商品有时提供高地租,有时提供低地租,有时完全不提供地租,是因为商品价格有高有低,有时大大超过足够支付这些工资和利润的价格,有时略为超过,有时完全不超过。”(第1卷第303—304页)}\end{quote}

我们首先来看结尾这句话。原来,只支付工资和利润的足够价格,费用价格,是排除地租的。如果产品的卖价大大超过足够价格,它就支付高地租。如果产品的卖价只是略为超过足够价格,它就支付低地租。如果产品正好按照足够价格出卖,它就不支付任何地租。如果产品的实际价格和支付利润、工资的足够价格相一致,它就不支付任何地租。地租始终是超过足够价格的余额。足够价格就其性质来说是排除地租的。这是李嘉图的理论。李嘉图从亚·斯密那里接受了足够价格,费用价格的观念;他避免了亚·斯密把足够价格同自然价格区别开来的那种前后矛盾的毛病,而是前后一贯地贯彻了足够价格的观念。斯密在犯了所有这些前后矛盾的毛病之后,还继续表现出前后矛盾,以致要求某些土地产品有一个超过足够价格的价格。但这种前后矛盾本身又是更正确的“observation”(“考察”)\fnote{《observation》一词既有“考察”的意思,又有“注意”的意思;马克思在这里暗指前面引用的斯密那一段话的开头“因此,应当注意”(《Ilfautdoncobserver》)。——编者注}的结果。

但是这一段话的开头的确天真得令人吃惊。在第七章,斯密先把价值分解为地租、利润和工资这一点颠倒为价值由地租、利润和工资的自然价格构成,然后说明,地租、利润和工资以同样的方式加入自然价格的构成。现在他说,地租以与利润、工资不同的方式加入“商品价格的构成”。但是地租以什么样的不同方式加入价格的构成呢?这就是以地租完全不加入价格的构成的方式。在这里我们第一次得到了对“足够价格”的真正解释。商品价格所以有贵贱高低,是因为工资和利润——它们的自然率——有高有低。如果这些高的或低的利润和工资得不到支付,商品就不能进入市场,就不能生产出来。而利润和工资构成商品的生产价格即费用价格;也就是说,它们实际上是商品的价值或价格的构成要素。相反,地租不加入费用价格,不加入生产价格。地租不是商品交换价值的构成要素。只有在商品的普通价格超过足够价格时,地租才得到支付。利润和工资,作为价格的构成要素,是价格的原因;相反,地租只是价格的结果,只是价格的后果。所以地租不象利润、工资那样作为要素加入价格的构成。用斯密的语言来说,这就是地租以与利润、工资不同的方式加入价格的构成。斯密似乎完全没有感觉到他推翻了他关于“自然价格”的全部学说。要知道,他所说的“自然价格”是什么呢?是市场价格所趋向的中心,是“足够价格”,——如果产品要较长时期地进入市场,进行生产,它是不能低于这个价格出卖的。

这样,地租现在是超过“自然价格”的余额,而以前是“自然价格”的构成要素;现在,它被说成是价格的后果,以前,它却被说成是价格的原因。

相反,斯密以下说法倒是没有什么矛盾的:对于某些土地产品来说,市场的情况始终使它们的普通价格必定超过它们的足够价格,换句话说,土地所有权有权力把价格抬到对资本家来说是足够的(如果他没有遇到对抗作用)水平以上。

[626]斯密就这样在第十一章把他在第五、六和七章所说的全部推翻之后,又心安理得地继续说,他现在言归本题,着手考察:(1)始终提供地租的土地产品;(2)有时提供地租有时又不提供地租的土地产品;最后,(3)在不同的社会发展时期,这两种产品相互之间的相对价值以及它们和工业品之间的相对价值所发生的变化。

\tsectionnonum{[(2)斯密关于对农产品的需求的特性的论点。斯密地租理论中的重农主义因素]}

第一节:论始终提供地租的土地产品。

斯密从人口论开始。食物据说始终创造对自己的需求。如果食物的数量增加了,食物消费者的人数也就增加。因此,这些商品的供给创造对它们的需求。

\begin{quote}{“因为象其他一切动物一样,人的繁殖自然同其生存资料相适应,所以对食物总是有或大或小的需求。食物总是能够购买或者说支配或多或少的劳动量,并且总是有人愿意为获得食物去做某种事情。”(第1卷第305页)(第1篇第11章)“但是{为什么?}土地几乎在任何情况下都能生产出较大量的食物,也就是说,除了以当时最优厚的条件维持使食物进入市场所必需的全部劳动外还有剩余。这个余额又始终超过那个足够补偿推动这种劳动的资本并提供利润的数量。所以这里始终有一些余额用来向土地所有者支付地租。”(同上,第305—306页)}\end{quote}

这完全是重农学派的口吻,而且既没有证明,也没有解释,为什么这种特殊商品的“价格”能提供超过“足够价格”的余额即地租。

斯密立即举出牧场和荒地作例子。接着是关于级差地租的话:

\begin{quote}{“不管土地的产品如何,地租随着土地的肥力而变动;不管土地的肥力怎样,地租随着土地的位置而变动。”(第1卷第306页)}\end{quote}

这里我们看到,地租和利润纯粹是产品中扣除以实物形式养活工人的那部分以后的余额。(这真正是重农学派的见解,这种见解实际上以下述情况为依据:在农业占统治地位的条件下,人几乎只靠农产品生活,而工业本身,即工场手工业,只作为农村的副业劳动,用来加工当地的自然产品。)

\begin{quote}{“这后一种产品\fnote{离市场远的偏僻地区的产品。——编者注},必须保证维持较大量的劳动,而作为租地农场主的利润和土地所有者的地租来源的余额就势必相应减少。”(第1卷第307页)}\end{quote}

因此,据说种植小麦提供的利润必定比牧场多:

\begin{quote}{“中等肥力的麦田,比同样面积的最好牧场,给人生产多得多的食物。”}\end{quote}

(可见,这里谈的不是价格,而是人的实物形式的食物的绝对量。)

\begin{quote}{“虽然耕种麦田要求较大量的劳动,但是补偿种子和维持全部劳动后剩下的余额还是大得多。”}\end{quote}

(虽然小麦耗费较大量的劳动,但是麦田所提供的食物在支付劳动报酬后剩下的余额,却超过畜牧场所提供的余额。这个余额所以有较大的价值,并不是因为小麦耗费了较大量的劳动,而是据说因为小麦的余额包含较多的食物。)

\begin{quote}{“因此,如果我们假定,一磅肉的价值从来不比一磅面包大,那末,这个较大的〈小麦〉余额〈因为同样的土地面积提供的小麦磅数比肉的磅数多〉就到处都代表一个较大的价值{因为已经假定,一磅面包(按价值)等于一磅肉,而在养活工人后,同样的土地面积剩下的面包的磅数大于肉的磅数},并给租地农场主的利润和土地所有者的地租构成一笔更大的基金。”(第1卷第308—309页)}\end{quote}

斯密用足够价格代替自然价格,并认定地租是超过足够价格的余额,随后他就忘记了这里一般谈的是价格,而从农业提供的食物数量和土地耕种者消费的食物数量的对比中得出了地租。

如果撇开这种重农学派的说明方法不谈,实际上斯密是假定:充当主要食物的农产品的价格,除了提供利润外,还提供地租。他从这个基础出发继续议论。随着耕作技术的发展,天然牧场的面积变得不能满足畜牧业的需要,不能满足对家畜肉类的需求。为了这个目的不得不利用耕地。[627]因此,肉的价格必须提高到不仅能够支付畜牧业所使用的劳动的报酬,而且能够支付

\begin{quote}{“这块土地用作耕地时能给租地农场主和土地所有者提供的利润和地租。在完全没有开垦的荒地上饲养的牲畜,和在耕种得很好的土地上饲养的牲畜,在同一市场上,就会按其重量和质量,以同样的价格出卖。这些荒地的所有者就利用这种情况,按照牲畜价格相应地提高自己土地的地租”。}\end{quote}

(这里,斯密正确地从市场价值超过个别价值的余额中得出了级差地租。但是在这种情况下市场价值提高,并不是因为从较好的土地推移到较坏的土地,而是因为从比较不肥沃的土地推移到比较肥沃的土地。)

\begin{quote}{“这样,随着土地耕作的进步,天然牧场的地租和利润,在一定程度上决定于已耕地的地租和利润,这种已耕地的地租和利润,又决定于麦田的地租和利润。”(第1卷第310—311页)“在没有……地方性优越条件的地方,小麦或充当人们主要植物性食物的任何其他产品所提供的地租和利润,自然要决定适宜于种植这种作物而现在却用作牧场的土地的地租和利润。利用人工牧场,种植芜菁、胡萝卜、大白菜等等,或者采用其他种种手段,使一定面积的土地饲养的牲畜多于天然牧场饲养的牲畜,这一切看来必定会促使农业发达的国家中自然比面包价格高的肉类价格有所降低。看来也已经产生了这样的结果”,等等。(第315页)}\end{quote}

斯密这样说明了畜牧业地租和农业地租的相互关系之后,继续写道:

\begin{quote}{“在一切大国中,大部分耕地都用来生产人的食物或牲畜的饲料。这些土地的地租和利润决定其他一切耕地的地租和利润。如果某种产品提供的地租和利润较少,种植这种产品的土地,就会立即用来种植小麦或改为牧场,如果某种产品提供的地租和利润较多,有一部分种植小麦或用作牧场的土地,就会立即用来种植这种产品。”(第1卷第318页)}\end{quote}

接着,斯密说到葡萄种植业、果园业、蔬菜业等等:

\begin{quote}{“为了使土地适于栽培这些作物,必须投下一笔较大的原始费用,或者逐年投下较大的耕作费用,虽然这些生产部门的地租和利润,往往大大超过从小麦或牧草得到的地租和利润,但是如果这种地租和利润只够弥补异常高昂的支出,它们实际上仍然是由这两种普通农产品的地租和利润决定的。”(第1卷第323—324页)}\end{quote}

在这以后,斯密又谈到殖民地的甘蔗和烟草的种植[然后说道:]

\begin{quote}{“就这样,生产人们食物的已耕地的地租,决定其他大部分耕地的地租。”(第1卷第331页)“在欧洲,小麦是直接充当人们食物的主要土地产品。所以,除一些特殊情况外,麦田的地租,在欧洲决定其他所有耕地的地租。”(第1卷第331—332页)}\end{quote}

然后,斯密又回到重农主义理论,并用了他自己的说法:食物本身为自己创造消费者。如果不种小麦而种植其他在最普通的土地上用同样的耕作方法能提供多得多的食物的作物,

\begin{quote}{“那末,土地所有者的地租,或者说,在支付劳动报酬并补偿租地农场主的资本及其普通利润后留给他的食物余额,也必然会多得多。不论这个国家维持劳动的普通开支如何,这个较大的食物余额总能够维持较大量的劳动,从而,使土地所有者能够购买,或者说,支配较大量的劳动”。(第1卷第332页)}\end{quote}

斯密举了大米作例子。

\begin{quote}{“在加罗林,也象在其他的英国殖民地一样,种植场主通常既是租地农场主,同时又是土地所有者,因此地租和利润就混在一起了。”(第1卷第333页)}\end{quote}

[628]但是稻田

\begin{quote}{“不适宜于种小麦,作牧场,或种葡萄,也不适宜于种其他任何对人有用的植物,而所有适宜于种这些作物的土地也不适宜于种稻子。所以,即使以大米为主要食物的国家,稻田的地租,也不能决定其他不能用来种稻子的耕地的地租”。(第1卷第334页)}\end{quote}

第二个例子是马铃薯(李嘉图对斯密这个观点的批判在前面引用过\fnote{见本册第386页。——编者注})。如果主要食物不是小麦,而是马铃薯,

\begin{quote}{“那末,同样面积的耕地就能养活多得多的人;因为工人通常都吃马铃薯,所以在补偿资本和养活所有种植马铃薯的工人外,就会有多得多的余额。而这个余额的更大部分也就会归土地所有者。人口会增加,地租将大大高于现在的水平”。(第1卷第335页)}\end{quote}

接着他对小麦面包、燕麦面包以及马铃薯作了一些进一步的说明,就结束了第十一章第一节。

我们看到,论述始终提供地租的土地产品的第一节可以概括如下:在假定主要植物性产品的地租已经存在的情况下,说明这种地租怎样调节畜牧业、葡萄种植业、果园业等等的地租。这里根本没有谈地租本身的性质,而只是泛泛地谈到——又是假定地租已经存在——土地的肥力和位置决定地租的高低。但是这里涉及的只是地租的差别,地租量的差别。然而,这里所考察的产品为什么始终提供地租呢?为什么它的普通价格始终超过它的足够价格呢?在这里斯密撇开价格,又陷入了重农主义。但是他到处都贯穿着这样一种思想:对农产品的需求始终这样大,是因为这种产品本身创造需求者,创造它自己的消费者。即使这样假定,也还是没有说明白,为什么需求一定超过供给,从而使价格高于足够价格。不过在这里又不知不觉地出现关于自然价格的影子,这个自然价格既包括利润和工资,也包括地租,而且,在供求相适应时就会得到支付:

\begin{quote}{“如果进入市场的数量恰好足够满足实际需求,那末,市场价格当然就会和自然价格完全一致……”(第1卷第114页)}\end{quote}

但是很典型的是,斯密在第十一章第一节没有一处谈到这个观点。而他在第十一章一开头恰恰是说,地租不作为价格的构成部分加入价格。矛盾太明显了。

\tsectionnonum{[(3)斯密关于各种土地产品的供求关系的论述。斯密对地租理论的结论]}

第二节:论有时提供地租有时又不提供地租的土地产品。

在这一节里才真正研究了地租的一般性质。

\begin{quote}{“人的食物看来是始终而且必然给土地所有者提供某些地租的唯一土地产品〈为什么是“始终”而且“必然”,却没有说明〉。其他各种产品,则根据不同的情况,有时能提供地租,有时又不能提供地租。”(第1卷第337页)“除了食物之外,衣服和住宅就是人类的两大需要。”(第1卷第338页)土地“在原始的未开垦的状态下”所能提供的衣服和住宅的材料,超过“它所能养活的”人数。由于“这些材料”同土地所能养活的人数相比,即同人口相比,“绰绰有余”,这些材料的“价格”就很低,或者根本没有“价格”。这些“材料”很大部分没有被利用或毫无用处,“而被利用的材料的价格,也被看成仅仅是为了使这些材料适于使用而必须花费的劳动和费用的等价物”。但是这个价格“不给土地所有者提供任何地租”。而土地在已开垦的状态下“所能养活的”人数即人口,超过土地提供的这些材料的数量,至少超过“人们希望得到和愿意支付的那些材料的数量”。于是这些材料就相对地显得“缺乏”,“而这就必然要提高它们的价值”。“对它们的需求量往往大于所能得到的数量。”那时人们对这些材料支付的价格,就会高于“使它们进入市场所必需的费用;因此,它们的价格始终能够给土地所有者提供一些地租”。(第1卷第338—339页)}\end{quote}

[629]可见,这里把地租解释成由于需求超过了供给——按照足够价格所能得到的供给。

\begin{quote}{最早的衣服材料是“大野兽”的毛皮。那些主要食用动物肉类的狩猎民族和游牧民族,“每一个人在获得食物的同时,也获得他穿不完的衣服材料”。没有对外贸易,其中大部分就被当作无用的东西丢掉。对外贸易提出对这些多余材料的需求,把它们的价格提到“高于把它们运到市场的费用。因此,这种价格也就能够给土地所有者提供一些地租……英格兰的羊毛由于在弗兰德找到了销路,使出产羊毛的土地的地租有了某些提高”。(第1卷第339—340页)}\end{quote}

这里是对外贸易提高了农业副产品的价格,以致生产这种产品的土地能够提供一些地租。

\begin{quote}{“建筑材料往往不能象衣服材料那样运到远地去,因而不那么容易成为对外贸易的对象。如果一国出产的建筑材料过多,即使在现代世界贸易的情况下,它们对土地所有者来说也往往没有任何价值。”例如采石场,在伦敦附近能够提供地租,但在苏格兰和威尔士的许多地方,却不能提供地租。建筑用的木材也是这样。“在人口稠密的文明国家”,木材可以提供地租,但在“北美的许多地区”,木材则就地烂掉。只要能把它弄走,土地所有者就很高兴了。“建筑材料既然这样充裕,所以被使用的那一部分材料的价值,就只不过相当于为了使这些材料适于使用而必须花费的劳动和费用。它不给土地所有者提供任何地租;只要有人愿意要,土地所有者通常都容许他们去采伐。但是当比较富裕的国家对这种材料有需求时,土地所有者有时也能从中得到地租。”(第1卷第340—341页)一国有多少人口,不是看“这个国家的产品能够保证多少人的衣服和住宅,而是看这个国家的产品能够保证多少人的食物。只要食物不缺,必要的衣服和住宅是不难找到的。但是常常衣服住宅有了,食物却依然很难找到。甚至在英吉利王国的一些地方,一个人只要用一天的劳动,就可以把一座当地所谓的房子建造起来”。在未开化的野蛮氏族中间,为了得到必需的衣服和住房,只要用全年劳动的百分之一就够了,其余百分之九十九,常常必须用来获得他们所需要的食物。“但是如果土地经过耕种和改良,一家人的劳动能为两家人提供食物,那末,社会半数人的劳动就足够为整个社会提供食物。”那时,另一半人就能满足人们的其他需要和嗜好。这些需要和嗜好的主要对象是衣服、住宅、家具,以及所谓奢侈品。食物的需要是有限的。上述这些需要是无限的。有多余食物的人,“总是愿意拿这部分多余的食物去交换”。“穷人为了获得食物”,就尽力满足富人的这些“嗜好”,并且还在这方面互相竞争。工人人数,随着食物数量的增加而增加,也就是说,随着农业的发展而增加。他们的“工作”允许实行“极细的分工”;所以他们加工的原料的数量就会比他们的人数增加快得多。“因此,对于任何一种材料,凡是人类的发明能把它用来改善或装饰住宅、衣服、车马、家具的,都产生了需求;对于地下蕴藏的化石和矿石,对于贵金属和宝石,也都产生了需求。这样一来,不仅食物是地租的原始源泉,而且,后来提供地租的其他任何土地产品,它的价值中的这个剩余部分,也都是土地的耕种和改良使生产食物的劳动生产力提高的结果。”(第1卷第342—345页)}\end{quote}

斯密这里所说的,也就是重农主义的真实的自然基础,即一切剩余价值(包括地租)的创造,都以农业的相对生产率为基础。剩余价值的最初的实在形式,就是农产品(食物)的剩余;剩余劳动的最初的实在形式,表现为一个人的劳动足以生产两个人的食物。除此以外,这一点对于分析以资本主义生产为前提的这个剩余价值的特殊形式地租,没有任何关系。

斯密继续说道:

\begin{quote}{“后来提供地租的其他土地产品〈食物除外〉并不是始终提供地租的。即使在土地耕种得最好的国家里,对于这些土地产品的需求,也不是始终大到足够使它们的价格除了支付产品生产和运到市场所花费的劳动以及补偿所用资本并提供普通利润以外还有一个余额。[630]需求是否那样大,取决于各种情况。”(第1卷第345页)}\end{quote}

这里又是说:地租的产生,是由于对土地产品的需求超过这些产品按足够价格——不包括地租,而只包括工资和利润——的供给。这不正是说,[在地租不存在的地方]土地产品按足够价格的供给很多,以致土地所有权不能对资本或劳动的平均化进行任何抵抗吗?这不也就是说,土地所有权即使在法律上存在,在这里实际上是不存在的,或者实际上不能作为土地所有权起作用吗?斯密的错误在于,不理解土地所有权按照超过足够价格的价格出卖产品,就是按照产品的价值出卖。斯密比李嘉图好的地方是他懂得,土地所有权能否显示自己的经济作用,取决于各种情况。因此,对他分析的这一部分,应当一步一步地跟着。他从煤矿开始,然后说到木材,然后又回到煤矿等等。因此我们先从他谈木材的地方开始:

\begin{quote}{木材价格,随着农业的状况而变动,变动的原因,同牲畜价格变动的原因一样。当农业还处于幼稚状态时,到处都是森林,这对土地所有者来说是一种障碍,谁愿意采伐,土地所有者是乐意让他采伐的。随着农业的进步,森林逐渐消失,一部分是由于耕地扩大,一部分是由于啃食树根和树苗的牲畜增加。“这些牲畜的头数,虽然不象完全是人类劳动产物的谷物数量增加得那样快,但是人的照料和保护,促进了牲畜的繁殖。”于是,森林逐渐稀少,它的价格也就提高。因此,森林能够提供很高的地租,以致耕地(或适于耕种的土地)也用来植树。大不列颠的情况就是这样。森林的地租决不能长久地超过耕地或牧场的地租。但是它可能达到同样的水平。(第1卷第347—349页)}\end{quote}

因此,森林的地租,就其性质来说,实际上和牧场的地租是一样的。它也属于这个范畴,虽然木材不能当作食物。经济范畴,不决定于产品的使用价值;这里它决定于这块土地能否变为耕地,或者相反。

煤矿。矿的富饶或贫瘠,一般说来,正如斯密正确指出的,取决于以同量劳动从不同矿开采出的矿产量是多还是少。矿的贫瘠,能把有利的位置抵销,以致这类矿完全不能开采。另一方面,位置不利也会把矿的富饶抵销,以致这些矿虽然天然富饶,却不宜于开采。特别是在没有好道路又没有航运的地方,往往是这样。(第1卷第346—347页)

有一些矿的产品仅够补偿足够价格。所以,它们能给企业主提供利润,但不能提供任何地租。因此,土地所有者不得不自己开采。这样,他可以获得“他所用资本的普通利润”。这一类煤矿在苏格兰很多。用其他方式来开采是不可能的:

\begin{quote}{“土地所有者不允许其他任何人不支付地租就去开采这些煤矿,而任何人又无法支付地租。”(第1卷第346页)}\end{quote}

斯密在这里正确地说明了,在土地已被占有的地方在什么情况下不支付地租。凡是一个人兼有土地所有者和企业主两种身分的地方往往是这样。以前斯密已经说过,殖民地的情况就是这样。租地农场主因为无法支付地租,也就不能在这里耕种土地。但是土地所有者耕种土地能得到利润,虽然土地不能给他提供任何地租。例如,美洲西部殖民地的情况就是这样,因为在这里始终有可能占有新地。土地本身不是一个阻碍的因素,自己耕种自己土地的土地所有者之间的竞争,在这里实际上是劳动者之间或资本家之间的竞争。至于煤矿或一般矿山,在前面假定的情况下,则不是这样。市场价值是由那些正好按照这个价值提供商品的矿决定的,它给比较不富饶或位置比较不利的矿提供较少的地租,或者完全不提供地租,而只补偿费用价格。在这里,这些矿只能由这样的人去开采,对他们说来,土地所有权的那种阻碍自由支配土地的作用是不存在的,因为他们一身兼有土地所有者和资本家两种身分;这种矿只有在土地所有权实际上不再作为与资本对立的独立因素的情况下才能开采。这种情况和殖民地的情况不同,在那里,土地所有者不能禁止任何人开垦新地。在这里他却能够这样做。他只允许他自己开矿。这并不能使他得到地租,却能使他排挤其他人,而从他自己投入矿山的资本中得到利润。

关于斯密所说的地租由最富饶的煤矿调节这一点,我在前面谈到李嘉图及其与斯密论战时\fnote{见本册第383—386页。——编者注}已经考察过了。这里只须指出下面一段话:

\begin{quote}{“煤炭正象其他一切商品一样,在一个较长时间内可以出卖的最低价格〈前面斯密说的是足够价格〉,就是仅仅足够补偿用于商品生产和运到市场的资本并提供普通利润的价格。”(第1卷第350页)}\end{quote}

我们看到,足够价格代替了自然价格。李嘉图把它们等同起来,是理所当然的。

[631]斯密断言:

\begin{quote}{煤矿的地租比农产品的地租少得多:农业中的地租通常达到总产品的1/3,对于煤矿来说,能占1/5就是很高的地租了,普通地租占1/10。金属矿受位置的影响较小,因为它们的产品比较容易运输,比较容易进入世界市场。所以它们的价值更多地取决于富饶程度,而不取决于位置,而煤矿的情形正好相反。彼此相隔最远的金属矿的产品可以互相竞争。“因此,世界上最富饶的矿山出产的普通金属的价格,尤其是贵金属的价格,必然会影响世界上其他各个矿山的同类金属的价格。”(第1卷第351—352页)“这样看来,因为每一个矿山的每一种金属价格,都在一定程度上由世界上当时开采的最富饶的矿山出产的该种金属的价格调节,所以绝大部分矿山所产的金属的价格,几乎都不超过补偿开采费用所需的价格,而且,很少能够向土地所有者提供高额地租。因此,对大多数矿山来说,地租只占金属价格的很小一部分,在贵金属价格中,它占的部分还要小得多。劳动和利润,在这两类金属的价格中都占大部分。”(第1卷第353—354页)}\end{quote}

斯密在这里正确地说明了C表的情况\fnote{见第302—303页。——编者注}。

谈到贵金属时,斯密又重复说明了他在谈到地租时用来代替自然价格的足够价格。在谈非农业生产的地方,他没有必要这样做,因为在这里,按照他最初的说明,足够价格和自然价格是一致的;这就是支付预付资本和平均利润的那个价格。

\begin{quote}{“贵金属在一个较长时间内可以出卖的最低价格……是由决定其他所有商品的最低普通价格的那些原则调节的。这种最低价格,是由贵金属从矿山进入市场通常所需要的资本决定的,也就是由这个劳动过程中通常所消费的食物、衣服、住宅决定的。这个价格必须至少足够补偿这笔资本并提供普通利润。”(第1卷第359页)}\end{quote}

说到宝石,斯密指出:

\begin{quote}{“对宝石的需求,完全是由它们的美丽引起的。它们只用于装饰。它们的美丽,又由于宝石稀少,或由于从矿山开采宝石困难和费用大,而显得更加珍贵。因此,在大多数情况下,工资和利润几乎占了宝石高昂价格的全部。地租在宝石价格中只占极小的份额,甚至常常不占任何份额。只有最富饶的矿山才能提供大一点的地租。”(第1卷第361页)}\end{quote}

这里只可能产生级差地租:

\begin{quote}{“因为全世界的贵金属和宝石的价格,是由最富饶的矿山的产品价格调节的,所以任何一个矿山能向土地所有者提供的地租,不是和该矿山的绝对富饶程度相适应,而是和它的所谓相对富饶程度,也就是它比其他同类矿山优越的程度相适应。如果发现了新矿山,它比波托西矿山优越的程度跟波托西矿山比欧洲矿山优越的程度一样,那末银的价值就会因此大大降低,以致连波托西矿山也不值得去开采了。”(第1卷第362页)}\end{quote}

比较不富饶的贵金属矿和宝石矿的产品,不提供任何地租,因为决定市场价值的始终是最富饶的矿山,并且不断有更富饶的新矿被开发,不断按上升序列运动。因而,比较不富饶的矿山的产品是低于它们的价值而仅仅按照它们的费用价格出卖的。

\begin{quote}{“如果一种产品的价值主要由它的稀少决定,那末产品的充裕必然使产品价值降低。”(第1卷第363页)}\end{quote}

在这以后,斯密又得出了多少是错误的结论。

\begin{quote}{“地面上的地产却是另外一种情况。它们的产品的价值和它们所提供的地租的价值,都是同它们的绝对肥力而不是同它们的相对肥力成比例。生产一定量食物、衣服材料和住房材料的土地,总能给一定的人数提供吃穿住;而且,不管土地所有者在这一产品中占有多大份额〈问题恰恰在于土地所有者在产品中能否占有份额和占有多大份额〉,这个份额[632]总是使他能相应地支配这些人的劳动和这种劳动所能给他提供的商品。”(第1卷第363—364页)“最贫瘠的土地的价值,并不因为邻近有最肥沃的土地而减少。相反,它的价值通常还因此而提高。肥沃土地养活的大量人口,为贫瘠土地的许多产品创造市场;这些产品决不能在靠贫瘠土地本身的产品养活的人们中间找到这种市场。”}\end{quote}

(但这只适用于这样的场合,即贫瘠土地所生产的和邻近肥沃土地所生产的不是同一种产品,贫瘠土地的产品不同比较肥沃的土地的产品竞争。就这样的场合来说,斯密是对的,这对理解各种土地产品的地租总额怎么会由于生产食物的土地肥沃而增加,确实是有重要意义的。)

\begin{quote}{“凡是能够使生产食物的土地的肥力提高的措施,不仅使经过改良的土地的价值增加{可以使这个价值减少,甚至化为乌有},而且还使其他许多土地的价值也同样增加,因为创造了对它们产品的新的需求〈或者,更确切地说,创造了对新产品的需求〉。”(第1卷第364页)}\end{quote}

斯密的上述一切仍然没有解释他假定对于生产食物的土地来说存在的绝对地租。斯密合理地指出,绝对地租对于其他土地例如矿山来说,也可能不存在,因为后者在数量上相对地说总是无限的(同需求相比),以致土地所有权在这里不可能对资本进行任何抵抗;土地所有权即使在法律上存在,在经济上也是不存在的。

(见第641页关于房租)\endnote{这一行是在马克思写完了论述斯密的房租观点的一段话(马克思手稿第641页)以后加进去的。——第415页。}[632]

\centerbox{※     ※     ※}

[641](见第632页)关于房租,亚·斯密说:

\begin{quote}{“全部房租中超过足够提供合理利润〈建造这所房屋的房主的利润〉的部分,自然归入地皮租;当土地所有者和房主是两个不同的人时,这一部分在大多数情况下全部付给前者。在远离大城市的乡村中的房屋,可以随意选择空地,只提供很少一点地皮租,或者说,不超过房屋所占土地用于农业时所能提供的地租。”(第5篇第2章)}\end{quote}

在房屋地皮租上,位置是级差地租的决定性因素,正象在农业地租上,土地肥力(和位置)是级差地租的决定性因素一样。

亚·斯密同重农学派一样,特别偏重农业和土地所有者,并持有重农主义观点,认为农业和土地所有者是最适当的课税对象。他说:

\begin{quote}{“地皮租和普通地租,都是土地所有者往往无须亲自操劳费心而唾手可得的一种收入。这种收入如有一部分拿去弥补国家开支,任何一种生产活动也不会因此受到损害。土地和社会劳动的年产品,即大部分居民的实际财富和收入,在实行这种税收以后,不会有任何变化。因此,地皮租和普通地租,大概是最宜于课以特别税的一种收入。”(第5篇第2章)\endnote{马克思在这里引用的斯密的两段话不是根据加尔涅的法译本(马克思在本册引用斯密的话都是根据这个译本),而是根据李嘉图《政治经济学和赋税原理》一书的英文本(第3版第14章)。——第415页。}}\end{quote}

与此相反,李嘉图(第230页)\endnote{马克思指李嘉图的《政治经济学和赋税原理》1821年伦敦第3版第230页。——第415页。}却提出了一种极其庸俗的反对意见。[641]

\tsectionnonum{[(4)斯密对于土地产品价格变动的分析]}

[632]第三节:论始终提供地租的产品的价值和有时提供地租有时又不提供地租的产品价值之间的比例的变动(第2卷第1篇第11章)。

\begin{quote}{“在土地自然肥沃但绝大部分完全没有耕种的国家,家畜、家禽、各种野生动物,耗费极少量的劳动就可得到,所以用它们也只能购买,或者说,支配极少量的劳动。”(第2卷第25页)}\end{quote}

斯密以多么奇特的方法把价值用劳动量来衡量同“劳动价格”,或者说,同某一商品所能支配的劳动量混淆起来,这从上面一段引文,特别是从下面一段引文可以看得很清楚。下面一段引文还表明,斯密竟然在有些地方把谷物看成价值尺度。

\begin{quote}{“在任何社会状态下,在任何社会文明发展阶段,谷物总是人类勤劳的产品。但是任何劳动部门的产品的平均量,总是多少准确地同平均消费相适应,即平均供给同平均需求相适应。此外,在不同的文明阶段,在同样的土地和同样的气候条件下,生产同量谷物,平均起来需要几乎同量的劳动,或者同样可以说,几乎同量劳动的价格。因为在耕作技术提高情况下劳动生产力的不断提高,或多或少会被作为农业主要工具的牲畜的价格的不断上涨所抵销。根据这一切,我们可以确信,在任何社会状态下,在任何文明阶段,同量谷物,和同量的其他任何土地原产品相比,都更恰当地成为同量劳动的代表或等价物。因此……在社会财富和文明的所有不同发展阶段,谷物同其他任何商品或其他任何一类商品比较起来,是更准确的价值尺度……此外,谷物或其他一般为人民喜爱的植物性食物,在每个文明国家,都是工人生存资料的主要部分……因此,劳动的货币价格取决于作为工人生存资料的谷物的平均货币价格的程度,远远超过取决于肉类或其他土地原产品的价格的程度。因此,金和银的实际价值,金和银所能购买或支配的实际劳动量,取决于它们所能购买或代表的谷物量的程度,远远超过取决于它们所能支配的肉类或其他土地原产品的数量的程度。”(第2卷第26—28页)}\end{quote}

在比较金和银的价值时,斯密又一次发挥了他的“足够价格”观点,并且[633]明确指出,足够价格不包括地租:

\begin{quote}{“我们说一种商品是贵还是贱,不仅要看它的普通价格是大是小,还要看这个普通价格超过使商品能在一个比较长的时间内进入市场的最低价格是多是少。这个最低价格,就是恰恰足够补偿商品进入市场所需资本并提供适中利润的价格。这个价格不给土地所有者提供什么东西;它的任何部分不由地租构成,它只分解为工资和利润。”(第2卷第81页)“金刚石和其他宝石的价格,和金的价格相比,大概更加接近于那个使它们能够进入市场的最低价格。”(第2卷第83页)}\end{quote}

按照斯密的说法,原产品有三类。(第2卷第89页)第一类产品的增加几乎不依赖或完全不依赖于人类劳动;第二类产品的数量能够根据需求而增加;第三类产品,其数量的增加,人类劳动“只能给以有限的或不经常的影响”。

第一类:鱼、罕见的鸟、各种野生动物、几乎所有的野鸟,特别是候鸟等等。随着财富和奢侈程度的增长,对于这类产品的需求则大大增加。

\begin{quote}{“因为这些商品的数量保持不变或几乎不变,而购买者间的竞争又日益扩大,所以它们的价格就可以涨到任何高度。”(第2卷第91页)第二类:“这包括在未耕地上天然成长的有用的植物和动物,它们十分丰富,以致只有很小的价值或全无价值,后来由于耕作的扩大,它们不得不让位于其他更加有利可图的产品。在长时期中,随着文明的不断进步,这类产品的数量不断减少,而同时对它们的需求却不断增加。这样,它们的实际价值,它们所能购买,或者说,支配的实际劳动量也越来越增加,最后将达到这样的高度,以致它们成为有利可图的产品,就象其他靠人的劳动在最肥沃的、耕种得最好的土地上获得的任何产品一样。如果这些产品的价值已经达到这样的高度,它也就不可能再提高了。否则人们马上就会用更多的土地和劳动来增加这些产品的数量。”(第2卷第94—95页)例如,家畜的情况就是这样。“在属于第二类原产品的各种商品中,家畜大概是随着文明的发展在价格上首先达到这种高度的商品。”(第2卷第96—97页)“如果说家畜最先达到这种价格{也就是使土地种植家畜饲料合算的价格},那末鹿肉大概就是最后达到这种价格的。尽管英国的鹿肉价格已经很高,但它还不够补偿鹿场的开支,这是有点养鹿经验的人都清楚的。”(第2卷第104页)“在每一个农场中,粮仓和牲口棚的残余食物可以用来饲养一定数量的家禽。因为家禽吃的东西,不利用也是浪费掉,所以饲养家禽只不过是废物利用;因为家禽几乎不花费租地农场主什么东西,所以他甚至能够以很低的价格出卖。”在供给充分时,家禽同家畜肉一样便宜。随着财富的增长,需求增大,家禽的价格就涨到牛肉或羊肉的价格以上,直到“专门耕种土地来饲养家禽变得有利可图”为止。法国的情况就是这样,等等。(第2卷第105—106页)猪和家禽一样,“最初饲养是为了废物利用”。猪吃的是糟粕。但是最后它的价格上涨到有必要专门耕种土地来饲养猪。(第2卷第108—109页)}\end{quote}

牛奶,牛奶场。(第2卷第110页及以下各页)(奶油、干酪;同上。)

按照斯密的意见,这些原产品价格的逐渐上涨,只是证明它们逐渐变成人类劳动产品,而在以前,它们几乎纯粹是自然产品。它们从自然产品变成劳动产品,只是耕作发展的结果,而耕作的发展,愈来愈缩小自然界的天然产品的范围。另一方面,在生产不大发达的条件下,上述产品很大部分都是低于自己的价值出卖的。它们一旦由副产品变成某一农业部门的独立产品,就立即按照自己的价值出卖(从而价格也上涨了)。

\begin{quote}{“显然,无论在哪个国家,如果靠人类劳动生产出来的任何土地产品的价格,没有高到足够补偿耕种土地和改良土地的费用,其土地是不可能得到充分的耕种和改良的。为了能够做到这点,每一单个产品的价格,第一,要足够支付好麦田的地租,因为其余大部分已耕地的地租正是由好麦田的地租决定的;第二,要足够支付租地农场主使用的劳动和费用,其标准不低于好麦田,换句话说,要足够补偿租地农场主所花费的资本并提供普通利润。每一单个产品价格的这种提高,显然应该[634]在种植这种产品的土地得到改良和耕种之前……现在,这些不同的原产品不仅比以前值较大量的银,而且值较大量的劳动和生存资料。因为要使这些产品进入市场必须花费较大量的劳动和生存资料,所以它们进入市场以后,就代表较大量的劳动和生存资料,或者说,值较大量的劳动和生存资料。”(第2卷第113—115页)}\end{quote}

在这里我们又看到,斯密只是在他把由可以买到的劳动量决定的价值跟由生产商品所必要的劳动量决定的价值混淆起来的时候,才使用前一种价值概念。

第三类:照斯密的说法,这一类包括这样一些原产品,

\begin{quote}{“对于这类产品数量的增加,人类劳动只能给以有限的或不经常的影响”。(第2卷第115页)}\end{quote}

毛和皮的数量受现有大小家畜头数的限制。但是这些最早的副产品,在家畜本身还没有广大市场的时候,就已经有了广大市场。家畜肉几乎总是限于国内市场。可是毛和生皮,甚至在文明初期,就已经多半有了国外市场。它们非常便于运输,并且是许多工业品的原料。因此,当本国工业还不需要它们时,工业比较发达的国家就已经可以充当它们的市场了。

\begin{quote}{“在耕作不发达因而人口稀少的国家,毛和皮的价格在整头动物价格中所占的比例,比在耕作较发达、人口较稠密因而对肉类有较大需求的国家,要大得多。”脂油的情形也是这样。随着工业的发展和人口的增长,家畜价格的提高对肉价的影响比对毛皮价格的影响大。因为随着一国工业和人口的增长,肉类市场不断扩大,而上述副产品的市场原先就已经超出国界了。但是随着本国工业的发展,毛皮等的价格也总会有某些提高。(第2卷第115—119页)鱼(第2卷第129—130页)。如果对鱼的需求增加,为了满足这一需求就要花更大量的劳动。“鱼通常要到较远的地方去捕,要用比较大的渔船和各种比较贵的捕鱼设备。”对鱼的需求,“如果不花费”比“过去使鱼上市所必需的”更多的“劳动,就不可能得到满足”。“因此,这种商品的实际价格,必然随着文明的发展而自然提高。”(第2卷第130页)}\end{quote}

可见,在这里,斯密是用生产商品所必要的劳动量来决定实际价格。

按照斯密的说法,随着文明的发展,植物性产品(小麦等)的实际价格必然下降:

\begin{quote}{“农业改良的推广和耕地的扩大,必然使各种动物性食物的价格同小麦价格相比有所提高,另一方面,我认为,它同样必然使各种植物性食物的价格有所降低。它使动物性食物的价格提高,是因为提供动物性食物的很大部分土地,改成适于生产小麦以后,现在必须向土地所有者和租地农场主提供麦田的地租和利润。它使植物性食物的价格降低,是因为它通过土地肥力的增加,使这种食物充裕起来。农业的改良,还会引进许多新的植物性食物品种,它们比小麦需要的土地少,而花费的劳动也不更多,所以,它们能以比小麦低得多的价格进入市场。如马铃薯、玉米就属于这一类……此外,在农业发展水平低的情况下,许多植物性食物,只限于在菜园中栽培,而且只使用锄;随着耕作技术的发展,这些植物性食物也开始在大田里种植,并且使用了犁。如芜菁、胡萝卜、大白菜等就属于这一类。”(第2卷第11章第145—146页)}\end{quote}

斯密看到,凡是在“原料的实际价格没有提高或提高得不多”(第2卷第149页)的地方,工业品的价格一般都降低了。

另一方面,斯密断言,劳动的实际价格即工资,随着生产的发展提高了。因此,他还认为,商品的价格不一定因为工资,或者说,劳动价格的提高而提高,虽然在他看来,工资也是“自然价格的构成部分”,甚至是“足够价格”的“构成部分”,或者换句话说,是“商品进入市场所需的最低价格”的“构成部分”。斯密怎样解释这一点呢?是因为利润降低了吗?不是(虽然他也认为,一般利润率会随着文明的发展而下降)。是因为地租降低了吗?也不是。他说:

\begin{quote}{“机器的改进,[635]技能的提高,劳动分工和劳动分配的更加合理(这一切是一个国家发展的必然结果),都使生产某种产品所需的劳动量大大减少;虽然由于社会繁荣,劳动的实际价格必然大大提高,但是生产每一物品所需的劳动量的大大减少,通常会把劳动价格所能出现的很大的提高抵销而有余。”(第2卷第148页)}\end{quote}

这样,商品价值降低,是因为生产商品所必要的劳动量减少,并且,尽管劳动的实际价格提高了,商品价值还是会降低。如果这里劳动的实际价格就是指它的价值,那末在商品价格因商品价值降低而降低时,利润必然会同时降低。如果劳动的实际价格是指工人得到的生活资料总额,那末,斯密的论点即使在利润提高的情况下也是正确的。

凡是斯密作出实际分析的地方,他都采用了正确的价值规定;这一点从这一章结尾他研究毛织品为什么在十六世纪[比十八世纪]贵的问题的地方也可以看到:

\begin{quote}{“那时,为了制造这些商品供应市场,要花费多得多的劳动量,因此商品上市以后,卖得或换得的价格必定是一个多得多的劳动量。”(第2卷第156页)}\end{quote}

这里的错误只在“价格”一词。

\tsectionnonum{[(5)斯密关于地租变动的观点和他对各社会阶级利益的评价]}

这一章的结束语。亚·斯密是以下面的评论来结束论地租这一章的:

\begin{quote}{“社会状况的任何改善,都有直接或间接提高实际地租的趋势。”“农业改良的推广和耕地的扩大可以直接提高实际地租。土地所有者得到的产品份额,必然随着这个产品数量的增加而增加。”(第2卷第157—158页)“原产品实际价格的提高,最初是农业改良的推广和耕地的扩大的结果,后来又成为农业改良的进一步推广和耕地进一步扩大的原因”。这些产品的实际价格例如家畜价格的提高,第一,会提高土地所有者所获得的份额的实际价值;第二,也会提高这个份额的相对量;因为“这种产品的实际价格提高以后,生产它所需的劳动并不比以前多。这样,产品中一个比过去小的份额,就足够补偿推动劳动的资本并提供普通利润。而产品中一个比过去大的份额就因此归土地所有者所得”。(第2卷第158—159页)}\end{quote}

李嘉图也完全用同样的方法来说明比较肥沃的土地的谷物价格上涨时地租份额的增大。但是这种涨价并不是由农业改良引起的,因此,李嘉图得出了和斯密相反的结论。

斯密随后还指出,工业劳动生产力的任何发展,都会给土地所有者带来好处:

\begin{quote}{“凡是降低后者\fnote{工业品。——编者注}实际价格的措施,都能提高前者\fnote{农产品。——编者注}的实际价格。”其次,随着社会实际财富的增加,人口也就增加,随着人口的增加,对农产品的需求也就增加,从而投在农业上的资本也增加,而“地租也就随着产品的增加而增加”。反之,凡是阻碍社会财富增长的相反情况,都会使地租下降,从而使土地所有者的实际财富减少。(第2卷第159—160页)}\end{quote}

斯密由此作出结论说,地主(土地所有者)的利益,始终同“整个社会的利益”一致。在斯密看来,工人的利益,也同整个社会的利益一致(第2卷第161—162页)。但是斯密毕竟诚实地指出了如下的区别:

\begin{quote}{“土地所有者阶级也许能够由于社会的繁荣而比他们〈工人〉得到更大的利益,但是没有一个阶级象工人阶级那样由于社会衰落而遭受那样大的苦难。”(第2卷第162页)}\end{quote}

相反,资本家(工业家和商人)的利益却同“整个社会的利益”不一致(第2卷第163页)。

\begin{quote}{“在任何一个商业或工业部门投资的实业家的利益,总是在某些方面和社会利益不同,有时甚至相反。”(第2卷第164—165页)“……[这是]这样一些人的阶级,这些人的利益[636]始终不会和社会的利益完全一致,通常他们的利益在于欺骗社会,甚至压迫社会,而他们因此也常常既欺骗社会又压迫社会。”(第2卷第165页)\endnote{手稿中接着有几段话,是分析李嘉图关于自己对地租的理解的论述的。这几段和上文用一条线隔开,它们是对考察李嘉图地租理论各章的补充;按其内容属于第十三章,所以本版放在第十三章(见第357—358页)。手稿中这几段话之后,有一个对李嘉图费用价格理论的分析的补充,放在圆括号内,马克思所作的分析在第十章,所以这个补充本版也移至第十章(见第239—240页)。——第422页。}[636]}\end{quote}

\tchapternonum{[第十五章]李嘉图的剩余价值理论}

\tsectionnonum{[A.李嘉图关于剩余价值的观点与他对利润和地租的见解的联系]}

\tsubsectionnonum{[(1)李嘉图把剩余价值规律同利润规律混淆起来]}

[636]李嘉图在任何地方都没有离开剩余价值的特殊形式——利润(利息)和地租——来单独考察剩余价值。因此,他对具有如此重要意义的资本有机构成的论述,只限于说明从亚·斯密(特别是从重农学派)那里传下来的,由流通过程产生的资本有机构成的差别(固定资本和流动资本);而生产过程本身内部的资本有机构成的差别,李嘉图在任何地方都没有涉及,或者根本就不知道。就是由于这个缘故,他把价值和费用价格混淆起来了,提出了错误的地租理论,得出了关于利润率提高和降低原因的错误规律等等。

只有在预付资本和直接花费在工资上的资本是等同的情况下,利润和剩余价值才是等同的。(这里不必考虑地租,因为剩余价值最初完全由资本家所占有,不管他以后要把其中多大部分分给他的同伙。李嘉图自己也认为地租是从利润中分离、分割出来的部分。)而李嘉图在论述利润和工资时,也就把不是花费在工资上的资本的不变部分撇开不谈。他是这样考察问题的:似乎全部资本都直接花费在工资上了。因此,就这一点说,他考察的是剩余价值,而不是利润,因而才可以说他有剩余价值理论。但另一方面,他认为他谈的是利润本身,的确他的著作中到处都可以看到从利润的前提出发,而不是从剩余价值的前提出发的观点。在李嘉图正确叙述剩余价值规律的地方,由于他把剩余价值规律直接说成是利润规律,他就歪曲了剩余价值规律。另一方面,他又想不经过中介环节而直接把利润规律当作剩余价值规律来表述。

因此,当我们谈李嘉图的剩余价值理论时,我们谈的就是他的利润理论,因为他把利润和剩余价值混淆起来了,也就是说,他只是从对可变资本即花费在工资上的那部分资本的关系来考察利润。至于李嘉图谈到同剩余价值有区别的利润的地方,我们留到后面再分析。

剩余价值只能从对可变资本即直接花费在工资上的资本的关系来考察,——而没有对剩余价值的认识,就不可能有任何利润理论,——这是如此符合事情的本质,以致李嘉图把全部资本看作可变资本,而把不变资本撇开不谈,虽然他有时也以预付资本的形式提到不变资本。

[637]李嘉图谈到(第二十六章《论总收入和纯收入》)

\begin{quote}{“利润同资本成比例,而不是同所使用的劳动量成比例的工商业部门”。(李嘉图《政治经济学和赋税原理》第418页)}\end{quote}

李嘉图的全部平均利润学说(他的地租理论是以此为基础的),除了归结为确认利润“同资本成比例,而不是同所使用的劳动量成比例”,还能是什么呢?如果利润“同所使用的劳动量成比例”,那末相等的资本就会提供极不相等的利润,因为这些资本的利润等于它们本部门生产出来的剩余价值,而剩余价值不取决于全部资本的量,而取决于可变资本的量,或者说,取决于“所使用的劳动量”。因此,怎么能说,利润同所投资本的量成比例,而不同所使用的劳动量成比例,仅仅是某种特殊投资部门即特殊生产部门所特有的例外情况呢?如果剩余价值率既定,对一定资本来说,剩余价值量就必然总是取决于所使用的劳动量,而不取决于资本的绝对量。另一方面,如果平均利润率既定,利润量就必然总是取决于所使用的资本的量,而不取决于所使用的劳动量。

李嘉图明确地谈到这样一些部门,如

\begin{quote}{“海运业、同遥远的国家进行的对外贸易,以及需要昂贵机器装备的部门”。(第418页)}\end{quote}

这就是说,他谈的是那些使用不变资本较多而可变资本较少的部门。同时,这些部门同其他部门相比,预付资本的总量大,换句话说,这些部门只有依靠大资本才能经营。如果利润率既定,利润量就完全取决于预付资本的量。但这决不是使用大资本和使用许多不变资本(这两者往往联系在一起)的部门不同于使用小资本的部门的特点,这不过是下述论点的一种运用,即等量资本提供等量利润,因而较大的资本能比较小的资本提供更多的利润。这同“所使用的劳动量”没有任何关系。但是,利润率一般是大还是小,确实取决于整个资本家阶级的资本所使用的劳动总量,取决于所使用的无酬劳动的相对量,最后取决于花费在劳动上的资本同只是作为生产条件再生产出来的资本之间的比例。

李嘉图本人就反驳了亚·斯密的下述看法,即认为对外贸易中的较高利润率,“个别商人在对外贸易中有时赚得的大量利润,会提高国内的一般利润率”。李嘉图说:

\begin{quote}{“他们断言,利润的均等是由利润的普遍提高造成的;而我却认为,特别有利的部门的利润会迅速下降到一般水平。”(第7章《论对外贸易》,第132—133页)}\end{quote}

李嘉图认为,特殊利润(如果不是由市场价格涨到价值以上所造成)虽然会平均化,但不会提高一般利润率;其次,他认为,对外贸易和市场的扩大不可能提高利润率,李嘉图的这些观点究竟正确到什么程度,我们留到后面再说\fnote{见本册第494—497页和第535—536页。——编者注}。但是,如果承认他的观点是正确的,如果一般承认“利润的均等”,那末,他又怎么能够把“利润同资本成比例”的部门与利润“同所使用的劳动量成比例”的部门区别开来呢?

在前面引用的第二十六章《论总收入和纯收入》中,李嘉图说:

\begin{quote}{“我承认,由于地租的性质,除了最后耕种的土地以外,任何一块土地上用于农业的一定量资本所推动的劳动量,都比用于工业和商业的等量资本所推动的劳动量大。”(第419页)}\end{quote}

这句话完全是无稽之谈。第一,按照李嘉图的说法,在最后耕种的土地上使用的劳动量比所有其他土地上使用的劳动量大。在他看来,其他土地上的地租就是由此产生的。因此,怎么能说,除了最后耕种的土地以外,一定量资本在所有其他土地上推动的劳动量,一定会比在工业和商业上推动的劳动量大呢?较好土地的产品的市场价值,超过用于耕种这种土地的资本使用的劳动量所决定的个别价值,这同一定量资本“所推动的劳动量,比用于工业和商业的等量资本所推动的劳动量大”,是不一样的吧?但是如果李嘉图说,撇开土地肥力的差别,地租的产生一般是由于,农业资本所推动的劳动量,就资本的不变部分而言,比非农业生产中的平均资本所推动的劳动量大;那当然就对了。

[638]李嘉图没有看到,在剩余价值既定时,有些原因会使利润提高或降低,总之会对利润发生影响。因为李嘉图把剩余价值和利润等同起来,所以,当他现在要证明利润率的提高和降低仅仅是由引起剩余价值率提高或降低的那些情况决定的时候,他是前后一贯的。其次,他没有看到,如果撇开在剩余价值量既定时影响利润率(虽然并不影响利润量)的那些情况不谈,利润率就取决于剩余价值量,而决不是取决于剩余价值率。如果剩余价值率,剩余劳动率既定,剩余价值量就取决于资本的有机构成,即取决于一定价值的资本例如100镑所雇用的工人人数。在资本有机构成既定时,剩余价值量就取决于剩余价值率。可见,剩余价值量决定于以下两个因素:同时雇用的工人人数和剩余劳动率。如果资本增大,那末,不管资本的有机构成如何,——假定资本虽然增大而其有机构成不变,——剩余价值量也会增加。但这丝毫不会改变下述情况:对于一定价值的资本例如100来说,剩余价值量保持不变。如果这里剩余价值量等于10,那末对于1000来说,剩余价值量就等于100,但是比例不会因此变动。

{李嘉图写道:

\begin{quote}{“在同一经济部门不可能有两种利润率;所以,在产品价值对资本的比例不同时,不同的将是地租,而不是利润。”(第212—213页)(第12章《土地税》)}\end{quote}

这只适用于“同一经济部门”的正常利润率。否则就同前面引文\fnote{见本册第225和354页。——编者注}中的论点直接矛盾:

\begin{quote}{“一切商品,不论是工业品、矿产品还是土地产品,它们的交换价值始终不决定于在只是享有特殊生产便利的人才具备的最有利条件下足以把它们生产出来的较小量劳动,而决定于没有这样的便利,也就是在最不利条件下继续进行生产的人所必须花在它们生产上的较大量劳动;这里说的最不利条件,是指为了把需要的产品量生产出来而必须继续进行生产的那种最不利的条件。”(第2章《论地租》,第60—61页)}}\end{quote}

在第十二章《土地税》中,李嘉图附带对萨伊提出了如下的反驳。在这里,我们也可以看到,这位英国人总是尖锐地看到了经济上的差别,而那位大陆人却经常忘记这种差别。

\begin{quote}{“萨伊先生[在他所举的例子中]假定,‘一个土地所有者由于勤劳、节俭和经营本领而使自己的年收入增加5000法郎’。但是,土地所有者如果不是自己经营,他就不可能在他的土地上发挥他的勤劳、节俭和经营本领;如果土地所有者自己经营,他就是以资本家和租地农场主的身分,而不是以土地所有者的身分来进行改良。他不预先增加用于这一农场的资本量,单凭自己的特殊经营本领{因而“经营本领”多少也只是一句空话},就能那样增加自己农场的产品,那是不可想象的。”(第209页)}\end{quote}

在第十三章《黄金税》(这一章对李嘉图的货币理论很重要)中,李嘉图提出了关于市场价格和自然价格的某些补充或进一步的规定。这些补充或规定可以归结为一点:这两种价格的平均化进行得较快或较慢,要看该经济部门所允许的供给的增加或减少是快还是慢,也就是说,要看资本向该部门流入或从该部门流出是快还是慢。李嘉图关于地租的论述,受到各方面(西斯蒙第等人)的指责,说他忽略了使用许多固定资本的租地农场主抽出资本的困难,等等。(1815—1830年英国的历史充分地证明了这一点。)不管这种指责如何正确,它根本没有涉及理论,完全没有触动理论,因为这里谈的只不过是经济规律发生作用的快慢程度问题。但对于向新地投入新资本的相反的指责,情况就完全不同了。李嘉图的前提是,向新地投入新资本只能在没有土地所有者干预的条件下进行,这里资本是[639]在它的运动没有遇到抵抗的环境中发挥作用的。然而这是根本错误的。为了证明这个前提,为了证明在资本主义生产和土地所有权已经发展的地方存在这种前提,李嘉图总是设想有以下的情况:土地所有权——或者实际上,或者法律上——并不存在,资本主义生产,至少农业本身的资本主义生产还不发展。

至于刚才谈到的李嘉图关于市场价格和自然价格的论点,那是这样的:

\begin{quote}{“商品价格由于课税或生产困难而上涨的现象,无论如何最终是要发生的;但市场价格和自然价格经过多长时间才会趋于一致,必然取决于这种商品的性质和它的数量能够减少的容易程度。如果被课税的商品数量不能减少,如果比方说租地农场主或制帽厂主的资本不能抽到别的部门去,那末,即使他们的利润因课税而降低到一般水平之下,也不会引起什么后果。除非对他们的商品的需求增加,租地农场主和制帽厂主决不可能把谷物和帽子的市场价格提高到这些商品增加了的自然价格的水平。即使他们扬言要放弃这个行业,把自己的资本转到更有利的部门中去,也会被看作是虚张声势,决不会实现;所以这类商品的价格不会靠缩减生产来提高。但是,实际上一切商品的数量都是可以减少的,资本也可以由利润较小的部门转到利润较大的部门,不过速度有所不同而已。一种商品的供给越是易于缩减而又无损于生产者,在由于课税或任何其他原因而使生产困难增加之后,该商品的价格就越是迅速地上涨。”(第214—215页)“一切商品的市场价格和自然价格的一致,总是取决于该商品的供给增减的容易程度。对于金、房屋、劳动以及其他许多物品来说,在某些情况下是不可能很快达到这种结果的。但是,象帽子、鞋子、谷物和衣服这样一些逐年消费又逐年再生产的商品,情况就不同了。这些商品的供给在必要时可以减少,并且不需要很长时间就能使供给缩减到与增加了的生产费用相适应”。(第220—221页)}\end{quote}

\tsubsectionnonum{[(2)利润率变动的各种不同情况]}

李嘉图在这第十三章《黄金税》中说:

\begin{quote}{“地租不是财富的创造,只是财富的转移。”(第221页)}\end{quote}

难道利润是财富的创造,或者说,利润倒不是剩余劳动从工人到资本家的转移吗?至于工资,它事实上也不是财富的创造,但也不是财富的转移。它是劳动产品的一部分由生产这个产品的人占有。

在这一章中,李嘉图说:

\begin{quote}{“……对地面上的原产品所课的税,会落在消费者身上,并且决不会影响地租,除非这种税通过削减维持劳动的基金而压低工资,缩减人口并减少对谷物的需求。”(第221页)}\end{quote}

李嘉图说,“对地面上的原产品所课的税”既不会落在土地所有者身上,也不会落在租地农场主身上,而会落在消费者身上,这是否正确,我们在这里暂且不论。但是,我敢断言,如果他是正确的,这种税就会提高地租,而李嘉图认为,这种税不会影响地租,除非它通过使生活资料等等涨价而减少资本、人口和对谷物的需求。问题在于,李嘉图以为,原产品的涨价只是在它使工人消费的生活资料涨价的限度内,才影响利润率。这里,说原产品涨价只是在这个限度内才能影响剩余价值率,因而影响剩余价值本身,并因此也影响利润率,那是对的。但是,在剩余价值既定时,“地面上的原产品”涨价,会提高不变资本(与可变资本相比)的价值,会增大不变资本对可变资本的比例,所以,就会降低利润率,因而就会提高地租。李嘉图的出发点是:[640]既然原产品无论涨价或跌价都不影响工资,它也就不会影响利润;因为他断言{有一段话除外,那一段话后面我们回过头来再谈\fnote{见本册第490—491页。——编者注}},不管预付资本的价值降低还是提高,利润率保持不变。因此,如果预付资本的价值增加,那末产品的价值也就增加,同样,产品中构成剩余产品即利润的那一部分也就增加。预付资本的价值降低时情况则相反。这种说法只有在下述场合才是正确的,即由于原料涨价、课税或其他原因,可变资本和不变资本的价值按同一比例发生变动。在这种场合,利润率保持不变,因为资本有机构成没有发生任何变动。即使在这种场合,也必须假定在出现暂时性变动时发生的情况,那就是——工资保持不变,尽管原产品可能涨价或跌价(也就是说,工资保持不变,不管工资的使用价值在价值既定不变时是提高还是降低)。

可能有以下一些情况。

首先说两种主要的差别。

(A)由于生产方式的变动,所使用的不变资本量和可变资本量之间的比例发生变动。在这种情况下,假定工资按价值来说{即按(它所代表的)劳动时间来说}不变,剩余价值率就保持不变。但是,如果同一资本所使用的工人人数,即可变资本发生变动,剩余价值本身就会发生变动。如果由于生产方式的变动,不变资本相对减少,那末剩余价值就会增加,因而利润率也就提高。反之,其结果也相反。

这里始终假定,一定量比如说100单位的不变资本和可变资本的价值保持不变。

在这种情况下,生产方式的变动在同样程度上影响不变资本和可变资本,也就是比如说,不变资本和可变资本在价值没有变动时必定以同样程度增加或减少,是不可能的。因为在这里不变资本和可变资本的减少和增加的必然性总是同劳动生产率的变动相联系的。生产方式的变动对不变资本和可变资本的影响是不同的而不是相同的,这一点,在资本有机构成既定的情况下,与必须使用大资本还是小资本毫无关系。

(B)生产方式不变。在不变资本和可变资本的相对量不变(也就是它们各自在总资本中所占的份额不变)的情况下,不变资本和可变资本之间的比例变动,是由于加入不变资本或可变资本的商品的价值有了变动而发生的。

这里可能有以下几种情况:

[1]不变资本的价值不变;可变资本的价值提高或降低。这总是会影响剩余价值,因此也会影响利润率。

[2]可变资本的价值不变;不变资本的价值提高或降低。于是,在前一场合利润率会降低,在后一场合则会提高。

[3]如果不变资本的价值和可变资本的价值同时降低,但降低的比例不同,那末,一个的价值同另一个的价值相比,总是或者提高,或者降低。

[4]不变资本和可变资本的价值按同一比例变动,不管两者同时提高或同时降低,都是如此。如果两者价值都提高,那末利润率就降低,但这不是因为不变资本的价值提高,而是因为可变资本的价值提高,从而剩余价值降低(因为这里只是可变资本的价值提高了,尽管这个资本所推动的工人人数照旧不变,甚至可能减少)。如果两者价值都降低,那末利润率就提高,但这不是因为不变资本的价值降低,而是因为可变资本(在价值上)降低,从而剩余价值增长。

(C)生产方式的变动以及构成不变资本或可变资本的各要素价值的变动。

这里一种变动可能和另一种变动相抵销,例如,如果不变资本的量增加,而它的价值降低或保持不变(因而一定量比如说100单位的价值也相应降低),或者,如果不变资本的量降低,而它的价值保持不变(因而一定量的价值就相应提高)或按同一比例提高。在后一种情况下,资本的有机构成不会发生任何变动。利润率保持不变。但是,不变资本的量与可变资本相对来说减少,而它的价值却增长,这种情况,除农业资本以外,是决不可能发生的。

一种变动对另一种变动的这种抵销作用,对可变资本来说是不可能的(在实际工资不变的条件下)。

因此,除上述那一种情况以外,只有一种可能:同可变资本相比,不变资本的价值和量同时相对地降低或提高;因而,同可变资本相比,不变资本的价值绝对地提高或降低。这种情况我们已经考察过了。如果不变资本的价值和量虽然同时降低或提高,[641]但是比例不同,那末根据假定,这总是可以归结为:同可变资本相比,不变资本的价值提高或降低。

这也包括另一种情况。因为,如果不变资本的量增加,可变资本的量就相对减少,反之,结果也相反。对价值来说,情况也完全一样。[641]

\tsubsectionnonum{[(3)不变资本和可变资本在价值上的彼此相反的变动以及这种变动对利润率的影响]}

[642]关于C的情况(第640页),还必须注意以下这一点:

可能有这种情况:工资提高了,而不变资本在价值上,不是在量上,却降低了。如果提高和降低这两端彼此相符,利润率就可能保持不变。例如,不变资本=60镑,工资=40镑,剩余价值率=50%,于是,产品=120镑,而利润率=20%。如果不变资本在它的量保持不变时降到40镑,如果工资提高到60镑,而剩余价值从50%降到[33+(1/3)]%,那末产品仍然会等于120镑,而利润率会等于20%。这是不对的。

根据假定,所使用的[活]劳动量创造的总价值为60镑。因此,如果工资提高到60镑,剩余价值,因而利润率,就会等于零。即使工资不提高这么多,工资的任何提高也总会引起剩余价值的降低。如果工资提高到50镑,剩余价值就等于10镑;如果工资提高到45镑,剩余价值就等于15镑,依此类推。可见,在一切情况下,剩余价值和利润率都以同样程度降低。因为剩余价值和利润率在这里是按保持不变的总资本来计算的。在资本(指总资本)量相同时,利润率必定不是随着剩余价值率一同提高和降低,而是随着剩余价值绝对量一同提高和降低。

如果在上述例子中[不变资本由亚麻构成],亚麻价格下降,由同一数量的工人纺成纱的那个亚麻量,可以用40镑买到,那末我们就会得出如下结果:

\todo{}

这里利润率降到20%以下。

如果不变资本的价值降低到30镑,我们就会得出:

\todo{}

如果不变资本的价值降低到20镑,我们就会得出:

\todo{}

在我们假定的前提下,不变资本价值的降低始终只是部分地抵销可变资本价值的提高。在这种前提下,不变资本价值的降低不可能全部抵销可变资本价值的提高,因为要使利润率等于20%,剩余价值10镑必须是整个预付资本的1/5。但是,在可变资本等于50镑的情况下,只有在不变资本等于0时才有这种可能。如果我们假定,可变资本只提高到45镑,那末剩余价值将是15镑。如果我们还假定,不变资本降低到30镑,那末,我们就会得出如下结果:

\todo{}

因而,在这里,两种运动完全相互抵销了。

[643]下面我们再举这样一种情况:

\todo{}

因而,在这里,即使剩余价值降低了\fnote{同最初的情况60c+40v+20m相比。——编者注},但由于不变资本价值降低得更多,利润率也可能提高。同样使用100镑资本,尽管工资提高了,剩余价值率降低了,却能雇用更多的工人。虽然剩余价值率降低了,但剩余价值本身,因而利润却增加了,因为工人人数增加了。根据上述20c+45v这个比例,在使用100镑资本时,我们得出如下比例:

\todo{}

剩余价值率和工人人数之间的比例在这里有极其重要的意义。李嘉图从来不考察这种比例。[634]

\centerbox{※     ※     ※}

[641]前面对于一个资本有机构成内部的变动所作的考察,显然对于各个不同资本,对于各个不同生产部门的资本之间有机构成的差别来说,也是适用的。

第一,代替一个资本的有机构成的变动的,将是各个不同资本的有机构成的差别。

第二,[代替]由一个资本的两部分价值变动引起的有机构成的变动的,将是各个不同资本之间在它们所使用的原料和机器的价值方面的完全一样的差别。这不适用于可变资本,因为我们假定各个不同生产部门的工资相等。各个不同部门中的不同工作日在价值上的差别和这个问题毫无关系。如果首饰匠劳动比粗工的劳动贵,那末首饰匠的剩余劳动时间也按同一比例,比粗工的剩余劳动时间贵。\endnote{手稿(第641页)中接着有几段话谈到斯密对房租的看法。这几段话本版移至第十四章(见第415页)。——第437页。}[641]

\tsubsectionnonum{[(4)李嘉图在他的利润理论中把费用人价格同价值混淆起来]}

[641]在第十五章《利润税》中,李嘉图说:

\begin{quote}{“对通称为奢侈品的那些商品所课的税,只会落在这些商品的消费者身上……但是,对必需品所课的税,落到消费者身上的负担,不是同他们的消费量成比例,而总是要高得多。”例如,谷物税[落到工厂主身上的负担,不仅要看他消费的谷物是多少,而且要看谷物涨价使工资提高了多少]。“这会改变资本的利润率。凡是使工资提高的一切东西,都会减少资本的利润;因此,对工人消费的任何一种商品所课的任何一种税,都有降低利润率的趋势。”(第231页)}\end{quote}

如果课税的对象不仅加入个人消费,而且加入生产消费,或者它只加入生产消费,那末,对消费者所课的税同时就是对生产者所课的税。但是,在这种情况下,这不仅仅适用于工人消费的必需品,而且适用于资本家在生产上消费的一切材料。每一种这样的税都会降低利润率,因为它会提高不变资本的价值(与可变资本相对而言)。

我们就拿对亚麻或羊毛所课的税作例子。[642]亚麻涨价了。因此麻纺业者用资本100就不可能买到和以前同样数量的亚麻来纺纱了。因为生产方式不变,所以,麻纺业者为了把原来数量的亚麻纺成纱,就需要和以前同样数量的工人。但是,与花费在工资上的资本相对而言,亚麻现在比以前具有更大的价值。因而利润率降低。在这种情况下,麻纱价格的上涨并不会给他带来好处。这个价格上涨的绝对量,对麻纺业者根本无关紧要。全部问题只在于产品价格超过预付资本价格的那个余额。如果麻纺业者想要提高整个产品的价格,以便不仅弥补亚麻价格的上涨,而且使同量的纱给他带来和以前一样多的利润,那末,由于麻纱的原料价格上涨而已经下降了的需求,现在由于为了提高利润而人为地提高产品的价格,就会更加降低。尽管平均利润率是既定的,这种加价在这里却是办不到的。\endnote{手稿(第642页末和第643页开头)中接着有几段话谈不变资本和可变资本在价值上的彼此相反的变动。这几段话是对马克思手稿第640—641页的补充,放在本册第434—436页。——第438页。}[642]

[643]也是在第十五章《利润税》中,李嘉图说:

\begin{quote}{“我们在本书前面一个部分,已考察过资本划分为固定资本和流动资本,或者更确切地说,划分为耐久资本和非耐久资本对商品价格的影响。我们曾经指出,两个工厂主使用的资本额可能完全相等,由此获得的利润额可能完全相等,但他们的商品的售价,将根据他们所用资本的消费和再生产的快慢而极不相同。其中一个工厂主的商品可能卖4000镑,而另一个工厂主的商品可能卖10000镑,虽然他们每人使用的资本都是10000镑,得到的利润都是20%即2000镑。一个工厂主的资本,比如说,可能由必须再生产的流动资本2000镑以及建筑物、机器等固定资本8000镑所构成;相反,另一个工厂主可能有流动资本8000镑,机器、建筑物等固定资本却只有2000镑。如果现在这两个资本家每人的收入都课税10%即200镑,那末,一个工厂主为了获得一般利润率,必须把自己的商品价格从10000镑提高到10200镑;另一个工厂主也必须把自己的商品价格从4000镑提高到4200镑。在课税前,一个工厂主出卖的商品比另一个工厂主的商品贵1.5倍;课税以后,则贵1.42倍。一种商品的价格提高2%,另一种商品则提高5%。因此,如果货币价值保持不变,所得税将改变商品的相对价格和价值。”(第234—235页)}\end{quote}

错误就在于最后“价格和价值”的这个“和”字。价格的这种变动只证明(在资本按不同比例分为固定资本和流动资本时也完全一样):为了确定一般利润率,由一般利润率决定、调节的价格或费用价格,与商品的价值必然是极不相同的,而这个极为重要的观点,李嘉图是根本没有的。

在同一章,李嘉图说:

\begin{quote}{“如果一个国家不收税,而货币价值又下降,那末货币的充裕在每一个市场上{这里李嘉图有一个可笑的想法:好象随着货币价值下降,每一个市场上都必然会出现货币的充裕}[644]会对每一种商品产生同样的影响。如果肉价上涨20%,那末,面包、啤酒、鞋子、劳动以及其他任何商品的价格也会上涨20%。只有这样才能使所有生产部门的利润率相等。但是,如果这些商品中有一种被课税,情况就不同了;如果这时所有商品的价格都按货币价值下降的比例上涨,那末利润就会不相等;对于被课税的商品来说,利润就会高于一般水平,在利润恢复平衡以前,资本就会从一个部门转移到另一个部门,但利润只有在相对价格发生变动之后才能恢复平衡。”(第236—237页)}\end{quote}

而利润的这种平衡一般是这样形成的:各种商品的相对价值,实际价值会发生变动,会互相适应,以致不是同自己的实际价值相一致,而是同它们必须提供的平均利润相一致。

\tsubsectionnonum{[(5)一般利润率和绝对地租率之间的关系。工资下降对费用价格的影响]}

在第十七章《原产品以外的其他商品税》中,李嘉图说:

\begin{quote}{“布坎南先生认为,谷物和原产品是按垄断价格出卖的,因为它们提供地租。他假定,一切提供地租的商品都必须按垄断价格出卖;他由此得出结论说,对原产品所课的一切税都会落在土地所有者身上,而不会落在消费者身上。布坎南说:‘因为总是提供地租的谷物的价格不论从哪一方面来说都不受它的生产费用的影响,所以这种费用必须从地租中支付;因此,当这种费用有所增减时,结果不是价格的涨落,而是地租的增减。从这个观点来看,对农业工人、马匹或农具所课的一切税,实际上都是土地税,这种税的负担在整个租佃期内都落在租地农场主身上,而在租约重订时,则落在土地所有者身上。同样,使租地农场主能够缩减生产费用的一切改良农具,例如脱粒机和收割机,以及便于租地农场主把产品运到市场的一切设施,例如良好的道路、运河和桥梁,虽然会减少谷物的实际生产费用,但不会降低谷物的市场价格。因此,由于这类改良而节省下来的一切,都作为地租的一部分归土地所有者所得。’很明显,〈李嘉图说〉如果我们承认布坎南先生立论的根据,即谷物价格总是提供地租,那末,当然就会由此得出他所主张的一切结论。”(第292—293页)}\end{quote}

这一点也不明显。布坎南立论的根据,并不在于一切谷物都提供地租,而是在于提供地租的一切谷物都按垄断价格出卖,在于亚·斯密所解释和李嘉图所理解的那种意义的垄断价格,就是“消费者购买商品愿意支付的最高价格”。\endnote{李嘉图在他的《原理》第十七章(第三版第289—290页)提出了关于垄断价格的这个定义。马克思在前面、在本册第396页,引用了亚·斯密关于垄断价格的类似定义。——第440页。}

但这恰好也是错误的。提供地租(把级差地租撇开不谈)的谷物,并不是按照布坎南所说的垄断价格出卖的。谷物只有在高于它的费用价格即按它的价值出卖的时候,才按垄断价格出卖。谷物的价格决定于物化在谷物中的劳动量,不决定于它的生产费用,而地租是价值超过费用价格的余额,因而是由费用价格决定的:与价值相比,费用价格越小,地租就越多,费用价格越大,地租就越少。一切改良都会使谷物的价值降低,因为它们使生产谷物所需要的劳动量减少。但它们会不会使地租降低,却取决于各种情况。如果谷物跌价,因而工资降低,那末剩余价值率就提高。在这种情况下,租地农场主用于种子、家畜饲料等等方面的费用也会降低。因此,其他一切非农业生产部门的利润率就会提高,从而农业的利润率也会提高。在非农业生产部门,直接劳动和积累劳动的相对量会保持不变;工人人数和以前一样(与不变资本相对而言),但可变资本的价值会降低,因而剩余价值[645]会提高,就是说,利润率也会提高。因此,在农业中,剩余价值和利润率也会提高。在这里地租会降低,因为利润率提高了。谷物便宜了,但它的费用价格增加了。因此,它的价值和它的费用价格之间的差额缩小。

根据我们的假定,平均的非农业资本的比例=80c+20v,剩余价值率=50%;所以剩余价值=10,而利润率=10%。因而,具有平均构成的资本100的产品价值等于110。

现在假定,由于谷物跌价,工资降低1/4;这样,用不变资本80镑即用同量原料和机器来劳动的同一工人人数,总共只花费15镑。而同量商品的价值将是80c+15v+15m,因为根据假定,这些工人所完成的劳动量等于30镑。因此,同量商品的价值仍旧等于110镑。但是,所花费的资本只有95镑,15镑比95镑,就是[15+(15/19)]%。如果花费的资本量照旧不变,或者说,按资本100镑计算,那就得出这样的比例:[84+(4/19)]c+[15+(15/19)]v。利润等于15+(15/19)镑。产品价值=115+(15/19)镑。但是,根据我们的假定,农业资本=60c+40v,而它的产品价值等于120镑。当费用价格是110镑时,地租等于10镑。现在地租总共只有4+(4/19)镑,因为115+(15/19)镑+4+(4/19)镑=120镑。

这里我们可以看到:具有平均构成的资本100镑生产的商品,其费用价格是115+(15/19)镑,而不是以前的110镑。[单位]商品的平均价格会不会因此而提高呢?

商品的价值仍然和以前一样,因为要把同样数量的原料和机器转化为产品,需要同样数量的劳动。但同样的100镑资本推动了较大量的劳动,现在不是把以前的80镑不变资本,而是把84+(4/19)镑不变资本转化为产品。但是在同量的[新加]劳动中,无酬劳动比以前多了。因此,利润以及资本100镑生产的全部商品量的总价值都增加了。单位商品的价值保持不变,但用资本100镑,生产出了更多的具有同一价值的单位商品。但是,各个不同的生产部门的费用价格情况会怎样呢?

假设非农业资本由下列资本构成:

\todo{}

(2)的差额=-10,(3)和(4)的差额加在一起=+10。对于全部资本400来说,这个差额是:0-10+10=0。如果资本400的产品卖440,那末,这笔资本生产的商品就是按它们的价值出卖。那就会得到10%的利润。但是,(2)的商品比它们的价值低10镑出卖,(3)的商品比它们的价值高2+(1/2)镑出卖,而(4)则比它们的价值高7+(1/2)镑出卖。只有(1)的商品在按照它的费用价格(即100镑资本加10镑利润)出卖时,才是按其价值出卖。

[646]但如果工资降低1/4,比例关系将会怎样呢?

对资本(1)来说,现在已不是80c+20v,而是[84+(4/19)]c+[15+(15/19)]v,利润——15+(15/19),产品价值——115+(15/19)。

对资本(2)来说,现在工资只花费30镑,因为40的1/4=10,40-10=30。产品价值是:60c+30v+剩余价值30(因为所使用的劳动创造的价值在这里等于60镑)。这里资本为90镑。工资占[33+(1/3)]%。对于资本100来说,得出的比例是[66+(2/3)]c+[33+(1/3)]v;产品价值=133+(1/3)。利润率=[33+(1/3)]%。

对资本(3)来说,现在工资只花费11+(1/4)镑,因为15的1/4=3+(3/4),而15-[3+(3/4)]=11。产品价值是:85c+[11+(1/4)]v+剩余价值11+(1/4)(所使用的劳动创造的价值在这里等于22+(1/2))。这里资本为96+(1/4)镑。工资占[11+(53/77)]%。对于资本100来说,得出的比例是[88+(24/77)]c+[11+(53/77)]v,利润率=[11+(53/77)]%,而产品价值=111+(53/77)。

对资本(4)来说,现在工资只花费3+(3/4)镑,因为5的1/4=1+(1/4),而5-[1+(1/4)]=3+(3/4)。产品价值是:95c+[3+(3/4)]v+剩余价值3+(3/4)(因为全部[新加]劳动所创造的价值在这里等于7+(1/2))。这里资本为98+(3/4)镑。工资占[3+(63/79)]%。对于资本100来说,得出的比例是[96+(16/79)]c+[3+(63/79)]v。利润率=3+(63/79)。产品价值=103+(63/79)。

这样,我们就得出:

\todo{}

利润是16%,更确切些说,略高于[16+(1/7)]%。计算是不完全准确的,因为我们在计算平均利润时,把分数省略了,在进一步计算时没有包括在内,因此,(2)的负差大了一些,(1)、(3)、(4)的[正差]小了一些。但是,我们看到,如果计算精确,正差和负差就会相互抵销。但是我们也看到,一方面,(2)的低于本身价值出卖的商品,[另一方面](3)特别是(4)的高于本身价值出卖的商品都会大大增加。固然,对单位产品来说,这种高于或低于价值的程度不象表上的数字那么大,因为在所有这四类里,都使用了[比以前]更多的劳动量,因而有更多的不变资本(原料和机器)转化为产品;所以上述这种高于或低于价值的数字是分摊在更大量的商品上。不过,这种高于或低于价值的情况还是很显著的。

由此可见,工资的降低,对(1)和(3)来说,会引起费用价格的上涨[与价值相比],对(4)来说,会引起费用价格的极大上涨。这就是李嘉图在考察流动资本和固定资本的差别时所引出的规律,\endnote{马克思指李嘉图的《政治经济学和赋税原理》一书(第三版)第一章第四节和第五节,李嘉图在这两节中研究了工资的涨落对具有不同有机构成的资本所生产的商品的“相对价值”的影响问题。马克思在本册第192—221页对这两节作了详细的批判分析。——第444页。}但是他丝毫没有证明,也不可能证明:这一规律同价值规律是可以并行不悖的,产品的价值对总资本来说保持不变[不管它在各个生产部门之间如何分配]。

[647]如果我们还注意到由流通过程产生的资本有机构成的差别,计算和平均起来会复杂得多。实际上,在我们计算时,我们是假定,全部预付不变资本都加入产品,也就是说,它只包含固定资本例如在一年内(因为我们必须按年度来计算利润)的损耗。如果我们不这样假定,产品量的价值就会极不相同,而这样假定时,产品量的价值只与可变资本一起变动。第二,在剩余价值率相同而流通时间不同的时候,与预付资本相对而言,所生产的剩余价值量会有很大的差别。这里,如果撇开可变资本的差别不谈,剩余价值量彼此之间的比例就与等量资本生产的不同价值量彼此之间的比例相同。在不变资本的较大部分由固定资本构成的地方,利润率会低得多,在资本的较大部分由流动资本构成的地方,利润率会高得多;在可变资本较大(与不变资本相比),同时在不变资本中固定资本部分又较小的地方,利润率最高。如果不变资本的流动部分和固定部分之间的比例在不同的资本中是相同的,那就只有可变资本和不变资本之间的差别是决定的因素了。如果可变资本与不变资本的比例是相同的,那就只有固定资本和流动资本之间的差别,即不变资本本身内部的差别是决定的因素了。

正如我们已经看到的,如果非农业资本的一般利润率由于谷物跌价而提高,那末,租地农场主的利润率无论如何都会提高。问题在于,租地农场主的利润率会不会直接提高,看来,这要取决于所实行的改良的性质。如果实行的这种改良使花费在工资上的资本与花费在机器等等上的资本相比大大减少,那末,租地农场主的利润率就不必直接提高。如果这种改良,比如说,使租地农场主需要的工人减少1/4,那末,租地农场主现在必须花费在工资上的就不是以前的40镑,而只是30镑。因而他的资本现在是60c+30v,或者以100计算,就是[66+(2/3)]c+[33+(1/3)]v。因为用40单位支付的劳动,提供剩余价值20,所以,用30支付的劳动提供15,而用33+(1/3)支付的劳动就提供16+(2/3)。这样一来,农业资本的有机构成与非农业资本的有机构成便接近了。在上述情况下,如果工资同时下降1/4,农业资本构成甚至可能成为非农业资本构成的个别场合。\endnote{马克思在这里举例说明可能发生农业资本有机构成接近工业资本有机构成的过程的一种趋向。马克思以下述情况为出发点:农业资本是60c+40v,非农业资本是80c+20v。马克思假定,由于农业劳动生产率提高,农业工人人数减少四分之一。因而,农业资本的有机构成发生变动:过去需要花费100单位资本——60c+40v的产品,现在只需要花费90单位资本——60c+30v,折合100计算,就是[66+(2/3)]c+[33+(1/3)]v。这样一来,农业的资本有机构成就会接近工业的资本有机构成。马克思进一步假定,在农业工人人数减少的同时,工资还因谷物减价而降低四分之一。在这种场合必须假定,在工业中工资也按同一比例降低。然而,工资的降低,对具有较低构成的农业资本比对非农业资本必定会产生更大的影响。这就会使农业资本构成和工业资本构成之间的差额又进一步缩小。农业资本[66+(2/3)]c+[33+(1/3)]v,在工资降低1/4时变为资本[66+(2/3)]c+25v,折合100计算,就是[72+(8/11)]c+[27+(3/11)]v。非农业资本80c+20v,在工资降低1/4时变为资本80c+15v,折合100计算,就是[84+(4/19)]c+[15+(15/19)]v。在农业工人人数进一步减少以及工资进一步降低时,农业资本的有机构成就越来越接近非农业资本的有机构成。马克思在考察这种假设的情况时,为了弄清楚农业劳动生产率的增长对农业资本有机构成的影响,在这里撇开不谈工业劳动生产率的同时的而且往往是更迅速的增长,这种增长表现在工业资本有机构成比农业资本有机构成有更进一步的提高。关于工业中的资本有机构成和农业中的资本有机构成之间的关系问题,见前面第7—8、11、95—96、107—108、111、116—118和270—271页。——第445页。}这时,地租(绝对地租)就会消失。

李嘉图在前面引用过的评论布坎南的那段话之后,继续写道:

\begin{quote}{“我希望我已经充分说明,在一个国家的土地尚未全部投入耕种,并且耕种尚未达到最高程度以前,总有一部分投在土地上的资本是不提供地租的,并且〈!〉正是这部分资本调节谷物的价格,这部分资本的产品,正象在工业中一样,分为利润和工资。因为不提供地租的谷物价格,受谷物生产费用的影响,所以这种生产费用不可能从地租中支付。因此,生产费用增加的结果,将是价格上涨,而不是地租降低。”(同上,第293页)}\end{quote}

既然绝对地租等于农产品价值超过它的生产价格的余额,那末,很明显,凡是能使谷物等等生产所需的劳动总量减少的东西,也能使地租减少,因为使价值减少,也就是使价值超过生产价格的余额减少。在生产价格由已支付的费用构成的情况下,生产价格的降低和价值的降低是一回事,而且是和价值的降低同时进行的。但是,在生产价格(或“费用”)等于预付资本加平均利润的情况下,情况却恰好相反。产品的市场价值会降低,但其中等于生产价格的那一部分,在一般利润率由于谷物市场价值降低而提高时,会提高起来。因而,地租降低在这里是因为这个意义上的“费用”(李嘉图谈到生产费用时通常对费用是这样理解的)有了提高。促使不变资本与可变资本相比不断增长的农业改良,即使在所使用的劳动[活劳动和物化劳动]总量只是略有减少,或者说只减少那么一点,以致对工资根本没有影响(对剩余价值没有任何直接影响)的情况下,也会使地租大大降低。如果由于这种改良,资本60c+40v变为[66+(2/3)]c+[33+(1/3)]v(例如由于移民、战争、新市场的发现、外国谷物的竞争、非农业生产部门的繁荣等等所引起的工资提高,租地农场主可能不得不设法使用较多的不变资本和较少的可变资本;而这些情况在实行改良之后还可能继续起作用,因此,尽管有这些改良,工资不会降低),[648]那末,农产品的价值就会从120降到116+(2/3),即减少3+(1/3)。利润率仍然等于10%。地租从10降到6+(2/3),而且地租的这种降低是在工资没有任何降低的情况下发生的。

由于工业的继续进步,一般利润率下降,因此绝对地租可能提高。由于农产品价值增加,从而农产品价值及其费用价格之间的差额增大,结果地租提高,因此利润率可能降低。(同时利润率还会由于工资提高而下降。)

由于农产品价值下降,一般利润率提高,绝对地租就可能降低。由于资本有机构成的变革,农产品价值下降,虽然利润率这时并不提高,绝对地租也可能降低。一旦农产品的价值和它的费用价格彼此相等,从而农业资本具有非农业资本的那种平均构成,绝对地租就会完全消失。

李嘉图的论点只有这样表达才是正确的:当农产品的价值等于它的费用价格的时候,不存在绝对地租。但是在李嘉图那里这个论点是错误的,因为他说:由于价值和费用价格一般是等同的,工业是这样,农业也是这样,\fnote{[663}(下面一段话说明,李嘉图有意识地把价值和生产费用等同起来:“马尔萨斯先生似乎认为,把某物的费用和价值等同起来,是我的学说的一部分。如果他说的费用是指包括利润在内的‘生产费用’,那确是如此。”(同上,第46页))[663]]所以,不存在绝对地租。实际上情况恰好相反:如果在农业中价值和费用价格等同,农业就是一种例外的生产了。

李嘉图承认可能不存在不支付任何地租的土地,同时他认为,即使这样,下面这种情况还是可以作为他的充分依据,即至少投在土地上的资本有些份额是不支付任何地租的。前一种情况和后一种情况对理论来说同样是无关紧要的。真正的问题在于:是由这种土地或这种资本的产品来调节市场价值呢?还是相反,这些产品由于它们的追加供给只能按照而不能高于并非由它们调节的市场价值出卖,因而不得不低于自己的价值出卖呢?关于后来使用的那些资本份额,问题很简单,因为这里在投入追加份额时,土地所有权对租地农场主来说是不存在的,作为资本家,租地农场主只注意费用价格;甚至当他自己是追加资本的所有者时,与其把这笔资本借出,只取得利息,而得不到利润,还不如把它投在他租种的土地上,即使取得的利润低于平均利润,对他更有利。至于地段,那末,这些不支付地租的土地,构成支付地租的整个地产的组成部分;这些地段是同整个地产不可分割的,它们同整个地产一起出租,虽然不能把这些地段单独租给任何一个资本主义农场主(但完全可以租给茅舍贫农以及小资本家)。这些小块土地也并不作为“土地所有权”与租地农场主相对立。或者土地所有者不得不自己耕种这些地段。租地农场主不可能为这些地段支付地租,而土地所有者也不会毫无代价地把它们租出去,除非他是想通过这种办法,自己不花费什么,就把自己的土地变为耕地。

如果在一个国家,农业资本的构成与非农业资本的平均构成相等,情况就不同了,而这是以农业的高度发展或工业发展水平很低为前提的。在这种场合,农产品的价值就会同它的费用价格相等。这时只可能支付级差地租。那些不提供级差地租、只能带来[真正的]农业地租的地段,这时就根本不可能支付任何地租了。因为当租地农场主把这些土地的产品按它们的价值出卖时,它们只抵补他的费用价格。因而租地农场主不支付任何地租。这样一来,土地所有者只好自己耕种这些土地,或者在租金的名义下,把他的租佃者的一部分利润甚至一部分工资刮走。一个国家可能发生这种情况,这并不妨碍另一个国家可能发生完全相反的情况。但是在工业,从而资本主义生产发展水平很低的地方,是不存在资本主义租地农场主的,因为资本主义租地农场主的存在是以农业中实行资本主义生产为前提的。因此,我们考察的就是与土地所有权仅仅作为地租才在经济上存在的那种经济组织完全不同的关系了。

李嘉图也是在第十七章中说:

\begin{quote}{“原产品没有垄断价格,因为大麦和小麦的市场价格,同呢绒和麻布的市场价格一样,是由它们的生产费用调节的。唯一的差别在于:谷物价格是由用于农业的资本的一部分,即不支付地租的那一部分调节的,而在工业品生产中,所用资本的每一部分都产生相同的结果;并且由于任何部分都不支付地租,所以每一部分都同样是价格的调节者。”(同上,第290—291页)}\end{quote}

认为在工业中所用资本的每一部分都产生相同的结果,并且任何部分都不提供地租(不过,工业中叫做超额利润),这种说法不仅是错误的,而且,[650]\endnote{在马克思编的手稿页码中漏了649这个页码。——第449页。}正如我们在前面看到的\fnote{见本册第225、354和428页。——编者注},已经被李嘉图自己所驳倒。

现在我们就来考察李嘉图的剩余价值理论。

\tsectionnonum{[B.李嘉图著作中的剩余价值问题]}

\tsubsectionnonum{(1)劳动量和劳动的价值。[劳动与资本的交换问题按照李嘉图的提法无法解决]}

李嘉图著作的第一章《论价值》一开始第一节就用了这样一个标题:

\begin{quote}{“商品的价值或这个商品所能交换的任何其他商品的量,取决于生产这个商品所必需的劳动的相对量,而不取决于付给这一劳动的报酬多少。”}\end{quote}

这里,李嘉图按照贯穿于他的全部研究中的风格,在他的书的开头就提出这样一个论点:商品价值决定于劳动时间这一规定与工资,或者说,对这种劳动时间即这种劳动量所支付的不同报酬,并不矛盾。李嘉图一开始就反对亚·斯密把商品价值决定于生产商品所必需的相应的劳动量这个规定与劳动的价值(或劳动的报酬)混淆起来。

显然,A和B两个商品包含的相应的劳动量,同生产商品A和B的工人从自己的劳动产品中得到多少,是绝对没有关系的。商品A和B的价值决定于生产它们所花费的劳动量,而不决定于商品A和B的所有者花费的劳动费用。劳动量和劳动价值是两个不同的东西。商品A和B包含的相应的劳动量,同A和B包含多少由A和B的所有者付酬的,甚至是他们自己完成的劳动,是毫无关系的。商品A和B不是按照它们所包含的有酬劳动的比例相互交换,而是按照它们所包含的既包括有酬劳动也包括无酬劳动的劳动总量的比例相互交换。

\begin{quote}{“亚当·斯密如此正确地规定了交换价值的真正源泉,要是前后一贯,他本来应该坚持一切物品价值的大小同生产它们所花费的劳动量的多少成比例的观点,可是他自己又提出了价值的另一个标准尺度,说一切物品价值的大小同它们所能交换的这种标准尺度的量的多少成比例……好象这是两种意思相同的说法;好象一个人由于他的劳动效率增加了一倍,因而能生产的商品量也增加一倍,他用它〈即他的劳动〉进行交换时所得到的量就必然会比以前增加一倍。如果这种说法确实是正确的,如果工人的报酬总是和他所生产的东西成比例,那末用于生产某种商品的劳动量和这种商品所能购买的劳动量就会相等,两者中的任何一个都可以准确地衡量其他物品的价值的变动;但是,它们不是相等的。”(第5页)}\end{quote}

亚·斯密在任何地方都没有说过,“这是两种意思相同的说法”。相反,他说:因为在资本主义生产中工人的工资已不再等于他所生产的产品,因而,一个商品所耗费的劳动量和工人用这一劳动所能购买的商品量,是两个不同的东西,正因为这样,商品所包含的劳动的相对量不再决定商品的价值,商品的价值宁可说是决定于劳动的价值,决定于我用一定量商品所能购买或支配的劳动量。因此,斯密认为,劳动的价值代替劳动的相对量成为价值尺度。李嘉图正确地回答亚·斯密说,两个商品所包含的劳动的相对量,同这种劳动的产品中有多少归工人自己所有毫无关系,同这种劳动的报酬如何毫无关系;因此,既然在工资(不同于产品本身的价值的工资)出现以前,劳动的相对量是商品价值的尺度,那就没有任何根据能够说明,为什么在工资出现以后,劳动的相对量就不再是商品价值的尺度。李嘉图正确地回答说,在这两种说法意思相同的时候,亚·斯密可以使用两种说法,但是,一旦两种说法意思不再相同时,这并不能成为用错误说法去代替正确说法的理由。

但是,李嘉图这些话丝毫没有解决构成亚·斯密的矛盾的内在基础的那个问题。只要我们谈的是物化劳动,劳动的价值和劳动量就依然是“意思相同的说法”。[651]一旦我们谈到物化劳动和活劳动交换,这两种说法就不再是这样的了。

两个商品按照它们包含的物化劳动进行交换。等量物化劳动互相交换。劳动时间是它们价值的“标准尺度”,正因为这样,所以它们的“价值的大小同它们所能交换的这种标准尺度的量的多少成比例”。如果商品A包含一个工作日,那末这个商品就可以与同样包含一个工作日的任何数量的其他商品交换;这个商品的“价值的大小”,同它换得的其他商品中的物化劳动量的多少成比例,因为这种交换比例是这个商品本身包含的劳动的相对量的表现,是和这种劳动的相对量相等的。

但是雇佣劳动是一种商品。它甚至是作为商品的产品进行生产的基础。原来,价值规律不适用于雇佣劳动。那就是说,这个规律根本不支配资本主义生产。这里有一个矛盾。这是亚·斯密遇到的一个问题。第二个问题是,一个商品(作为资本)的价值增殖不是同它所包含的劳动成比例,而是同它所支配的别人的劳动成比例,它所支配的别人的劳动量大于它本身所包含的劳动量,后面我们将会看到,这个问题在马尔萨斯著作中有了更充分的发挥。这实际上是斯密下述说法的第二个秘密动机,他说,自从资本主义生产出现以后,商品价值就不决定于商品所包含的劳动,而决定于商品所支配的活劳动,因而决定于劳动的价值。

李嘉图简单地回答说,在资本主义生产中情况就是这样。他不仅没有解决问题,甚至没有发觉亚·斯密著作中的这个问题。他根据自己研究的整个性质只限于证明,变动着的劳动价值——简单说,就是工资——并不会推翻如下的论点:不同于劳动本身的商品的价值由商品所包含的劳动的相对量决定。“它们不是相等的”,就是说,“用于生产某种商品的劳动量和这种商品所能购买的劳动量”不是相等的。李嘉图满足于确定这一事实。但是,劳动这种商品和其他商品有什么区别呢?一个是活劳动,另一个是物化劳动。因此这只是劳动的两种不同形式。既然这里只是形式的不同,那末,为什么规律对其中一个适用,对另一个就不适用呢?李嘉图没有回答这个问题,他甚至没有提出这个问题。

他说的下面这段话对这个问题也没有什么帮助:

\begin{quote}{“难道劳动的价值不……发生变动吗?它不仅象其他一切物品〈应读作商品〉一样,受始终随着社会状况的每一变动而变动的供求关系的影响,而且受用工资购买的食物和其他必需品的价格变动的影响。”(第7页)}\end{quote}

劳动的价格象其他商品的价格一样随着需求和供给的变动而变动,这一点,照李嘉图自己的意见,在涉及劳动价值的地方,是什么也说明不了的,正如其他商品的价格随着需求和供给的变动而变动,对这些商品的价值是什么也说明不了的一样。但是,“工资”(这只不过是劳动价值的另一种说法)要受“用工资购买的食物和其他必需品的价格变动”的影响这一事实,同样也不能说明为什么决定劳动的价值和决定其他商品的价值不一样(或看起来不一样)。因为其他商品也受加入它们的生产并和它们交换的其他商品的价格变动的影响。而花费在食物和必需品上的工资支出只不过表明劳动的价值同食物和必需品进行交换而已。问题正是在于:劳动同劳动所交换的商品为什么不按价值规律进行交换,不按劳动的相对量进行交换?

这样提出问题,既然以价值规律作为前提,问题本身就无法解决,所以不能解决,是因为这里把劳动本身同商品对立起来了,把一定量直接劳动本身同一定量物化劳动对立起来了。

我们后面将会看到,李嘉图的解释的这个弱点促进了李嘉图学派的瓦解,并且引出了荒谬的假设。

[652]威克菲尔德说得对:

\begin{quote}{“如果把劳动看成一种商品,而把资本,劳动的产品,看成另一种商品,并且假定这两种商品的价值是由相同的劳动量来决定的,那末,在任何情况下,一定量的劳动就都会和同量劳动所生产的资本量相交换;过去的劳动就总会和同量的现在的劳动相交换。但是,劳动的价值同其他商品相比,至少在工资取决于[产品在资本家和工人之间的]分配的情况下,不是由同量劳动决定,而是由供给和需求的比例决定。”(爱·吉·威克菲尔德给他在1835年伦敦出版的亚·斯密的《国富论》一书第1卷第230页所加的注)}\end{quote}

这也是贝利爱好的题目之一;我们在后面将加以考察。萨伊也是这样,他对于在这里突然承认供给和需求的决定性作用是非常高兴的。\endnote{马克思在他的手稿的下一页(第653页)上又回过头来谈到萨伊的“幸灾乐祸”,说这是因为李嘉图在用维持工人生活所必需的生存资料决定“劳动价值”时,引证了供求规律。这里马克思引用的李嘉图著作是康斯坦西奥译、萨伊加注的法译本。马克思在这里是不确切的。萨伊在给李嘉图著作所加的注释中“幸灾乐祸”,是因为李嘉图用供给和需求来决定货币的价值。马克思在《哲学的贫困》(见《马克思恩格斯全集》中文版第4卷第126页)中曾引了萨伊注释中有关的这段话。这段话的出处是:大·李嘉图《政治经济学和赋税原理》,康斯坦西奥译自英文,附让·巴·萨伊的注释和评述,835年巴黎版第二卷第206—207页。——第454、455页。}

\centerbox{※     ※     ※}

[652]关于(1)还要指出,李嘉图著作的第一章第三节用了这样一个标题:

\begin{quote}{“影响商品价值的,不仅是直接花费在商品上的劳动,而且还有花费在协助这种劳动的器具、工具和建筑物上的劳动。”}\end{quote}

因此,商品的价值既决定于为生产该商品所需要的物化的(过去的)劳动量,也决定于为生产该商品所需要的活的(现在的)劳动量。换句话说:劳动量完全不受劳动是物化劳动还是活劳动、过去劳动还是现在(直接)劳动这种形式差别的影响。如果这种差别在规定商品价值时是没有意义的,为什么当过去劳动(资本)同活劳动交换时,这种差别就有了决定性意义呢?既然这种差别本身,正象在对商品的关系上表现出来的那样,对于规定价值没有意义,它为什么在这里就一定会使价值规律失效呢?李嘉图没有回答这个问题,他甚至没有提出这个问题。[652]

\tsubsectionnonum{(2)劳动能力的价值。劳动的价值。[李嘉图把劳动同劳动能力混淆起来。关于“劳动的自然价格”的见解]}

为了规定剩余价值,李嘉图象重农学派、亚·斯密等人一样,必须首先规定劳动能力的价值,或者——按照他跟着亚·斯密和他的先行者使用的说法——劳动的价值。[652]

[652]那末,劳动的价值,或者说,劳动的自然价格,是怎样决定的呢?因为照李嘉图的意见,自然价格不过是价值的货币表现。

\begin{quote}{“劳动正象其他一切可以买卖并且在数量上可以增加或减少的物品一样〈就是说,同其他一切商品一样〉,有它的自然价格和市场价格。劳动的自然价格是使工人大体上说能够生存下去并且能够在人数上不增不减地〈应当说,按照生产的平均增长所需要的增长率〉延续其后代所必需的价格。工人养活自己以及养活为保持工人人数所必需的家庭的能力……取决于工人养活自己及其家庭所必需的食物、必需品和舒适品的价格。随着食物和其他必需品价格的上涨,劳动的自然价格也上涨;随着这些东西的价格下降,劳动的自然价格也下降。”(第86页)“不能认为劳动的自然价格是绝对固定不变的,即使用食物和必需品来计算也是一样。劳动的自然价格在同一国家的不同时期会发生变动,而在不同的国家会有很大差别。它主要取决于人民的风俗习惯。”(第91页)}\end{quote}

可见,劳动的价值是由在一定社会中为维持工人生活并延续其后代通常所必需的生活资料决定的。

但是,为什么?根据什么规律劳动的价值这样决定呢?

李嘉图除了说供求规律把劳动的平均价格归结为维持工人生活所必需(在一定社会中生理上或社会上所必需)的生活资料以外,实际上没有回答这一问题。[653]李嘉图在这里,在其整个体系的一个基本点上,正象萨伊幸灾乐祸地指出的那样(见康斯坦西奥的译本),是用需求和供给来决定价值。\endnote{马克思在他的手稿的下一页(第653页)上又回过头来谈到萨伊的“幸灾乐祸”,说这是因为李嘉图在用维持工人生活所必需的生存资料决定“劳动价值”时,引证了供求规律。这里马克思引用的李嘉图著作是康斯坦西奥译、萨伊加注的法译本。马克思在这里是不确切的。萨伊在给李嘉图著作所加的注释中“幸灾乐祸”,是因为李嘉图用供给和需求来决定货币的价值。马克思在《哲学的贫困》(见《马克思恩格斯全集》中文版第4卷第126页)中曾引了萨伊注释中有关的这段话。这段话的出处是:大·李嘉图《政治经济学和赋税原理》,康斯坦西奥译自英文,附让·巴·萨伊的注释和评述,835年巴黎版第二卷第206—207页。——第454、455页。}

李嘉图本来应该讲劳动能力,而不是讲劳动。而这样一来,资本也就会表现为那种作为独立的力量与工人对立的劳动的物质条件了。而且资本就会立刻表现为一定的社会关系了。可是,在李嘉图看来,资本仅仅是不同于“直接劳动”的“积累劳动”,它仅仅被当作一种纯粹物质的东西,纯粹是劳动过程的要素,而从这个劳动过程是决不可能引出劳动和资本、工资和利润的关系来的。

\begin{quote}{“资本是一国财富中用于生产的部分,由推动劳动所必需的食物、衣服、工具、原料、机器等组成。”(第89页)“较少的资本,也就是较少的劳动。”(第73页)“劳动和资本,即积累劳动”。(第499页)}\end{quote}

贝利正确地觉察到了李嘉图在这里[在劳动价值问题上]所作的飞跃:

\begin{quote}{“李嘉图先生相当机智地避开了一个困难,这个困难乍看起来似乎会推翻他的关于价值取决于在生产中所使用的劳动量的学说。如果严格地坚持这个原则,就会得出结论说,劳动的价值取决于在劳动的生产中所使用的劳动量。这显然是荒谬的。因此,李嘉图先生用一个巧妙的手法,使劳动的价值取决于生产工资所需要的劳动量;或者用他自己的话来说,劳动的价值应当由生产工资所必需的劳动量来估量,他这里指的是为生产付给工人的货币或商品所必需的劳动量。那我们同样也可以说,呢绒的价值不应当由生产呢绒所花费的劳动量来估量,而应当由生产呢绒所换得的银所花费的劳动量来估量。”(《对价值的本质、尺度和原因的批判研究》1825年伦敦版第50—51页)}\end{quote}

这种反驳逐字逐句都是正确的。李嘉图区别了名义工资和实际工资。名义工资是用货币表现的工资,是货币工资。

\begin{quote}{“名义工资”是“一年内付给工人的镑数”,而“实际工资”是“为获得这些镑所必需的工作日数”。(李嘉图,同上第152页)}\end{quote}

既然工资等于工人消费的必需品,而这种工资(“实际工资”)的价值等于这些必需品的价值,那末,显然,这些必需品的价值也等于“实际工资”,等于这一工资所能支配的劳动。如果必需品的价值发生变动,“实际工资”的价值也要变动。假定工人消费的必需品完全由谷物构成,他的必要的生活资料数量是每月一夸特谷物。那末,他的工资的价值[一个月]就等于一夸特谷物的价值;如果一夸特谷物的价值提高或降低,一个月劳动的价值也要提高或降低。但是,不论一夸特谷物的价值怎样提高或降低(不论一夸特谷物包含多少劳动),它的价值总是等于一个月劳动的价值。

这里我们看到了一个隐蔽的理由,说明为什么亚·斯密说,自从资本以及雇佣劳动出现以后,产品的价值已不由花费在产品上的劳动量决定,而由该产品所能支配的劳动量来决定。由劳动时间决定的谷物(和其他必需品)的价值会发生变动;但是,只要劳动的自然价格得到支付,一夸特谷物所能支配的劳动量就保持不变。因此,劳动同谷物相比具有不变的相对价值。就是由于这个缘故,在斯密的著作中,劳动的价值和谷物的价值(谷物在这里代表一般食物;见迪肯·休谟的著作\endnote{马克思引用的是詹姆斯·迪肯·休谟的小册子《关于谷物法的看法》(1815年伦敦版)。休谟在谈到亚当·斯密的“谷物价格决定劳动价格”这一论点时写道:“当斯密博士谈到谷物时,他是指一般食物……一切农产品”。(第59页)——第457页。})都是价值的标准尺度,因为一定量的谷物,只要它支付劳动的自然价格,它就支配一定量的劳动,不管生产一夸特谷物所花费的劳动量是多大。同量劳动总是支配同一使用价值,或者,确切些说,同一使用价值总是支配同量劳动。

李嘉图自己就是这样来规定劳动的价值、劳动的自然价格的。李嘉图说:一夸特谷物具有极不相同的价值,虽然一夸特谷物总是支配[654]同量的劳动,或者说,同量的劳动总是支配这一夸特谷物。是的,亚·斯密说,不管由劳动时间决定的一夸特谷物的价值怎样变动,工人为了购买这一夸特谷物,总是要付出(牺牲)同量劳动。因此,谷物的价值会变动,但是,劳动的价值不会变动,因为一个月的劳动等于一夸特谷物。就是谷物的价值,也只有在我们考察的是生产谷物所需要的劳动的时候才变动。如果我们考察的是这一夸特谷物所交换的劳动量,是谷物所推动的劳动量,那末谷物的价值也不会变动。正因为这样,一夸特谷物所换得的劳动量是价值的标准尺度。而其他商品的价值和劳动的关系,同它们和谷物的关系是一样的。一定量谷物支配一定量劳动。一定量的任何其他商品支配一定量的谷物。因此,任何其他商品,或者,确切些说,任何其他商品的价值,都是用该商品所支配的劳动量来表现的,因为这一商品的价值是用该商品所支配的谷物量来表现的,而这一谷物量又是用该谷物所支配的劳动量来表现的。

但是,其他商品和谷物(必需品)的价值比例由什么决定呢?由这些商品所支配的劳动量决定。而它们所支配的劳动量又由什么决定呢?由这一劳动所支配的谷物量决定。这里,斯密必然陷入循环论证。(不过,我们要顺便指出,斯密在进行真正的分析的地方,从来不采用这个价值尺度。)此外,斯密在这里把劳动同货币混淆起来,李嘉图也常常这样,正如斯密和李嘉图所说,劳动是

\begin{quote}{“商品价值的基础”,而“生产商品所必需的相对劳动量”是“确定商品相互交换时各自必须付出的相应商品量的尺度”。(李嘉图,同上第80页)}\end{quote}

斯密把价值的这一内在尺度同已经以价值规定为前提的外在尺度即货币混淆起来了。

亚·斯密根据一定量的劳动可以换得一定量的使用价值,得出结论说,这一定量的劳动是价值尺度,它总是具有同一价值,可是,同量使用价值却可以代表极不相同的交换价值。亚·斯密这样做是错了。但李嘉图犯了双重的错误,因为第一,他不懂得导致斯密犯错误的问题,第二,他自己完全忘记了商品的价值规律,而求助于供求规律,因而不是用花费在生产劳动力[forceoflabour]上的劳动量来决定劳动的价值,却用花费在生产付给工人的工资上的劳动量来决定劳动的价值,从而他实际上是说:劳动的价值决定于支付劳动的货币的价值!而后者由什么决定呢?支付劳动的货币量由什么决定呢?由支配一定量劳动的使用价值量决定,或者说,由一定量劳动支配的使用价值量决定。结果,李嘉图就一字不差地重犯了他指责亚·斯密犯过的那种前后矛盾的错误。

同时,我们看到,这一点妨碍他理解商品和资本之间的特殊区别,商品同商品的交换和资本同商品的交换(按商品交换规律)之间的特殊区别。

前面举的例子是:1夸特谷物=1个月劳动。假定1个月劳动等于30工作日。(1工作日等于12小时。)在这种情况下,1夸特谷物的价值小于30工作日。如果1夸特谷物是30工作日的产品,那末,劳动的价值就等于劳动的产品。因此就不会有任何剩余价值,因而也不会有任何利润。这样就不会有任何资本。就是说,如果1夸特谷物是支付30工作日的工资,实际上1夸特谷物的价值就总是小于30工作日。剩余价值就取决于这1夸特谷物的价值究竟比30工作日小多少。比如说,1夸特谷物是25工作日的产品。那末,剩余价值=5工作日,=全部劳动时间的1/6。如果1夸特(8蒲式耳)=25工作日,那末,30工作日=1夸特又1+(3/5)蒲式耳。可见,30工作日的价值(即支付30工作日的工资)总是小于包含30工作日的产品的价值。所以,谷物的价值不决定于[655]谷物所支配的劳动,谷物所换得的劳动,而决定于谷物所包含的劳动。相反,30日劳动的价值[即支付30工作日的工资]始终决定于1夸特谷物,而不论1夸特谷物的价值是多少。

\tsubsectionnonum{(3)剩余价值。[李嘉图没有分析剩余价值的起源。李嘉图把工作日看作一个固定的量]}

如果撇开把劳动和劳动能力混淆起来这一点不谈,那末李嘉图倒是正确地规定了平均工资,或者说,劳动的价值。这就是,李嘉图说,平均工资既不决定于工人得到的货币,也不决定于工人得到的生活资料,而是决定于为生产这些生活资料所花费的劳动时间,决定于物化在工人得到的生活资料中的劳动量。李嘉图把这个叫做实际工资。(见下文。)

并且,他得出这一规定是必然的。既然劳动的价值决定于该价值必须花费在上面的必要生活资料的价值,而必要生活资料的价值同一切其他商品的价值一样,决定于生产它们所花费的劳动量,那末,由此自然会得出结论说,劳动的价值等于必要生活资料的价值,等于生产这些必要生活资料所花费的劳动量。

但是,不管这个公式多么正确(撇开劳动和资本的直接对立不谈),它还是不充分的。单个工人为补偿他的工资进行再生产,——就是说,如果考虑到这个过程的连续性,——他生产的虽然不是直接供他维持生活的产品{他可能生产完全不加入他的消费的产品;即使他生产必要生活资料,但由于分工,他只生产其中的一种,例如谷物,并且只赋予它一种形式(例如只是粮食,不是面包)},但他生产的商品具有他的生活资料的价值,或者说,他生产的是他的生活资料的价值。这就是说,——如果我们考察的是工人的一日平均消费的话,——他一日消费的必要生活资料包含的劳动时间,是他的工作日的一部分。他用一日的一部分为再生产他的必要生活资料的价值而劳动;在工作日的这一部分中生产出来的商品,同工人一日消费的必要生活资料具有同样的价值,或者说,包含同样多的劳动时间。他的工作日中有多大一部分用于再生产或生产他的生活资料的价值,即他的生活资料的等价物,这取决于这些必要生活资料的价值(因而,取决于劳动的社会生产率,而不取决于他劳动的那一个别生产部门的生产率)。

李嘉图自然假定,一日的必要生活资料所包含的劳动时间,等于工人每天为了再生产这些必要生活资料的价值而必须劳动的劳动时间。但是,这样一来,由于他在阐述问题时没有把每个工人的工作日的一部分直接表现为用于再生产工人自己劳动能力的价值的时间,他就给这个问题的研究造成了困难,模糊了对这里存在的关系的理解。由此便产生了双重的混乱。剩余价值的起源变得不清楚了,因此后来的经济学家责备李嘉图没有理解、没有阐明剩余价值的性质。部分地也是由于这个原因,产生了他们经院式地解释剩余价值的尝试。但是,因为剩余价值的起源和性质在这里没有得到明确的表述,所以,把剩余劳动加必要劳动——简单说总工作日——看作某种固定的量,忽略了剩余价值量的差别,不理解资本的生产性,不理解资本强迫进行剩余劳动,——一方面是强迫进行绝对剩余价值的生产,其次是资本所固有的缩短必要劳动时间的内在渴望,——因而没有阐明资本在历史上的合理性。相反,亚·斯密已经提出了正确的公式。把剩余价值归结为剩余劳动,和把价值归结为劳动,具有同等重要的意义,并且措辞明确。

李嘉图是从资本主义生产的现有事实出发的。劳动的价值小于劳动所创造的产品的价值。因此,产品的价值大于生产产品的劳动的价值,或者说,大于工资的价值。产品的价值超过工资的价值的余额,就是剩余价值。(李嘉图错误地说成利润,前面已经指出,他在这里把利润和剩余价值等同起来了,而他说的实际上是后者。)在李嘉图看来,产品的价值大于工资的价值,这是事实。这个事实究竟是怎样产生的,仍然不清楚。整个工作日大于工作日中生产工资所需要的部分。为什么呢?李嘉图仍旧没有说明。因此,李嘉图错误地假定总工作日的量是固定的,并从这里直接得出了错误的结论。因此,李嘉图只能用生产必要生活资料的社会劳动的生产率的提高或降低来说明剩余价值的增加或减少。换句话说,李嘉图只知道相对剩余价值。

[656]显然,如果工人为生产他自己的生活资料(即同他自己的生活资料在价值上相等的商品)要用去他一整天工夫,那就不可能有任何剩余价值,因而也不可能有资本主义生产和雇佣劳动。为了使资本主义生产能够存在,社会劳动生产率就必须有相当的发展,使总工作日中除了再生产工资所必须的劳动时间以外还有余额,也就是说,要有或多或少的剩余劳动存在。但是,同样明显的是,如果说在劳动时间既定(工作日长度既定)的情况下,劳动生产率可以大不相同,那末,另一方面,在劳动生产率既定的情况下,劳动时间即工作日长度也可以大不相同。其次,很明显,如果说,为了使剩余劳动能够存在,必须以劳动生产率的一定发展水平为前提,那末,仅仅这种剩余劳动的可能性(就是说,那种必需的最低限度的劳动生产率的存在)还不能造成它的现实性。为此,必须事先强迫工人进行超过上述限度的劳动,而强迫工人这样做的就是资本。这一切在李嘉图著作中都没有谈到,而争取规定正常工作日的整个斗争却正是由此产生的。

在劳动的社会生产率发展的低级阶段,也就是说,在剩余劳动相对说来还少的那个阶段,靠别人劳动过活的阶级同劳动者的人数相比一般是小的。这个阶级随着劳动生产率的增长,也就是说,随着相对剩余价值的增长,可能大大(相对地说)增长起来。

其次,据认为,劳动的价值在同一国家的不同时期和在同一时期的不同国家是有很大变化的。但是,资本主义生产的故乡是温带。劳动的社会生产力可能很不发达,可是,正是在必要生活资料的生产上,一方面由于自然因素(如土地)富饶,另一方面由于居民消费水平极低(由于气候等),这一点能够得到补偿,例如在印度,这两种情况都存在。在社会的原始状态中,由于社会需要还不发展,最低限度的工资可能很少(从使用价值的数量来看),可是要花费许多劳动。但是,即使为生产最低限度的工资所必需的劳动只是中等的量,生产出来的剩余价值,虽然同工资(必要劳动时间)相比占很大的比例,就是说,即使剩余价值率很高,可是表现为使用价值,却同工资本身一样,还是极其微少的(相对地说)。

假定必要劳动时间=10小时,剩余劳动=2小时,整个工作日=12小时。如果必要劳动时间等于12小时,剩余劳动等于2+(2/5)小时,整个工作日是14+(2/5)小时,那末,生产出来的价值就大不相同。在第一种场合,生产出来的价值=12小时,在第二种场合,生产出来的价值等于14+(2/5)小时。剩余价值的绝对量也大不相同。在第一种场合,剩余价值等于2小时,在另一种场合,等于2+(2/5)小时。可是,剩余价值率,或者说,剩余劳动率,是相同的,因为2∶10=[2+(2/5)]∶12。如果在第二种场合花费的可变资本更多,那末它占有的剩余价值或剩余劳动也会更多。如果在后一种场合剩余劳动不是增加2/5小时,而是5/5小时,从而剩余劳动就等于3小时,而总工作日就等于15小时,那末,虽然必要劳动时间,或者说,最低限度的工资增加了,剩余价值率还是会提高(因为2∶10=1/5,而3∶12=1/4)。如果由于谷物等涨价而最低限度的工资从10小时增加到12小时,这两种情况就可能同时发生。可见,即使在这种场合,剩余价值率不仅能够保持不变,而且还能够和剩余价值量一起增长。

但是,我们假定,必要工资和以前一样等于10小时,剩余劳动等于2小时,其他一切条件不变(因此,这里完全不考虑不变资本的生产费用的减少)。如果工人现在多劳动2+(2/5)小时,其中2小时他自己占有,2/5小时成为剩余劳动,那末,在这种情况下,无论工资还是剩余价值都会同样增加,但是前者代表的量大于必要工资,或者说,必要劳动时间。

如果我们取一个既定量,把它分成两个部分,那末,很清楚,其中一个部分只有在另一个部分减少的情况下才能增加,反过来也是一样。但是,如果这是增长量(变量),情况就决不是这样。而工作日(在没有争得正常工作日以前)正是这样的一个增长量。如果是这种增长量,那末,两个部分都可以增长,或者以同样程度增长,或者以不同程度增长。一个部分的增长不是以另一个部分的减少为条件,反过来也是一样。这也就是工资和剩余价值两者就其交换价值来看能够同时增长,而在一定条件下甚至可能以同一程度增长的唯一情况。至于它们的使用价值,那是不言而喻的;[657]虽然劳动的价值,比如说,减少了,使用价值也可能增加。在1797年到1815年间,英国的谷物价格和名义工资都大大提高,那时,在正处于迅猛发展阶段的主要工业部门中,工作日长度也大大增加,我认为,这种情况阻止了利润率的降低(因为它阻止了剩余价值率的降低)。但是,在这种场合,正常工作日不管怎样都延长了,工人的正常寿命,也就是说,他的劳动能力的正常期限,也相应地缩短了。如果工作日的延长是经常的,就必然产生这种结果。如果这种延长只是暂时的,只是为了补偿暂时的工资涨价,那末,它除了在按劳动性质可能延长劳动时间的企业中阻止利润率降低以外,也可能不产生(妇女和儿童劳动除外)其他后果。(在农业中这种情况最少发生。)

李嘉图完全没有注意到这一点,因为他既不研究剩余价值的起源,也不研究绝对剩余价值,因而把工作日看作某种既定的量。可见,对于上述的情况,他所说的剩余价值和工资(他错误地说成利润和工资)就交换价值来看只能按反比例增加或减少这个规律是错误的。

我们假定,必要劳动时间不变,剩余劳动也不变。因而得出10+2;工作日=12小时,剩余价值=2小时;剩余价值率=1/5。

现在假定,必要劳动时间仍然不变;而剩余劳动从2小时增加到4小时。工作日便是10+4,即14小时;剩余价值=4小时;剩余价值率=4∶10=4/10=2/5。

在两种场合,必要劳动时间是一样的;但剩余价值在一种场合比另一种场合多一倍,工作日在第二种场合比第一种场合大六分之一。其次,虽然工资相等,生产出来的价值,根据各自耗费的劳动量,却大不相同;在第一种场合生产出来的价值等于12小时,在另一种场合=12+12/6=14。因此,那种认为假定工资相等(就价值来说,就必要劳动时间来说),两个商品所包含的剩余价值之比就等于两个商品所包含的劳动量之比的说法,是错误的。只有在正常工作日不变的情况下,这种说法才是正确的。

其次,假定由于劳动生产力提高,必要工资从10小时减到9小时(虽然它表现为所购买的使用价值仍然不变),剩余劳动时间也从2小时减到1+(4/5)小时(即9/5小时)。在这种情况下,10∶9=2∶[1+(4/5)]。因此,剩余劳动时间和必要劳动时间以同一比例减少。在两种场合,剩余价值率是一样的,因为2=10/5,1+(4/5)=9/5,而[1+(4/5)]∶9=2∶10。根据假定,用剩余价值可以买到的使用价值的量仍然不变。(但是,这一点只适用于作为必要生活资料的使用价值。)工作日从12小时减到10+(4/5)小时。在第二种场合生产出来的价值量小于第一种场合。尽管劳动量不等,剩余价值率在两种场合却是一样的。

在考察剩余价值时,我们把剩余价值和剩余价值率区别开来了。就一个工作日来看,剩余价值等于它所代表的绝对时数,比如说,2、3小时等。剩余价值率等于这一时数和构成必要劳动时间的时数之比。这个区别非常重要,因为它指明了工作日的不同长度。如果剩余价值等于2小时,那末,必要劳动时间为10小时,它就等于1/5,必要劳动时间为12小时,它就等于。在一种场合工作日是12小时,在另一种场合工作日是14小时。在第一种场合剩余价值率较大,而工人一天劳动的时数较少。在第二种场合剩余价值率较小,劳动力的价值较大,而工人一天劳动的时数较多。这里我们看到,在剩余价值不变(但工作日不等)时,剩余价值率可能不同。而在前面10∶2和9∶[1+(4/5)]的情况下,我们看到,在剩余价值率不变(但工作日不等)时,剩余价值本身可能不同(在一种情况下是2,在另一种情况下是1+(4/5))。

我在前面(第二章)曾经指出,在工作日既定(工作日的长度既定),必要劳动时间既定,因而剩余价值率既定的条件下,剩余价值量取决于由同一资本同时雇用的工人人数。\endnote{马克思指的是他的手稿第III本从第95b页上开始的一节,标题是《(2)绝对剩余价值》。马克思引的一段在这一节的(d)小节内,标题是“同一时间的工作日”(马克思手稿第102—104页)。——第466页。}这个论点本来是同义反复。因为,如果1工作日给我提供2小时的剩余劳动,那末,12工作日就给我提供24小时的剩余劳动,或者说,提供2剩余工作日。可是,在决定利润(利润等于剩余价值同预付资本之比,因而它取决于剩余价值的绝对量)时,这个论点具有极其重要的意义。这个论点所以具有重要意义,是因为量相等而有机构成不同的资本,使用的工人人数不等,因而生产的剩余价值就一定不等,也就是说,生产的利润就一定不等。在剩余价值率降低时利润可能提高,而在剩余价值率提高时利润可能降低,或者,如果剩余价值率的提高或降低由使用的工人人数的相反运动所抵销,利润可能不变。这里,我们一开始就看到,把剩余价值提高和降低的规律[658]与利润提高和降低的规律等同起来是极端错误的。如果仅仅考察单纯的剩余价值规律,那末,说在剩余价值率既定(以及工作日既定)时,剩余价值的绝对量取决于所使用的资本量,这似乎是同义反复。因为根据假定,这个资本量的增长和同时雇用的工人人数的增加是一回事,或者说,只是同一事实的不同表现。但是如果进而考察利润,在这里使用的总资本量和使用的工人人数对于同量资本来说是大不相同的,那末,就可以看出上述规律的重要性了。

李嘉图是从考察具有一定价值的商品即代表一定量劳动的商品出发的。而从这一点出发,绝对剩余价值和相对剩余价值似乎总是一致的。(这无论如何说明了他的考察方法的片面性,而且也符合他的整个研究方法——从由商品中包含的劳动时间决定的商品价值出发,然后研究工资、利润等在什么程度上影响这个价值。)但是,这是假象,因为这里不是单纯地谈商品,而是谈资本主义生产,谈作为资本的产物的商品。

假定有一笔资本使用一定数量的工人,比如说20人,而工资等于20镑。为了简便起见,我们假定固定资本等于零,就是说,完全不把它计算在内。假定这20个工人一天劳动12小时,把价值80镑的棉花纺成纱。如果1磅棉花值1先令,20磅棉花就值1镑,80镑=1600磅棉花。如果20个工人用12小时纺1600磅棉花,那末1小时就纺1600/12磅=133+(1/3)磅。因此,如果必要劳动时间等于10小时,剩余劳动时间就等于2小时,它提供266+(2/3)磅纱。1600磅纱的价值等于104镑;因为,如果10劳动小时=20镑,那末1劳动小时=2镑,2劳动小时=4镑;因此,12劳动小时=24镑。(80镑[原料价值]+24镑[新加劳动所创造的价值]=104镑)

但是,假定工人的剩余劳动时间等于4小时,那末他们的产品就等于8镑(我指的是工人创造的剩余价值,他们的产品实际上=28镑\endnote{马克思指由20个工人新创造的价值:这20个工人每一劳动小时创造价值2镑,14小时的工作日创造价值28镑。——第468页。})。总产品的价值是121+(1/3)镑,\endnote{总产品的价值包括转移到产品中去的价值(c)和新创造的价值(v+m)。因为马克思在这里撇开了固定资本,所以转移的价值只是原料的价值。在被考察的例子中,原料价值等于93+(1/3)镑(一小时把133+(1/3)磅棉花加工成纱,14小时加工1866+(2/3)磅;1磅棉花是1先令)。加上新创造的价值(28镑)便是121+(1/3)镑。——第468页。}而这121+(1/3)镑等于1866+(2/3)磅纱。因为生产条件不变,所以,1磅纱的价值仍然和过去一样;它包含的劳动时间仍然一样。根据假定,必要工资(它的价值、它所包含的劳动时间)也保持不变。

不论这1866+(2/3)磅纱是在第一种条件下还是在第二种条件下生产的,就是说,不论剩余劳动是2小时还是4小时,在两种情况下这些纱都具有同一价值。那就是,除了以前的1600磅棉花以外,多纺的266+(2/3)磅棉花值13+(1/3)镑。如果把它加到用于1600磅棉花的80镑上,就是93+(1/3)镑,而在两种情况下20个人的4追加劳动小时等于8镑。全部[新加]劳动就是28镑,因而1866+(2/3)磅纱的价值等于121+(1/3)镑。在两种情况下,工资都是一样的。1磅纱在两种情况下都值1+(3/10)先令。因为1磅棉花的价值=1先令,所以,在1磅纱中包含的新加劳动在两种情况下都是3/10先令,或3+(3/5)便士(或18/5便士)。

可是,在假定的条件下,每磅纱中价值和剩余价值之比是大不相同的。在第一种情况下,因为必要劳动=20镑,剩余劳动=4镑,或者说,前者=10小时,后者=2小时,剩余劳动和必要劳动之比是2∶10,或者说,1∶5。(同样可以说,4镑∶20镑=4/20=1/5。)因此,在这种情况下,[物化]在1磅纱中的[新加劳动]3+(3/5)便士中,包含着1/5无酬劳动=18/25便士,或72/25法寻,即2+(22/25)法寻。而在第二种情况下,必要劳动是20镑(10劳动小时),剩余劳动=8镑(4劳动小时)。剩余劳动和必要劳动之比是8∶20=8/20=4/10=2/5。因此,[物化]在1磅纱中的[新加劳动]3+(3/5)便士中,无酬劳动是这一数目的2/5,即5+(19/25)法寻,或1便士1+(19/25)法寻。[659]虽然在两种情况下1磅纱具有同一价值,并且在两种情况下支付同样的工资,但是1磅纱包含的剩余价值,在一种情况下比在另一种情况下多一倍。在作为产品的一定部分的单位商品中,劳动价值和剩余价值的比例,自然应当同全部产品中的比例一样。

在一种情况下[在12小时的工作日中把1866+(2/3)磅棉花纺成纱],用于棉花的预付资本等于93+(1/3)镑,而用于工资的预付资本是多少呢?这里,用于把1600磅棉花纺成纱的工资等于20镑,因而用于把追加的266+(2/3)磅棉花纺成纱的工资等于3+(1/3)镑。因此,工资总共用了23+(1/3)镑。而全部支出等于[不变]资本93+(1/3)镑+[23+(1/3)]镑=116+(2/3)镑。产品=121+(1/3)镑。这里[可变]资本的追加支出3+(1/3)镑只提供13+(1/3)先令(或2/3镑)的剩余价值。(20镑∶4镑=[3+(1/3)]镑∶(2/3)镑。)

相反,在另一种情况下[在14小时的工作日中把1866+(2/3)磅棉花纺成纱],预付资本只有93+(1/3)镑+20镑=113+(1/3)镑,而在4镑剩余价值上又加上了4镑。在两种情况下生产出来的纱的数量一样,纱的价值一样,就是说,这两种纱代表相等的劳动总量;但是这两个相等的劳动总量,虽然工资相同,却是由大小不等的两笔资本推动的;相反,工作日的长度是不等的,因而,无酬劳动的量就不同。就单独每一磅纱来考察,花费在它上面的工资的量,或者说,它所包含的有酬劳动的量,是不等的。这里,同样多的工资分配在较大量的商品上,不是因为劳动的生产率在一种情况下比在另一种情况下高,而是因为在一种情况下被推动的无酬剩余劳动总量比在另一种情况下大。因此,用同量的有酬劳动,在一种情况下生产的纱的磅数比在另一种情况下多,虽然在两种情况下总共都生产了等量的纱,两种等量的纱代表等量的总劳动(有酬劳动和无酬劳动)。相反,如果在第二种情况下劳动生产率提高的话,那末,无论如何(不论剩余价值同可变资本的比例怎样)1磅纱的价值都要下降。

因此,在这种情况下,如果说,因为1磅纱的价值是既定的,等于1先令3+(3/5)便士,其次,新加劳动的价值是既定的,等于3+(3/5)便士,并且因为根据假定,工资是相同的,即必要劳动时间不变,所以剩余价值一定会相同,两笔资本在其他条件相同的情况下生产出来的纱一定会带来相同的利润,那末,这种说法是错误的。如果谈的是1磅纱,那倒是对的,但这里谈的是生产了1866+(2/3)磅纱的一笔资本。为了要知道这笔资本从1磅纱中得到多大的利润(其实是剩余价值),我们必须知道工作日有多长,或者说,这笔资本(在生产率既定的情况下)推动的无酬劳动的量有多大。但是这从单位商品上是看不出来的。

可见,李嘉图只是研究了我称为相对剩余价值的东西。他是从工作日长度既定这一前提出发的(斯密和他的前辈似乎也是从这一前提出发的)。(至多,在斯密著作中提到过不同的劳动部门中工作日长度的差别,而这种差别已由劳动的较大强度、困难、使人厌恶的性质等抵销或补偿。)从这个前提出发,李嘉图总的说来正确地阐明了相对剩余价值。但是在谈到他的研究的主要论点之前,我们还要引几段引文来说明李嘉图的观点。

\begin{quote}{“工业中100万人的劳动总是生产出相同的价值,但并非总是生产出相同的财富。”(同上,第320页)}\end{quote}

这就是说,他们一天劳动的产品总是100万工作日的产品,包含同一劳动时间,而这种说法是错误的,或者,只有在考虑到不同劳动部门的不同困难程度等情况而普遍确立同一正常工作日的时候才可能是正确的。

可是,即使在这样的时候,在这里用一般形式表述出来的这个论点还是错误的。假定正常工作日等于12小时。假定一个人的年劳动产品用货币表示等于50镑,而且货币的价值不变。在这种情况下,100万人的劳动产品一年总是等于5000万镑。假定必要劳动等于6小时,那末用于这100万人的资本一年就等于2500万镑。剩余价值也等于2500万镑。不管工人得到2500万,3000万还是4000万,产品总是等于5000万。可是剩余价值在第一种情况下等于2500万,在第二种情况下等于2000万,在第三种情况下等于1000万。如果预付资本仅仅由可变资本组成,就是说,仅仅由用于这100万人的工资的资本组成,那末,李嘉图就对了。因此,他只有在一种情况下,即在全部资本等于可变资本的情况下才是对的,——在李嘉图的著作中就象在亚·斯密的著作中一样,这个前提贯穿着全部研究,[660]只要他谈的是整个社会的资本;但是,在资本主义生产条件下,无论在哪一个生产部门中,尤其在整个社会生产中,这种情况是不存在的。

加入劳动过程但不加入价值形成过程的那一部分不变资本,不加入产品(产品的价值),因此,在这里,当我们讲的是年产品的价值的时候,无论这一部分不变资本对决定一般利润率多么重要,它都是同我们无关的。加入年产品的那一部分不变资本,却是另外一种情况。我们已经看到,这部分不变资本中的一部分,或者说,在一个生产领域表现为不变资本的东西,在另一个生产领域的同一个生产年度却表现为劳动的直接产品;因而,一年花费的资本中有很大一部分,从单个资本家或特殊生产领域的角度来看表现为不变资本,而从整个社会或整个资本家阶级的角度来看却归结为可变资本。因此,这一部分是包含在前面所说的5000万之内的,是包含在5000万中构成可变资本或用于工资的部分之内的。

但是,对于为了补偿工农业中已消费的不变资本而消费的那一部分不变资本,对于生产不变资本——最初形式的原料、固定资本和辅助材料——的生产部门使用的不变资本已消费的部分,却是另外一种情况。这一部分不变资本的价值会在产品中重新表现出来,会在产品中被再生产出来。这一部分的价值以什么比例加入全部产品的价值,完全取决于它现有的量(假定劳动生产率保持不变;但是不管劳动生产率怎样变动,这一部分的价值总是具有一定的量)。(如果不把农业中的某些例外计算在内,平均说来,就连产品的量,即100万人生产出来的、李嘉图认为和价值不同的那个财富的量,当然也取决于这个作为生产前提的不变资本的量。)如果没有100万人的新的年劳动,产品的这一部分价值就不会存在。另一方面,如果没有这个不以他们的年劳动为转移而存在的不变资本,100万人的劳动就不能提供同一产品量。这个不变资本作为生产条件加入劳动过程,但是为了把全部年产品的这一部分价值再生产出来,无须再花费哪怕是一小时的劳动。因此,作为价值,这一部分不是年劳动的结果,虽然没有这一年劳动它的价值就不能在产品中再生产出来。

假定加入产品的那部分不变资本等于2500万,那末,100万人的产品的价值就等于7500万;如果前者等于1000万,后者就只等于6000万,依此类推。因为在资本主义发展过程中,不变资本对可变资本的比例在增长,所以100万人的年产品的价值,就有同作为因素参加100万人一年生产活动的过去劳动的增长成比例地不断增长的趋势。从这里已经可以看到,李嘉图既不能理解积累的实质,也不能理解利润的本质。

随着不变资本对可变资本的比例的增长,劳动生产率也增长,由人生产出来的、社会劳动借以发挥作用的生产力也增长。诚然,由于劳动生产率的这一增长,现有不变资本的一部分将不断贬值,因为它的价值不是决定于它原先已花费的劳动时间,而是决定于把它再生产出来所必须花费的劳动时间,而这种劳动时间随着劳动生产率的增长会不断减少。因此,不变资本的价值虽然不是同它的量成比例地增长,但毕竟是在增长,因为不变资本的量的增长比它的价值的减少快。不过,关于李嘉图的积累观点,我们到后面再谈。

无论如何,这里已经很明显,在工作日既定的情况下,100万人年劳动的全部产品的价值,将根据加入产品的不变资本的量的不同而大不相同,尽管劳动生产率在增长,这个价值在不变资本构成总资本的很大部分的地方,比在不变资本构成总资本的较小部分的社会条件下要大。因此,随着社会劳动生产率的进步以及实际上与它同时发生的不变资本的增长,资本本身的份额,在劳动的全部年产品中相对说来将占越来越大的部分,因而作为资本的财产(且不说资本家的收入)将不断增大,单个工人甚至整个工人阶级[的新加劳动]所创造的那部分价值所占的份额,[661]与现在作为资本同他们对立的他们的过去劳动的产品相比,将越来越减少。因此,劳动能力和作为资本而独立存在的劳动的客观条件之间的分离和对立不断增长。(我们这里不谈可变资本,即年劳动的产品中为再生产工人阶级所必需的那一部分;但是,就连工人阶级的这些生存资料本身也是作为资本同他们对立的。)

李嘉图把工作日看作既定的、有限的、固定的量的观点,在他书中的其他地方也谈到过,例如

\begin{quote}{“它们〈工资和资本的利润〉加在一起总是具有同一价值”。(同上,第499页,第32章《马尔萨斯先生的地租观点》)}\end{quote}

换言之,这只不过是说:劳动时间(工作日)——其产品在工资和资本的利润之间进行分配——总是同一的,是不变的量。

\begin{quote}{“工资和利润加在一起具有同一价值。”(第491页注)}\end{quote}

这里我无需重复,利润在这些地方都应读作剩余价值。

\begin{quote}{“工资和利润加在一起将总是同一价值。”(第490—491页)“工资应当按照它的实际价值计算,就是说,按照生产工资时使用的劳动和资本的量计算,不应按照它用衣服、帽子、货币或谷物来表示的名义价值计算。”(第1章《论价值》,第50页)}\end{quote}

工人得到的(他用自己的工资购买的)生活资料(谷物、衣服等)的价值,决定于生产它们所需要的总劳动时间,这里既包括生产它们所必需的直接劳动量,也包括生产它们所必需的物化劳动量。但是,李嘉图把问题搞糊涂了,因为他没有把问题表达清楚,他不是说:“它的(工资的)实际价值,即工作日中为再生产他(工人)自己的必要生活资料的价值、为再生产以工资形式支付给他,或者说,用以交换他的劳动的必要生活资料的等价所需要的那一部分”。“实际工资”应由工人为生产或再生产他自己的工资在一天中必须劳动的平均时间来决定。[而李嘉图做了这样的表述:]

\begin{quote}{“工人只有在用他的工资能买到大量劳动的产品时,他的劳动才是得到真正高的价格。”(第322页)}\end{quote}

\tsubsectionnonum{(4)相对剩余价值。[对相对工资的分析是李嘉图的科学功绩]}

相对剩余价值——这实际上是李嘉图在利润名义下研究的剩余价值的唯一形式。

为生产商品所需要的并包含在商品中的劳动量,决定商品的价值,因而商品的价值是一个既定的、一定的量。这个量在雇佣工人和资本家之间分配。(李嘉图同斯密一样在这里没有考虑不变资本。)很明显,一个分配者的份额的增加或减少,只能同另一个分配者的份额的减少或增加成比例。既然商品的价值全靠工人的劳动来创造,那末在任何情况下都要有这种劳动本身作为前提,但是工人必须活着,维持自己的生命,也就是说,必须得到必要工资(与劳动能力价值相等的最低限度的工资),否则就不可能有这种劳动。因此,工资和剩余价值——照李嘉图看来,商品的价值或产品本身分成这两个范畴——不仅彼此成反比,而且最初的、决定性的因素是工资的变动。工资的提高或降低引起利润(剩余价值)方面的相反的运动。工资提高或降低,不是因为利润(剩余价值)降低或提高,相反,因为工资提高或降低,剩余价值(利润)才降低或提高。工人阶级从自己的劳动所创造的年产品中取得他自己的份额以后剩下的剩余产品(其实应当说剩余价值),成为资本家阶级赖以生活的实体。

既然商品的价值决定于商品包含的劳动量,而工资和剩余价值(利润)只不过是两个生产者阶级彼此之间分配商品价值的份额,比例,那末,很明显,工资的提高或降低虽然决定剩余价值率(在李嘉图那里是利润率),但是并不影响商品的价值或价格(商品价值的货币表现)。一个整体在两个分配者之间进行分配的比例,既不会使这个整体本身变大,也不会使它变小。因此,认为工资的提高会提高商品的价格的看法,是一种错误的成见;工资提高只能使利润(剩余价值)降低。甚至李嘉图所举的一些例外情况即工资提高似乎会引起一些商品的交换价值降低,并引起另一些商品的交换价值提高——如果说的是价值,那也是错误的,只有对费用价格来说才是正确的。

[662]既然剩余价值(利润)率决定于工资的相对高度,那末后者又是由什么决定的呢?如果撇开竞争不谈,工资决定于必要生活资料的价格。必要生活资料的价格又取决于劳动生产率,而土地越肥沃,劳动生产率就越高(这里,李嘉图假定是资本主义生产)。每一种“改良”都使商品、生活资料的价格降低。因此,工资,或者说,“劳动的价值”的提高或降低同劳动生产力的发展成反比,只要这一劳动所生产的是加入工人阶级日常消费的必要生活资料。因此,剩余价值(利润)率的降低或提高同劳动生产力的发展成正比,因为这种发展使工资降低或提高。

工资不提高,利润(剩余价值)率就不可能降低;工资不降低,利润率就不可能提高。

工资的价值不是按照工人得到的生活资料的量来计算的,而是按照这些生活资料所耗费的劳动量(实际上就是工人自己占有的那部分工作日),按照工人从总产品中,或者更确切地说,从这个产品的总价值中得到的比例部分来计算的。可能有这种情况,工人的工资用使用价值(一定量的商品或货币)来衡量,是提高了(在劳动生产率提高的情况下),可是按价值却降低了,也可能有相反的情况。分析相对工资,或者说,比例工资,并把它作为范畴确定下来,是李嘉图的巨大功绩之一。在李嘉图以前,始终只对工资作了简单的考察,因而工人被看作牲畜。而这里工人是被放在他的社会关系中来考察的。阶级和阶级相互之间的状况,与其说决定于工资的绝对量,不如说更多地决定于比例工资。

现在从李嘉图著作中引几段话,以证实前面所表述的论点。

\begin{quote}{“猎人一天劳动的产品鹿的价值恰好等于渔夫一天劳动的产品鱼的价值。不管产量多少,也不管普通工资或利润的高低,鱼和野味的比较价值完全由它们自身包含的劳动量决定。如果……渔夫……雇用10个人,他们的年劳动值100镑,他们劳动一天可捕得鲑鱼20条;如果……猎人也雇用10个人,他们的年劳动值100镑,他们一天为他捕鹿10只;那末,不论全部产品中归捕获者的份额是多少,一只鹿的自然价格是两条鲑鱼。用来支付工资的份额对利润问题是极为重要的;因为一眼就可以看出,利润的高低恰好同工资的高低成反比;但是这丝毫不会影响鱼和野味的相对价值,因为在这两个行业中,工资要高就会同时都高,要低就会同时都低。”(第1章《论价值》,第20—21页)}\end{quote}

我们看到,李嘉图从被雇用者的劳动中得出商品的全部价值。在被雇用者和资本之间分配的,就是被雇用者自己的劳动,或者说,这一劳动的产品,或者说,这一产品的价值。

\begin{quote}{“工资的任何变动不可能引起这些商品的相对价值的变动,因为,假定工资提高,这两个行业中的任何一个并不因此就需要更大的劳动量,虽然对这一劳动将支付更高的价格……工资可能提高百分之二十,因此利润会以或大或小的幅度降低,但这决不会使这些商品的相对价值发生丝毫变动。”(同上,第23页)“劳动的价值提高,利润就不能不降低。如果把谷物在租地农场主和工人之间分配,后者得到的份额越大,留给前者的就越小。同样,如果把呢绒和棉织品在工人和雇主之间分配,分给前者的份额越大,留给后者的就越小。”(同上,第31页)[663]“亚当·斯密和一切追随他的著作家,据我所知,无一例外地都认为,劳动价格的上涨,必然会引起一切商品价格的上涨。我希望,我已成功地证明了这种意见是毫无根据的。”(同上,第45页)“工资的提高,如果是由于工人得到比较优厚的报酬,或者由于那些用工资购买的必需品的生产发生困难,那末,除了某些情况以外,不会引起价格的提高,但对于利润降低却有很大的影响。”如果工资的提高是由“货币价值的变动”引起的,那是另一回事。“在一种场合{即上述的后一种场合},国家的年劳动中并没有花费更大的份额来维持工人生活,在另一种场合,却花费了更大的份额。”(同上,第48页)[663]“随着食物和其他必需品价格的上涨,劳动的自然价格也上涨;随着这些东西的价格下降,劳动的自然价格也下降。”(同上,第86页)“现有人口的需要满足之后剩下来的剩余产品,必然同生产的容易程度成比例,也就是说,从事生产的人数越少,剩余产品就越多。”(第93页)“不论是耕种调节价格的那一等级土地的租地农场主,还是生产工业品的工厂主,都没有牺牲自己产品的任何部分来支付地租。他们的商品的全部价值只分成两部分:一部分构成资本的利润,另一部分构成工资。”(第107页)“假定丝绸、天鹅绒、家具以及其他任何不是工人需要的商品由于所费劳动增加而涨价,这会不会影响利润呢?当然不会。因为只有工资提高才能影响利润;丝绸和天鹅绒不为工人所消费,所以它们价格的上涨就不能提高工资。”(第118页)“如果10个工人的劳动在一定质量的土地上可以获得小麦180夸特,每夸特价值4镑,共计720镑(第110页)……在任何情况下,这720镑都必定分成工资和利润……不论工资或利润是提高还是降低,这两者都必定由720镑这个总额中提供。一方面,利润决不能提高到从这720镑中取出那样大一部分,以致余数不足以给工人提供绝对必需品;另一方面,工资决不能提高到使这个总额不剩下一部分作为利润。”(第113页)“利润取决于工资的高低,工资取决于必要生活资料的价格,而必要生活资料的价格主要取决于食物的价格,因为其他一切必需品的数量是可以几乎无止境地增加的。”(第119页)“虽然生产了一个较大的价值〈在土地变坏的情况下〉,但这一价值在支付地租以后剩下的部分中却有较大的份额是由生产者消费的{李嘉图在这里把工人和生产者等同起来了},而这一点,并且只有这一点,却调节着利润的大小。”(第127页)“改良的实质就是使生产商品所需要的劳动量比以前减少;而这种减少不能不使商品的价格,或者说,相对价值下降。”(第70页)“如果减少帽子的生产费用,尽管对帽子的需求增加一倍、两倍或者三倍,帽子的价格最后总要降到其新的自然价格的水平。如果降低维持人的生活的食物和衣服的自然价格,从而减少人所必需的生存资料的生产费用,尽管对工人的需求[664]可能大大增加,工资最后还是会降低。”(第460页)“工资分得的份额越小,利润分得的份额就越大,反过来也是一样。”(第500页)“本书的目的之一就是要说明,必需品的实际价值每有降低,工资也就降低,而资本利润则提高;换句话说,在任何一定的年价值中,归工人阶级所得的份额会减少,而归用基金使用这个阶级的人所得的份额会增加。”}\end{quote}

{只是在最后这句非常通俗的话里,李嘉图即使没有猜到,但毕竟说出了资本的本质。不是积累的劳动被工人阶级,被工人自己使用,而是“基金”,“积累的劳动”“使用这个阶级”,使用现在的、直接的劳动。}

\begin{quote}{“假定某工厂生产的商品价值为1000镑,这一价值在老板和他的工人之间分配{这句话又反映了资本的本质;资本家是老板,工人是他的工人},工人得800镑,老板得200镑;如果这些商品的价值降到900镑,同时由于必需品降价在工资上节省了100镑,那末,老板的纯收入丝毫不会减少。”(第511—512页)“如果由于机器改良,生产供工人穿的鞋子和衣服所需要的劳动量只等于现在的四分之一,那末这些东西的价格也许会降低75%;但是,绝不能由此得出结论说,工人因此就可以不再只消费一件上衣和一双鞋子,而可以经常消费四件上衣和四双鞋子了;由于竞争的影响和人口急剧增加的刺激,他的工资也许不久就会同用工资购买的各种必需品的新价值相适应。如果这种改良推广到工人的一切消费品,大概过了几年以后我们就会看到,虽然与任何其他商品相比这些商品的交换价值已经大大降低,虽然这些商品现在是已经大大减少了的劳动量的产品,但工人的消费量即使有所增加也是十分有限的。”(第8页)“工资的增加,总是靠减少利润,工资降低时,利润总是提高。”(第491页注)“在本书中,我始终力图证明:工资不降低,利润率就决不会提高;用工资购买的各种必需品不跌价,工资就不能持久降低。因此,如果由于对外贸易的扩大或机器的改良,工人消费的食物和其他必需品能按较低廉的价格进入市场,利润就会提高。如果我们不自己种植谷物,不自己制造工人所用的衣服和其他必需品,而是发现一个新的市场,可以用较低廉的价格从那里取得这些商品,工资就会降低,利润就会提高;但是,如果由于对外贸易的扩大或机器的改良而以较低廉的价格取得的商品仅仅是供富人消费的商品,利润率就不会发生任何变动。葡萄酒、天鹅绒、丝绸及其他贵重商品的价格即使降低50%,工资率也不会受到影响,因而利润也会保持不变。所以,对外贸易虽然对国家极为有利,因为它增加了用收入购买的物品的数量和种类,并且由于商品丰富和价格低廉而为节约和资本积累提供刺激{为什么不是为浪费提供刺激?},但是,如果进口的商品不属于用工人工资购买的那一类商品,就根本没有提高资本利润的趋势。以上关于对外贸易的看法同样适用于国内贸易。利润率决不会由于分工的改进、机器的发明、道路和运河的兴修或者在商品制造或运输上采用任何其他节约劳动的方法而提高。”}\end{quote}

{李嘉图刚刚讲过完全相反的话;他的意思显然是说,除非由于上述改良减少了劳动的价值,否则利润率决不会提高。}

\begin{quote}{“所有这些原因都影响商品价格,总是对消费者极为有利,因为它们使消费者能够用同样的劳动换得更多的在生产上实行了改良的商品;但是它们对于利润绝对没有任何影响。另一方面,[665]工人工资的任何降低都使利润提高,但是对于商品价格毫无影响。前一种情况对一切阶级都有利,因为一切阶级都是消费者”}\end{quote}

{但是,这怎么会对工人阶级有利呢?因为根据李嘉图的假定,如果这些商品属于用工资购买的物品,它们会使工资降低,如果它们的减价不会使工资降低,它们就不属于用工资购买的物品};

\begin{quote}{“后一种情况只对生产者有利;他们会得到更多的利润,但一切物品的价格仍旧不变”}\end{quote}

{这又怎么可能呢?因为根据李嘉图的假定,使利润提高的工人工资的降低,正是因为必要生活资料价格降低才发生的,因此决不能说“一切物品的价格仍旧不变”}。

\begin{quote}{“在前一种情况下,他们得到的数额同以前一样,但是用他们的所得来购买的一切物品〈这又错了;应该说除了必要生活资料以外的一切物品〉的交换价值减少了。”(第137—138页)}\end{quote}

我们看到,整个这一段都写得极为草率。但是,撇开这些形式上的缺点不谈,这一切,就象在整个关于相对剩余价值的这种研究中一样,只有在把“利润率”读成“剩余价值率”的情况下才是正确的。即使对奢侈品来说,上述改良也可以提高一般利润率,因为这些生产领域的利润率,同其他一切生产领域的利润率一样,也参加一切特殊利润率平均化为平均利润率的过程。如果在这种情况下,由于上述种种影响,不变资本的价值同可变资本相比降低了,或者周转时间的长度缩短了(就是说,流通过程有了变化),那末,利润率就会提高。其次,李嘉图在这里对于对外贸易的影响作了非常片面的解释。对于资本主义生产来说,非常重要的是产品发展成为商品,而这同市场的扩大,同世界市场的建立,因而同对外贸易,有极为重要的联系。

如果撇开这一点不谈,李嘉图倒是提出了一个正确的原理,就是说:一切不论是由分工、机器的改进、运输工具的完善还是由对外贸易引起的改良,一句话,一切缩短制造和运输商品的必要劳动时间的方法,由于并且只要它们降低劳动的价值,都会增加剩余价值(就是说,也会增加利润),从而使资本家阶级发财致富。

最后,我们在这一节里还必须引用李嘉图阐明相对工资的本质的几段话。

\begin{quote}{“如果我必须雇用一个工人劳动一星期,我不是付给他10先令而是付给他8先令,而货币的价值并没有发生任何变动,这个工人现在用8先令买到的食物和其他必需品,可能比以前用10先令买到的还多。但是,这不是象亚·斯密和最近马尔萨斯先生所说的那样,由于他的工资的实际价值提高了,而是由于工人用他的工资购买的那些物品的价值降低了,这是完全不同的两回事。但是,当我把这叫做工资的实际价值降低时,有人却说我使用了同这门科学的真正原理不相容的新奇说法。”(同上,第11—12页)“要正确地判断利润率、地租率和工资率,我们不应当根据任何一个阶级所获得的产品的绝对量,而应当根据获得这一产品所需要的劳动量。由于机器和农业的改良,全部产品可能加倍;但是,如果工资、地租和利润也增加一倍,那末三者之间的比例仍然和以前一样,其中任何一项也不能说有了相对的变动。但是,如果工资没有如数增加,如果它不是增加一倍而只增加一半……那末,在这种情况下,我认为,说……工资已经降低而利润已经提高,那是对的;因为,如果我们有一个衡量产品价值的不变的标准,我们就会发现,现在归工人阶级……所得的价值比以前少了,而归资本家阶级所得的价值比以前多了。”(第49页)“工资的这种降低,仍然是真正的降低,尽管它〈工资〉现在能够为工人提供廉价商品的量比他以前的工资所提供的还多。”(第51页)}\end{quote}

\centerbox{※     ※     ※}

德·昆西指出了李嘉图所发挥的一些论点,并把它们同其他经济学家的观点作了对照。

\begin{quote}{李嘉图以前的经济学家的情况是这样的:“当有人问他们究竟是什么决定一切商品的价值的时候,回答是:价值主要由工资决定。再问:究竟是什么决定工资?他们就会指出,工资必须同用它购买的商品的价值相适应;这个回答实际上就是说,工资由商品的价值决定。”(《三位法学家关于政治经济学的对话,主要是关于李嘉图先生的〈原理〉》,[666]载于1824年《伦敦杂志》第9卷第560页)}\end{quote}

就在这个《对话》中,谈到用劳动量衡量价值的规律和用劳动价值衡量价值的规律:

\begin{quote}{“这两个公式决不能认为仅仅是同一规律的两个不同表现,李嘉图先生的规律(即关于A的价值和B的价值之比等于生产它们的劳动量之比的论点)用否定式来表达,最好是说:A的价值和B的价值之比不等于生产A的劳动的价值和生产B的劳动的价值之比。”[同上,第348页]}\end{quote}

(如果在A和B两个部门中资本的有机构成相同,那末,确实可以说,这两种资本的产品价值之比等于生产这两种产品的劳动的价值之比。因为在这种情况下,产品A和产品B中包含的积累劳动量之比等于这两种产品包含的直接劳动量之比。两个部门的有酬劳动量之比等于这两种资本所使用的直接劳动总量之比。假定两种资本的构成都是80c+20v,剩余价值率都等于50%。如果一笔资本等于500,而另一笔等于300,那末前者的产品等于550,而后者的产品等于330。于是,两种产品之比也等于工资5×20(即100)和工资3×20(即60)之比。因为100∶60=10∶6=5∶3。产品A和产品B的价值之比是550∶330,或55∶33,或(55/11)∶(33/11)(因为5×11=55,3×11=33),因而等于5∶3。但是,即使在这种情况下,也只是知道它们之间的比例,而不知道所考察的两种产品的实际价值,因为各种极不相同的价值量都可以符合于5∶3这一比例。)

\begin{quote}{“如果产品的价格为10先令,那末工资和利润加在一起就不能超过10先令。但是,难道不是恰好相反,是工资加利润决定价格吗?不,那是陈旧的、过时的学说。”(托·德·昆西《政治经济学逻辑》1844年爱丁堡和伦敦版第204页)“新的政治经济学证明,任何商品的价格都由、并且仅仅由生产该商品的劳动的相对量决定。既然价格本身已经决定,价格也就决定那个无论工资还是利润都必须从中取得自己的特殊份额的基金。”(同上)“凡是可能破坏工资和利润之间的现有比例的变动,必定从工资中发生。”(同上,第205页)“李嘉图给地租学说增添了新的东西:他把地租学说归结为地租是否真的取消了价值规律的问题。”(同上,第158页)}\end{quote}

\tchapternonum{[第十六章]李嘉图的利润理论}

\tsectionnonum{[(1)李嘉图把利润和剩余价值区别开来的个别场合]}

已经详细证明:剩余价值规律,或者更确切地说,剩余价值率规律(假定工作日既定),不是象李嘉图所解释的那样,直接地、简单地同利润规律相一致,或者说,可以直接地、简单地适用于利润规律;李嘉图错误地把剩余价值和利润等同起来;只有在全部资本都由可变资本组成,或者说,全部资本都直接用于工资的场合,剩余价值和利润才是等同的;因此,李嘉图在“利润”名义下考察的,一般说来只是剩余价值。也只有在上述这种场合,总产品才会简单地归结为工资和剩余价值。李嘉图显然同意斯密关于年产品的总价值归结为收入的观点。因此,他也就把价值和费用价格混淆起来了。

这里没有必要重复说,利润率不是由支配剩余价值率的那些规律直接支配的。

第一,我们已经看到,利润率可能由于地租的降低或提高而提高或降低,同劳动价值的任何变动无关。

第二,利润的绝对量等于剩余价值的绝对量。但是,后者不仅决定于剩余价值率,而且决定于所使用的工人人数。因此在剩余价值率降低而工人人数增加的情况下,利润量可能不变,反过来也是一样。

第三,在剩余价值率既定的条件下,利润率取决于资本的有机构成。

第四,在剩余价值既定(从而假定每100单位的资本的有机构成也既定)的条件下,利润率取决于资本的不同部分的价值比例,这些不同部分可能由于不同原因而发生变动:部分地由于使用生产条件时节省了力等等;部分地由于价值变动,这种价值变动可能影响资本的一部分而不影响资本的其他部分。

最后,还要考虑到从流通过程产生的资本构成的差别。

[667]从李嘉图著作中已经隐约透露出来的一些想法,本来应该促使他把剩余价值和利润区别开来。由于他没有这样作,看来,——正如在分析第一章(《论价值》)时已经指出的,——他在有些地方就滑到认为利润只是商品价值的附加额这样一种庸俗观点上去了;例如,他在谈到固定资本占优势的资本的利润如何决定等等的时候就是如此\fnote{见本册第199—200页。——编者注}。他的追随者们的极其荒谬的言论,就是从这里产生的。说平均起来等量资本提供等量利润,或者说,利润取决于所使用的资本量,这个论点实际上是正确的,但是,如果不用一系列中介环节把它同一般价值规律等联系起来,简而言之,如果把利润和剩余价值等同起来(这只有对全部资本来说才是正确的),那末,就必然会产生庸俗观点。正因为如此,李嘉图才没有找到确定一般利润率的任何途径。

李嘉图懂得,商品价值的变动如果象货币价值的变动那样以同一程度影响资本的一切部分,这种变动就不影响利润率。李嘉图本来应该由此得出结论说,商品价值的变动如果不是以同一程度影响资本的一切部分,这种变动就影响利润率;因此,在劳动价值不变的情况下,利润率可能变动,甚至可能朝着同劳动价值变动相反的方向变动。但是,首先他必须注意到,剩余产品,或者在他看来也可以说,剩余价值,或者他还可以说,剩余劳动,只要他是从利润角度来考察的,他就不能单单按它同可变资本的比例来计算,而要按它同全部预付资本的比例来计算。

关于货币价值的变动,他说:

\begin{quote}{“不论货币价值的变动有多大,它都不会引起利润率的任何变动;因为,假定工厂主的商品价格从1000镑上涨到2000镑,即涨价100%,如果他的资本(货币价值的变动对他的资本和对产品的价值所起的影响是一样的),如果他的机器、建筑物和商品储备同样涨价100%,他的利润率将照旧不变……如果工厂主用一定价值的资本,通过节约劳动的办法,能使产品数量增加一倍,而产品价格下跌到原先价格的一半,那末产品同生产产品的资本的比例将照旧不变,因而利润率也将照旧不变。如果在他用同一资本使产品量增加一倍的同时,货币价值由于某种原因降低一半,产品就将按两倍于以前的货币价值出卖;但是用来生产这种产品的资本也将具有两倍于以前的货币价值;因此,在这种情况下,产品价值同资本价值的比例也将照旧不变。”(同上,第51—52页)}\end{quote}

如果李嘉图这里说的产品是指剩余产品,那是对的。因为利润率=剩余产品(剩余价值)/资本。所以,如果剩余产品=10,资本=100,那末利润率=10/100=1/10=10%。但是,如果他指的是全部产品,问题就说得不确切。那样的话,李嘉图所谓产品价值同资本价值的比例,显然是指商品价值超过预付资本价值的余额。无论如何可以看出,李嘉图在这里没有把利润同剩余价值等同起来,没有把利润率同剩余价值率(等于剩余价值/劳动价值,或者说,剩余价值/可变资本)等同起来。

李嘉图说:

\begin{quote}{“假定用来制造某些商品的原产品跌价,这些商品也将因此跌价。不错,这些商品将跌价,但是随着商品的跌价,生产者的货币收入并不会有任何减少。如果他卖出商品得到的货币减少,那只不过是因为用来制造商品的材料之一的价值已经减少。如果毛织厂主卖出呢绒不是得到1000镑,而是只得到900镑,那末,在用来纺织呢绒的羊毛的价值降低了100镑的情况下,他的收入仍然不会减少。”(同上,第32章第518页)}\end{quote}

(其实,李嘉图在这里讲的问题——原产品价值下降在某一实际场合的影响——同我们这里毫无关系。羊毛价值突然下降对于那些毛织厂主的货币收入当然有(不利)影响,因为他们在仓库里存有大批成品,这些成品是在羊毛昂贵的时候制造的,却要在羊毛价值[668]下降之后拿去出卖。)

按照李嘉图这里的假定,如果毛织厂主推动的劳动量同以前一样{其实他们可以推动更大的劳动量,因为一部分游离出来的、以前仅仅用于原料的资本,现在就可以用于原料加劳动了},那末,很明显,这些工厂主的“货币收入”按其绝对量来说“不会减少”,而他们的利润率却比以前增长了;因为同以前一样的那个量,比如说10%,即100镑,现在就不是按1000镑,而是按900镑计算。在第一种情况下利润率是10%,在第二种情况下就等于1/9,即[11+(1/9)]%。何况李嘉图还假定用来制造商品的原产品普遍跌价,那就不仅仅是某一个别生产部门的利润率,而且一般利润率都会提高。李嘉图不理解这一点是令人奇怪的,尤其是因为相反的情况他倒理解。

那就是,李嘉图在第六章(《论利润》)中考察了这样的情况:由于必需品涨价(因为较坏的土地投入耕种从而使级差地租提高),第一,工资会提高,第二,一切地面上的原产品的价格会上涨。(这个假定决不是必然的。虽然谷物涨价,棉花、丝,甚至羊毛和亚麻却完全可能跌价。)

第一,李嘉图说,租地农场主的剩余价值(即他所说的利润)将降低,因为他雇用的10个工人劳动的产品的价值仍然等于720镑,他必须从这个720镑的基金中拿出较大的一部分作为工资。李嘉图接着说:

\begin{quote}{“但是,利润率会降低得更多,因为……租地农场主的资本在很大程度上是由原产品,例如他的谷物、干草、未脱粒的小麦和大麦、马和牛等组成的,这一切都将因产品涨价而涨价。他的绝对利润将从480镑降到445镑15先令;但是,如果由于我刚才说的原因,他的资本从3000镑增加到3200镑,那末在谷物价格是5镑2先令10便士时,他的利润率将会降到14%以下。如果一个工厂主在他的企业中也投资3000镑,由于工资增加,他要继续经营这一企业,就不得不增加资本。如果他的商品以前卖720镑,现在他仍然要按同样的价格出卖;但是,工资原来是240镑,当谷物价格是5镑2先令10便士时就会增加到274镑5先令。在第一种情况下,他还剩下480镑作为3000镑资本的利润,在第二种情况下,他的资本增加了,而利润却只有445镑15先令。因此,他的利润会同租地农场主的已经变动了的利润率相一致。”(同上,第116—117页)}\end{quote}

可见,李嘉图这里把绝对利润(即剩余价值)和利润率区别开来了,并且指出,利润率由于预付资本价值变动而降低的幅度大于绝对利润(剩余价值)由于劳动价值提高而降低的幅度。这里,即使劳动价值保持不变,利润率也要降低,因为同一绝对利润要按更大的资本来计算。因此,在前面引用的他的例子即原产品价值降低的例子中,就会发生一个相反的情况,就是利润率提高(它不同于剩余价值的提高,或者说,不同于绝对利润的提高)。因此,这就说明,利润率的提高和降低,除了决定于绝对利润的增减和按用于工资的资本计算的绝对利润率的提高和降低以外,还决定于其他条件。

李嘉图在刚才引证的那个地方接着说:

\begin{quote}{“珠宝制品、铁器、银器和铜器不会涨价,因为它们的成分中没有地面上的原产品。”(同上,第117页)}\end{quote}

这些商品的价格不会上涨,但是,这些部门的利润率会高于其他部门的利润率。因为在其他部门,(由于工资增加而)减少了的剩余价值是同由于双重原因——第一,工资支出增加,第二,原料支出增加——而增大了价值的预付资本相比。在第二种情况下[即生产珠宝制品等的情况下],[669]减少了的剩余价值是同由于工资增加而只增大了可变部分的预付资本相比。

在这几个地方,李嘉图自己推翻了他的以错误地把剩余价值率和利润率等同起来作为基础的整个利润理论。

\begin{quote}{“所以,在任何情况下,如果随着原产品价格上涨工资也同时增加,农业利润和工业利润都会降低。”(第113—114页)}\end{quote}

从李嘉图自己所说的话中可以得出结论,在原产品价格上涨时,即使工资不随之增加,由于由原产品组成的那部分预付资本涨价,利润率也会降低。

\begin{quote}{“假定丝绸、天鹅绒、家具以及其他任何不是工人需要的商品由于所费劳动增加而涨价,这会不会影响利润呢?当然不会。因为只有工资提高才能影响利润;丝绸和天鹅绒不为工人所消费,所以它们价格的上涨就不能提高工资。”(同上,第118页)}\end{quote}

这些特殊部门的利润率当然要降低,尽管劳动价值——工资——保持不变。丝织厂主、钢琴厂主和家具厂主等的原料会变贵,因此保持不变的剩余价值同支出的资本的比例会降低,从而利润率会降低。而一般利润率是所有工业部门的特殊利润率的平均比率。或者,上述那些工厂主会提高他们的商品的价格,以便象以前那样获得平均利润率。价格的这种名义上的提高并不直接影响利润率,但是影响利润的支出。

李嘉图再次回到前面考察的情况:剩余价值(绝对利润)降低是因为必要生活资料的价格(以及地租)提高。

\begin{quote}{“我必须再次指出,利润率的降低比我在计算中假定的要迅速得多;因为如果产品的价值象我在前面假定的情况下说过的那样高,租地农场主的资本的价值就会大大增加,因为他的资本必须由许多价值已经增加的商品组成。在谷物价格可能从4镑上涨到12镑以前,租地农场主的资本的交换价值也许就已经增加一倍,等于6000镑而不是3000镑了。因此,如果租地农场主的利润原来是180镑,或者说,是他原有资本的6%,那末现在实际利润率不会高于3%;因为6000镑的3%是180镑,而且一个持有6000镑的新租地农场主要经营农业,就只有接受这种条件。许多行业都会从这里得到或大或小的好处。啤酒业者、烧酒业者、毛织厂主、麻织厂主减少的利润,由于他们储存的原料和成品价值提高,会得到部分的补偿;但是,金属制品、珠宝制品和其他许多商品的工厂主以及资本完全由货币组成的人,就要承担利润率降低的全部损失而得不到任何补偿。”(同上,第123—124页)}\end{quote}

这里重要的只是李嘉图所没有觉察到的一点,那就是:他推翻了自己把利润和剩余价值等同起来的观点,并且确认,不管劳动价值怎样,利润率可能受不变资本价值变动的影响。此外,他的例子只有部分是正确的。租地农场主、毛织厂主等等从他们现有的和上了市场的商品储备涨价得到的好处,到他们把这些商品一脱手,自然就没有了。一旦他们的资本消费完了,必须进行再生产了,这笔资本价值的提高就同样不再给他们带来任何好处了。到那时候,他们的处境就都和李嘉图自己提到的新的租地农场主一样,为了获得3%的利润,就不得不预付6000镑资本。相反,[XIII—670]珠宝业者、金属制品厂主、货币资本家等的损失起初虽然没有得到任何补偿,但是他们会实现高于3%的利润率,因为价值有了提高的只是他们用于工资的资本,而不是他们的不变资本。

这里,在李嘉图提到的利润降低由资本价值的提高来补偿的问题上,还有一点是重要的,就是对资本家来说,——以及一般地就年劳动产品的分配来说,——问题不仅在于产品在参与收入分配的不同人们之间进行分配,而且在于这种产品分为资本和收入。

\tsectionnonum{[(2)一般利润率(“平均利润”,或者说,“普通利润”)的形成]}

\tsubsectionnonum{[(a)事先既定的平均利润率是李嘉图利润理论的出发点]}

李嘉图的理论观点在这里决不是清楚的。

\begin{quote}{“我曾经指出,某种商品的产量可能不敷新的需求,因此它的市场价格可能超过它的自然价格,或者说,必要价格。但是,这只是暂时的现象。用来生产这种商品的资本所获得的高额利润,自然会把资本吸引到这个生产部门中来;一旦有了必要的基金,商品量有了相当的增加,商品价格就会下跌,这一生产部门的利润就会同一般水平相一致。一般利润率的降低同个别部门的利润的局部提高决不是不相容的。正由于利润不等,资本才由一个部门转移到另一个部门。因此,当一般利润由于工资提高以及向日益增长的人口供应必需品的困难增加而降低并逐渐稳定在较低的水平时,租地农场主的利润可能在一个短时间内超过原来的水平。对外贸易和殖民地贸易的个别部门在一定时间内也可能获得非常的刺激。”(第118—119页)“应当记住,市场上价格经常变动,这首先取决于供求关系。虽然呢绒可以按每码40先令的价格供应,并为资本提供普通利润,但由于时装样式改变……它可能上涨到60或80先令。毛织厂主将暂时得到非常利润,但资本将自然流入这个工业部门,直到供求再达到适当的水平为止,那时呢绒的价格将再降到40先令,也就是降到它的自然价格,或者说,必要价格。同样,每当谷物的需求增加时,其价格也可能上涨到使租地农场主的利润高于普通利润。如果肥沃的土地很多,那末,在使用了必要的资本量来生产谷物之后,谷物的价格将再降到它原来的水平,利润也将和以前一样;但是,如果肥沃的土地不多,如果为了生产追加的谷物量需要比通常更多的资本和劳动,那末谷物的价格就不会降到它原来的水平。它的自然价格就会上涨,租地农场主就不能长久地获得较高的利润,而不得不满足于降低了的利润率,这是必需品涨价使工资提高的必然结果。”(第119—120页)}\end{quote}

如果工作日既定(或者说,如果在不同生产部门只有工作日长度的差别,而这种差别又为不同种类的劳动的特点抵销),那末,一般剩余价值率,即一般剩余劳动率也是既定的,因为工资平均起来是相同的。李嘉图念念不忘这一点。所以,他把这种一般剩余价值率同一般利润率混淆起来了。我已经指出,在一般剩余价值率相同的情况下,如果商品按照各自的价值出卖,各个不同生产部门的利润率必然是完全不同的。

一般利润率是用社会的(资本家阶级的)总资本除生产出来的全部剩余价值而得出来的;因此,每一个别生产部门的每一笔资本,都表现为具有同一[671]有机构成(不论从不变资本和可变资本的构成来说,还是从流动资本和固定资本的构成来说)的总资本的相应部分。这笔资本作为这样的相应部分,按照它的量的大小,从资本总额所生产的剩余价值中获得自己的股息。这样分配的剩余价值,即在一定时期(比如说一年)内分给一定量(比如说100)的资本的一份剩余价值,就形成平均利润,或者说,形成加入每一部门的费用价格的一般利润率。如果这一份等于15,那末普通利润就是15%,费用价格=115。如果,比如说,只有一部分预付资本作为损耗加入价值形成过程,那末费用价格可能小些。但是费用价格总是等于已消费的资本加15,即加预付资本的平均利润。如果在一种情况下有100加入产品,在另一种情况下只有50加入产品,那末在一种场合费用价格等于100+15=115,而在另一种场合等于50+15=65;这样,两种资本就会按照同一费用价格,即按照为两种资本提供同一利润率的价格出卖自己的商品。显然,一般利润率的出现、实现和确立,使得价值必然转化为不同于价值的费用价格。李嘉图却相反,他假定价值和费用价格是等同的,因为他把利润率和剩余价值率混淆起来了。因此,他一点也没有想到,早在有可能谈论一般利润率以前,确立一般利润率的过程已经引起商品价格的普遍变动。他把这种利润率当作一种最初的东西,因此在他的著作里它甚至包含在价值规定中。(见第一章《论价值》。)李嘉图从一般利润率这个前提出发,考察的只是为了保持这种一般利润率,使这种一般利润率继续存在下去所必需的、他作为例外来解释的价格变动。他一点也没有想到,为了创造这个一般利润率,必须先有价值向费用价格的转化;因此,他由于把一般利润率作为基础,就不会再直接同商品的价值发生关系了。

在前面引用的一段话里也只有斯密的观点,但是连斯密的观点也作了片面的阐述,因为李嘉图内心始终抱有一般剩余价值率的思想。在他看来,在个别部门,利润率所以会高于平均水平,只是因为在个别生产部门中由于供求关系,由于生产不足或生产过剩,商品的市场价格会高于自然价格。那时,竞争,新资本流入一个生产部门或旧资本从另一部门抽出,就会使市场价格同自然价格趋于一致,并使个别生产部门的利润恢复到一般水平。这里,利润的实际水平被假定为不变的,既定的东西,问题只在于使个别生产部门中由于供求关系而高于或低于这个水平的利润恢复到这个水平。同时,李嘉图甚至总是假定,如果商品价格提供的利润大于平均利润,这种商品就是高于其价值出卖,如果商品价格提供的利润低于平均利润,这种商品就是低于其价值出卖。如果通过竞争,商品的市场价值同它的价值达到一致,那末,利润的平均水平就确立起来了。

李嘉图认为,这个水平本身,只有在工资(相对稳定地)降低或提高的时候,也就是在相对剩余价值率降低或提高的时候,才能提高或降低;而这在价格没有变动的时候也会发生。(虽然这里李嘉图自己就承认,在不同的生产部门,根据其流动资本和固定资本的构成不同,价格会发生很显著的变动。)

但是,即使在一般利润率已经确立,因而费用价格也已经确立的情况下,在个别生产部门,由于工作日较长,也就是由于绝对剩余价值率提高,利润率也可能提高。工人之间的竞争并不能把这种差别拉平,国家的干涉已经证明这一点。在这里,在这些个别生产部门,即使市场价格并不高于“自然价格”,利润率也会提高。当然,资本之间的竞争可能而且终将使这种超额利润不是完全落入这些个别生产部门的资本家手中。他们不得不把自己商品的价格降到其“自然价格”之下,或者其他生产部门将把自己的价格提高一些(如果事实上没有提高,——这种提高可能被这些商品的价值的降低抵销,——那末无论如何,[672]总不致把它们的价格降得象本部门中劳动生产力的发展所要求的那样低)。利润的一般水平将提高,费用价格将发生变动。

其次,如果出现一个新的生产部门,使用的活劳动很多,同积累劳动不成比例,因此这个部门的资本构成大大低于决定平均利润的平均构成,那末,供求关系就可能容许这个新的部门高于产品的费用价格,以比较接近于产品实际价值的价格出卖产品。竞争要把这种差别拉平,只有通过提高利润的一般水平才有可能,而提高利润的一般水平又以资本实现、推动更大的无酬的剩余劳动量为条件。在上述情况下,供求关系不是象李嘉图所认为的那样,使商品高于它的价值出卖,而只是使商品按接近它的价值、高于它的费用价格的价格出卖。因此,平均化的结果不可能是使利润恢复到原来的水平,而是确立一个新的水平。

\tsubsectionnonum{[(b)李嘉图在殖民地贸易和一般对外贸易对利润率的影响问题上的错误]}

例如在殖民地贸易上,情况也是如此;在殖民地,由于奴隶制和土地的自然肥力(或者由于土地所有权在实际上或在法律上还不发达),劳动价值比在宗主国低。如果宗主国的资本可以自由地转入这个新的部门,这些资本固然会压低这个部门的特殊超额利润,但是将提高利润的一般水平(正如亚·斯密正确地指出的那样)。

李嘉图在这里经常求助于这样一种说法:要知道,在旧的生产部门,使用的劳动量以及工资是保持不变的。但是,一般利润率决定于无酬劳动对有酬劳动和对预付资本的比例,这不是就某个生产部门,而是就资本可以自由转入的所有部门来说的。这个比例,在十个部门中可能有九个保持不变;但如果十个部门中一个有了变动,一般利润率在所有十个部门中都必然要发生变动。每当一定量资本所推动的无酬劳动量有了增加的时候,竞争的结果只能是:等量资本取得相等的股息,即在这个增大了的剩余劳动中的相等的一份;竞争的结果不可能是:尽管剩余劳动同全部预付资本相比已经增加,每一笔资本的股息却保持不变,仍然是剩余劳动中原来的那一份。既然李嘉图承认这一点,他就没有任何理由反驳亚·斯密的下述观点:单是因资本积累而加剧的资本竞争就会使利润率降低。因为在这里他自己就承认,即使剩余价值率有所提高,单单由于竞争,利润率也会降低。李嘉图的这个观点当然是同他的第二个错误前提联系着的,那就是,利润率(撇开工资的降低或提高不谈)所以能够提高或者降低,仅仅是由于市场价格暂时偏离自然价格。而什么是自然价格呢?就是等于预付资本加平均利润的那种价格。所以,又归结到这样一个前提:除非相对剩余价值降低或提高,否则平均利润决不可能降低或提高。

因此,李嘉图用下面的话来反对斯密,是错误的。他说:

\begin{quote}{“从对外贸易的一个部门转移到另一个部门,或者从国内贸易转移到对外贸易,据我看来,都不能影响利润率。”(同上,第413页)}\end{quote}

李嘉图认为,因为利润不影响价值,所以利润率也不影响费用价格,他这种看法也是错误的。

李嘉图认为,如果对外贸易的某一部门条件特别有利,那末,利润的一般水平总是通过使那里的利润降到原来水平的办法,而不是通过提高利润的原来水平的办法来确立的。他这种看法是错误的。

\begin{quote}{“他们断言,利润的均等是由利润的普遍提高造成的;而我却认为,特别有利的部门的利润会迅速下降到一般水平。”(第132—133页)}\end{quote}

由于李嘉图对利润率抱着完全错误的观点,他就根本不懂得对外贸易在不直接降低工人食物价格时所发生的影响。他看不到,对于象英国这样的国家,取得[673]较低廉的工业用原料具有多么重大的意义,他不了解,在这种情况下,正如我前面指出的那样,[20][用较低廉的原料制成的产品]价格虽然下跌,利润率却会提高,相反,[用较贵的原料制成的产品]价格上涨了,利润率却可能降低,即使工资在这两种情况下保持不变,也是如此。

\begin{quote}{“因此,利润率不会由于市场扩大而提高。”(第136页)}\end{quote}

利润率不是取决于单位商品的价格,而是取决于用一定的资本能够实现的剩余劳动量。李嘉图在其他地方对市场的重要意义也估计不足,因为他不了解货币的本质。

\centerbox{※     ※     ※}

[673](除了前面所说的以外,还必须指出:

李嘉图所以犯这一切错误,是因为他想用强制的抽象来贯彻他把剩余价值率和利润率等同起来的观点。庸俗经济学家由此得出结论说,理论上的真理是同现实情况相矛盾的抽象。相反,他们没有看到,因为李嘉图在正确抽象方面做得不够,才使他采取了错误的抽象。\endnote{马克思在这里指的是象让·巴·萨伊这样一些批评李嘉图的人,萨伊指责李嘉图把“赋予过分普遍意义的抽象原则”作为自己论述的基础,因此得出了不符合实际情况的结论(让·巴·萨伊《论政治经济学》1841年巴黎第6版第40—41页)。——第497页。})[673]

\tsectionnonum{[(3)]利润率下降规律}

\tsubsectionnonum{[(a)李嘉图关于利润率下降的见解的错误前提]}

利润率下降规律是李嘉图体系中最重要的观点之一。

利润率具有下降的趋势。为什么呢?亚·斯密说:这是由于资本积累的增长和随之而来的资本竞争的加剧。李嘉图反驳说:竞争能使不同生产部门的利润平均化(前面我们已经看到,他在这里是前后矛盾的),但它不能使一般利润率下降。在李嘉图看来,这种下降只有在下述情况下才是可能的,即由于资本的积累,资本的增殖比人口的增长快,以致对劳动的需求经常超过劳动的供给,因而工资在名义上、实际上以及按使用价值来说都不断提高——不论按价值还是按使用价值来说,都不断提高。但这种情况是不会有的。李嘉图不是一个相信这种寓言的乐观主义者。

因为利润率和剩余价值率(指相对剩余价值率,因为李嘉图假定工作日不变)在李嘉图看来是等同的,所以,利润率的不断下降或利润率下降的趋势,他只能用决定剩余价值率(即工作日中工人不是为自己而是为资本家劳动的那一部分)不断下降或下降趋势的同样原因来说明。但这是些什么样的原因呢?假定工作日既定,那末只有在工人为自己劳动的那一部分工作日增大的条件下,工人无代价地为资本家劳动的那一部分工作日才能减少、缩短。而这(假定劳动的价值能得到支付)只有在用工人工资购买的必需品即生活资料的价值增大的情况下才有可能。但是,由于劳动生产力的发展,工业品的价值在不断减少。因此,在李嘉图看来,利润率的下降,就只能用生活资料的主要组成部分——食物——的价值的不断提高来说明。而这据说又是由于农业生产率不断降低引起的。这就是李嘉图在分析地租时用来说明地租存在和地租增长的那个前提。因此,在李嘉图看来,利润率的不断下降是同地租率的不断提高联系在一起的。我已经指出,李嘉图对地租的理解是错误的。因而,他用来说明利润率下降的根据之一也就不能成立。第二,他关于利润率下降的观点是建立在错误的前提之上的,他认为,剩余价值率和利润率是等同的,从而,利润率的下降和剩余价值率的下降也是等同的,其实,这后一种下降只有按照李嘉图的方式才能解释。因此他的理论就被推翻了。

利润率下降——虽然剩余价值率这时保持不变或提高——是因为随着劳动生产力的发展,可变资本同不变资本相比减少了。因此,利润率下降不是因为劳动生产率降低了,而是因为劳动生产率提高了。利润率下降不是因为对工人的剥削减轻了,而是因为对工人的剥削加重了,不管这是由于绝对剩余时间增加,还是——在国家对此进行阻挠时——由于资本主义生产的本质必然要使劳动的相对价值降低,从而使相对剩余时间增加。

所以,李嘉图的理论是建立在两个错误前提之上的:

第一个错误前提是:地租的存在和增加以农业生产率不断降低为条件;

第二个错误前提是:利润率(在李嘉图那里就等于相对剩余价值率)的提高或下降只能同工资的提高或下降成反比。

[674]现在我首先把李嘉图发挥上述见解的地方收集在一起。

\tsubsectionnonum{[(b)对李嘉图关于增长的地租逐渐吞并利润这个论点的分析]}

我们预先还要作几点说明,指出李嘉图是怎样从自己的地租见解出发来阐明地租逐渐吞并利润率这个观点的。

为此,我们想利用第574页上的表\fnote{见第302—303页。——编者注},但是要作一些必要的修改。

在这些表内,都是假定使用的资本等于60c+40v,剩余价值率等于50%,因此,不管劳动生产率如何,产品的[个别]价值都等于120镑。其中10镑是利润,10镑是绝对地租。我们假定,40镑可变资本支付20个工人(比如说,支付他们一周的劳动,或者,因为谈的是利润率,最好算作支付一年的劳动;这在这里是完全无关紧要的)。依照A表,决定市场价值的是土地I,[煤的]吨数[或谷物的夸特数]等于60;所以,60吨值120镑,1吨[或1夸特]值,120/60,即2镑。工资是40镑,也就是说,等于20吨[煤]或20夸特谷物。因而这就是100镑资本所使用的工人数量的必要工资。如果现在必须推移到较坏等级的土地,为了生产48吨,需要资本110镑(60镑不变资本和推动20个工人的可变资本,即60镑不变资本和50镑可变资本),那末,在这种场合,剩余价值就是10镑,每吨的价格就等于2+(1/2)镑。如果我们推移到更坏等级的土地,那里120镑提供40吨,那末,每吨的价格就等于120/40,即3镑。在这里,在最坏的土地上,任何剩余价值都将没有了。在所有这些场合,由20个工人的劳动新创造的价值,总是等于60镑(3镑=1工作日,不管其长度如何)。因此,如果工资从40镑增加到60镑,那末,任何剩余价值都会消失。在这里总是假定,1夸特谷物是一个工人的必要工资。

我们假定,在上述两种[推移到较坏土地的]场合,必须花费的资本都只是100镑。或者同样可以说,不管在每一种场合花费多少资本,对于100镑资本来说会有怎样的比例呢?这就是说,我们不是假定,在工人人数相同和不变资本量相同的情况下,花费的资本是110镑,还是120镑,而是要计算,在假定有机构成相同(不是按价值,而是按使用的劳动量和不变资本量)的情况下,资本为100时有多少不变资本和能雇用多少工人(以便有可能把这100单位和其他的资本[价值]构成进行比较)。

110比60等于100比54+(6/11),110比50等于100比45+(5/11)。20个人推动60单位不变资本;那末54+(6/11)单位要由多少人来推动呢?

事情是这样。60镑是被雇用的一定数量的工人(比如说,20个工人)所创造的价值。在这里,如果每吨或每夸特=2镑,那末给这一定数量的工人支付20夸特或20吨,就=40镑。如果每吨的价值增加到3镑,剩余价值就会消失。如果每吨的价值增加到2+(1/2)镑,构成绝对地租的那一半剩余价值就会消失。

第一种场合,花费资本120镑(60c+60v)时,产品=120镑=40吨(40×3)。第二种场合,花费资本110镑(60c+50v)时,产品=120镑=48吨{48×[2+(1/2)]}。

第一种场合,花费资本100镑(50c+50v)时,产品=100镑=33+(1/3)吨{3镑×[33+(1/3)]=100镑}。同时,因为这里只是推移到较坏的土地,资本中并没有发生任何变化,推动50镑不变资本的工人人数,和以前推动60镑资本的工人人数的比例一样。因此,如果以前60镑资本由20个工人推动(在每吨的价值等于2镑时,他们得到40镑),现在的50镑资本就由16+(2/3)个工人推动,他们从每吨的价值增加到3镑时起得到50镑。一个人照旧得到1吨或1夸特=3镑,因为[16+(2/3)]×3=50。如果16+(2/3)个工人创造的价值=50镑,20个工人创造的价值就=60镑。因此,原先20个工人的一天劳动创造价值60镑的前提仍然有效。

现在再看第二种场合。花费资本100镑时,产品=109+(1/11)镑=43+(7/11)吨{[2+(1/2)]×[43+(7/11)]=109+(1/11)}。不变资本=54+(6/11)镑,可变资本=45+(5/11)镑。这45+(5/11)镑雇用多少工人呢?18+(2/11)个工人。[675]这时,如果20个工人一天劳动创造的价值=60镑,18+(2/11)个工人一天劳动创造的价值就=54+(6/11)镑,因而产品的价值=109+(1/11)镑。

这样我们就看到,在两种场合,同一资本[100镑]推动的工人人数比过去少了,可是现在花费在工人身上的费用贵了。他们劳动的时间还是那样多,但是他们提供的剩余劳动时间少了,或者完全没有了,因为他们在花费同样劳动的情况下生产的产品少了(而这种产品是由他们所消费的必需品构成的);因此,虽然他们劳动的时间和以前一样,他们生产1吨或1夸特所用的劳动时间却增加了。

李嘉图在他的计算中总是假定,资本推动了更多的劳动,因此必须花费更多的资本,即不是以前的100镑,而是120镑或110镑。这只有在假定生产的产品数量相同的情况下才是正确的,也就是说,在上述两种场合都是生产60吨,而不是第一种场合生产40吨(花费120镑),第二种场合生产48吨(花费110镑)。所以,花费100镑时,第一种场合生产33+(1/3)吨,第二种场合生产43+(7/11)吨。从上述假定出发,李嘉图就抛弃了正确的观点,这个正确观点不是[在劳动生产率降低的情况下]必须使用更多的工人来生产同样数量的产品,而是一定数量的工人生产的产品少了,而这种产品中构成工资的那部分却比以前大了。

现在把两个表比较一下:一个是第574页上的A表,一个是根据上述假定得出的新表。

\todo{}

如果现在这个表按照李嘉图的下降序列,用相反的顺序来表述,也就是说,如果我们从III开始,同时假定,最先耕种的最肥沃的土地[当它是唯一的一个等级的耕地时]不提供任何地租,那末,在III这个等级,我们首先会有100镑资本,它生产120镑的价值,其中60镑为不变资本,60镑为新加劳动。其次,根据李嘉图的前提,必须假定利润率比A表上的高,因为在每吨煤(或每夸特小麦)的价格降低时情况是这样的:当每吨值2镑时,20个工人得到20吨=40镑;而现在,当每吨只值1+(3/5)镑,或者说,值1镑12先令时,这20个工人总共只得到32镑(=20吨)。工人人数相同,而花费的资本将是60c+32v=92镑,全部产品的价值将是120镑,因为20个工人的劳动所创造的价值照旧等于60镑。按照[c和v之间的]这种比例,100镑资本应创造价值130+(10/23)镑(因为92∶120=100:[130+(10/23)],或23∶30=100∶[130+(10/23)]),并且这100镑资本的构成将是:[65+(5/23)]c+[34+(18/23)]v。因此,资本等于[65+(5/23)]c+[34+(18/23)]v;产品价值是130+(10/23)镑;工人人数是21+(17/23);剩余价值率是[87+(1/2)]%。

(1)这样,我们就得出下表:

\todo{}

用吨来表示,工资=21+(17/23)吨,利润=19+(1/46)吨。

[676]我们始终根据李嘉图的前提,现在假定由于人口增加,市场价格提高,因此必须耕种等级II,这里每吨价值等于1+(11/13)镑。

这里决不是象李嘉图所设想的那样,21+(17/23)个工人始终生产同样的价值,即65+(5/23)镑(工资和剩余价值算在一起)。因为[由于产品涨价]III的资本家能够雇用、从而能够剥削的工人人数,按照他自己的前提,将会减少,也就是说,剩余价值总额也将减少。

同时,农业资本的[有机]构成始终保持不变。为了推动60c,不管工资多少,始终需要20个工人(工作日既定)。

因为这20个工人得到20吨,而每吨现在值1+(11/13)镑,所以20个工人就花费20×[1+(11/13)]镑=20镑+[16+(12/13)]镑=36+(12/13)镑。

不管这20个工人的劳动生产率如何,他们生产的价值总是等于60镑。这样,预付资本等于96+(12/13)镑,产品价值等于120镑;所以,利润是23+(1/13)镑。因此,资本100提供的利润将等于23+(17/21),资本构成将是[61+(19/21)]c+[38+(2/21)]v,雇用的工人人数将是20+(40/63)。

如果产品总价值[在花费资本100镑时]=123+(17/21)镑,而III的每吨的个别价值=1+(3/5)镑,那末,这一等级的产品现在是多少吨呢?是77+(8/21)吨。剩余价值率是[62+(1/2)]%。

但是III的每吨是按1+(11/13)镑出卖的。因此,每吨的差额价值是4+(12/13)先令,或16/65镑,以77+(8/21)吨计算,是[77+(8/21)]×(16/65),即19+(1/21)镑。

III的产品不是卖123+(17/21)镑,而是卖123+(17/21)镑+[19+(1/21)]镑=142+(6/7)镑。这19+(1/21)镑就构成地租。

这样,对于III,我们就得出下表:

\todo{}

以吨计算,工资=20+(40/63)吨,利润=12+(113/126)吨。

现在我们转过来谈等级II。这里[根据李嘉图的前提]根本没有地租。市场价值和个别价值相符。II生产的吨数是67+(4/63)吨。

所以,对于II,我们就得出下表:

\todo{}

以吨计算,工资=20+(40/63)吨,利润=12+(113/126)吨。

[677](2)所以,对于第二种场合,即当等级II出现并产生地租时,我们就得出下表:

\todo{}

现在我们转过来谈第三种场合,我们同李嘉图一起假定,较次等级的煤矿(I)必须开采,而且能够开采,因为产品的市场价值已经提高到每吨2镑。因为不变资本为60镑时,需要20个工人,这20个工人现在要花费40镑,所以我们的资本的[价值]构成就和第574页A表上的一样,即60c+40v。20个工人生产的价值总是等于60镑。因此,不管资本的生产率如何,100镑资本所生产的产品的总价值[个别价值,实际价值]是120镑。这里利润率=20%,剩余价值=50%。以吨计算,利润=10吨。现在我们就要看一看,由于市场价值的这种变动以及决定利润率的I的出现,III和II会发生什么变化。

虽然III耕种的是最肥沃的土地,但它有100镑资本也只能使用20个工人,花费40镑,因为在不变资本为60镑时需要20个工人。因此,100镑资本使用的工人人数降到20。而它的产品的实际总价值现在等于120镑。但是因为III生产的每吨的个别价值等于1+(3/5)镑(或8/5镑),那末,它生产多少吨呢?75吨,因为120除以8/5等于75。III生产的吨数减少了,因为它用相同的资本,只能使用较少的劳动,而不是较多的劳动(李嘉图却经常错误地认为,能使用更多的劳动,因为他经常注意的只是为了生产同样数量的产品需要多少劳动,而不是按新的资本[价值]构成能使用多少活劳动;而这一点在这里正是唯一重要的问题)。但这75吨,III按150镑(而不是按构成其[实际]价值的120镑)出卖,这样,III的地租就提高到30镑。

至于II,这里产品的[实际]价值也等于120镑等等。但是因为每吨个别价值等于1+(11/13)镑(或24/13镑),所以这个等级生产65吨(因为120除以24/13等于65)。简单地说,我们在这里得到的是第574页上的A表。但是因为在这里我们需要有几个新项目来适应我们的目的,所以当出现I这个等级而产品市场价值提高到2镑时,我们就要为这第三种场合制一个新表。

(3)第三种场合:

\todo{}

[678]这样,这第三种场合就和第574页上的A表相符合(如果不计算绝对地租,绝对地租在这里作为利润的一部分出现),只是顺序颠倒了。

现在我们转过来谈我们假定的新的场合\fnote{见本册第500—502页。——编者注}。

首先我们来看看那个还提供利润的等级;我们把它叫做Ib。在资本为100镑时,它只提供43+(7/11)吨。

每吨价值提高到2+(1/2)镑。资本构成是[54+(6/11)]c+[45+(5/11)]v。产品价值是109+(1/11)镑。可变资本45+(5/11)镑用来支付18+(2/11)个工人的工资。因为20个工人一天劳动创造的价值等于60镑,所以18+(2/11)个工人创造的价值是54+(6/11)镑。因此,产品价值等于109+(1/11)镑。利润率=[9+(1/11)]%。利润=3+(7/11)吨。剩余价值率=20%。

因为III、II、I的资本有机构成和Ib的一样,而且它们必须支付的工资也同后者一样,所以,在资本为100镑时,它们也只能使用18+(2/11)个工人,这18+(2/11)个工人生产的总价值是54+(6/11)镑,所以,也同Ib一样,剩余价值为20%,利润率为[9+(1/11)]%。这里,产品的总价值[实际价值],也和Ib一样,等于109+(1/11)镑。

但是,因为III的每吨个别价值等于1+(3/5)镑(或8/5镑),所以III生产[109+(1/11)]∶(8/5),即68+(2/11)吨(换句话说,这里109+(1/11)镑等于68+(2/11)吨)。其次,每吨市场价值和每吨个别价值的差额现在是2+(1/2)镑减1+(3/5)镑,或者说,2镑10先令减1镑12先令,即18先令。以68+(2/11)吨计算,就是[68+(2/11)]×18先令=1227+(3/11)先令=61+(4/11)镑。III的产品不是卖109+(1/11)镑,而是卖170+(5/11)镑。这个余额就是III的地租。用吨来表示,地租=24+(6/11)吨。

因为II的每吨个别价值等于1+(11/13)镑(或24/13镑),所以它生产[109+(1/11)]:(24/13),即59+(1/11)吨。II的每吨市场价值和它的[个别]价值的差额是2+(1/2)镑减1+(11/13)镑,即17/26镑。以59+(1/11)吨计算,是38+(7/11)镑。这就是地租。总产品市场价值=147+(8/11)镑。用吨来表示,地租等于15+(5/11)吨。

最后,因为I的每吨个别价值等于2镑,所以在这里109+(1/11)镑等于54+(6/11)吨。市场价值和个别价值的差额等于2+(1/2)镑减2镑,即1/2镑。以54+(6/11)吨计算,是27+(3/11)镑。所以,总产品市场价值=136+(4/11)镑。用吨来表示,地租价值等于10+(10/11)吨。

如果我们现在把所有这些综合起来构成第四种场合,便得出下表:

[679](4)第四种场合:

\todo{}

最后,我们来考察一下最后一种场合,按照李嘉图的说法,在这里一切利润都会消失,剩余价值完全没有了。

在这里,产品价值提高到每吨3镑,所以,在使用20个工人时,他们的工资等于60镑,等于他们生产的价值。资本构成将是50c+50v。在这种场合使用的就是16+(2/3)个工人。20个工人生产的价值是60镑,16+(2/3)个工人生产的价值就是50镑。因而,工资会把这全部价值吞没。一个工人和以前一样得到一吨。产品价值=100镑。这样,生产的吨数=33+(1/3)。其中一半只能补偿不变资本的价值,而另一半只能补偿可变资本的价值。

既然III的每吨个别价值等于1+(3/5)镑(或8/5镑),那末,它生产多少吨呢?100除以8/5,即62+(1/2)吨,其价值=100镑。每吨市场价值和个别价值的差额等于3镑-[1+(3/5)]镑=1+(2/5)镑。以62+(1/2)吨计算,差额就是87+(1/2)镑。因而在这里总产品市场价值等于187+(1/2)镑。用吨来表示,地租就等于29+(1/6)吨。

II的每吨个别价值等于1+(11/13)镑。因此,每吨差额价值等于3镑减1+(11/13)镑,即1+(2/13)镑。因为在这里每吨个别价值等于1+(11/13)镑,或24/13镑,所以100镑的资本就生产100:(24/13),即54+(1/6)吨。以这个吨数计算,差额就是62+(1/2)镑。产品市场价值=162+(1/2)镑。用吨来表示,地租就等于20+(5/6)吨。

I的每吨个别价值=2镑。因此,每吨差额价值等于3镑减2镑,即1镑。因为在这里每吨个别价值=2镑,所以100镑的资本生产50吨。[产品市场价值和个别价值的]差额就是50镑。产品市场价值=150镑,以吨计算,地租=16+(2/3)吨。

现在我们转过来谈Ib,它到现在为止不提供任何地租。这里每吨个别价值=2+(1/2)镑。因而每吨差额价值等于3镑减2+(1/2)镑,即1/2镑。因为这里每吨个别价值=2+(1/2)(或5/2)镑,所以100镑的资本就生产40吨。以这个吨数计算,差额价值就是20镑,所以总产品的市场价值=120镑,以吨计算,地租=6+(2/3)吨。

这样,我们就有了第五种场合,按照李嘉图的说法,在这里利润会消失,我们现在用一个统一的表来说明。

[680](5)第五种场合:

\todo{}

在下页我把所有五种场合列一总表加以比较。\fnote{见第512—513页的表。——编者注}[680]

\todo{}

\tsubsectionnonum{[(c)一部分利润和一部分资本转化为地租。地租量的变动取决于农业中使用的劳动量的变动]}

[683]如果我们从上页的表\fnote{见第512—513页。——编者注}中首先考察E表,我们就会看到,在这里,最后一个等级Ia的情况是很明显的。在这里,工资吞并了[新加]劳动的全部产品和它创造的全部价值。任何剩余价值都没有了,从而利润和地租也没有了。产品的价值等于预付资本的价值,所以,在这里自己拥有资本的劳动者,能够不断地把自己的工资和自己的劳动条件再生产出来,但是不能再生产出更多的东西。关于最后这一个等级,决不能说地租吞并了利润。这里既没有地租,也没有利润,因为没有任何剩余价值。工资吞并了剩余价值,因而也吞并了利润。

至于其他四个等级,初看起来情况并不明显。既然没有剩余价值,又怎么可能有地租存在呢?而且在Ib、I、II和III这四个等级的土地上劳动生产率并没有发生变动。所以,剩余价值没有了,应该只是一种表面现象。

随后就会发现另一种初看起来同样不可理解的现象。以[煤的]吨数或谷物的夸特数来表示的地租,在III是29+(1/6)吨(或夸特),而在只有III的土地被耕种的A表,却没有任何地租。此外,那里的工人人数是21+(17/23),而现在在E表,工人人数只有16+(2/3);在A表,(吞并了全部剩余价值的)利润只是19+(1/46)吨。

在II可以发现同样的矛盾,在E表,II的地租是20+(5/6)吨(或夸特),而在B表,吞并了全部剩余价值的利润(而且使用的工人人数是20+(40/63),不是现在的16+(2/3)只等于12+(113/126)吨(或夸特)。

在I也可以发现同样的矛盾,在E表,I的地租等于16+(2/3)吨或夸特,而在C表,I的吞并了全部剩余价值的利润只等于10吨(而且使用的工人人数是20,不是现在的16+(2/3))。

最后,在Ib也有同样的现象,在E表,Ib的地租是6+(2/3)吨(或夸特),而在D表,它的吞并了全部剩余价值的利润只等于3+(7/11)吨或夸特(而且使用的工人人数是18+(2/11),而不是现在的16+(2/3))。

但是很明显,III、II、I、Ib的产品的市场价值高于个别价值,虽然也能够改变产品的分配,促使产品从一类分享者手里转移到另一类分享者手里,但是这种市场价值的提高,决不能使补偿工资后留下的剩余价值所借以表现的产品本身增加。我们这里列举的各个等级土地的生产率以及资本的生产率既然保持不变,那末,仅仅由于在市场上出现了比较不肥沃的土地或比较不富饶的矿井Ia的产品,III至Ib怎么就能够变得比较肥沃或富饶起来,也就是能够提供更多的吨数或夸特数呢?

用下述方法可以解开这个谜:

如果20个工人一天的劳动等于60镑,16+(2/3)个工人就生产50镑。因为在等级III,1+(3/5)镑(或8/5镑)所包含的劳动时间表现为1吨或1夸特,所以,50镑就表现为31+(1/4)吨或夸特。其中16+(2/3)吨或夸特用于工资;因而就有14+(7/12)吨留作剩余价值。

其次,因为每吨市场价值由1+(3/5)镑提高到3镑,所以为了补偿不变资本的价值[50镑],从62+(1/2)吨或夸特的产品中只要拿出16+(2/3)吨或夸特来就够了。不过,如果由在等级III生产的1吨(或1夸特)本身决定市场价值,因而市场价值等于它的个别价值,为了补偿50镑不变资本,就需要31+(1/4)吨或夸特。这个吨数或夸特数是在每吨价值等于1+(3/5)镑时为了补偿[不变]资本所必需的那部分产品,现在为了同一个目的,只要有16+(2/3)吨就够了。因此,[684]31+(1/4)吨减16+(2/3)吨,即14+(7/12)吨(或夸特),就游离出来,归入地租份内。

在III,16+(2/3)个工人在不变资本为50镑时生产的剩余价值等于14+(7/12)吨或夸特;以前用来补偿不变资本,而现在以剩余产品形式出现的那一部分产品也等于14+(7/12)吨或夸特,如果我们把这两部分加起来,总剩余产品就是28+(14/12)=29+(2/12)=29+(1/6)吨或夸特。这恰好就是E表中III以[煤的]吨数或谷物的夸特数来表示的地租。在E表中II、I、Ib这些等级以[煤的]吨数或谷物的夸特数来表示的地租在量上的表面矛盾,也完全可以用同样的方法加以解决。

这样,我们就看到,在较好等级的土地上由于它们产品的市场价值和个别价值的差额而产生的级差地租,在它的实在形态上,作为实物地租,作为剩余产品,或者说,象上述例子那样,作为以[煤的]吨数或谷物的夸特数表示的地租,是由两个要素构成,并由两种转化决定的。[第一,]表现工人剩余劳动的剩余产品即剩余价值,经历了从利润形式到地租形式的转化,因而归土地所有者,而不归资本家所有。第二,以前,当较好等级的土地或矿井的产品按其本身价值出卖时,有一部分产品必须用来补偿不变资本的价值,现在,当产品的每一部分都有了更高的市场价值时,那部分产品中有一部分就会游离出来,也以剩余产品的形式出现,因而也归土地所有者,而不归资本家所有。

剩余产品转化为地租而不转化为利润,以前用于补偿不变资本价值的产品的某一相应部分转化为剩余产品,因而转化为地租,——这两个过程在实物地租是级差地租的情况下构成实物地租。后一种情况,即产品的一部分不转化为资本,而转化为地租,李嘉图和所有后来的经济学家都没有注意到。他们只看到剩余产品转化为地租,而没有看到以前归入资本份内(而不归入利润份内)的那部分产品中有一部分也会转化为剩余产品。

这样构成的剩余产品,或者说,级差地租,它的名义价值决定于(根据假定)最坏土地或最次矿井生产的产品的价值。但是这种市场价值只能引起这种产品的另行分配,而不能创造产品。

这两个要素在一切有超额利润的场合都是存在的;例如,由于采用新机器等等,某种产品的生产变得便宜了,但它按照超过其本身价值的市场价值出卖,就是这种情况。工人的一部分剩余劳动不是作为利润出现,而是作为成为超额利润的剩余产品出现。一定量产品,在工业品按照其本身的较低的价值出卖时,本来必须用来为资本家补偿不变资本的价值,现在有一部分就会游离出来,因为没有什么需要补偿了;这一部分成了剩余产品,因而就使利润大大增加。[684]

\centerbox{※     ※     ※}

[688]{但是,当我们谈到资本主义生产发展过程中利润率下降的规律时,我们在这里把利润理解为剩余价值总额,它首先为产业资本占有,不管以后产业资本要怎样和借贷资本家(利息所得者)以及土地所有者(地租所得者)瓜分它。所以,这里利润率=剩余价值/预付资本。在这个意义上说,利润率可能下降,尽管,比如说,产业利润同利息相比提高了,或者相反;同样,尽管地租同产业利润相比提高了,或者相反,利润率也可能下降。如果利润=P,产业利润=P′,利息=Z,地租=R,那末P=P′+Z+R。很明显,无论P的绝对量如何,P′、Z和R彼此相对来说可能提高或下降,而不管P的大小,不管P是提高还是下降。P′、Z和R彼此相对来说的提高,仅仅是P在不同人之间的不同分配。进一步考察引起P在不同人之间的这种或那种分配(而分配与P本身的提高和下降决不是一回事)的种种情况,并不是这里的任务,这要留到分析资本竞争时再说。但是,如果说R在量上能够达到P本身不会有的高度——要是P只分为P′和Z——那末,正如已经证明的,这是一种表面现象,它是由下述情况造成的,即一部分产品,在其[市场]价值增加时,不是再转化为不变资本,而是游离出来,转化为地租。}[688]

\centerbox{※     ※     ※}

[684]在前面的所有论述中都假定,已涨价(按市场价值来说)的产品,不以实物形式加入不变资本的构成,而只加入工资,只加入可变资本。如果已涨价的产品加入不变资本,那末,在李嘉图看来,利润率因此就会降得更低,地租就会提得更高。这一点必需加以研究。

在此以前,我们一直假定,不变资本的价值,即上述场合的50镑,应由产品的价值补偿。所以,在每吨或每夸特值3镑的时候,为补偿上述价值所需的吨数或夸特数,当然比每吨或每夸特仅值1+(3/5)镑等等的时候少。但现在我们假定,煤或谷物,或其他任何土地产品——由农业资本生产的产品——以实物形式加入不变资本的形成,比如说,加入一半。在这种场合,很明显,不管煤或谷物的价格如何,[685]一定量的不变资本,即由一定数量工人推动的一定量的不变资本,总是要求用总产品的一定部分以实物形式来补偿自己,因为根据假定,表现为积累劳动量和活劳动量之比的农业资本构成保持不变。

假定,比如说,不变资本一半由煤或谷物构成,而另一半由其他商品构成,那末,50镑不变资本中有25镑是由其他商品构成,25镑是[由煤或谷物,]即在每吨[煤]值1+(3/5)(或8/5)镑时,由15+(5/8)吨[煤]或夸特[谷物]构成。无论每吨[煤]或每夸特[谷物]的市场价值如何变动,16+(2/3)个工人所需要的不变资本总是25镑[其他商品]加15+(5/8)夸特[谷物]或吨[煤],因为不变资本的物质构成不变,推动这个资本所需要的相应的工人人数也不变。

如果现在每吨[煤]或每夸特[谷物]的[市场]价值象在E表上那样提高到3镑,16+(2/3)个工人需要的不变资本就等于25镑+[15+(5/8)]×3镑=25镑+45镑+(15/8)镑=71+(7/8)镑。因为16+(2/3)个工人在这里要花费50镑,所以在这种场合需要花费的总资本就是[71+(7/8)]镑+50镑=121+(7/8)镑。

在有机构成相同的情况下,农业资本按其价值比例来说将会发生变动。

那就是[71+(7/8)]c+50v(工人人数为16+(2/3))。100镑资本的构成将是[58+(38/39)]c+[41+(1/39)]v。工人人数将是13+(79/117)(比13+(2/3)多1/117)。因为16+(2/3)个工人推动15+(5/8)夸特或吨不变资本,13+(79/117)个工人就推动12+(32/39)吨或夸特,即价值38+(6/13)镑。剩下的那部分不变资本等于20+(20/39)镑,由其他商品构成。在一切情况下,从产品中都应扣除12+(32/39)吨或夸特,来补偿它们以实物形式加入的那一部分不变资本。因为20个工人生产的价值等于60镑,13+(79/117)个工人生产的价值就等于41+(1/39)镑。但在E表上,13+(79/117)个工人的工资也是41+(1/39)镑。所以,这里任何剩余价值都没有了。

[在这种场合,E表的III的]总吨数将是[51+(11/39)吨\endnote{51+(11/39)吨这一数字是这样算出来的:如果E表III的16+(2/3)个工人生产62+(1/2)吨,那末13+(79/117)个工人在劳动生产率相同的情况下将生产[13+(79/117)]×[62+(1/2)]/16+(2/3)即51+(11/39)吨。——第519页。},其中]12+(32/39)吨会再生产出来[以补偿它们以实物形式加入的那部分不变资本],13+(79/117)吨用于工人的工资,6+(98/117)吨用于剩下的那部分不变资本(每吨3镑)。这三个部分合在一起是33+(1/3)吨。所以剩下的地租份额是17+(37/39)吨。

为了计算简便,我们假定出现对李嘉图最有利的极端情况,也就是假定不变资本完全和可变资本一样,仅仅由农产品构成,其价值由于等级Ia在市场上占统治地位而提高到每夸特或每吨3镑。

资本的技术构成保持不变,就是说,可变资本所代表的活劳动即工人人数(因为假定正常工作日不变),同使用这一数量的工人时需要的、根据我们假定现在是由煤的吨数或谷物的夸特数构成的劳动资料总量之间的比例保持不变。

因为在原来的资本构成60c+40v的条件下,当每吨的价格等于2镑时,40v代表20个工人或者20夸特或吨,所以,60c就代表30吨;因为III的这20个工人生产75吨,所以,13+(1/3)个工人(每吨价格为3镑时,40v相当于13+(1/3)吨或13+(1/3)个工人)就生产50吨,并推动不变资本[686]60/3,即20吨或夸特。

其次,因为20个工人生产价值60镑,所以13+(1/3)个工人就生产价值40镑。

因为资本家为购买20吨[不变资本]必须付出60镑,为雇用13+(1/3)个工人必须付出40镑,而这些工人只生产价值40镑,所以产品的价值=100镑;支出也是100镑。剩余价值和利润=0。

但是,因为III的生产率不变,所以象已指出的那样,13+(1/3)个工人生产50吨或夸特。但是实物支出以吨或夸特计算只有33+(1/3)吨:20吨用于不变资本,13+(1/3)吨用于工资。因而III生产的50吨包含16+(2/3)吨的剩余产品,这一剩余产品就形成地租。

但这16+(2/3)吨代表什么呢?

因为产品的价值=100镑,而产品本身=50吨,这里生产的每吨价值实际上是100/50,即2镑。只要以实物形式得到的产品数量大于以实物形式补偿资本所必需的数量,每吨的个别价值,即使在这种[缩小了的]生产规模下,也必然低于它的市场价值。

租地农场主必须花费60镑,以补偿20吨[不变资本]。这20吨,他是按每吨3镑计算的,因为每吨的市场价值就是如此,每吨就是按照这一价格出卖的。同样,他必须花费40镑,以支付13+(1/3)个工人的工资,或者说,以支付他付给工人的那个吨数或夸特数。因此,那些工人只得到13+(1/3)吨。

但在实际上,就等级III来说,20吨只值40镑,而13+(1/3)吨只值26+(2/3)镑。但是13+(1/3)个工人生产价值40镑,因此,创造剩余价值13+(1/3)镑。按每吨2镑计算,合6+(4/6)(或6+(2/3))吨。

因为III的20吨[不变资本]只值40镑,所以剩下一个余额20镑,等于10吨。

因此,构成地租的16+(2/3)吨分为:转化为地租的剩余价值6+(2/3)吨,以及转化为地租的资本10吨。但是,由于每吨的市场价值提高到3镑,20吨[不变资本]要花费租地农场主60镑,13+(1/3)吨[工资]要花费40镑,而作为市场价值超过租地农场主产品的[个别]价值的余额,作为地租出现的这16+(2/3)吨,就等于50镑。

等级II的13+(1/3)个工人提供多少吨呢?这里20个工人提供65吨,因而13+(1/3)个工人提供43+(1/3)吨。产品的价值和前面一样,等于100镑。但是这43+(1/3)吨中有33+(1/3)吨必须用于补偿资本。剩下作为剩余产品或地租的是43+(1/3)-[33+(1/3)]=10吨。

这10吨地租可以这样来解释:

II的产品价值等于100镑,产品是43+(1/3)吨,因而每吨价值=2+(4/13)镑。也就是说,13+(1/3)个工人花费30+(10/13)镑,[在支付13+(1/3)吨市场价值的40镑中]作为剩余价值剩下9+(3/13)镑。其次,20吨不变资本花费46+(2/13)镑,从支付它们的[市场价值的]60镑中剩下13+(11/13)镑。这和剩余价值加在一起就是23+(1/13)镑,正好相当于10吨的[实际]价值[每吨按2+(4/13)镑计算]。

只有等级Ia,为了补偿不变资本和工资,确实需要有33+(1/3)吨或夸特的实物,即需要全部总产品,因而事实上既没有剩余价值,也没有剩余产品,既没有利润,也没有地租。如果情况不是这样,如果得到的产品比以实物形式补偿资本所必需的多,就会发生利润(剩余价值)和资本向地租的转化。只要以前在价值较低时本来必须用来补偿资本的那部分产品中有一部分现在游离出来,或者本来必须转化为资本和剩余价值的那部分产品现在归入地租份内,在这里就会发生这种转化。

但是,同时我们也看到,不变资本的涨价如果是农产品涨价的结果,那就会使地租大大降低。例如,III和II的地租从[E表的]50吨(在每吨市场价值3镑时合150镑)降到26+(2/3)吨,也就是几乎降了一半。这种降低是必然的,[687]因为在这里同一资本100镑所使用的工人人数由于后面两个原因减少了:第一,因为工资提高,即可变资本的价值增加;第二,因为生产资料即不变资本的价值提高。工资提高本身的结果是100镑所使用的劳动少了,因此(在加入不变资本的商品价值不变的情况下)不变资本也相应地减少,所以整个这100镑总合起来就代表较少的积累劳动和较少的活劳动。但是,除此以外,加入不变资本的商品价值提高带来的结果就是,因为积累劳动和活劳动之间的技术比例不变,现在花费同量的货币能够使用的积累劳动量少了,因此能够使用的活劳动量也少了。因为在土地的生产率相同和资本技术构成既定的情况下,总产品量取决于所使用的劳动量,所以,随着使用的劳动量减少,地租也必然减少。

这种情况只有在利润消失的时候才会表现出来。当利润还存在的时候,尽管所有等级的产品都绝对减少,正如第681页上的表\fnote{见本册第512—513页。——编者注}所说明的,地租仍能增加。一般来说,很明显,在只有地租存在时,随着产品的减少,从而随着剩余产品的减少,地租本身也必然减少。如果不变资本的价值同可变资本的价值一起增长,那末这种情况一开始就会更快地出现。

但是,除此以外,第681页上的表还说明,在农业生产率降低的情况下,随着级差地租的增长,即使在较好等级的土地上,与一定量的预付资本,比如说100镑相比,总产品量也总是减少的。李嘉图对这一点毫无所知。利润率降低,是因为同一资本,比如说100镑所推动的劳动量越来越少,而支付这一劳动的费用贵了,从而用来积累的余额越来越小。但是在生产率既定的条件下,实际得到的产品量也和剩余价值一样,取决于资本所使用的工人人数。李嘉图没有看到这一点,就象他没有看到地租的形成不仅依靠剩余价值转化为地租,而且依靠资本转化为剩余价值一样。当然,资本这样转化为剩余价值只是表面上的。如果市场价值由III等等的产品价值决定,那末剩余产品的每一个极小部分就都代表剩余价值,代表剩余劳动。其次,李嘉图总是只注意到为了生产同样数量的产品,必须使用更多的劳动,但是他忽略了对确定利润率以及所生产出来的产品量有决定意义的东西,那就是,花费同一资本所使用的活劳动量会不断减少,在这种活劳动量中[假定按下降序列]成为必要劳动的部分会越来越大,成为剩余劳动的部分会越来越小。

从这一切可以得出结论说,即使把地租只看成是级差地租,李嘉图在这个问题上也丝毫没有比他的前辈前进一步。他在这方面的重大功绩就是德·昆西所指出的,对问题作了科学的阐述。但是李嘉图在解决问题的时候接受了他的前辈的观点。德·昆西说:

\begin{quote}{“李嘉图给地租学说增添了新的东西:他把地租学说归结为地租是否真的取消价值规律的问题。”(托·德·昆西《政治经济学逻辑》1844年爱丁堡和伦敦版第158页)}\end{quote}

其次,在该书第163页,德·昆西说:

\begin{quote}{“地租是土地(或其他任何生产因素)的产品的一部分,这一部分是为了使用土地的各种不同的力而支付给土地所有者的,而土地的各种不同的力是通过与同一市场上出现的同类因素的力相比较而衡量出来的。”}\end{quote}

接着在第176页,德·昆西写道:

\begin{quote}{“有人反驳李嘉图说,一等地的所有者不会白白地交出土地。但在只耕种一等地的时期{在这个神话时期!}还不能形成与土地所有者阶级不同的租地农场主和租佃者的特殊阶级。”}\end{quote}

[688]因此,在德·昆西看来,这个“土地所有权”规律只是在任何现代意义上的土地所有权都不存在的时候才起作用。

现在我们转过来谈摘自李嘉图著作的引文。

\tsubsectionnonum{[(d)在农产品价格同时提高的情况下利润率提高的历史例证。农业劳动生产率增长的可能性]}

(关于级差地租,首先还要作如下说明:事实上上升序列和下降序列是相互交替、相互交叉、彼此交织在一起的。

但这决不是说,如果在个别短暂时期(例如从1797年到1813年)下降序列的运动占很大优势,利润率因此就必然下降(就利润率由剩余价值率决定的情况而言)。相反,我认为,在1797年到1813年那个时期,在英国虽然小麦和一般农产品的价格都急剧上涨,但利润率还是异乎寻常地提高了。据我所知,没有一个英国统计学家不认为上述时期利润率提高了。有些经济学家,例如查默斯、布莱克等人,曾引用这一事实来证实自己的理论。首先我还必须指出,企图用货币贬值来解释这一时期发生的小麦价格的上涨,是十分荒谬的。研究过这一时期商品价格史的人没有一个会同意这种观点。此外,远在货币发生任何贬值以前,价格就开始上涨,并且达到很高的程度。在货币贬值以后就应当纯粹从价格中作相应的扣除。如果要问,为什么谷物价格上涨了,利润率还会提高?这可以用下述情况来解释:工作日延长,这是采用新机器的直接后果;加入工人消费的工业品和殖民地商品跌价;工资降低(虽然名义工资提高了),降到它的普通平均水平以下{这一事实对所考察的时期来说,是大家公认的;帕·詹·斯特林(在1846年爱丁堡出版的《贸易的哲学》一书中)大体上接受了李嘉图的地租理论,但他企图证明,谷物经常的(不是由偶然的歉收造成的)涨价的直接后果总是平均工资降低\endnote{帕·詹·斯特林《贸易的哲学,或利润和价格理论概要》1846年爱丁堡版第209—210页。——第25、525页。}};最后,利润率的提高还可以这样来解释:由于公债和国家的开支,对资本的需求比资本的供给增加得快,这就引起商品名义价格的提高,因此工厂主就从地租的搜刮者和其他领取固定收入的人那里把以地租等形式支付的那部分产品的一部分夺了回来。这种行动不是我们这里研究的对象,我们这里考察的是基本关系,因此我们只注意三个阶级,即土地所有者阶级、资本家阶级和工人阶级。但是正象布莱克所指出的那样\endnote{马克思指的是威·布莱克的书:《论限制现金支付期内政府支出的影响》1823年伦敦版。与正文中涉及的问题有关的该书摘录以及马克思的评语,见马克思1857—1858年的经济学手稿(见卡·马克思《政治经济学批判大纲》1939年莫斯科版第672—673页)。——第526页。},在相应的情况下,这种行动在实践中起着相当大的作用。)[688]

\centerbox{※     ※     ※}

\begin{quote}{[689]{布莱顿的哈利特先生在1862年的博览会\endnote{指1862年5月1日在伦敦开幕的国际博览会,会上展出了工农业产品的样品,艺术作品和科学新成就。——第526页。}上展出了“小麦良种”。“哈利特先生断言,谷物的穗也和跑马一样需要细心照管,然而往往培育得很马虎,不注意自然选择的理论。现举几个突出的例子来说明怎样才能管好小麦。1857年哈利特先生种出了一穗质量最好的红小麦,穗长4+(3/8)英寸,有47颗籽粒。1858年哈利特先生从他得到的不多的收成中,又选了最好的一穗,长6+(1/2)英寸,有79颗籽粒;1859年又同样从中选出最好的一穗,这次穗长7+(3/4)英寸,有91颗籽粒。次年即1860年,对耕作技术的试验不利,麦穗没有长得更好更大;但是一年以后,即1861年,最好的穗长8+(3/4)英寸,一根茎秆上有123颗籽粒。这样,在五年中麦穗的长度增加了一倍,而籽粒的数量几乎增加了两倍。所以能够获得这样的结果,是由于采用了哈利特先生所说的小麦培育的‘自然方法’,即各颗籽粒前后左右都相距约9英寸,使每颗籽粒都有充分发展的余地……哈利特先生断定,如果播种‘良种小麦’并且采用培育的‘自然方法’,英国的谷物生产可以增加一倍。他声称,下种适时,每平方英尺的土地只播种一粒,他从每粒种子收获的作物平均有23穗,每穗约有36颗籽粒。根据这种情况,一英亩土地的产品按精确的计算是1001880穗,而用普通方法播种,每英亩所费种子量达二十倍以上,却只能收934120穗小麦,即少67760穗……”}}\end{quote}

\tsubsectionnonum{[(e)李嘉图对利润率下降的解释以及这种解释同他的地租理论的联系]}

[李嘉图是这样证明利润率的下降的:]

\begin{quote}{“随着社会的进步,劳动的自然价格总是有上涨的趋势,因为调节劳动自然价格的一种主要商品由于生产困难不断增大而有涨价的趋势。但是,由于农业的改良和可以提供进口粮食的新市场的发现,能在某一个时期内阻止必需品价格上涨的趋势,甚至能使其自然价格下降,所以,这些同样的原因也会对劳动的自然价格产生相应的影响。除原产品和劳动外,一切商品的自然价格都有随着财富和人口的增加而下降的趋势;因为,一方面它们的实际价值虽然会由于制造它们所用的原料的自然价格上涨而增加,但是,机器的改良、劳动分工和劳动分配的改进、生产者在科学和技艺方面熟练程度的提高,会把这种增加的趋势抵销而有余。”(同上,第86—87页)“随着人口的增加,必需品的价格将不断上涨,因为生产它们需要较多的劳动……所以,工人的货币工资不会下降,而会提高,但提高的程度却不足以使工人能够买到商品涨价前他能买到的那样多的舒适品和必需品……尽管工人的报酬实际上比以前差了,工人工资的这种增加还是必然会减少工厂主的利润,因为工厂主不能按较高的价格出卖他的商品,而这些商品的生产费用却提高了……因此,可以看出,使地租提高的同一原因(即用同一比例的劳动量来生产追加的食物量越来越困难),也会使工资提高。所以,在货币价值不变的情况下,地租和工资都会有随着财富和人口的增加而提高的趋势。但是,地租的增加和工资的增加有根本的区别。地租的货币价值提高时,[690]它在产品中所占的份额也会随之增加:不仅土地所有者的货币地租会增加,而且他的谷物地租也会增加……工人不会这样幸运:不错,他得到的货币工资会更高,但以谷物来表示,他的工资却减少了;这时不仅他支配的谷物数量会减少,而且他的一般状况也会恶化,因为他将发现,市场工资率要保持在自然工资率以上是更困难了。”(同上,第96—98页)“假定谷物和工业品始终按同一价格出售,那末利润的高低就会同工资的高低成反比。但是,我们假定谷物价格提高是因为生产谷物需要更多的劳动量;这一原因并不会使工业品的价格提高,因为生产工业品不需要追加劳动量……如果工资随着谷物价格上涨而提高(这是肯定会发生的),那末他们[工厂主]的利润就一定会下降。”(同上,第108页)“但是人们也许要问,租地农场主虽然要支付一个工资的追加额,他是否至少还能得到以前的利润率呢?当然不能,因为他不仅要给他所雇用的每个工人支付较高的工资(就象工厂主所要做的那样),而且要支付地租,或者为了得到同量产品而使用追加工人。而原产品价格的上涨只会与这种地租或与追加的工人人数相适应,它不会补偿由于工资的提高而给租地农场主带来的损失。”(同上,第108页)“我们已经指出,在社会发展的早期阶段,在土地产品的价值中,无论是土地所有者还是工人所占的份额都不大,他们所占的份额是随着社会财富的增长以及生产食物困难的增加而成比例地增长的。”(第109页)}\end{quote}

这是关于“社会发展的早期阶段”的一个奇特的资产阶级幻想。在这种早期阶段,劳动者或者是奴隶,或者是靠自己劳动生活的农民等等。在第一种场合,他和土地一起属于土地所有者;在第二种场合,他就是他自己的土地所有者。在这两种场合,都没有介入土地所有者和农业工人之间的资本家。农业从属于资本主义生产,从而奴隶或农民变为雇佣工人,以及在土地所有者和工人之间介入了资本家,——这一切只不过是资本主义生产的最后结果,而李嘉图却看做是“社会发展的早期阶段”所固有的现象。

\begin{quote}{“因此,利润有下降的自然趋势,因为随着社会的进步和财富的增长,为了生产必需的追加食物量,必须花费越来越多的劳动。利润的这种趋势,这种可以说是重力作用,幸而由于生产必需品所使用的机器的改良以及农业科学上的发现而时常受到抑制,这些改良和发现使我们能够减少一部分以前所需要的劳动量,因而能降低工人生活必需品的价格。”(同上,第120—121页)}\end{quote}

李嘉图的下面一段话就直截了当地说,他所理解的利润率就是剩余价值率:

\begin{quote}{“虽然生产了一个较大的价值,但这一价值在支付地租以后剩下的部分中却有较大的份额是由生产者消费的,而这一点,并且只有这一点,却调节着利润的大小。”(同上,第127页)}\end{quote}

这就是说,撇开地租不谈,利润率等于商品价值超过在生产这种商品的时期所支付的劳动价值的余额,或者说,超过生产者所消费的那部分商品价值的余额。李嘉图在这里只把工人叫做生产者\endnote{马克思在这里再一次指出,李嘉图的《原理》中这个地方的“生产者”(《producer》)一词是指“工人”(马克思在前面第478页上第一次指出李嘉图著作中把“生产者”和“工人”这两个概念等同起来)。在李嘉图著作的另外一些地方,“生产者”这个词指“产业资本家”(例如正文第480、487和627页上李嘉图著作的引文)。——第528页。}。他认为所生产的价值是他们生产的。因此,在这里他把剩余价值解释为工人自己生产的价值中为资本家生产的部分。\fnote{[691}关于剩余价值的来源,李嘉图说:“……资本在货币形式上是不会产生利润的;而在它所能交换的材料、机器和食物的形式上,却可以产生收入。”(同上,第267页)“有价证券持有者的资本[692]决不可能成为生产资本——这实际上根本不是资本。如果有价证券持有者想把有价证券卖掉,并将卖得的资本生产地加以使用,他就只有使购买他的有价证券的人的资本离开某种生产用途才能做到。”(第289页注)[692]]

但是,如果李嘉图把剩余价值率和利润率等同起来,如果他同时又假定(而他正是这样做的)工作日的长度是既定的,那末利润率下降的趋势就只能用引起剩余价值率下降的那些原因来解释。在工作日的长度既定时,剩余价值率只有在工资率不断提高的条件下才可能下降。而工资率的不断提高只有在必需品的价值不断提高的情况下才可能发生,必需品价值的提高又只有在农业生产条件不断恶化的情况下,也就是在假定李嘉图的地租理论是正确的条件下才会发生。因为李嘉图把剩余价值率和利润率等同起来,[691]又因为剩余价值率只是按照它同花费在工资上的可变资本之比来计算的,所以李嘉图也和亚·斯密一样,假定全部产品的价值扣除地租之后,是在工人和资本家之间分配的,也就是说,分为工资和利润。换句话说,李嘉图作了一个错误的假定:全部预付资本只由可变资本构成。例如,在前面引用的那段话后面,他接着说:

\begin{quote}{“当较坏土地投入耕种时,或者当花费在老地上的追加的资本量和劳动量收益减少时,上述影响将是持久的:支付地租后剩下的、要在资本所有者和工人中间进行分配的那部分产品中,将有更大的份额归工人所得。”(同上,第127—128页)}\end{quote}

李嘉图紧接着又说:

\begin{quote}{“每个工人所得到的产品的绝对量也许会、甚至很可能会减少;但是,因为同租地农场主留下的全部产品相比,雇用的工人会增加,所以在全部产品中为工资所吸收的那部分价值会增大,因而产品中用来支付利润的那部分价值会减少。”(第128页)}\end{quote}

在前面不远的地方,李嘉图写道:

\begin{quote}{“土地产品在支付土地所有者和工人以后余下的数量,必然属于租地农场主,成为他的资本的利润。”(第110页)}\end{quote}

李嘉图在《论利润》那一部分(第六章)的结尾说,即使假定商品价格随着工人货币工资的提高而一起提高,——这是错误的假定,——他对利润率下降的分析仍然是正确的:

\begin{quote}{“在论工资的那一章,我们已经力求说明……商品的货币价格不会由于工资提高而提高。但是即使情况不是这样,即使高工资引起商品价格持久上涨,认为高工资必然会影响那些使用雇佣劳动的人,使他们失去一部分实际利润的说法仍然是正确的。假定制帽业者、织袜业者、制鞋业者在生产一定量商品时,每人多付10镑工资,而帽子、袜子和鞋子的价格上涨的总额足以补偿他们各人的这10镑。在这种情况下,他们的景况并不会比商品价格没有提高时好些。如果织袜业者的袜子卖得110镑,而不是100镑,他的利润的货币额就恰好和以前一样;但是因为他用这一相同的货币额换得的帽子、鞋子和其他一切商品的数量将会少十分之一,因为他用过去积蓄的数额〈即用同样的资本〉所能雇用的工人会由于工资提高而减少,所能购买的原料也会由于原料价格上涨而减少,所以他的景况并不会比他的货币利润总额实际减少而一切工业品价格不变的时候好些。”(第129页)}\end{quote}

李嘉图在其他地方论证问题的时候总是只强调,在较坏的土地上,为了生产同量的产品必须雇用数量更多的工人,而在这里,他终于提出了对利润率具有决定意义的因素,那就是,用同量资本所雇用的工人由于工资提高而减少。在其他方面,他并不完全正确。如果帽子等等的价格提高10%,资本家的景况不会改变,但是土地所有者在购买这一切商品的时候必须从他的地租中付出较大的数目。例如他的地租从10镑增加到20镑。但是他用这20镑买得的帽子等等的数量,比以前用10镑买得的成比例地减少了。

李嘉图说得完全对:

\begin{quote}{“在社会向前发展的情况下,土地的纯产品同土地的总产品相比,会不断减少”。(第198页)}\end{quote}

李嘉图的这个论点的意思是,在社会向前发展的情况下,地租不会提高。[纯产品同总产品相比会减少的]真实原因在于,在社会向前发展的情况下,可变资本同不变资本相比会减少。[691]

[692]随着生产的进步,不变资本同可变资本相比会增加,这一点李嘉图自己也承认,不过他采取的形式是,固定资本同流动资本相比会增加:

\begin{quote}{“在富强的国家,大量的资本都投在机器上;而在较贫穷的国家,按比例来说,固定资本少得多,流动资本多得多,因而很多工作要靠人的劳动来进行。因此,在富强的国家,商业和工业上的突然变动所带来的灾难,比在较贫穷的国家大。把流动资本从使用它的部门中抽出来,不象固定资本那样困难。为一个工业部门制造的机器,往往完全不能用于其他工业部门;相反,一个部门的工人的衣服、食物和住房却可以用来维持另一个部门的工人的生活}\end{quote}

(因此,这里的流动资本只能理解为用于工资的可变资本),

\begin{quote}{或者说,同一个工人虽然改换了自己的职业,但可能得到同样的食物、衣服和住房。然而,这是富裕国家必须容忍的不幸;为这种不幸而埋怨,就好比一个富商为了他的船只在海上会遇到各种危险,可是他的穷邻居的茅屋完全没有这种危险而长吁短叹一样,是没有道理的。”(同上,第311页)}\end{quote}

李嘉图自己指出了一个与农产品价格提高完全无关的地租提高的原因:

\begin{quote}{“任何固定在土地上的资本,到租佃期满时,都必然属于土地所有者而不属于租地农场主。土地所有者在重新出租他的土地时由于这一资本而得到的任何报酬都将以地租形式出现。但是,如果用一定量资本能从国外购得的谷物比在国内这种土地上生产的多,那就不会有人支付地租。”(同上,第315页注)}\end{quote}

关于同一个问题,李嘉图说:

\begin{quote}{“在本书的前面一个部分,我曾经指出本来意义的地租和土地所有者因支出自己的资本给租地农场主带来各种好处而在地租名义下得到的报酬之间的区别。但是我也许还没有充分说明由于这种资本的使用方法不同而产生的区别。因为这种资本的一部分一旦用来改良农场,就同土地不可分离地结合在一起,并会提高土地的生产力,所以,为了使用土地而支付给土地所有者的报酬完全具有地租的性质,并且受一切地租规律的支配。无论这种改良是由土地所有者还是由租地农场主出钱进行,除非从改良的土地上得到的收益很可能同其他任何等量投资所能获得的利润至少相等,否则这种改良一开始就不会进行。但是只要进行这种改良,以后从改良的土地上得到的收益就会完全具有地租的性质,并将经历地租所经历的一切变动。但是这种费用中有些只能在有限的时期内改良土地,不能长久地提高土地的生产力:比如说,这种费用如果用于建筑物或其他临时性的改良,就需要不断更新,因此它不能使土地所有者的实际地租持久地增加。”(第306页注)}\end{quote}

李嘉图说:

\begin{quote}{“在任何国家,在任何时候,利润都取决于在不提供地租的土地上或者用不提供地租的资本生产工人必需品所需要的劳动量。”(第128页)}\end{quote}

根据这种观点,租地农场主在李嘉图所说的不支付地租的土地即最坏的土地上的利润,调节一般利润率。李嘉图是这样推论的:最坏土地的产品按其价值出卖,并且不会带来任何地租;因此,这里可以看得很清楚,在扣除了只是作为给工人的等价物的那部分产品价值以后,留给资本家的剩余价值究竟有多少;而这种剩余价值就是利润。这种推论所依据的前提是,费用价格和价值是等同的,因为这一产品是按费用价格出卖的,所以它是按照价值出卖的。

从历史上和理论上来看,这是不正确的。我曾经指出\fnote{见本册第330页。——编者注},在资本主义生产和土地所有权存在的地方,最坏的土地或最次的矿井之所以能够不提供地租,只是因为在这种场合,它的产品按市场价值(不由这种产品本身调节)出卖时,是低于它的[个别]价值出卖的。因为这里产品的市场价值正好抵补它的费用价格。但是这种费用价格由什么调节呢?由非农业资本的利润率调节,自然,谷物价格也参与决定利润率,不过绝不能说,利润率仅仅是由谷物价格决定的。李嘉图的论断只有在价值和费用价格等同的情况下才是正确的。[693]从历史上看——由于资本主义生产在农业上比在工业上出现得晚些——也是农业利润由工业利润决定,而不是相反。只有说,在提供利润而不提供地租、其产品按费用价格出卖的最坏的土地上,平均利润率会出现,会明显地表现出来,那才是正确的,但如果说平均利润是由此调节的,那就完全不正确了。这完全是另一回事。

利息率和地租率不提高,利润率也可能下降。

\begin{quote}{“从我们对资本的利润所作的分析中可以看出,如果没有某种引起工资提高的持久的原因,任何资本积累都不能使利润\fnote{李嘉图在这里所说的利润,是指资本家拿去的那部分剩余价值,但决不是全部剩余价值。认为由于资本积累,剩余价值可能降低,那是错误的,而对利润[率}来说,却是正确的。]持久地降低……如果工人消费的必需品的量能够持久地、同样容易地增加,那末无论资本积累达到什么程度,利润率或工资率〈应当说剩余价值率和劳动价值率〉都不会有经常的变动。但是,亚当·斯密却把利润的下降完全归因于资本的积累和由此产生的竞争,而从来不去注意为追加资本所雇用的追加工人提供食物的困难在日益增加。”(同上,第338—339页)}\end{quote}

这种说法只有在利润和剩余价值等同的情况下,才是正确的。

亚·斯密说,随着资本的积累,利润率会由于资本家之间的竞争日益加剧而下降;而李嘉图则说,利润率会由于农业生产条件的不断恶化(必需品涨价)而下降。我们反驳了他的观点,他的这种观点只有在剩余价值率和利润率等同的情况下,也就是说,在利润率只是因为工资率提高(假定工作日不变)才能下降的情况下,才是正确的。斯密的见解所依据的是:他(根据他的错误的、已被他自己驳倒了的价值观点)把价值看作是工资、利润和地租相加的结果。按照他的看法,资本的积累通过降低商品价格的方法迫使任意规定的、本身没有任何内在尺度的利润降低,根据这种观点,对商品价格来说,利润纯粹是一种名义上的附加额。

李嘉图反驳斯密说,资本的积累不会使商品的价值规定发生变动,这一论据在理论上自然是正确的;但是,李嘉图企图用一个国家不可能发生生产过剩这一点来反驳斯密,这就大错特错了。李嘉图否认资本过多的可能性,但在他以后的时期,这种可能性在英国的政治经济学上已经成为公认的原理了。

第一,李嘉图没有看到,在现实生活中不仅资本家和工人彼此对立,而且[产业]资本家、工人、土地所有者、货币资本家、从国家领取固定收入的人等等,都彼此对立;在这里,商品价格的下降,使产业资本家和工人双方都受到打击,而对其他阶级却有利。

第二,李嘉图没有看到,资本主义生产决不是以随便什么样的规模进行都行的,资本主义生产越是发展,它就越是不得不采取与直接的需求无关而取决于世界市场的不断扩大的那样一种规模。李嘉图求助于萨伊的荒谬的前提,似乎资本家进行生产不是为了利润,不是为了剩余价值,而是直接为了消费,为了使用价值——为了他自己的消费。李嘉图没有看到,商品必须转化为货币。工人的需求是不够的,因为利润之所以存在,正是由于工人所能提出的需求小于他们的产品的价值,而相对说来,这种需求越小,利润就越大。资本家彼此提出的需求同样是不够的。生产过剩不会引起利润的持续下降,但是它经过一定时期会不断重复。随着生产过剩,就出现生产不足等等。生产过剩的起因恰好在于:人民群众所消费的东西,永远也不可能大于必要生活资料的平均数量,因此人民群众的消费不是随着劳动生产率的提高而相应地增长。不过,整个这一节都属于资本竞争的问题。关于这一点,李嘉图所说的一切是毫无价值的。(这就是第二十一章《积累对于利润和利息的影响》。)

\begin{quote}{“只有一种情况可能引起利润率在食物价格低廉时随着资本的积累而下降,那就是维持劳动的基金比人口增加快得多,这时工资高,而利润率却低;但这种情况也只具有暂时的性质。”(第343页)}\end{quote}

李嘉图[在《积累对于利润和利息的影响》这一章中]就利润和利息的关系讽刺萨伊说:

\begin{quote}{“萨伊先生承认利息率取决于利润率;但由此不能得出结论说,利润率取决于利息率。前者是因,后者是果,任何情况都不能使因果倒置。”(同上,第353页注)}\end{quote}

但是,使利润下降的那些原因能够使利息提高,反过来也是一样。\endnote{这里引用的李嘉图对萨伊关于利润和利息之间关系的观点的评论,马克思在他的手稿第736页上再次引用了,但把它当作与第736页所谈的问题无关的东西放在方括号里,并在李嘉图的结束语(“任何情况都不能使因果倒置”)后面反驳了一句:“最后这句话‘在某种情况下’肯定是不正确的。”马克思在《资本论》第三卷(第二十二章)中指出了在资本主义周期的不同阶段上利润率和利息率相互对立运动的可能性。马克思写道:“如果我们考察一下现代工业在其中运动的周转周期……我们就会发现,低利息率多数与繁荣时期或有额外利润的时期相适应,利息的提高与繁荣到周期的下一阶段的过渡相适应,而达到高利贷极限程度的最高利息则与危机相适应。”(见马克思《资本论》第3卷第22章)。——第535页。}

\centerbox{※     ※     ※}

[在《论殖民地贸易》一章中,李嘉图写道:]

\begin{quote}{“萨伊先生承认,生产费用是价格的基础,但他在他的著作的不同地方却说价格是由供求关系调节的。”(同上,第411页)}\end{quote}

[否认需求和供给的决定性作用的]李嘉图本来应该从[萨伊把生产费用的见解同需求和供给的见解结合起来的]这种论点中看到,[694][萨伊所谓的]生产费用与用于生产某种商品的劳动量是大不相同的。但他没有这样做,却继续说:

\begin{quote}{“真正地和最后地调节任何两种商品的相对价值的,是它们的生产费用”。(同上)}\end{quote}

[在《论殖民地贸易》这一章中,李嘉图写道:]

\begin{quote}{“亚当·斯密说:‘商品的价格,或者说,金银同商品相比较的价值,取决于使一定量金银进入市场所必需的劳动量和使一定量任何其他商品进入市场所必需的劳动量之间的比例。’他说这句话时难道不是同意这种观点{价格既不是由工资调节,也不是由利润调节}吗?不论利润是高还是低,也不论工资是低还是高,这种劳动量都不会变动。所以,高额利润怎么能够提高商品的价格呢?”(第413—414页)}\end{quote}

在上面引用的这段话里,亚·斯密所说的价格无非是指商品价值的货币表现。商品的价值以及用来交换商品的金银的价值由生产这两类商品{一方面是商品,另一方面是金银}所需要的劳动的相对量决定,这一事实同“高额利润能够提高”商品的实际价格即商品的费用价格这一点决不矛盾。当然,不是象斯密所想的那样,一下子全都如此。但是由于高额利润,的确会有一部分商品的价格比平均利润水平低时更高于这些商品的价值,而另一部分商品的价格则比利润低时低于它们的价值的程度要小些。\endnote{马克思这里回到本册第425—426页和第495—497页上所谈的问题,即从殖民地贸易和一般对外贸易中得到的比在宗主国得到的更高的利润对平均利润率因而对费用价格的影响问题。正如马克思所指出的,在这个问题上,亚当·斯密所持的观点比李嘉图正确。并参看马克思《资本论》第3卷第14章。——第536页。}

\tchapternonum{[第十七章]李嘉图的积累理论。对这个理论的批判。从资本的基本形式得出危机}

\tsectionnonum{[(1)斯密和李嘉图忽视不变资本的错误。不变资本各部分的再生产]}

我们先把李嘉图分散在全书中的论点搜集在一起。

\begin{quote}{“……一个国家的全部产品都是要消费掉的,但究竟由再生产另一个价值的人消费,还是由不再生产另一个价值的人消费,这中间有难以想象的区别。我们说收入节约下来加入资本,我们的意思是,加入资本的那部分收入,是由生产工人消费,而不是由非生产工人消费。〈李嘉图这里所说的区别,也是亚·斯密所说的区别。〉认为资本由于不消费而增加,那就大错而特错了。如果劳动价格大大提高,以致增加资本也无法使用更多的劳动,那我就要说,这样增加的资本仍然是非生产地消费的。”(第163页注)}\end{quote}

可见,这里全部问题只归结为,产品是由工人消费还是不由工人消费。这和亚当·斯密等人的看法一样。而实际上,这里必定也涉及这样一些商品的生产消费,这些商品构成不变资本并作为劳动工具或劳动材料被消费,或者说,这些商品通过消费转化为劳动工具和劳动材料。认为资本积累是收入转化为工资,就是可变资本的积累,这种见解从一开始就是错误的,也就是片面的。这样,对整个积累问题就得出了错误的解释。

首先,必须弄清不变资本的再生产。我们在这里就考察年再生产,也就是把一年作为再生产过程的时间尺度。

不变资本的很大一部分——固定资本——加入年劳动过程,但不[全部]加入年价值形成过程。[不加入价值形成过程的这部分]固定资本不会被消费。所以这部分固定资本不需要再生产。由于它一般加入生产过程并同活劳动接触,它就被保存下来,而且它的交换价值也同它的使用价值一起被保存下来。一个国家当年的这部分资本愈大,下一年这部分资本的纯粹形式上的再生产(保存)相对地也就愈大;假定生产过程即使只以原来的规模更新、继续、前进,情况就是如此。修理和为保存固定资本所必需的其他一切,我们算在原来花费在固定资本上的劳动费用中。这与上述意义上的保存毫无共同之处。

不变资本的第二部分每年在商品生产中会被消费掉,因此必须再生产出来。这里包括每年加入价值形成过程的那部分固定资本的全部,还包括由流动资本构成的那部分不变资本的全部,即原料和辅助材料。

至于不变资本的这第二部分,还应当进一步加以区分。

[695]在一个生产领域内表现为不变资本——劳动资料和劳动材料——的东西,有很大一部分同时就是某个并行的生产领域的产品。例如,棉纱是织布业者不变资本的一部分;棉纱又是纺纱业者的产品,也许前一天它还在制造过程中。这里所说的同时,是指在同一年内进行生产。在同一年内,同一些商品在其不同阶段通过不同的生产领域。它们作为产品从一个领域出来,又作为形成不变资本的商品进入另一个领域。而且它们全都作为不变资本在这一年内被消费掉:或者是作为固定资本只以它们的价值加入商品,或者是作为流动资本连它们的使用价值也加入商品。当一个生产领域生产出来的商品加入另一个生产领域,在这里作为不变资本被消费的时候,在有这一种商品加入的生产领域的序列之外,又有这种商品的不同要素或它的不同阶段同时并行地被生产出来。在同一年内,它不断在一个领域作为不变资本被消费掉,又不断在另一个并行的领域作为商品被生产出来。这样作为不变资本在一年内被消费的同一些商品,又同样不断在同一年内被生产出来。机器在A领域被磨损,同时会在B领域被生产出来。生产生活资料的生产领域在一年内所消费的不变资本,会同时在另一些生产领域被生产出来,因而会在一年内或在年终以实物形式重新得到补偿。无论是生活资料还是这部分不变资本,两者都是新的劳动、在一年内发挥作用的劳动的产品。

我在前面曾经说明\fnote{见本卷第1册第111—126和238—248页。——编者注},生产生活资料的那些生产领域的产品的一部分价值,即补偿这些生产领域的不变资本的那部分价值,是怎样形成这种不变资本的生产者的收入的。

但是,还有一部分不变资本,它每年都被消费掉,却不作为组成部分加入生产生活资料(供[个人]消费的商品)的那些生产领域。因此,这一部分也不能从这些领域中得到补偿。我们指的是不变资本——劳动工具、原料、辅助材料——的一部分,就是在不变资本——机器、原料和辅助材料——本身的形成过程即生产过程中用于生产消费的那部分。我们已经看到\fnote{见本卷第1册第126—140、182—195和248—258页。——编者注},这一部分是以实物形式得到补偿的,或者直接由这些生产领域本身的产品(例如种子、牲畜、一部分煤炭)来补偿,或者通过不同生产领域的那些形成不变资本的产品的一部分[在生产资料生产者之间]进行交换来补偿。这里就发生资本同资本的交换。

由于这部分不变资本的存在和消费,不仅产品量增加了,而且年产品的价值也增大了。和这部分消费掉的不变资本的价值相等的那部分年产品价值,会把必须以实物形式补偿消费掉的不变资本的那一部分,以实物形式从年产品中买回或者抽回。例如,播种时由种子构成的那部分价值,决定着收获时必须作为不变资本归还给土地即归还给生产的那部分价值(同时也决定着谷物量)。没有一年内新加的劳动,这部分就不能再生产出来;但在事实上,这部分是由去年的劳动或[一般说来]过去的劳动生产的,而且——如果劳动生产率不变——由这部分加在年产品中的价值,并不是当年劳动的结果,而是去年劳动的结果。一国使用的不变资本的比例愈大,生产不变资本所消费的那部分不变资本也就愈大,这部分不变资本不仅表现为较大的产品量,而且使这个产品量的价值提高。可见,这部分价值不仅是现在劳动、当年劳动的结果,而且同样是去年劳动、过去劳动的结果,虽然没有当年的直接劳动,它就不能重新出现,正如它所加入的产品不能出现一样。如果这部分不变资本增加了,那末不仅年产品量会增加,而且年产品的价值也会增加,即使年劳动量保持不变。这种增加就是资本积累的形式,理解这种形式非常重要。可是李嘉图的下述论点简直和这种理解相差太远了:

\begin{quote}{“工业中100万人的劳动总是生产出相同的价值,但并非总是生产出相同的财富。”(同上,第320页)}\end{quote}

假定工作日是既定的,这100万人不仅会因劳动生产率不同而生产出极不相同的商品量,而且这个商品量也会由于生产它时花费的不变资本的大小不同,从而由于加到它上面的,由去年劳动、过去劳动创造的价值的大小不同,而具有极不相同的价值。

\tsectionnonum{[(2)不变资本的价值和产品的价值]}

在这里,凡是谈到不变资本的再生产的地方,为简单起见,我们总是先假定劳动生产率不变,因而生产方式也保持不变。在生产规模既定的情况下,应当作为不变资本来补偿的是一定量的实物形式的产品。如果生产率不变,这个量的价值[696]也就保持不变。如果劳动生产率发生变动,因而把同量产品再生产出来,可能付出较贵或较廉的代价,花费较多或较少的劳动,那末,不变资本的价值也就发生变动,这种变动会影响产品在扣除不变资本以后剩下的余额的大小。

例如,假定播种需要20夸特[小麦],每夸特3镑,共计60镑。如果再生产一夸特所花费的劳动减少1/3,每夸特就只值2镑。应当作为播种费用从产品中扣除的仍然是20夸特,但它们在全部产品的价值中所占的部分现在只等于40镑。这样,为补偿同量不变资本就只需要总产品的一个较小的价值部分和总产品的一个较小的实物部分,虽然作为种子归还给土地的仍然应当是20夸特\endnote{这个例子是根据这样的假定,即在劳动生产率提高的情况下,从20夸特小麦的种子得到的收成比以前增加50%。例如,以前收成是100夸特,现在花费同量劳动,收成是150夸特,但是这150夸特和以前100夸特的价值一样,即300镑。以前种子占收成的20%(无论就夸特数来说,还是就价值来说都是这样),现在只占[13+(1/3)]%。——第541页。}。

如果每年消费的不变资本在一个国家是1000万镑,在另一个国家只是100万镑,而100万人一年内新加的劳动表现为1亿镑,那末产品价值在前一个国家就是11000万镑,在后一个国家就只是10100万镑。在这种情况下,第一个国家的单位商品不但可能而且毫无疑问会比第二个国家便宜,因为第二个国家花费同量的[直接]劳动生产出来的商品量少得多,比10与1之差少得多。当然,和第二个国家相比,第一个国家要拿出产品的更大一部分价值,因而要拿出总产品的更大一部分,用于补偿资本。但是第一个国家的总产品也多得多。

就工业品来说,大家知道,拿英国比如说同俄国相比,100万人生产的产品,不仅数量多得多,而且产品价值也大得多,虽然英国的单位商品便宜得多。但就农业来说,看来在资本主义发达的国家和比较不发达的国家之间就不存在这样的关系。落后国家的产品比资本主义发达的国家的产品便宜。这是就货币价格来说的。然而,看来发达国家的产品比起落后国家的产品来,则是劳动量(一年内花费的劳动量)少得多的产品。例如,在英国从事农业的人口不到三分之一,在俄国从事农业的人口却有五分之四,在英国是5/15,在俄国则是12/15。这些数字不应当从字面上去理解。例如在英国,在机器制造业、商业、运输业等等非农业经济部门,有大批的人从事农业生产各要素的制造和输送,而在俄国就没有。可见,从事农业的相对人数,不能简单地由直接从事农业的人数来决定。在进行资本主义生产的国家,有许多人间接地参加这种农业生产,而在不发达的国家,这些人都是直接从属于农业的。因此,表现出来的差别要比实际的差别大。但是对于一国文明的总的水平来说,这个差别极为重要,那怕这个差别只在于,有相当大一部分参与农业的生产者不直接参加农业,而摆脱了农村生活的愚昧,属于工业人口。

首先,我们不谈这一点。其次,我们也不谈这样一种情况,就是大多数农业民族不得不低于自己产品的价值出卖产品,而在资本主义生产发达的国家,农产品的价格却提高到它的价值的水平。无论如何,有一部分不变资本的价值加入英国土地耕种者的产品的价值,却没有这样一部分不变资本的价值加入俄国土地耕种者的产品的价值。

假定这部分价值等于10个人的日劳动。再假定这个不变资本由1个英国工人推动。我所说的是农产品中不是用花费[土地耕种者的]新劳动来补偿的那部分不变资本,如农具。如果1个英国人用[等于10工作日的]不变资本生产出来的产品,需要5个俄国工人才能生产出来,如果俄国人使用的不变资本等于1工作日,那末,英国人的产品就等于10+1=11工作日,俄国人的产品就等于1+5=6工作日。如果俄国的土地比英国肥沃,以致不使用不变资本或只使用十分之一的不变资本生产出来的谷物,就和英国人使用十倍资本生产出来的一样多,那末,同量的英国谷物的价值和同量的俄国谷物的价值之比将是11∶6。如果俄国谷物每夸特卖2镑,那末英国谷物每夸特就要卖3+(2/3)镑,因为2∶[3+(2/3)]=6∶11。可见,英国谷物的货币价格和价值比俄国谷物的货币价格和价值高得多,然而英国谷物是花费较少量的[直接]劳动生产出来的,因为过去劳动无论是在产品量中,还是在产品价值中再现出来,都无须花费任何追加的新劳动。只要英国人比俄国人使用较少的直接劳动而使用较多的不变资本,并且,只要这种不变资本——它无须英国人花费什么[在花费新劳动的意义上说],虽然它曾经花费过[一定的费用],并且必须得到支付,——没有把劳动生产率提高到足以抵销俄国土壤的自然肥力的程度,英国谷物的价格和价值较高的情况就会始终存在。因此,在进行资本主义生产的国家,农产品的货币价格可能比[697]不发达的国家高,虽然实际上这种产品花费的劳动量较少。这种产品包含较多的总劳动——直接劳动加过去劳动,但再现在这种产品中的过去劳动不需要任何[新]花费。如果不是自然肥力的差别发生影响,产品就会比较便宜。[发达的资本主义国家中]工资的较高的货币价格也可以用这种情况来说明。

到现在为止,我们谈的只是现有资本的再生产。工人补偿自己的工资,同时提供剩余产品或剩余价值,剩余价值形成资本家的利润(包括地租)。工人补偿重新用作他的工资的那一部分年产品。资本家已在一年内把利润吃光,但是工人又生产了可以重新作为利润被吃掉的这部分产品。在生活资料的生产中消费的那部分不变资本,由一年内新劳动生产的不变资本来补偿。生产这部分新的不变资本的生产者,在一部分生活资料上实现自己的收入(利润和工资),这部分生活资料的价值同生产生活资料时所消费的不变资本的价值相等。最后,在生产不变资本即生产机器、原料和辅助材料时消费的不变资本,由生产不变资本的各个生产领域的总产品以实物形式或通过资本同资本的交换来补偿。

\tsectionnonum{[(3)资本积累的必要条件。固定资本的折旧及其在积累过程中的作用]}

资本增殖,即与再生产不同的资本积累,即收入转化为资本,情况又怎样呢?

为使问题简单起见,假定劳动生产率不变,生产方式没有任何变化,因此,生产同量商品需要同量的劳动,也就是说,资本增殖花费的劳动量,和去年生产同量资本花费的劳动量一样。

剩余价值的一部分必须转化为资本,而不是作为收入被消费。它必须一部分转化为不变资本,一部分转化为可变资本。它分成资本的这两个不同部分的比例,取决于资本已有的有机构成,因为生产方式不变,两部分之间的价值比例也不变。生产愈发展,转化为不变资本的那部分剩余价值,同转化为可变资本的那部分剩余价值相比,就愈大。

首先,剩余价值的一部分(以及与这一部分相应的由生活资料构成的那部分剩余产品)必须转化为可变资本,即必须用来购买新劳动。这只有在工人人数增加或工人的劳动时间延长的情况下才有可能。后一种情况例如在一部分工人人口只是半就业或三分之二就业的时候,或者在一个或长或短的时期内绝对延长工作日但必须对此支付报酬时,都会发生。但是不能把这看作是积累的经常的手段。如果原来的非生产劳动者变成生产劳动者,或者原来不劳动的那部分人口如妇女、儿童、贫民被吸收到生产过程中来,工人人口就可能增加。这里我们把后一点撇开不谈。最后,由于工人人口随着整个人口的增加而绝对增加,[就业工人的人数也可能增加。]只有在人口这样绝对增加(虽然和使用的资本相比,人口相对减少了)的条件下,积累才能成为经常的不断的过程。人口增加表现为积累这个经常过程的基础。但是这就需要有一种不仅能够再生产工人人口,而且能够使工人人口不断增加的平均工资。为了应付突然情况,资本主义生产已作了准备:它迫使一部分工人人口进行过度劳动,又使另一部分工人人口陷于赤贫或半赤贫状态,作为后备军储备起来。

然而,另一部分必须转化为不变资本的剩余价值,情况又怎样呢?为了简单起见,我们就撇开对外贸易,考察一个与外界隔绝的国家。我们举一个例子。假定一个麻织厂主生产的剩余价值等于1万镑,他想把半数即5000镑转化为资本。根据机器织布业的资本有机构成,这个金额的五分之一要花费在工资上。这里我们把资本周转撇开不谈,如果考虑到资本周转,工厂主也许只要有够五周用的金额就行了,五周之后,他把自己的产品卖出去,就可以从流通领域中把用于工资的资本收回来。我们假定,他必须把1000镑存在银行家那里,以支付(20个工人的)工资,并在一年内作为工资逐渐花完。然后,4000镑必须转化为不变资本。第一,工厂主必须购买够20个工人在一年内加工织成麻布的纱。(我们始终把资本的流动部分的周转撇开。)其次,工厂主必须增加自己工厂中的织机,也许还要添置新的蒸汽机,或者加大旧机器的功率等等。但是要买到所有这些东西,他必须在市场上找到现成的纱、织机等等。他必须把他的4000镑变成纱、织机、煤炭等等,[698]即购买所有这些东西。但要能买到这些东西,这些东西必须已经存在着。因为我们已经假定,旧资本的再生产是在原有条件下进行的,所以,为了提供织布业者上一年所需要的那么多的纱,纺纱业者必定已经支出他的全部资本。那末,他怎样才能供给更多的纱来满足追加的需求呢?

提供织机等等的机器制造厂主的情况也正是这样。他生产的新织机数量,只够织布业补偿机器的平均损耗。但是,满怀积累欲的织布厂主拿3000镑去定购纱,拿1000镑去定购织机、煤炭(因为煤炭业者的情况也是这样)等等。或者说,他给纺纱厂主3000镑,给机器制造业者和煤炭业者等等1000镑,让他们替他把这些货币变成纱、织机和煤炭。因此,他必须等到这个过程结束后,才能开始自己的积累,开始自己新麻布的生产。这是第一个中断。

然而,得到3000镑的纺纱厂主现在的处境,也和拥有4000镑的织布厂主一样,区别只在于他会马上从得到的3000镑中扣下自己的利润。他可能会找到追加数量的纺纱工人,但是他需要亚麻、纱锭、煤炭等等。煤炭业者也一样,他除需要新工人以外,还需要新的机器或工具。而那个必须提供新的织机、纱锭等等的机器制造厂主,除需要追加的工人以外,还需要铁等等。亚麻生产者的情况最糟,他只有在下一年才能把追加量的亚麻提供出来,如此等等。

可见,织布厂主为了能够不拖延地、不间断地每年把他的一部分利润转化为不变资本,——并且为了使积累成为不断的过程,——就必须在市场上找到现成追加量的纱、织机等等。如果他以及纺纱厂主、煤炭业者等等在市场上能找到现成的亚麻、纱锭和机器,那他们就只需要雇用更多的工人了。

每年算作损耗并作为损耗加入产品价值的那部分不变资本,事实上并没有消耗掉。我们举一台机器为例,这台机器能用12年,价值12000镑;这台机器每年应当计算的平均损耗等于1000镑。既然每年有1000镑加入产品,那末到12年结束时就会再生产出12000镑的价值,并且能够用这个价格购买一台同一类型的新机器。这12年中必要的修理和日常维修,我们算入机器的生产费用,这些同我们的问题毫无关系。然而在事实上,实际的情况和这种平均的计算是不同的。机器在第二年可能比第一年好用。不过12年后它毕竟不能再使用了。这里的情况和家畜一样,一头家畜平均寿命为10年,但它并不因此每年死去十分之一,虽然10年后必须换一头新的。当然,在同一年中,总有一定数量的机器等等会达到确实必须换新机器的阶段。因此,每年都有一定数量的旧机器等等确实需要在实物形式上用新机器来替换。机器等等每年的平均生产就是与此相适应的。用来支付这些机器的那些价值按照它们(机器)再生产的时间从商品的卖款取得。但事实仍然是:虽然年产品价值(每年用来支付年产品的价值)有相当一部分比如说在12年后必须用来购买新机器以替换旧机器,但实际上决不需要每年都在实物形式上换掉旧机器的十二分之一,而且事实上也办不到。这个基金的一部分,在商品卖出或被支付以前可以用来发放工资或购买原料,因为商品不断地投入流通领域,但并不是立即从流通领域中回来。不过每一次使用这个基金都不可能延续一整年,因为一年周转一次的商品必须完全实现其价值,即必须支付,实现它所包含的工资、原料、机器损耗和剩余价值。

可见,凡是使用许多不变资本,因而也使用许多固定资本的地方,补偿固定资本损耗的这部分产品价值就是积累基金,这个基金可以被使用它的人用来作为新固定资本(或流动资本)的投资,而且这部分积累根本不是从剩余价值中扣除的。(见麦克库洛赫的著作。)\endnote{括号中的话“见麦克库洛赫的著作”是马克思后来(用铅笔)加的。马克思在1862年8月20日给恩格斯的信中第一次提出关于折旧基金用于积累的思想,他在1867年8月24日给恩格斯的信中提到前封信时告诉恩格斯,他后来在麦克库洛赫的著作中发现了有关这方面的一些暗示。马克思指的是麦克库洛赫的《政治经济学原理》1825年爱丁堡版第181—182页。马克思在《剩余价值理论》第三册中,在手稿第777页和第781页上又谈到了这个问题。——第548页。}这种积累基金在那些没有大量固定资本的生产阶段和国家是不存在的。这是重要的一点。这是一个不断用于改良、扩大等方面的基金。

\tsectionnonum{[(4)积累过程中各生产部门之间的联系。剩余价值的一部分直接转化为不变资本是农业和机器制造业中积累的特点]}

但是,我们这里所要研究的问题是这样的。即使投在机器制造业的全部资本仅够补偿机器每年的损耗,它所生产的机器也会比每年所需要的机器多得多,因为损耗有一部分只是在观念上存在,而在现实中只是过若干年之后才要以实物形式补偿。可见,这样使用的资本每年会提供大量的机器,这些机器可以用于新的投资,并且使这种新的投资提前实现。例如,一个机器制造厂主在本年内开始他的生产。假定他在这一年内提供12000镑的机器。这样,如果要把他所生产的机器简单再生产出来,在以后11年中,他每年只须生产1000镑的机器就行了,而且连这个年产量也不是每年都被消费掉。如果他使用的是他的全部资本,那末他的产品中被消费掉的部分就更小了。为了使他的资本保持运动,并且每年只实现[699]资本的简单再生产,那些需要这种机器的部门就必须继续不断地扩大生产。(如果这个机器制造厂主自己也进行积累,那就更是如此了。)

因此,即使在这个生产领域中投入的资本只是进行再生产,其他生产领域就必须不断进行积累。另一方面,只是由于机器制造业进行简单再生产,其他生产领域的不断积累才能不断在市场上现成地找到自己的要素之一。这里,即使一个生产领域本身进行的只是现有资本的简单再生产,在这个生产领域也经常有商品储备,供其他各生产领域用于积累,用于新的追加的生产消费。

至于被资本家比如说织布厂主转化为资本的那5000镑利润或剩余价值,可能有两种情况。我们始终假定,他在市场上找得到劳动,而他必须从这5000镑中拿出1000镑购买劳动,以便按照他这个生产领域的条件把这5000镑转化为资本。这部分[资本化的剩余价值]转化为可变资本,花费在工资上。但是为了使用这种劳动,工厂主就需要有纱、追加的机器和追加的辅助材料。{只有在工作日不延长的情况下,才需要追加的机器。在工作日延长的情况下,机器只是磨损得快些,机器再生产的时间会缩短,但同时会生产出更多的剩余价值;虽然机器的价值必须分摊到在较短时间内生产出来的商品上,然而生产出来的商品多得多;所以虽然磨损得快些,可是加入单位商品的价值或价格的那部分机器价值却小些。在这种情况下,不必把新资本直接花费在机器本身。只要补偿机器的价值稍快些就行了。但是在这种情况下,辅助材料需要预付追加的资本。}或者织布厂主能够在市场上现成地找到他的这些生产条件。这时,购买这些商品和购买其他商品不同的地方,只在于他购买商品是为了生产消费,而不是为了个人消费。或者他在市场上找不到这些现成的东西。这时,他就得定购这些东西(例如要买新结构的机器),就象他不得不定购市场上不能现成找到的那些个人消费品一样。如果原料(亚麻)只是根据定购进行生产{如靛蓝、黄麻等等,印度农民就是根据英国商人的定购和预付来生产的},那末织布厂主当年要在他自己的企业中进行积累就不可能了。另一方面,假定纺纱厂主把他的5000镑变成资本,而织布厂主不进行积累,那末,虽然市场上存在着生产纱的一切条件,但纱将卖不出去,这5000镑固然已转化为纱,但是没有转化为资本。

(关于信用,我们在这里不需要详谈。信用使积累资本可以不用在把它生产出来的那个领域,而用在它的价值增殖的机会最多的地方。但是,每个资本家都宁愿把他积累的资本尽量投在自己的部门。如果他把资本投在别的部门,他就成了货币资本家,得不到利润,只得到利息;或者他不得不去进行投机。但我们这里是谈平均积累,并且只是为了举例才假定积累资本投入这个或那个特殊部门。)

另一方面,如果亚麻种植业者扩大了生产,即进行了积累,而纺纱厂主、织布厂主、机器制造厂主等却没有进行积累,那末亚麻种植业者的仓库里就会有过剩的亚麻,下一年也许就会减少生产。

{这里我们暂时把个人消费完全撇开,只考察生产者之间的联系。如果存在这种联系,那末首先,对于生产者必须互相补偿的那些资本来说,他们会互相成为市场;新就业的或就业情况较好的工人会成为一部分生活资料的市场;因为剩余价值在下一年会增长,所以资本家能够消费自己收入中增长的部分,从而在一定程度上又会互相成为市场。可是本年的产品有相当一部分仍然不能实现。}

现在问题应该这样来表述:假定普遍进行积累,即假定在所有部门中都进行或多或少的资本积累,——而这实际上是资本主义生产的条件,资本家作为资本家来说强烈追求这一点,正象货币贮藏者强烈追求货币积累一样(不过这也是资本主义生产向前发展所必需的),——那末这种普遍积累的条件是什么,普遍积累究竟是什么意思呢?或者说,因为可以把织布厂主看作全体资本家的代表,那末为了使他能顺利地把5000镑剩余价值再转化为资本,并且逐年不断地把积累过程继续下去,需要些什么条件呢?积累5000镑,无非是把这些货币,把这个数额的价值,转化为资本。可见,资本积累的条件同原来生产或再生产资本的条件是完全一样的。

而这些条件就是:用一部分货币购买劳动,用另一部分货币购买能由这种劳动进行生产消费的商品(原料、机器等等)。{某些商品,例如机器、原料、半成品等等,只能供生产消费。其他一些商品,例如房屋、马匹、小麦、黑麦(可以用来造酒或制淀粉等)等等,既可供生产消费,也可供个人消费。}为了能够买到这些商品,它们就必须作为商品存在于[700]市场上,即存在于已经结束的生产和尚未开始的消费之间的中间阶段,存在于卖者手中,存在于流通阶段;或者根据定购可以得到供应(例如建造新工厂等等,就用定购的办法)。由于在资本主义生产条件下存在着社会规模的分工(劳动和资本在各个不同部门之间进行分配),由于生产、再生产在所有领域同时进行,情况也就是如此,这是进行资本生产和再生产的前提。这是市场的条件,是资本生产和再生产的条件。资本愈多,劳动生产率愈高,总之,资本主义生产的规模愈大,存在于从生产到消费(个人消费和生产消费)的过渡阶段,存在于流通中,存在于市场上的商品量就愈多,每一笔资本在市场上现成地找到自己再生产条件的把握也就愈大。情况之所以必然是这样,还因为按照资本主义生产的本质,第一,每一笔资本活动的规模,并不决定于个人需求(定购等等,私人需要),而是决定于力求实现尽可能多的劳动,因而实现尽可能多的剩余劳动,并用现有的资本提供尽可能多的商品的欲望;第二,每一笔资本都力求在市场上占据尽可能大的地盘,并竭力排挤、排除自己的竞争者。资本竞争。

{交通工具愈发达,市场上的存货就愈能减少。

\begin{quote}{“凡是生产和消费比较大的地方,在任何时候自然都会有比较多的剩余存在于中间阶段,存在于市场上,存在于从生产者到消费者的道路上,除非物品卖出的速度大大加快,消除了生产的增加本来会引起的这些后果。”(《论马尔萨斯先生近来提倡的关于需求的性质和消费的必要性的原理》1821年伦敦版第6—7页)}}\end{quote}

可见,新资本的积累只能在和已有资本再生产条件相同的条件下进行。

{我们在这里完全不谈这样一种情况:积累的资本大于能够投入生产的数量,例如资本以货币形式存放在银行家手里而不使用。由此会产生向国外贷款等等,一句话,产生投资的投机。我们也不考察有大量生产出来的商品不能卖出,出现危机等情况。这属于论述竞争的那一部分。这里我们要研究的只是资本在它运动的各个阶段上所采取的形式,而且总是假定商品会按其价值出卖。}

如果织布厂主除了用1000镑购买劳动外,还能在市场上找到现成的纱等等,或者能够定购到这些东西,他就能把5000镑剩余价值再转化为资本。为了能买到这些东西,就必须有追加的产品生产出来,这种追加的产品包括加入他的不变资本的商品,特别是包括需要较长时间才能生产出来而产量不能迅速增加或根本不能在当年增加的商品;原料,例如亚麻,就是这样的情况。

{这里商人资本就出现了,商人资本为日益增长的消费——个人消费或生产消费——把现成的商品贮存在仓库中;但这只是中介形式之一,因而不是这里所要谈的,而是考察资本竞争时所要谈的。}

正如一个领域中现有资本的生产和再生产以其他领域中并行的生产和再生产为前提,一个部门中的积累,或者说,追加资本的形成,也以其他部门中同时或并行地进行的追加生产为前提。因此,在所有提供不变资本的领域中,生产规模必须同时扩大(按照由需求决定的、每个特殊领域在整个生产增长中应承担的平均份额来扩大);所有不为个人消费提供成品的领域,都提供不变资本。其中最重要的是机器(工具)、原料、辅助材料的增加,因为当这些条件具备的时候,有这些东西加入的其他一切生产部门,不论是提供半成品还是提供成品,就只须推动更多的劳动了。

因此,为了能进行积累,看来所有领域都必须不断追加生产。

这一点还要稍为详细地加以说明。

其次,第二个重大问题:

再转化为资本的那部分剩余价值,或者说,——因为这里谈的是利润,——再转化为资本的那部分利润(包括地租;如果土地所有者想进行积累,想把地租转化为资本,那末剩余价值就总是落到产业资本家手中;甚至在工人把他自己的一部分收入转化为资本时,情况也是如此),只是由前一年的[701]新加劳动构成。现在要问,这笔新资本是否全部花费在工资上,是否只和新劳动交换?

赞成的说:一切价值最初都由劳动产生。一切不变资本最初都完全象可变资本一样是劳动的产品。看来,在这里我们又成了资本直接由劳动产生的见证人。

反对的说:难道追加资本的形成必须在比旧资本的再生产更坏的生产条件下进行吗?难道追加资本的形成必须回到生产方式的更低阶段吗?可是,如果新价值只花费在直接劳动上,因而这种直接劳动在没有固定资本等等条件下,必须先把这种资本生产出来,正象劳动最初不得不先把自己的不变资本创造出来那样,那末,情况就一定是这样。这纯粹是无稽之谈。但是,这是李嘉图等人的前提。这一点要较详细地谈谈。

这里产生的第一个问题是:

如果资本家不把一部分剩余价值卖掉,或者更确切地说,不把代表这部分剩余价值的剩余产品卖掉,而是把它直接当作资本使用,这一部分剩余价值能够由此转化成资本吗?如对这个问题作肯定的回答,那就包含着这样的结论:应当转化为资本的剩余价值并不是全都转化为可变资本,或者说,花费在工资上。

就谷物和牲畜构成的那部分农产品来说,这个问题从一开始就很清楚。收成中代表租地农场主的剩余产品或剩余价值的那部分谷物(一部分牲畜也是一样),可以不拿去卖,而立即作为种子或役畜再当作生产条件来使用。土地本身所生产的一部分肥料也是如此,这种肥料同时也可以作为商品在商业中流通,即可以出卖。租地农场主可以把作为剩余价值,作为利润得到的这部分[未进入流通的]剩余产品,立即在他自己的生产领域内再转化为生产条件,即直接转化为资本。这部分并不花费在工资上,并不转化为可变资本。这部分从个人消费中抽出来,而又不是在斯密和李嘉图所说的那种意义上生产地消费掉。这部分用于生产消费,然而是作为原料来消费的,不是作为生产劳动者或非生产劳动者的生活资料来消费的。谷物不仅可以用作生产工人等等的生活资料,而且可以用作牲畜的饲料,用作酿酒、制淀粉等等的原料。牲畜(肉用牲畜或役畜)也不仅可以用作生活资料,而且可以为许多工业部门提供原料,即毛皮、皮革、油脂、骨、角等等,同时还可以部分地为农业本身,部分地为运输业提供动力。

有一些生产部门,再生产时间超过一年(如大部分畜牧业,林业等等),但是这些部门的产品同时必须不断再生产,即不断要求投入一定量的劳动;在所有这些部门中,因为不仅代表有酬劳动,而且代表无酬劳动的新加劳动必须以实物形式积累到产品能够出卖时为止,所以积累和再生产是一致的。

(这里所谈的不是每年按照一般利润率归并[到资本中]的利润的积累;这不是实际的积累,而只是一种计算方法。这里所谈的是在若干年内反复进行的总劳动的积累,因而,在这种积累中,不仅有酬劳动,而且无酬劳动也以实物形式积累起来,并且立即再转化为资本。至于在这种情况下利润的积累,它反而不[直接]取决于这里新加的劳动量。)

经济作物(不论它们是提供原料,还是提供辅助材料)的情况也是这样。经济作物的种子,以及能够再作为肥料等等使用的那部分经济作物,都代表总产品的一部分。如果它们不拿去出卖,那丝毫也不会改变这样一个事实:它们一旦作为生产条件重新加入生产,就形成[新产品的]总价值的一部分,并以[702]这样的一部分构成新的生产中的不变资本。

这样就已经说明了一个主要问题——关于作为本来意义的农产品的原料和生活资料(食物)的问题。可见,这里积累是同更大规模的再生产直接一致的,因此,剩余产品的一部分直接在本生产领域内再用作生产资料,而不用经过同工资或其他商品相交换。

第二个主要问题是机器。这里指的并不是生产商品的机器,而是生产机器的机器,是机器制造业的不变资本。如果已经有了这种机器,那就只需要花费在采掘工业中为各种容器和机器提供原料(铁等等)的劳动了。有了机器制造机,也就有了对原料本身加工的机器。我们在这里遇到的困难,是不陷入前提的循环论证中。这种循环论证就是:为了生产更多的机器,就必须有更多的材料(铁等等,煤炭等等),而为了生产这种追加的材料,又必须有更多的机器。无论我们是否假定,生产机器制造机的工业家和生产机器(用机器制造机来生产)的工业家是属于同一个资本家集团,都不会使问题本身发生任何改变。有一点是明显的。剩余产品的一部分表现为机器制造机(剩余产品表现为这种形式,至少取决于机器制造厂主)。这种机器制造机不一定非卖不可,它们能够以实物形式作为不变资本再加入新的生产。因此,在这里我们又看到了第二类作为不变资本直接(或通过同一生产领域内的交换)加入新的生产(积累),而不用经过先转化为可变资本的过程的剩余产品。

剩余价值的一部分能否直接转化为不变资本的问题,首先归结为这样一个问题:代表剩余价值的剩余产品的一部分能否直接作为生产条件再加入本生产领域,而无须先让渡出去。

一般的规律是:

只要产品的一部分,从而剩余产品(即代表剩余价值的使用价值)的一部分,能够直接地,不通过中介,作为生产条件,作为劳动资料或劳动材料再加入它从中出来的那个生产领域,那末,这个生产领域中的积累就可能并且必定采取这种形式:剩余产品的一部分不拿去出卖,而是直接(或通过与同一生产领域内以同样方式进行积累的其他部门的资本家相交换)作为再生产的条件重新加入生产过程,所以在这里,积累和更大规模的再生产直接一致。它们两者必然到处都是一致的,但是不一定采取这种直接的方式。

一部分辅助材料的情况也是如此。例如一年内生产出来的煤炭就是这样,剩余产品的一部分能够被用来重新生产煤炭,因而,能够被煤炭业者直接地,不通过任何中介,作为不变资本用于更大规模的生产。

在工业界,有一种为工厂主建造整座工厂的机器制造业者。假定[他们的产品的]1/10是剩余产品,或者说,无酬劳动。显然,这1/10即剩余产品,究竟是表现为替第三者建造并卖给第三者的工厂建筑物,还是表现为机器制造业者替自己建造并卖给自己的工厂建筑物,都不会使问题发生任何改变。这里,问题只在于代表剩余劳动的那种使用价值的性质,只在于这种使用价值能不能作为生产条件再加入拥有这个剩余产品的[703]资本家的生产领域。这里我们又有了一个例子,表明使用价值这个范畴对于决定经济形式具有重要的意义。

这样,这里我们已经看到,剩余产品(因而也就是剩余价值)有很大一部分能够并且必须直接转化为不变资本,以便作为资本被积累起来,没有这部分剩余产品,就根本不可能有任何资本积累。

第二,我们已经看到,在资本主义生产高度发展,因而劳动生产率水平很高,因而不变资本很大,特别是由固定资本构成的那部分不变资本很大的地方,一切领域的固定资本的简单再生产,以及与此并行的生产固定资本的现有资本的再生产,就会形成一个积累基金,也就是为更大规模的生产提供机器,提供不变资本。

第三,剩下还有一个问题:

剩余产品的一部分能不能通过比如说机器、工具等等的生产者和原料即铁、煤炭、金属、木材等等的生产者之间的(间接)交换,即通过不变资本不同组成部分之间的交换,再转化为资本(不变资本)?例如,如果生产铁、煤炭、木材等等的工厂主向机器制造业者购买机器或工具,而机器制造业者向这些原料生产者购买金属、木材、煤炭等等,那末他们就是通过这种交换来互相补偿各自不变资本中相互有关的组成部分,或形成新的不变资本。这里的问题是,这种情况在多大程度上适用于剩余产品。

\tsectionnonum{[(5)资本化的剩余价值转化为不变资本和可变资本]}

我们在前面\fnote{见本卷第1册第126—140、182—195和248—258页。——编者注}已经看到,在现有资本进行简单再生产的条件下,在不变资本的再生产中所使用的那部分不变资本,或者直接以实物形式补偿,或者通过不变资本生产者之间的交换来补偿;这是资本同资本的交换,不是收入同收入的交换,或收入同资本的交换。其次,在消费品(加入个人消费的物品)的生产中所使用的,或者说,用于生产消费的不变资本,则由作为新加劳动的结果因而归结为收入(工资和利润)的同类新产品来补偿。与此相适应,在生产消费品的那些领域中,其价值用于补偿这些领域中所使用的不变资本的那部分产品,代表不变资本生产者的收入,相反,在生产不变资本的那些领域中,代表新加劳动,因而形成这种不变资本生产者的收入的那部分产品,则代表生活资料生产者的不变资本(用于补偿[在生活资料生产中消费掉的不变]资本)。因此,这就要求不变资本的生产者用他们的剩余产品(在这里,是他们产品中超过同他们的不变资本相等的那部分产品的余额)去交换生活资料,把这种剩余产品的价值用于个人消费。同时,这种剩余产品包括:

(1)工资(或再生产出来的工资基金),这一部分必须(由资本家)仍旧用在工资上,即用于个人消费(如果假定是最低限度的工资,那末工人就只能把他得到的工资实现在生活资料上);

(2)资本家的利润(包括地租)。这一部分如果很大,那就可以一部分用于个人消费,一部分用于生产消费。而在用于生产消费的情况下,不变资本的生产者之间就会进行产品交换;但这里已经不是代表他们应当互相补偿的不变资本的那部分产品的交换,而是一部分剩余产品,即收入(新加劳动)的交换,这部分剩余产品直接转化为不变资本,由于这种转化,不变资本的量就增加,再生产的规模就扩大。

因此,就是在这样的情况下,也有一部分现有的剩余产品,即一部分一年内新加的劳动,直接转化为不变资本,而不用先转化为可变资本。因此,这里也可以看出,剩余产品用于生产消费,或者说,积累,绝不等于全部剩余产品都花费在生产工人的工资上。

可以设想这样一种情况:机器制造业者把自己的商品(一部分)卖给比如说布匹生产者;布匹生产者付给他货币;机器制造业者用这些货币购买铁、煤炭等等,而不是购买生活资料。但是把整个过程加以考察就会明白,如果用于补偿不变资本的各要素的生产者不向生活资料生产者购买他生产出来的生活资料,因而,如果这个流通过程实质上不是生活资料和不变资本之间的交换,那末生活资料的生产者就不能购买机器或原料来补偿自己的不变资本。由于买和卖的行为的分离,当然在这些结算过程中可能发生极大的紊乱和麻烦。

[704]如果一个国家自己不能把资本积累所需要的那个数量的机器生产出来,它就要从国外购买。如果它自己不能把所需数量的生活资料(用于工资)和原料生产出来,情况也会如此。在这里,一旦有国际贸易参与,那就可以看得一清二楚,一个国家的剩余产品——如果它用于积累,——有一部分并不转化为工资,而是直接转化为不变资本。但是那时仍然会有一种看法,认为在国外为此而预付的货币会全部花费在工资上。我们已经看到,即使把对外贸易撇开,情况也不是这样,而且不可能是这样。

剩余产品究竟以怎样的比例分为可变资本和不变资本,这取决于资本的平均构成,而且资本主义生产愈发达,直接花费在工资上的那部分相对地也就愈小。有人认为,剩余产品既然只是一年内新加劳动的产品,它[在积累的情况下]也就只转化为可变资本,只花费在工资上,这种看法总的说来同那种认为因为产品只不过是劳动的结果或劳动的化身,所以产品的价值全部都归结为收入,即工资、利润和地租的错误观念是一致的,而斯密和李嘉图就是这样错误地认为的。

不变资本的很大一部分,即固定资本,可能由这样一些东西组成:有的直接加入生活资料、原料等等的生产过程;有的或者用来缩短流通过程,如铁路、公路、通航的运河、电报等等,或者用来保存和储备商品,如货栈、仓库等等;还有的只是在经过较长的再生产时期后才能增加土地的肥力,如土地平整、泄水渠等等。剩余产品究竟是较大一部分还是较小一部分花费在这几种固定资本中的某一种上,这对于生活资料等等的再生产所产生的直接的、最近的后果是极不相同的。

\tsectionnonum{[(6)危机问题(引言)。发生危机时资本的破坏]}

如果有不变资本的追加生产——比补偿旧资本所必需的、因而也就是生产原有数量的生活资料所必需的生产大的生产——作为前提,那末,在使用机器、原料等[来生产个人消费品]的领域进行追加生产,或者说,进行积累,就再也没有任何困难了。如果有必要的追加劳动,上述领域的资本家就会在市场上找到形成新资本,即把他们的代表剩余价值的货币转化为新资本的一切资料。

但是,整个积累过程首先归结为这样的追加生产,它一方面适应人口的自然增长,另一方面形成在危机中显露出来的那些现象的内在基础。这种追加生产的尺度,是资本本身,是生产条件的现有规模和资本家追求发财致富和扩大自己资本的无限欲望,而决不是消费。消费早就被破坏了,因为,一方面,人口的最大部分,即工人人口,只能在非常狭窄的范围内扩大自己的消费,另一方面,随着资本主义的发展,对劳动的需求,虽然绝对地说是在增加,但相对地说却在减少。此外还有一点:一切平衡都是偶然的,各个领域中使用资本的比例固然通过一个经常的过程达到平衡,但是这个过程的经常性本身,正是以它必须经常地、往往是强制地进行平衡的那种经常的比例失调为前提。

我们这里要考察的,只是资本在它向前发展的不同阶段所经历的形式。因此,不准备对实际生产过程在其中进行的现实关系加以分析。这里总是假定商品按其价值出卖。不考察资本的竞争,不考察信用,同样不考察实际的社会结构,——社会决不仅仅是由工人阶级和产业资本家阶级组成的,因此,在社会中消费者和生产者不是等同的:第一个范畴即消费者范畴(消费者的收入有一部分不是第一性的,而是第二性的,是从利润和工资派生的)比第二个范畴[即生产者范畴]广得多,因而,消费者花费自己收入的方式以及收入的多少,会使经济生活过程,特别是资本的流通和再生产过程发生极大的变化。可是,我们在考察货币时\endnote{指《政治经济学批判》第一分册。见《马克思恩格斯全集》中文版第13卷第87—88、131—132和135—137页。——第562页。}已经看到,就货币一般是一种与商品的实物形式不同的形式来说,就它作为支付手段的形式来说,货币本身就包含着危机的可能性,而这一点,在考察资本的一般性质时,用不着对成为实际生产过程的一切前提的进一步的现实关系加以说明,就更加清楚地表现出来了。

[705]大卫·李嘉图接受了庸俗的萨伊的(其实是属于詹姆斯·穆勒的)观点(我们谈这个微不足道的人物时,还要讲到这种观点),认为生产过剩,至少市场商品普遍充斥是不可能的。这种观点是以产品同产品交换\endnote{让·巴·萨伊《论政治经济学》1814年巴黎第二版第二卷第382页:“产品只是用产品购买的”。萨伊的这个公式几乎被李嘉图一字不改地重述了,见下面(第570页)从李嘉图的《原理》(1821年伦敦第3版第341页)中引用的一段话。马克思在《剩余价值理论》的后面正文中对这个公式进行了批判(见本册第570—572页和第3册《李嘉图学派的解体》一章,马克思手稿第811页)。——第563页。}这一论点为基础的,或者,正如穆勒所想象的那样,是以“卖者和买者之间的形而上学的平衡”\endnote{马克思指詹姆斯·穆勒关于生产和消费之间、供给和需求之间、购买量和销售量之间的经常和必要的平衡的论述,见穆勒的《政治经济学原理》1821年伦敦版第3篇第4章第186—195页。马克思在《政治经济学批判》第一分册关于商品的形态变化一节中,对詹姆斯·穆勒的这个观点(他最早是在1808年伦敦出版的《为商业辩护》这本小册子里提出的)作了更详细的分析(见《马克思恩格斯全集》中文版第13卷第87—88页)。——第563、575页。}为基础的,由此还进一步得出结论说,需求仅仅决定于生产本身,或者说,需求和供给完全一致。这种论点也采取李嘉图所特别喜爱的形式,即认为任何数额的资本在任何国家都能够生产地加以使用。

\begin{quote}{李嘉图在第二十一章(《积累对于利润和利息的影响》)中说:“萨伊先生曾经非常令人满意地说明:由于需求只受生产限制,所以任何数额的资本在一个国家都不会得不到使用。任何人从事生产都是为了消费或出卖,任何人出卖都是为了购买对他直接有用或者有助于未来生产的某种别的商品。所以,一个人从事生产时,他不成为自己产品的消费者,就必然成为他人产品的买者和消费者。不能设想,他会长期不了解为了达到自己所追求的目的,即获得别的产品,究竟生产什么商品对他最有利。因此,他不可能继续不断地〈这里根本不是说永远〉生产没有需求的商品。”(第339—340页)}\end{quote}

到处都力求做到前后一贯的李嘉图,发现他的权威萨伊在这里跟他开了个玩笑。他在上述引文的注释中说:

\begin{quote}{“萨伊先生说:‘和资本的使用范围相比,闲置资本越多,资本借贷的利率就越下降。’(萨伊[《论政治经济学》1814年巴黎第2版]第2卷第108页)萨伊先生的这句话同他的上述论点完全一致吗?如果任何数额的资本在一个国家都能够得到使用,怎么能说,和资本的使用范围相比资本会过多呢?”(第340页注)}\end{quote}

因为李嘉图引证萨伊的话,我们在后面考察萨伊这个骗子本人的观点时将批判萨伊的论点。

这里暂时只指出:

在进行资本的再生产时,同在进行资本积累时完全一样,问题不仅在于以原有的规模或者(在积累的时候)以扩大的规模补偿构成资本的同量使用价值,而且在于要补偿预付资本的价值,并且实现普通利润率(普通剩余价值)。因此,如果由于某种情况或某些情况的结合,商品(是全部还是大部都毫无关系)的市场价格大大降到它的费用价格之下,那末,一方面,资本的再生产就会尽量缩小。但是积累将更加停滞。以货币形式(金或银行券)积累起来的剩余价值如果转化为资本就只会带来损失。因此,这些剩余价值就以贮藏货币或者信用货币的形式闲置在银行里,而这丝毫不会改变问题的本质。如果缺少再生产的现实前提(例如在谷物涨价的时候,或者由于实物形式的不变资本还没有积累到足够的数量),相反的原因也可能引起同样的停滞。再生产会发生停滞,因此流通过程也会发生停滞。买和卖互相对立起来,不使用的资本就以闲置货币的形式出现。这种现象(这大都出现在危机之前)也可能发生在这样的时候:追加资本的生产进行得非常快,由于追加资本再转化为生产资本,就大大增加了对生产资本的一切要素的需求,以致实际生产赶不上,因而加入资本形成过程的一切商品涨价。在这种情况下,不管利润怎样增长,利率都要大大降低,而这种利率的降低就引起最冒险的投机活动。由于再生产停滞,可变资本就减少,工资就下降,使用的劳动量就减少。这些又反过来重新影响价格,使价格继续下跌。

任何时候都不应该忘记,在实行资本主义生产的条件下,问题并不直接在于使用价值,而在于交换价值,特别在于增加剩余价值。这是资本主义生产的动机。为了通过论证来否定资本主义生产的矛盾,就撇开资本主义生产的基础,把这种生产说成是以满足生产者的直接消费为目的的生产,这倒是一种绝妙的见解。

其次:因为资本流通过程不是一天就完了,相反,在资本流回[生产领域]以前,要经历一个相当长的时期,因为这个时期同市场价格[706]平均化为费用价格的时期是一致的,因为在这个时期内市场上发生重大的变革和变化,劳动生产率发生重大的变动,因而商品的实际价值也发生重大的变动,所以,很明显,从起点——最初的资本——到它经过一个这样的时期以后回来,必然会发生一些大灾难,危机的要素必然会积累和发展,这些决不是用产品同产品交换这一句毫无价值的空话就可以排除得了的。相反,一批商品在一个时期的价值和同一批商品在较后一个时期的价值的比较(贝利先生认为这是经院式的虚构\endnote{[赛·贝利]《对价值的本质、尺度和原因的批判研究》1825年伦敦版第71—93页。——第565页。})倒是资本流通过程的基本原则。

说到危机引起的资本的破坏,要区别两种情况。

只要再生产过程停滞,劳动过程缩短或者有些地方完全停顿,实际资本就会被消灭。不使用的机器不是资本。不被剥削的劳动等于失去了的生产。闲置不用的原料不是资本。建好不用的建筑物(以及新制造的机器)或半途停建的建筑物,堆在仓库中正在变质的商品,这一切都是资本的破坏。这一切无非是表示再生产过程的停滞,表示现有的生产条件实际上没有起生产条件的作用,没有发挥生产条件的效能。这时,它们的使用价值和它们的交换价值都化为乌有。

第二,危机所引起的资本的破坏意味着价值量的贬低,这种贬低妨碍价值量以后按同一规模作为资本更新自己的再生产过程。这就是商品价格的毁灭性的下降。这时,使用价值没有被破坏。一个人亏损了的东西,被另一个人赚了去。作为资本发挥作用的价值量不能在同一个人手里作为资本更新。原来的资本家遭到破产。如果某个资本家靠出卖自己的商品把他的资本再生产出来,而他的商品的价值本来等于12000镑,其中比如说2000镑是利润,如果这些商品的价格现在降到6000镑,那末,这个资本家就不能支付他的契约债务,即使他没有债务,用这6000镑也不能按以前的规模重新开始他的营业,因为商品价格又会回升到它的费用价格的水平。这样一来,6000镑资本被消灭了,虽然购买这些商品的人因为照商品的费用价格的半价购买,在营业再活跃时可以大有所为甚至因而发财。社会的名义资本,也就是现存资本的交换价值,有很大一部分永远消灭了,虽然由于不殃及使用价值,这种消灭正好可以大大促进新的再生产。这同时也是货币所有者靠牺牲产业家而发财致富的时期。至于纯粹的虚拟资本(公债券、股票等)的跌价,只要它不导致国家和股份公司的破产,不因此而动摇持有这类证券的产业资本家的信用,从而不阻碍再生产,那末这种跌价就只是财富从一些人的手里转到另一些人的手里,总的来说对再生产起着有利的影响,因为那些用廉价把这些股票或证券弄到手的暴发户大多数比原来的所有者更有事业心。

\tsectionnonum{[(7)在承认资本过剩的同时荒谬地否认商品的生产过剩]}

李嘉图在他自己提出的那些论点上总是前后一贯的。因此,在他的著作中,关于(商品的)生产过剩不可能的论点,同关于资本过多或资本过剩不可能的论点,是一回事。\fnote{这里必须区分:当斯密用资本过剩、资本积累来说明利润率下降时,说的是永久的影响问题,而这是错误的;相反,暂时的资本过剩、生产过剩、危机则是另一回事。永久的危机是没有的。}

\begin{quote}{“因此,在一个国家中,除非必需品的涨价使工资大大提高,因而剩下的资本利润极少,以致积累的动机消失,否则积累的资本不论多少,都不可能不生产地加以使用。”(第340页)“从上述各点可以看出,需求是无限的,只要资本还能带来某种利润,资本的使用也是无限的,无论资本怎样多,除了工资提高以外,没有其他充分原因足以使利润降低。此外还可以补充一句:使工资提高的唯一充分而经常的原因,就是[707]为越来越多的工人提供食品和必需品的困难越来越大。”(第347—348页)}\end{quote}

在这种情况下,李嘉图对于他的门徒们的愚蠢该怎么说呢?他们对于一种形式的生产过剩(市场商品普遍充斥)加以否认,同时,对于另一种形式的生产过剩,即资本的生产过剩、资本过多、资本过剩,却不仅加以承认,而且把它作为自己学说的一个基本点。

李嘉图以后时期的理智健全的经济学家中,没有一个否认资本过多。相反,他们都用资本过多来说明危机(只要不是用信用现象来说明)。因此,他们都承认一种形式的生产过剩,但是否认另一种形式的生产过剩。所以,剩下的只是这样一个问题:生产过剩的两种形式彼此之间的关系,即被否认的形式和被确认的形式的关系是怎样的?

李嘉图自己对于危机,对于普遍的、由生产过程本身产生的世界市场危机,确实一无所知。对于1800—1815年的危机,他可以用歉收引起谷物涨价,用纸币贬值、殖民地商品跌价等等来解释,因为,大陆封锁使市场由于政治原因而不是经济原因被迫缩小。对于1815年以后的危机,他也可以解释为部分由于荒年造成谷物缺乏,部分由于谷物价格下降,——因为,根据李嘉图自己的理论,在战争以及英国同大陆切断联系的时候必然引起谷物价格上涨的那些原因不再起作用,——部分由于从战争到和平的转变以及由此产生的“商业途径的突然变化”(见他的《原理》第十九章《论商业途径的突然变化》)。

后来的历史现象,特别是世界市场危机几乎有规律的周期性,不容许李嘉图的门徒们再否认事实或者把事实解释成偶然现象。他们——更不必说那些拿信用来说明一切,以便后来宣称他们自己也将不得不以资本过剩为前提的人了,——不这样做了,却臆造出了一个资本过多和生产过剩之间的美妙的差别。他们搬出他们手中的李嘉图与斯密的词句和论据来反对生产过剩,同时他们企图用资本过多解释他们否则就无法解释的现象。例如,威尔逊用固定资本过多来解释某几次危机,用流动资本过多来解释另外几次危机。资本过多本身,为优秀的经济学家们(例如富拉顿)所承认,而且已经成为大家所接受的偏见,以致这个说法在博学的罗雪尔先生的概论\endnote{威·罗雪尔《国民经济体系》,第一卷《国民经济学原理》1858年斯图加特和奥格斯堡第3版第368—370页。——第568页。}中竟作为一种不言而喻的东西出现了。

因此就要问:资本过多是什么?它同生产过剩有什么区别?

(不过,为了公正起见,需要指出,其他经济学家,如尤尔、柯贝特等,则认为生产过剩——只要考察的是国内市场——是大工业的正常状态。因此可以得出结论说,这种生产过剩只有在某种条件下,当国外市场也缩小时,才会引起危机。)

按照这些经济学家的看法,资本等于货币或者商品。因而,资本的生产过剩就等于货币或商品的生产过剩。可是,据说这两种现象彼此毫无共同之点。这些经济学家甚至不可能谈到货币的生产过剩,因为货币在他们看来就是商品,所以整个现象都归结为他们在一个名称下加以承认而在另一个名称下则加以否认的商品的生产过剩。如果进一步说到存在固定资本或流动资本的生产过剩,那末,这种说法的基础就是:商品已经不是在这个简单的规定上被考察,而是在它作为资本的规定上被考察。但是另一方面,这种说法也承认,在资本主义[708]生产及其种种现象中——例如在生产过剩中,问题不仅在于使产品作为商品出现并具有商品的规定的那种简单关系;而且在于产品的这样一些社会规定,由于这些规定,产品不止是商品,并且不同于简单的商品。

总之,可以认为,用“资本过多”的说法代替“商品生产过剩”的说法不仅仅是一种遁辞,或者说,不仅仅是一种昧着良心的轻率——同一现象,称作a,就认为是存在的和必要的,称作b,就加以否认,因而实际上怀疑和考虑的只是现象的名称,而不是现象本身;或者是想用这种办法来回避说明现象的困难:在现象采取某种形式(名称)而同这些经济学家的偏见发生矛盾时就加以否认,只有在现象采取另一种形式而变得毫无意义时才加以承认。如果撇开这一切不谈,那末,从“商品生产过剩”的说法转到“资本过多”的说法,实际上是个进步。进步表现在哪里?在于承认商品生产者不是作为单纯的商品所有者,而是作为资本家彼此相互对立。

\tsectionnonum{[(8)李嘉图否认普遍的生产过剩。在商品和货币的内在矛盾中包含着危机的可能性]}

再引李嘉图的几个论点:

\begin{quote}{“……会使人认为,亚当·斯密断定:我们似乎在一定程度上不得不〈其实情况就是如此〉生产出过剩的谷物、呢绒和金属制品,似乎用来生产这些商品的资本不能移作别用。但是,一笔资本的使用方式总是可以随便选择的,因此,任何商品都决不可能长期有剩余;因为如果有剩余,商品价格将跌到它的自然价格之下,资本就会转移到某些更有利的行业中去。”(第341—342页注)“产品总是用产品或服务购买的;货币只是进行交换的媒介。”}\end{quote}

(这就是说,货币只是流通手段,而交换价值本身只是产品同产品交换的转瞬即逝的形式,——这是错误的。)

\begin{quote}{“某一种商品可能生产过多,可能在市场上过剩,以致不能补偿它所花费的资本;但是不可能所有的商品都是这种情况。”(第341—342页)“生产的这种增长和由此引起的需求的增加是否会使利润降低,这完全取决于工资是否增加;而工资是否增加,除了短期的增加以外,又取决于为工人生产食品和必需品的容易程度。”(第343页)“商人把他们的资本投入对外贸易或海运业时,他们总是出于自由选择而不是迫不得已;他们这样做是因为在这些部门中他们的利润比在国内贸易中要大一些。”(第344页)}\end{quote}

至于危机,所有描写价格的实际运动的著作家或所有在危机的一定时候进行写作的实践家,都有理由藐视那些貌似理论的空谈,有理由满足于说:认为市场商品充斥等等不可能的学说,在抽象理论上是正确的,但在实践上是错误的。危机有规律的反复出现把萨伊等人的胡说实际上变成了一种只在繁荣时期才使用,一到危机时期就被抛弃的空话。

[709]在世界市场危机中,资产阶级生产的矛盾和对抗暴露得很明显。但是,辩护论者不去研究作为灾难爆发出来的对抗因素何在,却满足于否认灾难本身,他们不顾灾难有规律的周期性,顽固地坚持说,如果生产按照教科书上说的那样发展,事情就决不会达到危机的地步。所以,辩护论就在于伪造最简单的经济关系,特别是在于不顾对立而硬说是统一。

如果比如说买和卖,或者说,商品的形态变化运动,代表着两个过程的统一,或者确切些说,代表着一个经历两个对立阶段的过程,因而,如果这个运动本质上是两个阶段的统一,那末,这个运动同样本质上也是两个阶段的分离和彼此独立。但因为它们毕竟有内在联系,所以,有内在联系的因素的独立只能强制地作为具有破坏性的过程表现出来。正是在危机中,它们的统一、不同因素的统一才显示出来。相互联系和相互补充的因素所具有的彼此的独立性被强制地消灭了。因此,危机表现出各个彼此独立的因素的统一。没有表面上彼此无关的各个因素的这种内在统一,也就没有危机。但是,辩护论经济学家说:不对。因为有统一,所以就不会有危机。而这种说法又无非是说,各个对立因素的统一排除它们的对立。

为了证明资本主义生产不可能导致普遍的危机,就否定资本主义生产的一切条件和它的社会形式的一切规定,否定它的一切原则和特殊差别,总之,否定资本主义生产本身;实际上是证明:如果资本主义生产方式不是社会生产的一个特殊发展的独特形式,而是资本主义最初萌芽产生以前就出现的一种生产方式,那末,资本主义生产方式所固有的对抗、矛盾,因而对抗、矛盾在危机中的爆发,也就不存在了。

\begin{quote}{李嘉图跟着萨伊说:“产品总是用产品或服务购买的;货币只是进行交换的媒介。”}\end{quote}

因此,这里,第一,包含着交换价值和使用价值的对立的商品变成了单纯的产品(使用价值),因而商品交换变成了单纯的产品的物物交换,仅仅是使用价值的物物交换。这就不仅是退回到资本主义生产以前,而且甚至退回到简单商品生产以前去了;并且通过否定资本主义生产的第一个条件,即产品必须是商品,因而必须表现为货币并完成形态变化过程,来否定资本主义生产最复杂的现象——世界市场危机。不说雇佣劳动,却说“服务”,在“服务”这个词里,雇佣劳动及其使用的特殊规定性——就是增大它所交换的商品的价值,创造剩余价值,——又被抛弃了,因而货币和商品借以转化为资本的那种特殊关系也被抛弃了。“服务”是一种仅仅作为使用价值来理解的劳动(这在资本主义生产中是次要的事情),完全象“产品”这个词掩盖了商品的本质和商品中包含的矛盾一样。于是货币也就前后一贯地被看作仅仅是产品交换的媒介,而不是被看作必然表现为交换价值,即表现为一般社会劳动的商品的本质的、必然的存在形式。因为这样把商品变为单纯的使用价值(产品)会抹杀交换价值的本质,[710]所以,货币作为商品的本质的、在形态变化过程中独立于商品最初形式的形态,也就同样轻而易举地可能被否定,确切地说,必然被否定。

因此,这里论证不可能有危机的办法就是,忘记或者否定资本主义生产的最初前提——产品作为商品的存在,商品分为商品和货币这种二重化,由此产生的在商品交换中的分离因素,最后,货币或商品对雇佣劳动的关系。

此外,有些经济学家(例如约·斯·穆勒)想用这种简单的、商品形态变化中所包含的危机可能性——如买和卖的分离——来说明危机,他们的情况并不更妙些。说明危机可能性的这些规定,还远不能说明危机的现实性,还远不能说明为什么[再生产]过程的不同阶段竟会发生这样的冲突,以致只有通过危机、通过强制的过程,它们内在的统一才能发生作用。这种买和卖的分离在危机中也表现出来;这是危机的元素形式。用危机的这个元素形式说明危机,就是通过以危机的最抽象的形式叙述危机存在的办法来说明危机的存在,也就是用危机来说明危机。

\begin{quote}{李嘉图说\fnote{见本册第563面。——编者注}:“任何人从事生产都是为了消费或出卖,任何人出卖都是为了购买对他直接有用或者有助于未来生产的某种别的商品。所以,一个人从事生产时,他不成为自己产品〈goods〉的消费者,就必然成为他人产品的买者和消费者。不能设想,他会长期不了解为了达到自己所追求的目的,即获得别的产品,究竟生产什么商品对他最有利。因此,他不可能继续不断地〈continuYally〉生产没有需求的商品。”}\end{quote}

这种幼稚的胡说,出自萨伊之流之口是相称的,出自李嘉图之口是不相称的。首先,没有一个资本家是为了消费自己的产品而进行生产的。当我们说到资本主义生产时,即使有人把他的产品的某些部分再用于生产消费,我们也有充分理由说:“任何人从事生产都不是为了消费自己的产品。”但是,李嘉图说的是私人消费。以前,李嘉图忘记了产品就是商品。现在,他连社会分工也忘记了。在人们为自己而生产的社会条件下,确实没有危机,但是也没有资本主义生产。我们从来也没有听说过,古代人在他们以奴隶制为基础的生产中见过什么危机,虽然在古代人中也有个别生产者遭到破产。在二者择一的说法[“为了消费或出卖”]中,前一部分是荒谬的。后一部分也是荒谬的。一个人已经进行了生产,是出卖还是不出卖,[在资本主义生产条件下]是没有选择余地的。他是非出卖不可。在危机中出现的正是这样的情况,他卖不出去或者只能低于费用价格出卖,甚至不得不干脆亏本出卖。因此,说他把产品生产出来是为了出卖,这对他、对我们究竟有什么用处呢?问题正是在于:要弄清楚究竟是什么东西阻碍他这个善良愿望的实现。

其次:

\begin{quote}{“任何人出卖都是为了购买对他直接有用或者有助于未来生产的某种别的商品。”}\end{quote}

把资产阶级关系描绘得多么美好呵!李嘉图甚至忘记了,有人可能是为了支付而出卖,忘记了这种被迫的出卖在危机中起着很重要的作用。资本家在出卖时的直接目的是把他的商品,确切些说,是把他的商品资本,再转化成为货币资本,从而实现他的利润。消费——收入——决不是这个过程的主导因素,对于仅仅为了把商品变成生活资料而出卖商品的人来说,消费确实是主导因素。但这不是资本主义生产,在资本主义生产中,收入[消费]是作为结果,而不是作为起决定作用的目的出现的。每一个人出卖,首先是为了出卖,就是说,为了把商品变成货币。

[711]在发生危机的时候,一个人只要把商品卖出去,他就会感到很满意了,至于买进,他暂时不会去考虑。当然,要使实现了的价值能再作为资本发生作用,这个价值就必须通过再生产过程,也就是必须再同劳动和商品进行交换。但是,危机恰恰就是再生产过程破坏和中断的时刻。而这种破坏是不能用在不发生危机的时候它并不存在这个事实来解释的。毫无疑问,谁也不会“继续不断地生产没有需求的商品”(第339—340页),但是谁都没有作过这种荒谬的假设。并且,这样的假设同问题也毫无关系。资本主义生产的目的首先不是“获得别的产品”,而是占有价值、货币、抽象财富。

这里成为李嘉图的论断的基础的,还是我在前面考察过的\endnote{马克思指詹姆斯·穆勒关于生产和消费之间、供给和需求之间、购买量和销售量之间的经常和必要的平衡的论述,见穆勒的《政治经济学原理》1821年伦敦版第3篇第4章第186—195页。马克思在《政治经济学批判》第一分册关于商品的形态变化一节中,对詹姆斯·穆勒的这个观点(他最早是在1808年伦敦出版的《为商业辩护》这本小册子里提出的)作了更详细的分析(见《马克思恩格斯全集》中文版第13卷第87—88页)。——第563、575页。}詹姆斯·穆勒关于“买和卖之间的形而上学的平衡”的论点,这个论点在买和卖的过程中只看见统一而看不见分离。李嘉图下面这个主张(追随詹姆斯·穆勒)也是从这里来的:

\begin{quote}{“某一种商品可能生产过多,可能在市场上过剩,以致不能补偿它所花费的资本;但是不可能所有的商品都是这种情况。”(第341—342页)}\end{quote}

货币不仅是“进行交换的媒介”(第341页),同时也是使产品同产品的交换分解为两个彼此独立的、在时间和空间上彼此分离的行为的媒介。但是,前面所说的李嘉图对货币的错误理解的根本原因在于,李嘉图总是只看到交换价值的量的规定,就是说,交换价值等于一定量的劳动时间,相反,他忘记了交换价值的质的规定,就是说,个人劳动只有通过自身的异化(alienation)才表现为抽象一般的、社会的劳动。\fnote{[718}(李嘉图把货币仅仅看成流通手段,同他把交换价值仅仅看成转瞬即逝的形式,看成对资产阶级生产,或者说,资本主义生产来说仅仅是形式上的东西,是一回事;因此,在李嘉图看来,资本主义生产不是特定的生产方式,而是唯一的生产方式。)[718]]

不是所有种类的商品,而只是个别种类的商品,才能“在市场上过剩”,因此生产过剩始终只能是局部的,这种论点是一种可怜的遁辞。首先,如果谈的只是商品的性质,那末没有什么东西妨碍所有商品在市场上都过剩,因而妨碍它们都降到自己的价格之下\endnote{对于这一点,马克思在《剩余价值理论》第一册中解释如下:“低于它的价格——就是说,低于代表它[商品]的价值的货币额”(见本卷第1册第336页)。——第575页。}。这里说的恰恰只是危机的因素。就是说,除了货币以外的所有商品。说这种商品必然表现为货币,这只是说:所有商品都有这种必然性。完成这个形态变化,个别商品有多少困难,所有商品同样有多少困难。商品形态变化(它既包括买和卖的分离,又包括两者的统一)的一般性质,不仅不排除市场商品普遍充斥的可能性,相反,它本身就是这种普遍充斥的可能性。

其次,李嘉图的和其他类似的论断,当然不仅是从买和卖的关系出发,而且是从需求和供给的关系出发,这等我们考察资本的竞争时再谈。照穆勒的说法,买就是卖,如此等等,那末,这样一来,需求就是供给,供给就是需求。但是,供给与需求同样是彼此分离并且可以彼此独立的。在一定的时刻,由于对一般商品即货币亦即交换价值的需求大于对所有特殊商品的需求,换句话说,由于商品表现为货币、实现商品交换价值的因素居优势,商品再转化为使用价值的因素居劣势,所有商品的供给就可能大于对所有商品的需求。

如果从更广泛和更具体的意义上来理解需求和供给之间的关系,就要把生产和消费的关系包括在内。这里仍然必须看到,这两个因素的潜在的、恰好在危机中强制地显示出来的统一,是与同样存在的、甚至表现为资产阶级生产特征的这两个因素的分离和对立相对的。

至于局部的生产过剩和普遍的生产过剩的对立,既然问题在于承认第一种生产过剩是为了逃避承认第二种生产过剩,那末,对于这个问题必须指出:

第一,在危机之前,所有属于资本主义生产的物品往往普遍涨价。因此,所有这些商品都卷进接着而来的崩溃之中;在按照它们在崩溃之前的价格出卖的情况下,它们就造成市场负担过重。这种按照以前的市场价格市场吸收不了的商品量,按照下降了的、已经降到商品费用价格之下的价格,市场却能够吸收。商品的过剩总是相对的,就是说,都是在一定价格条件下的商品过剩。在这种情况下使商品能被吸收的那种价格,对生产者或商人来说,是引起破产的价格。

[712]第二,危机(因而,生产过剩也是一样)只要包括了主要交易品,就会成为普遍性的。

\tsectionnonum{[(9)李嘉图关于资本主义条件下生产和消费的关系的错误观点]}

我们就更仔细地看一看,李嘉图是怎样试图论证不可能有市场商品普遍充斥的:

\begin{quote}{“某一种商品可能生产过多,可能在市场上过剩,以致不能补偿它所花费的资本;但是不可能所有的商品都是这种情况。对谷物的需求受食用者人数的限制,对鞋子和衣服的需求受穿着者人数的限制。但是即使一个社会或社会的一部分可能有它能够消费或愿意消费的那样多的谷物和鞋帽,但不能说每一种自然或人工生产的商品都是这样。有些人如果可能的话会消费更多的葡萄酒。另一些人有了足够的葡萄酒,又会想添置家具或改进家具的质量。还有一些人可能想装饰自己的庭园或扩建自己的住宅。每一个人的心中都怀有做这一切或做其中一部分的愿望;所需要的只是钱,但是除了增加生产以外再没有别的方法可以提供钱。”(第341—342页)}\end{quote}

还能有比这个更幼稚的论证吗?它的意思就是:个别商品已生产出来的数量可能比能够消费的要多。但不可能所有的商品同时都这样。因为用商品来满足的需要是无限的,而所有这些需要不能同时得到满足。相反,满足一种需要的过程会使另一种需要转入可以说是潜在状态。因此,除了满足这些需要的钱以外什么都不需要,而这种钱又只有用增加生产的办法才能获得。这就是说,普遍生产过剩是不可能的。

这一切能说明什么呢?在生产过剩的时候,很大一部分国民(特别是工人阶级)得到的谷物、鞋子等比任何时候都少,更不用说葡萄酒和家具了。如果仅仅在一个国家的全体成员的即使最迫切的需要得到满足之后才会发生生产过剩,那末,在迄今资产阶级社会的历史上,不仅一次也不会出现普遍的生产过剩,甚至也不会出现局部的生产过剩。如果,比如说鞋子、棉布、葡萄酒或者殖民地产品充斥市场,难道这就是说,国民,哪怕只是三分之二的国民,对于鞋子、棉布等的需要已经得到满足而有余了吗?生产过剩同绝对需要究竟有什么关系呢?生产过剩只同有支付能力的需要有关。这里涉及的不是绝对的生产过剩,不是同绝对需要或者占有商品的愿望有关系的生产过剩本身。在这种意义上,既不存在局部的也不存在普遍的生产过剩,它们彼此根本不对立。

但是,李嘉图会说,如果有一批人需要鞋子和棉布,他们为什么不去设法弄到购买这些东西的钱呢?他们为什么不生产一些可以用来购买鞋子和棉布的东西呢?干脆说为什么他们不自己生产鞋子和棉布,不是更简单吗?而在发生生产过剩的时候尤其令人奇怪的是,正是充斥市场的那些商品的真正生产者——工人——缺乏这些商品。这里不能说,他们要得到这些东西,就得去生产这些东西,因为这些东西他们已经生产出来了,但他们还是没有。也不能说,某一种商品之所以充斥市场,是因为对这种商品没有需要。因此,既然甚至不能用充斥市场的商品的数量超过了对这些商品的需要这一点来说明局部的生产过剩,那末,无论如何也不能用市场上的许多商品还有需要,还有未能满足的需要,就否定普遍的生产过剩。

我们仍然以棉织厂主为例\fnote{见本册第545页及以下各页(那里不用“棉织厂主”,而用“麻织厂主”,这丝毫不改变问题的实质)。——编者注}。只要再生产不断进行,——因而,这一再生产中作为商品,作为待出卖的商品而存在的产品即棉布按其价值再转化为货币的阶段也不断进行,——可以说,生产棉布的工人也就消费掉棉布的一部分,并且随着再生产的扩大,也就是随着积累,他们消费的棉布也就相应地增多,或者说,也就有更多的工人来从事棉布生产,而他们同时也就是一部分棉布的消费者。

\tsectionnonum{[(10)危机的可能性转化为现实性。危机是资产阶级经济的一切矛盾的表现]}

在进一步考察之前,我们要指出:

在考察商品的简单形态变化时\endnote{马克思指《政治经济学批判》第一分册《商品的形态变化》一节。(见《马克思恩格斯全集》中文版第13卷第77—88页,特别是第86—88页。)——第579页。}已经显露出来的危机可能性,通过(直接的)生产过程和流通过程的彼此分离再次并且以更发展了的形式表现出来。一旦两个过程不能顺利地互相转化[713]而彼此独立,就发生危机。

在商品的形态变化中,危机的可能性表现为:

首先,实际上作为使用价值存在而在观念上以价格形式作为交换价值存在的商品,必须转化为货币:W—G。如果这个困难——出卖——已经解决,那末,购买,G—W,就再没有什么困难了,因为货币可以同一切东西直接交换。必要的前提就是,商品具有使用价值,商品所包含的劳动是有用的,否则它就根本不是商品。其次,假定商品的个别价值等于它的社会价值,就是说,物化在商品中的劳动时间等于生产该商品的社会必要劳动时间。因此,危机的可能性,就其在形态变化的简单形式中的表现来说,仅仅来自以下情况,即商品形态变化在其运动中经历的形式差别——阶段——第一,必须是相互补充的形式和阶段,第二,尽管有这种内在的必然的相互联系,却是过程的互不相干地存在着、在时间和空间上彼此分开、彼此可以分离并且已经分离、互相独立的部分和形式。因此,危机的可能性只在于卖和买的分离。只是在商品的形式上商品必须克服这里所遇到的困难。一旦它具有货币形式,这种困难就算度过了。但是,往前走,这又是卖和买的分离。如果商品不能以货币形式退出流通领域,或者换句话说,不能推迟自己再转化为商品的时间,如果——就象直接的物物交换中一样——买和卖彼此一致,那末,在上述假定下的危机的可能性就会消失。因为已经假定商品对别的商品所有者来说是使用价值了。在直接的物物交换的形式中,商品只有当它不是使用价值,或者在对方没有别的使用价值可以同它交换的时候,才不能进行交换。因此,只有在两个条件下才不可能进行交换:或者是一方生产了无用之物,或者是对方没有有用之物可以作为等价物同前者的使用价值交换。不过,在这两种情况下根本不会发生交换。然而只要发生交换,它的因素就不是彼此分离的。买者就是卖者,卖者就是买者。所以,既然交换就是流通,从交换形式产生的危机因素就消失了,如果我们说形态变化的简单形式包含着危机的可能性,那只不过是说,在这种形式本身包含着本质上相互补充的因素彼此割裂和分离的可能性。

但是,这也涉及到内容。在进行直接的物物交换的时候,从生产者方面来说,产品的主要部分,是为了满足他自己的需要,或者,到分工有了一些发展以后,是为了满足他所知道的他的协作生产者的需要。作为商品拿来交换的是剩余品,而这个剩余品是否进行交换,却是不重要的。在商品生产的情况下,产品转化为货币,出卖,就成了必不可少的条件。为满足自己需要而进行的直接生产已成为过去。如果商品卖不出去,就会发生危机。商品(个人劳动的特殊产品)转化为它的对立物货币,即转化为抽象一般的社会劳动的困难,在于货币不是作为个人劳动的特殊产品出现,在于已经卖掉了商品而现在持有货币形式的商品的人并不是非要立刻重新买进、重新把货币转化为个人劳动的特殊产品不可。在物物交换中不存在这种对立。在那里不是买者就不能是卖者,不是卖者就不能是买者。卖者——假定他的商品具有使用价值,——的困难仅仅是由于买者可以轻易地推迟货币再转化为商品的时间而产生的。商品转化为货币即出卖商品的这种困难,仅仅是由于商品必须转化为货币,货币却不立即必须转化为商品,因此卖和买可能彼此脱离而产生的。我们说过,这个形式包含着危机的可能性,也就是包含着这样的可能性:相互联系和不可分离的因素彼此脱离,因此它们的统一要通过强制的方法实现,它们的相互联系要通过强加在它们的彼此独立性上的暴力来完成。[714]危机无非是生产过程中已经彼此独立的阶段强制地实现统一。

危机的一般的、抽象的可能性,无非就是危机的最抽象的形式,没有内容,没有危机的内容丰富的起因。卖和买可能彼此脱离。因此它们是潜在的危机。它们的一致对商品来说总是危机的因素。但是它们也可能顺利地相互转化。所以,危机的最抽象的形式(因而危机的形式上的可能性)就是商品的形态变化本身,在商品形态变化中,包含在商品的统一中的交换价值和使用价值的矛盾以至货币和商品的矛盾,仅仅作为展开的运动存在。但是,使危机的这种可能性变成危机,其原因并不包含在这个形式本身之中;这个形式本身所包含的只是:危机的形式已经存在。

而这对于考察资产阶级经济是重要的。世界市场危机必须看作资产阶级经济一切矛盾的现实综合和强制平衡。因此,在这些危机中综合起来的各个因素,必然在资产阶级经济的每一个领域中出现并得到阐明。我们越是深入地研究这种经济,一方面,这个矛盾的越来越新的规定就必然被阐明,另一方面,这个矛盾的比较抽象的形式会再现并包含在它的比较具体的形式中这一点,也必然被说明。

总之,可以说:危机的第一种形式是商品形态变化本身,即买和卖的分离。

危机的第二种形式是货币作为支付手段的职能,这里货币在两个不同的、彼此分开的时刻执行两种不同的职能。

这两种形式都还是十分抽象的,虽然第二种形式比第一种形式具体些。

因此,在考察资本的再生产过程(它同资本的流通是一致的)时,首先要指出,上述两种形式在这里是简单地再现,或者更确切地说,在这里第一次获得了内容,获得了它们可以表现出来的基础。

现在我们就来考察资本从它作为商品离开生产过程然后重新以商品形式从生产过程中产生的时候起所经历的运动。如果我们这里把所有对内容的进一步的规定撇开不谈,那末,总商品资本和它包含的每一单个商品都要经历W—G—W过程,都要完成商品的形态变化。因此,只要资本也是商品并且只是商品,那末包含在这个形式中的危机的一般可能性,即买和卖的分离,也就包含在资本的运动中。此外,鉴于不同商品的形态变化是相互联系的,所以,一种商品转化为货币是因为另一种商品从货币形式再转化为商品。因此,买和卖的分离在这里进一步表现为:一笔资本从商品形式转化为货币形式,相应地另一笔资本就必须从货币形式再转化为商品形式,一笔资本发生第一形态变化,相应地另一笔资本就必须发生第二形态变化,一笔资本离开生产过程,相应地另一笔资本就必须回到生产过程。不同资本的再生产过程或流通过程的这种相互连结和彼此交叉,一方面,由于分工而成为必然的,另一方面,又是偶然的,因此,对危机的内容的规定已经扩大了。

但是,第二,至于由作为支付手段的货币形式产生的危机的可能性,那末,在考察资本时,这种可能性转化为现实性的更现实得多的基础已经显露出来了。例如,织布厂主必须支付全部不变资本,这种不变资本的要素是纺纱厂主、亚麻种植业者、机器制造厂主、制铁厂主、木材业者和煤炭业者等提供的。只要后面这些人生产的不变资本,只加入不变资本的生产而不加入最后商品——布,他们就是通过资本同资本的交换互相补偿各自的生产条件。现在假定[715]织布厂主把布卖给商人,作价1000镑,但用的是一张汇票,所以货币是作为支付手段出现。这个织布厂主又把这张汇票卖给银行家,他在银行家那里用它偿付了一笔什么债务,或者银行家给他办理了汇票贴现。同样,亚麻种植业者凭汇票卖给纺纱厂主,而纺纱厂主又凭汇票卖给织布厂主,同样,机器制造厂主凭汇票卖给织布厂主,制铁厂主和木材业者凭汇票卖给机器制造厂主,煤炭业者凭汇票卖给纺纱厂主、织布厂主、机器制造厂主、制铁厂主和木材业者。此外,制铁厂主、煤炭业者、木材业者和亚麻种植业者之间也用汇票互相支付。现在,如果商人支付不出,织布厂主就不能向银行家支付自己的汇票。

亚麻种植业者开出了由纺纱厂主支付的汇票,机器制造厂主开出了由织布厂主和纺纱厂主支付的汇票。由于织布厂主不能支付,纺纱厂主也就不能支付,他们两人都不能向机器制造厂主支付,而机器制造厂主则不能向制铁厂主、木材业者和煤炭业者支付。他们由于都没有实现自己商品的价值,就全都不能使补偿不变资本的那部分价值得到补偿。这样就要发生普遍的危机。这不过是在考察货币作为支付手段时展现的危机的可能性,但是,在这里,在资本主义生产中,我们已经看到了使危机可能性可能发展成为现实性的相互债权和债务之间、买和卖之间的联系。

在所有情况下:

如果买和卖彼此不发生梗阻,因而没有必要强制地加以平衡,另一方面,如果货币作为支付手段发挥作用的结果是彼此的债权互相抵销,也就是说作为支付手段的货币中潜在地包含着的矛盾没有成为现实;因此,如果危机的这两种抽象形式本身并没有实际地表现出来,那就不会有危机。只要买和卖不彼此脱离,不发生矛盾,或者只要货币作为支付手段所包含的矛盾不出现,因而,只要危机不是同时以其简单的形式——买和卖矛盾的形式和货币作为支付手段的矛盾的形式——出现,那就不可能发生危机。但是,这终究只不过是危机的形式,危机的一般可能性,因而也只不过是现实危机的形式,现实危机的抽象形式。危机的存在以这些形式出现就是以危机的最简单的形式出现,也是以危机的最简单的内容出现,因为这种形式本身就是危机的最简单的内容。但是,这还不是有了根据的内容。有简单的货币流通,甚至有作为支付手段的货币流通——这两者早在资本主义生产以前很久就出现了,却没有引起危机——而没有危机是可能的,也是现实的。因此,单单用这些形式不能说明,为什么这些形式会转向其危机的方面,为什么这些形式潜在地包含着的矛盾会实际地作为矛盾表现出来。

从这里可以看出有些经济学家的极端的庸俗,他们在再也不能用推理来否定生产过剩和危机的现象时,就安慰自己说,在上述形式中既定的[只]是发生危机的可能性,所以,不发生危机是偶然的,发生危机本身也不过是偶然的事。

在商品流通中,接着又在货币流通中发展起来的矛盾,——因而还有危机的可能性,——自然会在资本中再现出来,因为实际上只是在资本的基础上才有发达的商品流通和货币流通。

但是,现在的问题是要彻底考察潜在的危机的进一步发展(现实危机只能从资本主义生产的现实运动、竞争和信用中引出),要就危机来自作为资本的资本所特有的,而不是仅仅在资本作为商品和货币的存在中包含的资本的各种形式规定,来彻底考察潜在的危机的进一步发展。

[716]单单资本的(直接)生产过程本身在这里不能添加什么新的东西。为了使资本的生产过程存在,就得假定这一过程的条件是既定的。因此,在论资本的第一篇——在论直接生产过程的那一篇,并未增加危机的任何新的要素。这里潜在地包含着危机的要素,因为生产过程就是剩余价值的占有,因而也是剩余价值的生产。但是在生产过程本身,这一点是表现不出来的,因为这里不仅谈不到再生产出来的价值的实现,也谈不到剩余价值的实现。

只有在本身同时就是再生产过程的流通过程中,这一点才能初次显露出来。

这里还要指出,我们必须在叙述完成了的资本——资本和利润——之前叙述流通过程或再生产过程,因为我们不仅要叙述资本如何进行生产,而且要叙述资本如何被生产出来。但是,实际运动——这里说的是以发达的、从自己开始并以自己为前提的资本主义生产为基础的实际运动——是从现有资本出发的。因此,对于再生产过程以及在这个过程中得到进一步发展的危机的萌芽,在论述再生产的这一部分只能作不充分的叙述,需要在《资本和利润》一章\endnote{马克思指他的研究中后来发展成为《资本论》第三卷的那一部分。——第586页。}中加以补充。

资本的总流通过程或总再生产过程是资本的生产阶段和资本的流通阶段的统一,也就是把上述两个过程作为自己的不同阶段来通过的过程。这里包含着得到进一步发展的危机的可能性,或者说,包含着得到进一步发展的危机的抽象形式。因此,否认危机的经济学家们只坚持这两个阶段的统一。如果这两个阶段只是彼此分离而不成为某种统一的东西,那就不可能强制地恢复它们的统一,就不可能有危机。如果它们只是统一的而彼此不会分离,那就不可能强制地把它们分离,而这种分离还是危机。危机就是强制地使已经独立的因素恢复统一,并且强制地使实质上统一的因素变为独立的东西。[716]

\tsectionnonum{[(11)危机的形式问题]}

[770a]对第716页的补充。

因此:

(1)危机的一般可能性在资本的形态变化过程本身就存在,并且是双重的。如果货币执行流通手段的职能,危机的可能性就包含在买和卖的分离中。如果货币执行支付手段的职能,货币在两个不同的时刻分别起价值尺度和价值实现的作用,——危机的可能性就包含在这两个时刻的分离中。如果价值在这两个时刻之间有了变动,如果商品在它卖出的时刻的价值低于它以前在货币执行价值尺度的职能,因而也执行相互债务尺度的职能的时刻的价值,那末,用出卖商品的进款就不能清偿债务,因而,再往上推,以这笔债务为转移的一系列交易,都不能结算。即使商品的价值没有变动,只要商品在一定时期内不能卖出,单单由于这一笔债务,货币就不能执行支付手段的职能,因为货币必须在一定的、事先规定的期限内执行支付手段的职能。但是,因为同一笔货币是对一系列的相互交易和债务执行这种职能,所以无力支付的情况就不止在一点上而是在许多点上出现,由此就发生危机。

这就是危机的两种形式上的可能性。在没有第二种可能性的情况下,第一种可能性也可能出现,就是说,在没有信用的情况下,在没有货币执行支付手段的职能的情况下,也可能发生危机。但是,在没有第一种可能性的情况下,即在没有买和卖彼此分离的情况下,却不可能出现第二种可能性。但是,在第二种场合所以发生危机,不仅是因为商品一般地卖不出去,而且是因为商品不能在一定期限内卖出去,在这里危机所以发生,危机所以具有这样的性质,不仅由于商品卖不出去,而且由于以这一定商品在这一定期限内卖出为基础的一系列支付都不能实现。这就是本来意义上的货币危机形式。

因此,如果说危机的发生是由于买和卖的彼此分离,那末,一旦货币执行支付手段的职能,危机就会发展为货币危机,在这种情况下,只要出现了危机的第一种形式,危机的这第二种形式就自然而然地要出现。因此,在研究为什么危机的一般可能性会变为现实性时,在研究危机的条件时,过分注意从货币作为支付手段的发展中产生的危机的形式,是完全多余的。正因为这个缘故,经济学家们乐于举出这个显而易见的形式作为危机的原因。(既然货币作为支付手段的发展是同信用和信用过剩的发展联系在一起,那末当然应该说明这些现象的原因,但是这里还不是这样说明的地方。)

(2)只要危机是由同商品的价值变动不一致的价格变动和价格革命引起的,它当然就不能在考察一般资本的时候得到说明,因为在考察一般资本时假定价格是同商品的价值一致的。

(3)危机的一般可能性就是资本的形式上的形态变化本身,就是买和卖在时间上和空间上的彼此分离。但是这决不是危机的原因。因为这无非是危机的最一般的形式,即危机本身的最一般的表现。但是,不能说危机的抽象形式就是危机的原因。如果有人要问危机的原因,那末他想知道的就是,为什么危机的抽象形式,危机的可能性的形式会从可能性变为现实性。

(4)危机的一般条件,只要不取决于和价值波动不同的价格波动(不论这种波动同信用有无关系),就必须用资本主义生产的一般条件来说明。[770a]

[716](危机可能发生在:第一,[货币]再转化为生产资本的时候;第二,由于生产资本的要素特别是原料的价值变动,如棉花收成减少,因而它的价值增加。我们这里所涉及的还不是价格,而是价值。)[716]

[770a]第一个因素。货币再转化为资本。假定这是生产或再生产的一定阶段。这里,可以把固定资本看成既定的、不变的,它不加入价值形成过程。既然原料的再生产不仅取决于花费在其中的劳动,而且取决于这一劳动的同自然条件有关的生产率,那末,产品量——甚至[XIV—771a]同一劳动量生产的产品量——就可能减少(在歉收时)。于是原料的价值增加,原料的量则减少。为了以原有规模继续生产,货币必须按一定比例再转化为资本的不同组成部分,而现在这个比例被破坏了。用于原料的部分必须增加,剩下用于劳动的部分就减少,因此就不能吸收和以前相同的劳动量。第一是物质上不可能,因为原料的量减少了;第二是因为产品价值中必须有比原来更大的一部分用于原料,因而只能有较小一部分转化为可变资本。再生产不能按原有规模重新进行。一部分固定资本要闲置起来,一部分工人会被抛到街头。利润率会下降,因为不变资本的价值同可变资本相比增加了,使用的可变资本减少了。以利润率和劳动剥削率不变为根据事先规定的固定提成——利息、地租——仍旧不变,有一部分不能支付。于是发生危机。劳动危机和资本危机。因此,这就是由于靠产品价值补偿的一部分不变资本的价值提高而引起的再生产过程的破坏。其次,虽然利润率下降,产品却会涨价。如果这种产品作为生产资料加入其他生产领域,那末这种产品的涨价会使其他领域的再生产遭到同样的破坏。如果这种产品作为生活资料加入一般消费,那末,它或者也加入工人的消费,或者不加入工人的消费。如果是前者,它的后果同后面要讲的可变资本遭到破坏时产生的后果一样。但是,在这种产品加入一般消费的情况下,由于这种产品涨价(如果这种产品的消费不减少),对其他产品的需求就会减少,因而其他产品就不能再转化为相当于其价值的货币额,这样,其他产品的再生产的另一方面——不是货币再转化为生产资本,而是商品再转化为货币——就会遭到破坏。无论如何,我们所考察的这个生产部门的利润量和工资量会减少,从而其他生产部门出卖商品所得的一部分必要收入也会减少。

但是,即使不受收成的影响,或者说,即使不受提供这种原料的劳动的受自然因素制约的生产率的影响,这种原料不足的情况也可能发生。就是说,如果某个生产部门花费在机器等等上的那部分剩余价值,那部分追加资本过多,那末,虽然按原来的生产规模原料是够的,但按新的生产规模就不够了。因此,这种情况是由于追加资本不按比例地转化为资本的不同要素而产生的。这是固定资本生产过剩的情况,它所产生的现象正好同第一种情况[即原料歉收时]所产生的现象完全一样。

[861a][…………………………………………………………………][注:在手稿中,这一页的左上角撕掉了。结果原文前九行只留下六行的右半截,全文不可能完全恢复,但是可以猜测,马克思在这里谈的是“由可变资本的价值革命”而产生的危机。由比如说歉收引起的“必要生活资料的涨价”,会导致用于“可变资本所推动的”工人的支出增加。“同时,也会导致对其他一切商品、即一切不加入”工人的“消费的商品”的需求“减少”。因此“这些商品按其价值出卖”就成为不可能,“它们的再生产的第一阶段”,即商品向货币的“转化”就遭到破坏。于是,生活资料的涨价“导致其他生产部门发生危机”。

这一页撕坏的部分的最后两行中包含着总结这全段议论的思想:“不论这种原料是作为材料加入不变资本还是作为生活资料”加入工人的消费,由于原料涨价都可能发生危机。——编者注]

或者,它们[危机]是以固定资本的生产过剩,因而,是以流动资本的相对的生产不足为基础的。

因为固定资本同流动资本一样,都是由商品组成的,所以,那些否认商品生产过剩而同时又承认固定资本生产过剩的经济学家们的立场是最可笑不过的了。

(5)由于再生产的第一阶段遭到破坏,也就是由于商品向货币的转化发生障碍,或者说出卖发生障碍而产生的危机。在发生第一种[由于原料涨价而引起的]危机的情况下,危机是由于生产资本的要素的回流发生障碍而产生的。[XIV-861a]

\tsectionnonum{[(12)资本主义条件下生产和消费的矛盾。主要消费品生产过剩转化为普遍生产过剩]}

[XIII—716]在开始考察危机的新形式\endnote{在这之后不久,马克思在手稿第XIII本封面(手稿第770a页)和手稿第XIV本封面(手稿第771a和861a页)上写了关于危机形式的短评。根据马克思的注解:“对第716页的补充”,把这几页的正文放在前面(第11节:《危机的形式问题》)。——第591页。}之前,我们再回过头来看看李嘉图的著作和前面举过的例子\fnote{见本册第545页及以下各页(那里不用“棉织厂主”,而用“麻织厂主”,这丝毫不改变问题的实质),并见第579页。——编者注}。

[716]只要棉织厂主进行再生产和积累,他的工人也就是他的一部分产品的买者,他们把自己的一部分工资花费在棉布上。正因为工厂主进行生产,所以,工人们就有购买他的一部分产品的钱,就是说,工人们部分地给他提供了出卖产品的可能性。工人作为需求的代表所能购买的,只是加入个人消费的商品,因为他不是自己使用自己的劳动,因而也不是自己占有实现自己劳动的条件——劳动资料和劳动材料。所以,这一点已经把生产者的最大部分(在资本主义生产发达的地方就是工人本身)排除在消费者、买者之外了。他们不购买原料和劳动资料,他们只购买生活资料(直接加入个人消费的商品)。因此,说生产者和消费者是一回事,那是最可笑不过的了,因为对于很大数量的生产部门——所有不生产直接消费品的部门——来说,大多数参加生产的人是绝对被排斥于购买他们自己的产品之外的。他们决不是自己的这很大一部分产品的直接消费者或买者,虽然他们支付包含在他们购买的消费品中的自己产品的一部分价值。这里也可以看出,“消费者”这个词是模糊不清的,把“消费者”这个词同“买者”这个词等同起来是错误的。从生产消费的意义来说,恰恰是工人消费机器和原料,在劳动过程中使用它们。但是工人并不是为了自己而使用机器和原料,因此,也就不是机器和原料的买者。对于工人来说,机器和原料不是使用价值,不是商品,而是一个过程的客观条件,而工人本身则是这个过程的主观条件。

[717]可是有人会说,雇用工人的企业主在购买劳动资料和劳动材料时是工人的代表。但是,企业主代表工人——指的是在市场上代表——与假定说工人自己代表自己,条件是不一样的。企业主必须出卖包含着剩余价值,即无酬劳动的商品量,要是工人的话,就只须出卖把生产中预付的价值——以劳动资料、劳动材料和工资的价值形式出现——再生产出来的商品量。因此,资本家需要的市场比工人需要的市场大。而且,企业主是否认为市场条件对于开始再生产已充分有利,这取决于企业主而不是工人。

因此,对于一切不是用于个人消费而必须用于生产消费的物品来说,即使再生产过程不遭到破坏,工人也是生产者而不是消费者。

因此,主张把资本主义生产中的消费者(买者)和生产者(卖者)等同起来,从而否定危机,是再荒谬不过的了。这两者是完全不一样的。在再生产过程继续进行的情况下,只是对3000个生产者之中的一个,即对资本家,才可以说是等同的。反过来,说消费者就是生产者,也同样是错误的。土地所有者(收取地租的人)不生产,可是他消费。所有货币资本的代表也是这种情况。

否认危机的各种辩护论言论所证明的东西,总是和它们想要证明的相反,就这一点说,它们是重要的。它们为了否认危机,在有对立和矛盾的地方大谈统一。因此,说它们是重要的,只是因为可以说:它们证明,如果被它们用想象排除了的矛盾实际上不存在,那就不会有任何危机。但是,因为这些矛盾存在着,所以实际上有危机。辩护论者为否定危机存在而提出来的每个根据,都是仅仅在他们想象中被排除了的矛盾,所以是现实的矛盾,所以是危机的根据。用想象排除矛盾的愿望同时就是实际上存在着矛盾的一个证明,这些矛盾按照善良的愿望是不应该存在的。

工人实际上生产的是剩余价值。只要他们生产剩余价值,他们就有东西消费。一旦剩余价值的生产停止了,他们的消费也就因他们的生产停止而停止。但是,他们能够消费,决不是因为他们为自己的消费生产了等价物。相反,当他们仅仅生产这样的等价物时,他们的消费就会停止,他们就没有等价物消费了。或者他们的劳动会停止,或者他们的劳动会缩减,或者,无论如何,他们的工资会降低。在后一种情况下——如果生产水平不变——他们就不是消费他们生产的等价物。但是,这时他们之所以缺少钱,不是因为他们生产的东西不够,而是因为他们从他们所生产的产品中得到的太少。

因此,如果把这里所考察的关系简单地归结为消费者和生产者的关系,那就忘记了从事生产的雇佣工人和从事生产的资本家是两类完全不同的生产者,更不用说那些根本不从事生产活动的消费者了。这里又是用把生产中实际存在的对立撇开的办法来否定对立。仅仅雇佣工人和资本家的关系本身就包含着:

(1)生产者的最大部分(工人),并不是他们所生产的产品的很大一部分,即劳动资料和劳动材料的消费者(买者);

(2)生产者的最大部分,即工人,只有在他们生产的产品大于其等价物时,即在他们生产剩余价值,或者说,剩余产品时,才可能消费这个等价物。他们始终必须是剩余生产者,他们生产的东西必须超过自己的[有支付能力的]需要,才能在[718]自己的这些需要的范围内成为消费者或买者。\endnote{手稿中接着有一小段关于李嘉图对货币和交换价值的观点的插话。马克思把这段插话放在括号中,并注明:这段话对叙述的直接联系有妨碍,应该移到别处去。本版以脚注形式把它移到前面第575页。——第594页。}

因此,就这个生产者阶级来说,说生产和消费是统一的这种论调,无论如何一看就知道是错误的。

如果李嘉图说,需求的唯一界限是生产本身,而生产只受资本的限制,\endnote{大·李嘉图《政治经济学和赋税原理》1821年伦敦第三版第339页:“……需求只受生产限制”(马克思在前面第563页连同较长的上下文一起引证过这句话)以及第347页:“需求是无限的……资本的使用也是无限的”(马克思在前面第567页连同较长的上下文一起引证过这段话)。——第594页。}那末,如果剥去错误假定的外衣,实际上这只不过是说,资本主义生产只以资本作为自己的尺度,同时这里所说的资本也包括作为资本的生产条件之一并入资本(为资本所购买)的劳动能力。可是,问题恰恰在于资本本身是否也是消费的界限。无论如何从消极意义上说它是消费的界限,就是说,消费的东西不可能多于生产的东西。但问题是,从积极意义上说它是不是消费的界限,是不是在资本主义生产的基础上生产多少,就能够或者必须消费多少。如果对李嘉图的论点作正确的分析,那末,这个论点所说的恰恰同李嘉图想说的相反,——就是说,进行生产是不考虑消费的现有界限的,生产只受资本本身的限制。而这一点确实是这种生产方式的特点。

因此,根据假定,市场上比如说棉织品充斥,以致一部分或者全部都卖不出去,或者要大大低于它的价格才卖得出去。(我们暂且说价值,因为在考察流通或再生产过程时,涉及的还是价值,而不是费用价格,更不是市场价格。)

此外,在整个这一分析中,不言而喻的是:不可否认,有些部门可能生产过多,因此另一些部门则可能生产过少;所以,局部危机可能由于生产比例失调而发生(但是,生产的合乎比例始终只是在竞争基础上生产比例失调的结果),这种生产比例失调的一般形式之一可能是固定资本的生产过剩,或者另一方面,也可能是流动资本的生产过剩。\fnote{[720}(当发明了纺机的时候,同织布业比较,曾经出现纱的生产过剩。一旦织布业采用机器织机,这种比例失调就消除了。)[720]]正如商品按其价值出卖的条件是商品只包含社会必要劳动时间一样,对于资本的某一整个生产领域来说,这种条件就是,这个特殊领域所花费的只是社会总劳动时间中的必要部分,只是为满足社会需要(需求)所必要的劳动时间。如果这个领域花费多了,即使每一单位商品所包含的只是必要劳动时间,这些单位商品的总量所包含的却会多于社会必要劳动时间,正如单位商品虽然具有使用价值,这些单位商品的总量在既定的前提下却会丧失它的一部分使用价值。

可是我们这里谈的,不是以生产的比例失调为基础的危机,就是说,不是以社会劳动在各生产领域之间的分配比例失调为基础的危机。这一点只有在谈到资本竞争的时候才能谈到。前面已经说过\fnote{见本册第229—234页。——编者注},由于这种比例失调而引起的市场价值的提高或降低,造成资本从一个生产领域抽出并转入另一个生产领域,造成资本从一个领域向另一个领域的转移。可是,这种平衡本身已经包含着:它是以平衡的对立面为前提的,因此它本身可能包含危机,危机本身可能成为平衡的一种形式。但是,这种危机是李嘉图等人所承认的。

我们在考察生产过程时\endnote{马克思指他的1861—1863年手稿中直接在《剩余价值理论》前面的第I—V本手稿,特别是指关于绝对剩余价值的生产和相对剩余价值的生产那两节。——第596页。}已经看到,资本主义生产竭力追求的只是攫取尽可能多的剩余劳动,就是靠一定的资本物化尽可能多的直接劳动时间,其方法或是延长劳动时间,或是缩短必要劳动时间,发展劳动生产力,采用协作、分工、机器等,总之,进行大规模生产即大量生产。因此,在资本主义生产的本质中就包含着不顾市场的限制而生产。

在考察再生产时,首先假定生产方式不变,而在生产扩大的时候,生产方式实际上在一段时间内也是保持不变的。这里生产出来的商品量所以增加,是由于使用了更多的资本,而不是由于更有效地使用了资本。但是单单资本的量的增加[719]同时也就包含资本的生产力的增加。如果说资本的量的增加是生产力发展的结果,那末反过来说,一个更广阔的、扩大了的资本主义基础又是生产力发展的前提。这里存在着相互作用。因此,在更加广阔的基础上进行的再生产即积累,即使它最初只表现为生产在量上的扩大(在同样的生产条件下投入更多的资本),但在某一点上也总会在质上表现为进行再生产的条件具有较大的效率。因此,产品量增加的比例要大于在扩大再生产即积累中资本增长的比例。

现在再回过头来看看我们那个棉布的例子。

由于棉布充斥而造成的市场停滞,会使织布厂主的再生产遭到破坏。这种破坏首先会影响到他的工人。于是,工人对于他的商品棉布和原来加入他们消费的其他商品来说,现在只在较小的程度上是消费者,或者根本不再是消费者了。他们当然需要棉布,但是他们买不起,因为他们没有钱,而他们之所以没有钱,是因为他们不能继续生产,而他们之所以不能继续生产,是因为已经生产的太多了,棉布充斥市场。李嘉图的劝告,不论是“增加他们的生产”也好,“生产别的东西”\endnote{大·李嘉图《政治经济学和赋税原理》1821年伦敦第三版第342页:“……除了增加生产以外再没有别的方法可以提供钱”(马克思在前面第577页连同较长的上文一起引证过这句话),以及第339—340页:“他不可能继续不断地生产没有需求的商品”(马克思在前面第563页和573页连同较长的上文一起引证过这句话)。——第597页。}也好,都不能帮他们的忙。他们现在代表着暂时的人口过剩的一部分,代表着工人的生产过剩的一部分,在这种场合,就是棉布生产者的生产过剩的一部分,因为市场上出现的是棉布的生产过剩。

但是,除了投入织布生产的资本所直接雇用的工人以外,棉布再生产的这种停滞还影响一批别的生产者:纺纱者、棉花种植业者、纱锭和织机的生产者、铁和煤的生产者等等。所有这些人的再生产同样都要遭到破坏,因为棉布的再生产是他们进行再生产的条件。即使在他们自己的生产领域里没有生产过剩,就是说,即使那里生产的数量没有超过棉布工业销路畅通时所确定的合理的数量,这种情况也会发生。所有这些生产部门有一个共同点,就是它们不是把自己的收入(工资和利润,只要利润是作为收入来消费,而不是用于积累)用在它们自己的产品上,而是用在那些生产消费品,其中包括棉布的生产领域的产品上。这样,正因为市场上棉布过多,对于棉布的消费和需求就会减少。但是,对于用棉布的这些间接生产者的收入购买的作为消费品的其他一切商品的需求也会减少。棉布的这些间接生产者用来购买棉布和其他消费品的钱所以会受到限制和减少,就是因为市场上棉布过多。这也影响到其他商品(消费品)。它们现在突然发生相对的生产过剩,因为用来购买它们的钱减少了,从而对于它们的需求减少了。即使这些生产部门生产的东西并没有过多,现在也要发生生产过剩。

如果不仅棉布,而且麻布、丝绸和呢绒都发生生产过剩,那末不难理解,这些为数不多但居主导地位的物品的生产过剩就会在整个市场上引起多少带普遍性的(相对的)生产过剩。一方面,是再生产的一切条件出现过剩,各种各样卖不出去的商品充斥市场;另一方面,是资本家遭到破产,工人群众忍饥挨饿,一贫如洗。

可是,这一论证是两面的。如果说,一些主要消费品的生产过剩必然引起多少带普遍性的生产过剩,是容易理解的,那末决不能因此就说,这些主要消费品的生产过剩究竟怎样发生的问题也是可以理解的了。因为,普遍生产过剩的现象是从这些工业部门直接雇用的工人和为这些部门的产品生产各种先行要素(即生产这些部门所使用的不变资本的不同阶段)的一切生产部门的工人相互依存中得出来的。就后面这些部门来说,生产过剩是结果。但是,在前面那些部门中,生产过剩又是从哪里产生的呢?因为,只要前面那些部门扩大生产,后面这些部门也就会扩大生产,而随着生产的这种扩大,收入的普遍增长,从而这些部门本身的消费的普遍增长似乎也就有了保证。\endnote{手稿中这里接着有一段放在括号里的插话,其中举了由于使用纺机而引起纱生产过剩的局部危机的例子。本版以脚注形式把这段插话移到前面第595页。——第598页。}[719]

\tsectionnonum{[(13)生产扩大和市场扩大的不一致。李嘉图关于消费增长和国内市场扩大有无限可能性的见解]}

[720]如果我们说,不断扩大的生产{生产逐年扩大是由于两个原因:第一,由于投入生产的资本不断增长;第二,由于资本使用的效率不断提高;在再生产和积累期间,小的改良日积月累,最终就使生产的整个规模完全改观;这里进行着改良的积累,生产力的日积月累的发展}需要一个不断扩大的市场,而生产比市场扩大得快,那末,这不过是把要说明的现象用另一种说法说出,不是用它的抽象形式,而是用它的现实形式说出而已。市场比生产扩大得慢;换句话说,在资本进行再生产时所经历的周期中,——在这个周期中,资本不是简单地以原来的规模把自己再生产出来,而是以扩大了的规模把自己再生产出来,不是画一个圆圈,而是画一个螺旋形,——会出现市场对于生产显得过于狭窄的时刻。这会发生在周期的末尾。但这也仅仅是说:市场商品充斥了。生产过剩现在变得明显了。假如市场的扩大与生产的扩大步伐一致,就不会有市场商品充斥,不会有生产过剩。

但是,只要承认市场必须同生产一起扩大,在另一方面也就是承认有生产过剩的可能性,因为市场有一个外部的地理界限,一个国内市场同一个既是国内又是国外的市场相比是有限的,而后者和世界市场相比也是有限的,世界市场在每个一定的时刻也是有限的,但是潜在地是能扩大的。因此,如果承认为了不发生生产过剩,市场必须扩大,那也就是承认生产过剩是可能发生的,因为既然市场和生产是两个彼此独立的因素,那末,一个扩大同另一个扩大就可能不相适应,市场的范围对于生产来说可能扩大得不够快,新的市场——市场的不断扩大——可能很快被生产超过,因而扩大了的市场现在表现为一个界限,正如原来比较狭窄的市场曾经表现为一个界限一样。

因此,李嘉图否定关于随着生产的扩大和资本的增长市场也必定会扩大的观点,他是前后一贯的。照李嘉图看来,一个国家现有的全部资本,在这个国家可以有利地加以使用。因此李嘉图反驳亚·斯密,因为亚·斯密一方面提出过同他(李嘉图)一样的观点,另一方面以自己惯有的理性本能也反对过这个观点。斯密还不知道生产过剩以及从生产过剩产生危机的现象。他所知道的仅仅是同信用制度和银行制度一起自然发生的信用危机和货币危机。实际上,他把资本积累看做普遍的国民财富和福利的绝对增加。另一方面,他认为,单单从国内市场发展为国外市场、殖民地市场和世界市场本身,就是国内市场上存在所谓相对的(潜在的)生产过剩的证明。

值得把李嘉图反驳斯密的话引在这里:

\begin{quote}{“商人把他们的资本投入对外贸易或海运业时,他们总是出于自由选择而不是迫不得已;他们这样做是因为在这些部门中他们的利润比在国内贸易中要大一些。亚当·斯密曾正确地指出:‘每一个人对于食物的欲望都要受人胃的有限容量的限制}\end{quote}

{亚·斯密这里是大错特错了,因为他不把农业中生产的奢侈品包括在食物内},

\begin{quote}{但对于住宅、衣服、车马和家具方面的舒适品和装饰品的欲望似乎是没有限制或确定的界限的。’”李嘉图继续说:“因此,自然界对于一定时间内可以有利地用在农业上的资本的量,必然作了限制}\end{quote}

{不是因此而有出口农产品的民族吗?好象人们就不能违反自然,把一切可能的资本投于农业,以便例如在英国生产甜瓜、无花果、葡萄之类,栽培花卉之类,繁殖飞禽走兽之类?(请看例如罗马人单单在人工养鱼业上投下的资本。)难道工业的原料不是农业资本生产的吗?},

\begin{quote}{但是,自然界对于能用来生产生活上的‘舒适品和装饰品’的资本的量却没有规定什么界限〈好象自然界真是同这件事有什么关系似的!〉。人们所考虑的目的就是最大限度地得到这些物品。只是由于对外贸易或海运业可以更好地达到这个目的,人们才宁愿从事对外贸易或海运业,而不愿自己在国内生产所需要的商品或其代用品。但是,如果由于特殊情况,我们不能投资于对外贸易或海运业,那末,虽然获利较少,我们也会投资于国内;既然对于‘住宅、衣服、车马和[721]家具方面的舒适品和装饰品的欲望’是没有界限的,那末除了使我们维持生产这些物品的工人生活的能力受到限制的界限以外,用来生产这些物品的资本是不可能有任何界限的。但是,亚当·斯密却说成好象从事海运业不是出于自由选择,而是迫不得已;好象资本不投入这一部门就会闲置起来;好象国内贸易中的资本不限制在一定限度之内就会过剩。他说:‘当任何一个国家的资本增加到已经无法全部用来供应本国的消费并维持本国的生产劳动时{这一段话的着重号是李嘉图自己加的},资本的剩余部分就自然流入海运业,被用来为其他国家执行同样的职能’……但是,大不列颠的这部分生产劳动难道不能用来生产其他种类的商品并用以购买国内有更大需求的物品吗?如果不能这样,那末,虽然获利较少,难道我们不能用这种生产劳动在国内生产这些有需求的商品或者至少生产其代用品吗?如果我们需要天鹅绒,难道我们不能自己试制吗?如果试制不成,难道就不能生产更多的呢绒或我们需要的某种其他物品吗?我们生产工业品并用来在国外购买其他商品,是因为这样做比在国内生产能获得数量更多的商品{没有质量的差别!}。如果我们进行这种贸易的可能性被剥夺,我们马上就会重新开始为自己制造这些商品。但是,亚当·斯密的这种看法和他关于这一问题的所有一般论点是矛盾的。‘如果某个外国{李嘉图引用斯密的话}供应我们某种商品,比我们自己生产这种商品便宜,那就不如把我们自己的劳动用于我们有某种优越性的部门,而用我们自己的劳动的一部分产品向这个国家购买这种商品。国家的劳动总量由于总是与使用它的资本成比例{极不成比例!}〈这句话的着重号也是李嘉图自己加的〉,它就不会因此而减少,只不过需要找到能够最有利地使用它的部门而已。’他又说:‘所以,那些拥有的食物多于其消费量的人,总是愿意拿这部分多余的食物,或者可以说,这部分食物的价格,去交换其他物品。在满足了这种有限的欲望以后剩下的一切,就被用来满足那些永远不能完全满足并且看来根本没有止境的欲望。穷人为了获得食物,就尽力满足富人的嗜好,为了更有把握获得食物,他们就互相在自己报酬的低廉和工作的熟练方面竞争。工人人数随着食物数量的增加或者随着农业的改良和耕地的扩大而增加。由于他们的工作性质允许实行极细的分工,他们能够加工的材料数量比他们的人数增加得快得多。因此,对于任何一种材料,凡是人类的发明能够把它用来改善或装饰住宅、衣服、车马或家具的,都产生了需求;对于地下蕴藏的化石和矿物,对于贵金属和宝石,也都产生了需求。’从上述各点可以看出,需求是无限的,只要资本还能带来某种利润,资本的使用也是无限的,无论资本怎样多,除了工资提高以外,没有其他充分原因足以使利润降低。此外还可以补充一句:使工资提高的唯一充分而经常的原因,就是为越来越多的工人提供食品和必需品的困难越来越大。”(同上,第344—348页)}\end{quote}

\tsectionnonum{[(14)生产力不可遏止的发展和群众消费的有限性之间的矛盾是生产过剩的基础。关于普遍生产过剩不可能的理论的辩护论实质]}

生产过剩这个词本身会引起误解。只要社会上相当大一部分人的最迫切的需要,或者哪怕只是他们最直接的需要还没有得到满足,自然绝对谈不上产品的生产过剩(在产品量超过对产品的需要这个意义上讲)。相反,应当说,在这个意义上,在资本主义生产的基础上经常是生产不足。生产的界限是资本家的利润,决不是生产者的需要。但是,产品的生产过剩和商品的生产过剩是完全不同的两回事。李嘉图认为,商品形式对于产品是无关紧要的,其次,商品流通只是在形式上不同于物物交换,交换价值在这里只是物质变换的转瞬即逝的形式,因而货币只是形式上的流通手段;这一切实际上都是来源于他的这样一个前提:资产阶级生产方式是绝对的生产方式,也就是没有更确切的特殊规定的生产方式,因此,这种生产方式的一切规定都只是某种形式上的东西。因此,李嘉图也就不能承认资产阶级生产方式包含着生产力自由发展的界限——在危机中,特别是在作为危机的基本现象的生产过剩中暴露出来的界限。

[722]李嘉图从他引用、赞同并因而复述的斯密论点中看到,追求各种各样使用价值的无限“欲望”,总是在这样一种社会状态的基础上得到满足,在这种社会状态中,广大的生产者仍然或多或少只限于获得“食品”和“必需品”,因此,只要财富超出必需品的范围,绝大多数生产者就或多或少被排斥于财富的消费之外。

当然,后一种情况在古代以奴隶制为基础的生产中也是存在的,并且更加如此。但是古代人连想也没有想到把剩余产品变为资本。即使这样做过,至少规模也极有限。(古代人盛行本来意义上的财宝贮藏,这说明他们有许多剩余产品闲置不用。)他们把很大一部分剩余产品用于非生产性支出——用于艺术品,用于宗教的和公共的建筑。他们的生产更难说是建立在解放和发展物质生产力(即分工、机器、将自然力和科学应用于私人生产)的基础上。总的说来,他们实际上没有超出手工业劳动。因此,他们为私人消费而创造的财富相对来说是少的,只是因为集中在少数人手中,而且这少数人不知道拿它做什么用,才显得多了。如果说因此在古代人那里没有发生生产过剩,那末,那时有富人的消费过度,这种消费过度,到罗马和希腊的末期就成为疯狂的浪费。古代人中间的少数商业民族,部分地就是靠所有这些实质上贫穷的民族养活的。而构成现代生产过剩的基础的,正是生产力的不可遏止的发展和由此产生的大规模的生产,这种大规模的生产是在这样的条件下进行的:一方面,广大的生产者的消费只限于必需品的范围,另一方面,资本家的利润成为生产的界限。

李嘉图和其他人对生产过剩等提出的一切反对意见的基础是,他们把资产阶级生产或者看作不存在买和卖的区别而实行直接的物物交换的生产方式,或者看作社会的生产,在这种生产中,社会好象按照计划,根据为满足社会的各种需要所必需的程度和规模,来分配它的生产资料和生产力,因此每个生产领域都能分到为满足有关的需要所必需的那一份社会资本。这种虚构,一般说来,是由于不懂得资产阶级生产这一特殊形式而产生的,而所以不懂又是由于一种成见,认为资产阶级生产就是一般生产。正象一个信仰某一宗教的人把这种宗教看成一般宗教,认为除此以外都是邪教一样。

相反,倒是应该问一问:在资本主义生产的条件下,每个人都为自己而劳动,而特殊劳动必须同时表现为自己的对立面即抽象的一般劳动,并以这种形式表现为社会劳动,——在这样的资本主义生产的基础上,不同生产领域之间的必要的平衡和相互联系,它们之间的限度和比例的建立,除了通过经常不断地消除经常的不协调之外,用别的办法又怎么能够实现呢?这一点在谈到通过竞争达到平衡时就已经得到承认,因为这种平衡总是以有什么东西要平衡为前提,就是说,协调始终只是消除现存不协调的那个运动的结果。

因此,李嘉图也承认个别商品有可能充斥市场。不可能的只是同时的普遍的市场商品充斥。因此,不否认任何特殊生产领域有生产过剩的可能性。他认为,[普遍的]生产过剩和由此而来的普遍的市场商品充斥所以不可能,是因为这种现象不可能在一切生产领域同时发生。“普遍的市场商品充斥”这个词总是应该有保留地来理解,因为在发生普遍生产过剩的时候,有些领域的生产过剩始终只是主要交易品生产过剩的结果、后果,始终只是相对的,只是因为其他领域存在着生产过剩才成为生产过剩。

辩护论恰好把这一点颠倒过来了。照这些辩护论者的说法,只出现主动的生产过剩的那些主要交易品(一般说这是只能大规模地和用工厂方式进行生产的物品,在农业上也一样),它们的生产过剩之所以成为生产过剩,仅仅因为会出现相对的,或者说,被动的生产过剩的那些物品存在着生产过剩。按照这种看法,生产过剩之所以存在,仅仅因为生产过剩不是普遍的。生产过剩的相对性,即一些领域中现实的生产过剩引起另一些领域的生产过剩这个事实,被表述如下:普遍的生产过剩并不存在,因为,如果生产过剩是普遍的,一切生产领域相互之间就会保持同样的比例;就是说,普遍的生产过剩等于按比例生产,而按比例生产是排除生产过剩的。据说,这是反对普遍生产过剩的论据。[723]就是说,因为绝对意义上的普遍生产过剩并不是生产过剩,而只是一切生产领域的生产力的发展都超过了通常的水平,所以,据说,现实的生产过剩(它恰恰不是这种不存在的、自我消除的生产过剩)并不存在。其实,只是因为现实的生产过剩不是这样的生产过剩,所以它是存在的。

这种可怜的诡辩如果更详细地加以考察,可以归结如下:

假定铁、棉布、麻布、丝绸、呢绒等等发生生产过剩,那末不能说,例如煤生产得太少,因而造成了上述生产过剩;因为铁等等的生产过剩也就包含着煤的生产过剩,正如棉布的生产过剩也就包含着棉纱的生产过剩一样。{和棉布相比,棉纱可能生产过剩,和机器等相比,铁可能生产过剩。这种生产过剩总是不变资本的相对生产过剩。}因此,如果有某些物品作为组成要素即原料、辅助材料或生产资料加入另一些物品(加入这样一些商品,这些“商品可能生产过多,可能在市场上过剩,以致不能补偿它所花费的资本”\fnote{见本册第570、575、577页,那些地方引用了李嘉图这段话的全文。——编者注}),而另一些物品的绝对生产过剩正是需要加以说明的事实,从而某些物品的生产过剩是不言而喻的,那末,就根本谈不上某些物品的生产过剩。因此,说得上生产不足的是其他一些物品,它们直接归属的生产领域,既不属于根据假定要发生生产过剩的主要交易品的生产领域,也不属于这样一些领域,这些领域由于为主要交易品进行中间生产,其生产规模必须至少同产品最后阶段的规模一样,——不过没有什么东西会妨碍这些领域的生产达到更大的规模,因此在生产过剩内部又可能发生生产过剩。例如,虽然煤生产的数量必须使一切以煤作为必要生产条件的工业部门都能进行生产,因而铁、棉纱等等的生产过剩已经包含了煤的生产过剩(即使煤生产的数量只与铁和棉纱的生产成比例),但是,生产出来的煤也可能比铁、棉纱等的生产过剩所要求的还多。这种情况不仅是可能的,而且是非常可能的。因为,煤和棉纱的生产,以及其他所有仅仅为那些必须在别的领域完成的产品提供条件或充当准备阶段的领域的生产,都不是依据直接的需求,不是依据直接的生产或再生产,而是依据它们自身不断扩大的程度、限度、比例来进行的。而在这种考虑的支配下可能超过目标,那是不言而喻的。由此可见,生产不足的[不是上述那些产品,而]是其他物品,例如钢琴、宝石等,生产不足是发生在这些其他物品的部门中。{当然,也会发生这样的生产过剩,在那里,非主要物品的生产过剩不是后果,相反,生产不足倒是生产过剩的原因,例如在谷物歉收或棉花歉收等的情况下。}

当这种[关于生产不足的]说法被应用到国际范围——就象萨伊\endnote{马克思指萨伊的下述论断(在他的《给马尔萨斯的信》1820年巴黎版第15页):例如,如果英国商品充斥意大利市场,那末,原因就在于能够同英国商品交换的意大利商品生产不足。萨伊的这些论断在匿名著作《论马尔萨斯先生近来提倡的关于需求的性质和消费的必要性的原理》(1821年伦敦版第15页0中引证过,在马克思的第XII本札记本第12页对这部著作所作的摘录中也有这些论断。并参看马克思在本卷第一册第276页分析的萨伊的这一论点:“某些产品的滞销,是由另一些产品太少引起的。”——第607页。}和继萨伊之后的其他经济学家所作的那样,——的时候,就更加暴露其荒谬了。例如,他们断言,不应说英国生产过剩,而应说意大利生产不足。如果第一,意大利有足够的资本来补偿以商品形式输出到意大利的英国资本;第二,意大利用自己这笔资本生产出了英国资本部分地为补偿它自己和部分地为补偿它所带来的收入所需要的特殊物品,那就不会发生任何生产过剩。因此,实际地——对意大利的实际生产来说——存在着的英国的生产过剩这个事实就不存在了,存在的只是想象的意大利的生产不足这个事实;其所以说是想象的,是因为它以意大利存在着那里并不存在的[724]资本以及生产力的发展为前提,其次,是因为这里还作了同样空想的假定,就是这笔在意大利并不存在的资本用得恰好符合需要,使英国的供给和意大利的需求、英国的生产和意大利的生产能互相补充。换句话说,这无非是意味着:如果需求和供给彼此相符,如果资本按这样的比例在一切生产领域之间进行分配,以致一种物品的生产就包含着另一种物品的消费,因而也就包含着它自己的消费,那就不会发生生产过剩。如果不发生生产过剩,那生产过剩就不会发生。但是,因为在一定的条件下资本主义生产只能在某些领域无限制地自由发展,所以,如果资本主义生产必须在一切领域同时地、均匀地发展,那就根本不可能有任何资本主义生产。因为在上述某些领域生产过剩绝对存在,所以在没有[绝对的]生产过剩的那些领域,也就相对地存在着生产过剩。

总之,这种用一方面的生产不足来说明另一方面的生产过剩的观点无非是说:如果生产按比例进行,那就不会发生生产过剩。如果需求和供给彼此相符,也就不会发生生产过剩。如果一切领域具有进行并扩大资本主义生产的同样的可能性,如分工、机器、向遥远的市场输出、大规模生产等等,如果互相贸易的一切国家具有进行生产(而且是彼此各不相同又互为补充的生产)的同样的能力,也就不会有生产过剩。因此,如果发生生产过剩,那是因为所有这些虔诚的愿望没有实现。或者更抽象地说:如果到处都均匀地发生生产过剩,那就不会在一处发生生产过剩。但是,现在资本没有大到足以使生产过剩带有这样普遍的性质,因此就只会发生局部的生产过剩。

如果更仔细地进行考察,这个幻想可归结如下:

承认每一单个的生产部门都可能发生生产过剩。根据前面的解释,唯一能够防止在一切部门同时发生生产过剩的情况,是商品同商品的交换,就是说,抱这种观点的人求助于假定存在的是物物交换的条件。但是通向这种遁辞的道路恰好被切断了:商品流通不是物物交换,因此一种商品的卖者完全不必同时又是另一种商品的买者。可见,这整个遁辞的基础是撇开货币,撇开这里的问题不是产品交换而是商品流通这一点,而对于商品流通来说,买和卖的彼此分离具有重大的意义。

{资本流通本身包含着破坏的可能性。例如,在货币再转化为资本的生产条件时,问题不仅在于货币重新转化为同样的(按种类来说)使用价值,而且,为了使再生产过程重复进行,十分重要的是能够按原来的价值(或者按更低的价值,那当然更好)得到这些使用价值。但是,这些再生产要素有很大一部分,即由原料组成的部分,可能由于下述两个原因涨价:第一,如果生产工具的数量增加得比那段时间内能够生产出来的原料数量快。第二,由于收成的不稳定。因此,正如图克正确地指出过的\endnote{托·图克《价格和流通状况的历史》1838—1857年伦敦版第1—6卷。图克在这部六卷著作中的许多地方,特别是在1848年出版的第四卷开头,谈到气候条件对价格的影响。——第609页。},气候在现代工业中起着如此重大的作用。(这句话同样适用于与工资有关的食物。)因此,货币再转化为商品,完全同商品转化为货币一样,也可能遇到困难,也可能造成危机的可能性。如果考察的是简单流通而不是资本流通,那就不会发生这些困难。}(危机还有许多因素、条件、可能性,只有在分析更加具体的关系,特别是分析资本的竞争和信用时,才能加以考察。)

[725]否认商品的生产过剩,却承认资本的生产过剩。可是资本本身就是由商品组成的,或者说,如果资本由货币组成,它就必须再转化为这种或那种商品,才能执行资本的职能。因此,什么叫作资本的生产过剩呢?就是预定用来生产剩余价值的那些价值量的生产过剩(或者,从资本的物质内容方面来考察,就是预定用来进行再生产的那些商品的生产过剩),——因此,就是再生产的规模太大,这同直截了当说生产过剩是一个意思。

更加明确地说,资本的生产过剩无非是,为了发财而生产的东西过多了,或者说,不是用作收入进行消费,而是用来获得盈利、进行积累的那部分产品太多了;这部分产品不是用来满足它的所有者的私人需要,而是用来为它的所有者提供抽象的社会财富即货币,提供更大的支配别人劳动的权力——资本,或者说,扩大这个权力。这是一方的说法。(李嘉图否认这一点。\fnote{见本册第567页。——编者注})而另一方用什么来解释商品的生产过剩呢?就用生产不够多种多样,某些消费品生产得不够大量来解释。很清楚,这里不可能涉及生产消费的问题;因为工厂主生产过多的麻布,他对纱、机器和劳动等的需求必然因而增加。因此,这里涉及的是私人消费的问题。麻布生产得太多了,但是橙子也许就生产得太少了。起初否定货币,是为了说明买和卖的彼此分离[并不存在]。现在否定资本,是为了把资本家说成是完成W—G—W这个简单动作并且为了个人消费而进行生产的人,而不是把他看作以发财为目的、以把一部分剩余价值再转化为资本为目的而进行生产的资本家。但是,资本太多这句话无非是说,作为收入被消费并且在既定的条件下可能被消费的产品太少。(西斯蒙第\endnote{西斯蒙第用“生产和消费之间日益不成比例”来解释危机(让·沙·列·西蒙·德·西斯蒙第《政治经济学新原理》1827年巴黎第2版第1卷第371页)。恩格斯在《反杜林论》中指出,“用消费水平低来解释危机,起源于西斯蒙第,在他那里,这种解释还有一定的意义”(见《马克思恩格斯全集》中文版第20卷第311页),而在《资本论》第二卷序言中,恩格斯从西斯蒙第《新原理》中引了下面一段作为西斯蒙第观点的例证:“可见,由于财富集中在少数所有者手中,国内市场就越来越缩小,工业就越来越需要到国外市场去寻找销路,但是在那里,它会受到更大的变革的威胁”(见《马克思恩格斯全集》中文版第24卷第23页)。马克思在《剩余价值理论》第三册中又回过头来谈到西斯蒙第对危机的观点,既指出西斯蒙第的概念中有价值的因素,又指出它所固有的根本缺点(特别是见论马尔萨斯的一章,马克思手稿第775页)。——第610页。}。)那末,为什么麻布生产者要求谷物生产者消费更多的麻布,为什么谷物生产者要求麻布生产者消费更多的谷物呢?为什么麻布生产者自己不把他的更大一部分收入(剩余价值)实现在麻布上,而租地农场主自己不把他的更大一部分收入实现在谷物上呢?就每一个人单独来说,人们承认,他们的资本化的需要妨碍这样作(且不说每种需要都有一定的限度),但是就全体总起来说,人们就不承认这一点了。

(这里,我们完全撇开了由于商品的再生产比商品的生产便宜而产生的危机因素。而市场上的现有商品的贬值就是由此而来的。)

资产阶级生产的一切矛盾,在普遍的世界市场危机中集中地暴露出来,而在局部的(按内容和范围来说是局部的)危机中只是分散地、孤立地、片面地暴露出来。

至于专门谈到生产过剩,那它是以资本的一般生产规律为条件:按照生产力的发展程度(也就是按照用一定量资本剥削最大量劳动的可能性)进行生产,而不考虑市场的现有界限或有支付能力的需要的现有界限。而这是通过再生产和积累的不断扩大,因而也通过收入不断再转化为资本来进行的,另一方面,[726]广大生产者的需求却被限制在需要的平均水平,而且根据资本主义生产的性质,必须限制在需要的平均水平。

\tsectionnonum{[(15)李嘉图关于资本积累的各种方式和积累的经济效果的观点]}

李嘉图在第八章(《论赋税》)中说:

\begin{quote}{“如果一个国家的年生产能补偿它的年消费而有余,人们就说,这个国家的资本增加了;如果一个国家的年消费甚至不能由它的年生产来补偿,人们就说,这个国家的资本减少了。因此,资本可能由于增加生产或由于减少非生产消费而增加。”(第162—163页)}\end{quote}

李嘉图这里所说的“非生产消费”,和他在这段话的注释(第163页)中所说的一样,是指非生产劳动者即“不再生产另一个价值的人”的消费。因此,年生产的增加是指年生产消费的增加。年生产消费在非生产消费保持不变或者甚至有所增加时可以通过它自身的直接增加而增加,也可以通过减少非生产消费而增加。

\begin{quote}{同一个注释中说:“我们说收入节约下来加入资本,我们的意思是,加入资本的那部分收入,是由生产工人消费,而不是由非生产工人消费。”}\end{quote}

我已经指出\fnote{见本册第537—561页。——编者注},收入转化为资本决不等于收入转化为可变资本或者说收入用于工资。可是,李嘉图的想法却正是这样。

李嘉图在同一个注释中说:

\begin{quote}{“如果劳动价格大大提高,以致增加资本也无法使用更多的劳动,那我就要说,这样增加的资本仍然是非生产地消费的。”}\end{quote}

因此,不是单单收入由生产工人消费就使这种消费成为“生产的”消费,而是收入由生产剩余价值的工人消费才使这种消费成为“生产的”消费。按照这种看法,资本只有在支配[比原来]更多的劳动的时候才会增加。

李嘉图在第七章(《论对外贸易》)中写道:

\begin{quote}{“积累资本有两条道路:或者增加收入,或者减少消费,都可以积蓄资本。如果我的利润从1000镑增加到1200镑,而我的支出保持不变,我每年就比以前多积累200镑。如果我从我的支出中节约200镑,而我的利润仍旧不变,结果也是一样:我的资本每年将增加200镑。”(第135页)“如果由于采用机器,用收入购买的一切商品的价值下降20%,我就能够象我的收入增加20%那样有效地进行节约;但是在一种场合是利润率保持不变,在另一种场合是利润率提高了20%。如果从国外输入廉价商品使我能够从我的支出中节约20%,其结果将同机器降低了这些商品的生产费用完全一样,但是利润不会提高。”(第136页)}\end{quote}

(就是说,如果比较便宜的商品既不加入可变资本,又不加入不变资本,利润就不会提高。)

因此,如果收入的支出情况不变,积累就是利润率提高的结果{但是积累不仅取决于利润的高低,而且取决于利润的量};如果利润率不变,积累就是支出减少的结果,而李嘉图在这里认为,支出的减少是“用收入购买的商品”降价(或者由于采用机器,或者由于对外贸易)的结果。

在第二十章(《价值和财富,它们的特性》)中说:

\begin{quote}{“一个国家的财富〈李嘉图指的是使用价值〉可以用两种方法增加:它可能通过把更大的一部分收入用于维持生产劳动来增加,这不仅能增加商品总量的数量,而且能增加其价值;或者它也可能不通过使用追加的劳动量,而通过提高原来劳动量的生产率的方法来增加,这能增加商品的数量,但不能增加商品的价值。在第一种情况下,不仅一个国家的财富会增加,而且财富的价值也会增加。国家变富是由于节约,由于减少奢侈品和享乐品方面的支出,并且把这种节约所得用在再生产上。[727]在第二种情况下,不必减少奢侈品和享乐品方面的支出,也不必增加所使用的生产劳动量;但是用同量劳动将生产出更多的产品;财富将增长,但其价值不增加。在这两种增加财富的方法中,第二种方法应该是更可取的,因为它可以避免第一种方法必然带来的享乐品的缺乏和减少,而得到同样的结果。资本是一个国家为了未来生产而使用的那部分财富,它可以用增加财富的同样方法来增加。追加资本无论是由于技艺和机器的改进而得到的,还是由于把更大的一部分收入用于再生产而得到的,在生产未来财富时都有同样的效力;因为财富总是取决于生产出来的商品量,而与制造生产中所使用的工具的容易程度无关。一定量的衣服和食物将维持并雇用同样的人数,因而将保障同样工作量的完成,无论这些食物和衣服是由100人的劳动还是由200人的劳动生产出来的;但是在生产它们时如果用了200人,它们就会有加倍的价值。”(第327—328页)}\end{quote}

李嘉图对问题的第一个提法是:

在支出不变的情况下,如果利润率提高,积累就会增加,或者说,在利润率不变的情况下,如果支出(按价值)由于用收入购买的商品降价而减少,积累就会增加。

现在,李嘉图提出了另一个对立的提法:

如果有更大一部分收入从个人消费领域抽出,转入生产消费,如果用这样节约下来的那部分收入去推动更多的生产劳动,积累就会增加,资本按量和价值来说都能进行积累。在这种情况下,是靠节约来积累。

或者说,[用于个人消费的]支出保持不变,也不使用任何追加的生产劳动量,但是同样的劳动会生产出更多的产品,劳动的生产力提高了。花费同样的劳动会生产出更大量、更好、因而也更便宜的构成生产资本的要素即原料、机器等{李嘉图在前面说是用收入购买的商品,现在却说是作为生产工具使用的商品}。在这种情况下,积累既不取决于利润率有所提高,又不取决于由于节约有更大一部分收入变成资本,也不取决于由于用收入购买的商品降价而用于非生产目的的那部分收入减少。在这里,积累取决于提供资本本身要素的生产领域中劳动生产率提高,就是说,取决于作为原料、工具等加入生产过程的商品降价。

如果劳动生产力的增长,是由于同可变资本相比,固定资本的生产有所增加,那末,不仅再生产的量,而且再生产的价值都会增加,因为固定资本的价值有一部分是加入当年再生产的。这可能同人口的增加和所使用的工人人数的增加同时发生,虽然所使用的工人人数,同这些工人推动的不变资本相比,相对地说在不断减少。因此,不仅财富会增加,价值也会增加,并且,虽然劳动的生产率提高了,虽然同生产出来的商品量相比,劳动量减少了,被推动起来的活劳动量却大了。最后,即使劳动生产率保持不变,可变资本和不变资本也可能和每年人口的自然增长一起以同一程度增长。在这种情况下,资本不仅按量而且按价值,都能进行积累。最后这几点李嘉图完全没有注意到。

李嘉图在同一章里说:

\begin{quote}{“工业中100万人的劳动总是生产出相同的价值,但并非总是生产出相同的财富。}\end{quote}

(这是完全错误的,100万人的产品的价值,不仅取决于他们的劳动,而且取决于他们借以进行劳动的资本的价值;因此,这个价值将根据他们借以进行劳动的过去已经生产出来的生产力的大小不同而大不相同。)

\begin{quote}{机器的发明,技艺的提高,分工的改进,或者能够进行更有利的交换的新市场的发现,——这一切使100万人在一种社会状态下能够生产的‘必需品、舒适品和享乐品’等财富的量,比他们在另一种社会状态下所能生产的量大一倍或两倍。但是他们不能因此就使价值有所增加}\end{quote}

(肯定有所增加,因为他们过去的[728]劳动以大得多的规模加入新的再生产),

\begin{quote}{因为每一种物品的价值的提高或降低,都取决于生产这种物品的容易程度,换句话说,都取决于生产这种物品所花费的劳动量。}\end{quote}

(每一单位商品可能会跌价,但是增长了的商品总量的价值却会增加。)

\begin{quote}{我们假定,一定数量的人的劳动用一定的资本生产出1000双袜子,由于发明了机器,同样数量的人能生产2000双,或者他们除了继续生产1000双袜子以外,还能生产500顶帽子。那末,2000双袜子的价值,或者1000双袜子和500顶帽子的价值将不多不少恰好等于采用机器前1000双袜子的价值,因为它们将是同量劳动的产品。}\end{quote}

(注意:如果新采用的机器毫无所值的话。)

\begin{quote}{不过商品总量的价值还是会减少;因为,由于技术改良而增加了的产品量的价值,虽然将恰恰等于技术改良前生产的较小量产品的价值,但是这种变动对于在技术改良前已经制造出来而还没有消费掉的那部分商品也会发生影响。这些商品的价值将减少,因为它们必须全部降到在技术改良后的各种优越条件下生产出来的商品的价值水平,而且,虽然商品量增加了,财富增加了,享乐品的量增加了,但是社会所拥有的价值量将会减少。由于不断提高生产的容易程度,我们就不断减少某些以前已经生产出来的商品的价值,虽然我们用这同一方法不仅增加了国家的财富,而且还增加了未来生产的能力。”(第320—322页)}\end{quote}

李嘉图这里谈的是,生产力的日益发展会使在比较不利条件下生产出来的商品贬值,不论这些商品是仍然停留在市场上,还是作为资本正在生产过程中发挥作用。但是从这里决不能得出结论说,“商品总量的价值会减少”,尽管这个总量的某一部分的价值会减少。这种结果只有在两种情况下才会产生,第一,如果由于技术进步而新增加的机器和商品的价值,小于原有同类商品的价值已经贬值的部分;第二,如果我们不考虑下面这一点,即随着生产力的发展,生产领域也不断增加,因而为投资开辟了以前根本没有的新部门。生产在发展进程中不仅会变得更便宜,而且会变得更加多样化。

李嘉图在第九章(《原产品税》)中写道:

\begin{quote}{“反对原产品税的第三种意见是,认为提高工资和降低利润妨碍积累,其作用同土壤自然贫瘠一样;关于这种意见,我在本书的另一部分已试图证明:在支出上和在生产上,通过减少商品价值和通过提高利润率,都能够同样有效地进行节约。当我的利润从1000镑增加到1200镑,而价格不变的时候,我通过节约来增加资本的能力会增大,如果我的利润不变,而商品大大跌价,使我用800镑能够买到以前用1000镑才能买到的东西,我通过节约来增加资本的能力也会增大,但是在前一种情况下增大的程度比不上后一种情况大。”(第183—184页)}\end{quote}

即使纯收入按其价值量来说并不减少,产品(或者确切些说,在资本家和工人之间分配的那部分产品)的全部价值也可能减少。(按所占的比例来说,纯收入还可能增加。)关于这一点,在第三十二章(《马尔萨斯先生的地租观点》)中说:

\begin{quote}{“但是,马尔萨斯先生的全部论证是建立在这样一个不可靠的基础上:它假定,既然国家的总收入减少,纯收入也一定按同一比例减少。本书的目的之一就是要说明,必需品的实际价值每有降低,工资也就降低,而资本利润则提高;换句话说,在任何一定的年价值中,归工人阶级所得的份额会减少,而归用基金使用这个阶级的人所得的份额会增加。假定某工厂生产的商品价值为1000镑,这一价值在老板和他的工人之间分配,工人得800镑,老板得200镑。[729]如果这些商品的价值降到900镑,同时由于必需品降价在工资上节省了100镑,那末,老板的纯收入丝毫不会减少,因而,他支付同额税款将同价格下降前一样容易。”(第511—512页)}\end{quote}

在第五章(《论工资》)中说:

\begin{quote}{“虽然工资有符合于它的自然率的趋势,但是在一个不断进步的社会里,工资的市场率却可能在一个不定的时期内经常高于它的自然率,因为资本的增加给对劳动的新需求造成的刺激还没有发挥完它的作用,资本的再一次增加却又开始发挥同样的作用了。所以,如果资本的增加是逐渐的和经常的,对劳动的需求就会不断地刺激人口的增加。”(第88页)}\end{quote}

从资本主义观点出发,一切都是颠倒着表现出来的。工人人口量和劳动生产率程度既决定资本的再生产,又决定人口的再生产。这里却颠倒过来,表现为资本决定人口。

李嘉图在第九章(《原产品税》)中说:

\begin{quote}{“资本的积累自然在劳动的雇主之间引起日益加剧的竞争,因而引起劳动价格的提高。”(第178页)}\end{quote}

这取决于资本的不同组成部分在资本积累时以什么样的比例增加。资本可能进行积累,而对劳动的需求却可能绝对地或相对地减少。

既然,按照李嘉图的地租理论,随着资本的积累和人口的增加,由于必需品价值提高或者说农业生产率下降,利润率有下降的趋势,那末,积累就有阻碍积累的趋势,利润率下降的规律——因为随着工业的发展,农业生产率越来越降低——就象恶运一样降临到资产阶级生产的头上。相反,亚·斯密却欣赏利润率下降。在他看来,荷兰是一个模范的国家。亚·斯密指出,利润率下降,迫使除了最大的资本家以外的大多数资本家把他们的资本用到生产上去,而不是靠利息过活,因而对生产是一种刺激。在李嘉图的门徒的著作中,对这种致命趋势的恐惧具有悲喜剧的形式。

现在我们把李嘉图有关这个问题的几段话引在这里。

第五章(《论工资》):

\begin{quote}{“在不同的社会阶段,资本,或者说,使用劳动的资金的积累,速度有快有慢,它在所有情况下都必定取决于劳动生产力。一般说来,在存在着大量肥沃土地时,劳动生产力最大:在这种时期,积累往往进行得很快,以致工人的供给赶不上资本增加的速度。据计算,在有利条件下,人口在25年内可能增加一倍;但是在同样有利的条件下,一个国家的全部资本可能在更短的时期内增加一倍。在这种情况下,工资在整个时期内都会有上涨的趋势,因为对劳动的需求将比劳动的供给增加得更快。在从先进得多的国家引进技艺和知识的新殖民地,资本可能有比人口增加得更快的趋势;如果工人的不足不能从人口更多的国家得到补充,这种趋势就会大大提高劳动的价格。随着这些国家的人口增多、质量较坏的土地投入耕种,资本增加的趋势就会减弱;因为现有人口的需要满足之后剩下来的剩余产品,必然同生产的容易程度成比例,也就是说,从事生产的人数越少,剩余产品就越多。因此,在最有利的条件下,生产力虽然仍然有可能增长得比人口快,但是这种情况不会持续很久;因为土地的数量有限,质量不同,只要投在土地上的资本有所增加,就会使所得产品的比率下降,而人口繁殖力却始终不变。”(第92—93页)}\end{quote}

(最后这句话是牧师的发明。人口繁殖力会随着劳动生产力的减退而减退。)

这里,首先要指出,李嘉图承认“资本的积累……在所有情况下都必定取决于劳动生产力”,因此,第一性的是劳动,而不是资本。

其次,根据李嘉图的论断可以认为,早就有人居住的工业发达的国家,从事农业的人比殖民地多,而实际情况恰恰相反。为了生产同量的产品,[730]例如,英国使用的农业工人就比其他任何国家——不论新国家还是老国家——都少。固然,这里有较大一部分非农业人口间接参加农业生产。但是,即使这部分人口较多,其程度也远远赶不上比较不发达国家的直接农业人口超过比较发达国家农业人口的那种程度。即使我们假定,英国的谷物较贵,生产费用较大。使用的资本较多。加入农业生产的过去劳动较多,不过活劳动较少。但是,由于已有的生产基础,农业资本的再生产所花费的劳动量较少,虽然这笔资本的价值也是在产品中得到补偿。

第六章(《论利润》)。

首先还要说几句。我们已经看到,剩余价值不仅取决于剩余价值率,而且取决于同时雇用的工人人数,因而取决于可变资本量。

积累也不是直接决定于剩余价值率,而是决定于剩余价值对预付资本总额之比,即决定于利润率;并且,与其说决定于利润率,不如说决定于利润总量;我们已经看到,对社会总资本来说,利润总量和剩余价值总量是相同的,但对不同部门所使用的单个资本来说,利润总量却可能和各个不同部门所生产的剩余价值量大不相同。如果把资本积累全部加以考察,那末,利润就等于剩余价值,利润率就等于剩余价值/资本,或者更确切地说,等于按每100单位的资本计算的剩余价值。

如果利润率(百分率)既定,利润总量就取决于预付资本量;因此,既然积累决定于利润,积累也就取决于预付资本量。

如果资本总额既定,利润总量就取决于利润率的高低。

因此,利润率高的小资本可能比利润率低的较大资本提供的利润量大。

举例来说:

\todo{}

如果资本的乘数和利润率的除数相等,就是说,如果资本量增加的比例和利润率下降的比例相同,利润量总额不变。100的10%得10,2×100的(10/2)%或5%同样得10。换句话说:如果利润率下降的比例和资本积累(增加)的比例相同,利润量不变。

如果利润率下降快于资本增加,利润量总额就减少。500的10%得利润量50。但是,六倍资本额即6×500或3000的(10/10)%或1%只得30。

最后,如果资本增加快于利润率下降,那末,尽管利润率下降,利润量还会增加。例如,100在利润为10%时得利润量10。但是300(3×100)的4%(即利润率降了60%)得利润量12。

现在回过头来看看李嘉图的论点。

李嘉图在第六章(《论利润》)中写道:

\begin{quote}{“因此,利润有下降的自然趋势,因为随着社会的进步和财富的增长,为了生产必需的追加食物量,必须花费越来越多的劳动。利润的这种趋势,这种可以说是重力作用,幸而由于生产必需品所使用的机器的改良以及农业科学上的发现而时常受到抑制,这些改良和发现使我们能够减少一部分以前所需要的劳动量,[731]因而能降低工人生活必需品的价格。可是,必需品价格和工资的提高是有限度的;因为……一旦工资达到720镑,即等于租地农场主的全部收入,积累就一定停止,因为那时任何资本都不可能提供利润,对追加劳动也不可能有任何需求,因此,人口也将达到最高点。事实上,在这以前很久,很低的利润率就会使一切积累停止,一个国家的全部产品在支付了工人的工资以后,几乎都将属于土地所有者以及什一税和其他税的所得者。”(第120—121页)}\end{quote}

这是李嘉图观念中的资产阶级的“神的毁灭”,是世界的末日。

\begin{quote}{“远在这种价格水平成为持久的状况以前,积累的一切动机就会消失,因为任何人从事积累,都只是为了把他的积累生产地加以使用……因此,这种价格水平是决不可能存在的。正如工人没有工资就不能生活一样,租地农场主和工厂主没有利润也不能生活。他们的积累动机将随利润的每次减少而减少,当他们的利润低到不能对他们的辛劳和他们在把资本生产地加以使用时必然遇到的风险提供足够的补偿的时候,积累的动机将完全消失。我必须再次指出,利润率的降低……要迅速得多;因为如果产品的价值象我在前面假定的情况下说过的那样高,租地农场主的资本的价值就会大大增加,因为他的资本必然是由许多价值已经增加的商品组成的。在谷物价格可能从4镑上涨到12镑以前,租地农场主的资本的交换价值也许就已经增加一倍,等于6000镑而不是3000镑了。因此,如果租地农场主的利润原来是180镑,或者说,是他原有资本的6%,那末现在实际利润率不会高于3%,因为6000镑的3%是180镑,而且一个持有6000镑的新租地农场主要经营农业,就只有接受这种条件。”(第123—124页)“我们也可以预计到,虽然资本的利润率会因农业中资本的积累和工资的提高而降低,利润总额仍然会增加。例如,假定连续多次进行积累,每次为10万镑,而利润率从20%降到19%,18%,17%,就是说,不断下降,那末,我们可以预计到,先后相继的资本所有者得到的利润总额会不断增加:资本为20万镑时的利润总额会大于资本为10万镑时的利润总额,资本为30万镑时的利润总额还会更大些,依此类推,因此,即使利润率不断下降,利润总额也会随着资本的每次增加而增加。但是这样的级数只在一定时间内有效。例如,20万镑的19%大于10万镑的20%,30万镑的18%又大于20万镑的19%;但是当资本积累到了很大的数额,而利润率又下降的时候,进一步的积累就会使利润总额减少。例如,假定积累达到100万镑,利润率为7%,利润总额就是7万镑。如果现在100万镑再加上10万镑资本,而利润率降到6%,那末,虽然资本总额从100万镑增加到110万镑,资本所有者得到的将只是66000镑,或者说,少了4000镑。然而,只要资本多少能提供一些利润,就不会有既不增加产品,又不增加价值的资本积累。在使用10万镑追加资本时,原有资本的任何一部分的生产率都不会降低。国内土地和劳动的产品一定会增加,产品的价值也会增加,这不仅是由于加上了除原有产量外新增产品的价值,而且是由于生产最后一部分产品的困难加大使全部土地产品得到了新的价值。不过,当资本积累已经很大时,尽管产品的价值增加了,产品进行分配的结果将是,归利润的部分比以前减少,而归地租和工资的部分则增加。”(第124—126页)“虽然生产了一个较大的价值,但这一价值在支付地租以后剩下的部分中却有较大的份额是由生产者消费的,而这一点,并且只有这一点,却调节着利润的大小。在土地获得丰收时,工资可能暂时提高,生产者的消费可能超过他们通常的份额;但是因此而产生的对人口增加的刺激,很快就会使工人的消费回到通常的水平。但是当较坏土地投入耕种时,或者当花费在老地上的资本和劳动增加而收益减少时,上述影响将是持久的。”(第127页)[732]“因此,积累的效果在不同国家是不同的,并且主要取决于土地的肥力。一个国家无论多么辽阔,如果土地贫瘠并禁止粮食输入,那末,即使是较少量的资本积累也将引起利润率的大大降低和地租的迅速提高;相反,一个小的但是土地肥沃的国家,特别是如果它允许自由输入粮食,却能够积累很大的资本,而又不引起利润率的大大降低或地租的大量增加。”(第128—129页)税收(第十二章《土地税》中说)也可能造成这样一种情况,以致“剩下的剩余产品不够用来鼓励那些通常以自己的节约来增加国家的资本的人的努力”。(第206页)“只有一种情况{第二十一章《积累对于利润和利息的影响》中说}可能引起利润率在食物价格低廉时随着资本的积累而下降,那就是维持劳动的基金比人口增加快得多,这时工资高,而利润率却低;但这种情况也只具有暂时的性质。如果每个人都不使用奢侈品而专心致志于积累,那末生产出来的必需品就会有一定数量无法立即被消费。这为数有限的几种商品无疑会发生普遍过剩现象,因而对这些商品的追加量不会有需求,使用追加资本也不会提供利润。如果人们停止消费,他们就会停止生产。”(第343页)}\end{quote}

李嘉图关于积累和关于利润率下降规律的思想就是这样。

\tchapternonum{[第十八章]李嘉图的其他方面。约翰·巴顿}

\tsectionnonum{[A.]总收入和纯收入}

纯收入,与总收入(它等于总产品或总产品价值)相对立,是重农学派最初用来表达剩余价值的一种形式。他们认为地租是剩余价值的唯一形式,因为他们把工业利润仅仅理解为一种工资;重农学派对这一问题的看法必然会找到支持者,这就是后来那些把利润说成是对劳动进行监督而得的工资,从而把利润掩盖起来的经济学家。

这样,纯收入实际上就是产品(或产品价值)超过它补偿预付资本即不变资本和可变资本的那一部分的余额。因此,纯收入只不过是由利润和地租构成,而地租本身又只是从利润中分割出来、落入一个不同于资本家阶级的阶级手中的一部分利润。

资本主义生产的直接目的不是生产商品,而是生产剩余价值或利润(在其发展的形式上);不是产品,而是剩余产品。从这一观点出发,劳动本身只有在为资本创造利润或剩余产品的情况下才是生产的。如果工人不创造这种东西,他的劳动就是非生产的。因此,所使用的生产劳动量只是在剩余劳动量由于它——或者比例于它——而增长的情况下,才会使资本感到兴趣。我们称为必要劳动时间的东西,只有在这样的情况下才是必要的。如果劳动不产生这种结果,它就是多余的,就要被制止。

资本主义生产的始终不变的目的,是用最小限度的预付资本生产最大限度的剩余价值或剩余产品;在这种结果不是靠工人的过度劳动取得的情况下,这是资本的这样一种趋势:力图用尽可能少的花费——节约人力和费用——来生产一定的产品,也就是说,资本有一种节约的趋势,这种趋势教人类节约地花费自己的力量,用最少的资金来达到生产的目的。

从这种理解来看,工人本身就象他们在资本主义生产中表现的那样,只是生产资料,而不是目的本身,也不是生产的目的。

纯收入不决定于总产品价值,而决定于总产品价值超过预付资本价值的余额,或者说,决定于与总产品相比的剩余产品量。尽管[733]产品价值减少,或者甚至产品总量也随同价值一起减少,只要这个余额增加,资本主义生产的目的就达到了。

李嘉图彻底地、无情地道破了这种趋势。由此引起庸俗的慈善家们对他的一片叫骂。

李嘉图在考察纯收入时又犯了一个错误,即把总产品归结为收入,也就是归结为工资、利润和地租,而把应得到补偿的不变资本撇开不谈。但是我们在这里不准备详细谈这一点。

李嘉图在第三十二章(《马尔萨斯先生的地租观点》)中写道:

\begin{quote}{“明确地区别总收入和纯收入是很重要的,因为一切赋税都必须从社会纯收入中支付。假定一个国家在一年中能够向市场提供的全部商品即全部谷物、原产品、工业品等等价值为2000万,为取得这个价值需要一定人数的劳动,而这些工人起码的生活必需品要花费1000万。那我就会说,这个社会的总收入是2000万,纯收入是1000万。根据这一假定决不能得出结论说,工人得到的劳动报酬只能是1000万;他们可能得到1200万、1400万或1500万,在这种情况下他们就会从纯收入中得到200万、400万或500万。余下的就会在土地所有者和资本家之间分配,但是全部纯收入不会超过1000万。假定这个社会纳税200万,它的纯收入就会减到800万。”(第512—513页)}\end{quote}

[而在第二十六章(《论总收入和纯收入》)中我们读到:]

\begin{quote}{“如果一个国家无论使用多少劳动量,它的纯地租和纯利润加在一起始终是那么多,那末,使用大量生产劳动对于该国又有什么好处呢?每一个国家的全部土地产品和劳动产品都要分成三部分:其中一部分是工资,一部分是利润,另一部分是地租。”}\end{quote}

{这是错误的,因为这里忘记了用于补偿生产中所使用的资本(工资除外)的那一部分。}

\begin{quote}{“赋税或积蓄只能出自后两部分;第一部分,如果它是适中的,就始终是必要的生产费用。”}\end{quote}

{李嘉图本人在给这句话加的注释中指出:

\begin{quote}{“这种说法可能过分,因为在工资的名义下归工人所得的部分通常高于绝对必要的生产费用。在这种情况下,工人得到国家纯产品的一部分,他可以把这一部分积蓄起来或者消费掉;或者可以捐献出来供国防之用。”}“对于一个拥有2万镑资本,每年获得利润2000镑的人来说,只要他的利润不低于2000镑,不管他的资本是雇100个工人还是雇1000个工人,不管生产的商品是卖1万镑还是卖2万镑,都是一样的。一个国家的实际利益不也是这样吗?只要这个国家的实际纯收入,它的地租和利润不变,这个国家的人口有1000万还是有1200万,都是无关紧要的。一国维持海陆军以及各种非生产劳动的能力必须同它的纯收入成比例,而不同它的总收入成比例。如果500万人能够生产1000万人所必需的食物和衣着,那末500万人的食物和衣着便是纯收入。假如生产同量的纯收入需要700万人,也就是说,要用700万人来生产足够1200万人用的食物和衣着,那对于国家又有什么好处呢?纯收入仍然是500万人的食物和衣着。使用更多的人既不能使我们的陆海军增加一名士兵,也不能使赋税多收一个基尼。”(第416—417页)}\end{quote}

为了更好地弄清李嘉图的观点,我们还要引证下面几段作补充:

\begin{quote}{“谷物价格相对低廉总会带来好处,也就是说,根据这种价格,现有产品的分配更可能增加维持劳动的基金,因为在利润的名义下归生产阶级的部分将更多,而在地租的名义下归非生产阶级的部分将减少。”(第317页)}\end{quote}

这里的生产阶级只是指产业资本家。

\begin{quote}{“地租是价值的创造,但不是财富的创造。如果谷物的价格由于一部分谷物生产困难而从每夸特4镑提高到5镑,那末100万夸特的价值就不是400万镑而是500万镑……整个社会将拥有更大的价值,从这种意义上说,地租是价值的创造。但是这种价值是名义上的,因为它丝毫不增加社会的财富,也就是说,不增加社会的必需品、舒适品和享乐品。我们所拥有的商品量同以前一样,而不是更多,谷物也仍然和以前一样是100万夸特;但是每夸特价格从4镑提高到5镑的结果,却使谷物和商品的一部分价值从原来的所有者手里转到土地所有者手里。因此,地租是价值的创造,但不是财富的创造;它丝毫不增加国家的资源。”(第485—486页)}\end{quote}

[734]假定由于[生产较为容易或者由于]谷物的进口,谷物价格下跌,使地租减少100万。李嘉图说,资本家的货币收入就会因此增加,接着又说:

\begin{quote}{“但是人们也许会说,资本家的收入不会增加;从土地所有者的地租中扣下来的100万将作为追加工资支付给工人。即使这样……社会状况也会得到改善,人们将能够比以前容易负担同样的税款。这只是证明了一件更合乎愿望的事,即由于新的分配而使状况得到改善的主要是另一个阶级,而且是社会上最重要的一个阶级。这个阶级所能得到的900万[即由必要生存资料决定的工资]以外的全部数额,构成国家纯收入的一部分,要把它花费掉,就一定会增加国家的收入、福利或力量。所以这笔纯收入你可以任意分配。你可以给一个阶级多一些,给另一个阶级少一些,但是你不会因此减少纯收入的总额;现在用同量劳动仍将生产出更多的商品,虽然这些商品的货币价值总额将会减少。但是,国家的纯货币收入,即交纳赋税和取得享乐品的基金,将比以前能够更好地维持现有居民的生活,为他们提供奢侈品和享乐品,使他们能够负担任何一定数量的赋税。”(第515—516页)}\end{quote}

\tsectionnonum{[B.]机器[李嘉图和巴顿论机器对工人阶级状况的影响问题]}

\tsubsectionnonum{[(1)李嘉图的观点]}

\tsubsubsectionnonum{[(a)李嘉图关于机器排挤部分工人的最初猜测]}

李嘉图在第一章(《论价值》)第五节中写道:

\begin{quote}{“假定……有一台机器在某一工业部门中使用,一年能做100个工人的工作,而且只能持续使用一年。再假定,这台机器值5000镑,每年支付给100个工人的工资也是5000镑。显然在这种情况下,购买机器还是雇用工人,对工厂主来说都一样。但是,假定劳动价值提高了,结果100个工人一年的工资为5500镑。显然,这时工厂主就不会再犹豫:用5000镑购买机器来为他完成同量工作对他是有利的。但是,机器的价格会不会也提高呢?它会不会由于劳动价值提高也值5500镑呢?如果制造机器时没有使用资本,也无须对机器制造业者支付利润,那末机器价格就会提高。例如,如果一台机器是工资均为50镑的100个工人劳动一年的产品,因而它的价格是5000镑,那末,在工资提高到55镑的情况下,机器价格就是5500镑。但这是不可能的。必须假定雇用的工人不到100人,否则机器就不可能卖5000镑,因为从这5000镑中必须给雇用工人的资本支付利润。因此,假定只雇用85个工人,每人工资50镑,即一年支出4250镑,机器售价中除支付工人工资以外的750镑,就是机器制造业者的资本的利润。当工资提高10%时,机器制造业者就不得不使用425镑追加资本,因此支出的资本就不是4250镑,而是4675镑;如果他仍然把机器卖5000镑,他用这笔资本就只能得到325镑利润。但是一切工厂主和资本家的情况都是一样:工资的提高对他们所有的人都有影响。因此,如果机器制造业者由于工资提高而提高机器价格,那就会有异常大量的资本被用来制造这种机器,直到机器价格只能提供普通利润率为止。因此我们可以看到,机器价格并不会因工资提高而提高。但是,那个在工资普遍提高时能够使用机器而又不增加自己商品的生产费用的工厂主,如果仍然可以按照过去的价格出卖自己的商品,他的情况就特别有利;不过我们已经看到,他将不得不降低自己商品的价格,否则资本就会流入他的生产部门,直到他的利润降到一般水平为止。因此,从采用机器中得到好处的是公众:生产这些不会说话的因素所花的劳动,总是比被它们排挤的劳动少得多,即使它们具有相同的货币价值。”(第38—40页)\endnote{在1817年出版的李嘉图的《原理》第一版中已有这段话。——第629页。}}\end{quote}

这一点完全正确。这也是对那些认为受机器排挤的工人能在机器制造业本身找到工作的人的回答;其实,这些人的看法不过是属于制造机器的工场还完全建立在分工的基础上并且还没有使用机器来生产机器的那个时代的看法。

假定一个工人的年工资是50镑,100个工人的工资就等于5000镑。如果这100个工人为机器所代替,而机器同样值5000镑,那末这台机器就必然是不到100个工人的劳动产品。因为机器中除包含有酬劳动外,还包含无酬劳动,这种无酬劳动恰好构成机器制造厂主的利润。如果值5000镑的机器是100个工人的劳动产品,那末它就只包含有酬劳动了。如果利润率为10%,那末在这5000镑中,预付资本就约占4545镑,利润约占455镑。如果一个工人的工资为50镑,那末4545镑就只代表90+(9/10)个工人。

[735]但是4545镑资本决不只代表可变资本(直接花费在工资上的资本)。它还代表机器制造业者所使用的固定资本的损耗和原料。因此,一台值5000镑、代替100个工人(这些工人的工资为5000镑)的机器不是90工人的产品,而是数量少得多的工人的产品。所以,只有在用于生产机器{至少是其中每年连利息一起加入产品即加入产品价值的那一部分}的工人人数(以年计算)比机器所代替的工人人数少得多的情况下,使用机器才是有利的。

工资的任何提高,都会使必须预付的可变资本增加,尽管产品价值——在它等于可变资本加剩余劳动的限度内——仍然不变(因为可变资本所推动的工人人数不变)。

\tsubsubsectionnonum{[(b)李嘉图论生产的改进对商品价值的影响。关于工资基金游离出来用于被解雇的工人的错误论点]}

李嘉图在第二十章(《价值和财富,它们的特性》)中指出,自然因素没有给商品价值增加什么,相反,它使商品价值减少。它恰恰因此使资本家唯一关心的剩余价值增加。

\begin{quote}{“同亚当·斯密的意见相反,萨伊先生在第四章中谈到了太阳、空气、气压等自然因素赋予商品的价值,这些自然因素有时代替人的劳动,有时在生产中给人以帮助。但是这些自然因素,尽管能够大大增加使用价值,却从来不会给商品增加萨伊先生所说的交换价值。只要我们借助于机器或自然科学知识使自然因素来完成以前由人完成的工作,这种工作的交换价值就会相应地降低。”(第335—336页)}\end{quote}

机器具有价值。自然因素本身没有什么价值。因此,它不可能给产品增加任何价值,而且相反,只要它能代替资本或劳动,不论是直接劳动还是积累劳动,它就会使产品的价值减少。只要自然科学教人以自然因素来代替人的劳动,而不用机器或者只用以前那些机器(例如利用蒸汽锅炉,利用许多化学过程等等,也许比以前还便宜),它就可以使资本家(以及社会)不费分文,而使商品绝对降价。

在以上引文之后,李嘉图接着说:

\begin{quote}{“如果原先用十个人推动磨面机,后来发现借用风力或水力可以节省这十个人的劳动,那末面粉(一部分是磨面机的工作产品)的价值就会立即按节约的劳动量相应地下降;并且社会会由于这十个人的劳动所能生产的商品而变得富些,同时预定用于维持这十个人的生活的基金并无任何减少。”(第336页)}\end{quote}

社会首先会由于面粉价格下降而变得富些。社会可以消费更多的面粉,也可以把以前预定用在面粉上的钱用在另一种商品上,这另一种商品或者是已经存在的,或者是因新的消费基金游离出来才出现的。

关于这部分以前用在面粉上、现在由于面粉价格下降而游离出来另作他用的收入,可以说,它原来由社会整个经济“预定用在”一定的物品上,现在则离开了这种“预定的用途”。这就好象积累了新资本一样。使用机器和自然因素就是用这种方法把资本游离出来,并使以前“潜在的需要”有可能得到满足。

相反,关于“预定用于维持”这十个由于新发现而失去了工作的人的“生活的基金”的说法是错误的。因为第一种基金是由新发现节约下来或创造出来的,它是社会以前用在面粉上、现在因面粉价格下降而节约下来的那一部分收入。而节约下来的第二种基金是磨坊主以前支付给十个现已解雇的工人的。这个“基金”,正如李嘉图所说的那样,的确并未因新发现和解雇十个工人而有任何减少。但是这个基金和这十个工人绝对没有任何自然的联系。他们可能成为贫民,饿死等等。只有一点是肯定无疑的,那就是本来应该接替这十个工人来磨面的下一代的十个人,现在必须到其他行业找工作,这样人口[和对劳动的需求相比]就相对地增加了(不管人口的平均增长如何),因为磨面机现在[不用人力]转动了,而这十个工人,假如没有这种发现,本来要去推动磨面机,现在则被雇去生产另一种商品了。所以,机器的发明和自然因素的利用使资本和人(工人)游离出来,创造了游离出来的资本,同时也创造了游离出来的人手(斯图亚特所说的“自由人手”\endnote{詹·斯图亚特《政治经济学原理研究》1770年都柏林版第一卷第396页。马克思在他的1857—1858年经济学手稿中引用了这一处(见卡·马克思《政治经济学批判大纲》)1939年莫斯科德文版第666页)。并参看本卷第1册第22页和马克思《资本论》第3卷第47章第1节。——第632页。}),这就有可能[736]创立新的生产领域,或者扩大旧的生产领域,扩大它们的生产规模。

磨坊主将用他的游离出来的资本建立新的磨坊,或者将这笔资本贷给别人,如果他自己不能将它作为资本花费掉的话。

但是在所有情况下这里根本没有什么“预定用于”十个被解雇工人的基金。我们还要回过头来谈\fnote{见本册第635—643页。——编者注}摆在我们面前的这个荒谬的前提,即如果采用机器(或者利用自然因素)不减少可以用作工资的生活资料的量(比如在农业上,用马代替人,或用畜牧业代替谷物业时,这种情况就部分地出现过),那末用上述方法游离出来的基金就必然要作为可变资本花掉(好象生活资料不可能出口,不可能用在非生产劳动者身上,或者在某些生产领域工资不可能提高等等),并且必然要恰恰用在被解雇的工人身上。机器经常不断地造成相对的人口过剩,造成工人后备军,这就大大增加了资本的权力。

在第335页的一个注中,李嘉图还反驳萨伊说:

\begin{quote}{“认为财富就在于有丰富的生活必需品、舒适品和享乐品的亚·斯密,虽然会承认机器和自然因素能大大增加一国的财富,但是不会承认它们能给这种财富的价值增加什么东西。”}\end{quote}

如果没有那些使地租能够形成的条件,自然因素确实不会给价值增加什么东西。但是机器总是会把它自己的价值加到已有的价值中去;第一,既然机器的存在便于[一部分]流动资本不断转化为固定资本,并使这一转化能在日益扩大的规模上进行,所以机器就不仅会增加财富,而且会增加由过去劳动加到年劳动产品上的价值;第二,因为机器的存在使人口有绝对增长的可能,从而使年劳动量也随之增长,所以机器通过这第二种方式也会增加年产品的价值。[736]

\tsubsubsectionnonum{[(c)李嘉图改正他对机器问题的看法表现了他在科学上的诚实。李嘉图对问题的新提法中仍保留了以前的错误前提]}

[736]第三十一章《论机器》。

李嘉图在他的著作第三版中新加的这一章,证明了他的诚实,这使他和庸俗经济学家有了本质的区别。

\begin{quote}{“我对这一问题{即“机器对社会各不同阶级的利益的影响”问题}的看法,在进一步思考后有了相当大的改变,所以我更加认为有责任加以说明。虽然我想不起在机器问题上我发表过什么须要收回的东西,但是我曾用其他方式{作为议员?}\endnote{李嘉图在这里很可能是指1819年12月16日他在英国下院就威廉·德·克雷斯皮尼的提案所作的发言;克雷斯皮尼提议成立一个委员会来研究罗伯特·欧文提出的消灭失业和改善工人阶级状况的计划。李嘉图在发言中断定“机器不会减少对劳动的需求”(皮·斯拉法编《大卫·李嘉图著作和通讯集》1952年剑桥版第5卷第30页)。——第633页。}支持过我现在认为是错误的学说,所以我认为有责任把我现在的看法及其理由提出来供读者研究。自从我开始注意政治经济学问题以来,我一直认为,在任何生产部门内采用机器,只要能节省劳动,就对大家都有好处,而唯一的不方便,在大多数情况下都是由于资本和劳动要从一个部门转移到另一个部门而引起的。”}\end{quote}

{这种“不方便”,如果象在现代生产中那样经常不断地发生,那末它对工人来说就够大的了。}

\begin{quote}{“在我看来,土地所有者所得的货币地租如果不变,他们用这种地租购买的某些商品价格的下跌将会使他们得到好处,而价格的下跌必然是采用机器的结果。我认为,资本家最后也会以完全相同的方式得到好处。不错,发明机器或首先使用机器的人会由于暂时获得很大的利润而得到额外的好处;但是随着机器的普遍采用,机器生产的商品的价格就会由于竞争而降到它的生产费用的水平,这时资本家所得到的货币利润就会和以前一样,他也只能[737]作为消费者分享一般的好处,因为他用同样的货币收入可以支配更多的舒适品和享乐品。我认为,工人阶级由于采用机器也会同样得到好处,因为工人用同样的货币工资可以购买更多的商品。同时我认为,工资不会缩减,因为资本家所能提出的对劳动的需求以及所能使用的劳动量仍然和以前一样,虽然他也许不得不使用这个劳动量来生产某种新的商品,或者至少是生产别的商品。如果由于机器的改良,使用同量劳动生产的袜子可以增加到四倍,而对袜子的需求只增加一倍,织袜业中的一些工人就必然会被解雇;但是由于雇用这些工人的资本仍然存在,而且由于资本的所有者把资本用在生产上是有利的,我认为这种资本将被用于生产其他某种对社会有用而社会对它也肯定有需求的商品……因此,由于我认为对劳动的需求仍然和以前一样,而工资又不会降低,我认为工人阶级将由于使用机器后商品普遍跌价而和其他阶级同样受益。这就是我原来的看法,涉及土地所有者和资本家的地方,我的看法依然不变;但我现在深信,用机器来代替人的劳动,对于工人阶级往往是非常有害的。”(第466—468页)}\end{quote}

首先必须指出,李嘉图是从下述错误的前提出发的:好象机器总是在资本主义生产方式已经存在的生产领域被采用。可是大家知道,机器织机最初是代替手工织工,珍妮机是代替手工纺工,而割草机、脱粒机、播种机也许是代替独立的小农等等。在这里不仅劳动者受到排挤,而且他的生产工具也不再是(李嘉图意义上的)资本。当机器在以前的仅以分工为基础的工场手工业中引起革命时,就出现了旧的资本的这种完全的或彻底的贬值。在这里,说“旧的资本”对劳动的需求仍然和以前一样,是荒谬的。

手工织工、手工纺工等使用的“资本”已经不是“仍然存在”了。

但是,为了使研究简便起见,我们假定只是在资本主义生产(工场手工业)已经占统治地位的领域,或者甚至在已经以机器生产为基础的工场中才采用机器{当然,我们这里就不谈在新的生产部门采用机器了},这样,问题就是提高机器的自动性,或者采用改良的机器,这种机器使得有可能或者解雇一部分现在雇用的工人,或者使用和以前一样多的工人,但是他们能提供更多的产品。当然,这后一种情况是最有利的。

为了减少混乱,必须把下面两种东西分开:(1)采用机器并解雇工人的资本家的基金;(2)社会的基金,即这个资本家的商品的消费者的基金。

第一点。关于采用机器的资本家,说他可以把和以前一样多的资本花费在工资上,这是错误的、荒谬的。(即使在他求助于借款的时候,如果不是就他本人而是就整个社会来说,这种说法同样是错误的。)他把自己资本的一部分转化为机器和别的固定资本,把另一部分转化为他以前并不需要的辅助材料,在我们假定他用较少的工人生产较多的商品,从而需要更大量的原料时,他还把比以前更多的一部分资本转化为原料。可变资本即花费在工资上的资本对不变资本的比例,在他的生产部门缩小了。即使这个资本家的企业扩大到新的生产水平,以致他又能给全部被解雇的工人或者甚至比以前更多的工人以工作,这个缩小了的比例也仍然有效(由于劳动生产力随着积累而发展,同不变资本相对来说,可变资本减少的幅度甚至还会增大)。{他的企业对劳动的需求将随同他的资本的积累一起增长,但与资本积累相比,程度要小得多,而他的资本就其绝对量来说已不再是以前那样的对劳动的需求的源泉了。由此产生的直接后果就是一部分工人被抛向街头。}

但是人们会说,对工人的需求会间接地保持不变,因为机器制造业对工人的需求会增加。可是李嘉图自己早已指出\fnote{见本册第628—629页。——编者注},机器所费的劳动量决没有它所代替的那样多。可能在制造机器的工场中工作日会暂时延长,[738]因此那里最初不会多雇一个工人。原料(比如说棉花)可能来自美国和中国,而对美国黑奴或中国苦力的需求是否增加,对于被抛向街头的英国工人来说是完全无关紧要的。即使假定原料在国内生产,那时在农业中将雇用更多的妇女和儿童,将使用更多的马匹等等,也许将生产更多的这种产品和更少的其他产品。但是对被解雇的产业工人的需求在这里不会产生,因为在这里,在农业中,造成经常的相对人口过剩的同一过程也在发生。

认为采用机器能使工厂主的资本在最初投入企业时就游离出来,这种说法一看就知道是难以置信的。采用机器只是使他的资本投入别的部门,根据这个前提,其直接后果就是解雇工人,把一部分可变资本转化为不变资本。

第二点。至于社会公众,那末由于用机器生产的商品跌价而游离出来的首先是他们的收入;资本只有在用机器生产的物品作为生产要素加入不变资本的限度内才会直接游离出来。{如果这种物品加入工人的一般消费,那末根据李嘉图自己的看法,这也一定会引起其他生产部门的实际工资\endnote{关于李嘉图的“实际工资”(《realwages》)的概念,见本册第456—457、459—460、474、482和497页。——第636页。}下降。}游离出来的一部分收入将消费在这种物品上,这或者是因为这种物品的跌价使得新的消费者阶层能够享用它(但是在这种场合用在这种物品上的收入不是游离出来的收入),或者是因为原先的消费者现在要消费更多的已减价的物品,例如现在消费四双线袜而不是一双。这样游离出来的收入的另一部分可以用来扩大那些采用机器的生产部门,或者建立生产别种商品的新部门,最后,或者用来扩大某个早已存在的生产部门。不管怎样,这样游离出来并再转化为资本的收入,未必能够吸收每年重新流入各生产部门而现在首先被旧生产部门排斥在外的那部分增加的人口。但是,也可能有一部分游离出来的收入将和外国的产品相交换或由非生产劳动者消费掉。无论如何,在游离出来的收入和从收入游离出来的工人之间没有任何必然的联系。

第三。然而作为李嘉图论据的基础的是下面的荒谬看法。

[正如我们所看到的,]采用机器的工厂主的资本不会游离出来。这种资本只是另作他用,也就是说,这时它不会象以前那样转化为现已被解雇的工人的工资。它的一部分从可变资本转化为不变资本。即使它有一部分游离出来,那也将被这样的生产领域所吸收,在这些生产领域中,被解雇的工人不可能有工作,那里最多只能给本来应该接替他们的人提供一个收容所。

而游离出来的收入(只要它游离出来而不被减价物品消费的增加所抵销,或者只要它不和来自国外的生活资料相交换),靠旧生产部门的扩大或新生产部门的建立,也只是为每年流来的、首先被采用机器的旧生产部门排斥在外的那部分增加的人口提供必要的机会(如果游离出来的收入真这样做的话!)。

但是,以隐蔽的形式构成李嘉图论据基础的那种荒谬看法恰恰在于:

现已被解雇的工人以前消费的那些生活资料依然存在,并且照旧存在于市场上。另一方面,这些人手也存在于市场上。因此,一方面存在着有可能成为可变资本的工人的生活资料(也就是支付手段),另一方面存在着失业工人。这样,也就有了用来推动这些失业工人的基金。因此,他们也就能够找到工作。

甚至象李嘉图这样的经济学家居然也会说出这种令人毛发悚然的荒唐言论!

照这样说来,在资产阶级社会内,凡是有工作能力并且愿意工作的人,当市场上、社会上有生活资料可以作为某种工作的报酬支付给他的时候,就都不会挨饿了。

首先必须指出,这些生活资料决不是作为资本而和被解雇的工人对立的。

我们假定,由于采用机器,有10万工人突然被抛向街头。那末,首先毫无疑问的是,[739]存在于市场上的、以前足够这些工人平均消费一年的农产品将照旧存在于市场上。如果对这些农产品没有需求,同时又不能将它们输出国外,那会产生什么结果呢?因为和需求相比,供给相对增加,所以这些产品就会跌价,即使被解雇的10万工人饿死,对产品的消费也会因跌价而增加。甚至食品用不着跌价,因为食品的进口可能减少或者出口可能增加。

李嘉图从幻想出发,以为资产阶级社会的整个结构非常精巧,比如说,如果有10个工人被解雇,那末他们那些现在已游离出来的生活资料必定还是被这10个工人这样或那样地消费掉,否则就根本不可能找到销路,——好象在这个社会的底层不存在忙于到处寻找工作的失业或半失业的群众,好象以生活资料形式存在的资本是一个固定的量。

如果谷物的市场价格由于需求的减少而下跌,那末以谷物形式存在的资本(在其货币表现上)就会减少,并且只要谷物不出口,它就会同社会的较少一部分货币收入相交换。对于以工业品形式存在的资本说来,就更是如此。在手工织工渐渐饿死的那些年代,英国的棉织品的生产和出口都大大增加了。同时(1838—1841年)食品价格上涨了。而这些织工既没有一件完整的、可以蔽体的衣服,也没有可以维持生命的食物。人为地不断制造出来的、只有在热病似的繁荣时期才能被吸收的过剩人口,是现代工业生产的必要条件之一。没有什么东西能阻止这样一些现象发生:一部分货币资本闲置不用,生活资料由于相对生产过剩而跌价,而被机器排挤的工人却活活饿死。

当然,游离出来的劳动和游离出来的一部分收入或资本,最终一定会在某一新的生产部门或在旧的生产部门扩大时找到出路,但这更多的是给那些本来应该接替被排挤的工人的人,而不是给被排挤的工人本身带来好处。收入直接花费在其中的、多少是非生产劳动部门的一些新部门不断产生。此外还有:固定资本(铁路等)的形成和由此产生的监督工作;奢侈品等的生产;使花费收入的物品种类越来越多的对外贸易。

李嘉图从他的荒谬观点出发,所以假定机器的采用只有在它减少总产品(从而减少总收入)的时候才对工人有害。当然,这种情况在大农业中,当那里使用马匹代替工人消费谷物或把谷物业变为养羊业等等的时候,是可能的。但把这种情况推广到本来意义的工业上,那就十分荒谬了,因为工业总产品的市场绝不限于国内市场的范围。(而且在一部分工人濒于饿死的时候,另一部分工人可能吃得好些,穿得好些;同样,非生产劳动者与介于工人和资本家之间的中间阶层也可能吃得好些,穿得好些。)

认为加入收入的物品的增加量(或一般量)本身就是为工人提供的基金,或者说,构成支付给工人的资本,这种说法本身就是错误的。这些物品的一部分为非生产劳动者或根本不劳动的人所消费。另一部分可能通过对外贸易从它用作工资的形式(从它粗糙的形式)转化为加入富人的收入或用作不变资本的生产要素的形式。最后,还有一部分由那些在习艺所或监狱中的被解雇的工人本身当作施舍物、赃物或他们的女儿卖淫的报酬来消费。

下面我将把李嘉图借以发挥谬论的一些论点概括一下。如他自己所说,他的这一谬论是从巴顿的著作中得到启发的,所以,在引用了李嘉图的著作之后,还必须简略地考察一下巴顿的著作。

[740]不言而喻,为了每年雇用一定数量的工人,每年必须生产一定数量的食品和其他必需品。在大农业、畜牧业等方面可能有以下这样的情况,即纯收入(利润和地租)增加,而总收入,用来维持工人生活的必需品的总量却减少。但是问题不在这里。加入消费的物品总量,或者用李嘉图的话说,加入总收入的物品总量,可能增加,而这一总量中转化为可变资本的那一部分却不会因此而增加。这个部分甚至可能减少。那时作为收入而由资本家、土地所有者、他们的奴仆、非生产阶级、国家、中间阶级(商人)等消费的将更多。

李嘉图(和巴顿)的论断的隐蔽基础是:他原来是从这样一个假设出发,即任何资本积累都是可变资本的增加,因而对劳动的需求将直接和资本的积累按同一比例增加。这是不正确的,因为随着资本的积累,资本有机构成会发生变化,资本的不变部分会比它的可变部分增长得更快。但是这并不妨碍收入在价值和数量方面不断增加。然而收入的这种增加并不会使总产品的相应增加部分花费在工资上。不直接靠自己劳动生活的阶级和阶层人数将会增加,他们的生活会比以前更好,而非生产劳动者人数同样会增加。

把一部分可变资本转化为机器的资本家(因而他在原料构成产品价值要素之一的所有生产领域中,同他所使用的劳动量相比,也一定会把他的资本的一个更大的份额用在原料上)的收入与我们研究的问题没有直接关系,所以我们撇开不谈。他的实际上已进入生产过程的资本以及他的收入,起初是以他自己生产的产品的形式,或者更确切地说,商品的形式(比如说,他是一个纺纱厂主,就是以棉纱的形式)存在的。在采用机器之后,他会把这些商品的一部分(或者把他出卖这些商品所得的货币的一部分)转化为机器、辅助材料和原料,而不会象以前那样,把这部分货币作为工资支付给工人,也就是说,间接地把它转化为工人的生活资料。除了农业上的少数例外,资本家将会比以前更多地生产这种商品,虽然被他解雇的工人已经不再象以前那样是他自己的产品的消费者,也就是购买者了。现在市场上有更多的这种商品,虽然这些商品已经不再为被抛向街头的工人而存在,或者说,已经不再象以前那样多地为他们而存在。因此,首先就这个资本家自己生产的产品来说,即使在这种产品加入工人消费的情况下,这种产品的一部分不再作为资本为工人而存在这件事与产品数量的增加也毫不矛盾。相反,总产品的一个更大的部分现在必须用来补偿转化为机器、辅助材料和原料的那部分不变资本,也就是说,总产品的一个更大的部分现在必须和数量比以前更多的这些再生产要素相交换。如果因采用机器而引起的商品量的增加与以前存在的对用这些机器生产的商品的需求(也就是被解雇工人方面的需求)的减少相矛盾,那末在大多数情况下就根本不可能采用机器了。所以,如果我们考察的是这样的资本,即它的一部分现在不是再转化为雇佣劳动,而是再转化为机器,那末,生产的商品量和这些商品中再转化为工资的份额之间首先就没有任何确定的关系或必然的联系。

至于整个社会,它的收入的补充,或者更确切地说,收入范围的扩大,首先是在那些由于采用机器而降价的物品方面发生的。这种收入可能仍然作为收入来花费,如果其中有相当大一部分转化为资本,那末除了人为地造成的人口过剩外,也总是已经存在着自然增长的人口,他们能把转化为可变资本的那部分收入吸收掉。

因此,初看起来就只剩下这样一点:所有其他物品的生产,尤其是在生产加入工人消费的物品的那些领域,尽管解雇了比如说100个工人,还是会按照以前的规模进行;毫无疑问,在这些工人被解雇时就是这种情况。因此,就被解雇的工人对上述物品有过需求来说,这种需求减少了,虽然供给仍旧不变。由此可见,如果需求的这种减少得不到弥补,相应的商品的价格就会下降(或者价格不下降,而是有更多的商品在市场上作为存货保留到下一年)。如果这种商品同时又不是出口货,如果对它的需求仍然低于以前的水平,那末这种物品的再生产就要减少,但是[741]用于这一领域的资本却不一定要减少。可能将生产更多的肉类或者更多的经济作物或者高级食品和更少的小麦,或者生产更多的饲养马匹等等用的燕麦,或者生产更少的绒布短上衣和更多的资产阶级用的常礼服等等。但是,如果由于例如棉织品减价,在业工人可以在食物等方面多花费一点,那就根本不会有产生上述任何后果的必然性。可能生产和以前一样多的,甚至更多的商品(其中包括加入工人消费的商品),尽管现在转化为可变资本即花费在工资上的是较少的资本,是总产品中一个更小的部分。

这里也不会产生这种情况,即对这些商品的生产者来说,他们的资本有一部分会游离出来。在最坏的情况下,对他们的商品的需求将减少,因而在他们的商品跌价的时候他们的资本的再生产将遇到困难。因此,他们自己的收入会立即减少,正如每当商品跌价时都会发生这种情况一样。但是不能说,他们的商品中的某一部分以前是作为资本和被解雇的工人对立的,现在则和这些工人一起“游离出来”。作为资本和工人对立的是现在用机器生产的那一部分商品;这部分商品以货币的形式流到他们手里,被他们用来和别的商品(生活资料)进行交换,这些别的商品不是作为资本和工人发生关系,而是仅仅作为商品和他们的货币对立的。因此,这是一种完全不同的关系。工人用自己的工资购买租地农场主或其他某一资本家的商品,这个租地农场主或资本家不是以资本家的身分和工人相对立,也不是把他们当作工人在生产中使用。现在他们不过不再是他的购买者了,如果没有其他情况来弥补的话,这就可能使他的资本暂时贬值,但是不会有任何资本游离出来用于雇用被解雇的工人。曾经使用他们来进行生产的那笔资本“仍然存在”,但已经不是以资本转化为工资(或者只是间接地在更小的程度上转化为工资)的形式存在了。

不然的话,任何因遭遇某种不幸而挣不到钱的人,都会因此而使一笔能给他自己提供工作的资本游离出来。

\tsubsubsectionnonum{[(d)李嘉图对采用机器给工人阶级带来某些后果的正确判断。在李嘉图对问题的说明中存在的辩护论观点]}

李嘉图认为总收入就是补偿工资和剩余价值(利润和地租)的那一部分产品;他认为纯收入就是剩余产品,剩余价值。李嘉图在这里就象在他自己的全部经济理论中一样,忘记了总产品中有一部分应该补偿机器和原料的价值,简单地说,就是补偿不变资本的价值。

\centerbox{※     ※     ※}

下面列举的李嘉图的一些论断之所以引人注意,部分是由于一些顺便提及的意见,部分是因为它们经过适当的修改,对于大农业,尤其对于养羊业,在实践上是重要的。所以这里又显露出了资本主义生产的界限。不但资本主义生产的决定性的目的不是为生产者(工人)而生产,而且它的唯一的目的就是纯收入(利润和地租),即使这个目的是靠减少产量,减少商品的生产量来达到。

\begin{quote}{“我的错误之所以产生,是由于假定每当社会的纯收入增加时,其总收入也一定增加。但是现在我有一切理由确信,土地所有者和资本家从中取得收入的那种基金可能增加,同时另一种基金即工人阶级主要依靠的那种基金却可能减少。因此,如果我没有错的话,那就可以得出结论说,使国家的纯收入增加的原因,同时也可能造成人口过剩,使工人状况恶化。”(第469页)}\end{quote}

这里首先要指出:李嘉图在这里承认,使资本家和土地所有者财富增加的那些原因“……可能造成人口过剩”,所以人口过剩,或者说,过剩的人口,在这里表现为致富过程本身和作为它的先决条件的生产力发展的结果。

至于说到资本家和土地所有者从中取得收入的基金,以及另一方面工人从中取得收入的基金,那末总产品首先就是这个总的基金。加入资本家和土地所有者消费的相当大一部分产品不会加入工人的消费。可是另一方面,几乎所有加入工人消费的产品,——实际上是所有产品,不过是多少程度不同罢了,——也都加入土地所有者和资本家的消费,其中也包括他们的奴仆、食客、猫狗的消费。不能认为,在这里似乎是彼此孤立地存在着两种性质不同的具有固定的量的基金。重要的是每一方从这个总的基金中获得多大的份额。资本主义生产的目的在于用一定量的财富得到尽可能多的剩余产品或剩余价值。达到这一目的的方法是:使不变资本相对地比可变资本增加得快些,也就是说,以尽量少的可变资本来推动尽量多的[742]不变资本。因此,从比李嘉图在这里讲的更普遍得多的意义上来说,同一个原因,通过工人从中取得收入的基金的减少,会促使资本家和土地所有者从中取得收入的基金增加。

由此不应得出结论说,工人从中取得收入的基金会绝对地减少。这种基金同他们所生产的总产品相比,只是相对地减少。而这对于决定他们从他们自己创造的财富中获得多大的份额来说,是唯一重要的。

\begin{quote}{“假定有一个资本家使用一笔价值20000镑的资本,他是租地农场主,同时也是生产必需品的工厂主。再假定这笔资本中有7000镑投在固定资本上,即投在建筑物、劳动工具等等上,其余的13000镑作为流动资本用来维持劳动。再假定利润为10%,因而这个资本家的资本每年都能保持原有的效率,并提供2000镑的利润。这个资本家每年开始营业时拥有价值13000镑的食品和其他必需品。在一年内,他按照这个货币额把这些食品和必需品全部卖给自己的工人;在同一时期内,他又把同额货币作为工资支付给工人。年终,工人补偿给他价值15000镑的食品和其他必需品,其中2000镑他自己消费或由他按自己最喜欢和最乐意的方式处理。”}\end{quote}

{在这里剩余价值的性质就表现得很明显。这段话在李嘉图的著作第469—470页上。}

\begin{quote}{“就这些产品而言,这一年的总产品是15000镑,纯产品是2000镑。现在假定下一年资本家用一半工人制造机器,另一半照旧生产食品和其他必需品。在这一年内他会照常付出工资13000镑,并将同一金额的食品和其他必需品卖给他的工人。但是下一年的情况又会怎样呢?制造机器时,食品和其他必需品的产量只有平常的一半,它们的价值也仅仅等于以前的一半。机器值7500镑,食品和其他必需品也值7500镑,所以这个资本家的资本还是和以前一样大;因为在这两个价值以外,他还有价值7000镑的固定资本,合计仍然是20000镑资本和2000镑利润。他把供他个人花费的后一金额扣除以后,剩下来继续经营业务的流动资本就只有5500镑了;所以他用来维持劳动的资金就从13000镑减少到了5500镑,因此,以前用7500镑雇用的全部劳动现在就会过剩。”}\end{quote}

{可是,如果现在用价值7500镑的机器[用5500镑的可变资本]生产的产品和以前用13000镑可变资本生产的产品一样多,这种情况也会发生。假定机器的损耗一年是十分之一,即750镑,那末以前是15000镑的产品价值,现在就会等于8250镑(原来的7000镑固定资本的损耗不算在内,对这笔资本的补偿问题李嘉图根本没有提及)。在这8250镑中有2000镑利润,就象以前15000镑中有2000镑利润一样。在租地农场主自己把食品和其他必需品作为收入来消费的情况下,他会得到好处。在他由于必需品跌价而能降低他所雇用的工人工资的情况下,他又会得到好处,他的一部分可变资本就会游离出来。这也就是在一定程度上能够用来雇用新的劳动的那部分可变资本,但这只是因为尚未被解雇的工人的实际工资降低了。所以,一小部分被解雇的工人要靠牺牲在业工人的利益才能重新得到工作。但是产品的数量和以前完全一样,这种情况本身对被解雇的工人毫无益处。如果工资保持不变,可变资本丝毫也不会游离出来。产品价值——8250镑——并不会由于它所代表的食品和其他必需品与以前的15000镑所代表的一样多而有所提高。租地农场主一方面为了补偿机器的损耗,另一方面为了补偿他的可变资本,必须把他的产品卖8250镑。如果食品和其他必需品的这种跌价并不引起工资的普遍下降或者并不引起加入不变资本再生产的组成部分的价格下降,那末社会收入就会随着它在食品和其他必需品上的花费而增加。一部分非生产劳动者和生产劳动者等的生活就会过得好些。如此而已。(这部分人甚至会有积蓄,但这总是将来的事情。)被解雇的工人照旧没饭吃,虽然维持他们生活的物质的可能性还是和以前完全一样地存在着。在再生产中还是使用和以前一样的资本。不过以前作为资本而存在的一部分产品(其价值已降低),现在则作为收入而存在。}

\begin{quote}{“当然,资本家现在所能雇用的已经减少的劳动量,借助于机器,在扣除机器维修费以后,必然会生产出等于7500镑的价值,必然能补偿流动资本,并且带来全部资本的利润2000镑。但是,如果做到这一点,[743]如果纯收入不减少,那末对资本家来说,总收入的价值究竟是3000镑,10000镑,还是15000镑,难道不都是一样吗?”}\end{quote}

{这绝对正确。总收入对资本完全无关紧要。它唯一关心的是纯收入。}

\begin{quote}{“因此,在这种情况下,虽然纯产品的价值不会减少,虽然纯产品对商品的购买力可能大大增长,但是总产品的价值将由15000镑降为7500镑。因为维持人口和雇用劳动的能力总是取决于一个国家的总产品,而不是取决于它的纯产品}\end{quote}

{亚·斯密对总产品的偏重就是由此而来的,李嘉图反驳了这一点。见第二十六章(《论总收入和纯收入》),李嘉图在这一章一开头就说:

\begin{quote}{“亚当·斯密经常夸大一个国家从大量总收入中得到的利益,而不是从大量纯收入中得到的利益。”(同上,第415页)},所以对劳动的需求就必然会减少,人口将会过剩,工人阶级的状况将会陷于穷困。”}\end{quote}

{因此,劳动将会过剩,因为对劳动的需求减少了,而需求的减少是由于劳动生产力的发展。这段话在李嘉图的著作第471页。}

\begin{quote}{“不过因为积蓄一部分收入并把它转化为资本的能力必然取决于纯收入满足资本家需要的能力,所以采用机器使商品价格降低后,只要资本家的需要不变{但他的需要会增加},他就可能增加自己的积蓄,从而使收入更容易转化为资本。”}\end{quote}

{照这种说法,一部分资本——不是就它的价值来说,而是就使用价值来说,从构成这部分资本的物质要素来说——首先要转化为收入,然后才能有一部分收入再转化为资本。例如,在可变资本花费13000镑时,总数为7500镑的一部分产品加入了租地农场主所雇用的工人的消费,而且这部分产品是租地农场主的资本的一部分。根据我们的假定,由于采用机器,生产的产品将和以前一样多,但它的价值只有8250镑,而不是以前的15000镑。这种降价的产品现在有较大一部分既加入租地农场主的收入,也加入食品和其他必需品的购买者的收入。他们现在把这样一部分产品作为收入来消费了,这部分产品以前固然也是由租地农场主的工人(现已被解雇)作为收入来消费,但是他们的雇主却是把它作为资本在生产上来消费的。以前作为资本来消费的一部分产品,现在作为收入来消费,这就造成收入的增加,由于收入的这种增加,就形成新资本,收入就再转化为资本。}

\begin{quote}{“但是,每当资本增加,资本家雇用的工人也就增多}\end{quote}

{无论如何也没有他的资本增加的总量那么多。租地农场主也许会买更多的马匹或者鸟粪或者新工具},

\begin{quote}{因此,原先失业的人中有一部分后来就可以就业;如果采用机器以后生产增加很多,以致以纯产品形式提供的食品和其他必需品的数量和以前以总产品形式存在的数量相等,那就有可能象以前那样给全体人口提供工作,因而就不一定{但是可能和也许!}会有过剩的人口出现了。”(第469—472页)}\end{quote}

这就是说,在最后几行,李嘉图直接说出了我在上面已指出的东西。为了使收入按上述途径转化为资本,资本首先要转化为收入。或者如李嘉图所说的,先要靠减少总产品来增加纯产品,然后才有可能把一部分纯产品再转化为总产品。产品就是产品。“纯”和“总”的名称在这里不会引起任何变化(虽然二者的对立也可能意味着,尽管产品总量即总产品减少,超过支出的余额即纯产品也会增加)。产品之成为纯产品或成为总产品,要看它在生产过程中所采取的特定形式而定。

\begin{quote}{“我想要证明的,只是机器的发明和应用可能伴随着总产品的减少;每当这种情形出现时,工人阶级就要受损害,因为其中一部分人将会失业,人口同雇用他们的基金相比将会过剩。”(第472页)}\end{quote}

但是,即使在总产品数量不变或者增加的时候,这种情况也可能发生,而且在大多数场合[744]一定会发生,不同之处仅仅在于总产品的一部分以前用作可变资本,现在则作为收入来消费。

在这后面(第472—474页)李嘉图荒谬地举了一个毛织厂主因采用机器而使生产减少的例子,在这里不必谈它。

\begin{quote}{“如果这种看法是正确的,那末从中就应得出如下结论:(1)机器的发明和有效使用总会增加一个国家的纯产品,虽然它可能不会而且在一个短时期后肯定不会增加这种纯产品的价值。”}\end{quote}

只要它减少劳动的价值,它就总会增加纯产品的价值。

\begin{quote}{“(2)一个国家的纯产品的增加和总产品的减少是可以并存的。采用机器虽然可能而且往往必然会减少总产品的数量与价值,但只要能增加纯产品,使用机器的动机就永远足以保证机器的使用。(3)工人阶级认为使用机器往往会损害他们的利益,这种看法不是以成见和误解为根据,而是符合政治经济学的正确原则的。(4)如果生产资料由于采用机器而得到改良,使一个国家的纯产品大大增加,以致总产品(这里我始终是指商品的数量,而不是指它们的价值)不会减少,那末所有阶级的状况便都会得到改善。土地所有者和资本家会得到好处,但不是由于地租和利润的增加,而是由于用同量的地租和利润可以购买价值大大下降的商品}\end{quote}

{这个论点和李嘉图的整个学说是矛盾的,按照他的学说,必需品的减价,从而工资的下降,会提高利润,而使人们花费较少劳动却能在同一土地上得到更多产品的机器,在李嘉图看来,必然会减少地租};

\begin{quote}{同时工人阶级的状况也会有相当大的改善,第一,由于对家仆的需求增加}\end{quote}

{采用机器的一个真正美妙的结果,就是把工人阶级的相当一部分,妇女和男人,变成了仆人};

\begin{quote}{第二,由于如此丰富的纯产品刺激人们将收入储蓄起来;第三,由于工人用工资购买的一切消费品价格低廉{这种低廉的价格会使他们的工资下降}。”(同上,第474—475页)}\end{quote}

在机器问题上的资产阶级辩护论解释并不否认:

(1)机器经常不断地——时而在这里,时而在那里——使一部分人口过剩,把一部分工人抛向街头。人口过剩(从而有时在这里,有时在那里,引起某些生产领域的工资下降)不是因为人口比生活资料增长得快,而是因为采用机器引起的生活资料数量的迅速增加,使人们能够采用更多的机器,从而减少对劳动的直接需求。人口过剩的产生,不是因为社会基金减少了,而是因为其中用于工资的部分由于这种基金的增长而相对地减少了。

(2)这种辩护论更不否认从事机器劳动的工人本身的被奴役,以及受机器排挤而濒于死亡的手工劳动者或手工业者的困苦。

这种辩护论断言——部分地也是正确的——[第一],由于机器(一般由于劳动生产力的发展),纯收入(利润和地租)会增加,资产者会比以前需要更多的家仆;如果说他以前要从自己的产品中拿出较大部分花费在生产劳动上,那末现在他就可以把较大部分花费在非生产劳动上,结果仆人和其他靠非生产阶级的钱过活的劳动者就会增加。不消说,美妙的前景就是日益增多地把一部分工人变为仆人。同样使工人们聊以自慰的是,由于纯产品的增加,为非生产劳动者开辟了更多的活动领域,这些非生产劳动者要消费生产工人的产品,他们在剥削生产工人的利害关系上也多少和那些直接从事剥削的阶级一致起来了。

第二,资产阶级的辩护论断言:由于新的生产条件需要的活劳动比过去的劳动少而产生的对积累的刺激,受排挤的赤贫化的工人,或者,至少是本来应该接替他们的那部分增加的人口,[745]也会被吸收到生产中来;这或者是由于制造机器的企业本身的生产扩大,或者是由于与机器制造业有关的、因制造机器才成为必要并产生出来的那些生产部门的生产扩大,或者是由于用新资本开辟的并能满足新需要的雇用劳动的新部门的生产扩大。第二个美妙的前景是:工人阶级必须忍受一切“暂时的不方便”——失业以及劳动和资本从一个生产领域转到另一个生产领域,——但是雇佣劳动决不会因而终止;相反,雇佣劳动还会以不断扩大的规模再生产出来,它将绝对地增加,虽然和雇用它的不断增长的总资本相比会相对减少。

第三,资产阶级的辩护论断言,由于机器,消费将变得更为讲究。满足直接生活需要的物品价格低廉,使奢侈品的生产范围能够扩大。这样,在工人面前又开辟了第三个美妙的前景:为了取得他们所必需的生活资料,也就是取得以前那种数量的生活资料,同一数量的工人必须能够使上层阶级扩大他们的享受范围,使享受更讲究,更多样化,从而加深工人和高踞于他们之上的人们之间的经济、社会和政治的鸿沟。这就是劳动生产力的发展将给工人带来的十分美妙的前景和非常令人羡慕的结果。

接着李嘉图还指出,劳动阶级的利益要求

\begin{quote}{“把用在购买奢侈品方面的收入尽量转用来维持家仆”。因为,不管我购买家具还是维持家仆,我都会对一定量[按价值来说]的商品提出需求,在一种场合推动的生产劳动和在另一种场合推动的差不多相等;但是在后一场合我在“原有对工人的需求”之外增加了新的需求,“而需求的这种增加,只是因为我选择了第二种花费我的收入的方式”。(第475—476页)}\end{quote}

当一个国家维持大量海军和陆军时,也会得到同样的结果:

\begin{quote}{“不论它〈收入〉用什么方式来花费,生产上使用的劳动量总是相同的;因为生产士兵的食物和衣服同生产更奢侈的商品所需要的劳动量是相同的。但在战时还需要更多的人去当兵,所以靠收入而不靠国家的资本来维持的战争有利于人口的增长。”(第477页)}\end{quote}

[接着李嘉图写道:]

\begin{quote}{“还有一种情况应当注意:当一个国家的纯收入乃至总收入的数量都增加时,对劳动的需求却可能减少,用马的劳动代替人的劳动时就是这样。如果我在自己的农场里本来雇用100个工人,后来发现,把原来用于50个人的食物用来养马,在支付买马资本的利息后还可以得到更多的原产品,也就是说用马来代替人对我是有利的,那我也就会这样做。但这对工人将是不利的,除非我的收入增加到足以使我能同时使用人和马,否则人口显然就会过剩,工人的状况就会普遍恶化。很明显,在任何情况下工人都不能在农业中找到工作{为什么不能?如果扩大耕地面积呢?};不过,如果土地的产品由于以马代替人而增加了,被解雇的工人也许能在工业中找到工作或者去当家仆。”(第477—478页)}\end{quote}

有两种不断交错的趋势:[第一,]使用尽量少的劳动来生产同样多的或更多的商品,同样多的或更多的纯产品,剩余价值,纯收入;第二,使用尽量多的工人(虽然和他们生产的商品数量相比也是尽量少的),因为——在生产力发展的一定阶段上——使用的劳动量增加,剩余价值和剩余产品的量也会增加。一种趋势把工人抛向街头,造成过剩的人口;另一种趋势又把他们吸收掉,并绝对地扩大雇佣劳动奴隶制。于是工人被他的命运东抛西扔,但始终还是摆脱不了这种命运。所以工人完全有理由把他自己的劳动生产力的发展看作某种与他敌对的东西;另一方面,资本家则把他当作一个必须不断从生产中离开的要素来对待。

李嘉图在这一章中要努力解决的正是这些矛盾。他忘记指出:[746]介于工人为一方和资本家、土地所有者为另一方之间的中间阶级不断增加,中间阶级的大部分在越来越大的范围内直接依靠收入过活,成了作为社会基础的工人身上的沉重负担,同时也增加了上流社会的社会安全和力量。

资产者把采用机器使雇佣奴隶制永久化这件事用来为机器“辩护”。

\begin{quote}{“在前面我还指出过:机器改良的结果,以商品计算的纯收入总会增加,这种增加会导致新的积蓄和积累。我们必须记住,这种积蓄是逐年进行的,不久就会创造出一笔基金,其数额远远大于原来因发明机器而损失的总收入。这时对劳动的需求将和以前一样大,人民的生活状况也将由于积蓄的增加而得到进一步改善,增加了的纯收入又使他们有可能增加积蓄。”(第480页)}\end{quote}

先是总收入有损失,纯收入有增加。然后一部分增加了的纯收入再转化为资本,从而再转化为总收入。工人就这样被迫不断增大资本的权力,以便在极大的动乱之后,可以被允许在更大的规模上重复这同一过程。

\begin{quote}{“资本和人口每有增加,食品价格一般总会上涨,因为生产这些东西更加困难了。”(第478—479页)}\end{quote}

紧接着李嘉图就说道:

\begin{quote}{“食品价格上涨会使工资提高,而工资的每次提高都会有一种趋势,就是把积蓄起来的资本比以前更多地用于使用机器方面。机器同劳动处于不断的竞争中,机器往往只是在劳动价格上涨时才能被应用。”(第479页)}\end{quote}

因此机器是制止劳动价格上涨的手段。

\begin{quote}{“为了说明原理,我曾假定改良的机器是突然发明出来并得到广泛应用的。但实际上这种发明是逐渐完成的,其作用与其说是使资本从现在的用途上转移,倒不如说是决定被积蓄和积累的资本的用途。”(第478页)}\end{quote}

实际情况是:由于资本的新积累而得以进入使用劳动的新领域的,主要不是被排挤的劳动,而是新提供的劳动,即本来应该接替被排挤的工人的那部分增加的人口。

\begin{quote}{“在人们容易取得食物的美国和其他许多国家里,使用机器的诱惑力远远不象在英国那样大{除了美国,在任何地方都没有那样大规模地使用机器,甚至在家庭日常生活中也使用机器},在英国,食品价格很高,生产食品要耗费很多劳动。”}\end{quote}

{美国使用机器比经常有过剩人口的英国相对说来要多得多,正是美国,表明机器的使用很少取决于食品的价格,虽然它的使用可能象在美国那样取决于工人的相对不足,在美国,分布在辽阔的国土上的人口相对说来是稀少的。例如我们在1862年9月19日《旗帜报》\endnote{《旗帜报》是英国保守派的日报,1827年在伦敦创刊。——第655页。}刊载的一篇关于博览会的文章\endnote{指1862年9月19日《旗帜报》第5—6版上刊登的《博览会上的美国》一文(未具名)。关于1862年的世界博览会,见注112。——第655页。}中就读到:

\begin{quote}{“人是制造机器的动物……如果把美国人当作人类的代表,那末这个定义……是无可非议的。能用机器做的事就不用手去做,这已成为美国人的一个主要观点。从摇摇篮到做棺材,从挤牛奶到伐木,从缝钮扣到选举总统的投票,他们几乎全都用机器。他们发明了一种机器,可用来节省咀嚼食物的劳动……劳动力的异常缺乏和由此而来的很高的劳动价值{虽然食品的价值很低},以及某种天生的灵敏激发了他们的发明精神……美国生产的机器一般说比英国造的价格便宜……总的来说,机器与其说是能做以前不能做的事的一种发明,不如说是节省劳动的装置。{那末汽船呢?}……在博览会的美国馆中展出有埃默里的轧棉机。从美国开始植棉以来很多年内,棉花的收获量并不太大,这不仅仅是因为对棉花的需求不大,而且是因为手工清棉的困难使种植棉花变得无利可图。但是当伊莱·维特尼发明了清棉的锯齿[747]轧棉机时,植棉面积立即增加,而且至今几乎还在按算术级数增加。的确,说维特尼创造了商业性的植棉业,这并不夸大。维特尼轧棉机经过或多或少重大而有效的改进一直还在使用,原始的维特尼轧棉机在现在的这些改进和补充发明以前,丝毫不逊于大多数妄图把它排挤出去的机器。现在带有沃耳巴尼(纽约州)埃默里商标的机器,无疑完全取代了为它奠定基础的维特尼机。这种机器同样简单,生产效率却更高;它轧出的棉花不仅更干净,而且一层层象铺好的棉絮一样,因此这一层层的棉花从机器出来后立即可以压紧打包……在博览会的美国馆中几乎全是机器……挤奶机……从一个滑轮到另一滑轮的传动皮带装置……大麻梳纺机,一下子就把包扎成捆的大麻直接打成麻绳……纸袋机,它可以裁纸、粘叠,一分钟做300个纸袋……霍斯的洗衣挤干机,它用两个橡皮滚筒把衣服上的水挤出,衣服就差不多干了,用这种机器节省时间,而且不损伤织物……装订机……制鞋机。大家都知道,在英国早已用机器制作鞋面,可是这里展出的还有绱鞋的机器,切鞋底的机器,还有做鞋后跟的机器……功率强大的碎石机,结构很灵巧,无疑会广泛使用于铺路和捣碎矿石……奥本(纽约州)的华德先生的航海信号系统……收割机和割草机是美国的发明,它们越来越获得英国方面的好评。麦考密克的机器是最好的……汉斯布劳先生的压力泵,曾荣获加利福尼亚奖章,它的结构简单和生产效率之高在博览会上是最突出的……与世界上任何水泵相比,它能以同样的力量抽更多的水……缝纫机……”}“使劳动价值提高的原因并不会提高机器的价值,所以资本每有增加,就会有越来越大的部分用在机器方面。对劳动的需求将随着资本的增加而继续增加,但不会按同一比例增加,其增加率一定是递减的。”(第479页)}\end{quote}

在最后一句话中李嘉图表述了正确的资本增加规律,虽然他用以论证这一规律的理由是非常片面的。李嘉图在他的著作的这个地方还加了一个注,表明他在这里是追随巴顿,所以我们还要简略地谈一谈巴顿的著作。

预先还要作一点说明。在这之前,李嘉图在谈到收入是花费在家仆上还是花费在奢侈品上这一问题时说:

\begin{quote}{“在这两种情况下,纯收入是相同的,总收入也是相同的,但是纯收入实现在不同的商品上。”(第476页)}\end{quote}

同样,总产品的价值也可能是一样的,但是,它是否会“实现”——工人对此是非常敏感的——“在不同的商品上”,那要看它应补偿更多的可变资本还是应补偿更多的不变资本。

\tsubsectionnonum{[(2)巴顿的见解]}

\tsubsubsectionnonum{[(a)巴顿关于资本积累过程中对劳动的需求相对减少的论点。巴顿和李嘉图不懂得这种现象同资本统治劳动有内在的联系]}

巴顿的著作叫作:

约翰·巴顿《论影响社会上劳动阶级状况的环境》1817年伦敦版。

我们先引一下巴顿著作中为数不多的理论观点:

\begin{quote}{“对劳动的需求取决于流动资本的增加,而不是取决于固定资本的增加。如果这两种资本的比例在任何时候和任何国家确实都是一样的话,那末由此的确可以得出结论说,就业工人的人数同国家的财富成比例。但是这种假定一点也不现实。随着技术的进步和文明的传播,固定资本与流动资本相比越来越大。英国生产一匹凡而纱所使用的固定资本额至少等于印度生产同样一匹凡而纱所使用的固定资本额的一百倍,也许是一千倍。而[748]流动资本的份额则小到百分之一或千分之一。我们很容易想象得到,在一定的情况下,一个勤劳的民族可能把一年的全部积蓄都加到固定资本上去,在这种情况下,这些积蓄也不会使对劳动的需求有任何增长。”(巴顿,同上第16—17页)}\end{quote}

{李嘉图在他的著作第480页的注中,就巴顿的这些话指出:

\begin{quote}{“我认为,在任何情况下资本增加而对劳动的需求不随之增加是难于想象的。至多只能说,对劳动的需求的增加率将愈来愈小。在我看来,巴顿先生在上述著作中关于固定资本的扩大对工人阶级状况的某些影响所持的看法是正确的。他的著作包含许多有价值的资料。”}}\end{quote}

对上面引的巴顿的论点,必须再加上下面这一段话:

\begin{quote}{“固定资本一经形成,就不再引起对劳动的需求了{这不对,因为固定资本必须再生产,虽然这种再生产只能经过一定的间隔时间,而且只能逐渐地进行},但是在固定资本形成的时候,它所使用的人手,和同额流动资本或收入所使用的人手一样多。”(第56页)}\end{quote}

还有:

\begin{quote}{“对劳动的需求,完全取决于收入和流动资本的总额。”(第34—35页)}\end{quote}

毫无疑问,巴顿有很大的功劳。

亚·斯密认为,对劳动的需求的增加同资本的积累成正比。马尔萨斯从资本积累不象人口增加那样快(资本不是以递增的规模再生产)出发,得出人口过剩的结论。巴顿第一次指出,资本各有机组成部分并不随着积累和生产力的发展而以同样程度增加;相反,在生产力增长的过程中,转化为工资的那部分资本同(巴顿称为固定资本的)另一部分相比会相对减少,而后者同自己的量相比只是稍微改变对劳动的需求。因此,他第一次提出这样一个重要论点:“就业工人的人数”不是“同国家的财富成比例”,工业不发达国家的就业工人的人数比工业发达的国家相对地多。

李嘉图在他的《原理》的前两版中,在这一点上还完全跟着斯密走,但在第三版第三十一章《论机器》中,却采用了巴顿的修正,而且采用了巴顿的片面的说法。李嘉图向前发展的唯一的一点——这一点有重要意义——就是:他不仅象巴顿那样提出,对劳动的需求的增加不是同机器的发展成比例,而且还断言,机器本身“造成人口过剩”\fnote{见本册第644、646、648、649和653页。——编者注},即造成过剩的人口。只不过李嘉图错误地把采用机器的这种结果局限于纯产品靠减少总产品而增加的场合——这种场合只有在农业中才会遇到,而他却认为工业中也会出现。但这也就简要地反驳了整个荒谬的人口论,尤其是反驳了庸俗经济学家关于工人必须努力把自己的繁殖限制在资本积累的水平以下的谰言。相反,从巴顿和李嘉图对问题的论述中可以得出结论说,这样限制工人人口的繁殖,会减少劳动的供给,因而会提高劳动的价格,这只会加速机器的采用,加速[花在工资上的]流动资本向固定资本的转化,从而人为地造成人口“过剩”;因为人口过剩通常不是与生存资料的数量相对而言的,而是与雇用劳动的资金的数量,与对劳动的实际需求相对而言的。

[749]巴顿的错误或缺点在于,他对资本的有机区别或资本的有机构成,只从它在流通过程中所表现的形式即固定资本和流动资本的形式来理解,——这是一种已为重农学派所发现的差别,亚·斯密对它作了进一步的阐述,在斯密之后,它成了经济学家们的偏见,其所以是偏见,是因为他们按照传统,把资本的有机构成只看成这种差别。这种从流通过程产生的差别一般说来对财富的再生产有重大的影响,因而对财富中构成“劳动基金”的那部分也有重大的影响。但是在这里这并不是决定性的东西。作为固定资本的机器、建筑物、种畜等等同流动资本的不同之处,并不在于它们直接同工资有什么样的关系,而只在于它们的流通和再生产的方式。

资本的不同组成部分对活劳动的直接关系,不是同流通过程的现象相联系,不是从流通过程产生,而是从直接的生产过程产生,并且是不变资本和可变资本之间的关系,而不变资本和可变资本之间的差别只以它们对活劳动的关系为基础。

例如巴顿说:对劳动的需求不取决于固定资本,只取决于流动资本。但是,流动资本的一部分,即原料和辅助材料,就象机器等等一样不同活劳动交换。在原料作为要素加入价值形成过程的一切生产部门中,原料——就我们考察的只是加入商品的那部分固定资本而言——构成不是花费在工资上的那部分资本的最大部分。[巴顿称为]流动资本的另一部分,即商品资本的一部分,由加入非生产阶级的收入(即不加入工人阶级的收入)的那些消费品构成。可见,这两部分流动资本的增加一点也不比固定资本的增加对劳动的需求有更大的影响。而且,由原料和辅助材料所构成的那部分流动资本和作为固定资本投入机器等等上面的那部分资本相比较,并不是以更小的比例增长,甚至还是以更大的比例增长。

拉姆赛进一步阐述了巴顿所指出的这种差别。他修正了巴顿的见解,但没有超出巴顿的表述方式。他实际上把巴顿所说的差别归结为不变资本和可变资本的差别,但仍然把不变资本叫做固定资本,不过把原料等等也加了进去;把可变资本叫做流动资本,然而把所有不直接花费在工资上的流动资本排除在外。关于这些,我们以后讲到拉姆赛时再谈。但是这证明了内在发展的必然性。

只要一弄清楚不变资本和可变资本的差别(这种差别完全来自直接的生产过程,来自资本的不同组成部分对活劳动的关系),也就会看到这种差别本身同所生产的消费品的绝对量没有任何关系,虽然它和总收入的一定量实现在什么物品上有很大关系。但是,总收入实现在不同商品上的这种方式,并不象李嘉图所说的和巴顿所暗示的那样,是资本主义生产的内在规律的原因,而是它的结果,这种规律使得产品中构成工人阶级再生产的基金的那一部分和产品总额相比越来越小。如果说资本的很大一部分是由机器、原料、辅助材料等等构成,那末工人阶级总人数中就只有一小部分人被用来再生产[750]加入工人消费的生活资料。但是可变资本再生产的这种相对减少,并不是对劳动需求相对减少的原因,反而是它的结果。同样,在那些从事一般加入收入的消费品生产的工人中,将有较大一部分人用来生产供资本家、土地所有者及其仆从(国家、教会等等)消费——即花费收入——的物品,而较少一部分人则用来生产用于工人收入的物品。但这依然是结果,而不是原因。只要工人和资本家的社会关系发生改变,只要支配资本主义生产的关系发生革命,这种情况就会立即发生变化。收入,用李嘉图的话来说,就会“实现在不同的商品上”。

在所谓的生产的物质条件中没有什么东西能强迫人们采用这种或那种实现收入的方式。如果工人居于统治地位,如果他们能够为自己而生产,他们就会很快地,并且不费很大力量地把资本提到(用庸俗经济学家的话来说)他们自己的需要的水平。重大的差别就在于:是现有的生产资料作为资本同工人相对立,从而它们只有在工人必须为他们的雇主增加剩余价值和剩余产品的情况下才能被工人所使用,是这些生产资料使用他们工人,还是工人作为主体使用生产资料这个客体来为自己生产财富。当然这里要以资本主义生产一般说来已把劳动生产力发展到能够发生这一革命的必要高度为前提。

{以1862年(今年秋天)为例。郎卡郡的失业工人处境困难。另一方面,在伦敦货币市场上,“货币难于找到用途”,结果几乎必然要出现投机公司,因为贷款连2%的利息也难得到。根据李嘉图的理论就会得出这样的结论:因为一方面伦敦有资本,另一方面曼彻斯特有失业的劳动力,所以“必定会开辟一个别的使用劳动的部门”。}

\tsubsubsectionnonum{[(b)巴顿对工资变动和工人人口增长的见解]}

巴顿接着论证,只要事先不是人口增长得这样厉害,以致工资水平很低,资本的积累就只会缓慢地提高对劳动的需求。

\begin{quote}{“一定时期的工资和劳动总产品之比决定资本用在这一方面〈固定资本〉还是那一方面〈流动资本〉。”(同上,第17页)“如果工资在商品价格不变时下降,或者如果商品价格在工资不变时上涨,那末企业主的利润就会增加,这就会推动他雇用更多的人手。相反,如果同商品相比工资上涨了,工厂主就会尽量少用人手,力求用机器来做一切事情。”(同上,第17—18页)“我们有可靠的材料证明,在工资逐渐上涨的上一世纪(即十八世纪)的上半叶比劳动的实际价格急遽下降的下半叶,人口的增长要缓慢得多。”(第25页)“可见,工资上涨本身决不会使工人人口增加;工资下降却能使人口十分迅速地增加。例如,如果英格兰人的需要降到爱尔兰人的需要水平,那末工厂主就会根据他们生活费用减少的程度而使用更多的工人。”(第26页)“寻找工作的困难比工资的低微对结婚的妨碍要大得多。”(第27页)“必须承认,财富的任何增加都有造成对劳动的新的需求的趋势。但是因为和所有其他商品相比,劳动的生产所需要的时间最长}\end{quote}

{根据同一原因,工资额可能长期保持在平均水平之下,因为同所有其他商品相比,使劳动离开市场,从而使它的供给降到当时的需求水平,是最难于做到的},

\begin{quote}{所以在一切商品中,[751]由于需求的增加,劳动的价格上涨得最多;并且因为工资一上涨,十分利润就会减少九分,所以很清楚,除非在这以前人口增长得使工资保持在很低的水平,资本的增加就只能对劳动的实际需求的增加产生缓慢的影响。”(第28页)}\end{quote}

巴顿在这里提出了各种论点。

第一,工资上涨本身不会使工人人口增加;可是工资下降却能很容易地迅速地使工人人口增长。证据是:十八世纪上半叶工资逐渐上涨而人口缓慢地增长;相反,十八世纪下半叶实际工资大大下降,工人人口却迅速增长。原因是:妨碍结婚的不是工资的低微,而是寻找工作困难。

第二,可是工资水平越低,寻找工作越容易。因为资本转化为流动资本还是固定资本,也就是转化为使用劳动的资本还是不使用劳动的资本,是同工资的高低成反比的。工资低,对劳动的需求就大,因为那时使用大量劳动对企业主有利,而且他用同量的流动资本能够使用更多的工人。工资高,工厂主就会尽量少用工人,并力求用机器来做一切事情。

第三,资本的积累本身只会缓慢地提高对劳动的需求,因为这种需求一提高,如果劳动的供给小于对它的需求,劳动的价格就会迅速上涨,十分利润就会减少九分。只有在这种情况下,即在积累之前工人人口已大大增长,以致工资水平极低,甚至在上涨以后仍然很低的情况下,积累才能迅速地反映在对劳动的需求上,因为[新的]需求主要是吸收失业的人手,而不是争夺完全就业的工人。

所有这些略加修正,也适用于充分发达的资本主义生产。但是这不说明资本主义生产的发展过程本身。

因此,巴顿提出的历史证据是和它应该证明的东西相矛盾的。

在十八世纪上半叶,工资逐渐上涨,人口缓慢地增长,也没有任何机器,和下半叶相比,也很少使用其他固定资本。

相反,在十八世纪下半叶,工资不断下降,人口惊人地增长,却出现许多机器。但正是机器,一方面,使现有的人口过剩,从而使工资降低,另一方面,由于世界市场的迅速发展,又把这些人口吸收,之后再使它过剩,再把它吸收;与此同时,机器异常地加快了资本的积累,增加了可变资本的数量,虽然这种可变资本无论是同产品的总价值相比,还是同它使用的工人数量相比,都相对地减少了。

相反,在十八世纪上半叶还没有大工业,只有以分工为基础的工场手工业。资本的主要组成部分仍然是花费在工资上的可变资本。劳动生产力的发展比十八世纪下半叶缓慢。对劳动的需求,以及工资,是和资本的积累一起增加的,并且几乎是同这种积累成比例地增加的。英国实质上还是一个农业国,那里还广泛地存在着(甚至继续发展着)农业人口所经营的家庭(纺织)工业。严格意义上的无产阶级还不可能产生,也还没有工业的百万富翁。

在十八世纪上半叶可变资本比较占优势,下半叶固定资本占优势;但是固定资本需要大量的人这样的材料。要大规模地运用固定资本,就必须先有人口的增长。可是发展的整个实际过程和巴顿对它的解释是矛盾的,因为很明显,这里生产方式一般发生了变化:适合于大工业的规律和适合于[752]工场手工业的规律不是一回事。工场手工业只是向大工业发展的一个阶段。

但是巴顿提出的一些历史材料,如关于工资变动的材料以及关于英国谷物价格变动的材料,在这里都是值得注意的,因为巴顿把十八世纪上半叶和下半叶做了对比。

\begin{quote}{“下表说明了〈可是“十七世纪中叶至十八世纪中叶[实际]工资上涨了,因为谷物价格在这个时期至少下跌了35%”〉最近70年农业工人的工资和谷物价格之比。(巴顿,同上第25—26页)“在上院[一个委员会]关于济贫法的报告中〈1816年?〉列有一表,载明从革命时期[1688年]以来议会各次会议所通过的有关圈地的法案的数目。从该表可以看出,在从1688年至1754年的66年中,这样的法案共通过了123件,而在1754年至1813年的69年\fnote{巴顿著作中如此。实际上从1754至1813年只有59年。——编者注}中却通过了3315件。谷物种植的发展速度后一时期几乎为前一时期的25倍。可是在前66年中,有越来越多的谷物不断生产出来用于出口,而在后69年的大部分时间中,原来用于出口的谷物都消费了,同时还进口越来越多的、最后达到很大数量的谷物供自己消费……所以,和后一时期相比,前一时期人口的增长,看来比谷物种植的发展所表现出来的还要慢。”(同上,第11—12页)“根据格雷哥里·金按照住房数目确定的数字,1688年英格兰和威尔士的人口为550万。根据马尔萨斯的计算,1780年的人口为770万。这就是说,在92年间人口增加了220万。在以后的30年内人口增加了270万以上。但是关于前一次人口的增长,很可能,主要是在1750年至1780年期间发生的。”(第13页)}\end{quote}

\todo{}

巴顿根据可靠材料计算,

\begin{quote}{“1750年居民人数为5946000人,这说明从革命时期[1688年]以来增加了446000人,或者说,每年增加7200人”。(第13—14页)“可见,按照最低的估计,近几年来人口增长的速度为一百年前的10倍。但是资本的积累要增长到10倍是不可思议的。”(第14页)}\end{quote}

问题不在于每年生产多少生活资料,而在于每年有多大一部分活劳动加入固定资本和流动资本的生产。与不变资本相对的可变资本量就是由此决定的。

巴顿用美洲矿山生产率的不断提高来说明近五、六十年来几乎整个欧洲人口的惊人增长,因为贵金属的这种充裕使商品价格的提高大于工资的提高,也就是说,实际上降低了工资,因而提高了利润率。(第29—35页)[XIII—752]

\tchapternonum{附录}

\tsectionnonum{[(1)关于农业中供求经常相符的论点的最初提法。洛贝尔图斯和十八世纪经济学家中的实践家]}

[\endnote{以《剩余价值理论》第二册附录形式发表的短评,是马克思在手稿第XI、XII和XIII本的封面上写的。它包括《理论》第二册正文中所考察的某些问题的补充材料。——第667页。}XII—580b]斯密“顺便提出的”关于谷物创造对谷物本身的需求等等\fnote{见本册第402页及以下各页。——编者注}的理论(这个理论后来马尔萨斯在他的地租理论中扬扬得意地加以复述,并且部分地成了他的人口论的基础),[在斯密之前]在下面一段话里已经十分简单明了地提出来了:

\begin{quote}{“谷物同它的消费或多或少是成比例的。如果人口多了,谷物也会多,因为会有更多的人手耕种土地;如果谷物多了,人口也会多,因为丰富将使人口增加。”([约翰·阿伯思诺特]《当前粮食价格和农场面积相互关系的研究》,一个租地农场主著,1773年伦敦版第125页)}\end{quote}

因此,

\begin{quote}{“在农业中不可能有生产过剩”。(同上,第62页)}\end{quote}

洛贝尔图斯关于种子等不作为资本项目加入\fnote{见本册第39—52页。——编者注}[租地农场主的支出]的幻想,已被十八世纪(特别是从六十年代起)的数百篇论文[所驳倒],其中有些是租地农场主自己写的。但是相反,认为地租作为费用项目加入租地农场主的支出,倒是正确的。租地农场主把地租算在生产费用之内(它也确实属于他的生产费用)。

\begin{quote}{“如果……谷物价格接近它应该达到的水平,那末,这只能由土地价值和货币价值的比例决定。”(同上,第132页)}\end{quote}

下面这段话表明,自从资本掌握了农业的时候起,地租在资本主义租地农场主本人的概念中,就仅仅成了利润的扣除,全部剩余价值就开始被看作实质上是利润:

\begin{quote}{“老办法是租地农场主的利润[他在总产品中占的份额]按三份地租计算〈分成制〉。在农业的幼年时期,这是分配财产的公平合理的办法。在世界不太开化的地方现在仍然采用这种办法……一方提供土地和资本,另一方提供知识和劳动。但是在耕作得好的和肥沃的土地上,地租现在具有极小的意义。重要的是一个人能够以资本的形式和自己劳动的年支出形式投入的那笔款项,他必须根据这笔款项来计算自己货币的利息,或者说,自己的收入。”(同上,第34页)[XII—580b]}\end{quote}

\tsectionnonum{[(2)纳萨涅尔·福斯特论土地所有者和工业家之间的敌对关系]}

[XIII—670a]“\textbf{土地所有者}和\textbf{工业家}彼此之间永远是敌对的,对对方的赢利是忌妒的。”([\textbf{纳萨涅尔·福斯特}]《论当前粮价昂贵的原因》1767年伦敦版第22页注释)[XIII—670a]

\tsectionnonum{[(3)霍普金斯对地租和利润之间的关系的看法]}

[XIII—669b]\textbf{霍普金斯}(见有关段落\fnote{见本册第52页。——编者注})天真地把\textbf{地租}看作剩余价值的原始形式,而把利润看作从地租派生的东西。

霍普金斯写道:

\begin{quote}{“当……生产者既是土地耕种者又是制造业者时,土地所有者得到10镑价值的\textbf{地租}。假定这个地租一半用原产品支付,另一半用工业品支付。假定生产者\textbf{分为}两个阶级(土地耕种者和制造业者)之后,这种情况能够照旧继续下去。但是,实际上更方便的是,由土地耕种者向土地所有者\textbf{交付全部地租},而在他拿自己的产品去同制造业者的劳动产品交换时把地租加到自己的产品上,以便两个阶级公平地分摊这笔款项,使两个部门的工资和利润保持在同一水平上。”(\textbf{托·霍普金斯}《关于调节地租、利润、工资和货币价值的规律的经济研究》1822年伦敦版第26页)[XIII—669b]}\end{quote}

\tsectionnonum{[(4)凯里、马尔萨斯和詹姆斯·迪肯·休谟论农业改良]}

\begin{quote}{[XI—490a]“应该指出,我们总是把土地所有者和租地农场主看成\textbf{同一个人}……在美国情况就是这样。”(\textbf{亨·查·凯里}《过去、现在和将来》1848年费拉得尔菲亚版第97页)“人总是从贫瘠的土地推移到较好的土地,然后再回到原来的贫瘠土地并且翻耕泥灰质或石灰质的土地,这样持续不断地反复进行……在这条道路的每一阶段上,人造出越来越好的机器\fnote{指被耕种和改良的土地。——编者注}……资本投入农业可以比投入\textbf{机器}得到\textbf{更大的}利益,因为同一种机器\textbf{仅仅}具有\textbf{同样的}效能,而土地的生产率却越来越\textbf{提高}……采用蒸汽机得到的好处是:它节约了比如说把毛织成呢的劳动的工资,但要\textbf{减去}机器磨损的损失。而用于耕种土地的劳动生产了工资,还\textbf{加上}由于土地这种机器的改良而得到的利益……因此,每年带来100镑收入的一个地段,要比带来同样收入的一台蒸汽机卖得贵……地段的买者知道,这块土地会付给他工资和利息,加上这块土地由于使用而增加的价值。蒸汽机的买者知道,蒸汽机会付给他工资和利息,减去这台机器由于使用而减少的价值。前者购买的,是一种随着使用而不断改良的机器。后者购买的,是一种随着使用而不断变坏的机器……土地是这样一种机器,新资本和劳动花费在它上面可以得到日益增长的利益,而要使花费在蒸汽机上的这种支出带来日益增长的收入却是不可能的。”(同上,第129—131页)}\end{quote}

\centerbox{※     ※     ※}

有的农业改良即使会使生产费用减少并且最终会使价格下降,而在初期——在价格还没有下降时——会引起农业\textbf{利润}的暂时提高,但是,

\begin{quote}{“\textbf{最后}也几乎总是\textbf{使地租增加}。由于有可能得到大量的暂时利润而投入农业的\textbf{增加的资本,在多数情况下不可能在租佃期内完全从租种的土地上抽回},而\textbf{在重订租约时},土地所有者就要通过\textbf{增加自己的地租}来从这些投资中得到利益”。(\textbf{马尔萨斯}《关于地租的本质和增长及其调整原则的研究》1815年伦敦版[第25—26页])}\end{quote}

\centerbox{※     ※     ※}

\begin{quote}{“如果说,在近几年谷物普遍涨价之前,耕地一般只提供\textbf{不多的地租}主要是因为有\textbf{公认的必要的经常休闲},那末现在就应该再减少地租量,以便有可能回到原来的休闲制。”(\textbf{詹·迪·休谟}《关于谷物法及其同农业、商业和财政的关系的看法》1815年伦敦版第72页)[XI—490a]}\end{quote}

\tsectionnonum{[(5)霍吉斯金和安德森论农业劳动生产率的增长]}

\begin{quote}{[XIII—670a]“随着人口的增长,为了给人们提供食物,只要有越来越少的土地面积就够了。”([\textbf{托马斯·霍吉斯金}]《财产的自然权利和人为权利的比较》1832年伦敦版第69页)(霍吉斯金的这部著作是匿名出版的。)}\end{quote}

在霍吉斯金之前,\textbf{安德森}已谈到了这一点\fnote{见本册第157—159页。——编者注}。[XIII—670a]

\tsectionnonum{[(6)]利润率的下降}

[XIII—670a]使用较多不变资本(机器、原料)的较大资本的利润,——因为要分摊到使用的活劳动占较小比例的总资本上,——[按其比率来说]小于在较小的总资本中占较大比例的活劳动所创造的[按量来说]较小的利润。可变资本的[相对]减少和不变资本的相对增加(虽然这两部分资本都在增长),只是\textbf{劳动生产率提高的}另一种\textbf{表现}。[XIII—670a]


\tchapternonum{[第十九章]托·罗·马尔萨斯}

\tchapternonum{[(1)马尔萨斯把商品和资本这两个范畴混淆起来]}

[\endnote{马克思在论马尔萨斯这一章中,考察了马尔萨斯在李嘉图的《原理》出版(1871年)以后所写的著作。在这些著作里,马尔萨斯企图用旨在维护统治阶级中最反动阶层的利益的庸俗辩护论来对抗李嘉图的劳动价值论,对抗李嘉图的千方百计发展生产力的要求,而照李嘉图的观点,这样发展生产力,就应当牺牲个人的甚至整个阶级的利益。关于作为“人口论”的鼓吹者的马尔萨斯,在本章中只是附带谈了一下。马克思在《剩余价值理论》第2册《对所谓李嘉图地租规律的发现史的评论》那一章中对马尔萨斯论人口的著作做了一般的评述(见本卷第2册第121、123、125—128、158页)。——第3页。}XIII—753]这里要考察的是马尔萨斯的下列著作:

(1)《价值尺度。说明和例证》1823年伦敦版。

(2)《政治经济学定义》1827年伦敦版(还要参看约翰·卡泽诺夫1853年在伦敦出版

的这一著作及其所附卡泽诺夫的《注释和补充评论》)。

(3)《政治经济学原理》1836年伦敦第2版(要参看1820年或1820年前后的第1版)。

(4)还要注意一个马尔萨斯主义者\endnote{后来查明,这一匿名著作的作者是约翰·卡泽诺夫。——第3页。}(一个反对李嘉图学派的马尔萨斯主义者)的下述著作:《政治经济学大纲》1832年伦敦版。

马尔萨斯在他《论谷物法的影响》(1814年)这一著作中谈到亚·斯密时还说:

\begin{quote}{“斯密博士作了这样一番论证[即断言谷物的实际价格永远不变],显然是由于他习惯于把劳动〈即劳动价值〉看作价值的标准尺度,而把谷物看作劳动的尺度……无论劳动或其他任何商品都不能成为实际交换价值的准确尺度,这在现在已被认为是政治经济学的最明白不过的原理之一。实际上这正是从交换价值的规定本身得出来的。”\fnote{本卷引文中凡是尖括号〈〉和花括号{}内的话都是马克思加的。——译者注}[第11—12页]}\end{quote}

但是,马尔萨斯在他1820年的著作《政治经济学原理》中,却借用斯密的这个“价值的标准尺度”来反对李嘉图,而斯密自己在他理论上真正有所发展的地方从来没有使用过它。\endnote{马克思在他的著作的前几章中批判了斯密把劳动价值看作价值的标准尺度的观点,并且证明这一观点与斯密对价值的其他更深刻的见解相矛盾。见本卷第1册第54—55和140页,第2册第457—459页。——第4页。}在上面引用的那本论谷物法的著作中,马尔萨斯自己采用了斯密的另一个价值规定:价值决定于生产某一物品所必需的资本(积累劳动)和劳动(直接劳动)的量。

总之,不能不承认,马尔萨斯的《原理》以及要在某些方面对《原理》作进一步发挥的上述另两部著作的产生,在很大程度上是由于马尔萨斯嫉妒李嘉图的著作\endnote{指李嘉图的主要著作《政治经济学和赋税原理》1817年伦敦版。——第4页。}所取得的成就,并且企图重新爬上他在李嘉图的著作问世前作为一个剽窃能手所骗取到的首席地位。此外,在李嘉图的著作中对价值所作的规定尽管还是抽象的,但它是反对地主及其仆从们的利益的,而马尔萨斯却维护这些人的利益,比维护工业资产阶级的利益更为直接。同时,不可否认,马尔萨斯有在理论上故弄玄虚的某种兴趣。不过,他所以能够反对李嘉图,以及能够以这种方式来反对,只是因为李嘉图有种种自相矛盾之处。

马尔萨斯在反对李嘉图时用来作为出发点的,一方面是剩余价值的产生问题\endnote{马克思在《剩余价值理论》第二册中指出,李嘉图没有分析剩余价值的产生,劳动与资本的交换问题按照李嘉图的提法无法解决。(见本卷第2册第449—454和459—474页)。——第4页。},另一方面是李嘉图把不同投资领域中费用价格\endnote{“费用价格”(“Kostenpreis”或“Kostpreis,”“costprice”)这一术语,马克思用在三种不同的意义上:(1)资本家的生产费用(c+v),(2)同商品价值一致的商品的“内在的生产费用”(c+v+m),(3)生产价格(c+v+平均利润)。这里正文中用的是第三种意义,即生产价格。在《剩余价值理论》第二册中,“费用价格”这一术语(以及“平均价格”和“生产价格”)总是指生产价格,而在本册中,它有时也用来指资本家的生产费用(c+v),在这种情况下,本册中都译为“生产费用”(例如,第38、39III40、410、516页)。《Kostenpreis》这一术语所以有三种用法,是由于《Kosten》(“费用”、“生产费用”)这个词在经济科学中被用在三种意义上,正如马克思在《剩余价值理论》第三册中(见本册第81—86页和第569—570页)特别指出的,这三种意义是:(1)资本家预付的东西,(2)预付资本的价格加平均利润,(3)商品本身的实在的(或内在的)生产费用。除了资产阶级政治经济学古典作家使用的这三种意义以外,“生产费用”这一术语还有第四种庸俗的意义,即让·巴·萨伊给“生产费用”下的定义:“生产费用是为劳动、资本和土地的生产性服务支付的东西。”(让·巴·萨伊《论政治经济学》1814年巴黎第2版第2卷第453页)马克思坚决否定了对“生产费用”的这种庸俗的理解(例如见本卷第2册第142、239和535—536页)。——第4页。}的平均化看作价值规律本身的变形的观点,以及他始终把利润和剩余价值混淆起来(把两者直接等同起来)的做法。马尔萨斯并没有解决这些矛盾和概念的混乱,而是从李嘉图那里把它们接受过来,以便依靠这种混乱去推翻李嘉图关于价值的基本规律等等,并作出使他的保护人乐于接受的结论。

马尔萨斯的上述三部著作的真正贡献在于,他强调了资本和雇佣劳动之间的不平等交换,而李嘉图实际上却没有阐明,按价值规律(按商品中所包含的劳动时间)进行的商品交换中,如何产生出资本和活劳动之间、一定量的积累劳动和一定量的直接劳动之间的不平等交换,也就是说,实际上没有说明剩余价值的起源(因为在李嘉图那里资本是直接和劳动相交换,而不是和劳动能力相交换)。[754]后来的为数不多的马尔萨斯信徒之一——卡泽诺夫,在为马尔萨斯的上述著作《政治经济学定义》所写的序言中觉察到了这一点,因此他说:

\begin{quote}{“商品的交换和商品的分配〈工资、地租、利润〉应当分开来考察……分配的规律不完全取决于同交换有关的那些规律。”(该书序言第VI、VII页)}\end{quote}

在这里这无非是说,工资和利润的相互关系,——资本和雇佣劳动,积累劳动和直接劳动的交换,——并不直接同商品交换的规律相符合。

如果考察货币或商品作为资本的价值增殖,也就是说,不是考察它们的价值,而是考察它们的资本主义的价值增殖,那末,很明显,剩余价值无非是资本——商品或货币——所支配的劳动超过商品本身所包含的劳动量的那个余额,即无酬劳动。除了商品本身所包含的劳动量(它等于商品中的生产要素所包含的劳动与加到这些要素上的直接劳动之和)以外,商品还买到商品中所不包含的劳动余额。这个余额构成剩余价值;资本价值增殖的程度取决于这个余额的大小。商品换得的这个活劳动余额是利润的源泉。利润(确切些说,剩余价值)并不是来源于似乎与等价物即等量活劳动相交换的物化劳动,而是来源于在这个交换中没有被支付等价物而占有的那部分活劳动,来源于资本在这个虚假的交换中占有的无酬劳动。因此,如果把这一过程的中介环节撇开不谈,——由于李嘉图著作中没有这种中介环节,马尔萨斯就更有权撇开不谈,——如果只考察这一过程的实际内容和结果,那末价值增殖,利润,货币或商品之转化为资本,都不是由于商品按价值规律进行交换,即与它们所花费的劳动时间成比例地进行交换而发生的,相反倒是由于商品或货币(物化劳动)同比它所包含的或者说耗费的劳动多的活劳动相交换的结果。

马尔萨斯在上述那些著作中的唯一贡献是强调指出了这一点,而李嘉图对这一点却说得不那么清楚,因为他始终是以在资本家和工人间分配的成品为前提,却不去考察导致这一分配的中介过程——交换。可是后来这一贡献却化为乌有了,因为马尔萨斯把作为资本的货币或商品的价值增殖,因而也就是把它们在执行资本的特殊职能时的价值,同商品本身的价值混淆起来。因此,我们将会看到,他在以后的论述中又退回到货币主义的荒谬概念——让渡利润,\endnote{“让渡利润”(“Profituponexpropriation”或“Profituponalie-nation”)是詹姆斯·斯图亚特的用语。马克思在《剩余价值理论》第一册引用和分析过这个用语(见本卷第1册第12—13页)。——第8页。}完全陷入最可悲的混乱之中。这样,马尔萨斯不但没有超过李嘉图,反而在他的论述中企图使政治经济学倒退到李嘉图以前,甚至倒退到斯密和重农学派以前。

\begin{quote}{“在同一国家和同一时间,只能分解为劳动和利润的那些商品的交换价值,是由生产这些商品实际耗费的积累劳动和直接劳动,加上以劳动表示的全部预付的不断变动的利润额而得出的那个劳动量来准确衡量的。但这必然和这些商品所能支配的劳动量相同。”(《价值尺度。说明和例证》1823年伦敦版第15—16页)“商品所能支配的劳动是价值的标准尺度。”(同上,第61页)“我在任何地方都没有看到过〈指马尔萨斯自己的著作《价值尺度》出版以前〉这样的表述:某一商品通常支配的劳动量,必定可以代表并衡量生产这一商品花费的劳动量加利润。”(《政治经济学定义》1827年伦敦版第196页)}\end{quote}

和李嘉图不同,马尔萨斯先生想一下子把“利润”包括在价值规定之中,以便使利润直接从这个规定得出。由此可见,马尔萨斯感到了困难之所在。

不过,他把商品的价值和商品作为资本的价值增殖等同起来,是极其荒谬的。当商品或货币(简单说,物化劳动)作为资本同活劳动相交换时,它们所换得[755]的劳动量总是比它们本身所包含的劳动量大;如果把交换前的商品同它与活劳动交换后所得到的产品二者加以比较,就会发现,商品所换得的,是商品本身的价值(等价物)加上超过商品本身价值的余额即剩余价值。但是,如果因此说商品的价值等于它的价值加超过这个价值的余额,那是荒谬的。所以,只要商品作为商品同另一商品相交换而不是作为资本同活劳动相交换,那末,由于这里它是同等价物相交换,它所换得的物化劳动量就和它自身所包含的物化劳动量相等。

可见,值得注意的只是,马尔萨斯认为利润已经直接地现成地包括在商品的价值之中,并且有一点对他来说是清楚的,这就是:商品所支配的劳动量始终大于它所包含的劳动量。

\begin{quote}{“正因为某一商品通常所支配的劳动,等于生产这一商品实际花费的劳动加利润,所以我们有理由认为它〈劳动〉是价值的尺度。因此,如果认为商品的一般价值决定于商品供给的自然的和必要的条件,那末毫无疑问,只有它通常所能支配的劳动才是这些条件的尺度。”(《政治经济学定义》1827年伦敦版第214页)“基本生产费用恰恰是商品供给条件的等价表现。”(卡泽诺夫出版的《政治经济学定义》1853年伦敦版第14页)“商品供给条件的尺度是商品在其自然和通常状况下所能交换的劳动量。”(同上)“某一商品所支配的劳动量,正好代表生产这一商品花费的劳动量加预付资本的利润,因此,它真正代表衡量商品供给的自然和必要的条件,代表那些决定价值的基本生产费用”。(同上,第125页)“对某一商品的需求虽然同买者愿意并能够用来和它交换的另一商品的量不相适应,但是它确实同买者为该商品付出的劳动量相适应;其原因是:某一商品通常所支配的劳动量正好代表对该商品的有效需求,因为该劳动量正好代表这种商品供给所必需的劳动和利润的意量;而商品在某一时间所支配的实际劳动量如果偏离了通常的数量,那就代表由于暂时原因而引起的需求过多或需求不足。”(同上,第135页)}\end{quote}

马尔萨斯在这里也是正确的。商品供给的条件,即在资本主义生产基础上的商品生产或者更确切说再生产的条件,就在于商品或它的价值(由商品转化成的货币)在它生产或再生产过程中换得比它本所身所包含的劳动量大的劳动量;因为生产商品仅仅是为了实现利润。

例如,一个棉织厂主卖掉了他的棉布。新棉布的供给条件是,在棉布的再生产过程中,厂主以货币(即棉布的交换价值)换得比棉布原来包含的或以货币表现的劳动量大的劳动量。因为棉织厂主是作为资本家生产棉布的。他要生产的不是棉布,而是利润。生产棉布只是生产利润的手段。由此应得出什么结果呢?所生产的棉布比所预付的棉布包含更多的劳动时间,更多的劳动。这种剩余劳动时间,剩余价值,也表现为剩余产品并不补偿用来交换劳动的棉布多的棉布。因此,一部分产品并不补偿用来交换劳动的那些棉布,而构成属于厂主的剩余产品。换句话说,如果我们考察全部产品,那末,每一码棉布中都有一定部分(或者说每码棉布的价值中都有一定部分)没有被支付任何等价物,它代表无酬劳动。可见,如果厂主将一码棉布按照它的价值出卖,就是说,用它同包含等量劳动时间的货币或商品相交换,那他就是实现了一定数额的货币或得到了一定数量的商品而未付任何代价。因为他出卖棉布不是按照他支付过报酬的劳动时间,而是按照这一码棉由里所包含的劳动时间,在这个劳动时间中有一部分[756]厂主并没有支付过报酬。棉布包含的劳动时间例如等于12先令。其中厂主只支付了8先令。如果他按棉布的价值出卖,卖得12先令,他就赚了4先令。

\tchapternonum{[(2)马尔萨斯所解释的庸俗的“让渡利润”见解。马尔萨斯对剩余价值的荒谬观点]}

至于说到买者,那末,根据假定,他在任何情况下都只支付棉布的价值,也就是说,他付出的货币额所包含的劳动时间和棉布所包含的一样多。这里可能有三种情况。(1)买者是资本家。他用来支付棉布的货币(即商品的价值)也包含一部分无酬劳动。因此,如果说一个出卖无酬劳动,那末另一个则用无酬劳动来购买。他们各自实现了无酬劳动,一个以卖者的身分实现,另一个以买者的身分实现。(2)或者买者是独立生产者。在这种情况下,他以等价物换取等价物。卖者以商品形式卖给他的劳动是否支付过报酬,与他根本无关。他得到的物化劳动和他付出的一样多。(3)最后,或者买者是雇佣工人。在这种情况下——假定商品按它的价值出卖,——他也和其他任何买者一样,用他的货币买得商品形式的等价物。他所得到的商品形式的物化劳动和他以货币形式付出的一样多。但是,他为换取构成他的工资的货币付出的劳动却比这些货币中包含的劳动要多。他补偿了货币中包含的劳动,还加上了他无偿地付出的剩余劳动。因此,他为货币支付的代价,超过了货币的价值,从而他为货币等价物(棉布等)支付的代价,也超过这种等价物的价值。可见,对他这个买者来说,费用要比对任何一个商品的卖者来说都大,尽管他在商品形式上为自己的货币取得等价物;但是,在货币形式上他却没有为他的劳动取得等价物,相反,他在劳动中付出的比等价物多。可见,工人是唯一高于商品价值来支付一切商品的买者,甚至在他按照商品价值购买商品时也是这样,因为他用超过货币价值的劳动量购买了一般等价物货币。对于卖商品给工人的人来说并没有因此得到任何好处。工人付给他的并不比其他任何买者付给的多,工人支付的是劳动创造的价值。资本家把工人生产的商品又卖给工人,他的确通过这种出卖实现了利润,但这只不过是他把商品卖给其他任何买者时所实现的那种利润。资本家把商品卖给这个工人时所得的利润,其来源并不是他高于商品价值把商品卖给工人,而是在此以前,事实上是在生产过程中,他低于商品的价值向工人购买了商品。

所以,马尔萨斯先生既然把商品作为资本的价值增殖变成商品的价值,也就前后一贯地把所有买者都变成雇佣工人,也就是说,他硬使所有买者不是用商品,而是用直接劳动同资本家相交换,硬使他们交回给资本家的劳动多于商品中包含的劳动,可是实际上,资本家的利润的产生却是由于他出卖的是商品中包含的全部劳动,而已经支付的只是商品中包含的一部分劳动。因此,如果说李嘉图的困难在于,商品交换规律无法直接解释资本和雇佣劳动之间的交换,反而似乎与这一交换相矛盾,那末,马尔萨斯却用把商品的购买(交换)变成资本和雇佣劳动之间的交换这样一个办法来解决这个困难。马尔萨斯所不理解的就是商品中包含的劳动总量和商品中包含的有酬劳动量之间的差额。正是这个差额构成利润的源泉。而马尔萨斯下一步就不可避免地这样得出利润:卖者出卖商品不仅高于他为商品所花费的(资本家正是这样做的),而且高于商品所值,这就是说,马尔萨斯回到“让渡利润”的庸俗观点,即认为剩余价值的产生是由于卖者高于商品价值出卖商品(也就是换得比商品中包含的劳动时间多的劳动时间)。这样一来,某人作为某一商品的卖者所赚得的,也就是他作为另一商品的买者所亏损的,因而完全不能理解,通过价格的这种普遍的名义上的提高,会有什么实际的“赢利”。[757]尤其不可理解的是,整个社会怎能由此而致富,真正的剩余价值或真正的剩余产品怎能由此而形成。这真是荒唐而愚蠢的见解。

我们已经看到\fnote{见本卷第1册第3章和第4章。——编者注},亚·斯密曾素朴地表述了一切相互矛盾的因素,因而他的学说成了截然相反的各种观点的源泉和出发点。马尔萨斯先生以亚·斯密的论点为依据,作了一种混乱的、然而是建立在正确地感觉和意识到有待克服的困难的基础上的尝试,企图用一种新的理论与李嘉图的理论相对抗,从而保持其“首席地位”。从这种尝试到荒谬的庸俗观点的过渡是这样实现的:

如果我们考察商品作为资本的价值增殖,即考察商品与活的生产劳动相交换,那末,商品除了它本身包含的劳动时间——即工人所再生产的等价物——之外,还支配形成利润源泉的剩余劳动时间。如果我们现在把商品的这种价值增殖变为商品的价值,那末商品的每一个买者都必须作为工人与商品发生关系,就是说,他在购买时除了商品中包含的劳动量之外,还要另外再付出一定数量的剩余劳动。既然除了工人以外其他买者都不是作为工人与商品发生关系{我们已经看到,即使工人单纯作为商品买者出现时,先前的、原有的差别仍然间接地存在},那末就必须假定:买者虽然不直接付出比商品中包含的更多的劳动量,但是——这其实是一回事——要付出一个包含更多劳动量的价值。上述的过渡就是靠这种“更多的劳动量或者说——这其实是一回事——包含更多劳动量的价值”实现的。总之,问题实?上可归结成这样:商品的价值就是买者为商品支付的价值,这个价值等于商品的等价物(价值)加超过这个价值的余额,即剩余价值。于是就得出这样一个庸俗观点:利润在于商品贱买贵卖。买者购买商品所花费的劳动或物化劳动多于卖者为商品所花费的。

但是,如果买者本身是资本家,是商品的卖者,而且他用来购买的货币只代表他所卖出的商品,结果就只能是:双方都过贵地出卖自己的商品,从而相互欺诈,而且只要双方都仅仅实现一般利润率,欺诈的程度也就相同。那末,应该到哪里去找付给资本家的劳动量等于资本家的商品中包含的劳动加资本家的利润的买者呢?举个例子。卖者为商品花费了10先令。他把商品卖了12先令。这样,他支配的劳动不只是10先令,而且多了2先令。但是买者同样把他值10先令的商品卖了12先令。这样,他们各自作为买者所亏损的正是他们作为卖者所赚得的。工人阶级是唯一的例外。因为,既然产品价格被提高到它的费用之上,工人就只能买回产品的一部分,这样,产品的另一部分,或该部分的价格,就构成资本家的利润。但是,既然利润恰恰是由于工人只能买回产品的一部分而获得的,那末资本家(资本家阶级)就决不能[仅仅]靠工人的需求来实现自己的利润,决不能靠全部产品同工资相交换来实现利润,相反只能靠工人的全部工资同仅仅一部分产品相交换来实现利润。可见,除了工人以外,还必须有其他需求和其他买者,否则就不会产生任何利润。这些买者从哪里来呢?如果他们本身是资本家,是卖者,那就会发生上述的资本家阶级的自相欺诈,因为他们互相在名义上提高他们商品的价格,他们各自作为卖者所赚得的正是他们作为买者所亏损的。因此,必须有不是卖者的买者,资本家才能实现他的利润,才能“按照商品的价值”出卖商品。所以就必须有地主、年金领取者、领干薪者、牧师等等以及他们的家仆和侍从。至于这些“买者”怎样占有[758]购买手段,也就是说,他们怎样必须不付等价物而先从资本家那里取得一部分产品,以便用这样取得的东西买回少于它的等价物的商品,马尔萨斯先生没有加以说明。不管怎样,由此就产生了他的为下述主张辩护的论据:尽可能多地增加非生产阶级,好让商品的卖者找到市场,为自己的供给找到需求。这样,再接下去,人口论小册子\endnote{这里指的是马尔萨斯的有名著作《人口原理》,该书第一版是1798年在伦敦匿名出版的。在这一著作中,马尔萨斯断言,劳动群众的贫困似乎是由于人口有按几何级数增加的趋势,而消费品的数量最多只能按算术级数增加。——第15页。}的作者就鼓吹,经常的消费过度和寄生者占有尽可能多的年产品是生产的条件。除了从他的理论必然产生的这一论据之外,还有一个进一步的辩护论据:资本代表对抽象财富的欲望,对价值增殖的欲望,但是,这种欲望只是由于有代表支出欲望、消费欲望、奢侈欲望的购买者阶级存在,也就是说,有那些是买者而不是卖者的非生产阶级存在才能实现。

\tchapternonum{[(3)十九世纪二十年代马尔萨斯主义者和李嘉图主义者之间的争论。他们在对待工人阶级的立场方面的共同点]}

在这个基础上,在二十年代(从1820年到1830年这段时间,总的说来,是英国政治经济学史上一个大的形而上学的时代),马尔萨斯主义者和李嘉图主义者之间发生了一场绝妙的争吵。李嘉图主义者象马尔萨斯主义者一样地认为,必须使工人自己不占有自己的产品,这个产品的一部分要归资本家所有,以便使他们(即工人)有生产的刺激,从而保证财富的增长。但是李嘉图主义者激烈地反对马尔萨斯主义者的下述观点:地主,国家和教会的领干薪者,以及一大帮游手好闲的仆从,必须首先占有资本家的一部分产品而不付任何等价物(正象资本家对工人那样),然后从资本家那里购买资本家自己的商品,并为他们提供利润,——虽然李嘉图主义者对于工人却持同样的主张。按照李嘉图主义者的学说,为了使积累从而也使对劳动的需求增加,工人必须把自己产品中尽可能大的一部分无偿地让给资本家,以便资本家把由此增加的纯收入再转化为资本。马尔萨斯主义者也是这样论证的。按照他们的意见,应该以地租、税收等等形式从产业资本家那里无偿地索取尽可能大的一部分,以便他们能把余下的部分卖给这些强加给他们的“分享者”而获得利润。工人不应占有自己的产品,这样才不致丧失劳动的刺激——李嘉图主义者和马尔萨斯主义者都这样说。[马尔萨斯主义者说]产业资本家必须把他的一部分产品让给只从事消费的阶级——“为享受果实而生的人们”\fnote{见贺雷西《书信集》。——编者注},——好让这些阶级再拿产业资本家让给他们的东西在对他们不利的条件下和产业资本家进行交换。否则资本家就要丧失生产的刺激,而这种刺激恰恰在于,资本家取得高额利润,大大高于商品价值出卖商品。以后我们还要回过来再谈这场滑稽的论战。

\tchapternonum{[(4)马尔萨斯片面地解释斯密的价值理论。他在同李嘉图论战中利用斯密的错误论点]}

现在我们首先来证明:马尔萨斯陷入了一种十分粗俗的观念。

\begin{quote}{“无论商品要通过多少次中间交换,无论生产者是把他们的商品运到中国,还是就在产地出卖,商品能否取得适当的市场价格的问题,完全取决于生产者能否补偿他们的资本并取得普通利润,从而能够顺利地继续他们的营业。但他们的资本是什么呢?正如亚·斯密指出的,资本是用来劳动的工具、被加工的材料、以及支配必要劳动量的手段。”}\end{quote}

(马尔萨斯认为,这也就是用在商品生产上的全部劳动。利润是超过这种用在商品生产上的劳动的余额。因而事实上这只不过是商品生产费用上的名义附加额。)为了使人们对他的看法不留下任何疑问,马尔萨斯还以赞同的态度引用了托伦斯上校的《论财富的生产》一书(第6章第349页)来证实他自己的观点。

\begin{quote}{“有效的需求在于,消费者{买者和卖者之间的对立在这里变成消费者和生产者之间的对立}[759]通过直接的或间接的交换能够和愿意付给商品的部分,大于生产它们时所耗费的资本的一切组成部分。”(卡泽诺夫出版的《政治经济学定义》第70—71页)}\end{quote}

而马尔萨斯的《定义》一书的出版者、辩护者和注释者卡泽诺夫先生自己则说:

\begin{quote}{“利润不取决于商品互相交换的比例}\end{quote}

{这就是说,如果考察的只是资本家之间的商品交换,那末,由于这里不存在资本家同工人之间的交换(工人除劳动之外没有任何其他商品可以交换),马尔萨斯的理论就会表现为这样一种谬论,即彼此单纯给商品价格加上一个附加额,一个名义上的附加额。因此,不得不撇开商品交换,而谈不生产任何商品的人之间的货币交换},

\begin{quote}{因为同一比例在任何利润高度上都能存在,而取决于对工资的比例,或者说对抵补原有费用所需的比例,这个比例在任何情况下都决定于买者为取得商品而作出的牺牲(或他付出的劳动的价值)超过生产者为使商品进入市场而作出的牺牲的程度。”(同上,第46页)}\end{quote}

为了获得这样奇妙的结果,马尔萨斯必须在理论上大耍花招。首先,在抓住亚·斯密学说的一个方面,即商品的价值等于商品支配的劳动量,或支配商品的劳动量,或商品交换的劳动量这一主张的同时,必须消除亚·斯密本人以及其后的经济学家,其中包括马尔萨斯,对商品的价值——价值——可以成为价值尺度这一论点所提出的异议。

马尔萨斯的《价值尺度。说明和例证》(1823年伦敦版)一书是愚蠢的真正典型,它用诡辩来自我陶醉,在自己内在的概念混乱中辗转迂回;它的晦涩、拙劣的叙述,给天真的、不内行的读者留下这样一个印象:如果读者弄不清楚这一团混乱,那末其困难不在于混乱与清楚之间的矛盾,而在于读者的理解力太差。

马尔萨斯首先必须把大卫·李嘉图在“劳动的价值”和“劳动量”之间所作的划分\endnote{关于李嘉图的“劳动的价值”和“劳动量”的概念,见本卷第2册第449—459页。——第18页。}重新抹掉,并把斯密的[不同价值规定的]并列归结到一个错误方面。

\begin{quote}{“一定的劳动量,必定具有同支配它或者它实际上交换的工资相等的价值。”(《价值尺度。说明和例证》1823年伦敦版第5页)}\end{quote}

这句话的目的就是把劳动量和劳动的价值这两个用语等同起来。

这句话本身纯粹是同义反复,是荒谬的陈辞滥调。既然工资或者说“它〈一定的劳动量〉所交换的”东西构成这个劳动量的价值,那末,说一定的劳动量的价值等于工资或等于这个劳动所交换的货币量或商品量,就是同义反复。换句话说,这不过意味着,一定的劳动量的交换价值等于这一劳动量的交换价值,或者叫作工资。但是,{且不说直接同工资相交换的不是劳动,而是劳动能力,正是这个混淆造成了谬误},决不能从上述同义反复中得出这样的结论:一定的劳动量等于工资中或者说构成工资的货币或商品中包含的劳动量。假定一个工人劳动12小时,得到6小时的产品作为工资,那末这6小时的劳动产品就构成12小时劳动的价值(因为它是用来换取12小时劳动的工资,商品)。不能由此推论说,6小时劳动等于12小时,或者代表6小时的商品等于代表12小时的商品;也不能说,工资的价值等于代表[同该工资相交换的]劳动的产品的价值。由此只能得出结论说,劳动的价值(因为它是用劳动能力的价值,而不是用劳动能力所完成的劳动来衡量的)、[760]一定劳动量的价值所包含的劳动,少于它所买到的劳动;因此,代表所买到的劳动的那些商品的价值和用来购买或支配这一定劳动量的那些商品的价值,是大不相同的。

马尔萨斯先生得出了直接相反的结论。因为一定劳动量的价值等于它的价值,按照马尔萨斯的意见,就可以得出结论:代表这个劳动量的价值等于工资的价值。按照马尔萨斯的意见,由此还可以得出结论:商品所吸收和包含的直接劳动(即扣除生产资料后剩下的劳动)创造的价值并不比为它支付的价值大;它只再生产工资的价值。所以不言而喻,如果商品的价值由商品所包含的劳动决定,利润就无法解释,而必须用别的源泉解释利润,——如果已经假定商品的价值必须包括它所实现的利润。因为用在商品生产上的劳动包括:(1)被磨损的因而再现于产品价值中的机器等等所包含的劳动;(2)使用的原料所包含的劳动。这两个要素自然不会因为它们成为新商品的生产要素而使它们在新商品生产前本来包含的劳动量有所增加。于是,剩下的是(3)包含在工资中的、与活劳动相交换的劳动。但是,按照马尔萨斯的意见,这种活劳动并不比它所交换的物化劳动多。因此,商品不包含任何无酬劳动部分,只包含补偿等价物的劳动。由此可以得出如下的结论:如果商品的价值由商品中所包含的劳动决定,它就不提供任何利润。如果它提供利润,那末,按照马尔萨斯的意见,这就是商品价格超过商品中所包含的劳动的余额。因而,为了使商品按照它的价值(包括利润在内的价值)出卖,商品必须支配这样一个劳动量,它等于用在商品生产上的劳动加上一个代表商品出卖时所实现的利润的劳动余额。

\tchapternonum{[(5)马尔萨斯对斯密关于不变的劳动价值这一论点的解释]}

其次,马尔萨斯为了证明劳动——不是生产所需要的劳动量,而是作为商品的劳动——是价值的尺度,他断言:

\begin{quote}{“劳动的价值是不变的。”(《价值尺度。说明和例证》第29页注)}\end{quote}

{这种说法决不是什么创见,而是亚·斯密《国富论》第1卷第5章(加尔涅的法译本,第1卷第65—66页)中下述论点的改写和进一步发挥:

\begin{quote}{“等量劳动,在任何时候和任何地方,对于完成这一劳动的工人必定具有相同的价值。在通常的健康、体力和精神状况下,在工人能够掌握通常的技能和技巧的条件下,他总要牺牲同样多的休息、自由和幸福。他所支付的价格总是不变的,不管他用这一价格换得的商品量有多少。诚然他用这个价格能买到的这些商品的量有时多有时少,但这里发生变化的是这些商品的价值,而不是购买商品的劳动的价值。在任何时候和任何地方,难于得到或者说要花费许多劳动才能得到的东西总是贵的,而容易得到或者说花费不多的劳动就能得到的东西总是便宜的。由此可见,劳动本身的价值永远不变,所以劳动是唯一真实的和最终的尺度,在任何时候和任何地方都可以用这个尺度来衡量和比较一切商品的价值。”}}\end{quote}

{其次,马尔萨斯如此引为骄傲的并扬言是他最早提出的一个发现(即价值等于商品中所包含的劳动量加代表利润的劳动量的论点),看来也只不过是把斯密以下两句话拼凑在一起(马尔萨斯始终不失为一个剽窃者):

\begin{quote}{“价格的各个不同构成部分的实际价值,是以每一构成部分所能购买或支配的劳动量来衡量的。劳动不仅衡量价格中归结为劳动的部分的价值,而且还衡量归结为地租的部分和归结为利润的部分的价值。”(第1卷第6章,加尔涅的译本,第1卷第100页)}}\end{quote}

[761]根据这一点马尔萨斯说:

\begin{quote}{“如果对劳动的需求增加了,那末,工人的较高工资就不是由劳动价值的提高,而是由劳动所交换的产品的价值的降低引起的。在劳动过剩的情况下,工人的低工资是由产品价值的提高,而不是由劳动价值的降低引起的。”(《价值尺度。说明和例证》第35页,并参看第33—34页)}\end{quote}

贝利很好地嘲笑了马尔萨斯对于劳动价值不变的论证(指马尔萨斯的进一步论证,而不是指斯密的论点;也指一般关于不变的劳动价值的论点):

\begin{quote}{“我们可以用同样的方法证明任何物品都具有不变的价值;可以以10码呢绒为例。因为不管我们对这10码呢绒付出5镑还是10镑,付出的金额在价值上总是等于用这笔钱换得的这块呢绒,或者换句话说,这个金额对这块呢绒来说具有不变的价值。但是,用来换取具有不变价值的物的东西,本身必须是不变的;所以这10码呢绒必须具有不变的价值……如果说,工资虽然在数量上有变化,但支配的劳动量不变,因此具有不变的价值,那末这种说法正同所谓买帽子付出的金额虽然时多时少,但总是买到一顶帽子,因此它具有不变的价值这种说法一样不足取。”(《对价值的本质、尺度和原因的批判研究,主要是论李嘉图先生及其信徒的著作》1825年伦敦版第145—147页)}\end{quote}

贝利在这同一本著作中,非常尖刻地嘲笑了马尔萨斯用来“说明”他的价值尺度的那些荒谬的、自以为高明的计算表格。

马尔萨斯在他的《政治经济学定义》(1827年伦敦版)中对贝利的讥讽大发雷霆,同时他试图这样来证明劳动价值不变:

\begin{quote}{“随着社会的进步,许多商品,如原产品,和劳动相比,价格上涨,而工业品的价格却下降。因此,差不多可以这样说:一定的劳动量在同一国家中支配的商品量,平均说来,在几百年的过程内不可能发生重大的变化。”(《定义》1827年伦敦版第206页)}\end{quote}

马尔萨斯还象证明“劳动价值不变”一样绝妙地证明:工资的货币价格的提高,必然引起商品的货币价格的普遍提高。

\begin{quote}{“如果货币工资普遍提高,货币的价值将相应地下降;而当货币的价值下降时……商品的价格总是上涨。”(同上,第34页)}\end{quote}

如果货币的价值同劳动相比降低了,那末恰恰需要证明:所有商品的价值同货币相比提高了,或者说,不是用劳动而是用其他商品计算的货币价值降低了。而马尔萨斯证明这一点的办法,却是事先就把这一点当作前提。

\tchapternonum{[(6)马尔萨斯利用李嘉图关于价值规律的变形的论点反对劳动价值论]}

马尔萨斯反对李嘉图的价值规定所持的论据,完全来自李嘉图本人最先提出的一个论点,即认为商品的交换价值的变动与生产商品时所花费的劳动量无关,它是因流通过程中产生的资本构成上的区别——流动资本和固定资本的比例不同,所用固定资本的耐久程度不同,流动资本的周转时间不同——引起的。简言之,马尔萨斯反对李嘉图所持的论据,来自李嘉图的费用价格和价值的混淆,因为李嘉图把不依各个生产领域使用的劳动量为转移的费用价格的平均化看作是价值本身的变形,从而把整个原理推翻了。马尔萨斯抓住李嘉图本人所强调的并且是他最先发现的那些违反价值决定于劳动时间这个规定的矛盾,不是为了解决矛盾,而是为了倒退到完全荒谬的观念上去,为了把说出互相矛盾的现象即用语言把这些现象表达出来,当成解决矛盾。我们在考察李嘉图学派的解体时,还会看到[詹姆斯·]穆勒和麦克库洛赫也使用了同样手法。他们试图用烦琐的荒谬的定义和区分,把与普遍规律相矛盾的现象胡说成直接同普遍规律一致,以便在自己的议论中避开这些现象,不过这样一来,基础本身也就不存在了。

在下面引用的马尔萨斯著作中的一段话里,马尔萨斯利用了李嘉图本人提供的违反价值规律的材料来反对李嘉图:

\begin{quote}{“斯密说过,谷物一年就可成熟,而肉用牲畜却需要喂养4—5年才能屠宰;因此,如果我们拿交换价值相等的一定数量的谷物和一定数量的肉相比较,那就可以肯定,不考虑其他因素,单是多出的3年或4年的利润(按生产肉类使用的资本15%计算)的差额,就会使一个少得多的劳动量[762]在价值上得到补偿。可见,两个商品的交换价值可以相等,而一个商品中的积累劳动和直接劳动却比另一个少40%或50%。对任何一个国家的大量最重要的商品来说,这是常见的事情;如果利润从15%降到8%,肉的价值和谷物相比就会降低20%以上。”(《价值尺度。说明和例证》第10—11页)}\end{quote}

既然资本由商品构成,并且加入资本或构成资本的商品有很大一部分具有这样一种价格(也就是普通意义上的交换价值),这种价格不仅包括积累劳动和直接劳动,而且——就我们考察的只是这种特殊商品来说,——还包括一个因加上平均利润而形成的纯粹名义上的价值附加额,所以马尔萨斯说:

\begin{quote}{“劳动不是用于生产资本的唯一要素。”(卡泽诺夫出版的《定义》第29页)“什么是生产费用呢?……就是生产商品所需要的和生产商品时消费的工具和材料中所包含的实物形式的劳动量,加上一个相当于预付资本在整个预付期间的普通利润的附加量。”(同上,第74—75页)“根据同样的理由,穆勒先生把资本叫作积累劳动是非常错误的。人们也许可以把资本叫作积累劳动加利润,但肯定不能单单叫作积累劳动,除非我们决定把利润叫作劳动。”(同上,第60—61页)“说商品的价值由生产商品所必需的劳动量和资本量来调节或决定,是完全错误的。说商品的价值由生产商品所必需的劳动量和利润量来调节,是完全正确的。”(同上,第129页)}\end{quote}

关于这一点,卡泽诺夫在第130页的注释中说:

\begin{quote}{“‘劳动和利润’的说法可能遭到这样的反驳,说这两者不是互相关连的概念,因为劳动是动因,利润是效果,一个是因,一个是果。因此,西尼耳先生用‘劳动和节欲’的说法取而代之〈按西尼耳的说法是:“谁把自己的收入转化为资本,谁就是节制了如果花费这笔资本就能获得的享受”〉……但是必须承认,利润的原因不在于节欲,而在于生产地使用资本。”}\end{quote}

绝妙的解释!商品的价值由包含在商品中的劳动加利润构成;由包含在商品中的劳动和不包含在商品中但购买商品时必须支付的劳动构成。

马尔萨斯继续反驳李嘉图说:

\begin{quote}{“李嘉图断言,利润随着工资价值的提高而按比例地下降,反之亦然。这种说法只有假定在其生产上耗费相等劳动量的商品始终具有相等价值的条件下才是正确的。而这种假定在五百次里难得有一次可以成立,而且必然如此,因为随着文明和技术的进步,使用的固定资本量不断增加,流动资本的周转时间则越来越不相同和不相等。”(《定义》1827年伦敦版第31—32页)〈在卡泽诺夫的版本第53—54页上,马尔萨斯的这段话,在文字上完全一样:“事物的自然状态”使李嘉图的价值尺度变了样,因为这种状态造成一种趋势:“随着文明和技术的进步,使用的固定资本量不断增加,流动资本的周转时间则越来越不相同和不相等。”〉“李嘉图先生自己也承认他的规则有相当多的例外;但是如果我们考察一下这些他所谓的例外的情况,即使用的固定资本量不同,耐久程度不同,使用的流动资本周转时间不同,那末我们就会发现,这些例外情况如此之多,以致规则可以看作例外,而例外可以看作规则。”(卡泽诺夫出版的《定义》第50页)}\end{quote}

\tchapternonum{[(7)马尔萨斯的庸俗的价值规定。把利润看成商品价值附加额。马尔萨斯对李嘉图相对工资见解的反驳]}

根据上面所说,马尔萨斯还提出了这样一个价值规定\endnote{这个价值规定是卡泽诺夫根据马尔萨斯和亚当·斯密的意见表述的,而马尔萨斯是从亚当·斯密那里借用了商品的价值决定于用这个商品可以买到的活劳动量这一规定。——第25页。}:

\begin{quote}{“价值是对商品的估价,这种估价的根据是买者为商品付出的费用,或者说买者为了得到它而必须作出的牺牲,这种牺牲用他为交换这一商品而付出的劳动量来衡量,或者也可以说用这一商品所支配的劳动来衡量。”(同上,第8—9页)}\end{quote}

卡泽诺夫还指出了马尔萨斯和李嘉图的区别:

\begin{quote}{[763]“李嘉图先生同亚·斯密一起,把劳动当作费用的真正尺度;但是他只是用它来衡量生产者的费用……它同样可以用作买者的费用的尺度。”(同上,第56—57页)}\end{quote}

换句话说:商品的价值等于买者所必须支付的货币额,这一货币额可以最准确地用它所能购买的普通劳动量来估量\fnote{马尔萨斯先假定利润的存在,然后就可以用一个外在尺度来衡量它的价值量。他没有涉及利润的产生和内在可能性的问题。}。但这一货币额又由什么决定,这一点当然没有说明。我们这里看到的是日常生活中人们对这种事情的十分粗俗的观念。用莫测高深的语言来表达的不过是肤浅的见解。换句话说,这无非是把费用价格和价值等同起来,——这种混同,在亚·斯密著作中,尤其是在李嘉图著作中是和他们的实际分析相矛盾的,而马尔萨斯却把它奉为规律。因此,这是沉湎于竞争、只看到竞争造成的表面现象的市侩所特有的价值观。费用价格究竟是由什么决定的呢?由预付资本的量加利润决定。而利润又是由什么决定的呢?利润的基金是从哪里来的呢?代表这一剩余价值的剩余产品是从哪里来的呢?如果问题只在于名义上提高货币价格,那末提高商品的价值是最容易的事了。预付资本的价值又由什么决定呢?马尔萨斯说,是由预付资本中包含的劳动的价值决定的。劳动的价值又由什么决定呢?是由花费工资购买的商品的价值决定的。而这些商品的价值又由什么决定呢?由劳动的价值加利润。这样,我们只好不断地在循环论证里兜圈子。假定付给工人的真是他的劳动的价值,也就是说,构成他的工资的那些商品(或货币额)等于他的劳动物化在其中的商品的价值(货币额),那末,他要是得到100塔勒工资,他加到原料等等上面的,简言之,加到预付[不变]资本上面的总共也就是100塔勒的价值。在这种情况下,利润无论如何只能由卖者在出卖商品时加在商品的实际价值上的附加额构成。所有的卖者都这样做。因此,只要是资本家彼此交换商品,那就谁也不能通过这种附加额实现任何东西,根本不能通过这种方法形成一个可供他们从中汲取收入的剩余基金。只有那些生产加入工人阶级的消费的商品的资本家,才能获得一个实际的、而不是虚构的利润,因为他们卖回给工人的商品比他们向工人购买的商品贵。他们用100塔勒从工人那里购买来的商品,又以110塔勒卖回给工人,也就是说,他们只把产品的10/11卖回给工人,而把1/11留给自己。但这仅仅意味着,例如工人做工11小时,只给他10小时的报酬,只给他10小时的产品,而1小时或者说1小时的产品,无代价地归资本家所有。而这也就意味着——就这里是同工人阶级发生关系而言——利润的产生是由于工人阶级把自己劳动的一部分白白送给资本家,因而“劳动量”和“劳动价值”不相等。但其他资本家却不会有这种出路,因而只能获得虚构的利润。

此外,下面这一段话令人信服地表明马尔萨斯多么不理解李嘉图的基本论点,他根本不懂得利润能够不通过价值附加额而通过别的办法获得。

\begin{quote}{“可以承认,直接制成并直接供人使用的最初商品是纯粹劳动的结果,因而它们的价值由这一劳动的量决定;但是这种商品作为资本来帮助生产其他商品时,资本家在一定时期内就必然不能使用他的预付,因之也就必然要求以利润形式取得报酬。在社会发展的早期阶段,用于生产商品的预付资本比较小,这种报酬是很高的,而且由于利润率高,这些商品的价值受到相当大的影响。在社会进一步发展的阶段,由于使用的固定资本量大大增加,由于很大一部分流动资本在资本家从卖得之款中得到补偿前的预付期加长,利润对资本和商品的价值也发生很大的影响。在这两种情况下,商品互相交换的比例都会受到不断变动的利润量的重大影响。”(卡泽诺夫出版的《定义》第60页)}\end{quote}

确立相对工资的概念是李嘉图的最大功绩之一。其要点就是:工资的价值(因而还有利润的价值)完全取决于工作日中工人为他自己劳动(为了生产或再生产他的工资)的那一部分和归资本家所有的那一部分劳动时间的比例。这一点在经济学上非常重要,事实上这只是对正确的剩余价值理论的另一种表达\endnote{关于李嘉图的“相对工资”的观念,见本卷第2册第475—483页。——第28页。}。其次,这一点对理解两个[764]阶级的社会关系是很重要的。马尔萨斯在这里嗅到了一些不大对头的味道,因而不得不提出这样的异议:

\begin{quote}{“在李嘉图先生以前我还没有见到有哪个著作家曾在比例的意义上使用工资或者实际工资这个术语。”}\end{quote}

(李嘉图谈的是工资的价值,它实际上也就表现为属于工人的那一部分产品\endnote{关于李嘉图的“实际工资”(“realwages”)的概念,见本卷第2册第456—457、459—460、474、482、497和636页。——第28、241页。}。)

\begin{quote}{“利润确实是指一种比例;而利润率始终被正确地表达为对预付资本的价值的百分比。”}\end{quote}

{要说清楚马尔萨斯所谓的预付资本的价值是指什么,那是很困难的,而要他本人说清楚甚至是不可能的。照马尔萨斯的说法,商品的价值等于商品中包含的预付资本加利润。既然预付资本中除了直接劳动外,还包括商品,所以预付资本的价值等于预付资本中包含的预付资本加利润。于是利润就等于预付资本的利润加利润的利润。如此等等,以至无穷。}

\begin{quote}{“至于工资,我们在考察它的增减时从来不是根据它对通过一定劳动量获得的全部产品的比例,而是根据工人所取得的某种产品量的多少,或者说根据这些产品支配必需品和舒适品的能力大小。”(《定义》1827年伦敦版第29—30页)}\end{quote}

因为在资本主义生产条件下,交换价值——交换价值的增殖——是直接目的,所以弄清怎样衡量交换价值是很重要的。由于预付资本的价值是用货币(实在货币或计算货币)表示的,所以这种增殖的幅度用资本本身的货币量来衡量,并以一定数量——100——的资本(货币额)作为标准。

\begin{quote}{马尔萨斯说:“资本的利润就是预付资本的价值和商品在出卖或被消费时所具有的价值之间的差额。”(《定义》1827年伦敦版第240—241页)}\end{quote}

\tchapternonum{[(8)马尔萨斯的生产劳动和积累的观点同他的人口论相抵触]}

\tsectionnonum{[(a)]生产劳动和非生产劳动}

\begin{quote}{“收入是用来直接维持生活和取得享受的,而资本是用来取得利润的。”(《定义》1827年伦敦版第86页)“工人和家仆是用于完全不同目的的两种工具,前者帮助获得财富,后者帮助消费财富。”\endnote{马尔萨斯的这一段话几乎是逐字重复亚当·斯密的论述。《剩余价值理论》第一册引用过亚当·斯密的这一论述(见本卷第1册第146页):“……制造业工人的劳动,通常把自己的生活费的价值和他的主人的利润,加到他所加工的材料的价值上。相反,家仆的劳动不能使价值有任何增加……一个人,要是雇用许多制造业工人,就会变富;要是维持许多家仆,就会变穷。”马克思用斯密所特有的术语“生产劳动和非生产劳动”作为这一节的标题,暗示马尔萨斯的这一观点是从斯密那里借用来的。——第29页。}(同上,第94页)}\end{quote}

下面这个对生产工人的定义倒是不错的:

\begin{quote}{“生产工人就是直接增加自己主人的财富的工人。”(《政治经济学原理》[第2版]第47页[注])}\end{quote}

此外,还可以引用下面一段话:

\begin{quote}{“唯一真正的生产消费,就是资本家为了再生产而对财富的消费和破坏……资本家使用的工人,自然把他不积蓄的那部分工资,作为用于维持生活和取得享受的收入来消费,而不是作为用于生产的资本来消费。他对于使用他的人、对于国家是生产的消费者,但严格说来,对自己本身就不是生产的消费者。”(卡泽诺夫出版的《定义》第30页)}\end{quote}

\tsectionnonum{[(b)]积累}

\begin{quote}{“现代任何政治经济学家都不能把积蓄看作只是货币贮藏;撇开这种做法的狭隘和无效不说,积蓄这个名词在涉及国民财富方面只能设想有一个用法,这个用法是从积蓄的不同用途中产生并以积蓄所维持的各种不同劳动的实际差别为基础的。”(《政治经济学原理》[第2版]第38—39页)“资本积累就是把收入的一部分当作资本使用。因此,现有的财产或财富不增加,资本也可能增加。”(卡泽诺夫出版的《定义》第11页)“在一个主要依靠工商业的国家里,如果在工人阶级中间盛行慎重地对待结婚的习惯,那对国家是有害的。”(《政治经济学原理》[第2版]第215页)}\end{quote}

这种话竟出自鼓吹制止人口过剩的人之口!

\begin{quote}{“生活必需品的缺乏,是刺激工人阶级生产奢侈品的主要原因;如果这个刺激消除或者大大削弱,以致花费很少劳动就能够获得生活必需品,那末我们就有充分理由认为,用来生产舒适品的时间将不会更多,而只会更少。”(《政治经济学原理》[第2版]第334页)}\end{quote}

但是,对于这个人口过剩论的说教者来说,最重要的是这样一段话:

\begin{quote}{“按人口的性质来说,即使遇到特殊需求,不经过16年或18年的时间,也不可能向市场供应追加工人。然而,收入通过节约转化为资本却可以快得多,一个国家用来维持劳动的基金比人口增长得快的情况,是经常有的。”(同上,第319—320页)}\end{quote}

[765]卡译诺夫正确地指出:

\begin{quote}{“当资本用于预付给工人的工资时,它丝毫不增加用来维持劳动的基金,而只不过是把这种已经存在的基金的一定部分用于生产的目的。”(卡泽诺夫出版的《政治经济学定义》第22页注)}\end{quote}

\tchapternonum{[(9)][马尔萨斯所理解的]不变资本和可变资本}

\begin{quote}{“积累劳动〈其实应当称为物质化劳动、物化劳动〉是花费在生产其他商品时使用的原料和工具上的劳动。”(卡泽诺夫出版的《政治经济学定义》第13页)“在谈到生产商品所花费的劳动时,应当把花费在生产商品所需的资本上的劳动称为积累劳动,以区别于最后的[即在生产商品最后阶段的]资本家所使用的直接劳动。”(同上,第28—29页)}\end{quote}

指出这种区别当然很重要。但是在马尔萨斯那里,这种区别却没有导致任何成果。

马尔萨斯试图把剩余价值或至少是剩余价值率(不过他总是把它们同利润和利润率混为一谈)解释为对可变资本之比,即对用在直接劳动上的那部分资本之比。但是在马尔萨斯那里,这一尝试是十分幼稚的,而且按照他的价值观,也只能是这样。他在他的《政治经济学原理》[第2版]中说:

\begin{quote}{“假定资本只用在工资上。如果100镑用在直接劳动上,年终收回110、120或130镑,显然,在任何一种情况下,利润决定于总产品价值中用来支付所使用的劳动的份额。如果在市场上产品的价值是110,那末用来支付工人的份额是产品价值的10/11,而利润就是10%。如果产品价值的120,那末支付劳动的份额是10/12,而利润是20%;如果产品价值是130,那末必须用来支付预付劳动的份额是10/13,而利润是30%。现在假定资本家预付的资本不单由劳动构成。资本家对于他所预付的资本的一切部分,都期望得到同样的利益。假定预付额的1/4,用于支付(直接)劳动,其余3/4则是积累劳动、利润以及因地租、赋税和其他支出而产生的利润的附加。在这种情况下,说资本家的利润将随着他产品的这1/4的价值与所使用的劳动量之比的变动而变动,这是完全正确的。例如,假定一个租地农场主在农业上花了2000镑,其中1500镑用于种籽、马饲料、固定资本的损耗、固定资本和流动资本的利息、地租、什一税、赋税等等,500镑用于直接劳动,而到年终收回2400镑。这个租地农场主的利润是由2000镑产生的400镑,即20%。同样明显的是,如果我们拿产品价值的1/4即600镑来同支付直接劳动的工资总额相比,结果得出的利润率完全一样。”(第267—268页)}\end{quote}

马尔萨斯在这里表现了邓德里厄里勋爵作风\endnote{邓德里厄里勋爵作风(或邓德里厄里作风)——指矫揉造作的浮华习气。邓德里厄里勋爵是英国作家汤姆·泰勒的喜剧《我们的美国亲戚》(《OurAmericanCousin》)里的人物,该剧于1858年首次上演。——第31页。}。他想(他模糊地感到,剩余价值,从而还有利润,与用在工资上的可变资本具有一定的关系)证明“利润决定于总产品价值中用来支付所使用的劳动的份额”。他最初说对了,因为他假定全部资本由可变资本即用于工资上的资本构成。在这种场合,利润和剩余价值确实是等同的。但是即使在这一场合,马尔萨斯也只是发表了一些十分荒唐的见解。如果所支出的资本是100,利润为10%,那末产品的价值等于110,利润占所支出的资本的1/10(即它的10%),占总产品价值(马尔萨斯已经把利润本身的价值也算在里面)的1/11。这样,利润是总产品价值的1/11,而预付资本是它的10/11。10%的利润同总产品价值的关系可以表述如下:总产品价值中不包括利润的那一部分等于总产品的10/11,或者说,有10%利润在内的、价值为110镑的产品,包含10/11的支出,利润就是由这些支出产生的。这个出色的数学推理使马尔萨斯感到如此有趣,以致他以利润为20%,30%等为例重复了同样的演算。到现在为止,我们看到的不过是同义反复。利润是对所支出的资本的百分比;总产品价值包含了利润的价值,而所支出的[766]资本是总产品的价值减去利润的价值。因而110—10=100。但100是110的10/11。让我们再看下去。

假定资本不仅由可变资本,而且由不变资本构成。“资本家对于他所预付的资本的一切部分,都期望得到同样的利益。”固然,这违背前面刚刚提出的论断,即利润(应该说是剩余价值)决定于用在工资上的资本的份额。但这有什么要紧呢?马尔萨斯这样的人是不会去违背“资本家”的“期望”或想法的。于是,他就大显身手。假定资本是2000镑,其中3/4即1500镑是不变资本,1/4即500镑是可变资本。利润是20%。这样利润就是400镑,产品的价值是2000十400=2400镑。\endnote{在手稿中有下面有三句话:“但是,600∶400=66+(2/3)%。总产品的价值=1000,其中用于工资的部分=6/10。而马尔萨斯先生的计算是怎样的呢?”最后一句话用以引出下文,但是马克思想用前两句话说明什么却不清楚。——第32页。}[马尔萨斯接着说,]拿总产品的1/4来看,这个1/4的价值等于600镑。所支出的资本的1/4等于500镑,即等于总预付资本中用于工资的部分;100镑则构成利润的1/4,等于总利润中分摊在资本家付出的工资总额上的部分。按照马尔萨斯的意见,这就能证明“资本家的利润将随着他产品的这1/4的价值与所使用的劳动量之比的变动而变动”。其实,这不过证明,一定的资本——例如4000镑——的一定比率的利润,例如20%的利润,形成这笔资本的每一个别部分的20%的利润,而这是同义反复。但这绝对证明不了这笔利润同用于工资的那部分资本之间存在某种确定的、独特的、特有的比例。如果我不象马尔萨斯先生那样取总产品的1/4而是取1/24即(2400镑中的)100镑为例,那末这100镑也包含了20%的利润,换句话说,其中的1/6是利润。在这种情况下,资本是83+(1/3)镑,利润是16+(2/3)镑。如果这83+(1/3)镑等于例如生产中使用的一匹马的价值,那末,按马尔萨斯的方法,就会证明利润随马的价值的变动而变动,即随总产品的28+(4/5)之一而变动。

马尔萨斯在他不能剽窃唐森、安德森或其他什么人而只好自己靠自己的时候,就表现得如此可怜。就实质而言(撇开此人的特点不谈),值得注意的倒是他的这种模糊猜测:剩余价值应按照用在工资上的那部分资本计算。

{在利润率已知的条件下,总利润即利润总额总是取决于预付资本量。而积累则是由利

润总额中再转化为资本的那部分来决定的。但是,因为这部分等于总利润减去资本家所消费的收入,所以它不仅取决于利润总额的价值,而且取决于资本家能用这笔利润总额购买的商品的低廉程度——一部分取决于加入他的消费,加入他的收入的商品的低廉程度,一部分取决于加入不变资本的商品的低廉程度。在这里,由于利润率假定为已知,所以工资同样假定为已知。}

\tchapternonum{[(10)马尔萨斯的价值理论[补充评论]}

在马尔萨斯看来,劳动的价值永远不会变动(这是从亚当·斯密那里继承来的),变动的只是我用劳动换得的商品的价值。\fnote{见本册第20—22页。——编者注}假定在一种情况下一个工作日的工资等于2先令,在另一种情况下等于1先令。资本家为同样的劳动时间付出的先令,在前一种情况下比在后一种情况下多一倍。但是工人为取得同样多的产品付出的劳动,在后一种情况下比在前一种情况下多一倍,因为在后一种情况下,他做完整个工作日才得到1先令,而在前一种情况下只要做半个工作日就可以得到1先令。马尔萨斯先生只看到资本家为换取同样多的劳动付出的先令有时多有时少。他没有看到工人为换取一定量产品付出的劳动也完全相应地有时多有时少。

\begin{quote}{“为一定量的劳动付出较多产品,或者用一定量的产品换取较多劳动,在他〈马尔萨斯〉看来都是一样。然而任何人都会认为这恰恰是相反的。”(《评政治经济学上若干用语的争论,特别是有关价值、供求的争论》1821年伦敦版第52页)}\end{quote}

这本书(《评政治经济学上若干用语的争论》1821年伦敦版)还非常正确地指出:劳动作为价值尺度,在马尔萨斯按亚·斯密的一种见解所理解的意义上,能完全和其他任何商品一样充当价值尺度,而它在这个意义上却不能成为象货币在实际上所充当的那样好的价值尺度。一般说来,这里只有在货币是价值尺度这个意义上才能谈价值尺度问题。

[767]一般说来,价值尺度(在货币的意义上)决不是使商品彼此可通约的东西,——参看我的著作的第一部分第45页\endnote{马克思指《政治经济学批判》第一分册。见《马克思恩格斯全集》中文版第13卷第57—58页。——第35页。}。

\begin{quote}{“相反,正是作为物化劳动时间的商品的可通约性使金成为货币。”}\end{quote}

各个商品作为价值是统一体,它们不过是同一统一体即社会劳动的表现。价值尺度(货币)先要有作为价值的商品为前提,而且只涉及这一价值的表现和数量。商品的价值尺度涉及的总是价值转化为价格,它已经把价值作为前提。

上面提到的《评政治经济学上若干用语的争论》中的那一段话是这样说的:

\begin{quote}{“马尔萨斯先生说:‘在同一地点和同一时间,不同的商品所能支配的不同的日劳动量,正好和这些商品的相对交换价值成比例,反过来也是一样。’\endnote{匿名著作《评政治经济学上若干用语的争论》的作者引用的是马尔萨斯《政治经济学原理》第1版(1820年伦敦版)第121页。——第35页。}如果这对劳动来说是正确的,那末对其他任何东西来说同样是正确的。”(《评政治经济学上若干用语的争论》第49页)“货币在同一时间和同一地点可以很好地执行价值尺度的职能……但是这〈指马尔萨斯的论点〉对劳动来说看来是不正确的。劳动甚至在同一时间和同一地点也不是尺度。我们拿一定量的谷物来说,假定它在同一时间和同一地点与一粒钻石在价值上相等;那末谷物和钻石,如果用它们的实物形式支付劳动,能否支配等量劳动呢?有人会说:不能,但是钻石可以购买货币,用货币便能支配等量的劳动……这种决定价值的方法是没有用的,因为这种方法如果不用似乎被它取代的另一种方法来校正,便不能采用。我们只能得出这样的结论:谷物和钻石所以能支配等量的劳动,是由于它们在货币形式上具有相同的价值。但是有人却要我们作出这样的结论:两者之所以具有相同的价值,是由于它们支配等量的劳动。”(同上,第49—50页)}\end{quote}

\tchapternonum{[(11)]生产过剩。“非生产消费者”等等。[马尔萨斯为“非生产消费者”的挥霍辩护,把它看成防止生产过剩的手段]}

从马尔萨斯的价值理论引出了他这个人口过剩(因生活资料不足而产生的人口过剩)论者如此狂热鼓吹的关于非生产消费必须不断增长的整个学说。商品的价值等于预付的材料、机器等等的价值加商品中包含的直接劳动量,而直接劳动量,照马尔萨斯的说法,则等于商品中包含的工资的价值加根据一般利润率的水平加在全部预付上的利润附加额。马尔萨斯认为,这一名义附加额构成利润,并且是商品供给即商品再生产的条件。这些要素构成不同于生产者价格的买者价格,而买者价格也就是商品的实际价值。现在要问,这一价格是怎样实现的呢?谁应该支付这一价格呢?这一价格应该从什么基金中支付呢?

在研究马尔萨斯的观点时我们必须先作如下的区分(这一点他没有做)。一部分资本家生产的商品直接加入工人的消费;另一部分资本家生产的商品或者只是间接地加入工人的消费(就这种商品作为原料和机器等等加入生产生活必需品所需的资本而言),或者根本不加入工人的消费,因为它们只加入非工人的收入。

我们首先来考察那些生产加入工人消费的物品的资本家。他们不只是工人的劳动的买者,而且是把工人生产的产品再卖给工人的卖者。如果工人加进的劳动量值100塔勒,那末资本家就付给他100塔勒。而[在马尔萨斯看来]这就是资本家所购买的劳动加在原料等等上的唯一的价值。因此工人得到了他的劳动的价值,他交给资本家的仅仅是这个价值的等价物。但是,工人虽然名义上取得了这个价值,他实际上得到的商品量却少于他所生产的商品量。实际上他只收回物化在产品中的自己劳动的一部分。为了简便起见,我们也象马尔萨斯本人经常做的那样,假定资本只由用于工资的资本构成。为了生产商品,预付给工人100塔勒(这100塔勒就是所购买的劳动的价值,而且是劳动加在产品上的唯一价值),可是,资本家却把这一商品卖110塔勒,而工人用100塔勒只能买回10/11的产品;1/11的产品,即10塔勒的价值或代表这10塔勒剩余价值的剩余产品量,则归资本家所有。如果资本家把商品卖120塔勒,工人得到的只有10/12,资本家则得到了2/12的产品及其价值。如果资本家把商品卖130塔勒(30%),工人就只得到10/13,资本家则得到10/13的产品。如果资本家加上50%的附加额,即把商品卖150塔勒,工人就只得到2/3的产品,[768]资本家则得到1/3的产品。资本家把商品的价格卖得越高,在产品的价值中,从而也在产品的量中,工人得到的份额就越小,资本家自己所占的份额就越大,工人用他的劳动的价值所能买回的那部分产品价值或产品本身就越少。即使在预付资本中除了可变资本外还有不变资本,例如除了100塔勒的工资外还有100塔勒的原料等等,事情也不会发生任何变化。在这种情况下,如果利润率为10%,资本家就不是把商品卖210塔勒,而是卖220塔勒(其中不变资本100塔勒,[直接]劳动的产品120塔勒)。

{西斯蒙第的《政治经济学新原理》,于1819年第一次出版。\fnote{见本册第51—52页。——编者注}}

在上例中,就生产直接加入工人消费的物品(生活必需品)的A类资本家来看,我们看到这样一种情况:通过名义附加额,即加在预付资本价格上的正常利润附加额,确实为资本家创造了一个剩余基金,因为资本家通过这种转弯抹角的方法只把工人产品的一部分还给工人,而把另一部分据为己有。但是,这种结果之所以产生,并不是由于资本家按照提高了的价值把全部产品卖给工人,而是由于产品价值的提高使工人没有可能用他的工资买回全部产品,而只能买回产品的一部分。因此非常明显,工人的需求在任何时候也不足以实现购买价格超过生产费用[costprice]\endnote{《costprice》这一术语,马克思在这里和有时在后面是用来指资本家的生产费用(c+v)。见注6。——第38页。}的余额,即不足以实现利润和商品的“价值”。相反,利润基金之所以存在,正是由于工人不能用他的工资买回他的全部产品,也就是他的需求和供给不相适应。于是A类资本家手里便有代表一定价值(在上例是20塔勒)的一定量商品,他不必用这些商品来补偿资本,而可以把其中的一部分当作收入花掉,把另一部分用于积累。注意:资本家手里有多少这样的基金,取决于他加在生产费用上的价值附加额,这一附加额决定资本家和工人分配总产品的比例。

现在我们来看B类资本家,他们给A类提供原料和机器等等,简言之,提供不变资本。B类只能把自己的产品卖给A类,因为他们既不能把他们自己的商品卖给与资本(原料、机器等等)毫无关系的工人,也不能卖给生产奢侈品(即不是生活必需品、不加入工人阶级日常消费的一切物品)的资本家,也不能卖给生产制造奢侈品所需的不变资本的资本家。

在上面我们已经看到,在A类预付的资本中有100塔勒用于不变资本。生产这一不变资本的工厂主,在利润率为10%时,用90+(10/11)塔勒的生产费用生产了它,但是卖100塔勒(90+(10/11)∶9+(1/11)=100:10)。于是他靠A类资本家得到自己的利润。也就是说,他从卖220塔勒的A类的产品中得到的是他的100塔勒,而不只是我们假定他用来购买直接劳动的90+(10/11)塔勒。B绝不是从他的工人那里得到利润的,他不能把价值90+(10/11)塔勒的工人的产品按100塔勒再卖给工人,因为工人根本不向他购买东西。然而,B的工人的情况仍然和A的工人一样。他们用90+(10/11)塔勒买到的只是名义上具有90+(10/11)塔勒的价值的商品量,因为A的产品的每一部分的价格都均等地提高了,或者说,它的价值的每一部分都由于相应的利润附加额而代表较少的一部分产品。

{但是,这种加价只能到一定程度,因为工人必须获得足够的商品来维持生活和再生产他的劳动能力。如果资本家A加上了100%,把他花费了200塔勒生产的商品卖400塔勒,工人就只能买回1/4的产品(假定工人得到100塔勒)。如果工人需要得到一半产品才能生活,资本家就必须付给他200塔勒。这样,资本家只留下100塔勒(不变资本100塔勒,工资200塔勒)。因此,结果和资本家按300塔勒出卖花费200塔勒生产的商品并付给工人100塔勒工资的情况是一样的。}

B类资本家不是(直接)靠他的工人,而是靠他把自己的商品卖给A类资本家获得利润基金的。A的产品不仅为实现B类资本家的利润服务,而且构成A自己的利润基金。可是,很清楚,资本家A靠工人获得的利润不可能通过把商品卖给资本家B来实现;资本家B也和资本家A自己的工人一样,不能对A的产品提出足够的需求(以保证产品按照它的价值出卖)。相反,这里出现了反作用。[769]资本家A加上的利润附加额越高,与他的工人相对比,他在总产品中所占有的、资本家B不能得到的那一部分就越大。

资本家B以和资本家A同样的幅度加上附加额。资本家B仍旧付给他的工人90+(10/11)塔勒,虽然工人用这笔钱只能买到较少的商品。但是,如果A取得20%,而不是10%,那末,B同样取得20%,而不是10%,并且按109+(1/11)塔勒,而不是按100塔勒出卖自己的商品。这样,A支出的这部分费用就增加了。

甚至完全可以把A和B作为一类来考察(B属于A的费用,资本家A从总产品中付给资本家B的部分越大,他自己剩下的部分就越小)。在290+(10/11)塔勒的资本中,B占90+(10/11),A占200。他们总共支出290+(10/11)塔勒,并取得29+(1/11)塔勒的利润。B从A买回的无论如何不可能多于100塔勒,而且其中还包括他的9+(1/11)塔勒的利润。正如刚才说过的,两者合在一起共有29+(1/11)塔勒的收入。

现在来谈C类和D类。C是生产为制造奢侈品所必需的不变资本的资本家,D是直接生产奢侈品的资本家。首先,这里很清楚,只有D向C提出直接需求,D类资本家是C类资本家的商品的买者。而资本家C只有把他的商品按照高价,通过把名义附加额加在商品生产费用上的办法卖给资本家D,才能实现利润。资本家D付给资本家C的东西,必须多于C为补偿他的商品的[生产费用的]一切组成部分所必需的东西。资本家D的利润则部分地加在资本家C为他生产的不变资本上,部分地加在D自己直接预付于工资的资本上。C可以用从D那里赚得的利润购买D的一部分商品,虽然他不能把自己的利润全都这样花掉,因为他还需要为自己取得生活必需品,而不仅是为工人(为他们,他是用在D那里实现的资本去交换的)。第一,C的商品的实现直接取决于能否把该商品卖给资本家D;第二,即使已经这样卖出,资本家C的利润所产生的需求也不能实现资本家D出卖的商品的全部价值,正象资本家B的需求不能实现A的商品的全部价值一样。因为资本家C的利润正是从资本家D那里赚来的,即使C把这个利润再用在D的商品上,而不用在其他商品上,他的需求也永远不会大于他从D那里赚得的利润。资本家C的利润永远比D的资本少得多,比D的全部需求少得多,并且C的利润永远不会构成D的利润源泉(顶多是D对C进行一些欺骗,把他卖回给C的商品加价),因为资本家C的利润是直接从资本家D的钱袋中来的。

其次,很清楚,如果在每一类(不管是C类还是D类)内部资本家互卖他们的商品,他们之中谁也不会因此赚得或实现任何利润。资本家m把只值100塔勒的商品按110塔勒卖给资本家n,而n对m也这样做。在交换之后,每个人都和交换前一样仍然拥有生产费用为100塔勒的商品量。每个人都用110塔勒只换得值100塔勒的商品。附加额并不使他拥有的对别人的商品的支配权大于附加额使别人拥有的对他的商品的支配权。至于说到价值,那末,即使m和n都不交换自己的商品而乐于把商品说成是值110塔勒而不是值100塔勒,其结果也会是完全一样的。

其次,很清楚,[按照马尔萨斯的观点]D类(因为C类包括在D类里)的名义剩余价值不代表实际的剩余产品。由于资本家A加上附加额,工人用100塔勒买到的生活必需品减少了,这一情况同资本家D没有直接关系。为了雇用一定数量的工人,资本家D必须照旧支付100塔勒。他付给工人的是工人劳动的价值;除此之外,工人[按照马尔萨斯的观点]没有把任何东西加在产品上,他们交给自己雇主的不过是他们所得工资的等价物。资本家D只有把品卖给第三者才能得到超过这个等价物的余额,因为他向第三者出卖商品的价格高于生产费用。

实际上,[生产奢侈品,例如]生产镜子的工厂主,在他的产品中有和租地农场主的一样的剩余价值和剩余产品。因为产品中包含着无酬劳动和有酬劳完全一样地表现在产品中。无酬劳动表现在剩余产品中。镜子的一部分,工厂主没有花费任何代价,但这一部分也有价值,因为这一部分,工厂主没有花费任何代价,但这一部分也有价值,因为这一部分完全和镜子的补偿预付资本的另一部分一样包含着劳动。剩余产品中的这种剩余价值在镜子出卖以前已经存在,并不是由于出卖才产生的。相反,如果工人在直接劳动中付出的只是他以工资形式取得的积累劳动的等价物,那就既不存在[770]剩余产品,也不存在与之相适应的剩余价值了。但是,在马尔萨斯看来,事情并不是这样,他认为工人还给资本家的只是工资的等价物。

很清楚,D类(包括C类)不能用A类的办法人为地为自己创造剩余基金,也就是说,不能像A类那样把自己的商品按照比向工人购买这种商品时贵的价格再卖给工人,从而在裣了支出的资本以后将总产品的一部分据为己有。因为工人不是D类的商品的买者。D类的剩余基金也不能通过D类资本家相互出卖自己的商品,换言之,通过D类内部进行商品交换而产生。因此,它只有通过D类把自己的产品卖给A类和B类才能实现。由于D类资本家把价值为100塔勒的商品按110塔勒出卖,A类资本家用100塔勒只能买到D类的10/11的产品,而D类资本家自己留下1/11,这一部分产品他们可以自己消费或者用来交换本类的其他商品。

一切不直接生产生活必需品、因而不是把自己的绝大部分或很在一部分商品再卖给工人的资本家,[按照马尔萨斯的观点]情况是这样的:

假定这种资本家的不变资本是100塔勒。其次,如果资本家在工资上付出100塔勒,他就把工人的劳动价值付给了工人。工人在100塔勒的价值上加上100塔勒,因此产品的总价值(生产费用)是200塔勒,而利润是从哪里来的呢?如果平均利润率是10%,资本家就把值200塔勒的商品卖220塔勒。如果他确实把商品卖220塔勒,那末很清楚,要再生产这种商品有200塔勒就够了:100塔勒用于原料等等,100塔勒用于工资;他把20塔勒装进自己的腰包。他可以把这20塔勒当作收入花掉,也可以用来积累资本。

但是,他以高于商品“生产价值”(按照马尔萨斯的说法,“生产价值”和“出售价值”或实际价值不同,所以实际上利润等于生产价值和出售价值之间的差额,或者说等于出售价值减生产价值)10%的价格把商品卖给谁呢?这些资本家靠彼此交换或出卖商品是不能实现丝毫利润的。如果A把价值为200塔勒的商品按220塔勒卖给B,那末B对A也会耍同样的花招。这些商品换一下手,既不改变它们的价值,也不改变它们的数量。以前在A手中的商品量,现在在B手中;反过来也是一样。以前用100表示的东西,现在用110表示,事情一点也没有变化。无论是A的购买力还是B的购买力,都没有任何变动。

但是,根据假定,这些资本家都不能把自己的商品卖给工人。

因此,他们必须把自己的商品卖给生产生活必需品的资本家。后者由于和工人进行交换,事实上拥有实际的剩余基金。名义剩余价值的形成,使他们实际上掌握了剩余产品。这是到目前为止存在的唯一的剩余基金。其他资本家的剩余基金,则只有靠他们把自己的商品以高于其生产价值的价格卖给这些剩余基金的占有者才会产生。

至于生产为制造生活必需品所需的不变资本的资本家,我们已经看到,生活必需品的生产者必须向他们购买。这些购买包括在他的生产费用之内。他的利润愈高,他所花费的附加上同样利润率的预付就愈贵。如果生活必需品的生产者按20%而不是10%的附加额出卖生活必需品,他的不变资本的生产者也会附加20%而不是10%。他要求用来换取90+(10/11)的不是100,而是109+(1/11),或者凑成整数就是110,因此现在产品的[生产]价值是210,这个数额的20%是42,因此整个产品的[出售]价值等于252。其中工人得到100。现在资本家取得的利润就超过总产品的1/11;他以前按220出卖产品时,只得到1/11。产品的量仍旧是那么多,但是归资本家支配的部分,在价值上和数量上都增加了。

至于既不生产生活必需品、也不生产加入这些生活必需品生产的资本的其他资本家,他们只有把自己的商品卖给前两类资本家,才[能]\fnote{手稿这一页缺左下角,因此,原文有几行缺少头几个字。所缺的字由编者根据意思补上并放在四角括号内。——编者注}取得利润。如果前两类资本家赚取20%,他们也赚取[这么多]。

但是,第一类资本家[同工人的交换]和两类资本家之间的交换有很大的区别。第一类资本家[由于]同工人进行[交换],形成了一个生活必需品的实际剩余基金,即[作为]资本的[增殖额]归他们支配的剩余产品,所以他们可以把其中一部分积累起来,一部分[作为收入花费],不管是用来购买他们自己的生活必需品,还是购买奢侈品。这里的剩余价值实际上[代表][XIV—771]剩余劳动和剩余产品,虽然这一结果[在马尔萨斯那里]是通过笨拙的、转弯抹角的办法,通过给商品价格加上附加额的办法得来的。假定生产生活必需品的工人的产品的价值实际上只等于100塔勒。但是,因为支付工资只要产品的10/11就够了,所以资本家只要花费90+(10/11)塔勒,他由此获得利润9+(1/11)塔勒。但是,如果他把劳动的价值和劳动的量设想为等同的,付给工人100塔勒,而把商品按110塔勒卖给他们,那末他仍旧会得到产品的1/11。产品的这个1/11现在不是值9+(1/11)塔勒而是值10塔勒,这对于资本家来说没有什么好处,因为他现在预付的资本已经不是90+(10/11)塔勒,而是100塔勒。

至于说到其他几类资本家,他们那里[按照马尔萨斯的观点]不存在任何现实的剩余产品,不存在任何代表剩余劳动时间的东西。他们把值100塔勒的劳动产品卖110,而只是由于这样给商品价格加上附加额,这笔资本才转化为资本加收入。

但是,从邓德里厄里勋爵\fnote{指马尔萨斯。——编者注}的观点来看,现在这两类资本家之间的情况又是怎样的呢?

生活必需品的生产者把价值100塔勒的新加劳动产品\endnote{手稿中这里用的是“剩余产品”(《Surplusprodukt》)一词,马克思在他的手稿第703页上曾专门谈到这个词的含义:“剩余产品在这里是产品中超过同不变资本相等的那部分产品的余额”(见本卷第2册第559页),也就是指新加劳动产品(v+m)。如果不变资本等于零,那末“新加劳动产品”就相当于产品的价值。——第45页。}卖110塔勒(因为他们支付的工资不是90+(10/11)塔勒,而是100塔勒)。但他们是唯一占有剩余产品的人。如果其他资本家同样把价值100塔勒的产品按110塔勒卖给他们,那末其他资本家实际上除补偿自己的资本外也会得到利润。为什么呢?因为他们有价值100塔勒的生活必需品就已经足以支付自己的工人,从而可以把10塔勒留给自己。或者更确切地说,因为他们实际上获得了价值100塔勒的生活必需品,而其中的10/11就足以支付自己的工人,因为他们这时所处的情况和A类、B类资本家是一样的。然而A类、B类资本家换回的仅仅是代表100塔勒价值的产品量。产品在名义上值110塔勒,并不能使他们多得分文,因为这些产品在数量上,作为使用价值,不能代表一个比100塔勒包含的劳动时间所提供的更大的产品量,[在价值上]他们也不能用这些产品在补偿100塔勒的资本之外再补偿10塔勒的资本。这种情况只有在他们再卖出商品的情况下才有可能。

虽然两类资本家都按110互相出卖值100的东西,但是在这种交换中只有在第二类手里的100才确实具有110的效用。第一类资本家用110的价值实际上只换得100的价值。他们按较高的价格出卖他们的剩余产品,仅仅是由于他们对加入他们收入的[奢侈]物品所支付的,高于这些物品的价值。但是,第二类资本家所实现的剩余价值,实际上仅仅限于他们从第一类所实现的剩余产品中分享的那一部分,因为他们自己并不创造任何剩余产品。

说到奢侈品的这种涨价,马尔萨斯非常及时地想起资本主义生产的直接目的不是挥霍,而是积累。因此,A类资本家会由于这种不合算的交易,——这种交易使他们从工人那里剥夺来的一部分果实重新失掉,——而减少他们对奢侈品的需求。但是,如果他们这样做了,并且更多地进行积累,那末,对他们所生产的生活必需品的有支付能力的需求即生活必需品的市场就会缩小,这个市场靠工人和不变资本生产者的需求是不可能充足地建立起来的。因此,生活必需品的价格会下降,然而A类资本家就是靠生活必需品价格的提高,靠加在价格上的名义附加额并且同这个附加额成比例地从工人身上榨取他们的剩余产品。如果生活必需品的价格从120降到110,他们的剩余产品和他们的剩余价值就会从2/12降到1/11。因而,奢侈品生产者的市场、对奢侈品的需求,还会按更大得多的比例缩小。

第一类资本家在他们的资本已得到补偿后同第二类资本家交换时出卖实际的剩余产品。相反,第二类只是出卖他们的资本,为的是通过这种交易使它从资本转化为资本加收入。这样,整个生产(特别是生产的增长)之所以能持续不断,全靠生活必需品的涨价,而同奢侈品的实际产量成反比的奢侈品价格,又应和这种涨价相适应。第二类资本家把值100的东西卖110,他们在这种交换中也不会得到任何利益,因为实际上他们换回的110也只值100。但是,这100(以生活必需品的形式)能补偿资本加上利润,而那100[以奢侈品的形式]只不过号称110罢了。所以结果就是,第一类资本家在这种交换中得到价值100的奢侈品。他们用110购买价值100的奢侈品。但是对第二类资本家来说,这110就有110的价值,因为他们用100支付劳动(补偿他们的资本),而把10作为余额留下来。

[772]交换双方按同样比率互相贵卖商品,在同样程度上互相欺骗,利润究竟怎么会由此而产生,这是难以理解的。

如果除了一类资本家同他们的工人交换以及各类资本家互相交换以外,还有第三类买者——从机器里出来的神\fnote{原文是与deusexmachina,直译是:“从机器里出来的神”(在古代的戏院里,扮演神的演员由特殊的机械装置送上舞台);转意是:突然出现以挽救危局的人。——编者注}——出现,那末困难就解决了。这第三类买者按照商品的名义价值付款,但他们自己不向任何人出卖商品,自己不以这一套骗人的把戏来回敬,就是说,这一类买者经历的过程是G—W,而不是G—W—G,他们购买商品不是为了补偿他们的资本并且得到利润,而是为了消费商品,他们买而不卖。在这种情况下,资本家实现利润不是靠相互交换商品,而是靠:(1)同工人交换,把总产品的一部分卖回给工人,卖得的货币等于先前用来向工人购买全部总产品(扣除不变资本之后)的货币;(2)把一部分生活必需品和奢侈品卖给第三类买者。因为这一类买者为100支付110,而不再把100按110出卖,所以资本家就会在事实上,而不仅仅在名义上实现10%的利润。利润可以由两种方法取得,即从总产品中尽可能少地卖回给工人,而尽可能多地卖给用现金支付、自己什么也不出卖、只是为消费而买的第三类买者。

但是,不兼卖者的买者,必须是不兼生产者的消费者,即必须是非生产消费者;正是这一类非生产消费者解决了马尔萨斯的矛盾。但是,这种非生产消费者必须同时是有支付能力的消费者,必须形成实际的需求,并且,他们所拥有的、每年支出的价值额,必须不仅足以支付他们购买和消费的商品的生产价值,而且除此以外还足以支付一个名义的利润附加额、剩余价值、出售价值和生产价值之间的差额。这一类买者在社会上代表为消费而消费,正象资本家阶级代表为生产而生产一样;前者代表“挥霍的热情”,后者代表“积累的热情”。(《政治经济学原理》[第2版]第326页)资本家阶级的积累欲望之所以能保持,是由于他们卖得之款经常大于他们的支出,而利润也就成为积累的刺激剂。尽管他们如此热中于积累,但不会弄到生产过剩的地步,或者说很难发生生产过剩,因为非生产消费者不仅是投入市场的产品的巨大排水渠,而且他们自己没有任何产品投入市场。所以,他们的人数不管怎样多,也不会造成对资本家的竞争;相反,他们所有的人都是只求不供的代表者,因此就会抵销资本家方面发生的供过于求。

但是,这一类买者每年的支付手段是从哪里来的呢?这里首先是土地所有者,他们在地租的名义下把很大一部分年产品价值据为己有,并把通过这种方法从资本家那里夺得的货币用于消费资本家生产的商品,他们在购买商品时受到资本家的欺骗。这些土地所有者自己不必从事生产,而且通常也的确不从事生产。根本的一点是:虽然他们花费货币购买劳动,他们雇用的不是生产工人,而只是那些帮助他们消耗财富的食客,家仆,这些人使生活必需品的价格保持在高水平上,因为他们只是购买生活必需品,而本身不会促使生活必需品或任何其他商品的供给有所增加。但是,这种地租所得者还不足以造成“足够的需求”。还必须求助于人为的手段。这就是征收高额的税,供养大批国家和教会的领干薪者,维持庞大的军队,支付大量年金,征收供养牧师的什一税,举借大量的国债,以及不时发动费用浩大的战争。这些就是马尔萨斯心目中的“灵丹妙药”。(《政治经济学原理》[第2版]第408页及以下各页)

总之,被马尔萨斯当作“灵丹妙药”的第三类买者——他们只买不卖,只消费不生产——先是不付代价地取得很大一部分年产品价值,并通过下述办法使生产者致富:生产者首先必须把购买他们商品所需的货币白白付给第三类买者,然后[773]再把这些货币取回,即把自己的商品按高于商品价值的价格卖给他们,或者说,以货币形式从他们那里收回的价值大于以商品形式向他们提供的价值。而这种交易是年年重复的。

\tchapternonum{[(12)马尔萨斯同李嘉图论战的社会实质。马尔萨斯歪曲西斯蒙第关于资产阶级生产的矛盾的观点。马尔萨斯对普遍生产过剩可能性的原理所作的解释的辩护论实质]}

马尔萨斯的结论是完全正确无误地从他的基本的价值理论中得出来的;不过,这个理论也十分明显地符合他的目的——为英国现状辩护,为大地主所有制、“国家和教会”、年金领取者、收税人、教会的什一税、国债、交易所经纪人、教区小吏、牧师和家仆(“国民支出”)辩护,而李嘉图学派恰好把这一切当作对资产阶级生产的无益的、陈腐的障碍,当作累赘来加以反对。李嘉图不顾一切地维护资产阶级生产,因为这种生产意味着尽可能无限制地扩大社会生产力,同时他不考虑生产承担者的命运,不管生产承担者是资本家还是工人。他承认这个发展阶段的历史的合理性和必然性。他完全生活在他那个时代的历史焦点上,就象他完全缺乏对过去的历史感一样。马尔萨斯也愿意资本主义生产尽可能自由地发展,只要这一生产的主要承担者即各劳动阶级的贫困是这一发展的条件;但是,这种生产同时应该适应贵族及其在国家和教会中的分支的“消费需要”,并且应该成为一种物质基础,以满足封建制度和君主专制制度遗留下来的利益的代表人物的过时要求。马尔萨斯愿意有资产阶级生产,只要这一生产不是革命的,只要这一生产不形成历史发展的因素,而只是为“旧”社会造成更广阔、更方便的物质基础。

因此,一方面,存在着工人阶级,由于人口规律的作用,他们同供他们使用的生活资料相比始终是过剩的,即由于生产不足而造成人口过剩;其次,存在着资本家阶级,由于这种人口规律的作用,他们始终能够把工人自己的产品按照这样的价格卖回给工人,使工人从中取回的仅仅能勉强维持他们的生存;最后,社会上还有很大一批奇生虫,一群专事享乐的雄蜂,他们一部分是老爷,一部分是仆役,他们部分地以地租的名义,部分地以政治的名义,无偿地从资本家阶级那里攫取一大批财富,但是,他们要用从资本家手里夺得的货币,按高于价值的价格支付向这些资本家购买的商品;资本家阶级受积累欲望的驱使从事生产,非生产者在经济上则只代表消费欲望,代表挥霍。而且这被描绘为避免生产过剩的唯一办法,而这种生产过剩又是和与生产相比的人口过剩同时存在的。处于生产之外的那些阶级的消费过度,被说成是医治这两种过剩的灵丹妙药。工人人口同生产之间的失调现象,通过根本不参加生产的游手好闲者吃掉一部分产品的办法得到消除。资本家引起的生产过剩的失调现象,则通过财富享受者的消费过度得到消除。

我们已经看到,当马尔萨斯企图根据亚·斯密观点的弱点建立一种对立的理论来反对李嘉图根据亚·斯密观点的优点建立的理论时,他显得多么幼稚、庸俗和浅薄。未必还有什么东西比马尔萨斯关于价值的著作所表现出的那种虚弱的挣扎更滑稽可笑的了。但是,一当他作出实际结论,从而重新进入他作为经济学方面的阿伯拉罕·圣克拉\endnote{阿伯拉罕·圣克拉是奥地利传教士和著作家乌尔利希·梅格尔勒(1644—1709年)的笔名,他力图用公众易懂的形式宣传天主教,并用所谓民间文体来进行“救人”的说教和写劝善的作品。——第51页。}从事活动的领域时,他又自由自在起来。不过,即使在这里,他也没有改变他那天生的剽窃者的本性。乍一看来,谁能相信马尔萨斯的《政治经济学原理》竟不过是西斯蒙第的《政治经济学新原理》一书的马尔萨斯化的译本呢?然而,事实就是如此。西斯蒙第的书于1819年出版。一年以后,马尔萨斯的拙劣的英文仿制品问世了。象过去剽窃唐森和安德森一样,他现在又在西斯蒙第那里为自己的一本厚厚的经济学论著找到了理论支柱,不过与此同时,他还利用了从李嘉图的《原理》一书中学来的新理论。

[774]如果说马尔萨斯攻击李嘉图的是李嘉图著作中对旧社会说来是革命的资本主义生产倾向,那末他凭着永无谬误的牧师本能从西斯蒙第的著作中抄来的,却只是对资本主义生产,对现代资产阶级社会说来是反动的东西。

在这里,我不把西斯蒙第列入我的历史述评之内,因为对于他的观点的批判,属于我写完这部著作以后才能着手的那一部分——资本的现实运动(竞争和信用)。

马尔萨斯利用西斯蒙第的观点来适应自己的目的,这从《政治经济学原理》的一章的标题就可以看出。这一章的标题是:

\begin{quote}{《生产力必须和分配手段相结合以保证财富的不断增长。》([第2版]第361页)}\end{quote}

[在这一章中写道:]

\begin{quote}{“只有生产力,还不能保证创造相应程度的财富。为了把生产力充分调动起来,还需要有一些别的东西。这就是对全部生产物的有效的和不受阻碍的需求。看来,最有助于达到这一目的的,是这样地分配产品并使这些产品这样地适应那些消费产品的人的需要,以致全部产品的交换价值能不断增加。”(《政治经济学原理》第2版第361页)}\end{quote}

其次,下面一段话同样是西斯蒙第式的和反对李嘉图的:

\begin{quote}{“一国的财富,部分地取决于靠本国的劳动所获得的产品的数量,部分地取决于这个数量与现有人口的需要和购买力的适应,这种适应要使它能具有价值。财富并不单单由这些因素中的一种因素决定,这是十分肯定无疑的。”(同上,第301页)“但是,财富和价值的最密切的联系,也许在于后者是前者的生产所必需的。”(同上)}\end{quote}

这段话是专门针对李嘉图,针对他的著作第二十章《价值和财富,它们的特性》的。李嘉图在那里说:

\begin{quote}{“因此,价值和财富在本质上是不同的,因为价值不取决于充裕程度,而取决于生产的困难或容易程度。”(李嘉图《原理》第3版第320页)}\end{quote}

{其实,价值也可能随着“生产的容易程度”的提高而增加。假定某一国家的人口从100万增加到600万。100万人过去每天工作12小时。600万人则把生产力发挥到每人工作6小时就能生产出以前用12小时生产的东西。那末,按照李嘉图本人的观点,财富就增加到六倍,价值增加到三倍。}

\begin{quote}{“富并不取决于价值。一个人的贫富取决于他所能支配的生活必需品和奢侈品的充裕程度……只是由于把价值的概念和财富即富的概念混淆起来,才会断言,减少商品的数量,即减少生活必需品、舒适品和享乐品的数量,可以增加财富。如果说价值是财富的尺度,那末这种说法是不能否定的,因为商品的价值会由于商品的稀少而增加;但是……如果财富是由生活必需品和奢侈品构成,它就不可能由于它们的数量的减少而增加。”(同上,第323—324页)}\end{quote}

换言之,李嘉图在这里是说,财富只是由使用价值构成。他把资产阶级生产变成单纯为使用价值而进行的生产,这对于交换价值占统治地位的生产方式是一种非常美妙的见解。他把资产阶级财富的特有形式只看成一种不触及这种财富内容的表面的东西。因此他也就否认在危机中爆发出来的资产阶级生产的矛盾。因此就产生了他对货币的完全错误的见解。因此他在考察资本的生产过程时也就完全不注意流通过程,——而流通过程却包括商品的形态变化,包括资本转化为货币的必然性。无论如何,没有一个人比李嘉图本人更好地、更明确地阐明了:资产阶级生产并不是为生产者(他不止一次地这样称呼工人)\endnote{李嘉图在其他一些地方使用的“生产者”(《producer》)一词是指“产业资本家”,马克思在《李嘉图的其他方面。约翰·巴顿》一章中指出了这一点(见本卷第2册第627页,并参看第480页和第487页马克思引自李嘉图《原理》一书的引文)。马克思在本卷第二册第478页和第528页上指出李嘉图把“生产者”和“工人”两个概念等同起来。在本卷第二册第527页和第622页马克思引用的引文中,李嘉图也是在上述意义上使用“生产者”一词的。——第54页。}生产财富,因此资产阶级财富的生产完全不是为“充裕”而生产,不是为生产生活必需品和奢侈品的人生产生活必需品和奢侈品,——如果生产只是满足生产者需要的一种手段,是一种仅仅由使用价值占统治地位的生产,那末情况本来应当是这样的。可是,同一个李嘉图说:

\begin{quote}{“如果我们生活在欧文先生的一个平行四边形\endnote{欧文在阐述他的空想的社会改革计划时证明:按平行四边形或正方形建立劳动公社新村,无论从经济上看,还是从组织家庭生活的观点看,都是最合适的。欧文在1817—1821年的一系列演说中都谈到了这些思想。——第54页。}里,共同享用我们的全部产品,那末谁也不会由于产品充裕而受害;但是,只要社会构成仍然象目前这样,充裕就往往对生产者有害,而匮乏倒对他们有利。”(《论农业的保护关税》1822年伦敦第4版第21页)}\end{quote}

[775]李嘉图把资产阶级的生产,确切些说,把资本主义的生产看作生产的绝对形式。这就是说,他认为,资本主义生产的生产关系的一定形式,在任何地方都不会同生产本身的目的即充裕发生矛盾或束缚这一目的,充裕既包括使用价值的量,也包括使用价值的多样性,这又决定作为生产者的人的高度发展,决定他的生产能力的全面发展。在这里李嘉图陷入了可笑的自相矛盾之中。当我们谈到价值和财富时,根据李嘉图的解释,我们只是指整个社会。而当谈到资本和劳动时,李嘉图认为“总收入”仅仅为了创造“纯收入”而存在,是不言而喻的事。实际上,他对资产阶级生产赞赏的,正是这种生产的一定形式同以前的各种生产形式相比能给生产力以自由发展的天地。当这种形式不再起这种作用的时候,或者当这种形式在其中起这种作用的那些矛盾显露出来的时候,李嘉图就否认矛盾,或者确切些说,他自己就以另一种形式表现矛盾,把财富本身,把使用价值总量本身说成是ultimaThule\fnote{最终目的,最终之物,极点,极限(直译是:极北的休里——古代人想象中的欧洲极北部的一个岛国)。——编者注},而不考虑生产者了。

西斯蒙第深刻地感觉到,资本主义生产是自相矛盾的;一方面,它的形式——它的生产关系——促使生产力和财富不受拘束地发展;另一方面,这种关系又受到一定条件的限制,生产力愈发展,这种关系所固有的使用价值和交换价值、商品和货币、买和卖、生产和消费、资本和雇佣劳动等等之间的矛盾就愈扩大。他特别感觉到了这样一个基本矛盾:一方面是生产力的无限制的发展和财富的增加——同时财富由商品构成并且必须转化为货币;另一方面,作为前一方面的基础,生产者群众却局限在生活必需品的范围内。因此,在西斯蒙第看来,危机并不象李嘉图所认为的那样是偶然的,而是内在矛盾的广泛的定期的根本爆发。他经常迟疑不决的是:国家应该控制生产力,使之适应生产关系呢,还是应该控制生产关系,使之适应生产力?在这方面,他常常求救于过去;他成为“过去时代的赞颂者”\fnote{见贺雷西《诗论》。——编者注},或者也企图通过别的调节收入和资本、分配和生产之间的关系的办法来制服矛盾,而不理解分配关系只不过是从另一个角度来看的生产关系。他中肯地批判了资产阶级生产的矛盾,但他不理解这些矛盾,因此也不理解解决这些矛盾的过程。不过,从他的论据的基础来看,他确实有这样一种模糊的猜测:对于在资本主义社会内部发展起来的生产力,对于创造财富的物质和社会条件,必须有占有这种财富的新形式与之适应;资产阶级形式只是暂时的、充满矛盾的形式,在这种形式中财富始终只是获得矛盾的存在,同时处处表现为它自己的对立面。这是始终以贫困为前提、并且只有靠发展贫困才能使自己得以发展的财富。

我们已经看到,马尔萨斯用多么巧妙的办法剽窃了西斯蒙第的观点。而马尔萨斯的理论又以夸张的、更丑恶得多的形式包含在托马斯·查默斯(神学教授)的《论政治经济学和社会的道德状况、道德远景的关系》(1832年伦敦第2版)中。在这里,不仅在理论上更明显地表现出牧师的成分,而且在实质上也更明显地表现出一个“从经济学方面”维护“法定教会”\endnote{“法定教会”(《EstablishedChurch》)是指英国国教会。——第56、344页。}的“尘世福祉”和“法定教会”与之共存亡的整套制度的“法定教会”教徒。

上面提到的马尔萨斯有关工人的论点如下:

\begin{quote}{“从事生产劳动的工人的消费和需求,决不能单独成为资本的积累和使用的动机。”(《政治经济学原理》[第2版]第315页)“如果租地农场主的全部产品在市场上出卖时卖价的增加部分正好等于他付给所雇10个追加工人的报酬,那末,就没有一个租地农场主会自找麻烦去监督这10个追加工人的劳动。在有关商品的过去的供求状况方面或在它的价格方面,必须——在新工人造成追加需求之前,因而与这种需求无关——出现某种东西证明雇用追加的工人来生产这种商品是合算的。”(同上,第312页)“由生产工人本身造成的需求,决不会是一种足够的需求,[776]因为这种需求不会达到同工人所生产的东西一样多的程度。如果达到这种程度,那就不会有什么利润,从而也就不会有使用工人的劳动的动机。任何商品的利润的存在本身,必须先有一种超过生产这种商品的工人的需求范围的需求。”(同上,第405页注)“工人阶级消费的剧增必然大大增加生产费用,因此,这一定会降低利润,削弱或破坏积累的动机。”(同上,第405页)“生活必需品的缺乏,是刺激工人阶级生产奢侈品的主要原因;如果这个刺激消除或者大大削弱,以致花费很少劳动就能够获得生活必需品,那末我们就有充分理由认为,用来生产舒适品的时间将不会更多,而只会更少。”(同上,第334页)}\end{quote}

马尔萨斯并不打算掩盖资产阶级生产的矛盾,相反,他是想要突出这些矛盾,以便一方面证明工人阶级的贫困是必要的(对这种生产方式说来,他们的贫困确实是必要的),另一方面向资本家证明,为了给他们出卖的商品创造足够的需求,养得脑满肠肥的僧侣和官吏是必不可少的。因此,马尔萨斯证明,要导致“财富的不断增长”,无论人口的增加,或资本的积累(同上,第319—320页),或“土地的肥力”(第331页及以下各页)、“节省劳动的发明”、“国外市场”的扩大(第352、359页),都是不够的。

\begin{quote}{“工人和资本同用它们获利的手段比较起来,都可能过剩。”(同上,第414页)}\end{quote}

因此,和李嘉图学派相反,马尔萨斯强调了普遍生产过剩的可能性(同上,第326页及其他各处)。

他在这方面提出的主要论点如下:

\begin{quote}{“需求总是由价值决定,而供给总是由数量决定。”(《政治经济学原理》[第2版]第316页)“商品不仅同商品相交换,而且也同生产劳动和个人服务相交换,而同这些东西相比,就象同货币相比一样,可能发生市场商品普遍充斥。”(同上)“供给必须始终同数量成比例,而需求必须始终同价值成比例。”(卡泽诺夫出版的《政治经济学定义》第65页)“詹姆斯·穆勒说:‘显然,一个人生产出来而不打算用于他自己消费的一切东西,就构成他可以用来交换其他商品的储备。因此,他的购买愿望和购买手段,换句话说,他的需求,正好等于他生产出来但不准备自己消费的东西的数量。’”\fnote{见本册第106页。——编者注}……[马尔萨斯反驳詹姆斯·穆勒说,]“很明显,他购买其他商品的手段,并不同他生产出来并打算销售的商品的数量成比例,而是同这些商品的交换价值成比例;除非某一商品的交换价值同该商品的数量成比例,否则说每一个个人的需求和供给永远相等,就不可能是正确的。”(同上,第64—65页)“如果每一个个人的需求都同他的确切意义上的供给相等,那末,这就证明,他永远能够按照生产费用(加上公平的利润)出卖他的商品;那时,甚至连市场商品的局部充斥也不可能有了。这种论点证明的东西太多了……供给必须始终同数量成比例,而需求必须始终同价值成比例。”(《政治经济学定义》1827年伦敦版第48页注)“在这里,穆勒把需求了解为他〈需求者〉的购买手段。但是,这种购买其他商品的手段,并不同他生产出来并打算销售的商品的数量成比例,而是同这些商品的交换价值成比例;除非某一商品的交换价值同该商品的数量成比例,否则说每一个个人的需求和供给永远相等,就不可能是正确的。”(同上,第48—49页)“托伦斯错误地认为,‘供给增长是有效需求增长的唯一原因’。\fnote{见本册第79页。——编者注}如果真是那样的话,人类在遇到食物和衣服暂时减少的情况时,将会多么难于恢复啊。但是,当食物和衣服的数量减少时,它们的价值会提高;剩下的食物和衣服的货币价格的增长程度,在一段时间内会超过它们的数量减少的程度,而劳动的货币价格可能保持不变。其必然结果是出现了推动比过去大的生产劳动量的力量。”(第59—60页)“一个国家的所有商品,同货币或劳动相比较,可能同时跌价。”(第64页及以下各页)“因此,市场商品普遍充斥是可能的。”(同上)“商品的价格可能全都跌到生产费用之下。”(同上)}\end{quote}

\centerbox{※     ※     ※}

[777]此外,还值得提到的只是马尔萨斯关于流通过程的观点:

\begin{quote}{“如果我们把所使用的固定资本的价值算作预付资本的一部分,我们就必须在年终时把这种资本的残余价值算作年收入的一部分……实际上,他〈资本家〉每年预付的资本只包括他的流动资本,他的固定资本的磨损,以及固定资本的利息和由用于按期支付各项年开支的货币构成的那一部分流动资本的利息。”(《政治经济学原理》[第2版]第269页)}\end{quote}

我认为,折旧基金,即补偿固定资本磨损的基金,同时也就是积累基金。

\tchapternonum{[(13)李嘉图学派对马尔萨斯关于“非生产消费者”的观点的批判]}

我还想从一本反对马尔萨斯理论的李嘉图学派的著作中摘引几段话。关于这本著作中从资本主义观点对马尔萨斯的全部非生产消费者特别是地主进行的抨击,我将在另外一个地方说明:这种抨击从工人的观点来看,也可以逐字逐句地用在资本家身上(这个说明要放在《对资本和雇佣劳动关系的辩护论的解释》那一篇\endnote{《对资本和雇佣劳动关系的辩护论的解释》这一篇,马克思没有写成。——第59页。})。

[这位匿名的李嘉图主义者写道:]

\begin{quote}{“马尔萨斯先生以及象他那样推论的人认为,除非能保证利润率等于或者大于以前的利润率,否则资本的使用就不可能增大,并且认为,单单是资本的增加本身并不能保证这样的利润率,而是适得其反,因此他们想找到一个不取决于生产本身并且处在生产之外的源泉,它能同资本一起不断增长,从它可以经常取得必要数量的超额利润。”(《论马尔萨斯先生近来提倡的关于需求的性质和消费的必要性的原理》1821年伦敦版第33—34页)在马尔萨斯看来,这种源泉就是“非生产消费者”。(同上,第35页)“马尔萨斯先生有时说什么存在着两类不同的基金:资本和收入,供给和需求,生产和消费,它们必须保持步调一致,不要互相超越。好象在生产出来的商品的总量之外还要有另外一个想必是从天上掉下来的总量,以便去购买这些生产出来的商品……马尔萨斯所要求的这种消费基金,只有牺牲生产才能取得。”(第49—50页)“他的〈马尔萨斯的〉论断使我们始终弄不清,究竟是应当增加生产还是限制生产。假如有人感到需求不足,那末马尔萨斯先生是否会劝他把钱付给别人,让别人用这笔钱购买他的商品呢?大概不会的。”(第55页)当然会的!“你出卖自己的商品,目的就是要得到一定数额的货币;如果你把这个数额的货币白送给另一个人,让他买你的商品,从而把这笔钱还给你,那就没有任何意义了。你还不如马上把你的商品烧掉,这样,你的情况也会是一样的。”(第63页)}\end{quote}

对马尔萨斯来说,匿名作者是正确的。但是,决不能从“生产出来的商品的总量”是同一基金——既是生产基金又是消费基金,既是供给基金又是需求基金,既是资本基金又是收入基金——这个论断中得出结论说,这个总基金在这些不同范畴之间怎样分配是无关紧要的。

这位匿名作者不理解,马尔萨斯所谓对于资本家来说工人的“需求”“不够”是指什么。

\begin{quote}{“至于来自劳动的需求,指的就是以劳动同商品交换,或者说……以将来在材料价值上追加的价值同现有的、现成的产品交换……这是实际的需求,它的增加对于生产者说来是十分重要的”……(同上,第57页)}\end{quote}

马尔萨斯指的不是劳动的供给(我们这位作者称之为“来自劳动的需求”),而是工人由于得到工资而能对商品提出的需求,就是工人作为买者在商品市场上出现时所拥有的货币。关于这种需求,马尔萨斯还正确地指出,对资本家的商品供给来说,它任何时候也不可能是足够的。否则,工人就能用自己的工资买回自己的全部产品。

[778]这位作者还说:

\begin{quote}{“他们〈工人〉[对工作的]需求的增加不过是表明他们甘愿自己拿走产品中更小的份额,而把其中更大的份额留给他们的雇主;要是有人说,这会由于消费减少而加剧市场商品充斥,那我只能回答说:市场商品充斥是高额利润的同义语。”(同上,第59页)}\end{quote}

按照作者的意思,这种说法象是开玩笑,但是实际上它包含着“市场商品充斥”的根本秘密。

关于马尔萨斯的《地租论》\endnote{这个匿名的李嘉图主义者所说的《地租论》,是指马尔萨斯的小册子《关于地租的本质和增长及其调整原则的研究》1815年伦敦版。——第61页。},这位作者说:

\begin{quote}{“马尔萨斯先生发表他的《地租论》,看来部分地是为了反对当时‘用红字写在墙上’的‘打倒地主!’的口号,为了起来保护这个阶级,并且证明他们与垄断者不同。他指出,地租不能废除,地租的增长通常是一种伴随财富和人口增长的自然现象;但是,‘打倒地主!’这个人民的呼声并不一定意味着不应该有象地租这样的东西,宁可说是意味着地租应该按照所谓‘斯宾斯计划’\endnote{“斯宾斯计划”是指英国空想社会主义者托马斯·斯宾斯从1775年开始鼓吹的土地国有化计划,他要求废除土地私有制,将地租(在扣除各种税款和公社的公用开支后)均等地分配给公社全体居民。——第61页。}在居民中间平均分配。但是当马尔萨斯先生着手为地主祛除垄断者这个可憎的名称和亚当·斯密关于‘他们喜欢在他们未曾播种的地方得到收获’的评语时,他似乎是在为一个名称而奋斗……在他的所有这些议论中,辩护士的气味太重了。”(同上,第108—109页)}\end{quote}

\tchapternonum{[(14)马尔萨斯著作的反动作用和剽窃性质。马尔萨斯为“上等”阶级和“下等”阶级的存在辩护]}

马尔萨斯的《人口原理》是一本攻击法国革命和与它同时的英国改革思想(葛德文等)的小册子。它对工人阶级的贫困进行辩解。理论是从唐森等人那里剽窃来的。

他的《地租论》是一本维护地主而反对产业资本的小册子。理论是从安德森那里剽窃来的。

他的《政治经济学原理》是一本维护资本家利益而反对工人,维护贵族、教会、食税者、谄媚者等等的利益而反对资本家的小册子。理论是从亚·斯密那里抄袭来的。至于他自己有所发明的地方,真是可怜之至。在进一步阐述理论时,西斯蒙第又成了依据。[XIV—778]

\centerbox{※     ※     ※}

[VIII—345]{马尔萨斯在他的《人口原理》(比·普雷沃从英文第5版译的法译本,1836年日内瓦第3版第4卷第104—105页)中以他惯用的“高深的哲理”发表了如下的见解,反对向英国茅舍贫农赠送乳牛的计划:

\begin{quote}{“有人指出有乳牛的茅舍贫农比没有乳牛的茅舍贫农更勤劳,生活更正规……现在大多数有乳牛的人,是用他们自己劳动所得购买乳牛的。所以,更正确地说,是劳动使他们得到乳牛,而不是乳牛使他们产生了对劳动的兴趣。”}\end{quote}

那末,还可以正确地说,勤劳(加上对别人劳动的剥削)使资产阶级暴发户得到乳牛,但是乳牛却使暴发户的子孙养成懒惰的习惯。如果去掉他们的乳牛支配别人无酬劳动的能力(不是产乳的能力),那末这对他们养成劳动的兴趣倒是十分有益的。

这位“高深的哲学家”说:

\begin{quote}{“很清楚,不能所有的人都属于中等阶级。有上等阶级和下等阶级是绝对必要的〈自然,没有两头就没有中间〉,而且有这两个阶级存在是非常有益的。如果在社会上人们不能指望上升,也不害怕下降,如果劳动没有奖赏,懒惰不受惩罚,人们就无法看到为改善自己的处境而表现出的那种勤奋和热情,而这是[346]社会幸福的极重要的动力。”(同上,第112页)}\end{quote}

必须有下等人,上等人才会害怕下降,必须有上等人,下等人才能指望上升。为了使懒惰受到惩罚,工人必须贫困,食利者和马尔萨斯十分心爱的土地所有者必须富有。可是马尔萨斯所谓的劳动的奖赏是什么呢?正如我们以后将要看到的\fnote{见本册第26—27、31、36—40等页。——编者注},马尔萨斯指的就是工人必须献出自己的一部分劳动而得不到任何等价物。如果成为刺激的是“奖赏”,而不是饥饿,那真是美妙的刺激了。上述一切最多不过是归结为:有的工人可以指望有朝一日也能剥削工人。

\begin{quote}{卢梭说:“垄断越扩大,被剥削者身上的锁链就越沉重。”\endnote{在卢梭的著作中没有找到这句话。——第63页。}}\end{quote}

“高深的思想家”马尔萨斯却不这样认为。他的最高希望是,中等阶级的人数将增加,无产阶级(有工作的无产阶级)在总人口中占的比例将相对地越来越小(虽然它的人数会绝对地增加)。马尔萨斯自己认为这种希望多少有点空想。然而实际上资产阶级社会的发展进程却正是这样。

\begin{quote}{马尔萨斯说:“我们应当抱这样的希望,近年来已有很大发展的节省劳动的方法,终将有一天能以比现在少的人类劳动满足最富裕的社会的一切需要;如果说,即使到那时工人还摆脱不了目前压在他们身上的重担〈他们仍将辛勤地做目前一样多的工作,并且,相对地说,为他人的越来越多,为自己的越来越少〉,那末,承担沉重的劳动的人数毕竟会减少。”(同上,第113页)}[VIII—346]}\end{quote}

\tchapternonum{[(15)匿名著作《政治经济学大纲》对马尔萨斯的经济理论原理的阐述]}

[XIV—778]1832年在伦敦匿名出版的《政治经济学大纲。略论财富的生产、分配和消费的规律》是一部阐述马尔萨斯原理的著作。

本书作者\fnote{即约翰·卡泽诺夫。——编者注}一开头就指出了马尔萨斯主义者反对价值由劳动时间决定的实际动机。

\begin{quote}{“关于劳动是财富的唯一源泉的学说,看来既是错误的,又是危险的,因为它不幸给一些人提供了把柄,他们可以断言一切财产都属于工人阶级,别人所得的部分仿佛都是从工人阶级那里抢来和骗来的。”(上述著作,第22页注)}\end{quote}

匿名作者下面说的话,比马尔萨斯著作更明显地表现出把商品的价值和商品或货币作为资本的价值增殖混为一谈。在后一意义上,它正确地表达了剩余价值的起源:

\begin{quote}{“资本的价值,即资本所值的劳动量或者说所能支配的劳动量,总是大于耗费在资本上的劳动量,这个差额就构成利润,或者说资本所有者的报酬。”(同上,第32页)}\end{quote}

从马尔萨斯那里吸收来的下述关于为什么在资本主义生产条件下利润应列入生产费用的论点,也是正确的:

\begin{quote}{“所使用的资本的利润{“如果得不到这种利润,那就没有生产商品的足够动机”}是供给的重要条件,而且它作为这样的条件成为生产费用的一个组成部分。”(同上,第33页)}\end{quote}

下面这段话,一方面包含着资本的利润直接产生于资本同劳动的交换这一正确思想,另一方面也阐述了马尔萨斯关于出卖中创造利润的学说:

\begin{quote}{“一个人的利润,不是取决于他对别人的劳动产品的支配,而是取决于他对这种劳动本身的支配。〈这里正确地区分了商品同商品的交换和作为资本的商品同劳动的交换。〉在工人的工资不变的情况下,如果他[779]〈在货币价值降低的情况下〉能以较高的价格出售他的商品,显然他就会从中获得利益,而不管其他商品是否涨价。他只要用他的产品的较小部分,就足以推动这种劳动,因而更大部分的产品就留给他自己了。”(同上,第49—50页)}\end{quote}

这种情况也会发生在下述场合:例如,一个资本家采用了新的机器、新的化学过程等等,生产出的商品低于原来的价值,而他却按照原来的价值出卖,或者至少高于现在降低了的个别价值出卖。在这种场合,当然工人不是直接为自己劳动了更短的时间,为资本家劳动了更长的时间。但是,在再生产过程中,“他只要用他的产品的较小部分,就足以推动这种劳动”。可见,工人实际上是用他的比过去更大的一部分直接劳动来换取他所得到的物化劳动。例如,他仍和以前一样得到10镑。但是,这10镑——尽管对社会来说代表同样多的劳动量——不再是以前同样多的劳动时间(也许少了一小时)的产品。因此,工人实际上为资本家劳动了更长的时间,为自己劳动了更短的时间。这就等于他现在总共只得到8镑,但是这8镑由于他的劳动生产率的提高所代表的使用价值量和原来10镑一样多。

对于上面提到的[詹姆斯·]穆勒关于需求和供给等同的论点\fnote{见本卷第2册第562—563、575—576页以及本册第57—58、106—109页。——编者注},匿名作者指出:

\begin{quote}{“每个人的供给,取决于他提供到市场上的数量;他对其他物品的需求,取决于他的供给的价值。供给是确定的,它取决于他自己;需求是不定的,它取决于别人。供给可能保持不变,需求则发生变化。某人向市场提供100夸特谷物,每夸特在一个时期可能值30先令,在另一时期可能值60先令。供给的数量在两种情况下是一样的,但是,这个人的需求,或者说,他购买其他物品的能力,在后一场合比前一场合大一倍。”(同上,第111—112页)}\end{quote}

关于劳动和机器的关系,匿名作者指出:

\begin{quote}{“当商品由于更合理的分工而增多时,无需比以前更大的需求,就可以维持先前使用的全部劳动}\end{quote}

(怎么会这样?如果分工更合理,那末,用同量的劳动就会生产更多的商品。因此,供给会增加,为了吸收供给,难道不要扩大需求吗?难道亚·斯密说的分工取决于市场规模不对吗?其实,谈到[必须增加]外来需求,那末,[实行更合理的分工和采用机器这两种情况]在这方面是没有差别的,只是在采用机器的情况下[需求的增加必须]有更大规模。不过,“更合理的分工”可能需要同以前一样多的甚至更多的工人,而采用机器,在任何情况下都会减少用在直接劳动上的那部分资本),

\begin{quote}{而当采用机器时,如果需求不增加,或者工资或利润不降低,那末部分工人无疑会失业。我们假定有价值1200镑的商品,其中1000镑是100个工人的工资(每人10镑),200镑(按利润率20%计算)为利润。现在假定,用50个使用机器的工人的劳动可以生产出同样的商品,机器的价值等于其余50个工人的劳动,并需要10个工人来维修;这时生产者可以把他所生产的商品的价格降低到800镑,而他的资本所得的报酬仍然不变。50个工人的工资…………………………500镑维修机器的10个工人的工资……………100镑500镑流动资本的20%利润200镑500镑固定资本的20%利润共计800镑”}\end{quote}

{(“维修机器的10个工人的工资”,在这里代表机器的年磨损。否则,这种算法就是错误的,因为维修机器的劳动应加在机器的最初生产费用上。)从前,企业主每年支出1000镑,但产品当时的价值是1200镑。现在,他一次就把500镑投在机器上,因而他用不着再以任何其他方式支出这笔钱了。他每年支出的,就是用于“维修机器”的100镑和用于工资的500镑(因为本例中没有原料一项)。他每年只须支出600镑,但是他的总资本照旧得到200镑利润。利润额和利润率都没有变。但是,他的年产品总共只有800镑。}

\begin{quote}{“从前为商品支付1200镑的人,现在可以节省400镑,这笔钱他可以花在别的方面,或者购买较多的同一种商品。如果这笔钱花在[780]直接劳动的产品上,只能给33.4个工人提供就业机会,但由于采用机器而失业的工人是40名。因为33.4个工人的工资(按每人10镑计算)…………………………334镑20%的利润………………………………………………………………66镑共计400镑”}\end{quote}

{换句话说,这就是:如果400镑花在作为直接劳动产品的商品上,而且每个工人的工资是10镑,那末价值400镑的商品应当是不到40个工人的劳动的产品。如果这些商品是40个工人的产品,那末,它们就只包含有酬劳动了。劳动的价值(或者说,物化在工资中的劳动量)就会等于产品的价值(物化在商品中的劳动量)。但是,400镑商品包含着正是构成利润的无酬劳动。所以,这些商品应当是不到40个工人的产品。如果利润是20%,那末只有5/6的产品可以由有酬劳动构成,约334镑,按每人10镑计算,相当于33.4个工人。而1/6的产品,约66镑,是无酬劳动。李嘉图完全以同样的方式证明:即使机器的货币价格同它所代替的直接劳动的价格一样高,机器在任何时候都不可能是同样多的劳动的产品。\fnote{参看本卷第2册第628—629页。——编者注}}

\begin{quote}{“如果这笔钱〈即上述的400镑〉用来购买更多的同一种商品,或者购买另一种用同样种类和同样数量的固定资本制造出来的商品,那末这笔钱只能给30个工人提供就业机会。因为25个工人的工资(每人按10镑计算)………………250镑维修机器的5个工人的工资……………………………50镑250镑流动资本和250镑固定资本的利润……………100镑共计400镑”}\end{quote}

{问题是这样的:在使用机器的情况下,生产价值800镑的商品要在机器上花费500镑;所以,生产价值400镑的商品在机器上只花费250镑。其次,操纵价值500镑的机器要50个工人;所以,操纵价值250镑的机器要25个工人(250镑)。再其次,“机器的维修”——价值500镑的机器的再生产——要10个工人;所以,价值250镑的机器的再生产只要5个工人(50镑)。这样一来,就是固定资本250镑和流动资本250镑,共计500镑。这笔资本的利润,按利润率20%计算,是100镑。于是,产品包含着300镑的工资和100镑的利润,共计400镑。在这种情况下,被雇的工人是30名。在这里无论如何应该假定,资本家(从事生产的)或者从消费者存在银行家那里的积蓄(400镑)中借用了资本,或者他自己除了等于消费者所积蓄的收入400镑之外还有资本。因为仅有400镑资本,他不可能在机器上花费250镑,又在工资上花费300镑。}

\begin{quote}{“当1200镑的总额花在直接劳动的产品上时,产品的价值分为工资1000镑和利润200镑〈工人人数为100名,工资为1000镑〉。当这笔钱一部分按一种方式用,一部分按另一种方式用时……产品的价值分为工资934镑和利润266镑〈即在使用机器的企业里的工人为60名,不使用机器劳动的工人为33.4名,工人总数为93.4名,他们总共得到934镑的工资〉。最后,按照第三种假定即全部款项花在机器与劳动的共同产品上,产品价值便分为900镑的工资〈因为在这种情况下工人人数为90名〉和300镑的利润。”(同上,第114—117页)[781]“资本家除非积累更多资本,否则在采用机器后他便不能使用和以前一样多的劳动。但是,这种物品的消费者在物品的价格下降后所积蓄的收入,将会增加他们对这种或其他某种物品的消费,从而能造成对一部分被机器排挤的劳动的需求,虽然不是对全部这种劳动的需求。”(同上,第119页)“麦克库洛赫先生认为,一个生产部门采用机器,必然会在其他某一生产部门造成同样大的或更大的对被解雇的工人的需求。为了证明这一点,他假定,到机器全部磨损以前为补偿机器价值所必需的年提成,每年都将造成越来越多的对劳动的需求。\endnote{匿名著作《政治经济学大纲》的作者(卡泽诺夫)在这里提到的麦克库洛赫的话,见约·雷·麦克库洛赫《政治经济学原理》1825年爱丁堡版第181—182页。参看本册第183页。——第68页。}但是,到一定时期末了,年提成加在一起只能等于机器的原有价值加机器使用期间的利息,因此很难理解,这种提成到底怎么会造成一个比不采用机器时更大的对劳动的需求。”(同上,第119—120页)}\end{quote}

当然,当机器的磨损只是在计算中而实际尚未发生作用的那段时间里,折旧基金本身也可以作积累之用。但是这样造成的对劳动的需求,无论如何也比全部投入机器的资本——而不只是补偿机器每年磨损所必须的年提成——用于工资时所产生的需求小得多。麦克彼得\fnote{把“麦克库洛赫”写作“麦克彼得”,含有嘲笑的意思。“彼得”一词,来自德文“dummerPeter”(直译是:“笨蛋彼得”,意为“蠢货”、“笨蛋”)。——编者注}始终是一头蠢驴。这段话所以值得注意,只是因为这里说出了折旧基金本身就是积累基金这个思想。

\tchapternonum{[第二十章]李嘉图学派的解体}

\tchapternonum{(1)罗·托伦斯}

\tsubsectionnonum{[(a)斯密和李嘉图论平均利润率和价值规律的关系]}

[782]罗·托伦斯《论财富的生产》1821年伦敦版。

对竞争——生产的外部表现——的考察表明,等量资本平均说来提供等量利润,或者说,如果平均利润率既定,利润量就取决于预付资本量(而平均利润率的含义也不过如此)。

亚·斯密记录了这个事实。关于这个事实同他提出的价值理论如何联系的问题,并没有引起他丝毫的内心不安;这个问题所以没有使他不安,尤其是因为除了他的所谓的内在理论以外,他还提出了其他各种各样的理论,并且可以随便采用其中这一种或那一种。这个情况使他产生的唯一反应,就是对那种试图把利润归结为监督劳动的工资的观点进行反驳,因为,撇开其他一切情况不谈,监督劳动并不是按生产规模扩大的程度增长的,而且生产规模不扩大,预付资本的价值也能增长(例如由于原料的涨价)\endnote{马克思在《剩余价值理论》第一册中引用并分析了亚当·斯密《国富论》中的这一段话(见本卷第1册第70—72页)。——第70页。}。在斯密那里没有决定平均利润和平均利润量本身的内在规律。他只限于说,竞争使这个x缩小。

李嘉图到处(除了少数的而且只是偶然的说明以外)都把利润和剩余价值直接等同起来。因此,在他看来,出卖商品之所以获得利润,并不是因为商品高于它的价值出卖,而是因为商品按照它的价值出卖。然而在考察价值方面(李嘉图的《原理》第一章),是他第一个一般地考虑到商品的价值规定同等量资本提供等量利润这一现象的关系。等量资本所以能够提供等量利润,只是因为它们生产的商品尽管不是按相同的价格出卖(然而可以说,如果把固定资本中没有被消费的部分的价值加到产品价值上,结果就会有相同的价格),但提供的剩余价值相同,提供的价格超过预付资本价格的余额相同。而且,李嘉图第一个注意到,同量资本决非具有相同的有机构成。他所理解的这种构成上的区别,是他从亚·斯密那里找到的区别即流动资本和固定资本,也就是说,他只看到从流通过程中产生的区别。

李嘉图根本没有直接说,有机构成不同从而推动的直接劳动量不同的各资本生产价值相同的商品并提供相同的剩余价值(他把剩余价值和利润等同起来)这一事实,同价值规律乍看起来是矛盾的。相反,他是以资本和一般利润率的存在为前提去研究价值的。他一开始就把费用价格和价值等同起来,而没有看到,这个前提一开始就同价值规律乍看起来是矛盾的。他只是根据这种包含着主要矛盾和基本困难的前提去考察个别的情况——工资的变动,即工资的提高或降低。为了使利润率保持不变,工资的提高或降低(与之相适应的是利润的下降或提高)必须对有机构成不同的资本发生不同的影响。如果工资提高,从而利润下降,那末用较大比例的固定资本生产的商品的价格就下降。反之,结果也相反。因此,各商品的“交换价值”在这种情况下不是由生产各该商品所需要的劳动时间决定。换句话说,有机构成不同的资本具有相同的利润率这个规定(不过,李嘉图只是在个别的情况下并且通过那样曲折的途径才得出这个结论),同价值规律是矛盾的,或者象李嘉图所说,成为价值规律的例外;对此马尔萨斯正确地指出,随着工业的发展[783],李嘉图的规则成了例外,而例外成了规则。\fnote{见本册第25页。——编者注}在李嘉图那里,矛盾本身没有表达清楚,即没有以下列形式表达:尽管一种商品比另一种商品包含的无酬劳动多,——因为在对工人的剥削率相同时,无酬劳动量取决于有酬劳动量,就是说,取决于所使用的直接劳动量,——但是它们提供的价值相同,或者说,提供的无酬劳动超过有酬劳动的余额相同。相反,矛盾在他那里只是以这种独特的形式出现:在某些情况下,工资——工资的变动——影响商品的费用价格(他说,影响交换价值)。

同样,资本的周转时间的区别,——资本不论是在生产过程中(即使不是在劳动过程中)\endnote{马克思在他的1857—1858年手稿中谈到关于特别是在农业中存在的生产时间和劳动时间的区别,以及与此有关的资本主义在农业中发展的特点(见卡·马克思《政治经济学批判大纲》1939年莫斯科版第560—562页)。生产期间(除了劳动时间以外,还包括劳动对象仅仅接受自然界的自然过程的作用的时间),这个概念马克思在《资本论》第二卷第二篇第十三章作了详细的阐述。参看本愿第2册第19页。——第72页。}还是在流通过程中停留时间较长,它为了本身的周转所需要的都不是更多的劳动,而是更多的时间,——对于利润的均等也毫无影响。这又和价值规律相矛盾,——照李嘉图说来,这又是价值规律的例外。

可见,李嘉图把问题阐述得非常片面。如果他以一般的形式来表达,他也就会使问题得到一般的解决。

但是,李嘉图仍然有很大的功绩:他觉察到价值和费用价格之间存在差别,并在一定的场合表述了(尽管只是作为规律的例外)这个矛盾:有机构成不同的资本,就是说,归根结蒂始终是那些使用不同量活劳动的资本,提供相同的剩余价值(利润),而且,——如果把一部分固定资本进入劳动过程而不进入价值形成过程这一情况撇开不谈,——提供相同的价值即具有相同价值(更确切地说是费用价格,但是李嘉图把它们混淆了)的商品。

\tsubsectionnonum{[(b)托伦斯在价值由劳动决定和利润源泉这两个问题上的混乱。局部地回到亚·斯密那里和回到“让渡利润”的见解]}

我们在前面已经看到\fnote{见本册第4页和第22—25页。——编者注},马尔萨斯利用这个[由大卫·李嘉图发现的关于价值规律和构成不同的资本有相同利润这一事实之间的矛盾]来否定李嘉图的价值规律。

托伦斯在他的著作一开头就从李嘉图的这个发现出发,但是决不是为了解决问题,而是为了把“现象”本身说成是现象的规律。

\begin{quote}{“假定所使用的是耐久程度不同的资本。如果一个毛织厂主和一个丝织厂主各使用2000镑资本,前者把1500镑花在耐用的机器上,500镑用在工资和材料上,而后者花在耐用的机器上的只有500镑,花在工资和材料上的是1500镑。假定这种固定资本每年消费1/10,利润率是10%。因为毛织厂主的2000镑资本必须有2200镑的进款,才给他提供10%的利润,又因为固定资本的价值经过生产过程从1500镑减少到1350镑,所以生产的商品必须卖850镑。同样,因为丝织厂主的固定资本经过生产过程减少了1/10,即由500镑减少到450镑,所以为了要给他的2000镑总资本提供普通利润率,所生产的丝就必须卖1750镑……如果所使用的是量相同而耐久程度不同的资本,那末,一个生产部门生产的商品连同资本余额,跟另一个生产部门生产的产品和资本余额,在交换价值上将是相等的。”(托伦斯《论财富的生产》1821年伦敦版第28—29页)}\end{quote}

这里只是指出了,记录了竞争中暴露出来的现象。同样,这里只是假定了一个“普通利润率”,而没有解释它从哪里来,甚至也没有觉察到必须加以解释。

\begin{quote}{“等量资本,或者换句话说,等量积累劳动,往往推动不等量的直接劳动;但是这丝毫不改变事情的本质”,(第29、30页)}\end{quote}

就是说,不改变下述情况:产品的价值加上没有被消费的资本余额提供相等的价值,或者同样可以说,提供相等的利润。

托伦斯这个论点的功绩不在于他在这里也只是再次把现象记录下来而不加解释,而是在于,他确定了资本之间的差别是等量资本推动不等量的活劳动,尽管他把这说成“特殊”情况而又把事情弄糟了。如果价值等于生产商品所花费的、物化在商品中的劳动,那就很清楚,在商品按它的价值出卖时,商品中包含的剩余价值只能等于其中包含的无酬劳动,或者说剩余劳动。但是在对工人的剥削率相同的情况下,这种剩余劳动量,对“推动不等量的直接劳动”的资本来说——不管这种不等是由直接的生产过程引起,还是由流通时间引起——是不可能相同的。因此,托伦斯的功绩就在于他作了这种表述。他由此作出什么结论呢?结论是,在这里,[784]在资本主义生产中,价值规律发生了一个突变,就是说,由资本主义生产中抽象出来的价值规律同资本主义生产的现象相矛盾。而他用什么来代替这个规律呢?什么也没有,他只不过对应该解释的现象作了粗浅的缺乏思考的文字上的表述。

\begin{quote}{“在社会发展的初期〈就是说,正好是交换价值——作为商品的产品——一般说来几乎没有发展,因而价值规律也没有发展的时期〉商品的相对价值是由花费在商品生产上的劳动(积累劳动和直接劳动)的总量决定的。但是一旦有了资本积累,并且有了资本家阶级和工人阶级的区别,一旦在某一工业部门作为企业主出现的人自己不劳动,而预付给别人生存资料和材料,商品的交换价值就由花费在生产上的资本量,或者说积累劳动量决定了。”(同上,第33—34页)“只要两笔资本相等,它们的产品的价值就相等,不管它们所推动的,或者说它们的产品所需要的直接劳动量如何不同。如果两笔资本不等,它们的产品的价值就不等,虽然花费在它们的产品上的劳动总量完全相同。”(第39页)“因此,在资本家和工人之间发生上述分离以后,交换价值就开始由资本量,由积累劳动量决定,而不象在这种分离以前那样,由花费在生产上的积累劳动和直接劳动的总量来决定了。”(同上,第39—40页)}\end{quote}

这里,又不过是确认了以下现象:等量资本提供等量利润,或者说,商品的费用价格等于预付资本的价格加平均利润;不过是暗示了,由于“等量资本推动不等量的直接劳动”,上述这种现象乍看起来同商品价值决定于商品中包含的劳动时间这一规定是不相容的。托伦斯说资本主义生产的这种现象,只有当资本存在——资本家阶级和工人阶级出现——时,当客观的劳动条件独立化为资本时才表现出来,这是同义反复。

但是,商品生产的[必要因素]——资本家和工人、资本和雇佣劳动——的分离是怎样推翻商品的价值规律的,这一点托伦斯只是从不理解的现象中“推论”出来的。

李嘉图试图证明,资本和雇佣劳动的分离丝毫没有改变——除了某些例外——商品的价值规定,托伦斯以李嘉图的例外为依据否定了规律本身。托伦斯回到了亚·斯密那里(李嘉图的论证是反对斯密的),按照斯密的看法,诚然,“在社会发展的初期”,当人们彼此还只是作为交换商品的商品所有者相对立时,商品的价值决定于商品中包含的劳动时间,但是资本和土地所有权一形成,就不是这样了。这就是说(正如我在第一部分\endnote{马克思指《政治经济学批判》第一分册。见《马克思恩格斯全集》中文版第13卷第49页。——第75页。}已经指出的),适用于作为商品的商品的规律,只要商品一被当作资本或当作资本的产品,只要一般说来一发生商品向资本的转变,就不适用于商品了。另一方面,只有整个产品全都转化为交换价值,产品生产的构成要素本身全都作为商品加入产品,产品才全面地具有商品的形式,就是说,只是随着资本主义生产的发展并在资本主义生产的基础上,产品才全面地成为商品。因此,商品的规律应该在不生产(或只是部分地生产)商品的生产中存在,而不应该在产品作为商品存在的那种生产中存在。这个规律本身,同作为产品的一般形式的商品一样,是由资本主义生产条件中抽象出来的,而它恰恰不适用于资本主义生产。

此外,关于“资本和劳动”的分离影响价值规定的议论——撇开所谓在资本还不存在的情况下资本不能决定价格这个同义反复不谈——又是对表现在资本主义生产表面的事实的非常肤浅的转述。只要每个人都用自己的工具劳动,都自己出卖自己生产的产品{但是实际上,产品按[785]全社会规模出卖的必然性,决不会同用自己的劳动条件进行的生产相一致},无论工具的费用或他自己从事的劳动的费用就都属于他的费用。资本家的费用是由预付资本,由他花费在生产上的价值总额构成,而不是由劳动构成,这种劳动是他没有从事过的而且他花费在这种劳动上的无非是他为它所支付的。从资本家的观点看来,这是一个很好的理由,可以用来说明他们不必按照一定资本所推动的直接劳动量,而应按照他们所预付的资本量彼此计算和分配(全社会的)剩余价值。但是这个理由决不能说明,这个应这样分配和被这样分配的剩余价值是从哪里来的。

托伦斯认为商品的价值由劳动量决定,就这一点来说,他还是坚持李嘉图理论的,但是他断言,只有花费在商品生产上的“积累劳动量”才能决定商品的价值。在这里,托伦斯又陷入极端混乱了。

因此,例如,呢绒的价值由织机、羊毛等等和工资中的积累劳动决定。这一切形成呢绒生产所需要的积累劳动的构成要素。在这里“积累劳动”一词不外是物化劳动,物化劳动时间。但是,当呢绒织成,生产结束的时候,花费在呢绒上的直接劳动也就转化为积累劳动,或者说物化劳动。因此,为什么织机和羊毛的价值应当由它们包含的物化劳动(这不外是物化在一个对象中,一个产品中,一个有用物中的直接劳动)决定,而呢绒的价值却不应当这样呢?如果呢绒也作为构成要素进入新的生产,例如进入染坊或缝纫工场,那它就是“积累劳动”,上衣的价值就由工人的工资、工人的工具和呢绒的价值决定,而呢绒的价值本身则由呢绒中的“积累劳动”决定。如果我把商品作为资本,就是说,在这里把它同时作为生产条件来考察,商品的价值就归结为直接劳动,这种直接劳动叫作“积累劳动”,因为它以物化的形式存在。相反,如果我把这同一商品作为商品来考察,作为产品和生产过程的结果来考察,这种商品的价值就不是由积累在商品本身的劳动决定,而是由积累在它的生产条件中的劳动决定了。

试图用资本的价值决定商品的价值,实际上是一个很妙的循环论证,因为资本的价值等于构成资本的那些商品的价值。詹姆斯·穆勒反驳这个家伙的话是对的,他说:

\begin{quote}{“资本就是商品,说商品的价值由资本的价值决定,就等于说,商品的价值由商品的价值决定。”\endnote{马克思引的是詹姆斯·穆勒的著作《政治经济学原理》。这段话见该书第1版第74页、第2版第94页。在这里马克思大概转引自赛米尔·贝利《对价值的本质、尺度和原因的批判研究》一书(第202页),在这本书中,这段话也被看作是反对托伦斯的。——第77页。}}\end{quote}

这里还要指出下面一点。因为[在托伦斯那里]商品的价值由生产商品的资本的价值决定,换句话说,由积累和物化在这个资本中的劳动量决定,所以只有两种情况是可能的。

[在托伦斯看来,]商品包含着:第一,消费掉的固定资本的价值;第二,原料的价值,换句话说,包含[消费掉的]固定资本和原料中所包含的劳动量;第三,还包含物化在用作工资的货币或商品中的劳动量。

这样,这里只有两种情况是可能的。

包含在固定资本和原料中的“积累”劳动量,在生产过程之后和在生产过程之前是一样的。至于预付的“积累劳动”的第三部分,工人则用他的直接劳动来补偿,就是说,在这一场合,加在原料等等上的“直接劳动”在商品中,在产品中所代表的积累劳动,正好和工资中所包含的一样多。或者,这种“直接劳动”代表更多的劳动量。如果它代表更多的劳动量,那末,商品就比预付资本包含更多的积累劳动。这样,利润就正好从商品包含的积累劳动超过预付资本包含的积累劳动的余额中产生。这样,商品的价值[786]照旧决定于商品中包含的劳动量(积累劳动加直接劳动,而后者现在在商品中也是作为积累劳动,而不再作为直接劳动存在了。它在生产过程中是直接劳动,在产品中是积累劳动)。

或者[也就是在第一种情况下],直接劳动代表的只是预付在工资中的劳动量,只是这个劳动量的等价物。(如果直接劳动比这个劳动量少,那末,要说明的就不是资本家为什么获利,而是资本家怎么不亏损的问题了。)在这种情况下,利润从哪里来呢?剩余价值,即商品价值超过商品生产的构成要素的价值,或者说超过预付资本的价值的余额,从哪里产生呢?它不是从生产过程本身产生(因此只有在交换或流通过程中实现),而是从交换,从流通过程产生了。这样,我们就回到马尔萨斯那里,回到粗浅的重商主义的“让渡利润”观念。托伦斯先生也前后一贯地得出了这种观念,虽然他又是那样前后不一贯,以致不是用一个无法解释的从天上掉下来的基金(这个基金不仅构成商品的等价物,而且构成超过这个等价物的余额,它由始终能够高于商品的价值支付商品而自己并不高于商品的价值出卖商品的这种买者的资金构成)来解释这个名义价值而把问题化为乌有。托伦斯不象马尔萨斯那样前后一贯地求助于这种虚构,相反地却认定“有效需求”即支付产品的价值额,仅仅由供给产生,因而也是商品;在这里,绝对无法理解,双方都既作为卖者又作为买者怎么能同样地相互欺诈。

\begin{quote}{“对某一商品的有效需求,总是由资本的组成部分,或者说,消费者为交换这个商品而能够和愿意提供的生产商品所必需的物品的量决定的,在利润率既定时,总是由这个量来衡量的。”(托伦斯,同上第344页)“供给增长是有效需求增长的唯一原因。”(同上,第348页)}\end{quote}

马尔萨斯从托伦斯的书中引了这句话,不无理由地对这种观点提出异议。(《政治经济学定义》1827年伦敦版第59页)\fnote{见本册第58页。——编者注}

下面托伦斯关于生产费用等的论述,表明他确实得出了上述荒谬的结论:

\begin{quote}{“市场价格〈马尔萨斯称之为“购买价值”〉总是包括某一时期的普通利润率。自然价格由生产费用构成,或者换句话说,由生产或制造商品时的资本支出构成,它不可能包括利润率。”(托伦斯,同上第51页)“一个租地农场主支出了100夸特谷物,而收回120夸特,这20夸特就是利润,把这个余额或者说利润,叫作他的支出的一部分,是荒谬的……同样,工厂主收回一定量成品。这些成品的交换价值高于材料等的价值。”(第51—53页)“有效的需求在于,消费者能够和愿意通过直接的或间接的交换付给商品的部分,大于生产商品时所耗费的资本的一切组成部分。”(第349页)}\end{quote}

120夸特谷物无疑比100夸特多。但是,如果人们象在这种场合一样,只考察使用价值和使用价值所经历的过程,其实也就是生长过程或生理[787]过程,那末,说它们(虽然不是说20夸特本身,但确实是说构成这20夸特的要素)不进入生产过程,就会是错误的。否则,这20夸特就不能从生产过程中出来。除了100夸特的谷物(种子\endnote{马克思在这里是从下述假定出发的:谷物的一切生产费用,即托伦斯提出的100夸特,都是种子的支出。实际上,生产120夸特谷物所花费的种子要少得多——比如说,20或30夸特。其余70或80夸特用于支付劳动工具、肥料、工人的工资等。但是,这种情况对于马克思的论证是毫无意义的。——第80页。})以外,进入使100夸特谷物转化为120夸特的过程的,还有由肥料提供的化学成分,土地内包含的盐,以及水、空气、阳光。这些要素、成分、条件(使100夸特转化为120夸特的自然界的支出)的转化和进入,是在生产过程本身进行的,而这20夸特的要素是作为生理的“支出”进入这个过程本身的,从100夸特转化为120夸特就是这个过程的结果。

单从使用价值的观点看,这20夸特不纯粹是利润。这不过是无机要素被有机部分同化并转化为有机物质罢了。没有作为生理支出的物质加入,无论如何决不会由100夸特变为120夸特。因此,事实上可以说(即使单从使用价值的观点来看,从谷物作为谷物来看),以无机的形式作为“支出”加入谷物的东西,则以有机的形式作为现有的结果20夸特,即收获的谷物超过播种的谷物的余额出现。

但是,这种考察方法本身,同利润问题毫无关系,就好比我们不能说,通过劳动过程把金属拔成一千倍长的金属丝,因为它的长度增加到一千倍,就代表一千倍的利润。就金属丝来说,长度增加了;就谷物来说,夸特数增加了。但是,只和交换价值有关的利润既不是由增加的长度也不是由增加的数量形成,虽然这个交换价值也表现为剩余产品。

至于交换价值,则无须再说明:90夸特谷物的价值可能丝毫不小于(甚至大于)100夸特的价值,100夸特的价值可能大于120夸特的价值,120夸特的价值可能大于500夸特的价值。

可见,托伦斯是根据一个同利润即同产品价值超过预付资本价值的余额毫无关系的例子得出关于利润的结论的。即使从生理方面,从使用价值的观点看,他的例子也是错误的,因为实际上在他那里,作为剩余产品出现的20夸特谷物已经以这种或那种方式(虽然是以另外的形式)存在于生产过程本身了。

不过,托伦斯最后还是脱口说出了利润是让渡利润这种陈旧的天才的观念。

\tsubsectionnonum{[(c)托伦斯和生产费用的概念]}

托伦斯一般地提出了什么是生产费用这个争论问题,这是他的功绩。李嘉图经常把商品的价值同生产费用(就它等于费用价格而言)混淆起来,因此,他看到萨伊虽然也认为生产费用决定价格但是得出不同的结论\fnote{见本卷第2册第535—536页。——编者注},就感到惊奇。马尔萨斯同李嘉图一样,认定商品的价格由生产费用决定,而且他同李嘉图一样把利润算在生产费用之内。但是他用完全不同的方式给价值下定义,就是说,他不用商品中包含的劳动量,而用商品能够支配的劳动量来决定价值。

生产费用这个概念的含混是由资本主义生产的性质本身引起的。

第一,对于资本家来说,他所生产的商品的费用,自然是他为商品所花费的东西。除了预付资本的价值以外,商品没有花费他任何东西,就是说,他没有在商品上支出其他任何价值。如果他为了生产商品,在原料、工具、工资等上面支出100镑,商品就花费他100镑,不会多些,也不会少些。除了包含在这笔预付中的劳动,就是说,除了包含在预付资本中的、决定为生产过程预付的商品的价值的积累劳动,商品不花费他任何劳动。他为直接劳动花费的,是他为直接劳动支付的工资。除了工资,直接劳动没有花费他任何东西,而除了直接劳动,他只预付不变资本的价值。

[788]托伦斯就是从这个意义上理解生产费用的,而每个资本家在计算利润时(不管利润率如何),也是从这个意义上理解生产费用的。

在这里,生产费用等于资本家的预付,等于预付资本的价值,等于为生产过程预付的商品中包含的劳动量。任何一个经济学家,李嘉图也在内,都是从预付、支出等意义上来使用生产费用的这个定义的。马尔萨斯把这叫作生产价格,而同购买价格相对立。剩余价值向利润形式的转化同预付的这个定义是相适应的。

第二,按第一个定义,生产费用是资本家在生产过程中为制造商品所支付的价格,因而是他为商品所花费的东西。但是,资本家为生产商品所花费的东西和商品生产本身所花费的东西是完全不同的两回事。资本家为生产商品支付过报酬的劳动(物化劳动和直接劳动)和生产商品所必要的劳动在量上是完全不同的。它们之间的差额也就形成预付价值和所得价值之间的差额,形成资本家购买商品的价格和商品出卖的价格(如果商品按照它的价值出卖)之间的差额。如果这个差额不存在,货币或商品就决不会转化为资本。随着剩余价值的消失,利润的源泉也就消失。商品生产本身的费用是由商品生产过程中消费的资本的价值,也就是由进入商品的物化劳动量加花费在商品生产上的直接劳动量构成的。在商品中消费的“物化劳动”和“直接劳动”的总量构成商品生产本身的费用。只有通过这个物化的和直接的劳动量的生产消费,商品才能制造出来。这也是商品作为产品,作为商品以及作为使用价值从生产过程中出来的必要条件。在现实的劳动过程的技术条件不变时,或者同样可以说,在劳动生产力的一定的发展水平毫无变化时,不管利润或工资怎样变动,商品的这个内在的生产费用保持不变。在这个意义上,商品的生产费用等于商品的价值。花费在商品上的活劳动和资本家支付过报酬的活劳动是不同的东西。所以,对于资本家来说的商品的生产费用(他的预付)从一开始就和商品生产本身的费用,和商品的价值不同。商品的价值(就是商品本身所花费的东西)超过预付资本的价值(即资本家为商品所花费的东西)的余额形成利润,因而,利润的产生不是由于商品高于它的价值出卖,而是由于商品高于资本家所支付的预付资本的价值出卖。

商品的生产费用,即商品的内在的生产费用等于商品的价值,也就是等于商品生产所必需的(物化的和直接的)劳动时间总量——这个定义表达了商品生产的基本条件,并且在劳动的生产力不变时保持不变。

第三,但是,我在前面已经指出\fnote{见本卷第2册第19—22、27、65—70、191—261页。——编者注},每一个别行业或个别生产部门的资本家决不是按照商品即特殊行业或特殊生产部门或特殊生产领域的产品本身所包含的价值出卖商品的,因此,这个资本家得到的利润量不会和剩余价值量,剩余劳动量,或者说物化在他所出卖的商品中的无酬劳动量相等。相反,资本家在他的商品中平均说来能实现的剩余价值,只是和这种商品作为社会资本一定部分的产品所分摊到的剩余价值相等。如果社会资本等于1000,某个[789]生产部门的资本等于100,如果剩余价值(从而剩余价值物化在其中的剩余产品)的总量等于200,即20%,那末,投入这个生产部门的资本100将按照120的价格出卖它的商品,而不管这个商品的价值是120,还是多于或少于120,就是说,不管这个商品中包含的无酬劳动是否等于预付在商品上的劳动的1/5。

这就是费用价格,如果谈到本来意义上的(经济学意义上的,资本主义意义上的)生产费用,那末这就是预付资本的价值加平均利润的价值。

很清楚,不管个别商品的这种费用价格怎样偏离商品的价值,它都是由社会资本的总产品的价值决定的。各个资本,由于它们利润的平均化,作为社会总资本的一定部分相互发生关系,并且作为这样的一定部分从剩余价值(剩余产品),剩余劳动或无酬劳动的总基金中获得股息。这丝毫没有改变商品的价值,丝毫没有改变下述情况:不管商品的费用价格等于、大于或小于商品的价值,只要商品的价值没有生产出来,就是说,只要生产商品所必要的物化劳动和直接劳动的总量没有花费在商品上,商品是决不能生产出来的。这个劳动(不仅是有酬劳动而且是无酬劳动)量必须花费在商品上;虽然有些生产部门的一部分无酬劳动由“资本家同伙”\endnote{马克思在《资本论》第三卷中论证了资本家们作为“同伙”的这个特点。在利润率平均化的过程中,“每一单个资本家,同每一个特殊生产部门的所有资本家总体一样,参与总资本对全体工人阶级的剥削,并参与决定这个剥削的程度”。(见马克思《资本论》第3卷第10章)马克思在研究了这个过程后写道:“……我们在这里得到一个象数学一样精确的证明:为什么资本家在他们的竞争中表现出彼此都是虚伪的兄弟,但面对着整个工人阶级却结成真正的共济会团体。”(同上)参看本卷第2册第21页。——第84页。}占有,而不是由推动这个特殊生产部门的劳动的资本家占有,但是这丝毫不改变资本和劳动之间的一般关系。其次,很清楚:不管商品的价值和费用价格之间是什么关系,费用价格总是随着价值的变动,也就是随着生产商品所必要的劳动量的变动而变动,而提高或降低。此外,很清楚:一部分利润始终必须代表剩余价值,代表物化在这个商品本身中的无酬劳动,因为在资本主义生产基础上,每个商品包含的在它上面花费的劳动,比推动这个劳动的资本家支付过报酬的劳动多。一部分利润可能由不是花费在一定行业所提供的或一定生产领域所生产的商品上的劳动构成;但是这样一来,就有其他某个生产领域生产的其他某个商品,它的费用价格降到它的价值以下,或者说,它的费用价格中计算和支付的无酬劳动,比它包含的无酬劳动少。

因此很清楚,虽然大多数商品的费用价格必定偏离它们的价值,就是说,虽然它们的“生产费用”必定偏离它们包含的劳动总量,但是,不仅这种生产费用和费用价格由商品的价值决定,并同价值规律相符合(而不是和它相矛盾),而且甚至生产费用和费用价格的存在本身,也只有在价值和价值规律的基础上才能理解,没有这个前提,它们的存在就是不可思议的和荒谬的。

同时我们也就可以理解,那些一方面看到了竞争中的实际现象,另一方面又不理解价值规律和费用价格规律之间的中介过程的经济学家,为什么求助于虚构,说是资本而不是劳动决定商品的价值,或者确切些说,价值根本不存在。

[790]利润加入商品的生产费用;亚·斯密正确地把利润作为构成要素包括在商品的“自然价格”中,因为在资本主义生产的基础上,如果商品不提供等于预付资本价值加平均利润的费用价格,它就——最终地、照例地——不会拿到市场上去。或如马尔萨斯所说(虽然他不理解利润的起源,不理解利润的真正原因),因为利润,从而包括利润在内的费用价格,是(资本主义生产的基础上的)商品供给的条件。商品要生产出来,要进入市场,它至少必须为卖者提供这个市场价格,这个费用价格,而不管它本身的价值比这个费用价格大还是小。对于资本家来说,只要他的商品的价格中包含的从无酬劳动或固定了无酬劳动的剩余产品的总基金中取得的量,同其他任何等量资本从这个总基金中获得的量相等就行了,至于他的商品比其他商品包含的无酬劳动多还是少,那是无关紧要的。在这个意义上,资本家是“共产主义者”。自然,在竞争中每个人都力求得到比平均利润多的利润,而这只有在别人得到的利润比平均利润少的情况下才是可能的。正是由于这种斗争,平均利润才得以形成。

以预付资本(不管是不是借的)的利息形式在利润中实现的一部分剩余价值,对资本家来说,也表现为支出,表现为他作为资本家的一项生产费用,就象利润根本就是资本主义生产的直接目的一样。而在利息上(特别是对借入的资本说)这一点也表现为资本家的生产活动的实际前提。

这一点同时表明了生产形式和分配形式的区别是怎么回事。利润、分配形式,在这里同时又是生产形式、生产条件、生产过程的必要的构成要素。因此,约·斯·穆勒等把资产阶级的生产形式看成绝对的,而把资产阶级的分配形式看成相对的,历史的,因而是暂时的,是多么愚蠢,——这一点以后还要回过头来谈。分配形式只不过是从另一个角度看的生产形式。构成资产阶级分配的界限的特征——也就是特殊的局限性——作为控制生产和支配生产的特定性质加入生产本身。但是,资产阶级的生产,由于它本身的内在规律,一方面不得不这样发展生产力,就好象它不是在一个有限的社会基础上的生产,另一方面它又毕竟只能在这种局限性的范围内发能生产力,——这种情况是危机的最深刻、最隐秘的原因,是资产阶级生产中种种尖锐矛盾的最深刻、最隐秘的原因,资产阶级的生产就是在这些矛盾中运动,这些矛盾,即使粗略地看,也表明资产阶级生产只是历史的过渡形式。

其次,这一点被例如西斯蒙第粗浅地但又相当正确地看成是为生产的生产同因此\fnote{即因为为生产的生产,而不是为工人生产者的生产。——编者注}而排除了生产率的绝对发展的分配之间的矛盾。

\tchapternonum{(2)詹姆斯·穆勒[解决李嘉图体系的矛盾的不成功的尝试]}

[791]詹姆斯·穆勒《政治经济学原理》1821年伦敦版(1824年伦敦第2版)。

穆勒是第一个系统地阐述李嘉图理论的人,虽然他的阐述只是一个相当抽象的轮廓。他力求做到的,是形式上的逻辑一贯性。“因此”,从他这里也就开始了李嘉图学派的解体。在老师[李嘉图]那里,新的和重要的东西,是在矛盾的“肥料”中,从矛盾的现象中强行推论出来的。作为他的理论基础的矛盾本身,证明理论借以曲折发展起来的活生生的根基是深厚的。而学生[穆勒]的情况却不是这样。他所加工的原料已不再是现实本身,而是现实经老师提炼后变成的新的理论形式了。一部分是新理论的反对者们的理论上的不同意见,一部分是这种理论同现实的往往是奇特的关系,促使他去进行把不同意见驳倒,把这种关系解释掉的尝试。在进行这种尝试时,他自己也陷入了矛盾,并且以他想解决这些矛盾的尝试表明,他教条式地维护的理论正在开始解体。穆勒一方面想把资产阶级生产说成是绝对的生产形式,并且从而试图证明,这种生产的真实矛盾不过是表面上的矛盾。另一方面,他力图把李嘉图的理论说成是这种生产方式的绝对的理论形式,并且同样用形式上的理由把有些已为别人所指出、有些是摆在他本人眼前的理论上的矛盾辩解掉。不过,穆勒在某种程度上也还是比李嘉图的观点前进了一步,越过了李嘉图本人阐述观点时所划的界限。穆勒还维护了李嘉图所维护的历史的利益——反对土地所有权的产业资本的利益,而且更加坚决地从理论中作出了实际结论,例如,他从地租理论做出了反对土地私有权存在的实际结论,他想或多或少直接地把土地私有变为国有。穆勒的这个结论和他这方面的观点,我们不打算在这里研究。

\tsectionnonum{[(a)把剩余价值同利润混淆起来。利润率平均化问题上的烦琐哲学。把对立的统一归结为对立的直接等同]}

在李嘉图的学生们那里,也象在李嘉图本人那里一样,看不到剩余价值和利润的区别。李嘉图本人只是在工资的变动可能对有机构成不同(这在李嘉图那里也只是涉及流通过程时谈到)的各资本产生的不同影响中,才注意到两者的区别。无论是李嘉图本人还是他的学生们都没有想到,即使我们考察的不是各个不同部门的资本,而是单独的每一笔资本,只要它不是仅仅由可变资本构成,不是只花在工资上的资本,那末,利润率和剩余价值率就有区别,因而利润就必然是剩余价值的一种进一步发展了的、发生了特殊变化的形式。只是在谈到不同生产领域的、由不同比例的固定部分和流动部分构成的各资本的相等的利润——平均利润率时,他们才注意到剩余价值和利润的区别。在这方面,穆勒只是把李嘉图在第一章《论价值》中陈述的东西重复一遍,加以通俗解释罢了。穆勒在这个问题上产生的唯一新的疑问是:

穆勒指出,“时间本身”(就是说,不是劳动时间,而是单纯时间)不生产任何东西,因此也不生产“价值”。在这种情况下,一笔资本,象李嘉图所说,由于需要更长的时间进行周转,和另一笔使用了更多的直接劳动但周转得更快的资本提供同样多的利润这个事实,怎么会同价值规律相符合呢?我们看到,穆勒在这里只是抓住了一个非常个别的情况,这个情况概括起来可以这样说:费用价格以及作为它的前提的[792]平均利润率(从而包含十分不同的劳动量的商品的相同价格),怎么会同利润无非是商品中包含的劳动时间的一部分,也就是资本家不付等价物而占有的一部分这种情况相符合呢?然而在考察平均利润率和费用价格时,提出的是同价值规定毫无关系的、纯粹外在的观点,例如这样的观点:如果有个资本家的资本因为要——譬如说——象葡萄酒那样较长久地停留在生产过程中(或者,在其他场合,较长久地停留在流通过程中),必须完成较长时间的周转,那末,这个资本家就应该得到不能使他的资本增殖的那段时间的补偿。但是,没有使价值增殖的时间怎么能创造价值呢?

穆勒关于“时间”的论点是:

\begin{quote}{“时间什么也做不出来……因此,它怎么能够增加价值呢?时间只不过是一个抽象的术语。它是一个词,一种音响。无论把一个抽象的单位说成是价值尺度,还是把时间说成是价值的创造者,在逻辑上都同样是荒谬的。”\endnote{詹姆斯·穆勒书中的这一段话,马克思是从贝利《对价值的本质、尺度和原因的批判研究》一书(第217页)转引的,这从该引文与穆勒书中的原文稍有出入可以看出来。——第89页。}(《政治经济学原理》第2版第99页)}\end{quote}

其实,在说明不同生产领域的资本之间的补偿理由时,问题并不涉及剩余价值的生产,却涉及剩余价值在不同类别的资本家之间的分配。因此,在这里有意义的是同价值规定本身绝对没有任何关系的观点。在这里,迫使某一特殊生产领域的资本放弃在其他领域可能生产更多剩余价值的条件的一切,都是补偿理由。例如,使用的固定资本多而流动资本少;使用的不变资本多于可变资本;资本必须较长久地停留在流通过程中;最后还有一种情况,就是资本必须较长久地停留在生产过程中而不经历劳动过程,这种情况,每逢生产过程按其工艺性质要求中断以便使制造中的产品经受自然力的作用时(例如,葡萄酒置于窖内),都会发生。在所有这些场合,——穆勒特别注意的是其中最后一种,可见,他把他所遇到的困难看得十分狭窄,只看作一种个别现象,——都会发生补偿。其他领域生产的剩余价值,有一部分会纯粹按照这些在直接剥削劳动方面条件比较不利的资本的数量转给这些资本(这种平均化是由竞争实现的,在平均化的条件下,每一笔个别资本都只是作为社会资本的一定部分出现)。只要理解剩余价值和利润的关系,其次理解利润平均化为一般利润率,这种现象是十分简单的。但是,如果想不经过任何中介过程就直接根据价值规律去理解这一现象,就是说,根据某一个别行业的个别资本所生产的商品中包含的剩余价值即无酬劳动(也就是根据直接物化在这些商品本身中的劳动)来解释这一资本所取得的利润,那末,这就是一个比用代数方法或许能求出的化圆为方问题更困难得多的问题。这简直就是企图把无说成有。但是,穆勒正是企图用这种直接的形式来解决问题。因此,这里实质上不可能解决问题,而只能口头上诡辩地把困难辩解掉,就是说,只能是烦琐哲学。穆勒开了这个头。而在麦克库洛赫这样一个无耻之徒那里,这种做法就具有故作高深的无耻性质了。

贝利的话最能说明穆勒提出的解决问题的办法:

\begin{quote}{“穆勒先生作了一次把时间的作用归结为耗费劳动的独特尝试。他说(《原理》1824年第2版第97页):‘如果窖中的葡萄酒因放置一年而价值增加1/10,那末,认为在葡萄酒上多耗费了1/10的劳动,是正确的。’……一件事只有当它确实发生了,[793]认为它已经发生才是正确的。在所举的例子中,根据假定,任何人都没有接近过葡萄酒,没有为它花费一刹那时间,或稍微动一动肌肉。”(《对价值的本质、尺度和原因的批判研究》1825年伦敦版第219—220页)}\end{quote}

在这里,一般规律同进一步发展了的具体关系之间的矛盾,不是想用寻找中介环节的办法来解决,而是想用把具体的东西直接列入抽象的东西,使具体的东西直接适应抽象的东西的办法来解决。而且是想靠捏造用语,靠改变事物的正确名称来达到这一点。(我们看到的确实是一场“用语的争论”\endnote{暗指反对“用语的争论”的匿名论战著作《评政治经济学上若干用语的争论,特别是有关价值、供求的争论》1821年伦敦版。马克思在后面(本章第3节)对这本匿名著作作了详细的评述。——第91页。},但是它之所以是“用语的”,是因为他们企图用空话来解决没有得到实际解决的实际矛盾。)这种手法在穆勒那里还只是处于萌芽状态;它比反对者的一切攻击更严重得多地破坏了李嘉图理论的整个基础,这一点在考察麦克库洛赫时可以看出来。

穆勒只是在他绝对找不到其他出路的时候,才求助于这种方法。但是,他的基本方法与此不同。在经济关系——因而表示经济关系的范畴——包含着对立的地方,在它是矛盾,也就是矛盾统一的地方,他就强调对立的统一因素,而否定对立。他把对立的统一变成了这些对立的直接等同。

例如,商品隐藏着使用价值和交换价值的对立。这种对立进一步发展,就表现为、实现为商品的二重化即分为商品和货币。商品的这种二重化作为过程出现在商品的形态变化中,在这种变化中,卖和买是一个过程的不同因素,但是这一过程的每一行为同时都包含着它的对立面。我在本书的第一部分\endnote{马克思指《政治经济学批判》第一分册。见《马克思恩格斯全集》中文版第13卷第87—88页。——第92页。}曾经指出,穆勒摆脱对立的办法是仅仅抓住买和卖的统一,从而把流通变成物物交换,又把从流通中搬来的范畴偷偷塞到物物交换里。还可参看我在那里关于他的货币理论所说的话,他在货币理论中对问题也采取了类似的态度。\endnote{同上,第169—172页。——第92页。}

在詹姆斯·穆勒那里有一些不适当的章节划分:《论生产》,《论分配》,《论交换》,《论消费》。

\tsectionnonum{[(b)穆勒使资本和劳动的交换同价值规律相符合的徒劳尝试。局部地回到供求论]}

关于工资,穆勒写道:

\begin{quote}{“人们发现,对工人说来,更加方便的是以预付的方式把工人的份额付给工人,而不是等到产品生产出来和产品的价值得到实现的时候。人们发现,适合于工人取得其份额的形式是工资。当工人以工资的形式完全得到了产品中他应得的份额时,这些产品便完全归资本家所有了,因为资本家事实上已经购买了工人的份额,并以预付的方式把这一份额支付给工人了。”(《政治经济学原理》,帕里佐的法译本,1823年巴黎版第33—34页)}\end{quote}

穆勒最大的特点是:正象货币在他看来只是为了方便而发明的一种手段一样,资本主义关系本身,按照他的意见,也是为了方便而想出来的。这种特殊的社会生产关系,是为了“方便”而发明出来的。商品和货币转化为资本,是由于工人不再以商品生产者和商品所有者的身分参加交换,相反,他们被迫不是出卖商品,而是把自己的劳动本身(直接把自己的劳动能力)当作商品卖给客观的劳动条件的所有者。工人与客观劳动条件的这种分裂,是资本和雇佣劳动的关系的前提,正象它是货币(或代表货币的商品)转化为资本的前提一样。穆勒以这种分离,这种分裂为前提,以资本家和雇佣工人的关系为前提,为的是以后再把以下的现象说成是方便的事情:工人不是出卖产品,不是出卖商品,而是出卖在他生产出产品之前他自己在产品(产品的生产丝毫不决定于他,生产的进行不由他作主)中所占的份额,[794]或者更确切地说,在资本家出卖、销出包含工人的份额的产品之前,工人在产品中所占的份额就已经由资本家支付了,就已经转化为货币了。

穆勒想通过这种对工资的观点来回避与这里所考察的关系的特殊形式相联系的特殊困难。对于认为工人是直接出卖自己的劳动(而不是出卖自己的劳动能力)的李嘉图体系来说,困难在于:既然商品的价值决定于生产该商品所耗费的劳动时间,那末在构成资本主义生产基础的、一切交换中最大的交换——资本家和雇佣工人之间的交换中,为什么这个价值规律不实现呢?为什么工人以工资形式取得的物化劳动量不等于他为换取工资而付出的直接劳动量呢?为了排除这个困难,穆勒把雇佣工人变成了商品所有者,说他向资本家出卖自己的产品,自己的商品,因为他在产品,商品中所占的份额是他的产品,他的商品,是他以特殊商品的形式生产出来的价值。他解决困难的方法是把包含着物化劳动和直接劳动的对立的、资本家和雇佣工人之间的交易,说成物化劳动的所有者之间、商品所有者之间的普通交易。

尽管穆勒由于耍了这样一个花招而使自己不可能理解资本家和雇佣工人之间发生的过程的特殊性质、特点,但是他决没有给自己减少困难,却增加了困难,因为,对于结果的特殊性,现在已经不可能依据工人所出卖的商品(它具有这样的特性:它的使用价值本身是形成交换价值的要素,因而这个商品的消费创造出比它本身所包含的更多的交换价值)的特殊性来理解了。

在穆勒看来,工人是和任何其他商品所有者一样的商品出卖者。例如,他生产6码麻布。这6码中2码所代表的价值等于他所加入的劳动。因此,他是把2码麻布卖给资本家的卖者。既然这时工人是和其他任何麻布所有者一样的麻布卖者,那他为什么不能象其他任何出卖2码麻布的卖者一样从资本家那里全部取得2码麻布的价值呢?恰恰相反,与价值规律的矛盾这时表现得尖锐得多。工人这时出卖的决不是和其他一切商品不同的特殊商品。他出卖的是物化在产品中的劳动,也就是这样的商品:它作为商品并没有由于具有某种特点而与其他任何商品不同。这样,如果1码麻布的价格——即代表1码麻布中包含的劳动时间的货币量——是2先令,那末为什么工人所得到的是1先令,而不是2先令呢?如果工人得到2先令,资本家就不能实现剩余价值,李嘉图体系也就全部被推翻了。我们也就会被迫退回到“让渡利润”。6码麻布使资本家花费的,等于它的价值,即12先令。但是资本家按照13先令出卖它。

或者事情是这样:当资本家出卖麻布的时候,它和其他一切商品一样按照自己的价值出卖,但是当工人卖它的时候,则低于它的价值出卖,这样,价值规律就被工人和资本家之间的交易破坏了。而穆勒所以求助于虚构,恰恰是为了逃避这一点。他想把工人和资本家之间的关系变成商品的卖者和买者之间的普通关系。那末,普通的商品价值规律为什么在这里不应决定这种交易呢?但是,据说工人的报酬是“以预付的方式”支付的。可见,这里我们所看到的毕竟不是普通的商品买卖关系。这种“预付”在这里应该是什么呢?(按照穆勒的假设和按照实际情况,)一个工人,譬如说按周领工资的工人,在他从资本家那里领到对一周产品中属于他的份额的“报酬”之前,就“预付了”自己的劳动,创造了这个份额,即已把自己一周的劳动物化在产品之中。资本家“预付了”原料和劳动工具,工人“预付了”劳动,而在周末支付工资的时候,工人把商品,自己的商品,即自己在全部商品中所占的份额卖给资本家。但是,穆勒会说,资本家还在他自己把6码麻布转化为货币,卖出去之前,就付款给工人了,也就是替工人把2码[795]麻布转化为银,转化为货币了!可是,如果资本家制造的是定货,如果他还在商品生产出来以前就已经把它卖出,那又当如何呢?更广泛地说,资本家向工人买这2码麻布是为了再卖,而不是为了自己消费,这对于工人——在这里是2码麻布的卖者,有什么关系呢?买者的动机同卖者有什么相干呢?买者的动机又怎么可能进而使价值规律也发生变化呢?如果前后一贯的话,那就必须承认,每个卖者都应当低于商品的价值出卖他的商品,因为他交给买者的是使用价值形式的产品,而买者交给他的是货币形式的价值,转化为银的产品形式。在这种情况下,麻织厂主也应当少付给麻纱商人、机器厂主、煤炭生产者等等。因为他们卖给他的商品,是他不过准备使之转化为货币的商品,而他不仅要在自己的商品卖出之前,而且要在自己的商品生产出来之前就把这个商品的组成部分的价值“以预付的方式”付给他们。工人供给他的是麻布,是制成的、适于出卖的形式的商品;相反,上述那些商品卖者供给他的是机器、原料等等,这些东西还必须经过一定的过程才能具有适于出卖的形式。对于穆勒这个极端的李嘉图主义者(在他看来,买和卖、供和求不过是等同的,货币则不过是一种形式)来说,妙不可言的是:商品转化为货币(在把2码麻布卖给资本家时,发生的无非就是这种转化)要以卖者不得不低于商品的价值出卖他的商品,而买者用自己的货币买到比货币价值大的价值为前提。

由此可见,穆勒把事情归结为这样一个荒谬的论断:在这笔交易中,买主购买是为了再卖并赚取利润,因此卖者不得不低于商品的价值出卖他的商品;这样,整个价值理论就被推翻了。穆勒为解决李嘉图的矛盾之一而作的这第二个尝试,实际上毁掉了李嘉图体系的整个基础,特别是毁掉了这个体系的优点:把资本和雇佣劳动的关系看作积累劳动和直接劳动之间的直接交换,即从这种关系的特定性质去考察它。

为了设法摆脱困难,穆勒必须再进一步说,这里所谈的不是单纯的商品买卖的交易;工人和资本家的关系不如说是出借货币或从事贴现业务的资本家(货币资本家)对产业资本家的关系,因为这里涉及的是支付,是把等于工人在总产品中所占份额的产品转化为货币。以生息资本(资本的特殊形式)的存在为前提来解释生产利润的资本(资本的一般形式);把剩余价值的派生形式(它已经以资本为前提)说成剩余价值产生的原因,——这倒是一种满不错的解释。此外,穆勒在这种情况下必须保持前后一贯,撇开李嘉图所发展的关于工资和工资水平的一切已确定的规律,而相反地从利息率中引伸出工资;在这种情况下,实际上却又不能说明利息率应该由什么决定,因为按照李嘉图学派和其他所有值得一提的经济学家的意见,利息率是由利润率决定的。

所谓工人在他自己产品中所占的“份额”这种空话,实质上是以下述情况为根据的:如果考察的不是资本家和工人之间的单独交易,而是他们在总的再生产过程中的交换,如果注意的是这个过程的实际内容,而不是它的表现形式,那末在实际上就会看到,资本家用来支付工人的东西(以及作为不变资本同工人对立的那部分资本),不外是工人本身的产品的一部分,并且不是尚须转化为货币的那部分产品,而是已经卖出、已经转化为货币的那部分产品,因为工资是以货币而不是以实物支付的。在奴隶制度等等的条件下,不存在由于花在工资上的那部分产品先要转化为货币而产生的假象,因此看得很清楚,奴隶作为工作报酬取得的东西,实际上不是奴隶主的“预付”,而只是奴隶的物化劳动中以生活资料的形式流回到奴隶手中的部分。在资本家那里情况也是如此。他只是在表面上“预付”。他作为工资预付给工人的,或者更确切地说,他付给工人的[796]报酬——因为他要到工作完成后才付给报酬——是工人已经生产出来而且已经转化为货币的产品的一部分。资本家所占有的、从工人那里夺来的工人产品,有一部分以工资形式,作为对新产品的预付——如果愿意用这个名称的话——流回到工人手里。

抓住资本家和工人之间的交易所特有的这种表面现象来解释交易本身,对于穆勒来说是完全不相称的(这种作法对于麦克库洛赫、萨伊或巴师夏来说倒是相称的)。资本家除了他以前从工人那里夺来的东西,也就是由其他人的劳动预付给他的东西之外,没有任何东西可以用来预付给工人。要知道,甚至马尔萨斯也说,资本家所预付的东西,不是“由呢绒”和“其他商品”,而是“由劳动”构成的\endnote{托·罗·马尔萨斯《价值尺度。说明和例证》1823年伦敦版第17—18页。——第97页。},也就是恰恰由资本家没有从事过的东西构成的。资本家预付给工人的是工人自己的劳动。

但是,所有这种代用语都丝毫不能帮穆勒的忙,就是说,丝毫不能帮助他回避解决这个问题:积累劳动和直接劳动之间的交换(李嘉图以及追随他的穆勒等人就是这样理解资本和劳动之间的交换过程的)如何同直接与它矛盾的价值规律相符合。从穆勒的以下论点中可以看出,上述代用语是丝毫帮不了他的忙的:

\begin{quote}{“产品按什么比例在工人和资本家之间进行分配,或者,工资水平按什么比例调节?”(穆勒《政治经济学原理》,帕里佐的法译本,第34页)“确定工人和资本家的份额,是他们之间的商业交易的对象,讨价还价的对象。一切自由的商业交易都由竞争来调节,讨价还价的条件随着供求关系的变化而变化。”(同上,第34—35页)}\end{quote}

付给工人的是工人在产品中所占的“份额”。穆勒这样说,是为了使工人在他同资本的相互关系中变成一个普通的商品(产品)卖者,是为了掩盖这种相互关系的特殊性质。工人在产品中所占的份额,可以认为是他的产品,即工人的新加劳动物化于其中的那部分产品。但是情况并不是这样。相反,现在我们问工人在产品中所占的“份额”是怎样的,即他的产品是怎样的(因为属于工人的那一部分产品,就是他所出卖的他的产品),这时,我们就听到说,他的产品和他的产品是两种完全不同的东西。我们还应该先搞清楚工人的产品(即他在产品中所占的份额,因而也就是属于他的那一部分产品)是什么。可见,所谓“他的产品”只是一句空话,因为工人从资本家那里得到的那一份价值,并不由他本身的产量决定。所以,穆勒只是把困难推远了一步。在解决困难方面,他仍在开始研究时的出发地点踏步不前。

这里表现了概念的混淆。如果认为资本和雇佣劳动之间的交换是不断的行为——凡是不把资本主义生产的个别行为、个别因素固定化、孤立化的人都会认为它是这样的行为,那末,工人所取得的就是他自己的产品的一部分价值,他已经补偿了这部分价值,还加上了他白白送给资本家的那部分价值。这是不断反复进行的。可见,实际上工人不断取得他自己的产品的价值的一部分,他所创造的价值的一个部分或份额。他的工资多少,不决定于他在产品中所占的份额,倒是他在产品中所占的份额决定于他的工资量。工人实际上取得产品价值中的一个份额。但是,他所取得的那个份额决定于劳动的价值,而不是劳动的价值反过来决定于他在产品中所占的份额。劳动的价值,即工人本身的再生产所需要的劳动时间,是一个已固定的量;这个量是由于工人的劳动能力出卖给资本家而固定下来的。实际上,工人在产品中所占的份额也是由此固定下来的。而不是相反,不是先把他在产品中所占的份额固定下来,然后由这个份额决定他的工资的水平或价值。其实,这也正是李嘉图的最重要的、最强调的论点之一,因为,不然的话,劳动的价格就会决定劳动所生产的商品的价格,而按照李嘉图的见解,劳动的价格只决定利润率。

那末,穆勒现在如何确定工人所取得的产品“份额”呢?他用供求,用工人和资本家之间的竞争来决定它。穆勒在这里提出的说法,可以适用于一切商品:

\begin{quote}{“确定工人和资本家〈卖者和买者〉的份额〈应读作:在商品价值中的份额〉,是他们[797]之间的商业交易的对象,讨价还价的对象。一切自由的商业交易都由竞争来调节,讨价还价的条件随着供求关系的变化而变化。”(同上,第34—35页)}\end{quote}

可见,问题就在这里!这就是穆勒说的话,他这个热诚的李嘉图主义者曾经证明:需求和供给固然能够决定市场价格在商品价值上下的波动,但是不能决定商品价值本身;需求和供给如果用来决定价值,就成了两个没有意义的字眼,因为它们本身的决定要以价值的决定为前提!而现在——萨伊早已在这一点上指责过李嘉图\endnote{马克思在《剩余价值理论》第二册(见本卷第2册第454和455页)提到萨伊的“幸灾乐祸”,说这是因为李嘉图在用维持工人生活所必需的生存资料决定“劳动价值”时,引证了供求规律。这里马克思引用的李嘉图著作是康斯坦西奥译、萨伊加注的法译本。马克思在这里是不确切的。萨伊在给李嘉图著作所加的注释中“幸灾乐祸”,是因为李嘉图用供给和需求来决定货币的价值。马克思在《哲学的贫困》(见《马克思恩格斯全集》中文版第4卷第126页)中曾引了萨伊注释中有关的这段话。这段话的出处是:大·李嘉图《政治经济学和赋税原理》,康斯坦西奥译自英文,附让·巴·萨伊的注释和评述,1835年巴黎版第二卷第206—207页。——第100页。}——穆勒为了决定劳动的价值,即一种商品的价值,竟求助于用需求和供给来决定它!

但是,问题还不止于此。

穆勒没有说——其实在这种情况下这是无关紧要的——双方当中哪一方代表供给,哪一方代表需求。但是,既然资本家付出货币,工人相反地提供某种东西来交换货币,我们就可以假定需求是在资本家方面,供给在工人方面。但是,这时工人“出卖”的是什么呢?他提供的是什么呢?是他在还不存在的产品中所占的“份额”吗?但是要知道,他在未来的产品中所占的“份额”恰恰还要由他和资本家之间的竞争,由“需求和供给”的关系来决定!这个关系的一个方面,即供给,不可能由本身不过是供求斗争结果的东西构成。那末,工人到底拿出什么来卖呢?自己的劳动吗?但是这样一来,穆勒就又遇到了他想回避的最初的困难,即积累劳动和直接劳动之间的交换。当他说这里发生的不是等价物的交换,或者说所卖的商品即劳动的价值不是用“劳动时间”本身来衡量,而是由竞争,由供求来决定的时候,他也就承认,李嘉图的理论遭到破产,而李嘉图的反对者是对的,后者认为商品价值决定于劳动时间的主张是错误的,因为最重要的一种商品即劳动本身的价值同商品价值的这个规律相矛盾。我们将在下文中看到,威克菲尔德就直接说过这样的话。\fnote{见本册第205页,并见本卷第2册第453—454页。——编者注}

穆勒愿意怎样打转转和兜圈子都可以,但是他找不到摆脱这个左右为难的窘境的出路。用他本人的表达方法来说,工人的竞争最多只能使他们按照这样的价格提供一定的劳动量,这个价格依据供求关系等于他们将要用这个劳动量生产出来的产品的一个较大的或较小的部分。但是,他们用自己的劳动换取的这个价格,这个货币量等于应当生产出来的产品的价值的一个较大的或较小的部分这一点,无论如何从一开始就不妨碍一定的活劳动(直接劳动)量在这里换得一个较大的或较小的货币(积累劳动,而且是以交换价值形式存在的劳动)量。从而,这也不妨碍不等量的劳动在这里相互交换,不妨碍较少的积累劳动换取较多的直接劳动。而这些恰恰是穆勒应该加以解释的现象,也是他为了避免违反价值规律而想用自己的解释搪塞过去的现象。这种现象丝毫也不会由于以下情况而改变它的性质并得到解释:工人用自己的直接劳动换得货币的那个比例,在生产过程结束后表现为付给工人的价值和工人所创造的产品的价值之间的比例。资本和劳动之间最初的不平等交换,在这里只是以另一种形式表现出来罢了。

穆勒进一步阐述的一段话,也可以表明他如何固执地回避劳动和资本之间的直接交换,而李嘉图还是毫无拘束地以此为出发点的。穆勒是这样说的:

\begin{quote}{[798]“假定有一定数目的资本家和一定数目的工人。假定他们分配产品的比例也通过某种方法确定了。如果工人人数增长了而资本量没有增加,增加的那一部分工人就会试图排挤原来在业的那一部分。他们只有按较低报酬提供自己的劳动,才能作到这一点。在这种情况下工资水平必然降低……在相反的情况下结果也相反……如果资本量同人口的比例不变,工资水平也就保持不变。”(同上,第35页及以下各页)}\end{quote}

穆勒应当确定的,恰恰也就是“他们〈资本家和工人〉分配产品的比例”。为了让竞争决定这个比例,穆勒就假定,这个比例已经“通过某种方法确定了”。为了让竞争决定工人的“份额”,他就假定,这个份额在竞争之前就已经“通过某种方法”确定了。这还不够。为了表明竞争如何改变已经“通过某种方法”确定了的产品分配,他还假定工人在他们的人数比资本量增加得快的时候,就“按较低报酬提供自己的劳动”。可见,穆勒在这里直接说出了工人的供给是由“劳动”构成的,工人提供这种劳动以换取“报酬”,即换取货币,换取一定量的“积累劳动”。为了避开劳动和资本之间的直接交换,避开直接出卖劳动,他求助于“产品分配”论。为了解释产品分配的比例,他又假定直接出卖劳动以换取货币,以致于资本和劳动之间的这种最初的交换后来就表现在工人在他的产品中所占的份额上,而不是这种最初的交换决定于工人在产品中所占的份额。最后,当工人人数和资本量不变时,“工资水平”也保持不变。但是,当需求和供给彼此适应时,这种工资水平是怎样的呢?这也正是应当说明的。说工资水平在这种供求的平衡遭到破坏时就会变动,在这里并不说明任何问题。穆勒的同义反复的说法只能证明,他在这里感到李嘉图的理论中有一种障碍,要克服这种障碍,只有根本脱离这个理论。

\centerbox{※     ※     ※}

反对马尔萨斯、托伦斯及其他人。穆勒反对商品价值决定于资本价值,他正确地指出:

\begin{quote}{“资本就是商品,说商品的价值由资本的价值决定,就等于说,商品的价值由商品的价值决定,商品的价值由它本身决定。”(《政治经济学原理》1821年伦敦第1版第74页)}\end{quote}

\centerbox{※     ※     ※}

{穆勒并不掩盖资本同劳动的对立。他说,为了使不依靠直接劳动的社会阶级壮大,利润率必须高;为此,工资也就应该相对地低。为了使人类的(社会的)能力能在那些把工人阶级只当作基础的阶级中自由地发展,工人群众就必须是自己的需要的奴隶,而不是自己的时代的主人。工人阶级必须代表不发展,好让其他阶级能够代表人类的发展。这实际上就是资产阶级[799]社会以及过去的一切社会所赖以发展的对立,是被宣扬为必然规律的对立,也就是被宣扬为绝对合理的现状。

\begin{quote}{“人类进步,即不断地从科学和幸福的一个阶段过渡到另一个更高的阶段的能力,看来在很大程度上取决于这样的人所组成的阶级:他们是自己的时代的主人,也就是说,他们相当富有,根本不必为取得过比较安乐的生活的资财而操心。科学的领域就是由这个阶级的人来培植和扩大的;他们传播光明;他们的子女受良好的教育,被培养出来去从事最重要、最高雅的社会职务;他们成为立法者、法官、行政官员、教师、各种技艺的发明家、人类赖以扩大对自然力的控制的一切巨大和有益的工程的领导者。”(《政治经济学原理》,帕里佐的法译本,1823年巴黎版第65页)“资本的收入应当大到足够使社会上很大一部分人能够享受余暇所提供的好处。”(同上,第67页)}}\end{quote}

\centerbox{※     ※     ※}

对以上所谈的再作一些补充。

在穆勒这个李嘉图主义者看来,劳动和资本之间的区别只是劳动的不同形式之间的区别:

\begin{quote}{“劳动和资本:前者是直接劳动,后者是积累劳动。”(《政治经济学原理》1821年伦敦英文第1版第75页)}\end{quote}

在另一个地方他说:

\begin{quote}{“关于这两种劳动应当指出,它们并不是始终按照同样比率取得报酬的。”(《政治经济学原理》,帕里佐的法译本,1823年巴黎版第100页)}\end{quote}

可见,他在这里接近了问题的实质。既然用来支付直接劳动的报酬的始终是积累劳动即资本,那末不按同样比率支付报酬,在这里只能意味着较多的直接劳动同较少的积累劳动交换,而且“始终”如此,因为,不然的话,积累劳动就不能作为“资本”同直接劳动交换,它不仅不能象穆勒所期望的那样提供足够大的收入,而且根本不会提供收入。因此,这里就承认了,——因为穆勒正象李嘉图一样,把资本同劳动的交换看作积累劳动同直接劳动的直接交换,——这两种劳动是按照不相等的比例进行交换的,而在这样交换的情况下,以等量劳动相互交换为内容的价值规律便遭到了破坏。

\tsectionnonum{[(C)穆勒不理解工业利润的调节作用]}

穆勒把李嘉图实际上为阐明自己的地租理论而假定的东西\fnote{见本卷第2册第532—533页。——编者注},作为一条基本规律提出来。

\begin{quote}{“农业的利润率调节其他利润率。”(《政治经济学原理》1824年伦敦第2版第78页)}\end{quote}

这是根本错误的,因为资本主义生产是在工业中,而不是在农业中开始的,而且是逐渐支配农业的;这样,农业利润只是随着资本主义生产的发展才和工业利润平均化,而且只是由于这种平均化才开始影响工业利润。所以,首先,上述论点从历史上看是错误的。其次,只要存在这种平均化,就是说,只要存在这样的农业发展状况,即资本随着利润率的高低从工业转入农业或从农业转入工业,那末,说农业利润从这时开始起调节作用,而不说这里是两种利润相互发生作用,同样是错误的。其实,李嘉图本人为了说明地租,是以相反的情况为前提的。谷物价格上涨,结果利润下降,但不是在农业中下降(在较坏土地或第二笔生产率较低的资本提供新的供给之前),——因为谷物价格的上涨绰绰有余地给租地农场主补偿谷物价格上涨所引起的工资的提高,——而是在工业中下降,因为这里不发生这种补偿或超额补偿。这样,工业利润率下降了,只得到这种较低利润率的资本就可以用于较坏土地。如果利润率不变,情况就不会这样。而且,只是由于工业利润下降对较坏土地上的农业利润的这种反作用,整个农业利润才下降,[800]较好土地上的一部分农业利润才以地租形式从利润中分出来。这就是李嘉图对这个过程的说明。可见,按照他的说明,是工业利润调节农业利润。

如果现在由于农业的改良,农业利润又提高了,那末工业利润也会提高。但是这决不能排除下述情况:正象最初工业利润的减少决定农业利润的减少一样,工业利润的提高也会引起农业利润的提高。每当工业利润的提高同谷物价格以及加入工人工资的其他农产必需品的价格无关的时候,也就是说,每当工业利润的提高是由于构成不变资本的商品等等价值降低的时候,情况就是这样。如果工业利润不调节农业利润,地租反而绝对不能得到解释。平均利润率在工业中是由于资本利润的平均化以及价值因此转化为费用价格而形成的。这种费用价格——预付资本的价值加平均利润——是农业从工业获得的前提,因为农业中由于土地所有权的存在,上述平均化是不可能发生的。如果农产品的价值因而高于由工业的平均利润决定的费用价格,那末这个价值超过费用价格的余额就形成绝对地租。但是,为了能对价值超过费用价格的这种余额进行衡量,费用价格应当是第一性的,也就是说,它应当作为规律由工业强加给农业。

\centerbox{※     ※     ※}

穆勒的下述论点值得注意:

\begin{quote}{“生产中消费的东西总是资本。这就是生产消费的一个特别值得注意的属性。生产中所消费的东西就是资本,并且通过消费才成为资本。”(《政治经济学原理》,帕里佐的法译本,1823年巴黎版第241—242页)}\end{quote}

\tsectionnonum{[(d)]需求,供给,生产过剩[直接把需求和供给等同起来的形而上学观点]}

\begin{quote}{“需求意味着购买愿望和购买手段……一个人所提供的等价物品〈购买手段〉就是需求的工具。他的需求量就是用这个等价物品的价值来衡量的。需求和等价物品是两个可以相互代替的用语……他的〈一个人的〉购买愿望和购买手段,换句话说,他的需求,正好等于他生产出来但不准备自己消费的东西的数量。”(同上,第252——253页)}\end{quote}

我们在这里看到,需求和供给的直接等同(从而市场商品普遍充斥的不可能性)是怎样被证明的。需求据说就是产品,而且这种需求的量是用这种产品的价值来衡量的。穆勒就是用这同样的抽象“证明方法”证明买和卖只是等同,而不是彼此相区别;他就是用这同样的同义反复证明价格取决于流通的货币量;他也就是用这同样的手法证明供给和需求(它们只是买者和卖者的关系的进一步发展的形式)必然是彼此平衡的。这还是同样的一套逻辑。如果某种关系包含着对立,那它就不仅是对立,而且是对立的统一。因此,它就是没有对立的统一。这就是穆勒用来消除“矛盾”的逻辑。

我们首先拿供给作出发点。我供给的是商品,是使用价值和交换价值的统一体,譬如说,一定量的铁,它等于3镑(这笔钱又等于一定量的劳动时间)。根据假设,我是一个制铁厂主。我供给一定的使用价值——铁,也供给一定的价值,即表现为铁的价格3镑的价值。但是,这里有下面一种小小的差别。一定量的铁确实是由我投入市场的。相反,铁的价值却只是作为铁的价格存在,这个价格还要由铁的买者去实现,买者对我来说代表对铁的需求。铁的卖者的需求,则是对铁的交换价值的需求,这种交换价值固然包含在铁里,但是还没有实现。同样大的交换价值可以表现在数量极不相同的铁上。由此可见,使用价值的供给和有待实现的价值的供给决不是等同的,因为数量完全不同的使用价值可以表现同量的交换价值。

[801]同样是3镑的价值,可能表现在1吨、3吨或10吨上。可见,我供给的铁(使用价值)的量和我供给的价值的量决不是互成比例的,因为无论前者怎样变化,后者的量可能始终不变。无论我供给的铁的量是大还是小,根据我们的假设,我始终要实现的是不以铁本身的量为转移,一般说来不以铁作为使用价值的存在为转移的价值。由此可见,我所供给的(但是还没有实现的)价值和我所供给的、已经实现的铁的量,决不是互成比例的。因此,没有丝毫理由认为一种商品按照自己的价值出卖的能力和我所供给的商品量是成比例的。对买者来说,我的商品首先是作为使用价值而存在的。买者把它作为使用价值来购买。但是他所需要的是一定量的铁。他对铁的需要并不决定于我所生产的铁的量,正象我的铁的价值本身不和这个量成比例一样。

当然,购买的人手中持有商品的转化形式,货币,即具有交换价值形式的商品,而且他之所以能作为买者出现,只是由于他或其他人曾经作为现在以货币形式存在的那个商品的卖者出现。但是,这决不能作为理由来说明他把自己的货币再转化为我的商品,或者说明他对我的商品的需要决定于我生产的商品量。就他对我的商品提出需求来说,他需要的量或者可能比我供给的量少,或者可能完全一样,但是要低于商品的价值。正象我供给的某种使用价值的量和我供给它时所依照的价值不是等同的一样,他的需求也可能和我的供给不相适应。

但是,全部关于供求的研究都不是这里所要涉及的。

既然我供给铁,我需求的就不是铁,而是货币。我供给的是某种特殊的使用价值,需求的是它的价值。因此,我的供给和需求,正象使用价值和交换价值一样,是不同的。既然我在铁上供给了某一价值,我需求的就是实现这一价值。可见,我的供给和需求就象观念和现实一样,是不同的。其次,我供给的量和它的价值绝不是互成比例的。而且,对我供给的某种使用价值的量的需求,不是决定于我想要实现的价值,而是决定于买者按照一定价格需要买到的量。

我们再引证穆勒如下的论点:

\begin{quote}{“显然,每个人加在产品总供给量上的,是他生产出来但不准备自己消费的一切东西的总量。无论年产品的一定部分以什么形式落到这个人的手里,只要他决定自己一点也不消费,他就希望把这一部分产品完全脱手;因此,这一部分产品就全部用于增加供给。如果他自己消费这个产品量的一部分,他就希望把余额全部脱手,这一余额就全部加在供给上。”(同上,第253页)}\end{quote}

换句话说,这无非是指,全部投入市场的商品构成供给。

\begin{quote}{“可见,因为每个人的需求等于他希望脱手的那一部分年产品,或者换一种说法,等于他希望脱手的那一部分财富}\end{quote}

{且慢!他的需求等于他希望脱手的那一部分产品的价值(在这个价值一旦实现时);他希望脱手的东西是一定量的使用价值,他希望取得的东西是这个使用价值的价值。这两种东西决不是等同的},

\begin{quote}{并且因为每个人的供给也完全与此相同}\end{quote}

{决不是这样,他的需求不是他希望脱手的东西,即不是产品,而是这种产品的价值;相反,他的供给现实地是这种产品,而这种产品的价值则只是在观念上被供给},

\begin{quote}{所以,每个人的供给和需求必然是相等的。”(第253—254页)}\end{quote}

{这就是说,他所供给的商品的价值和他以这个商品去要求、但并不拥有的那个价值是相等的。如果他按照商品的价值出卖商品,那末他供给的价值(以商品形式)和取得的价值(以货币形式)就是相等的。但是,不能由于他希望按照商品的价值出卖商品,就得出结论说实际发生的情况就是这样。一定量的商品由他供给,并出现在市场上。他想要得到他所供给的商品的价值。}

\begin{quote}{“供给和需求处于[802]一种特殊的相互关系之中。每一个被供给的、被运往市场的、被出卖的商品,始终同时又是需求的对象,而成为需求对象的商品,始终同时又是产品总供给量的一部分。每一个商品都始终同时是需求的对象和供给的对象。当两个人进行交换时,其中一个人不是为了仅仅创造供给而来,另一个人也不是为了仅仅创造需求而来;他的供给对象,供给品,必定给他带来他需求的对象,因此,他的需求和他的供给是完全相等的。但是如果每一个人的供给和需求始终相等,那末,一个国家的全体人员的供给和需求,总起来说,也是这样。因此,无论年产品总额如何巨大,它永远不会超过年需求总额。有多少人分配年产品,年产品总量就分成多少部分。需求的总量,等于所有这些部分的产品中所有者不留归自己消费的东西的总额。但是,所有这些部分的总量,恰恰等于全部年产品。”(同上,第254—255页)}\end{quote}

既然穆勒已经假定每个人的供给和需求相等,那末,说全体人员的供给和需求因而也彼此相等的全部冗长的高明议论,便是完全多余的了。

\centerbox{※     ※     ※}

和穆勒同时代的李嘉图学派是如何理解穆勒的,例如,从下面的引文就可以看出:

\begin{quote}{“可见,在这里{这是指穆勒关于劳动的价值规定}我们看到至少有这样一种情况:价格(劳动的价格)经常决定于供求关系。”(普雷沃《评李嘉图体系》,载于麦克库洛赫《论政治经济学》,普雷沃译自英文,1825年日内瓦—巴黎版第187页)}\end{quote}

麦克库洛赫在上述《论政治经济学》一书中说,穆勒的目的在于——

\begin{quote}{“对政治经济学原理进行逻辑演绎。”(第88页)“穆勒几乎阐述了所有成为讨论对象的问题。他善于解开和简化最复杂最困难的问题,并且把各种不同的科学原理纳入自然秩序。”(同上)}\end{quote}

从穆勒的逻辑中可以得出这样的结论:他把我们在前面\fnote{见本卷第2册,特别是第180—187页。——编者注}分析的李嘉图著作的十分不合逻辑的结构幼稚地当作“自然秩序”原封不动地保存下来了。

\tsectionnonum{[(e)]普雷沃[放弃李嘉图和詹姆斯·穆勒的某些结论。试图证明利润的不断减少不是不可避免的]}

至于上面提到的普雷沃(他把穆勒对李嘉图体系的说明作为他的《评李嘉图体系》一文的依据),他的某些反对意见是纯粹幼稚无知地误解李嘉图的结果。

但是,下面一段谈到地租的话值得注意:

\begin{quote}{“如果象应该做的那样,注意到较坏土地的相对数量,就会对这种土地在决定价格上所发生的影响提出怀疑。”(普雷沃,同上第177页)}\end{quote}

普雷沃引用了穆勒以下一段话,这段话对于我所作的分析也很重要,因为穆勒在这里为自己设想了一种情况,在这种情况下,级差地租之所以产生,是因为新的需求——追加的需求——通过向较好土地而不是向较坏土地推移,即按上升序列运动而得到满足:

\begin{quote}{“穆勒先生用了这样一个例子:‘假设某一国家的全部耕地质量相同,并且对投入土地的各资本提供同样多的产品,只有一英亩例外,它提供的产品六倍于其他任何土地。’(穆勒《政治经济学原理》第2版第71页)毫无疑问,正象穆勒先生所证明的那样,租用这一英亩土地的租地农场主并不能提高他的租地收入〈即不能比其他租地农场主取得更高的利润;普雷沃把这个思想表达得很拙劣\endnote{在普雷沃翻译的麦克库洛赫那本书所附的正误表上,这句话被改为:“租佃这最后一英亩土地的农场主不能逃避交纳相应的地租”。——第111页。}〉,六分之五的产品会属于土地所有者。”}\end{quote}

{可见,这里我们看到的是在利润率不降低和农产品价格不提高的条件下的级差地租。这种情况一定会更加常见,因为[803]不论自然肥力如何,土地的位置随着一国工业的发展、交通工具的发展和人口的增长必然不断改善,而位置(比较好的位置)是和自然肥力同样发生作用的。}

\begin{quote}{“但是,如果这位机灵的作者想到对相反的情况作同样的假设,他就会相信结果是不同的。我们假设全部土地具有相同的质量,只有一英亩较坏土地除外。在这唯一的一英亩土地上,所花费的资本的利润只等于其他任何一英亩土地的利润的六分之一。能不能设想,千百万英亩土地的利润会因此而降低到普通利润的六分之一呢?这唯一的一英亩土地想必根本不会对价格发生丝毫作用,因为,任何进入市场的产品(特别是谷物)不会由于如此微不足道的数量的竞争而受到显著的影响。因此我们说,对李嘉图拥护者关于较坏土地的影响的主张,应该考虑到不同肥力的土地的相对数量而加以修正。”(普雷沃,同上第177—178页)}\end{quote}

\centerbox{※     ※     ※}

{萨伊为康斯坦西奥翻译的李嘉图著作所加的注释,只有一个关于对外贸易的正确意见。\endnote{马克思指的是萨伊给李嘉图《政治经济学原理》第七章《论对外贸易》所加的注释。萨伊在这个注释中举了一个例子:法国从安的列斯群岛进口的糖在法国的价格,比法国本国生产的糖便宜。——第112页。}通过欺骗行为,由于一个人得到了另一个人失掉的东西,也可能获得利润。在一个国家内,亏损和盈利是平衡的。在不同国家的相互关系中,情况就不是这样。即使从李嘉图理论的角度来看,——这一点是萨伊没有注意到的,——一个国家的三个工作日也可能同另一个国家的一个工作日交换。价值规律在这里有了重大的变化。或者说,不同国家的工作日相互间的比例,可能象一个国家内熟练的、复杂的劳动同不熟练的、简单的劳动的比例一样。在这种情况下,比较富有的国家剥削比较贫穷的国家,甚至当后者象约·斯·穆勒在《略论政治经济学的某些有待解决的问题》一书中所指出的那样从交换中得到好处\endnote{约·斯·穆勒在他的《略论政治经济学的某些有待解决的问题》(1844年伦敦版)第一篇中考察了“各国相互交换的规律以及商业世界各国商业利益的分配”,并且指出:“我们通过对外国人的贸易取得他们的商品,而花费的劳动和资本,往往少于他们自己为这些商品所花费的。然而,这种贸易对外国人还是有利的,因为他们从我们这里换得的商品,如果他们自己去生产,就要花费较高的代价,尽管我们为它花费的代价较少。”(第2—3页)——第112页。}的时候,情况也是这样。}

\centerbox{※     ※     ※}

[关于农业利润和工业利润的相互关系问题,普雷沃说道:]

\begin{quote}{“我们承认,总的说来,农业利润率决定工业利润率。但是,我们同时应该注意到,后者必然也对前者发生反作用。如果谷物的价格达到一定的高度,工业资本就会流入农业,不可避免地使农业利润降低。”(普雷沃,同上第179页)}\end{quote}

反驳是正确的,但是提得十分狭隘。参看前面所说的\fnote{参看本册第104—105页。——编者注}。

李嘉图学派认为,只有工资增长,利润才会下降,因为随着人口的增长,生活必需品的价格提高,而这种情况又是资本积累的结果,因为随着资本的积累,较坏土地逐渐投入耕种。但是李嘉图本人承认,当资本增加得比人口快的时候,也就是当资本相互竞争使工资提高的时候,利润也会下降。这是亚·斯密的观点。普雷沃说:

\begin{quote}{“如果资本的需求的增长使工人的价格即工资提高,那末,认为这些资本的供给的增长会使资本的价格即利润降低,难道是不对的吗?”(同上,第188页)}\end{quote}

按照李嘉图的观点,利润降低只能由于剩余价值减少,也就是由于剩余劳动减少,也就是由于工人消费的生活必需品的价格上涨,也就是由于劳动价值提高,尽管工人得到的实际报酬这时不但不会提高,反而可能降低;普雷沃就以这种错误的观点为依据,试图证明利润的不断降低并不是不可避免的。

第一,他说:

\begin{quote}{“繁荣状态首先使利润提高}\end{quote}

(这里指的正是农业利润:随着繁荣状态的到来,人口增加,从而对农产品的需求也增加,从而租地农场主的超额利润也增加),

\begin{quote}{而且这是在新地投入耕种以前很久的时候,所以,当这种新地开始影响地租,使利润降低的时候,利润尽管马上降低下来,但依然和繁荣以前一样高……为什么在某个时候会转而耕种质量较坏的土地呢?这样做只是指望至少能获得等于普通利润的利润。那末,什么情况能使质量较坏的土地创造这种利润率呢?人口的[804]增长……增加的人口形成对现有的生存资料的压力,因而使食品(特别是谷物)价格上涨,结果是农业资本获得高额利润。其他资本流入农业:但是因为土地面积有限,所以这种竞争也有限度,最终结果是耕种较坏土地仍然获得比商业或工业更高的利润。从这时起(在这种较坏土地有足够数量的前提下)农业利润便不能不决定于投入土地的最后一批资本的利润。如果以财富开始增长时〈利润开始分为利润和地租时〉存在的利润率作为出发点,就会发现利润决没有降低的趋势。利润会和人口一起增长,直至农业利润增长到这样的程度,以致利润(由于耕种新的、较坏的土地)会显著下降,但是决不会降到低于它原来的比率,或者(更确切地说)降到低于各种不同的情况所决定的平均比率。”(同上,第190—192页)}\end{quote}

普雷沃显然错误地理解了李嘉图的观点。在普雷沃看来,由于繁荣,人口增加,这又使农产品的价格提高,从而使农业利润提高(尽管令人不解的是:如果农业利润的这种提高是经常性的,地租为什么在租佃期满后不会提高,这种农业上的超额利润为什么不会甚至在较坏土地投入耕种之前就以地租形式被人占有)。但是,促使农业利润提高的农产品价格的上涨,会使一切工业部门的工资提高,因而引起工业利润的下降。这样,工业中会形成一个新的利润率。即使质量较坏的土地在现行农产品市场价格下只提供这个降低了的利润率,资本也会投入质量较坏的土地。把资本吸引到这里来的,是高的农业利润和高的谷物市场价格。只要还没有足够多的资本转入农业,这些被吸引来的资本甚至还能象普雷沃所说的那样提供比已经降低的工业利润更高的利润。但是,一旦追加供给够了,市场价格就下降,因而较坏土地只能提供普通的工业利润。较好土地的产品所提供的超过这种利润的东西,就转化为地租。这就是李嘉图的观点,普雷沃接受了这个观点的基础,并且以这个观点作为自己立论的出发点。现在,谷物比农业利润提高之前贵。但是,它给租地农场主提供的超额利润则转化为地租。这样,较好土地的利润也降到因农产品涨价而下降的、较低的工业利润率水平。令人不解的是,如果没有某些其他情况出现并引起变化,利润为什么就不会因此降到低于它“原来的比率”。当然,其他情况是可能出现的。根据假定,不管怎样,农业利润在生活必需品涨价之后都要提高到工业利润以上。但是,如果这时工人消费的生活必需品中由工业生产的那一部分由于生产力的发展而降价,以致工人的工资(即使它按照它的平均价值支付)因农产品涨价而提高的程度,没有达到不存在这种起抵销作用的情况时所应达到的高度;其次,如果同样由于生产力的发展,采掘工业所提供的产品的价格以及不加入食物的农产原料的价格也降低了(固然,这种假定未必是现实的),那末,工业利润就可能不下降,尽管它还会低于农业利润。这时,因资本转移到农业中以及因形成地租而引起的农业利润的下降,[805]只会使利润率恢复原来的水平。

[第二,]普雷沃还试图用另一个论据:

\begin{quote}{“质量较坏的土地……只有当它提供的利润同工业资本的利润一样高或者更高时,才会被投入耕种。在这种情况下,尽管新地投入耕种,谷物和其他农产品的价格往往仍旧很高。这种高价格使工人人口陷于穷困,因为工资的提高赶不上雇佣工人消费的物品的价格的提高。农产品的高价格会成为全体居民或多或少的负担,因为工资的提高和生活必需品价格的提高几乎影响一切商品。这种普遍的穷困加上人口过多造成的死亡率的提高,引起雇佣工人人数的减少,并因而造成工资的提高和农业利润的下降。从这时起,进一步的发展方向便同以前相反。资本从较坏土地抽出,又流回工业。但是,人口规律很快又发生作用:一旦贫困消失,工人人数就增加,他们的工资就下降,因而利润就提高。这种向两个不同方向的波动会交替发生,但是并不触动平均利润。利润可能由于其他原因或者就由于这一原因而提高或下降,可能轮流地时而朝这一方向时而朝另一方向变动,但是不能认为利润的平均提高或下降是由于新地必须投入耕种造成的。人口是一个调节器,它可以恢复自然秩序并把利润控制在一定的范围内。”(同上,第194—196页)}\end{quote}

尽管叙述的方式非常混乱,但是从“人口规律”来看,这是正确的。不过这同下面的假定不一致:农业利润不断增长,直到与人口的增长相适应的追加供给创造出来为止。既然这里发生的是农产品价格的不断提高,那末由此得出的就不是人口减少,而是利润率普遍下降,这又引起积累的减少,只有这样,才造成人口的减少。根据李嘉图—马尔萨斯的观点,人口的增长是比较慢的。而普雷沃立论的基础是这样的思路:上述过程会使工资降到它的平均水平以下;随着工资的这种下降和工人的贫困会出现谷物价格的下跌,因而利润又会提高。

但是,这条思路会把我们引向与这里的问题无关的研究,因为我们在这里的前提是:劳动的价值总是被全部支付的,就是说,工人得到他本身的再生产所必需的生活资料。

普雷沃的这些论断之所以重要,是由于它们表明了,李嘉图的观点以及李嘉图所接受的马尔萨斯的观点虽然能够解释利润率的波动,但是不能解释利润率为什么不断下降而无回升:要知道,在谷物价格达到一定高度和利润降到一定程度时,工资就会降到它的水平以下,这又会造成强制性的人口减少,因此造成谷物价格和全部生活必需品价格下降,而这又必然会引起利润的提高。

\tchapternonum{(3)论战著作}

[806]从1820年到1830年这个时期是英国政治经济学史上形而上学方面最重要的时期。当时进行了一场拥护和反对李嘉图理论的理论斗争,出版了一系列匿名的论战著作;这里引用了其中一些最重要的著作,特别是只涉及那些和我们论题有关的论点的著作。不过,同时这些论战著作的特点也是,它们事实上都只是围绕价值概念的确定和价值对资本的关系进行论战的。

\tsectionnonum{(a)《评政治经济学上若干用语的争论》}

[政治经济学上的怀疑论;把理论的争论归结为用语的争论]

《评政治经济学上若干用语的争论,特别是有关价值、供求的争论》1821年伦敦版。

这部著作不无一定的尖锐性。书名很说明特点——《用语的争论》。

它部分是反对斯密、马尔萨斯的,但是也反对李嘉图。

这部著作的基本思想是:

\begin{quote}{“……争论的产生,完全是因为不同的人用语含义不同,是因为这些争论者象故事中的骑士那样从不同方面去看盾脾。”(第59—60页)}\end{quote}

这种怀疑论总是某种理论解体的预兆,也是某种无思想、无原则的适合家庭需要的折衷主义的先驱。

关于李嘉图的价值理论,匿名作者首先谈到:

\begin{quote}{“假设当我们谈到价值或者与名义价格相对立的实际价格的时候,我们指的是劳动,那就会出现一个明显的困难;因为我们常常要谈到劳动本身的价值或价格。如果我们把作为某物的实际价格的劳动理解为生产该物的劳动,那末就产生另一个困难;因为我们常常要谈到土地的价值或价格;但是土地不是由劳动生产出来的。因此,这个规定只适用于商品。”(同上,第8页)}\end{quote}

谈到劳动,这里对李嘉图的反驳是正确的,因为李嘉图认为资本直接购买劳动,也就是说,他直接谈论劳动的价值,而实际上,被买卖的是劳动能力——它本身是一种产品——的暂时使用权。匿名作者在这里并没有解决问题,而只是强调问题没有解决罢了。

说不是劳动产品的“土地的价值或价格”,表面看来直接同价值概念相矛盾,不能直接从其中得出来,这也是完全正确的。但是这句话用来反驳李嘉图就格外没有意义了,因为匿名作者并不反对李嘉图的地租理论,而李嘉图恰恰在那里阐明了怎样在资本主义生产的基础上形成土地的名义价值,以及土地的名义价值和价值规定并不矛盾。土地的价值不过是支付资本化的地租的价格。因此,这里假定的关系比从商品及其价值的简单考察中乍一看就得出的关系要深刻复杂得多;这正象虚拟资本\endnote{马克思在这里所说的虚拟资本是指国债资本,也就是说,国家(资产阶级的或地主资产阶级的)把贷款不是作为资本支出,而用从居民那里征收的税款来支付利息。参看马克思《资本论》第3卷第29章。——第118页。}(这种资本是交易所投机的对象,而且事实上不过是对部分年税的某种权利的买卖)不能用生产资本的简单概念去说明一样。

第二个反驳——说李嘉图把价值由某种相对的东西变为某种绝对的东西——在后来出版的另一部论战著作(赛米尔·贝利著)中,成了攻击李嘉图整个体系的出发点。我们在论述贝利的著作时还将提到《评政治经济学上若干用语的争论》中与此有关的观点。

关于支付劳动的资本产生的源泉,作者在一个顺便作出的评论中作了中肯的表述,但是他是不自觉的(相反,他是想借此证明下面那些我没有加上着重号的话,即劳动的供给本身阻碍劳动下降到它的自然价格的水平的趋势)。

\begin{quote}{“已经增长的劳动供给就属已经增长的用来购买劳动的东西的供给。因此,如果我们和李嘉图先生一起,说劳动总是具有下降到他称为劳动的自然价格的水平的趋势,那末我们就应该想到:引起这种趋势的劳动供给增长本身,就是阻碍这种趋势发生作用的对抗原因之一。”(同上,第72—73页)}\end{quote}

如果不从劳动的平均价格即劳动的价值出发,理论就不可能进一步展开;这就象不从一般商品价值出发,理论也不可能展开一样。只有那样,才能理解价格波动的实际现象。

\begin{quote}{[807]“这并不是说,他〈李嘉图〉主张,如果两类不同商品中分别取出的两件商品,例如,一顶帽子和一双鞋,是由等量劳动生产的,那末,这两件商品就能互相交换。这里所说的‘商品’,应该理解为‘一类商品’,而不是单独一顶帽子,一双鞋等等。英国生产所有帽子的全部劳动,为此必须看作是分配在所有帽子上面的。在我看来,这一点从一开始以及在李嘉图学说的一般阐述中都没有表示出来。”例如,李嘉图谈到,“机器制造工人的一部分劳动”,包含在例如一双袜子上。“可是制造每一双袜子的全部劳动,——如果我们说的是个别的一双袜子,——包含机器制造工人的全部劳动,而不只是他的一部分劳动;因为,虽然一台机器织出许多双袜子,但是缺少机器的任何一部分,连一双袜子也制造不出来。”(同上,第53—54页)}\end{quote}

后一段话是以误解为基础的。全部机器进入劳动过程,但只有一部分机器进入价值形成过程。

除此之外,这个评论中也有些正确的东西。

我们现在从作为资本主义生产的基础和前提的商品——产品的这个特殊的社会形式——出发。我们考察个别的产品,分析它们作为商品所具有的,也就是给它们打上商品烙印的形式规定性。在资本主义生产以前——在以前的生产方式下——很大一部分产品不进入流通,不投入市场,不作为商品生产出来,不成为商品。另一方面,在这个时期,加入生产的很大一部分产品不是商品,不作为商品进入过程。产品转化为商品,只发生在个别场合,只涉及产品的剩余部分等等,或只涉及个别生产领域(加工工业产品)等等。产品既不是全部作为交易品进入过程,也不是全部作为交易品从过程出来。但是产品发展为商品,一定范围的商品流通,因而一定范围的货币流通,也就是说,相当发达的贸易,是资本形成和资本主义生产的前提和起点。我们就是把商品看成这样的前提,因为我们是从商品出发,并把它作为资本主义生产的最简单的元素的。但是,另一方面,商品是资本主义生产的产物、结果。表现为资本主义生产元素的东西,后来表现为资本主义生产本身的产物。只有在资本主义生产的基础上,商品才成为产品的普遍形式,而且资本主义生产愈发展,具有商品形式的产品就愈作为组成部分进入资本主义生产过程。从资本主义生产中出来的商品,与我们据以出发的、作为资本主义生产元素的商品不同。在我们面前的已经不是个别的商品,个别的产品。个别的商品,个别的产品,不仅实在地作为产品,而且作为商品,表现为总产品的一个不仅是实在的、而且是观念的部分。每个个别的商品都表现为一定部分的资本和资本所创造的剩余价值的承担者。

在总产品例如1200码棉布的价值中,包含预付资本的价值加资本家占有的剩余劳动——譬如说120镑的价值(假设预付资本是100镑,剩余劳动等于20镑)。每码棉布等于120/1200镑,即1/10镑或2先令。作为过程的结果表现出来的,不是个别商品,而是商品总量,其中总资本的价值被再生产出来并加上了剩余价值。所生产的总价值除以产品数,决定个别产品的价值,而且个别产品只有作为总价值的这种相应部分才成为商品。现在决定个别产品的价值并使个别产品成为商品的,不再是花费在个别的特殊的商品上的劳动(这种劳动在大多数情况下根本无法计算出来;它在某一个商品中可以比在另外一个商品中多),而是总劳动,总劳动的相应部分,即总价值除以产品数得出的平均数。因此,为了补偿总资本连同剩余价值,商品总量中的每一商品也都必须按其由上述方式决定的价值出卖。如果1200码只卖出800码,资本就得不到补偿,更得不到利润。但是,每一码也都低于它的价值出卖,因为它的价值不是孤立地,而是作为总产品的一定部分决定的。

\begin{quote}{[808]“如果你们把劳动叫做商品,那末它也还是不同于一般商品。后者最初为交换的目的而生产,然后拿到市场上去,和同时在市场上出售的其他商品按照适当的比例相交换。劳动只有当它被带到市场上去的那一瞬间才被创造出来,或者不如说,劳动是在它被创造出来以前被带到市场上去的。”(同上,第75—76页)}\end{quote}

实际上,被带到市场上去的不是劳动,而是工人。工人卖给资本家的不是他的劳动,而是对他自身作为劳动力[workingpower]的暂时使用权。在资本家和工人订立的合同中,在他们商定的买卖中,这才是直接的对象。

如果实行计件工资制,工人按件得到报酬,而不是按劳动能力受资本家支配的时间得到报酬,那末,这只是决定这种时间的另一种方式。时间用产品来计量,在这里一定量的产品被看作社会必要劳动时间的表现。在伦敦许多盛行计件工资制的工业部门中,工资就是这样按[社会必要劳动时间的]小时支付的,但是这件或那件劳动产品是否代表“一个小时”,常常引起争执。

不管个别的工资形式如何,劳动能力虽然在被使用以前按一定条件出卖,却要在完成劳动以后,才得到报酬(无论是按日、按周等等),不仅在计件工资制下如此,而且普遍如此。在这里,货币先在观念上作为购买手段,然后成为支付手段,因为商品在名义上转移到买者那里和实际的转移是不同的。商品(劳动能力)的出卖,使用价值在法律上的转让和它在实际上的转让,在这里从时间上说是不一致的。因此,价格的实现迟于商品的出卖(见我的著作的第一部分,第122页)\endnote{马克思在这里引用了《政治经济学批判》第一分册。见《马克思恩格斯全集》中文版第13卷第132—134页。——第121页。}。这里也表明,不是资本家,而是工人在预付;正如出租房子,不是租赁人,而是出租人预付使用价值。诚然,工人在他生产的商品卖出以前得到工资(或者,至少可能得到工资,如果商品不是预先订购,等等)。但是,在他(工人)得到工资以前,他的商品,他的劳动能力,已经消费在生产上,已经转到买者(资本家)的手里。问题不在于,商品的买者打算怎样处理商品,不在于他购买商是为了把它保存下来当作使用价值,还是为了把它再出卖。问题在于第一个买者和卖者之间的直接交易。

[李嘉图在他的《原理》中说:]

\begin{quote}{“在不同的社会阶段,资本,或者说,使用劳动的手段的积累,速度有快有慢,它在所有情况下都必定取决于劳动生产力。一般说来,在存在着大量肥沃土地的地方,劳动生产力最大。”(李嘉图《政治经济学和赋税原理》1821年伦敦第3版第92页)}\end{quote}

关于李嘉图的这个论点,匿名著作的作者说:

\begin{quote}{“假如第一句话中的劳动生产力,是指每一产品中属于亲手生产该产品的人的那一部分很小,那末这句话就是同义反复,因为其余部分形成一个基金,只要它的所有者高兴,便可以用来积累资本。”}\end{quote}

(因此,就不言而喻地承认:从资本家的观点来看,“劳动生产力是指每一产品中属于亲手生产该产品的人的那一部分很小”。这句话非常好。)

\begin{quote}{“但是,在土地最肥沃的地方,大多不会有这种情况。”}\end{quote}

(这个反驳是愚蠢的。李嘉图是以资本主义生产为前提的。他不是研究,资本主义生产是在土地肥沃的地方容易发展,还是在土地相对来说不肥沃的地方容易发展。在资本主义生产存在的地方,资本主义生产在土地最肥沃的地方生产率最高。劳动的自然生产力,即劳动在无机界发现的生产力,和劳动的社会生产力一样,表现为资本的生产力。李嘉图本人在上面一段话中,把劳动生产力和生产资本的劳动——它生产的是支配劳动的财富、而不是归劳动所有的财富——等同起来,这是正确的。他的用语“资本,或者说,使用劳动的手段”,实际上是他把握资本的真正本质的唯一用语。他本人局限于[809]资本主义观点,以致对他来说这种颠倒,这种概念的混淆是不言而喻的。劳动的客观条件(而且是劳动本身创造的),原料和劳动工具,不是劳动作为自己的手段来使用的手段,相反,它们是使用劳动的手段。不是劳动使用它们,而是它们使用劳动。劳动是这些物作为资本进行积累的手段,而不是给工人提供产品、财富的手段。)

\begin{quote}{“在北美是这种情况,但这是一种人为的情况”}\end{quote}

(即资本主义的情况)。

\begin{quote}{“在墨西哥和新荷兰\fnote{澳大利亚的旧称。——编者注}不是这种情况。从另一种意义来说,在有许多肥沃土地的地方,劳动生产力确实最大,——这里是指人(只要他愿意)生产出与他所完成的总劳动相比是大量的原产品的能力。人能生产出超过维持现有人口生活所必需的最低限度的食物,这实际上是自然的赐予。”}\end{quote}

(这是重农学派学说的基础。这种“自然的赐予”是剩余价值的自然基础,它在农业劳动(最初几乎所有的需要都由农业劳动满足)中表现得最明显。在工业劳动中不是这样明显,因为工业劳动的产品首先必须作为商品出卖。最先分析剩余价值的重农学派,就是在剩余价值的实物形式上理解剩余价值的。)

\begin{quote}{“但是‘剩余产品’(李嘉图先生的用语,第93页)一般是指某物的全部价格超过生产该物的工人所得部分的余额}\end{quote}

(这个蠢驴没有看到,在土地肥沃,因而在产品价格中工人所得的份额虽然不大却能购买足够的生活必需品的地方,资本家所得的份额是最大的),

\begin{quote}{是指一种由人的协议确定而不是由自然规定的关系。”(《评政治经济学上若干用语的争论》第74—75页)}\end{quote}

如果最后结尾的这句话有什么意义,那就是,资本主义意义上的“剩余产品”应该同劳动生产率本身严格区别开来。劳动生产率只有当它对资本家来说作为利润实现时,才引起资本家的关心。这正是资本主义生产的局限性,资本主义生产的界限。

\begin{quote}{“如果对某一物品的需求超过了从供给的现状来看的有效需求,因而价格上涨,那末,或者,能够在生产费用的比率保持不变的情况下,增大供给的规模,——在这种情况下,供给的规模将一直增大到这种物品同其他物品按原先那样的比例进行交换为止。或者,第二,不能增大原来的供给规模,这样,上涨的价格将不下降,而是如斯密所说,将继续为生产这种物品所使用的特殊的土地、资本或劳动提供更多的地租,或利润,或工资(或所有三者)。或者,第三,供给的可能增加,相应地要求比原先供给的商品量的周期生产〈注意这个用语!〉有更多的土地,或资本,或劳动,或所有三者。这样,在需求增加到足以(1)按提高的价格支付追加供给;(2)按提高的价格支付原先的供给量以前,供给就不会增加。因为生产追加商品量的人,不会比生产原先商品量的人有更多的可能获得商品的高价……这样,这个行业就会得到超额利润……超额利润或者只落到一些特殊的生产者手里……或者在追加的产品和其余的产品不能区别时,由大家分享……人们为了加入能得到这种超额利润的行业,将付出一些东西……他们为此而付出的就是地租。”(同上,第79—81页)}\end{quote}

这里要注意的只是,在这一著作中地租第一次被看作固定的超额利润的一般形式。

\begin{quote}{[810]“‘收入转化为资本’这一用语,是这些用语争论的另一个根源。一个人以为这是指资本家把他的资本所赚得的一部分利润用于增加他的资本,而不是象他在另一种情况下可能做的那样,用于个人消费。另一个人则以为这是指某人作为资本支出的,决不是他作为他自己的资本的利润得到的,而是作为地租、工资、薪金得到的。”(同上,第83—84页)}\end{quote}

最后这些说法——“这些用语争论的另一个根源”,“一个人以为这是指”,“另一个人则以为这是指”——表明了这个自作聪明的拙劣作者的手法。

\tsectionnonum{(b)《论马尔萨斯先生近来提倡的关于需求的性质和消费的必要性的原理》[匿名作者的资产阶级的局限性。他对李嘉图的积累理论的解释。不理解引起危机的资本主义生产的矛盾]}

《论马尔萨斯先生近来提倡的关于需求的性质和消费的必要性的原理》1821年伦敦版。

李嘉图学派的著作。对马尔萨斯驳斥得好。表现出这些人的无限局限性。这里暴露出,当他们考察的不是土地所有权而是资本的时候,他们的敏锐性就变为无限的局限性了。虽然如此,这部著作还是上面提到的十年内最好的论战著作之一。

\begin{quote}{“如果用于刀的生产的资本增加1%,并且只能按同样的比例增加刀的生产,那末,假定其他物品的生产不增加,刀的生产者支配一般物品的可能性,将按较小的比例增加;正是这种可能性,而不是刀的数量的增加,构成企业主的利润,或增加他的财富。但是,如果其他所有行业的资本同时也增加1%,并且产品同样增加,那末结果就不同了,因为一种产品和另一种产品交换的比例不变,从而每种产品的一定部分所能支配的其他产品和以前一样多。”(上述著作,第9页)}\end{quote}

首先,如果象假定的那样,除了刀的生产,其他的生产(以及用于生产的资本)都不增加,那末,刀的生产者的收入将不是“按较小的比例”增加,而是根本不增加,甚至绝对亏损。这时,刀的生产者只有三条路可走。或者,他必须拿已增加的产品去交换,就象他拿较少量的产品去交换一样;这样,他的增产将造成真正的亏损。或者,他必须努力找到新的消费者;如果他限于原先的消费者范围,那末要做到这一点,就只有从其他行业把买主吸引过来,把自己的亏损转嫁到别人身上;或者,他必须超越原先的界限扩大他的市场,——但是,这两种办法都既不取决于他的美好愿望,也不仅仅取决于已增加的刀的数量的存在。或者,最后,他必须把他的产品的剩余部分转到下一年去,并相应地减少下一年的新的供给,这样,如果他的资本追加额不仅包括追加的工资,而且包括追加的固定资本,也会造成亏损。

其次,如果其他所有资本都按相同的比例积累,决不能由此得出结论说,它们的生产也按相同的比例增加。即使是这样,也不能由此得出结论说,它们需要多用1%的刀,因为它们对刀的需求,既同它们自己产品的增加没有什么联系,也同它们对刀的购买力的增长没有什么联系。这里只会得出同义反复:如果每一个别行业的资本的增加,与社会需要所造成的对每一个别商品的需求的增加成比例,那末,一种商品的增加就会为其他商品的增加供给提供市场。

因此,这里假定:(1)是资本主义生产,其中每一个别行业的生产以及这种生产的增加,都不是直接由社会需要调节,由社会需要[811]控制,而是由各个资本家离开社会需要而支配的生产力调节的;(2)尽管如此,生产却是这样按比例地进行,好象资本直接由社会根据其需要使用于各个不同的行业。

按照这个自相矛盾的假定,即假定资本主义生产完全是社会主义的生产,那末,实际上就不会发生生产过剩。

此外,在资本积累相等的不同行业内(说资本在不同行业按相等的比例积累,又是一个不妥当的假定),与所用资本的这

种增加相应的产品量,是极不相同的,因为不同行业的生产力,或者说,所生产的使用价值量与所使用的劳动之比,是大不相同的。这里和那里生产出相同的价值,但是同一价值表现出来的商品量却大不相同。因此,当A行业的价值增加1%,商品量增加20%,而B行业的价值同样增加1%,但商品量只增加5%时,就完全无法理解,为什么A的商品量必定在B行业找到市场。在这里忽视了使用价值和交换价值的区别。

萨伊的伟大发现——“商品只能用商品购买”,\endnote{在萨伊的著作(《论政治经济学》1814年巴黎第2版第2卷第382页)中说过:“产品只是用产品购买的”。对萨伊这个论点的批判见本卷第2册第563—564页和第569—574页。——第127页。}只不过是说,货币本身是商品的转化形式。这决不能证明,因为我只能用商品购买,所以我就能用我的商品购买,或者说,我的购买力和我所生产的商品量成比例。同一价值可以表现为极不相同的商品量。但是使用价值——消费——和产品价值无关,而和产品量有关。完全不能理解,为什么因为现在六把刀的价钱和以前一把刀一样,我就要买六把刀。且不说工人出卖的不是商品,而是劳动,而且有许多人不生产商品,但是用货币购买。商品的买者和卖者不是同一的。土地所有者和货币资本家等在货币形式上获得其他生产者的商品。他们是“商品”的买者,却不是“商品”的卖者。不仅产业资本家之间有买卖,而且他们还把自己的商品卖给工人和不是商品生产者的收入所有者。最后,他们作为资本家进行的买卖和他们花费自己的收入的购买,是大不相同的。

\begin{quote}{“李嘉图先生(第二版第359页)在引证了斯密关于利润下降的原因的观点之后,补充说:‘但是,萨伊先生曾经非常令人满意地说明:由于需求只受生产限制,所以任何数额的资本在一个国家都不会不加以使用。’”}\end{quote}

(多么聪明!当然,需求受生产限制。对那种不可能按定货生产的东西,或需求不能现成地在市场上找到的东西,是不可能产生需求的。但是,绝不能因为需求受生产限制就得出结论说,生产受需求限制或曾经受它限制,生产永远不能超过需求,特别是不能超过与当前市场价格适应的需求。这是萨伊式的敏锐思想。)

\begin{quote}{“‘在一个国家中,除非工资由于必需品的涨价而大大提高,因而剩下的资本利润极少,以致积累的动机消失,否则积累的资本不论多少,都不可能不生产地加以使用’〈匿名作者自己在括号内写道:〉(我认为,这是指“为所有者带来利润”)(同上,第360页)。”}\end{quote}

(在这里李嘉图把“生产地”和“有利润地”等同起来,而在资本主义生产中,只有“有利润地”才是“生产地”,这正是资本主义生产同绝对生产的区别,以及资本主义生产的界限。为了“生产地”进行生产,必须这样生产,即把大批生产者排除在对产品的一部分需求之外;必须在同这样一个阶级对抗中进行生产,[812]这个阶级的消费决不能同它的生产相比,——因为资本的利润正是由这个阶级的生产超过它的消费的余额构成的。另一方面,必须为那些只消费不生产的阶级生产。必须不仅仅使剩余产品具有成为这些阶级的需求对象的形式。另一方面,资本家本人,如果想要积累,也不应当对自己的产品按其生产的数量提出需求——就这些产品加入收入来说。否则他就不能积累。因此,马尔萨斯把那些任务不是积累而是消费的阶级同资本家对立起来。一方面假定所有这些矛盾是存在的,另一方面又假定,生产的进行完全没有冲突,好象这些矛盾都不存在。买和卖是分离的,商品和货币、使用价值和交换价值是分离的。可是又假定,这种分离是不存在的,存在的是物物交换。消费和生产是分离的;生产者不消费,消费者不生产。可是又假定,消费和生产是同一的。资本家进行生产是直接为了增加他的利润,为了交换价值,而不是为了享受。可是又假定,他进行生产是直接为了享受,而且仅仅为了享受。如果假定资本主义生产中存在的矛盾——这些矛盾诚然不断在平衡,但是这一平衡过程同时表现为危机,表现为互相分离、彼此对立、但又互相联系的各因素的通过暴力的结合——不存在,那末这些矛盾自然就不可能发生作用。在每个行业,每个资本家都按照他的资本进行生产,而不管社会需要,特别是不管同一行业其他资本的竞争性供给。可是又假定,他好象是按社会的定货进行生产的。如果没有对外贸易,据说,奢侈品就会不管生产费用多少而在国内生产。在这种情况下,除了生活必需品的生产以外,劳动就确实是非常不生产的了。因此,资本的积累也不多了。这样,每个国家就可以使用全部在国内积累的资本,因为按照假定,在国内只积累少量资本。)

\begin{quote}{“如果李嘉图前一句中的‘不会不加以使用’是指‘不可能不生产地加以使用’,或者更确切地说,‘不可能不有利润地加以使用’,那末后一句就把前一句限定了(不说同它矛盾)。如果它单单指‘加以使用’,这一论断就没有意义了,因为,我想,无论亚当·斯密或其他任何人都没有否认:如果不计较利润的多少,资本是能够‘加以使用’的。”(同上,第18—19页)}\end{quote}

实际上,李嘉图是说,在一个国家中,一切资本,无论是以什么样规模积累起来的,都能有利润地加以使用;另一方面,资本的积累又阻碍“有利润地”使用资本,因为资本的积累必定引起利润的减少,亦即积累率的缩减。

\begin{quote}{“他们〈工人〉[对工作的]需求的增加\fnote{见本册第60页。——编者注}不过是表明他们甘愿自己拿走产品中更小的份额,而把其中更大的份额留给他们的雇主;要是有人说,这会由于消费减少而加剧市场商品充斥,那我只能回答说:市场商品充斥是高额利润的同义语。”(同上,第59页)}\end{quote}

这的确是市场商品充斥的隐秘基础。

\begin{quote}{“只要由于使用机器而价格变得便宜的物品,不是工人因为便宜就能使用的东西,那末,作为消费者的工人在繁荣时期并不能(如萨伊先生在《论政治经济学》第四版第一卷第60页上所说的那样)从机器得到任何好处。从这方面来看,脱粒机和风磨,对于工人来说,可能是很重要的;但是截夹板机,滑轮制造机或花边织机的发明很少使他们的状况得到改善。”(同上,第74—75页)“在分工发达的地方,工人的技艺只能在他学得这种技艺的特殊领域应用;工人本身就是一种机器。在这种情况下,有一个很长的失业时期,就是说,一个失去劳动,即从根本上失去财富的时期。因此,象鹦鹉那样喋喋不休地说,事物都有找到自己的水准的趋势,是丝毫无济于事的。我们必须看看周围,我们会发现,事物[813]长时期都不能找到自己的水准;即使找到了,也比过程开始时的水准低。”(同上,第72页)}\end{quote}

这位李嘉图主义者,效法李嘉图,正确地承认了由商业途径的突然变化引起的危机\endnote{李嘉图的《原理》第十九章标题是《论商业途径的突然变化》,在这里,“商业”不仅指某个国家的商业,而且指某个国家的生产活动。参看本卷第2册第567—568页。——第130页。}。1815年战争以后英国的情况就是这样。因此,所有以后的经济学家每次都认为,每次危机的最明显的导火线就是引起每次危机的唯一可能的原因。

他也认为信用制度是危机的原因。(第81页及以下各页)(好象信用制度本身不是由“生产地”即“有利润地”使用资本的困难产生的。)例如,英国人为了开辟市场,不得不把他们自己的资本贷到国外去。在生产过剩、信用制度等上,资本主义生产力图突破它本身的界限,超过自己的限度进行生产。一方面,它有这种冲动。另一方面,它只能忍受与有利润地使用现有资本相适应的生产。由此就产生了危机,它同时不断驱使资本主义生产突破自己的界限,迫使资本主义生产飞速地达到——就生产力的发展来说——它在自己的界限内只能非常缓慢地达到的水平。

匿名作者对萨伊的评论非常正确。这在分析萨伊时应该引用。(见第VII本第134页\endnote{马克思指他的关于政治经济学的札记本之一。马克思在这里提到的第VII本的前63页是1857—1858年经济学手稿的结尾部分(见卡·马克思《政治经济学批判大纲》1939年莫斯科版第586—764页)。从第VII本第63a页起(马克思在这里注明:“从1859年2月28日开始”)。是路德、兰盖、加利阿尼、维里、帕奥累蒂、马尔萨斯、理查·琼斯以及其他作者的著作的摘录。在第VII本第134页马克思从《论马尔萨斯先生近来提倡的关于需求的性质和消费的必要性的原理》这一著作(第110和112页)中摘录了有关匿名作者批判和讽刺萨伊的段落。——第130页。})

\begin{quote}{“他〈工人〉同意用他的一部分时间为资本家劳动,或者——其结果一样——同意把生产出来并拿去交换的总产品的一部分归资本家所有。他不得不这样做,否则资本家将不给他提供帮助}\end{quote}

(即提供资本。妙极了,按照匿名作者的意见,不论资本家占有全部产品而以其中一部分作为工资付给工人,还是工人把自己的一部分产品留下给资本家,“其结果一样”)。

\begin{quote}{但是,因为资本家的动机是盈利,并且因为这种利益在一定程度上总是既取决于积蓄的能力,又取决于积蓄的意愿,所以资本家愿意提供这种帮助的追加量;而同时,因为他将发现,需要这种追加量的人比过去需要原有量的人少,所以他只能指望,归他自己的那一部分利益少些;他不得不同意把他的帮助所创造的利益的一部分作为(可以说是)礼物〈!!!〉送给工人,否则他就得不到另外一部分利益。这样,利润就由于竞争而降低了。”(同上,第102—103页)}\end{quote}

真妙极了!如果由于劳动生产力的发展,资本积累十分迅速,以致对劳动的需求使工资提高,工人白白为资本家劳动的时间少些,并在一定程度上分到自己生产率较高的劳动所创造的利益,那末这就是资本家送给工人“礼物”!

同一位作者详细地证明,高工资对工人是一种不良刺激,虽然在谈到土地所有者时,他认为低利润会使资本家心灰意懒。(见第XII本第13页\endnote{马克思指他的第XII本札记本。在这个札记本的封面上马克思亲笔写着:“1851年7月于伦敦”。在第13页上对匿名著作《论马尔萨斯先生近来提倡的关于需求的性质和消费的必要性的原理》第97、99、103—104、106—108和111页做了摘录。在马克思的第XII本札记本第12页上摘录了上述著作第54—55页,其中谈到土地所有者(土地所有者的地租减低资本家的利润)。——第131页。})

\begin{quote}{“亚·斯密认为,资本的一般积累或增加会降低一般利润率,其原理与每个个别行业的资本的增加会降低该行业的利润是一样的。但是实际上,个别行业的资本的这种增加,意味着这里资本增加的比例比其他行业同期内资本增加的比例大。”(同上,第9页)}\end{quote}

驳萨伊。(见第XII本第12页\endnote{马克思在他的第XII本札记本第12页上从《论马尔萨斯先生近来提倡的关于需求的性质和消费的必要性的原理》(第15页)摘录了匿名作者对萨伊关于英国生产过剩的原因是意大利生产不足的论断的批评意见。参看本卷第1册第237页,第2册第606—607页,第3册第277页。——第131页。})

\begin{quote}{“可以说,劳动是资本的直接市场或直接活动场所。在一定的时间,在一定的国家,或在全世界,能按照不低于既定利润率投放的资本量,看来主要取决于因支出该资本而可能推动当时实有人数去完成的劳动量。”(同上,第20页)[814]“利润不是取决于价格,而是取决于同费用比较而言的价格。”(同上,第28页)“萨伊先生的论点\endnote{在这以前匿名作者从萨伊的著作(《给马尔萨斯先生的信》1820年巴黎版第46页)引用了萨伊的论点:“产品只是用产品购买的”。这个论点在萨伊那里还有另外的说法:“产品总是为自己开辟市场”(《论马尔萨斯先生近来提倡的关于需求的性质和消费的必要性的原理》1821年伦敦版第13、110页)。——第132页。}决不证明,资本为自己开辟市场,而只证明,资本和劳动相互为对方开辟市场。”(同上,第111页)}\end{quote}

\tsectionnonum{(C)托马斯·德·昆西[无法克服李嘉图观点的真正缺陷]}

[托马斯·德·昆西]《三位法学家关于政治经济学的对话,主要是关于李嘉图先生的〈原理〉》(载于1824年《伦敦杂志》第9卷)。

试图反驳一切对李嘉图的攻击。从下面这句话可以看出,他是知道问题所在的:

\begin{quote}{“政治经济学的一切困难可以归结为:什么是交换价值的基础?”(上述著作,第347页)}\end{quote}

在这一著作中常常尖锐地揭露李嘉图观点的不充分,虽然在这样做的时候,其辩证法的深度与其说是真实的,不如说是矫揉造作的。真正的困难(这些困难不是由价值规定产生的,而是由于李嘉图在这个基础上所作的说明不充分,由于他强制地和直接地使比较具体的关系去适应简单的价值关系)根本没有解决,甚至根本没有觉察到。但是,这本著作具有它出版的那个时期的特征。可以看出,那时在政治经济学中人们对待前后一贯性和思维还是严肃的。

(同一作者后来一本较差的著作;《政治经济学逻辑》1844年爱丁堡版。)

德·昆西尖锐地表述了李嘉图观点和前人观点不同之处,并且没有象后来人们所作的那样,企图通过重新解释来削弱或抛弃问题中所有独特的东西,只在文句上加以保留,从而为悠闲的无原则的折衷主义敞开大门。

李嘉图学说中有一点德·昆西特别强调,在这里我们也必须指出,因为它在我们马上就要考察的同李嘉图的论战中起作用,这个论点就是:一种商品支配其他商品的能力(它的购买力;事实上就是它用其他商品表示的价值)和它的实际价值根本不同。

\begin{quote}{“如果得出结论说,实际价值大是因为它购买的量大,或者实际价值小是因为它购买的量小,那完全是错误的……如果商品A的价值增加一倍,它支配的商品B的量并不因此就比以前增加一倍。情况可能如此,但是也可能支配的量是500倍或只是1/500……谁也不否认,商品A由于本身的价值加倍,所支配的一切价值不变的物品的量也将加倍……但是,问题在于,是不是在一切情况下,商品A在它的价值加倍时所支配的量都将加倍。”(散见《三位法学家的对话》第552—554页)}\end{quote}

\tsectionnonum{(d)赛米尔·贝利}

\tsectionnonum{[(a)《评政治经济学上若干用语的争论》的作者和贝利在解释价值范畴中的肤浅的相对论。等价物问题。否认劳动价值论是政治经济学的基础]}

[寨米尔·贝利]《对价值的本质、尺度和原因的批判研究;主要是论李嘉图先生及其信徒的著作》,《略论意见的形成和发表》一书的作者著,1825年伦敦版。

这是反对李嘉图的主要著作(也反对马尔萨斯)。试图推翻学说的基础——价值。除了“价值尺度”的定义,或者更确切地说,具有这一职能的货币的定义以外,从积极方面来看,没有什么价值。(并参看同一作者的另一著作:《为〈韦斯明斯特评论〉杂志上一篇关于价值的论文给一位政治经济学家的信》1826年伦敦版)

因为正如前面讲的\fnote{见本册第118页。——编者注},这部著作的基本思想是赞同《评政治经济学上若干用语的争论》的,所以这里还要回头去谈后一著作并引用其中有关的地方。

《评政治经济学上若干用语的争论》的作者责备李嘉图,说他把价值由商品在其相互关系中的相对属性变成某种绝对的东西。

在这方面,李嘉图应该受责备的只是,他在阐述价值概念时没有把不同的因素,即没有把在商品交换过程中出现或者说表现出来的商品交换价值和商品作为价值的存在(这种存在与商品作为物、产品、使用价值的存在不同)严格区分开来。

[815]《评政治经济学上若干用语的争论》中谈到:

\begin{quote}{“如果用来生产大部分商品或除一种商品以外的所有商品的绝对劳动量增加了,那末能不能说,这一种商品的价值仍然不变?因为它将同较少量的其他各种商品相交换。如果实际上断定,应当把价值的增加或减少理解为生产这种商品的劳动量的增加或减少,那末,我刚才加以反驳的结论,可能在一定程度上是正确的。但是象李嘉图先生那样,说生产两种商品的相对劳动量是这两种商品相互交换的比例的原因,即两种商品的交换价值的原因,这就同所谓每种商品的交换价值表示生产该商品的劳动量,而完全与其他商品或其他商品的存在毫无关系的说法完全不一样。”(《评政治经济学上若干用语的争论》第13页)“李嘉图先生的确告诉我们,‘他希望引起读者注意的这个研究,涉及的是商品相对价值的变动的影响,而不是商品绝对价值的变动的影响’\fnote{见本卷第2册第189页。——编者注},在这里他好象认为,有一种是交换价值而又不是相对价值的东西。”(同上,第9—10页)“李嘉图先生离开了他对价值这个词的最初用法,使价值成为某种绝对的东西而不是相对的东西。这一点在他的《价值和财富,它们的特性》这一章中表现得尤其明显。那里讨论的问题,其他经济学家也曾讨论过,那纯粹是毫无益处的用语的争论。”(同上,第15—16页)}\end{quote}

我们在评论这个人以前,还要谈谈李嘉图。他在其《价值和财富》一章中证明,社会财富不取决于所生产的商品的价值,虽然后一点对于单个生产者来说具有决定性的意义。因此,他更应该理解,仅仅以剩余价值为目的即以生产者群众的相对贫困为基础的生产形式,绝不能象他一再说明的那样,是财富生产的绝对形式。

现在,我们来谈这位在“用语”上自作聪明的人是怎样“评”[注;讽刺性地暗喻这个作者的著作《评政治经济学上若干用语的争论》。——编者注]的。

如果除了一种商品以外,所有商品都因为比以前花费了更多的劳动时间而增加了价值,那末,劳动时间没有变动的这种商品,就同较少量的其他所有商品相交换。这种商品的交换价值(就它实现在其他商品上来说),即表现在其他所有商品的使用价值上的交换价值减少了。“然而能不能说这一种商品的交换价值仍然不变?”这只是提出了所谈的问题,这里既没有肯定的回答,也没有否定的回答。如果生产一种商品所需要的劳动时间减少了,而生产其他所有商品的劳动时间不变,结果仍旧一样:一定量的这种商品将同较少量的其他所有商品相交换。在这里,在两种情况下发生了同样的现象,虽然发生的原因是直接相反的。反之,如果生产商品A所需要的劳动时间不变,而生产其他所有商品的劳动时间减少了,那末,它将同较大量的其他所有商品相交换。由于相反的原因,即生产商品A所需要的劳动时间增加了,而生产其他所有商品的劳动时间不变,也会产生同样的结果。因此,在第一种情况下,商品A同较少量的其他所有商品相交换,而这可能是由于两种相反的原因。在第二种情况下,它同较大量的其他所有商品相交换,这也可能是由于两种相反的原因。但是请注意,按照假定,它每次都是按它的价值进行交换,因而是同等价物相交换。商品A每次都把它的价值实现在它所交换的一定量的其他使用价值上,而不管这些使用价值的量怎样变动。

由此显然可以得出结论:商品作为使用价值相互交换的量的比例,诚然是商品价值的表现,是商品的实现了的价值,但不是商品价值本身,因为同样的价值比例可以表现在完全不同的使用价值量上。商品作为价值的存在不表现在商品本身的使用价值上——不表现在商品作为使用价值的存在上。商品的价值是在商品用其他使用价值来表现时显现出来的,也就是在其他使用价值同这一商品相交换的比例中显现出来的。如果1盎斯金=1吨铁,也就是说,如果少量的金和大量的铁交换,难道表现在铁上的一盎斯金的价值因此就比表现在金上的铁的价值大吗?商品按它们所包含的劳动进行交换,也就是说,就它们代表等量劳动来说,它们是相等的,同一的。因而这也是说,每一商品,就本身来看,是和它[816]自己的使用价值,和它自己作为使用价值的存在不同的东西。

同一商品的价值,依照我把它表现在这种或那种商品的使用价值上,可以表现为极其不同的使用价值量,但是价值本身不变。这虽然使价值的表现改变了,但是没有使价值发生变动。同样,所有可以表现商品A的价值的不同使用价值的不同量,都是等价物,它们不仅作为价值,而且作为等量的价值互相发生关系,因此,当这些极不相同的使用价值量互相代替时,价值仍然不变,就象它没有在极不相同的使用价值上获得表现一样。

如果商品按照它们代表等量劳动时间的那种比例进行交换,那末它们作为物化劳动时间的存在,它们作为物体化劳动时间的存在,就是它们的统一体,它们的同一要素。作为这样的劳动产品,商品在质上是同一的,只是在量上根据它们代表的同一物即劳动时间的多少而有所不同。它们作为这个同一要素的表现,是价值,而就它们代表等量劳动时间来说,它们是相等的价值,是等价物。为了可以在量上把它们加以比较,它们必须首先是同名的量,是在质上同一的。

正是作为这种统一体的表现,这些不同的物是价值,并且作为价值互相发生关系,而它们的价值量的差别,它们的内在的价值尺度,也就由此得出来。而且只是因为如此,一种商品的价值,才能体现、表现在作为它的等价物的其他商品的使用价值上。因此,单个商品本身——完全撇开它的价值在其他商品上的表现不谈——作为价值,作为这个统一体的存在,也和它本身作为使用价值,作为物不同。作为劳动时间的存在,商品是价值;作为一定量的劳动时间的存在,它是一定的价值量。

因此,我们这位自作聪明的人的下面说法是很典型的:“如果我们是这样理解,那我们就不是这样理解”,反之亦然。我们的“理解”和我们所说的事情的本质特征没有一点关系。当我们说某物的交换价值时,我们当然首先把它理解为能够同某一种商品相交换的其他任何一种商品的相对量。但是,经过更进一步的考察,我们将发现:要使某物同其他根本不同的无数物品——即使它们之间有自然的或其他的相似之处,在交换时也不会加以注意——相交换的比例成为稳定的比例,所有这些不同的各种各样的物品都必须看成是同一的共同的统一体的相应表现,即看成与它们的自然存在或外表完全不同的要素的相应表现。其次,我们还将发现,如果我们的“理解”有一些意义,那末某一商品的价值就不仅是某种使该商品与其他商品不同或相同的东西,而且是某种使该商品与它本身作为物,作为使用价值的存在不同的质。

\begin{quote}{“A物的价值的提高,只是指用B、C等物衡量的价值,即同B、C等物交换时的价值。”(同上,第16页)}\end{quote}

为了A物(例如书)的价值可以用B物煤和C物葡萄酒来衡量,A、B、C作为价值必须是与它们作为书、煤或葡萄酒的存在不同的东西。为了A的价值可以用B来衡量,A必须具有不以B对这种价值的衡量为转移的价值,并且二者[在质上]都必须等于表现在二者上的第三物。

说商品的价值因此就由某种相对的东西变为某种绝对的东西,是完全错误的。正好相反。作为使用价值,商品表现为某种独立的东西。而作为价值,它仅仅表现为某种设定的东西\endnote{设定的东西——黑格尔哲学术语,是指和无条件的、原初的、第一性的东西相区别的某种受制约的东西,不以本身为根据而以他物为根据的某种东西。——第138页。},某种仅仅由它与社会必要的、同一的、简单的劳动时间的关系决定的东西。这是相对的,只要商品再生产所必要的劳动时间发生变动,它的价值也就变动,虽然它实际包含的劳动时间并未变动。

[817]我们这位自作聪明的人陷入了拜物教多深,以及他怎样把相对的东西变为某种肯定的东西,下面的话是最清楚的说明:

\begin{quote}{“价值是物的属性,财富是人的属性。从这个意义上说,价值必然包含交换,财富则不然。”(同上,第16页)}\end{quote}

在这里财富是使用价值。当然,使用价值对人来说是财富,但是一物之所以是使用价值,因而对人来说是财富的要素,正是由于它本身的属性。如果去掉使葡萄成为葡萄的那些属性,那末它作为葡萄对人的使用价值就消失了;它就不再(作为葡萄)是财富的要素了。作为与使用价值等同的东西的财富,它是人们所利用的并表现了对人的需要的关系的物的属性。相反,在我们的作者看来,“价值”竟是“物的属性”!

商品作为价值是社会的量,因而,和它们作为“物”的“属性”是绝对不同的。商品作为价值只是代表人们在其生产活动中的关系。价值确实包含交换,但是这种交换是人们之间物的交换;这种交换同物本身是绝对无关的。物不论是在A手中还是在B手中,都保持同样的“属性”。“价值”概念的确是以产品的“交换”为前提的。在共同劳动的条件下,人们在其社会生产中的关系就不表现为“物”的“价值”。产品作为商品的交换,是劳动的交换以及每个人的劳动对其他人的劳动的依存性的一定形式,是社会劳动或者说社会生产的一定方式。

我在我的著作的第一部分\endnote{指《政治经济学批判》第一分册。见《马克思恩格斯全集》中文版第13卷第22—23页,以及第37—39页。——第139页。}曾经谈到,以私人交换为基础的劳动的特征是:劳动的社会性质以歪曲的形式“表现”为物的“属性”;社会关系表现为物(产品,使用价值,商品)互相之间的关系。我们这位拜物教徒把这个假象看成为真实的东西,并且事实上相信物的交换价值是由它们作为物的属性决定的,完全是物的自然属性。直到目前为止,还没有一个自然科学家发现,鼻烟和油画由于什么自然属性而彼此按照一定比例成为“等价物”。

可见,正是他这位自作聪明的人,把价值变为某种绝对的东西,变为“物的属性”,而不是把它仅仅看成某种相对的东西,看成物和社会劳动的关系,看成物和以私人交换为基础的社会劳动的关系,在这种社会劳动中,物不是作为独立的东西,而只是作为社会生产的表现被规定的。

但是,“价值”不是绝对的东西,不能把它看成某种独立存在的东西,这跟下面一点完全不同:商品必然会使它的交换价值具有一种不同于它的使用价值,或者说不同于作为实际产品的存在并且不依赖于这种存在的独立的表现,也就是说,商品流通必然导致货币的形成。商品使它的交换价值在货币中,首先是在价格中具有这种表现,在价格中,所有商品都表现为同一劳动的物化,都不过是同一实体在量上的不同表现。商品的交换价值在货币上的独立化本身,是交换过程的产物,是商品中包含的使用价值和交换价值的矛盾以及商品中同样包含的矛盾——一定的、特殊的私人劳动必然表现为它的对立面,表现为同一的、必要的、一般的并且在这种形式上是社会的劳动——发展的结果。商品表现为货币,不仅包含这样的意思,即商品的不同价值量,是通过它们的价值在一种特殊商品的使用价值上的表现来衡量,同时也包含下面的意思:所有商品都表现为一种形式,在这种形式中,它们作为社会劳动的化身而存在,因而可以同其他任何商品交换,可以随意转化为任何使用价值。所以,它们表现为货币——价格——最初只是观念的,只有通过实际的出卖才能实现。

李嘉图的错误在于,他只考察了价值量,因而只注意[818]不同商品所代表的、它们作为价值所包含的物体化的相对劳动量。但是不同商品所包含的劳动,必须表现为社会的劳动,表现为异化的个人劳动。在价格上,这种表现是观念的。只有通过出卖才能实现。商品中包含的单个人的劳动转化为同一的社会劳动,从而转化为可以表现在所有使用价值上,可以同所有使用价值相交换的劳动——这种转化,交换价值的货币表现中包含的这个问题的质的方面,李嘉图没有加以阐述。商品中包含的劳动必须表现为同一的社会劳动即货币,这种情况被李嘉图忽视了。

资本的发展,从它自己这方面看,已经是以商品交换价值的充分发展,因而也是以商品交换价值在货币上的独立化为前提。在资本的生产过程和流通过程中,作为独立形式的价值是出发点;这个价值保存下来,得到增加,在它赖以表现的商品所经历的一切变化中通过与其原有量的比较来衡量自己的增加程度,并且更换着作为它的躯壳的商品,而不管价值本身表现在极不相同的使用价值上。作为生产的先决条件的价值和由生产中产生的价值二者之间的关系,——而作为先决条件的价值的资本是同利润相对立的资本,——构成整个资本主义生产过程的包罗一切的和决定性的因素。这里不仅是象货币形式那样的价值的独立表现,而且是处于运动过程中的价值,也就是在使用价值经历极不相同的形式的过程中保存下来的价值。因此,价值在资本上的独立化程度比在货币上要高得多。

从以上所说可以判断我们这位在“用语”上自作聪明的人是多么高明,他把交换价值的独立化看成是空洞的词句、表达的手法、经院式的虚构。

\begin{quote}{“‘价值’或法文的valeur,不仅被绝对地,而不是相对地当作物的属性,甚至被一些作者当作可衡量的商品。‘占有价值’,‘转移价值的一部分’〈固定资本的一个非常重要的因素〉,‘价值总额或总和’等等,我不知道这一切都说明什么。”(同上,第57页)}\end{quote}

因为货币本身是商品,从而具有可变的价值,所以独立化的价值本身,在货币上也只获得相对的表现,这一点丝毫不会使问题发生变化,而是一种缺陷,这种缺陷的产生是由于商品的性质,由于商品的交换价值必然具有和商品的使用价值不同的表现而产生的。我们的这位作者已经充分暴露了他的“我不知道”。这种情况从他的批判的全部性质可以看出。这种批判企图用空谈来回避事物本身的矛盾的规定性中包含的困难,并把困难说成是思考的产物或定义之争。

\begin{quote}{“‘两物的相对价值’可能有两种含义:指两物互相交换或将要互相交换的比例,或者指各自交换或将要交换的第三物的相对量。”(同上,第53页)}\end{quote}

不用说,这是个绝妙的定义。如果3磅咖啡今天或明天同1磅茶叶相交换,那绝不是说,这里是等价物相交换。照这种说法,每种商品能够永远只按它的价值进行交换,因为它的价值是它偶然交换的另一种商品的任何量。但是当人们说3磅咖啡同它们的等价物茶叶交换时,通常不是“指”这个意思。这里是假定,在交换后和交换前一样,每个交换者手里都有价值相等的商品。不是两种商品互相交换的比例决定它们的价值,而是它们的价值决定它们互相交换的比例。如果价值不过是商品A偶然交换的商品量,A的价值怎样表现在商品B、C等等上呢?因为[819]在这种情况下两种商品之间既然没有内在的尺度,在A同B交换之前,A的价值就不能表现在B上。

相对价值,第一,指价值量——不同于是价值这种质。因此,后者也不是某种绝对的东西。第二,指一种商品表现在另一种商品的使用价值上的价值。这只不过是它的价值的相对表现——即就价值和它借以表现的商品的关系而言。一磅咖啡的价值只是相对地表现在茶叶上。要绝对地表现一磅咖啡的价值,——即使以相对的形式,即不按它和劳动时间的关系,而按它和其他商品的关系,——就必须把它表现在它和其他所有商品的无限系列的等式上。这将是咖啡的相对价值[在相对形式上]的绝对表现。价值的绝对表现就是价值在劳动时间上的表现,通过这种绝对表现,价值就会表现为某种相对的东西,然而是在那种使价值成为价值的绝对关系中表现的。

\centerbox{※     ※     ※}

现在我们来谈贝利。

他的著作只有一个积极的贡献:他最早比较正确地阐明了价值尺度,实际上就是阐明了货币的一种职能,或者说,阐明了具有特殊的形式规定性的货币。为了衡量商品的价值——为了确定外在的价值尺度——不一定要使衡量其他商品的商品的价值不变。(相反,我在第一部分\endnote{指《政治经济学批判》第一分册。见《马克思恩格斯全集》中文版第13卷第56—58页。——第143页。}已经证明,它必定是可变的,因为价值尺度本身是商品,而且必须是商品,否则它和其他商品就没有共同的内在尺度了。)例如,货币的价值变了,那它的变动对其他所有商品来说都是相同的。因此,其他商品的相对价值就象货币保持不变一样正确地表现在货币上。

这样,就把寻求“不变的价值尺度”的问题排除了。但是,这个问题本身(把不同历史时期的商品价值加以比较的兴趣,实际上不是经济学本身的兴趣,只是纯学术的兴趣\fnote{见本册第166—167页。——编者注})是由误解产生的,它隐藏着一个深刻得多和重要得多的问题。“不变的价值尺度”首先是指一种本身价值不变的价值尺度,就是说,因为价值本身是商品的规定性,“不变的价值尺度”就是指价值不变的商品。例如,金和银或谷物,或劳动,是这种商品,那我们就可以通过同这种商品的比较,通过其他商品同这种商品交换的比例,用其他商品的金价格、银价格、谷物价格或它们和工资的比例,准确地衡量这些其他商品的价值的变动。因此,在这样提出问题时,一开始就假定,“价值尺度”只指其他所有商品赖以表现其价值的商品——不管这是指其他所有商品真正赖以表现其价值的商品即货币,具有货币职能的商品,还是指由于自己价值不变而成为理论家用于计算的货币的那种商品。但是,很显然,无论如何这里涉及的仅仅是作为价值尺度——在理论上或实际上——而本身价值不会变动的货币。

但是要使商品能把它们的交换价值独立地表现在货币上,表现在第三种商品,即特殊的商品上,其前提已经是存在商品价值。余下的问题只在于在量上比较它们。为了使它们的价值和价值差别能够得到这种表现,已经有了一个使它们成为相同的东西(价值),使它们作为价值在质上相同的统一体作为前提。例如,一切商品都用金表现它们的价值,那末它们在金上的这种表现,它们的金价格,它们和金的等式,就是可以说明并计算它们之间的价值比例的等式,因为现在它们表现为不等量的金,并且商品以这种方法在它们的价格中表现为[820]同名称的可以比较的量。

但是要这样表现商品,商品必须已经作为价值而成为同一的。如果商品和金,或任何两种商品,不能作为价值,作为同一的统一体的代表互相表现,那末,每种商品的价值都用金来表现的问题就不能解决。换句话说,问题本身已经包含了这个前提。在谈得上用一种特殊的商品来表现商品的价值以前,商品已经被假定为价值,被假定为和商品的使用价值不同的价值。为了使两个不同的使用价值量可以作为等价物彼此相等,已经假定,它们都等于第三物,它们在质上相同,并且都只是这个等质物的不同量的表现。

因此,“不变的价值尺度”的问题,实际上只是为探索价值本身的概念、性质,价值规定——它本身不再是价值,因此也就不会作为价值发生变动——所作的错误表达。这种价值规定就是劳动时间——在商品生产中特殊地表现出来的社会劳动。劳动量没有价值,不是商品,而是使商品转化为价值的东西,是商品中的统一体,而商品作为这个统一体的表现,在质上相同,只是在量上不同。商品是一定量的社会劳动时间的表现。

假定金具有不变的价值。这样,如果一切商品的价值用金表现,我就能够用商品的金价格来衡量商品价值的变动。但是要用金来表现商品的价值,商品和金作为价值必须是同一的。金和商品只有作为这个价值一定量的表现,作为一定的价值量,才能是同一的。金的不变价值和其余各种商品的可变价值,并不妨碍它们作为价值是同一的,由同一实体构成。在金的不变价值帮助我们哪怕前进一步以前,商品的价值首先必须用金表现,用金估计——就是说,把金和商品当作同一的统一体的表现,当作等价物。

{为了用商品中包含的劳动量衡量商品,——时间是劳动量的尺度,——商品中包含的不同种类的劳动就必须还原为相同的简单劳动,平均劳动,普通的非熟练劳动。只有这样,才能用时间,用一个相同的尺度衡量商品中包含的劳动量。这种劳动在质上必须相同,才能使它的差别成为纯粹量上的差别,纯粹大小的差别。但是,还原为简单的平均劳动,这不是这种劳动(一切商品的价值都还原为这种作为统一体的劳动)的质的唯一规定。某种商品所包含的劳动量是生产该商品的社会必要量,因而劳动时间是必要的劳动时间,这是一种只和价值量有关的规定。但是构成价值统一体的劳动不只是相同的简单的平均劳动。劳动是表现在一定产品中的私人劳动。可是,产品作为价值应该是社会劳动的化身,并且作为社会劳动的化身应该能够由一种使用价值直接转化为其他任何使用价值(劳动赖以直接表现的一定的使用价值,对劳动来说应该是无关紧要的,这样,产品就能够由使用价值的一种形式转化为使用价值的另一种形式)。因此,私人劳动应该直接表现为它的对立面,即社会劳动;这种转化了的劳动,作为私人劳动的直接对立面,是抽象的一般劳动,这种抽象的一般劳动因此也表现为某种一般等价物。个人劳动只有通过异化,才实际表现为它的对立面。但是,商品必须在它让渡以前具有这种一般的表现。个人劳动必然表现为一般劳动,就是商品必然表现为货币。就这些货币当作尺度,当作商品价值的价格表现来说,商品得到了这种表现。但是商品只有实际转化为货币,只有通过出卖,才作为交换价值得到自己的这种适当的表现。第一个转化只是理论的过程,第二个转化才是实际的过程。

[821]因此,谈到作为货币的商品的存在时,应该指出,不仅商品在货币形式上取得了衡量其价值量的一定尺度,——因为它们都把自己的价值表现在同一商品的使用价值上,——而且它们都表现为社会的抽象的一般劳动的存在。这是这样一种形式,通过这种形式,它们都取得相同的外形,它们都表现为社会劳动的直接化身;并且作为这样的化身它们都起着社会劳动的存在的作用,能够直接地——与它们的价值量成比例地——同其他一切商品相交换;其实,商品在已经把商品转化为货币的人手中,不是表现为具有特殊使用价值形式的交换价值的存在,而仅仅表现为作为交换价值承担者的使用价值(例如金)的存在。商品可以低于或高于它的价值出卖。这只和它的价值量有关。但是当它每一次出卖,转化为货币时,它的交换价值都具有一种独立的、和它的使用价值不同的存在。现在它只是作为一定量的社会劳动时间存在,它用来证明这一点的就是它能够直接同任何商品相交换,能够(按照它的量)转化为任何使用价值。在考察货币时,这一点就同商品中包含的劳动作为商品的价值要素所经历的形式上的转化一样,不能忽视。但是通过货币——通过作为货币的商品所具有的这种绝对可交换性,通过它作为交换价值的绝对效能(这和价值量毫无关系,这不是量的规定,而是质的规定)——可以看到:由于商品本身所经历的过程,它的交换价值独立化了,实在地表现在某种和商品的使用价值并列的、独立的形式中,就同曾经观念地表现在它的价格上一样。

这一切表明,《评政治经济学上若干用语的争论》的作者和贝利都不懂得价值和货币的本质,因为他们把价值的独立化看成是经济学家的一种经院式的虚构。价值的这种独立化在资本中表现得更加明显,资本在某种意义上,可以称为处于运动过程中的价值,——这样一来,因为价值只是在货币中独立地存在,——又可以称为处于运动过程中的货币,这种货币经历一系列过程,在其中保存下来,从自身出发并以加大的量回到自身。现实的怪异也表现为用语的怪异,它和人们的常识相矛盾,和庸俗经济学家所指的以及他们认为是他们所说的相矛盾,这是不言自明的。在商品生产的基础上,私人劳动表现为一般的社会劳动;人与人的关系表现为物与物的关系并表现为物——由此而产生的矛盾存在于事物本身,而不是存在于表达事物的用语中。}

看起来李嘉图经常认为,事实上有时也谈到,好象劳动量解答了错误的或者说被错误地理解的“不变的价值尺度”问题,就象从前把谷物、货币、工资等看作解决这个问题的秘方而提出来一样。在李嘉图那里所以会发生这种错误的假象,是因为确定价值量,对于他来说,是决定性的任务。因此,他不懂得劳动在其中成为价值要素的特殊形式,特别是不懂得,个别劳动表现为抽象的一般劳动,并以这个形式表现为社会劳动的必然性。因此,他不懂得货币的形成同价值的本质,同价值由劳动时间决定这一规定有什么联系。

贝利的著作有一些贡献,因为他通过自己的反驳,揭露了表现为货币——一种与其他商品并列的商品——的“价值尺度”同价值的内在尺度和实体的混淆。但是,如果他本人把货币作为“价值尺度”,不只是作为量的尺度,而且作为商品的质的转化来分析,那末他本人就会对价值作出正确的分析。他没有这样做,却满足于对已经以价值为前提的外在“价值尺度”作表面的考察,停留在毫无意义的议论上。

[822]但是在李嘉图著作中还是可以找到个别段落,在那里他直接强调,商品中包含的劳动量所以是衡量它们的价值量、它们的价值量的差别的内在尺度,只是因为劳动是使不同的商品成为相同的东西,是它们的统一体,它们的实体,它们的价值的内在基础。他只是忘掉去研究,劳动在什么样的一定形式上才是这种东西。

\begin{quote}{“如果我们把劳动作为商品价值的基础,把生产商品所必需的相对劳动量作为确定商品相互交换时各自必须付出的相应商品量的尺度,不要以为我们否定商品的实际价格或者说市场价格对商品的这种原始自然价格的偶然和暂时的偏离。”(李嘉图《政治经济学和赋税原理》1821年伦敦第3版第80页)“[德斯杜特·德·特拉西说:]‘衡量……就是找出它们〈被衡量的物〉包含……多少同类的单位。’如果法郎和要衡量的物不能还原为某个对两者共同的另一尺度,法郎就只是衡量铸成法郎的金属本身数量的价值尺度。我认为,它们是可以这样还原的,因为它们两者都是劳动的结果;并且,因此劳动〈因为劳动是它们的动因〉是共同的尺度,用这个尺度可以计量它们的实际价值和相对价值。”(李嘉图《原理》1821年伦敦第3版,第333—334页)}\end{quote}

一切商品都可以还原为劳动即它们的统一体。李嘉图没有研究的,是作为商品的统一体的劳动赖以表现的特殊形式。因此他不懂得货币。因此,在他那里,商品转化为货币,纯粹是形式的、没有深入到资本主义生产内部实质的东西。但是,他告诉我们一点:只因为劳动是商品的统一体,只因为一切商品都是同一统一体——劳动——的表现,所以劳动是商品的尺度。劳动是商品的尺度,不过因为劳动是作为价值的商品的实体。李嘉图对表现在使用价值上的劳动和表现在交换价值上的劳动没有加以应有的区别。作为价值基础的劳动不是特殊的劳动,不是具有特殊的质的劳动。在李嘉图那里,到处都把表现在使用价值上的劳动同表现在交换价值上的劳动混淆起来。诚然,后一种形式的劳动只是以抽象形式表现的前一种形式的劳动。

在上面引用的段落中,李嘉图所谓的实际价值是指作为一定劳动时间的体现的商品。他所谓的相对价值是指一种商品中所包含的劳动时间在其他商品的使用价值上的表现。

现在来谈贝利。

贝利紧紧抓住作为商品的商品的交换价值赖以体现、表现的形式。如果一种商品的交换价值表现在充当货币的第三种商品(其他一切商品同样把它们的价值表现在它上面)的使用价值上,即表现在商品的货币价格上,商品的交换价值就表现为一般的形式。如果我把任何一种商品的交换价值表现在其他任何一种商品的使用价值上,即表现为谷物价格,麻布价格等等,商品的交换价值就表现为特殊的形式。事实上,一种商品的交换价值,对其他商品来说,始终只表现为它们进行交换的量的关系。单个商品本身不能表现一般劳动时间,或者说,单个商品只能以它和充当货币的商品的等式,即以它的货币价格的形式,表现一般劳动时间。但是在这种情况下,商品A的价值始终表现为执行货币职能的商品G的一定量的使用价值。

这是直接的现象。贝利就是紧紧抓住了这种现象。交换价值表现为商品进行交换的量的关系这种最表面的形式,在贝利看来,就是商品的价值。从表面进入深处,是不允许的。贝利甚至忘记一个简单的道理:如果y码麻布=x磅麦秆,那末,不同物品即麻布和麦秆间的这个等式就使它们成为等量。它们作为相等的东西的这种存在,必须不同于它们作为麦秆和麻布的存在。[823]它们不是作为麦秆和麻布相等,而是作为等价物相等。因此,等式的一方必须表现和等式的另一方相同的价值。因此,麦秆和麻布的价值必须既不是麦秆,也不是麻布,而是二者共同的同时又跟二者作为麦秆和麻布不同的东西。这是什么呢?贝利没有回答这个问题。他没有这样做,而是把政治经济学的所有范畴逐一论述,以便不断重复千篇一律的老调:价值是商品的交换比例,因而不是别的什么东西。

\begin{quote}{“如果某个物品的价值就是它的购买能力,那末就必须有供购买的东西。因此,价值除了仅仅表示两个物品作为可交换的商品相互间的比例之外,不表示任何肯定的或内在的东西。”(《对价值的本质、尺度和原因的批判研究》第4—5页)}\end{quote}

事实上贝利的全部智慧已经包含在这段话里了。“如果价值无非是购买能力”(一个绝妙的定义,因为“购买”不仅以价值,而且以价值的货币表现为前提),“那它就表示”等等。但是,我们首先要从贝利这段话中去掉荒谬地偷运进来的东西。“购买”就是把货币转化为商品。货币已经以价值和价值的进一步发展为前提。因此,首先必须抛开“购买”这个用语。否则就是用价值解释价值。我们必须用“同其他物品交换”代替“购买”。“必须有供购买的东西”,是一个完全多余的说明。如果“物品”作为使用价值被它的生产者消费,如果它不是仅仅占有其他物品的手段,不是“商品”,那自然就谈不上价值。

贝利首先谈的是“物品”。但是接着,“两个物品相互间的”比例在他那里变成“两个物品作为可交换的商品相互间的比例”。其实这里所谈的物品相互间只处于交换关系中或者说可交换的物品的关系中。正因为如此,它们才是“商品”,是和“物品”不同的东西。另一方面,“可交换的商品的比例”,或者是废话(因为“不可交换的物品”不是商品),或者是贝利先生自相矛盾。物品不应随便按照什么比例进行交换,它们应该作为商品进行交换,也就是说,应该作为可交换的商品,作为各自具有价值并应按照自己的等价程度相交换的物品,互相发生关系。这样,贝利就承认了:它们交换的比例,因而每种商品购买其他商品的能力,是由它的价值决定的,而不是这种能力决定它的价值,这种能力只是价值的结果。

总之,如果我们从贝利这段话里,去掉一切错误的,偷运进来的或没有意义的东西,这段话就是下面这样。

且慢!我们还必须去掉另外的陷阱和废话。在我们面前有两种用语:一种是“物品的交换能力”等等(因为“购买”一词在没有货币概念的情况下是不成立的,没有意义的),另一种是“一个物品同其他物品交换的比例”。如果“能力”应该表示某种和“比例”不同的东西,那就不能说,“交换能力”“仅仅表示比例”,等等。如果两个用语应该表示同一个东西,那末同一个东西用两个彼此迥然不同的用语来表示,只能产生混乱。一物对另一物的比例是两物间的比例,不能说这个比例是属于其中某一物。相反,一物的能力是该物内在的东西,尽管它这个内在的属性只能[824]表现在它对其他物的关系上。例如,引力是物本身的能力,虽然这种能力在没有东西可以吸引时是“潜在的”。在这里贝利试图把“物品”的价值说成是它内在的,而同时又只是作为“比例”才存在的东西。因此他先用“能力”这个词,然后又用“比例”这个词。

因此,贝利思想的精确表达是这样的:

\begin{quote}{“如果某个物品的价值就是它同其他物品交换的比例,那末,因此〈即因为“如果”〉,价值除了表示两个物品作为可交换的物品相互间的比例之外,不表示任何东西。”}\end{quote}

这个同义反复谁也不会否认。不过由此可以得出结论:物品的“价值”“不表示任何东西”。例如,1磅咖啡=4磅棉花。在这里,什么是1磅咖啡的价值呢?4磅棉花。什么是4磅棉花的价值呢?1磅咖啡。既然1磅咖啡的价值是4磅棉花,而4磅棉花的价值=1磅咖啡,所以很清楚,1磅咖啡的价值=1磅咖啡(因为4磅棉花=1磅咖啡)。A=B,B=A;所以A=A。因此,从这种说明中可以得出以下的结论:某个使用价值的价值等于该使用价值的一定量。因此,1磅咖啡的价值不过是1磅咖啡。如果1磅咖啡=4磅棉花,那很清楚,1磅咖啡>3磅棉花,1磅咖啡<5磅棉花。1磅咖啡>3磅棉花以及<5磅棉花,也表示咖啡和棉花之间的比例,就同1磅咖啡=4磅棉花表示这种比例完全一样。这个=并不比>或<表示更多的比例,而只是表示另一种比例。为什么正是等号(=)关系把咖啡的价值表现在棉花上,把棉花的价值表现在咖啡上?难道这个等号纯粹是由于一般地说这些量相互交换而得出来的吗?这个=只是表示交换这个事实吗?不能否认,如果咖啡随便按照什么比例和棉花交换,那末它们就是相互交换,如果商品之间的比例只由交换这个事实来确定,那末咖啡无论是和2磅、3磅、4磅或5磅棉花交换,咖啡的价值同样都表现在棉花上。但是比例这个词是指什么呢?咖啡本身决不包含什么“内在的、肯定的东西”来决定它按什么比例同棉花交换。贝利所说的比例,不是由咖啡内在的并和实际交换不同的某种属性决定的。这样,比例这个词有什么用呢?贝利所说的比例是什么呢?就是同一定量咖啡交换的棉花量。严格地说,贝利没有理由说,咖啡按照什么比例进行交换,而只能说,它现在或过去是按照什么比例进行了交换的。因为如果比例的确定先于交换,那末交换就由“比例”决定,而不是比例由交换决定了。因此,我们也必须把作为某种超越于咖啡和棉花之外并和它们脱离的东西的比例抛开。

[这样,上面引证的贝利的话就具有以下的形式:]

\begin{quote}{“如果某个物品的价值就是同它交换的另一物品的量,那末,因此,价值除了表示同它交换的另一物品的量之外,不表示任何东西。”}\end{quote}

一种作为商品的商品,只能把它的价值表现在其他商品上,因为对于它作为[单个]商品来说,一般劳动时间是不存在的。[于是,贝利认为,]如果一种商品的价值表现在另一种商品上,这种商品的价值就无非是它和另一种商品的等式。贝利不知厌倦地到处玩弄他的聪明(在他的表述中,这就是同义反复,因为他[实质上是]说:如果一种商品的价值无非是它和另一种商品的交换比例,那末价值就无非是这个比例),这就格外使读者厌倦。

他用下面这段话表明了他的哲学的深奥:

\begin{quote}{“如果某物没有另一物同它存在距离的关系,我们就无法谈某物的距离,同样,如果某种商品没有另一种商品[825]同它相比较,我们也就无法谈某种商品的价值。一物如果不同另一物发生关系,其本身就不能有距离,同样,一物如果不同另一物〈同商品的价值有关的社会劳动不是另一物吗?〉发生关系,其本身就不能有价值。”(同上,第5页)}\end{quote}

一物和另一物有距离,这个距离的确是该物和另一物之间的关系;但是距离同时又是跟两物之间的这种关系不同的东西。这是空间的一维,一定的长度,它除了能够表示我们的例子中两物的距离外,同样能够表示其他两物的距离。但是还不止于此。当我们说距离是两物之间的关系时,我们是以物本身的某种“内在的”东西,某种能使物互相存在距离的“属性”为前提的。语音A和桌子之间有什么距离呢?这个问题是没有意义的。当我们说两物的距离时,我们说的是它们空间位置的差异。因此,我们假定,它们二者都存在于空间,是空间的两个点,也就是说,我们把它们统一为一个范畴,都作为空间的存在物,并且只有在空间的观点上把它们统一以后,才能把它们作为空间的不同点加以区别。它们同属于空间,这是它们的统一体。\fnote{[XV—887}{关于贝利的荒谬观点,还要指出:

当他说A物和B物有距离时,他并不是比较它们,不是把它们统一为一个范畴,而是在空间上区别它们。据说,它们不是占有同一空间。但是,关于二者,他[实质上是]说:它们是空间的并且作为空间的物而不同。可见,他已预先把它们统一为一个范畴,使它们有了同一的统一体。而这里讨论的正是纳入统一范畴的问题。

如果我说,三角形A的面积等于平行四边形B的面积,意思不只是说,三角形的面积表现在平行四边形上,平行四边形的面积表现在三角形上。而且是说,如果三角形的高=h,底=b,则A=h·b/2,这是它本身具有的一种属性,平行四边形也具有这种属性,它同样=h·b/2。在这里,三角形和平行四边形作为面积,是同一的,是相等物,虽然它们作为三角形和平行四边形是不同的。为了使这些不同的东西相等,每一个都必须独自表现同一的统一体。如果几何学,象贝利先生的政治经济学一样,只满足于说,三角形和平行四边形相等是指三角形表现在平行四边形上,平行四边形表现在三角形上,那几何学就不可能有什么成就了。}[XV—887]]

但是互相可以交换的物品的这个统一体是什么呢?这种交换不是物品作为自然物互相保持的关系。它也不是物品作为自然物同人的需要的关系,因为不是物品的效用程度决定物品互相交换的量。那末使它们能按照一定比例互相交换的同一性是什么呢?它们作为什么才变得能够互相交换呢?

事实上,贝利在这整个问题上都只是追随《评政治经济学上若干用语的争论》的作者。

\begin{quote}{“它〈价值〉不能对相比较的物品中的一个物品来说变动了,而对另一个物品来说又没有变动。”(同上,第5页)}\end{quote}

这仍然只是说:一种商品的价值在另一种商品上的表现只能作为这种表现发生变化;而这种表现本身不是以一种商品,而是以两种商品为前提的。

贝利先生认为,如果谈的只是在互相交换中的两种商品,那末人们自然而然地就会发现他所谓的价值的纯粹相对性。蠢驴!似乎在两种商品互相交换,两种产品作为商品互相发生关系时,就用不着象在千万种商品互相交换时那样,说明它们的同一性在什么地方。此外,在只有两种产品存在的地方,产品决不会发展成商品,因此商品的交换价值也决不会发展。包含在产品I中的劳动就没有必要表现为社会劳动。因为产品不是作为生产者的直接消费品生产出来,而只是作为价值的承担者,也可以说,是作为支取所有社会劳动体现物的一定量的凭证生产出来,所以一切产品作为价值都必须具有一种和它作为使用价值的存在不同的存在形式。正是它们中包含的劳动作为社会劳动的这种发展,它们的价值的发展,决定了货币的形成,决定了商品必须互相表现为货币,即表现为交换价值的独立的存在形式;产品所以能这样,那只是因为它们把一种商品从商品总额中分离出来,所有商品都用这种分离出来的商品的使用价值来衡量自己的价值,从而把这种特殊商品中包含的劳动直接转化为一般的社会劳动。

贝利先生用他那种只抓住现象表面的古怪的思维方法,得出了相反的结论:价值概念所以会形成,——这个概念把价值由商品进行交换的纯粹量的关系,变为某种同这种关系无关的东西(他认为,这是把商品的价值变为某种绝对的东西,变为一种和商品分离的、烦琐的本质)——只是因为在商品之外存在货币,使我们习惯于不是从商品的相互关系来考察商品的价值,而是把商品的价值看成和第三物的关系,看成一种[826]和商品相互的直接关系不同的第三种关系。在贝利看来,不是产品作为价值的规定性,导致货币的形成,并表现为货币,而是货币的存在导致价值概念的虚构。下面一点历史地看是完全正确的:对价值的研究最初是根据商品作为价值的可以看得见的表现,根据货币,因此,探索价值规定就(错误地)表现为探索“价值不变”的商品,或探索作为“不变的价值尺度”的商品。因为贝利先生证明,货币作为价值的外在尺度——和价值表现——虽然具有可变的价值,却执行着它的任务,所以他认为这样就排除了价值概念——它不受商品价值量的可变性的影响——的问题,并且事实上根本用不着再去考虑价值是什么了。因为商品的价值在货币上——在特殊的第三种商品上——的表现并不排除这第三种商品的价值的变动,因为“不变的价值尺度”的问题消失了,所以价值范畴本身的问题也就消失了。贝利非常得意地用成百页的篇幅写出这么一些空空洞洞的废话。

在下面一些段落中,他喋喋不休地重复着同样的意思,其中一部分是逐字逐句从《评政治经济学上若干用语的争论》上抄来的。

\begin{quote}{“假定只有两种商品,它们按照它们所包含的劳动量的比例互相交换。如果……在后来一个时期生产商品A需要的劳动量增加一倍,而生产商品B所需要的劳动量不变,商品A的价值就会比商品B增加一倍……但是,虽然商品B是用和过去一样多的劳动量生产的,它的价值却不会保持不变,因为它只和商品A——根据假定,它是商品B可以相比较的唯一商品——的半数相交换。”(同上,第6页)“当我们谈两种商品之间的比例时,经常同其他商品〈不是把价值仅仅看成两种商品之间的比例〉或同货币比较,这就产生了关于价值是某种内在的和绝对的东西的观念。”(第8页)“我的主张是:如果所有商品都是在完全相同的条件下生产出来的,例如都只是由劳动生产的,那末,始终需要花费同量劳动的商品的价值,在其他各种商品的价值都发生变动时,不会保持不变。{即该商品的价值在其他商品上的表现不会保持不变。这是同义反复。}”(同上,第20—21页)“价值决不是内在的和绝对的东西。”(同上,第23页)“除了通过一定量的另一种商品,就无法表示或表现一种商品的价值。”(同上,第26页)}\end{quote}

(同样,除了通过一定量的音节,就无法“表示”或“表现”一种思想。贝利由此得出结论:思想不过是音节。)

\begin{quote}{“他们〈李嘉图及其信徒〉不是把价值看成两个物之间的比例,而是把价值看成由一定量劳动生产出来的肯定的成果。”(同上,第30页)“因为按照他们的学说,商品A和商品B的价值相互之间是作为生产它们的劳动量发生关系,或者说……是由生产它们的劳动量决定的,所以,看来他们作出了结论:商品A的价值,撇开同其他任何东西的关系,等于生产它的劳动量。最后这个论断无疑是没有任何意义的。”(同上,第31—32页)李嘉图及其信徒“把价值说成是一种一般的和独立的属性”。(第35页)“商品的价值必定是它在某物上的价值。”(同上)}\end{quote}

我们看到,为什么把价值限定在两种商品上,把价值看成两种商品之间的关系,对贝利来说是如此重要。但是这里发生了困难:

\begin{quote}{“因为每种商品的价值都表示该商品和另一商品的交换比例}\end{quote}

(在这里“交换[827]比例”是什么意思呢?为什么不是商品的“交换”呢?但同时在交换中应该表现一定的比例,而不只是交换的事实。因此,价值=交换比例),

\begin{quote}{所以根据它用来比较的商品,我们可以称它的价值为货币价值、谷物价值、呢绒价值;因此有千万种价值,有多少种商品,就有多少种价值,它们都同样是现实的,又都同样是名义的。”(同上,第39页)}\end{quote}

原来如此!价值=价格。在它们之间没有区别。在货币价格和其他任何价格表现之间也没有“内在的”区别,虽然实际上正是货币价格,而不是呢绒价格等等,表现商品的名义价值,一般价值。

但是,虽然商品有千万种价值,或者说千万种价格,有多少种商品,就有多少种价值,这千万种表现都始终表示同一价值。最好的证明就是:所有这些不同表现都是等价物,它们不仅在表现上可以互相代替,而且在交换本身中也互相代替。我们谈到其价格的商品的这种关系可以表现为所有不同商品的千万种不同的“交换比例”,然而这里始终表现同一关系。因此,这种始终同一的关系和它的上千种不同的表现是不一样的,或者说,价值和价格是不一样的,价格只是价值的表现:货币价格是价值的一般表现,其他各种价格是特殊表现。但是,甚至这个简单的结论,贝利也没有得出。在这里,不是李嘉图是虚构家,而是贝利是拜物教徒,因为他即使没有把价值看成(被孤立地考察的)个别物的属性,毕竟把价值看成物和物之间的关系,而实际上价值只不过是人和人之间的关系、社会关系在物上的表现,它的物的表现,——人们同他们的相互生产活动的关系。

\tsectionnonum{[(β)劳动价值和资本家利润问题上的混乱。贝利把内在的价值尺度同价值在商品或货币上的表现混淆起来]}

[关于劳动的价值,贝利说:]

\begin{quote}{“李嘉图先生相当机智地避开了一个困难,这个困难乍看起来似乎会推翻他的关于价值取决于在生产中所使用的劳动量的学说。如果严格地坚持这个原则,就会得出结论说,劳动的价值取决于在劳动的生产中所使用的劳动量。这显然是荒谬的。因此,李嘉图先生用一个巧妙的手法,使劳动的价值取决于生产工资所需要的劳动量;或者用他自己的话来说,劳动的价值应当由生产工资所必需的劳动量来估量;他这里指的是为生产付给工人的货币或商品所必需的劳动量。那我们同样也可以说,呢绒的价值不应当由生产呢绒所花费的劳动量来估量,而应当由生产呢绒所换得的银所花费的劳动量来估量。”(同上,第50—51页)}\end{quote}

以上所述,对李嘉图关于资本直接和劳动相交换而不是和劳动能力相交换的错误观念来说,是正确的。这也就是我们以前在别的形式上听到的\fnote{见本册第117页。——编者注}那种指责。仅此而已。对劳动能力来说,贝利的类比是不适用的。他不应该拿呢绒,而应该拿一种生物产品例如羊肉来和活的劳动能力比较。生产家畜所必需的劳动,除了照料家畜所花费的劳动以及生产其生活资料所花费的劳动以外,不应指家畜本身花费在消费行为即饮食行为上,一句话,花费在消化这些产品或生活资料的行为上的“劳动”。劳动能力的情况也完全一样。生产劳动能力所花费的劳动是什么呢?除了在培养劳动能力、教育、学徒上花费的劳动——这在谈到非熟练劳动时几乎是用不着考虑的——以外,劳动能力的再生产所花费的,不过是工人消费的生活资料的再生产所花费的劳动。生活资料的消化并不是“劳动”,[828]正如呢绒中包含的劳动,除了织布工人的劳动和羊毛、染料等等包含的劳动以外,并不是还包含羊毛本身的化学作用或物理作用——由于这种作用,羊毛象工人或家畜吸收食物那样吸收染料等等。

其次,贝利企图推翻李嘉图关于劳动的价值同利润成反比的规律。而且他企图推翻的恰恰是这个规律的正确部分。问题在于,他和李嘉图一样,把剩余价值和利润等同起来。他没有提到这个规律的唯一可能的例外,那就是:工作日延长,工人和资本家均等地分得工作日延长的成果。但是即使在这种情况下,因为劳动力[workingpower]的价值将更快地(在更少的年份内)被消费掉,剩余价值也会靠牺牲工人的生命增长起来,工人的劳动力同它给资本家提供的剩余价值相比就贬值了。

贝利的论据极为肤浅。他是从他的价值概念出发的。在贝利看来,商品的价值是商品价值在一定量的其他使用价值(其他商品的使用价值)上的表现。因此,劳动的价值等于劳动所交换的其他商品(使用价值)量。{商品A的交换价值怎么能表现在商品B的使用价值上这个实际问题,他根本没有考虑。}这样一来,只要工人得到同量商品,劳动的价值就仍然不变,因为它仍旧表现在同量的其他有用物上。利润则表示对资本的比例,或者说,也是对总产品的比例。但是,虽然在劳动生产率提高时资本家所得的总产品的比例增大了,工人所得的产品份额却可能仍旧不变。既然资本家得到的东西的价值不是由比例决定,而是由“这一价值在其他商品上的表现”决定,那就不能理解,贝利在谈到资本时怎么突然得出一个比例,这个比例对资本家有什么用处。

事实上,这就是我们在考察马尔萨斯时已经谈到的那种妙论\fnote{见本册第27—29页。——编者注}。工资等于一定量的使用价值。而利润是价值的比例(但是贝利不得不回避这种说法)。如果我按使用价值来计量工资,而按交换价值来计量利润,那就很明显,在二者之间既不存在反比,也根本不存在任何比例,因为在这种情况下,我是拿两个不能相比的量,两个没有共同基础的物来互相比较了。

但是,贝利在这里所说的关于劳动价值的观点,按照他的原则,也适用于其他任何商品的价值。任何一种商品的价值无非是同它交换的其他物的一定量。如果我用1镑换得20磅棉纱,那末,在贝利看来,即使用来生产1磅棉纱的劳动这一次比另一次多一倍,这1镑的价值也始终是同一的,就是说始终得到支付的。一个最普通的商人也不会相信,如果在物价昂贵时和产品丰富时都用1镑买得1夸特谷物,他这1镑换得了相同的价值。在这里,任何价值概念都消失了。剩下的只是一个没有解释也无法解释的事实:若干量商品A和若干量商品B按照随便什么样的比例相交换。不管这个比例怎样,它总是表示等价物。这样,连贝利关于“表现在商品B上的商品A的价值”这一说法本身,也失掉了任何意义。如果商品A的价值表现在商品B上,那末就必须假定,同一价值一次表现在商品A上,另一次表现在商品B上,因而A的价值当它表现在B上时,仍和原先一样。但是照贝利看来,不存在可以表现在B上的A的价值,因为除了这种表现之外,无论A或B都没有价值。照贝利的看法,表现在B上的A的价值和表现在C上的A的价值,必定是完全不同的,就好象B和C是不同的一样。我们在这里看到的,不是在两种表现上等同的同一价值,而是A的两种比例,这两种比例彼此没有任何共同之处,而且要说它们是等价表现,那是荒谬的。

\begin{quote}{[829]“劳动价值的提高或降低,意味着用以交换劳动的商品量的增加或减少。”(同上,第62页)}\end{quote}

真是胡说![从贝利的观点看来]劳动的价值或其他任何物的价值都不可能提高或降低。我今天用1A换得3B,明天换得6B,后天换得2B。但是在这一切情况下,[照贝利看来]A的价值都无非是A所换得的B量。它以前是3B,现在是6B。贝利怎么能说A的价值提高或降低呢?表现在3B上的A,和表现在6B或2B上的A,有不同的价值。不过,在这种情况下,就不是同一个A在同一个时间换得3B或2B或6B了。同一个A在同一个时间总是表现在同量的B上。只有就不同的时间而言,才能说A的价值变动了。但是A只能和“同时存在的”商品相交换,并且只有和其他商品相交换这个事实(而不只是交换的可能性)[照贝利的看法]才使A成为价值。只有现实的“交换比例”形成A的价值,而现实的“交换比例”当然只有对同一个时间的同一个A才能发生。因此贝利宣称,把不同时期商品的价值加以比较,是荒谬的。\endnote{[赛·贝利]《对价值的本质、尺度和原因的批判研究》1825年伦敦版第71—93页;参看本卷第2册第565页,第3册第165—167、175—176页。——第163页。}但是,由此他本来应当宣称,价值的提高或降低也是荒谬的(既然商品在一个时间的价值同它在另一个时间的价值不能比较,价值就不可能有提高或降低),——因而“劳动价值的提高或降低”也是荒谬的。

\begin{quote}{“劳动是一种可交换的物,即在交换中支配其他物的物;‘利润’这一用语却只意味着商品的份额或比例,而不是一种可以同其他物品相交换的物品。我们问工资是否提高了,我们的意思是:一定量的劳动是否换得比以前更多的其他物。}\end{quote}

(因此,当谷物贵了,劳动的价值就是降低了,因为它换得的谷物少了;另一方面,如果与此同时呢绒贱了,劳动的价值同时就是提高了,因为它换得的呢绒多了。这样,劳动的价值在同一个时间又提高又降低;它的价值的两种表现——在谷物上的和在呢绒上的——不是等同的,不是等价的,因为它的提高了的价值不可能等于它的降低了的价值);

\begin{quote}{但是我们问利润是否提高了,我们指的是……资本家的收入对所使用的资本是否有更大的比例。”(同上,第62—63页)“劳动的价值不单取决于总产品中为换得工人的劳动而给予工人的那个份额,而且也取决于劳动生产率。”(同上,第63—64页)“劳动的价值提高时利润必定下降的论点,只有在这种提高不是由劳动生产力的增长引起的情况下,才是正确的。”(同上,第64页)“如果劳动生产力增长,就是说,如果同一劳动在同一时间生产更多的商品,那末,劳动的价值可能提高而利润不降低;后者甚至还可能提高。”(同上,第66页)}\end{quote}

(按照这个观点,对于其他任何商品也可以说,它的价值的提高不会引起和它交换的其他商品的价值的降低,甚至还会引起对方价值的提高。例如,假定同一劳动以前生产1夸特谷物,现在生产3夸特。以前生产1夸特花费1镑,现在生产3夸特也花费1镑。如果现在2夸特和1镑交换,货币的价值就提高了,因为它现在表现在2夸特上,不是表现在1夸特上。这样,谷物的买者就用他的货币换得更大的价值。但是,谷物的卖者,把他只花费2/3镑的东西卖1镑,赚了1/3镑。结果,他的谷物的价值就在谷物的货币价格降低的同时提高了。)

\begin{quote}{[830]“不管6个工人劳动的产品是多少,不管它是100夸特谷物还是200夸特或300夸特,只要资本家在产品中所占的比例始终是四分之一,这四分之一表现在劳动上就始终是不变的。”}\end{quote}

(归工人所得的3/4产品,如果把它们表现在劳动上,也可以这么说。)

\begin{quote}{“如果产品是100夸特,就会有75夸特付给6个工人,因而归资本家所得的25夸特将支配2个工人的劳动。”}\end{quote}

(而付给工人的75夸特将支配6个工人的劳动。)

\begin{quote}{“如果产品是300夸特,6个工人就会得到225夸特,归资本家所得的75夸特,仍然将仅仅支配2个工人的劳动,不会更多。”}\end{quote}

(同样,归6个工人所得的225夸特仍然将仅仅支配6个工人的劳动,不会更多。)(既然如此,为什么万能的贝利不许李嘉图把工人得到的产品份额,也象资本家得到的产品份额那样表现在劳动上,并且把表现在劳动上的这两份产品的价值互相比较呢?)

\begin{quote}{“归资本家所得的比例的这种增加,就是表现在劳动上的利润的价值的增长,}\end{quote}

(既然“利润意味着……不是一种可以同其他物品相交换的物品”(见上述),因而不是意味着“价值”,贝利怎么能说利润的价值和利润的价值的增长呢?另一方面,归工人所得的比例不减少,归资本家所得的比例难道能够增加吗?)

\begin{quote}{或者换句话说,也就是利润支配劳动的能力的增加。”(同上,第69页)}\end{quote}

(资本家占有别人劳动的能力的这种增加,和工人占有自己劳动的能力的减少,岂不是正好一致吗?)

\begin{quote}{“对于利润和劳动的价值同时增长的学说,如果有人反驳说,生产出来的商品是资本家和工人能够取得他们的报酬的唯一源泉,从而一方得到的必然是另一方失掉的,那末,对这种反驳的回答是明确的。当产品量保持不变时,这种反驳不可否认是正确的;但是同样不可否认,如果产品增加一倍,即使一方所得的比例减少而另一方所得的比例增加,归双方所得的产品份额也可能都增加。”}\end{quote}

(这正好是李嘉图所说的。双方的比例是不能同时增加的;即使归双方所得的产品份额同时增加,它们也不能按同一比例增加,因为不然的话,份额和比例就成了一回事了。一方比例的增加,只能靠另一方比例的减少。\endnote{在整个这一论断中,所谓归工人所有的(以及归资本家所有的)产品“份额”,是指新加劳动物化在其中的那部分产品的实物单位量;所谓“比例”,是指这种产品归一方或他方的百分比。例如,如果工人的新加劳动物化在100实物单位的产品中,其中工人所得的部分占60%,资本家所得的部分占40%,那末,在产品量增加一倍(由于劳动生产率的增长)而工人和资本家按原来的比例分配产品时,工人所得的“份额”就会增加60实物单位,而资本家所得的“份额”只增加40实物单位。但是,如果这时资本家的部分从40%增加到48%,那末,工人的部分就会从60%减少到52%,虽然他们所得的“份额”还是会增加44实物单位(资本家所得的“份额”同时增加56实物单位)。——第165页。}贝利先生把归工人所得的产品份额叫作“工资的价值”,而把资本家所得的比例叫作“利润”的价值,换句话说,他认为同一商品有两个价值——一个在工人手里,另一个在资本家手里,这是他自己的胡说。)

\begin{quote}{“当产品量保持不变时,这种反驳不可否认是正确的;但是同样不可否认,如果产品增加一倍,即使一方所得的比例减少而另一方所得的比例增加,归双方所得的产品份额也可能都增加。而正是归工人所得的产品份额的增加,形成工人劳动价值的增长}\end{quote}

(因为这里所说的价值是指一定量的物品),

\begin{quote}{然而,正是归资本家所得的比例的增加,形成资本家的利润的增长}\end{quote}

(因为这里所说的价值,是指不按量而按所花费的劳动来估量的同一些物品)。

\begin{quote}{由此}\end{quote}

(就是说,由荒谬的双重尺度:一次是物品,另一次是同一些物品的价值)

\begin{quote}{可以十分明确地得出结论说,关于二者同时增加的假定一点也没有矛盾的地方。”(同上,第70页)}\end{quote}

这个针对着李嘉图的荒谬论断完全没有击中[831]目标,因为李嘉图只是断言,两个份额的价值的提高和下降必定成反比。贝利却只是反来复去地说:价值是同某一物品相交换的物品量。他在考察利润时不可避免地会陷入困境,因为这里是资本的价值同产品的价值相比较。于是他就寻找一条出路:他在这里把价值理解为物品表现在劳动上的价值(照马尔萨斯的样子)。

\begin{quote}{“价值是同时存在的各商品之间的比例,因为只有这样的商品能够互相交换;而如果我们把商品在一个时间的价值同它在另一个时间的价值相比较,那末所比较的就只是该商品在这些不同时间内对其他某种商品的比例。”(同上,第72页)}\end{quote}

因此,如前面所说的,既没有价值的提高,也没有价值的降低,因为价值的提高和降低总是意味着商品在一个时间的价值同它在另一个时间的价值相比较。同样,商品既不能低于它的价值,也不能高于它的价值出卖,因为它的价值就是它卖得的东西。价值和市场价格是等同的。实质上,甚至不能说“同时存在的”商品,现在的价值,而只能说过去的价值。一夸特小麦的价值是什么呢?就是它昨天卖得的一镑。因为它的价值只能是它所换得的东西,在它没有被交换的时候,它“对货币的比例”不过是想象的比例。但是交换一经完成,我们持有的就不是一夸特谷物,而是一镑,因而已经不能再说这一夸特谷物的价值了。贝利在谈到把不同时期的价值相比较时,只是指,比如说,对十八世纪和十六世纪的商品的不同价值的学术研究,这里就产生了一个困难,因为价值的同一货币表现,由于货币本身价值的变动,在不同时间表示不同的价值。这里的困难就在于把货币价格还原为价值。但是贝利真是一头蠢驴!在资本的流通过程或再生产过程中,把一个时期的价值同另一个时期的价值相比较,难道不正是生产本身赖以进行的经常业务吗?

贝利先生根本不懂得“商品价值决定于劳动时间”和“商品价值决定于劳动价值”这两种说法是什么意思。他根本不懂得这两者之间的差别。

\begin{quote}{“请不要以为我不是主张商品价值相互之比等于生产这些商品所必要的劳动量相互之比,就是主张商品价值相互之比等于劳动价值相互之比。我只是主张:如果前一种说法是正确的,后一种说法也就不会是错误的。”(同上,第92页)}\end{quote}

各种商品的价值由一种商品的价值决定(如果它们由“劳动价值”决定,那末,它们就是由另一种商品决定;因为劳动价值是以劳动作为商品为前提的),和各种商品的价值由一种没有价值、本身不是商品而是价值实体并且最先使产品成为商品的第三物决定,——在贝利看来是一回事。可是,在第一种情况下,谈的是商品的一种价值尺度,实际上也就是货币,就是其他商品借以表现自己价值的商品。要使这一点成为可能,必须已经有各种商品的价值存在作为前提。无论是计量的商品还是被计量的商品,在第三物上必须已经是同一的。相反,在第二种情况下,最先确定了这种同一性本身,然后它表现在价格上,表现在货币价格或其他任何价格上。

贝利把“不变的价值尺度”和对内在的价值尺度的寻求,也就是和价值概念本身等同起来。只要把这两个东西混为一谈,寻求“不变的价值尺度”甚至就成为一种理性的本能。而可变性正是价值的特点。对“不变的东西”的寻求表达了这样的思想:内在的价值尺度本身不能也是商品,也是价值,相反,它必须是某种构成价值,因而形成内在的价值尺度的东西。贝利证明说,[832]商品价值可以获得货币表现,而且如果商品的价值比例是既定的,一切商品都可以把自己的价值表现在一种商品上,虽然这种商品的价值也会变动。但是这种商品的价值不管怎么变动,它在同一时间内对其他商品来说总是不变的,因为它是对一切商品同时发生变动的。贝利由此得出结论说,不需要什么商品之间的价值比例,因而也用不着去寻找它。因为他发现它已在货币表现上反映出来,所以他就用不着去“了解”这种表现为什么是可能的,它是怎么决定的,它在事实上表示什么。

一般说来,以上所述,既可以用来反驳马尔萨斯,也可以用来反驳贝利,因为贝利认为,无论以劳动量还是以劳动价值作为价值尺度,涉及的是同一个问题,同一回事。其实,在后一种情况下,价值已经作为前提,问题是要找到衡量这些已经作为前提的价值的尺度,找到它们的外在尺度,它们作为价值的表现。在前一种情况下,研究的是价值本身的发生和内在性质。在后一种情况下,研究的是商品到货币的发展,或交换价值在商品交换过程中取得的形式。在前一种情况下,研究的对象是价值,它不依这种表现为转移,相反地是这种表现的前提。贝利和其他蠢驴都认为,决定商品的价值就是指找到商品价值的货币表现,找到商品价值的外在尺度。但是,其他蠢驴出于理性的本能,说在这种情况下,这个尺度必须具有不变的价值,即在实际上它本身必须处于价值的范畴之外。贝利则说,这里没有什么需要进一步考虑的,因为我们在实践中已经找到了现成的价值表现,这种表现本身具有并且可以具有可变的价值而不损害它的职能。

[问题的一般提法就是这样。]特别是贝利本人在前面曾经告诉我们,6个工人劳动的产品,即同一劳动量的产品,可以是100夸特或200夸特或300夸特谷物,而“劳动价值”,在贝利看来,只是这6个工人从100、200或300夸特中得到的份额。这个份额可以是每个工人50、60或70夸特\endnote{“每一个工人50、60或70夸特”是随便举的数字,如果拿符合贝利上述例子(第163—165页)的数字来代替,那就是:“每一个工人12+(1/2)、25或37+(1/2)夸特”。——169页。}。这样,就是照贝利本人所举的例子看来,劳动量和这个劳动量的价值也是两种极不相同的表现。怎么能认为,价值表现在劳动量上,同表现在与劳动量根本不同的劳动量的价值上是一样的呢?如果同样的劳动以前提供3夸特谷物,现在提供1夸特,而同样的劳动以前提供20码麻布(或3夸特谷物),现在仍旧提供20码麻布,那末,用劳动时间来计量,1夸特谷物现在就等于20码麻布,或20码麻布等于1夸特谷物,而3夸特谷物就等于60码麻布而不是等于20码麻布。因此,1夸特谷物的价值和1码麻布的价值,相对地变动了。但是它们按“劳动价值”来说却丝毫没有变动,因为1夸特谷物和20码麻布仍旧是同以前一样的使用价值。并且很可能1夸特谷物现在支配的劳动量不比以前多。

如果拿单个商品来说,那末,贝利的论断是毫无意义的。如果生产一双长靴所必要的劳动时间减少十分之九,那末一双长靴的价值就减少十分之九;表现在其他一切商品上——生产这些商品所必要的劳动保持不变或不按同一程度减少——也相对地减少。但是劳动价值——例如制靴业以及其他一切生产部门的日工资——可能保持不变,甚至可能提高。现在在一双长靴中包含的劳动少了,因而包含的有酬劳动也少了。但是当谈到劳动价值时,这并不是说,对一小时劳动,一般地说对较小量的劳动,要比对较大量的劳动,支付较少的报酬。贝利的命题只有对资本的总产品来说才会有某种意义。假定200双长靴和以前100双长靴一样是同量资本(和同量劳动)的产品。在这种情况下,200双长靴的价值就和以前100双长靴的价值一样。于是这里可以说,200双长靴对1000码麻布(假定这是200镑资本的产品)之比,等于这两笔资本所推动的劳动的价值之比。在什么意义上呢?难道在一双长靴对一码麻布之比也[833]可以这样说的意义上吗?

劳动价值是商品包含的劳动时间中由工人自己占有的那一部分;是产品中体现属于工人自己的劳动时间的那一部分。所以,如果商品的全部价值分解为有酬劳动时间和无酬劳动时间,并且无酬劳动时间对有酬劳动时间之比是同一的,就是说,如果一切商品中的剩余价值在总价值中占有同一比例,那末很明显,既然各商品相互之比等于它们中包含的总劳动量之比,各商品相互之比同时必定等于这些总劳动量中相同比例部分之比,因而也必定等于一种商品中的有酬劳动时间对另一种商品中的有酬劳动时间之比。

W∶W′=GA∶G′A′,这里,GA表示总劳动时间。GA/x=W中的有酬劳动时间,G′A′/x=W′中的有酬劳动时间,因为我们假定,这两种商品中的有酬劳动时间在总劳动时间中占有相同的比例部分。

W∶W′=GA∶G′A′;

GA∶G′A′=GA/x:G′A′/x。

所以,

W∶W′=GA/x∶G′A′/x,

即各商品相互之比等于它们包含的有酬劳动时间之比,或者说,等于它们包含的劳动的价值之比。

但是在这种情况下,劳动价值并不是象贝利所希望的那样来决定,而是它本身由[商品中包含的]劳动时间决定。

其次,——撇开价值转化为生产价格不谈而只考察价值,——各个资本是由不同比例的可变资本部分和不变资本部分构成的。所以,在考察价值时可以看出,不同商品包含的剩余价值在量上是不相同的,或者说,有酬劳动在总预付劳动中所占的比例对各种商品来说是不相同的。

总之,工资——或者说,劳动价值——在这里是商品价值的指数,并不是因为它是价值,并不是因为工资会提高或降低,而是因为某种商品包含的表现在工资上的有酬劳动量,是该商品所包含的劳动总量的指数(与其他商品相比)。

一句话,全部问题归结为:既然商品价值相互之比等于A∶A′(即商品中包含的劳动时间量之比),那末,它们相互之比也等于A/x∶A′/x,即商品中包含的有酬劳动时间量之比,——如果一切商品中有酬劳动时间对无酬劳动时间之比是相同的,就是说,如果不管总劳动时间是多少,有酬劳动时间总是等于总劳动时间除以x。但是,这个“如果”是不符合实际情况的。即使假定各生产部门工人的剩余劳动时间是相同的,各生产部门中有酬劳动时间对所耗费的劳动时间之比也是不同的,因为耗费的直接劳动对耗费的积累劳动之比不同。比如有两笔资本:50v+50c和10v+90c。假定在这两种情况下无酬劳动都等于有酬直接劳动的十分之一。这样,在第一种商品中包含的价值是105,在第二种商品中是101。有酬劳动时间在第一种情况下占预付劳动的1/2,在第二种情况下只占1/10。

[834]贝利说:

\begin{quote}{“如果商品相互之比等于生产它们的劳动量之比,它们相互之比也必定等于这种劳动的价值之比;因为否则就必然含有这样的意思:两种商品A和B可能在价值上相等,虽然在一种商品上所耗费的劳动的价值比另一种商品上所耗费的劳动的价值大或小;或者说,商品A和B在价值上可能不等,虽然它们各自耗费的劳动在价值上是相等的。但是,由价值相等的劳动生产出来的两种商品在价值上的这个差别,就会和公认的利润的均等相矛盾,而利润的均等是李嘉图先生和其他作者一致承认的。”(同上,第79—80页)}\end{quote}

在最后一句话里,贝利无意中摸索到对李嘉图的正确的反驳,李嘉图是直接把利润和剩余价值等同起来,把价值和费用价格等同起来的。这个反驳的正确表述就是:如果商品按自己的价值出卖,它们就提供不同的利润,因为这时利润等于商品本身包含的剩余价值。这个反驳是正确的。但是它不是反对价值理论,而是反对李嘉图在应用这个理论方面的错误。

不过,在上面引用的话里,贝利本人对问题的理解多么不正确,可以从下面这段话看出来:

\begin{quote}{“相反,李嘉图认为,‘劳动在价值上可以提高或降低而不影响商品的价值’。这个论断和另一个论断显然有很大的不同。它是否正确,实际上要看另一个论断是否错误,或者说,要看相反的论断如何。”(同上,第81页)}\end{quote}

这个蠢驴自己以前说过,同量劳动的结果可以是100、200或300夸特谷物。这一点决定一夸特谷物对其他商品的比例,而不管劳动价值如何变动,就是说,不管100、200或300夸特中归工人自己所得的是多少。假如这个蠢驴要在某种程度上保持前后一贯,他就应该说:劳动价值可以提高或降低,但是商品价值相互之比仍然等于劳动价值之比,因为——按照错误的假定——工资的提高或降低是普遍的,而且工资的价值在所耗费的劳动总量中始终占相同的比例部分。

\tsectionnonum{[(γ)把价值同价格混淆起来。贝利的主观主义观点。关于费用价格和价值的差额问题]}

[贝利说:]

\begin{quote}{“表现商品价值的能力同商品价值的不变性没有关系}\end{quote}

(确实没有关系!但是它同表现价值以前首先找出价值大有关系;同找出彼此极不相同的使用价值怎样归入价值这个共同的范畴和共同的名称,从而使一物的价值可以由另一物表现,大有关系},

\begin{quote}{无论是把商品互相比较,还是把它们同所使用的尺度比较,都是如此。同样,把这些价值表现加以比较的能力也同价值的不变性没有关系。”}\end{quote}

{如果不同商品的价值都表现在同一的第三种商品上(不管后者的价值如何变动),那末,把这些已经具有共同名称的表现加以比较,当然是很容易的。}

\begin{quote}{“A值4B还是6B}\end{quote}

{困难是怎样使A和若干数量的B相等,这只有当A和B有一个共同的统一体,或者说,A和B是同一个统一体的不同体现物时,才有可能。如果所有商品都必须表现在金上,表现在货币上,困难仍然一样。在金和其他每种商品之间必须有一个共同的统一体},

\begin{quote}{以及C值8B还是12B,这对于把A和C的价值表现在B上的能力是无关紧要的,而且——既然A和C的价值都表现在第三种商品B上——对于把A和C的价值加以比较的能力,当然也是无关紧要的。”(同上,第104—105页)}\end{quote}

但是,A怎样表现在B或C上呢?必须把A、B、C看成某种和它们作为物、产品、使用价值不同的东西,才能使“它们”互相表现,换言之,才能把它们当作同一的统一体的等价表现。A=4B。因而,A的价值表现在4B上,而4B的价值表现在A上,结果等式的两方表现同一的东西。它们是等价物。它们两者都是价值的相等的表现。如果它们是不相等的表现,如A>4B或A<4B,也是一样。在所有这些情况下,只要[835]它们是价值,它们就只是在量上不同或相等,但是始终是同一个质的量。困难在于找到这个质。

\begin{quote}{“在对价值进行计量的过程中,必要条件是使被计量的商品具有共同的名称}\end{quote}

{例如,为了把三角形和其他一切多角形加以比较,只须把多角形化为三角形,把它们表现在三角形上。但是要这样做,三角形和多角形事实上就被看成等同的东西,看成同一个东西——空间——的不同表现形式},

\begin{quote}{这在任何时候都可以同样轻而易举地做到;或者更确切地说,这是现成的,因为这就是记录下来的商品价格,或者说商品对货币的价值比例。”(同上,第112页)“决定价值也就是表现价值。”(同上,第152页)}\end{quote}

这里我们把这个家伙弄清了。我们看到价值已用价格来计量和表现了。因此,[贝利认为]我们也就可以满足于不知道什么是价值。贝利把价值尺度到货币的发展,进而把货币作为价格标准的发展,同价值作为商品交换的内在尺度的发展中价值概念本身的确立混为一谈。他正确地认为,这种货币没有必要成为价值不变的商品;但是他由此得出结论:独立于商品本身之外、与商品本身不同的价值规定是没有必要的。

只要把商品的价值作为商品的共同的统一体,商品相对价值的计量和这种价值的表现就一致了。但是,在我们找到和商品的直接存在不同的统一体以前,我们将看不到这种表现。

就拿贝利关于物品A和B之间的距离的例子\fnote{见本册第154—155页。——编者注}来说,也可以看出这一点。当我们说它们之间有距离时,我们已经假定,它们二者是空间的点(或线)。如果把它们看成点,而且是同一线上的点,那它们的距离就可以用寸、尺等表示。A和B这两种商品的统一体,乍看起来,就是它们的可交换性。它们是“可交换的”物品。作为“可交换的”物品,它们是同一名称的量。但是,“它们”作为“可交换的”物品的存在必须和它们作为使用价值的存在不同。这种存在是什么呢?

货币本身已经是价值的表现,是以价值为前提的。货币作为价格标准,又已经以商品转化(理论上)为货币作为前提。如果所有商品的价值都表现为货币价格,我就可以比较它们;事实上它们已经被比较了。但是要把价值表现为价格,商品的价值必须先表现为货币。货币只不过是商品价值在流通过程中借以表现的形式。但是我怎样才能把x棉花表现在y货币上呢?这个问题可以归结为:一般地说,我怎样才能把一种商品表现在另一种商品上,或者说,把商品表现为等价物?只有离开一种商品在另一种商品上的表现去分析价值,才能回答这个问题。

\begin{quote}{“认为……不同时期的商品之间可以存在价值比例,是错误的,按照事物的性质,也是不可能的;而既然不存在这种比例,也就不能进行计量。”(同上,第113页)}\end{quote}

这种谬论前面已经有过\fnote{见本册第163、166—167页。——编者注}。在货币执行支付手段职能时,就已经存在“不同时期的商品之间的价值比例”。整个流通过程都是不同时期商品价值不断比较的过程。

\begin{quote}{“如果它〈货币〉不是不同时期商品进行比较的好手段,那末,这就意味着:它不能在不存在任何可以由它执行的职能的地方执行职能。”(同上,第118页)}\end{quote}

作为支付手段和贮藏货币,货币就是要执行这种比较不同时期商品价值的职能。

事实上,这全部谬论的秘密由下面一段话透露出来了,这段话完全是从《评政治经济学上若干用语的争论》的作者那里抄来的\fnote{同上,第138页。——编者注},它使我相信,贝利作为剽窃者利用了被他小心翼翼地隐瞒起来的《用语的争论》。

\begin{quote}{[836]“财富是人的属性,价值是商品的属性。人或共同体是富的;珍珠或金刚石是很有价值的。”(《对价值的本质、尺度和原因的批判研究》第165页)}\end{quote}

珍珠或金刚石所以有价值,是因为它们是珍珠或金刚石,也就是由于它们的属性,由于对人有使用价值,——也就是由于它们是财富。但是在珍珠或金刚石中没有什么东西可以确定它们和其他[使用价值]之间的交换比例。

贝利突然成了高深莫测的哲学家:

\begin{quote}{“在作为价值原因的劳动和作为价值尺度的劳动之间,总之,在价值的原因和尺度之间,是有区别的。”(第170页及以下各页)}\end{quote}

的确,在“价值尺度”(指货币而言)和“价值原因”之间有非常显著的区别(而且被贝利忽略了)。价值的“原因”把使用价值转化为价值。外在的价值尺度已经以价值的存在为前提。例如,金所以能成为棉花的价值尺度,只是因为金和棉花作为价值具有和二者都不相同的统一体。价值的“原因”是价值的实体,因而也是内在的价值尺度。

\begin{quote}{“一切在商品交换中间接或直接对人的意识起决定性影响的……情况,都可以看作价值的原因。”(第182—183页)}\end{quote}

实际上,这不过是说:那些使卖者或者使买者和卖者把某种东西看成商品的价值或等价物的情况,是商品价值的原因或两种商品等价的原因。把决定商品价值的“情况”,说成影响交换者的“意识”而本身又存在于交换者的意识中(也许不存在,也许以歪曲的形式存在),这样,就根本不能进一步认识它。

这些(虽然影响意识、但是独立于意识之外的)迫使生产者把他们的产品作为商品出卖的情况,——这些使一种社会生产形式区别于另一种社会生产形式的情况,——赋予他们的产品(也给他们的意识)一种与使用价值无关的交换价值。这些产品的生产者的“意识”可以完全不知道他们的商品的价值实际上是怎样决定的,或者说,是什么东西使他们的产品成为价值的,——对于意识来说,这甚至可能不存在。产品的生产者被置于决定他们的意识的条件下,而他们自己却不一定知道。每个人都可以把货币作为货币使用,而不知道货币是怎么一回事。经济范畴反映在意识中是大大经过歪曲的。贝利所以把问题转入意识领域,是因为他在理论上走进了死胡同。

贝利不说,他自己所理解的“价值”(或“价值的原因”)是什么,而对我们说,这是买者和卖者在交换活动中所想象的东西。

但是,实际上作为这个貌似哲理的词句的基础的是:

(1)市场价格是由表现在供求关系中的不同情况决定的,而这些情况本身影响市场上的交易者的“意识”。一个非常重要的发现!

(2)在商品价值转化为费用价格时要考虑到作为“补偿理由”影响意识或在意识中出现的“不同的情况”。但是,所有这些补偿理由只影响作为资本家的资本家的意识,并且它们是由资本主义生产本身的性质产生的,而不是由买者和卖者的主观理解产生的。在买者和卖者的头脑中,它们毋宁说是作为不言而喻的“永恒真理”存在的。

贝利和他的前辈一样,抓住李嘉图把价值和费用价格混淆起来这一点来证明,价值不是由劳动决定的,因为费用价格同价值相偏离。用这一点反对李嘉图的[把价值和费用价格]等同是完全正确的,但用来反对[价值决定于劳动的]论点本身则是不正确的。

在这方面,贝利首先引证了李嘉图本人谈到的商品相对价值[837]由于劳动价值提高而变动的论点\fnote{见本卷第2册第196—221页。——编者注}。其次,他引证了“时间的影响”(在不延长劳动时间情况下生产时间的差别),即已经引起穆勒疑问的同一情况\fnote{见本册第89—91页。——编者注}。贝利没有看到真正的普遍的矛盾——虽然资本构成不同,资本周转时间不同等等,却存在着平均利润率。他只是重复了李嘉图本人和后来的作者已经注意到的这个矛盾的个别表现形式。因此,他在这里不过是一个应声虫:他没有使批判前进一步。

其次,他强调生产费用是“价值”的主要原因,因而是价值的主要要素。但是,他象李嘉图以后的其他作者一样,正确地指出,生产费用概念本身有不同的含义。最后他本人宣称,他同意托伦斯的价值由预付资本决定的观点,这个观点对费用价格来说是正确的,但是如果它不从价值本身的发展得出,也就是说,如果想由更发达的关系,即由资本的价值得出商品的价值,而不是相反,那是毫无意义的。

他最后的一个反驳是:如果一个行业的劳动时间不等于另一个行业的劳动时间,以致例如体现工程师12小时劳动的商品的价值比体现农业工人12小时劳动的商品的价值大一倍,那末,商品的价值就不能用劳动时间计量。这可以归结为:例如,简单劳动日如果有其他劳动日作为复杂劳动日与之相比,就不是价值尺度。李嘉图已经证明,如果简单劳动和复杂劳动之比是既定的,上述事实并不妨碍用劳动时间计量商品。\endnote{大·李嘉图《政治经济学和赋税原理》1821年伦敦第3版第13—15页。——第179页。}诚然他没有说明,这种比例是怎样发展和决定的。这属于对工资问题的说明,这归根到底就是劳动能力本身的价值的差别,即劳动能力的生产费用(由劳动时间决定)的差别。

下面就是贝利对前面已经概括的观点加以表述的段落:

\begin{quote}{“事实上,说生产费用是决定这类商品〈不存在垄断,而且只要扩大生产就可以增加产量的商品〉交换量的主要情况,是不会引起异议的;但是什么是生产费用,我们最优秀的经济学家的理解是不完全一致的;有些人主张,耗费在商品生产上的劳动量构成它的费用;另外一些人则主张,应该把用在这上面的资本叫做生产费用。”(同上,第200页)“劳动者没有资本而生产的东西,花费他的是他的劳动;资本家生产的东西,花费他的是他的资本。”(第201页)}\end{quote}

(正是这个理由决定了托伦斯的观点。资本家使用的劳动,除了他用在工资上的资本以外,没有花费他任何东西。)

\begin{quote}{“大部分商品的价值是由用在商品上的资本决定的。”(第206页)}\end{quote}

贝利对商品价值仅仅决定于商品中包含的劳动量的论点提出了以下反驳意见:

\begin{quote}{“只要我们能够找到任何下面这类例子,这种看法就不可能是正确的。第一种情况是:由同量劳动生产的两种商品,卖得不同量的货币;第二种情况是:以前价值相等的两种商品,虽然使用的劳动量没有任何变动,但在价值上变得不等了。”(第209页)“如果我们和李嘉图先生一样,说‘对不同性质的劳动的估量,能够在市场上迅速地而且对所有实际目的都十分准确地确定’;或者和穆勒先生一样,说‘在估量等量劳动时当然要注意不同的繁重程度和不同的熟练程度’,这都不是〈对第一种情况的〉回答。这种例子完全破坏了规则的普遍适用性。”(第210页)“比较一个劳动量和另一个劳动量,只可能有两种方法;一种是按照耗费的时间,另一种是按照生产出来的结果〈这种方法用于计件工资制〉。前一种方法适用于一切种类的劳动;后一种方法只能用于比较耗费在同类物品上的劳动。因此,如果在估量两种不同劳动时,所耗费的时间不决定劳动量之间的[839]\endnote{马克思在编手稿页码时把“838”误写为“839”。——第180页。}比例,那末这种比例就必然始终是不确定的和无法确定的。”(第215页)“关于第二种情况:试举任何两种价值相等的商品A和B为例;一种是用固定资本生产的,另一种是不用机器由劳动生产的,并且假定,在固定资本或劳动量没有任何变动的情况下,劳动价值提高了。按照李嘉图先生自己的论据,A和B之间的价值比例马上会发生变化,就是说,它们的价值将变得不等了。”(第215—216页)“对这两种情况我们还可以加上时间对价值的影响。如果生产一种商品比生产另一种商品需要的时间多,那末,即使它不需要较多的资本和劳动,它的价值也较大。李嘉图先生承认这个原因的影响。但是穆勒先生主张”……(第217页)}\end{quote}

最后,贝利还谈到下面一点,这是他在这方面提出的唯一的新东西:

\begin{quote}{“上述三类商品{这一点,即这三类商品,又是从《评政治经济学上若干用语的争论》的作者那里抄来的}〈即(1)在绝对垄断下生产的商品;(2)在有限的垄断下生产的商品,如谷物;(3)在完全自由竞争下生产的商品〉不可能绝对分开。它们不仅毫无区别地互相交换,而且在生产过程中混合在一起。因此,一种商品的一部分价值可能由垄断造成,而另一部分价值则可能由那些确定非垄断产品价值的原因造成。例如,一种物品可以在最自由的竞争下用原料生产者依靠完全的垄断按照六倍于实际费用的价格出卖的原料生产出来。”(第223页)“在这种情况下很清楚,尽管可以正确地说,物品的价值由工厂主花费在它上面的资本量决定,但是任何分析也不能把这笔资本的价值归结为劳动量。”(第223—224页)}\end{quote}

这个意见是正确的。但是垄断在这里和我们没有关系,因为我们所涉及的只是两个范畴,即价值和费用价格。很明显,价值转化为费用价格有双重作用。第一,加到预付资本上的利润可以高于或低于商品本身包含的剩余价值,即利润所代表的无酬劳动可以大于或小于商品本身所包含的。这一点适用于商品中的可变资本部分及其再生产。但是,除此之外,不变资本——或者说,作为原料、辅助材料和劳动工具,总之作为劳动条件加入新生产的商品的价值的商品——的费用价格,同样可以高于或低于它们的价值。因此,加入新生产的商品的价值的,是偏离了价值的价格部分,这个价格部分不取决于新加劳动量,或者说,不取决于使这些具有一定费用价格的生产条件转化为新产品的劳动量。总之很清楚,对商品本身——作为生产过程的结果的商品——的费用价格和价值之间的差额适用的东西,同样适用于以不变资本的形式,作为组成部分,作为前提进入生产过程的商品。可变资本,无论它的价值和费用价格之间有多大差额,总是由构成新商品的价值组成部分的一定劳动量补偿的,至于新商品的价值是恰好表现在新商品的价格上,还是高于或低于价格,那是无关紧要的。相反,如果说的是不依赖新商品本身的生产过程而加入该商品的价格的价值组成要素,那末费用价格和价值的这种差额将作为先决要素转入新商品的价值。

因此,商品的费用价格和价值之间的差额是由双重原因产生的:一方面是那些构成新商品生产过程的前提的商品的费用价格和价值之间的差额,另一方面是实际加到生产费用上的剩余价值和[按预付资本]计算的利润之间的差额。但是,每一种作为不变资本加入另一种商品的商品本身都是作为结果,作为产品从另一个生产过程出来的。因此,一种商品交替地时而表现为其他商品的生产的前提,时而表现为生产过程的结果,在这个过程中其他商品的存在又是这种商品的生产的前提。在农业(畜牧业)中,同一商品时而表现为产品,时而表现为生产条件。

费用价格对价值的这种有重要意义的偏离——这种偏离是由资本主义生产决定的——丝毫没有改变费用价格照旧是由价值决定这个事实。

\tchapternonum{(4)麦克库洛赫}

\tsectionnonum{[(a)在彻底发展李嘉图理论的外表下使李嘉图理论庸俗化和完全解体。肆无忌惮地为资本主义生产辩护。无耻的折衷主义]}

[840]麦克库洛赫是李嘉图经济理论的庸俗化者,同时又是使这个经济理论解体的最可悲的样板。

他不仅是李嘉图的庸俗化者,而且是詹姆斯·穆勒的庸俗化者。

而且,他在一切方面都是庸俗经济学家,是现状的辩护士。使他担心到可笑地步的唯一事情,就是利润下降的趋势;他对工人的状况是完全满意的,总而言之,他对沉重地压在工人阶级身上的资产阶级经济的一切矛盾都是满意的。在这里,一切都生气勃勃。在这里,他甚至知道,

\begin{quote}{“一个生产部门采用机器,必然会在其他某一生产部门造成同样大的或更大的对被解雇的工人的需求”。\endnote{约·雷·麦克库洛赫《政治经济学原理》1825年爱丁堡版第181—182页。这段引文,马克思是从卡泽诺夫《政治经济学大纲》(1832年伦敦版)一书中转引来的。见本册第68页。——第183页。}}\end{quote}

在这个问题上他背离了李嘉图,正象他在后来的一些著作中开始对土地所有者大加奉承一样。但是,鉴于利润率下降的趋势,他把全部温情脉脉的关怀都倾注在可怜的资本家身上。

\begin{quote}{“麦克库洛赫先生看来和其他科学代表人物不同,他不是寻求具有特征的区别,而只是寻求类似之处;按照这个原则,他就把物质对象和非物质对象、生产劳动和非生产劳动、资本和收入、工人的食物和工人本身、生产和消费以及劳动和利润,统统混淆起来。”(马尔萨斯《政治经济学定义》1827年伦敦版第69—70页)“麦克库洛赫先生在他的《政治经济学原理》(1825年伦敦版)一书中,把价值区分为实际价值和相对价值即交换价值。他在第211和225页上说,前者‘取决于耗费在占有或生产商品上的劳动量,而后者取决于商品换得的劳动或其他任何商品的量’;而且,他说(第215页),在通常状况下,即当市场上的商品供给和对商品的有效的需求完全一致的时候,这两种价值是等同的。但如果它们是等同的,那末他谈的两个劳动量也应该是等同的。但是,他在第221页告诉我们,它们不是等同的,因为一个包括利润,另一个不包括利润。”([卡泽诺夫]《政治经济学大纲》1832年伦敦版第25页)}\end{quote}

麦克库洛赫在他的这本《政治经济学原理》第221页上是这样说的:

\begin{quote}{“事实上它〈商品〉换得的总是更多{即比生产该商品所用的劳动更多的劳动},而且正是这个余额构成利润。”}\end{quote}

这是这个苏格兰大骗子所用的手法的鲜明例证。

马尔萨斯、贝利等人的争论,迫使他把实际价值和交换价值即相对价值区别开来。但是他所作的这种区别实际上就是他在李嘉图那里发现的区别。实际价值,就是从生产商品所必需的劳动来看的商品;相对价值就是各种不同商品的比例,这些商品可以用同样的时间生产出来,因而它们是等价物,因此,其中一种商品的价值,可以用花费同样多劳动时间的另一种商品的使用价值量来表现。商品的相对价值,按李嘉图的这种见解,不过是它的实际价值的另一种表现,不过意味着各种商品按照它们包含的劳动时间进行交换,或者说,它们各自包含的劳动时间是相等的。因此,如果商品的市场价格等于它的交换价值(在需求和供给相符时就是如此),那末买进的商品包含的劳动就同卖出的商品包含的一样多。如果在交换时商品换回的和在商品中付出的劳动量相同,那末商品仅仅实现它的交换价值,或者说,商品不过按它的交换价值出卖。

这一切,库洛赫都加以确认,象鹦鹉学舌那样正确地加以重复。不过,他在这里走过了头,因为马尔萨斯的交换价值规定——交换价值是商品支配的雇佣劳动量——已经深入他的内心。他因此把相对价值规定为“商品换得的劳动或其他任何商品的量”。李·嘉图在考察相对价值时,始终只谈劳动以外的商品,因为在商品交换时,利润所以实现,仅仅因为商品同劳动交换并不是等量劳动相交换。李嘉图在其著作一开头就特别强调指出:商品价值决定于[841]商品中包含的劳动时间,和商品价值决定于商品可以买到的劳动量,这两者是根本不同的。\endnote{大·李嘉图《政治经济学和赋税原理》1821年伦敦第3版第1—12页。——第184页。}这样,他一方面把商品包含的劳动量同商品支配的劳动量区别开来;另一方面,他从商品的相对价值中排除了商品同劳动的交换。因为一种商品同另一种商品相交换,是等量劳动相交换。商品同劳动本身相交换,则是不等量劳动相交换,而资本主义生产正是以这种交换的不平等为基础的。李嘉图没有解释这个例外如何同价值概念相符合。李嘉图以后的经济学家们的争论就是由此产生的。但是,正确的本能使他看到了这种例外(事实上这根本不是例外,只是他把它理解为例外)。由此可见,库洛赫比李嘉图走得还远,表面上比李嘉图还彻底。

在他那里毫无破绽。一切完美无缺。无论商品同商品相交换,还是商品同劳动相交换,这种交换比例都同样是商品的相对价值。如果交换的商品按它们的价值出卖(也就是说,如果需求和供给相符),这种相对价值就始终是实际价值的表现,也就是说,在交换的两极有相同的劳动量。因此,“在通常状况下”,商品所交换的也仅仅是和该商品包含的劳动量相等的雇佣劳动量。工人以工资形式得到的物化劳动,恰好等于他在交换时以直接劳动的形式还给资本家的劳动。这样,剩余价值的源泉就消失了,李嘉图的整个理论也就瓦解了。

可见,库洛赫先生一开头是在使李嘉图理论贯彻到底的外表下破坏这个理论。

下一步怎么样呢?下一步,他无耻地从李嘉图投奔到马尔萨斯那里去了,——按照马尔萨斯的学说,商品的价值决定于商品买到的劳动量,这个劳动量必须始终大于商品包含的劳动量。麦克库洛赫和马尔萨斯的区别仅仅在于,马尔萨斯把这一点按其本来面目,即把它作为李嘉图的对立面说出来,而库洛赫先生却以一种使李嘉图理论失去意义的表面的彻底性(即彻底的浅薄无知)采用李嘉图的说法,然后又采用这个对立面。因此,李嘉图学说的最内部的核心——在商品按其价值进行交换的基础上利润如何实现——库洛赫是不理解的,而且对他来说这个核心是不存在的。既然交换价值——按照库洛赫的说法,交换价值“在市场的通常状况下”等于实际价值,但是“事实上”总是大于实际价值,因为利润就建立在这个余额上(借“事实上”一词作了一个出色的对比和出色的分析)——是商品换得的“劳动或其他任何商品的量”,所以,适用于·“劳动”的,也适用于“其他任何商品”。换句话说,商品不仅同比它包含的劳动量大的直接劳动量相交换,而且同比它包含的劳动量大的其他商品中的物化劳动量相交换;这就是说,利润是“让渡利润”,这样,我们就又回到重商主义者那里去了。马尔萨斯直截了当地作出了这个结论。在库洛赫那里,这个结论则是不言自明的,不过他却把这妄称为李嘉图体系的发展。

而李嘉图体系的这种完全解体(变成一堆废话)——被自夸为李嘉图体系的彻底发展的这种解体——却被那些无知之徒,尤其是大陆上的无知之徒(其中当然包括罗雪尔先生)当作从这个体系出发而得出的走得太远、走到极端的结论,他们因而相信库洛赫先生所学到的李嘉图的“咳嗽和吐痰”\endnote{暗指席勒的《华伦斯坦》(第六场)中华伦斯坦的一个士兵的话:“他怎样咳嗽,怎样吐痰,你学得满象!但他的天才,我是说他的精神,却没有办法模仿。”——第186页。}的姿态(库洛赫用这种姿态来掩盖自己的不可救药的、浅薄无知的和无耻的折衷主义),真的就是把李嘉图体系贯彻到底的科学尝试!

麦克库洛赫纯粹是一个想利用李嘉图的经济理论来发财的人,而他确实令人吃惊地做到了这一点。萨伊也曾经这样利用斯密的理论,不同的是,他至少还有点贡献:他使斯密的理论有一定的形式上的条理化,而且,除了误解的情况之外,有时他还敢于提出一些理论上的疑问。因为库洛赫起先是靠李嘉图的经济理论在伦敦登上教授的席位,所以他最初势必以李嘉图主义者的身分出现,特别是还要参加反对土地所有者的斗争。一旦他站住了脚,并踏着李嘉图[842]的肩膀获得了一定的地位,他就主要致力于在辉格主义范围内讲述政治经济学,特别是李嘉图的政治经济学,而把其中使辉格党人讨厌的一切结论全部剔除。他的最后论货币、税收等等的著作,不过是为当时的辉格党内阁作的辩护词而已。此人由此谋得了一个肥缺。他的统计著作纯粹是骗钱的东西。在这里,对理论的浅薄无知的糟蹋和庸俗化,也暴露出此人本身就是一个“庸夫俗子”,关于这一点,下面我们在结束有关这位苏格兰投机家的问题之前,还要谈到一些。

1828年麦克库洛赫出版了斯密的《国富论》。这个版本的第四卷包括麦克库洛赫本人所写的“注释和论述”,其中一部分是为了增加篇幅把他从前发表过的、与问题毫无关系的蹩脚文章,例如关于“长子继承制”等等的文章,重新刊印出来;另一部分几乎逐字逐句重复他的政治经济学史讲义,或者象他自己所说的,“有许多是从其中借用来的”;还有一部分则竭力把穆勒以及李嘉图的反对者在这段时间里提出的新东西按照自己的方式加以同化。

麦克库洛赫先生在他的《政治经济学原理》\endnote{马克思显然是指1830年出版的麦克库洛赫《政治经济学原理》一书第二版,因为马克思通常引用的该书第一版,是在1825年,即在附有麦克库洛赫的“注释和论述”的斯密《国富论》问世前发表的。——第187页。}一书中,只是把他的“注释”和“论述”抄了一遍,而这些“注释”和“论述”又是他本人从自己过去的“零散的著作”中抄下来的。不过在《原理》中情况更加糟糕一些,因为在“注释”中,前后矛盾的地方还比在所谓的系统叙述中容易过得去。所以,上面从麦克库洛赫的《原理》引述的一些论点,有一部分虽然是从“注释和论述”中一字不改地抄来的,但是它们在这些“注释”中毕竟不象在《原理》中那样显得前后矛盾。{此外,他的《原理》还包括从穆勒那里抄来并加上极其荒谬的解释的东西,以及重新刊印的论谷物贸易等等的文章;这些文章他大概已经用二十个不同的标题在各种不同的期刊上,甚至往往在不同时间的同一刊物上一字不改地一再发表过。}

麦克在上面提到的他出版的亚·斯密著作的第四卷(1828年伦敦版)中说(他在《政治经济学原理》中逐字逐句重复了这些话,但是他在“注释”中还认为是必要的那些区别却没有了):

\begin{quote}{“必须把商品或产品的交换价值和实际价值(即费用价值)区别开来。前者,即商品或产品的交换价值,是指它们交换其他商品或劳动的能力或可能性;后者,即商品的实际价值或费用价值,是指为生产或占有商品所必需的劳动量,更确切地说,是指在所考察的时间内生产或占有同种商品所必需的劳动量。”(麦克库洛赫出版的亚·斯密的《国富论》1828年伦敦版第4卷第85—86页)“用一定量劳动生产的商品{在商品的供给和有效的需求相等的情况下},始终将交换或者说购买用同样多的劳动生产的其他任何商品。但是,它决不会交换或者说购买和生产它所用的劳动正好同样多的劳动;但是,尽管它不会这样做,它交换或者说购买的劳动量,总是象其他任何在相同条件下(即用和它本身相同的劳动量)生产出来的商品交换或者说购买的一样多。”(同上,第96—97页)“事实上〈麦克库洛赫在《原理》中一字不差地重复了这个词,因为这个“事实上”事实上构成他的全部论据〉它〈商品〉换得的总是更多{即比生产该商品所用的劳动更多的劳动},而且正是这个余额构成利润。资本家不会有任何动机〈好象在进行商品交换和考察商品价值时,问题就在于买者的“动机”〉去用一定的已完成的劳动量的产品交换[843]相同的待完成的劳动量的产品。这就等于贷款{“交换”竟等于“贷款”!}而不收任何利息。(同上,第96页)}\end{quote}

让我们从末尾谈起。

如果资本家取回的劳动不比他在工资上预付的多,他就是“贷款”而没有“利润”。问题是要解释,如果商品(劳动或其他商品)都按照它们的价值进行交换,利润怎么可能产生。麦克库洛赫的解释是:如果是等价物进行交换,利润就不可能产生。起初假定资本家同工人进行“交换”。然后,为了解释利润,又假定他们“不是”进行交换,而是其中一方贷出(即付出商品),另一方借入,即在取得商品之后之才付出。或者,为了解释利润,说资本家如果没有利润,他就没有“任何利息”。在这里,问题本身的提法就是错误的。资本家用来支付工资的商品,与他作为劳动成果取回的商品,是不同的使用价值。因此,他取回的并不是他预付的东西,正象他用一种商品交换另一种商品时取回的不是原来那种商品一样。他是购买另一种商品还是购买为他生产这另一种商品的特殊[商品——]劳动,这都是一回事。正象在一切商品交换的情况下一样,他付出了一种使用价值,而换取了另一种使用价值。相反,如果考虑的只是商品的价值,那末,用“一定的已完成的劳动量”去交换“相同的待完成的劳动量”(尽管资本家实际上只是在劳动已经完成之后才支付的),就没有任何矛盾,正象用一定的已完成的劳动量去交换相同的已完成的劳动量是不矛盾的一样。后一种情况是毫无意义的同义反复。前一种情况的前提是:“待完成的劳动”物化于和已完成的劳动不同的使用价值之中。所以在这种场合,[交换对象之间]存在差别,因而也就存在由这种关系本身产生的交换动机;而在另一种场合就不存在这种动机,因为,在这种交换中问题[仅仅]在于劳动量,A只是同A相交换。因此,麦克先生求助于动机。资本家的动机,是要取回比他付出的更多的“劳动量”。利润的产生用资本家有赚取“利润”的动机来解释。但是,在商人出卖商品的情况下,在一切不以消费而以利润为目的出卖商品的情况下,也完全可以这样说:卖者没有用一定的已完成的劳动量去交换相同的已完成的劳动量的动机。他的动机是要换得比他付出的更多的已完成的劳动。因此,他必须以货币或商品形式取得比他以商品或货币形式付出的更多的已完成的劳动。从而,他必须贵卖贱买,贱买贵卖。这样,我们看到的便是“让渡利润”,其产生的原因,并不在于它符合价值规律,而在于买者和卖者据说都没有按照价值规律来买或卖的“动机”。这就是麦克的第一个“卓越的”发现,这在力图阐明价值规律如何不顾买者和卖者的“动机”而为自己开辟道路的李嘉图体系中真是个绝妙的发现。

[844]此外,麦克在“注释”中的叙述和他在《原理》中的叙述只有以下的不同:

在《原理》中,他区别了“实际价值”和“相对价值”,并且说,“在通常情况下”两者是相等的,但是“事实上”,如果必须取得利润,两者就不能相等。可见,他不过是说:“事实”和“原则”相矛盾。

在“注释”中,他区别了三种价值:“实际价值”,商品在同其他商品交换时的“相对价值”,同劳动交换的商品的“相对价值”。商品在同其他商品交换时的“相对价值”,是商品表现在其他商品即“等价物”上的实际价值。相反,商品在同劳动交换时的相对价值,则是商品表现在另一种实际价值上的实际价值,而这另一种实际价值比商品的实际价值本身大。这就是说,商品的价值是同一个更大的价值进行交换,同非等价物进行交换。如果商品同劳动等价物进行交换,那就不会有利润了。商品在它同劳动交换时的价值,是一个更大的价值。

问题:李嘉图的价值规定同商品和劳动的交换相矛盾。

麦克的解答:在商品同劳动交换时,不存在价值规律,存在的是它的对立面。否则就无法解释利润。[然而]对于他这个李嘉图主义者来说,利润是应该用价值规律来解释的。

解答:价值规律(在这个场合)就是利润。“事实上”,麦克所说的只是李嘉图理论的反对者所说的话:如果价值规律在资本和劳动的交换中起支配作用,那就不存在任何利润了。他们说,因此李嘉图的价值规律是错误的。他说,就这个场合而言(这个场合他本来是应该根据价值规律加以解释的),这个规律是不存在的,在这个场合“价值”“意味着”某种别的东西。

由此清楚地看出,麦克库洛赫对李嘉图的规律丝毫也不理解。不然的话他就应该说:在按照本身包含的劳动时间相交换的商品进行交换时产生利润,是由于商品中包含“无聊”劳动。因此,资本和劳动的不平等交换可以说明商品按其价值相交换和在这种商品交换中实现的利润。麦克库洛赫却不是这样,他说:包含同样多的劳动时间的商品,可以支配同样多的不包含在它们之中的劳动余额。他想用这个方法把李嘉图的论点和马尔萨斯的论点调和起来,硬把商品价值决定于劳动时间和商品价值决定于支配劳动的能力等同起来。但是,包含同样多的劳动时间的商品,可以支配同样多超过它们包含的劳动的劳动余额,这意味着什么呢?这仅仅意味着,包含一定的劳动时间的商品,可以支配一定量的超过它包含的劳动的剩余劳动。不仅包含x劳动时间的商品A是如此,而且同样包含x劳动时间的商品B也是如此,——这一点已经包含在马尔萨斯的公式的表述中了。

可见,矛盾在麦克那里是这样解决的:如果李嘉图的价值规律发生作用,就不可能有利润,也就是说,不可能有资本和资本主义生产。这正是李嘉图的反对者的论断。而麦克也正是用这一点来回答他们,反驳他们。在这里,他完全没有觉察到,对于同劳动[相交换]的交换价值的解释——价值就是同某种非价值的交换——是多么妙不可言。

\tsectionnonum{[(b)通过把劳动的概念扩展到自然过程而对劳动的概念进行歪曲。把交换价值和使用价值等同起来。把利润解释为“积累劳动的工资”的辩护论观点]}

[845]麦克先生在这样抛弃了李嘉图政治经济学的基础以后,还更进一步,破坏了这个基础的基础。

李嘉图体系的第一个困难是,资本和劳动的交换如何同“价值规律”相符合。

第二个困难是,等量资本,无论它们的有机构成如何,都提供相等的利润,或者说,提供一般利润率。实际上这是一个没有被意识到的问题:价值如何转化为费用价格。

困难是从这里产生的:具有不同构成(不管这是由不变资本和可变资本的比例不同或固定资本和流动资本的比例不同引起的,还是由周转时间不同引起的)的等量资本,推动不等量的直接劳动,从而也推动不等量的无酬劳动,所以,它们在生产过程中不可能占有相等的剩余价值或相等的剩余产品。因此,既然利润无非是按总预付资本的价值计算的剩余价值,它们就不能得到相等的利润。如果剩余价值是某种别的东西而不是劳动(无酬的),那末,劳动也就根本不是商品价值的“基础和尺度”。\endnote{李嘉图在他的《原理》中好几处(例如第三版第80页)把劳动称为“商品价值的基础”。在《原理》第三版第333—334页上李嘉图把劳动说成“价值的尺度”。参看本册第148—149页,马克思从李嘉图的《原理》中引了相应的话。——第192页。}

这里产生的困难,李嘉图自己已经发现(尽管不是在其一般形式上),并且把它们当做价值规则[即规律]的例外。马尔萨斯把这些例外连同规则一起抛弃,因为例外成了规则。也同李嘉图论战的托伦斯,至少在某种程度上表述了这个问题,说等量资本虽然推动不等量的劳动,但是仍然生产出“价值”相等的商品,因此,价值不是由劳动决定的。贝利等人也是这样。至于穆勒,则承认李嘉图所确认的例外是例外,而且这些例外,除了唯一的一个形式外,没有使他发生任何怀疑。他发现只有一个造成资本家利润平均化的理由是和规则相矛盾的。这个情况就是:某些商品停留在生产过程中(例如,葡萄酒置于窖内)而没有在它们上面花费任何劳动;这是一段使它们经受某种自然过程的作用的时期。(例如,在农业和制革业中,在开始采用某些新的化学药剂以前劳动长时间中断,就是这种情况,而这一点穆勒没有提到。)然而这段时间仍被算作提供利润的时间。商品不经受劳动过程的这段时间也被算作劳动时间。(在流通时间比较长的场合,情况也总是这样。)穆勒可以说是这样“摆脱了”困境:他说,例如葡萄酒置于窖内的时间,可以算作它吸收劳动的时间,尽管根据假定,实际上并非如此。不然,[穆勒指出,]就得说“时间”创造利润了,而时间本身,据说“不过是一种音响和烟云”\endnote{歌德的《浮士德》第一部第十六场(《玛尔特的花园》)中浮士德的话。马克思在前面第89页从詹姆斯·穆勒的书中引了相应的话。——第193页。}而已。库洛赫附和穆勒的这种胡说,更确切些说,他以其惯用的、矫揉造作的剽窃手法,以一般的形式重复了这种胡说,在这种形式下,隐蔽的荒谬思想就暴露出来了,李嘉图体系的以及整个经济思想的最后残余也就被顺利地抛弃了。

上述种种困难,加以进一步考察,可以归结为下面这个困难:

以商品的形式作为材料或工具进入生产过程的那部分资本加在产品上的价值,始终不会大于它在这个生产过程开始前所具有的价值。因为,这部分资本只是由于它是物体化的劳动才具有价值,而它包含的劳动并不由于它进入生产过程而发生任何变化。它根本不取决于它所进入的生产过程,而完全取决于生产它本身所需要的社会规定的劳动,因而,在再生产它所需要的劳动时间多于或少于它包含的劳动时间时,它本身的价值才发生变动。因此,这一部分资本作为价值,原封不动地进入生产过程,又原封不动地从生产过程中出来。如果说它毕竟实际进入生产过程并且发生了变动,那末,这是它的使用价值所经受的变动,是它本身作为使用价值所经受的变动。原料所经受的或者劳动工具所完成的一切操作,只不过是它们作为一定的原料和一定的劳动工具(纱锭等等)所经历的过程,是它们的使用价值的过程,这个过程本身同它们的交换价值毫不相干。交换价值在这个[846]变动中保持不变。全部情况就是这样。

同劳动能力交换的那部分资本,则不是这样。劳动能力的使用价值,是劳动,是创造交换价值的要素。因为劳动能力在它的生产消费中所完成的劳动,比劳动能力本身的再生产所需要的劳动多,比提供工资等价物的劳动多,所以,资本家以付给工人的工资从工人那里换得的价值,大于他为这个劳动支付的价格。因此,在劳动剥削率相同的前提下,可以得出这样的结论:两个等量资本中推动较少的活劳动的那个资本,——这无论是由于它的可变部分对不变部分的比例本来就小,还是由于它的流通时间,或者说它不同劳动交换、不接触劳动、不吸收劳动的生产时间[较长],——创造较少的剩余价值,并且一般说来创造价值较小的商品。在这种条件下,创造出来的价值怎么还会相等,而剩余价值怎么还会同预付资本成比例呢?李嘉图没有能够回答这个问题,因为这样提出的问题是荒谬的,实际上,这里既没有生产出相等的价值,也没有生产出相等的剩余价值。但是,李嘉图不理解一般利润率的起源,因而也不理解价值怎样转化为和它迥然不同的费用价格。

麦克依靠穆勒的荒谬的“遁词”排除了困难。排除困难的方法是,用空洞的辞句避开了困难所由产生的具有特征的区别。这个具有特征的区别是:劳动能力的使用价值就是劳动,因此它也创造交换价值。其他商品的使用价值,则是和交换价值不同的使用价值,因此,这种使用价值所经受的任何变动,都不影响商品的预先决定的交换价值。排除困难的方法是,把商品的使用价值称为交换价值,而把这些商品作为使用价值所经历的各种操作,把它们作为使用价值在生产中提供的各种服务,称为劳动。是啊,在日常生活中也确实谈到役畜劳动和机器劳动,而在诗的语言中还有这样的说法:铁在熊熊烈火中劳动,或者在锻锤的锤击下呻吟,劳动。甚至铁在呼号呢。可以最容易不过地证明,一切“操作”都是劳动,因为劳动是一种操作。同样可以证明,一切有形体的东西都有感觉,因为一切有感觉的东西都是有形体的东西。

\begin{quote}{“有充分理由可以把劳动下定义为任何一种旨在引起某一合乎愿望的结果的作用或操作,而不管它是由人,由动物,由机器还是由自然力完成的。”(麦克库洛赫《为斯密〈国富论〉写的注释和补充论述》第4卷第75页)}\end{quote}

而这决不[仅仅]适用于劳动工具。实质上,这同样适用于原料。羊毛在吸收染料时要经受物理的作用,即物理的操作。总而言之,对任何物施加物理的、机械的、化学的等等作用以“引起某一合乎愿望的结果”,物本身都必然发生反应。这就是说,它在经受加工的同时本身必然也在劳动。于是,一切进入生产过程的商品之所以增加价值,不仅因为它们本身的价值被保存下来,而且因为它们依靠本身“劳动”——不单单是作为物化劳动——而创造了新的价值。这样一来,当然一切困难都被排除了。实质上,这不过是萨伊的“资本的生产性服务”、“土地的生产性服务”等说法的改头换面;李嘉图始终反对这种说法,而麦克,说来奇怪,就在这同一“论述”或“注释”中也反对这种说法,他在这里傲慢地捧出了他从穆勒那里抄来并加以修饰的发现。在同萨伊的论战中,麦克库洛赫对李嘉图倒还念念不忘,他还记得这种“生产性服务”实际上只是作为使用价值的物在生产过程中表现出来的属性。但是,当他把“劳动”这个神圣的名称赋予这种“生产性服务”时,一切当然就完全改变了。

[847]在麦克顺利地把商品变为工人之后,不言而喻,这些“工人”也要取得工资,而且除了它们作为“积累劳动”具有的价值外,对它们的“操作”或者说“作用”也必须付给工资。商品的这种工资,资本家受权装入自己的腰包,它是“积累劳动的工资”,换句话说就是利润\fnote{见本册第202页。——编者注}。[按照麦克库洛赫的看法]这就证明,相等的资本提供相等的利润(不管这些资本推动的劳动多少),是直接从价值决定于劳动时间得出来的。

最令人吃惊的是,如上面已经指出的,麦克就在他从穆勒的理论出发剽窃萨伊的观点的同时,又用李嘉图的话去反对同一个萨伊。从下面李嘉图反驳萨伊的一些话里,可以最清楚不过地看到麦克是怎样逐字逐句地抄袭萨伊的,所不同的只是在萨伊谈到作用的地方,他把这种作用叫作劳动:

\begin{quote}{“萨伊先生……硬说他〈亚·斯密〉犯了一个错误,说‘他把生产价值的能力仅仅归于人的劳动。更正确的分析告诉我们,价值是由人的劳动的作用,确切地说,是由人的勤劳的作用,同自然所提供的各种因素的作用以及同资本的作用结合起来产生的。斯密不懂得这一原理,所以他就不能提出有关机器在财富生产中所发生的影响的正确理论’。\endnote{这段话引自萨伊《论政治经济学》1814年巴黎第2版第1卷第51—52页。——第197页。}同亚当·斯密的看法相反,萨伊先生……谈到了自然因素赋予商品的价值”等等……“但是,这些自然因素尽管能够大大增加使用价值,却从来不会给商品增加萨伊先生所说的交换价值。”(李嘉图《政治经济学原理》第3版第334—336页)“机器和自然因素能大大增加一国的财富……但是……它们不能给这种财富的价值增加任何东西。”(同上,第335页注)}\end{quote}

李嘉图,象所有值得提到的经济学家一样,象亚·斯密一样(虽然斯密有一次出于幽默把牛称为生产劳动者)\fnote{见本卷第1册第271页。——编者注},强调指出劳动是人的、而且是社会规定的人的活动,是价值的唯一源泉。李嘉图和其他经济学家不同的地方,恰恰在于他前后一贯地把商品的价值看作仅仅是社会规定的劳动的“体现”。所有这些经济学家都多少懂得(李嘉图更懂得)应该把物的交换价值看作仅仅是人的生产活动的表现,人的生产活动的特殊的社会形式,看作一种和物及其作为物在生产消费或非生产消费中的使用完全不同的东西。在他们后来,价值实际上不过是以物表现出来的、人的生产活动即人的各种劳动的相互关系。李嘉图引用德斯杜特·德·特拉西的下面一段话来反驳萨伊,这段话,正如他明确地声明的那样,也表达了他本人的见解:

\begin{quote}{“很清楚,我们的体力和智力是我们唯一的原始的财富,因此,这些能力〈人的能力〉的运用,某种劳动〈可见,劳动是人的能力的实现〉,是我们唯一的原始的财宝;凡是我们称为财富的东西,总是由这些能力的运用创造出来的……此外,这一切东西确实只代表创造它们的劳动,如果它们有价值,或者甚至有两种不同的价值,那也只能来源于……创造它们的劳动的价值。”(李嘉图,同上第334页)}\end{quote}

由此可见,商品所以有价值,一般说,物所以有价值,仅仅由于它们是人的[848]劳动的表现——不是因为它们本身是物,而是因为它们是社会劳动的化身。

可是有人竟敢于说可悲的麦克把李嘉图的观点发展到了极端。就是这个麦克,轻率地力图把李嘉图的理论同相反的见解折衷主义地混在一起加以“利用”,把李嘉图理论的原理和整个政治经济学的原理,把作为人的活动而且是社会规定的人的活动的劳动本身,与作为使用价值、作为物的商品所具有的物理等等的作用等同起来!就是他,把劳动的概念本身都丢掉了!

麦克库洛赫凭着穆勒的“遁词”而变得厚颜无耻,他抄袭萨伊的观点,同时又用李嘉图的话来反驳萨伊,而他抄袭萨伊的那些话,恰巧就是李嘉图在第二十章《价值和财富》中作为同他本人的观点以及斯密的观点根本对立的东西坚决加以驳斥的。(罗雪尔当然要重复说,麦克是发展到了极端的李嘉图。\endnote{威·罗雪尔《国民经济体系》,第1卷《国民经济学原理》1858年斯图加特和奥格斯堡第3版第82、191页。——第198页。})不过,麦克比萨伊更荒谬,因为萨伊并没有把火、机器等的“作用”称作劳动。而且麦克更加前后矛盾。在萨伊那里,风、火等可以创造“价值”,而麦克认为只有那些可以被独占的使用价值,物,才创造“价值”。风或蒸汽或水在不占有风磨、蒸汽机、水车的情况下,好象也可以被当作动力使用!占有和独占那些为使用自然力所必须占有的物的人,好象并没有把这些自然力也独占下来!空气、水等等,我要多少就能有多少。但是它们只有在我占有了能用来使它们起生产因素作用的那些商品、那些物的时候,对我来说才是生产因素!由此可见,麦克在这方面还比不上萨伊。

所以,在这样一些把李嘉图的观点庸俗化的言论中,我们看到了对李嘉图理论的最彻底、最无知的败坏。

\begin{quote}{“但是,既然这种结果〈由任何一种东西的作用或者说操作产生的结果〉是那些不能被一定数目的个人在排斥他人的情况下独占或占有的自然力的劳动或者说作用创造出来的,那末,这种结果就没有任何价值。这些自然力所完成的东西,是它们无代价地完成的。”(麦克库洛赫《为斯密〈国富论〉写的注释和补充论述》第4卷第75页)}\end{quote}

似乎棉花、羊毛、铁或机器所完成的东西,并不是同样“无代价地”完成的。机器有价值,机器的作用则不要付报酬。任何商品的使用价值在商品的交换价值被支付后,就什么也不值了。

\begin{quote}{“卖油的人并不要求为油的自然属性付任何费用。他在估计油的生产费用时考虑的是为获得油而使用的劳动的价值,这也就是油的价值。”(凯里《政治经济学原理》1837年费拉得尔菲亚版第1卷第47页)}\end{quote}

李嘉图在反驳萨伊时恰恰强调,例如机器,它的作用同风或水的作用一样什么也不值:

\begin{quote}{“自然力和机器为我们提供的服务……由于增加了使用价值,对我们是有用的;但是,由于它们做工不需要费用……它们为我们提供的帮助就不会使交换价值有丝毫增加。”(李嘉图,同上第336—337页)}\end{quote}

可见,麦克连李嘉图的最简单的原理都不懂。但是这个狡猾的家伙这样想:如果说棉花、机器等等的使用价值什么也不值,如果说除了它们的交换价值外,它们的使用价值不要另付报酬,那末,这种使用价值却会由使用棉花、机器等等的人出卖,——他们出卖对他们来说什么也不值的东西。

[849]这个家伙的极端浅薄无知从下面这一点可以看出:他接受了萨伊的“原理”,然后利用非常详细地从李嘉图那里抄来的东西大讲其地租理论。

因为土地是“一定数目的个人在排斥他人的情况下独占或占有的自然力”所以它的自然的生长作用或者说“劳动”,即它的生产力,具有价值,从而地租就可以象重农学派那样用土地的生产力来解释。这个例子清楚地说明了麦克把李嘉图观点庸俗化的手法。一方面,他抄袭了李嘉图的只有在李嘉图提出的前提基础上才有意义的论点,另一方面,他又接受了别人的(他自己保留的只是“名词术语”或者小小的更动)直接否定这些前提的东西。他想必会说:“地租是”被土地所有者装进腰包的“土地的工资”。

\begin{quote}{“如果一个资本家在支付工人工资、饲养马匹或租用机器上花费同样金额,又如果这些工人、马匹和机器能够完成同样的工作量,那末,工作无论是由工人、马匹还是机器来完成,它的价值显然都是相同的。”(麦克库洛赫《为斯密〈国富论〉写的注释和补充论述》第4卷第77页)}\end{quote}

换句话说,产品的价值与所花费的资本的价值相适应。这是有待解决的问题。照麦克看来,问题的提出“显然”就是问题的解决。但是,既然比如说机器完成的工作量比被它排挤的工人完成的工作量大,那末更加“显然”的是:机器产品的价值和“完成同样工作”的工人的产品的价值相比不会降低,而只会提高。因为机器在一个工人制造一件产品的时间里可能制造出一万件来,而且每件都具有相同的价值,所以机器的产品比工人的产品一定会贵一万倍。

麦克竭力要表示和萨伊不同,——他认为创造价值的不是自然力的作用,而只是被独占的或由劳动产生的力的作用;不过,他还是无法在用词上克制自己,又回到了李嘉图式的用语上去。例如,他写道:

\begin{quote}{“风的劳动对船产生了合乎愿望的作用,使船发生一定的变化。但是这种变化的价值不会由于有关的自然力的作用或者说劳动而增大,它根本不取决于它们,而取决于参与生产这一结果的资本量或者说过去劳动的产品,这正象小麦的磨粉费用不取决于推动磨的风或水的作用,而取决于在这种操作中所耗费的资本量一样。”(同上,第79页)}\end{quote}

这里,磨粉之所以增加小麦的价值,忽然又只是由于资本即“过去劳动的产品”在磨粉的操作中被“耗费”。这就是说,不是因为磨盘“劳动”了,而是因为在“耗费”磨盘的时候,也“耗费”了它所包含的价值,即物化在其中的劳动。

麦克在发表了这番堂皇的议论以后,把他从穆勒和萨伊那里借来的、他用以使价值概念同一切与之矛盾的现象调和起来的深奥道理归纳如下:

\begin{quote}{“在有关价值的一切讨论中……劳动一词表示……人的直接劳动或人所生产的资本的劳动,或兼指两者。”(同上,第84页)}\end{quote}

可见,劳动[850]应理解为人的劳动,其次应理解为人的积累劳动,最后还应理解为使用价值的有益利用,即使用价值在消费(生产消费)中表现出来的物理等等的属性。而离开了这些属性也就无所谓使用价值。使用价值只有在消费中才实际表现出来。这就是说,要我们把劳动产品的交换价值理解为这些产品的使用价值,因为这种使用价值仅仅在于它在消费(不管是生产消费或非生产消费)中的实际表现,或者如麦克所说,在消费中的“劳动”。但是,使用价值的“操作”、“作用”或“劳动”的种类以及它们的自然尺度,都象这些使用价值本身一样是各不相同的。那末,什么是我们能够用来把它们加以比较的统一依据即尺度呢?[在麦克库洛赫那里]这个统一依据是由一个共同的词“劳动”来造成的,在把劳动本身归结为“操作”或“作用”这些词之后,就用这个词暗中替换了使用价值的这些完全不同的表现。可见,对李嘉图观点的这种庸俗化的结果,就是把使用价值和交换价值等同起来,因此,我们必须把这种庸俗化看成是这个学派作为一个学派解体的最后的最丑恶的表现。

\begin{quote}{“资本的利润只是积累劳动的工资的别名”,(麦克库洛赫《政治经济学原理》1825年爱丁堡版第291页)}\end{quote}

也就是对商品作为使用价值在生产中提供服务而付给商品的工资的别名。

而且,这种“积累劳动的工资”在麦克库洛赫先生那里还有一种独特的奥妙的含义。我们已经提到过,除了他从李嘉图、穆勒、马尔萨斯和萨伊那里抄来的、构成他的著作的基本内容的那些东西以外,他自己还不断把他的“积累劳动”以不同的标题一再翻印出售,经常从他以前已经得过报酬的著作中“大量抄录”。对于这种赚取“积累劳动的工资”的手法,早在1826年就有一本专门著作进行过详细的分析,而从1826年到1862年,麦克库洛赫在赚取积累劳动的工资这方面又进一步取得了多么大的成就啊!\endnote{关于麦克库洛赫这一节,以及《李嘉图学派的解体》全章(除了约·斯·穆勒一节写于1862年春以外),是马克思于1862年10月写的(马克思自己在包括该章的第XIV本的封面上注明了这一点)。——第202页。}(作为修昔的底斯的罗雪尔也使用过“积累劳动的工资”这个可悲的词句。\endnote{威·罗雪尔《国民经济体系》,第1卷《国民经济学原理》1858年斯图加特和奥格斯堡增订第3版第353页。马克思用古希腊大历史学家修昔的底斯的名字来称呼罗雪尔,这是因为,如马克思在后面(见本册第558页)所说,“罗雪尔教授先生谦虚地宣称自己是政治经济学的修昔的底斯”。“修昔的底斯·罗雪尔”这个称呼具有辛辣的讽刺性:马克思在许多地方指出,罗雪尔既严重歪曲了经济关系的历史,又严重歪曲了经济理论的历史。参看本卷第2册第130—132页。——第202页。})

上面提到的著作叫作:莫迪凯·马利昂《对麦克库洛赫先生的〈政治经济学原理〉的若干说明》1826年爱丁堡版\endnote{这本小册子的真实作者是英国政论家约翰·威尔逊,他曾以不同的笔名发表著作。——第202页。}。这本著作说明我们这位骗子手是怎样成名的。他9/10是从亚·斯密、李嘉图和其他作者那里抄来的,其余1/10则是不断地从他自己的积累劳动中抄来的,“他最无耻最恶劣地一再重复这种积累劳动”。[第4页]例如,马利昂指出,麦克库洛赫不仅把同一些文章当作自己的“论述”,当作新的著作,轮流卖给《爱丁堡评论》\endnote{《爱丁堡评论,或批评杂志》是1802年至1929年发行的英国资产阶级的文学、政治杂志。在十九世纪二十年代和三十年代每三个月发行一期,是辉格党的机关报。这一时期发表的有关经济问题的文章大多数是麦克库洛赫写的。——第202页。}、《苏格兰人报》\endnote{《苏格兰人报,或爱丁堡政治文学报》是1817年开始发行的英国资产阶级报纸。十九世纪上半叶是辉格党的机关报。这个报纸从创刊到1827年发表了麦克库洛赫论述经济问题的文章。1818年至1820年麦克库洛赫是该报的编辑。——第202页。}、《英国百科全书》\endnote{《英国百科全书》是一部多卷的英国(现在是英美)百科词典。从1768年起不断以新版刊行。十九世纪末之前一直在爱丁堡出版。——第202页。},而且他比如说还在不同年份的《爱丁堡评论》杂志上把同一些文章一字不差地重新发表,只是多少颠倒一下次序,换上新的招牌。在这方面,马利昂是这样评论“这个最不可相信的修鞋匠”[第31页]、“这位所有经济学家中最经济的经济学家”[第66页]的:

\begin{quote}{“麦克库洛赫先生的文章不管和天体多么不一样,但是有一点却和星辰相似,就是它们总是定期再现。”(第21页)}\end{quote}

麦克库洛赫信仰“积累劳动的工资”,这是毫不奇怪的!

麦克先生获得的名声,说明这种骗子手的卑鄙行为可以有多么大的神通。

[850a]只要顺便看一下1824年3月《爱丁堡评论》(那篇拙劣文章的名称是《论资本积累》),就知道麦克库洛赫怎样利用李嘉图的某些论点来抬高自己。在那篇文章里,这位“积累劳动的工资”之友对利润率的下降发出了真正的哀鸣。

\begin{quote}{“作者……这样表达了他对利润下降的忧虑:‘英国所表现的繁荣外貌是虚假的;贫困的瘟疫悄悄地侵害着市民大众,国家富强的基础已被动摇……在象英国这样利息率低的地方,利润率也是低的,国家的繁荣已经越过了它的顶点。’这种论断不能不使每一个熟悉英国美好状况的人感到吃惊。”(普雷沃《评李嘉图体系》第197页)}\end{quote}

麦克先生不必对“土地”比“铁、砖等”得到优厚的“工资”感到不安。原因想必是土地“劳动”得更勤快。[XIV—850a]

\centerbox{※     ※     ※}

[XV—925]{瞎眼睛的猪有时也能找到橡实。麦克库洛赫有一次就是这样。但是即使如此,照他那样表达,这也不过是一些前后矛盾的说法,因为他没有把剩余价值和利润区别开来;其次,这是他的又一轻率的折衷主义的剽窃。照托伦斯之流看来(他们认为价值是由资本决定的),同样照贝利看来,利润应该从它对(预付)资本的比例加以考察。和李嘉图不同,他们不是把利润和剩余价值等同起来,但是这只是因为他们根本不感到需要在价值的基础上解释利润,因为他们把剩余价值借以表现出来的形式——作为剩余价值对预付资本的比例的利润——看作原始形式,实际上不过是把表现出来的形式用文字表达出来。

下面麦克著作中的两段话,说明(1)他是李嘉图主义者;(2)他直接抄袭李嘉图的反对者:

\begin{quote}{“利润只能由于工资降低而提高、只能由于工资提高而降低这一李嘉图的规律,只有在劳动生产率不变的情况下才是正确的”(麦克库洛赫《政治经济学原理》1825年爱丁堡版第373页)。这里是指提供不变资本的生产部门的劳动生产率。“利润取决于它对生产它的资本的比例,而不取决于它对工资的比例。如果所有生产部门的劳动生产率普遍提高了一倍,由此得到的额外产品在资本家和工人之间分配,那末,虽然按预付资本计算利润率提高了,资本家和工人之间的比例仍旧不变。”(同上,第373—374页)}\end{quote}

即使在这种情况下,正象麦克也指出的,可以说工资同产品相比也相对地降低了,因为利润提高了。(然而在这种情况下利润的提高正是工资降低的原因。)但是这种计算是以工资作为产品的一部分这种错误算法为依据的,我们在上面已经看到,约翰·斯图亚特·穆勒先生就企图用这种诡辩的办法把李嘉图的规律普遍化。\endnote{马克思指的是1861—1863年手稿第VII本和第VIII本(手稿第319—345页)中篇幅很长的关于约·斯·穆勒的插入部分。按照马克思在稿本封面上所编的《剩余价值理论》目录以及他在手稿第VII本正文中所作的指示,把关于约·斯·穆勒这一节移至本册(第208—258页)。关于“以工资作为产品的一部分这种错误算法”,马克思在后面第244—248页谈到过。——第204页。}}[XV—925]

\tchapternonum{(5)威克菲尔德[在“劳动价值”和地租问题上对李嘉图理论的局部反驳]}

[XIV—850a]威克菲尔德在理解资本上的真正功绩,已在前面《剩余价值转化为资本》这一节中阐明了\endnote{马克思在1862年初着手写作《剩余价值理论》时,打算把它作为关于资本生产过程的研究的第五节即最后一节,直接放在绝对剩余价值和相对剩余价值的结合这一节之后。但是马克思在写作《理论》的过程中认为有必要在第四节(《绝对剩余价值和相对剩余价值的结合》)和《剩余价值理论》之间再插进两节《剩余价值再转化为资本》和《生产过程的结果》(见本卷第1册第446页)。这一点也说明马克思为什么提到1862年10月还没有写的应对威克菲尔德的一些观点加以阐述的一节《剩余价值转化为资本》。马克思在《资本论》第一卷第二十二章即标题为《剩余价值转化为资本》的这一章的注22中,引了威克菲尔德的论点:“在资本使用劳动以前,劳动就已经创造了资本”(见《马克思恩格斯全集》中文版第23卷第639页注22)。——第205页。}。这里只涉及和“本题”直接有关的地方。

\begin{quote}{“如果把劳动看成一种商品,而把资本,劳动的产品,看成另一种商品,并且假定这两种商品的价值是由相同的劳动量来决定的,那末,在任何情况下,一定量的劳动就都会和同量劳动所生产的资本量相交换;过去的劳动就总会和同量的现在的劳动相交换。但是,劳动的价值同其他商品相比,至少在工资取决于[产品在资本家和工人之间的]分配的情况下,不是由同量劳动决定,而是由供给和需求的关系决定。”(威克菲尔德在他出版的亚·斯密《国富论》1835年伦敦版第1卷第230页上所加的注)}\end{quote}

因此,照威克菲尔德看来,如果劳动的价值被支付了,利润就无法解释。

威克菲尔德在上述他出版的斯密著作第二卷中指出:

\begin{quote}{“剩余产品\endnote{威克菲尔德所谓的剩余产品,是指产品中“补偿资本和普通利润”后余下的部分(威克菲尔德为亚·斯密《国富论》第2卷所加的注释,第215、217页)。——第205页。}总是形成地租。但不是由剩余产品构成的地租也还是可能被支付的。”(第216页)“如果象在爱尔兰那样,大多数人弄得只能吃马铃薯,住小屋,穿破衣,并且为了求得过上这种生活,必须把他们除了小屋、破衣和马铃薯之外所能生产的一切都交出来,那末,即使资本或劳动的产品照旧,他们赖以生活的土地的所有者得到的东西,也会随着他们聊以为生的东西的减少,而增加起来。贫困的佃户所交出的,都被土地所有者占有。所以,土地耕种者生活水平的下降是剩余产品的另一原因……当工资降低时,它对剩余产品的影响,和生活水平下降所起的影响是一样的:总产品不变,剩余部分增大了;生产者得到的更少,土地所有者得到的更多。”(第220—221页)}\end{quote}

在这种情况下利润称为地租,就同例如在印度,劳动者用资本家的贷款从事劳动(纵然劳动者在名义上是独立的),把全部剩余产品交给资本家时利润称为利息完全一样。

\tchapternonum{(6)斯特林[用供求关系对资本家的利润作庸俗解释]}

\begin{quote}{“每种商品的量必须这样调节,也就是使该商品的供给与商品的需求之比小于劳动的供给与劳动的需求之比。商品的价格或价值,同耗费在商品上的劳动的价格或价值之间的差额形成利润或余额,这种利润或余额是李嘉图根据他的理论所不能解释的。”(帕特里克·詹姆斯·斯特林《贸易的哲学》1846年爱丁堡版第72—73页)}\end{quote}

[851]同一个作者对我们说:

\begin{quote}{“如果商品的价值按照它们的生产费用互成比例,这就可以叫作价值水准。”(同上,第18页)}\end{quote}

因此,如果劳动的需求和供给相符,劳动就会按照它的价值出卖(不管斯特林如何理解这个价值)。如果劳动耗费在其上的商品的需求和供给相符,商品就会按照它的生产费用出卖,这个生产费用就是斯特林所谓的劳动的价值。于是,商品的价格等于耗费在商品上的劳动的价值。而劳动的价格又和劳动本身的价值处于同一水准。从而,商品的价格等于耗费在商品上的劳动的价格。因此在这种场合就不会有利润或余额。

于是,斯特林这样来解释利润或余额:

劳动的供给与劳动的需求之比必须大于劳动耗费在其上的商品的供给与这个商品的需求之比。必须设法使商品的卖价高于商品中包含的劳动的被支付的价格。

斯特林先生把这叫作对余额现象的解释,其实这不过是对必须解释的现象的另一种表达。进一步考察,只有三种情况是可能的。(1)劳动的价格合乎“价值水准”,就是说,劳动的需求和劳动的供给相符,劳动的价格等于劳动的价值。这时,商品必须高于它的价值出卖,就是说必须设法使商品的供给低于商品的需求。这是纯粹的“让渡利润”,不过加上了它得以实现的条件。(2)劳动的需求超过劳动的供给,劳动的价格高于劳动的价值。这时,资本家付给工人的多于这些工人所生产的商品的价值,买者必须付给资本家双重余额:第一,资本家起初付给工人的余额;第二,资本家的利润。(3)劳动的价格低于劳动的价值,劳动的供给超过劳动的需求。这时,余额的产生是由于劳动低于它的价值被支付,而[以商品的形式]按照它的价值或至少高于它的价格被出卖。

如果从斯特林的议论中把无稽之谈去掉,那末,在斯特林那里,余额的产生是由于劳动低于它的价值被资本家购买,而以商品形式高于它的价格再被出卖。

如果前面两种情况把所谓生产者必须“设法”使他的商品高于它的价值或高于“价值水准”出卖这一可笑的形式去掉,那无非是:如果一种商品的需求超过商品的供给,市场价格就提高到价值以上。这当然不是什么新的发现,它所解释的这种“余额”,无论对李嘉图还是对其他任何人从来没有造成丝毫困难。[XIV—851]

\tchapternonum{(7)约翰·斯图亚特·穆勒[直接从价值理论中得出李嘉图关于利润率和工资量成反比的原理的徒劳尝试]}

\tsectionnonum{[(a)把剩余价值率同利润率混淆起来。“让渡利润”见解的因素。关于资本家的“预付利润”的混乱见解]}

[VII—319]前面引证过的那本小册子\endnote{指约翰·斯图亚特·穆勒的著作《略论政治经济学的某些有待解决的问题》1844年伦敦版。马克思在《关于生产劳动和非生产劳动的理论》这一章中引用了该书(见本卷第1册第176页)。——第208页。},实际上包括了约翰·斯图亚特·穆勒先生关于政治经济学问题的全部创见(这与他的大部头的概论\endnote{马克思指约翰·斯图亚特·穆勒的著作《政治经济学原理及其对社会哲学的某些应用》,两卷集,1848年伦敦版。——第208页。}不同)。在这本小册子的第四篇题为《论利润和利息》的“论文”中写道:

\begin{quote}{“工具和原料象其他物一样,最初除劳动外并不花费别的任何东西……制造工具和原料所耗费的劳动,加上以后依靠工具加工原料所耗费的劳动,就是生产成品所耗费的劳动总量……因此,补偿资本无非是补偿所耗费的劳动的工资。”(约·斯·穆勒《略论政治经济学的某些有待解决的问题》1844年伦敦版第94页)}\end{quote}

这一点本身就是错误的,因为所耗费的劳动和所支付的工资决不是等同的。确切地说,所耗费的劳动等于工资和利润之和。补偿资本意味着既补偿有酬劳动(工资),也补偿资本家没有付酬但被他出卖的劳动(利润)。在这里,穆勒先生把“所耗费的劳动”和其中由使用该劳动的资本家付酬的那一部分混淆起来了。这种混淆本身,对于他理解他自称在传授的李嘉图理论,并不那么有利。

关于不变资本,还要顺便指出:尽管不变资本的每个部分都可以归结为过去劳动,因而可以设想它在某个时候曾经代表利润或工资,或者代表两者,但是,这个不变资本一旦形成,它的一部分(例如种子等等)就既不可能再归结为利润,也不可能再归结为工资。

穆勒没有把剩余价值同利润区别开来。因此他宣称,利润率(这对于已经转化为利润的剩余价值来说是正确的)等于产品的价格对花费在产品上的生产资料(包括劳动在内)的价格之比。(同上,第92—93页)同时,他又想直接从李嘉图关于“利润取决于工资,工资下降则利润提高,工资提高则利润下降”[同上,第94页]的规律得出利润率的规律,而李嘉图在他的这个原理中把剩余价值同利润混淆起来了。

穆勒先生本人甚至对于他试图解决的问题也不十分清楚。因此,我们在听取他的解答之前,先把他的问题简要地表述一下。利润率是剩余价值对预付资本总额(不变资本加可变资本)之比,而剩余价值本身则是工人所完成的劳动量超过以工资形式预付给他的劳动量的余额;就是说,剩余价值只是就它对可变资本或者说对花费在工资上的资本的关系,而不是就它对全部资本的关系来考察的。因此,剩余价值率和利润率是两种不同的比率,虽然利润本身不过是从一定角度来考察的剩余价值。就剩余价值率来讲,说它完全“取决于工资,工资下降则提高,工资提高则下降”,那是正确的。(就剩余价值量来讲,这样说就不对了,因为它在同一时间内不仅取决于单个工人的剩余劳动被占有的比率,而且取决于同时被剥削的工人人数。)既然利润率是剩余价值对预付资本总价值之比,它当然要受到剩余价值的下降或提高,也就是受到工资的提高或下降的影响,并由这种情况决定;但是,除了由这种情况决定之外,利润率还包括[320]不取决于工资的提高或下降并且不能直接归结为这种情况的其他因素。

约翰·斯图亚特·穆勒先生一方面同李嘉图一起把利润和剩余价值直接等同起来,另一方面(在同反李嘉图派的论战中)又不是在李嘉图的意义上,而是在利润率的真正意义上,把利润率理解为剩余价值对预付资本(可变资本加不变资本)总价值之比,因而煞费苦心地力图证明,利润率直接由决定剩余价值的规律决定,这个规律简单地归结为:工人在自己的工作日中占有的那部分越小,归资本家所有的那部分就越大,反之亦然。现在我们就来看一看他这种煞费苦心的努力,而在这番努力中最糟糕的是,他自己也不清楚他要解决的究竟是什么问题。如果他把问题本身正确地表述出来,那他就不会错误地以这种方法去解决了。

所以,他说:

\begin{quote}{“尽管工具、原料和建筑物本身都是劳动的产物,然而,它们的价值总量毕竟不能归结为生产它们的工人的工资。{他在上面说过,补偿资本就是补偿工资。}资本家因付出工资而取得的利润必须计算在内。生产成品的资本家,不仅应该用成品补偿他自己和工具生产者付出的工资,而且应该补偿他从自己的资本中预付给工具生产者的利润。”(同上,第98页)因此,“利润不单单代表[成品生产者]在补偿费用之后的余额;它还加入费用本身。[成品生产者的]资本一部分用于支付或补偿工资,一部分用于支付其他资本家的利润,这些资本家的协力是取得生产资料所必需的”。(第98—99页)“因此,一种物品可能是和以前同量的劳动的产品,而如果最后的生产者应付给先前那些生产者的利润的某一部分能够节约下来,物品的生产费用还是会减少的……然而,利润率的变动和工资的生产费用成反比这一点,仍然是正确的。”(第102—103页)}\end{quote}

在这里,我们当然始终是从商品的价格等于它的价值这个前提出发的。穆勒先生本人也是在这个基础上进行研究的。

首先必须指出,在刚才引证的穆勒的论述中,利润看上去和“让渡利润”十分相似。但是我们且不去说它。硬说一种物品(如果按照它的价值出卖)可能“是和以前同量的劳动的产品”,而同时由于某种情况,“物品的生产费用”会“减少”,这是再荒谬不过的了。{这一点只有在我最先提出的那个意义上才是可能的,也就是说把物品的[实际]生产费用和[它对]资本家[来说]的生产费用区别开来,因为这种生产费用中有一部分是资本家不支付的。\endnote{马克思在1861—1863手稿第II本的《货币转化为资本》这一节(手稿第88页)中,说明了这一区别:“资本家的生产费用只是他所预付的价值总额,因而产品的价值等于预付资本的价值。另一方面,产品的实际生产费用等于产品中包含的劳动时间的总和。但是产品中包含的劳动时间的总和大于资本家预付的或者说支付过代价的劳动时间的总和,产品价值中超过由资本家支付过代价的或预付的价值的这个余额,恰好形成剩余价值”。马克思在手稿第XIV本关于托伦斯的一节(见本册第81—86页)和第XV本关于庸俗政治经济学的一节中(见本册第569—570页),又回过来谈了这个问题。——第211页。}在这种情况下,说资本家靠自己工人的无酬剩余劳动获取利润,正象他可能通过对提供不变资本给他的资本家支付不足而获取利润,就是说,通过对这个资本家商品中包含的、未由这个资本家付酬的剩余劳动(这种劳动正是因此形成他的利润)的一部分不支付而获得利润一样,——这确实是对的。这始终归结为:他低于商品的价值对商品支付。利润率(即剩余价值对预付资本总价值之比)的提高,既可以由于同量预付资本在客观上变便宜了(生产不变资本的生产部门中劳动生产率提高的结果),也可以由于它对买者来说在主观上变便宜了,即买者低于它的价值支付。在这种情况下,对买者来说,它始终是较小劳动量的结果。}

[321]在上面引证的那段话里,穆勒首先表述的是这样一个思想:生产成品的资本家的不变资本,不仅分解为工资,而且分解为利润。在这里他的思路是这样的:

如果最后的资本家所预付的不变资本仅仅归结为工资,那末利润就是他在补偿构成预付资本总额的全部工资后剩下的余额{而预付在成品生产上的全部(支付的)费用就归结为工资}。预付资本总价值就等于产品中包含的全部工资的价值。利润就是超过这个总额的余额。既然利润率等于这个余额对预付资本总价值之比,那末,它的提高或降低显然就取决于预付资本总价值,即取决于全部构成预付资本的工资的价值。{如果考察的是利润和工资的一般关系,这个论据本身事实上就是荒诞无稽的。其实,穆勒先生只要把总产品中分解为利润的部分(不管这笔利润是付给最后的资本家,还是付给参与商品生产的先前的那些资本家,都无关紧要)放在一边,把分解为工资的部分放在另一边,那末,利润额就仍然会等于超过工资额的价值的余额,而李嘉图的“反比例”就可以直接适用于利润率了。但是,说预付资本总额分解为利润和工资,是不正确的。}然而,最后的资本家预付的资本不仅分解为工资,而且分解为预付利润。因此,最后的资本家的利润不仅是超过预付工资的余额,而且是超过预付利润的余额。可见,利润率不仅由超过工资的余额决定,而且由留在最后的资本家手里的超过工资和利润的总额(根据假定,这个总额构成全部预付资本)的余额决定。因此,这个比率显然不仅会因工资的提高或降低而变动,而且会因利润的提高或降低而变动。如果我们把利润率由于工资的提高或降低而发生的变动撇开不谈,如果我们假定,——其实在实践中时常会遇到这种情况,——工资的价值(即工资的生产费用,包含在工资中的劳动时间)保持不变,那末,跟着穆勒先生走下去,就会得出一条绝妙的规律:利润率的提高或降低取决于利润的提高或降低。

\begin{quote}{“如果最后的生产者应付给先前那些生产者的利润的某一部分能够节约下来,物品的生产费用还是会减少的。”}\end{quote}

这一点在事实上是很正确的。假定先前那些生产者的利润中没有任何一部分是纯粹的附加额,或如詹姆斯·斯图亚特所说的让渡利润,那末,“利润的”某一“部分”的节约{只要这种节约不是由于后来的生产者欺骗先前的生产者,也就是说不是由于没有把先前的生产者的商品中包含的价值全部付给他}都是商品生产所需的劳动量的节约。{在这里,我们把,比如说,为资本在生产期间闲置不用的时间等等而支付的利润撇开不谈。}举例来说,为了把原料,比如煤,从矿井运到工厂去,以前需要两天,而现在只要一天就够了,这样就“节约了”一个工作日;但是,这一点既与这个工作日中分解为工资的部分有关,也与其中分解为利润的部分有关。

穆勒先生在自己弄清楚了最后的资本家的余额的比率,或一般地说利润率,不仅取决于工资和利润的直接比例,而且取决于最后的利润或每个特定的资本家的利润对预付资本总价值,即(花费在工资上的)可变资本加不变资本的总和之比以后,换句话说,[322]在弄清楚了利润率不仅仅决定于利润对花费在工资上的资本部分之比,即不仅仅决定于工资的生产费用或者说工资的价值以后,又接着说:

\begin{quote}{“然而,利润率的变动和工资的生产费用成反比这一点,仍然是正确的。”}\end{quote}

尽管这是错误的,“然而……仍然是正确的”。

穆勒在这方面所作的例证,可以看成是政治经济学家所特有的例证方法的典范,而且由于这个例证的作者还写过一本逻辑学的书\endnote{马克思指约翰·斯图亚特·穆勒的著作《推论和归纳的逻辑体系,证明的原则与科学研究方法的关系》,两卷集,1843年伦敦版。——第213页。},这就更加令人惊异了。

\begin{quote}{“例如,假定有60个农业工人,他们领取60夸特谷物作为工资,他们用去价值也是60夸特的固定资本和种子;他们的劳作的产品是180夸特。假定利润率是50%,那末,生产180夸特谷物所用的种子和工具就必然是40个工人劳动的产品;因为这40个工人的工资连同他们的雇主的利润共60夸特。因此,如果产品是180夸特,那就是总共100个工人的劳动结果。现在再假定,仍旧是100个工人的劳动,但是由于某种发明,不需要任何固定资本和种子了。以前只有支出120夸特才能取得180夸特的结果;现在只支出100夸特就可以了。180夸特谷物仍然是和以前同量的劳动,即100个工人的劳动的结果。因此,一夸特谷物仍然是一个工人劳动的10/18的产品。因为作为一个工人的报酬的一夸特谷物,实际上是和以前同量的劳动的产品;然而,它的生产费用却减少了。现在一夸特谷物是一个工人劳动的10/18的产品,再没有别的;以前生产一夸特谷物,则需要这一劳动量同补偿[先前那个资本家的]利润的费用结合在一起,即多支出1/5。如果工资的生产费用仍然和以前一样,利润就不可能提高。每个工人以前取得1夸特谷物;但是以前1夸特谷物是现在1+(1/5)夸特的生产费用的结果。因此,为了使每个工人取得和以前一样多的生产费用,每人就应该有1夸特加1/5夸特的谷物。”(同上,第99—103页)“因此,假定付给工人的是工人自己生产的产品,那就很明显,当这种产品的生产费用有了某种节约而工人照旧得到以前的生产费用时,他得到的产品就必然会同资本的生产力的提高成比例地增加。但如果是这样,资本家的支出和他所得到的产品之间的比例,就会和以前完全一样,利润也不会提高。因此,利润率的变动和工资的生产费用的变动是同时发生的,是不可分割的。由此可见,如果李嘉图所说的低工资不仅指作为较小劳动量的产品的工资,而且指用较少的费用——包括劳动和先前的资本家的利润在内——生产的工资,那末他的意见就是完全正确的。”(同上,第104页)}\end{quote}

关于这个出色的例证,我们首先要指出:在这个例证中假定,由于某种发明,谷物不用种子(原料),不用固定资本就可以生产出来;不用原料,不用劳动工具,就是说,光凭两只手,就可以用空气、水和土地制造出来。[323]在这个荒谬的假定下掩盖的不外是这样一个假定:不用不变资本,就是说,只靠新加劳动,就可以生产产品。在这种情况下,当然可以证明原来应该证明的东西,即利润和剩余价值是等同的,从而利润率也仅仅取决于剩余劳动对必要劳动之比。可是,困难正好是由于下面这一点产生的:因为剩余价值[不仅和资本的可变部分,而且]和资本的不变部分发生比例关系,——这个比例关系我们称为利润率,——所以剩余价值率和利润率彼此是不同的。因此,如果我们假定不变资本等于零,那末,由于不变资本的存在而产生的困难,就被我们用撇开这个不变资本的存在的办法排除了。换句话说,我们用假定困难不存在的办法排除了困难。这倒是一个行之有效的办法。

现在我们把问题,或者说穆勒对问题的例证,正确地表述一下。

在第一个假定中,我们看到:

\todo{}

在这个例子中假定:加到不变资本上的劳动等于120夸特。因为每一夸特是一个工作日(或一个工人的年劳动,它可以看作由365工作日构成的一个工作日)的工资,所以180夸特总产品只包含60工作日[新加劳动],其中30补偿工人的工资,30构成利润。可见,我们实际上是假定,一个工作日物化在两夸特中;因此,60个工人把他们的60工作日物化在120夸特中,其中60是他们的工资,60形成利润。换句话说,工人用半个工作日为自己工作,补偿工资,用另外半个工作日为资本家工作,从而为资本家创造剩余价值。因此,剩余价值率在这里是100%,而不是50%。但是,因为可变资本只占全部预付资本的一半,所以利润率就不是表现为60夸特比60夸特,而是60夸特比120夸特,就是说,不是100%,而只是50%。如果资本的不变部分等于零,全部预付资本就仅仅由60夸特,仅仅由预付在工资上的等于30工作日的资本构成;利润和剩余价值,从而利润率和剩余价值率也就等同了。这样,利润就是100%,而不是50%。两夸特谷物是一个工作日的产品,120夸特谷物是60工作日的产品,尽管作为一个工作日的工资只有一夸特谷物,作为60工作日的工资只有60夸特谷物。换句话说,工人只得到自己的产品的一半(50%),而资本家得到的却是自己的费用的两倍,即得到100%的利润。

那60夸特不变资本的情况又是怎样的呢?它同样是30工作日的产品。假定在这笔不变资本中,它的生产要素相互间的比例和上面假定的一样,即1/3是不变资本,2/3是新加劳动;其次,假定剩余价值率和利润率也和上面假定的一样,我们就会得出如下的计算数字:

\todo{}

在这里,利润率又是50%,剩余价值率100%。总产品是[324]30工作日的产品,但是其中10工作日(=20夸特)是过去劳动(不变资本),20工作日是20个工人新加的劳动,但是他们每人只得到自己的产品的一半作为工资。两夸特仍然是一个工人劳动的产品,尽管其中一夸特仍然是一个工人劳动的工资,一夸特是资本家的利润——资本家占有了工人劳动的一半。

生产成品的资本家作为剩余价值取得的60夸特,形成50%的利润率,因为这60夸特的剩余价值不仅按预付在工资上的60夸特,而且按预付在种子和固定资本上的60夸特,即总共按120夸特计算。

由此可见,如果穆勒把生产种子和固定资本(总价值为60夸特)的资本家的利润也按50%计算,如果他还假定这里不变资本和可变资本的比例也同生产180夸特谷物时一样,那末他就应当说,不变资本生产者的利润等于20夸特,工资等于20夸特,不变资本等于20夸特。既然工资等于一夸特,60夸特就包含30工作日,正象120夸特包含60工作日一样。

但穆勒是怎样说的呢?

\begin{quote}{“假定利润率是50%,那末,生产180夸特谷物所用的种子和工具就必然是40个工人劳动的产品;因为这40个工人的工资连同他们的雇主的利润共60夸特。”}\end{quote}

第一个资本家雇用了60个工人,付给每人每天1夸特(就是说,这个资本家在工资上花费60夸特),另外,他还在不变资本上花费60夸特;在这个资本家那里,60工作日物化在120夸特中,但是工人从其中仅仅得到60夸特作为工资。换句话说,工资只占60个工人劳动产品的一半。因此,60夸特的不变资本总共只等于30个工人劳动的产品;如果这60夸特仅仅由利润和工资构成,那末工资占30夸特,利润占30夸特;因此,工资等于15个工人的劳动,利润也是如此。但是,如果利润只占50%,那是因为,按照假定,在60夸特不变资本所包含的30天中,10天是过去劳动(不变资本),分解为工资的只有10天。总之,10天包含在不变资本中,20天是新加的工作日,然而在这20天中,工人仅仅为自己劳动10天,另外10天则为资本家劳动。但是,穆勒先生却硬说这60夸特是40个工人劳动的产品,而以前120夸特却是60个工人劳动的产品。以前一夸特包含半个工作日(虽然一夸特是整个工作日的工资),而现在则是3/4夸特等于半个工作日。其实,花费在不变资本上的1/3产品(60夸特)具有的价值,即包含的劳动时间,同其他任何一个1/3的产品完全一样。即使穆勒先生想把60夸特的不变资本完全分解为工资和利润,那也丝毫改变不了其中包含的劳动时间量。其中包含的仍旧是30工作日;只不过利润和剩余价值一致了,因为这里没有不变资本需要补偿。因此,利润是100%,而不象以前那样是50%。以前剩余价值也是100%,但是利润只是50%,这正是由于在计算利润时把不变资本算进去了。

这样,我们在这里就看到穆勒先生的双重错误手法。

就前一个180夸特来说,困难在于剩余价值同利润不一致,因为60夸特的剩余价值不仅应当按60夸特(总产品中等于工资的部分)计算,而且[325]应当按120夸特即60夸特的不变资本加60夸特的工资计算。因此,剩余价值是100%,而利润只是50%。就构成不变资本的那60夸特来说,穆勒先生是这样排除上述困难的:他假定在这里全部产品是在资本家和工人之间分配的,也就是说,在创造由总价值60夸特的种子和劳动工具构成的不变资本时,没有任何不变资本参加。对资本I来说应该加以说明的情况,对资本II来说被事先假定不存在,从而问题本身就消失了。

其次,穆勒先假定在构成I的不变资本的60夸特的价值中只包括[直接]劳动,这里不存在过去劳动,不存在不变资本,从而利润同剩余价值——也就是说,还有利润率同剩余价值率——是一致的,它们之间没有任何区别;然后又反过来假定,象在I的场合一样,它们之间是有区别的,因此利润也象在I的场合一样只占50%。如果在I的场合产品中1/3不是由不变资本构成,那末利润同剩余价值就一致了。总产品只有120夸特,等于60工作日,其中工人取得30(=60夸特),资本家取得30(=60夸特)。利润率便等于剩余价值率——100%。实际上它等于50%,因为60夸特的剩余价值不是按60夸特(工资),而是按120夸特(工资、种子和固定资本)计算的。在II的场合,穆勒假定生产是不用任何不变资本的。他还假定这里的工资也没有变动,仍是一夸特。然而他还是设想利润和剩余价值在这里是彼此不同的,即利润只占50%,虽然剩余价值达100%。实际上穆勒所假定的是:占总产品1/3的这60夸特所包含的劳动时间,比总产品其他的1/3包含的多;这60夸特是40工作日的产品,而其余120夸特仅仅是60工作日的产品。

可是,实际上这里流露出关于“让渡利润”的陈旧谬见,这种利润同产品中包含的劳动时间以及同李嘉图的价值规定,是毫无关系的。这就是说,穆勒假定:一个工人的一个工作日的工资,等于他的一个工作日的产品,或者说,他劳动了多少时间,工资中就包含多少劳动时间。如果作为工资付出的是40夸特,利润等于20夸特,那末,作为工资付出的40夸特就包含40工作日。40工作日的报酬等于40工作日的产品。如果60夸特总产品有50%即20夸特的利润,那末,由此得出结论,40夸特等于40个工人的劳动的产品,因为根据上述假定,40夸特构成工资,而且1个工人1天领取1夸特。那末其余20夸特是从哪里来的呢?这40个工人劳动了40工作日,因为他们取得了40夸特。因而,1夸特是1工作日的产品。40工作日也就只生产40夸特,不会多生产1蒲式耳。那构成利润的20夸特又是从哪里来的呢?穆勒的这个例子所依据的,是关于“让渡利润”——产品价格超出产品价值之外的纯粹名义上的余额——的陈旧谬见。但是在这里,价值不是以货币表现,而是以产品本身的相应部分表现的,因此这种“让渡利润”是绝对荒谬的,是不可能的。如果40夸特谷物是40个工人的劳动的产品,他们每人每天(或每年)取得1夸特的工资,即取得自己的全部产品作为工资,如果以货币表示,1夸特谷物等于3镑,40夸特等于120镑,那末,资本家把这40夸特卖了180镑,就取得了60镑的利润,即50%的利润(60镑=20夸特谷物)——这样设想是最容易的。但是,如果硬说资本家从他的40个工人在40工作日中生产的、由他支付了40夸特工资的40夸特谷物中卖出了60夸特,那末上述设想本身就显得妄诞无稽了。他手里只有40夸特,可是他卖出了60夸特,也就是说,比他拥有供出卖的量多卖了20夸特。

[326]总之,穆勒首先是企图利用下列美妙的假定来证明李嘉图的规律(即李嘉图把剩余价值同利润混淆起来的错误规律):

(1)假定生产不变资本的资本家本身不需要任何不变资本;这样,穆勒就用他的这个假定把不变资本带来的困难全部排除了;

(2)假定在没有不变资本的情况下,不变资本带来的剩余价值和利润之间的区别仍然继续存在,尽管这里没有任何不变资本;

(3)假定生产40夸特小麦的资本家可以卖出60夸特小麦,因为他的总产品是作为不变资本卖给另一个资本家的,而那个资本家的不变资本等于60夸特;还因为资本家II用这60夸特取得一笔50%的利润。

后面这个谬论就是“让渡利润”的观念,它所以在这里表现得如此荒唐,只是因为应该构成利润的不是以货币表现的名义价值,而是所出卖的产品本身的一部分。这样,穆勒先生想为李嘉图辩护,却离开了李嘉图的根本观点,并远远落在李嘉图、亚·斯密以及重农学派的后面。

可见,穆勒为李嘉图学说作的第一个辩护,就是他从一开始就推翻了这个学说,也就是说推翻了它的这样一个根本原理:利润只是商品价值的一部分,就是说,只是商品所包含的劳动时间中由资本家随着他的产品出卖但没有给工人付报酬的那一部分。穆勒认为,资本家对工人的全部工作日付了报酬,但是仍然取得利润。

我们再看看穆勒接着是怎样做的。

他假定由于某种发明在生产谷物时不再需要使用种子和农具了;也就是说,根据他的这一假定,对最后的资本家来说,不再需要不变资本了,就象他已为用在种子和固定资本上的先前那60夸特的生产者作的假定一样。现在,穆勒本应这样推论下去:

资本家I现在就不必在种子和固定资本上花费60夸特,因为我们已经说过,他的不变资本等于零。因而,他只须在劳动60工作日的60个工人的工资上花费60夸特。这60工作日的产品等于120夸特。工人只得到60夸特。因此,资本家得到60夸特的利润,即100%。他的利润率恰恰等于剩余价值率,也就是恰恰等于工人不是为自己而是为资本家劳动的劳动时间[对他们为自己劳动的劳动时间之比]。他们一共劳动了60天。他们生产120夸特,得到60夸特作为工资。因而,尽管他们劳动了60天,可是他们作为工资得到的是30工作日的产品。2夸特所需的劳动时间量,仍然等于1工作日。由资本家付酬的工作日,仍然等于1夸特,即等于已完成的工作日的半数。产品减少了1/3,从180夸特减少到120夸特;但是利润率提高了50,即从50%增长到100%。为什么?以前180夸特有1/3是专门补偿不变资本的支出的,因此,它既不加入利润,也不加入工资。另一方面,工人为资本家生产的那60夸特(工人为资本家劳动的那30工作日),不是按花在工资上的60夸特(不是按工人为自己劳动的那30工作日),而是按花在工资、种子和固定资本上的120夸特(60工作日)计算的。因此,尽管工人在60天中为自己劳动了30天,为资本家劳动了30天,尽管资本家在工资上的60夸特支出为他提供了120夸特产品,他的利润率也不是100%,而只是50%,这是因为利润率的计算方法不同:在一种情况下按2×60计算,在另一种情况下则按60计算。剩余价值[327]在这两种情况下是相同的,但是利润率不同。

可是穆勒是怎样对待这个问题的呢?

他不是假定资本家[在采用了那个可以不用不变资本的发明之后]在支出60夸特时得到120夸特(从60工作日中占有30工作日);根据他的假定,资本家现在雇用100个工人,他们向他提供180夸特,而且始终假定1工作日的工资等于1夸特。这样,就得出如下的计算数字:

\todo{}

这样,资本家现在得到80%的利润。利润在这里等于剩余价值。因此,剩余价值率也仅仅等于80%;它以前等于100%,即比现在高20。于是我们在这里就看到这样一种现象:利润率提高了30,而剩余价值率降低了20。

如果资本家在工资上仍然只花费60夸特,那就会得出如下的计算数字:

100夸特提供80夸特的剩余价值

10夸特提供8夸特的剩余价值

60夸特提供48夸特的剩余价值

但是以前60夸特提供了60夸特的剩余价值(可见降低了20%)。或者用另一个方式来表示,以前:

\todo{}

由此可见,剩余价值(在这两种情况下,我们都应该按100夸特计算)从100夸特降低到80夸特,即降低了20%。

\todo{}

下面我们分析一下物化在一夸特中的劳动时间,即一夸特的价值。以前是2夸特等于1工作日,或者说1夸特等于半个工作日,即等于一个工人一天劳动的9/18。而现在,180夸特是100工作日的产品;因此,1夸特是100/180工作日的产品,即10/18工作日的产品。换句话说,产品贵了1/18工作日,或者说劳动的生产率降低了,因为以前一个工人生产1夸特,只需要9/18工作日,而现在需要10/18工作日。尽管剩余价值降低了,劳动生产率因而也降低了,也就是说,尽管工资的实际价值(生产费用)提高了1/18,即11+(1/9)%,但是利润率提高了。以前,180夸特是90工作日的产品(1夸特是90/180工作日即9/18或1/2工作日的产品)。而现在,180夸特则是100工作日的产品(1夸特是100/180即10/18工作日的产品)。假定1工作日等于12小时,或60×12=720分。[328]这样,1/18工作日就等于720/18分,即40分。从这720分中,工人在第一种场合提供给资本家720分的半数,即360分。因此,60个工人一共提供给他360×60分。在第二种场合,工人从720分中提供给资本家8/18,即仅仅320分。但是,第一个资本家雇用60个工人,从而占有360×60=21600分。第二个资本家雇用100个工人,从而占有320×100=32000分。于是第二个资本家的利润就多于第一个,因为100个工人每人每天提供320分,多于60个工人每人每天提供360分。可见,所以发生这种情况,仅仅因为第二个资本家多雇用了40个工人;但是他从每个工人身上取得的却少于第一个资本家。第二个资本家获得了更多的利润,尽管这里剩余价值率和劳动生产率降低了,实际工资的生产费用(即工资中包含的劳动量)提高了。然而穆勒先生却想证明与此截然相反的事情。\endnote{按照前面第213页所引的穆勒的错误论断,使用了不变资本和60个工人的资本家I,为了生产1夸特谷物(一个工人的工资),花费6/9夸特谷物(120/180=2/3=6/9),而不用不变资本、光使用100个工人的资本家II,在1夸特谷物上仅仅花费5/9夸特(100/180=5/9),就是说,资本家II的每个工人“工资的生产费用”低1/9夸特,换句话说,资本家I的这些“工资的生产费用”比资本家II的多五分之一(20%)。——第224页。}

假定没有采用这种使生产谷物可以不用种子和固定资本的“发明”的资本家I,也(象资本家II那样)使用了100工作日(而在上面的计算中他只使用了90工作日)。这样,他就要多使用10工作日,其中3+(1/3)用于他的不变资本(种子和固定资本),3+(1/3)用于工资。在生产保持原来的发展水平的条件下,这10工作日的产品等于20夸特,但是其中6+(2/3)夸特补偿不变资本,12+(4/3)夸特是6+(2/3)工作日的产品。在这些产品里,6+(2/3)夸特构成工资,6+(2/3)夸特构成剩余价值。

于是,我们就得出如下的计算数字:

不变资本

资本家I在100工作日的总产品中取得33+(1/3)工作日的利润,即在200夸特的总产品中取得66+(2/3)夸特的利润。或者说,如果我们把他支出的资本按夸特来计算,那末他就是用133+(1/3)夸特(66+(2/3)工作日的产品)取得66+(2/3)夸特的利润。然而资本家II支出100夸特取得80夸特的利润。可见,资本家II的利润多于资本家I的利润。但是,资本家I生产200夸特所用的劳动时间,与资本家II生产180夸特所用的劳动时间一样多。资本家I的一夸特等于半个工作日,而资本家II的一夸特等于10/18(即5/9)工作日,即多包含1/18的劳动时间,因此比前者贵1/18。所以,资本家I就必定会挤垮资本家II。后者则不得不放弃他的发明,安于象以前一样使用种子和固定资本来生产谷物。

资本家I用120夸特取得的利润是60夸特,即50%(这和用133+(1/3)夸特取得利润66+(2/3)夸特是一样的)。

资本家II用100夸特取得的利润则是80夸特,即80%。

II的利润:I的利润=80∶50=8∶5=1∶5/8。

与此相反,II的剩余价值:I的剩余价值=80∶100=8∶10=1∶10/8=1∶1+(2/8)=1∶1+(1/4)。

II的利润率比I的利润率高30。

II的剩余价值则比I的剩余价值低20。

资本家II多雇用了66+(2/3)%的工人,而资本家I则每个工作日从每个工人那里只多占了1/8(即12+(1/2)%)的劳动。

[329]总之,穆勒先生事实上证明了:资本家I——他共使用90工作日,其中1/3包含在不变资本(种子、机器等等)中,他雇用60个工人,但是只用30天的产品付报酬——用半个(即9/18)工作日生产1夸特谷物,用90工作日生产180夸特,其中60夸特补偿不变资本中包含的30工作日,60夸特补偿60工作日的工资(即30工作日的产品),60夸特补偿剩余价值(即30工作日的产品)。这个资本家I的剩余价值等于100%。他的利润则等于50%,因为60夸特剩余价值不是按60夸特,不是按花在工资上的那部分资本计算,而是按120夸特,也就是按多一倍的资本(可变资本加不变资本)计算的。

其次,穆勒证明了:资本家II——他使用100工作日,但是他(由于自己的发明)丝毫没有把工作日花费在不变资本上——生产180夸特的产品,因此1夸特等于10/18工作日,即比资本家I的夸特贵1/18工作日(40分)。他的工人的劳动生产率低1/18。既然工人仍旧一工作日取得1夸特作为报酬,那末他的工资就实际价值来说,即就生产它所需的劳动时间来说,提高了1/18。尽管现在工资的生产费用提高了1/18,尽管资本家II的总产品同所耗费的劳动时间相比减少了,而且他所生产的剩余价值只占80%,而资本家I的剩余价值是100%,但是资本家II的利润率还是80%,而资本家I的利润率只是50%。为什么呢?因为在资本家II那里,尽管工资的生产费用提高了,可是他雇用工人多了;还因为资本家II的剩余价值率同利润率相等,其所以如此,则是由于他所生产的剩余价值只按花在工资上的资本计算,而不变资本等于零。但是,穆勒恰恰与此相反,想要证明,利润率的提高,根据李嘉图的规律,是工资的生产费用减少的结果。我们已经看到,尽管工资的生产费用已经增加,利润率还是提高了;因此,如果把利润和剩余价值直接等同起来,而又把利润率理解为剩余价值或总利润(它等于剩余价值)对全部预付资本总价值之比,那末李嘉图规律就是错误的。

穆勒先生接着说:

\begin{quote}{“以前只有支出120夸特才能取得180夸特的结果;现在只支出100夸特就可以了。”}\end{quote}

穆勒先生忘了,在第一种场合,支出120夸特等于耗费60工作日,而在第二种场合,支出100夸特等于耗费55+(5/9)工作日(也就是说,在第一种场合,1夸特等于9/18工作日,而在第二种场合,等于10/18工作日)。

\begin{quote}{“180夸特谷物仍然是和以前同量的劳动,即100个工人的劳动的结果。”}\end{quote}

对不起!这180夸特,以前是90工作日的结果,而现在是100工作日的结果。

\begin{quote}{“一夸特谷物仍然是一个工人劳动的10/18的产品。”}\end{quote}

对不起!以前这是一个工人劳动的9/18的产品。

\begin{quote}{“因为作为一个工人的报酬的一夸特谷物,实际上是和以前同量的劳动的产品。”}\end{quote}

对不起!第一,1夸特谷物现在“实际上”是10/18工作日的“产品”,而以前是9/18工作日的产品;因此多耗费了1/18工作日的劳动;第二,一个工人的报酬,不管它耗费了9/18工作日还是10/18工作日,任何时候都不能同他的劳动的产品混为一谈;它始终只是这种产品的一部分。

\begin{quote}{“现在一夸特谷物是一个工人劳动的10/18的产品,再没有别的〈这是正确的!〉;以前生产一夸特谷物,则需要这一劳动量同补偿[先前那个资本家的]利润的费用结合在一起,即多支出1/5。”[同上,第100—103页]}\end{quote}

且慢!第一,[330]我们再说一遍,认为一夸特以前耗费10/18工作日,那是错误的;它只花费9/18工作日。更加错误的是(如果绝对错误的东西可以有程度差别的话),这9/18工作日再加上“利润的补偿,即多支出1/5”。90工作日(不变资本和可变资本计算在一起)生产了180夸特。180夸特=90工作日;1夸特=90/180=9/18=1/2工作日。可见,在这9/18工作日上面,即在I的场合1夸特所耗费的半个工作日上,没有任何“附加额”。

但是,我们在这里发现一种十足的谬论,所有上述一派胡言都是以隐蔽的形式围绕着它提出的。穆勒愚弄了自己,首先是因为他假定:如果说120夸特是60工作日的产品,并且这项产品在资本家和60个工人之间平分,那末,构成不变资本的那60夸特便是40工作日的产品。实际上,不管资本家和生产这60夸特的工人按照什么比例分配产品,它们只能是30工作日的产品。

但是,我们且不谈这一点。为了把穆勒的谬论彻底弄清楚,我们假定归结为利润的不是60夸特不变资本的1/3(即不是20夸特),而是全部60夸特。由于这种假定对穆勒有利而不是对我们有利,由于它使问题简化,我们就更可以作这样的假定。此外,与其认为穆勒的资本家有一种“发明”,使他不用种子和固定资本就能生产出180夸特谷物,还不如认为生产60夸特不变资本的资本家有一种发明,使他能够强迫30个工人白白地、不要任何报酬地(就象徭役劳动形式那样)劳动30工作日,生产出60夸特或60夸特的价值。总之,我们假定:上述60夸特仅仅包含给资本家I生产不变资本的资本家II的利润;资本家II应该卖出30工作日的产品,而对他的30个各劳动了一个工作日的工人不付分文。能不能说这笔仅仅归结为利润的60夸特加入资本家I的工资生产费用,并同他的工人所完成的劳动时间“结合在一起”呢?

当然,资本家I和他的工人如果没有那构成他们的不变资本并仅仅归结为利润的60夸特,是不能生产出120夸特的(根本连一夸特也生产不出来)。对于他们说来,这是必要的生产条件,而且是应该付酬的生产条件。可是,他们需要这60夸特,是为了生产出180夸特。在这180夸特中,有60夸特是用来补偿上述60夸特的。对不变资本的这种补偿,丝毫也不触动他们的120夸特,即他们的60工作日的产品。如果他们不用构成他们的不变资本的那60夸特也能生产出120夸特,那末他们的产品,即60工作日的产品,会照旧是那些;可是,总产品少了,这正是由于以前的那60夸特没有再生产出来。资本家的利润率提高了,因为在使他有可能取得60夸特剩余价值的生产条件上的支出,或者说费用,不加入他的生产费用。绝对利润照旧,即60夸特。但是这60夸特只要他花60夸特的支出。而现在他却为此支出120夸特。因而,这种花在不变资本上的支出,加入资本家的生产费用,而不加入工资的生产费用。

假定由于某种“发明”,资本家III也可以不给自己的工人付酬而用15[不是30]工作日生产出60夸特,这部分地是因为他采用了更好的机器等等。这个资本家III会把资本家II排挤出市场,并把资本家I变成自己的固定买者。这样,资本家I的支出现在便[331]由60工作日减少到45工作日。可是,工人为了用60夸特[不变资本]生产180夸特的产品,仍旧要花费60工作日。而且他们还是需要用30工作日来生产自己的工资。对他们来说,1夸特等于半个工作日。然而这180夸特要资本家花费的只是45工作日,而不是60工作日。但是,认为谷物在称为种子时比称为谷物时花费较少的劳动时间,那是荒谬的,因此,我们应该假定:在第一个60夸特中,种子花费的代价和以前是一样的,但是需用的种子少了;或者,作为固定资本包含在这60夸特中的那个价值组成部分便宜了。

\centerbox{※     ※     ※}

我们首先把迄今分析穆勒的“例证”得出的结果写下来。

第一,弄清楚了以下一点:

我们曾经假定,120夸特谷物是在没有任何不变资本的条件下生产出来的,并且仍旧是60工作日的产品,而以前,包括60夸特不变资本在内的180夸特,是90工作日的产品。在这种情况下,花在工资上的60夸特资本——它等于30工作日,但是支配了60工作日——仍旧提供同样多的产品,即120夸特。这项产品的价值也保持不变,即1夸特等于半个工作日。固然,产品在以前是180夸特,而不是现在的120夸特;但是这60夸特差额只是代表不变资本中包含的劳动时间。可见,工资的生产费用和工资本身——无论是它的使用价值,还是它的交换价值——都保持不变:1夸特谷物等于半个工作日。剩余价值也不变,即60夸特比60夸特,或1/2工作日比1/2工作日。剩余价值率在两种场合都是100%。但是利润率在第一种场合只是50%,而现在是100%。这仅仅是因为60∶60=100%,而60∶120=50%。在这里,利润率的提高,并不是由于工资的生产费用变动,而只是由于假定不变资本等于零。在下述情况下,结果也大致相同:不变资本的价值、从而预付资本总价值减少,剩余价值对资本之比因此提高,而这个比例也就是利润率。

作为利润率,剩余价值不仅按实际增加的、创造剩余价值的那部分资本计算,即不仅按花在工资上的那部分资本计算,而且按仅仅是重新出现在产品中的原料和机器的价值计算,并且,还要按全部机器设备的价值,即不仅按机器设备中确实进入价值形成过程、因而其磨损应该得到补偿的那部分价值,而且按仅仅进入劳动过程的那部分价值计算。

第二,

穆勒的第二个例子假定:资本I提供了180夸特的产品,这些产品等于90工作日(从而60夸特或30工作日构成不变资本;60夸特为可变资本,即60工作日的工资,从60工作日中工人得到30工作日的报酬;60夸特构成剩余价值),资本II也提供了180夸特,但是这180夸特等于100工作日,从而180夸特中100夸特为工资,80夸特为剩余价值。在这里,全部预付资本用于工资。这里的不变资本等于零;工资的实际价值提高了,尽管工人所取得的使用价值仍旧是1夸特;但是现在1夸特等于10/18工作日,而以前1夸特只等于9/18工作日。剩余价值从100%下降到80%,即降低了1/5或20%。利润率则从50%上升到80%,即提高了3/5或60%。由此可见,工资的实际生产费用在这种情况下甚至不是[象在“第一”的情况下那样]仅仅保持不变,而是增加了。劳动的生产率降低了,因而工人所完成的剩余劳动量减少了。虽然如此,利润率却提高了。为什么呢?第一,因为这里没有不变资本,利润率因而等于剩余价值率。在资本不是完全花在工资上——而在资本主义生产的条件下,资本完全花在工资上几乎是不可能的——的一切情况下,利润率始终应该低于剩余价值率,而它低于剩余价值率的程度,应该同预付资本总价值高于资本中花在工资上的那个组成部分的价值的程度相适应。第二,利润率提高了,是因为资本家II比资本家I雇用的工人多得多,比补偿这两个资本家各自使用的劳动在生产率上的差额所需的工人人数还多得多。

第三,如果从事情的一个方面来看,那末,上文中“第一”和“第二”两段所引证的事例就足以证明利润率的变动可以完全不取决于工资的生产费用的论点。因为“第一”这一段已经指出,尽管劳动的生产费用保持不变,利润率也会提高。“第二”这一段则指出,资本II与资本I相比,尽管劳动生产率降低,因而工资的生产费用提高,但是利润率还是提高了。因此,这种[VIII—332]情况本身向我们证明,如果我们反过来以资本I与资本II相比,那末,尽管剩余价值率提高,劳动生产率增长,因而工资的生产费用下降,但是利润率降低了。对I来说,[一夸特]工资的生产费用只是9/18工作日,对II来说,则是10/18工作日,尽管如此,II的利润率还是比I高60%。在所有这些场合,利润率的变动不仅不决定于工资生产费用的[方向相反的]变动,反而与它的方向一致。诚然,必须指出,不能因此认为一个运动是另一运动的原因(例如,不能认为利润率降低是由于工资的生产费用降低;也不能认为利润率提高是由于工资的生产费用提高),而只能认为其他情况抵销了这些变动的对立作用。但是无论如何,李嘉图的下述规律是错误的:利润率按照和工资变动相反的方向变动,一方因另一方降低而提高,或者反过来。这个规律只有对剩余价值率来说才是正确的。但是利润率与工资的价值不是按照相反的方向,而是按照相同的方向提高或降低,这里甚至有(固然不是永远有)某种必然的相互联系。哪里劳动的生产率较低,哪里就使用更多的手工劳动。而凡是劳动的生产率较高的地方,使用不变资本就更多。因此,在这里,引起剩余价值率提高或降低的同一些情况,必定引起利润率相反的变动,也就是说,必定在剩余价值率提高时引起利润率降低,等等。

\tsectionnonum{[(b)成品的生产和生产这个成品的不变资本的生产结合在一个资本家手里会不会影响利润率]}

我们现在且按照穆勒原来的设想讲一讲这个情况,尽管他没有正确表述它。这样同时也可以弄清他关于资本家的预付利润的种种议论的真正意思。

无论依靠什么样的“发明”和“结合”\fnote{见本册第214页:“……则需要这一劳动量同补偿利润的费用结合在一起……”——编者注},照穆勒那样举出的例子是站不住脚的,因为这个例子包含着绝对的矛盾和绝对的荒谬,它本身的各种前提是互相抵触的。

按照穆勒的意见,180夸特产品中有60夸特(种子和固定资本)等于利润20夸特和40工作日的工资40夸特。这样,如果20夸特的利润不存在,40工作日仍旧存在。根据这个假定,工人也就取得自己劳动的全部产品,这样就绝对无法理解,20夸特的利润及其价值是从哪里来的。如果假定它只不过是价格的名义附加额,也就是说,如果它不代表被资本家据为己有的劳动时间,那末它不存在,也应当盈利,完全象在60夸特中有20夸特算作没有作过工的工人的工资一样。其次:60夸特在这里只是不变资本的价值的表现。但是按照穆勒的意见,这是40工作日的产品。而在另一方面,其余120夸特又被假定为60工作日的产品。但是,这里应该把工作日理解为相等的平均劳动。因此,这个假定是荒谬的。

由此可见,首先必须假定180夸特的产品只包含90工作日,构成不变资本价值的60夸特只包含30工作日。假定等于20夸特即10工作日的利润可以不存在,也是荒谬的。因为,这等于假定用来生产不变资本的30个工人虽然不是为资本家工作,但是又甘愿只领取相当于他们劳动时间的一半的工资,而不把其余一半算在自己的商品内。一句话,他们要比价值低50%出卖自己的工作日。所以,这个假定同样是荒谬的。

但是我们假定,资本家I不是向资本家II购买自己所需的不变资本然后进行加工,而是在他自己的企业中把不变资本的生产和加工结合起来。这样,他就是自己向自己提供种子、农具等等。此外,我们也把那种可以不用种子和固定资本的“发明”撇开不谈。总之,假定这个资本家在生产他的不变资本所需的不变资本上花费20夸特(等于10工作日),在[每]10工作日的工资上花费10夸特,其中5日是工人白白工作的。

于是,就得出如下的计算数字:

[333]

\todo{}

工资的实际生产费用仍旧是那些,也就是说,劳动生产率也没有发生变化。总产品仍旧是那样多,即180夸特,并且和以前一样具有180夸特的价值。剩余价值率也没有变化:80夸特比80夸特。剩余价值的绝对额即绝对量则从60夸特增加到80夸特,即增加20夸特。预付资本从120夸特减少到100夸特。以前是以120夸特赚得60夸特,即利润率为50%。现在则是以100夸特赚得80夸特,即利润率为80%。预付资本总价值120夸特减少了20夸特,利润率则从50%提高到80%。撇开利润率不谈,利润本身现在是80夸特,而以前是60夸特;所以,它增加了20夸特,这就是说,其增加程度和剩余价值量(不是剩余价值率)是相同的。

因此,在这里,实际工资的生产费用没有任何变动,没有任何改变。这里利润率的增长是由于:

(1)尽管剩余价值率没有提高,但是剩余价值绝对量从60夸特增加到80夸特,即增加了1/3;它所以增加1/3或33+(1/3)%,是因为资本家不是象以前那样雇用60个工人,而是现在直接在他自己的企业里雇用80个工人,即多剥削1/3或33+(1/3)%的活劳动,此外,他是在剩余价值率和以前只雇用60个工人时相同的条件下雇用这80个工人的。

(2)当剩余价值绝对量(因此,还有总利润)这样提高33+(1/3)%即从60夸特增加到80夸特时,利润率则从50%增加到80%,增加了30,即提高了3/5(因为:50的1/5是10,50的3/5是30)或60%。问题在于,尽管资本中花在工资上的那个组成部分从60夸特增加到80夸特(从30工作日增加到40工作日),但是所花费的资本的价值却从120夸特减少到100夸特。资本的前一部分增加了10工作日(=20夸特)。相反,资本的不变部分却从60夸特减少到20夸特,或者说从30工作日减少到10工作日,即减少了20工作日。这样,如果我们从这20工作日中扣除所增加的用于工资的那一部分资本10工作日,结果,所花费的全部资本就减少10工作日(=20夸特)。它以前是120夸特(=60工作日),而现在总共只有夸特(=50工作日)。可见,它减少了1/6,即16+(2/3)%。

不过,利润率的这一切变动,仅仅是一种表面现象,只不过是从一本账簿转到另一本账簿上。资本家I现在取得了80夸特的利润,而不是60夸特,即多取得了20夸特的利润;但是这恰恰是不变资本的生产者以前得到而现在失去的利润,因为资本家I自己生产不变资本,而不再购买不变资本,就是说,不再[334]付给不变资本的生产者20夸特(10工作日)的剩余价值(这笔剩余价值是不变资本的生产者从他雇用的20个工人身上榨取来的),而把它装进自己的腰包。

180夸特象以前一样,有80夸特的利润,只不过以前这笔利润是在两个人中间分配的。利润率看起来高了,这是因为资本家I以前把上述60夸特仅仅看作不变资本,而且对于他来说,这60夸特也确实是不变资本,这样,他就没有把注意力集中在不变资本生产者所取得的利润上。利润率也象剩余价值或包括劳动生产率在内的任何生产条件一样,没有变动。以前[也是这样],[不变资本]生产者所花费的资本是40夸特(20工作日),而资本家I支出的[可变资本]是60夸特(30工作日),共计100夸特(50工作日)。前者的利润是20夸特,后者的利润是60夸特,共计80夸特(40工作日)。相当于90工作日(180夸特)的全部产品,给花在工资和不变资本上的100夸特提供80夸特的利润。对社会来说,这时仍象以前一样,来自利润的收入没有变化;剩余价值对工资之比,也没有变化。

差别是这样产生的:当资本家作为买者出现在商品市场上的时候,他只是一个商品所有者;他必须支付商品的全部价值,支付商品所包含的全部劳动时间,而不管资本家和工人过去或现在按照什么比例参加分配这些劳动时间的果实。可是当他作为买者出现在劳动市场上的时候,他实际上买到的劳动就多于他所支付的。由此可见,当他不再购买他所需要的原料和机器,而自己也生产这些东西的时候,他把否则就得向原料和机器的卖者支付的剩余劳动据为己有了。

对单个资本家来说,——不是对利润率来说,——是他自己取得一笔利润,还是他把这笔利润付给别人,当然不是没有差别的。(因此,在计算利润率因不变资本增加而降低的情况时,永远要采用整个社会的平均数,即社会在某一时间作为不变资本使用的价值总量,以及这个总价值对直接花在工资上的资本量之比。)但是,这个着眼点,甚至对于单个资本家实行联合化——譬如,往往同一个资本家兼营纺纱和织布,而且还自己烧制所需要的砖,等等——时也很少起(并且很少能起)决定性作用。在这里起决定性作用的,是运输时间的节约所造成的生产费用的实际节约,厂房、燃料、动力等的节约,以及对原料质量加强监督,等等。如果资本家想要自己制造所需要的机器,他就得象为了满足自己的需要或满足若干固定用户的个人需要而工作的小生产者那样,小规模地生产机器,而且他为机器花费的代价,比他向为市场生产的机器制造业者购买机器要高。如果他同时纺纱、织布和制造机器,不仅是为了自己,而且是为了供应市场,那他就需要一笔更大的资本,可是,他如果把这笔资本投入他原来的事业,也许更加合算(分工)。只有当资本家为自己建立了相当大的市场,以致能够以有利可图的规模自己生产自己的不变资本的时候,上述着眼点才具有决定性作用。为此,他本身对不变资本的需求应该是相当大的。在这种情况下,即使他的劳动的生产率低于专门从事这种不变资本的生产的人,他也把一部分否则就得向另一个资本家支付的剩余劳动据为己有。

我们看到,这和利润率没有任何关系。因此,如果象穆勒所举的例子那样,以前使用了90工作日和80个工人,那末,尽管产品中包含的40工作日(=80夸特)的剩余劳动以前由两个资本家占有,而现在由一个资本家独吞,生产费用也决不会因而有丝毫节约,20夸特的利润(10工作日)在一本账簿上消失,仅仅是为了在另一本账簿上重新出现。

因此,离开劳动时间的节约,从而离开工资的节约,这种先前利润的节约就不过是一种幻觉。\endnote{马克思就在这个第VIII本(其中结束了关于约·斯·穆勒的一节)的第368页手稿上(见本卷第1册第221页),以及在第X本中间部分关于洛贝尔图斯的一章的第461—464页手稿上(见本卷第2册第43—52页),回过来谈了成品的生产与生产这个成品的不变资本的生产结合在一个资本家手里,会不会影响利润率的问题。——第238页。}

\tsectionnonum{[(c)关于不变资本价值的变动对剩余价值、利润和工资的影响的问题]}

[335]可是第四,现在还剩下这样一种情况:不变资本的价值因劳动生产率提高而降低;在这里要研究,这种情况是否关系到以及在多大程度上关系到工资的实际生产费用,或者说劳动的价值。因此,问题在于:不变资本实际价值的变动会在多大程度上同时引起利润和工资之间的比例的变动?不变资本的价值——它的生产费用——可能保持不变,然而产品中包含的这种不变资本的量却可能有时较多,有时较少。甚至假定不变资本的价值是不变的,不变资本的总量也会随着劳动生产率和大规模生产的发展而增长。因此,所使用的不变资本的相对量在它的生产费用不变甚至增长的条件下发生的变化,即对利润率始终发生影响的变化,从一开始就被排除在这里研究的范围之外。

其次,产品既不直接、又不间接加入工人消费的一切生产部门,也被排除在对这一问题的考察的范围之外。但是,这一类生产部门中发生的实际利润率的变动(即这一类生产部门中实际生产的剩余价值对所花费的资本之比的变动),却同产品直接或间接加入工人消费的生产部门中的利润率的变动完全一样,对因利润平均化而形成的一般利润率发生影响。

此外,问题必须归结为:不变资本价值的变动怎样才会对剩余价值本身发生反作用?因为既然假定剩余价值是既定的,那末剩余劳动对必要劳动之比也就是既定的,从而工资的价值,即工资的生产费用也就是既定的。在这种情况下,不变资本的价值的任何变动,都根本不会触动工资的价值以及剩余劳动对必要劳动之比,尽管这种变动在任何情况下都会影响利润率,影响资本家的剩余价值的生产费用,而且在一定情况下(即当产品加入工人消费时)还会影响体现工资的使用价值的数量,虽然并不影响它的交换价值。

假定工资是既定的。譬如说,棉纺织厂中的工资等于10劳动小时,剩余价值等于2劳动小时。再假定子棉由于丰收而落价一半。同量棉花以前要工厂主花费100镑,现在只花费50镑。同量棉花所吸收的纺纱和织布的劳动量同以前一样。这样,资本家现在在棉花上只花费50镑,就能吸收同以前花费100镑时一样多的剩余劳动;或者,如果他还是在棉花上花费100镑,那他现在用同样的价格所换取的棉花数量,就可以使他多吸收一倍的剩余劳动。在这两种情况下,剩余价值率、即剩余价值对工资之比,依然保持不变;但是剩余价值量在后一种情况下增长了,因为,尽管剩余劳动率相同,所使用的劳动却增加了一倍。在这两种情况下,利润率都提高了,尽管这里工资的生产费用没有任何变动。利润率所以提高,是因为在利润率中,剩余价值按资本家所支付的生产费用,按他所花费的资本的总价值计算,是因为这种生产费用减少了。现在资本家花费比以前少的资本,就可以生产出象以前一样多的剩余价值。在后一种情况下,不仅利润率提高,而且利润量也增加,因为剩余价值本身由于使用较多劳动而增加了,可是原料费用这时并没有增加。而且,在这种情况下,利润率和利润量也是在劳动价值没有任何变动时提高的。

另一方面我们假定,棉花的价值由于歉收提高了一倍,因此,同样多的[336]棉花,以前值100镑,而现在值200镑。在这种情况下,利润率无论如何都会降低,而利润量,或者说利润的绝对量,在一定的条件下,也可能减少。如果资本家使用同以前一样多的工人,这些工人的劳动同以前一样多,而且条件同以前完全一样,那末,尽管剩余劳动对必要劳动之比,从而剩余价值率和剩余价值量都保持不变,资本家的利润率还是下降。利润率所以下降,是因为对资本家来说,剩余价值的生产费用增加了,也就是说,他必须在原料上多花费100镑,才能占有同以前一样多的别人的劳动时间。但是,如果资本家现在不得不把以前用在工资上的钱的一部分花在棉花上,譬如说,用150镑购买棉花,而其中50镑以前是加入工资的,那末,无论利润率或利润量都会下降,而利润量所以下降,是因为使用的劳动少了,尽管这时剩余价值率没有变动。如果由于歉收而没有足够的棉花用来吸收同以前一样多的活劳动,情形也会如此。在这两种情况下,利润量和利润率都下降,尽管劳动的价值,也就是说,剩余价值率,或资本家取得的无酬劳动量与他以工资付酬的劳动之比,都没有任何变动。

总之,在剩余价值率不变,也就是说,劳动的价值不变的条件下,不变资本价值的变动必然引起利润率的变动,而且可能伴随着利润量的变化。

另一方面,现在就工人来说,情况如下:

棉花价值降低了,因此有棉花加入的产品的价值也降低了,而工人仍象以前一样取得等于10劳动小时的工资。但是,他自己消费的那部分棉纺织品,现在可以按便宜的价格买到了,因此,他以前在棉纺织品上的支出,有一部分可以用在其他东西上。他能够得到的生活资料的数量,只能与此相应地增加,即与他在棉纺织品价格上的节约相应地增加。因为就其他方面来说,他现在用较多的棉纺织品换得的,并不比他以前用较少的棉纺织品换得的多。与棉纺织品价值的降低相应,其他商品的相对价值提高了。简言之,现在较多的棉纺织品所具有的价值,并不比以前较少的棉纺织品所具有的价值多。因而,在这种情况下,工资的价值仍会和以前一样,不过它代表较多的其他商品(使用价值)。但是利润率还是会提高,尽管剩余价值率在前提不变的情况下不可能提高。

在棉花涨价时,情况则相反。如果工人仍然一天劳动那样多时间,所取得的工资仍然等于10劳动小时,那末,他的劳动的价值就仍然是那些,可是[他的工资的]使用价值,就他自己消费棉纺织品来说,却减少了。在这种情况下,工资的使用价值减少,工资的价值保持不变,尽管利润率下降了。由此可见,虽然剩余价值和(实际)工资12的降低和提高彼此永远成反比例(工人参加分配靠劳动时间绝对延长而获得的果实这种情况例外;但是在这种情况下,工人的劳动能力会损耗得更快),利润率却可能在工资价值不变而其使用价值增长的情况下提高,在工资价值不变而其使用价值减少的情况下降低。

因此,利润率由于不变资本的价值降低而提高,这与工资的实际价值(工资中包含的劳动时间)的任何变动毫无直接关系。

这样,如果棉花的价值象上面假定的那样降低50%,那末,[穆勒著作中的]以下说法就是再错误不过的了:工资的生产费用在这种情况下降低了;或者说,既然以棉纺织品形式领取工资的工人取得和以前一样多的价值,也就是说,取得比以前更多数量的棉纺织品(因为10劳动小时仍和以前一样等于譬如说10先令,但由于子棉的价值降低了,现在我用这10先令就能够买到比以前多的棉纺织品),那末利润率就仍和以前一样。剩余价值率仍和以前一样,但[337]利润率提高了。产品的生产费用会减少,因为产品的一个组成部分,即产品中包含的原料,比以前耗费较少的劳动时间。工资的生产费用仍和以前一样,因为工人为自己劳动的时间和为资本家劳动的时间仍象以前一样多。(要知道,工资的生产费用并不取决于工人用来进行劳动的生产资料所耗费的劳动时间,而取决于他为补偿自己的工资而花费的劳动时间。在穆勒先生看来,由于工人加工的譬如说是铜而不是铁,或者是亚麻而不是棉花,他的工资的生产费用就会贵些;或者说,如果工人种的是亚麻种子而不是棉花种子,或者如果他劳动时使用昂贵的机器,而不是根本不用机器或只用简单的手工业工具,那末,他的工资的生产费用就会贵些。)利润的生产费用会减少,这是因为,为生产剩余价值而预付的资本的总量即总额会减少。剩余价值的费用,永远不会大于花在工资上的那部分资本的费用。相反,利润的费用则等于为创造这个剩余价值而预付的资本的费用总额。因而,利润的费用就不仅决定于花在工资上并创造剩余价值的资本组成部分的价值,而且决定于为推动和活劳动交换的资本组成部分所必需的那些资本组成部分的价值。穆勒先生把利润的生产费用和剩余价值的生产费用混淆起来了,也就是说,他把利润和剩余价值混淆起来了。

综上所述,可以看出原料的贵贱对于加工这种原料的工业的重要性(至于机器的相对跌价\fnote{所谓机器的相对跌价,我是指所使用的机器总量的绝对价值增加了,但是增加的程度不如这些机器的总量和工作效率增长的程度。}就不必说了),——甚至假定市场价格等于商品的价值,即商品的市场价格下降的程度同商品中包含的原料的价值下降程度完全一致,也是这样。

因此,托伦斯上校关于英国的说法是正确的:

\begin{quote}{“对于英国这样的国家来说,某个国外市场的重要性不应当根据它从这个国家得到的成品的数量,而应当根据它还给这个国家的再生产要素的数量来衡量。”(罗·托伦斯《就英国状况和消除灾难原因的手段致尊敬的从男爵、议员罗伯特·皮尔爵士的信》1849年伦敦第2版第275页)}\end{quote}

{但是托伦斯论证这一点所使用的方法是拙劣的。那是关于供求的老生常谈。在托伦斯那里,事情被归结为这样:从事例如棉花加工的英国资本的增长速度,高于从事棉花种植的例如美国资本,棉花的价格就会上涨。托伦斯说,这时,

\begin{quote}{“棉纺织品的价值和用于它的各要素的生产费用相比将会降低”。[同上,第240页]}\end{quote}

也就是说,当原料的价格由于英国的需求增长而提高的时候,因原料价格提高而变贵的棉纺织品的价格将会降低;例如我们在目前(1862年春)实际上就看到,棉纱比子棉贵不了多少,棉布比棉纱贵不了多少。然而托伦斯假定:棉花尽管贵,但是足够供英国工业消费。棉花的价格涨到它的价值之上。因此,如果说棉纺织品按照它的价值出卖,那末,这所以可能只是由于:棉花种植者从全部产品中取得多于他应得的剩余价值,实际上占有了应该属于棉纺织厂主的剩余价值的一部分。后者不能通过提高价格来为自己补偿这一部分剩余价值,因为价格提高,需求就会减少。由于需求减少,棉纺织厂主的利润甚至反而会下降到低于因棉花种植业者加价而应下降的程度。

对原料例如棉花的需求,每年不仅决定于实际的、当时存在的需求,而且决定于当年的平均需求,从而,它不仅决定于开工工厂的需求,而且决定于来年——根据现有的经验——开办的工厂所增加的需求,就是说,决定于一年内工厂的相对增加,或者说与工厂相对增加相适应的[338]追加需求。

相反,如果棉花等等的价格下跌了(例如,由于收成特别好),那末它多半会降低到棉花的价值之下——这依然是由于供求规律的作用。因此,利润率,有时如上所述还有利润量,不仅仅按照它们在降低的棉花价格等于棉花价值的情况下所应提高的程度而提高,它们还会由于以下原因而提高:成品价格的降低并未完全达到棉花种植业者出卖棉花时其价格低于价值的程度,也就是说,棉纺织厂主把应当归棉花种植业者的一部分剩余价值装进了自己的腰包。这不会减少对他的产品的需求,因为产品价格反正由于棉花价值降低而降低了。不过,它降价的程度并没有达到子棉价格降到子棉本身价值以下的程度。

此外,需求在这时还会由于以下原因而增长:工人的就业率和报酬都很高,以致他们本身也在很大程度上成为消费者,成为他们自己的产品的消费者。在原料价格的下降不是由于它的平均生产费用继续不断降低,而是由于收成特别好(天气条件)的情况下,工人的工资不会下降;相反,对工人的需求还会增加。这种需求造成的结果,不只是与它的增长成比例地发生作用。相反,如果产品价格突然上涨,一方面许多工人被解雇,另一方面工厂主设法把工资压低到它的正常水平以下,借以避免遭受损失。这样,工人的正常需求下降,而且,这就进一步加快已经发生的需求普遍下降,以及加强需求普遍下降对市场价格的影响。}

穆勒关于产品在工人和资本家之间分配的(李嘉图式的)观念,主要使他产生这样一个想法:不变资本价值的变动可以引起劳动的价值即劳动的生产费用的变动,因此,例如预付的不变资本价值降低,就会引起劳动的价值,劳动的生产费用的降低,就是说也会引起工资的降低。由于原料譬如说棉花的价值降低,棉纱的价值就降低。它的生产费用减少了。它包含的劳动时间减少了。举例来说,如果一磅棉纱是一个工人一个(12小时的)工作日的产品,而这一磅棉纱所包含的棉花的价值降低了,那末,这一磅棉纱的价值的减少数就正好等于用在棉纱上的棉花价值的减少数。例如,一磅二级品40支细纱的价格在1861年5月22日是12便士(1先令)。它在1858年5月22日是11便士(实际上是11+(6/8)便士,因为棉纱价格的降低并没有达到子棉价格降低的幅度)。但是在第一种情况下,一磅标准质量的子棉的价格为8便士(实际上是8+(1/8)便士),在第二种情况下为7便士(实际上是7+(3/8)便士)。可见,这里棉纱价值的降低数正好等于子棉价值即棉纱原料价值的降低数。因此,穆勒说,劳动仍和以前一样;如果以前劳动是12小时,那末产品仍然和以前一样是这12小时的结果;但是在第二种情况下,所加的过去劳动比在第一种情况下少1便士;劳动没有变化,可是劳动的生产费用减少了(即减少1便士)。

尽管一磅棉纱作为棉纱,作为使用价值,仍和以前一样是12小时劳动的产品,然而1磅棉纱的价值无论现在或过去都不[仅仅]是纺纱工人12小时劳动的产品。在第一种情况下,12便士中有2/3即8便士是子棉的价值,而不是纺纱工人的产品;在第二种情况下,11便士中有2/3即7便士不是纺纱工人的产品。在第一种情况下,作为12小时劳动的产品的是4便士,在第二种情况下同样是4便士。在这两种情况下,[纺纱工人的]劳动仅仅加进了棉纱价值的1/3。所以说,在第一种情况下,一磅棉纱中只有1/3磅是纺纱工人的产品(如果撇开机器不谈的话),在第二种情况下也是如此。工人和资本家应该象以前一样仅仅分享等于1/3磅棉纱的4便士。如果工人用4便士购买棉纱,那末他在第二种情况下得到的棉纱比第一种情况下多,但是现在较多的棉纱的价值,和过去较少的棉纱的价值完全相同。可是,4便士在资本家和工人中间的分配,仍和以前一样。如果工人花费在自己工资的再生产或生产上的时间是10小时,那末他的剩余劳动就等于2小时,这和以前的情况是一样的。工人仍和以前一样从4便士即1/3磅棉纱中取得5/6,而资本家从其中取得1/6。由此可见,产品即棉纱的分配没有发生任何[339]变化。虽然如此,利润率还是提高了,因为原料的价值降低了,因而剩余价值对总预付资本即对资本家的生产费用之比提高了。

如果我们为了使例子简化而抛开机器等等不谈,那末这两种情况就可以表示如下:

可见,这里的利润率提高了,尽管劳动的价值没有变动,而以棉纱表示的工资的使用价值增加了。利润率提高了(在工人自己占有的劳动时间没有发生任何变化的情况下),仅仅是由于棉花的价值、从而还有资本家的生产费用的总价值降低了。2/3便士对11+(1/3)便士的支出之比,当然小于2/3便士对10+(1/3)便士的支出之比。

\centerbox{※     ※     ※}

根据以上所说,可以看到,穆勒在结束他的例证时提出的以下论点\fnote{见本册第214页。——编者注}是错误的:

\begin{quote}{“如果工资的生产费用仍然和以前一样,利润就不可能提高。每个工人以前取得1夸特谷物;但是以前1夸特谷物是现在1+(1/5)夸特的生产费用的结果。因此,为了使每个工人取得和以前一样多的生产费用,每人就应该有1夸特加1/5夸特的谷物。”(同上,第103页)“因此,假定付给工人的是工人自己生产的产品,那就很明显,当这种产品的生产费用有了某种节约而工人照旧得到以前的生产费用时,他得到的产品就必然会同资本的生产力的提高成比例地增加。但如果是这样,资本家的支出和他所得到的产品之间的比例,就会和以前完全一样,利润也不会提高。〈这一点恰恰错了。〉因此,利润率的变动和工资的生产费用的变动是同时发生的,是不可分割的。由此可见,如果李嘉图所说的低工资不仅指作为较小劳动量的产品的工资,而且指用较少的费用——包括劳动和先前的资本家的利润在内——生产的工资,那末他的意见就是完全正确的。”(同上,第104页)}\end{quote}

由此可见,按照穆勒的例证,李嘉图的观点只有在这样的情况下才是完全正确的:所说的低工资(或一般说来,工资的生产费用)指的不仅同李嘉图所说的相反,而且简直是一种极端荒谬的东西,也就是说,工资的生产费用不是指工人用来补偿自己的工资的那一部分工作日,而且还是指工人加工的原料和使用的机器的生产费用,即工人既没有为自己也没有为资本家劳动过的劳动时间。

\centerbox{※     ※     ※}

第五,现在来谈谈本来的一个问题:不变资本价值的变动会对剩余价值发生什么影响?

如果我们说,平均日工资的价值等于10小时,换句话说,在工人劳动的整个工作日譬如说12小时中,要用10小时来生产和补偿他的工资,他在这10小时之外劳动的时间才是无酬的劳动时间,才构成资本家[340]没有付酬而取得的价值,那末,这无非是说,在工人所消费的生活资料总量中包含10小时的劳动时间。这10劳动小时表现为工人用来购买他所必需的生活资料的一定的货币额。

但是商品的价值决定于它所包含的劳动时间,而不问这种劳动时间是包含在原料中,在磨损的机器中,还是在工人利用机器新加到原料上的劳动中。因此,如果加入该商品的原料或机器的价值发生了永久性的(而不只是暂时性的)变动,——造成这种变动的原因,是生产这些原料和机器的劳动生产率,简言之,即生产商品中包含的不变资本的劳动生产率发生了变化——而且如果由于这种变动,现在要用比以前多的或比以前少的劳动时间来生产商品的各该组成部分,那末,商品本身就会因而变贵或变贱(在把原料变为产品的劳动的生产率不变,以及工作日的长度不变的条件下)。结果,劳动能力的生产费用,即劳动能力的价值,会提高或降低;从而,如果工人以前在12小时中为他自己劳动10小时,那末现在他就必须为自己劳动11小时,或者在相反的情况下只劳动9小时。在前一种情况下,他为资本家完成的工作,即剩余价值,将减少一半,从2小时减少到1小时;在后一种情况下,剩余价值将增加一半,从2小时增加到3小时。在后一种情况下,资本家的利润率和利润量都会增长:前者增长是由于不变资本的价值减少了,两者都增长是由于剩余价值率(及其绝对量)提高了。

不变资本的价值的变动,仅仅以这种方式影响劳动的价值,影响工资的生产费用,或者说影响工作日在资本家和工人之间的划分,从而也影响剩余价值。

但是这仅仅说明,对于资本家,例如纺纱的资本家来说,他自己的工人的必要劳动时间不仅决定于纺纱业中的劳动生产率,而且决定于棉花、机器等等生产部门中的劳动生产率,同样地还决定于所有这样一些生产部门的生产率:这些部门的产品虽然不作为不变资本——既不作为原料,也不作为机器等等——加入这个资本家的产品(根据假定,这种产品加入工人消费)即棉纱之内,但是构成花在工资上的流动资本的一部分;也就是说,同样地还决定于生产食品等等的劳动生产率。在某一生产部门中作为产品的东西,在另一部门会成为劳动材料或劳动资料;因此,某一生产部门的不变资本是由另一生产部门的产品构成的,它在另一生产部门不是不变资本,而是那个部门的生产的结果。对于单个资本家来说,劳动生产率的提高(从而还有劳动能力价值的下降)是发生在他自己那个生产部门,还是发生在那些为他的企业提供不变资本的部门,不是没有区别的。而对资本家阶级——对资本整体——来说则是一样的。

由此可见,这种情况{即不变资本的价值下降(或相反的变动)不是由于使用这种不变资本的生产部门扩大了生产规模,而是由于不变资本本身的生产费用变动了},完全没有超出就剩余价值所阐明的那些规律的范围。\endnote{马克思指他的1861—1863年手稿第I—V本,在这几本里马克思阐述了他的剩余价值理论(见《货币转化为资本》、《绝对剩余价值》、《相对剩余价值》各节)。——第249页。}

一般说来,当我们谈到利润和利润率的时候,总是假定剩余价值是既定的,因而影响剩余价值的一切因素都已经起过作用。这都是假定了的前提。

\centerbox{※     ※     ※}

第六,这里还可以再研究一下,不变资本对可变资本之比,因而还有利润率,如何由于剩余价值的特殊形式,即由于劳动时间延长到正常工作日以上而发生变动。[341]由于这种情况,不变资本的相对价值,或者说,不变资本在产品总价值中所占价值的比例部分减少了。不过我们把这一点留在第三章\endnote{马克思指他的研究中后来发展成为《资本论》第三卷的那一部分。——第250页。}里讲,因为这里所研究的东西,一般说来,大部分是属于那一章的。

\centerbox{※     ※     ※}

穆勒先生根据他所作的出色例证,提出了一个(李嘉图式的)一般原理:

\begin{quote}{“利润规律的唯一的表现……是利润取决于工资的生产费用。”(同上,第104—105页)}\end{quote}

应该说情况正好相反:利润率{穆勒所谈的也正是利润率}只是在唯一的一个情况下才仅仅取决于工资的生产费用,这个情况就是:剩余价值率和利润率等同。不过这只有在以下那种在资本主义生产中几乎不可能的情况下才是可能的:全部预付资本直接预付在工资上,任何不变资本——不论是原料还是机器、建筑物等等——都不加入产品;或者,原料等等虽然加入产品,但是它们本身都不是劳动的产品,都没有价值。只有在这种情况下,利润率的变动才和剩余价值率的变动等同,或者换句话说,才和工资的生产费用的变动等同。

但是一般说来(刚才谈到的例外情况也包括在内),利润率等于剩余价值对预付资本总价值之比。

如果我们用M代表剩余价值,用C代表预付资本的价值,那末利润率就是M∶C,或M/C。这个比例既取决于M的量{而决定M的量时,是把决定工资的生产费用的一切情况包括进去的},也取决于C的量。但是C,预付资本的总价值,是由不变资本c与(花在工资上的}可变资本v构成的。因此,利润率等于M/(v+c),即M/C。但是M本身,剩余价值,不仅决定于它本身的比率,即剩余劳动对必要劳动之比,或者说工作日在资本和劳动之间的划分,工作日划分为有酬劳动时间和无酬劳动时间。剩余价值量,即剩余价值的绝对量,还决定于同时被资本剥削的工作日数。而这个按一定无酬劳动率使用的劳动时间量,对一定的资本来说,则取决于产品在不再投入劳动或不再投入以前那样多劳动的情况下停留在生产过程本身(例如,在酒窖中酿制的葡萄酒,已经种在地里的谷物,在一定的时间内经受化学作用的皮革和其他材料,等等)的时间,也取决于商品流通时间的长短,取决于商品形态变化时间的长短,即从它作为产品被制成时到它作为商品投入再生产时这段时间的长短。多少工作日可以同时使用{在工资的价值从而剩余价值率既定的情况下},总的说来,取决于花在工资上的资本量。而刚刚谈到的情况总的来说可以改变一定量资本在一定时期譬如说一年内所能使用的活劳动时间的总量。这些情况也决定着一定资本能够使用的劳动时间的绝对量。但是这并没有改变以下事实:剩余价值仅仅决定于剩余价值率乘以同时使用的工作日数。上述那些情况只决定这两个因素中的后一个因素——所使用的劳动时间量。

剩余价值率等于剩余劳动在一个工作日中所占的比例,也就是等于一个工作日所生产的剩余价值。举例来说,如果一个工作日等于12小时,剩余劳动等于2小时,那末这2小时就等于12小时的1/6,或者更正确地说,我们应当拿这2小时按必要劳动(或者说按支付必要劳动的工资——同量的物化形式上的劳动时间)来计算,这样,剩余劳动所占的份额是1/5(10小时的1/5是2小时;1/5=20%)。在这里(一个工作日的)剩余价值量完全决定于剩余价值率。如果现在资本家使用100这样的[342]工作日,那末剩余价值(它的绝对量)就等于200劳动小时。剩余价值率仍和以前一样:200小时比1000小时必要劳动,即1/5或20%。如果剩余价值率是既定的,那末剩余价值量完全取决于所使用的工人人数,就是说取决于花在工资上的资本的绝对量,取决于可变资本。如果所使用的工人人数,即花在工资上的资本或可变资本的量是既定的,那末剩余价值量就完全取决于剩余价值率,即取决于剩余劳动对必要劳动之比,取决于工资的生产费用,取决于工作日在资本家和工人之间的划分。如果100个工人(他们一天劳动12小时)给我提供200劳动小时,那末剩余价值的绝对量就等于200小时,剩余价值率则是一个[有酬的]工作日的1/5,即2劳动小时。剩余价值在这里等于2小时乘以100。如果50个工人给我提供200劳动小时,那末剩余价值的绝对量就等于200劳动小时,剩余价值率则等于一个(有酬的)工作日的2/5,即4劳动小时。剩余价值在这里等于4小时乘以50。既然剩余价值的绝对量等于剩余价值率和工作日数相乘之积,那末,在两个因数按反比例变化时,它也会保持不变。

剩余价值率永远表现为剩余价值对可变资本之比。因为可变资本等于有酬劳动时间的绝对量,而剩余价值等于无酬劳动时间的绝对量。所以,剩余价值对可变资本之比,永远表示工作日的无酬部分对有酬部分之比。假定上例中的10小时的工资等于1塔勒,这里1塔勒就是包含10劳动小时的白银量。在这种情况下,100工作日的报酬是100塔勒。如果这里剩余价值等于20塔勒,那末剩余价值率就是20/100,即1/5或20%。换句话说,资本家用10劳动小时(等于1塔勒)取得2劳动小时,用10×100即1000劳动小时则取得200劳动小时,等于20塔勒。

总之,尽管剩余价值率仅仅决定于剩余劳动时间对必要劳动时间之比,换言之,决定于一个工作日中工人为生产他的工资所需的相应部分,即决定于工资的生产费用,但是剩余价值量此外还决定于工作日的数量,决定于按一定的剩余价值率使用的劳动时间的绝对量,也就是说,决定于用在工资上的资本的绝对量(如果剩余价值率是既定的话)。但是,既然利润是剩余价值的绝对量(而不是剩余价值率)对全部预付资本总价值之比,那末利润率显然就不仅仅决定于剩余价值率,而且决定于剩余价值的绝对量,而这个绝对量则取决于剩余价值率和工作日数的复比例,取决于用在工资上的资本量和工资的生产费用。

如果剩余价值率是既定的,那末剩余价值量就仅仅取决于(用在工资上的)预付资本量。平均工资到处都是相同的。换句话说,假定一切生产部门中的工人都领取譬如说10小时的工资。(在工资高于平均工资的部门中,这对于我们的研究以及对于问题本身来说,就好象资本家使用了较多的普通工人一样。)这样,假定剩余劳动到处都是相同的,就是说整个正常工作日也是相同的(不相同的程度,由于一小时的复杂劳动等于譬如两小时的简单劳动,得以部分地拉平),[343]那末,剩余价值量就仅仅取决于[花在工资上的]预付资本量。因此,可以说,剩余价值量的相互之比等于(花在工资上的)预付资本量的相互之比。但是对利润就不能这样说,因为利润是剩余价值对全部预付资本总价值之比,而在同量资本中,花在工资上的资本组成部分,或者说可变资本对全部资本之比,可能是而且常常是极不相同的。利润量在这里则取决于——不同资本中的——可变资本对全部资本之比,即取决于v/(c+v)。因此,如果剩余价值率是既定的——它始终表现为m/v,表现为剩余价值对可变资本之比,——那末利润率就仅仅决定于可变资本对全部资本之比。

总之,利润率首先决定于剩余价值率,或者说,决定于无酬劳动对有酬劳动之比;它随着剩余价值率的变动而变动——提高或降低(只要这种作用没有被其他决定因素的运动抵销)。剩余价值率的提高或降低则同劳动生产率成正比,而同工资的生产费用,或者说同必要劳动量成反比,也就是同劳动的价值成反比。

其次,利润率决定于可变资本对全部资本之比,决定于v/(c+v)。问题在于,剩余价值的绝对量在剩余价值率既定的条件下仅仅取决于可变资本量,而可变资本量,在我们假定的前提下,决定于或者说只是表示同时使用的工作日数,即所使用的劳动时间的绝对量。利润率则取决于这个由可变资本提供的剩余价值绝对量对全部资本之比,也就是取决于可变资本对全部资本之比,取决于v/(c+v)。既然在计算利润率时剩余价值M假定是既定的,因而v也假定是既定的,那末v/(c+v)的一切变动就只能由c,即不变资本量的变动产生。这是因为:如果v是既定的,那末c+v的总额,即C,只有当c变动时才能变动,而随着c的变动,v/(c+v),即v/C的比例也会变动。

如果v=100,c=400,那末v+c=500,而v/(c+v)=100/500=1/5=20%。可见,如果剩余价值率等于5/10,即1/2,剩余价值就等于50。但是,既然可变资本仅仅等于全部资本的1/5,那末利润就等于全部资本的1/5的1/2,即1/10。实际上,500的1/10,等于50。利润率就是10%。v/(v+c)这一比例随着c的每一次变动而变动,当然不是按相同数值变动。我们假定v和c最初都等于10,也就是说,总资本的半数是可变资本,半数是不变资本,那末v/(v+c)=10/(10+10)=10/20=1/2。这样,剩余率[dieMehrrate]如果等于v的1/2,那就等于C的1/4。或者说,如果剩余价值为50%,那末在可变资本为C/2的这一情况下,利润率就等于25%。现在假定不变资本增加一倍,从10增加到20,那末v/(c+v)=10/(20+10)=10/30=1/3。(以前剩余率是10的1/2,现在等于C的1/3的1/2,也就是说等于30的1/6,即等于5。而10的半数也等于5。5比10,得50%。5比30,得16+(2/3)%。而在以前,5比20,得1/4,即25%。)不变资本增加一倍,即从10增加到20;但是c+v的总额只增加一半,即从20增加到30。不变资本增加100%,c+v的总额只增加50%。v/(c+v)这一比例最初等于10/20,现在不过减少到10/30,从1/2减少到1/3,即从3/6减少到2/6,也就是只减少了1/6,而不变资本增加一倍。不变资本的增加或减少怎样影响v/(c+v)这一比例,显然取决于c和v在最初构成全部资本C(c+v)的两个部分时的比例。

[344]首先,不变资本(也就是它的价值)在所使用的原料、机器等等的数量保持不变的条件下可以增加(或减少)。因此,在这种情况下,不变资本的变动不是决定于它作为不变资本进入的那个生产过程的生产条件,而是与它无关。但是,无论不变资本价值的这种变动的原因是什么,它们总会影响利润率。在这种情况下,同量原料、机器等等的价值所以比以前多或少,是因为生产它们所需要的劳动时间比以前多或少。在这里,价值的变动是由不变资本的各组成部分作为产品从中出来的那些过程的生产条件决定的。我们在前面\fnote{见本册第238—246页。——编者注}已经考察了这种情况是如何影响利润率的。

但是,当同一个生产部门的不变资本(例如原料)的价值由于这种不变资本本身的生产变贵或变便宜而提高或降低时,这种情况对利润率的影响,同在下述情况下产生的影响完全一样:在某一生产部门(甚至是同一生产部门),在工资支出相同的条件下,一种商品比另一种商品采用了较贵的原料。

如果工资支出相同,而某一资本加工的原料(例如小麦)比另一资本加工的原料(例如燕麦)贵(或者是银和铜、羊毛和棉花,等等),这两笔资本的利润率都应该同原料价格的提高成反比。因此,如果这两个生产部门平均说来取得相同的利润,那末所以能够如此,仅仅是因为在资本家阶级内部,剩余价值在各个资本家之间的分配,不是根据每笔资本在它的特殊的生产领域内生产的剩余价值,而是根据所使用的资本的大小。这可能有以下两种情况。加工比较便宜的材料的A,按照商品的实际价值出卖他的商品,从而以货币形式取得他自己所生产的剩余价值。他的商品的价格等于商品的价值。加工比较贵的材料的B,高于商品的价值出卖他的商品,并规定这样的价格,[这个价格能使商品提供给他同样多的利润,]就象他加工比较便宜的材料一样。如果后来A同B交换他们的产品,那末,对A来说,这无异于他算在自己商品的价格里的剩余价值少于商品中实际包含的剩余价值。或者A和B两者事先按照所花费的资本量来规定利润率,也就是说,他们按照他们支出的资本量来分配总剩余价值,结果也一样。而这正是一般利润率的含义\endnote{马克思在这里第一次表述了他关于剩余价值转化为平均利润以及商品的价值变成与它不同的生产价格这一学说的基本思想。这一部分手稿写于1862年春(见正文第243页)。还可参看本卷第一册第76页,那里第一次出现“平均价格”这一术语,以表示与价值不同的生产价格。马克思在1862年6月(在关于洛贝尔图斯的一章中——见本卷第2册第19—22、25、63—70页)和1862年7—8月(在对李嘉图的经济观点体系的批判分析中——见本卷第2册第191—240页),更详尽地阐述了平均利润和生产价格的学说。——第256页。}。

十分明显,当某一资本的不变部分(例如原料)的价值由于收成好坏以及诸如此类的影响而有短时期的降低或提高的时候,是不会出现这种平均化的,虽然譬如说纺纱厂主在棉花收成特别好的年份取得的特别多的利润,毫无疑问会把大量新资本吸引进这个工业部门,促使建造大批新的工厂和棉纺织机。因此,如果继之而来的是棉花歉收的年景,[棉花价格突然提高所带来的]损失就会更大。

其次,在机器和原料——简言之,不变资本——的生产费用没有变动的条件下,可能需要更多的这种不变资本,从而它的价值会相应增加,其原因在于:不变资本的上述各组成部分作为生产资料进入的那些过程的生产条件改变了。在这种情况下,和在上述情况下一样,不变资本的价值的增加当然会引起利润率的下降;但是,从另一方面来看,生产条件的这种变化本身却证明,这里劳动具有更高的生产率,从而剩余价值率提高了。要知道,原来那样多的活劳动,现在消费比以前多的原料,只是因为它加工这些原料所需的时间少了;现在使用更多的机器,也只是因为机器的价值比它所代替的劳动的价值低。可见,利润率的降低在这里由于剩余价值率的提高,从而也由于剩余价值绝对量的提高,而多少得到补偿。

最后一点,引起不变资本价值变动的两种情况可能通过极不相同的结合共同发生作用。举例来说,[345]子棉的平均价值降低了,但是与此同时,在一定时间内加工的棉花总量的价值却更增加了。再举一个例子:一磅羊毛的价值和在一定时间内加工的羊毛总量的价值都增加了。第三个例子:更大量地使用机器,绝对地说变贵了,但是与其效率相对来说便宜了,等等。

到目前为止一直是假定可变资本保持不变。但是,可变资本本身也有可能不仅相对地——与不变资本量相比——减少,而且还绝对地减少,例如,在农业中就是这样。此外,可变资本也可能绝对地增加。但是,那时的结果仍旧同它保持不变时一样,只要不变资本由于上述原因以更大程度或以同样程度增加。

如果不变资本保持不变,那末它同可变资本相比每次增加或减少的原因,仅仅在于不变资本的这种相对增加或减少是可变资本绝对减少或增加的结果。

如果可变资本保持不变,那末不变资本每次增加或减少的原因,只在于它本身的绝对增加或减少。

如果两种资本同时发生变动,那末在除去这两者相同的变动之后,所得的结果仍旧和一种资本保持不变而另一种资本有所增加或减少一样。

但是,如果利润率是既定的,那末利润量就取决于所使用的资本量。利润率低的大量资本提供的利润,多于利润率高的小量资本。

\centerbox{※     ※     ※}

这个插入部分到此可以结束了。

此外,在约·斯·穆勒的著作中还应当注意的只有以下两个论点:

\begin{quote}{“严格说来,资本并不具有生产力。唯一的生产力是劳动力,当然,它要依靠工具并作用于原料。”(同上,第90页)}\end{quote}

严格说来,穆勒在这里把资本与构成资本的物质组成部分混为一谈了。可是,这个论点对于那些同样把两者混为一谈,但又认为资本有生产力的人来说,却是好的。当然,这里说穆勒的论断正确,也仅仅就所指的是价值的生产而论。要知道,如果指的只是使用价值,那自然界也是会生产的。

\begin{quote}{“资本的生产力不外是指资本家借助于他的资本所能支配的实际生产力的数量。”(同上,第91页)}\end{quote}

在这里,资本被正确地看作生产关系。[VIII—345]

\centerbox{※     ※     ※}

[XIV—85]在以前的一本稿本中,我曾经详细地分析了穆勒在《略论政治经济学的某些有待解决的问题》(1844年伦敦版)一书中,如何对剩余价值和利润不加区分,粗暴地试图直接从价值理论中得出李嘉图关于利润率(关于利润与工资成反比例)的规律。

\tchapternonum{[(8)结束语]}

以上关于李嘉图学派的全部叙述表明,这个学派的解体是在这样两点上:

(1)资本和劳动之间按照价值规律交换。

(2)一般利润率的形成。把剩余价值和利润等同起来。不理解价值和费用价格的关系。

\tchapternonum{[第二十一章]以李嘉图理论为依据反对政治经济学家的无产阶级反对派}

[852]在政治经济学上的李嘉图时期,同时也出现了[资产阶级政治经济学的]反对派——共产主义(欧文)和社会主义(傅立叶、圣西门)(社会主义还只是处在它的发展的最初阶段)。但是,依照我们的计划,这里要考察的只是本身从政治经济学家的前提出发的反对派。

从我们在下面引用的著作中可以看出,所有这些人实际上都是从李嘉图的形式出发的。

\tchapternonum{(1)小册子《国民困难的原因及其解决办法》}

\tsectionnonum{[(a)把利润、地租和利息看成工人的剩余劳动。资本的积累和所谓“劳动基金”之间的相互关系]}

《根据政治经济学基本原理得出的国民困难的原因及其解决办法。致约翰·罗素勋爵的一封信》1821年伦敦版(匿名)。

这本几乎没有人知道的小册子(约40页),是在“这个不可相信的修鞋匠”\endnote{“这个不可相信的修鞋匠”(《thisincrediblecobbler》)——《对麦克库洛赫先生的,〈政治经济学原理〉的若干说明》这一小册子的作者对麦克库洛赫的称呼。见前面正文第203页。——第260、294页。}麦克库洛赫开始被人注意的时候出现的,它包含一个超过李嘉图的本质上的进步。它直接把剩余价值,或李嘉图所说的“利润”(李嘉图常常也把它叫作“剩余产品”),或这本小册子作者所说的“利息”,看作“剩余劳动”,即工人无偿地从事的劳动,也就是工人除了补偿他的劳动能力价值的劳动量,即生产他的工资的等价物的劳动量以外而从事的劳动。把体现在剩余产品中的剩余价值归结为剩余劳动,同把价值归结为劳动是一样重要的。这一点其实亚·斯密已经说过\fnote{见本卷第1册第57—64页和第2册第461页。——编者注},并且成为李嘉图的阐述中的一个主要因素。但是,李嘉图从来没有以绝对的形式把它说出来并确定下来。

李嘉图和其他政治经济学家的兴趣仅仅在于理解资本主义生产关系,并把它说成是生产的绝对形式,而我们所考察的这本小册子以及要在这里考察的其他这一类著作,则是要掌握李嘉图和其他政治经济学家所揭露的资本主义生产的秘密,以便从工业无产阶级的立场出发来反对资本主义生产。

[小册子的作者说:]

\begin{quote}{“无论资本家得到的份额有多大〈从资本的立场出发〉,他总是只能占有工人的剩余劳动,因为工人必须生活。”(上述著作,第23页)}\end{quote}

这些必要的生活条件,工人能够维持生活所需要的这种最低限度,从而能够从工人身上榨取的剩余劳动量,的确都是相对的量。

\begin{quote}{“如果资本的价值\endnote{从马克思下面的说明中可以看出,小册子《根据政治经济学基本原理得出的国民困难的原因及其解决办法》的作者把“资本的价值”(“thevalueofcapital”)理解为“资本利息”率,即资本的所有者占有的剩余劳动量和他所使用的资本量之比(小册子的作者把“资本利息”理解为马克思叫作剩余价值的东西,但是小册子的作者在这里把剩余价值率同利润率混淆起来了:他把从工人身上榨取的剩余劳动直接和整个预付资本相比)。——第261页。}不按照资本量增加的比例而减少,资本家就会超过工人能够维持生活所需要的最低限度从工人那里榨取每一个劳动小时的产品。不管这种情况看起来多么可怕和多么令人讨厌,资本家最后还是可以把希望寄托在只须花费极少量劳动就能生产出来的那些食物上,并且最后可以对工人说:你不应当吃面包,因为大麦面更便宜;你不应当吃肉,因为吃甜菜和马铃薯也可以过活。我们已经到了这个地步。”(同上,第23—24页)“如果工人能够做到用马铃薯代替面包生活,那就毫无疑问,从他的劳动中可以榨取更多的东西。这就是说,如果靠面包生活,他要维持自己和他的家庭,他必须为自己保留星期一和星期二的劳动,如果靠马铃薯生活,他就只需要为自己保留星期一的一半。星期一的另一半和星期二的全部就可以游离出来,以使国家或资本家得利。”(同上,第26页)}\end{quote}

这里利润等等直接被归结为对工人没有得到任何等价物的那部分劳动时间的占有。

\begin{quote}{“谁都承认,支付给资本家的利息,无论是采取地租、借贷利息的性质,还是采取企业利润的性质,都是用别人的劳动来支付的。”(同上,第23页)}\end{quote}

由此可见,地租、借贷利息和企业利润都只是“资本利息”的不同形式,这种资本利息又归结为“工人的剩余劳动”。这种剩余劳动体现在剩余产品中。资本家是剩余劳动或剩余产品的所有者。剩余产品就是资本。

\begin{quote}{“假定……没有剩余劳动,因而也就没有什么东西可以作为资本积累起来。”(同上,第4页)}\end{quote}

他马上接着说:

\begin{quote}{“剩余产品的所有者,或者说,资本的所有者……”(同上)}\end{quote}

作者用与伤感的李嘉图学派截然不同的口气说:

\begin{quote}{“资本增加的自然和必然的结果是资本价值的减少。”(第22页)}\end{quote}

关于李嘉图,他说:

\begin{quote}{“既然已经发现,如果人口不是随着资本的增加而增加,工资就会由于资本和劳动之间的不平衡而提高,如果人口增加,工资就会由于得到食物的困难而提高,那末,为什么要竭力向我们证明,说因为只有工资的提高才能使利润降低,所以资本的任何积累都不会使利润降低呢?”(第23页)}\end{quote}

[853]如果“资本价值”,即“资本利息”,也就是资本所支配、占有的剩余劳动,不随资本量的增加而减少,那末复利就会按几何级数增长;这个级数用货币计算(见普莱斯),就要以不可能有的积累(不可能有的积累率)为前提,同样,如果把这个级数归结为它的真正要素,即归结为劳动,它就不仅会把剩余劳动,而且会把必要劳动作为资本“得到的份额”一齐吸收。(关于普莱斯的幻想,还要回过头来在收入及其源泉一节\endnote{《收入及其源泉》一节,马克思在1863年1月就已计划放在《资本论》第三部分(见本卷第1册第447页)。但是在1862年10月写的手稿第XIV本封面上,这一节附在《剩余价值理论》最后一章的《补充部分》(见本卷第1册第5页)。而实际上,在1862年10月和11月写成的手稿第XV本中,有一大节是探讨与批判庸俗政治经济学有关的收入及其源泉问题。但是那里根本没有谈到“普莱斯的幻想”。马克思在《资本论》第三卷第二十四章对这一幻想作了批判分析。——第263页。}中谈到。)

\begin{quote}{“如果能使资本不断增加,并使资本价值保持不变(其标志是借贷利息率不变),那末,为使用资本而支付的利息很快就会超过全部劳动产品……资本有快于算术级数增加资本的趋势。谁都承认,支付给资本家的利息,无论是采取地租、借贷利息的性质,还是采取企业利润的性质,都是用别人的劳动来支付的。因此,如果资本继续积累,在利息率保持不变的情况下,为使用资本而支付的劳动必然越来越增多,直到社会上全体工人的全部劳动都被资本家吸收为止。但这是不可能发生的;因为无论资本家得到的份额有多大,他总是只能占有工人的剩余劳动,因为工人必须生活。”(第23页)}\end{quote}

但是“资本的价值”怎样减少,小册子的作者是不清楚的。他自己说,照李嘉图的看法,这种情况之所以发生,或者是由于资本积累比人口增加快时工资提高了,或者是由于人口增加比资本积累快时(或甚至两者[同样地]同时增加时),工资的价值(但不是用生活资料表示的工资的量)因农业生产率降低而增加了。但是我们的匿名作者怎样来说明这一点呢?后一种说法他没有接受;照他的看法,工资会越来越降低,直至降到可能的最低限度。他说,[资本“利息”降低]之所以可能,只是由于虽然工人被剥削得更厉害,或者仍旧那样厉害,用来交换活劳动的那部分资本却相对减少。

不管怎样,匿名作者把利息按几何级数增长这句毫无意义的话还原为它的真正意义,即还原为毫无意义,这是他的功绩。\fnote{[XV—862a}由于剩余价值和剩余劳动的同一性,资本积累就有了质的界限,这种界限是由整个工作日的长度(24小时内劳动能力能起作用的时间)、当时生产力的发展程度以及能够限制同时遭受剥削的工人人数的人口数目决定的。相反,如果在不可理解的利息形式上来考察剩余价值,也就是把剩余价值看作是资本通过一种神秘的魔术而使自身增长的比例,那末,资本积累的界限就仅仅是量的,就绝对不能理解,为什么资本不天天早上把利息作为新的资本一次又一次地并入自身,从而创造出复利的无穷级数。[XV—862a]]

此外,匿名作者认为,有两种办法可以阻止资本在剩余产品或剩余劳动增加时把它掠夺来的赃物的越来越大的部分交还给工人。

第一种办法是把剩余产品转化为固定资本,这就可以阻止“劳动基金”,或者说,工人消费的那一部分产品必定随着资本的积累而增长。

第二种办法是对外贸易,它使资本家能够拿剩余产品去交换外国的奢侈品,从而自己把它消费掉。因此,即使是由必需品构成的那部分产品,也完全可以增加,而不必以工资的形式按其增加的某种比例流回给工人。

必须指出,第一种办法只是定期发生作用,而随后又失去作用(至少在固定资本由加入必需品生产的机器等等构成的情况下是这样),它以剩余产品转化为资本为条件,而第二种办法则以资本家消费剩余产品的部分越来越大,资本家的消费不断增加为条件,而不以剩余产品再转化为资本为条件。如果这种剩余产品以它直接存在的形式保留下来,那末其中就会有很大一部分必须作为可变资本同工人相交换,其结果就会提高工资和降低绝对或相对剩余价值。马尔萨斯宣扬“富人”必须增加消费,以便使那部分用来同劳动交换、转化为资本的产品具有很高的价值,带来很多的利润,吸收大量的剩余劳动,其真正的秘密也就在此。不过马尔萨斯不是让工业资本家本身增加消费,而是把这一职能给了土地所有者、领干薪者等等,因为积累的欲望和消费的欲望结合在一个人身上便会互相干扰。这里也暴露出巴顿、李嘉图等人观点中的错误。工资不是由产品总量中可能作为可变资本被消费,或者说,可能转化为可变资本的那一部分决定,而是由产品总量中实际转化为可变资本的部分决定。这些产品中有一部分甚至可能以实物形式被各种食客吃掉,另外一部分则可能通过对外贸易等等作为奢侈品消费掉。

我们这位小册子的作者忽略了以下两件事:

由于采用机器,大批工人经常失业,这就造成过剩人口;于是剩余产品找到了可以同它交换的现成的新劳动,而人口不必增加,绝对劳动时间无须延长。假定以前雇用500个工人,现在雇用300个工人,这300个工人提供相对来说更多的剩余劳动。只要剩余产品有足够的增加,其余的200个工人便可以用剩余产品来雇用。原有[可变]资本的一部分转化为固定资本,另一部分用来雇用较少量的工人,但是同他们的人数相比,却从他们身上榨取了更多的剩余价值,特别是榨取了奥多的剩余产品。其余的200个工人就是为了使新的剩余产品资本化而创造出来的材料。

[853a]正如这个小册子所说的,必需品通过对外贸易变成奢侈品,本身是很重要的:

(1)因为这种情况结束了这样一种谬论:似乎工资取决于所生产的必需品的量,似乎这些必需品必然以这种形式由它们的生产者或者甚至由从事生产的全体民众所消费,也就是说,必然再转化为可变资本,或者说,象巴顿和李嘉图所说的那样,再转化为“流动资本”;

(2)因为这种情况决定了某些同建立在资本主义生产基础上的世界市场有联系的落后国家——例如北美合众国奴隶占有制各州(见凯尔恩斯\endnote{马克思指的是当时刚出版的凯尔恩斯的著作《奴隶劳力:它的性质、经过及其可能的前途》(1862年伦敦版)。这本书他不止一次地在《资本论》第一卷和第三卷中引证过。——第266页。})或波兰等等——的整个社会形式(老毕希已经理解到了这一点,如果他不是从斯图亚特那里剽窃来的话)。无论这些国家从它们的奴隶的剩余劳动中榨取的简单形式即子棉或谷物形式的剩余产品量有多大,它们仍然能够保持这种简单的、没有等级差别的劳动,因为对外贸易使它们能够把这种简单的产品变成任何形式的使用价值。

说年产品中必须以工资形式花费的部分取决于“流动资本”量,这就等于说,当产品中有很大一部分由“建筑物”构成时,当与工人人口相比建筑了大量的工人住宅时,由于住宅的供给比对住宅的需求增加得快,工人一定会得到良好的和便宜的住宅。

相反,以下的说法是正确的:如果剩余产品很多,资本家又打算把其中很大一部分用作资本,那末(假定这么多剩余产品本身不是通过把大批工人抛向街头的办法取得的),对劳动的需求一定会增长,因而剩余产品中作为工资来交换的部分也必然会增长。无论如何,不是剩余产品(不管它以什么形式存在,甚至以必需品的形式存在)的绝对量迫使人们把剩余产品用作可变资本,因而使工资增加。而是除非机器经常造成过剩人口,除非资本的越来越大的部分(特别是也通过对外贸易)和资本交换而不是和劳动交换,资本化的欲望就会迫使人们把剩余产品的很大一部分用作可变资本,因而随着资本的积累,引起工资的增长。剩余产品中以只能用作资本的形式直接生产出来的部分,以及其中由于同外国交换而取得这种形式的部分,比其中必须和直接劳动相交换的部分增长得快。

工资取决于现有的资本,因而资本的迅速积累是引起工资提高的唯一手段,这一句话可归结如下:

一方面,如果把劳动条件表现为资本的形式撇开不谈,那就是这样一个同义反复:在工人生活条件不降低的情况下工人人数能够增加多快,取决于一定数量的工人所实现的劳动生产率。他们生产的原料、劳动工具和生活资料越多,他们就会有越多的钱,不仅用来抚养自己还不能工作的子女,而且用来实现新的正在成长的一代人的劳动,从而使人口的增长和生产的增长相一致,甚至使生产的发展超过人口的增长,因为随着人口的增长,工人的技能会提高,分工会增多,采用机器的可能性会增大,不变资本会增加,一句话,劳动生产率会提高。

如果人口的增长取决于劳动生产率,那末劳动生产率便取决于人口的增长。这里是互相发生作用。但是用资本主义的术语来表达,这就是说,工人人口的生活资料取决于资本的生产率,取决于工人的产品中尽可能大的一部分作为工人劳动的支配者同他们相对立。李嘉图本人正确地表达了这一点(我指的是同义反复),他认为工资取决于资本的生产率,而资本的生产率取决于劳动生产率。\fnote{见本卷第2册第618页和本册第121—123页。——编者注}

劳动取决于资本的增长这一点,一方面只不过是意味着如下的同义反复,[854]即工人人口的生活资料和就业手段的增长取决于他们自身的劳动生产率,第二,用资本主义的术语来表达,劳动取决于这样一种情况,即他们自己的产品作为别人的财产和他们相对立,因此,他们自己的生产率作为他们所创造的物的生产率和他们相对立。

这一点实际上意味着,工人在自己产品中占有的部分必须尽可能小,以便使他们的产品中作为资本和他们相对立的部分尽可能大;工人无偿地让给资本家的东西必须尽可能多,以便使资本家用来再购买工人劳动的资金(无偿地从工人那里榨取来的东西)尽可能多地增加。在这种场合,可能出现这样的情况:如果资本家让工人无代价地劳动得太多了,现在,为了换取这些没有付给等价物而得到的东西,他就可能让工人无代价地劳动得略微少一些。但是,因为这样做的结果正好妨碍资本家所追求的目的,即尽可能快地积累资本,所以工人必须在这样的条件下生活,以至于工人无酬劳动的减少又会因工人人口的增加(无论是由于采用机器而造成的相对增加,还是由于早婚而造成的绝对增加)而停止。(这也就是马尔萨斯主义者作为土地所有者和资本家之间的关系来宣扬,而为李嘉图学派所嘲笑的那样一种关系。)工人必须把自己产品中尽可能大的一部分无代价地交给资本,以便在较为有利的条件下用自己的新劳动买回这样让出的一部分产品。但是,由于这种有利的转变会同时消灭有利转变的条件,所以它只能是暂时的,它一定会再转化为它自己的对立面。

(3)适用于必需品通过对外贸易转化为奢侈品的,一般也适用于奢侈品的生产,但是,要使奢侈品花样繁多和增加,对外贸易的确是一个相当重要的条件。虽然从事奢侈品生产的工人为他们的雇主生产资本,但是他们的产品不能以实物形式再转化为资本,既不能再转化为不变资本,也不能再转化为可变资本。

如果把运往国外交换必需品(这些必需品全部或部分加入可变资本)的那部分奢侈品除外,那末,奢侈品所代表的只是一种剩余劳动,并且是直接以富人作为收入来消费的剩余产品形式出现的剩余劳动。诚然,奢侈品不只是代表生产它们的那些工人的剩余劳动。相反,这些工人完成的剩余劳动平均来说同其他生产部门的工人所完成的一样多。但是,正象可以把包含1/3剩余劳动的1/3产品看作这些剩余劳动的体现,而把产品的其余2/3看作预付资本的再生产一样,构成奢侈品生产者的工资的必需品生产者的剩余劳动,也可以体现为整个工人阶级的必要劳动。整个工人阶级的剩余劳动体现在:(1)资本家及其仆从所消费的那一部分必需品上;(2)全部奢侈品上。对单个资本家或单个生产部门来说,这就表现得不同了。对于单个资本家来说,他生产的奢侈品的一部分只是预付资本的等价物。

如果剩余劳动中直接表现为奢侈品形式的部分过大,那末,很明显,它一定会妨碍积累和扩大再生产,因为剩余产品中再转化为资本的部分太小。如果剩余劳动中表现为奢侈品形式的部分过小,那末,资本(即剩余产品中能够以实物形式再用作资本的部分)的积累将快于人口的增加,利润率将会下降,除非有必需品的国外市场存在。

\tsectionnonum{[(b)简单再生产条件下和资本积累条件下资本和收入的交换问题]}

我在解释资本和收入的交换时\fnote{见本卷第1册第233—258页。——编者注}把工资也看作收入,并且一般说来只考察了不变资本和收入的关系。工人的收入同时表现为可变资本,这一情况只有在如下的条件下才是重要的,那就是,在积累过程中(在新资本形成过程中),生产生活资料的资本家的由生活资料(必需品)构成的余额,能够同生产不变资本的资本家的由原料或工具构成的余额直接交换。在这里,一种形式的收入同另一种形式的收入交换,[855]这种交换一经完成,资本家A的收入就转化为资本家B的不变资本,而资本家B的收入就转化为资本家A的可变资本。

在考察资本的这种流通、再生产和相互补偿方式等等的时候,首先必须把对外贸易撇开不谈。

其次,必须区别以下两种现象:

(1)既定规模的再生产,

(2)扩大规模的再生产,或者说,积累——收入转化为资本。

关于(1)。

我曾经指出:

生活资料生产者必须补偿(1)他们的不变资本,(2)他们的可变资本。他们的产品中代表超过这两部分的余额的那一部分价值,构成剩余产品,构成剩余价值的物质存在,这种剩余价值又不过是剩余劳动的代表。

可变资本——生活资料生产者的产品中代表可变资本的部分——构成工资,构成工人的收入。这一部分在这里已经以实物形式存在,它以这种实物形式重新用作可变资本。这一部分,即工人再生产出来的等价物,被用来重新购买工人的劳动。这是资本和直接劳动之间的交换。工人以货币形式得到这一部分,他用这些货币买回他自己的产品或同一部类的其他产品。这是在工人以货币形式得到他应得的那一份产品的凭证以后可变资本各个不同组成部分相互之间的交换。这是同一部类(生活资料)内部新加劳动的一部分同另一部分的交换。

剩余产品(新加劳动)中由(生产生活资料的)资本家自己消费的部分,或者是被他们以实物形式消费,或者是在他们之间用一种可消费形式的剩余产品同另一种相交换。这是收入同收入的交换,同时这两种收入都归结为新加劳动。

上述交易其实不能说是收入同资本的交换。资本(必需品)是同劳动(劳动能力)交换。因此,这里不是收入和资本相交换。当然,工人一得到工资,就会把它消费掉。但是他用来同资本交换的不是他的收入,而是他的劳动。

[生活资料生产者的产品中的]第三部分[代表他们的]不变资本,它同生产不变资本的生产者的产品的一部分相交换,也就是同他们的产品中代表新加劳动的那一部分相交换。不变资本生产者的产品的这一部分,是由工资的等价物(也就是由[这些生产者的]可变资本)和剩余产品,剩余价值,即以只能用于生产消费而不能用于个人消费的形式存在的资本家的收入组成。所以,一方面,这是这些生产者的可变资本同生活资料中代表[生活资料生产者的]不变资本的部分相交换。实际上是不变资本生产者的产品中代表他们的可变资本而以不变资本形式存在的那一部分,同生活资料生产者的产品中代表不变资本而以可变资本形式存在的部分相交换。这里是新加劳动同不变资本的交换。

另一方面,不变资本生产者的产品中代表剩余产品而以不变资本形式存在的那一部分,同生活资料中代表生活资料生产者的不变资本的部分相交换。这里是收入同资本的交换。生产不变资本的资本家的收入,同生活资料相交换,并补偿生产生活资料的资本家的不变资本。

最后,生产不变资本的资本家的产品中本身代表不变资本的部分,部分地以实物形式得到补偿,部分地通过不变资本生产者之间的(被货币掩盖了的)实物交换得到补偿。

这一切都是在假定再生产规模和原有生产规模相同的情况下发生的。

如果我们现在要问,全部年产品中哪一部分代表新加劳动,那末,计算是非常简单的:

(A)[个人]消费品。分为三部分。[第一,]资本家的收入,等于一年内加进的剩余劳动。

第二,工资,即可变资本,等于工人用以再生产自己的工资的新加劳动。

最后,第三部分是原料、机器等等。这是不变资本,即产品价值中只被保存而不被生产的部分。因此,这不是一年内的新加劳动。

[856]如果我们用c′表示[这一部类的]不变资本,用v′表示可变资本,用r′表示剩余产品,表示收入,那末,这一部类就是由c′和v′十r′组成。

c′只是保存的价值,而不是新加劳动(这个c′代表产品的一部分);相反,v′十r′是一年内加进的劳动。

[A部类的]总产品(或它的价值)Pa扣除c′,就代表新加劳动。

因此,如果从A部类的产品中,即从Pa中扣除c′,我们便得出一年内的新加劳动。

(B)生产消费品。

v″+r″在这里也是代表新加劳动。在这一领域里执行职能的不变资本c″不代表新加劳动。

但是,v″+r″=它们所交换的c′。c′转化为B部类的可变资本和收入。另一方面,v″和r″转化为c′,转化为A部类的不变资本。

如果从B部类的产品中,即从Pb中扣除c″,我们便得出一年内的新加劳动。

但是,Pb-c″=c′。因为全部产品Pb扣除c″即B部类使用的不变资本后,同c′相交换。

在v″+r″同c′交换后,情况可以表述如下:

Pa只由新加劳动构成,新加劳动的产品分解为利润和工资,分解为必要劳动的等价物和剩余劳动的等价物。因为现在代替c′的v″+r″等于B部类的新加劳动。

因此,全部产品Pa,不论是它的剩余产品,还是它的可变资本和它的不变资本,都由一年内新加劳动的产品组成。

相反,全部产品Pb可以这样来看:它不代表新加劳动的任何部分,而只代表被保存的过去劳动。因为它的c″部分不代表任何新加劳动。同样,它的用v″+r″换得的c′部分也不代表新加劳动,因为这个c′在A部类代表预付不变资本,不代表新加劳动。

由此可见,年产品中所有作为可变资本构成工人收入,作为剩余产品构成资本家的消费基金的部分都归结为新加劳动,而产品中其余所有代表不变资本的部分只归结为被保存的过去劳动,仅仅补偿不变资本。

因此,那种把年产品中所有作为收入,作为工资和利润(包括利润的分枝——地租和利息等,也包括非生产劳动者的工资)消费的部分都归结为新加劳动的看法是正确的,而把全部年产品都归结为收入,归结为工资和利润,即只归结为新加劳动中某些部分的总和的看法却是错误的。年产品中有一部分归结为不变资本,它按价值来说不代表新加劳动,而作为使用价值,既不加入工资,也不加入利润。这部分产品,按其价值来说,代表真正意义上的积累劳动,按其使用价值来说,代表这种积累的过去劳动的消费。

另一方面,认为产品中归结为工资和利润的部分不能全部代表一年内加进的劳动,这种看法同样是正确的。因为这种工资和利润可以用来购买服务,即购买不加入代表工资和利润的产品的劳动。这种服务,这种劳动,是人们在消费产品的过程中使用的,它们不加入产品的直接生产。

[857]关于(2)。

关于积累,关于收入转化为资本,关于扩大规模的再生产(就这种再生产的发生不单单是由于更有效地使用原有资本而言),情况就不同了。在这里,全部新资本是由新加劳动构成的,而且是由利润等等形式的剩余劳动构成的。不过,说这里新生产的全部要素都是由新加劳动——工人的剩余劳动的一部分——构成和产生,虽然是正确的,但是象政治经济学家们又一次假定的那样,认为剩余劳动转化为资本时只归结为可变资本或工资,却是错误的。例如,假定租地农场主的一部分剩余产品同机器厂主的一部分剩余产品交换。这种交换使机器厂主能够直接或间接地把小麦转化为可变资本,雇用更多的工人。另一方面,由于这种交换,租地农场主就把他的一部分剩余产品转化为不变资本,由于这种转化,他可能不是雇用新的工人而是解雇一部分原有的工人。其次,租地农场主可能耕种更多的土地。那样一来,一部分小麦就将不转化为工资,而转化为不变资本,等等。

只有在进行这种积累时才能看出,所有的一切,不论是收入还是可变资本和不变资本,都是被占有的别人劳动,不论是工人赖以工作的劳动条件还是工人用自己的劳动换得的等价物,都是资本家不付等价物而得到的工人劳动。

甚至在原始积累的条件下也是这样。假定我从工资中节约500镑。那末,这500镑实际上所代表的不是单纯的积累劳动,而是和资本家的“积累劳动”不同的、我自己的、由我自己和为我积累的劳动。我把它转化为资本,购买原料等等和雇用工人。假定利润是20%即每年100镑。在五年中(如果始终没有新的积累,并且每年得到的100镑都被吃掉),我以收入的形式把我的资本“吃掉”。到第六年我的这500镑资本本身就代表不付等价物而占有的别人劳动了。如果我总是把我的利润的一半积累起来,那末[把我的原有资本吃掉的]过程就会慢一些,因为我不吃掉那么多,并且[占有别人劳动的过程]会快一些。

\todo{}

到第八年,虽然我吃掉的比原有资本多,我的资本却几乎增加了一倍。在972镑资本中,已经没有丝毫有酬劳动,或者说,我曾为之支付过等价物的劳动了。我以收入的形式把我的全部原有资本消费掉了。就是说,我得到了原有资本的等价物,我又把这个等价物消费掉了。新资本仅仅是由被占有的别人劳动所构成。

在考察剩余价值本身的时候,产品的实物形式,从而剩余产品的实物形式,是无关紧要的。在考察实际再生产过程的时候,它却具有重要意义,一方面是为了理解产品形式本身,另一方面是为了弄清楚奢侈品等等的生产对再生产过程的影响。这里我们又有了一个说明使用价值本身具有经济意义的例子。

\tsectionnonum{[(c)小册子作者的功绩及其观点在理论上的混乱。他提出的关于资本主义社会中的对外贸易的作用以及“自由时间”是真正的财富等问题的意义]}

[858]现在再回过头来谈我们的小册子。

[小册子的作者写道:]

\begin{quote}{“假定一个国家的全部劳动所生产的恰好足够维持全部人口的生活;在这种情况下,很明显,就没有剩余劳动,因而也就没有什么东西可以作为资本积累起来。假定一个国家的全部劳动一年中所生产的足够维持该国人口两年的生活;在这种情况下,很明显,或者是足够维持全部人口生活一年的消费资料必须毁掉,或者是人们必须停止一年的生产劳动。但是,剩余产品——或者说资本——的所有者在下一年中既不会让人们无事可做,也不会让这些产品毁掉;他们会把人们的劳动用于某种不是直接生产的工作,例如用来安装机器等等。但是到第三年,全部人口会重新从事直接生产劳动,并且,由于上一年安装的机器现在已经开始运转,所以很明显,这一年的产品将比第一年多,因为还要加上机器的产品。因此,这种剩余产品就更加要或者毁掉,或者象上面所说的那样被使用;而这种使用会重新增加社会的劳动生产力,直到人们必须停止一段时间生产劳动,否则他们的劳动产品就要毁掉。这便是最简单的社会状态下[资本积累]的明显后果。”(第4—5页)“其他国家的需求不仅受我们的生产能力的限制,而且受他们的生产能力的限制。”}\end{quote}

{这是对萨伊的论断的回答,萨伊认为,不是我们生产得太多,而是其他国家生产得太少\endnote{马克思指萨伊的下述论断(在他的《给马尔萨斯先生的信》1820年巴黎版第15页):例如,如果英国商品充斥意大利市场,那末,原因就在于能够同英国商品交换的意大利商品生产不足。萨伊的这些论断在匿名著作《论马尔萨斯先生近来提倡的关于需求的性质和消费的必要性的原理》(1821年伦敦版第15页)中引证过,在马克思的第XII本札记本第12页对这部著作所作的摘录中也有这些论断。参看本卷第1册第237页,第2册第607页和本册第131页。——第277页。}。他们的生产能力不一定和我们的生产能力相等。}

\begin{quote}{“因为,不管我们多么努力,在若干年内整个世界从我们这里拿走的未必会比我们从世界取得的多,所以我们备受赞扬的整个对外贸易从来没有、从来不能、也决不可能为我国的财富增加一先令或一文钱,因为每有一包丝绸、一箱茶叶、一桶酒进口,就有价值相等的某种东西出口,甚至我们的商人从他们的对外贸易中取得的利润,也由这里用出口换得的进口商品的消费者支付。”(第17—18页)“对外贸易只是为了资本家舒适和享乐而进行的一种商品交换:资本家没有一百个躯体和一百双脚,他不能以衣服和袜子的形式把国内生产的全部呢绒和棉针织品都消费掉,因此它们被用来交换酒和丝绸。但是这些酒和丝绸象那些呢绒和袜子一样代表我们本国人的剩余劳动;通过这种办法资本家的破坏力无限度地增大了:由于对外贸易,资本家得以巧胜自然,突破自然对他们的剥削要求和剥削愿望设置的成千的自然限制;现在无论对于他们的实力或者对于他们的愿望,都没有什么限制了。”(第18页)}\end{quote}

我们看到,小册子的作者接受了李嘉图的对外贸易学说。在李嘉图的著作中这一学说只是用来证明他的价值理论,或者说明这一学说和价值理论并不矛盾。在小册子里则着重指出,体现在对外贸易结果上的不仅是国民的劳动,而且是国民的剩余劳动。

如果剩余劳动和剩余价值只表现在国民的剩余产品中,那末,为了价值而增加价值,从而榨取剩余劳动,就会受到[国民]劳动所创造的价值借以表现的使用价值的局限性或狭隘范围的限制。但是只有对外贸易才使作为价值的剩余产品的真正性质显示出来,因为对外贸易使剩余产品中包含的劳动作为社会劳动发展起来,这种劳动表现在无限系列的不同的使用价值上,并且在实际上使抽象财富有了意义。

\begin{quote}{“只有需要和满足这些需要所必需的商品种类的无限多样性{因而还有生产这些不同种类的商品的具体劳动的无限多样性},才使对财富的贪欲{从而占有他人劳动的贪欲}成为无止境的和永远无法满足的。”(威克菲尔德在他出版的亚·斯密《国富论》1835年伦敦版第1卷第64页上所加的注)}\end{quote}

但是,只有对外贸易,只有市场发展为世界市场,才使货币发展为世界货币,抽象劳动发展为社会劳动。抽象财富、价值、货币、从而抽象劳动的发展程度怎样,要看具体劳动发展为包括世界市场的各种不同劳动方式的总体的程度怎样。资本主义生产建立在价值上,或者说,建立在包含在产品中的作为社会劳动的劳动的发展上。但是,这一点只有在对外贸易和世界市场的基础上[才有可能]。因此,对外贸易和世界市场既是资本主义生产的前提,又是它的结果。

[859]这本小册子不是理论性论著。它是对政治经济学家们为当时的贫困和“国民困难”所找到的虚假原因的抗议。因此,它在这里并没有奢望,而且也不能对它提出要求:把剩余价值理解为剩余劳动,就要对经济范畴的整个体系进行总的批判。相反,作者以李嘉图体系为依据,只是前后一贯地作出了这一体系本身中所包含的结论,并且为了工人阶级的利益而提出这一结论来反对资本。

可是,这位作者为既有的经济范畴所束缚。就象李嘉图由于把剩余价值同利润混淆起来而陷入令人不快的矛盾一样,他也由于把剩余价值命名为资本利息而陷入同样的矛盾。

诚然,他在以下方面超过了李嘉图:首先,他把一切剩余价值都归结为剩余劳动,其次,他虽然把剩余价值叫作资本利息,同时又着重指出,他把资本利息理解为剩余劳动的一般形式,而与剩余劳动的特殊形式即地租、借贷利息和企业利润相区别:

\begin{quote}{“支付给资本家的利息,无论是采取地租、借贷利息的性质〈应当说:形式〉,还是采取企业利润的性质……”(第23页)}\end{quote}

可见,小册子的作者把剩余劳动或剩余价值的一般形式和它们的特殊形式区别开来了,李嘉图和亚·斯密却没有做到这一点,至少是没有有意识地和前后一贯地做到这一点。但是,他还是把这些特殊形式之一的名称——利息,当作一般形式的名称。这就足以使他重新陷入经济学的费解的行话中。

\begin{quote}{“在一个巩固地建立起来的社会里,资本的不断增长会由借贷利息的下降表现出来,或者同样可以说,会由为使用资本而付出的他人的劳动量的减少表现出来。”(第6页)}\end{quote}

这有点象凯里的话。但是小册子的作者认为,不是工人使用资本,而是资本使用工人。既然他把利息理解为任何形式的剩余劳动,那末全部问题(即“解决我们的国民困难”)就归结为提高工资,因为利息的减少也就是剩余劳动的减少。但他的意思是:在劳动同资本交换的时候,对别人劳动的占有必须减少,或者说,工人从他自己的劳动中占有的必须多些,而资本占有的必须少些。

要求减少剩余劳动可能有两方面的意思:

(1)工人除了再生产劳动能力、创造工资的等价物所必需的时间以外,从事的劳动必须少些;

(2)在劳动总量中采取剩余劳动(即工人无代价地为资本家劳动的时间)的形式的部分必须少些;从而,在体现劳动的产品中采取剩余产品形式的部分必须少些;也就是说,工人从他自己的产品中得到的必须比以前多些,而资本家从这一产品中得到的必须比以前少些。

作者自己对这个问题是不清楚的,这一点也可以从下面一段话——其中实际上包含着他的著作中的结论性东西——看出来:

\begin{quote}{“一个国家只有在使用资本而不支付任何利息的时候,只有在劳动6小时而不是劳动12小时的时候,才是真正富裕的。财富就是可以自由支配的时间,如此而已。”(第6页)}\end{quote}

因为这里“利息”被理解为利润、地租、借贷利息,一句话,被理解为任何形式的剩余价值,因为在小册子的作者本人看来,资本只不过是劳动产品,是积累的劳动,用它来交换,不仅能够得到等量劳动,而且能够得到剩余劳动,所以在他看来,“资本不提供利息”这种说法的意思就是没有任何[860]资本存在。产品不转化为资本。既没有剩余产品,也没有剩余劳动。只有到那时国家才真正富裕。

但是,这一点的意思可能是:除了工人的再生产所需要的产品和劳动以外,既没有产品,也没有劳动。或者是:工人自己占有这个余额,无论是产品的余额,还是劳动的余额。

不过作者所指的不只是后面一点,这从以下的事实可以看出:他把“一个国家只有在劳动6小时而不是劳动12小时的时候,才是真正富裕的”,“财富就是可以自由支配的时间,如此而已”这两个论点和“使用资本而不支付任何利息”这一论点结合起来了。

这可能是这样的意思:

如果所有的人都必须劳动,如果过度劳动者和有闲者之间的对立消灭了,——而这一点无论如何只能是资本不再存在,产品不再提供占有别人剩余劳动的权利的结果,——如果把资本创造的生产力的发展也考虑在内,那末,社会在6小时内将生产出必要的丰富产品,这6小时生产的将比现在12小时生产的还多,同时所有的人都会有6小时“可以自由支配的时间”,也就是有真正的财富,这种时间不被直接生产劳动所吸收,而是用于娱乐和休息,从而为自由活动和发展开辟广阔天地。时间是发展才能等等的广阔天地。大家知道,政治经济学家们自己认为雇佣工人的奴隶劳动是合理的,说这种奴隶劳动为其他人,为社会的另一部分,从而也为[整个]雇佣工人的社会创造余暇,创造自由时间。

或者这一点也可能有这样的意思:

工人现在除了自己的再生产(现在)所需要的以外劳动6小时。(不过这大概不会是小册子作者的观点,因为他把工人现在所需要的说成是非人道的最低限度。)如果资本不再存在,那末工人将只劳动6小时,有闲者也必须劳动同样多的时间。这样,所有的人的物质财富都将降到工人的水平。但是所有的人都将有自由时间,都将有可供自己发展的时间。

显然,小册子的作者本人对这一点是不清楚的。不过下面这段话无论如何仍不失为一个精彩的命题:

\begin{quote}{“一个国家只有在劳动6小时而不是劳动12小时的时候,才是真正富裕的。财富就是可以自由支配的时间,如此而已。”}\end{quote}

李嘉图在《价值和财富,它们的特性》一章中也说,真正的财富在于用尽量少的价值创造出尽量多的使用价值,换句话说,就是在尽量少的劳动时间里创造出尽量丰富的物质财富。这里,“可以自由支配的时间”以及对别人劳动时间里创造出来的东西的享受,都表现为真正的财富,但是正象资本主义生产中的一切东西一样,因而正象资本主义生产的解释者所认为的那样,这是以对立的形式表现出来的。财富和价值的对立后来在李嘉图的著作里表现为这样的形式,即纯产品在总产品中占的比例应当尽量的大,而这(又是在这种对立的形式上)意味着,社会上那些虽然享受物质生产成果、但是其时间只有一部分被物质生产吸收或者完全不被物质生产吸收的阶级,与时间全部被物质生产吸收、因而其消费仅仅构成生产费用的一个项目、仅仅构成一种使其充当上层阶级的驮畜的条件的那些阶级比较起来,人数应当尽可能地多。这一点总是意味着期望社会上注定陷入劳动奴隶制即从事强制劳动的部分尽可能地小。这就是那些站在资本主义立场上的人所能达到的最高点。

小册子的作者批驳了这一点。即使交换价值消灭了,劳动时间也始终是财富的创造实体和生产财富所需要的费用的尺度。但是自由时间,可以支配的时间,就是财富本身:一部分用于消费产品,一部分用于从事自由活动,这种自由活动不象劳动那样是在必须实现的外在目的的压力下决定的,而这种外在目的的实现是自然的必然性,或者说社会义务——怎么说都行。

不言而喻,随着雇主和工人之间的社会对立的消灭等等,劳动时间本身——由于限制在正常长度之内,其次,由于不再用于别人而是用于我自己——将作为真正的社会劳动,最后,作为自由时间的基础,而取得完全不同的、更自由的性质,这种同时作为拥有自由时间的人的劳动时间,必将比役畜的劳动时间具有高得多的质量。

\tchapternonum{(2)莱文斯顿[把资本看成工人的剩余产品。把资本主义发展的对抗形式同资本主义发展的内容本身混淆起来。由此产生的对生产力的资本主义发展成果的否定态度]}

[861]皮尔西·莱文斯顿硕士《论公债制度及其影响》1824年伦敦版。

这是一部非常出色的著作。

小册子《国民困难的原因及其解决办法》的作者是就剩余价值的原始形式,即剩余劳动形式来考察剩余价值的。所以劳动时间的长短成了他的主要着眼点。他主要是就剩余劳动或剩余价值的绝对形式,即在工人本身的再生产所必要的劳动时间以外延长劳动时间的形式,而不是通过劳动生产力的发展缩减必要劳动的形式,来考察剩余劳动或剩余价值的。

缩减这种必要劳动是李嘉图的主要着眼点,但是,在资本主义生产的情况下,缩减必要劳动是延长属于资本所有的劳动时间的一种手段。与此相反,小册子的作者却把缩短生产者的劳动时间和停止为剩余产品的所有者劳动宣布为最终目的。

莱文斯顿似乎以工作日既定为前提。因此,他论述的主要对象——在这些论述中,也和小册子《国民困难的原因及其解决办法》的作者一样,只是附带涉及到一些理论问题——是相对剩余价值,或者说,由于劳动生产力的发展而归资本所有的剩余产品。就象抱这种观点的人一般所做的那样,这里多半是就剩余产品的形式来考察剩余劳动,而小册子的作者则多半是就剩余劳动的形式来考察剩余产品。

\begin{quote}{“教导人们说一国的富强取决于它的资本,就是要使劳动从属于财富,使人变成财产的奴仆。”(第7页)}\end{quote}

李嘉图的理论在它自己的前提基础上产生的对立面具有如下的特点:

政治经济学,随着它的不断发展,——这种发展,就基本原则来说,在李嘉图的著作里表现得最突出,——越来越明确地把劳动说成是价值的唯一要素和使用价值的唯一[积极的]创造者,把生产力的发展说成是实际增加财富的唯一手段,而把劳动生产力的尽可能快的发展说成是社会的经济基础。实际上,这也就是资本主义生产的基础。特别是李嘉图的著作,在它证明价值规律既不受土地所有权也不受资本积累等等的破坏的时候,其实只是企图把一切和这种见解矛盾或似乎矛盾的现象从理论中排除出去。但是,正象劳动被理解为交换价值的唯一源泉和使用价值的积极源泉一样,“资本”也被同一些政治经济学家,特别是大卫·李嘉图(在他以后,托伦斯、马尔萨斯、贝利等人更是这样)看作是生产的调节者、财富的源泉和生产的目的,而在他们的著作里,劳动表现为雇佣劳动,这种雇佣劳动的承担者和实际工具必然是赤贫者(而且这里还有马尔萨斯的人口论在起作用),他们只是生产费用的一个项目和单纯的生产工具,注定只能拿最低限度的工资,每当工人对资本来说成为“多余的”时候,还不得不降到这一最低限度以下。在这个矛盾中,政治经济学只是说出了资本主义生产的本质,或者也可以说,雇佣劳动,即从本身中异化出来的劳动的本质,这种劳动创造的财富作为别人的财富和它相对立,它自己的生产力作为它的产品的生产力和它相对立,它的致富过程作为自身的贫困化过程和它相对立,它的社会力量作为支配它的社会力量和它相对立。但是,这些政治经济学家把社会劳动在资本主义生产中表现出来的这种一定的、特殊的、历史的形式说成是一般的、永恒的形式,说成是自然的真理,而把这种生产关系说成是社会劳动的绝对(而不是历史地)必然的、自然的、合理的关系。由于受到资本主义生产视野的局限,他们把社会劳动在这里借以表现的对立形式说成和摆脱了上述对立的这一劳动本身一样是必然的。这样,他们一方面把绝对意义上的劳动(因为在他们看来,雇佣劳动和劳动是等同的),另一方面又把同样绝对意义上的资本,把工人的贫困和不劳动者的财富同时说成是财富的唯一源泉,他们不断地在绝对的矛盾中运动而毫不觉察。(西斯蒙第由于觉察到了这种矛盾而在政治经济学上开辟了一个时代。)“劳动,或者说,资本”——在李嘉图的这种说法\endnote{马克思在《剩余价值理论》第二册引用和分析了李嘉图著作中包含这一说法的引文(见本卷第2册第200、202和456页)。——第285页。}中,矛盾本身以及把这种矛盾当作等同的东西说出来的那种天真,明显地表现出来了。

但是很明显,既然使资产阶级政治经济学在理论上作了这种毫不留情的表述的那同一种现实的发展,又发展了现实本身所包含的实际矛盾,特别是发展了英国日益增长的“国民”财富和日益增长的工人贫困之间的对立,其次,既然这些矛盾在李嘉图以及其他政治经济学家的理论中得到了理论上中肯的、尽管是无意识的表现,那末,站到无产阶级方面来的思想家[XV—862]抓住了在理论上已经给他们准备好了的矛盾,是十分自然的。劳动是交换价值的唯一源泉和使用价值的唯一的积极的创造者。你们这样说。另一方面,你们说,资本就是一切,而工人算不了什么,或者说,工人仅仅是资本的生产费用的一个项目。你们自己驳倒了自己。资本不过是对工人的诈骗。劳动才是一切。

这实际上是从李嘉图的观点,从李嘉图自己的前提出发来维护无产阶级利益的一切著作的最后的话。李嘉图不懂得他的体系中所论述的资本和劳动的等同,同样,这些著作的作者也不懂得他们所论述的资本和劳动之间的矛盾。因此,即使是他们中间最出色的人物,如霍吉斯金,也把资本主义生产的一切经济前提看作是永恒的形式,他们所希望的只是消灭资本——这些前提的基础,同时也是必然结果。

莱文斯顿的主要思想是:

劳动生产力的发展创造了资本,或者说,财产,即为“有闲者”——游手好闲者、非劳动者——创造剩余产品,同时劳动还生出了它的寄生赘瘤;劳动生产力越发展,这个寄生赘瘤就越把劳动的骨髓吸尽。非劳动者获得占有这种剩余产品的权利,或者说,获得占有别人劳动产品的权力,是由于他已经拥有财富,还是由于他有土地、土地所有权,这并不会使事情发生变化。两者都是资本,即都是对别人劳动产品的支配权。财产——property——在莱文斯顿看来,只是对别人劳动产品的占有,而这一点只有在生产劳动发展的情况下才有可能,而且只可能是与生产劳动发展的程度相适应。莱文斯顿把生产劳动理解为生产必需品的劳动。非生产劳动,“消费劳动”\endnote{莱文斯顿所说的“消费劳动”(“industryofconsumption”),是指奢侈品的生产和为财产所有者进行的各种服务。——第286页。}是资本或财产发展的结果之一。莱文斯顿和小册子《国民困难的原因及其解决办法》的作者一样,表现为一个禁欲主义者。在这里,他本身又是为政治经济学家的概念所束缚。没有资本,没有财产,工人消费的必需品便会生产得极其丰富,但不会有奢侈品的生产。或者也可以说,既然在小册子的作者看来,资本生产维持工人生活所必需的劳动以外的剩余劳动,并且引起机器(小册子的作者称为“固定资本”)的制造以及对外贸易和世界市场的建立,部分是为了利用从工人那里榨取的剩余产品去增进生产力,部分是为了使这种剩余产品成为必需品以外的多种多样的使用价值,——既然如此,那末莱文斯顿同小册子的作者一样,是理解,或者至少是在实际上承认资本的历史必然性的。同样,在莱文斯顿看来,没有资本和财产,就既不会有“舒适品”、机器或奢侈品生产出来,也不会有自然科学的发展,也不会有靠余暇或靠富人从非劳动者那里取得自己“剩余产品”的等价物的欲望才能存在的精神产品。

小册子的作者和莱文斯顿说这些话并不是为资本辩护,而是以此作为攻击资本的出发点,因为所有这一切都纯粹是违背工人的利益而不是为了工人。但是他们这样实际上也就承认这是资本主义生产的结果,承认资本主义生产因而是社会发展的一种历史形式,尽管这种历史形式是和构成整个这一发展基础的那一部分人口的利益相矛盾的。在这方面他们(虽然是从相反的一极出发)也具有政治经济学家们的局限性,即把这一发展的对立形式和这一发展的内容本身混淆起来。一些人为了这种对立的成果而希望这种对立永世长存。另一些人则为了摆脱对立而决心牺牲在这种对立形式范围内产生的成果。这就使这种对于[资产阶级]政治经济学的反对不同于同一时期的欧文等人,另一方面也不同于为了摆脱尖锐形式的对立而想回到古老的对立形式的西斯蒙第。

[莱文斯顿写道:]

\begin{quote}{“穷人的贫困创造了他的〈富人的〉财富……如果一切人都是平等的,那末谁也不会为别人劳动。必需品将会有余,而奢侈品将会绝迹。”(第10页)“生产产品的劳动是财产的父亲,帮助别人消费产品的劳动是财产的孩子。”(第12页)“财产的增加,维持有闲者和非生产劳动的能力的增长,这就是政治经济学上称为资本的东西。”(第13页)“因为财产的使命就是花费,因为没有花费,财产对于它的所有者来说就完全是无用的东西,所以财产的存在是和消费劳动的存在[863]密切地联系在一起的。”(同上)“如果每个人的劳动刚够生产他自己的食物,那就不会有任何财产了,也就不会有任何一部分人民的劳动用来满足想象的需要了。”(第14—15页)“在社会发展的每一个阶段上,随满每个人的劳动生产率由于人口的增长和技术设备的改良而提高,劳动的人数会逐渐减少……财产由于生产资料的改良而增加;财产的唯一使命就是鼓励懒散。当每一个人的劳动勉强够维持他自己的生活的时候,因为不可能有财产,所以不会有有闲者。如果一个人的劳动能够养活五口人,那末一个从事生产的人就将负担四个有闲者的生活,因为只有这样产品才能消费掉……社会的目标就是牺牲勤劳者来抬高有闲者,从富裕中创造出实力。”(第11页)}\end{quote}

{莱文斯顿关于地租所说的话(不完全正确,因为正是在这里须要说明为什么地租落到土地所有者手里而不落到租地农场主,产业资本家手里)适用于因劳动生产力增长而发展的一般剩余价值:

\begin{quote}{“在社会发展的早期阶段,人们还没有人造的辅助手段来促进他们的劳动生产力,他们的收益中可以作为地租支出的部分是极小的;因为土地没有自然价值,它的全部产品都靠劳动。但是劳动技能每有提高,都会增加可以用来支付地租的那部分产品。在维持十个人的生活需要九个人的劳动的地方,总产品中只有1/10可以用作地租。在一个人的劳动足够维持五个人的生活的地方,就会有4/5的产品用作地租或用于国家的只能由劳动的剩余产品来满足的其他需要。前者似乎是英国被征服时期的情况,后者就象英国的现状,现在只有1/5的人口从事农业。”(第45—46页)“社会把每一种改良都只用来增加懒散,这种情况是千真万确的。”(第48页)}}\end{quote}

注。莱文斯顿的著作是独特的。它的直接的题目,正如书名写的,是现代的公债制度。其中莱文斯顿谈到:

\begin{quote}{“反对法国革命[后来反对拿破仑]的全部战争,除了把一些犹太人变成绅士和把一些笨蛋变成政治经济学家以外,没有作出任何高尚的事情。”(第66—67页)“公债制度也有一个好的结果:尽管它从国内老贵族那里夺取其一大部分财产,以便把这些财产转交给新出现的西班牙式绅士,作为对他们的欺骗和盗窃国库的高超手段的奖励……既然它鼓励欺骗和卑鄙行为,给招摇撞骗和自命不凡披上智慧的外衣,把全体人民变成从事证券投机的民族……既然它破坏了关于等级和门第的一切偏见,使货币成为人与人之间唯一的区别标志……它也就破坏了财产的永恒性。”(第51—52页)}\end{quote}

\tchapternonum{(3)霍吉斯金}

《保护劳动反对资本的要求,或资本非生产性的证明。关于当前雇佣工人的团结》,一个工人著,1825年伦敦版。

托马斯·霍吉斯金《通俗政治经济学。在伦敦技术学校的四次演讲》1827年伦敦版。

第一部匿名的著作也是霍吉斯金写的。如果说前面谈到的那些小册子以及其他许多类似的小册子都无声无息地过去了,那末霍吉斯金的这两部著作,特别是第一部著作,却引起了强烈的反应,至今仍然可算是(参看约翰·莱勒《货币和道德》1852年伦敦版[第XXIV页和第319—322页])英国政治经济学方面的重要著作。这里我们将依次考察这两部著作。

\tsectionnonum{[(a)资本的非生产性的论点是从李嘉图理论中得出的必然结论]}

在《保护劳动反对资本的要求,或资本非生产性的证明》这本小册子中,作者正象书名所表明的那样,想证明“资本的非生产性”。

李嘉图从来没有断言资本就生产价值的意义来说是生产的。李嘉图认为,资本加到产品上的只是它自己的价值,而它自己的价值则取决于它的再生产所需要的劳动时间。它只是作为“积累劳动”(更确切地说,作为[864]物化劳动)才具有价值,它只是把它的这一价值加到它所加入的产品中去。的确,李嘉图在一般利润率问题上犯了前后矛盾的错误。而这也正是他的反对者们用来抓住他的一个矛盾。

至于谈到生产使用价值意义上的资本的生产性,那末,在斯密、李嘉图等人看来,一般说,在政治经济学家看来,它不过是指过去的有用劳动的产品重新用作生产资料,即劳动对象、劳动工具和工人的生活资料。劳动的客观条件不是象在原始状态下那样表现为简单的自然物(作为简单的自然物,它们从来不会是资本),而是表现为已被人类的活动改造过的自然物。但是在这个意义上,“资本”这个词完全是多余的,是什么也说明不了的。小麦可以食用并不是因为它是资本,而是因为它是小麦。羊毛的使用价值是它作为羊毛所固有的,而不是作为资本所固有的。同样,蒸汽机的作用同它作为资本的存在毫无共同之处。如果蒸汽机不是“资本”,不属于工厂主而属于工人,它会提供同样的服务。在实际的劳动过程中,所有这些物之所以能提供服务,是因为它们作为使用价值,而不是作为交换价值,更不是作为资本,同加到它们身上的劳动发生关系。它们在这里是生产的,或者更确切地说,劳动生产率实现在作为自己的物质的它们身上,原因在于它们作为实际劳动的客观条件的属性,而不在于它们作为独立地同工人相对立、同工人相异化的条件、作为体现在资本家身上的活劳动支配者的那种社会存在。按照霍普金斯(不是我们的霍吉斯金)的正确说法\endnote{霍普金斯的著作《论地租及其对生存资料和人口的影响》(1828年伦敦版第126页)中有关的段落,马克思在《对所谓李嘉图地租规律的发现史的评论》一章中引用过(见本卷第2册第151页)。——第291页。},它们在这里是作为财富,而不是作为“纯”财富,是作为产品,而不是作为“纯”产品被消费和被使用的。固然,在政治经济学家的头脑里,也和在资本家的头脑里一样,这些物在同劳动的关系上所采取的一定的社会形式,是和它们作为劳动过程的因素的实际特性交织在一起,并且彼此不可分割地结合在一起的。但是,政治经济学家们一到着手分析劳动过程的时候,他们便不得不把“资本”这个用语完全抛开,而去谈论劳动材料、劳动资料和生活资料。但是在产品作为材料、工具和工人的生活资料这种特性中所反映的只是它们作为物的条件同劳动的关系;劳动本身在这里表现为支配它们的活动。在这方面绝对没有劳动和资本的关系,而只有人类合乎目的的活动在再生产过程中同它自己的产品的关系。它们仍然是劳动产品,仍然不过是劳动自由支配的对象。它们只是表示一种关系,在这种关系的范围内,劳动把它本身所创造的,至少是在这种形式上所创造的物的世界占为己有;但是,除了表示活动必须和它的材料相适应外,它们决不表示这些物对劳动的任何其他支配权,否则它就不是合乎目的的活动,就不是劳动了。

只有把资本看作一定的社会生产关系的表现,才能谈资本的生产性。但是如果这样来看资本,那末这种关系的历史暂时性质就会立刻显露出来,对这种关系的一般认识,是同它的继续不断的存在不相容的,这种关系本身为自己的灭亡创造了手段。

但是政治经济学家们没有把资本看成是这样一种关系,因为他们不敢承认它的相对性质,也不理解这种性质;相反,他们只是从理论上反映了为资本主义生产所束缚的、受资本主义生产支配的、同资本主义生产有利害关系的实际家们的观念。

霍吉斯金本人在[反对资产阶级政治经济学的]论战中是从政治经济学家们的狭隘的观念出发的。既然这些政治经济学家把资本描述成永恒的生产关系,他们就把资本归结为劳动对它的物质条件的一般关系,这种一般关系是一切生产方式所共有的,并不包含资本的特殊性质。在他们认为资本创造“价值”时,他们中间一些最优秀的人——[特别是]李嘉图——都承认,它除了创造它从劳动中过去得到以及现在不断得到的价值以外,并没有创造别的价值,因为一个产品所包含的价值是由再生产它所必需的劳动时间决定的,即由产品作为活的现在劳动的结果而不是过去劳动的结果这样一种关系决定的。而劳动生产率的增长,正象李嘉图着重指出的那样,恰好表现在过去劳动的产品的不断贬值上。另一方面,政治经济学家们经常把这些物借以表现为资本的一定的特殊形式同它们作为物以及作为一切劳动过程的简单因素的属性混为一谈。作为“劳动的使用者”\endnote{马克思指英国的流行的说法“capitalemployslabour”(“资本使用劳动”),这种说法反映了资本与雇佣劳动的关系的实质本身。马克思在1861—1863年手稿第XXI本上在揭示“资本的生产性”的含义时写道“……生产资料,劳动的物的条件——劳动材料、劳动资料(以及生活资料)——[在资本主义生产方式下]也不是从属于工人,相反,是工人从属于它们。不是工人使用它们,而是它们使用工人。正因为这样,它们才是资本”(见本卷第1册第419页;并参看第1册第73页、第2册第479页和本册第122—123页)。——第292页。}的资本所含的奥妙,他们却没有说明,他们只是不断无意识地把这种奥妙说成是某种同资本的物的性质不可分离的东西。

[867]\endnote{手稿这个地方的页码弄乱了:在第864页之后,正文转入第867页,接着是第868、869、870、870a页,然后是第865、866页,最后是第870b、871、872页等等。从不衔接的一页转到另一页,是马克思自己指出的。——第292页。}第一本小册子\fnote{指匿名小册子《根据政治经济学基本原理得出的国民困难的原因及其解决办法》。——编者注}从李嘉图的理论中作出了正确的结论,把剩余价值归结为剩余劳动。这一点同李嘉图的反对者和追随者所做的相反,因为他们死抱住李嘉图对剩余价值和利润的混淆不放。

第二本小册子\fnote{指莱文斯顿的小册子《论公债制度及其影响》。——编者注}也和李嘉图的反对者和追随者相反,更准确地规定了取决于劳动生产力发展程度的相对剩余价值。李嘉图也谈到了这一点,但是他避开了莱文斯顿所作的结论:劳动生产力的提高只是增加了别人的、支配劳动的财富即资本。

最后,第三本小册子\fnote{指霍吉斯金的匿名小册子《保护劳动反对资本的要求,或资本非生产性的证明》。——编者注}把作为李嘉图论述问题的必然结果的总的论点表达出来了:资本是非生产的。这一点是针对托伦斯、马尔萨斯等人的,因为这些人继续发展李嘉图学说的一个方面,把他的“劳动是价值的创造者”这个论点变成了“资本是价值的创造者”这样一个相反的论点。同时,在这本小册子里还反驳了劳动绝对取决于作为劳动存在条件的现有资本量这个论点,这个论点贯穿在从斯密到马尔萨斯的著作中,特别是被马尔萨斯(也被詹姆斯·穆勒)奉为绝对教条。

第一本小册子是以这样的命题结束的:

\begin{quote}{“财富就是可以自由支配的时间,如此而已。”\fnote{见本册第279—282页。——编者注}}\end{quote}

\tsectionnonum{[(b)反驳李嘉图的资本是积累劳动的定义。关于并存劳动的见解。对物化的过去劳动的意义估计不足。现存财富同生产运动的关系]}

霍吉斯金认为“流动资本”只不过是不同种类的社会劳动的并存(“并存劳动”),而积累只不过是社会劳动生产力的积累,所以工人本身的技能和知识(科学力量)的积累是主要的积累,比和它一同进行并且只是反映它的那种积累,即这种积累活动的现存客观条件的积累,重要得多,而这些客观条件会不断重新生产和重新消费,只是名义上进行积累:

\begin{quote}{“生产资本和熟练劳动……是一个东西……资本和工人人口完全是一个意思。”[《保护劳动反对资本的要求》第33页]}\end{quote}

这一切都只是加利阿尼的命题的进一步发展:

\begin{quote}{“真正的财富是……人”(《货币论》,库斯托第编,现代部分,第3卷第229页)。}\end{quote}

整个客观世界,“物质财富世界”,在这里不过是作为从事社会生产的人的因素,不过是作为从事社会生产的人的正在消失而又不断重新产生的实践活动而退居次要地位。请把这种“理想主义”同李嘉图的理论在“这个不可相信的修鞋匠”\endnote{“这个不可相信的修鞋匠”(《thisincrediblecobbler》)——《对麦克库洛赫先生的,〈政治经济学原理〉的若干说明》这一小册子的作者对麦克库洛赫的称呼。见前面正文第203页。——第260、294页。}麦克库洛赫的著作中变成的粗野的物质拜物教比较一下,在他的著作中,不仅人和动物的区别不见了,甚至连有生物和物之间的区别也不见了。让人们还去说什么在崇高的资产阶级政治经济学的唯灵论面前,无产阶级反对派所鼓吹的只是以满足鄙俗的需要为目的的粗野的唯物主义吧!

霍吉斯金的错误在于:在研究资本的生产性时,他没有区别在什么程度上涉及到使用价值的生产,在什么程度上涉及到交换价值的生产。

其次(但是从历史上说这是有它的理由的),他所考察的资本是他在政治经济学家们那里发现的资本。一方面,资本(在它加入实际劳动过程的情况下)被说成是劳动的单纯的物的条件,或者说只具有劳动的物质要素的意义;而且(在价值形成过程中)只不过是用时间来计算的一定的劳动量,也就是和这一劳动量本身没有什么区别的东西。另一方面,尽管资本在它出现在实际生产过程中的情况下,实际上不过是劳动本身的名称,是劳动的别名,它却被说成是支配劳动和决定劳动的力量,是劳动生产率的基础,是同劳动无关的财富。而这是没有任何中介过程的。这就是霍吉斯金在他的前辈那里发现的东西。他针对资产阶级关于经济发展的欺人之谈,阐述了这一发展的真实情况。

\begin{quote}{“资本是一种神秘的词,就象教会或者国家,或者由宰割其他人的人为了掩盖拿刀的手而发明的普通术语中的其他任何词一样。”(《保护劳动》第17页)}\end{quote}

其次,霍吉斯金按照他在政治经济学家们那里发现的传统,区分了流动资本和固定资本,而且他所理解的流动资本,主要是指流动资本中由工人的生活资料构成的或用作这种生活资料的部分。

\begin{quote}{“政治经济学家们断言,没有过去的资本积累,分工是不可能的”……但是“那些被认为由名叫流动资本的商品储备产生的结果,是由并存劳动引起的”。(第8—9页)}\end{quote}

面对着政治经济学家们的粗陋的理解,霍吉斯金有权说“流动资本”只是特殊“商品”的“储备”的“名称”。因为政治经济学家们没有说明在商品的形态变化中表现出来的特殊社会关系,所以他们只能从物质上去理解“流动”资本。从流通过程中产生的资本的一切[868]区别,——其实是资本流通过程本身,——事实上都不过是作为再生产过程因素的(由于同雇佣劳动的关系而取得资本性质的)商品的形态变化。

从某种意义上说,分工无非是并存劳动,即表现在不同种类的产品(或者更确切地说,商品)中的不同种类的劳动的并存。在资本主义的意义上,分工就是生产某种商品的特殊劳动分为一定数量的简单的、在不同工人之间分配而又相互联系的工序,它以行业划分这种社会内部即作坊外部的分工为前提。另一方面,作坊内部的分工又扩大了社会内部的分工。产品本身越片面,它所交换的商品越多样化,表现它的交换价值的使用价值的系列越大,它的市场越大,产品就越能在更充分的意义上作为商品来生产,它的交换价值就越不取决于它作为使用价值的直接存在,或者说,它的生产就越不取决于它的生产者对它的消费,越不取决于它作为它的生产者的使用价值的存在。情况越是这样,产品就越能作为商品来生产,因而也就越能大量地进行生产。产品的使用价值对产品生产者无关紧要这一事实,会在产品生产的总量中在量上表现出来,即使该产品的生产者同时又是他自己产品的消费者,这个总量同该产品生产者的消费需要也没有任何关系。但是,作坊内部的分工是这种大规模生产的方法之一,因而也是[作为商品的]产品的生产方法之一。因此,作坊内部的分工是以社会内部的行业划分为基础的。

市场的大小有两层意思:第一,消费者的数量,他们的人数;第二,也包括彼此独立的行业的数量。即使前者的数量不增加,后者的数量也可能增加。例如,当纺纱和织布从家庭工业和农业中分离出来时,所有土地耕种者就都成了纺纱者和织布者的市场。同样,后两者由于他们的行业划分现在也互为市场。社会内部分工的前提首先是不同种类劳动的相互独立,即它们的产品必须作为商品相互对立,并且通过交换,完成商品的形态变化,作为商品相互发生关系。(因此,在中世纪城市禁止农村从事尽可能多的职业。其目的不仅是为了排除竞争,——亚·斯密在这里只看到这一点,——而且是为了给自己开辟市场。)另一方面,社会内部的分工要得到适当的发展,就必须以一定的人口密度为前提。作坊内部分工的发展更是以这种人口密度为前提。以前一种分工的一定发展程度为前提的这后一种分工,又从自己这方面同前一种分工相互发生作用,并增进前一种分工,因为它把从前相互有联系的行业分为彼此独立的行业,增加和分化它们所间接需要的准备工作,同时,由于生产和人口的增加,资本和劳动的游离,它还创造出新的需要和满足这些需要的新的方法。

因此,霍吉斯金说“分工”不是被称为流动资本的商品储备的结果,而是“并存劳动”的结果,如果这里他所说的分工是指行业划分,那就是同义反复。这只是意味着分工是分工的原因或结果。因此,霍吉斯金所指的只能是:作坊内部的分工是以行业划分、社会分工为条件,并且在一定意义上是社会分工的结果。

不是“商品储备”造成这种行业划分,从而造成作坊内部的分工,而是上述行业划分(和分工)表现在商品的储备上,或者更确切地说,表现在产品的储备变为商品的储备这一点上。{但在政治经济学家们的著作中,总是不可避免地把资本主义生产方式的属性、特征,即资本本身的属性、特征(就资本表示生产者相互之间以及同自己产品之间的一定关系而言),说成是物的属性。}

[869]但是,如果在经济意义上(见杜尔哥、斯密等的著作)说“过去的资本积累”就是分工的条件,那末,这是指作为资本的商品储备事先在劳动的买者手里的积聚,因为作为分工的一个特点的这种协作形式,是以工人的集结,因而以他们在劳动期间所需的生活资料的积累为前提,以劳动生产率的提高,因而以劳动不断进行所必需的原料、工具和辅助材料数量的增加为前提(因为劳动不断需要大量这些东西),一句话,以大规模生产的客观条件为前提。

在这里,资本积累不可能指“作为分工条件的生活资料、原料和劳动工具数量的增加”,因为,如果把资本积累理解为这种资料的积累,那它就是分工的结果,而不是分工的前提。

在这里,资本积累也不可能意味着在新的生活资料生产出来以前,工人的生活资料一般就必须具备,或者说,工人已经生产出来的劳动产品必须用作新的生产的原料和劳动资料。因为这是一般的劳动条件,在分工发展以前也同样是如此。

一方面,从物质要素的观点来看,积累在这里无非是指:分工使生活资料和劳动资料的积聚成为必要,而这些生活资料和劳动资料在以前,当劳动者在各个行业中(在这种假定下,行业不可能是很多的)自己一个接一个地去完成生产一种或几种产品所需要的所有不同工序的时候,是零星分散的。这里的前提不是绝对的增加,而是积聚:把较大量的生活资料和劳动资料集结在一点,而且比集结在一起的工人人数相对地多。例如,在工场手工业中工人所需要的亚麻(与其人数相比)就比一切以副业方式纺麻的农民和农妇所需要的多。因此就有工人的集结以及原料、工具和生活资料的积聚。

另一方面:在产生这一过程的历史基础上(工场手工业,即以分工为特点的工业生产方式,就是从中发展起来的),这种积聚只能以这样的形式进行,即这些工人作为雇佣工人,作为被迫出卖自己劳动能力的工人集结在一起,因为他们的劳动条件作为别人的财产,作为别人的力量独立地同他们相对立,而这一点包括:这些劳动条件作为资本同他们相对立,也就是上述生活资料和劳动资料,或者同样可以说,依靠货币而拥有的对它们的支配权,掌握在单个的货币所有者或商品所有者手里,他们因此变成了资本家。劳动者丧失劳动条件表现为这些劳动条件作为资本离开劳动者而获得独立,或者说,表现为资本家支配这些劳动条件。

所以,象我指出的那样\endnote{马克思引用他在当时(1862年10月)还没有写成的关于原始积累的一节,按照马克思的计划(见本卷第1册第446页),这一节应当放在《剩余价值理论》一节之前。这一节的初稿包括在1857—1858年经济学手稿里(见卡·马克思《政治经济学批判大纲》1939年莫斯科版第363—374页)。——第299页。},原始积累无非是那些作为同劳动和工人对立的独立力量的劳动条件的分离。历史的过程使这种分离成为社会发展的因素。既然资本已经存在,那末,这种分离的保持和再生产就从资本主义生产方式本身中以越来越大的规模发展起来,直到发生历史变革。

使资本家成为资本家的不是对货币的占有。要使货币转化为资本,必须具备资本主义生产的前提,上述分离就是资本主义生产的第一个历史前提。在资本主义生产本身的范围内,这种分离,因而作为资本的劳动条件的存在,是既定的;这是生产本身的不断再生产出来和不断扩大的基础。

积累现在通过把利润,或者说剩余产品,再转化为资本而成为经常的过程,因此,数量已经增加了的、同时是劳动的客观条件、再生产条件的劳动产品,经常作为资本,作为从劳动异化出来的、支配劳动的和在资本家身上个性化了的力量同劳动相对立。但是这样一来,积累,即把一部分剩余产品再转化为劳动条件,就成了资本家的特殊职能。愚蠢的政治经济学家由此得出结论说:这种事情如果不在这种对抗的特殊形式上进行,就根本不可能进行。在他的脑子里,扩大规模的再生产是和这种再生产的资本主义形式——积累——分不开的。

[870]积累只是把原始积累中作为特殊的历史过程,作为资本产生的过程,作为从一种生产方式到另一种生产方式的过渡出现的东西表现为连续的过程。

政治经济学家们为资本主义生产代理人的观念所束缚,陷入了双重的、但是互为条件的概念的混淆。

一方面,他们把资本从一种关系变成一种物,变成“商品储备”(这时他们已经忘掉商品本身不单纯是物),这些商品由于被用作新劳动的生产条件而被称为资本,并按其再生产方式被称为流动资本。

另一方面,他们又把物变成资本,即把表现在物上并通过物表现的社会关系,看成物本身只要作为要素加入劳动过程或工艺过程就具有的属性。

因此,[一方面,]作为支配劳动的力量,作为分工的先决条件的原料和对生活资料的支配权在不劳动者手里的积聚(后来,分工不仅使积聚增多,而且由于劳动生产力的提高,也使被积聚的总量增多),就是说,作为分工条件的资本的预先积累,在政治经济学家们看来意味着生活资料和劳动资料的量的增加或积聚(他们没有区别这两者)。

另一方面,在他们看来,如果生活资料和劳动资料不具有成为资本的属性,如果构成劳动条件的劳动产品不消费劳动本身,如果过去劳动不消费活劳动,如果这些物属于工人而不属于自己本身或受委托的资本家,那末,这些生活资料和劳动资料就不会作为生产的客观条件起作用。

如果劳动条件属于联合起来的工人,如果这些工人同劳动条件的关系,就象同自然的劳动条件的关系一样,也就是象同他们自己的产品和他们自己活动的物的要素的关系一样,那末,分工似乎就不是同样可能的(虽然分工在历史上不可能从一开始就以它只有作为资本主义生产发展的结果才能表现出来的那种形式出现)。

其次,因为在资本主义生产的条件下,资本占有工人的剩余产品,因为资本已经占有的那些劳动产品现在因此而以资本的形式同工人相对立,所以很明显,剩余产品转化为劳动条件,只能从资本家那里开始,并且只能采取这样的形式,即资本家把不付等价物而占有的劳动产品变成获取新的不付等价物的劳动的生产资料。因此,扩大再生产就表现为利润转化为资本,表现为资本家的节约,资本家不是把他无代价地得到的剩余产品吃光,而是把它重新变为剥削劳动的手段,但是要做到这一点,只能通过把它重新转化为生产资本,其中也包括把剩余产品转化为劳动资料。因此,政治经济学家得出结论说:如果剩余产品事先不从工人的产品转化为他的雇主的财产,以便以后重新用作资本并重复过去的剥削过程,剩余产品就不能充当新的生产的要素。一些蹩脚的政治经济学家把贮藏和货币贮藏的观念也归入这一点。甚至一些优秀的政治经济学家,如李嘉图,也把关于禁欲的观念从货币贮藏者那里移到资本家身上。

政治经济学家们没有把资本看作是一种关系。他们不可能这样看待资本,因为他们没有同时把资本看作是历史上暂时的、相对的而不是绝对的生产形式。霍吉斯金本人也没有这样来理解资本。只要这样的理解是为资本辩护的话,它就不是为政治经济学家们对资本的辩护进行辩护,而是相反地否定他们的辩护。因此,霍吉斯金同对资本的这种看法没有任何关系。

就霍吉斯金和政治经济学家们之间存在的情况来看,他的论战的性质看来是预先确定了的并且是很简单的。霍吉斯金本来只是应该借助政治经济学家们“科学地”发展了的一个方面,来反对他们不加考虑地、无意识地和天真地从资本主义的思想方式接受来的拜物教观念,并且大致这样说:

如果工人想利用自己的产品来进行新的生产,那就必须把过去劳动的产品(一般说,劳动产品)当作材料、工具和生活资料来使用。他的产品的这种一定的消费方式是生产性的。但是,对工人的产品的这种使用,工人消费自己产品的这种方式,同这种产品对工人本身的支配,同这种产品作为资本的存在,同原料和生活资料的集中掌握[870a]在个别资本家手中,以及同工人被剥夺了对他们产品的所有权,究竟有什么关系呢?这同工人首先必须白白地把自己的产品交给第三者,以便后来用自己的劳动再从第三者那里把它赎回来,为此他不得不付给第三者比产品里包含的劳动更多的劳动来交换这一产品,并且这样来为资本家创造新的剩余产品,又有什么关系呢?

在这里,过去劳动表现在两种形式上。第一,表现为产品,使用价值。生产过程要求工人把这一产品的一部分[作为生活资料]消费,而把另一部分用作原料和劳动工具。这一点属于工艺过程,它只是表明,工人为了把他们的产品变成生产资料,他们在工业生产中应当怎样对待他们自己的劳动产品,怎样对待他们自己的产品。

第二,过去劳动表现为价值。这一点只是表明工人的新产品的价值不只是代表他们的现在劳动,而且代表他们的过去劳动,表明工人以自己的劳动扩大旧价值,同时正因为他们扩大了旧价值,于是就保存了旧价值。

资本家的要求同这一过程本身没有任何关系。当然,既然资本家占有劳动产品,占有过去劳动的产品,他就因此拥有占有新产品和活劳动的手段。但这正好是引起抗议的行动方式。“分工”所必需的预先的积聚和积累恰恰不一定表现为资本的积累。从它们是必需的这一点出发,决不能得出结论,说资本家必须支配那些由昨天的劳动为今天的劳动创造的条件。如果资本的积累[根据政治经济学家们的意见]无非就是劳动的积累,那末这决不包含它必须是别人劳动的积累这样一种意思。

但是霍吉斯金没有走这条简单的道路,初看起来这是很奇怪的。在反对资本的生产性(首先反对流动资本的生产性,但是更反对固定资本的生产性)的论战中,他好象是在反对或者否定过去劳动本身或它的产品作为新劳动的条件对再生产的重要性,也就是反对或者否定过去的、物化在产品中的劳动对于作为当前正在进行的活动的劳动的重要性。这样的转变是怎样引起的呢?

因为政治经济学家们把过去劳动同资本等同起来——过去劳动在这里既从具体的、物化在产品中的劳动的意义上来理解,也从社会劳动,即物化劳动时间的意义上来理解,——所以很明显,他们作为资本的品得\fnote{歌颂者、赞美者(品得是古希腊诗人)。——编者注},当然会把生产的物的要素提到首位,并且同主观要素即活的、直接的劳动相比,过高地估计物的要素的意义。在他们看来,只有当劳动成为资本,当它和自身相对立,当它的被动的一面和它的能动的一面相对立的时候,它才是适合的。因此,产品支配生产者,物支配主体,已实现的劳动支配正在实现的劳动,等等。在所有这些见解当中,过去劳动不是仅仅表现为活劳动的物的因素,从属于活劳动的物的因素,而是相反;不是表现为活劳动的权力要素,而是表现为支配这种劳动的权力。为了也从工艺上为特殊的社会形式即资本主义形式(在这种形式中,劳动和劳动条件的相互关系被颠倒了,以致不是工人使用这些条件,而是劳动条件使用工人)辩护,政治经济学家们赋予劳动的物的因素以一种和劳动本身相对立的虚假的重要性。正因为这样,霍吉斯金才相反地坚持认为,这种物的因素——从而一切物化财富——同活的生产过程比较起来,是极不重要的,它实际上只是作为活的生产过程的因素才具有价值,而它本身是没有任何价值的。这里,霍吉斯金有点低估过去劳动对现在劳动的意义,不过这一点在反对政治经济学家们的拜物教时是很自然的。

如果在资本主义生产中,从而在它的理论表现上,即在政治经济学上,过去劳动只表现为劳动本身给劳动创造的基础等等,那末这种争论便不可能发生。争论之所以存在,只是因为在资本主义生产的现实生活中,以及在它的理论中,物化劳动表现为同劳动本身的对立,同活劳动的对立。正象在受宗教束缚的思维过程中,思维的产品不仅要求支配思维本身,而且实现了这种支配一样。

[865]因此,霍吉斯金的命题

\begin{quote}{“那些被认为由名叫流动资本的商品储备产生的结果,是由并存劳动引起的”(第9页)}\end{quote}

其意思首先是说:

活劳动的同时并存,引起了大部分被认为由名叫流动资本的过去劳动产品产生的结果。

例如,流动资本的一部分是由生活资料的储备构成的,资本家积累这些生活资料,照政治经济学家们的说法,是为了在工人劳动时维持工人的生活。

因为在资本主义生产的条件下,生产和消费都最大,所以在市场上(在流通领域)商品量也最大,虽然如此,储备的形成却根本不是资本主义生产的特点。在资本家积累生活资料的储备这一见解中,仍然流露出对货币贮藏者所实现的积累即贮藏的回忆。

这里首先应当把消费基金撇开,因为这里谈的是资本和工业生产。一切属于个人消费范围的东西,无论它消费得较快还是较慢,都不再成为资本{虽然其中一部分可能再转化为资本,例如房屋、停车场、容器等等}。

\begin{quote}{“当时,欧洲所有的资本家是否都拥有供给他们所雇用的全部工人一个礼拜的食物和衣服呢?让我们首先来考察食物问题。人民的一部分食物是面包,它经常只是在食用以前几小时才烤出来……面包业主的产品不能贮藏。做面包的原料,无论是小麦还是面粉,没有不断的劳动就根本不能保存……纺纱工人确信在需要面包的时候就能得到面包,他的雇主确信他付给工人的钱能使工人买到面包,这些都不过是由以下事实产生的:在需要面包的时候,总是可以得到面包。”(第10页)“工人的另一种食物是牛奶,而牛奶的生产……一天两次。如果说乳牛已经有了,那末对于这一点应当这样来回答:它需要经常的照料和经常的劳动,它的饲料在一年的大部分时间里都是饲料作物每天生长的结果。它放牧的田野需要人手……肉类的情况也是一样。肉类不能贮藏,因为肉类刚一上市,就已经开始要坏。”(第10页)甚至拿衣服来说,由于怕虫蛀,“衣服的储备,同衣服的总消费比较起来,只是一个很小的数量”。(第11页)“穆勒说得对:‘一年内生产出来的东西一年内就被消费掉’,所以实际上不能积累起使人们能够完成持续一年以上的全部工作所需的商品储备。因此,从事这些工作的人不应当指望已经生产出来的商品,而应当指望由其他人劳动和生产出他们在完成自己产品的劳动期间为自己生存所必需的东西。所以,即使工人同意,为了在一年内完成的工作,必须积累一些流动资本……那末很明显,在进行持续一年以上的全部工作的过程中,工人不指望也不可能指望积累的资本。”(第12页)“如果我们适当地注意到那些创造财富的、不能在一年内完成的工作的数量和重要性,也注意到维持生存所必需的、无数的、每天劳动的产品,而这些产品在生产出来以后又立即被消费掉,那末,我们就会懂得,每一项不同种类劳动的成效和生产力取决于其他人的并存生产劳动的程度,总是比取决于流动资本的任何积累的程度大。”(第13页)“资本家能够养活,并因而雇用其他劳动者,不是由于他拥有商品储备,而是由于他有支配一些人的劳动的权力。”(第14页)“可以说储存起来和预先准备好的唯一的东西,就是工人的技能。”(第12页)“通常被认为是由流动资本的积累产生的一切结果,都是由于熟练劳动的积累和储存,这种最重要的工作,对大部分工人来说,不要任何流动资本也可以完成。”(第13页)“工人人数总是必须取决于流动资本的量,或者,照我的说法,取决于允许工人消费的并存劳动的产品的量。”(第20页)[866]“流动资本……只是为了消费才创造出来;固定资本……不是为了消费,却是为了帮助工人生产消费的物品而生产出来。”(第19页)}\end{quote}

所以,我们首先指出:

\begin{quote}{“每一项不同种类劳动的成效和生产力取决于其他人的并存生产劳动的程度,总是比取决于流动资本的任何积累(即“已经生产出来的商品”)的程度大。”这些“已经生产出来的商品”是和“并存劳动的产品”对立的。}\end{quote}

{在每一单个的生产部门内部,资本中归结为劳动工具和劳动材料的部分总是作为“已经生产出来的商品”而成为前提。不能纺还没有“生产出来的”棉花,不能使尚待制造的纱锭转动,不能烧还未从矿井里开采出来的煤。因此,它们总是作为过去劳动的存在形式加入[生产]过程。在这个意义上,现存的劳动取决于以前的劳动,而不只是取决于并存劳动,尽管这种以前的劳动,无论以劳动资料还是劳动材料的形式出现,总是只有作为活劳动的物的因素(仅仅作为生产消费即劳动消费的因素)同活劳动相接触,才具有某种用处(生产上的用处)。

但是在考察流通和再生产过程时,我们同时还看到,商品被制造出来并转化为货币以后,它之所以能再生产出来,只是因为它的一切要素被“并存劳动”同时生产和再生产出来\endnote{马克思在《剩余价值理论》的前几章批判地分析亚·斯密和大·李嘉图的观点时谈到了再生产过程的基本要素。特别提到某种商品的一切要素必须同时生产和再生产出来,(见本卷第1册第96—98和136—137页以及第2册第538—539、552和553页)。——第307页。}。

在生产中有两种运动。我们拿棉花作为例子。它从一个生产阶段转到另一个生产阶段。最初它作为子棉生产出来,然后经过许多道工序,直到适合出口,或者,如果是在本国进一步加工,它就要直接转到纺纱者手里。然后,它从纺纱者手里转到织布者手里,从织布者手里转到漂白者、染色者、整理者手里,从他们手里又转到各种各样为了专门目的而把它加工为衣服、床单等等的工厂。最后,如果不是作为劳动资料(不是材料)进入生产消费,它便从最后的生产者手里转到消费者手里,即转为个人消费。但是这样一来,无论是为了生产消费还是个人消费,棉花都取得了它的使用价值的最终形式。在这里作为产品从一个生产领域出来的东西,又作为生产条件进入另一个生产领域,这样经过连续的阶段,直到最后制成为使用价值。在这里过去劳动不断表现为现在正在进行的劳动的条件。

但是在产品这样地从一个阶段转到另一个阶段,在它完成这一现实的形态变化的同时,它又在每一个阶段上被生产出来。当织布者在加工纱,纺纱者在纺棉花的时候,新的子棉又处在自己的生产过程当中。

因为不断的、重新开始的生产过程就是再生产过程,所以它同样是由并存劳动决定的,当产品完成自己的形态变化,从一个阶段转到另一个阶段的时候,并存劳动就同时生产出产品的不同阶段。棉花、棉纱和布——所有这一切不只是一个在一个之后,一个由另一个生产出来,而且也是同时并行地生产出来和再生产出来。当我在考察单个商品的生产过程时,表现为以前的劳动的结果的东西,在我考察该商品的再生产过程时,也就是说,当我从该商品的不断进行的生产过程,从这个生产过程的条件的总和,而不只是从一个孤立的行为或有限的空间来考察该商品的生产过程时,就同时表现为并存劳动的结果。这不只是经过不同阶段的循环,而且是商品在其属于特殊生产领域和形成不同劳动部门的一切阶段上的并行生产。如果同一个农民先种亚麻,然后把它纺成纱,再把它织成布,那末这些工序就有连续性,但是没有同时性,而同时性则以建立在社会内部分工基础上的生产方式为前提。

如果从单个商品的生产过程中的某一阶段来考察单个商品的生产过程,那末以前的劳动,固然,只是由于它为之提供生产条件的活劳动才具有意义。但是另一方面,这些生产条件(没有它们,活劳动就不能实现)总是作为以前的劳动的已完成的结果加入这一过程。因此,提供生产条件的那些劳动部门的协作劳动总是表现为被动的,并且作为这种被动的因素而成为前提。政治经济学家们都强调这一方面。相反,在再生产和流通中,每一个特殊领域的商品生产过程所依靠的和作为其先决条件的社会中介劳动,则表现为现在的、并存的、同时的劳动。商品以它的最初形式和它的已完成形式或连续形式同时生产出来。没有这一点,商品在完成了它的现实的形态变化以后,就不能从货币再转化为它的生存条件。[870b]因此,商品只有同时表现为同时的活劳动的产品,它才是以前的劳动的产品。从这个意义上说,资本家所认定的全部物质财富只是包括流通过程在内的总生产源流中的一种迅速消逝的因素。}

\tsectionnonum{[(c)]所谓积累不过是一种流通现象(储备等是流通的蓄水池)}

霍吉斯金只是从流动资本的一个组成部分来考察流动资本。但是一部分流动资本会不断转化为固定资本和辅助材料,只有另一部分才转化为消费品。而且,即使那部分最终转化为供个人消费的商品的流动资本,除了它作为从终结阶段出来的最终产品所具有的最后形式外,在它较早的各个生产阶段,也一直同时以还不能进入消费的最初形式存在,也就是以在不同程度上有别于产品最终形式的原料或半成品的形式存在。

霍吉斯金所谈的问题是,工人现在给资本家提供的劳动与由工资转化成的物品(这些物品实际上就是构成可变资本的使用价值)所包含的劳动之间有什么关系。必须承认,如果没有这些供消费的物品,工人就无法劳动。所以政治经济学家们说,流动资本——过去劳动,资本家积累的已经生产出来的商品——是劳动的条件,其中也包括分工的条件。

谈到生产条件,特别是谈到霍吉斯金所说的流动资本,通常是说,在工人生产出新的商品之前,也就是在工人劳动期间,在工人自己生产的商品还处在形成状态的时候,资本家就应当积累起工人消费所需的生活资料。这里透露出一种看法,即认为资本家就象货币贮藏者那样从事积累,或者说,他就象蜜蜂采蜜那样收集生活资料的储备。

但是这只不过是一种说法而已。

首先,我们这里谈的不是做生活资料买卖的零售商。他们当然经常要有充足的商品储备。他们的栈房、店铺等只不过是蓄水池,商品在可以进入流通之后就分配在这里。这种积累不过是商品从流通转入消费之前所处的中间阶段。这是商品作为商品在市场上的存在。其实,它作为商品也只有以这种形式存在。至于它是不是已经不在第一个卖者(生产者)手里,而是在第三个或第四个卖者手里,它是不是最终转入把它卖给真正消费者的卖者手里,这对问题毫无影响。这只关系到:在中间阶段商品代表着资本(其实是资本加利润,因为生产者在商品中出卖的不仅是资本,而且还有他的资本所赚得的利润)同资本的交换,在最后阶段商品代表着资本同收入的交换(就是说,如果商品象在这里假设的那样预定不转入生产消费,而转入个人消费)。

已经最后成为使用价值并已进入可以出卖状态的商品,作为商品处于市场,处于流通阶段;一切商品,当它们必须完成它们的第一形态变化,即转化为货币时,都处于这个阶段。如果这叫作“积累”,那末积累就无非是商品作为商品的“流通”或存在。因此,这种“积累”就会同货币贮藏正好相反,因为货币贮藏是要使商品永远保持在这种可以流通的状态,而这也只有以货币的形式把商品从流通中抽出来才能办到。如果生产,从而还有消费,都是多种多样和大规模的,那末就会有大量的各种各样的商品经常处于这种停顿状态,即处于这种中间阶段,一句话,处于流通中,或者说,处于市场上。所以,如果从量的方面来考察,那末,大量的积累在这里无非是指大量的生产和大量的消费。

商品的停顿——商品停留在过程的这一时刻,它存在于市场而不存在于工厂或私人家里(作为消费品),即存在于商人的店铺、栈房中——只是[871]它生命过程中的一个很短暂的时刻。这种“财物世界”,“实物世界”的静止的、独立的存在只是一种表面现象。驿站始终客满,但始终都是新的旅客。同样的商品(同一种商品)不断地在生产领域中重新生产出来,出现在市场上并被消费掉。它们,不是同一些商品,而是同一种商品,始终同时存在于这三个阶段上。如果中间阶段延长,以致新商品从生产领域出来时,市场还是被旧商品占据着,那末就会产生停滞,阻塞;出现市场商品充斥,商品贬值;出现生产过剩。所以,流通的中间阶段在什么地方成为一种独立的存在,而不只是向前运动的源流中一个短暂的停留,以及商品在流通阶段的存在在什么地方表现为积累[Aufhaufung],这绝不是生产者的一种自由行动,绝不是生产的目的或者生产的内在的生命因素,正如血液涌向头部引起中风并不是血液循环的内在因素一样。资本作为商品资本(在这个流通阶段,在市场上,它就是以这种形式出现的)不应该停滞不动,而应该只是在运动进程中作短暂的停留。否则再生产过程就会遭到破坏。整个机构就会紊乱。所以,这种在个别点上以集中形式出现的物质财富同生产和消费的持续不断的源流相比,是微不足道的,也只能是微不足道的。因此,斯密也认为,财富是“年度的”再生产。所以,它所注明的日期不是什么遥远的过去,而只不过是昨天。另一方面,如果再生产由于受到某些干扰而停顿下来,那末仓库等等就会空起来,就会出现匮乏,就立刻会显示出:现存财富看起来所具有的那种经常性不过是它的更替、它的再生产的经常性,是社会劳动的不断的物化。

在商人那里也存在着W—G—W的过程。商人从中获取“利润”这一点,在这里和我们没有关系。他出卖商品,又购买同样的商品(同一种商品)。他把商品卖给消费者,又从生产者那里把商品买进来。同样的商品(同一种商品)在这里不断地转化为货币,货币又不断地再转化为同样的商品。但是这种运动只不过是不断的再生产,即不断的生产和消费;因为再生产包含着消费。(为了能够进行商品的再生产,商品就必须卖掉,必须加入消费。)商品必须用事实证明自己是使用价值。(因为对于卖者来说是W—G,对于买者来说就是G—W,也就是货币转化为作为使用价值的商品。)再生产过程既然是流通和生产的统一,它就包含着本身是流通因素的消费。消费本身就是再生产过程的因素和条件。如果就整个过程来考察,商人向生产者购买商品所支付的货币,实际上是消费者向商人购买商品所用的货币。对于生产者来说,商人代表消费者,而对于消费者来说,商人就代表生产者;他是同一商品的买者和卖者。他用来购买商品的货币,纯粹从形式上看,实际上就是消费者的商品的终结形态变化。消费者把他的货币转化为作为使用价值的商品。所以,货币转入商人之手就意味着商品的消费,或者从形式上看,意味着商品从流通转入消费。只要商人再用这些货币向生产者购买,这就是生产者的商品的第一形态变化,表示商品转入中间阶段,在这个阶段它作为商品停留在流通中。只要W—G—W这个过程是商品转化为消费者的货币,并且是现在为商人所有的货币再转化为同样的商品(同一种商品),那末这一过程就无非表示商品不断地转入消费,因为进入消费的商品所空出的位置为此就必须由从生产过程出来现在进入这一中间阶段的商品所填补。

[872]商品在流通中停留以及它被新商品所取代,当然还要取决于商品处在生产领域的时间的长度,因而取决于商品再生产时间的长度,随着这种时间长度的不同,商品停留的时间也不同。例如,谷物的再生产需要一年时间。例如,今年(1862年)秋季收获的谷物,只要不再用作种子,就必须足够供来年全年——直到1863年秋——的消费。它立即被投入流通(即使在农场主的粮仓里,它也是已经处于流通中了),在这里它被流通的各种蓄水池——仓库、谷物商、磨坊主等等——所吸收。这些蓄水池既是生产的排水渠,又是消费的引水渠。只要商品处于蓄水池中,它就是商品,因而就处于市场上,处于流通中。它只是点点滴滴地被年消费从流通中抽出。把它排挤出去的新商品所进行的补充,新商品的源流,只有在一年以后才会到来。因此,这些蓄水池也只是随着对已消费的商品的补充的到来而逐渐地变空。如果还有剩余,如果新的收成超过平均收成,那末就会发生阻塞。这种一定的商品在市场上占有的空间就会显得充斥。为了都能在市场上给自己找到位置,商品就会降低自己的市场价格,这样就会使它们重新运动起来。如果商品作为使用价值的量太大,那末它们就会通过降低自己的价格的办法来适应它们应占的空间。如果这个量太小,那末它们就会用提高自己的价格的办法来扩大自己。

另一方面,作为使用价值会迅速坏掉的那些商品,在流通的蓄水池中也只有瞬息间的停留。它们必须转化为货币和必须被再生产出来的时间,是由它们的使用价值的性质所规定的,这种使用价值如果不是每天或几乎每天被消费掉,就会坏掉,因而也就不再是商品。因为如果使用价值的消失本身不是生产行为,交换价值就会和它的承担者即使用价值一起消失。

一般说来很清楚,虽然聚集在流通蓄水池中的商品的绝对量会随着国民经济的发展而增长,但是由于生产和消费的增长,这个量同年生产和年消费的总量相比,还是会减少。商品从流通到消费的转移会加速,而且是由于一系列的原因。再生产的速度在下列场合会加快:

(1)商品迅速地通过它的各个生产阶段,生产过程在每个生产阶段缩短;这取决于商品在它的每一种形式上的生产所必需的劳动时间的缩短;所以,这是和分工、机器、化学过程的应用等等的发展同时发生的。{随着化学的发展,人为地加速了商品从一种聚集状态到另一种聚集状态的转变,加速了它和其他物体的结合,例如染色;加速了它和其他物质的分离,例如漂白,——一句话,无论是同一些物质的形式(它们的聚集状态)的变化,还是必然产生的物质变换,都人为地加速了;至于会给植物和动物提供较便宜的物质,即花费很少劳动时间的物质,以进行植物性的和有机的再生产,那就更不用说了。}

(2)部分地由于不同生产部门的联合,即由于形成了把一定生产部门联合起来的生产中心,[部分地]由于交通工具的发展,商品迅速地从一个生产阶段转到另一个生产阶段;换句话说,缩短了间歇期间,减少了商品在一个生产阶段和另一个生产阶段之间的中间阶段的停留时间,或者说,缩短了从一个生产阶段到另一个生产阶段的转移。

(3)所有这些发展——各个不同生产阶段的缩短以及从一个阶段到另一个阶段的转移的加快——都是以大规模生产,大量生产为前提,同时也以大量不变资本,特别是固定资本基础上的生产为前提;因而也以生产的不断进行为前提,所谓不断,不是指我们刚才考察这种不断进行时所说的不断,即不是指通过各个生产阶段的彼此接近和相互渗透而形成的不断,而是指在生产中不会发生有意的中断。这种中断在为订货而生产的情况下总是会发生的,就象在[873]手工业者那里出现的那样,在本来意义上的工场手工业中(只要工场手工业本身尚未被大工业改造)也还是那样。而现在生产是按资本所容许的规模进行的。这个过程并不等待需求,而是资本的一种职能。资本不断以同样的规模(且不说积累或扩大)进行工作,同时生产力不断发展和提高。因此,生产不仅进行得很快,使得商品很快就获得适合于流通的形式,而且是不断地进行。生产在这里仅仅表现为不断的再生产,同时也是大量的生产。

因此,如果商品长期滞留在流通的蓄水池中,如果商品积存在这里,那末,由于生产浪潮迅速地一个接着一个涌来,由于它们不断注入流通蓄水池大量材料,这些蓄水池很快就会充斥。例如,柯贝特正是在这个意义上说:“市场总是商品充斥。”\endnote{柯贝特关于市场总是商品充斥和关于供给总是超过需求的看法,在他的著作《个人致富的原因和方法的研究;或贸易和投机原理的解释》1841年伦敦版第115—117页上作了说明。——第315页。}可是造成再生产这样迅速、这样大量的这些情况,也会减少商品在这些蓄水池中聚集的必要性。就生产消费来说,这种情况已部分地包含在商品本身或其组成部分所必须通过的各个生产阶段的彼此接近中。如果煤炭每天大量生产,并且经由铁路、轮船等运送到工厂主的大门口,那末工厂主就不需要储备煤,或者只需要储备少量的煤,或者,如果有一个商人介入其中,情况也是一样,这个商人除了每天卖出和每天得到补给的以外,也只需要有很少的储备。纱、铁等的情况也是如此。可是,把生产消费(在生产消费领域中,商品储备,即商品各组成部分的储备,必然会这样减少)撇开不谈,[经营个人消费品的]商人也同样有:第一,迅速的交通工具,第二,可靠的、不断的、迅速的更新和供给。因此,虽然他的商品储备在数量上可能增加,但是这种储备的每一个要素存在于他的蓄水池,即存在于这种过渡状态的时间会缩短。同他出卖的全部商品量相比,也就是同生产量和消费量相比,他的仓库在每个一定时刻所保存的、聚集的商品储备是不大的。在生产比较不发达的阶段,情况就不一样了,在这些阶段再生产进行缓慢,——因而必然有较多的商品滞留在流通的蓄水池中,——交通工具缓慢,联络困难,因此储备的更新往往发生中断,从蓄水池变空到它重新装满,即商品储备的更新,这中间要经过很长的间歇期间。这时就会发生和下述产品类似的情况:这些产品由于其使用价值的性质,它们的再生产要经过一年或半年,总之,要经过比较长的期间才能实现。

{交通工具对于蓄水池变空所产生的影响,可以棉花为例来说明。由于利物浦和美国之间经常有船舶来往,——交通的迅速是一个因素,经常性是另一个因素,——所以用不着把全部棉花一下子运出去。棉花可以逐渐上市。(生产者也不希望商品一下子充斥市场。)棉花存在利物浦货栈内,诚然已经是在流通的蓄水池中,但是其数量——同这种商品的总消费量相比——已不象在船舶要经半年的航程、一年只从美国开来一两次时所需要的那样多了。曼彻斯特的工厂主等可以大致根据他直接消费的多少来充实他的仓库,因为有了电报和铁路,就有可能随时把棉花从利物浦运到曼彻斯特。}

流通蓄水池的特殊的充满现象(不是由于市场负担过重造成的充满,市场负担过重在这种情况下发生,要比在宗法式的生产速度缓慢的情况下容易得多)只是投机性的,只有在与价格的实际涨落或预料中的涨落有关的例外场合才会发生。

关于储备的这种相对减少,即处在流通中的商品量同生产和消费的总量相比而言的相对减少,见莱勒的著作、《经济学家》\endnote{《经济学家》(《TheEconomist》)是英国经济、政治问题周刊,1843年起在伦敦出版,大工业资产阶级的机关刊物。——第316页。}、柯贝特的著作(有关的引文放在霍吉斯金之后)。[874]西斯蒙第错误地认为这是值得遗憾的事(也请参阅他的著作)\endnote{西斯蒙第在他的《政治经济学概论》1837年布鲁塞尔版第一卷第49页及以下各页谈到随着贸易和交通工具的发展现有商品储备减少的问题。——第316页。}。

(诚然,另一方面,我们也会看到市场的不断扩大,随着商品在市场停留的间歇期间的缩短,空间的范围相应扩大,或者说,市场在空间上相应扩大,以商品生产领域为中心画出的圆的半径越来越大。)

“挣多少吃多少”的消费者,改变衣着就象改变意见一样迅速,而不是一件上衣等等一穿就是十年,这种情况和再生产的速度有关,或者说,不过是再生产速度的另一种表现。甚至那些不受使用价值的性质制约的物品的消费,也越来越在时间上和生产趋于一致,因而也越来越依附于现在劳动,并存劳动(因为实际上这里是并存劳动的交换),这一切都是同过去劳动越来越成为生产的重要因素的程度相适应的,虽然这种过去本身总是很近的,而且只是相对的。

(下面一个例子说明储备的建立同生产的不发展是多么紧密相联。在牲畜很难过冬的时候,冬天就没有鲜肉。一旦畜牧业克服了这一困难,由于必须以腌肉或熏肉代替鲜肉而产生的储备也就会自行停止。)

产品只有在它进入流通的场合,才成为商品。产品作为商品的生产,因而还有流通,会由于以下原因随着资本主义的生产而异常扩大:

(1)大规模的生产,量,大批,也就是同生产者[对他自己的产品]的需要在数量上丝毫没有关系的生产;事实上,他是不是哪怕在最小的程度上消费自己的产品,这纯粹是偶然的。生产者只有在他生产自己资本的一部分构成要素时,才会大量地消费自己的产品。相反,在社会发展的较早阶段,只是——或者主要是——超过自己需要的多余的产品才成为商品。

(2)同需要的日益增加的多样性成反比的产品的质的单一性。这一点会引起以前彼此联系着的生产部门较大程度的分离和独立,一句话,会使社会内部的分工增多,此外还会引起新的生产部门的建立和商品种类的多样性的增加。(最后,在论述霍吉斯金之后,还要列举威克菲尔德对这个问题的看法。)商品的这种多样化即分化,有两类。第一,同一产品的不同阶段,以及加在产品上的中间劳动(也就是生产它的构成要素等的劳动)分化为不同的彼此独立的劳动部门;换句话说,同一产品在它的不同阶段转化为不同种类的商品。第二,由于有劳动和资本(或者说,劳动和剩余产品)游离出来,另一方面,由于发现利用同一使用价值的新方法[从而出现新的种类的商品]。由于第一点中所谈到的那些变化,于是产生新的需要(例如,随着蒸气在工业中的利用,就出现对迅速和全面的交通工具的需要),因而也产生满足这些需要的新方法,——或者是发现利用同一使用价值的新方法,或者是发现新原料,或者是发现对旧原料进行不同处理的新加工方法(如电铸术),等等。

这一切归结为一个产品在其一个接一个的阶段或者说状态中转化为不同的商品,或者归结为创造作为商品的新产品或者说新的使用价值。

(3)以前以实物形式消费\fnote{这里指自然经济条件下的消费。——编者注}大量产品的人口中的大多数转化为雇佣工人。

(4)租地农民转化为产业资本家{地租随之转化为货币地租,总之,所有的实物交纳(赋税等,地租)转化为货币支付}。总之:土地以工业方式经营,因而它的化学和机械的生产条件,甚至种子等等,牲畜等等,肥料等等都要新陈代谢,而不象从前那样只限于使用自己的粪肥。

(5)大量以前“不可让渡的”财物的变卖使它们转化为商品,仅仅由流通券构成的财产形式被创造出来。一方面是地产的让渡(在广大群众变得连任何财产都没有的情况下,也出现了他们例如把自己的住房当作商品的现象)。另一方面是铁路股票,简言之,各种各样的股票。

\tsectionnonum{[(d)霍吉斯金对资本家为工人“积累”生活资料的见解的驳斥。霍吉斯金不了解资本拜物教化的真正原因]}

[875]现在我们回过头来谈霍吉斯金。

所谓资本家为工人“积累”[生活资料],当然不是指商品从生产转入消费时存在于流通蓄水池中,存在于流通中,存在于市场上这种情况。如果这样来解释这种“积累”,那就等于说,产品是为了工人而流通,为了工人而成为商品,总之,产品作为商品的生产是为了工人而进行的。

同其他任何人[商品所有者]一样,工人必须首先把他实际上(虽然不是在形式上)出卖的商品即他的劳动转化为货币,然后才能把这些货币再转化为供消费的商品。非常明显,如果没有消费品以及生产资料作为商品存在于市场,那末,分工(既然它以商品生产为基础),雇佣劳动,总之,资本主义生产,就不可能存在;如果没有商品流通,没有商品停留在流通蓄水池中,这种生产就不可能进行。因为真正说来,产品只有在流通中才是商品。工人必须在商品形式上取得他的生活资料,这对他来说,就象对其他任何人一样。

此外,工人与经营生活资料的商人相对立不是作为工人与资本家相对立,而是作为货币与商品,作为买者与卖者相对立。这里不存在雇佣劳动与资本的关系,除非是涉及商人自己的工人。然而即使是这样的工人,只要他们是向商人购买,他们就不是作为工人与商人相对立。只有在商人向他们购买时才会发生这种情况。所以我们要把这种流通的当事人撇开。

至于工业资本家,那末构成他的储备,即他的“积累”的是:

第一,他的固定资本——建筑物、机器等等,这些东西工人是不消费的,或者,他如果消费,那是在劳动过程中为资本家生产地消费;这些东西虽然是工人的劳动资料,但绝不是工人的生活资料。

第二,他的原料和辅助材料;不直接加入生产的那部分原料和辅助材料的储备,正如我们所看到的,有减少的趋势。这些东西也不是工人的生活资料。资本家为工人进行这种“积累”,不过是表示资本家为工人效劳,从工人那里夺走他的劳动条件的所有权,并把他的这些劳动资料(这些东西本身不过是他的劳动的转化了的产品)变为剥削劳动的手段。当工人把机器和原料当作劳动资料使用时,他无论如何不是靠它们生活的。

第三,他的进入流通以前存在于仓库、货栈中的商品。这些商品是劳动的产品,而不是在生产期间为维持劳动自身而积累的生活资料。

因此,资本家为工人“积累”生活资料,不过是表示资本家必须拥有足以支付工资的货币,工人用这些货币从流通蓄水池中取得自己的消费资料(如果就整个阶级来考察,就是买回工人自己的一部分产品)。但是这些货币不过是工人所出卖和提供的那种商品的转化形式。从这个意义上说,生活资料是为工人“积累”,如同生活资料是为他的资本家积累一样,因为资本家也用货币(同一商品的转化形式)购买消费资料等等。这些货币可以是单纯的价值符号;所以它们根本不一定是“过去劳动”的代表,而只是在每个人手中表示他所实现的价格——不是过去劳动(或以前的商品)的价格,而是这个人所出卖的同时劳动或商品的价格。是单纯的形式存在\endnote{这里马克思把货币描述为“单纯的形式存在”(《blossesFormdasein》)是在如下的意义上:货币的使用价值“虽然是实际存在的,但在[交换]过程本身中却表现为单纯的形式存在,它还需要通过转化为真正的使用价值才得到实现”(见《马克思恩格斯全集》中文版第13卷第37页)。——第321页。}。或者说,因为在以前的生产方式下工人也必须吃饭,而且不管他的产品的生产时间有多长,在生产时他总得消费生活资料,所以为工人“积累”生活资料,就是指工人必须首先把自己的劳动产品转化为资本家的产品,转化为资本,然后才能以货币的形式再拿回一部分这样的产品作为报酬。

[876]在这一过程中(对于这一过程本身说来,工人得到的是同时劳动的产品还是过去劳动的产品,是并行劳动的产品还是自己以前的产品,实际上完全是无关紧要的)使霍吉斯金感兴趣的是下面一点:

工人每天消费的产品(不管他自己的产品是否已经制成,他都必须消费)的一大部分以至绝大部分决不是以前的积累劳动。相反,这在很大程度上是工人在他生产自己的商品的同一天、同一周所生产的劳动产品。面包、肉、啤酒、牛奶、报纸等等就是这样。霍吉斯金也许还会说,其中有一部分是未来劳动的产品,因为工人要用六个月内积攒的工资来购买只是在这六个月的末尾才制成的上衣等等。(我们已经看到,全部生产都以加入其中的各组成部分和表现为原料、半成品等不同形式的产品的同时再生产为前提。一切固定资本则以未来劳动作为其再生产的前提,它也要以未来劳动作为再生产自己等价物的前提,没有这个等价物,它就不能进行再生产。)霍吉斯金说,在一年之内,工人(由于谷物的再生产的性质,由于植物性原料等的生产的性质)不得不在一定程度上“指靠”过去劳动。{例如关于住房就不能这样说。有的使用价值,因其性质,只是逐渐磨损,它不是一下子被消费掉,而只是被使用,在这种情况下,以前的劳动的这种产品存在于“市场”,决不是为工人而想出的某一特别行动的结果。工人在资本家为他“积累了”脏得要命的贫民窟之前,就早已“有了住房”。(关于这一点见兰格的著作\endnote{马克思指(小)赛米尔·兰格的著作《国家的贫困,贫困的原因及其防止办法》1844年伦敦版第149—154页。马克思在《资本论》第一卷第二十三章注115中引了这本书里描写资本主义大城市中工人居住条件极为恶劣的一段话。——第322页。}。)}(且不说特别对工人具有决定意义的大量日常需要,而工人是几乎只能满足自己的日常需要的,——我们已经看到,生产和消费一般说来在时间上越来越趋于一致,所以,如果就整个社会来考察,社会全体成员的消费就越来越依赖于他们的同时生产,或者更确切地说,依赖于同时生产的产品。)但是如果劳动操作延续若干年,工人就只得“指靠”自己的生产,“指靠”生产其他商品的工人的同时劳动和未来劳动。

工人总是必须在市场上取得作为商品的生活资料(因而他所购买的这些“服务”只是在它们被购买时才被创造出来),因此,这些生活资料相对地说是以前的劳动(即在它们作为产品存在之前就存在的劳动,但决不是在工人自己的劳动——即工人用其价格购买这些产品的劳动——之前存在的劳动)的产品。这些生活资料可能是与这种劳动在时间上一致的产品,对于“挣多少吃多少”的人来说,它们在大多数场合正是这样的产品。

如果考虑到这一切,那末资本家为工人“积累”生活资料可归纳为如下几点:

(1)商品生产的前提是,人们可以在市场上取得他们自己所不生产的作为商品的消费品,或者说,商品一般作为商品被生产出来。

(2)工人消费的绝大部分商品,在其作为商品同工人相对立的最后形式上,实际上是同时劳动的产品(因此,它们根本不是由资本家积累的)。

(3)在资本主义生产的条件下,工人自己生产的劳动资料和生活资料是同工人相对立的,前者作为不变资本,后者作为可变资本同他相对立;他的所有这些生产条件都表现为资本家的财产;而这些生产条件从工人手里转到资本家手里以及工人的产品或其产品的价值部分地流回到工人手里,就叫作为工人“积累”流动资本。工人在他的产品完成之前总是必须消费的这些生活资料所以成为“流动资本”,是因为工人不是以自己过去的产品的价值或未来的[877]产品来直接购买生活资料或者进行支付,而是必须先从资本家那里得到领取生活资料的凭证,即货币;资本家只是由于工人过去生产的、将来生产的或现在生产的产品才能够发给这种凭证。

霍吉斯金在这里力图证明工人是依靠其他工人的并存劳动,而不是依靠过去劳动,

(1)以便消除“积累这个用语”,

(2)因为“现在劳动”是同资本相对立的,而“过去劳动”则一直被政治经济学家们看作就是资本,是一种异化的、同劳动本身敌对的、独立的劳动形式。

但是对同时劳动,普遍地从它与过去劳动相对立的意义上来理解它,这本身就是一个十分重要的因素。

所以,霍吉斯金得出如下的结论:

资本或者仅仅是一种名称和托词,或者它表现的不是物,而是关系:一个人的劳动同其他人的并存劳动的社会关系,这种关系的后果,结果,被认为是由构成所谓流动资本的物造成的。商品在作为货币的一切存在上能否实现为使用价值,取决于同时劳动。(全年的[劳动]本身就是同时的[劳动]。)只有一小部分加入直接消费的商品是一年以上的产品,即使它们是这样的产品(例如牲畜等等),它们每年也需要新的劳动。所有需要一年以上时间的劳动操作都是建立在继续不断的年生产的基础上。

\begin{quote}{“资本家能够养活,并因而雇用其他劳动者,不是由于他拥有商品储备,而是由于他有支配一些人的劳动的权力。”(第14页)}\end{quote}

然而是货币给每个人以“权力”,去支配“一些人的劳动”,支配已经物化在他们的商品中的劳动,以及支配这种劳动的再生产——在这个限度内也就是支配劳动本身。

在霍吉斯金看来,真正“积累”起来的,但不是作为死的物质,而是作为活的东西“积累”起来的,是工人的技能,是劳动的发展程度。{诚然(霍吉斯金没有强调这一点,因为和政治经济学家们的粗陋见解相反,对他来说重要的是把重点放到与物相对立的主体上,也可以说放到主体中的主观方面),每一特定时刻所具有的、作为出发点的劳动生产力发展程度,不仅以工人的技能和能力的形式存在,而且同时存在于这种劳动为自己创造的、并且每天都在更新的物质工具之中。}这是形成出发点的真正的前提,而且这个前提是一定发展进程的结果。积累在这里就是把已承受下来的、被实现了的东西加以同化、继续保存并进行改造。正是在这个意义上,达尔文把通过一切有机体即植物和动物的遗传而进行的“积累”看作促使有机体形成的动因;这样,不同的有机体本身就是通过“积累”而形成,并且只是活的主体的“发明”,是活的主体的逐渐积累起来的发明。但是对生产来说,这并不是唯一的前提。对动物和植物来说,这种前提就是它们外部的自然界,——因而既包括无机的自然界,也包括它们同其他动植物的关系。在社会上从事生产的人,也同样遇到一个已经发生变化的自然界(特别是已经转化为他自己活动的工具的自然要素)以及生产者彼此间的一定关系。这种积累一部分是历史过程的结果,一部分就单个工人来说是技能的代代相传。霍吉斯金说,在这种积累的情况下,任何流动资本都不会对大多数工人有什么帮助。

霍吉斯金指出,“商品〈生活资料〉储备”同总消费和生产比较起来单是不大的。而现有人口的熟练程度却始终都是总生产的前提,因而是财富的主要积累,是以前劳动的被保存下来的最重要的结果,不过这种结果是存在于活劳动本身中的。

\begin{quote}{[878]“通常被认为是由流动资本的积累产生的一切结果,都是由于熟练劳动的积累和储存,这种最重要的工作,对大部分工人来说,不要任何流动资本也可以完成。”(第13页)}\end{quote}

政治经济学家们说,工人人数(从而现有工人人口的幸福或贫困)取决于现有的流动资本量,对于这种说法霍吉斯金正确地作了如下的评论:

\begin{quote}{“工人人数总是必须取决于流动资本的量,或者,照我的说法,取决于允许工人消费的并存劳动的产品的量。”(第20页)}\end{quote}

被认为由“流动资本”、由某种“商品储备”造成的东西,是“并存劳动”的结果。

所以,霍吉斯金用另外的话说:劳动的一定社会形式的作用被认为是由物,由这一劳动的产品造成的;关系本身被幻想为物的形式。我们已经看到,这是以商品生产,以交换价值为基础的劳动所固有的特点,这种混淆表现在商品上和货币上(霍吉斯金没有看到这一点),而且更多地表现在资本上。\endnote{关于商品、货币和资本的拜物教性质,马克思在《政治经济学批判》第一分册(见《马克思恩格斯全集》中文版第13卷第21—25、37—39和144—146页)中谈到过。——第326页。}物作为劳动过程的物的因素所产生的作用,被认为是由这些物在资本中造成的,就象这些物在自己的人格化中,在和劳动对立的自己的独立性中所具有的作用一样。假如它们不再以这种异化的形式和劳动相对立,它们[在政治经济学家们看来]就不再能够产生这种作用。资本家作为资本家只不过是资本的人格化,是具有自己的意志、个性并与劳动敌对的劳动产物。霍吉斯金认为这纯粹是主观的幻想,在这种幻想后面隐藏着剥削阶级的欺诈和利益。他没有看到这种表述方法是怎样从现实关系本身中产生的,没有看到后者不是前者的表现,而是相反。英国的社会主义者就是在这个意义上说:“我们需要的是资本,而不是资本家”。\endnote{大约在这些关于霍吉斯金的论述以前半年,马克思在未写完的关于英国社会主义者布雷的一节中顺便引用了布雷这样几句话:“对生产者的操作具有重大意义的不是资本家,而是资本。资本和资本家之间的区别就象船上装的货物和提货单之间的区别一样大。”(见本册第356页)。——第326页。}但是如果他们排除了资本家,他们也就使劳动条件丧失了资本性质。

\centerbox{※     ※     ※}

{《评政治经济学上若干用语的争论》一书的作者、贝利和其他人指出\fnote{见本册第138—139和176页。——编者注},“value,valeur”\fnote{价值。——编者注}这两个词表示物的一种属性。的确,它们最初无非是表示物对于人的使用价值,表示物的对人有用或使人愉快等等的属性。事实上,“value,valeur,Wert”\fnote{价值。——编者注}这些词在词源学上不可能有其他的来源。使用价值表示物和人之间的自然关系,实际上是表示物为人而存在。交换价值则代表由于创造交换价值的社会发展后来被加在Wert(=使用价值)这个词上的意义。这是物的社会存在。

\begin{quote}{“梵文Wer的意思是‘掩盖、保护’,由此有‘尊敬、敬仰’和‘喜爱、珍爱’的意思。从这个词派生的形容词Wertas是‘优秀的,可敬的’意思;哥特文wairth,古德文wert,盎格鲁撒克逊文weorth,vordh,wurth,英文worth,worthy,荷兰文waard,waardig,德文wert,立陶宛文wertas(“可敬的,有价值的,贵重的,受器重的”)。梵文Wertis,拉丁文virtus\fnote{力量,优点,优秀的品质。——编者注},哥特文wairthi,德文Wert。”[夏韦《试论哲学词源学》1844年布鲁塞尔版第176页]}\end{quote}

物的Wert\fnote{价值。——编者注}事实上是它自己的virtus\fnote{力量,优点,优秀的品质。——编者注},而它的交换价值却和它的物的属性完全无关。

\begin{quote}{“梵文Wal的意思是‘掩盖,加固’;[拉丁文]vallo\fnote{用堤围住,加固,保护。——编者注},valeo\fnote{成为有力的,坚固的,健康的。——编者注};val-lus\fnote{堤。——编者注}——起掩护和保护作用的东西;valor——是力量本身。”由此有[法文]valeur,[英文]value;“请把Wal同德文walle,walte\fnote{我支配,我照料,我管理。——编者注},英文wall\fnote{墙。--编者注},wield\fnote{掌握,拥有。--编者注}作一比较。”\endnote{马克思在1864年6月16的信里告诉恩格斯说,这些不同的印欧语词的对照,是从“一个比利时词源学家”那里引来的,而从这封信中可以看出,马克思自己不相信这些对照都有充分根据。“一个比利时词源学家”就是《试论哲学词源学》(1844年布鲁塞尔版)一书的作者奥诺莱·约瑟夫·夏韦。在引自夏韦著作的第二段引文中法文“valeur”和英文“value”是马克思自己加上去的。——第327页。}[夏韦《试论哲学词源学》1844年布鲁塞尔版第70页]}}\end{quote}

\centerbox{※     ※     ※}

接着霍吉斯金转到固定资本。这是被生产出来的生产力,是在大工业里,在这种资本的发展过程中,由社会劳动为自己创造的工具。

下面是关于固定资本的一段话:

\begin{quote}{“所有的工具和机器都是劳动产品……当它们只是过去劳动的结果而不由工人加以适当使用时,它们就不能补偿制造它们的费用……如果它们闲置不用,其中大部分就会失去价值……固定资本之所以有用不是由于过去劳动,而是由于现在劳动,它给自己的所有者提供利润不是因为它被积累,而是因为它是获得对劳动的支配权的手段。”(第14—15页)}\end{quote}

这里终于正确地抓住了资本的性质。

\begin{quote}{[879]“各种工具制成以后,它们本身能生产什么呢?什么也不生产。相反,如果它们不由劳动利用或使用,它们就会开始生锈和毁坏……是否应当把某一工具看成是生产资本,这完全要看它是否被某个生产工人所使用。”(第15—16页)“很容易理解,为什么……道路修建者应当得到一部分只有道路使用者才能从道路得到的利益;但是我不理解,为什么所有这些利益都应当属于道路本身,并且由那些既不修建道路也不使用道路的人以他们的资本的利润为名据为己有。”(第16页)“蒸汽机的巨大效用并不是取决于铁和木料的积累,而是取决于对自然力的实际的活的知识,这种知识使一些人能够制造机器,使另一些人能够操纵机器。”(第17页)“没有知识,它们〈机器〉就不可能发明,没有机器制造工人的灵巧和技能,它们就不可能制造出来,而没有技能和劳动,它们就不能在生产上使用。但是知识、技能和劳动却是资本家能够据以要求获得产品的一个份额的唯一因素。”(第18页)“当人们把若干代人的知识继承下来并且大群地生活在一起时,他们就有可能用他们的智力来完成自然界所做的事情。”(第18页)“一个国家的生产劳动不是取决于固定资本的量,而是取决于固定资本的质。”(第19页)“作为供养和维持人的生活的手段的固定资本,在其效率方面完全取决于工人的熟练程度,因此一个国家的生产劳动,就固定资本来说,是和人民的知识和技能成比例的。”(第20页)}\end{quote}

\tsectionnonum{[(e)]复利;根据复利说明利润率的下降}

\begin{quote}{“只要略微看一看,任何人都会相信,随着社会的发展,简单利润不会减少,只会增加,也就是说,同量劳动,前一时期生产100夸特小麦和100台蒸汽机,现在会生产更多一些……实际上我们看到,在我们国内现在靠利润过富裕生活的人比过去多得多。然而很清楚,任何劳动,任何生产力,任何发明才能,任何技术,都不能满足复利的压倒一切的要求。但一切积蓄都是从资本家的收入中来的〈也就是从“简单利润”中来的〉,因此,这些要求实际上不断地提出,而劳动生产力同样不断地拒绝满足它们。因此,不断有一种平衡创造出来。\endnote{在霍吉斯金的小册子里,紧接这段话的一个句子说明,霍吉斯金在这里说的“有一种平衡创造出来”,指的是:“资本家允许工人有生存资料,因为他们没有工人的劳动不行,而且他们宽宏大量地满足于占有产品中不是为实现这一目的”(即保证体力的最低工资)“所必需的每一个细小部分”。——第329页。}”(第23页)}\end{quote}

例如,如果利润不断重新积累起来,资本100,按10%计算,过20年后就是约673,因为小的差数在这里没有什么意义,所以我们也可以说是700。这样一来,资本在20年内就增加了六倍。照这样的规模,如果仅仅是单利,资本每年应该提供的就不是10%,而是30%,也就是说提供大两倍的利润,我们把年数增加得越多,在计算每年的单利的时候利息率或利润率就提高得越多,资本越大,这种提高也就总是越快。

但是,事实上资本主义积累无非是利息再转化为资本(因为这里对于我们的目的,即对于这种计算的目的来说,利息和利润被看作是等同的),——因而是复利。今天资本是100;它产生利润(或利息)10。把它加到资本上,得110,这就是现在的资本。因此,它提供的利息就不只是资本100的利息,而是(100K+10Z)的利息,即复利。这样,在第二年末就是(100K+10Z)+10Z+1Z=(100K+10Z)+11Z=121。现在这就是第三年开始时的资本。在第三年是:

(100K+10Z)+11Z+12.1Z,于是资本在第三年末便是133.1。

[880]我们在复利上加上一撇,就得出下表:

\todo{}

换句话说,在第九年就已经有一半以上的资本[这时资本等于214.358881]是由利息构成的,可见资本中由利息构成的部分是按几何级数增加的。

我们看到,二十年后资本就会增加六倍,然而即使按照马尔萨斯的“最极端的”假定,人口也只能在二十五年中增加一倍。但是我们且假定,人口在二十年中增加一倍,因而工人人口也增加一倍。如果算出每年的平均结果,那末利息应当是30%,比它原来大两倍。但是在剥削率不变的情况下,在二十年中已增加一倍的人口(在这二十年的很大一部分时间中,新的一代还不能劳动;尽管有儿童参加劳动,这新的一代在这个期间也几乎有一半时间不能劳动)只能比以前完成多一倍的劳动,因而也只能完成多一倍而不是多两倍的剩余劳动。

利润率(因而还有利息率)是这样决定的:

(1)假定剥削率不变,利润率决定于在业工人人数,决定于所使用的工人的绝对量,因而决定于人口的增长。虽然所使用的工人的绝对量增加了,但是随着资本的积累和工业的发展,它对所使用的资本的总额的比率却降低了(因此,在剥削率不变的情况下,利润率会下降)。同样,整个人口也绝对不会象复利那样按照几何级数增长。在工业发展的一定阶段,人口的增长可以说明剩余价值量和利润量的增加,但同时又可以说明利润率的下降。

(2)利润率决定于正常工作日的绝对量,即剩余价值率的提高。因此,利润率能够由于劳动时间超出正常工作日以外的延长而提高。但是这有它的身体界限和——不久以后——它的社会界限。随着工人推动更多的资本,同一资本会支配更大量的绝对劳动时间,——[881]这是没有疑问的。

(3)如果正常工作日不变,剩余劳动能够随着劳动生产力的发展,通过必要劳动时间的缩短和加入工人消费的生活资料的跌价而相对增加。但是劳动生产力的这种发展使可变资本和不变资本相比减少了。比方说用两个人代替20个人,不管绝对剩余劳动时间或相对剩余劳动时间怎样增加,要使这两个人的剩余劳动时间等于20个人的剩余劳动时间,这在体力上是不可能的。即使这20个人每天只完成两小时的剩余劳动,他们提供的剩余劳动就有40小时,而两个人一天生活的全部时间只有48小时。

劳动能力的价值不是按劳动或资本的生产力提高的比例降低的。生产力的这种提高也会在一切不(直接或者间接)生产必需品的部门提高不变资本对可变资本的比例,而不引起劳动价值的任何变化。生产力的发展是不平衡的。资本主义生产的性质的特点是,它发展工业比发展农业快。这并不是由于土地的性质,而是由于土地需要其他社会关系,以便按照它的性质实际加以利用。资本主义生产只是在它的影响使土地贫瘠并使土地的自然性质耗尽以后,才把注意力集中到土地上去。此外,由于存在土地所有权,农产品比其他商品贵,因为农产品是按其价值支付的,而不会降低到费用价格的水平。但是,农产品是必需品的主要组成部分。其次还有一点:由于竞争的规律,如果有1/10的土地在耕种时花费较贵,其余9/10的耕地也会“人为地”受到这种相对不肥沃的严重影响。

为了在资本积累时利润率保持不变,利润率实际上就必须提高。如果资本总是提供10%的剩余劳动,那末,在按照复利进行积累以及所使用的资本因而增加的情况下,同一个工人就必须按照复利增长的级数多提供两倍、三倍、四倍的剩余劳动,——这是荒谬的。

工人推动的、其价值通过工人的劳动保存和再生产的资本量,是和工人追加的价值即剩余价值完全不同的。如果资本量=1000,追加劳动=100,再生产出来的资本便是1100。如果资本量=100,追加劳动=20,再生产出来的资本便是120。利润率在前一场合=10%,在后一场合=20%。然而从100中可以比从20中积累得更多。因此,资本的源流{撇开资本由于生产力的提高而贬值的情况不谈}——或者说资本的“积累”——将比例于资本已有的量而不是比例于利润率的高度滚滚向前。这一点可以说明,尽管利润率下降,积累(按量来说)还是增加,至于在生产率不断提高而利润率即使降低的情况下,可能比在生产率低而利润率高的情况下积累更大一部分收入,那就更不用说了。高利润率(只要它以高剩余价值率为基础)在劳动生产率虽然不高但工作日很长的情况下是可能的。高利润率之所以可能,[还]因为劳动生产率虽然不高,但是工人的需要很小,因而工资的最低额也很小。与工资最低额的微小相适应的是劳动精力的缺乏。在这两种情况下,尽管利润率高,资本的积累却很慢。人口停滞,而生产产品所耗费的劳动时间很多,虽然支付给工人的工资很少。

[882]虽然剩余价值率不变甚至提高,利润率也会下降,对于这一点我曾这样解释过:可变资本同不变资本相比减少了,也就是说,活的现在劳动同所使用的和再生产出的过去劳动相比减少了。\fnote{见本卷第2册第498页和676页。——编者注}霍吉斯金和《国民困难的原因及其解决办法》小册子的作者则用工人不可能满足“复利”的要求,即不可能满足资本积累的要求来解释利润率的下降。

\begin{quote}{“任何劳动,任何生产力,任何发明才能,任何技术,都不能满足复利的压倒一切的要求。但一切积蓄都是从资本家的收入中来的〈也就是从“简单利润”中来的〉,因此,这些要求实际上不断地提出,而劳动生产力同样不断地拒绝满足它们。因此,不断有一种平衡创造出来。”\fnote{见本册第329页。——编者注}(同上,第23页)}\end{quote}

从总的意思来说这是一样的。我说,利润率会随着资本的积累而下降,因为不变资本同可变资本相比会增加,这就是说,如果撇开资本各部分的一定形式不谈,所使用的资本同所使用的劳动相比会增加。利润下降并不是因为工人被剥削得少了,而是因为同所使用的资本相比,所使用的劳动总的来说是少了。

例如,假定可变资本与不变资本之比=1∶1。在这种情况下,如果总资本=1000,那末,c=500,v=500;如果剩余价值率=50%,那末,500的50%=50×5,即250。因此,利润率将是1000分之250,即250/1000,或1/4,即25%。

如果总资本=1000,c=750,而v=250,那末,在剩余价值率为50%的情况下,250提供125。而利润率将是125/1000,即1/8,或12+(1/2)%

但是在第二种情况下使用的活劳动比第一种情况下[少]。如果我们假定,一个工人的工资一年等于25镑,那末在第一种情况下,工资为500镑时就雇用20个工人,在第二种情况下,工资为250镑时就雇用10个工人。同一笔资本1000镑在一种情况下雇用20个工人,在另一种情况下只雇用10个工人。在第一种情况下,资本总量和工作日数之比是1000∶20;在第二种情况下是1000∶10。在第一种情况下,20个工人中每个工人摊到所使用的资本(不变资本和可变资本)50镑(因为20×50=500×2=1000)。在第二种情况下,每个工人摊到所使用的资本100镑(因为100×10=1000)。与此相应,资本中用于一个工人的工资部分,在两种情况下却是一样的。

我提出的公式包含一个新的论据,它说明为什么在进行积累时,较少的工人会摊到同量的资本上,或者同样可以说,为什么较大量的资本会摊到同一劳动上。无论我是说,在第一种情况下,1个工人摊到的所使用的资本等于50,在另一种情况下,1个工人摊到100单位的资本,也就是只要1/2个工人就摊到50单位的资本;因此,无论我是说,在一种情况下1个工人摊到50单位的资本,在另一种情况下,1/2个工人摊到50单位的资本,还是说,在一种情况下50单位的资本摊到1个工人身上,而在另一种情况下,50×2单位资本摊到1个工人身上,这都是一回事。

霍吉斯金等人正是运用了这后一个公式。在他们看来,积累一般来说就是要求复利,就是说,有更多的资本摊到同一个工人身上,这个工人现在应当按照摊到他身上的资本量提供更多的剩余劳动。因为摊到一个工人身上的资本按复利增加了,而他的劳动时间却相反地具有十分明确的界限,“任何生产力”也不能把他的必要劳动时间缩短到符合这些复利所要求的程度,所以这里“经常会遇到一种平衡”。这时“简单利润”则保持不变或者甚至会增加(这种“简单利润”实际上是剩余劳动或剩余价值)。但是随着资本的积累,在单利形式的背后开始隐藏着复利。

[883]其次,很明显:如果复利=积累,那末,撇开积累的绝对界限不谈,利息的这种形成取决于积累过程本身的规模和强度等等,即取决于生产方式。要不然,复利就无非是以利息形式占有他人的资本(他人的财产),就象过去在罗马以及在一般放高利贷的情况下发生的一样。

霍吉斯金的看法是:原来摊到一个工人身上的资本比方说是50镑,同时假定工人要为50镑资本提供25镑利润。过了几年,由于一部分利息转化为资本,并且年年这样重复,摊到一个工人身上的资本已经是200镑了。如果每年的利息是50%,而且总是全部变成资本,那末这个过程不到四年就可以完成。工人象过去要为50镑资本提供25镑利润一样,现在应当为200镑资本提供100镑利润,即比过去提供的多三倍。但这是不可能的。为此他必须多劳动三倍的时间,就是说,如果以前他一天劳动12小时,现在就要劳动48小时,或者劳动价值必须由于劳动生产力的发展而减少四分之三。

如果工作日等于12小时,年工资是25镑,工人一年提供25镑利润,那末,他为资本家劳动的时间必须同为他自己劳动的时间相等,即6小时,或者说,半个工作日。如果工人必须提供100镑利润,那他就要在12小时当中为资本家劳动4×6小时,而这是荒谬的。假定工作日延长到15小时。即使是在这种情况下,工人也不能在15小时劳动当中提供24小时。他更不能在15小时的工作日中提供30小时,而这30小时是必需的,因为他要为资本家劳动24小时,为自己劳动6小时。如果他们自己的全部劳动时间都用来为资本家劳动,他也只能提供50镑,即只能使“利息”增加一倍——为200镑资本提供50镑利润,而他以前为50镑提供了25镑利润。以前利润率是50%,现在是25%。但是在资本为200镑的情况下,要得到25%是不可能的,因为工人还须生活。不论生产力怎样增长,如果12小时所创造的价值仍然如上例那样等于75镑,那末24小时所创造的价值等于2×75,即150镑。因为工人必须生活,所以他无论如何也不能提供150镑利润,更不用说提供200了。他的剩余劳动始终不过是他的工作日的一部分,但是绝不能由此得出结论,就象洛贝尔图斯先生那样\fnote{见本卷第2册第89—90页。——编者注},认为利润永远不可能等于100%。如果利润按整个工作日计算,利润便永远不能等于100%(因为在整个工作日中利润本身已计算在内),但是就工作日中被支付的部分来说,利润完全可能等于100%。

例如在上例中,利润是50%:

\todo{}

这里占工作日一半的利润等于全部产品的1/3。

[884]如果工人把3/4的工作日给资本家,那就是:

\todo{}

折算成100便得出:

\todo{}

现在我们更详细地来考察一下,在这种见解的背后会隐藏着什么东西,根据这种见解,利润下降是因为在积累进程中利润不是“简单利润”(因此,对工人的剥削率不会降低,却象霍吉斯金所说的会提高),而是“复合利润”,但劳动在任何情况下都赶不上复利的要求。

首先应当指出,这一点需要有进一步的规定才能一般具有意义。当作积累(即占有剩余劳动)的产物来看——这种看法就整个再生产来说是必要的——一切资本都是由利润构成(由“利息”构成,如果这个词被看作和利润等同而不是和“借贷利息”等同)。因此,如果利润率=10%,那末这就是“复利”,利润的利润。完全不能理解的是,在经济上10/100和11/110究竟有什么区别。这样就会得出结论:“简单利润”也是不可能的,或者至少是简单利润也应当下降,因为这种简单利润实际上同复合利润一样是复合的。如果把问题看得狭窄些,即仅仅指生息资本,那末,复利会吞没利润而且吞没的比利润还多;生产者(资本家或非资本家)必须付给放债人复利,这意味着他除利润外不得不逐渐把他的一部分资本也付给放债人。

所以,首先必须指出,霍吉斯金的见解只有在假定资本比人口,即比工人人口增长得快的情况下才有意义。(就是后一种增长也是相对的。资本的本性就是使一部分工人过度劳动,把他们弄得疲惫不堪,把另一部分变为赤贫者。)如果人口和资本增加的程度相同,那就没有任何理由可以说明,为什么我能用100镑从x个工人身上取得的剩余劳动,就不能用800镑从8x个工人身上取得。[885]8×100K对8x个工人提出的要求,不会比100K对x个工人提出的要求更多。因此,这里霍吉斯金的理由不能成立。(实际上完全不是这样。即使人口和资本增加的程度相同,资本主义的发展,由于不变资本靠减少可变资本而发展的结果,也会使一部分人口成为过剩人口。)

\begin{quote}{{“你分配它们〈商品〉是为了促使劳动的供给更多还是更少,你是在它们将成为劳动条件的地方分配它们,还是在它们将鼓励游手好闲的地方分配它们,这一点对劳动来说具有十分重大的意义。”(《论马尔萨斯先生近来提倡的关于需求的性质和消费的必要性的原理》1821年伦敦版第57页)“不断增加的人口的数目会促进这种劳动供给的增加。”(同上,第58页)“如果商品不能支配它以前所支配的那样多的劳动量,那末这只有在这一劳动生产的产品不比过去多的地方才有意义。如果劳动的生产率提高了,那末尽管现有的商品量现在支配的劳动量比过去少,生产也不会缩减。”(同上,第60页)}\end{quote}

这一点是针对马尔萨斯的。的确,生产不会缩减,但利润率会减低。“商品量支配劳动”这样一些厚颜无耻的说法包含了马尔萨斯的价值规定\fnote{见本册第8—10页和第24页。——编者注}中所具有的同样的厚颜无耻。“商品支配劳动”这种说法对资本的性质是极好的和充分的说明。

就是这个作者,对威斯特作了正确的评论:

\begin{quote}{“《论资本用于土地》的作者说,如果资本大量增加,劳动将获得较高的报酬,而这种情况……只有在资本利润很高的时候才会发生。他补充说:‘资本利润越多,劳动工资就越高。’这种说法的错误在于这里漏掉了几个字:‘已经得到的资本利润越多……劳动工资就越高’……高利润和高工资不会同时发生;它们不会在同一桩交易里发生;一个妨碍另一个,并降低其水平。同样可以这样来论述:‘商品的价格最高时,商品的供给也增加得最快,因此大量的供给和高的价格是一起前进的。’这是把因果混为一谈。”(同上,第100—101页)}}\end{quote}

因此,只有当(由于积累过程)同一个工人必须推动更多的资本,或者说,当资本同劳动相比增加了的时候,也就是当例如原来是100的资本由于积累变为110,而原来提供剩余价值10的同一个工人必须适应于资本的增长提供剩余价值11,即提供复利的时候,霍吉斯金的论点才有意义。所以,不仅工人过去推动的同一资本在它被再生产出来以后必须提供相同的利润(“简单利润”),而且这个资本已经由工人的剩余劳动增大起来,工人现在必须第一,为原有资本(或资本的价值)提供剩余劳动,第二,还要为他自己的已被积累起来的即资本化了的剩余劳动提供剩余劳动。既然这一笔资本每年都在增长,同一个工人就必须不断提供越来越多的劳动。

但是一般说来,要在同一工人身上摊到比过去更多的资本,只有[在下列两种情况下]才有可能:

第一种情况。如果劳动生产力不变,要在同一个工人身上摊到比过去更多的资本,就只有使工人延长他的绝对劳动时间,例如不是劳动12小时,而是劳动15小时,或者增加劳动强度,即在12劳动小时内完成15小时的劳动,在4小时内完成5小时的劳动,或者说,在4/5小时内完成5/5小时的劳动。因为工人在一定的时数里把自己的生活资料再生产出来,所以在这种情况下,资本家会得到3小时,就象劳动生产力提高了一样,而实际上这里是劳动增加了,而不是劳动生产力提高了。如果劳动的这种强化推广到一切劳动部门,那末商品的价值就必然按照它所化费的劳动时间的减少而下降。这种强度就会成为劳动的平均强度,成为劳动的自然性质。如果劳动的这种强度[886]只是发生在某些部门,这就等于复杂劳动,即自乘的简单劳动。于是较紧张的一小时劳动的某一部分就会等于较松弛的一小时劳动,它们创造同样的价值。例如,在上述情况下,4/5小时的较紧张劳动,就和5/5即1小时的较松弛劳动创造一样多的价值。

延长劳动时间和通过更大的劳动强度,即通过所谓的压缩劳动空隙来增加劳动,这两者都有其界限(尽管例如伦敦的面包工人通常都是劳动17小时,有时还更多),有十分明确的身体界限,而当达到界限的时候,复利,“复合利润”就会停止。

在这些界限内会出现下列情况:

如果资本家对劳动的延长或强化不予支付,他的剩余价值(利润也一样,如果不变资本的价值不变的话,因为我们假定生产方式不变)——(在上述情况下)他的利润——将比他的资本增长得快。他则不为增长的资本支付任何必要劳动。

如果他按照与过去相同的比例支付追加劳动,剩余价值就会和资本的增加成比例地增长。利润就会增长得更快。因为这里固定资本的周转更快;同时机器的磨损加快的程度不会象它的使用加快的程度一样。固定资本的支出会减少,因为同时劳动的200个工人比延长工作日的100个工人需要更多的机器、建筑物等等。同样,在后一种情况下所需的监工等等也较少。(这种情况给资本家造成了一种极其惬意的机会:他可以不再经过任何困难就能根据市场的情况扩大或缩减他的生产。此外,这种情况会增加他的权力,因为一部分工人劳动负担过重,相应地就会有失业的和半失业的后备军,因而工人间的竞争就会加剧。)

虽然在这种情况下必要劳动和剩余劳动之间的纯粹的算术比例没有被破坏,而且这里的唯一情况是它们两者都能以相同的程度增加,但是对劳动的剥削却增加了,——在工作日延长的情况下是这样,在工作日强化(紧张化)的情况下也是这样,只要在这种强化的同时工作日不缩短(如实行十小时工作日法案)。工人缩短了他的劳动能力的存在期限,在比他的工资的增加大得多的程度上消耗了他的劳动能力,而且更加变成一种单纯的工作机器。但是这后一种情况且撇开不说,如果工人在正常工作日的情况下,假定能活20年,而在工作日延长或强化的情况下只能活15年,那末,在一种情况下他是在15年内出卖他的劳动能力的价值,而在另一种情况下,他是在20年内出卖他的劳动能力的价值。在一种情况下劳动能力的价值必须在15年内被补偿,在另一种情况下则在20年内被补偿。

如果每年支付5%,价值100经过20年将得到补偿,因为5×20=100。如果每年支付6+(2/3)%,价值100经过15年将得到补偿。但是在上面所考察的情况下,工人从追加的3小时中得到的只是相当于按20年计算的他的劳动能力的一天的价值。假定他的劳动是8小时必要劳动和4小时剩余劳动,那末他从每1小时中得到2/3小时,因为12×2/3=8。与此相应,他从3小时的额外时间中得到2小时,或者说,从每1小时额外时间中得到2/3小时。但是只有在假定他的劳动能力存在20年的情况下,这才是他1小时劳动能力的价值。如果工人出卖劳动能力的时间只有15年,那末每小时的劳动能力的价值还要相应提高。

对未来的预支——真正的预支——一般说来在财富生产上只有对工人和对土地来说才有可能。由于过早的过度紧张和消耗,由于收支平衡的破坏,工人和土地的未来实际上可能被预支和被破坏。在资本主义生产条件下两者都会发生这种情况。至于所谓的预支,例如公债,那末关于对未来的这样一种预支,莱文斯顿正确地指出:

\begin{quote}{[887]“他们[公债制度的维护者们]宣称,他们打算把今天的开支转嫁到未来,并且坚决主张为了满足现在这一代人的需要可以加重后一代人的负担,这在实际上等于荒谬地认为,可以消费尚未存在的东西,可以在种子播入土地以前就吃粮食。”(莱文斯顿《论公债制度》第8页)“我们的政治家们的全部智慧就是大规模地把一批人的财产转到另一批人的手里,就是建立巨额基金以奖励投机和盗窃国库。”(同上,第9页)}\end{quote}

工人和土地的情况却不是这样。这里被支出的东西是作为力量而存在的,由于这种力量的加速支出,它的寿命就缩短了。

最后,如果资本家对额外时间,比对正常劳动时间不得不支付更多的报酬,那末照上面所说的,这绝不是工资的提高,而只是对额外时间的提高了的价值的补偿,而且追加的工资很少能达到为此所必需的高度。实际上,在工人进行额外劳动的时候,不仅额外时间应当有较好的报酬,而且每一个劳动小时都应当有较好的报酬,以便劳动能力的较快消耗也能多少得到点报酬。

因此,在所有情况下都是对劳动的更大剥削。同时在所有情况下,剩余价值都会随着资本的积累而[相对]减少,而且利润率也会下降,只要这一点不被不变资本的节约抵销的话。[887]

[887]因此,这就是随着资本积累——随着“复合利润”的出现——利润率必然下降的一种情况。如果资本300(第一笔)的利润率等于10%(因而利润是30),而追加资本100的利润率等于6%,那末资本400的全部利润便是36。因此总的来说,100的利润是9。利润率从10%降到9%。

但是,已经说过,在这个基础上(即在劳动生产率不变的情况下)达到一定点以后,追加资本的利润不仅一定会减少,而且会完全消失,于是以这种“复合利润”为基础的一切积累一定会停止。在这种场合,利润的减少是和对劳动的剥削的加重联系在一起的,利润在一定点上的消失,不是因为工人或其他什么人得到了自己的全部产品,而是因为在体力上劳动不可能超过一定量的劳动时间,也不可能把劳动强度增加到超过一定的程度。

第二种情况。在工人数量不变时,每个工人可以摊到比过去更多的资本,因而追加资本可以被用于、被花费于加强对同一数量工人的剥削的唯一的另外一种情况,[888]就是提高劳动生产率,改变生产方式。这种情况决定了不变资本和可变资本之间有机比例的改变。换句话说,这里资本同劳动相比的增加,和不变资本同可变资本以及一般来说同可变资本使用的活劳动量相比的增加,是等同的。

因此,这里霍吉斯金的见解可以归结为我所阐述的一般规律。

剩余价值即对工人的剥削增加了,但是同时利润率下降了,因为可变资本同不变资本相比减少了,活劳动量同推动它的资本相比,一般来说相对地减少了。在劳动的年产品中,一个较大的部分会在资本的名义下为资本家所占有,一个较小的部分会在利润的名义下为资本家所占有。

{这样就产生了查默斯牧师的幻想:年产品中资本家用作资本的量越小,他们吞掉的利润就越大;\endnote{指查默斯的著作《论政治经济学和社会的道德状况、道德远景的关系》1832年格拉斯哥、爱丁堡、都柏林和伦敦第2版第88—89页及其他各页。——第344页。}于是,“法定教会”\endnote{“法定教会”(《EstablishedChurch》)是指英国国教会。——第56、344页。}就来帮助他们,要他们把很大一部分剩余产品用于消费,而不要把它资本化。这个该死的牧师把原因和结果混淆了。而且利润量在利润率较低时也会随着所花费的资本量的增加而增加。此外,这一较小百分比代表的使用价值量增加了。但是,这同时需要资本的集中,因为现在各种生产条件都要求使用大量资本。这需要由大资本家吞并小资本家,使小资本家“丧失资本”。这不过又是劳动条件和劳动本身在另一种形式上的分离(因为小资本家还有较多的自己的劳动。总的说来资本家的劳动和他的资本量成反比,就是说,和他成为资本家的程度成反比。如果没有抵销这种向心力的离心倾向同向心力一起在经常起作用,那末这个过程很快就会使资本主义生产告终;对于这种离心倾向,这里我们不去考察,因为这是属于论资本的竞争那一章),——这种分离,形成资本和原始积累的概念,然后在资本的积累中表现为不断的过程,最后在这里表现为现有资本集中在少数人手中和许多人丧失资本。}

劳动生产率的提高不能完全补偿劳动量的(相对)减少,或者说,剩余劳动和所花费的资本的比例不是按照所使用的劳动的相对量减少的同一比例增长,这种情况之所以造成,部分地是由于:只有当一定的投资领域的劳动生产率有了发展,劳动价值,或者说,必要劳动量才会减少,即使在这些领域,劳动生产率的发展也是不平衡的,并且还会有各种抵销因素发生作用;例如,工人本身虽然不能阻止工资下降(就价值来说),但是他们不会容许工资绝对降到最低限度,反而会努力争取在量上分享一些增长的共同财富。

但是剩余劳动的这种增加也是相对的,并且只有在一定界限内才有可能。要使它适应复利的要求,必要劳动时间在这种情况下就必须等于零,就象在前面所考察的那种情况下[剩余劳动时间]必须无止境地延长那样。

利润率的提高或降低——由[劳动的]供求的变化,或者由必需品价格(同奢侈品相比)暂时的提高或降低(这种暂时的提高或降低又是由供求的这种变化和由此造成的工资的提高或降低引起的)造成的工资的提高或降低所决定的利润率的提高或降低——同利润率提高或降低的一般规律[889]没有任何关系,正象商品市场价格的提高或降低同商品价值的规定根本没有任何关系一样。这一点应当放在工资的现实运动那一章来考察。如果供求关系对工人有利,工人的工资提高,那末某些必需品的价格,特别是食品的价格就可能(但决不是必然)随之暂时提高。关于这一点,《论马尔萨斯先生近来提倡的关于需求的性质和消费的必要性的原理》一书的匿名作者正确地指出:

\begin{quote}{在这种情况下,“对必需品的需求同对非必需品的需求相比会增加,以致这两种需求之间的比例,同他行使这种权力〈即资本家支配商品的权力〉以获得物品供自己消费时的情况完全不同。必需品将因此同数量更多的一般物品交换……这些必需品至少有一部分会是食物”。(第21—22页)}\end{quote}

接着,匿名作者正确地发挥了李嘉图的见解:

\begin{quote}{“于是,不管怎样,谷物价格的提高并不是工资提高(工资提高使利润降低)的最初的原因,而是相反,首先工资的提高是谷物价格提高的原因,其次,土地的性质(由于这种性质,耕作加强时收成相对地越来越少)使一部分这种价格的提高成为永久性的,并阻止人口规律对已有的工资增加产生充分的反作用。”(第23页)}\end{quote}

霍吉斯金和《国民困难的原因及其解决办法》这一小册子的作者都是用活劳动不可能满足“复利”的要求来解释利润的下降,他们对这个问题虽然没有作更进一步的分析,但是比起斯密和李嘉图来,还是大大接近于真理,因为斯密和李嘉图是用工资上涨来解释利润下降的:一个是用实际工资和名义工资的上涨来解释,另一个是用名义工资的上涨,其实不如说是实际工资的降低来解释。霍吉斯金和所有这些[政治经济学家的]无产阶级反对派都以健全的理智指出了这样一个事实:随着资本的发展,靠利润过活的人数相对地增加了。

\tsectionnonum{[(f)霍吉斯金论劳动的社会性质以及资本与劳动之间的关系]}

现在还要从霍吉斯金的小册子《保护劳动反对资本的要求》中举几个结论性的论点。

对产品的交换价值,即对包含在商品里的作为社会劳动的劳动的论述:

\begin{quote}{“几乎每一个艺术和技能的产品都是联合劳动和结合劳动的结果。”}\end{quote}

(这是资本主义生产的结果。)

\begin{quote}{“人是依赖于人的,这种依赖性随着社会的发展而增长,以致任何个人的任何劳动如果不构成大的社会劳动的一部分,这种劳动就未必……会有丝毫价值。”}\end{quote}

{这段话可以用来说明下面这一论点:商品生产,或者说,作为商品的产品生产,只有在资本的基础上才具有包罗万象的性质,才触及产品的实质本身。}

\begin{quote}{“……在实行分工的地方,在工人能够实现他的收入以前,已有别人对这种或那种劳动的评价参加进来,再也没有什么东西可以叫做个人劳动的自然报酬。每个工人只生产整体的一个部分,由于每个部分单独就其本身来说没有任何价值或用处,因此没有东西工人可以拿来说:‘这是我的产品,我要留给我自己。’从某种联合操作例如制造呢绒的操作开始,直到在共同努力制造这一产品的各种不同的人之间分配其产品为止,这中间不止一次地有人对这种或那种劳动的评价参加进来,问题在于在这个共同产品之中有多少应当归于用联合劳动来生产它的每一个个人。[890]除了把这个问题交给工人自己去自由评价外,我不知道还有别的解决这一问题的办法。”(第25页)“我必须补充一点,未必有一种劳动会比别的劳动更有价值。毫无疑问,一切劳动都是同样必需的。”(第26页)}\end{quote}

最后,霍吉斯金谈到资本和劳动之间的关系:

\begin{quote}{“师傅和他们的帮工一样是工人。在这一点上,他们的利益和他们帮工的利益完全相同。但除此以外,他们还是资本家或是资本家的代理人,在这方面,他们的利益和他们工人的利益则截然相反。”(第27页)“这个国家的产业工人的教育已广为普及,这就使得几乎所有师傅和老板的劳动和技艺的价值日益降低,因为教育的广为普及,使拥有这种专门知识的人数增加了。”(第30页)“资本家是在各种工人之间从事压迫的中介人。”如果排除了资本家,那末“非常清楚,资本,或者说使用劳动的能力,和并存劳动就是一个东西;生产资本和熟练劳动也是一个东西。因此,资本和工人人口完全是一个意思。在自然界的体系中,口是同手和智慧结合在一起的”。(第33页)}\end{quote}

资本主义生产方式同社会劳动的不同因素在相互关系中所具有的并以资本为代表的异化形式一起消失。这便是霍吉斯金的结论。

\centerbox{※     ※     ※}

资本的原始积累。包括劳动条件的集中。它是劳动条件对工人和劳动本身的独立化。它的历史活动就是资本产生的历史活动——把劳动条件转化为资本、劳动转化为雇佣劳动的历史的分离过程。这样就提供了资本主义生产的基础。

在资本本身基础上、因而也是在资本和雇佣劳动关系基础上的资本积累。它以越来越大的规模再生产出物质财富同劳动的分离和独立。

资本的积聚。大资本通过消灭小资本而进行的积累。吸引力。资本和劳动的中间结合体的丧失资本。这不过是下述过程的最后一级和最后形式:把劳动条件转化为资本,然后把这种资本和某些资本以更大的规模再生产出来,最后把社会上许多地方形成的资本同它们的所有者分离开来,并把它们集中在大资本家手里。生产在取得这种对立和矛盾的极端形式的同时,转化为社会生产,尽管是以异化的形式。这就是社会劳动以及在实际劳动过程中生产工具的公共使用。资本家作为上述过程,即同时加速这一社会生产,从而加速生产力发展的过程的职能执行者,就依照他们以社会名义为自己刮取收入以及作为这一社会财富的所有者和社会劳动的指挥者而飞扬跋扈的程度日益成为多余的人。他们的情况也和封建主一样,封建主的要求连同他们的服务,就曾经随着资产阶级社会的产生而成为多余的东西,变成了纯粹是过时的和不适当的特权,从而迅速趋于消灭。[XV—890]

\tsectionnonum{[(g)霍吉斯金的基本论点在其《通俗政治经济学》一书中的表达]}

[XVIII—1084]托·霍吉斯金《通俗政治经济学。在伦敦技术学校的四次演讲》1827年伦敦版。

\begin{quote}{“容易的劳动只是留传下来的技能。”(第48页)“因为由分工产生的一切利益自然集中在工人那里并属于工人,如果工人被剥夺了这些利益,如果在社会发展的进程中由于工人的技能不断提高而发财致富的只是那些从来都不劳动的人,那末造成这种情况的原因一定是非正义的占有,是发财致富的人的篡夺和抢劫,是陷于贫困的人的同意俯首听命。”(第108—109页)[1085]“如果把工人的繁殖仅仅同资本家对他们的服务的需求相比较,那末工人确实是繁殖得太快了。”(第120页)“马尔萨斯指出,工人人数的增加对每个工人从年产品中得到的份额的减少有影响,其假定是:这一产品在工人中间进行分配的那一部分是一定的和固定的量,决不是由工人一年中所生产的东西调节的。”(第126页)“劳动是价值的唯一尺度;但是劳动,这个一切财富的创造者,不是商品。”(第186页)}\end{quote}

关于货币对财富增加的影响,霍吉斯金正确地指出:

\begin{quote}{“如果一个人能够用小量的容易毁坏的产品来换取某种不易毁坏的东西,那末他就不会被诱惑去抛弃那些容易毁坏的产品了。这样,货币的使用就会防止浪费,从而增加财富。”(第197页)“零售商业的主要好处是由这种情况决定的:商品最适于生产的量并不是商品最适于分配的量[对个人消费来说]。”(第146页)“关于资本的理论,以及使劳动停在除工人生活费用之外还能为资本家生产利润的那个点上的实践,看来,都是同调节生产的自然法相违背的。”(第238页)}\end{quote}

关于资本积累,霍吉斯金所说的同他在前一部著作中所说的几乎一样。但是为了完整起见,这里还是把那些主要段落引证如下:

\begin{quote}{“我们现在仅仅来考察一下例如固定资本——这是对那些断定资本有助于生产的人最有利的题目。必须区别资本积累的结果极不相同的下列三种情况:(1)生产资本和使用资本的是同一个人。非常明显,他生产和使用的工具在他手里的任何积累都会减轻他的劳动。工人生产和使用这种工具的能力就是这种积累的界限。(2)生产资本和使用资本的是按公平的比例分配共同劳动产品的不同的人。资本可以由一个工人生产,而由另一个工人使用;他们按照每一个人在生产商品时所花费的劳动的比例分配商品……但是这一事实我宁可这样来说明:社会的一部分生产工具,而另一部分却使用工具,这样便形成能够提高生产力和增长公共财富的一定的分工部门。只要这两类工人的产品在他们之间进行分配,他们生产和使用的工具的积累和增加,就会象生产工具和使用工具的是同一个人时一样地有益。(3)资本是既不生产它也不使用它的那一类人的财产……资本家只是工具的所有者,他本身不是劳动者。他无论怎样也不能促进生产。”}\end{quote}

{换句话说,促进生产的是工具,而不是某个A对这种工具所具有的所有者的头衔,不是工具属于非劳动者这种情况。}

\begin{quote}{“资本家占据一个工人的产品并把它转交给另一个工人——或者象多数种类的固定资本那样转交一个时期,或者象工资那样永远转交出去,——只要资本家认为这一产品的利用或消费可以给他带来好处,他就转交。资本家如果不是为了自己的利益,那他绝不容许落到他手里的一个工人的产品被另一个工人利用或消费。他使用或出借自己的财产,为的是在工人的产品或者说自然收入中得到一份;这种财产在他手里的任何积累,都不过是他支配劳动产品的权力的扩大,并且会阻碍国民财富的发展。目前的情况就是这样……因为资本家,整个产品的所有者,只要他除了维持工人生活的费用以外得不到利润,他就既不会允许工人生产工具,也不会允许工人使用工具,所以很明显,这里对生产劳动设置的界限比自然界规定的界限要狭窄得多。随着资本在第三者手里的积累,资本家所要求的全部利润额增加了,从而给生产和人口的增加制造了人为的障碍……在工人从来不是资本所有者的当前社会状况下,资本的任何积累都会使要求于工人的利润额增加,并且使仅能保证工人过舒适生活的一切劳动成为不可能……既然承认劳动生产一切,甚至生产资本,那末把生产力说成是由劳动所生产和使用的工具造成的,便是荒谬的了。”(第243—247页)“工资不会象工具那样使生产变得容易……劳动,而不是资本,支付一切工资。”(第247页)[1086]“资本家的大部分预付是由支付的诺言构成的……纸币的发明和使用显示了资本决不是积蓄的某种东西。只要资本家为了实现自己的财富或支配他人的劳动而不得不拥有真正积累起来的贵金属或商品,我们就可以认为资本的积累是真正积蓄的结果,认为社会的进步取决于资本的积累。但是,当发明了纸币和印在羊皮纸上的有价证券,当只有这么一张羊皮纸的人就能得到纸片形式的年收入,并且由于有了这些小纸片,他就可以得到供他使用或消费的一切所需的东西,而如果他不把所有这些小纸片都花光,他在年终就比年初更富,或者说,就有权在第二年得到更大量的纸片,于是他就有更大的支配劳动产品的权力,——这时就非常明显,资本不是积蓄的结果,单个资本家发财致富不是由于真正的物质的积蓄,而是由于他做了一件使他能够……从他人的劳动产品中得到更多东西的事情……呢绒厂主有用以支付工资的铸币或纸币。他的工人用这种工资去交换别的工人的产品,这种工资不论是铸币还是纸币,后者都不会保存起来;它又回到厂主那里,厂主又拿出他自己的工人制造好的呢绒来和工资交换。他又用返回来的铸币或纸币支付工资,这些铸币或纸币又进行同样的循环……把体现在机器上的知识和技能给予劳动的一切巨大帮助都只归功于他的〈资本家的〉财产,而不管这种财产是用来支付工资还是表现为有用的工具……矿工、熔炼工、锻工、机械工、司炉和无数其他的人的联合劳动,而不是死的机器,完成着蒸汽机所做的一切……按照通常的说法,工人的这种技能的生产力被认为是由它的有形产品即劳动工具造成的,既不生产工具,也不使用工具,而只是工具的所有者的人,却自认为是最生产的人。”(第248—251页)}\end{quote}

霍吉斯金对于“资本流往国外的危险”的议论的反驳,对于把资本利息看作发展生产的必要刺激的观点的反驳,或者说对于积蓄理论的反驳,见第IX本札记本第47页\endnote{马克思引的是他1851年的第IX本札记本。在这个札记本的第47页摘录了霍吉斯金《通俗政治经济学》一书第252—256页上的话。——第352页。},要在论庸俗经济学家一章中谈这一点。

\begin{quote}{“随着人口的增长,生产和消费两者也都增长,国民财富的积累或增长的概念无非就是这样。”(第257页)[XVIII—1086]}\end{quote}

\tsectionnonum{[(h)霍吉斯金论资本的权力以及论财产权利的变革]}

[XIII—670a][霍吉斯金]《财产的自然权利和人为权利的比较》1832年伦敦版。

\begin{quote}{“现在,社会的一切财富首先落入资本家手中,甚至大部分土地也被资本家买去。他对土地所有者支付地租,对工人支付工资,对赋税和什一税的征收者支付他们要求的东西,而留给自己的是年劳动产品的很大一部分,其实是最大的而且日益增长的一部分。现在,资本家可以看作是全部社会财富的最先所有者,虽然没有任何一项法律给予他这种所有权。”(第98页)“所有权方面的这种变化是由于资本的取息、由于复利的增长而产生的,同样值得注意的是,整个欧洲的立法者都想用取缔高利贷的法律来阻止这件事。”(第98页注)“资本家支配国家的全部财富的权力是所有权上的一种彻底的革命;然而这个革命是靠哪一项法律或者哪一套法律来实行的呢?”(第99页)[XIII—670a]}\end{quote}

\tchapternonum{[(4)]政治经济学家的反对派布雷}

[\endnote{关于布雷的一节马克思没有写完。实际上马克思只是收集了作为“政治经济学家的反对派”的布雷的一些最重要的见解。布雷关于“平等交换”是消除使工人阶级成为牺牲品的那种不公正现象的手段的空想学说,马克思早在《哲学的贫困》(1847年)第一章题为《构成价值或综合价值》的第二节(见《马克思恩格斯全集》中文版第4卷第88—117页)作了批判分析。关于布雷对货币的本质和作用的看法,见马克思的1847年手稿《工资》(《马克思恩格斯全集》中文版第6卷第641页);《政治经济学批判大纲》(1939年莫斯科版)第55、690、754页;1858年4月2日马克思给恩格斯的信;《政治经济学批判》(《马克思恩格斯全集》中文版第13卷第76页)。——第353页。}X—441]约·弗·布雷《对待劳动的不公正现象及其消涂办法》1839年里子版。

\begin{quote}{因为人的存在以劳动为条件,而劳动又以劳动资料为前提,所以“土地这个一切活动的巨大场所和一切财富的原料,必须是它的所有居民的共同财产”。(第28页)“生活有赖于食物,而食物有赖于劳动。这种依赖性是绝对的。因此,一个人要回避劳动,只有在其他大批人的劳动增加的情况下才有可能。”(第31页)“人们所加予别人的或自己遭受的一切不公正现象和痛苦,归根到底都是由于某些个人和阶级篡夺了土地的权利并剥夺了其他个人和其他阶级的这种权利……人们占有了土地所有权以后,下一步便是占有对人本身的所有权。”(第34页)}\end{quote}

布雷宣布自己的目的是:

\begin{quote}{“在政治经济学家们自己的基础上并用他们自己的武器来同他们进行斗争〈为了证明不是在任何社会制度下贫困都必然是工人的命运〉。政治经济学家们要推翻用这种方法作出的结论,就必须先否认或推翻他们自己的论点所依据的那些已确立的真理和原则。”(第41页)“根据政治经济学家们本身的意见,为了生产财富,必须有:(1)劳动,(2)过去劳动的积累,或者说资本,(3)交换……”这就是他们所认为的一般生产条件。“这些生产条件对整个社会都是有效的,它们的性质是:任何个人或任何阶级都不能不受它们的影响。”(第42页)“金科玉律:必须劳动!——对一切创造物来说是同样有约束力的……只有人能够回避这一规律;这一规律的性质是:一个人只有靠牺牲别人才能回避这一规律。”(第43页)“按照劳动和交换的真正性质来说,严格的公正态度要求{布雷在这里引用了政治经济学家们提出的商品交换价值的定义}交换双方的利益不仅是相互的,而且是相等的……在公正的交换制度下,一切商品的价值都会由它们的生产费用的总和来确定,并且相等的价值应该总是换得相等的价值……直到今天,工人们交给资本家一年的劳动,但只换得半年劳动的价值,现在在我们周围存在着的权力和财富的不平等就从这里产生。交换的不平等(按一种价格买进,按另一种价格卖出)的必然结果是:资本家继续是资本家,工人继续是工人,一个是暴君阶级,另一个是奴隶阶级。”(第47—49页)“在现在的制度下,交换不仅没有象政治经济学家们所断定的那样给交换双方的每一方提供相互的利益,而且可以有把握地说,在资本家和生产者之间的大多数交易中根本没有进行交换……工厂主或土地所有者用什么来支付工人的劳动呢?用劳动?不是,因为资本家不劳动。用资本?不是,因为他的财富的储备不断增长……因此,资本家不能用属于他自己的任何东西来交换。因此,整个这种交易明显地表明,资本家和土地所有者所做的只是:他们对工人一星期劳动的偿付,是他们上星期从工人那里取得的财富的一部分;而这一点正好说明他们同工人以无易有……资本家好象用来和工人的劳动相交换的财富既不是资本家的劳动创造的,也不是他的财富创造的,它最初由工人的劳动得来,并且通过欺骗性的不平等交换制度每天又从工人那里被夺走。生产者和资本家之间的全部交易是明显的欺骗,纯粹是一幕滑稽剧。”(第49—50页)“宣称‘必须积累!’的法律只有一半得到执行,它的提出有利于一个特殊阶级而有损于整个其余社会。”(第50页)“在现在的社会制度下,整个工人阶级在劳动资料方面依赖资本家或雇主;而在一个阶级由于自己的社会地位而在劳动资料方面依赖另一个阶级的地方,它在生活资料方面也同样依赖那个阶级。而这种状况同社会的目的本身是如此矛盾,并且是如此违背理性……因此一分钟也不能为它辩解,为它辩护。它赋予个别的人以任何一个凡人所不应有的权力。”(第52页)“我们的日常经验告诉我们,如果我们从一个面包上切下一块,这一块就不能再长大。一个面包只是若干块的总和,我们吃掉的块数越多,留下来吃的就越少。工人的面包的情况就是这样,[442]但是资本家的面包却不遵循这种规则。他的面包不是变小,而是不断增大;资本家不断地切,但面包总是在增大……如果交换是平等的,现在的资本家的财富就会逐渐地由他们那里转到工人阶级的手里;富人花掉的每一个先令都会使他的富有少一个先令。”(第54—55页)}\end{quote}

布雷在同一个地方还指出:

\begin{quote}{“一个资本家要从自己的属于工人阶级的祖先真正积累起来的劳动中继承哪怕是一千镑,也几乎是完全不可能的。”(第55页)“从政治经济学家们自己的学说中可以得出这样的结论:没有积累就不可能有交换,没有劳动就不可能有积累。”(第55页)“在现在的制度下,每个工人至少给予雇主六日劳动以换取一个只值四日或五日劳动的等价物,雇主的利益必然是工人的损失。”(第56页)“因此,不论怎样试图用赠予、个人积累、交换或继承来论证财富的起源,我们都会发现一个又一个的证据,说明在富人的所有权的这种论证方面有一个缺陷,这个缺陷使论证一下子便失去任何公正的外貌和任何意义……所有一切财富都是好多世纪以来在工人阶级的骨肉上生长起来的,并且是通过欺骗性的和奴役性的不平等交换制度从工人那里夺走的。”(第56—57页)“在现在的制度下,如果工人想成为富人,他就必须成为资本家,或者说,成为交换他人劳动的人,而不是交换自己的劳动,那时,他就会用别人掠夺他自己的办法,即通过不平等交换的办法来掠夺别人,从而有可能从别人的不大的损失中获得大量的利益。”(第57页)“政治经济学家们和资本家们写了并出版了很多书,目的是给工人灌输一种错误的观念,似乎‘资本家的利润并不是生产者的损失’。他们对我们说,劳动离了资本寸步难行,资本就象挖土工人手里的铁铲一样,资本对于生产就象劳动本身对于生产一样必要……资本和劳动的这种相互依赖性与资本家和工人的关系毫无共同之处,并不证明前者必须靠后者生活……对生产者的操作具有重大意义的不是资本家,而是资本。资本和资本家之间的区别就象船上装的货物和提货单之间的区别一样大。”(第59页)“从资本和劳动的相互关系中可以明显地看出,一个国家的资本越多,或者说,积累的产品越多,生产就越容易,为达到某一(一定的)结果所需要的劳动就越少。例如,不列颠人民利用他们现在的巨大的资本积累(他们的建筑物、机器、船舶、运河和铁路),在一个星期内所能生产的工业财富,比一千年前他们的祖先在半个世纪里所能生产的还多。使我们能够这样做的,不是我们的体力的优越,而是我们的资本。因为凡是资本缺乏的地方,生产就发展得缓慢而吃力,反过来也是一样。由此可以清楚地看出,资本的一切利益同样是劳动的利益,资本的任何增加都会减轻劳动的繁重程度,因此资本的一切损失必然也是劳动的损失。虽然这一真理早被政治经济学家们所发现,但是他们还从来没有作过公正的阐述。”}\end{quote}

{事实上,这些家伙是这样议论的:

积累的劳动产品——即未消费的产品——减轻劳动,并使劳动更有成效。因此,这种减轻等等的成果应当不是对劳动本身有利,而是对积累有利。因此,积累不应当是劳动的财产,而劳动却应当是积累的财产——劳动自己的产品的财产。因此,工人不应当为自己积累,而应当为别人积累,积累应当作为资本同他相对立。

在政治经济学家们那里,资本的物质要素和它的作为资本的社会的形式规定性(即和它的作为支配劳动的劳动产品的对抗性质)是如此地生长在一起,以致他们提出的任何一个论点都不能不自相矛盾。}

\begin{quote}{“政治经济学家们总是把资本和社会的一个阶级等同起来,把劳动和另一个阶级等同起来,虽然这两种力量都没有这种自然的联系,同样也不应当有这种人为的联系。政治经济学家们总是把事情说成这样:似乎工人的幸福,甚至工人的生存本身,只有在工人用自己的劳动来维持资本家的奢侈和懒散生活的情况下才有可能。他们不愿让工人吃饭,直到工人生产出两份饭——一份为自身,一份为他的老板,后者则是间接地即通过不平等交换得到自己的一份。”(第59—60页)“当工人生产出某种物品的时候,它已经不属于工人,而属于资本家,通过不平等交换的无形魔术,它从一个人手里转到另一个人手里。”(第61页)“在现在的制度下,资本和劳动,铁铲和挖土工人,是两种分离的和对抗的力量。”(第60页)[443]“但是,即使所有的土地、房屋和机器都属于资本家,而不存在工人阶级,资本家也不能回避伟大的条件‘必须劳动!’。尽管他们有一切财富,他们也只能在劳动和饿死之间进行选择。他们不能吃土地或房屋;没有人的劳动加进去,土地就不会长出食物,机器也不会做出衣服。因此,如果资本家和私有者说,工人阶级应该养活他们,那末他们实际上也就是说,生产者完全象土地和房屋一样属于他们,工人只是为了有钱人的需要才创造出来。”(第68页)“生产者用他所给予资本家的东西进行交换时,得到的不是资本家的劳动,也不是资本家的劳动产品,而是工作。在货币的帮助下,工人阶级不仅不得不完成为活命自然要完成的劳动,而且还得为其他阶级负担劳动。生产者从非生产阶级那里得到的是金银还是其他商品,那是无关紧要的;全部实质在于,工人阶级完成他自己的劳动并养活他自己,此外还要完成资本家的劳动和养活资本家。不论生产者从资本家那里得到的名义报酬是什么,他们的实际报酬却是:本来应当由资本家完成的劳动现在转到了他们身上。”(第153—154页)“我们假定联合王国的人口是2500万。假定他们的生活费平均每人每年至少15镑。联合王国全部人口的生活费的年价值总额是37500万镑。但是我们生产的不只是生活资料,因为我们的劳动也创造出许多不供个人消费的物品。我们增加了我们的房屋、船舶、工具、机器、道路和其他供未来的生产使用的设备,此外还修复了一切磨损了的东西,因此我们每年都在增加我们的积累储备,或者说资本。所以,虽然我们的生活费的价值象上面所说的每年只有37500万镑,但是人民所创造的财富总的年价值却不少于5亿镑……只有1/4的人口即大约600万从14岁至50岁的男人可以算作真正的生产者。可以说,在目前的情况下,这个数字中参加生产的恐怕还不到500万人〈布雷接着说,直接参加物质生产的只有400万人〉;因为成千上万有劳动能力的男人被迫坐着无事可做,而本应由他们去做的工作却由妇女和儿童去做;在爱尔兰就有几十万男人根本找不到工作。这样一来,不到500万男人连同几千名儿童和妇女却必须为2500万人进行生产……现有的工人人数如果不使用机器,便不能养活自己和养活现有的游惰者以及非生产劳动者。现在在农业和工业上使用的各种机器,据统计可以完成近1亿有劳动能力的男人的劳动……这些机器及其在现在的制度下的使用,产生了几十万现在压迫工人的游惰者和食利润者……机器使现在的社会制度富有成效,机器也将使它遭到破坏……机器本身是好的,没有机器不行;但是机器的使用,它们为个别人占有而不为整个国家占有这种情况却不好……现在参加生产的500万男人中,有些人一天只劳动5小时,而另一些人却劳动15小时;如果此外我们还注意到由于生意萧条时期大量工人被迫无事可做而造成的时间损失,我们就会发现我们的年产品是由社会上不到五分之一的每天平均劳动10小时的人创造和分配的……假定各种有钱的非生产者以及他们的家属和奴仆只有200万人,他们的生活费平均和工人的生活费一样多,即每人15镑,那末这200万人每年就将花掉工人阶级3000万镑……但是按照最低的估计,他们的生活费每人至少50镑。这样,作为社会上完全非生产的纯粹不劳而食者的生活费的年价值总额就是1亿镑……此外,还有各种小私有者阶级、产业家阶级和商人阶级以利润和利息的形式[444]获得的加倍的和四倍的收入。根据最低的估计,社会上这个人数众多的部分所消费的那部分财富一年不少于14000万镑,超过了相同人数的报酬最高的工人的平均消费量。这样,游惰者和食利润者这两个阶级(他们可能占总人口的1/4)连同他们的政府一起,每年就要吞食近3亿镑,即超过生产出来的全部财富的半数。这一点使帝国的每个工人平均一年损失50镑以上……剩下在国家其余3/4的人口中分配的大约平均每人每年至多11镑。根据1815年的统计,联合王国全体人民的年收入约为43000万镑,其中工人阶级得到99742547镑,而靠地租、年金和利润生活的阶级得到330778825镑。当时国内全部财产的价值约计30亿镑。”(第81—85页)}\end{quote}

参看金的图表\endnote{马克思指英国最初的统计学家之一格雷哥里·金所编的《1688年英格兰不同家庭的收支表》,这个图表被查理·戴韦南特收进他的著作《论使一国人民在贸易差额中成为得利者的可能的方法》(1699年伦敦版)。马克思在《剩余价值理论》第一册(见本卷第1册第171—172页)谈到这个图表。——第359页。}等等。

\begin{quote}{1844年英国的人口是:大小贵族——1181000人,商人、工业家、农场主等——4221000人(以上两类共5402000人),工人、贫民等——9567000人。(托·查·班菲尔德《产业组织》1848年伦敦第2版[第22—23页])[X—444]}\end{quote}

\tchapternonum{[第二十二章]拉姆赛}

\tchapternonum{[(1)区分不变资本和可变资本的尝试。关于资本是不重要的社会形式的观点]}

[XVIII—1086]乔治·拉姆赛(三一学院)《论财富的分配》1836年爱丁堡版。

说到拉姆赛,我们又回到政治经济学家们这里来了。

{为了把商业资本列入生产领域,拉姆赛把它称为“商品从一个地点向另一个地点的运输”。(拉姆赛,同上第19页)这样,他就把商业和运输业混淆了。}

拉姆赛的主要功绩在于:

首先,他事实上区分了不变资本和可变资本。诚然,这种区分是以如下的方式作出的:他把从流通过程得出的固定资本和流动资本的区别作为唯一的区别在名称上保留下来,但是对固定资本作了这样的解释,说它包括不变资本的一切要素。因此,他所理解的固定资本,不仅是机器和工具、劳动用或保存劳动成果用的建筑物、役畜和种畜,而且包括各种原料(半成品等等)、“土地耕种者的种子和制造业者的原料”。(第22—23页)此外,被拉姆赛列入固定资本的还有“各种肥料、农业用的篱笆和工厂中消费的燃料”。(第23页)

\begin{quote}{“流动资本只由在工人完成他们的劳动产品以前已经预付给工人的生活资料和其他必需品构成。”(同上)}\end{quote}

由此可见,他所谓的“流动资本”,无非是[1087]归结为工资的那部分资本,而固定资本则是归结为客观条件——劳动资料和劳动材料——的那部分资本。

当然,拉姆赛的错误在于,他把这种从直接生产过程得出的资本的划分与从流通过程中产生的区别等同起来。这是他墨守政治经济学传统的结果。

另一方面,拉姆赛又把按照上面那样解释的固定资本的纯粹物质构成和它作为“资本”的存在混淆起来。流动资本(即可变资本)不进入实际劳动过程;进入这个过程的,是用流动资本买来的东西,也就是用来代替它的东西——活劳动。除此之外,进入这个过程的还有不变资本,即物化在客观的劳动条件(劳动材料和劳动资料)中的劳动。因此,拉姆赛说道:

\begin{quote}{“严格地说,只有固定资本,而不是流动资本,才是国民财富的源泉。”(第23页)“劳动和固定资本是生产费用的所有要素。”(第28页)}\end{quote}

在生产商品时实际耗费的,是原料、机器等等以及推动它们的活劳动。

“流动”资本是多余的,它处在生产过程之外。

\begin{quote}{“如果我们假定工人不是在完成产品之前得到报酬,那就根本不需要流动资本。生产还会保持同样的规模。这证明,流动资本既不是生产的直接因素,甚至对生产也毫无重要意义,它只是由于人民群众可悲的贫困而成为必要的一个条件。”(第24页)“从国民的观点来看,只有固定资本才是生产费用的要素。”(第26页)}\end{quote}

换句话说,被拉姆赛称为“固定资本”的、物化在劳动条件(劳动材料和劳动资料)中的劳动以及活劳动,简言之,实现了的、物化了的劳动以及活劳动,是生产的必要条件,是国民财富的要素。反之,[在拉姆赛看来,]工人的生活资料一般地采取“流动资本”的形式,这纯粹是由“人民群众可悲的贫困”产生的“一个条件”。劳动是生产的条件,而雇佣劳动则不是;从而工人的生活资料作为“资本”,作为“资本家的预付”同工人相对立,这也不是生产的条件。拉姆赛忽视了这样一个情况:如果生活资料不作为“资本”(按他的说法,不作为“流动资本”)同工人相对立,客观的劳动条件也就同样不作为“资本”(按他的说法,不作为“固定资本”)同工人相对立。拉姆赛认真地,而不象其他政治经济学家那样只是在口头上,把资本归结为“国民财富中用于或预定用于促进再生产的部分”[第21页]。因此他宣称,雇佣劳动,从而资本——再生产资料在雇佣劳动的基础上取得的社会形式——是不重要的,它们只是由人民群众的贫困产生的。

总之,我们现在已经接近这样一点,即政治经济学本身根据它的分析宣布:生产的资本主义形式,从而资本,对生产来说并非绝对的条件,而只是“偶然的”、历史的条件。

然而,拉姆赛的分析还不够,还不足以从自己的前提中,从他给直接生产过程中的资本所下的新定义中,得出正确的结论。

\tchapternonum{[(2)拉姆赛关于剩余价值和价值的观点。剩余价值归结为利润。关于不变资本和可变资本的价值变动对利润率和利润量的影响问题的不能令人满意的说明。资本的有机构成,积累和工人阶级的状况]}

拉姆赛确实接近于正确地理解剩余价值。

\begin{quote}{“流动资本所使用的劳动,总是要多于先前用于它自身的劳动。因为,如果它使用的劳动不能多于先前用于它自身的劳动,那它的所有者把它作为流动资本使用,还能得到什么好处呢?”(第49页)“或许有人会说,任何一笔流动资本所能使用的劳动量,不过等于先前用于生产这笔资本的劳动。这就意味着,所花费的资本的价值等于产品的价值。”(第52页)}\end{quote}

因此,这就是说,资本家用较少的物化劳动同较多的活劳动相交换,这个无酬的活劳动余额,构成产品价值超过产品生产中消费掉的资本价值的余额,换句话说,构成剩余价值(利润等等)。如果资本家以工资支付的劳动量等于他在产品上从工人那里收回的劳动量,产品的价值就不会大于资本的价值,也就不会有利润了。尽管拉姆赛在这里如此接近于剩余价值的真正起源,然而他毕竟受政治经济学传统的束缚太甚,以致又立即走入歧途。首先,他对可变资本[1088]和劳动之间的这种交换的解释方法本身是模棱两可的。如果他对这种交换十分明确,就不可能产生进一步的误解。他说:

\begin{quote}{“一笔比如说由100个工人的劳动创造的流动资本,将推动150个工人。因此,在这种情况下,年终的产品将是150个工人劳动的结果。”(第50页)}\end{quote}

在什么条件下,100个工人的产品能够雇150个工人呢?

如果一个工人得到的12劳动小时的工资等于12劳动小时所创造的价值,那末,用他的劳动的产品只能重新买到一个工作日,用100工作日的产品只能买到100工作日。但是,如果他一天的劳动产品的价值等于12劳动小时,而他一天得到的工资的价值只等于8劳动小时,那末用他一天的产品的价值就可以支付(可以重新买到)1+(1/2)个工作日或1+(1/2)个工人。用100工作日的产品可以雇100(1+1/2个工人或工作日)=100+50=150个工人。由此可见,为使100个工人的劳动产品能推动150个工人,这100个工人中的每一个,或者总的说来,每一个工人都必须用相当于他为自己劳动的时间的一半白白为资本家劳动,或者说,他必须白白劳动1/3个工作日。在拉姆赛那里,这一点没有讲清楚。他的模棱两可表现在结论上:“因此,在这种情况下,年终的产品将是150个工人劳动的结果。”当然,它将是150个工人劳动的结果,正象100个工人的劳动产品曾经是100个工人劳动的结果一样。模棱两可(以及无疑由于拉姆赛或多或少地效法马尔萨斯而产生的含糊不清)在于,好象利润的产生只是由于现在使用的是150个工人,而不是100个工人。这就等于说,从150个工人那里获得利润,是由于现在用这150个工人的产品推动了225个工人(100∶150=150∶225;20∶30=30∶45;4∶6=6∶9)。但问题不在这里。

如果用x表示100个工人的全部工作日,他们提供的劳动量就是x。他们得到的工资则是2/3x。因此,他们的产品的价值=x,他们的工资的价值=x-1/3x,他们创造的剩余价值=1/3x。

如果把100个工人劳动的全部产品重新花在工资上,那末就可以雇150个工人,而这150个工人的产品又等于225个工人的工资。100个工人的劳动时间就是100个工人的劳动时间。但是,他们的有酬劳动却是66+(2/3)个工人劳动的产品,或者说仅仅等于100个工人产品中包含的价值的2/3。模棱两可是这样产生的:好象100个工人或100个工作日(按年或按日计算工作日都一样)提供了150工作日,即包含着150工作日创造的价值的产品;实际上则是100工作日创造的价值够支付150工作日的报酬。如果资本家仍旧使用100个工人,他的利润就仍旧那么多。他就仍旧用等于66+(2/3)个工人的劳动时间的产品支付100个工人,而把余下的部分装入腰包。如果他把100个工人的全部产品重新花在工资上,他就实现了积累,并且可以占有等于50工作日的剩余劳动,而以前占有的只是33+(1/3)工作日。

拉姆赛在这个问题上并没有搞清楚,这一点从下述事实立刻就可以看出来:他为了反对价值决定于劳动时间,又把那个否则就“无法解释”的现象——对于剥削不等量劳动的各资本,利润率是相同的——提了出来。

\begin{quote}{“固定资本的使用,大大改变了价值取决于劳动量的原则。因为一些耗费了等量劳动的商品,要成为可供消费的成品,却需要很不相同的时间。但是因为在这段时间里资本不带来收入,所以,为了使该项投资不比其他项投资——这些项投资的产品成为可供消费的成品所需的时间较短——获利少,当商品最后进入市场时,它必须提高价值,提高的数额相当于少得的利润。这一点表明,资本可以撇开劳动而调节价值。”(第43页)}\end{quote}

更确切地说,这一点表明资本撇开特殊产品的价值而调节平均价格\endnote{“平均价格”这一术语,马克思在这里是指“生产价格”,就是指生产费用(c+v)加平均利润。“平均价格”这一术语本身说明,这里所指的是“一个相当长的时期内的平均市场价格,或者说,市场价格所趋向的中心”(见本卷第2册第359页)。马克思用的这个术语最初见于本卷第1册第76页。在本卷第2册关于洛贝尔图斯和李嘉图的那几章,这个术语多次跟“费用价格”和“生产价格”同时并用。——第365页。},表明资本交换商品不是按照商品的价值,而是按照“使该项投资不比其他项投资[1089]获利少”的办法。拉姆赛也没有放弃重复自[詹姆斯·]穆勒以来就已经出名的例子“葡萄酒置于窖内”\fnote{见本册第89—91、193、251页。——编者注},因为在政治经济学中,不用脑子的传统比在其他任何一门科学中都更加顽固。于是,他得出结论:“资本是不取决于劳动的价值源泉”,(第55页)其实他至多可以作这样的结论:资本在某一特殊部门中实现的剩余价值,不取决于该特殊资本所使用的劳动量。[1089]

[1090]拉姆赛的错误观点在这里尤其令人奇怪,是因为他一方面理解到可以说是剩余价值的自然基础,而另一方面,在一个场合断定,剩余价值的分配——剩余价值平均化为一般利润率——并不增加剩余价值本身。

[首先,拉姆赛说道:]

\begin{quote}{“利润的存在决定于物质世界的规律,按照这个规律,自然界的恩惠,在得到人们的劳动和技艺的配合和指引时,就会充分报答国民的劳动,使国民劳动提供的产品,除了以实物形式补偿已消耗的固定资本并繁衍受雇的工人的种族所绝对必需的数额以外,还有一个余额……”[第205页]}\end{quote}

{“繁衍工人的种族”,这也[1091]是资本主义生产的美妙结果!当然,如果劳动生产率只够再生产劳动条件和维持工人的生活,就不可能有余额;从而也就没有利润,没有资本。但是,自然界同下述事实是毫不相干的:不管是否存在这个余额,工人的种族在繁衍着,而余额采取利润的形式并在这个基础上“繁衍”资本家的种族——这一点拉姆赛本人也承认了,因为他宣称,“流动资本”(在他那里是指工资,雇佣劳动)不是生产的本质条件,而只是由“人民群众的可悲的贫困”产生的。他并没有得出资本主义生产“繁衍”这种“可悲的贫困”的结论,虽然他说到资本主义生产“繁衍工人的种族”并给工人留下恰恰够这种繁衍所必需的数额时,也承认了这一点。在上述意义上可以说,剩余价值等等是以某种自然规律为基础的,是以与自然界相互作用的人类劳动的生产率为基础的。但是,拉姆赛自己指出劳动时间的绝对延长是剩余价值的源泉(第102页),也指出由于工业进步而提高的劳动生产率是源泉。}

\begin{quote}{“……只要总产品中除去用于上述目的所绝对必需的以外还有一点儿余额,就有可能从产品总量中分离出一种属于另外一个阶级的叫做利润的特殊收入。”(第205页)“资本主义企业主作为一个特殊阶级的存在本身是取决于劳动生产率的。”(第206页)}\end{quote}

其次,当谈到工资的[普遍]提高引起一些部门价格上涨而使利润率平均化时,拉姆赛说道:

\begin{quote}{“工资的普遍提高引起一些产业部门价格上涨,决不能使资本主义企业主避免利润的减少,甚至一点也不减轻他们的总的损失,而仅仅促使比较平均地把这种损失分配在这个阶级的不同阶层之间。”(第163页)}\end{quote}

如果有一个资本家,他的葡萄酒是100个工人的产品(拉姆赛举的例子),另一个资本家,他的商品是150个工人的产品;当前者的葡萄酒卖得的价格和后者的商品一样,以便“使该项投资不比其他项投资获利少”时,那末显而易见,在葡萄酒和另一种商品中包含的剩余价值并没有因此增多,而只是平均地分配在不同的资本家阶层中。[1091]

[1089]拉姆赛还再次援引了李嘉图[价值决定于劳动时间这一规定]的“例外”。这些例外将要在我们的正文中谈到价值转化为生产价格时加以探讨。\endnote{马克思指《资本和利润》这一篇,他是在1861—1863年手稿第XVI本上开始写的(该本注明:1862年12月,而包括论拉姆赛、舍尔比利埃、琼斯那几章的第XVIII本则注明:1863年1月)。马克思本想在《资本和利润》这一篇的第二章考察价值转化为生产价格的问题,这一点从写完论拉姆赛一章后不久写的第二章计划草稿中可以看出(见本卷第1册第447页)。《资本和利润》这一篇以后发展为《资本论》第三卷。关于大卫·李嘉图所表述的价值决定于劳动时间这一规定的“例外”,马克思在《剩余价值理论》第二册详细地谈到过(见本卷第2册第191—224页;并参看本册第72页)。——第367页。}就是说,这里只简单地谈一下。假定不同生产部门中的工作日的长度(在不被劳动的强度、劳动的不愉快等等抵销的情况下)相等,或者更确切些说,假定剩余劳动以及剥削率相等,那末,剩余价值率,只有在工资提高或降低时才可能发生变动。剩余价值率的这种变动,亦即工资的提高或降低,依资本有机构成的不同而对商品的生产价格发生不同的影响。可变部分大于不变部分的资本,当工资降低时,取得的剩余劳动多于不变部分大于可变部分的资本,而当工资提高时,占有的剩余劳动则少于后者。可见,工资的提高或降低对两个部门的利润率发生相反的作用,或者说,造成对一般利润率的两种相反的偏离。因此,为了维持一般利润率,在工资提高时,前一类商品的价格将上涨,后一类商品的价格则下降。(当然,各类资本只是按照它所使用的活劳动同所花费的全部资本量的比例直接受工资波动的影响。)相反,在工资下降时,前一类商品的价格将下降,后一类商品的价格则上涨。

严格说来,这一切在考察价值向生产价格的最初转化和一般利润率的最初形成时,是不必探讨的,因为更确切地说,这里涉及的问题是,工资的普遍提高或降低怎样影响受一般利润率调节的生产价格。

这种情况跟固定资本和流动资本之间的区别就更无关系了。银行家和商人几乎只使用流动资本,但很少使用可变资本,就是说,他们花在活劳动上的资本是比较少的。相反,矿主使用的固定资本比起缝纫业资本家不知要大多少倍。但是,他是否也按同样的比例使用活劳动,那就大成问题了。只是因为李嘉图把这个特殊的、比较不重要的情况作为区别生产价格和价值的唯一事例(或象他错误地表述的那样,作为价值决定于劳动时间这一规定的例外)提出来,并且是以固定资本和流动资本的区别的形式提出的,所以这一谬误才作为重要的教条——而且是以错误的形式——进入以后的全部政治经济学。(应该同矿主对比的不是缝纫业主,而是银行家和商人。)

[拉姆赛说道:]

\begin{quote}{“工资的提高受劳动生产率的限制。换句话说……一个工人劳动一天或一年所得到的,决不可能多于他依靠财富的其他任何源泉在这个时间里所能生产出来的……他的报酬必定低一些,因为总产品中的一部分始终要补偿固定资本〈按照拉姆赛的意思,就是不变资本:原料和机器等等〉及其利润。”(第119页)}\end{quote}

在这里,他把两个不同的东西混起来了。在日产品中包含的“固定资本”量,不是工人日劳动的产品,或者说,由一部分实物形式的产品代表的这部分产品价值,不是日劳动的产品。而利润倒确实是工人的这种日产品,或者说这种日产品的价值的扣除部分。

如果说,拉姆赛没有研究清楚剩余价值的本质,尤其是,他在价值同生产价格的关系上,在剩余价值转化为平均利润上,完全拘泥于旧的偏见,那末,他从自己对于固定资本和流动资本的理解中,反而得出了另外一个正确的[1090]结论。

在谈这一点之前,先再引证一段话:

\begin{quote}{“价值不仅要与实际消费掉的资本,而且要与还未变动的资本成比例,一句话,要与所使用的全部资本成比例。”(第74页)}\end{quote}

这段话的意思应该是:利润,从而生产价格,要与所使用的全部资本成比例,而价值则显然不能随着没有加入产品价值的资本部分发生变动。

[拉姆赛从自己对于固定资本和流动资本的理解中,得出如下的结论:]

随着社会的进步(即资本主义生产的进步),资本的固定部分靠缩小流动部分,即缩小用于劳动的部分而增大。因此,随着财富的增加或资本的积累,对劳动的需求相对地减少。在工业中,生产力的发展给工人带来的“祸害”是暂时的,但它们会一再重复。在农业中,特别是在耕地变为牧场时,这些祸害则是永久的。总的结果是:随着社会的进步,即随着资本的发展(也就是随着国民财富的发展),这种发展对工人状况的影响越来越小,换句话说,按照一般财富增长即资本积累的比例,或者同样可以说,按照再生产规模扩大的比例,工人状况相对恶化。我们看到,这个结论和亚·斯密的素朴见解或庸俗政治经济学的辩护论见解大不相同。在亚·斯密那里,资本的积累是和对劳动需求的增长,和工资的不断提高,从而和利润的下降等同的。在他那个时候,对劳动的需求,确实是至少和资本的积累按同样比例增长,因为当时工场手工业还占支配地位,而大工业则处在襁褓之中。

[拉姆赛说道:]

\begin{quote}{“对劳动的需求仅仅取决于〈直接地、不需任何媒介地〉流动资本量。”(第87页)〈这是拉姆赛的同义反复,因为在他那里流动资本等于花在工资上的资本。〉“随着文明的进步,国家的固定资本靠减少流动资本而增长。”(第89页)“因此,对劳动的需求,并不总是随资本的增长而增长,至少不是按同样的比例增长。”(第88页)“只有当流动资本由于新的发明而比原来数额增多时}\end{quote}

{在这里,再次流露出这样一个错误观点:似乎生活资料总量的增多和生活资料中供工人用的部分的增多是一回事},

\begin{quote}{对劳动的较大需求才会出现。那时需求会提高,但并不是同总资本的积累成比例地提高。在工业十分先进的国家,固定资本同流动资本相比总是越来越大。因此,在社会进步的过程中,用于再生产的国民资本的每次增加,对工人状况的影响会越来越小。”(第90—91页)“固定资本的每次增加,都是靠减少流动资本,亦即靠减少对劳动的需求来达到的。”(第91页)“机器的发明给工业中在业工人人口带来的祸害,可能只是暂时的,但是它们会经常重复发生,因为新的改进经常推动劳动的节约。”[第91页]}\end{quote}

在工业中这些祸害所以是暂时的,照拉姆赛看来(第91—92页),是由于以下原因:[第一,]使用新机器的资本家,得到超额利润;因此他们进行节约以及扩大资本的能力增长了。节约下来的一部分也作为流动资本使用。第二,工业品的价格按照生产费用减少的比例下跌;因此,消费者节约了,从而也促进了资本的积累,一部分资本可能进入商品生产费用减少的工业部门。第三,这些产品的价格的下跌,增大了对它们的需求。

\begin{quote}{“可见,尽管机器会使相当数量的人失业,然而,经过一段或长或短的时间,这些人,甚至更大数量的工人,可能重新被雇用。”(第92—93页)“在农业方面,情况完全不同。对原料的需求增长得不象对工业品的需求那么快……对农村人口来说,耕地变为牧场是最致命的……从前用以养活工人的基金,现在几乎全部用在牛、羊和固定资本的其他要素上。”(第93页)[1090]}\end{quote}

[1091]拉姆赛正确地指出:

\begin{quote}{“工资和利润一样,都应该看成成品的一部分,从国民的观点看来,这部分与成品的生产费用是完全不同的。”(第142页)“固定资本……撇开它被使用的结果不谈……是一种纯粹的损失……除所消费的固定资本外,只有劳动(不是工资,不是对劳动支付的东西)是生产费用的要素。劳动是一种牺牲。它在一个经济部门花费的越多,留给另一个经济部门的就越少。因此,如果把劳动用在无收益的事业中,国民就要由于主要的财富源泉的滥用而蒙受损失……劳动报酬并不构成费用要素。”(第142—143页)}\end{quote}

(把劳动,而不是把有酬劳动或者说工资作为价值要素,这是十分正确的。)

拉姆赛正确地描述了实际的再生产过程。

\begin{quote}{“怎样才能把产品和花费在产品上的资本加以比较呢?……如果指整个国民而言……那末很清楚,花费了的资本的各个不同要素应当在这个或那个经济部门再生产出来,否则国家的生产就不能继续以原有的规模进行。工业的原料,工业和农业中使用的工具,工业中无数复杂的机器,生产和贮存产品所必需的建筑物,这一切不仅应当成为一国所有资本主义企业主的全部预付的组成部分,而且应当成为该国总产品的组成部分。因此,总产品的量可以同全部预付的量相比较,因为每一项物品都可以看成是与同类的其他物品并列的。”(第137—139页)“至于单个资本家}\end{quote}

{这是错误的抽象。国民只是作为资本家阶级存在,而整个这一阶级的活动和单个资本家的活动完全一样。两种考察方式的区别仅仅在于,一种是把使用价值,另一种则把交换价值紧紧抓住并孤立起来},

\begin{quote}{由于他不是以实物来补偿自己的支出,他的支出的大部分必须通过交换来取得,而交换就需要一定份额的产品,由于这种情况,单个资本主义企业主不得不把更大的注意力放在自己产品的交换价值上,而不是放在产品的量上。”(第145—146页)[1092]“他的产品的价值愈高于预付资本的价值,他的利润就愈大。因此,资本家计算利润时,是拿价值同价值相比,而不是拿量同量相比。这一点是在国民和个人计算利润的方式上应该看到的第一个区别。”}\end{quote}

{假定国民跟全体资本家有所不同,国民在某种意义上也可以把价值同价值这样相比:国民可以计算用于补偿消费了的不变资本部分和加入个人消费的产品部分的全部劳动时间,以及花在创造用来扩大再生产规模的余额上的劳动时间。}

\begin{quote}{“第二个区别是,由于资本主义企业主总是向工人预付工资,而不是用成品支付工资,企业主就把这个预付看成和所消费的固定资本一样,是他的支出的一部分,虽然从国民的观点看来,工资并不是费用要素。”}\end{quote}

{事实上,对总再生产过程来说,这个区别也消失了。资本家总是用成品支付工资,就是说,他用工人昨天生产出来的商品支付工人明天的工资;换句话说,他以工资形式给工人的,事实上只是一种凭证,用来取得在未来制成或者是接近制成,就是说到被购买时要最后完成的产品。在再生产中,即在生产的连续过程中,仅仅作为表面现象的预付,也就消失了。}

\begin{quote}{“因此,资本主义企业主的利润率,取决于他的产品价值超过预付资本——固定资本和流动资本——的价值的余额。”(第146页)}\end{quote}

{从“国民的观点”来看也是这样。资本主义企业主的利润始终取决于他本人为产品所支付的,而不管他支付工资时产品是否制成。}

拉姆赛的功绩在于,首先,他反驳了自亚·斯密以来广为流行的错误观点,即认为总产品的价值分解为各种名称不同的收入;其次,他以双重方式,通过工资率即剩余价值率和通过不变资本的价值,决定利润率。但是,他犯了一个正好和李嘉图相反的过错。李嘉图想强行使剩余价值率同利润率相等。拉姆赛则相反,提出了利润率的二重性的规定:(1)利润率决定于剩余价值率(即工资率)和(2)利润率决定于这个剩余价值对总预付资本之比(就是说,事实上拉姆赛是以不变资本与总资本之比决定利润率的),——并且未弄清问题实质而把这二重性的规定看成是决定利润率的两个平行的情况。他没有看到剩余价值在成为利润之前所发生的转化。因此,如果说李嘉图为了贯彻价值理论,试图强行把利润率归结为剩余价值率,那末拉姆赛就是试图把剩余价值归结为利润。此后我们将看到,他叙述不变资本价值对利润率的影响所用的方法,是很不充分的,甚至是错误的。

[拉姆赛说:]

\begin{quote}{“利润的上升或下降,同总产品或它的价值中用来补偿必要预付的那个份额的下降或上升成比例……因此,利润率决定于以下两个因素:第一,全部产品中归工人所得的那个份额;第二,为了以实物形式或通过交换来补偿固定资本而必须储存的那个份额。”(第147—148页)}\end{quote}

因此,换句话说,利润率决定于产品价值超过流动资本和固定资本总额的余额;也就是说,决定于第一,流动资本和第二,固定资本在全部产品价值中所占的份额。如果我们知道这笔余额是哪里来的,问题就简单了。但是,如果我们只知道利润取决于余额对这些支出的比例,我们就可能得出关于这笔余额的来源的极其错误的看法,例如,就可能象拉姆赛那样,以为它部分地来源于固定(不变)资本。

\begin{quote}{[1093]“构成固定资本的各种物品在生产上变得容易,肯定会使这个份额\fnote{即总产品中用来补偿“固定资本”的份额。——编者注}减少从而提高利润率,就象在前一种场合,由于用以维持劳动的流动资本要素的再生产变得便宜而使利润率提高一样。”(第164页)}\end{quote}

例如,以租地农场主为例:

\begin{quote}{“无论总产品的数额是多少,其中用来补偿在生产中以不同形式消费了的全部东西的那个量,不应当有任何变动。只要生产以原有的规模进行,这个量就必须看成是不变的。因此,总产品越多,租地农场主必须为上述目的拨出的份额必然越小。”(第166页)“生产食物和诸如亚麻、大麻、木材之类的原料的租地农场主,把这些东西再生产出来越容易,他的利润提高得就越多。租地农场主的利润由于他的产品数量的增加而提高;产品的总价值保持不变,但是,租地农场主用来补偿他可以自己供应自己的各种固定资本要素在总产品中,从而在它的价值中的份额,比从前减少了。至于工业资本家则会由于他的产品具有较大的购买力而获利。”(第166—167页)}\end{quote}

假定收成等于100夸特,种子等于20夸特,即等于收成的1/5。再假定第二年收成增加一倍(支出同量的劳动);现在它等于200夸特。如果生产规模保持原有水平,种子就仍然等于20夸特,但现在这20夸特只占收成的1/10。然而必须考虑到,先前100夸特的价值等于现在200夸特的价值;也就是说,前一年收成的一夸特[按价值来说]等于后一年收成的两夸特。在前一场合,剩余80夸特,在后一场合,剩余180夸特。因为这里讨论的是不变资本价值的变动对利润率的影响问题,不涉及工资,所以就假定工资在价值上保持不变。这样一来,在前一场合工资等于20夸特,在后一场合等于40夸特。最后,再假定租地农场主不能以实物形式再生产的其他不变资本组成部分,在前一场合其价值等于20夸特,从而在后一场合等于40夸特。

于是,我们得出如下的计算数字:

(1)产品=100夸特,种子=20夸特。其他不变资本=20夸特,工资=20夸特,利润=40夸特。

(2)产品=200夸特,种子=20夸特。其他不变资本=40夸特,工资=40夸特,利润=100夸特。这100夸特在价值上等于(1)中的50夸特。因此,在这个场合有10夸特的超额利润。

可见,在这里,由于不变资本的价值变动,不[仅]利润率,而且利润本身也提高了。尽管在(1)、(2)两个场合工资是相同的,利润与工资之比,即剩余价值率,却提高了。然而这只是表面现象。在第二个场合,利润中首先有80夸特——这80夸特[按价值来说]等于(1)的40夸特——对工资之比与第一个场合相同;其次有20夸特——这20夸特只等于

(1)的10夸特——从不变资本转化为收入。

但是这个计算正确吗?我们必须假定,第二个场合[收成加倍]的结果属于下一年,尽管劳动是在和第一个场合相同的条件下进行的。为了更清楚起见,我们假定,在第一个场合1夸特等于2镑。在第二个场合,租地农场主为得到200夸特的收成,支出如下:种子20夸特(40镑),其他不变资本20夸特(40镑),工资20夸特(40镑)。总计120镑,而产品=200夸特。在第一个场合他也只是支出120镑(60夸特),而产品是100夸特,等于200镑。剩下的是利润80镑或40夸特。因为第二个场合的200夸特[和第一个场合的100夸特一样]是同量劳动的产品,所以也只值200镑。因此在第二个场合剩下的也只有80镑利润,但是这80镑现在等于140夸特\endnote{如果说,在最初计算中曾假定,在第二个场合,用于劳动工具和劳动力的生产费用,已经按每夸特谷物降低了一半的价值(由于收成增加一倍而造成)计算,那末,马克思现在注意到下面这种情况:每夸特谷物价值的这种降低,只是在第二年秋天才发生,而在秋天以前每夸特的价值要高一倍。因此,如果在最初计算中,第二个场合的生产费用以20c+40c+40v=100夸特的数额表示,那末,它现在就用和第一个场合相同的数额即20c+20c+20v=60夸特来表示。因为第二个场合的收成等于200夸特,所以剩下的是利润140夸特。——第376页。}。因此,每夸特[对租地农场主来说]仅仅值4/7镑而不是1镑。换句话说,每夸特的价值从2镑降到4/7镑,即减少1+(3/7)镑,而不是象上面第二个场合与第一个场合对比中所假定的那样,从2镑降到1镑,或者说只减少一半。

在第二个场合,租地农场主的全部产品等于200夸特,价值200镑。但其中120镑补偿他在生产上支出的60夸特,每夸特花费他2镑。因此,剩下的是80镑利润,等于剩下的140夸特。这是怎么回事呢?在第二个场合,每夸特值1镑,但是在生产上支出的60夸特,每夸特则值2镑。它们使租地农场主花费的,就等于他从新的收成中支出120夸特。这样,剩下的140夸特值80镑,或者说,具有的价值并不比第一场合剩下的40夸特多。诚然,租地农场主对这200夸特的每1夸特都是按1镑出卖的(假定他出卖自己的全部产品),这样他就卖得200镑。但是在这200夸特中,有60夸特,每夸特花费他2镑;因此,剩下的每1夸特只给他提供4/7镑[或者说,约1/2镑]。

如果他现在重新支出[种子]20夸特(10镑[按每夸特1/2镑计算])、工资40夸特(20镑)和其他不变资本40夸特(20镑),也就是总共支出100夸特以代替从前的60夸特,而得到180夸特的收成,那末这180夸特所具有的价值和从前100夸特所具有的价值[按每夸特1镑计算]是不相等的。诚然,他使用了同从前一样多的活劳动,从而[1094]可变资本的价值和从前相同,剩余产品的价值也和从前相同。但是,他支出的物化劳动却较少,因为同样的20夸特,从前值20镑,现在只不过值10镑。

因此,得出如下的计算数字:

\todo{}

第一个场合的产品等于100夸特(100镑)。

第二个场合的产品等于180夸特(90镑)。

然而[尽管产品的价值下降],利润率提高了:因为在第一个场合40镑利润是靠60镑支出获得的,而在第二个场合40镑利润是靠50镑支出获得的。前者为66+(2/3)%,后者为80%。

无论如何,利润率的提高不象拉姆赛假定的那样,是由于价值保持不变。因为在产品生产上支出的劳动的一部分,即不变资本(在这里是种子)中包含的那部分劳动减少了,所以,如果生产以原有的规模继续进行,产品的价值就要下降,正象100磅纱中包含的棉花降价时这些纱的价值也要下降一样。但是,可变资本对不变资本之比提高了(虽然可变资本的价值并没有提高)。换句话说,所花费的资本总额对剩余价值之比降低了。利润率的提高就是由此而来的。

如果拉姆赛所说的是正确的,也就是说,如果价值保持不变,那末利润,利润量从而还有利润率就会提高。利润率的单独提高是根本谈不上的。

但是,[不变资本的价值变动对利润率的影响]问题就特殊情况[一部分不变资本以实物形式得到补偿]来说,还没有解决。这种特殊情况在农业中的表现如下:

一定量的种子在收成中是按产品的原价计算的,而且这一部分以实物形式加入收成。其余的支出通过按原价出卖谷物而得到抵补。通过这些原有的支出,产品增加一倍。比如说,在前面讲到的场合,支出种子20夸特(40镑),其余支出等于40夸特(80镑),现在的收成是200夸特而不是原来的100夸特(200镑),在这100夸特中40夸特(80镑)是全部支出60夸特(120镑)的利润。这一次收成所支出的和上一次支出的绝对量相等,都是60夸特,价值120镑,但是现在的余额不是40夸特,而是140夸特。在这里,实物形式的余额大大增加。但是由于在两个场合支出的劳动是相同的,所以现在200夸特具有的价值并不比从前100夸特多。因此,这200夸特值200镑,即每夸特的价值从2镑降为1镑。但是,既然余额等于140夸特,那末看来它应该值140镑,因为其中每一夸特所值同别的一夸特是完全一样的。

如果我们先撇开再生产过程来观察问题,假定租地农场主不再经营,把全部产品出卖,那问题就会变得再简单不过了。那时,为了抵补(补偿)自己的120镑支出,他实际上应该卖出120夸特。这样预付资本就得到抵补。因此,余额是80夸特,而不是140夸特,并且因为这80夸特等于80镑,所以它所值和第一个场合的余额完全一样。

然而,再生产过程使问题多少起了变化。就是说,租地农场主从自己的产品中以实物形式补偿20夸特的种子。[按价值来说,]这20夸特使他得到40夸特产品的补偿。但是,在再生产过程中,他仍旧只须以20夸特的实物来支付。他的其余支出[以夸特表示]随着每夸特价值减少而相应增加(假定工资不降低)。为了补偿不变资本的其余部分,他现在需要40夸特,而不是原来的20夸特,为了补偿工资,也需要40夸特,而不是20夸特。从前他支出60夸特,现在一共要支出100夸特;但是,他不必按谷物减价所要求的那样,支出120夸特,因为他现在是用价值20镑的20夸特补偿(因为这里只涉及到这20夸特的使用价值)从前值40镑的20夸特[种子]。这样一来,他显然[1095]赚了现在值20镑的这20夸特。他的余额不是80镑,而是100镑,不是80夸特,而是100夸特。(如果按原有价值以夸特表示这个余额,它现在就不是40夸特,而是50。)这是一个不容置疑的事实,如果谷物的市场价格不因谷物丰足而下降,他就可以按新价值多出卖20夸特,赚得20镑。

他之所以通过再生产用相同的支出取得这20镑余额,是因为劳动的生产率高了,虽然剩余价值率在这里并没有提高,就是说,工人提供的剩余劳动并没有比从前多,或者他从产品的再生产部分(代表活劳动的部分)中得到的份额并不比从前少。相反,可以这样假定,工人在再生产中得到40夸特,而从前只得到20夸特。可见,这是一个独特的现象。这个现象没有再生产是不会发生的,但是,它的发生是同再生产联系着的,而且它之所以发生,是因为租地农场主以实物形式补偿自己的一部分预付。在这种情况下,不仅利润率会提高,而且利润也会增加。(至于再生产过程本身,租地农场主现在或者可以按原有规模继续进行再生产,这时,如果收成又是同样好,产品价格就下降——,因为一部分不变资本的价值减少了,——但是利润率会提高;或者他可以扩大生产规模,用相同的支出扩大播种,这时,利润和利润率都会提高。)

现在谈工厂主。假定他在棉纱生产上支出100镑,利润为20镑。因此产品等于120镑。假定在100镑中用于棉花的支出是80镑。如果现在棉花的价值下降一半,他就只须在棉花上支出40镑,其余一切则支出20镑,就是说,总共须支出60镑,而不是100镑。利润仍旧等于20镑。总产品等于80镑(假定他不扩大生产规模)。这样一来,40镑留在他的腰包里;他可以把这40镑自己花掉,或者作为追加资本投放。在后一种情况下,按照新的生产规模,他将在棉花上[追加]支出26+(2/3)镑,在劳动等等上[追加]支出13+(1/3)镑。[40镑追加支出的]利润是13+(1/3)镑。总产品现在=60+40+33+(1/3)=133+(1/3)镑。

可见,这里的问题不在于租地农场主以实物形式补偿自己的种子,因为工厂主所用的棉花是购买的,而不是用自己的产品补偿的。可见,这种现象可以归结如下:从前作为不变资本被束缚的那部分资本中,有一部分游离出来,或者说,一部分资本转化为收入。如果在再生产过程中花费的资本同从前正好一样多,其结果就会同在原有生产规模上使用追加资本完全一样。因此,这是一种由于提供资本各组成部分的生产部门的生产率提高而发生的积累。然而,原料价值的这种下降,如果是丰收造成,就会被歉收时原料的涨价抵销。因此,在一次或几次丰收时以上述方式游离出来的资本,在某种程度上成了预防歉收的准备资本。例如,某个工厂主的[固定资本]周转期为12年,他就必须这样安排,使他在这12年期间至少能够以同样的规模继续生产。因此,势必估计到,补偿[原料]时所支付的价格会发生波动,而且在比较长的年限内得到平衡。

资本各组成部分价格的上涨所起的作用,跟它们价格的下降所起的作用相反。(在这里,我们把可变资本撇开不谈,虽然工资降低时,必须支出的可变资本按价值来说减少了;而工资提高时必须支出的可变资本增多了。)现在为了能按原有规模继续生产,必须支出更多的资本。因此,撇开利润率下降不谈,这里必须使用准备资本,或者把一部分收入转化为资本,虽然它不是作为追加资本起作用。

在一种场合[在价格下降时]发生了积累,虽然预付资本的价值保持不变(但是它的物质组成部分增加了)。资本的价值增殖率和利润的绝对量增长了,因为这就同在原有生产规模上投入追加资本一样。在另一种场合[在价格上涨时],积累的发生是由于预付资本的价值,即总产品价值中执行资本职能的部分增长了。但是资本的物质组成部分没有增加。利润率下降了。(利润量只有在现在雇用的工人数量和从前不一样,或者工人的工资也提高的情况下才会减少。)

上述资本转化为收入的现象是值得注意的,因为它造成一种假象,似乎利润量的增加(或者反之——减少)不取决于剩余价值量。我们曾经看到\fnote{见本卷第2册第518—523页。——编者注},在[1096]一定的情况下,这种现象可以解释部分地租。

在前面考察的场合(等于20镑的20夸特余额不是立即重新用于扩大生产规模,也就是说,不是用于积累),一笔20镑的货币资本游离出来了。这是一个例子,说明尽管商品价值量保持不变,仍然可以有多余的货币资本从再生产中沉淀下来。这是由于一部分先前以固定(不变)资本形式存在的资本转化为货币资本而产生的。

前面讲到的现象[一部分资本转化为收入]和[拉姆赛的]利润率的规定是多么不相干,这一点,只要我们设想一个在新的生产条件下开始经营的租地农场主(或工厂主),就很清楚了。以前,为了开始经营需要120镑的资本:40镑用于购买20夸特种子,40镑用于其他不变资本要素,40镑用于支付工资。他的利润是80镑。80镑比120镑支出,即2比3,等于66+(2/3)%。

现在,租地农场主预付20镑来购买20夸特种子,40镑象以前一样购买其他不变资本,40镑支付工资,这样,他的资本支出为100镑。而80镑利润与100镑支出之比,为80%。利润量保持不变,但是利润率提高了20%。因此,我们看到,种子价值(或者说,补偿种子时所支付的价格)的下降本身同利润的增加毫无关系,而仅仅包含利润率的提高。

此外,租地农场主(或在另一场合,工厂主)本身也不把这件事看作是他的利润增加,而看作是一部分以前被束缚在生产中的资本游离出来。而且他这样考察问题是由于下面的简单计算。以前在生产上预付的资本等于120镑,现在等于100镑,而20镑则作为闲置资本、作为可以随意使用的货币留在租地农场主的腰包里。但是,在这两种场合,他的全部资本都只等于120镑,就是说,他的资本量没有增加。诚然,资本的1/6从那种被束缚于再生产过程的形式中游离出来,起着同追加资本一样的作用。

拉姆赛没有抓住这个问题的实质,因为他根本没有弄清价值、剩余价值和利润之间的关系。

\centerbox{※     ※     ※}

拉姆赛正确地阐述了机器等等怎样——在它们影响可变资本的范围内——对利润和利润率发生作用。就是说,它们通过降低劳动能力的价值,通过增加相对剩余劳动,或者——就总再生产过程来考察——通过减少总产品中用以补偿工资的份额发生作用。

\begin{quote}{“在那些不加入固定资本的商品的生产中,劳动生产率的提高或降低,不能对利润率产生任何影响,除非使总产品中用以维持劳动的份额发生变动。”(同上,第168页)“如果工厂主通过机器的改良使他的产品增加一倍,那末,他的商品的价值,归根到底,必然按商品数量增加的比例减少。”}\end{quote}

{这里假定,事实上,把机器的磨损计算在内,增加了一倍的产品数量所值并不比以前此数的一半多。不然的话,单位产品的价值会下降,但不是按产品数量增加的比例下降。产品数量可能增加一倍,而它的价值、单位商品的价值(在总产品价值提高的情况下),却可能不是从2降到1,而只是从2降到1+(1/4),等等。}

\begin{quote}{“……工厂主不过是由于他可以使工人的衣着更便宜,从而使工人在总收益中所得的份额更小,才会获利……租地农场主也不过是{由于工厂主那里的劳动生产率提高}在他的一部分支出用于供给工人衣着而现在他能够比较便宜地买到这一部分的情况下,才会获利,就是说,他用和工厂主相同的方式获利。”(第168—169页)}\end{quote}

不变资本各组成部分的价值的降低[或提高]所以影响利润率,是因为它影响剩余价值与所花费的资本总额的比例。至于工资的降低(或相反)影响利润率,则是因为它直接影响剩余价值率。

例如,假定在前面讲到的场合(假定租地农场主是亚麻种植业者),种子的价格保持不变,等于40镑(20夸特),花在其他不变资本上的仍旧是40镑(20夸特),但是工资——即同样工人人数的工资——从40镑降到20镑(从20夸特降到10夸特)。在这个场合,[新创造的]价值(工资加剩余价值)量保持不变。因为工人人数相同,他们的劳动仍旧体现在等于40镑+80镑=120镑的价值中。但是现在这120镑中,归工人的是20镑,属于剩余价值的是100镑。{换句话说,这里假定没有进行过任何能影响这个部门的在业工人人数的改良。}

现在预付资本是100镑,而不是120镑,这和种子价值降低一半的场合一样。但是现在利润是100镑,也就是说利润率是100%,而在另一场合[种子价值降低],所花费的资本也从120降为100镑,但利润率是80%。跟这另一场合一样,现在[1097]有20镑或1/6的资本游离出来。但是在这另一场合剩余价值保持不变,等于80镑(就是说,剩余价值率为200%,因为工资是40镑)。现在剩余价值量提高到100镑(就是说,剩余价值率提高到500%,因为工资是20镑)。

在这里,不仅利润率,而且利润本身也提高了,因为剩余价值率,从而剩余价值本身提高了。这就是现在这个场合不同于另一个场合的地方,而拉姆赛并没有看到这一点。只要利润的增长没有被由于不变资本的价值同时发生变动而造成的利润率相应降低所抵销,这种增长总是要出现的。例如在前面讲到的场合,所花费的资本是120镑,利润是80镑,即66+(2/3)%。在我们这个场合,所花费的资本等于1利润是100镑,即100%。但是,如果由于不变资本的价格发生变动,支出从100增加到150镑,那末,利润虽从增长到100镑,却仍旧只提供66+(2/3)%的利润率。

[拉姆赛继续说道:]

\begin{quote}{“既不加入固定资本也不加入流动资本的那些商品,不可能由于它们的生产率发生任何变化而影响利润。这类商品是各式各样的奢侈品。”(第169—170页)“资本主义企业主由于奢侈品充裕而得到好处,因为他们的利润将支配较大数量的奢侈品供他们个人消费;但是,这个利润的比率不会因这些商品的丰富或不足而受到任何影响。”(第171页)}\end{quote}

首先应该指出,一部分奢侈品可以作为不变资本要素进入生产过程。例如,葡萄进入葡萄酒的生产,金进入奢侈品的生产,金刚石用于磨玻璃,等等。但是,拉姆赛既然说“不加入固定资本的商品”,就把这种情况排除了。这样,他的下一句话:“这类商品是各式各样的奢侈品”,是错误的。

然而,说到奢侈品工业的劳动生产率,它增长的原因也只能和其他所有生产部门一样:要么由于取得奢侈品原料的自然仓库如矿山、土地等等的生产率提高了,或者发现较富饶的这类自然仓库;要么由于采用分工,或者特别是使用机器(以及改进的工具)和自然力。{工具的改进和工具的分化一样属于分工。}(化学过程也不应当忘记。)

现在假定,通过机器(或化学过程)缩短了奢侈品的生产时间;生产它们所需要的劳动减少了。这一点对于工资,对于劳动能力的价值不会有丝毫的影响,因为奢侈品不加入工人消费(至少从来不加入他们的决定其劳动能力价值的那部分消费)。{奢侈品生产时间的缩短对工人的市场价格可能产生影响,如果工人因此被抛到街头,从而使劳动市场上的供给增加的话。}因此,奢侈品生产时间的缩短,对剩余价值率不产生影响,从而在利润率决定于剩余价值率的情况下,对利润率也不产生影响。可是,只要它触动剩余价值量,或者触动可变资本对不变资本以及对总资本之比,它当然会对利润率产生影响。

例如,[在某奢侈品的生产中]如果从前雇用20个工人,现在使用机器只需要10个工人,那末剩余价值率实际上并没有受到任何影响。奢侈品变得便宜并不能使工人的生活费用变得便宜。为了再生产自己的劳动能力,他仍然需要和从前相同的劳动时间。

{因此,事实上,生产奢侈品的工厂主力图把劳动的报酬压到劳动的价值之下,压到它的最低限度之下,他所以能够做到这一点,是因为相对的人口过剩(例如刺绣女工的情况),而这种过剩又是由于其他生产部门劳动生产率的增长造成的。或者生产奢侈品的工厂主力图延长绝对劳动时间——在其他部门也是这样;在这种情况下,他实际上创造了绝对剩余价值。只有一点是正确的:奢侈品工业的劳动生产率的提高不能压低劳动能力的价值,不能创造相对剩余价值,总之,不能创造由劳动生产率本身的增长决定的剩余价值形式。}

但是,剩余价值量决定于两个因素:[第一,]剩余价值率,即单个工人的剩余劳动(绝对的或相对的);第二,同时使用的工人人数。因此,如果奢侈品工业的劳动生产率的增长使一定量资本所推动的工人人数减少,它就会使剩余价值量减少;从而在其他所有条件保持不变的情况下,它也会使利润率降低。如果工人人数减少了,或者虽然工人人数保持不变但用在机器和原料上的资本增加了,就是说,在可变资本与总资本相比出现任何减少,而这种减少在这里[根据假定]没有被工资的下降拉平或部分抵销时,利润率也会下降。但是,因为这个部门的利润率,和其他任何部门的利润率一样,也[1098]参加一般利润率的平均化,所以,奢侈品工业的劳动生产率的提高在这里会引起一般利润率的下降。

相反,如果劳动生产率的提高不是在奢侈品工业本身,而是在向它提供不变资本的那些部门,那末,奢侈品工业的利润率就会提高。

{剩余价值(也就是它的大小、它的量、它的总额)决定于剩余价值率乘在业工人人数。有些情况可能在同一个方向或者在相反的方向同时影响两个因素,也可能仅仅影响其中一个因素。撇开工作日的绝对延长不谈,奢侈品工业的劳动生产率的提高只影响在业工人人数。因此,其必然结果是剩余价值量减少,从而利润率下降——即使不变资本没有增加。如果不变资本增加了,那末,减少了的剩余价值则按照增大了的总资本来计算。}

\centerbox{※     ※     ※}

拉姆赛比其他人更接近于正确地理解利润率。因此,[传统观念的]缺陷在他那里也比在其他人那里表现得明显。他提出了所有的要点,但是提得片面,因而是错误的。

拉姆赛用以下的话总括了他对利润的观点:

\begin{quote}{“因此,单个资本家的利润率决定于下述因素:(1)生产工人衣食等等生活必需品的劳动的生产率;(2)生产加入固定资本的物品的劳动的生产率;(3)实际工资率{实际工资在这里应该是指工人得到的生活必需品等等的数量,而不管属于这些必需品的商品的价格如何}。上述第一个和第三个因素的变化,通过改变总产品中归工人的份额而影响利润。第二个因素的变化,则通过改变用于补偿——直接或经过交换——生产中消费了的固定资本的份额而影响利润;因为利润实质上是个份额问题。”(同上,第172页)}\end{quote}

拉姆赛公正地指责李嘉图(尽管他自己的说明也有缺陷):

\begin{quote}{“李嘉图忘记了,全部产品不仅分为工资和利润,而且还必须有一部分补偿固定资本。”(第174页注)}\end{quote}

\centerbox{※     ※     ※}

{只要对积累,即对剩余价值转化为资本进行初步考察,就可以看到,全部剩余劳动表现为资本(不变资本和可变资本)和剩余劳动(利润、利息、地租)。因为在剩余价值向资本的转化中显出:剩余劳动本身采取资本的形式,工人的无酬劳动作为客观的劳动条件的总和同工人相对立。在这种形式中,客观的劳动条件的总和作为他人的财产同工人相对立,以致作为工人劳动的前提的资本看来似乎和这种劳动无关。资本表现为现成的价值量,而工人只是必须增加它的价值。至于说到剥削,则不是指工人过去劳动的产品(也不是指以下任何情况,这种情况影响或提高过去劳动的[产品的]价值,而与这种过去劳动所进入的特殊劳动过程无关)或这种产品的补偿,而始终只是指工人现在劳动被剥削的方式和程度。只要单个资本家按原有的(或扩大的)规模继续生产,资本的补偿就好象是一种对工人没有影响的行为,因为即使劳动条件归工人所有,他自己也必须用总产品的一部分补偿这些劳动条件,以便按原有的规模继续再生产或者扩大再生产(而后者由于人口的自然增长也是必需的)。但是,资本的这种补偿在三个方面影响工人:(1)劳动条件作为不属于工人的财产,作为资本的永恒化,使工人作为雇佣工人的地位永恒化,从而使工人始终要用自己的一部分劳动时间白白为他人劳动的命运永恒化;(2)这些生产条件的扩大,换句话说,资本的积累,使得靠工人的剩余劳动为生的阶级的数量和人数增多;资本的积累通过使资本家及其同伙的相对财富增多而使工人的状况相对恶化,此外,还通过使工人的相对剩余劳动量增加(由于分工等等),使总产品中归结为工资的份额减少的办法使工人的状况恶化;(3)最后,由于劳动条件以愈来愈庞大的形式,愈来愈作为社会力量出现在单个工人面前,所以,对工人来说,象过去在小生产中那样,自己占有劳动条件的可能性已经不存在了。}

\tchapternonum{[(3)拉姆赛论“总利润”分为“纯利润”(利息)和“企业主利润”。在他关于“监督劳动”、“补偿风险的保险费”和“超额利润”等观点中的辩护论因素]}

[1099]拉姆赛把我仅仅称之为利润的东西称为总利润。他把这个总利润分为纯利润(利息)和企业主利润(企业主收入,产业利润)\fnote{[1130}{西尼耳先生的《大纲》和拉姆赛的《论财富的分配》是大致同时出版的,在后一著作[第二部分]第四章已经详尽地论述了利润分为“企业主利润”和“资本的纯利润”即“利息”;为什么这个在1821年和1822年已经是人所周知的利润划分却被认为是西尼耳先生发明的呢?——这一点只能这样来解释:西尼耳作为纯粹的现状辩护论者,从而作为庸俗经济学家,是深得罗雪尔先生同情的\endnote{马克思指德国庸俗经济学家罗雪尔的著作《国民经济学原理》1858年版第385页。罗雪尔在这里谈到利润分为企业主利润和利息时,引用了西尼耳的《大纲》。马克思指出利润分为“企业主利润”和“资本的纯利润”这一点早在1821和1822年就已经人所周知,这可能是指匿名著作《论马尔萨斯先生近来提倡的关于需求的性质和消费的必要性的原理》(1821年伦敦版〉第52—53页,以及霍普金斯的著作《关于调节地租、利润、工资和货币价值的规律的经济研究》(1822年伦敦版)第43—44页。——第389页。}。}[1130]]。

在一般利润率下降的问题上,拉姆赛同李嘉图一样,也和亚·斯密论战。他反驳亚·斯密说:

\begin{quote}{“诚然,资本主义企业主之间的竞争,可以使大大超过普通水平的利润平均化{这种平均化决不足以解释一般利润率的形成},但是,认为竞争会降低这个普通水平本身,则是错误的。”(第179—180页)“假定每一种商品(原料和成品)的价格,由于生产者之间的竞争而下降是可能的话,那末这一点决不会影响利润。每个资本主义企业主都会把他的产品卖较少的钱,但另一方面,他的每一项支出,不管它属于固定资本还是属于流动资本,都会相应地减少。”(第180—181页)}\end{quote}

拉姆赛也反对马尔萨斯:

\begin{quote}{“利润由消费者支付这种想法显然是十分荒谬的。消费者又是谁呢?他们必定是土地所有者、资本家、企业主、工人,或者领取薪金的人。”(第183页)“唯一能够影响一般总利润率的竞争,是资本主义企业主和工人之间的竞争。”(第206页)}\end{quote}

在最后这一句话里,表达了李嘉图的论点中正确的东西。利润率的下降可以不取决于资本和劳动之间的竞争,但是唯一能够使利润率下降的竞争,却是这种竞争。不过拉姆赛本人并没有给我们指出一般利润率具有下降趋势的原因。他所说的唯一东西是,——这一点是正确的,——利率的下降可以完全不取决于国内的总利润率。就是说:

\begin{quote}{“即使我们假定,借入资本除了用于生产之外,决不用于其他目的,那末,在总利润率没有任何变化的时候,利息仍然可能变化。因为,随着一个民族的财富不断增长,有一类人产生出来并不断增加,他们靠自己祖先的劳动{剥削,掠夺}占有一笔只凭利息就足以维持生活的基金。还有许多人,他们在青壮年时期积极经营,晚年退出,靠他们自己积蓄的钱的利息过安逸的生活。随着国家财富的增长,这两类人都有增加的趋势;这是因为那些在开始时已有相当资本的人,比那些在开始时只有少数资本的人,更容易获得独立的财产。因此,在老的富有的国家,不愿亲自使用资本的人所占有的国民资本部分,在社会全部生产资本中所占的比例,比新垦殖的贫穷的国家大。在英国,食利者阶级的人数是多么多啊!随着食利者阶级的增大,资本贷放者阶级也增大起来,因为他们是同一些人。只是由于这个缘故,利息在老的国家也必定有下降的趋势。”(第201—202页)}\end{quote}

关于纯利润(利息)率,拉姆赛说道:

\begin{quote}{“它部分地取决于总利润率,部分地取决于总利润分为利息和企业主利润的比例。这个比例取决于资本的贷出者和借入者之间的竞争。这种竞争受预期的总利润率的影响,但不是完全由它调节。竞争所以不是完全由它调节,一方面是因为有许多人借钱并不打算用在生产上,另一方面又因为全部国民资本中可贷出的份额,随着国家的财富而变化,不以总利润的任何变化为转移。”(第206—207页)“企业主利润取决于资本的纯利润,而不是后者取决于前者。”(第214页)}\end{quote}

[1100]除开前面提到的情况,拉姆赛还正确地指出:

\begin{quote}{“只有在文明程度已达到不必提出保证偿还贷款的要求的地方,借贷利息才是纯利润的尺度……例如在英国,目前我们不会考虑把承担风险的补偿加进利息中去,因为贷出的资金都有所谓良好的保证。”(第199页注)}\end{quote}

他在谈到他称之为资本主义企业主的产业资本家时,指出:

\begin{quote}{“资本主义企业主是财富的总分配者;他付给工人工资,付给[货币]资本家借贷利息,付给土地所有者地租。一方是企业主,另一方是工人、[货币]资本家和土地所有者。这两大类人的利益正好彼此相反。雇劳动、借资本和租土地的是企业主,他当然力图以尽可能低的报酬使用它们,而这些财富源泉的所有者则力求以尽可能高的价格出租它们。”(第218—219页)}\end{quote}

产业利润(监督劳动)。

总的说来,拉姆赛关于产业利润,特别是关于监督劳动的论述,是这部著作中提出的最合理的东西,尽管他的一部分论证是从施托尔希那里\endnote{马克思指施托尔希的著作《政治经济学教程》1823年巴黎版第1卷第3篇第13章。——第391页。}抄来的。

剥削劳动是要花费劳动的。就资本主义企业主所从事的劳动仅仅由于资本和劳动对立才成为必要这一点来说,这种劳动加入他的监工(工业军士)的费用,并且已经算在工资项下,这种情况跟奴隶监工和监工所用的鞭子的费用算在奴隶主的生产费用中完全一样。这种费用跟大部分商业费用完全一样,属于资本主义生产的非生产费用。凡是谈到一般利润率的地方,资本家之间的竞争以及他们的尔虞我诈所花费的劳动,也不在考察之列;同样,一个资本主义企业主同另一个相比,在花最少费用从自己工人身上榨取最大量剩余劳动并在流通过程中实现这种榨取来的剩余劳动方面,有多大技巧,也不在考察之列。对这一切的考察,属于对资本竞争的研究。这种研究,总的来说,涉及资本家之间以及他们为攫取最大数量的剩余劳动所作的斗争和努力,而且只涉及剩余劳动在不同的单个资本家间的分配,但同剩余劳动的来源及其一般大小无关。

对监督劳动来讲,只剩下组织分工和组织某些个人间的协作这种一般的职能。这种劳动在较大的资本主义企业里完全由经理的工资代表。它已经从可供形成一般利润率的东西中扣除了。英国的工人合作工厂\endnote{关于英国的工人合作工厂,见《资本论》第三卷第五章、第二十三章和第二十七章。并参看本册第552—553页。——第392页。}提供了最好的实际证明,因为这种工厂尽管支付较高的利息,提供的利润还是大于平均利润,即使在扣除了经理的工资——当然,它由这种劳动的市场价格决定——以后也是如此。本身就是经理的那些资本主义企业主,节省了一项生产费用,把工资支付给自己,从而取得高于平均利润率的利润率。如果辩护论者[关于企业主利润是监督工资]的这种说法,明天被认真地实现,如果资本主义企业主的利润只是管理和指挥的工资,那末占有他人剩余劳动并把这种剩余劳动转化为资本的资本主义生产,后天就完结了。

但是,即使我们把监督劳动[的报酬]看成是隐藏在一般利润率中的工资,拉姆赛[同上,第227—231页]以及其他经济学家阐述的规律在这里仍然适用。这个规律就是:在利润(企业主利润和总利润[包括利息])同所花费的资本量成比例时,监督劳动所占的份额同资本量成反比——资本大,这个份额就非常小;资本小,就是说,在资本主义生产仅仅名义上存在的地方,这个份额就非常大。一个几乎完全亲自从事企业中所需要的劳动的小资本家,拿他的资本来比,看来获得很高的利润率,而实际情况却是,他既然没有雇用什么工人,占有他们的剩余劳动,实际上就没有取得丝毫利润,而只是名义上从事资本主义生产(不管是工业还是商业)。这个“资本家”和雇佣工人的区别在于,他由于自己的名义上的资本,实际上是自己劳动条件的主人和所有者,因此没有主人压在头上,[1101]他的全部劳动时间都由他自己占有,而不是被他人占有。在这里,作为利润出现的,只是超过普通工资的余额,这个余额恰恰是由于[这个小所有者]占有自己的剩余劳动而造成的。不过,这种形式仅仅属于资本主义生产方式实际上还没有占支配地位的领域。

[拉姆赛说:]

\begin{quote}{“企业主利润可以分解为(1)企业主薪金;(2)补偿其风险的保险费;(3)他的超额利润。”(同上,第226页)}\end{quote}

至于(2),同这里丝毫没有关系。柯贝特(以及拉姆赛本人[同上,第222—225页])说过\endnote{马克思指柯贝特的著作《个人致富的原因和方法的研究;或贸易和投机原理的解释》1841年伦敦版第100—102页。——第393页。},补偿风险的保险费,只是把资本家的损失平均分摊,或者说,更普遍地在整个资本家阶级中分摊。从这个平均分摊的损失中,必须扣除保险公司的利润,即扣除投在保险事业中并担负这种平均分摊职能的资本的利润。这些保险公司以和商业资本家或货币资本家同样的方式取得一部分剩余价值,而不直接参加剩余价值的生产。这是一个如何在不同种类的资本家中间分配剩余价值以及因此对单个资本进行扣除的问题。它既同剩余价值的性质无关,也同剩余价值量无关。工人当然不可能提供比他的剩余劳动更多的东西。他不可能再另外付给资本家一笔钱,为资本家占有这种剩余劳动的果实保险。至多可以说,即使不谈资本主义的生产,生产者在这方面也会有一定的支出,就是说,他们必须支出自己的一部分劳动或者说一部分劳动产品,以防自己的产品、财富或财富的要素遇到意外等等。代替每个资本家自行保险的是,他用[总]资本的一定部分专门担负这项业务,这样,就更可靠、更便宜地取得相同的结果。保险费以一部分剩余价值支付;剩余价值在资本家之间的分配和剩余价值的保险,跟剩余价值的来源和数量无关。

因此,有待考察的是:第一,企业主的“薪金”;第二,“超额利润”——拉姆赛在这里用来称呼剩余价值的一部分,这部分属于和食利的资本家不同的产业资本家,因此它的绝对量也决定于利息与产业利润之比,即属于资本(不同于土地所有权)的那个剩余价值部分所分解成的两个部分之比。

至于说到企业主的“薪金”,那末首先不言而喻的是,在资本主义生产中,资本作为劳动的统治者的职能,落在资本家或者由资本家付酬的职员——资本家的代理人的身上。这种职能只要不是由合作劳动的性质,而是由劳动条件对劳动本身的统治产生的,它也就要随着资本主义生产一起消失。然而,拉姆赛本人把[企业主利润的]这个组成部分抛掉了,或者把它降低到不值一谈的地步:

\begin{quote}{“不管企业大小,企业主的薪金也和监督劳动一样,几乎是相同的。”(同上,第227页)“一个工人决不能说,他能够完成两个、三个或者更多象他一样的工人所能完成的工作。但是一个工业资本家或租地农场主却可以代替十个或更多象他一样的人。”(第255页)}\end{quote}

企业主利润的第三部分是“超额利润”(包括风险费,这种风险只是可能的,只是利润和资本的可能的损失,而实际上表现为保险费,从而表现为特定部门的一定资本在总剩余价值中所取得的份额)。

\begin{quote}{拉姆赛说道:“这种超额利润不折不扣地代表那种从支配资本使用权的权力中{换句话说,从支配他人劳动的权力中}产生的收入,不管这个资本是属于这个资本家本人还是从别人那里借来……纯利润〈利息〉完全随资本的大小而变化;反过来,资本越大,超额利润对所使用的资本之比也就越大。”(第230页)}\end{quote}

换句话说,这不过意味着,“企业主的薪金”与资本的大小成反比。资本活动的规模越大,生产方式越是资本主义的,产业利润中可以归结为“薪金”的组成部分就越微小,产业利润就越清楚地表现出它的真正性质:它是“超额利润”,即剩余价值,亦即无酬剩余劳动的一部分。

产业利润和利息的全部对立,只有基于食利者和产业资本家的对立,才有意义,但是它完全不涉及工人和资本的关系,不涉及资本的性质,也不涉及资本的利润的来源等等。

关于不是生产谷物、而是生产其他农产品的土地的地租,拉姆赛说:

\begin{quote}{“这样一来,为一种产品支付的地租,成了其他产品价值高的原因。”(同上,第279页)}\end{quote}

在最后一章[《论国民收入》]中,拉姆赛说:

\begin{quote}{“收入和年总产品的区别仅仅在于,收入中没有用于维持固定资本〈在拉姆赛那里就是指不变资本,即各生产阶段上的原料、辅助材料和机器等等〉的一切东西。”(第471页)}\end{quote}

[1102]拉姆赛在前面已经讲过\fnote{见本册第361—362页。——编者注}并且在最后一章再次讲到,

\begin{quote}{“流动资本”〈在他那里就是指花在工资上的资本〉是多余的,“它既不是生产的直接因素,甚至对生产也毫无重要意义”。(第468页)}\end{quote}

拉姆赛不过没有由此得出如下的明显结论:否定雇佣劳动和花在雇佣劳动上的资本,也就否定了整个资本主义生产的必要性,从而劳动条件就不再作为“资本”,或者用拉姆赛的术语,不再作为“固定资本”同工人相对立了。劳动条件的一部分之所以表现为固定资本,只是因为另一部分表现为流动资本。但是,一经把资本主义生产假定为事实,拉姆赛就宣布了工资和资本的总利润(包括产业利润,或按他的说法,企业主利润)是收入的必要形式(第475、478)。

当然,这两种形式的收入实际上把资本主义生产以及作为其基础的两个阶级的本质最简单最一般地概括起来了。可是,他却把地租,即土地所有权,说成对资本主义生产是多余的形式(第472页),他忘记了,地租是这种生产方式的必然产物。以上所述也适用于他的另一个论点,即“资本的纯利润”,或者说,利息,不是一种必要的形式:

\begin{quote}{“食利者[在总利润急剧下降的情况下]只好转变为产业资本家。这对于国民财富是无关紧要的……给资本的所有者和使用者提供各自的收入,无疑地不需要那么高的纯利润。”(第476—477页)}\end{quote}

在这里,拉姆赛又忘记了他自己说过的话:随着资本的发展,必然形成一个不断增大的食利者阶级。\fnote{见本册第390—391页。——编者注}

[拉姆赛说:]

\begin{quote}{“资本的总利润和企业主利润……对于生产的过程是必需的。”(第475页)}\end{quote}

当然。没有利润就没有资本,而没有资本就没有资本主义生产。

\centerbox{※     ※     ※}

总之,从拉姆赛那里得出的结论是,第一,建立在雇佣劳动基础上的资本主义生产方式,不是社会生产的必然的即绝对的形式。(拉姆赛本人仅仅以一种带局限性的说法来叙述这个观点,他说,如果不是人民大众那么穷,以致不得不在产品完成以前预支自己在产品中应得的份额,“流动资本”和“工资”就是多余的。)第二,与产业利润不同的利息,和地租(即由资本主义生产本身创造的土地所有权形式)一样,对资本主义生产来说,是不必要的,而且是可以被它扔掉的累赘。如果这种资产阶级的理想真正实现的话,结果只能是,全部剩余价值直接落在产业资本家手中,社会(在经济方面)就会归结为资本与雇佣劳动的简单对立——这种简化无疑会加速这种生产方式的灭亡。[1102]

\centerbox{※     ※     ※}

[1102]{在1862年12月1日的《晨星报》\endnote{《晨星报》(《ThemorningStar》)是英国的一家日报,自由贸易派的机关报;1856年到1869年在伦敦出版。——第397页。}上,一个工厂主抱怨说:

\begin{quote}{“从总产品中扣除工资、地租、资本利息、原料费用以及经纪人、商人的赢利,剩下的才是工厂主、郎卡郡居民、业主的利润,而且他们还得为这么多参与总产品分配的人负担工人的生活费。”}\end{quote}

如果把价值放在一边,来考察实物形式的总产品,那就很明显,在补偿了不变资本和花在工资上的资本之后,剩下的是代表剩余价值的产品部分。但是,从这个剩余中,要扣除一部分作为地租和经纪人、商人的赢利,不论这些人是否使用自己的资本——这一切都取自总产品中代表剩余价值的部分。因此,这一切对工厂主来说是一种扣除。如果工厂主的资本是借来的,那他的利润本身也要分成产业利润和利息。}

{关于级差地租:在比较肥沃的土地上劳动的工人,比起在比较不肥沃的土地上劳动的工人,劳动生产率要高些。因此,如果前一个工人以实物形式得到报酬,那末他在总产品中取得的份额就小于在比较不肥沃的土地上劳动的工人。或者同样可以说,尽管他每天劳动的时数相同,他的相对剩余劳动却大于另一个工人。但是,他的工资和另一个工人的工资的价值是相同的。因此他的雇主的利润也并不比另一个雇主的利润大。在他的产品的超额部分中包含的剩余价值,他的较高的相对劳动生产率,或者说,他的级差剩余劳动,被土地所有者装进了腰包。}[1102]

\tchapternonum{[第二十三章]舍尔比利埃}

[1102]舍尔比利埃《富或贫》1841年巴黎版(从日内瓦版翻印)。

(我们是把这个人专门归入[政治经济学家]这个行列呢,——因为他的观点大部分是西斯蒙第的,——还是把他的比较中肯的见解在适当场合作为引文列举出来,这还是一个大问题。\endnote{这里指《剩余价值理论》的计划。马克思在手稿第XIV本的封面所写的《理论》最后几章的计划包括《(n)舍尔比利埃》(见本卷第1册第5页),本章就是根据这一点写成的。至于西斯蒙第,马克思不打算在《剩余价值理论》中研究他的观点,而打算在自己著作的下一部分,即阐述《资本的竞争和信用》的部分来研究(见本册第52页〉。——第399页。})[1102]

\tchapternonum{[(1)把资本区分为两部分:由机器和原料构成的部分以及由工人的“生活资料基金”构成的部分]}

\begin{quote}{[1103]舍尔比利埃说:“资本就是原料、工具、生活资料基金[approvisionnement]。”(第16页)“资本同财富的其他任何部分之间没有任何区别。只是由于特殊的使用方式,物才成为资本,就是说,只有它被当作原料、工具或生活资料基金在生产行为中加以使用,它才成为资本。”(第18页)}\end{quote}

可见,这是一个普通的方法,即把资本归结为它在劳动过程中所表现的物质要素:劳动资料和生活资料。而且,把资本归结为生活资料的说法是不确切的,因为生活资料虽然是生产者在生产中生存的条件、前提,但是并不加入劳动过程本身;加入劳动过程的,只有劳动对象、劳动资料和劳动本身。因此,劳动过程的客观因素——它们对一切生产形式都是共同的——在这里被称为资本,虽然“生活资料基金”(工资已包含在内)默默地以这些劳动条件的资本主义形式为前提。

舍尔比利埃和拉姆赛完全一样,认为“生活资料基金”——拉姆赛称之为流动资本——会减少(同资本总量相比至少会相对减少,在机器不断排挤工人的情况下则会绝对减少)。但是,他和拉姆赛似乎都认为,可以作为生产资本使用的生活资料即生活必需品的量必然会减少。情况完全不是这样。在这里,总产品中补偿资本并当作资本使用的部分和代表剩余产品的部分,总是被混为一谈。“生活资料基金”会减少,是因为资本(即总产品中当作资本使用的部分)中有一大部分已经不是以可变资本的形式,而是以不变资本的形式被再生产出来。[另一方面,]较大一部分由生活资料构成的剩余产品,则被非生产劳动者和完全不劳动的人吃掉,或者用来交换奢侈品。如此而已。

当然,总资本中转化为可变资本的部分越来越小这个事实,也可以用另外的方式来表示。资本中由可变资本组成的部分,等于总产品中工人自己占有、为自己生产的部分。因此,这部分越小,再生产它所需的工人人数在工人总数中占的比例也就越小(单个工人情况也是这样:他为自己劳动的劳动时间也就越少)。同总劳动一样,工人的总产品也分解为两部分。一部分是工人为自己生产的,另一部分是为资本家生产的。同单个工人的劳动时间可以分成两部分一样,整个工人阶级的劳动时间也可以分成两部分。如果剩余劳动等于半个工作日,那末这就象是工人阶级中有一半为工人阶级生产生存资料,而另一半则为资本家——他们一方面作为生产者,一方面作为消费者——生产原料、机器和成品。

可笑的是,舍尔比利埃和拉姆赛都以为,总产品中能够由工人消费、能够以实物形式加入工人消费的部分必然会减少,或者一般说来会减少。会减少的只是以这种形式,即作为可变资本被消费的部分。相反,被仆人、士兵等等吃掉,或者被输出国外换取更讲究的生活资料的那一部分则会更大。

在拉姆赛和舍尔比利埃的著作中只有一点是重要的,即他们实际上把可变资本和不变资本相对立,而不是停留在从流通中得出的固定资本和流动资本的区分上。因为,舍尔比利埃把资本中归结为“生活资料基金”的部分,同由原料、辅助材料和劳动资料(工具、机器)组成的部分对立起来。不过,不变资本中的两个组成部分——原料和辅助材料,就它们的流通形式来说,都属于流动资本。

在资本组成部分的变动中,重要的不是生产原料和机器的工人相对来说多于直接生产生活资料的工人(这只是分工而已),重要的是,产品应按什么比例补偿过去劳动(即不变资本)和支付活劳动。资本主义生产的规模越大,——从而积累资本越大,——用来生产机器和原料的资本所转化成的机器和原料,在[总]产品价值中占的份额也就越大。因此,以实物形式,或者通过不变资本各不同部分的生产者之间的交换,必然返回生产的产品部分就越大。属于生产的产品部分的比例也就更大,代表活劳动,新加劳动的部分相对来说也就更小。当然,这后一部分表现在商品上,表现在使用价值上,也会增加,因为上述事实和劳动生产率的提高有相同的意义。但是,相对来说,这部分中归工人所有的部分,还会更加减少。而且这同一过程会引起工人人口经常的相对过剩。

\tchapternonum{[(2)关于工人人数同不变资本量相比不断减少的问题]}

[1104]{随着资本主义生产的发展,投在机器和原料上的资本部分在增加,花在工资上的资本部分在减少,这是不容争辩的事实。这是使拉姆赛和舍尔比利埃唯一感到兴趣的问题。但对我们来说,主要的是:这个事实是否说明利润率降低(而且这种降低远不象所说的那么厉害)?同时这里问题不仅涉及量的比例,而且涉及价值比例。

如果现在一个工人纺的棉花能和过去100个工人纺的一样多,原料就必须增加到100倍,此外,这个过程只有依靠一台能使一个工人看管100枚纱锭的纺纱机才能实现。但是,如果与此同时一个农业工人现在生产的棉花和过去100个工人生产的一样多,一个机器制造业的工人现在是生产一整台纺纱机,而不是生产一枚纱锭,那末价值比例保持不变,也就是说现在花在纺纱、棉花和纺纱机上的劳动和过去花在纺纱、棉花和纱锭上的劳动完全相同。

至于机器,它的费用不象它所代替的劳动的费用那么大,虽然纺纱机比纱锭贵得多。有一台纺纱机的单个资本家,所拥有的资本必然大于购买一架纺车的单个纺纱者。但是,如果把纺纱机所需的工人人数考虑在内,使用纺纱机就比使用纺车便宜。否则,纺纱机就排挤不了纺车。资本家取代了纺纱者。但是,纺纱者花在纺车上的资本,与产品量相对来说,大于资本家花在纺纱机上的资本。}

劳动生产率的增长(由于使用机器)和工人人数的减少(同使用的机器的数量与功率相比),二者是一回事。代替简单而便宜的工具的是这类工具(虽然形式有了改变)的组合,此外,还加上由发动装置和传动装置组成的整套机器;然后是用来生产动力(如蒸汽)的材料(如煤等等),最后是建筑物。如果一个工人看管1800枚纱锭,而不是转动一架纺车,那末如果问,为什么这1800枚纱锭不象一架纺车那样便宜,那就是再荒唐不过了。在这里,生产率正是取决于以机器形式使用的资本量。机器磨损的比例只和商品有关;工人同全部机器相对立,因而花在劳动上的资本的价值也同花在机器上的资本的价值相对立。

毫无疑问,机器变得便宜是由于两个原因:由于制造机器的原料是用机器生产的;由于在把这种原料变成机器时使用机器。但是这样说包含着两层意思:第一,在这两个部门,拿它们采用的机器和工场手工业生产中使用的工具相比,花在机器上的资本同花在劳动上的资本相对来说,在价值上增加了。第二,单个机器和它的组成部分变得便宜了,但是发展起一个机器体系:代替工具出现的不仅是单个机器,而且是整个体系,从前可能是起主要作用的工具,例如(织袜机或类似的机器上的)织针,现在是成千上万地结合在一起。同工人相对立的每一台机器,都是工人从前一个个单独使用的工具的庞大组合,例如1800枚纱锭代替了一枚纱锭。但是除此以外,机器还包含旧工具所没有的要素等等。尽管各单个要素便宜了,机器的总体在价格上却大大提高了,生产率的增长就是由于这个总体的不断扩大。

其次,机器所以变得便宜,除了由于它的组成部分变得便宜以外,还由于动力源泉(例如蒸汽锅炉)和传动装置也变得便宜了。动力节约了。但是所以能够节约,正是因为同一个发动机,由于规模的不断增大,能够推动更大的机器体系。发动机相对地说变得便宜了,或者说,它的费用不是和用它来推动的机器体系的扩大成比例地增长;它本身随着自己的规模的增大而变贵,但它的价格不是和它的规模的增大成比例地提高;即使它的费用绝对地说是增长了,但是相对地说还是减少了。因此,撇开单个机器的价格不谈,这是一个使得与劳动对立的机器资本增大的新的因素。机器运转速度增加,会大大提高生产力,但是同机器价值本身毫无关系。

因此,机器价值的增长(与使用的劳动量相比较,因而也是与劳动价值,可变资本相比较),同机器引起的劳动生产率的增长相适应,这种说法,是不言而喻的,或者说是同义反复。

[1105]造成商品由于使用机器而降价的一切情况,首先可以归结为单位商品吸收的劳动量的减少,其次可以归结为机器磨损(这种磨损的价值加入单位商品)的减少。机器磨损得越慢,再生产它们所需的劳动就越少。这就使得由机器构成的资本的量和价值,同以劳动形式存在的资本相比,有了增加。

这样一来,剩下的就只是原料问题。很明显,原料量必须同劳动生产率成比例地增长,也就是说,原料量必须同劳动量成比例。这个比例比它表面上看起来要大。

例如,假定每个星期消费10000磅棉花。一年按50个星期计算,全年共消费10000×50=500000磅。假定全年的工资=5000镑。每磅棉花比如值6便士,全年就是250000先令=12500镑。假定资本每年周转5次。这样,全年的1/5就消费100000磅棉花,价值2500镑。在这1/5年中,支出工资1000镑,即等于投在棉花上的资本价值的1/3强。但是,这丝毫不影响原料量同劳动量的比例。如果现在每1/5年中所消费的棉花的价值等于10000镑,劳动的价值等于1000镑,那它们的比例就是1∶10。(如果考察全年的产品,即一方面是50000镑,另一方面是5000镑,这个比例同样是1∶10。)

{商品的价值就其与机器有关来说,决定于加入商品的机器磨损;因此,只是在机器的价值本身加入价值形成过程,即机器的价值在劳动过程中被消费的情况下,商品的价值才决定于机器的价值。相反,利润却决定于(撇开原料不谈)进入劳动过程的全部机器的价值,而不管这个价值被消费的程度如何。因此,利润必然随着[活]劳动总量的减少(同花在机器上的资本部分相对来说)而下降。利润并不是按照相同的比例下降,因为剩余劳动在增加。}

在原料方面,可以提出这样一个问题:假如纺纱业的生产力提高十倍,也就是一个工人现在纺的纱和过去十个工人纺的纱一样多,那末,为什么一个黑人现在生产的棉花不可以和过去十个黑人生产的棉花一样多,也就是说,为什么不可以使价值比例在这里保持不变呢?纺纱者在同一时间里纺掉十倍的棉花,但是黑人在同一时间里也生产十倍的棉花。因此,十倍的棉花量,并不比以前等于它的1/10的棉花量贵。所以,尽管原料量增加了,它对可变资本的价值比例却可以保持不变。实际上,这个工业部门一般说来所以能够这样发展起来,完全是棉花大降价的结果。\fnote{[1105}{如果明天棉花降价90%,那末后天,纺纱业就会发展得更快,等等。}[1105]]材料(例如金和银)越贵,用它来制作奢侈品时使用机器和分工就越少。这是因为在原料上资本支出太大,而且由于原料昂贵,对这些产品的需求有限。

对上面提出的问题,可以非常简单地回答如下:一部分原料,如毛、丝、皮革,是通过动物性有机过程生产出来的,而棉、麻之类是通过植物性有机过程生产出来的;资本主义生产至今不能,并且永远不能象掌握纯机械方法或无机化学过程那样来掌握这些过程。象皮革等等以及动物的其他组成部分这类原料所以变得昂贵,部分原因就在于不合理的地租规律随着文明的进步使这些产品的价值提高了。至于煤和金属(以及木材),它们随着生产的发展已变得非常便宜;然而在矿源枯竭时,金属的开采也会成为比较困难的事情,等等。

{关于谷物地租和矿山地租,如果可以说,它们并没有提高产品的价值(只是提高了它的市场价格),它们只不过是产品价值的表现(产品价值超过生产价格的余额),那末相反,毫无疑问的是,牲畜租、房租等等就不是这些产品价值提高的结果,而是原因。}

原料、辅助材料等的降价,使资本的这个部分的价值增长变慢,但没有使增长停止。这种降价在一定程度上抑制了利润率的下降。

这个讨厌的问题到此结束。[1105]

[1105]{在考察利润时,假定剩余价值是既定的。只考察不变资本的变动对利润率的影响。只有一种方法使剩余价值直接影响不变资本,那就是通过绝对剩余劳动,通过延长工作日,使不变资本在[产品]价值中占的比例减小。相对剩余劳动——在工作日保持不变的场合(撇开劳动强度增大不谈)——通过剩余价值本身的提高使利润对总资本的价值比例增大。绝对剩余劳动时间则使不变资本的费用相对减少。}

\tchapternonum{[(3)舍尔比利埃关于利润率取决于资本有机构成的猜测;他在这个问题上的混乱。舍尔比利埃论资本主义条件下的“占有规律”]}

[1106]现在回过头来谈舍尔比利埃。

他提出的利润率的公式,或者说是用数学来表示通常所理解的利润,本身并不包含任何规律;或者说甚至是绝对错误的,尽管他对这个事物有某种模糊的概念,接近于对它的了解。

\begin{quote}{“商业利润\endnote{“商业利润”(《profitmercantile》)是舍尔比利埃对单个资本家的利润的称呼,以区别于整个社会的利润。--第407页。}决定于同生产资本各不同要素相比的产品价值。”[同上,第70页]}\end{quote}

{实际上,利润是产品的剩余价值与总预付资本价值之比,而与资本各要素的区别无关。但是剩余价值本身决定于可变资本的量和它的价值增殖率;而这个剩余价值与总资本之比,又决定于可变资本与不变资本之比,也决定于不变资本的价值变动。}

\begin{quote}{“这种规定的两个主要要素,显然是原料价格和加工这些原料所必需的生活资料基金的数量。社会的经济进步,以相反的方向作用于这两个要素。这种进步具有使原料变贵的倾向,因为它使在面积有限的私人土地上经营的采掘业\endnote{舍尔比利埃所说的“采掘业”(《industriesextractives》)不仅指采矿、伐木、捕鱼、狩猎,而且指生产农产原料的各种农业生产。——第407页。}的一切产品的价值提高。”相反,生活资料基金却随着社会进步而(相对)减少,关于这一点我们以后再谈。“产品总量减去为获得这些产品而消费的资本总量,就得出一定时期内的利润总量。产品总量同使用的资本,而不是同已消费的资本成比例地增长。因此,利润率,或者说,利润与资本之比,是另外两个比——使用的资本与已消费的资本之比以及已消费的资本与产品之比——结合的结果。”(同上,第70页)}\end{quote}

舍尔比利埃一开始说对了,利润决定于同生产资本“各不同要素”相比的产品价值。突然他又跳到产品本身,跳到产品量上去了。但是第一,产品量的价值不增加,产品量也可以增加。第二,在把产品量同构成已消费的和未消费的资本的产品量进行比较时,最多也只能按照拉姆赛的办法去做,这就是,把总国民产品同它的以实物形式耗费的构成要素相比较。\fnote{见本册第371—372页。——编者注}但是,就每一单个生产领域的资本来说,产品的形式和它的构成要素是不同的(即使在农业这一类生产部门里也是如此,在那里,一部分产品以实物形式构成该产品的生产要素)。为什么舍尔比利埃会走上这条歧路呢?因为,尽管他猜测到资本的有机构成对利润率有决定意义,但是他完全没有利用他探索到的不变资本和资本的另一组成部分之间的对立来说明剩余价值;正如他没有说明价值本身一样,他也根本没有说明剩余价值。他没有指出剩余价值是从哪里来的,所以就去求助于剩余产品,即求助于使用价值。

虽然任何剩余价值都表现为某种剩余产品,但是剩余产品本身不代表剩余价值。{假定产品根本不包含剩余价值,例如,一个农民有自己的工具(再加上自己的土地),他劳动的时间正好只是一个雇佣工人为补偿自己的工资而劳动的时间,比方说6小时。如果是丰年,他的产品可能加倍,但是全部产品的价值仍然和过去一样。在这种情况下,虽然有剩余产品,却没有剩余价值。}

舍尔比利埃用“生活资料基金”这种“被动的”、纯物质的形式,即用可变资本在工人手中转化成的使用价值形式来表示可变资本,这本身就已经是错误的。相反,如果他按照可变资本的实际表现形式来看待可变资本,就是说把它看作货币(交换价值的即一定量社会劳动时间本身的存在),那末对资本家来说,可变资本就会转化为他用可变资本交换得来的劳动(而在物化劳动同活劳动的这种交换中,可变资本会发生变动,它会增长);可变资本是作为劳动,不是作为“生活资料基金”,而成为生产资本的要素的。而“生活资料基金”是使用价值,是可变资本借以实现为工人的收入的使用价值的物质存在。所以,作为“生活资料基金”,可变资本完全同舍尔比利埃称为“被动的”要素的另外两个资本部分\fnote{[1110}在第59页上,舍尔比利埃把原料和机器称为“资本的两个被动的要素”同“生活资料基金”相对立。[1110]]一样是“被动的”要素。

同一个错误观念妨碍舍尔比利埃通过这个主动要素与被动要素之比,去说明利润率和随着社会发展而出现的利润率的下降。事实上,他得出的无非是这样一个结论:“生活资料基金”[1107]由于生产力的发展而减少,同时工人人口却在增长;因此,工资由于人口过剩而降到它的价值以下。他没有在价值交换的基础上,即在劳动能力按价值支付的基础上,说明任何问题,这样,利润实际上(尽管他没有说出来)就表现为工资的扣除部分;当然,实际的利润有时也可能包含这个扣除部分,但是后者永远不可能成为利润范畴的根据。

首先,让我们把舍尔比利埃的第一个论点还原为它的正确的表达:

\begin{quote}{“产品总量的价值减去为获得〈生产〉这些产品而消费的资本总量的价值,就得出一定时期内的利润总量。”}\end{quote}

这就是利润的第一个(通常的)表现形式,对资本主义意识来说也是这样。换句话说:利润是在一定时期内获得的产品价值超过已消费的资本价值的余额。或者说:是产品价值超过产品的生产费用的余额。甚至这个“一定时期”在舍尔比利埃那里也是突如其来的,因为他没有向我们说明资本的流通过程。因此,他的第一个论点不外是利润的普通定义,是利润的直接表现形式。

舍尔比利埃的第二个论点:

\begin{quote}{“产品总量同使用的资本,而不是同已消费的资本成比例地增长。”}\end{quote}

换句话说,这依然是:

\begin{quote}{“产品总量的价值同预付资本〈不管它是否已经消费〉成比例地增长。”}\end{quote}

这里的目的只是想用狡猾的手法得出利润量决定于使用的资本量这样一个完全未经证明,而且在直接表述上也是错误的论点(因为这个论点已经把[个别利润]平均化为一般利润率这一点当作前提)。但是,由于“产品总量同使用的资本,而不是同已消费的资本成比例地增长”,就必然形成一种表面的因果关系。

让我们看一下这个论点的两种表述,一种是舍尔比利埃的表述,一种是它应该有的表述。从它的上下文来看,并且根据它被当作:mediusterminus\fnote{(推理的)中词。这里是第二个前提的意思。——编者注}而推出的结论来看,它应该是:

\begin{quote}{“产品总量的价值同使用的资本,而不是同已消费的资本成比例地增长。”}\end{quote}

这里显然是想用下列说法来巧妙地解释剩余价值:使用的资本超过已消费的资本的余额构成产品的价值余额。但是,未消费的资本(机器等等)保存着自己的价值(因为,“未消费”,正是说它的价值未消费),它在生产过程结束后,仍然保存着它在生产过程开始前具有的价值。如果发生了价值变动,这种变动也只能发生在已消费的,从而已加入价值形成过程的资本部分。舍尔比利埃解释利润的方法,从下面这一点来看实际上也是错误的:例如一笔三分之一未消费、三分之二在生产中已消费的资本,在劳动剥削率相等(撇开利润率的平均化不谈)时,必然比另一笔三分之二未消费、三分之一已消费的资本提供更多的利润。因为后一笔资本包含的机器等等和其他的不变资本显然多一些,而前一笔资本包含的这个要素少一些,推动的活劳动量多一些,从而推动的剩余劳动也多一些。

如果我们看一下舍尔比利埃本人对自己的论点所作的表述,那末首先应该指出,这种表述不会给他带来什么好处,因为产品量或者说使用价值量本身,不论是对于价值、剩余价值,还是对于利润,都根本不起决定作用。然而在这一切的背后隐藏的究竟是什么呢?由机器等等组成的不变资本部分,加入劳动过程,不加入价值形成过程,因而它有助于产品量的增加,却不在其价值上附加任何东西。(因为在它通过本身的磨损给产品附加价值的情况下,它自己也就属于已消费的资本,不属于同已消费的资本相区别的使用的资本。)但是不变资本的这个未消费的部分本身并不造成产品量的增长。它有助于在一定的劳动时间内创造更大量的产品。因此,如果劳动只是在“生活资料基金”所包含的那么多劳动时间内进行,产品量就会保持不变。所以,产品的余额不是由使用的资本超过已消费的资本的余额构成,而是由这个已消费的资本部分发生的变动造成的(前提是,这里讲的不是象农业这一类生产部门,在这类部门中,产品量不取决于或者可以不取决于所支出的资本量,劳动生产率部分地取决于无法控制的自然条件)。

如果舍尔比利埃把不变资本——已消费的或未消费的——看成同劳动时间[长度]无关,同可变资本在价值增殖过程中发生的变动无关,他同样可以说:

\begin{quote}{“产品总量[1108]的增长〈至少在加工工业中〉,同已消费的资本中由原料组成的部分的增长成比例。”}\end{quote}

因为产品的增长和资本的这个部分的增长在物质上是等同的。另一方面,在农业中(在采掘业中也是一样),土地比较肥沃时,在未消费的资本(即不变资本)用得少、已消费的资本(例如工资)用得比较多的地方,产品量可能比先进的国家大得多,在先进的国家,使用的资本与已消费的资本之比要高得多。

这样,舍尔比利埃的第二个论点就是企图用巧妙的手法偷运剩余价值(利润的必要基础)。

[舍尔比利埃得出的结论是:]

\begin{quote}{“因此,利润率,或者说,利润与资本之比,是另外两个比例——使用的资本和已消费的资本之间的比例以及已消费的资本和产品之间的比例——结合的结果。”(第70页)}\end{quote}

应当先说明利润。但是,代替这个说明的只是对利润下了这样一个定义,这个定义只是表示了利润的表现方式,只是表示了利润是总产品的价值超过产品的生产费用,或者说超过已消费的资本价值的余额这样一个事实,就是说,对利润下了一个平平常常的定义。

现在应当说明利润率。但是,又只是下了一个平平常常的定义:利润率是利润与总资本之比,或者也可以说是产品价值超过它的生产费用的余额与预付在生产上的总资本之比。可见,对于资本要素的近似正确的区分加以歪曲的理解和拙劣的运用,以及对于利润和利润率同这些要素的比例的直接关系所作的猜测,只是使舍尔比利埃以更加明显的学理主义的形式重复那些人所共知的词句,事实上,这些词句只是确认了利润和利润率的存在,关于它们的本质却什么也没有谈到。

舍尔比利埃用代数的方法来表示他的学理主义的公式,这也无济于事:

\begin{quote}{“用P表示一定时期的总产品,C表示使用的资本,π表示利润,r表示利润和资本的比例(利润率),c表示已消费的资本。这样,P-c=π,r=π/C,即Cr=π。因此P-c=Cr;r=(P-c)/C。”(第70页注)}\end{quote}

这一切只是表示,利润率等于利润与资本之比,而利润等于产品价值超过产品生产费用的余额。

一般说来,当舍尔比利埃说到已消费的资本和未消费的资本时,他脑子里想的是固定资本和流动资本的区别,而并不坚持他自己所确认的、与这种区别不同的从生产过程产生的资本的区别。剩余价值在流通之前就已被假定了;不管从流通中产生的区别怎样影响利润率,这些区别与利润的来源都毫无关系。

\begin{quote}{“生产资本由可消费的部分[生活资料基金、原料、辅助材料]和不可消费的部分[工具、器具、机器]组成。随着财富和人口的增长,可消费的部分有增长的趋势,因为采掘业需要越来越大的劳动量。另一方面,这同一个发展又使得使用的资本量的增长程度大大高于已消费的资本量的增长程度。因此,虽然已消费的资本总量有增长的趋势,但这一过程的影响会受到抑制,因为产品量会以更快的速度增长,并且必须承认,利润总量至少是和使用的资本总量一样快地增长的。”(第71页)“增长的是利润量,而不是利润率即这个量与使用的资本之比,r=(P-c)/C。显然,如果C比P-c增长得快,即使r下降,P-c或者说利润(因为P-c=π)也可能增长。”(第71页注)}\end{quote}

在这里,还算在某种程度上接触到了利润率下降的原因;但是,有了先前的歪曲以后,这只能导致混乱和自相抵触的矛盾。起先是已消费的资本量增长,但是产品量增长得更快(就是说,在这里产品价值超过产品生产费用的余额增长得更快),因为产品量同使用的资本成比例地增长,而后者比已消费的资本增长得快。为什么固定资本例如比原料量增长得快,这一点在任何地方都没有说明。但是且不管它。利润量同使用的资本,同总资本成比例地增长,可是[1109]利润率据说还是要下降,因为总资本比产品量,或者不如说,比利润量增长得快。

舍尔比利埃先是说利润量至少是和“使用的资本总量”按同样的比例增长,可是后来又说利润率下降,因为使用的资本总量比利润量增长得快。起先是P-c“至少是”和C成比例地增长,后来又是(P-c)/C下降,因为C比“至少是和C一样快地增长的”P-c增长得更快。如果去掉这一切混乱,剩下的就只是如下的同义反复:即使P-c增长,(P-c)/C也可能下降,也就是说,如果利润率下降,即使利润增长,利润率也可能下降。利润率只是指P-c与C之比,如果资本比利润量增长得快,[这个比例就下降]。

于是便得出如下的聪明的结论:如果资本比利润量增长得快,或者说,如果利润量尽管绝对增长,但是和资本相比却相对减少,那末,利润率可能下降,即增长的利润量与资本之比可能下降。这无非是利润率下降的另一种表现。对这种现象的可能性,甚至对它的存在,从来都没有人怀疑过。这里涉及的唯一的问题恰恰是要说明这种现象的原因,而舍尔比利埃却用利润量的增长至少是同资本的增长成比例,来说明利润率的下降,说明与总资本相比利润量的下降!他显然模糊地猜测到,使用的活劳动量,尽管绝对地说增加了,但是与过去劳动相比还是相对减少;因此利润率必然下降。但是他的这种猜测还不是清醒的理解。越接近入门,而实际上并未入门,表述上的歪曲程度就越大,并且认为已经入门的错觉就越大。

相反,舍尔比利埃关于一般利润率的平均化所说的,倒很中肯。\endnote{在手稿中接下去是《资本论》第三部分第二章的计划草稿,作为插入部分放在方括号内。在这里,马克思打算研究一般利润率的形成问题。本版把这个计划收入本卷第1册的《附录》(第447—448页)。——第415页。}[1109]

\begin{quote}{[1109]“扣除地租之后,利润量——即产品超过已消费的资本的余额——的剩余部分,在资本主义生产者之间,按照他们每人使用的资本的比例进行分配,而与已消费的资本相应的并确定用来补偿它的那一部分产品,则按照他们实际消费掉的资本的比例进行分配。这种二重分配规律由于力图把各方面使用的资本的收益平均化的竞争的作用而得以实现。这种二重分配规律最终决定不同种类产品的相应的价值和价格。”(第71—72页)}\end{quote}

这一段很好。只是最后一句话,即一般利润率的这种形成决定商品的价值和价格(应该说生产价格)是错误的。相反,价值规定是第一性的,是利润率的前提,也是生产价格形成的前提。“利润量”——即剩余价值,[1110]它本身只是商品总价值的一部分——的某种分配,又怎么能够决定这个“利润量”,因而决定剩余价值,因而也决定商品价值本身呢?只有把商品的相对价值理解为商品的生产价格,舍尔比利埃的说法才是正确的。舍尔比利埃的全部错误都是由于他没有独立地考察价值和剩余价值的起源和规律。

此外,对雇佣劳动和资本的关系,他的理解在一定程度上也是正确的:

\begin{quote}{“没有通过让渡〈合法转让财产、继承等等〉得到什么东西,也没有什么东西可以拿去进行交换的人,只有向资本家提供自己的劳动,才能得到他们所需要的东西。他们只有权得到作为劳动价格付给他们的东西,而无权得到这种劳动的产品以及他们附加在产品上的价值。”(第55—56页)“无产者为换取一定量的生活资料出卖自己的劳动,也就完全放弃了对资本其他部分的任何权利。这些产品的占有还是和以前一样;并不因上述的[无产者和资本家之间的]契约而发生变化。产品完全归提供原料和生活资料的资本家所有。这是占有规律的严格结果,相反地,这个规律的基本原则却是每个劳动者对自己的劳动产品拥有专门的权利。”(第58页)}\end{quote}

照舍尔比利埃的说法,这个基本原则就是:

\begin{quote}{“劳动者对于作为自己劳动的结果的价值,拥有专门的权利。”(第48页)}\end{quote}

由于商品规律,商品形成等价物,并按照它们的价值,即按照它们包含的劳动时间彼此交换。这个规律怎么一下子变了样子,以致资本主义生产(对于产品来说,只有在资本主义生产的基础上,作为商品来进行生产,才具有本质的意义)竟然反过来建立在一部分劳动不经交换就被占有的基础上,——这一点舍尔比利埃既不理解,也没有加以说明。他只是感到,这里发生了某种转变。

舍尔比利埃所说的“基本原则”纯粹是一种虚构。它是由商品流通造成的假象产生的。商品按照它们的价值,即按照它们包含的劳动彼此交换。单个人在这里只是作为商品所有者互相对立,所以,只有让出自己的商品,才能占有别人的商品。因此形成一种似乎他们能交换的只是自己的劳动的假象,因为包含别人劳动的商品的交换,在这些商品本身又不是用自己的商品换得的情况下,是以与[简单]商品所有者即买者和卖者的关系不同的另一种人与人之间的关系为前提的。在资本主义生产当中,资本主义生产表面上反映出来的这种假象消失了。但是有一种错觉并没有消失:似乎最初人们只是作为商品所有者互相对立,因而每个人只有在他是劳动者的情况下才是所有者。如上所述,这“最初”就是由资本主义生产的假象产生的错觉,——这种现象在历史上从来不曾有过。一般说来,人(不论是孤立的还是社会的)在作为劳动者出现以前,总是作为所有者出现,即使所有物只是他从周围的自然界中获得的东西(或者他作为家庭、氏族或公社的成员,部分地从周围的自然界中获得,部分地从公共的、已经生产出来的生产资料中获得)。最初的动物状态一终止,人对他周围的自然界的所有权,就总是事先通过他作为公社、家庭、氏族等等的成员的存在,通过他与其他人的关系(这种关系决定他和自然界的关系)间接地表现出来。“没有所有权的劳动者”作为“基本原则”,倒不如说只是文明的产物,而且是“资本主义生产”这个一定的历史阶段上的产物。这是“剥夺”规律,不是“占有”规律,至少不是舍尔比利埃所想象的一般占有规律,而是和一定的、特殊的生产方式相适应的占有规律。[1110]

[1111]舍尔比利埃说:

\begin{quote}{“产品在转化为资本以前就被占有了;这种转化并没有使它们摆脱那种占有。”(第54页)}\end{quote}

这句话不仅适用于产品,而且适用于劳动。原料等等和劳动资料属于资本家;它们是他的货币的转化形式。另一方面,如果资本家用等于6劳动小时的产品的货币额,购买劳动能力或一天(例如12小时)的劳动能力的使用权,那末,这12小时的劳动就属于资本家,这个劳动在实现以前就已被资本家占有。它通过生产过程本身转化为资本。不过,这种转化是在它被占有以后发生的行为。

“产品”转化为资本:如果产品在劳动过程中执行劳动条件、生产条件(劳动对象和劳动资料)的职能,就是在物质上转化;如果不仅产品的价值被保存,而且产品本身还成了吸收劳动和剩余劳动的手段,也就是说,产品实际上执行劳动吸收器的职能,那就是在形式上转化。[1112]另一方面,在生产过程之前被占有的劳动能力,在生产过程中会直接转化为资本,因为它转化成了劳动条件和剩余价值,因为它物化为产品时既保存不变资本,又补偿可变资本并附加剩余价值。[1112]

\tchapternonum{[(4)关于作为扩大再生产的积累问题]}

[舍尔比利埃说:]

\begin{quote}{[1110]“财富的任何积累,都为加速进一步的积累提供手段。”(同上,第29页)}\end{quote}

{李嘉图(从斯密那里继承下来的)关于任何积累都归结为工资的支出的观点,即使在积累的任何部分都不是以实物形式(例如,租地农场主播下更多的种子,畜牧业者增加种畜或肥育牲畜的头数,机器制造业者在机器制造机上占有一部分剩余价值)实现的情况下也是错误的。这个观点即使在下述情况下也是错误的:即使不存在这样一种现象,即生产某个资本部分的构成要素的所有生产者,由于考虑到年积累的事实,即考虑到一般生产规模的扩大,都是经常地进行超额生产。此外,土地耕种者可以用他的一部分剩余谷物和畜牧业者进行交换,畜牧业者可以把这部分谷物转化为可变资本,而土地耕种者则[通过这种交换]把自己的谷物转化为不变资本。亚麻种植业者[1111]出卖他的一部分剩余产品给纺纱业者,纺纱业者把它转化为不变资本;亚麻种植业者可以用这笔货币购买工具,而工具生产者又可以用这笔货币购买铁等等,这样一来,所有这些要素都直接成了不变资本。但是,撇开这一点不谈。

假定机器厂主想把一笔1000镑的追加资本转化为生产要素。在这种情况下,他当然要把其中的一部分花在工资上,比如说,200镑。他用其余的800镑购买铁、煤等等。假定这些铁、煤还有待于生产。如果制铁业者或煤炭业者在这个时候既没有剩余的(积累的)商品储备,又没有追加的机器,而且也不能直接购买机器(因为在这种情况下,又会发生不变资本同不变资本的交换),那末,制铁业者和煤炭业者只有使他们的旧机器延长工作时间,才能为机器厂主生产铁和煤的追加量。于是旧机器就要加速补偿,但是它们的一部分价值会加入新产品。不过这一点也撇开不谈。制铁业者无论如何也需要更多的煤;因此,这里他必须至少把800镑中属于他的份额的一部分直接转化为不变资本。但是他们两人——煤炭业者和制铁业者——出卖他们的煤和铁时,也使其中包含了无酬剩余劳动。如果这种劳动占1/4,在800镑中就已经有200镑不归结为工资,更不用说产品价值中归结为旧机器磨损的那一部分了。

剩余产品总是由各种特定资本生产的实物形式的东西如煤、铁等等组成。有些生产者的产品互为生产要素,如果他们互相交换这些产品,那末一部分剩余产品就直接转化为不变资本。而同生活资料生产者生产的产品交换并补偿其不变资本的那一部分,则形成必要的可变资本。有些生活资料已不能作为要素(除了作为可变资本)加入自身的生产,这些生活资料的生产者,正是通过其他生产者借以获得追加的可变资本的同一过程来获得追加的不变资本。

再生产——就它是积累而言——和简单再生产的区别如下:

第一,积累的生产要素——它们的属于可变资本的部分和属于不变资本的部分——由新加劳动构成;它们不完全转化为收入,虽然它们是由利润产生的;利润,或者说剩余劳动,转化为所有这些生产要素。而在简单再生产中,一部分产品代表过去劳动(也就是说,这里指的不是当年完成的劳动)。

第二,不言而喻,如果某些生产部门劳动时间延长了,就是说,在那里没有使用追加的工具或机器,那末新产品就要部分地支付旧的工具或机器的更快的磨损,而旧的不变资本的这种加速消费也是积累的因素。

第三,在[扩大]再生产过程中,部分地由于资本的游离,部分地由于一部分产品转化为货币,部分地只是由于生产者[用自己的追加商品]收回货币而使对其他人——例如对出卖奢侈品的人——的商品的需求减少,从而形成追加货币资本;因为有了这种资本,就完全没有必要象在简单再生产中那样系统地补偿生产要素。

每个人都可以用剩余的货币购买产品或支配产品,尽管他向之购买产品的生产者既不把自己的收入花在买者的产品上,也不用这种产品补偿自己的资本。}{每当追加资本(可变的或不变的)不是相互补充的时候,它必然在某一方面作为货币资本沉淀下来,即使这种货币资本只是以债权形式存在。}

\tchapternonum{[(5)舍尔比利埃的西斯蒙第主义因素。关于资本有机构成问题。比较发达的资本主义生产部门的可变资本绝对减少。在资本有机构成保持不变情况下不变资本和可变资本的价值比例的变动。资本的有机构成以及固定资本和流动资本之间的不同比例。资本周转的差别及其对利润的影响]}

在其他方面,舍尔比利埃的观点是西斯蒙第和李嘉图的互相排斥的见解的奇怪混合物。[1111]

[1112]下面的话是西斯蒙第的东西:

\begin{quote}{“关于资本不同要素之间的比例不变的假设,在社会经济发展的任何阶段都不会实现。它们之间的比例实质上是可变的,而且是由于两个原因:(1)分工;(2)人力由自然力代替。这两个原因使生活资料基金与资本的另外两个要素之比有下降的趋势。”(第61—62页)“在这种状况下,生产资本的增加,不一定会引起用来形成劳动价格的生活资料基金的增长;在这种增加的同时,资本的这个要素至少是暂时地会绝对减少,从而劳动价格会下降。”(第63页)}\end{quote}

{这是西斯蒙第的东西;这种[生活资料基金的减少]对工资高度的影响,是舍尔比利埃的唯一着眼点。如果研究是以劳动按其价值支付为前提,而劳动的市场价格在这一点(价值)的上下波动则不考虑在内,这个着眼点也就完全失去意义。}

\begin{quote}{“一个生产者想要在自己的企业中采用新的分工或者利用某种自然力,他不会等到积累的资本足以在这些条件下使用他以前所需要的全部工人时才这样做;在分工的场合,他也许会满足于用五个工人来生产他以前用十个工人生产的东西;在使用自然力的场合,他也许只要使用一台机器和两个工人。因此,生活资料基金[以前等于3000],在第一种场合将减少到1500,在第二种场合将减少到600。但是因为工人现有人数保持不变,所以他们的竞争会很快使劳动价格降到它原来的水平以下。这是占有规律的极其惊人的结果之一。财富即劳动产品的绝对增多,并没有引起工人生活资料基金的相应增多,甚至还能引起这一基金的减少,使各种产品中应归于工人的份额减少。”(第63—64页)“决定劳动价格〈这里始终只是指劳动的市场价格〉的原因,是生产资本的绝对量以及资本不同要素之间的比例,这是工人的意志不能给予任何影响的两个社会事实。”(第64页)“一切机会几乎都是对工人不利的。”(同上)}\end{quote}

生产资本不同要素之间的比例,是由两种方式决定的。

第一,生产资本的有机构成。我们指的是技术构成。在劳动生产力既定的情况下,——只要不发生什么变化,就可以假定它是不变的,——在每个生产领域中,原料和劳动资料的量,也就是与一定的活劳动量(有酬的和无酬的),即一定的可变资本的物质要素量相应的、表现为物质要素的不变资本量,是一个确定的量。

如果与使用的活劳动相比,物化劳动小,代表活劳动的产品份额就大,而不管这个产品部分在资本家和工人之间怎样分配。反之则相反。因此,如果劳动剥削率是既定的,剩余劳动在前一场合也就大,在后一场合也就小。只是由于生产方式发生变化,而这种变化又改变资本两个部分的技术比例,这里才能发生变化。即使在这种场合,如果各资本的量不同,使用较多不变资本的资本所使用的活劳动的绝对量也可能相同,或者甚至更大。但是相对来说,它必然要小一些。对于等量的资本来说,或者以总资本的一定的相应部分(例如100)来计算,不论从绝对还是相对来说,它都必然要小一些。由于劳动生产力的发展(不是下降)而出现的一切变化,都使代表活劳动的产品部分减少,使可变资本减少。因为假定工资到处相同,所以在考察不同生产部门的资本[1113]时,我们可以说,上述变化会使处在较高生产发展阶段的部门的可变资本绝对减少。

这就是由生产方式的变化产生的变化。

但是第二,如果把资本的有机构成和由资本有机构成的差别产生的资本之间的差别假定为既定的,那末尽管技术构成保持不变,[不变资本和可变资本之间的]价值比例也能发生变动。这里可能有以下几种情况:(a)不变资本的价值发生变动;(b)可变资本的价值发生变动;(c)二者按相同的或不同的比例同时变动。

(a)如果技术构成保持不变,不变资本的价值发生变动,那末,这个价值或者下降或者提高。如果它下降,并且只使用原有的活劳动量,就是说,如果生产的阶段或规模保持不变,从而照旧使用例如100个工人,那末,在物质上就照旧需要同量的原料和劳动资料。但是,剩余劳动与总预付资本之比将比以前大。利润率会提高。在相反的场合利润率就下降。在前一场合,对于某个生产领域已经使用的资本来说(不是指那些在不变资本要素的价值发生变动以后新投入该领域的资本),使用的资本总量会减少,或者说,这个资本的某一部分会游离出来,尽管生产继续以原有的规模进行;或者这样游离出来的资本会追加投入生产,起着资本积累的作用。生产规模会扩大,剩余劳动的绝对量会相应地增长。在这种生产方式下,不管剩余价值率如何,任何的资本积累都会导致剩余价值总量的增长。

相反,如果不变资本要素的价值提高,那末,或者生产规模(从而总预付资本量)必须扩大,才能使用和以前同量的劳动(价值没有变动的同一可变资本);这时,尽管剩余价值的绝对量和剩余价值率保持不变,剩余价值与总预付资本之比却减少,因而利润率下降。或者生产规模和预付资本总量不扩大。在这种场合,可变资本在任何情况下都必然会减少。

如果花在[变贵了的]不变资本上的数额和以前一样,这个数额就代表不变资本的较少量的物质要素,因为技术比例保持不变,所以使用的劳动必须减少。这样,总预付资本中就减少了一个游离出来的劳动量;预付资本的总价值减少了;但是在这个减少了的资本中,不变资本占的比例(从价值上说)比以前大。剩余价值绝对减少,因为使用的劳动减少了,剩下来的剩余价值与总预付资本之比下降,因为和不变资本相比可变资本减少了。

另一方面,如果使用的总资本和以前一样,——减少了的可变资本价值(代表减少了的使用的活劳动总量)被增大了的不变资本价值吸收(前者减少的比例和后者增大的比例相同),那末,剩余价值的绝对量会减少,因为使用的劳动减少了,同时,这个剩余价值与总预付资本之比也会下降。因此,利润率下降在这里是由于两个原因:剩余劳动量减少和这个剩余劳动与总预付资本之比下降。

在第一种场合(在不变资本要素的价值下降时),利润率不管怎样都会提高,要使利润额增加,生产规模就必须扩大。假定资本等于600,其中一半是不变资本,一半是可变资本。如果不变部分价值下降一半,那末,可变资本仍旧是300,而不变资本只是150。使用的总资本就只有450,150就会游离出来。如果把这150再加入资本,那末其中有100现在将会作为可变[1114]资本支出。因此,在这里,如果在生产中继续使用和以前相同的资本,生产规模就要扩大,使用的劳动量就要增多。

在相反的场合(在不变资本要素的价值提高时),利润率不管怎样都会下降,要使利润额不减少和使用的劳动量(从而剩余价值量)保持不变,生产规模就必须扩大,也就是说,预付资本必须增多。如果生产规模不扩大,如果预付的资本只是和以前一样多,或者甚至比以前还少,那末不仅利润率要下降,而且利润量也要减少。

在以上两种场合,剩余价值率都保持不变;而在资本的技术构成发生变化时,它就会变化:不变资本增加时,它会提高(因为这时劳动生产率提高了),不变资本减少时,它会下降(因为这时劳动生产率降低了)。

(b)如果可变资本价值的变动与有机构成无关,那末这种情况之所以能发生,仅仅因为不是这个生产领域生产的、而是作为商品从外部进入该领域的生活资料在价格上下降或提高了。

如果可变资本的价值下降,那末这个可变资本仍旧代表相同的活劳动量,只是这个活劳动量的所值现在减少了。因此,如果生产规模保持不变(因为不变资本的价值没有变),总资本中预付在购买劳动上的部分就减少。现在只需花费较少的资本,就可支付同样数量的工人。可见,在这里,在生产规模保持不变的情况下,所花费的资本额会减少。利润率会提高,这是由两种原因产生的。剩余价值增大了;活劳动与物化劳动之比保持不变,但是增大的剩余价值却是与减少的总资本相应的。如果把游离出来的部分附加在资本上,这就等于积累。

如果可变资本的价值提高,那末,为了使用原有数量的工人,就必须花费更多的总资本,因为不变资本的价值保持不变,而可变资本的价值提高了。使用的劳动量保持不变,但是剩余劳动在使用的劳动总量中占的部分比以前小了,而这个较小的部分是与一个比以前大的资本相应的。这是在生产规模保持不变而总资本价值提高的场合下发生的。如果总资本价值不提高,那末生产规模就必须缩减。使用的劳动量减少了,在这个减少了的劳动量中,剩余劳动部分比以前小,它与总预付资本之比也小了。

有机变化和由价值变动引起的变化,在某种情况下,能够对利润率产生相同的影响。但是,它们之间有如下的区别:如果价值变动不单是由市场价格的波动引起,就是说,如果它们不是暂时的,那末它们就始终必然是由提供不变资本或可变资本要素的领域发生的有机变化引起的。

(c)对第三种情况这里不需要作进一步的考察。

在不同生产领域的资本相等的情况下,——或者按总资本的等量部分计算,例如都按100计算,——有机构成可能相同,虽然不变资本和可变资本的价值比例将随着使用的辅助材料和原料量的价值不同而不同。例如,铜代替铁、铁代替铅、羊毛代替棉花等等。

另一方面,如果价值比例相同,有机构成可能不同吗?

[不同生产领域的两笔资本]有机构成相同时,每100单位的资本中不变资本和活劳动的相对量也相同——它们之间的量的比例也相同。很可能是,不变资本的价值相同,虽然被推动的相对劳动量不同。如果机器或原料在一种场合比另一种场合贵(或者相反),那末所需要的劳动,比如说,可能减少;但是那时可变资本的价值也将相对减少(或者相反)。

[1115]举资本A和B为例。假定c′和v′是资本A的组成部分(按价值来说),c和v是资本B的组成部分(按价值来说)。如果c′∶v′=c∶v,那末,c′v=v′c。因此c′/c=v′/v。

在不变资本和可变资本之间的价值比例相同时,只可能出现下述情况。如果一个领域比另一领域完成了更多的剩余劳动{例如农业中就不能打夜工,虽然单个农业工人可能被迫过度劳动,但是在地块等等的大小既定的条件下所能使用的劳动总量,却受到需要生产的对象(谷物)的限制,可是,在工厂的大小既定的条件下,生产的产品量(有可能)取决于劳动小时的数量,——就是说,由于生产方式不同,在生产规模既定时,一个领域可以比另一领域使用更多的剩余劳动},那末不变资本和可变资本之间的价值比例可能相同,但是使用的劳动量与总资本之比会不同。

或者假定,材料和劳动(由于它属于较高级的劳动)按同一比例变贵。在这种情况下,资本家A在资本家B使用25个工人的地方使用5个工人,这5个工人的费用是100镑,和那25个工人的费用一样,因为他们的劳动变贵了(从而他们的剩余劳动的价值也变贵了)。同时,这5个工人加工价值500镑的原料y100磅,而资本家B的工人加工价值500镑的原料x1000磅,因为在资本家A那里材料较贵,劳动生产力较不发达。这里,在两种场合可变资本和不变资本的价值比例都是100镑比500镑,但是资本A和B的有机构成不同。

价值比例相同:资本家A的不变资本的价值等于资本家B的不变资本的价值,A花在工资上的资本,和B一样多。但是他的产品量较少。虽然他需要的劳动量绝对地说和资本家B需要的一样,但是相对地说他需要的却多一些,因为他的不变资本贵一些。在同一时间里,A加工的原料等等较少,但是这个较少量和B的较多量的原料的价值相同。在这种场合价值比例相同,有机构成则不同。在另一种场合,如果价值比例相同,这种情况只有在剩余劳动量不同或各种劳动的价值不同时才有可能发生。

资本有机构成的概念可以这样表述:这是在不同生产领域为吸收同量劳动而必须花费的不变资本的不同比例。同量劳动与劳动对象的结合,在一种场合比在另一种场合需要更多的原料和机器设备,或者只是其中之一。

{在固定资本和流动资本之间的比例很不相同的情况下,不变资本和可变资本之间的比例可能相同,从而剩余价值也可能相同,虽然一年内生产的价值必然不同。假定在不使用任何原料(辅助材料撇开不谈)的煤炭工业中,固定资本占总资本的一半,可变资本占另一半。假定在裁缝业中固定资本等于零(和上一场合一样,辅助材料撇开不谈),但是原料等于一半,可变资本和上一场合一样也等于总资本的一半。这样,两笔资本(在对劳动的剥削相同的情况下)将实现相同的剩余价值,因为按100单位的资本计算,它们使用的劳动量相同。但是,假定煤炭工业中的固定资本十年周转一次,而两种场合的流动资本的周转毫无差别。如果剩余价值等于50,那末,裁缝业主到年终生产的总价值(假定两种场合的可变资本都是一年周转一次)将等于150。相反,煤炭业者到第一年年终生产的价值等于105(即固定资本5,可变资本50,剩余劳动50)。他的产品的总价值加固定资本等于150(也就是说,产品等于105,剩下的固定资本等于45),和裁缝业主那里的情况一样。可见,生产的价值量不同,并不排除生产的剩余价值相同。

第二年,煤炭业者的固定资本将等于45,可变资本等于50,剩余价值等于50。因此,预付资本将等于95,利润等于50。利润率会提高,因为固定[1116]资本的价值由于固定资本在第一年磨损十分之一而减少。因此,毫无疑问,对所有使用很多固定资本的资本来说,——在生产规模不变的情况下,——利润率必然提高,其提高的程度等于机器即固定资本的价值由于已经补偿的磨损而每年下降的程度。如果煤炭业者在十年内总是按同一价格出卖自己的产品,那末,他在第二年得到的利润率必然高于第一年,依此类推。或者必须假定,维修工作等等同磨损成正比,以致在固定资本项目下每年预付的资本部分的总额保持不变。上述超额利润也能得到平衡,因为固定资本的价值(与磨损无关),由于旧机器必须同较完善的、较晚发明的新机器相竞争而会逐渐下降。但是另一方面,这种由于磨损,由于固定资本价值的减少而自然产生的不断提高的利润率,使旧机器能够同较完善的新机器竞争,因为新机器还要按全部价值进行计算。最后,如果煤炭业者[在第二年年终]卖得便宜些,就是说这样计算:50比预付资本100得50%利润,95乘50%得47+(1/2);即如果他出卖同量产品得到[不是105,而是]102+(1/2),那末,和比如说还只是第一年把机器投入生产的人相比,他就卖得便宜些。固定资本的大量投入以拥有大资本为前提。因为这些大资本所有者控制着市场,所以看来他们只是由于上述原因才获得超额利润(租)。这种租在农业中是因为在相对来说比较肥沃的土地上劳动而获得的,在这里,则是因为利用相对来说比较便宜的机器进行劳动而获得的。}

{许多被说成是由固定资本和流动资本之间的比例造成的情况,实际上是与可变资本和不变资本之间的差别有关。第一,[某些生产部门的]不变资本和可变资本之间的比例可能相同,尽管固定资本和流动资本之间的比例会不同。第二,当我们说到不变资本和可变资本时,指的是资本最初的划分为活劳动和物化劳动,而不是流通过程或流通过程对再生产的影响所引起的这种比例的变化。

首先,显而易见,固定资本和流动资本之间的差别只有在它影响总资本的周转时,才能影响剩余价值(撇开同可变资本与不变资本之比有关的在所用活劳动量上的差别不谈)。因此,必须研究资本周转怎样影响剩余价值。显然有两种情况同这个问题密切相关:(1)剩余价值不能那么快(那么经常)地积累起来,再转化为资本;(2)预付资本必须增长,既为了继续推动同一数量的工人,等等,也由于资本家对本身消费不得不作的预付的时间延长。这两种情况在考察利润时很重要。但是这里首先应该考察的,只是它们怎样影响剩余价值的问题。而这两种情况始终必须清楚地区别开来。}

{凡是使预付增长而没有使剩余价值相应增长的情况,都会使利润率下降,即使剩余价值保持不变;凡是使预付减少的情况,其作用则相反。因此,只要同流动资本相比的较大固定资本量——或资本的不同周转——影响预付量,它也就影响利润率,即使它丝毫不影响剩余价值。}

{利润率不是单纯地按预付资本计算的剩余价值,而是在既定的期间,即在一定的流通时间实现的剩余价值量。因此,只要固定资本和流动资本的差别影响了一定的资本在既定的期间实现的剩余价值量,它也就影响利润率。这里有两个因素:第一,(同实现的剩余价值相比的)预付的量的差别;第二,在这些预付连同剩余价值流回以前,生产这些预付所必需的时间的长度的差别。}

[1117]{实质上影响再生产时间,或者更确切地说,影响一定时间内的再生产次数的,有两种情况:

(1)产品在生产领域本身停留的时间较长。

第一,一件产品本身所需要的生产时间比另一件产品所需要的可能长一些,可能需要一年中较长的一段时间,也可能是整整一年,或者是一年以上。(例如,在建筑业、畜牧业和某些奢侈品的生产中就需要一年以上。)在这种情况下,按照生产资本的构成,即按照它的不变资本和可变资本的划分,产品不断吸收劳动,同不变资本相比,往往(例如在奢侈品生产中、在建筑业中)吸收很多的劳动。这样,随着产品生产时间的延长(然而这种生产也是劳动过程的均衡的持续),劳动和剩余劳动就不断地被吸收。例如,在畜牧业或建筑业中,比如说建筑业需要一年以上的时间。产品只有在完成以后才能进入流通,也就是说出卖,投入市场。第一年的剩余劳动,和其他劳动一起,在第一年的未完成产品中客体化了。它既不小于也不大于具有同样的不变资本和可变资本比例的其他生产部门的剩余劳动。但是,这个产品的价值不能实现,就是说不能转化为货币,从而剩余价值也不能实现。因此,这个剩余价值既不能作为资本积累起来,也不能用于消费。预付资本和剩余价值可以说都成了进一步生产的基础。它们是进一步生产的前提,它们在某种程度上作为半成品,以某种方式作为原料,加入第二年的生产。

假定预付资本等于500,劳动等于100,剩余价值等于50,这样,第二年用于生产的预付资本就等于550加上第二年追加的预付500。剩余价值仍然等于50。这样,到第二年年终产品的价值就等于1100镑,其中100是剩余价值。在这种场合,剩余价值就象在第一年资本全部再生产出来、第二年又重新投入500镑的场合一样。可变资本在第一年和第二年都是100,剩余价值都是50。但是利润率不同。第一年利润率是50/500,即10%。但是第二年预付是550+500=1050,这个数额的十分之一等于105。因此,如果假定利润率相同,产品的价值就是:第一年550,第二年550+500+55+50=1155。产品的价值在第二年年终就等于1155。在另一种情况下它只等于1100。在这里,利润大于生产出来的只是100的剩余价值。如果把资本家在两年内必须预付的自己消费的费用也计算在内,那末支出的资本同剩余价值相比就更大了。不过第一年的全部剩余价值确实在第二年也都已转化为资本。此外,花在工资上的资本增大了,因为第一年预付的100镑到第一年年终没有再生产出来,因此在第二年必须为同样的劳动预付200镑,否则,用第一年再生产出来的100镑就够了。

第二,在劳动过程结束之后,产品可能还必须停留在生产领域里,以便经受自然过程的作用,这些过程不需要任何劳动或者只需要相对来说非常少的劳动,例如葡萄酒置于窖内。只有这个时期过去以后,资本才能再生产出来。显然,这里不管可变资本和不变资本的比例如何,得出的结果都和支出较多不变资本和较少可变资本时的情况一样。剩余劳动和这里在一定期间使用的全部劳动一样,量比较小。如果利润率相同,那末,这是由于利润的平均化,而不是由于这个领域里生产的剩余价值。为了维持再生产过程——生产的连续性——必须事先预付较多的资本。又是由于这个原因,在这里剩余价值与预付资本之比会下降。

第三,当产品还处在生产过程中的时候,劳动过程可能中断,例如农业中就有这种情形,还有象制革之类的过程也是这样,在那里,在产品能够从一个生产阶段转到下一个较高的生产阶段以前,化学过程造成劳动过程的中断。在这种场合,如果有化学方面的发现来缩短这种中断的持续时间,那末劳动生产率就会提高,剩余价值就会增大,向生产过程预付物化劳动的时间就会缩短。在劳动过程发生中断的所有场合,剩余价值减少,预付资本增大。

(2)当某个流动资本的周转速度由于离市场远而比平均的周转速度慢时,也会发生同样的情况。在这里,资本预付也增大,剩余价值也减少,剩余价值与预付资本之比也下降。}{在这一场合,资本在流通领域里停滞的时间较长,在前面所说的场合,则在生产领域里停滞的时间较长。}

[1118]{假定在运输业的某个部门里,预付资本等于1000;固定资本等于500,五年耗损完;可变资本等于500,每年周转四次。这样,年产品的价值就是100+2000+100(如果[年]剩余价值率等于20%),总计是2200。另一方面,假定在裁缝业的某个部门里,不变流动资本等于500(固定资本等于0),可变资本等于500,剩余价值等于100。假定资本每年周转四次。这样,年产品的价值是4(500+500)+100=4100。两种场合的剩余价值相同。后一笔资本一年全部周转四次,或者说,每季度周转一次。在第一笔资本中,每年周转的有600[其中有500每年周转四次]。因此每季度有500+100/4,即525在周转。因此一个月有175,两个月有350,8个月有1400。总资本(1000)周转一次需要5+(5/7)个月。它一年只周转2+(1/10)/次。

有人会说,第一笔资本,为取得10%的利润,每个季度加在价值1000上的附加额少于第二笔资本。但是这里的问题不在于附加额。一个资本家获得较多的剩余价值,是靠他已消费的资本,而不是靠他使用的资本。这里的差别来自剩余价值[相对量],不是来自利润附加额。这里差别在于价值,不在于剩余价值。两笔资本中的可变资本500,每年都周转四次。两笔资本每年获得的剩余价值都等于100,[年]剩余价值率都等于20%。但是每季度的剩余价值是25镑,——是不是说,百分比较高呢?每季度25比500等于每季度5%,因此全年就是20%。

第一个资本家有一半资本一年周转四次,另一半一年只有五分之一在周转。四次的一半是二次。因此,他的资本一年周转2+(1/10)次。第二个资本家的全部资本一年周转四次。但是这绝对改变不了剩余价值。如果第二个资本家不间断地继续再生产过程,那他就必须不断把500转化成原料等等,而他用于支付劳动的始终只是500,但是第一个资本家用于支付劳动的也是500,其余的500却一劳永逸地(即为期5年)被赋予一种不需要他再来转化的形式。但是这一点[剩余价值的均等]只有在尽管固定资本和流动资本的量有差别,但是[两笔资本的]可变资本和不变资本的比例却相同的情况下才会出现。

如果在两笔资本中,都是一半为不变资本,一半为可变资本,那末[第一笔资本]有一半可能只由固定资本构成,如果流动不变资本等于零;而[第二笔资本]有一半可能只由流动不变资本构成,如果固定资本等于零。应当看到,虽然流动不变资本可能等于零,例如在采掘业和运输业中(在那里,不过是辅助材料代替原料构成流动不变资本),然而固定资本(除了在银行家等等那里)永远不会等于零。但是,如果不变资本在两种场合都和可变资本处于相同的比例,尽管不变资本在一种场合包含较多的固定资本和较少的流动不变资本,在另一种场合则包含较少的固定资本和较多的流动不变资本,情况也不会有所改变。这里有的只是一半资本的再生产时间上的差别和总资本的再生产时间上的差别。一个资本家必须在他的500镑流回之前把它预付5年,另一个则预付一个季度或一年。对资本的支配能力不同。预付没有什么不同,但是预付的时间不同。这种差别在这里和我们无关。如果就全部预付资本来考察,剩余价值和利润在这里是相同的:第一年预付1000镑得100镑。第二年不如说是在固定资本方面出现较高的利润率,因为可变资本保持不变,而固定资本的价值减少了。第一个资本家第二年只预付400固定资本和500可变资本,照旧获得100镑利润。但是100比900等于11+(1/9)%,而第二个资本家继续进行再生产,照旧预付1000,获得100镑利润,等于10%。

当然,如果同可变资本相比,整个不变资本随同固定资本一起增加了,或者说,如果为了推动同量劳动,而不得不总的来说预付较多的资本,那末情况就会改变。在上面举的例子中,问题不在于总资本周转有多快或预付有多大,问题在于那部分足以推动同量生产劳动而在另一种场合足以更新生产过程的资本周转有多快。但是,如果在上述例子中,固定资本[不是等于500,而是]等于1000,而流动资本[仍旧]只等于500,那末情况就会改变。但是发生这种变化,并不因为是固定资本。要知道,如果第二种场合的流动不变资本(例如由于材料变贵)值1000[而不是500],那末情况也会同样改变。因为在[两个例子中的]第一种场合,固定资本越大,总预付资本同可变资本相比的相对量也就越大,二者也就会混淆起来。此外,周转这种事情,实际上来源于商业资本,在那里这由其他规律决定:在商业资本中,如我指出的那样\endnote{马克思指他在1861—1863年手稿第XV和XVII本中,特别是在第964页(第XV本)和1030页(第XVII本)对商业资本的研究。——第435页。},利润率实际上由周转的平均数决定,而不管该资本的构成如何,不过它主要是由流动资本构成。因为商业资本的利润决定于一般利润率。}

[1119]{整个情况如下。

假定固定资本等于x。如果它15年只周转一次,那末,它一年有1/15在周转,但是每年需要补偿的也只是这笔资本的1/15。如果它一年中补偿了15次,那末情况并不会因此而有丝毫改变。它的量仍旧保持不变。但是产品因此而变贵了。当然,比起以流动资本形式预付的同量资本来,对资本的支配能力就较小,贬值的风险就较大。但是这丝毫不影响剩余价值,虽然资本家先生们在计算利润率时把这一点也计算在内,因为在计算磨损时这种贬值的风险是被计算在内的。

至于资本的另一部分,那末假定不变资本的流动部分(原料和辅助材料)一年等于25000镑,工资等于5000镑。这样,如果这笔资本一年只周转一次,全年就必须预付30000镑;如果剩余价值等于100%,即5000镑,那末到年终利润就等于5000镑比30000,即1/6,或16+(2/3)%。

但是如果这笔资本每1/5年周转一次,那末在不变流动资本上只须预付5000镑,在工资上只须预付1000镑。利润——1000镑,5/5年——5000镑。但是这个剩余价值是用6000镑资本获得的,因为决不会预付大于这个数额的资本。因此,利润是5000比6000,即5/6,也就是第一种场合的5倍:83+(1/3)%(撇开固定资本不谈)。于是,这里就出现利润率的极大差别,因为实际上5000镑的劳动是用1000镑的资本买来的,而25000镑的原料等等是用5000镑的资本买来的。如果在这种周转率不同的情况下资本的量相同,那末在第一种场合就只能预付6000镑。或者说每月只预付500镑,其中5/6由不变资本构成,1/6由可变资本构成。这1/6=83+(1/3)镑,用它获得100%的剩余价值是83+(1/3)镑,全年是(83+1/3)12=12/3(或4)+996=1000。但是1000比6000等于16+(2/3)%。}

\tchapternonum{[(6)李嘉图和西斯蒙第的互相排斥的见解在舍尔比利埃著作中的折衷主义的结合]}

现在回过头来谈舍尔比利埃。

下面的论述是西斯蒙第的见解:

\begin{quote}{“社会的经济进步只要是以生产资本的绝对增长和这一资本不同要素之间的比例发生变化为特征,它就会给工人提供若干好处:(1)劳动生产率的提高——特别是由于使用机器——引起生产资本非常迅速的增长,以致尽管生活资料基金和这一资本的其他要素之间的比例发生了变化,但这个基金还是有了绝对的增长,这样就不仅允许使用原有数量的工人,而且允许雇用追加数量的工人,所以,进步的结果,如果撇开一些暂时的中断不谈,对工人来说就意味着生产资本的扩大和对劳动的需求的增长。(2)资本生产率的提高有着使一系列产品的价值大大下降的趋势,从而使它们成为工人可以获得的东西,工人的消费范围由此而得到扩大……但是:(1)构成劳动价格的生活资料基金有时会减少,即使这种减少是短暂的、局部的,它也会对工人造成极其有害的后果。(2)促使某一社会的经济进步的情况大部分是偶然的,是不以生产资本家的意志为转移的。因此,这些原因的作用不是经常不变的”……。“(3)使工人的状况变得幸福或不幸福的,与其说是工人的绝对消费,不如说是工人的相对消费。如果工人无法获得的产品的数量以更大的比例增加了,如果把他和资本家隔开的距离只是增大了,如果他的社会地位变得更低和更不利了,那末,对他来说,能够获得一些他们这样的人以前无法获得的产品又有什么意义呢?除了维持体力所绝对必需的消费品以外,由我们消费的消费品的价值完全是相对的。人们忘记了,雇佣工人是有思想的人,是赋有和劳动资本家同样的才能、被同样的动机推动的人。”(第65—67页)[1120]“社会财富的迅速增长,不管能给雇佣工人带来怎样的好处,也消除不了他们贫困的原因……他们照旧被剥夺了对资本的任何权利,因而不得不出卖自己的劳动,并且放弃对这种劳动的产品的任何要求。”(第68页)“这是占有规律的根本缺陷……弊病在于雇佣工人和由他的劳动推动的资本之间这种完全缺乏联结的链条。”(第68—69页)}\end{quote}

这最后一句关于“联结的链条”的话纯粹是西斯蒙第的见解,同时是荒谬的。

关于标准人即资本家,等等——见同书第74—76页。

关于资本积聚和排除小资本家——第85—88页。

\begin{quote}{“如果在目前状况下实际利润是从资本家的节约中得到的,那末[在其他的分配制度下]它同样也可以从雇佣工人的节约中得到。”(第88—89页)}\end{quote}

[另一方面,]舍尔比利埃

(1)赞同[詹姆斯·]穆勒关于一切赋税都应从地租征收的观点\endnote{马克思指詹姆斯·穆勒在他的《政治经济学原理》(1821年伦敦版第4章第5节《地租税》)中的论断。穆勒在书中证明,国家的全部费用在土地还不是私有财产的情况下用全部地租来支付,在土地已经成了私有财产以及地租同原来的水平相比有所增加的情况下用地租的增长额来支付,是合理的。——第438页。}(第128页),但是因为不可能“规定一种真正从地租征收,而且只触及地租的赋税”,又因为很难把利润和地租区分开,——如果土地所有者本人耕种土地,就根本不可能区分开,——所以舍尔比利埃就

(2)继续前进,接近了李嘉图学说的正确结论:

\begin{quote}{“为什么不再前进一步,废除土地私有制呢?”(第129页)“土地所有者是有闲者,他们靠公众的费用来养活自己,对生产或社会的一般福利毫无益处。”“使土地具有生产能力的,是使用在农业上的资本。土地所有者对此毫无贡献。他的存在只是为了收取地租,而地租并不构成他的资本的利润的一部分,它既不是劳动的产物,也不是土地生产力的产物,而是由消费者的竞争抬高的农产品价格的结果”……(第129页)“因为废除土地私有制丝毫不会改变产生地租的原因,所以地租还会继续存在;不过地租将由国家征收,因为全部土地将属于国家,国家将把可耕种的地块租给拥有足够的资本来经营这些地块的私人。”(第130页)地租将取代国家的全部收入。“最后,获得解放的、摆脱了一切枷锁的工业将得到空前的发展”……(第130页)}\end{quote}

但是,怎样使这个李嘉图式的结论和西斯蒙第的给资本和资本主义生产拴上“链条”的虔诚愿望协调起来呢?怎样使它和下面这种悲叹协调起来呢:

\begin{quote}{“如果没有一场变革来阻止我们的社会在占有规律的[目前]统治下实现的发展进程,资本最终将成为世界的主宰。”(第152页)“资本将到处消灭旧的社会差别,以便用一种简单的人类划分来代替它,这就是把人类划分为富人和穷人,富人享乐和统治,穷人劳动和服从。”(第153页)“生产基金和产品的普遍占有,向来是把人数众多的无产者阶级降低到屈从和政治上无权力的状态,但是这种占有曾经和整套限制性法律结合在一起;这些法律阻碍产业的发展和资本的积累,[1121]限制被剥夺继承权阶级的增长,把他们的公民自由约束在狭小的范围内,从而用各种方法使这个阶级无能为害。今天,资本已把这些枷锁的一部分打碎了。它正在准备把它们全部打碎。”(第155—156页)“无产者的堕落是[目前]财富分配的第二个后果。”\endnote{舍尔比利埃把富人和穷人之间的不平等称为“目前财富分配的第一个后果”。——第438页。}(第156页)}\end{quote}

\tchapternonum{[第二十四章]理查·琼斯}

\tchapternonum{(1)理·琼斯《论财富的分配和税收的源泉》,第一部分:《地租》,1831年伦敦版。[地租历史观的因素。琼斯在地租理论的个别问题上胜过李嘉图之处以及他在这方面的错误]}

琼斯的这第一部论地租的著作就已经有一个特点,那是詹姆斯·斯图亚特爵士以来一切英国经济学家所没有的,这就是:对各种生产方式的历史区别有了一些理解。(对各种历史形式所作的这种正确的区分,总的说来同已被指出的琼斯的考古学的、语言学的和历史学的非常重要的错误并不矛盾。例如,见《爱丁堡评论》第54卷第4篇文章。\endnote{马克思指的是该杂志第54卷(1831年8月至12月)上刊登的一篇没有署名的书评,评论当时刚出版的琼斯的著作《论财富的分配》。——第439页。})

琼斯在李嘉图以后的现代经济学家的著作中发现,地租被规定为超额利润,这一规定的前提是:农场主是资本家(或者说,农业资本家经营土地),他期望从资本的这种特殊使用中得到平均利润;农业本身从属于资本主义生产方式。简言之,这里所考察的仅仅是土地所有权的改变了的形式,即资本这一占统治地位的社会生产关系赋予它的那种形式,亦即它的现代资产阶级形式。琼斯完全没有资本自有世界以来就已存在这样一种错觉。

琼斯关于地租的起源的见解,一般说来概括在以下的论述中:

\begin{quote}{“甚至在人们从事最原始的劳动时,土地也有能力提供多于土地耕种者维持生活所必需的东西,这样就使他有可能向土地所有者交纳贡物,这就是地租的起源。”(第4页)“由此可见,地租起源于那样一个时代的土地占有,在那个时代,大多数居民或者不得不在所能得到的任何条件下耕种土地,或者饿死,而且那时这些人的微薄的资本,如工具、种子等,由于不可克服的必然性而同他们一起被束缚在土地上,因为,他们若不从事农业,而去从事任何别的,他们的资本就根本不够维持他们的生活。”[第11页]}\end{quote}

琼斯研究了地租的一切变化:从最原始的徭役劳动形式到现代的租地农场主地租。他到处都发现,地租的一定形式,即土地所有权的一定形式,与劳动和劳动条件的一定形式相适应。所以他依次考察了劳役地租或农奴地租,考察了劳役地租向实物地租的转化,考察了分成制地租、莱特\endnote{莱特(Ryot)——印度农民。琼斯用这个术语来称呼印度和亚洲其他国家的这样一些农民,他们向君主,即向被认为是全部土地的最高所有者缴纳实物租税。——第440页。}地租等,他的这种研究的细节,我们在这里不感兴趣。在一切较早的地租形式中,直接占有别人剩余劳动的人不是资本家,而是土地所有者。地租(正如重农学派根据回忆所理解的那样)在历史上(在亚洲各民族中还是在最大范围内)表现为剩余劳动的普遍形式,即无偿地完成的劳动的普遍形式。与资本主义关系不同,在这里,对这种剩余劳动的占有不是以交换为中介,而是以社会的一部分人对另一部分人的暴力统治为基础。(由此而来的还有直接的奴隶制、农奴制或政治的依附关系。)

因为我们在这里考察土地所有权只是由于理解土地所有权是理解资本的先决条件,所以我们也就不去详细叙述琼斯的论证,而立即转到那个十分有利于把他同所有他的前辈们区别开来的结果上去。

但是,在此之前还要附带谈几点意见。

琼斯在谈到徭役劳动——以及或多或少与此相适应的农奴制(或奴隶制)的各种形式——的时候,[1122]无意中突出了任何剩余价值(任何剩余劳动)都可以归结成的两种形式。总的说来值得注意的是:本来意义的徭役劳动在其最粗野的形式中最鲜明不过地显示了雇佣劳动的本质。

\begin{quote}{“地租〈在有徭役劳动的地方〉在这样的情况下只有用以下两种办法才能增加:或者是更巧妙更有效地使用农奴的劳动〈这是相对剩余劳动〉,然而这将由于土地所有者这个阶级无力发展农业科学而受到阻碍;或者是增加从农奴身上榨取的劳动的量,在这种情况下,如果所有者的土地会耕种得好一些,那末,农奴的土地就会因劳动被夺去而耕种得坏一些。”(同上,第II章[第61页])}\end{quote}

琼斯的这本论地租的书同我们将在第二节中加以考察的他的《大纲》有以下区别:在第一本著作中,琼斯把土地所有权的各种不同形式当作某种既定的东西,并以这些形式为出发点,而在第二本著作中,他以土地所有权的各种不同形式与之相适应的劳动的各种形式为出发点。

琼斯还指出,劳动的社会生产力的不同发展程度怎样和这些不同的生产关系相适应。

徭役劳动(奴隶劳动也完全一样),就地租这一点来说,同雇佣劳动有一个共同点,那就是,地租是用劳动支付,不是用实物支付,更不是用货币支付。

\begin{quote}{在分成制地租的情况下,“资本由土地所有者预付,并让实际劳动者自由地耕种土地,表明这里依然没有起中介作用的资本家阶级”。(第74页)“莱特地租是从土地取得工资的劳动者向作为土地所有者的君主交纳的实物地租。”(第IV章[第109页])(这种地租主要见于亚洲。)“莱特地租往往和劳役地租及分成制地租结合在一起。”(第136页及以下各页)在这里主要的土地所有者是君主。“在亚洲,城市的繁荣,或者更确切地说,城市的存在,完全依赖于政府的地方性开支。”(第138页)“茅舍贫农\endnote{琼斯所理解的“茅舍贫农”(《Cottier》)是爱尔兰的无地农民,他们从地主那里租一小块耕地,交一定的货币地租。——第442页。}地租……这就是从土地取得生存资料的佃农按照契约以货币形式支付的地租。”(第143页)(爱尔兰。)“在地球的大部分地区不存在货币地租。”[同上]“所有这些形式〈劳役地租、莱特地租、分成制地租、茅舍贫农地租等等,一句话,农民地租的一切形式〉都阻碍土地生产力的充分发展。”[第157页]“不同的人的劳动生产率的差别取决于以下两点:第一,在使用手工劳动的情况下,利用发明的程度,第二,人的纯体力活动在多大程度上得到过去劳动的积累结果的帮助,也就是说,这种差别取决于生产中使用的技能、知识和资本的差别。”[第157—158页]“非农业阶级的人数不多。显然,不从事农业劳动而能生活的人的相对数,完全取决于土地耕种者的劳动生产率。”(第VI章[第159—160页])“在英国农村,在农奴劳动废止以后,出现了在土地所有者的领地上从事耕作的租佃者。那是自由民。”(同上[第166页])}\end{quote}

最后,我们要谈到这里使我们最感兴趣的一点,即租地农场主地租。正是在这里,琼斯的优越之处突出地显示了出来:他证明,李嘉图等人看作是土地所有权的永恒形式的东西,却是土地所有权的资产阶级形式,这种形式一般只在以下情况才出现,第一,土地所有权不再是支配生产从而支配社会的关系;第二,农业本身以资本主义方式经营,而这又是以城市的大工业(至少是工场手工业)的发展为前提。琼斯指出,李嘉图所说的地租只存在于[1123]以资本主义生产方式为基础的社会。随着地租转化为超额利润,土地所有权对工资的直接影响也就终止,换句话说,这只是意味着,今后直接占有剩余劳动的人不是土地所有者,而是资本家。地租的相对量现在仅仅取决于剩余价值在资本家和土地所有者之间的分配,而不取决于对这种剩余劳动的榨取本身了。这层意思实际上在琼斯那里已经有了,尽管他没有明确地把它说出来。

同李嘉图相比,琼斯不论在历史地解释现象方面,还是在经济学的细节问题上,都向前迈出了重要的一步。我们将逐步考察他的理论。当然他的理论中也有错误。

琼斯在下面的论述中正确地说明,在什么历史条件和经济条件下地租是超额利润,或者说,是现代土地所有权的表现。

\begin{quote}{“只有当社会各阶级的最重要的相互关系不再从土地所有权和土地占有产生的时候,租地农场主地租才能存在。”(第185页)}\end{quote}

资本主义生产方式开始于工业,只是到后来才使农业从属于自己。

\begin{quote}{“最先受资本家支配的是手艺人和手工业者。”(第187页)“这种制度的直接结果之一,就是有可能随意把用于农业的劳动和资本转移到其他行业中去。”}\end{quote}

{只有具备了这种可能性,才谈得上农业利润和工业利润的平均化。}

\begin{quote}{“当租佃者自己是劳动农民,由于缺少其他生存资料而被迫从土地获取这些资料时,他被穷困束缚在这块土地上;他可能拥有的少量资本实际上也同它的所有者一起被束缚在土地上,因为这笔资本如果不完全用来耕种土地,就不够维持他的生活。这种对土地的依赖性随着资本主义企业主的出现而终止了,如果在农业中使用工人所能赚得的,不如在那种社会情况下的其他各种行业里从工人劳动中赚得的多,就会停止经营农业。在这种情况下,地租必然完全由超额利润构成。”(第188页)“地租不再对工资发生影响了。”[同上]“当一个劳动者为资本家雇用时,他对土地所有者的依赖性就终止了。”(第189页)}\end{quote}

下面我们将看到,琼斯并没有真正说明超额利润是怎样产生的,或者更确切地说,他不过是按李嘉图的方式去说明,也就是用各种土地的自然肥力的差别去说明。

\begin{quote}{“当地租由超额利润构成时,特定的一块土地的地租可能由于以下三种原因而增加:(1)由于在生产中使用更多的积累资本而使产品增加;(2)更有效地使用已有投资;(3)在资本和产品保持不变的情况下,各生产阶级所得的产品份额减少,而土地所有者的份额相应增加。这些原因也可以按不同的比例结合起来发生作用。”(第189页)}\end{quote}

现在让我们来看看这几种原因都是些什么情况。首先,它们都是以地租来自超额利润为前提。其次,毫无疑问,这些原因中的第一个原因是完全正确的,这个原因李嘉图只有一次顺便提到过\fnote{见本卷第2册第112—113、138、358页。——编者注}。如果农业上使用的资本增加,地租的量也就增加,尽管谷物等等的价格不提高,并且一般说来也不发生其他任何变动。显然,在这种情况下土地价格也会提高,尽管谷物价格不提高,并且一般说来在谷物价格方面也不发生其他任何变动。

琼斯把最坏的土地的地租解释为垄断价格。所以他把地租的真正源泉归结为:或者是垄断价格(如布坎南、西斯蒙第、霍普金斯等人的主张),如果存在(不是由各种土地的肥力的差别产生的)绝对地租的话;或者是级差地租(如李嘉图的主张)。

{关于绝对地租。拿金矿为例。假定使用的资本等于100镑,平均利润等于10镑,地租等于10镑。再假定资本的半数由不变资本(在这一场合是机器和辅助材料)构成,半数由可变资本构成。50镑不变资本只是表示,它所包含的劳动时间[1124]和50镑金包含的劳动时间相等。所以与50镑相等的那一部分产品将补偿已消费的不变资本。如果剩下来的产品等于70镑,并且用50镑可变资本去推动50个工人,那末50个工人[的劳动](假定工作日等于12小时)就必须表现为70镑金,其中50镑支付工资,20镑体现无酬劳动。在这种情况下,有机构成相同的所有资本的产品价值都等于120镑。于是产品就等于50c+70,而后面这70镑代表50个工作日,并且等于50v+20m。一笔100镑的资本,如果它使用的不变资本较多,使用的工人人数较少,它生产出来的将是价值较小的产品。但是,一切普通的产业资本,即使它们的产品价值在这种情况下等于120镑,也只会按产品的生产价格110镑来出卖产品。但是,对金矿来说,即使撇开土地所有权不谈,这也是不可能的,因为在这里价值表现在产品的实物形式上。因此,在这里必然会产生10镑地租。}

\begin{quote}{“谷物能够按照垄断价格(即按照超出在最不利条件下生产谷物的人的费用和利润的价格)出卖,或者按照仅仅支付普通利润的价格出卖。如果是第一种情况,那末,撇开耕地肥力的一切差别不说,由于资本增加而达到的产品增加(在价格不变的情况下),就能使地租同所花费的资本的增加成比例地提高。例如,假定普通利润率为10%。如果花100镑生产出的谷物能卖115镑,那末地租就是5镑。如果由于耕作水平的提高,在这同一块土地上使用的资本增加一倍,并且产品也增加一倍,那末200镑的资本就会提供230镑的产品,地租将是10镑,就是说也增加一倍。”(第191页)}\end{quote}

{对于绝对地租是这样,对于级差地租也是这样。}

\begin{quote}{“在小的社会内,谷物总是能够按照垄断价格出卖……在比较大的国家,如果人口增长的速度总是比农产品增长的速度快,这种情况也是可能的。但是对于土地异常辽阔而又多种多样的国家来说,谷物的垄断价格则是极不平常的现象。如果谷物价格显著提高,就会有更多的土地被耕种,或者有更多的资本投到原有的耕地上去,直到价格所提供的利润不再多于所花费用的普通利润为止。那时农业的发展就会停下来,在这样的国家,谷物通常出售的价格只够补偿在最不利条件下使用的资本,并得到该资本的普通利润率;而比较肥沃的土地所支付的地租,则按这些土地的产品超过花费同样资本耕种的最坏土地的产品的余额来计算。”(第192页)“如果一个国家拥有各种质量的土地,要在这个国家的整个土地面积上增加地租,所必需的条件就是:较好的土地必须给随着农业的发展而投到它上面的追加资本提供多于显然较坏的土地所提供的产品;因为,在能够找到办法把新资本按普通利润率使用在A和Z\fnote{A和Z是拉丁字母的第一个和最后一个,这里用来表示这个国家的最坏的和最好的土地。——编者注}之间的任何一块土地上的时候,凡是质量比这块特定的土地好的土地的地租都会增长。”(第195页)“如果经营土地A花费100镑,每年获得110镑,——其中普通利润是10镑,——土地B花费100镑,获得115镑,土地C花费100镑,获得120镑,依此类推,直到土地Z,那末,土地B就提供地租5镑,土地C则提供地租10镑。现在假定,这些土地中的每一块土地都花费200镑来经营。这时A将提供220镑,B—230镑,C—240镑,依此类推,于是地租在土地B就是10镑,在土地C是20镑,等等。”(第193页)“在农业中使用的资本的一般积累,会使一切等级的土地的产品都或多或少地与这些土地的原有质量相应地增长,同时它本身也必定会提高地租,而不管从使用的劳动和资本中得到的收益怎样日益减少,而且事实上也与其他任何原因毫无关系。”(第195页)}\end{quote}

琼斯的功绩在于他最先明确地指出,既然已经假定地租是存在的,那末一般说来它就会{始终要假定生产方式不发生任何变革}由于农业资本,即使用在土地上的资本的增加而增加。这种情况不仅在价格保持不变时可能发生,而且甚至在价格下跌到原来水平以下时也可能发生。

[1125]对于[农业]生产率递减的论断,琼斯反驳说:

\begin{quote}{“英国谷物的平均收获从前每英亩不超过12蒲式耳。现在增加了将近一倍。”(第199页)“依次投入土地的资本和劳动,都会比前一次使用得更经济和更有效。”(第199—200页)“当投在原来那块土地上的资本增加一倍、两倍、三倍时,在收益不减少,耕地的相对肥力不发生变化的情况下,地租也会增加一倍、两倍、三倍等等。”(第204页)}\end{quote}

这就是琼斯胜过李嘉图的第一点。地租既然已经存在,它就能够由于在土地上使用的资本的单纯增加而增加,既不管各种土地的相对肥力怎样变化,也不管相继使用的各笔资本的收益怎样变化,也不管农产品价格怎样变化。

琼斯的另外一点是:

\begin{quote}{“对于地租的增长来说,各种土地肥力的比例完全不变,并不是绝对必要的。”(第205页)}\end{quote}

{琼斯在这里没有看到:正好相反,甚至在全部农业资本使用得更有效时,土地肥力的差别增大也必定会,而且确实会使级差地租的量增大。反之,土地肥力的差别缩小,必定会使级差地租即从这些差别产生的地租减少。去掉原因,也就去掉了结果。然而地租(撇开绝对地租不说)还是能够增长,但那仅仅是由于农业上使用的资本增加了。}

\begin{quote}{“李嘉图没有看到,追加资本在肥力不同的土地上必然带来不同的结果。”(同上)}\end{quote}

(可见,这无非是说,追加资本的使用,会扩大土地的相对肥力的差别,从而使级差地租提高。)

\begin{quote}{“如果用同一个数去乘几个相互之间有一定比例的数,各乘积之间的比例仍将和原数之间的比例相同,但是各个乘积在量上的差额将逐次增大。如果10、15、20各乘以2或4,得数将是20、30、40或40、60、80,它们的相互比例并没有破坏;80和60同40的比例,与20和15同10的比例一样,但是它们的乘积在量上的差额每次都将增大:最初差额是5和10,后来差额是10和20,而最后差额是20和40。”(第206—207页)}\end{quote}

这个规律可简单表述如下:

\todo{}

各项之间的差额在(2)是两倍,在(3)是四倍。差额总和,也是在(2)是两倍,在(3)是四倍,等等。

这就是第二个规律。

第一个规律(琼斯只把它用在级差地租上)是:地租量和使用的资本量一同增加。如果资本为100时地租量等于5,那末资本为200时地租量就等于10。

[1126]第二个规律:如果所有其他情况保持不变,在各种土地上使用的资本[的收益]的差额的比例保持不变,那末,这些差额的总量,从而总地租量或这些差额总和,就会和由于使用的资本增加而引起的这些差额的绝对量的增长一同增长。所以,第二个规律是:在各种土地的肥力的比例不变,但是使用在这些土地上的资本以同等程度增加的情况下,级差地租的量同这些土地上的产品的差额的增长成比例地增长。

\begin{quote}{再往下看:“如果在A、B、C三个等级的土地上各使用100镑,所得产品分别为110、115和120镑,而后来使用200镑,总收入为220、228和235镑,那末产品的相对差额减小了,而且这些土地在肥力上相互接近了。然而,它们的产品量的差额还是从5和10增加到8和15,地租也因而提高了。由此可见,具有使耕地肥沃程度互相接近趋势的那些改良,即使没有其他任何原因起促进作用,也完全能够使地租提高。”(第208页)“种植芜菁和饲养羊,以及在这方面使用的新资本,给较坏的土地的肥力带来的变化,比给较好的土地的肥力带来的变化要大。但是这使较坏的土地和较好的土地的绝对产量增加了,因而也使地租提高了,尽管这时耕地肥力的差别缩小了。”(同上)“至于李嘉图的看法,即改良能够引起地租下降,那末这里就应当想起,农业改良的发现、完善和推广实际上是非常缓慢的。”(第211页)}\end{quote}

{最后这句话只有实践的意义,没有涉及事情的本质;它仅仅指出这些改良进行得不够快,以致不能使供给较之需求有很大的增加,不能使市场价格因而下降。}

最初我们看到:

(1)ABC

10,15,20。

每一个等级使用的资本都等于100。产品等于110、115、120。差额是5+10=15。

由于进行了改良,现在使用的资本增加了一倍,在A、B、C三个等级的每一个等级使用的都不是100,而是200,但是这个资本在不同等级的土地上起着不同的作用,得到的产品等于220(即A的产品的一倍)、228和235。

因此得出:

(2)ABC

20,28,35。

每一个等级使用的资本现在都等于200。产品等于220、228和235。差额是8+15=23。但是这个差额的比率减小了。5∶10(即在第一种情况下B—A[的差额同A的比例])=1/2,10∶10=1,而8∶20仅仅等于2/5,15∶20=3/4。差额的比率减小了,但是差额本身的量增大了。然而,这并不会构成任何新规律,而不过象在第一个规律中那样,证明地租随着使用的资本的增加而增加,虽然产品的增加,在A、B、C,都不是与这些土地肥力的原有差别成比例的。如果由于肥力的这种提高(然而,对于B和C来说,这意味着肥力的[相对]减小,因为不然的话,它们的产品就应当等于230和240),价格会下降,那末,地租提高或者仅仅保持不变,就决不是必然的。

[1127]作为第二个规律的结果,作为这一规律的进一步运用,得出了

第三个规律:如果“那些提高农业上所用资本的效率的改良”,会增加某些地段上获得的超额利润,那末这些改良也会增加地租。

与此有关的有琼斯以下的(以及前面的)论述:

\begin{quote}{“因此,租地农场主地租提高的第一个源泉,是不断增长的积累以及资本在不同土地上产生的不同效果。”(第234页)}\end{quote}

{但是,这里所说的只能是那些直接影响土地肥力的改良,如肥料、轮作制等等。}

\begin{quote}{“那些提高农业上所用资本的效率的改良,会使地租提高,因为这些改良会增加某些地块上获得的超额利润。除非这些改良使土地的产品量增加得那样快,以致超过了需求的增长,它们就总是会引起超额利润的这种增加。提高所用资本的效率的这些改良,通常是随着农业技术的进步和较大量辅助资本〈不变资本〉的积累出现的。地租由于这种原因而提高,随着地租的这种提高而来的通常是把耕作扩展到较坏的土地上去,但一点也不减少在最坏的耕地上使用的农业资本的收益。”(第244页)}\end{quote}

{琼斯十分正确地指出,利润的下降并不证明农业生产率降低。但他本人对于利润下降的可能性解释得非常不完善。他说,或者是产品的数量可能发生变动,或者是产品在工人和资本家之间的分配可能发生变动。在这里对于利润率下降的真正规律还毫无所知。

\begin{quote}{“利润的下降不是农业生产率降低的证据。”(第257页)“利润部分地取决于劳动产品量,部分地取决于劳动产品在工人和资本家之间的分配;所以利润量能够由于这两个因素中的任何一个因素发生变动而变动。”(第260页)}\end{quote}

由此也就产生了琼斯所表述的一条错误规律:

\begin{quote}{“撇开课税的影响不谈,如果所有的生产阶级合起来看,其收入有了明显的减少〈这里没有说,什么是收入,是使用价值还是交换价值,是指利润量还是利润率〉,如果出现利润率下降而没有通过提高工资得到补偿,或者出现工资下降而没有通过提高利润率得到补偿〈这正是错误的李嘉图规律〉,那就可以得出结论说,劳动和资本的生产力已经有些减低。”(第273页)}}\end{quote}

琼斯正确地理解到,尽管绝对地说农业实际上是在不断进步,但是随着社会的发展,农产品价值同工业品相比会有相对的增长:

\begin{quote}{“在一个国家的发展过程中通常可以看到:在人口不断增长的情况下,工业的能力和技能的增长程度大于可以期待于农业的增长程度。这是不容争辩的真理。所以,随着国家的进步,没有农业生产率的任何绝对的下降,也能期待农产品的相对价值增长。”(第265页)}\end{quote}

但这并不说明农产品的货币价格的绝对的上涨,除非金的价值下降,而这种下降在工业中由于工业品价格的更大下降而得到平衡和超过平衡,但是这样的平衡在农业中不会发生。甚至[1128]在不发生金(货币)的价值普遍下降的时候,上述情况也会出现,例如某个国家用自己的日劳动换取的货币多于同它竞争的国家时就是如此。

琼斯不相信李嘉图规律在英国的作用,但承认这一规律的抽象的可能性,理由如下:

\begin{quote}{“如果地租增长仅仅是由于李嘉图提出的那个原因,也就是说,由于‘使用追加劳动量带来比较少的收益’,因而较好土地的部分产品会转到土地所有者手里,那末,总产品中被土地所有者当作地租拿去的平均份额就必然要增长。第二,在这种情况下,就必然会有更大部分人口的劳动使用在农业上。”(第280—281页)}\end{quote}

(后面一点是不确切的。有可能:更多的间接[secondary]劳动被使用,即更多的由工商业提供的商品加入了农业过程,可是总产品并不相应增加,使用的直接[农业]劳动量也不增多,甚至会更少。)

\begin{quote}{“我们在英国的统计中发现三个事实:随着耕地面积的扩大,全国的地租总额增加了;从事农业的那部分人口减少了;土地所有者从产品中得到的份额减少了。”(第282页)}\end{quote}

(最后一点,完全可以和利润率下降一样,用补偿不变资本的那部分产品增加来解释。这时地租在量和价值上都可能增长。)

\begin{quote}{“亚·斯密说:‘随着农业改良的发展,地租同耕地面积相比虽然增加了,同土地产品相比却减少了。’\endnote{亚·斯密的这段话载于他的《国富论》第二篇第三章。——第452页。}”(第284页)}\end{quote}

琼斯把不变资本叫作“辅助[auxiliary]资本”。

\begin{quote}{“从农业部在不同时期提出的各种报告可以看出,在英国,使用在农业上的全部资本同用于维持工人的资本之比是5∶1,即所使用的辅助资本,比用来维持直接使用在农业上的劳动的资本多三倍。在法国这种比是2∶1。”(第223页)“如果有一定量的追加资本,以过去劳动的结果的形式被使用,以便促进当前使用的工人的劳动,那末,要使这种资本的使用有利可图,因而成为经常可行的,只要有较少的年收益就够了,可是,如果用同量的新资本来维持追加工人,那就需要有较多的年收益。”(第224页)“假定在土地上花100镑来维持三个工人,他们生产自己的工资和10%的利润,即总共110镑。假定花费的资本数量增加一倍。起初有三个新工人被使用。增加的产品应当等于110镑,即三个追加工人的工资加10镑利润。现在假定,追加的100镑以工具、肥料的形式,或者以过去劳动的其他任何结果的形式被使用,而所使用的工人人数保持不变。就算这笔辅助资本平均够用五年。在这种情况下,资本家的年收益必须能够支付[追加资本的]10%的利润,并且用20镑抵补这笔资本的年损耗;因此,要使第二个100镑的继续使用有利可图,所必需的年收益是30镑,而用这100镑来使用直接劳动所必需的金额则是110镑。所以,很明显,在不再能用同量资本来维持追加劳动的时候,农业上的辅助资本的积累也是可能的,而且农业上的这种资本的积累能够在无限长的时期内继续下去。”(第224—225页)“可见,辅助资本的增长,一方面,在直接或间接地花费在土地耕种上的劳动量[1129]相同的情况下,会提高人对地力的支配权,另一方面会减少使进一步使用一定量新资本能获利所必需的年收益。”(第227页)“我们假定,例如有100镑的农业资本,全用来支付工资,并提供10%的利润。在这种情况下,租地农场主的收入等于工人收入的十分之一。如果这种资本增加一倍、两倍等等,那末租地农场主的收入将同工人的收入保持原有的比例。但是,如果工人人数保持不变,而资本量增加一倍,那末利润就变成20镑,或者说,是工人收入的五分之一。如果资本增加三倍,利润就变成40镑,或者说,是工人收入的五分之二。如果资本增加到500镑,利润就变成50镑,或者说,是工人收入的一半。资本家在社会上的财富、影响,也许在某种程度上还有他们的人数,都会与此相应地增长……随着资本的增长,一定数量追加直接劳动的使用,往往也成为必要。然而这种情况并不妨碍辅助资本的连续不断的、相对的增长。”(第231—232页)}\end{quote}

在这段话中首先有一点是重要的,即随着资本的增长,“辅助资本”同可变资本相比会增加,换句话说,可变资本同不变资本相比会相对地减少。

当“辅助资本”中由固定资本构成的部分,即不变资本中其周转历时数年,其价值仅仅以损耗的形式逐年加入产品的那一部分增长的时候,年收益同预付资本相比减少的现象到处都会发生,而不仅仅是在农业中。诚然,在工业中一年内加工的原料量的增加,要比固定资本量的增加快得多(例如,试把一台纺纱机每周以及每年用掉的棉花量,同一架纺车用掉的棉花量比较一下)。但是如果假定,例如(大规模的)裁缝业加工的原料的价值,和纺纱业加工的一样多(裁缝业加工的原料的量虽然少,但是比纺纱业的原料贵),那末裁缝业的年收益就应当比纺纱业的年收益大得多,因为在纺纱业中有较大部分已耗费的资本(固定资本)只是以年折旧的形式加入产品。

在资本增长,但增长的只是不变资本而不是可变资本的时候,农业上(这里可以被看作原料的东西,如种子,不会和不变资本的其他部分,特别是固定资本按同一比例增长)的年收益的价值自然会减少。因为,可变资本必须在产品中全部得到补偿,而固定资本只是以年折旧的形式,按照它每年消费的程度得到补偿。假定谷物价格既定,如果一夸特等于1/2镑,那末要在利润是10%的时候补偿100镑可变资本,就需要220夸特,而补偿20镑的磨损和10镑的利润,只需要60夸特(=30镑)。较少的绝对收益(在这里,和在类似条件下的工业中的情况一样)提供同样的利润。但是在这里琼斯毕竟有种种错误。

首先,不能说(在既定的前提下)土地的生产力增长了。它们增长了,是同直接使用的劳动相比,而不是同使用的全部资本相比而言。只能说,现在需要较少的总产品,就可以提供与以前相同的纯产品,即相同的利润。

[1130]其次,在这种特殊的领域内,同工人的收入相比,租地农场主收入的增长,在这里的总产品中转化为利润的部分同工人所得到的部分相比日益增加的情况下,确实有重要的意义。这样,农场主-资本家的“财富和影响”,同他的工人的“财富和影响”相比,才会不断地增长和扩大。但是,琼斯却似乎是这样计算的:10比100是1/10;20镑比120(即100花在劳动上,20表示损耗)是1/6,而这20镑等于付给工人的工资的1/5,等等。但是笼统地说,花在劳动上的资本减少时利润率会提高,那是再错没有了。恰好相反。在这种情况下剩余价值会相对减少,因而利润率会降低。至于特殊的单个租地农场主(每个单个企业也一样),利润率却能够保持不变,不管在那里200镑资本是使用三个工人,还是使用六个工人。

为了使地租等于超额利润即超过平均利润的余额,前提是,不仅农业要在形式上从属于资本主义生产,而且利润率在各个生产部门中,特别是在农业和工业之间,要平均化。否则地租就会等于超过工资的余额(也就是利润)。地租也可以代表利润的一部分,或者甚至是工资的扣除部分。

\tchapternonum{(2)理·琼斯《1833年2月27日在伦敦皇家学院讲述的政治经济学绪论。附工资讲座大纲》1833年伦敦版。[“国家的经济结构”的概念以及用它来说明社会制度的不同类型的尝试。关于“劳动基金”的混乱思想]}

[琼斯在《绪论》中写道:]

\begin{quote}{“在一个民族的一定历史时期,土地所有权几乎普遍都是或者掌握在国家政府的手里,或者掌握在从政府得到权利的人的手里。”(第14页)“我所说的国家的经济结构,是指各不同阶级之间的关系,这些关系最初由于土地所有权的制定和土地剩余产品的分配而建立起来,后来由于资本家的出现而(在或大或小的程度上)发生了变化和变动,资本家则是作为从事财富的生产和交换并向工人人口提供食物和工作的当事人出现的。”(第21—22页)[1130]\endnote{手稿中接着有一段属于论拉姆赛那一章的简短插话,这段插话以脚注形式放在本册第389页上。——第456页。}}\end{quote}

[1130]琼斯所说的“劳动基金”[《LaborFund》]是指

\begin{quote}{“劳动者所消费的收入总量,不管这些收入的源泉是什么”。(《大纲》第44页)}\end{quote}

琼斯的主要论点(“劳动基金”这个术语也许是属于马尔萨斯的?)\endnote{琼斯称为“劳动基金”(《LaborFund》)的东西,在马尔萨斯那里叫作“用来维持劳动的基金”(《fundsforthemaintenanceoflabour》)。这个术语已经多次出现在马尔萨斯的《人口原理》第一版中(1798年伦敦版第303、305、306、307、312、313页及其他各页)。在第五版(1817年伦敦版)中,它主要出现在第三篇第五章和第六章。在马尔萨斯的《政治经济学原理》中也出现过这个术语,例如在本册第30页所引用的地方。——第456页。}如下:整个社会经济结构是围绕着劳动形式旋转的,也就是说,是围绕着劳动者借以占有自己的生活资料,或者说,占有其产品中他赖以生存的那一部分产品的形式旋转的;这个“劳动基金”有各种不同的形式,资本仅仅是其中的一种形式,是历史上出现较晚的一种形式。亚·斯密提出的那个重大区别——劳动是由资本支付还是直接由收入支付——只有在琼斯那里才得到它能够得到的充分阐明,并且成为理解社会上各种经济结构的一个重要关键。与此同时,这样一种荒诞的观念也因此消失了:似乎因为在资本中工人的收入一开始就以资本家的占有物或积蓄物的形式出现,所以这就不仅仅是一种形式上的区别。

\begin{quote}{“甚至在西欧各国我们还能发现由它们的土地和劳动的产品的特殊分配方式所产生的社会制度的影响,这种分配方式是[1131]在它们作为农业国存在的初期形成的〈也就是说,分配是在下面几个阶级之间进行:(1)农业劳动者阶级,(2)土地所有者阶级,(3)仆人、侍从和手工业者,这些人直接或间接地分享土地所有者的收入〉。”(《绪论》第16页)“这种经济结构经受的变动,其基本因素和动力是资本,即为了赚取利润而使用的积累财富……在一切国家中,这里所指的财富的特殊部分,对于社会各个不同阶级之间的联系的变化起着极大的作用,并且对于这些阶级的生产力发生决定的影响……在亚洲以及在欧洲的一部分(以前是在全欧洲)非农业阶级几乎完全靠其他阶级的收入,主要是靠土地所有者的收入维持生活。如果你需要一个手工业者的劳动,你就供给他材料;他到你家里来,你管他饭,并且付给他工资。过了一段时间,出现了资本家;他备置材料,预付工人的工资,成为工人的雇主,并且是生产出来的产品的所有者,他用这种产品交换你的货币……这样,在土地所有者和一部分非农业劳动者之间就有了一个中间阶级,这些非农业劳动者现在要靠这个中间阶级来得到工作和生存资料了。以前联结社会的纽带现在削弱和瓦解了;另外的联系,另外的相互依赖原则现在联结着社会的各个不同阶级,新的经济关系出现了”……“这里,在英国,不仅绝大多数非农业劳动者几乎完全靠资本家雇用,而且农业劳动者也成了资本家的仆人。”(同上,第16页及以下各页)}\end{quote}

琼斯的《工资讲座大纲》和他的论地租的书有以下区别。在论地租的书中考察的是土地所有权的各种不同形式,和这些形式相适应的则是劳动的各种不同社会形式。在《大纲》中,琼斯从劳动的这些不同形式出发,并且把土地所有权的各种不同形式和资本当作它们的产物来考察。劳动者的劳动的社会规定性,和劳动条件——特别是土地、自然界,因为这个关系包括其他一切关系——对劳动者所采取的形式相适应。但是,实际上劳动者的劳动的这个社会规定性只是在上述形式中得到自己的客观表现。

因此,我们将看到,“劳动基金”的各种不同形式,是和劳动者同他自己的生产条件发生关系的不同方式相适应的。他以什么方式占有自己的产品(或产品的一部分),要看他同他的生产条件发生什么关系。

\begin{quote}{琼斯说:“劳动基金可以分为三类:(1)由劳动者自己生产并由他们自己消费的收入,这些收入决不属于其他任何人。{在这种情况下,劳动者实际上必然是他自己的生产工具的所有者,不管他的收入具有什么样的特殊形式。}(2)属于和劳动者不同的那些阶级的收入,这些阶级花费这些收入来直接维持劳动。(3)真正的资本。劳动基金的所有这些不同的种类都可以在我们本国观察到;但是,如果我们看看其他国家,我们就会发现,这个基金的某些部分在我们这里极为有限,在其他一些国家却是居民生存的主要源泉,并且决定着多数国民的性格和状况。”(《大纲》第45—46页)关于第一点。“农业劳动者,或者说,占有土地的农民的工资……这些农业劳动者,或者说农民,是份地的继承者、私有者、佃农。佃农就是农奴、分成制佃农、茅舍贫农。后者是爱尔兰特有的。所有这几种土地耕种者-农民的收入中往往混有地租或利润之类的东西,但是,如果他们主要是依靠自己体力劳动的报酬生活,他们就应当被看作工资劳动者[wageslabourers]。因此,劳动农民中包括:(α)份地的继承者,他们是农业[1132]劳动者。古代的希腊。现今的亚洲,尤其是印度。(β)农民-私有者。法国、德国、美国、澳大利亚、古代的巴勒斯坦。(γ)茅舍贫农。”(第46—48页)}\end{quote}

这里值得注意的是,劳动者为自己再生产“劳动基金”。这种“劳动基金”不转化为资本。劳动者直接生产自己的“劳动基金”,他也直接占有它,尽管他的剩余劳动,按照他和他自己的生产条件发生关系的特殊形式,由他自己占有全部或一部分,或者全部由其他阶级占有。琼斯把这类劳动者叫作“工资劳动者”,这纯粹是经济学上的偏见。他们并没有工资劳动者即雇佣工人的任何特征。既然在资本统治下归工人自己占有的那部分产品是工资,那末归任何一个劳动者自己消费的那部分产品也就必然是工资,——这是一种资产阶级政治经济学的美妙的概念。

\begin{quote}{关于第二点。[靠这种劳动基金维持生活的人口,]“在英国限于家仆、士兵、水手和少数独立从事劳动并从其雇主的收入中得到支付的手工业者。在地球上相当大的地区,这种劳动基金维持着几乎所有的非农业劳动者。以前这种基金在英国占优势。沃里克——国王制造者\endnote{琼斯在《关于劳动和资本的讲义》中关于沃里克是这样说的:“据说,著名的沃里克伯爵,‘国王制造者’,不得不每天在自己的城堡和家里供养四万人。”——第459页。}。英国的贵族。如今这种基金在东方占优势。手工业者、仆人。靠这种基金维持的庞大军队。在整个亚洲这种基金集中于君主之手所产生的后果。某些城市的突然兴盛。突然衰落。撒马尔汗、坎大哈等地”。(第48—49页)}\end{quote}

琼斯忽略了两个主要的形式:第一,具有农业和工业合一特点的亚洲村社;第二,中世纪的城市行会制度,这种制度部分地在古代世界也存在过。

\begin{quote}{关于第三点。“资本决不应和世界上的一般劳动基金混淆起来,劳动基金的大部分是由收入构成的。国家的各种收入……都参加资本赖以形成的积累。在不同的国家和不同的社会发展阶段,它们以不同的程度参加这种积累。例如有这样的情况,即主要依靠工资和地租进行积累。”(第50页)}\end{quote}

剩余劳动转化为资本(而不是直接作为收入同劳动交换),这就造成一种印象,似乎资本是收入的积蓄。这就是琼斯的主要观点。确实,在社会发展过程中,资本量是由再转化为资本的收入构成的。但是在资本主义生产中,连最初的“劳动基金”本身也表现为资本家的积蓄。再生产出来的“劳动基金”本身,不象在第一种情况下那样为劳动者所占有,而是表现为资本家的财产,表现为对工人来说是别人的财产。而这一点是琼斯没有阐明的。

琼斯在这个教学大纲里关于利润率及其对积累的影响的论述是软弱无力的:

\begin{quote}{“在其他一切条件相同的情况下,一个国家从它的利润中进行积蓄的能力,随着利润率的变化而变化:这种能力在利润率高时就大,在利润率低时就小;但是在利润率下降时,其他一切条件就不会保持不变。使用的资本量和人口数目相比可能增加。”}\end{quote}

(琼斯不懂得,怎样由于使用的资本“可能”增加而发生如下情况:正是因为“使用的资本量和人口数目相比增加了”,利润率才下降。但是他正在接近于正确的观点。}

\begin{quote}{“积累的动因和便利条件可能增加……在利润率低时积累的速度通常会比人口的增加快,例如在英国;在利润率高时积累的速度通常会比人口的增加慢,[1133]例如在波兰、俄国、印度等地。”(第50—51页)}\end{quote}

在利润率高的地方(撇开北美的情况不谈,在那里,一方面,资本主义生产占统治地位,另一方面,一切农产品的价值都低),利润率高通常是由于,资本主要由可变资本构成,即直接劳动占优势。假定资本是100,其中五分之一是可变资本,并且假定剩余劳动是三分之一工作日。在这种情况下利润等于10%。现在假定五分之四的资本由可变资本构成,而剩余劳动是六分之一工作日。在这种情况下利润等于16%。

\begin{quote}{“有一种理论错误地以为,在利润率随着国家的发展而降低的地方,日益增长的人口的生存资料必定减少。这种错误的基础是:(1)错误的概念,即利润的积累在利润率低的地方必定会慢,在利润率高的地方必定会快;(2)错误的假定,即利润是积累的唯一源泉;(3)错误的假定,即地球上所有劳动者的生活都是依靠积累和收入的积蓄,而决不是依靠收入本身。”(第51页)}\end{quote}

[琼斯指出]

\begin{quote}{“当资本负起预付工资的责任时,国家的经济结构中就会发生变动”。[1133]}\end{quote}

[1157]理·琼斯在下面的论述中作了正确的概括:[1157]

\begin{quote}{[1133]“用于维持劳动的资本量可以独自发生变化,而不管资本总量的变化如何〈这是一个重要的论点〉……有时可以看到,当资本本身变得更加充裕的时候,在业人数的大变动,以及由此而来的大灾难,就会变得更加频繁。”(第52页)\endnote{这段引文马克思在手稿第XVIII本的最后一页(手稿第1157页)又引了一次,并且增加了这里所引用的马克思的补充评论。——第460页。}[1133]}\end{quote}

[1157]总资本可以保持不变,但是可变资本可以发生变化(特别是减少)。资本两个组成部分的比例的变化,不一定意味着总资本量方面的变化。

另一方面,总资本的增长不仅可以和可变资本的相对减少有关,而且可以和它的绝对减少有关;总资本的增长总是和可变资本的剧烈变动有关,因此,也和“在业人数的变动”有关。[1157]

[接着,琼斯在这个关于工资的教学大纲中写道:]

\begin{quote}{[1133]“劳动者从依赖一种基金逐渐向依赖另一种基金过渡的各个时期……农业劳动者向着由资本家支付报酬转变……非农业阶级向着受资本家雇用转变。”(第52—53页)}\end{quote}

琼斯在这里所说的“转变”,就是我所说的“原始积累”。只有形式上的区别。它也是和庸俗的“积蓄”观点对立的。

\centerbox{※     ※     ※}

\begin{quote}{“奴隶制。奴隶可以划分为牧羊奴隶、耕作奴隶、家庭奴隶,最后,还有一种既是耕作奴隶又是家庭奴隶的混合型奴隶。我们看到,有的奴隶是耕种土地的农民,有的是靠富人的收入维持生活的仆人或手工业者,有的是靠资本维持生活的工人。”(第58—59页)}\end{quote}

但是只要奴隶制占统治地位,资本主义关系就每次只能偶然地作为从属关系出现,决不能作为统治的关系出现。

\tchapternonum{(3)理·琼斯《国民政治经济学教程》1852年哈特福版}

\tsectionnonum{[(a)资本主义生产方式的历史观的萌芽同关于资本只是“积累的储备”的资产阶级拜物教观点的结合。生产劳动和非生产劳动问题]}

[琼斯在《国民政治经济学教程》中写道:]

\begin{quote}{“国民的劳动生产率实际上取决于两种情况。第一,取决于他们所生产的财富的原始源泉〈土地和水〉是富饶还是贫乏。第二,取决于在利用这些源泉或者在对取自这些源泉的商品进行加工时,他们所使用的劳动的效率如何。”(第4页)“人的劳动效率取决于:(1)劳动的连续性;(2)劳动用来实现生产者的目的所具备的知识和技能;(3)帮助劳动的机械力。”(第6页)“劳动者在生产财富时所使用的力……可以由以下几点来增强:(1)让比他们自身的动力大的动力为他们服务……(2)用更能发挥机械效益的方法去使用他们所拥有的某一数量或某一种类的动[1134]力。例如,40马力的蒸汽机在铁路上发挥的效力,和在公路上发挥的就不同。”(第8页)“用两匹马拉一种较好的犁,可以和用四匹马拉一种较坏的犁完成同样多和同样好的劳动。”(第9页)“蒸汽机不是一种单纯的工具;它能提供追加动力,而不单是提供更能发挥机械效益地使用工人已经拥有的力的手段。”(第10页注)}\end{quote}

可见,在琼斯看来,工具和机器的差别就是如此。工具向工人提供更能发挥机械效益地使用他已经拥有的力的手段;机器能使动力增加。(?)

\begin{quote}{“资本……就是由收入中积蓄起来并用来获取利润的财富所构成的。”(第16页)“资本的可能的源泉……显然是构成社会的所有个人的所有可以积蓄起来的收入。最有助于国民资本进步的几种特殊收入,在它们各个不同发展阶段上是不同的,因此它们在处于这种发展的不同阶段的各个国家里也是截然不同的。”(同上)“因此,利润决不是资本形成和增加的唯一源泉,而在社会的初期阶段,同工资和地租相比,利润甚至是一个不重要的积累源泉。”(第20页)“当国民劳动的力量真正得到显著发展时,利润作为积累的源泉就相当重要了。”(第21页)}\end{quote}

按照这种说法,资本是构成收入的那种财富的一部分,这一部分不是作为收入被消耗,而是用来生产利润。利润已经是剩余价值的一种形式,这种形式专门以资本为前提。如果以资本主义生产方式的存在即资本的存在为前提,那末琼斯的解释是对的。换句话说,在应当解释的东西被当作前提的时候,他的这个解释是对的。但是琼斯在这里所指的是一切不作为收入被消耗的收入,而是为达到致富目的即在生产上被消耗的收入。

不过,在这里有两点是重要的。

第一,在经济发展的一切阶段上都有一定的财富积累,也就是说,一部分采取扩大生产规模的形式,一部分采取货币贮藏之类的形式。当“工资”和地租在社会上占优势的时候,——也就是按照上面所说,这时,一般不归劳动者自己所有的剩余劳动和剩余产品的大部分归土地所有者所有(在亚洲,归国家所有),另一方面,劳动者则自己再生产自己的“劳动基金”,不仅自己生产自己的“工资”,而且自己把它付给自己,并且在大多数情况下(在这种社会状况下,几乎是经常)他至少能够使自己得到自己的剩余劳动和剩余产品的一部分,——在这样的社会状况下,“工资”和地租也是积累的主要源泉。(在这里利润只限于商人等等的利润。)只有当资本主义生产占统治地位,当它不只是偶尔存在,而是使社会的生产方式从属于它时;当资本家实际上把全部剩余劳动和全部剩余产品首先直接占为己有,尽管他不得不把其中的一部分付给土地所有者等等时,——只是从这时候起,利润才成为资本的主要源泉,积累的主要源泉,由收入中积蓄起来并用来获取利润的财富的主要源泉。同时这要有一个先决条件(在资本主义生产方式占统治地位的情况下,这是不言而喻的),即“国民劳动的力量真正得到显著发展”。

有一些蠢驴,他们以为,没有资本的利润,就不会有积累,或者他们这样来为利润辩护,说资本家做出了牺牲,为了生产的目的,由自己的收入中进行积蓄;对于这些人,琼斯回答说,“积累”这个职能,正是在这种特殊的生产方式(资本主义生产方式)下,才主要由资本家承担,而在以前的生产方式下,在这个过程中起主要作用的是劳动者自己,部分是土地所有者,利润在那里几乎不起任何作用。

当然,积累的职能总是会转到这样的人的身上:(1)占有剩余价值,(2)特别是占有剩余价值,同时又是生产本身的当事人。因此,如果有人说,[1135]因为资本家通过由利润中进行“积蓄”来积攒自己的资本,因为他执行积累的职能,所以利润是合理的,那末,这只是说,资本主义生产方式因为事实上是存在的,所以是合理的,这种说法对过去的和以后的生产方式也同样适用。如果有人说,用其他办法不可能进行积累,那就是忘记了,这个特定的积累方法——通过资本家进行积累——有其发生的特定的历史日期,并且会朝着其消亡的(也是历史的)日期走去。

第二,既然有那么多积累的财富通过各种手段转到资本家手中,以致他们能够支配生产,那末最大量的现有资本——经过一定期间——可以被认为完全是由利润(收入)产生的,也就是由资本化的剩余价值产生的。

有一点琼斯提得不够,说实在的,他只是作了暗示,这就是:如果劳动生产者自己付给自己“工资”,并且他的产品不是先采取由他人的收入中“积蓄起来”的形式,然后再由他人付还给劳动者,那末劳动者就必须占有自己的生产条件(无论是作为私有者,还是作为佃农或份地的继承者等等)。要使他的“工资”(以及“劳动基金”)作为别人的资本同他相对立,就必须预先使他丧失这些生产条件,而这些生产条件必须采取别人的财产的形式。只有在劳动者的“劳动基金”连同他的生产条件被夺去,并且作为与工人对立的资本独立出来以后,进一步的过程(这个过程涉及的不是这些原有条件的单纯的再生产,而是它们的进一步发展)才会开始,以致生产条件和“劳动基金”都是作为由他人的收入中“积蓄起来”以便转化为资本的东西出现在工人面前。工人丧失了对自己的生产条件的占有,因而也丧失了对自己的“劳动基金”的占有以后,也就丧失了积累的职能,他在财富上所追加的一切,都表现为他人的收入,而这种收入必须预先被这些人“积蓄起来”,即不应作为收入消耗掉,才能执行资本的职能以及工人的“劳动基金”的职能。

因为琼斯本人所叙述的是这样的社会状况,在这种社会状况下,事情还不是这个样子,当时[劳动者和劳动条件之间]还存在着统一,所以他自然必须把上述的“分离”当作资本真正的形成过程提出来。一旦有了这种“分离”,资本的形成过程自然就会发生,——它将继续并且扩大,——因为工人的剩余劳动现在总是作为别人的收入同工人相对立,也只有通过这种收入的“积蓄”才能发生财富的积累和生产规模的扩大。

收入再转化为资本。如果资本{即生产条件和劳动者相分离)是利润的源泉(也就是说,剩余劳动表现为资本的收入,而不表现为劳动的收入},那末现在利润又成了资本的源泉,成了新资本形成的源泉,也就是说,追加的生产条件作为资本同工人相对立,作为手段来保持工人的工人身分并一再占有工人的剩余劳动。劳动者和劳动条件之间原有的统一(我们不谈奴隶关系,因为当时劳动者自身属于客观的劳动条件}有两种主要形式:亚洲村社(原始共产主义)和这种或那种类型的小家庭农业(与此相结合的是家庭工业)。这两种形式都是幼稚的形式,都同样不适合于把劳动发展为社会劳动,不适合于提高社会劳动的生产力。因此,劳动和所有权(后者应理解为对于生产条件的所有权)之间的分离、破裂和对立就成为必要的了。这种破裂的最极端的形式(在这种形式下社会劳动的生产力同时会得到最有力的发展)就是资本的形式。原有的统一的恢复,只有在资本创造的物质基础上,并且只有通过工人阶级和整个社会在这个创造过程中经历的革命,才有可能实现。

琼斯提得不够的还有下面一点:

直接作为收入同劳动交换的那种收入,只要它不是雇用次要劳动者的独立劳动者的收入,那就是土地所有者的收入,这种收入来源于独立劳动者付给他的地租,他同他的仆人和侍从没有以实物形式把这种地租完全消费掉,而用其中的一部分来购买次要劳动者的产品和服务。因此,收入同劳动的这种交换总是以第一种关系[土地所有者和向土地所有者交付地租的独立劳动者之间的关系]为前提。

[1136]{即使产业资本家使用他自己的资本,也有一部分利润被看作是利息,这仅仅是因为,这种收入具有单独的存在形式,同样,在资本主义生产方式的基础上,即使劳动者拥有自己的生产资料,并且不雇用其他任何劳动者,这些生产资料仍被看作是资本,而劳动者自己的在普通工资以外实现的那部分劳动,表现为由他的资本产生的利润。在这种情况下,劳动者本身将分解为不同的经济身分。他作为他自己的工人得到自己的工资,又作为资本家得到自己的利润。这个评论属于《收入及其源泉》\endnote{关于收入及其源泉,马克思在1861—1863年手稿第XV本后半部分作了论述,他在这方面揭示了庸俗政治经济学的阶级根源和认识论根源(见本册第499—600页)。这个“补充部分”(这是马克思在手稿第XIV本封面上对它的称呼),也就是对《剩余价值理论》正文的补充,马克思后来决定放在《资本论》第三部分,这从他在1863年1月拟定的这一部分的计划可以看出;按照这一计划,第九章的标题应该是《收入及其源泉》(见本卷第1册第447页)。——第466页。}那一章。}

\begin{quote}{“有两种财富,它们对于国民生产力的影响是有区别的,一种财富是积蓄起来,并且作为工资支出来获取利润,一种财富是从收入中预付来维持劳动。说到这种区别时,我用资本这个词,只是为了表示由收入中积蓄起来并用来获取利润的那部分财富。”(第36—37页)“我们也许可以……把资本理解为一切被用来维持劳动的财富,而不问这种财富是否经过了预先的积蓄过程……在这种情况下,如果我们仔细研究一下不同国家中和不同条件下的劳动阶级以及向劳动阶级支付的人的状况,我们就必须把积蓄起来的资本和没有经过积累过程的资本加以区别;一句话,把是收入的资本和不是收入的资本加以区别。”(第36页)“除了英国和荷兰,在旧大陆的一切国家中,农业劳动者的工资不是从由收入积蓄和积累的基金中预付,而是由劳动者自己生产出来,并且除了以供他们自己直接消费的储备形式存在之外,从来不以其他任何形式存在。”(第37页)}\end{quote}

琼斯和其他政治经济学家(也许西斯蒙第除外)不同的地方是,他把资本的社会的形式规定性作为本质的东西强调出来,并把资本主义生产方式和其他生产方式之间的一切区别归结为这个形式规定性。资本的这个社会的形式规定性就是,劳动直接转化为资本,另一方面,这个资本购买劳动不是为了它的使用价值,而是为了增加自己本身的价值,为了创造剩余价值(更高的交换价值),“用来获取利润”。

但是,这里同时也表明,为使收入转化为资本而进行的“收入的积蓄”本身,以及“积累”本身,仅仅在形式上不同于使“财富被用来维持劳动”的其他条件。领取由资本“预付的”工资的英国和荷兰的农业工人,如同法国农民或俄国的独立经营的农奴一样,也是“自己生产自己的工资”。如果从生产过程的连续性来考察生产过程,那末资本家今天作为“工资”预付给工人的,不过是这个工人昨天“生产出来的”产品的一部分。因此,[资本主义生产方式和其他生产方式之间的]区别并不在于,在一种场合工人生产他自己的工资,而在另一种场合不生产。区别在于,他的这个产品[在一种场合]表现为工资;还在于,在一种场合[在资本主义生产方式的条件下]工人的产品(工人的产品中构成“劳动基金”的那一部分),第一,表现为别人的收入;但是,第二,不当作收入来花费,也不花费在收入借以直接消费掉的劳动上面,第三,作为资本和工人相对立,资本把产品的这一部分归还给工人,不是简单地去交换等价物,而是去交换比物化在给工人的产品中的劳动量更大的劳动量。因此,工人的产品表现为,第一,别人的收入,第二,收入的“积蓄”,用来购买劳动以便获得利润,也就是说,它表现为资本。

工人自己的产品作为资本和他相对立的这个过程,也就是琼斯所说的:“劳动基金”“经过预先的积蓄过程”,“经过积累过程”,即在再转化为工人的生存资料以前,“以另一种形式存在”(琼斯在这里直接谈判的只是形式变化),而不是“以供劳动者直接消费的储备形式存在”。全部区别在于工人所生产的“劳动基金”在以工资形式重新回到工人手里以前所经受的那个形式转化。因此,在独立农民或独立手工业者那里,“劳动基金”决不会采取“工资”的形式。

[1137]就“劳动基金”来说,“积蓄”和“积累”在这里不过是劳动者的产品经受的那些形式转化的名称。独立劳动者和雇佣工人一样,消费自己的产品,或者更确切地说,后者和前者一样,消费自己的产品。区别只在于,在雇佣工人那里,他的产品表现为由另一个人——资本家——的收入中积蓄起来,或者说,积累起来的东西。实际情况则是,这个过程使资本家能够为自己“积蓄”,或者说,“积累”工人的剩余劳动;因此,琼斯也非常有力地强调了这种情况:在非资本主义生产方式下,积累不是来源于利润,而是来源于“工资”,即来源于独立农业劳动者的收入或用自己的劳动直接同收入交换的手工业者的收入(否则怎样会从这些独立农业劳动者和独立手工业者当中产生出资产者呢?),并且来源于土地所有者得到的地租。但是为了使“劳动基金”经过这些转化,还必须使生产条件象资本那样和劳动者相对立,在其他形式中情况就不是这样。在这一场合,财富的增加,不表现为来源于工人,而表现为通过积蓄,通过剩余价值再转化为资本而从利润得来,——正如“劳动基金”本身(在由于新的积累而增加以前)作为资本和工人相对立一样。

“积蓄”,就这个词的本来意义讲,只有在谈到那种把自己的收入资本化的资本家对那种把自己的收入作为收入吃光花光的资本家的关系时才有意义,它对于说明资本家和工人之间的关系则毫无意义。

有两个主要事实可以说明资本主义生产的特点:

第一,生产资料积聚在少数人手中,因此不再表现为单个劳动者的直接财产,而表现为社会的生产能力,尽管首先表现为不劳动的资本家的财产。在资产阶级社会里,这些资本家是生产资料的受托人,并享受从这种委托中得到的全部果实。

第二,劳动本身由于协作、分工以及劳动同社会对自然力支配的结果相结合,而组织成为社会的劳动。

从这两方面,资本主义生产把私有财产和私人劳动取消了,虽然还是处在对抗的形式中。

在亚·斯密看来,生产劳动和非生产劳动的主要区别是,前者直接同资本交换,后者直接同收入交换,——这一区别的意义,只是在琼斯那里才得到充分的阐明。这里指出,第一种劳动说明资本主义生产方式的特征;第二种劳动,在它占统治的地方,属于以前的各种生产方式,而在它只是间或出现的地方,则限于(或者应当限于)那些不直接生产财富的领域。

\begin{quote}{“资本是一种工具,借助这种工具,可以把能够提高人的劳动效率和国民生产力的一切因素推动起来……资本是过去劳动的积累起来的结果,这个结果被用来在财富生产工作的某个部分获得某种效果。”(第35页)}\end{quote}

(琼斯在对这段话作的注释中说:

\begin{quote}{“我们可以并且有理由认为,在所生产的商品没有到达要消费它的人手中以前,生产行为并没有结束;在此以前所做的一切,都是为着这一目的。给我们把茶从哈特福运到我们学院来的小店主的马和车,对于我们要得到茶以满足消费的目的来说,是同采茶和焙茶的中国人的劳动一样必要的。”)“但是……这个资本……并不是在每个社会中都完成它所能完成的一切任务。在一切场合,它都是逐渐地、依次地着手完成这些任务,一个值得注意并且极端重要的事实是,有一种特殊职能,它的执行对于资本的力量在资本的其他一切职能中的重大发展是非常重要的,这种特殊职能正是资本对地球上大部分劳动者还从来没有执行过的那种职能。”(第35—36页)“我指的是工资的预付。”(第36页)“由资本家预付工资的劳动者,在地球上还不到四分之一……这一事实……在说明各国的进步程度时有头等重要意义。”(同上)[1138]“资本,或者说,积累的储备,只是后来,当它在财富的生产中执行了其他各种职能以后,才担负起向劳动者预付工资的职能。”(第79页)}\end{quote}

在琼斯的最后这句话(第79页)中,资本实际上被说成是“关系”,被说成不仅是“积累的储备”,而且是完全确定的生产关系。“储备”不可能“担负起预付工资的职能”。琼斯着重指出,资本的基本形式,就是资本和雇佣劳动相对立,并支付工资;资本的这种形式赋予整个社会生产过程一种特征,支配它,使社会劳动生产力达到全新的发展,同时使一切社会的和政治的关系革命化。琼斯着重指出,资本在担负起这个具有决定意义的职能以前,执行了其他的职能,表现为其他一些从属的但在历史上是较早的形式,但是只有随着它作为产业资本出现,它的力量在它的一切职能中才得到充分的发挥。另一方面,在他的第三篇讲义《论资本或者资本家{在这里,问题就在这个“或者”上:只是由于这种人格化,积累的储备才成为资本}如何逐渐地担负起财富生产中的一连串职能》中,琼斯并没有告诉我们,这些较早的职能是什么。实际上,这只能是商业资本或货币经营资本的职能。但是,尽管琼斯如此接近了正确的观点,并在一定形式上说出了这种观点,另一方面,他作为政治经济学家,却仍然在很大程度上为资产阶级拜物教所束缚,以致连魔鬼也不能担保,他是否理解“积累的储备”本身所能完成的各种职能。

琼斯说:

\begin{quote}{“资本,或者说,积累的储备,只是后来,当它在财富的生产中执行了其他各种职能以后,才担负起向劳动者预付工资的职能。”(第79页)}\end{quote}

这句话充分表现出这样一种矛盾:一方面,琼斯对资本有正确的历史的理解,另一方面,这种理解又被政治经济学家所固有的狭隘见解即“储备”本身就是“资本”弄得模糊不清。因此,“积累的储备”在琼斯那里成为一个“担负起”向劳动者“预付工资的职能”的人。琼斯在破除政治经济学家所固有的偏见时自己还是为这种偏见所束缚,——既然资本主义生产方式被看作历史上一定的生产方式,而不再是生产的永恒的自然关系,那末破除这种偏见就是必要的。

我们看到,从拉姆赛到琼斯有了多么大的飞跃。正是资本的那个使资本成为资本的职能——预付工资——被拉姆赛说成是偶然的,只是由大多数人的贫困引起的,对于生产过程本身是无关紧要的。拉姆赛用这种狭隘的形式否认资本主义生产方式的必然性。而琼斯{奇怪的是,他们两人都是英国国教会的牧师\endnote{这里提到的两个英国经济学家当中,只有琼斯是牧师。——第472页。}。看来英国的牧师比大陆上的还是考虑得多些}指出,正是这个职能使资本成为资本,并且决定资本主义生产方式的特征。琼斯指出,这个形式怎样在生产力发展的一定阶段才会发生,并且那时就会创造出全新的物质基础。但是琼斯因此也以比拉姆赛更深刻的完全不同的方法理解这个形式的“可废止性”,理解它的仅仅是历史上暂时的必然性。琼斯决没有把资本主义的关系看作永恒的关系。

\begin{quote}{“将来可能出现这样一种情况,——世界各大洲可能会逐渐接近这样一种情况,——在这种情况下,劳动者和积累的储备的所有者将是同一的;但是在各国的发展中……这种情况至今还从未有过,为了探索和理解这种发展,我们必须考察劳动者怎样逐渐地从用自己的收入支付劳动者报酬的主顾的支配下,转到用资本(它的所有者指望从它的总产品中为自己实现一种特殊的收入)的预付支付劳动者报酬的企业主的支配下。也许,这种情况同劳动者和资本家是同一个人的情况相比,还不是那么令人满意;但是我们必须仍然把它看作生产发展进程中的一定阶段,这个阶段直到现在还是先进国家的发展的特征。亚洲的居民还没有达到这个阶段。”(第73页)}\end{quote}

[1139]在这里琼斯直截了当地宣称,他把资本和资本主义生产方式只“看作”社会生产发展中的一个过渡阶段,从社会劳动生产力的发展来看,这个阶段同一切过去的形式相比是一个巨大的进步,但是这个阶段决不是最终的结果,而是相反,在它固有的对抗形式中,即在“积累的财富的所有者”和“实际的劳动者”之间的对抗形式中,包含着它灭亡的必然性。

琼斯曾在海利贝里任政治经济学教授,是马尔萨斯的继任者。在这里我们看到,政治经济学这门实际科学是怎样结束的:资产阶级生产关系被看作仅仅是历史的关系,它们将导致更高级的关系,在那里,那种成为资产阶级生产关系的基础的对抗就会消失。政治经济学以自己的分析破坏了财富借以表现的那些表面上相互独立的形式。它的分析(甚至在李嘉图那里就已经)进行得如此远了:

(1)财富的独立的物质形式趋于消灭,财富不过表现为人的活动。凡不是人的活动的结果,不是劳动的结果的东西,都是自然,而作为自然,就不是社会的财富。财物世界的幻影消逝了,这个世界不过表现为不断消失又不断重新产生的人类劳动的客体化。任何物质上持久的财富都只是这个社会劳动的转瞬即逝的物化,是生产过程的结晶化,生产过程的尺度是时间,即运动本身的尺度。

(2)财富的不同组成部分,通过各种各样的形式流入社会的不同部分,这些形式正在丧失自己的表面的独立性。利息,仅仅是利润的一部分,地租,仅仅是超额利润。因此,不论利息还是地租都溶解在利润里面,而利润本身则归结为剩余价值即无酬劳动。但是商品价值本身只归结为劳动时间。李嘉图学派甚至走得这样远,以至把这个剩余价值的占有形式之一——土地所有权(地租)——当作无用的形式加以否定,只要得到它的是私人[而不是国家]。李嘉图学派不承认土地所有者是资本主义生产的职能执行者。这样,对抗就归结为资本家和雇佣工人之间的对抗。但是李嘉图学派的政治经济学把资本家和雇佣工人之间的这种关系看作某种既定的东西,看作生产过程本身所依据的自然规律。后来的经济学家,象琼斯这样的经济学家,超过了这一点,他们只承认这种关系的历史的合理性。但是,自从资产阶级生产方式以及与它相适应的生产关系和分配关系被认为是历史的以来,那种把资产阶级生产方式看作生产的自然规律的谬论就宣告破产了,并且开辟了新社会的远景,开辟了新的经济社会形态的远景,而资产阶级生产方式只构成向这个形态的过渡。\endnote{在手稿中接下去是《资本论》第三部分或第三篇的计划草稿——《资本和利润》,作为插入部分放在方括号内。本版把这个计划收入本卷第1册《附录》(第447页)。——第474页。}[1139]

[1139]关于琼斯,我们还要考察几个问题:

(1)资本主义生产方式——由资本预付工资——究竟怎样改变[生产]形式和生产力。(2)琼斯关于积累和利润率的论断。

但是这里首先还要指出下面一点:

\begin{quote}{[1140]“资本家只是一种中介人,他使劳动者在新的形式和新的情况下从周围的主顾所支出的收入中得到利益。”(第79页)}\end{quote}

这里说的是过去直接靠土地所有者等等的收入生活的非农业劳动者。现在不是这些劳动者用自己的劳动(或自己劳动的产品)直接同这种收入交换,而是资本家用这些劳动者的劳动产品——收集和集中在他手中的——同这种收入交换,或者说,收入转化为资本,同资本交换,同时构成资本的收益。这种收入现在不是构成劳动的直接收益,而是构成使用工人的资本的直接收益。\endnote{在手稿(第1140—1144页)中接下去是:《资本论》第一部分或第一篇的计划草稿——《资本的生产过程》,本版把这个计划收入本卷第1册《附录》(第446页),再往下是从报刊杂志和书籍上摘录的关于利率的高度、资本家对工人阶级的剥削、不变资本和可变资本之间的各种比例等问题的材料。马克思在《资本论》第一卷和第三卷中引用了这里的某些摘录。第1142页上有关庸俗政治经济学把利润看作资本家的“工资”的辩护论观点的一小段话,收入本册《附录》(第553页)。——第474页。}[1140]

[1144]琼斯把资本作为特殊的生产关系来描述,认为这种生产关系的主要特征是:积累的财富表现为预付的工资,“劳动基金”本身则表现为“由收入中积蓄起来并用来获取利润的财富”,然后,他就从生产力的发展中考察这一生产方式所特有的变化。琼斯很好地论述了,怎样随着物质生产力的变化,经济关系以及与此相连的国民的社会状况、道德状况和政治状况,也都在发生变化:

\begin{quote}{“随着各社会改变自己的生产力,它们也必然改变自己的习俗。社会上所有各个不同的阶级在其发展进程中都会发觉,新的关系已把它们同其他阶级联系起来,它们处在新的地位,并被新的道德的和社会的危险所包围,被社会进步和政治进步的新条件所包围。”(第48页)}\end{quote}

在考察琼斯怎样说明资本主义生产形式对生产力发展的影响之前,还要引几段同我们上面所引的有联系的话。

\begin{quote}{“随着社会的经济组织以及生产任务借以完成的因素和手段(丰富的或贫乏的)的变化,会发生大的政治的、社会的、道德的和精神的变化。这些变化发生在居民当中,必然对居民的各种政治要素和社会要素产生决定性的影响;这种影响将涉及国民的精神面貌、习惯、风俗、道德和幸福。”(第45页)“英国是这样一个唯一的大国,它……作为一个生产的机构,在向着完善前进的过程中迈出了第一步;只有这个国家,它的居民,不论是农业的还是非农业的,都受资本家的指挥,在这个国家里,不仅在它的财富的巨大增长中,而且在它的居民的一切经济关系和地位中,处处可以感觉到资本家所拥有的手段和唯有资本家才能执行的那些特殊职能所产生的影响。但是英国——我遗憾地但毫不犹豫地说——对于用这种方法发展自己的生产力的人民的经历[1145]来说,决不是一个成功的范例。”(第48—49页)“一般劳动基金由以下几部分构成:(1)劳动者自己生产的工资;(2)其他阶级用于维持劳动的收入;(3)资本,或者说,由收入中积蓄起来并用来预付工资以便获取利润的财富。我们把靠第一部分劳动基金维持生活的人叫作非雇佣劳动者。靠第二部分劳动基金维持生活的人叫作领薪金的服务人员。靠第三部分劳动基金维持生活的人叫作雇佣工人。从这三部分劳动基金中的哪一部分领取工资,这决定劳动者和社会其他阶级的相互关系,并从而决定——有时直接地,有时或多或少间接地——完成生产任务的连续性、技能和力量的程度。”(第51—52页)“世界上的劳动人口有半数以上,也许甚至三分之二以上,是靠第一部分劳动基金,即劳动者自己生产的工资维持生活的。这些劳动者到处都由占有并耕种土地的农民构成……第二部分劳动基金,即用于维持劳动的收入,维持着东方绝大部分非农业生产劳动者。这一部分劳动基金在欧洲大陆有一定的重要性,但在英国它只包括人数不多的做零工的手工业者,他们是一个人数众多的阶级的残余……第三部分劳动基金,即资本,在英国雇用了大多数劳动者,然而在亚洲它只维持着不多的人,在欧洲大陆这部分基金只维持着非农业劳动者,他们总共也许不到全部生产人口的四分之一。”(第52页)“我没有把奴隶劳动作为一个特别的范畴……劳动者的公民权对于他们的经济地位不发生影响。可以看到,奴隶象自由民一样,靠某种形式的劳动基金维持生活。”(第53页)}\end{quote}

但是,如果说劳动者的“公民权”对于“他们的经济地位”不发生影响,那末他们的经济地位对于他们的公民权却发生影响。只有在工人有人身自由的地方,国家范围内的雇佣劳动,从而还有资本主义生产方式,才是可能的。它是建立在工人的人身自由之上的。

琼斯正确地把斯密的生产劳动和非生产劳动还原为它们的本质,即还原为资本主义劳动和非资本主义劳动,因为他正确地运用了斯密关于由资本支付的劳动者和由收入支付的劳动者的区分。但是琼斯自己把生产劳动和非生产劳动显然理解为加入物质[财富]生产的劳动和不加入这种生产的劳动。这是根据[上面引用过的]那段话[第52页]得出的,琼斯在那里谈到依靠别人花费的收入维持生活的生产劳动者。还根据以下的话:

\begin{quote}{“社会上不生产物质财富的那一部分,可能是有用的,也可能是无用的。”(第42页)“我们可以并且有理由认为,在所生产的商品没有到达要消费它的人手中以前,生产行为并没有结束。”(第35页注)}\end{quote}

靠资本生活的劳动者和靠收入生活的劳动者之间的区别,同劳动的形式有关。资本主义生产方式和非资本主义生产方式的全部区别就在这里。相反,如果从较狭窄的意义上来理解生产劳动者和非生产劳动者,那末生产劳动就是一切加入商品生产的劳动(这里所说的生产,包括商品从首要生产者到消费者所必须经过的一切行为),不管这个劳动是体力劳动还是非体力劳动(科学方面的劳动);而非生产劳动就是不加入商品生产的劳动,是不以生产商品为目的的劳动。这种区分决不可忽视,而这样一种情况,即其他一切种类的活动都对物质生产发生影响,物质生产也对其他一切种类的活动发生影响,——也丝毫不能改变这种区分的必要性。

\tsectionnonum{[(b)琼斯论资本主义生产形式对生产力发展的影响。关于追加固定资本的使用条件问题]}

[1146]现在我们来谈资本主义生产方式影响下的生产力的发展问题。

[琼斯说:]

\begin{quote}{“这里最好指出这样一点,即这个事实{由资本预付工资}怎样影响劳动者的生产力,或者说,怎样影响劳动的连续性、知识和力量……向工人支付工资的资本家,能够促进工人劳动的连续性。第一,他使这种连续性成为可能,第二,他对此进行监督和强制。世界上有人数很多的各种各样的劳动者,他们经常徘徊街头,寻找主顾,他们的工资取决于人们的偶然需要,就是说有人恰好在这个时候需要他们的服务,或者需要他们所制造的物品。最早的传教士在中国看到过这样的情况:‘那里的手工业者从早到晚在城里到处奔走,寻找主顾。大部分中国工人都是在私人家里劳动。例如,你需要衣服吗?裁缝便从早上到你家里来,到晚上才回家。其他一切手工业者的情况也是这样。他们经常为了寻找工作而走街串巷,甚至铁匠也担着他的锤子和炉子沿街寻找普通的零活。理发匠也是……肩上扛着靠椅,手里提着盆子和烧热水的小炉子走街串巷。’\endnote{琼斯在这里引用的是重农学派的月刊《公民历书》(《EphemeridesduCitoyen》)1767年第三卷第56页。——第477页。}这种情况至今在整个东方仍然是常见的现象,在西方世界也有一部分是这样。所以说,这种劳动者不可能在任何长时间内连续地劳动。他们必须象出租马车那样在街头招揽主顾,如果找不到主顾,他们就不得不闲起来。如果经过一段时间他们的经济地位发生了变化,并且成了资本家的工人,由资本家预付给他们工资,那就会产生两种结果:第一,他们现在能够连续地劳动;第二,出现了这样一种代理人,他的职能和利益就是迫使工人真正连续地劳动……资本家……所拥有的资财允许他等待主顾……因此,所有这类人的劳动就有了更大的连续性。他们每天从早到晚地劳动,他们的劳动不致因为等待或寻找那个必须消费他们所制造的物品的主顾而中断。但是,因此就成为可能的、工人劳动的连续性由于资本家的监督而得到了保障和增加。他预付他们的工资,他应当得到他们劳动的产品。他的利益和他的特权就是留心监视,不让他们工作中断或懈怠。既然劳动的连续性因此得到了保障,那末单是这个变化对劳动生产力的影响就非常大……生产力增加一倍。两个从早到晚连续劳动一年的工人所生产的东西,可能多于四个没有固定工作的工人所生产的,因为后者要把很多时间消耗在寻求主顾和恢复中断了的工作上面。”(第37—38页)}\end{quote}

[关于琼斯在这里所说的,必须指出:]

第一,关于从做临时活(如在土地所有者家中缝衣等等)的劳动者转变成受资本雇用的工人的事,杜尔哥已经作了很好的阐述。

第二,劳动的这种连续性虽然把资本主义劳动和琼斯所描述的劳动形式很好地区别开来,但是没有把它和大规模的奴隶劳动区别开来。

第三,把由于劳动持续时间的增加和工作中断现象的消除而引起的劳动本身的增加,叫作劳动生产力的增加是不正确的。劳动生产力的增加,只有在劳动的连续性提高工人个人技能的限度内才会发生。我们所理解的劳动生产力[增加],是指使用一定量劳动时具有更大的效率,而不是指使用的劳动的量的任何变化。琼斯所说的那种情况,宁可说是劳动对资本的形式上的隶属,这种情况只有随着固定资本的发展才获得充分的发展。(关于这一点,马上就要谈。)

琼斯正确地着重指出:资本家把劳动视为自己的财产,丝毫也不让它白白耗费。至于直接依赖收入的劳动,那末说的只是劳动的使用价值。

[1147]琼斯继而完全正确地指出,非农业工人从早到晚继续不断地埋头劳动,决不是天然如此,这种劳动本身是经济发展的产物。中世纪的城市劳动,与亚洲的劳动形式和西方的农村劳动形式(以前占统治地位,现在还部分地可以看到)不同,它已经前进了一大步,并且对于资本主义生产方式,对于劳动的连续性和经常性来说,是一所预备学校。

{关于劳动的这种连续性,在1821年伦敦出版的匿名小册子《论马尔萨斯先生近来提倡的关于需求的性质和消费的必要性的原理》中写道:

\begin{quote}{“资本家好象还掌管着一个劳动介绍所;他保险劳动不会没有把握找到销路。如果没有资本家,这种没有把握的事,就会使劳动在很多情况下得不到雇用。由于他的资本,那些寻找买者和奔跑市场的麻烦事就比较少了。”(第102页)}\end{quote}

在这本小册子里我们还读到:

\begin{quote}{“在资本在很大程度上由固定资本构成的地方,或者在资本投于土地的地方……企业主在更大得多的程度上(和使用较少固定资本时相比)不得不继续使用和过去几乎同样多的流动资本,以便不致于失去固定资本部分的任何利润。”(第73页}}\end{quote}

{[琼斯还说:]

\begin{quote}{“关于在中国由于劳动者依赖他们的主顾的收入而造成的状况,你大概可以在一个由美国人举办的中国展览会上看到一幅极其引人注目的图景,这个展览会在伦敦展出了很久。展览会充满了对手工业者的描绘,他们携带着自己的一套简单工具到处寻找主顾,如果找不到主顾,就得闲起来。这里明显地呈现出,在他们的情况下,必然没有作为劳动生产率的三大要素之一的劳动的连续性;任何一个有见识的观众都能看出,这里也缺乏固定资本和机器,它们不见得是劳动生产率的次要的要素。”(第73页)“类似的情景在印度的城市也能看到,欧洲人的出现并没有改变这种状况。不过,在印度的农村地区,手工业者是靠特别的方法维持的……确实为某个村庄所需要的手工业者和其他非农业劳动者,靠这个村庄居民的公共收入的一部分来维持生活。在全国范围内有一大批世代相传的劳动者靠这种基金生活,他们的劳动满足了农业劳动者用自己的劳动满足不了的简单的需要和嗜好。这些农村手工业者的地位和权利,象东方的一切权利一样,很快就成为世代继承的了。手工业者在别的农村居民那里找到主顾。农村居民是定居的、不变动的,为他们服务的手工业者也是这样……城市手工业者过去和现在都是处于完全不同的地位。他们实质上是从同一个基金——土地的多余收入——获得自己的工资,但是在这里,基金的分配方法以及分配者是不同的,因此,手工业者不再能是永久定居的了,他们不得不进行频繁的和往往是灾难性的迁移……这样的手工业者不会由于依赖大量的固定资本而被限定在一个地方。(例如,在欧洲,棉纺织业和其他企业被限定在富有水力,或者富有生产蒸汽的燃料的地方,欧洲已有大量的财富转化为建筑物、机器等等。)……如果劳动者完全[1148]依靠从那些消费他所生产的商品的人的收入中直接领取一部分来维持生活,那末情况就不同了……这种劳动者不会被限定在有任何固定资本的地方。如果他们的主顾在较长时期内,有时甚至在短时期内,迁移了自己的住地,非农业劳动者为了不致饿死,就不得不跟随他们一起去。”(第73—74页)“为手工业者预备的这种基金的大部分,在亚洲由国家及其官吏来分配。分配的主要中心自然是首都。”(第75页)“由撒马尔汗往南到比贾普尔和塞林格帕塔姆,我们可以看到一些消失了的首都的遗迹,只要国王的收入(也就是土地的全部多余收入)的新的分配中心一形成,这些首都就被它们的居民突然舍弃了(而不象在其他国家那样是由于逐渐的衰落)。”(第76页)}\end{quote}

请看一看贝尔尼埃博士的书\endnote{马克思指的是法国医生和旅行家弗朗斯瓦·贝尔尼埃的书《大莫卧儿等国游记》,1670—1671年在巴黎初次出版,后来曾多次再版。马克思在1853年6月2日致恩格斯的信中从贝尔尼埃的这本书中引了很长的两段,其中包括把印度的城市比作军营那句话(见《马克思恩格斯全集》中文版第28卷第256页)。——第480页。},他把印度的城市比作军营。可见,这是以亚洲的土地所有制形式为基础的。}

\centerbox{※     ※     ※}

现在我们要从劳动的连续性转到分工、知识的发展、机器的使用等等。

[琼斯写道:]

\begin{quote}{“支付劳动者的雇主的变动对劳动的连续性产生的影响,决不限于以上所说。现在可以对生产上的各种工作作进一步的划分……如果他〈资本家〉使用的不是一个人,而是几个人,那末他就能在他们当中分配工作;他就能使每个工人固定地去完成整个工作中他完成得最好的那一部分……如果资本家是富有的,并且雇用了足够数量的工人,那末,只要工作还能细分,它就会尽量细分下去。这时,劳动的连续性也就达到了完善的程度……资本取得了预付工资的职能以后,现在已逐步地使劳动的连续性趋于完善。同时,资本也使这种劳动为产生一定效果而应用的知识和技能增加了。资本家阶级最初部分地摆脱了体力劳动的必要性,最后完全摆脱了体力劳动的必要性。他们的利益要求他们使用的工人的生产力尽可能地大。所以他们的注意力放在,而且几乎完全放在这种力量的增加上面。思想越来越集中于寻找最好的手段以达到人类劳动的一切目的;知识扩大了,增大了它的应用范围,并且几乎在所有生产部门中协助了劳动……但是我们再往下看一看机械力。不是用来支付劳动,而是用来协助劳动的资本,我们要称为辅助资本。”}\end{quote}

{可见,琼斯所理解的“辅助资本”,是不变资本中那个不是由原料构成的部分。}

\begin{quote}{“一个国家的辅助资本量,在具备一定条件时,能够无限地增加,即使工人人数保持不变。在这方面每前进一步,人类劳动效率的第三个要素,即它的机械力,都会增大……因而,辅助资本量同人口相比会增加……必须具备什么样的条件,才能使用于协助他们{资本家雇用的工人}的辅助资本量增加呢?必须同时具备三个条件:(1)积蓄追加资本的手段;(2)积蓄追加资本的愿望;(3)某种发明,由此有可能通过使用辅助资本来提高劳动生产力,而且提高到这样的程度,以致劳动在它以前生产的财富之外,还把使用的追加辅助资本按其消费的程度,连同其利润再生产出来……如果在现有的知识状况下能有利地加以使用的整个辅助资本量已经具备……那末只有知识水平的提高才能指出使用更大量资本的手段。其次,这种使用只在如下场合才有可能,即所发明的手段要把劳动力提高到能够把追加资本在它被消耗期间再生产出来。如果不是这样,资本家就定会损失自己的财富……但是,除此以外,工人的提高了的劳动效率应当还能生产一些利润,否则资本家把自己的资本用于生产的动机就完全没有了……只要通过使用新的辅助资本量能够达到这两个目的,对于进一步使用这种新的资本量就不会有固定的和最终的界限。资本的增长能够和知识的增长一起前进。但是知识永远不会停滞不前。由于知识每时每刻都在各个方面向前发展,所以每时每刻都能出现新工具、新机器、新动力,这就使社会能够有利可图地追加一些协助劳动的辅助资本量,并以此来扩大它的劳动生产率同那些较贫穷的、技能较差的国家的劳动生产率之间的差别。”(同上[第38—41页])}\end{quote}

[1149]首先,我们来看看琼斯的意见,他认为新的发明、装置或设备必须能够“把劳动生产力提高到这样的程度,以致劳动在它以前生产的财富之外,还把使用的追加辅助资本按其消费的程度再生产出来”,或者说,使劳动“把追加资本在它被消耗期间再生产出来”。可见,这仅仅意味着,磨损是按照磨损的程度得到补偿的,或者说,追加资本在它被消费期间平均得到补偿。产品价值的一部分——或者也可以说,产品的一部分——必须补偿已消费的“辅助资本”,而且必须在这样的期限内补偿,也就是说,在它完全被消费掉时,它就能完全被再生产出来,或者说,同一种新资本就能代替已消费的资本。但是做到这一点的条件是什么呢?劳动生产率必须由于使用追加的“辅助资本”而提高到这样的程度,以致产品的一部分能够分出来,或者以实物形式,或者通过交换,来补偿这个组成部分。

如果劳动生产率有这样高,也就是说,如果在同样一个工作日生产出的产品量增加了这样多,以致单位商品比原来生产过程中的单位商品便宜,即使这时商品总额要用自己的总价格来抵补机器的(比如说)年磨损,但摊在单位商品上的相应的磨损部分极小,那末“辅助资本”也会被再生产出来。如果我们从总产品中扣除,第一,补偿磨损的部分,第二,补偿原料价值的部分,那末剩下的就是支付工资的部分,以及抵补利润的部分,这个部分甚至会在单位商品价格不变的情况下提供更多的剩余价值。

不具备这种条件,产品也有可能增加。例如,如果棉纱的磅数[仅仅]增加到10倍(而不是100倍,等等),而摊到单位产品价值上的补偿机器磨损的附加额,从1/6减少到1/10,那末用机器生产的棉纱就会比用纺车生产的棉纱贵。\endnote{马克思在这里考察的是使用新固定资本的盈利性问题。只有在补偿磨损的补充费用,因产品数量增多引起单位产品成本降低而得到补偿的情况下,资本家才会使用追加固定资本。马克思的意思可以用下面的例子来说明。假定用手工纺纱生产的10磅棉纱总价值为10镑,其中8镑用在原料上,2镑用在劳动力上(马克思在这里撇开利润不谈)。因此,用手工纺的1磅棉纱的价值等于1镑。又假定,由于使用了机器,所生产的棉纱数量增加到100倍(是1000磅而不是10磅),原料的花费同样也增加到100倍,而劳动力的花费增加较少,例如增加到10倍。在这种情况下,1000磅棉纱的价值等于800镑(原料的花费)+20镑(劳动力的花费)+164镑(按马克思的假设,固定资本的损耗在这里是棉纱价值的1/6),也就是说等于984镑。在这种情况下,1磅棉纱的价值大致是9/10镑,也就是说,同手工纺纱相比棉纱落价了。这表明在这里机器的使用是有利的。假如棉纱数量只增加到10倍(是100磅而不是10磅),那末棉纱的价值等于80镑(原料的花费)+12镑(劳动力的花费假定增加到6倍)+10+(2/9)镑(按马克思的假设,固定资本的损耗量现在减少到棉纱价值的1/10),也就是说,等于102+(2/9)镑。这样,1磅棉纱的价值就超过了1镑。这表明:尽管用以补偿固定资本损耗的费用相对减少了(从1/6减少到1/10),在这种情况下机器生产的棉纱还是比手工生产的棉纱贵。因此,在这里机器的使用对资本家来说是不利的。——第483页。}如果用100镑追加资本购买鸟粪投在农业上,如果这些鸟粪必须在一年内被补偿,而一夸特产品的价值(在旧的生产方法下)等于2镑,那末,仅仅为了补偿损耗\endnote{马克思在《资本论》第二卷中指出,为改良土壤而投下的物质的一部分,在较长的时期内“继续作为生产资料存在,因而取得固定资本的形式”(见《马克思恩格斯全集》中文版第24卷第179页)。马克思正是在这个意义上在正文中谈到对投在土地上的鸟粪的“补偿损耗”问题。——第483页。},就必须生产50追加夸特。否则这笔追加资本就不会被使用(这里我们撇开利润不谈)。

琼斯认为,追加资本必须“在它被消耗期间再生产出来”(当然,通过出卖产品或者以实物形式),他的这种意见仅仅意味着,商品必须补偿它所包含的损耗。为了重新开始再生产,商品所包含的一切价值要素,都必须在商品的再生产重新开始时就得到补偿。在农业上这种再生产时间是由自然条件决定的,而在什么时间内必须补偿损耗,也完全和在什么时间内必须补偿比如说谷物的其他一切价值要素一样,在这里是决定了的。

为了使再生产过程能够开始,也就是说,为了使本来的生产过程能够得到更新,必须经过流通过程,即商品必须出卖(除非它是以实物形式自己补偿自己,就象种子那样),而商品卖得的货币必须重新转化为生产要素。就谷物和其他农产品来说,对于这种再生产,存在着一定的、由四季更替所规定的期限,因此,对于流通过程的持续时间,也存在着最终的界限,即肯定的界限。[这是第一点。]

第二,流通过程的这种肯定的界限,一般来自作为使用价值的商品的性质。所有的商品都会在一定的时间内变坏,尽管它们存在的ultimaThule\fnote{极限,极点,最终之物,最终目的(直译是:极北的休里——古代人想象中的欧洲极北部的一个岛国)。——编者注}各不相同。如果人不消费它们(为了生产或者为了个人消费),天然的自然力就会消费它们。它们会逐渐变质,最后完全毁坏。如果商品的使用价值失去了,它的交换价值也就见鬼去了,它的再生产也就停止了。因此,商品流通时间的最终的界限决定于作为使用价值的商品所固有的再生产时间的自然期限。

第三,为了使商品的生产过程连续不断,也就是说,为了使资本的一[1150]部分不间断地处在生产过程中,另一部分不间断地处在流通过程中,就必须按照再生产时间的自然界限,按照各种不同使用价值存在的界限,或者按照资本的各种不同的作用领域,对资本进行极不相同的划分。

第四,上述一切同时适用于商品的所有价值要素。但是,对于那些有很多固定资本参与生产的商品来说,除了由商品本身的使用价值给流通时间规定的界限外,固定资本的使用价值也具有决定的作用。固定资本在一定时间内被损耗,因此,它必须在一定期间再生产出来。比方说,一只船在10年内用坏,或者一架纺纱机在12年内用坏。在这10年内所获得的运费,或者在这12年内卖的纱,必须足以在10年后用一只新船来代替旧船,或者在12年后用一架新纺纱机来代替旧纺纱机。如果固定资本在半年内消费掉,产品就必须在半年内从流通中返回。

因此,除了作为使用价值的商品的自然毁灭期限(这个期限对于不同的使用价值是极不相同的),除了生产过程连续性的要求(由于这种要求,根据商品必须在生产领域停留时间的长短,根据商品能够在流通领域停留时间的长短,又有流通时间的各种不同的最终界限),还要加上第三点,即加入商品生产的“辅助资本”的各种不同的毁灭期限以及由此引起的再生产的必要性。

琼斯认为[使用“辅助资本”的]第二个条件是“辅助资本”必须“生产”的“利润”,而这是任何资本主义生产的必要条件,不论使用的资本有怎样特殊的形式规定性。其实,琼斯在任何地方都没有向我们说明,他对这种利润的产生是怎样理解的。但是,因为他只从“劳动”中引出这种利润,只从提高了的工人劳动效率中引出“辅助资本”所提供的利润,所以在琼斯那里,任何利润都必然归结为绝对的或相对的剩余劳动。一般说利润是这样产生的:产品的一部分以实物形式或通过交换去补偿资本中那些由原料和劳动资料构成的部分,资本家在扣除这一部分产品之后,第一,由余下的产品部分中支付工资,第二,把一部分产品作为剩余产品占为己有,他出卖这部分产品,或者以实物形式消费它。(后一种情况,在资本主义生产条件下不必考虑,只是少数直接生产必要生活资料的资本家除外。)可是这个剩余产品正象产品的其他部分一样,是工人的物化劳动,不过是无酬劳动,是资本家不付等价而占有的劳动产品。

在琼斯对问题的论述中,有一点是新的,即他指出“辅助资本”在一定限度以上的增加取决于知识的增加。琼斯说,要使“辅助资本”增加,必须有:(1)积蓄追加资本的手段,(2)积蓄追加资本的愿望,(3)某种发明,由此有可能把劳动生产力提高到这样的程度,以致能够再生产出追加资本,并且生产出追加资本的利润。

在这里首先必需的是剩余产品的存在,不管它是以实物形式存在,还是已经转化为货币。

以棉花生产为例,曾经有过这样的时候,那时在美国(如同现在在印度)种植场主能够种植大面积的棉花,但是他们没有办法通过清棉及时把子棉变成棉纤维。一部分长好的棉花便在地里腐烂。轧棉机的发明结束了这种情况。现在一部分产品转化为轧棉机,但是轧棉机不仅能补偿自己的费用,而且能增加剩余产品。新市场的出现也有同样的作用,例如,它能把皮革转化为货币。(运输工具的改良也有同样的作用。)

每一种消费煤的新机器,都是一种把以煤的形式存在的剩余产品转化为资本的手段。把一部分剩余产品转化为“辅助资本”,可以通过两种方式:[第一,]通过现有“辅助资本”的增加,也就是它的扩大规模的再生产;[第二,]通过新使用价值的发现,或者通过旧使用价值的新应用,以及通过新机器或动力的发明,从而创造出新的种类的“辅助资本”。在这里,知识的扩大当然是“辅助资本”增加的条件之一,或者同样可以说,是剩余产品或剩余货币转化为(在这里对外贸易具有重要意义)追加的“辅助资本”的条件之一。例如,电报的发明为投入“辅助资本”开辟了完全新的范围,铁路等等也是这样,古塔波胶或印度胶的整个生产也是这样。

[1151]关于知识的扩大这一点是重要的。

积累完全不一定要直接推动新劳动,它可以仅限于给旧劳动提供新方向。例如,同一个机械厂,过去生产[手工]织布机,现在制造机械织布机,一部分[手工]织布工人转到这个改变了的生产上来,其余部分则被抛弃街头。

当一种机器代替劳动的时候,它(为了它本身的生产)所需要的新劳动不管怎样都少于它所代替的劳动。也许仅仅给旧劳动提供新方向。不管怎样,都会有劳动游离出来,这种劳动经过或多或少的流浪和苦难之后可能被用到其他的方向。这样就为新生产领域提供了人身材料。至于资本的直接游离,这里游离出来的不是购买机器的资本,因为它就是投到机器上的。即使假定,机器比被它排挤的工人的工资便宜,那也会需要更多的原料等等。如果被解雇的工人一年花费500镑,新机器也值500镑,那末资本家以前每年都必须花费500镑,而现在机器也许能用10年,资本家实际上每年只花费50镑。但是不管怎样,游离出来的(扣除在机器生产及其辅助材料例如煤的生产上使用的追加工人的费用以后)或者是曾构成[被解雇]工人的收入的资本,或者是工人曾用自己的工资与之交换的资本。这种资本依然存在。如果工人仅仅作为动力被代替,而机器本身没有什么显著的变化,举例说,如果现在机器用水或风推动,而过去是用工人推动,那末就有双重资本游离出来:一种是以前用于支付工人的资本,一种是工人曾用自己的货币收入与之交换的资本。这样的例子李嘉图已经用过\fnote{见本卷第2册第630—633页。——编者注}。

但是,一部分以前转化为工资的产品,现在总是被作为“辅助资本”再生产出来。

一大部分以前直接用于生活资料生产的劳动,现在用于“辅助资本”的生产。这和亚·斯密的观点也是相矛盾的,按照斯密的观点,资本的积累等于使用要多的生产劳动。除开上面所说的,这里发生的只能是劳动使用的改变,以及劳动由直接生产生活资料转移到生产生产资料,即铁路,桥梁、机器、运河等等。

\centerbox{※     ※     ※}

{现有的生产资料量和现有的生产规模对于积累是多么重要,这从下面一段引文可以看出:

\begin{quote}{“在郎卡郡能用如此惊人的速度建造起一座包括纺纱间和织布间的大棉纺织厂,这是由于,在工程师、设计师、机器制造者那里大量搜集有各种模型,从巨大的蒸汽机、水车、铁梁、铁柱,直到翼锭精纺机或织机的最小零件。在最近一年内,费尔贝恩先生在他的一个机器制造厂中(不依赖他的大的机器制造厂和蒸汽锅炉制造厂)就建造了700马力的水车和400马力的蒸汽机。每当增大的商品需求吸引新资本的时候,有利地使用新资本的手段就如此迅速地制造出来,以致在法国、比利时或德国的同类工厂能够开工以前,新资本就能实现同它自身价值相等的利润。”(安·尤尔《工厂哲学》1836年巴黎版第1卷第61—62页)}\end{quote}

[1152]工业的发展导致机器降价,部分是相对降价(同机器的功率相比),部分是绝对降价;但同时与此相联的是在一个工厂里集中有大量的机器,因此机器设备的价值同被使用的活劳动相比增大了,虽然它的个别组成部分的价值减少了。

动力——生产动力的机器——随着动力传送机械和工作机的改进,即随着磨擦力的减少等等而逐渐降价。

\begin{quote}{“使用自动工具所带来的优越性,不仅改进了工厂的机器设备的精度,加速了它的制造,而且在很大程度上降低了它的价格,增大了它的灵便性。现在可以买到最好的翼锭精纺机,每枚纱锭9先令6便士;自动走锭精纺机也可以买到,每枚纱锭约8先令,包括它的专利税在内。棉纺织厂的纱锭运行时磨擦很小,以致一马力就能带动精纺机的500枚纱锭,自动走锭精纺机的300枚纱锭,翼锭精纺机的180枚纱锭;这一马力还带动一切准备机器,即梳棉机、粗纺机等等。三马力足以带动30台大织机连同它们的浆纱机。”(安·尤尔《工厂哲学》1836年巴黎版第1卷第62—63页)}}\end{quote}

\centerbox{※     ※     ※}

[琼斯进一步指出:]

\begin{quote}{“在地球上绝大部分地区,劳动阶级的大多数还根本不是从资本家那里得到自己的工资;他们或者自己生产它,或者从自己的主顾的收入中得到它。在这里,保证他们的劳动连续性的第一个大步骤还没有完成。在劳动中帮助他们的,只是为了生计而用自己的双手劳动的人所能掌握的那种知识和那样一种数量的机械力。较发达的国家的技能和科学,巨大的动力,这种动力所能带动的积累的工具和机器,在那些仅有这种劳动者参加的劳动中是没有的。”(第43页)}\end{quote}

{甚至在英国:

\begin{quote}{“以农业为例……很好地经营农业所必需的知识,在全国传播得少而且不普遍。非常小的一部分农业人口享用着……能够在国民劳动的这个部门使用的全部资本……在我们的非农业劳动者中,只有很小一部分人在大工厂中工作。在农村作坊中,在那些通过小的组合完成自己的单项工作的手工业者和手艺人那里,分工是不充分的,因而劳动的连续性也是不完善的……走出大城市的圈子,看看国家的广阔原野,那就可以看到,国民劳动的很大一部分,无论在劳动的连续性方面,还是在劳动的技能和力量方面,都距离完善还很远很远。”(第44页)}}\end{quote}

资本主义生产的发展势必引起科学和劳动的分离,同时使科学本身被应用到物质生产上去。

\centerbox{※     ※     ※}

关于地租,琼斯正确地指出:

完全依赖于利润的现代意义上的地租的前提是:

\begin{quote}{“资本和劳动从一个生产部门转移到另一个生产部门的可能性……资本和劳动的灵活性,在农业资本和农业劳动没有这种灵活性的那些国家……我们根本不能指望看到在英国看到的那些纯粹由这种灵活性产生的结果。”(第59页)}\end{quote}

这种“资本和劳动的灵活性”,一般说来是形成一般利润率的现实前提。这种灵活性以劳动的确定形式无关紧要为前提。在这里,实际上发生了(靠损害工人阶级的利益)以下两种情况之间的磨擦:一方面,分工和机器赋予劳动能力以片面性,另一方面,这种劳动能力只作为任何一种劳动的现实的可能性和资本相对立{这就使资本同它在行会工业中的不发达形式有了区别},劳动投向这个方向还是投向另一个方向,要看在这个或那个生产领域里能获得什么样的利润,因此,各种不同的劳动量能够从一个领域转到另一个领域。

\begin{quote}{在亚洲等地,“主要的人口由劳动农民构成。他们所采用的落后的耕作制[1153]提供了长的闲暇时间。农民正如生产自己的食物一样……也生产大部分自己消费的其他生活必需品——自己的衣服,自己的工具,自己的家具,甚至自己的房屋,因为在这个阶级中只有很少的行业划分。这些人的风俗习惯是不变的;它们从父母传到子女;没有任何东西能改变或破坏它们”。(第97页)}\end{quote}

相反,资本主义生产的特征是,资本和劳动的灵活性,生产方式的不断变革,从而,生产关系、交往关系和生活方式等方面的不断变革,与此同时,在国民的风俗习惯和思想方式等等方面也出现了很大的灵活性。

让我们把刚才引用的关于“落后的耕作制”条件下的“闲暇时间”那段话,同下面两段话比较一下:

\begin{quote}{(1)“如果在农场使用蒸汽机,那它就会构成在农业中使用最多的工人的体系的一部分,并且不管怎样,马的数目必定会减少。”(《论农业中使用的动力》,约翰·查默斯·摩尔顿先生1859年12月7日在艺术和手工业协会所作的报告\endnote{艺术和手工业协会(SocietyofArtsandTrades)是资产阶级教育性质和慈善性质的团体,于1754年在伦敦成立。摩尔顿的报告发表在该协会的周刊《艺术协会杂志》(《JournaloftheSocieyofArts》)1859年12月9日那一期上。马克思引用的那段话在该期第56页上。——第490页。})(2)“生产农产品和生产其他劳动部门的产品所需要的时间是有差别的,这种差别就是农民具有很大依赖性的主要原因。他们不能在不满一年的时间内就把商品送到市场上去。在这整个期间内,他们不得不向鞋匠、裁缝、铁匠、马车制造匠以及其他各种生产者,赊购他们所需要的、可以在几天或几周内完成的各种产品。由于这种自然的情况,并且由于其他劳动部门的财富的增长比农业快得多,那些垄断了全国土地的土地所有者,尽管还垄断了立法权,但仍旧不能使他们自己和他们的奴仆即租地农民摆脱成为国内依赖性最强的人的命运。”(霍吉斯金《通俗政治经济学》1827年伦敦版第147页注)}\end{quote}

资本家和资本的区别在于,资本家必须生活,也就是说,必须每日每时把剩余价值的一部分作为收入来消费。因此,在资本家能把自己的商品运到市场以前,生产所经历的时间越长,或者说,他从市场得到出卖商品的收益所需要的时间越长,资本家就越是不得不在这一段时间靠借债生活(这一点我们在这里不必去考察),或者说,他就越是必须积累有更多的作为收入花费的货币储备。他就越是必须在较长的时间内向自己预付自己的收入。他的资本就必须越多。他不得不把自己的一部分资本经常闲置不用,以便作为消费基金。

{所以在小农业中,家庭工业是和农业结合在一起的;必须有一年的储备等等。}

\tsectionnonum{[(c)琼斯论积累和利润率。关于剩余价值的源泉问题]}

现在我们转到琼斯的积累学说。在此以前只是指出琼斯对积累的看法中有两个特点:第一,积累的源泉完全不一定是利润;第二,“辅助资本”的积累取决于知识的进步。琼斯把这种进步限于新的机器设备、动力等等的发明。但是这具有一般的意义。例如,如果把谷物用作制烧酒的原料,就会产生一个新的积累源泉,因为在这种情况下,剩余产品能被转化为新的形式,能用来满足新的需要,并且能作为生产要素进入新的生产领域。用谷物制造淀粉等等的情况也是如此。这些商品以及一切商品的交换领域因而扩大了。如果煤炭被用于照明等等,也会发生同样的情况。

当然,对外贸易——通过增加使用价值的多样化和商品量——也是积累过程中的重要因素。

琼斯这里所说的积累首先是指积累和利润率之间的关系(关于利润率的产生,他远远没有弄清楚):

\begin{quote}{“国家由利润积累资本的能力,不是随着利润率的变化而变化……相反,由利润积累资本的能力通常是按照同利润率相反的方向发生变化,即利润率低的地方,积累能力大,利润率高的地方,积累能力小。亚·斯密说:[1154]‘居民由利润得来的那部分收入,在富国总是比在贫国大得多,这是因为富国的资本大得多;但利润同资本相比,富国的利润通常又低得多’(《国富论》第2篇第3章)在英国和荷兰,利润率比欧洲其他任何地方都低。”(第21页)“在它的〈英国的〉财富和资本增长最快的时期,利润率逐渐下降。”(第21—22页)“所生产的利润的相对量……不是仅仅取决于利润率……而是取决于与使用的资本相对量结合起来考察的利润率。”(第22页)“较富国家的资本量的增长……通常还引起利润率下降,或者说,从使用的资本得到的年收入同这个资本总量之间的比例下降。”(同上)“如果有人说,在其他条件相等的情况下,由利润进行积累的能力决定于利润率,那末对这种说法的回答应当是,这种情形,即使实际上有可能发生,也是非常少见,因而不值得考虑。我们从观察中知道,利润率的下降是这样一种现象,它通常是由各国使用的资本量差额的增长引起的,因此,在较富国家中利润率下降时,所有其他条件并不是相等的。如果有人断言,利润可能下降得非常厉害,以至完全不可能由利润积累,那末对这种说法的回答应当是,从利润会如此下降的假定出发来加以论证是荒谬的,因为在利润率达到这个水平之前很久,资本就已向国外流走,以便在别的国家得到更高的利润,而资本输出的可能性总是会确立某种界限,只要还存在利润率较高的其他国家,任何一个国家的利润都不会下降到低于这个界限。”(第22—23页)“除了积累的原始源泉……还有派生源泉,例如,公债券所有者、官吏等等的收入。”(第23页)}\end{quote}

所有这些都很好。说[利润的]积累量决不是仅仅取决于利润率,而是取决于乘以使用的资本的利润率,也就是说,同样取决于使用的资本量,那是完全正确的。如果使用的资本=C,利润率=r,那末[最大限度的]积累=Cr,很明显,如果乘数C的增加比乘数r的减少迅速,这个乘积就会增加。通过观察所确定的事实的确就是这样。但是我们对于这一事实的原因还是一无所知。不过琼斯本人已经非常接近于这个原因,因为他已观察到,“辅助资本”和推动它的工人人口相比是在不断增长。

如果利润的下降是由于李嘉图所说的原因,即由于地租的增加,那末总剩余价值对使用的资本的比例会保持不变。区别只在于,总剩余价值的一部分——地租——靠牺牲另一部分即利润而增长,这就使得总剩余价值[对总资本]的比例保持不变,因为利润、利息和地租只是总剩余价值的单个范畴。可见,李嘉图实际上否定了这种现象。

另一方面,单是利息率的下降不能说明任何问题,正如它的上升不能说明任何问题一样,尽管利息率自然始终是一个最低比率的指标,利润不能低于这个最低比率。因为利润必须始终大于平均利息率。

[1155]撇开利润率下降规律使政治经济学家感到恐惧这点不谈,它的最重要的后果就是它以不断增长的资本积聚为前提,因而以较小的资本家日益丧失资本为前提。一般说来,这是资本主义生产的所有规律的结果。如果我们摘除这个事实的对抗性质,即摘除它在资本主义生产基础上所具有的那个特点,那末这个事实,即这个不断向前发展的集中化过程,将表明什么呢?不外是,生产丧失自己的私有性质并成为社会过程,并且这是现实的,而不只是形式上的,即不象在任何交换中那样,生产具有社会性是由于生产者的绝对的相互依赖性,由于他们必须把自己的劳动表现为抽象的社会劳动(货币)。因为生产资料现在是作为公共的生产资料被使用,因而——不是由于它们是单个人的财产,而是由于它们对生产的关系——作为社会的生产资料被使用;各个企业的劳动现在同样也是以社会规模来完成。

琼斯书中专门有一节,标题是《决定积累倾向的各种原因》。[琼斯把这些原因归结为以下五点:]

\begin{quote}{“(1)民族的气质和意向方面的差别;(2)国民收入在各居民阶级之间的分配有差别;(3)可靠地使用积蓄起来的资本的保障程度有差别;(4)有利而可靠地用连续的积蓄进行投资的难易程度有差别;(5)不同居民阶层通过积蓄改善自己地位的可能性有差别。”(第24页)}\end{quote}

这五点原因实质上可以归结为,积累取决于某一特定国家所达到的资本主义生产方式的发展阶段。

我们首先看一看第(2)点。资本主义生产发达的地方,利润是积累的主要源泉,即资本家把最大部分国民收入集中在自己手里,甚至一部分土地所有者也力图把自己的收入资本化。

第(3)点。资本家越是把管理国家的权力抓到自己手里,法律的和警察的保障就越增加。

第(4)点。随着资本的发展,一方面,生产领域会增大,另一方面,信用组织会发展,它使贷款人(银行家)能够把积蓄的每一文钱都集中在自己手里。

第(5)点。在资本主义生产条件下,人的社会地位的改善仅仅取决于金钱,而每个人都能幻想他有一天会成为路特希尔德。

还有第(1)点。并非一切民族都有相同的从事资本主义生产的才能。某些原始民族,例如土耳其人,既没有这方面的气质,也没有这方面的意向。但这是例外。随着资本主义生产的发展,会形成资产阶级社会的平均水平,与此同时,也会在极不相同的民族之间形成气质和意向的平均水平。资本主义生产,象基督教一样,本质上是世界主义的。所以,基督教也是资本所特有的宗教。在这两个方面只有人是重要的。一个人就其自身来说,他的价值不比别人大,也不比别人小。对于基督教来说,一切取决于人有没有信仰,而对于资本来说,一切取决于他有没有信用。此外,当然在第一种场合还要附加上天命,而在第二种场合要附加上一个偶然因素,即他是否生下来就有钱。

\centerbox{※     ※     ※}

剩余价值的源泉和最初的地租:

\begin{quote}{“当土地被占有并被耕种以后,它向花费在它上面的劳动所提供的,几乎总是多于用旧方法继续耕种它所必需的。土地在此以外所生产的[1156]一切,我们将称为它的剩余产品。这个剩余产品就是最初的地租的源泉,并由它来规定土地的所有者(他们不同于租种土地的人)经常能从土地上获得的那些收入的界限。”(第19页)}\end{quote}

这些最初的地租是剩余价值借以表现的最早的社会形式,这个隐秘的观点是重农学派学说的基础。

绝对剩余价值和相对剩余价值之间有一个共同点,即二者都以劳动生产力的一定发展程度为前提。如果一个人的(每一个人的)整个工作日(可支配的劳动时间)只够养活他自己(至多还有他的一家),那末也就不再有剩余劳动、剩余产品和剩余价值了。劳动生产力的一定发展程度这个前提,是以财富的自然源泉(土地和水)的天然富饶程度为基础的,而这种天然富饶程度在不同的国家等等是不同的。起初,需要是简单的,原始的,因而必须用来维持生产者本身生存的产品最低量也是很少的。这里的剩余产品同样是很少的。另一方面,在这样的条件下,靠剩余产品为生的人数也很少,因此,剩余产品在这里是人数较多的生产者的较少的剩余产品的总额。

绝对剩余价值的基础,即它赖以存在的现实条件,是土地(即自然)的天然富饶程度,而相对剩余价值则以社会生产力的发展为基础。

\tchapternonum{收入及其源泉。庸俗政治经济学}

\tchapternonum{[(1)]生息资本在资本主义生产基础上的发展[资本主义生产方式的关系的拜物教化。生息资本是这种拜物教的最充分的表现。庸俗经济学家和庸俗社会主义者论资本利息]}

[\endnote{在手稿第XIV本封面上草拟的《剩余价值理论》最后几章的计划中,紧接着《理查·琼斯》(这第五部分结束)这一章之后,马克思写了:《补充部分:收入及其源泉》(见本卷第1册第5页),而在第XV本封面上草拟的这一本的目录中,有《庸俗政治经济学》这一题目(同上,第6页)。这两个题目(《收入及其源泉》和《庸俗政治经济学》)占手稿第XV本很大一部分篇幅,而且二者是紧密地交错在一起加以论述的。在这一本(写于1862年10月—11月)里,马克思在手稿第890页上中断了对霍吉斯金观点的分析,转而写关于收入及其源泉和关于庸俗政治经济学的《补充部分》,庸俗政治经济学抓住收入及其源泉的拜物教形式的外表,在这个基础上建立了自己的辩护士理论。在进一步阐述有关问题(也在第XV本)的过程中,这个《补充部分》首先转到对借贷资本的分析,这种分析同对于庸俗政治经济学的批判紧密地结合在一起,然后又转到对商业资本这种不创造剩余价值而只分配剩余价值的资本主义经济领域的分析。这样一来,马克思就逐渐地越出了作为自己著作的历史批判部分的《剩余价值理论》的直接对象的范围。对商业资本的研究马克思一直继续到第XV本结尾。但下一本手稿即第XVI本(1862年12月)却以《第三章。资本和利润》这个标题开始。这一本的主要内容,是马克思在1865年写《资本论》第三卷第一、二两篇时广为利用的、对于剩余价值转化为利润和剩余价值率转化为利润率以及利润转化为平均利润的研究。在第XVI本结尾,马克思转到——如他自己所说——“这一篇最重要的问题”,即分析利润率随着资本主义生产方式的发展而下降的原因。马克思后来把这个阐述加以改写,用于《资本论》第三卷第三篇(《利润率趋向下降的规律》),这个阐述是到下一本手稿即第XVII本(1862年12月—1863年1月初)的开头才完成的。在第XVII本(从手稿第1029页开始),马克思重新回过头来分析商业资本,继续手稿第XV本的正文。但是在这里马克思也中断了对商业资本问题的阐述,这一次是为了写题为《资本主义再生产过程中货币的回流运动》的《补充部分》。这个篇幅相当大的《补充部分》,只是到第XVIII本(1863年1月)的开头,马克思才用下面的话作了结束:“对这个问题的进一步考察应该推后”。接着,马克思又重新(在手稿第1075页)回过头来研究商业资本,这一次他考察了不同的作者对这个问题的观点。对商业资本的所有这些研究包括在手稿第XV、XVII和XVIII本中;马克思在1865年写作《资本论》第三卷第四篇时在很大程度上利用了这些研究材料。结束了对商业资本的研究之后,马克思(在手稿第1084页)又回到《剩余价值理论》,回到在第XV本中断的关于霍吉斯金那一节。从这里列举的1861—1863年手稿第XV—XVIII本内容丰富的材料中,按照马克思本人的计划,只有包含在第XV本中(第891—950页)的《收入及其源泉。庸俗政治经济学》这一部分,作为附录收入本版《剩余价值理论》。马克思著作的全部历史批判部分以此结束。——第499页。}XV—891]收入的形式和收入的源泉以最富有拜物教性质的形式表现了资本主义生产关系。这是资本主义生产关系从外表上表现出来的存在,它同潜在的联系以及中介环节是分离的。于是,土地成了地租的源泉,资本成了利润的源泉,劳动成了工资的源泉。现实的颠倒借以表现的歪曲形式,自然会在这种生产方式的当事人的观念中再现出来。这是一种没有想象力的虚构方式,是庸人的宗教。庸俗经济学家——应该把他们同我们所批判的经济学研究者严格区别开来——实际上只是[用政治经济学的语言]翻译了受资本主义生产束缚的资本主义生产承担者的观念、动机等等,在这些观念和动机中,资本主义生产仅仅在其外观上反映出来。他们把这些观念、动机翻译成学理主义的语言,但是他们是从[社会的]统治部分即资本家的立场出发的,因此他们的论述不是素朴的和客观的,而是辩护论的。对必然在这种生产方式的承担者那里产生的庸俗观念的偏狭的和学理主义的表述,同诸如重农学派、亚·斯密、李嘉图这样的政治经济学家渴求理解现象的内部联系的愿望,是极不相同的。

然而,在所有这些形式中,最完善的物神是生息资本。在这里,我们看到的是资本的最初起点——货币,以及G—W—G′这个公式,而这个公式已被归结为它的两极G—G′。创造更多货币的货币。这是被缩简成了没有意义的简化式的资本最初的一般公式。

土地,或者说自然,是地租即土地所有权的源泉,——这具有充分的拜物教性质。但是,由于把使用价值和交换价值随意地混淆起来,通常的观念就还有可能求助于自然本身的生产力[来解释地租],而这种生产力借助某种魔术在土地所有者身上人格化了。

劳动是工资(即工人在他的产品中所占有的由劳动的特殊社会形式决定的份额)的源泉;劳动是下述事实的源泉:工人用自己的劳动从产品(即从物质上考察的资本)中为自己购买从事生产的许可权,并在劳动中占有一个源泉,由于有了这个源泉,他的一部分产品才以报酬的形式从这个作为雇主的产品中流回他那里,——这种说法也是够妙的。但是,在这里,通常的观念在如下的限度内还算是符合事实的,即尽管它把劳动同雇佣劳动混淆起来,从而把雇佣劳动的产品即工资同劳动的产品混淆起来,然而对健全的人类理智来说,有一点仍然是清楚的,这就是劳动本身创造它的工资。

至于资本,如果就生产过程来进行考察,那末认为它是猎取别人劳动的工具这样一种观念,总是或多或少地保存着。无论把这一点看作是“合理的”还是“不合理的”,有根据的还是无根据的,——这里总是以资本家和工人的关系为前提,总是指的这种关系。

就资本出现在流通过程来说,通常的看法所特别注意的是,它表现为商人资本,这是一种仅仅从事这种业务的资本,所以利润在这里有时用普遍欺诈这个含糊的观念来说明,有时用比较明确的观念来说明,即:商人欺诈产业资本家,就象产业资本家欺诈工人那样,或者说,商人欺诈消费者,就象生产者相互欺诈那样。不管怎样,这里利润是用交换,就是说,用社会关系而不是用物来解释的。

相反,在生息资本上物神达到了完善的程度。这是一个已经完成的资本,——因而是生产过程和流通过程的统一,——因此,它在一定的期间提供一定的利润。在生息资本的形式上,只剩下了这种规定性,而没有生产过程和流通过程作媒介。在资本和利润中,还存在着对过去的回忆,尽管由于利润和剩余价值的不同,由于所有资本具有形式单一的利润——一般利润率,资本已经[892]非常模糊不清了,已经变得难以理解和神秘莫测了。

在生息资本上,这个自动的物神,自行增殖的价值,创造货币的货币,达到了完善的程度,并且在这个形式上再也看不到它的起源的任何痕迹了。社会关系最终成为物(货币、商品)同它自身的关系。

对于利息以及利息与利润的关系,这里不作进一步的研究,对于利润按怎样的比例分为产业利润和利息,这里也不研究。有一点是清楚的,这就是:在资本和利息上,资本作为利息的神秘的、自行创造的源泉,即作为资本自行增长的源泉已达到了完善的程度。正因为如此,照[通常的]观念看来,资本主要存在于这种形式中。这就是真正意义上的资本。

既然在资本主义生产的基础上,体现在货币或商品中——真正说来是体现在货币即商品的转化形式中——的一定价值额提供了一种权力,使人有可能白白地从工人身上榨取一定量的劳动,占有一定的剩余价值,剩余劳动,剩余产品,那末很清楚,货币本身可以作为资本,作为特殊种类的商品出卖,或者说,资本可以在商品或货币的形式上被购买。

资本可以作为利润的源泉出卖。通过货币等等,我使另一个人能够占有剩余价值。因此,我取得这个剩余价值的一部分,是很自然的。土地具有价值,是由于它使我能够获得一部分剩余价值,因此我在土地上不过是为借助于土地所获得的这部分剩余价值而支付;同样,我在资本上不过是为借助于资本所创造的剩余价值而支付。因为在资本主义生产过程中,除了实现剩余价值外,资本的价值还会永恒化,会再生产出来,所以自然而然,货币或商品作为资本出卖时,会在一定时期之后又流回卖者手中,卖者永远不会象转让商品那样转让货币,而是保留自己对货币的所有权。在这种场合,货币或商品不是作为货币或商品出卖,而是作为它的二次方,作为资本,作为自行增殖的货币或商品价值来出卖了。它不仅会自行增长,而且会在总生产过程中把自己保存下来。因此,对于卖者来说,它仍旧是资本,会流回卖者手中。在这里,出卖就在于:一个把它作为生产资本使用的第三者,必须从他只是因有这笔资本而获取的利润中,支付一定的部分给资本所有者。象土地一样,货币是作为创造价值的物贷出的,这个物在这个创造价值的过程中被保存下来,不断地流回,因而也可能流回最初的卖者手中。只是由于流回最初的卖者手中,货币才成为资本。否则,他就是把它作为商品来卖,或是用它作为货币来买了。

但是不管怎样,形式就其本身来考察(实际上,货币作为榨取劳动的手段,作为获得剩余价值的手段,是定期转让的)是这样的:物现在表现为资本,资本也表现为单纯的物,资本主义生产过程和流通过程的全部结果则表现为物所固有的一种属性;究竟是把货币作为货币支出,还是把货币作为资本贷出,取决于货币所有者,即处在随时可以进行交换的形式上的商品的所有者。

这里我们看到的是作为本金的资本和作为果实的资本的关系,资本提供的利润由资本自己的价值来决定,并且资本本身不会因这个过程而消失(这是符合资本的性质的)。

由此可以明白,为什么肤浅的批判完全象它想要保存商品而反对货币那样,现在却要用它那改良派的智慧去反对生息资本,同时毫不触动现实的资本主义生产,而只是攻击这种生产的一个结果。这种从资本主义生产的立场出发对于生息资本的反驳,今天竟自诩为“社会主义”,其实这种反驳,作为资本本身的发展因素,例如在十七世纪就已出现,那时,产业资本家还必须首先同当时还比自己强大的旧式高利贷者进行斗争,以夺取自己的地位。

[893]作为生息资本的资本,它的充分的物化、颠倒和疯狂,——不过,在生息资本上,资本主义生产的内在本性,它的疯狂性,只是以最明显的形式表现出来,——就是生“复利”的资本,在这里,资本好象一个摩洛赫,他要求整个世界成为献给他的祭品,然而由于某种神秘的命运,他永远满足不了自己理所当然的、从他的本性产生的要求,总是到处碰壁。

货币或商品流回它们的起点即资本家手中,是资本在生产过程和流通过程中具有特征的运动,这一方面表示现实的形态变化,即商品转化为它的生产条件,生产条件再转化为商品形式:再生产;另一方面又表示形式上的形态变化,即商品转化为货币,货币再转化为商品。最后,这还表示价值的增长,G—W—G′。原有的、但是已在过程中增大了的价值始终保留在同一个资本家手中。改变的只是资本家占有这个价值的形式——或者是货币形式,或者是商品形式,或者是生产过程本身的形式。

资本流回到它的起点,在生息资本的场合,取得了一个完全表面的、同现实运动(资本的回流就是这种运动的形式)相分离的形态。A把他的货币不是作为货币,而是作为资本支出。在这里,货币没有发生任何变化。它不过转手而已。它只是在B手中才实际转化为资本。但对A来说,货币变成资本是由于它从A手中转到了B手中。资本由生产过程和流通过程实际流回的现象,是对B来说的。而对A来说,流回是在和让渡相同的形式上进行的。货币由B手中再回到A手中。A是贷出货币,而不是支出货币。

货币在资本的实际生产过程中的每一次换位,都表示再生产的一个要素:或者是货币转化为劳动,或者是完成的商品转化为货币(生产行为的结束),或者是货币再转化为商品(生产过程的更新,再生产的重复)。在货币作为资本贷出时,就是说,在它不是转化为资本,而是作为资本进入流通时,货币的换位不过表示货币本身的转手。所有权留在贷款人手中,而对货币的支配则转到产业资本家手中。但对贷款人来说,货币转化为资本是从他把货币不是作为货币而是作为资本支出时开始的,即从他把货币交到产业资本家手中开始的。(对贷款人来说,即使他把货币不是贷给产业家,而是贷给浪费者,或者贷给交不起房租的工人,货币也仍然是资本。全部典当业就是建立在这个基础上的。)诚然,另外一个人把货币转化为资本,但这个行为是在贷款人与借款人发生的行为之外完成的。在这个行为中,这种中介过程消失了,看不见了,不直接包含在内了。这里表现出来的不是货币向资本的实际转化,只是这种转化的毫无内容的形式。正如劳动能力的情况一样,在这里,货币的使用价值就是:货币创造交换价值,创造比它本身所包含的更大的交换价值。货币作为自行增殖的价值贷出,作为商品贷出,不过是作为这样一种商品,它恰恰由于自己的这种属性而同商品本身相区别,从而也具有特殊的让渡形式。

资本的起点是商品所有者,货币所有者,简单地说,是资本家。因为资本的起点和终点是一致的,所以资本又流回到资本家手中。但是在这里,资本家是以双重身分存在的:既作为资本所有者,又作为把货币实际转化为资本的产业资本家。事实上,[894]资本是从产业资本家那里流出,然后又流回到他那里,但他仅仅是暂时的所有者。资本家是以双重身分存在的:法律上的和经济上的。因此,资本作为所有物,也就回到法律上的资本家那里,回到非正式的丈夫那里。然而资本的回流(这种回流包含着资本价值的保存,它使资本成为自行保存的和永久化的价值)只是对资本家II起中介作用,而不是对资本家I起中介作用。因此,资本的回流在这里也不是表现为一系列经济过程的归宿和结果,而是表现为买者和卖者之间的特殊的法律上的交易的结果,这就是,资本在这种场合是被贷出,不是被卖出,即只是暂时让渡。事实上,被卖出的只是它的使用价值,使用价值在这里就在于生产交换价值,提供利润,生产比它本身所包含的价值更多的价值。作为货币,资本并不由于使用而改变。但它是作为货币被付出,也是作为货币再流回。

资本流回的形式,取决于它的再生产方式。如果资本作为货币贷出,它就以流动资本的形式流回,流回的量等于它的全部价值加剩余价值(在这个场合就是剩余价值或利润中归结为利息的部分),即贷出的货币额加由它产生的增长额。

如果资本以机器、建筑物等形式贷出,简单地说,以资本在生产过程中必须借以执行固定资本职能的物质形式贷出,那末,它就以固定资本的形式,例如作为年支付流回(这个年支付等于对损耗的补偿额,即固定资本中进入流通的价值部分),再加上剩余价值中算作固定资本(不是因为它是固定资本,而是因为它是一定量的资本一般)利润的部分(在这里是利润的一部分,即利息)。

在利润本身,剩余价值,从而利润的真正源泉,已经模糊不清了,神秘化了:

(1)因为,从形式上考察,利润是以全部预付资本计算的剩余价值,因此资本的每个部分,不管是固定资本还是流动资本,是花在原料、机器上还是花在劳动上,都提供相同的利润;

(2)因为,某一单个的已知资本,例如500,如果它的剩余价值等于50,资本的每个部分,例如每五分之一,就都提供10%,这样,由于一般利润率的确立,现在每个500或100的资本,不管它用于哪个领域,不管其中可变资本和不变资本的比例如何,也不管它的周转时间如何不同等等,它同其他任何一个在完全不同的有机条件下活动的资本一样,在相同的期间,总要提供相同的平均利润,例如10%。这就是说,因为孤立地加以考察的各个单个资本的利润和由这些资本本身在其各自的生产领域创造的剩余价值,实际上是不等的量。

其实,第二点只是把第一点已经包括的东西作了进一步的阐述。

不过,作为利息的基础的,正是剩余价值的这种外表化的形式,也就是剩余价值作为利润而存在的形式,这种形式不同于它的最初的简单形态(这时它还带着出生的脐带)而且绝非一眼就可以辨认出来。利息不是直接以剩余价值为前提,而是直接以利润为前提,利息本身只是被归入特殊范畴、特殊项目内的一部分利润。因此,在利息上比在利润上识别剩余价值要困难得多,因为只有当剩余价值以利润形式出现时,利息才同它直接发生关系。

资本回流的时间取决于实际的生产过程;就生息资本来说,它作为资本的回流看来仅仅取决于贷款人和借款人之间的契约。所以,就这种交易来看,资本的回流不再表现为由生产过程决定的结果,而是表现为资本似乎一刻也没有丧失货币的形式。当然,这些交易是由资本的实际的回流决定的,但是这一点不会在交易本身中表现出来。

[895]利息和利润不同,它代表单纯的资本所有权的价值,就是说,它使货币(价值额,任何形式的商品)所有权潜在地成为资本所有权,从而使商品或货币本身成为自行增殖的价值。当然,劳动条件只有当它们作为工人的非所有物,从而作为别人的所有物同工人相对立来执行职能的时候,才是资本。但是只有同劳动相对立,它们才能作为别人的所有物执行职能。这些劳动条件和劳动的对立存在,使它们的所有者成为资本家,使资本家占有的这些劳动条件成为资本。但是,在货币资本家A手中,资本不具有这种使自己成为资本,从而也使货币所有权表现为资本所有权的对立性质。货币或商品借以成为资本的现实的形式规定性消失了。货币资本家A决不是同工人相对立,他只是同另一个资本家B相对立。他卖给B的,事实上只是货币的“使用权”,是货币转化为生产资本时将会产生的结果。但是他直接出卖的事实上并不是使用权。如果我出卖商品,我就是出卖一定的使用价值。如果我用商品购买货币,那我就是购买了作为商品的转化形式的货币所具有的执行职能的使用价值。我不是在出卖商品的交换价值的同时出卖商品的使用价值,我也不是在购买货币本身的同时购买货币的特殊的使用价值。但是,作为货币的货币,在转化为资本并执行资本的职能(货币在贷款人手中没有执行这种职能)之前,它所具有的使用价值,不外是它作为商品(金、银——货币的物质实体)或作为货币,作为商品的转化形式所具有的使用价值。事实上,贷款人卖给产业资本家的,即在这次交易中发生的,不过是贷款人把货币所有权让给产业资本家一段时间。他在一定期间让渡自己的所有权,也就是产业资本家在一定期间购买这个所有权。因此,贷款人的货币在被让渡之前就已经作为资本出现:同资本主义生产过程分离的、单纯的货币或商品所有权就已经作为资本出现。

货币只有在让渡之后才表现为资本,这对于事情本身毫无影响,正象棉花的使用价值实际上只有在棉花让渡给纺纱业者之后才表现出来,或者肉的使用价值实际上只有在肉从肉铺里转到消费者的餐桌上才表现出来,并不改变棉花或肉的使用价值一样。货币一旦不用于消费,商品一旦不再为它的所有者的消费服务,它们就会使它们的所有者成为资本家,而它们自己——同资本主义生产过程相分离,并且在转化为“生产”资本之前——就作为资本出现,也就是说,作为自行增殖、自行保存、自行增长的价值出现。创造价值、提供利息是它们内在的属性,就象梨树的属性是结梨子一样。贷款人就是把自己的货币作为这种生息的东西出卖给产业资本家的。因为货币会自行保存,是自行保存的价值,所以产业资本家能够按照随意约定的期限把它归还。因为货币每年创造一定的剩余价值,一定的利息,确切些说,因为在每一段时间内它的价值都在增长,所以,产业资本家也能够每年或在契约规定的其他任何期限内把这个剩余价值支付给贷款人。要知道,作为资本的货币,和雇佣劳动完全一样,每天都提供剩余价值。利息虽然只是利润中固定在特殊名称下的部分,它在这里却表现为这样一种剩余价值的创造,这种剩余价值的创造是资本本身所固有的,同生产过程是分离的,因而是单纯的资本所有权即货币和商品的所有权所固有的,同造成这种所有权和劳动之间的对立从而使这种所有权具有资本主义所有权性质的那些关系是分离的,——这种剩余价值的创造是单纯的资本所有权所固有的,因而是本来意义上的资本所固有的。相反,产业利润在这里只不过表现为利息的附加额,这个附加额是借款人把借来的资本用在生产上,即用这笔资本对工人进行剥削而挣得的(或者象人们所说的:通过自己作为资本家的劳动;资本家的职能在这里被说成等于劳动,甚至被说成和雇佣劳动等同,因为[896]真正在生产过程中执行职能的产业资本家,事实上是作为从事活动的生产当事人,作为劳动者而与游手好闲、无所事事的贷款人相区别,贷款人同生产过程相分离并且处在这个过程之外执行所有者的职能)。

这样,利息,而不是利润,表现为从资本本身,因而从单纯的资本所有权中产生的资本的价值创造;因此利息表现为由资本本能地创造出来的收入。庸俗经济学家就是在这种形式上理解利息的。在这种形式上,一切中介过程都消失了,资本的物神的形态也象资本物神的观念一样已经完成。这种形态之所以必然产生,是由于资本的法律上的所有权同它的经济上的所有权分离,由于一部分利润在利息的名义下被完全离开生产过程的资本自身或资本所有者所占有。

对于要把资本说成是价值和价值创造的独立源泉的庸俗经济学家来说,这个形式自然是他们求之不得的,在这个形式上,利润的源泉再也看不出来了,资本主义过程的结果也离开过程本身而取得了独立的存在。在G—W—G′中,还包含有中介过程。在G—G′中,我们看到了资本的没有概念的形式,看到了生产关系的最高度的颠倒和物化。

一般利息率,或者说,一般利率当然是和一般利润率相适应的。在这里,我们不打算进一步阐明这个问题,因为对生息资本的分析不属于概论这一篇,而属于论信用那一篇。\endnote{马克思说的“概论这一篇”是指《资本一般》(《dasKapitalimallge-meinen》)那一篇。按照1858—1862年计划,这一篇应由三部分组成(《资本的生产过程》、《资本的流通过程》和《二者的统一,或资本和利润》),紧接着这一篇之后应该是具有更专门性质的三篇:《资本的竞争》、《信用》和《股份资本》。马克思在后来写作《资本论》的过程中,还把很多按照原来计划不属于《资本一般》这一篇问题范围的东西逐渐地收入《资本的生产过程》、《资本的流通过程》和《二者的统一,或资本和利润》这几部分中。其中包括有很多与信用和信用制度有关的问题被收入由《资本和利润》这一篇发展而来的《资本论》第三卷。——第512页。}但是,为了完全弄清楚资本的这个表现形式,指出如下一点是重要的:一般利润率远远不象利息率,或者说利率那样表现为可以捉摸的、明确的事实。诚然,利率在不断地波动。今天(在向产业资本家贷款的货币市场上,我们所谈的只是这个方面)利率是2%,明天是3%,后天又是5%。但这个2%,3%或5%的利率是适用于所有贷款人的。提供2%,3%,5%,是任何一个100镑货币额的一般比率,同一个实际执行资本职能的价值额,在不同的生产领域所提供的实际利润却很不相同,这些实际利润对利润的观念上的平均水平的偏离,使利润的这个平均水平始终只有通过某种过程,通过某种反作用才能确立下来,而这一点又始终只有在较长的资本流通期间才能做到。在若干年间,一定领域的利润率较高,而在以后若干年间则较低。把这若干年或一系列这样的演变综合在一起,平均起来就得出平均利润。但这样一来,平均利润就不表现为直接既定的东西,而只表现为各种相互矛盾的波动的平均结果。利率却不是这样。它普遍地是每天确定的事实,这个事实对产业资本家来说,甚至是他们从事活动时计算上的前提和项目。一般利润率在用来估计实际利润时,事实上仅仅作为观念上的平均数存在;在它被固定为现成的、确定的、既定的东西时,它仅仅作为平均数、作为抽象物存在;在现实生活中,它仅仅作为在各种不同的实际利润率的平均化运动中起决定作用的趋势存在,而不管这些利润率是属于同一领域的单个资本,还是属于不同生产领域的不同资本。

[897]贷款人向资本家要求的,是根据一般利润率(平均利润率)计算,而不是根据单个资本家那里出现的对一般利润率的个别偏离。平均数在这里成了前提。利息率本身在变动,但是这种变动对所有的贷款人都适用。

相反,确定的、相同的利息率不仅按平均数来说是存在的,而且事实上也是存在的(虽然它根据借款人是否被认为第一流的债务人而在最低限度和最高限度之间变动),对它的偏离宁可说是由特殊情况所造成的例外。同记载气压状况的气象报告相比,这种不是为这个或那个资本编制,而是为货币市场上现有的资本即借贷资本编制的记载利息率状况的证券交易报告,其准确性毫不逊色。

借贷资本的利息率具有较大的固定性和等同性,它和一般利润率的较难捉摸的形式不同,而且相反。这种情况从何而来,这里不去阐述。这样的阐述属于论信用那一篇。不过有一点是明显的:每个领域内的利润率的波动,——同一个生产领域内的单个资本家享有的特殊利益完全撇开不谈,——都取决于当时市场价格的状况和市场价格围绕费用价格的波动。不同领域的利润率的差别,只有通过不同领域的市场价格即不同商品的市场价格和这些商品的费用价格的比较才能知道。某个特殊领域的利润率下降到观念上的平均水平之下,如果时间拖得很久,就足以使资本离开这个领域,或者使新资本不可能按平均规模流入这个领域。因为,新的追加资本的流入,同已经投入的资本的再分配相比,更能使资本在各特殊领域的分配平均化。而特殊领域的超额利润只有通过市场价格和费用价格的比较才能知道。只要差别以某种方式表现出来,资本就开始从一些生产领域流出而流入另一些生产领域。撇开这种平均化行为需要时间这一点不谈,每个特殊领域的平均利润本身,也只有根据资本的性质,通过例如在7年等等的周期内所实现的利润率的平均数表现出来。因此,单是上下波动,如果不超过平均程度,不采取异常的形式,就不足以引起资本的转移,何况固定资本还会给资本的转移带来困难。一时的行情只能在有限的程度上产生影响,而且它对追加资本的流人或流出的影响,要大于对已经投入不同领域的资本的再分配的影响。

我们看到,所有这一切是一个非常复杂的运动,这里要考察的,不仅有每个特殊领域的市场价格、不同商品的比较费用价格、每个领域的供求状况,而且有不同领域的资本家的竞争!此外,平均化的快慢在这里取决于资本的特殊有机构成(例如,固定资本多还是流动资本多)和它们的商品的特殊性质,就是说,要看商品作为使用价值的性质是否易于允许按照市场价格的状况把它们较快地撤出市场、减少或增加它们的供给。

货币资本的情况则完全不同。在货币市场上,互相对立的只是两个范畴:买者和卖者,需求和供给。一方面是借款的资本家阶级,另一方面是贷款的资本家阶级。商品具有同一形式——货币。资本因投在特殊生产领域或流通领域而具有的一切特殊形态,在这里都消失了。在这里,资本是存在于独立的交换价值即货币的没有差别的彼此等同的形态上。特殊领域之间的竞争在这里停止了;它们全体一起作为借款人出现,资本则以这样一个形式与它们全体相对立,在这个形式上,按怎样的方式使用的问题对资本来说还是无关紧要的事。如果说生产资本[898]只是在特殊领域之间的运动和竞争中把自己表现为整个阶级共有的资本,那末,资本在这里现实地有力地在对资本的需求中表现为整个阶级共有的资本。另一方面,货币资本(货币市场上的资本)也实际具有这样一个形态,在这个形态上,它是作为共同的要素,而不问它的特殊使用方式如何,根据每个特殊领域的生产需要,被分配在不同领域之间,被分配在资本家阶级之间。并且,随着大工业的发展,出现在市场上的货币资本,会越来越不由个别的资本家来代表,即越来越不由市场上现有资本的这个部分或那个部分的所有者来代表,而是由把它集中起来,组织起来,并且以完全不同于实际生产的方式把它控制起来的银行家来代表。因此,就需求的形式来说,和货币资本相对立的是整个阶级的力量;但就供给来说,这个资本整个地表现为借贷资本,表现为集中在少数蓄水池里的全社会的借贷资本。

这就是为什么一般利润率同固定的利息率相比,表现为模糊不清的景象的一些理由;利息率的大小固然也会变动,但并不妨碍它对所有借款人来说都一样地发生变动,所以它在他们面前总是表现为固定的、既定的量,象货币的价值的变动并不妨碍它对一切商品来说都具有相同的价值一样;象商品的市场价格虽然每天发生波动,但并不妨碍它逐日都有牌价一样,利息率的变动也不妨碍它作为货币的价格有规则地在牌价中标示出来。这是因为资本本身在这里是作为一种特殊的商品——货币——提供的;因此,它的价格的确定,和其他一切商品的情形一样,就是它的市场价格的确定;因此,利息率总是表现为一般利息率,表现为这样多的货币取得这样多的利息。而利润率甚至在同一个领域内,在商品市场价格相等的情况下,也可能不同(因为各单个资本生产相同的商品时的条件不同;因为特殊利润率不是由商品的市场价格决定的,而是由市场价格和生产费用之间的差额决定的),而不同领域的利润率,只是在过程中通过不断的波动才能达到平均化。一句话:只是在货币资本上,在借贷货币资本上,资本才成为商品,这种商品的自行增殖的属性具有一个固定的价格,由当时的利息表示出来。

因此,资本作为生息资本,而且正是在它作为生息货币资本的直接形式上(生息资本的其他形式在这里与我们无关,这些其他形式也是由这个形式派生出来的,并以这个形式为前提),取得了它的纯粹的拜物教形式。第一,这是由于资本作为货币的不断存在;在这样的形式上,资本的一切规定性都已经消失,它的现实要素也看不出来;它仅仅作为独立的交换价值、作为获得独立存在的价值而存在。在资本的现实过程中,货币形式是一个转瞬即逝的形式。在货币市场上,资本总是以这个形式存在。第二,资本所产生的剩余价值,又是在货币形式上,表现为资本本身应得的东西,表现为货币资本,即同它完成的过程相脱离的资本的单纯所有者应得的东西。G—W—G′在这里成了G—G′,而且,正象资本形式在这里是没有差别的货币形式一样,——因为货币正好是这样一个形式,在这个形式上,商品作为使用价值的差别消失了,从而由这些商品的存在条件构成的生产资本的差别,生产资本本身的特殊形式的差别也消失了,——货币资本所产生的剩余价值,它所转化成或表现出来的剩余货币,也表现为根据货币额本身的量来计算的一定的比率。利息率是5%时,作为资本的100镑就等于105镑。这样就得出一个自行增殖的价值的,或者说,创造货币的货币的十分明显的形式。它同时又是毫无内容的形式,不可理解的、神秘的形式。我们在分析资本时是从G—W—G出发的,G—G′不过是它的结果而已。\endnote{马克思指他的1861—1863年手稿第I本,这一本从《货币转化为资本》这一节开始。该节第一小节是《G—W—G。资本的最一般形式》(1861—1863年手稿第1—6页)。——第517页。}现在我们发现G—G′作为主体。正象生长是树木固有的属性一样,生出货币(τοχοs)\fnote{产物,利息。——编者注}是资本在其作为货币的纯粹的形式上固有的属性。我们在外表上发现的、因而曾经作为我们分析的出发点的这个不可理解的形式,现在又作为一个过程的结果被我们碰到了,在这个过程中,资本的形态越来越和它的内在本质相异化,并且越来越与之失去联系。

[899]我们从作为商品的转化形式的货币出发。现在我们到达作为资本的转化形式的货币。这和我们曾经把商品看成是资本生产过程的前提和结果完全一样。

资本在自己这种最奇特同时又和普通观念最接近的形态上,既是庸俗经济学家的“基本形式”,又是肤浅的批判的最直接的攻击点。就前者来说,部分地是因为内在联系在这里最少表现出来,而且资本是以一种好象是价值的独立源泉的形式出现;部分地是因为在这种形式上资本的对立性质完全被掩盖了,被抹杀了,资本和劳动的对立不见了。另一方面,这种形式的资本所以受到攻击,是因为它在这里以最不合理的形式表现出来,给庸俗社会主义者提供了最容易突破的攻击点。

十七世纪资产阶级经济学家(柴尔德、卡耳佩珀等人)反对把利息看作剩余价值的独立形式,这种论战只是新兴的产业资产阶级反对旧式高利贷者——当时货币财富的垄断者——的斗争。在这里,生息资本还是一种洪水期前的资本形式,这种形式只是刚刚不得不从属于产业资本,处于依附产业资本的地位,这是生息资本在资本主义生产基础上从理论和实践上都必须占有的地位。资产阶级在这里也象在其他场合一样,毫不迟疑地去求助于国家,使现存的、旧时遗留下来的生产关系适合于它自己的需要。

显然,按另一种办法在不同种类的资本家之间分配利润,即靠降低利率来提高产业利润或者相反,都绝不会触动资本主义生产的本质。因此,把生息资本当作资本的“基本形式”来反对的社会主义,就不仅是本身完全局限于资产阶级视野的问题。而且,就它的论战并非一种出于误解的、盲目向资本本身发起的攻击和批判来说(不过,在这里把资本和资本的一种派生形式等同起来了),它无非是一种披着社会主义外衣的、要求发展资产阶级信用的愿望,因此,它只是表示,在这种论战披上社会主义外衣的国家里,现存关系是不发达的。这种社会主义本身只是资本主义发展的一个理论上的征兆,尽管这种资产阶级的努力可能采取非常惊人的形式,例如“无息信贷”的形式\endnote{指蒲鲁东在他同巴师夏的论战(1849—1850年)中为“无息信贷”辩护。马克思在后面,在他的手稿第935—937页(见本册第581—585页)对蒲鲁东的这个观点进行了批判。——第518页。}。圣西门主义及其对于银行制度的赞美就是属于这一类(以后又出现过“动产信用公司”\endnote{动产信用公司(CreditMobilier,全称SocietegeneraleduCreditMobilier)是法国的一家大股份银行,创办于1852年。该银行以进行金融投机活动著称,最后于1867年破产。1856—1857年,马克思为伦敦的宪章派报纸《人民报》(《ThePeople’sPaper》)和美国的报纸《纽约每日论坛报》(《New-YorkDailyTribune》)写了六篇文章评论该银行的投机活动(见《马克思恩格斯全集》中文版第12卷第23—40、218、227、313—317页以及第13卷第85—86、186页)。——第518页。})。

\tchapternonum{[(2)]生息资本和商业资本同产业资本的关系。更为古老的形式。派生的形式}

商业形式和利息形式比资本主义生产的形式即产业资本更古老。产业资本是在资产阶级社会占统治地位的资本主义关系的基本形式,其他一切形式都不过是从这个基本形式派生的,或者与它相比是次要的,——派生的,如生息资本;次要的,也就是执行某种特殊职能(属于资本的流通过程)的资本,如商业资本。所以,产业资本在它的产生过程中还必须使这些形式从属于自己,并把它们转化为它自己的派生的或特殊的职能。产业资本在它形成和产生的时期碰到了这些更为古老的形式。产业资本碰到它们时把它们作为前提,但不是作为由它本身确立的前提,不是作为它自己生活过程的形式。这如同它最初碰到了商品,但不是作为它自己的产品,它碰到了货币流通,但不是作为它自己的再生产要素。一旦资本主义生产在它的所有形式上发展起来,成了占统治地位的生产方式,生息资本就会从属于产业资本,商业资本就会仅仅成为产业资本本身的一种从流通过程派生的形式。但是,作为独立形式的[900]生息资本和商业资本必须先被摧毁并从属于产业资本。对生息资本使用行政权力(国家),强行降低利率,使生息资本再也不能把条件强加于产业资本。但是,这是资本主义生产一些最不发达的阶段所特有的形式。产业资本为了使生息资本从属于自己而使用的真正方式,是创造一种产业资本所特有的形式——信用制度。强行降低利率还是产业资本本身从以前的生产方式的方法中借用来的形式,一旦产业资本强大了,夺取了地盘,它就把这个形式当作无用的、不合目的的东西扔掉。信用制度是它自己的创造,信用制度本身是产业资本的一种形式,它开始于工场手工业,随着大工业而进一步发展起来。信用制度最初是反对旧式高利贷者(英国的金匠、犹太人、伦巴第人等等)的论战形式。十七世纪揭示信用制度的最初秘密的著作,全是以这种论战形式写成的。

至于商业资本,它以各种不同的形式从属于产业资本,或者也可以说,它成了后者的职能,成了执行某种特殊职能的产业资本。商人不是购买商品,而是购买雇佣劳动,用以生产供他进行商业销售的商品。但是,这样一来,商业资本本身就失去了它和生产相比所具有的固定形式。工场手工业通过商人向商品生产者的这种转化来反对中世纪的行会,并把手工业限制在比较狭小的范围。在中世纪,商人(意大利、西班牙等国工场手工业发达的个别分散的点除外)不过是城市行会或农民所生产的商品的包买商。\endnote{马克思在《资本论》第三卷第二十章中指出:波佩认为,中世纪的商人不过是行会手工业者或农民所生产的商品的“包买商”。见约·亨·摩·波佩的著作《从科学复兴至十八世纪末的工艺学历史》1807年哥丁根版第1卷第70页。——第520页。}

商人向产业资本家的这种转化,同时也是商业资本向单纯的产业资本形式的转化。另一方面,生产者成了商人。例如,呢绒生产者不是逐渐地一小批一小批地从商人那里获得自己所需要的材料并为之加工,而是自己按照自己资本的大小去购买材料等等。各种生产条件都作为他自己买来的商品进入生产过程。呢绒生产者现在已经不是为个别商人或某些顾客生产,而是为商业界生产了。

在第一种形式上,商人统治着生产,商业资本统治着由它推动的城市手工业者的劳动和农民家庭手工业。手工业和家庭手工业是从属于它的。在第二种形式上,生产转化为资本主义生产。生产者自己就是商人;商业资本在这里只是在流通过程中起中介作用,在资本的再生产过程中执行一定的职能。这是两种形式。商人作为商人成为生产者、产业家。产业家、生产者成为商人。

起初,商业是行会的、农村家庭的和封建的农业生产转化为资本主义生产的前提。它使产品发展成为商品,这部分地是因为它为产品创造了一个市场,部分地是因为它提供了新的商品等价物,部分地是因为为生产提供了新的材料,并由此开创了一些生产部门,它们一开始就以商业为基础:既以替市场生产为基础,也以世界市场造成的生产要素为基础。

一旦工场手工业(尤其大工业)相当巩固了,它就又为自己创造市场,夺取市场,一部分是采用暴力手段来开辟市场,但市场是它用自己的商品本身来夺取的。以后,商业就只不过是工业生产的奴仆,而对工业生产来说,市场的不断扩大则是它的生活条件,因为不断扩大的大量生产不受现有的商业界限(就商业仅仅反映现有需求而言)的限制,而是仅仅受现有的资本量和劳动生产力发展水平的限制,它不断地使现有市场商品充斥,从而不断地促使市场界限扩大和改变。在这里,商业是产业资本的奴仆,它执行从产业资本的生产条件中产生的一项职能。

产业资本在其发展的初期,试图用殖民制度(同时用禁止性关税制度)以暴力手段为自己确保一个市场和若干市场。产业资本家面对着世界市场;因此,他要[901]把自己的生产费用不仅同国内的市场价格相比较,而且同整个世界市场的市场价格相比较,同时必须经常这样做。他在生产时总是要考虑世界市场的市场价格。以前,这种比较只是商人的事,这样就保证了商业资本对生产资本的统治。[901]

\centerbox{※     ※     ※}

[902]可见,利息无非是利润的一部分(利润本身又无非是剩余价值,无酬劳动),它是由完全地或部分地借助别人的资本从事“劳动”的产业资本家支付给这笔资本的所有者的。利息是利润——剩余价值——的一部分,这一部分作为一种特殊的范畴被固定下来,以特有的名称和总利润相分离;这种分离和利息的起源毫无关系,只和它的支付或占有的方式有关。尽管产业资本家直接掌握全部剩余价值,不管剩余价值以地租、产业利润和利息的名义在产业资本家和其他人之间怎样进行分配,产业资本家总不是自己占有这部分利润,而是把它从自己的收入中扣除,支付给资本所有者。

如果利润率是既定的,利息率的相对高度就取决于利润分割为利息和产业利润的比例;如果这种分割的比例是既定的,利息率的绝对高度(即利息对资本的比例)就取决于利润率。这种分割比例是怎样确定的,这里不打算研究。这是属于对资本的现实运动,亦即对各个资本的现实运动的考察问题,而我们这里涉及的是资本的一般形式。

生息资本的形成,它和产业资本的分离,是产业资本本身的发展、资本主义生产方式本身的发展的必然产物。货币(即总是可以转化为生产条件的价值额)——或生产条件(货币随时都可以转化为生产条件,货币不过是生产条件的转化形式)——作为资本来使用,就可以支配一定量的别人劳动,支配比它本身所包含的更多的劳动。货币在同劳动交换时不仅保存自己的价值,而且增加自己的价值,产生剩余价值。作为资本的货币或商品,其价值不是由它们作为货币或商品所具有的价值来决定,而是由它们为自己的所有者“生产”的剩余价值的量来决定。资本的产品是利润。在资本主义生产的基础上,货币是作为货币支出还是作为资本支出,只是货币的不同的用途。在资本主义生产的基础上,货币(商品)从可能性来说是资本(正象劳动能力从可能性来说是劳动完全一样),因为,第一,它可以转化为生产条件,而且实际上也只是这些生产条件的抽象表现,是它们作为价值的存在,第二,财富的物质要素从可能性来说具有成为资本的属性,因为这些要素的对立面——雇佣劳动,也就是使它们成为资本的东西——是作为社会生产的基础存在的。

地租也只是产业资本家必须支付给另一个人的一部分剩余价值的名称,正如利息是由产业资本家虽然收进来(和地租一样)但是必须支付给第三者的另一部分剩余价值一样。然而,这里有很大的区别。土地所有者利用土地所有权阻止资本按照农产品的费用价格使它们的价值平均化。对土地所有权的垄断使他有可能这样做。它使他有可能把价值和费用价格间的差额装进自己的腰包。另一方面——在涉及级差地租的情况下——这种垄断还使他有可能拿走产品的市场价值超过一定土地上产品的个别价值的余额,而不是象在其他部门那样,这个差额作为超额利润落入在比平均条件有利的条件下从事经营的资本家的腰包,因为平均条件满足需求的基本量,决定生产的主要量,从而调节每个特殊生产领域的市场价值。

土地所有权是夺取产业资本生产的一部分剩余价值的手段。相反,贷出的资本——在资本家用借来的资本从事经营的情况下——是生产全部[903]剩余价值本身的手段。货币(商品)可以作为资本贷出这种情况,不过意味着,货币从可能性来说是资本。李嘉图所说的废除土地所有权,即把土地所有权变为国家所有权,把地租交付给国家而不是交给地主,是一种理想,是资本从它最内在的本质中产生的内心愿望。资本不可能废除土地所有权。但是,通过把土地所有权转化为[交给国家的]地租,资本作为阶级占有了地租,以抵补自己的国库开支,就是说,资本通过迂回的办法占有了它不能直接拿到手的东西。可是废除利息和生息资本,就是废除资本和资本主义生产本身。只要货币(商品)可以用作资本,它就可以作为资本出卖。因此,那些要商品而不要货币、要产业资本而不要生息资本、要利润而不要利息的人,真不愧为小资产阶级空想主义者。

生息资本和提供利润的资本——这并不是两种不同的资本,而是同一个资本,它在生产过程中执行资本的职能,提供利润,利润在两种不同的资本家之间进行分配:一种是处在生产过程之外、作为所有者代表资本自身的资本家{不过,资本由私有者代表是资本的基本条件;不然的话,它就不成其为与雇佣劳动相对立的资本了},一种是代表执行职能的资本即处于生产过程中的资本的资本家。

\tchapternonum{[(3)剩余价值的各个部分独立化为不同形式的收入。利息和产业利润之间的比例。收入的拜物教形式的不合理性]}

利润分割的进一步“硬化”或独立化是这样表现出来的:每个资本的利润,从而以资本互相平均化为基础的平均利润,都分成或被割裂成两个互不依赖或互相独立的部分,即利息和产业利润,后者现在往往也被简单地称为利润或取得“监督劳动的工资”这样的新教名。如果利润率(平均利润)等于15%,利息率(如我们已看到的那样,它总是固定在一般的形式上)等于5%(一般利息率在货币市场上总是作为货币的“价值”或“价格”标示出来),那末资本家——即使他是资本的所有者,资本的任何部分都不是借来的,从而利润不必在两种资本家中间进行分配——就会这样来考虑问题:在这15%当中,5%代表他的资本的利息,只有10%代表他把资本用于生产而取得的利润。这5%的利息是他作为“产业资本家”对作为资本“所有者”的他自己所负的债;这个利息应付给他的资本自身,因而也应付给作为资本自身的所有者的他(这个资本自身同时也是资本的自为的存在,或者说资本作为资本家的存在,作为从自身中排他的所有权的存在),这个抽掉了生产过程的资本与执行职能的资本、处于生产过程中的资本不同,与代表这种执行职能的、“劳动的”资本的“产业资本家”不同。“利息”是资本的果实,即不进行“劳动”、不执行职能的资本的果实,而利润则是“劳动的”、执行职能的资本的果实。这和下述情况相类似:农场主-资本家——他同时又是土地所有者,用资本主义方式经营土地的土地所有者——把他的利润中形成地租的那一部分,把这个超额利润,不是归于作为资本家的他自己,而是归于作为土地所有者的他自己,不是归于资本,而是归于土地所有权,也就是说,他作为资本家欠了作为土地所有者的他自己的“地租”。这样一来,表现在一种规定性上的[生息的]资本,就和表现在另一种规定性上的[提供产业利润的]同一个资本,在固定的形式上对立起来,就象土地所有权和资本相互对立一样,事实上,土地所有权和资本是以两种本?上不同的生产资料为基础的、占有别人劳动的权利。

如果在一种场合有5个股东经营一家棉纺厂,棉纺厂代表100000镑资本,提供10%的利润,即10000镑,那末,每个股东各得利润的1/5,即2000镑。如果在另一种场合单独一个资本家把同量资本投入工厂,获得同量利润10000镑,那末,这个资本家不会认为:他得到的2000镑是一个股东的利润,得到的8000镑是四个不存在的股东的合伙利润。因此,在不同的[904]资本家——他们对同一笔资本具有不同的法律权利,以这种或那种形式表现为同一笔资本的共有者——之间进行的单纯的利润分割本身,绝不会给利润的这些部分建立不同的范畴。那末,为什么利润在资本贷出者和资本借入者之间的偶然的分割会建立这些范畴呢?

乍看起来,这里谈的只是利润的这样一种分割,在进行这种分割时存在两个资本所有者,两个具有不同权利的所有者,而这一点乍看起来是个法律因素而不是经济因素。资本家是用自有资本还是用别人的资本从事生产,或者他是按什么比例用自有资本和别人的资本从事生产,这个问题本身是完全无关紧要的。可是,利润分为[产业]利润和利息的这种分割并不表现为偶然的分割,不是取决于偶然的情况(即一个资本家是否实际上要同另一个资本家分割利润,他在这一场合是用自有资本经营还是用别人的资本经营),而是相反,即使他仅仅用自有资本进行生产,他也无论如何都要分裂为资本的单纯所有者和资本的使用者,分裂为生产过程外的资本和生产过程内的资本,分裂为自身提供利息的资本和作为处于[生产]过程中的、提供利润的资本,——这种情况是怎样产生的呢?

在这里,有一个现实的因素作为基础。货币(作为商品一般的价值表现)在生产过程中所以能占有剩余价值(不管它叫什么,不管它分解成哪些部分),只是因为在生产过程之前货币就已经被假定为资本。在生产过程中,货币作为资本把自己保存、生产和再生产出来,而且是在不断扩大的规模上这样做。但是,如果资本主义生产方式已经存在,如果生产是在这种生产方式的基础上并且在与它相适应的社会关系的范围内进行,就是说,如果所涉及的问题不单单是资本的形成过程,那末,早在生产过程之前,货币按其性质来说就已经作为资本自身存在了,尽管这种性质只是在过程中才实现,而且一般说来只有在过程本身中才具有现实性。如果货币不是作为资本进入过程,它也就不会作为资本,就是说,不会作为提供利润的货币,作为自行增殖的价值,作为生产剩余价值的价值从过程中出来。

这里的情况和货币的情况一样。例如,铸币无非是一块金属。它成为货币只是由于它在流通过程中的职能。但是,商品的流通过程一旦作为前提,铸币就不仅执行货币的职能,而且在它进入流通过程之前,就作为货币在每个单独的场合充当这个过程的前提了。

资本不仅是资本主义生产的结果,而且是它的前提。因此,货币和商品就其自身来说,潜在地是资本,在可能性上是资本:一切商品就其可能转化为货币而言,货币就其可能转化为形成资本主义生产过程要素的商品而言,都是这样的资本。可见,货币,——作为商品和劳动条件的纯粹的价值表现,——自身作为资本,是资本主义生产的前提。不作为过程的结果,而作为过程的前提来考察的资本是什么呢?是什么使它在进入过程之前就成为资本,从而过程只是使它的内在性质得到发展呢?是它借以存在的社会规定性:过去劳动同活劳动相对立,产品同活动相对立,物同人相对立,劳动本身的物的条件作为别人的、独立的、自我孤立的主体或人格化,一句话,作为别人的所有物,而且在这个形式上作为劳动本身的“使用者”和“支配者”(它们占有劳动而不是被劳动占有)同劳动相对立。价值(无论它是作为货币还是作为商品而存在),而在进一步的发展中则是劳动条件,作为别人的所有物,作为自我的所有物,同劳动者相对立,这无非是说,它们是作为非劳动者的所有物同劳动者相对立,或者至少是说,在劳动条件的所有者是资本家的情况下,他也不是作为劳动者,而是作为价值等等的所有者,作为主体(这些物就是在这个主体上具有自己的意志,自己属于自己,人格化为独立的力量)同这些劳动条件相对立。资本,作为生产的前提,资本,在它不是从生产过程中出来,而是在它进入生产过程之前的形式上,是一种对立性,在这种对立性中,劳动作为别人的劳动同资本相对立,资本本身作为别人的所有物同劳动相对立。在这里表现出来的,是离开过程本身的、已表现为资本所有权本身的那种对立的社会性质。

[905]这个因素是资本主义生产过程不断产生的结果,并且作为这样的结果又是它的不断需要的前提;这个因素离开资本主义生产过程本身,现在表现在这样的事实上:货币和商品就其自身来说,潜在地是资本;它们能够作为资本出售;在这个形式上,它们代表单纯的资本所有权,代表作为单纯所有者的资本家(撇开他的资本主义职能不谈);就它们本身来考察,它们是对别人劳动的支配权,因而是自行增殖的价值,并且提出占有别人劳动的要求。

这里也清楚地表明了:占有别人劳动的根据和手段,就是这种关系,而不是资本家方面提供的任何劳动或对等价值。

因此,利息表现为由作为资本的资本,由单纯的资本所有权产生的剩余价值,资本之所以从生产过程中得到这个剩余价值,是因为资本作为资本进入生产过程,也就是说,这个剩余价值属于资本本身,而不以生产过程为转移,——尽管剩余价值只是在生产过程中才出现,——因此,这个剩余价值是资本作为资本就已经潜在地包含着的。相反,产业利润则表现为不是属于作为资本所有者的资本家,而是属于作为执行职能的所有者的资本家,即属于执行职能的资本的那一部分剩余价值。就象在这种生产方式中一切看来都是颠倒的那样,在利息和利润的关系上的这种最后的颠倒也终于出现了,以致利润中划为特殊项目[利息]的部分反而表现为专门属于资本的产物,而产业利润却不过是在这个部分上增长起来的追加额。

因为货币资本家处于生产过程本身之外,实际上仅仅作为资本所有者取得剩余价值中属于他自己的一份;因为资本的价格,即单纯的资本所有权的价格会在货币市场上以利息率表示出来,就象其他任何商品的市场价格会在市场上表示出来一样;因为剩余价值中由资本自身即单纯的资本所有权决定的份额由于这一点而是一个既定的量,而利润率却是波动的,在不同的生产领域每时每刻都不同,在每个领域内在各个资本家之间也不同(部分是由于他们进行生产的有利条件不同,部分是由于他们对劳动实行资本主义剥削的本领和能力不同,部分是由于他们在欺骗商品的买者或卖者时走运和狡黠的程度——让渡利润——不同),——因为上述种种原因,所以在资本家看来,不管他们是不是处于过程中的资本的所有者,利息自然是由资本本身,由资本所有权,由资本所有者(不管资本所有者是他们自己还是第三者)产生的;相反,产业利润在他们看来则是他们的劳动的产物。他们是作为执行职能的资本家——资本主义生产的实际当事人——同作为资本的单纯的、不活动的存在的他们自己或第三者相对立,从而作为劳动者同作为所有者的他们自己或其他人相对立。既然他们是劳动者,他们实际上就是雇佣工人,而且由于他们的特殊的优越地位,他们不过是报酬较高的雇佣工人,他们所以能这样,部分地应归功于他们是自己给自己支付工资。

因此,当利息和作为生息资本的资本表示物质财富同劳动的单纯对立,因而表示物质财富作为资本的存在时,在通常的观念中这一点恰恰被颠倒过来了,因为在表面现象上货币资本家乍看起来同雇佣工人毫无关系,而只同另一个资本家发生关系,这另一个资本家又不是同雇佣劳动相对立,而是自己作为劳动者同作为资本的单纯存在、单纯所有者的他自己或另一个资本家相对立。此外,单个资本家可以把他的货币作为资本贷出,也可以自己把它作为资本来使用。在他从自己的货币中取得利息时,他得到的只是货币的价格,这个价格,在他不是作为资本家“执行职能”,在他不“劳动”时,也是可以得到的。因此很明显,就他实际上从生产过程中得到的只是利息来说,他只应归功于资本,而不应归功于生产过程本身和[906]代表执行职能的资本的他自己。

由此也产生了某些庸俗经济学家的绝妙的论调:如果产业资本家除利息外得不到其他利润,他就会用他的资本去生息而过食利者的生活。这样一来,所有的资本家就会停止生产,所有的资本就会停止执行资本的职能,但是他们仍然可以靠它的利息为生!杜尔哥就已经有类似的议论:如果资本家得不到利息,他就会购买土地(资本化的地租),就会靠地租为生。\endnote{安·罗·雅·杜尔哥《关于财富的形成和分配的考察》1766年版第73节和第85节。——第530页。}但是,因为地租在重农学派那里代表真正的剩余价值,所以在这里利息仍然是从剩余价值来的;而在上述那些庸俗见解中,真实关系却被颠倒过来了。

还有另一种情况必须指出:对于借钱的产业资本家来说,利息加入费用,在这里费用是指预付的价值。一笔例如1000镑的资本,不是作为价值1000镑的商品,而是作为资本加入他的生产;因此,如果1000镑资本每年提供10%的利息,那末它就作为1100镑的价值加入年产品。所以,这里清楚地表明,价值额(以及体现它的商品)不是在生产过程中才成为资本,而是作为资本形成生产过程的前提,从而它自身已经包含了属于它(单纯的资本)的剩余价值。对于用借来的资本经营的产业家来说,利息,或者说,作为资本的资本,加入他的费用,它所以表现为这样的资本,只是因为它产生剩余价值(所以,它作为商品,例如值1000,作为资本就值1100,即1000+1000/10,C+C/x)。如果在产品中只得到利息,那末,它虽然是超过作为单纯商品计算的预付资本的价值的余额,但不是超过作为资本计算的商品的价值的余额;产业家必须把这个剩余价值付给别人,它属于他的预付,属于他生产商品时的支出。

至于用自有资本经营的产业家,他必须把资本的利息支付给自己,并把它看成是预付。实际上,他预付的不仅是例如价值为1000镑的资本,而且是作为资本的1000镑的价值,如果利息等于5%,这个价值就是1050镑。这对他来说也决不是什么空想。因为,如果他不把这1000镑用于生产而把它借出,它作为资本就会给他带来1050镑。因此,只要他把这1000镑作为资本预付给自己,他就是给自己预付1050镑。总得在某一个人身上弥补自己的损失,哪怕在自己身上!

价值为1000镑的商品作为资本具有1050镑的价值。就是说,资本不是简单的数字:它不是简单的商品,而是自乘的商品;不是简单的量,而是量的比例。资本是它作为本金、作为既定的价值同作为[生产了]剩余价值[的本金]的它自己的比例。资本C的价值(按一年计算)等于C(1+1/x)\endnote{马克思在论马尔萨斯这一章中,考察了马尔萨斯在李嘉图的《原理》出版(1871年)以后所写的著作。在这些著作里,马尔萨斯企图用旨在维护统治阶级中最反动阶层的利益的庸俗辩护论来对抗李嘉图的劳动价值论,对抗李嘉图的千方百计发展生产力的要求,而照李嘉图的观点,这样发展生产力,就应当牺牲个人的甚至整个阶级的利益。关于作为“人口论”的鼓吹者的马尔萨斯,在本章中只是附带谈了一下。马克思在《剩余价值理论》第2册《对所谓李嘉图地租规律的发现史的评论》那一章中对马尔萨斯论人口的著作做了一般的评述(见本卷第2册第121、123、125—128、158页)。——第3页。}或C+C/x。正象用简单的计算方法不可能理解或推算出等式ax=n中的x一样,从基本的概念中也无法理解或推算出自乘的商品,自乘的货币,资本。

同利润的一部分,即资本所创造的剩余价值的一部分以利息的形式表现为资本家的预付完全一样,在农业生产中,另外一部分——地租,也表现为资本家的预付。但是在这里,这种看法的不合理性却不那么引人注目,因为地租在这里表现为土地的年价格,土地因此作为商品加入生产。“土地价格”固然比资本价格包含的不合理性更大,但这种不合理性不是包含在形式本身之中。因为土地在这里表现为一种商品的使用价值,而地租则表现为这种商品的价格。(不合理性在于,不是劳动产品的东西——土地,却有了价格,即表现在货币上的价值,也就是说,有了价值,因而,应该被看作物化社会劳动。)因此,就外在的形式来说,我们在这里也和在任何其他商品那里一样,看到了双重表现:表现为使用价值和交换价值,交换价值观念地表现为价格,表现为某种绝对不同于商品使用价值的东西。而在“1000镑=1050镑”或“50镑是1000镑的年价格”这种说法中,却是相同的东西即交换价值和交换价值发生关系,而且交换价值要作为和自身不同的东西成为它自身的价格,即表现在货币上的交换价值本身。

[907]所以,这里有两种形式的剩余价值——利息和地租,资本主义生产的结果——作为前提,作为预付加入生产;这种预付是资本家本人投入的,所以它对于资本家来说决不代表剩余价值,即超过他的预付价值的余额。就剩余价值的这些形式而言,在单个资本家本人看来,剩余价值的生产属于资本主义生产的生产费用,而占有别人的劳动和占有超过在过程中消费掉的商品(不论这些商品是加入不变资本还是加入可变资本)价值的余额,是这种生产方式的必要条件。当然,这一点也表现在:平均利润形成商品费用价格的一个要素,因而形成商品供给的条件,形成商品生产本身的条件。然而,产业资本家把这个余额,这部分剩余价值,——尽管它形成生产本身的一个要素,——公正地看成超过自己的费用的余额,不认为它象利息和地租那样属于自己的预付。事实上,在危机时刻,当价格的下跌使产业利润消失或显著减少,因而生产缩减或停顿的时候,利润也会作为生产条件同他本人相对立。从这里可以看出那些把剩余价值的不同形式单纯看作分配形式的人的愚钝。它们同样是生产形式。[907]

\centerbox{※     ※     ※}

[937]在“土地—地租,资本—利润(利息),劳动—工资”这个三位一体的公式中,可能看起来最后一个环节还是最合理的:这里至少说出了工资产生的源泉。但是实际上正相反,最后这个形式却是最不合理的,它是另外两个形式的基础,就象雇佣劳动一般说来以作为土地所有权的土地和作为资本的产品为前提一样。只有当劳动条件以这种形式同劳动对立的时候,劳动才是雇佣劳动。但是在“劳动—工资”这个公式中,劳动正是作为雇佣劳动来表现的。因为工资在这里表现为劳动的特殊产品,表现为劳动的唯一产品(工资对于雇佣工人来说确实也是劳动的唯一产品),所以价值的其他部分——地租、利润(利息)——也必然表现为是从其他的特殊源泉产生的;正象产品中归结为工资的价值部分必须理解为劳动的特殊产品一样,归结为地租和利润的价值部分也必须理解为它们为之存在并归其所有的那些因素的特殊结果,就是说,这些价值部分必须理解为土地和资本各自的果实。[937]

\tchapternonum{[(4)剩余价值的转化形式的硬化过程以及这些形式同它们的内在实质即剩余劳动日益}

分离的过程。生息资本是这个过程的最终阶段。把产业利润看成“资本家的工资”的辩护论观点]

[910]我们来考察一下资本在以生息资本形式出现之前所经历的道路。

在直接的生产过程中,情况还简单。剩余价值除了剩余价值本身这种形式外,还没有取得特殊的形式,剩余价值本身这种形式,只不过使剩余价值有别于产品的另一部分价值,即构成产品中再生产出来的价值等价物的那一部分价值。正如一般价值归结为劳动一样,剩余价值归结为剩余劳动,即无酬劳动。因此,剩余价值也只是以实际会改变自己价值的那部分资本——可变资本,花在工资上的资本——来计量的。不变资本不过是使资本的可变部分能够发生作用的条件。情况很简单,如果用等于10个人的劳动的100镑购买20个人的劳动(即包含20个人的劳动的商品),产品价值就等于200镑,而100镑剩余价值就是10个人的无酬劳动。或者说,如果有20个人劳动,那末每人只有半天为自己劳动,另外半天为资本家劳动。20个半天等于10天。这等于是,只有10个人的劳动得到报酬,而另外10个人则是白白为资本家劳动。

这里,在这种胚胎状态中,关系还是很清楚的,或者更确切地说,完全不会误解。这里的困难只是在于说明这种不付等价物便能占有劳动是怎样由商品交换规律(即商品按其包含的劳动时间互相交换)产生的,首先是,怎样与商品交换规律不发生矛盾。

[911]流通过程已经抹掉了、已经掩盖了实际存在的联系。因为剩余价值量在这里同时还决定于资本的流通时间,所以看起来,这里还加进了一种与劳动时间不同的要素。

最后,如果考察一下完成了的资本(它表现为一个整体,表现为流通过程和生产过程的统一,它是再生产过程的表现,即一定的价值额,这个价值额在一定期间,在一定的流通阶段,生产出一定的利润即剩余价值),那末,在这种形态上,生产过程和流通过程还只是作为一种回忆和作为在同等程度上决定剩余价值的因素而存在,因此,剩余价值的单纯性质就被模糊了。剩余价值现在表现为利润。这里必须注意以下几点:(1)这种利润与不同于劳动时间的资本的一定流通阶段有关;(2)剩余价值在计算时,不是同直接产生它的那部分资本相比,而是不加区分地同整个总资本相比;这样一来,剩余价值的源泉完全看不见了;(3)虽然在利润的这种最初形式上,利润量在数量上还与单个资本生产的剩余价值量相等,但是利润率从一开始就不同于剩余价值率,因为剩余价值率等于,m/v,而利润率等于m/(c+v);(4)在剩余价值率既定的情况下,利润率可能提高或降低;利润率甚至可能朝着与剩余价值率变动相反的方向变动。

可见,剩余价值在利润的最初形式上已经具有这样一种形式,这种形式不仅使人不能直接辨认它与剩余价值、剩余劳动的同一性,而且好象是直接与这种同一性相矛盾的。

其次,由于利润转化为平均利润,由于一般利润率的形成,以及与此有关的或由此决定的价值转化为费用价格,单个资本的利润,不仅在表现上(即在利润率和剩余价值率的区别上),而且在实体上(这里也就是在数量上)都和单个资本在其特殊生产领域里所生产的剩余价值本身不同。如果我们考察单个资本,而且也考察某个特殊领域的总资本,那末,利润现在就不仅在表面上,而且在实际上都和剩余价值不同。等量资本提供等量利润,或者说,利润与资本的量成比例。或者说,利润由预付资本的价值决定。在所有这些表现上,利润同资本有机构成的关系完全被掩盖了;这种关系在这里已经无法辨认了。相反,一眼就能看到的是,等量资本推动的劳动量极不相同,从而支配的剩余劳动极不相同,创造的剩余价值量极不相同,但是提供的利润量相同。由于价值转化为费用价格,商品价值决定于商品中包含的劳动时间这种基础本身,似乎也被取消了。

正是在利润的这种完全异化的形式上以及在利润的形式愈来愈掩盖自己的内核的情况下,资本愈来愈具有物的形态,愈来愈由一种关系转化为一种物,不过这种物是包含和吸收了社会关系的物,是获得了虚假生命和独立性而与自身发生关系的物,是一个可感觉而又超感觉的存在物;而且在资本和利润的这种形式上,资本表面上是作为现成的前提出现的。这就是资本的现实性的形式,或者更确切地说,是资本的现实存在的形式。资本也正是以这种形式存在于其承担者即资本家的意识中,反映在他们的观念中。

这种固定的和硬化的(变了形的)利润形式(从而也是利润的创造者即资本的形式,因为资本是根据,利润是归结,资本是原因,利润是结果,资本是实体,利润是偶性;资本所以成为资本,只是因为它生产利润,只是因为它是创造利润即创造追加价值的价值)——从而也是作为利润根据的资本的形式,作为资本保存下来并通过利润来增殖的资本的形式——更加固定在它的外表性上了,因为赋予利润以这种平均利润形式的资本的平均化过程,使一部分独立存在的、似乎是在其他基础上(即在土地上)生长出来的利润,在地租形式上同利润脱离了。诚然,地租起初是作为租地农场主向土地所有者支付的一部分利润出现的。但是,因为他(租地农场主)既不能把这种超额利润装入自己的腰包,而他所使用的资本,作为资本来说,又与其他资本不论在哪一点上都毫无区别(租地农场主之所以把超额利润交给土地所有者,是因为他并不认为作为资本的资本是超额利润的源泉),所以土地本身在这里就表现为商品的这一部分价值(它的这一部分剩余价值)的源泉,而土地所有者不过是[912]土地在法律上的人格化。

如果地租按预付资本计算,那还有一点线索,可以使人想起地租的来源就是利润即一般剩余价值的一个特殊部分。(当然,在土地所有权直接剥削劳动的社会制度下,情况不是这样。在那里,要认识剩余财富的源泉,那是毫无困难的。)但是地租是按一定量的土地支付的;地租会资本化为土地价值;这个价值会比例于地租的涨落而涨落;地租则比例于保持不变的土地面积(可是用在土地上的资本的量却会变动)而涨落;土地等级的差别在必须按一定单位面积支付的地租的高度上表现出来;地租总额按总面积计算,这就能确定出比如说每平方尺的平均地租;如同资本主义生产所创造的这一生产的其他一切形式一样,地租也表现为一种固定的、既定的、任何时候都存在的、从而对个人来说是独立存在的前提。租地农场主必须支付地租,并且要按照土地的单位面积,根据土地的质量来支付一定的数额。如果土地质量有了提高或降低,他为若干英亩土地必须支付的地租也要提高或降低,而不管他在土地上使用了多少资本。正如他必须支付利息,而不管他获得了多少利润一样。

按产业资本来计算地租,是政治经济学的又一个批判性的公式,这个公式保持了地租同作为产生地租的基础的利润之间的内在联系。但是在现实中这种联系是不表露在外的;相反,地租在这里是以实际的土地来计量,——因此,一切中介过程都被砍去了,而地租的纯粹外表的独立形态却完成了。地租只有在这种外表化上,在完全脱离它的中介过程的情况下,才是独立的形态。多少平方尺的土地就提供多少地租。在这种说法中,剩余价值的一部分——地租——表现为同某种特殊的自然要素的关系,而与人的劳动无关,在这里,不仅剩余价值的性质完全被掩盖了(因为价值本身的性质被掩盖了),而且利润本身的存在现在也要归功于作为一种特殊的、物的生产工具的资本,正如地租的存在要归功于土地一样。土地作为自然界的一部分而存在,并提供地租。资本由人们生产的产品构成,这些产品提供利润。一种由人们生产的使用价值提供利润,另一种不是由人们生产的使用价值提供地租,——这只是物创造价值的两种不同的形式,前者与后者一样都是既可理解又不可理解。

显然,只要剩余价值分解成各个不同的特殊部分,而这些部分又与各种不同的、只是在物?上不同的生产要素——自然界、劳动产品、劳动——发生关系,只要剩余价值一般获得特殊的、彼此无关、互不依赖、由各种不同的规律调节的形态,那末,剩余价值所有这些形态的共同的统一体(即剩余价值本身),从而这个共同的统一体的性质,也就愈来愈无法辨认,不再通过现象表示自己,而必须当作某种隐藏的秘密来发现了。剩余价值各个特殊部分的形态的这种独立化,它们作为独立形态的相互对立,由于以下的事实而完成了:这些部分中的每一部分都可以归结为作为其尺度和特殊源泉的某种特殊要素,或者说,剩余价值的每一部分都表现为某种特殊原因的结果,某种特殊实体的偶性。这就是:利润—资本,地租—土地,工资—劳动。

就是这些完成了的关系和形式,在实际生产中表现为前提,因为资本主义生产方式是通过它本身所创造的各种形态运动的,这些形态即它的结果,又同样地在再生产过程中作为完成了的前提同它相对立。它们就是以这样的身分实际上决定着单个资本家等等的行动,成为他们的动机,并作为这样的动机反映在他们的意识之中。庸俗政治经济学无非是以学理主义的形式来表达这种在其动机和观念上都囿于资本主义生产方式的外在表现的意识。而庸俗政治经济学愈是肤浅地抓住现象的表面,仅仅用一定的方式把这种现象的表面复制出来,它就愈觉得自己“合乎自然”,而与任何抽象的空想无关。

[913]前面我们谈到流通过程的地方\fnote{见本册第534—535页。——编者注},还应当指出一点,由流通过程产生的一些规定,结晶为一定种类的资本(固定资本、流动资本等等)的属性,这样也就表现为一定商品在物质上所固有的既定的属性。

如果说在利润表现为资本主义生产的既定前提的这种最终形态中,利润所经历的许多转化和中介过程都消失了,无法认识了,从而资本的性质也消失了,无法认识了;如果说这种形态由于以下的事实而更加固定化:使它得以完成的同一过程,会使利润的一部分作为地租同它相对立,从而使它成为剩余价值的一种特殊形式而同作为特殊物质生产工具的资本发生关系,完全象地租同土地发生关系一样;那末,这种由于大量看不见的中间环节而与自己的内在实质相分离的形态,就会获得更加外表化的形式,或者不如说,就会在生息资本上,在利润和利息的划分上,在作为资本的简单形态(这种形态使资本成为它自己的再生产过程的前提)的生息资本上获得绝对外表化的形式。一方面,这里表现出资本的绝对形式:G—G′,自行增殖的价值。另一方面,甚至在纯粹商业资本中也存在的中间环节,即G—W—G′公式中的W在这里消失了。在G—G′公式中,只有G同它自身的关系,这种关系是用它自身来衡量的。这是绝对地从过程中抽出、脱离过程、处于过程之外的资本,它是这样一个过程的前提,对这个过程来说它又是结果,它只有在这一过程之中并通过这一过程,才成为资本。

{我们把利息可能是单纯的财产转移而不一定表示实际的剩余价值这一点撇开不谈。譬如说,当货币贷给“挥霍者”,也就是说当它用于消费时,利息就不表示实际的剩余价值。但是,当货币被借来用于支付时,情况也会是这样。在这两种场合,货币都是作为货币而不是作为资本贷出的,但是对于货币所有者来说,仅仅由于贷放行为,货币才成为资本。在第二种场合,当票据贴现或以当时卖不出去的商品作抵押进行贷款时,借来供支付用的货币,就能够同资本的流通过程,同商品资本向货币资本的必要的转化过程发生关系。只要这种转化过程的加快——这是在信贷中按照信贷的一般性质会发生的情况——能使再生产的速度加快,也就是使剩余价值的生产速度加快,借入的货币就是资本。但是,如果借入的货币仅仅是用来偿还债务,并不加速再生产过程,甚至可能使再生产过程无法进行或者缩小其规模,那末,这笔货币就只是支付手段,对借款人来说只是货币,而对贷款人来说,则是在实际上不依赖资本过程的资本。在这种场合,利息同“让渡利润”一样,是不依赖资本主义生产本身——不依赖剩余价值的创造——的事实。在货币的这两种形式,即作为获得商品以供消费的购买手段和作为偿还债务的支付手段的形式上,利息完全同“让渡利润”一样,表现为这样一种形式:它虽然是在资本主义生产中再生产出来,却不依赖资本主义生产,属于更早的生产方式。但是资本主义生产的性质就包括这样一点:货币(或商品)能够在生产过程之外成为资本并作为资本出卖,这种情形在货币不转化为资本而只起货币作用的更古老的形式上也会发生。

生息资本的第三种更古老的形式以这样的事实为基础:资本主义生产还不存在,而利润还是以利息的形式被占有,资本家纯粹以高利贷者的身分出现。这包括以下两点:(1)生产者还是独立地利用自己的生产资料进行劳动,而不是生产资料利用生产者进行劳动(虽然奴隶也属于这种生产资料,但是奴隶在这里也同役畜一样,并不形成特殊的经济范畴,或者,最多也只是存在物?上的差别:不会说话的工具;有感觉的、会说话的工具);(2)生产资料只是在名义上属于生产者,也就是说,生产者会由于某些偶然情况而不能用出卖自己商品的所得来再生产这些生产资料。因此,生息资本的这些形式在存在商品流通和货币流通的一切社会形式中都会出现,而不管其中占统治地位的是奴隶劳动、农奴劳动,还是自由劳动。在上述形式的最后一种形式中,生产者以利息的形式向资本家支付自己的剩余劳动,因而这种利息也包含着利润。在这里,有了整个[914]资本主义生产,却没有它的优越性,即没有劳动的社会形式的发展和由这些形式中产生的劳动生产力的发展。这种形式在农民中占决定性的优势,他们的一部分生活资料和生产工具已经必须作为商品来购买,也就是说,除他们外已经有独立的城市工业,此外,他们还必须用货币纳税、交付地租等等。}

生息资本只有在借贷货币实际转化为资本并生产一个余额(利息是其中的一部分)时,才成为生息资本。但这一点并不能排除:利息和生息这种属性,不管有没有[生产]过程,都同生息资本长在一起。同样,下面这样一个事实,即为了实际证明棉花的有用属性,必须把棉花纺成纱或进行其他某种加工,也不能排除棉花作为棉花的使用价值。资本也是这样,只有转入生产过程,才能实际证明自己的生息能力。而劳动能力,也只有当它在过程中作为劳动被使用,被实现时,才表明它有创造价值的能力。这一点并不能排除:劳动能力自身作为一种能力,是创造价值的活动,并且作为这样的活动,它不是从过程中才产生的,而相反地是过程的前提。它是作为创造价值的能力被人购买的。购买它的人也可以不让它去从事劳动(例如,剧院经理有时购买一个演员,并不是为了要他演戏,而是为了使他不能再为自己的竞争者的剧院演戏)。购买劳动能力的人是否利用他支付过报酬的劳动能力的属性,即它创造价值的属性,这与卖者或所卖商品无关,正如购买资本的人是否把这些资本作为资本来使用,也就是说,他是否在过程中使这种资本所固有的创造价值的属性发挥作用,这与卖者或所卖商品无关一样。在这两种场合,他为之支付的东西,是那个就自身来说,在可能性上,就所买商品(在一种场合是劳动能力,另一种场合是资本)的性质来说,已经包含在这两种商品中的剩余价值和它们保存它们自身的价值的能力。因此,用自有资本经营的资本家也把剩余价值的一部分看成利息,即看成这样的剩余价值,它之所以从生产过程产生出来,是因为资本把它带进了生产过程而与这个过程无关。

地租和“土地—地租”关系,可以表现为比利息和“资本—利息”关系更加神秘的形式。但是地租形式上的不合理性,也并不在于表示资本本身的关系。因为土地本身是生产的(就使用价值来说),本身是活的生产力(具有使用价值或用来生产使用价值),所以这里可能有两种见解:或者是迷信地把使用价值同交换价值混淆起来,把物同产品中包含的劳动的某种特殊社会形式混淆起来(于是,不合理性就在自身中为自己找到理由,因为这里地租作为某种特殊的东西同资本主义过程本身没有任何关系),或者是“启蒙的”政治经济学的见解:认为既然地租与劳动无关,也与资本无关,那末地租就根本不是剩余价值的一种形式,而只是价格的附加额,是土地占有的垄断使土地所有者可能获得的附加额。

生息资本的情况却不同。这里涉及的不是某种与资本无关的关系,而是资本主义关系本身,是由资本主义生产产生的、它所特有的、反映资本实质本身的关系,是资本借以表现为资本的那种资本形态。利润仍然包含着对处于过程中的资本的关系,对生产剩余价值、生产利润本身的过程的关系。生息资本的情况与利润不同,在利润上,剩余价值的形态成了某种异化的、离奇的东西,使人不能直接认清剩余价值的简单形态,从而不能认清它的实体和产生的原因;相反,在利息上,这种异化形式却明显地作为本质的东西出现、存在和表现。这种形式作为某种同剩余价值的实际性质相对立的东西独立化并固定化了。在生息资本上,资本同劳动的关系消失了。实际上利息是以利润为前提的,利息只是利润的一部分,剩余价值[915]怎样在利息和利润之间、在不同种类的资本家之间进行分配,这实际上与雇佣工人完全无关。

利息明确地表现为离开资本主义过程本身的、独立于过程的、处于过程之外的资本的果实。它应付给作为资本的资本。它进入生产过程,因而也从生产过程中出来。资本孕育着利息。资本不是从生产过程中得出利息,而是把利息带进生产过程。因此,利润中超过利息的余额,即资本只是靠生产过程得到的、只是作为执行职能的资本生产出来的那个剩余价值量,就获得一种不同于利息(即资本自身、资本本身、作为资本的资本所固有的价值创造)的产业利润这样一种特殊形式(即企业利润,至于是产业利润还是商业利润,那要看重点是在生产过程上还是在流通过程上)。这样一来,连剩余价值的最后一种形式,即在一定程度上还能使人想起其起源的形式,也分离为并被理解为不仅是异化的形式,而且是直接同剩余价值本身相对立的形式,因此,资本和剩余价值的性质,也和一般资本主义生产的性质一样,终于被完全神秘化了。

产业利润,与利息相对立,代表着过程中的资本而与过程外的资本相对立,代表着作为过程的资本而与作为所有权的资本相对立,——因而代表着作为执行职能的资本家、作为劳动资本的代表者的资本家而与只是作为资本的人格化、只是作为资本的所有者的资本家相对立。这样,它就作为劳动资本家而与作为资本家的自身相对立;进而作为劳动者而与只是作为所有者的自己相对立。因此,如果说这里还保存着剩余价值同过程的关系,那末,这恰好是以剩余价值概念本身被否定的形式表现出来的。产业利润被归结为劳动,但不是归结为别人的无酬劳动,而是归结为雇佣劳动,即归结为付给资本家的工资,这样一来,资本家就同雇佣工人落入一个范畴,就不过是一种报酬较高的雇佣工人,正如工资一般就存在着各种差别一样。

实际上,货币转化为资本,并不是由于货币同商品的物质生产条件相交换,也不是由于这些生产条件——劳动材料、劳动资料、劳动——在劳动过程中进入发酵状态、相互作用、相互结合,即进入某种化学过程,并把商品作为这个过程的结晶沉淀下来。如果情况仅仅是这样,那末我们就决不会有资本,决不会有剩余价值。劳动过程的这种抽象形式是一切生产方式所共同的,而不管它们的社会形态和历史规定性如何。这种过程成为资本主义过程,货币转化为资本,只是由于:(1)商品生产,即作为商品的产品生产,是生产的普遍形式;(2)商品(货币)同作为商品的劳动能力(即实际上同劳动)相交换,因而劳动是雇佣劳动;(3)但是后者只有在下述情况下才会发生:客观条件,也就是(就整个生产过程来考察)产品本身,作为独立的力量,作为不是劳动的财产,作为他人的财产,因而按形式来说是作为资本,同劳动相对立。

作为雇佣劳动的劳动和作为资本的劳动条件(从而作为资本家的所有权:它们人格化为资本家,在资本家身上,它们表现为它们本身的所有者,它们代表着资本家对它们的所有权,即它们对本身的所有权而与劳动相对立),是同一种关系的表现,不过是从这种关系的不同的两极出发而已。这种资本主义生产的条件,是资本主义生产的经常的结果。这是资本主义生产本身为它自己提供的前提:资本主义生产本身就是它自己的前提,也就是说,一当它发展起来并在与它相适应的关系中发挥作用时,它就连同它的条件一起被作为前提。但是资本主义生产过程不是一般生产过程;它的各个要素的对抗的社会规定性,只有在过程本身中才能发展和实现,这种规定性是该过程的贯彻始终的特征,并使该过程正好成为这种社会规定的生产方式即资本主义生产过程。

[916]当资本——不是某种特定的资本,而是一般资本——刚一开始形成,它的形成过程就是在它之前的社会生产方式的解体过程和这一生产方式瓦解的产物。因而,这是一个历史过程和属于一定的历史时期的过程。这是资本的历史创始时期。(例如,人的存在是有机生命所经历的前一个过程的结果。只是在这个过程的一定阶段上,人才成为人。但是一旦人已经存在,人,作为人类历史的经常前提,也是人类历史的经常的产物和结果,而人只有作为自己本身的产物和结果才成为前提。)在这里,劳动还只是必须同旧形式的劳动条件分离,而在旧形式下,劳动和劳动条件是一个统一的整体。只有这样,劳动才成为自由劳动,只有这样,劳动条件才转化为与劳动相对立的资本。资本成为资本的过程,或者说,资本在资本主义生产过程本身出现之前的发展过程,和资本在这个过程中的实现,在这里是属于历史上两个不同的时期。在后一时期,资本是前提,它的存在是作为一种自行起作用的东西而成为前提。在前一时期,资本是另一个社会形式解体过程的沉淀物。这里资本是另一个形式的产物,而不是象后来那样,它是它自己再生产的产物。资本主义生产是在雇佣劳动这个资本主义的、现存的、但是同时又是被它不断再生产出来的基础上进行的。因而资本主义生产也是在作为劳动条件的形式、作为资本主义生产的既定前提的资本这个基础上进行的,但是这种前提,也象雇佣劳动一样,是资本主义生产的经常的创造,是它的经常的产物。

在这个基础上例如货币自身就是资本,因为生产条件自身具有与劳动相对立的异化形式,表现为他人的所有权而与劳动相对立,并作为这样的所有权对劳动进行统治。这时资本也可以作为具有这种属性的商品出卖,也就是资本可以作为资本出卖,当资本作为有息贷款贷放时就是这样。

但是,因为资本和资本主义生产的独特的社会规定性的因素——这种独特的社会规定性在法律上通过资本表现为一种所有权,通过资本所有权表现为一种独特的所有权形式——已经固定下来,利息又因此表现为资本在这种规定性上(与作为一般生产过程的规定性的这种规定性无关)生出的剩余价值的一部分,所以很明显,剩余价值的另一部分,即利润中超过利息的余额,即产业利润,就必然表现为这样一种价值,这种价值不是由作为资本的资本生出的,而是由同它的、已经以“资本利息”这个名称取得独特存在方式的社会规定性相分离的生产过程生出的。但是,生产过程同资本相分离,就是一般的劳动过程。因此,同作为资本家的本身相区别的产业资本家,同作为资本家即资本所有者的本身相区别的产业家,不过是劳动过程中单纯的职能执行者,不是执行职能的资本,而是与资本无关的职能执行者,即一般劳动过程的特殊承担者,即劳动者。这样,产业利润就顺利地转化为工资,同普通的工资落入同一个范畴,不同于普通工资的只是数量和支付的特殊形式,也就是资本家自己给自己支付工资,而不是由别人给他支付工资。

在利润分为利息和产业利润这最后一次分裂中,剩余价值的性质(从而资本的性质)不仅完全消失了,而且显然表现为一种完全不同的东西。

利息表示剩余价值的一部分,这不过是在特殊名称下从利润中分出的一个份额,这个份额是给资本的单纯所有者的,是由他夺去的。但是这个单纯的量的分割转化成了质的分割,这种质的分割赋予两个部分一种转化形式,在这种转化形式上,它们的原始实质的痕迹已经看不见了。[917]这种情况得以固定下来,首先是因为利息不是表现为同生产无关的、仅仅在产业家用别人的资本从事经营时才“偶然”发生的分割。即使产业家用自有的资本从事经营,他的利润也会分为利息和产业利润,因此,不管产业家是不是他的资本的所有者这种偶然情况,单纯量的分割已经固定化为质的分割,固定化为由资本本身和资本主义生产本身的性质产生的质的分割。这不仅是在不同的人之间进行分配的利润的两个部分,而且还是利润的两种特殊范畴,它们和资本有不同的关系,也就是说,和资本的不同规定性有关。利润的各部分以独立范畴出现的这种独立化,比较容易地固定下来,原因是(撇开以前已阐明的原因不谈)生息资本作为一种历史形式是出现在产业资本之前,并在它的旧形式上继续同产业资本并存,只是在产业资本的发展进程中才被产业资本作为它本身的一种特殊形式置于从属资本主义生产的地位。

这样,从单纯的量的分割中就产生了质的分裂。资本本身被分裂。只要它是资本主义生产的前提,从而,只要它表示劳动条件的异化形式,表示某种特殊的社会关系,它就在利息上得到实现。它在利息上实现它的作为资本的性质。另一方面,只要它在过程中执行职能,这个过程就表现为脱离自己的特殊资本主义性质,脱离自己的特殊社会规定性的过程——表现为单纯的一般劳动过程。因此,只要资本家参加劳动过程,他就不是作为资本家来参加(因为他的这个性质体现在利息中),而是作为一般劳动过程的职能执行者,作为劳动者来参加,他的工资就表现为产业利润。这是一种特殊的劳动方式——管理的劳动,而劳动方式一般来说是彼此各不相同的。

这样,在这两种剩余价值形式上,剩余价值的性质、资本的实质以及资本主义生产的性质,不仅完全消失了,而且转到了自己的反面。但是,由于物的主体化、主体的物化、因果的颠倒、宗教般的概念混淆、资本的单纯形式G—G′在这里被荒诞地、不经过任何中介过程地展示和表现出来,资本的性质和形态也就完成了。同样,各种关系的硬化以及它们表现为人同具有一定社会性质的物的关系,在这里也以完全不同于商品的简单神秘化和货币的已经比较复杂的神秘化的方式表达出来了。变体和拜物教在这里彻底完成了。

利息自身正好表现出,劳动条件作为资本而存在,同劳动处于社会对立中,并且转化为同劳动相对立并且支配着劳动的私人权力。利息概括了劳动条件对主体活动的关系上的异化性质。利息把资本的所有权,或者说单纯的资本所有权,表现为占有别人劳动产品的手段,表现为支配别人劳动的权力。但是,它是把资本的这种性质表现为某种在生产过程本身之外属于资本的东西,而不是表现为这个生产过程本身的独特的规定性的结果。它不是把资本的这种性质表现为同劳动对立,而是相反地同劳动无关,只是表现为一个资本家对另一个资本家的关系,也就是说,表现为一种存在于资本对劳动本身的关系之外的、与这种关系无关的规定性。利润在资本家之间的分配,与工人本身毫无关系。因此,在利息上,在利润的这个形态上,资本的对立性质固然得到了特殊的表现,但是表现成这样:这种对立在其中已经完全消失,而且明显地被抽掉了。利息除了表现货币、商品等等增殖自己价值的能力以外,还把剩余价值表现为从货币和商品中生长出来的某种东西,表现为它们的自然果实,也就是说,利息不过是资本的神秘化在最极端的形式上的表现,——只要它一般表现社会关系本身,它表现的[918]就只是资本家之间的关系,而决不是资本与劳动之间的关系。

另一方面,这个利息形式又使利润的另一部分取得产业利润这种质的形式,即产业资本家——不是作为资本家而是作为劳动者(产业家)——的劳动工资形式。资本家作为资本家在劳动过程中所要完成的、恰好使他同工人相区别的特殊职能,被表现为单纯的劳动职能。他创造剩余价值,不是因为他作为资本家进行劳动,而是因为他,即资本家,也进行劳动。这好比一个国王,他作为国王在名义上指挥着军队,于是有人就说,不是因为他作为王位所有者进行指挥,起着统帅的作用,他才指挥军队,而是因为他指挥军队,执行统帅的职能,所以他才是国王。因此,如果说,剩余价值的一部分在利息的形式上完全同剥削过程相分离,那末另一部分在产业利润的形式上就表现为剥削过程的直接的对立面,即不是对别人劳动的占有,而是自己劳动的价值创造。因此,剩余价值的这一部分也就不再是剩余价值,而是一种和剩余价值相反的东西,是所完成的劳动的等价物。因为资本的异化性质,它同劳动的对立,处于剥削过程之外,处于这种异化的现实行动范围之外,所以一切对立性质也就从这个过程本身排除了。因此,现实的剥削,即实现并实际表现对立性质的东西,就表现为它的直接对立面,表现为一种在物质上是独特的、但是属于劳动的同一社会规定性的劳动,即雇佣劳动,即属于同一劳动范畴的劳动。在这里剥削的劳动和被剥削的劳动被等同起来了。

利润的一部分转化为产业利润,正如我们看到的,是由利润的另一部分转化为利息引起的。与利润的一部分相适应的是资本的社会形式,即资本是所有权;与利润的另一部分相适应的是资本的经济职能,即资本在劳动过程中的职能,不过这种职能已摆脱并抽掉了使资本得以执行这种职能的社会形式,即对立形式。至于有人怎样用聪明的理由进一步为这一点作辩护,我们将在分析把利润解释为“监督劳动”的报酬的辩护论观点时作更详细的考察。在这里人们把资本家和他的经理混同起来了,这一点斯密已经指出过\endnote{马克思指的是斯密《国富论》第一篇第六章。——第550页。}。

当然,产业利润中也包含一点属于工资的东西(在不存在领取这种工资的经理的地方)。资本家在生产过程中是作为劳动的管理者和指挥者(captainofindustry)出现的,在这个意义上说,资本家在劳动过程本身中起着积极作用。但是只要这些职能是产生于资本主义生产的特殊形式,(也就是说,产生于资本对作为它的劳动的劳动的统治,从而对作为它的工具的工人的统治;产生于作为社会的统一体,作为在资本上人格化为支配劳动的权力的社会劳动形式的主体而表现出来的资本的性质),那末,这种与剥削相结合的劳动(这种劳动也可以转给经理)当然就与雇佣工人的劳动一样,是一种加入产品价值的劳动,正如在奴隶制下奴隶监工的劳动,也必须和劳动者本人的劳动一样给予报酬。如果一个人把他对自己的本性、对外部自然界以及对其他人的关系以宗教形式想象成某些独立存在的力量,以致被这些想象所统治,那末,他就需要祭司和祭司的劳动。但是随着意识的宗教形式以及与此相联系的关系的消失,这种祭司的劳动也就不再进入社会生产过程。祭司的劳动与祭司一起消失了,而资本家作为资本家所完成的或他委托别人完成的劳动,也会与资本家一起同样消失。(奴隶制的例子用几段引文加以说明。\endnote{马克思在两三年后写的《资本论》第三卷第二十三章里列举了关于奴隶监工的引文。——第551页。})

可是,把利润归结为作为监督劳动的报酬的工资这一辩护论观点,本身又转过来反对辩护士;因为英国[919]社会主义者曾以充分的理由回答说:很好,以后你们就只应拿普通经理的工资;你们的产业利润不仅在口头上而且在实际上都应归结为监督或管理劳动的工资。

{自然,不可能详细地研究辩护士的这些愚蠢的废话及其种种矛盾。例如,产业利润的提高和下降不论同利息还是同地租都成反比。但是对劳动的监督,即资本家实际完成的一定量劳动,却与此无关,就象与工资的下降无关一样。这种工资的特点正是:它的下降和提高同实际的工资成反比(在利润率由剩余价值率决定的情况下,如果全部生产条件保持不变,它就完全由剩余价值率决定)。但是诸如此类的“小矛盾”并没有消除持辩护论观点的庸俗经济学家头脑中的等同性。不管资本家付出的工资是少还是多,不管工人得到的工资是较高还是较低,资本家完成的劳动都绝对地保持不变。(正如按一个工作日支付的工资丝毫不改变劳动本身的量一样。)不仅如此。工人的工资较高时,他的劳动强度就较大。相反,资本家的劳动则是个完全确定的东西:它在质上和量上都是由资本家应管理的劳动的量决定,而不是由这一劳动量的报酬决定。资本家不会强化自己的劳动,正如工人不会加工出多于他在工厂中得到的棉花一样。}

英国社会主义者接着还说:管理劳动和监督劳动也同其他任何劳动能力一样,现在可以在市场上购买,并且可以同样比较便宜地生产出来,因而可以同样比较便宜地买到。资本主义生产本身已经使那种完全同资本所有权(不管是自有的资本还是别人的资本)分离的管理劳动比比皆是。因此,这种管理劳动就完全无需资本家亲自担任了。这种劳动实际上是同资本分离而存在的,但这不是表现在产业资本家同货币资本家那种表面上的分离上,而是表现在产业管理人员等等同各种资本家的分离上。最好的证明就是:第一,工人们自己创办的合作工厂\fnote{参看本册第392页。——编者注}。它们提供了一个实例,证明资本家作为生产上的职能执行者对工人来说已经成为多余的了,就象在资本家本人看来,土地所有者的职能对资产阶级的生产是多余的一样。第二,只要资本家的劳动不是由作为资本主义过程的那种[生产]过程引起,因而这种劳动并不随着资本的消失而自行消失;只要这种劳动不是剥削别人劳动的职能的名称,也就是说,只要这种劳动是由劳动的社会形式(协作、分工等等)引起,它就同资本完全无关,就象这个形式本身一旦把资本主义的外壳剥去,就同资本完全无关一样。说这种劳动作为资本主义的劳动,作为资本家的职能是必要的,这无非就是说,庸俗经济学家不能设想在资本内部发展起来的劳动的社会生产力和劳动的社会性质,能够脱离它们的这种资本主义形式,脱离它们的各因素的异化、对立和矛盾的形式,脱离它们的颠倒和混乱。而这正是我们所要坚持的。[XV—919]

\centerbox{※     ※     ※}

[XVIII—1142]{资本家的实际利润,有很大一部分是“让渡利润”,而且资本家的“个人劳动”在不是涉及剩余价值的创造,而是涉及整个资本家阶级的总利润通过商业途径在其各个成员之间进行分配的场合,有着特别广阔的活动余地。这一点在这里与我们无关。某些种类的利润——例如,以投机为基础的利润——只有在这种场所才能获得。因此,这里就不去考察这些利润了。庸俗政治经济学(特别是为了把利润说成是“工资”)把这种“让渡利润”同来源于剩余价值的创造的利润混为一谈,这表明庸俗政治经济学象畜生一样愚蠢。例如,请看看可敬的罗雪尔。因此,对于这类蠢驴来说,他们把分配整个资本家阶级的总利润时不同生产领域的资本家在计算上的考虑和补偿的理由,同资本家剥削工人的理由,同所谓的利润本身的来由混为一谈,这也是十分自然的。}[XVIII—1142]

\tchapternonum{[(5)古典政治经济学和庸俗政治经济学的本质区别。利息和地租是商品市场价格的构成要素。庸俗经济学家企图赋予利息和地租的不合理形式以合理的外观]}

[XV—919]在生息资本上,——由于利润分为利息和[产业]利润,——资本取得了它的最彻底的物的形式,它的纯粹的拜物教形式,剩余价值的性质表现为一种完全丧失了它自身的东西。正象土地表现为地租的源泉,劳动表现为工资(部分是真正的工资,部分是产业利润)的源泉一样,资本——作为物——在这里表现为价值的独立的源泉,表现为价值的创造者。诚然,这种观点的代表者始终认为,商品的价格应当支付工资、利息和地租,但它支付它们是因为加入商品的土地创造地租,加入商品的资本创造利息,加入商品的劳动创造工资;是因为它们创造了落入它们各自的所有者或代表[920]——土地所有者、资本家和劳动者(雇佣工人和产业家)——手里的这几部分价值。因此,从这个观点来看,说一方面商品的价格决定工资、地租和利息,另一方面利息、地租和工资的价格决定商品的价格,在理论上也没有什么矛盾,或者说,如果有矛盾,那也是价格的实际运动的矛盾或循环论证。

不错,利率会波动,但它只是和其他任何商品市场价格的波动一样,取决于供求关系。这不会使利息不再成为资本内在的东西,就象商品价格的波动不会使价格不再成为商品固有的规定一样。

因此,一方面,只要土地、资本和劳动被看作地租、利息和工资的源泉,而地租、利息和工资被看作商品价格的构成要素,土地、资本和劳动就表现为创造价值的要素;另一方面,只要它们归于每一种生产价值的工具的所有者,并把它们创造的那部分产品价值归于他,它们就表现为收入的源泉,而地租、利息和工资的形式则表现为分配形式。(庸俗经济学家把分配形式实际上只当作从另一角度看的生产形式,而批判的经济学家却把它们彼此分开,并且否认它们的同一性,这一点表明,正如我们以后将看到的,和批判的政治经济学比较起来,庸俗经济学家真是愚蠢透顶。)

在生息资本上,资本表现为它作为货币或商品所具有的价值或剩余价值的独立源泉。而且它是在本身,在自己的物的形式上成为这样的源泉的。诚然,资本为了实现它的这种属性必须加入生产过程,但是土地或劳动也必须这样做。

因此,很明显,为什么庸俗政治经济学宁愿采取“土地—地租,资本—利息,劳动—工资”这样的公式,而不愿采取斯密等人用来说明价格要素(更确切地说,价格分解成的各部分)的公式,在这一公式里出现的是“资本—利润”的关系,所有的古典经济学家一般都用这种关系来说明资本关系本身。在利润中还包含着同[生产]过程的[使庸俗政治经济学]感到为难的联系,剩余价值和资本主义生产的真正性质(和它们的外部表现不同)还多少可以辨认。当利息被说成是资本的真正产物,从而剩余价值的另一部分即产业利润完全消失并归入工资范畴时,情况就不再是如此了。

古典政治经济学力求通过分析,把各种固定的和彼此异化的财富形式还原为它们的内在的统一性,并从它们身上剥去那种使它们漠不相关地相互并存的形式;它想了解与表现形式的多样性不同的内在联系。因此,它把地租还原为超额利润,这样,地租就不再作为特殊的,独立的形式而存在,就和它的虚假的源泉即土地分离开来。它同样剥去了利息的独立形式,证明它是利润的一部分。于是,它把非劳动者借以从商品价值中获取份额的一切收入形式,一切独立的形式或名义都还原为利润这一种形式。但是利润归结为剩余价值,因为全部商品的价值都归结为劳动;商品中包含的有酬劳动量归结为工资;因此,超过这一数量的余额归结为无酬劳动,归结为在各种名义下被无偿地占有的、然而是由资本引起的剩余劳动。在进行这种分析的时候,古典政治经济学有时也陷入矛盾;它往往试图不揭示中介环节就直接进行这种还原和证明不同形式的源泉的同一性。但这是它的分析方法的必然结果,[921]批判和理解必须从这一方法开始。它感兴趣的不是从起源来说明各种不同的形式,而是通过分析来把它们还原为它们的统一性,因为它是从把它们作为已知的前提出发的。但是,分析是说明起源,理解实际形成过程的不同阶段的必要前提。最后,古典政治经济学的缺点和错误是:它把资本的基本形式,即以占有别人劳动为目的的生产,不是解释为社会生产的历史形式,而是解释为社会生产的自然形式,不过它自己已通过它的分析开辟了一条消除这种解释的道路。

庸俗政治经济学的情况就完全不同了,正当政治经济学本身由于它的分析而使它自己的前提瓦解、动摇的时候,正当政治经济学的对立面也已经因此而多少以经济的、空想的、批判的和革命的形式存在的时候,庸俗政治经济学开始嚣张起来。因为政治经济学和由它自身产生的对立面的发展,是同资本主义生产固有的社会矛盾以及阶级斗争的现实发展齐头并进的。只是在政治经济学达到一定的发展程度(即在亚·斯密以后)和形成稳固的形式时,政治经济学中的一个因素,即作为现象观念的单纯的现象复写,即它的庸俗因素,才作为政治经济学的特殊表现形式从中分离出来。例如萨伊就把亚·斯密著作中这里或那里渗透的庸俗观念分离出来,并作为特殊的结晶和亚·斯密并存。随着李嘉图的出现和由他引起的政治经济学的进一步发展,庸俗经济学家也得到了新的营养(因为他自己什么也不生产),政治经济学越是接近它的完成,也就是说它越是走向深入和发展成为对立的体系,它自身的庸俗因素,由于用它按照自己的方法准备的材料把自己充实起来,就越是独立地和它相对立,直到最后在学术上的混合主义和无原则的折衷主义的编纂中找到了自己至上的表现。

随着政治经济学的深入发展,它不仅自己表现出矛盾和对立,而且它自身的对立面,也随着社会经济生活中的现实矛盾的发展而出现在它的面前。与这种情况相适应,庸俗政治经济学也就有意识地越来越成为辩护论的经济学,并且千方百计力图通过空谈来摆脱反映矛盾的思想。因此,萨伊同例如巴师夏比较起来还算是一个批评家,还算无所偏袒,因为他在斯密的著作里发现的矛盾相对说来还是未发展的,而巴师夏却是一个职业的调和论者和辩护论者,虽然他不仅在李嘉图的政治经济学中发现了经济学本身在内部已经形成的矛盾,而且发现了在社会主义和当时日常的阶级斗争中正在形成的矛盾。再加上,庸俗政治经济学在其较早的发展阶段,找到的材料还没有完全加工好,因此它本身在参与解决经济问题的时候还或多或少地从政治经济学的观点出发,例如萨伊就是这样,而那位巴师夏却只有剽窃,并且力图用自己的论据把古典政治经济学中不合口味的方面消除掉。

但巴师夏还不代表最后的阶段。他还有一个特点,这就是学识贫乏,对于他为了统治阶级的利益而加以粉饰的那门科学的认识十分肤浅。他搞辩护论还是很热情的,这是他的真正的工作,因为政治经济学的内容,只要是合他心意的,他可以从别人那里取来。最后的形式是教授形式,这种形式是“从历史的角度”进行工作的,并且以明智的中庸态度到处搜集“最好的东西”,如果得到的结果是矛盾,这对它说来并不重要,只有完备才是重要的。这就是阉割[922]一切体系,抹去它们的一切棱角,使它们在一本摘录集里和平相处。在这里,辩护论的热忱被渊博的学问所抑制,这种渊博的学问宽厚地俯视着经济思想家的夸张的议论,而只是让这些议论作为稀罕的奇物漂浮在它的内容贫乏的稀粥里。因为这类著作只有在政治经济学作为科学已走完了它的道路的时候才会出现,所以它们同时也就是这门科学的坟墓。(至于它们完全以同样的方式超然耸立于社会主义者的空想之上,那就不用说了。)甚至斯密、李嘉图和其他人的真正的思想(不仅是他们本身的庸俗因素)在这里也好象是毫无内容,变成了庸俗的东西。罗雪尔教授先生就是这样的大师,他谦虚地宣称自己是政治经济学的修昔的底斯。\endnote{罗雪尔在他的著作《国民经济学原理》(1854年)第一版序言中,不知羞耻地引证了修昔的底斯。——第558页。}他把自己比作修昔的底斯,可能是因为他对修昔的底斯有这样一种看法,即修昔的底斯似乎经常把原因和结果相混淆。

诚然,资本不花费任何劳动就占有别人的劳动成果这一事实,非常明显地表现在生息资本的形式上:因为在这里资本以它借以与生产过程本身脱离的形式表现出来。但是在这个形式上,资本所以能够这样,只是因为它本身实际上并不花费任何劳动,而是作为自行创造价值的、成为价值源泉的要素加入劳动过程。如果说生息资本不花费任何劳动便占有一部分产品价值,那末它不花费任何劳动也创造了这部分价值,由自身、由自身内部创造了这部分价值。

异化形式使古典的,因而也使批判的政治经济学家感到困难,他们试图通过分析来剥去这种形式,可是庸俗政治经济学却正好是在产品价值的各个不同部分相互对立的异化中第一次感到十分自在:正如一个经院哲学家在谈到“圣父、圣子和圣灵”这一公式时感到十分自在一样,庸俗经济学家在谈到“土地—地租,资本—利息,劳动—工资”这一公式时也感到十分自在。因为这正是这样一种形式,在这种形式中,这些关系在现象上似乎直接相互联系着,因而也在受这种生产方式束缚的资本主义生产当事人的观念和意识中存在着。庸俗政治经济学认为它越是实际上仅仅从事于把普通观念译成学理主义的语言,它就越是单纯、合乎自然和对公众有益,就和一切理论上的吹毛求疵离得越远。因此,它越是在异化的形式上来认识资本主义生产的各种形态,它就越是接近于普通观念的要素,也就是越在它自己的自然要素中浮游。

此外,这给辩护论帮了很大的忙。因为,例如在“土地—地租,资本—利息,劳动—工资”这一公式中,剩余价值的各种不同形式和资本主义生产的各种不同形态,不是作为异化形式相互对立,而是作为相异的和彼此无关的形式、作为只是彼此不同但无对抗性的形式相互对立。不同的收入来自完全不同的源泉,一个来自土地,另一个来自资本,第三个来自劳动。因此,它们不是处于相互敌对的关系,因为它们根本没有任何内在联系。如果说它们还是在生产上共同起作用,那末,这是一种协调的动作,是协调的表现;这好比农民、牛、犁和土地,尽管它们彼此不同,但它们却在农业中,在实际的劳动过程中协调地共同劳动。如果它们之间发生了对抗,那末,这种对抗只是由于生产当事人中谁应当从产品,从它们共同创造的价值中多占一些而引起的竞争造成的。如果有时会发展到冲突,那末,土地、资本和劳动之间这一竞争的最后结果终归还是这样:在它们[923]对分割的争执过程中,它们由于竞争而大大增加了产品的价值,以致每一个都获得了更大的一份,所以它们的竞争本身只是刺激所有生产当事人的协调的表现。

例如阿伦德先生批评劳说:

\begin{quote}{“作者受他的某些前辈的影响,把企业主的收入作为第四种要素和国民财富的三种要素(工资、资本的租金和地租)并列;这样,由亚·斯密如此谨慎地建立起来的、我们的科学〈!〉的任何进一步发展的整个基础被破坏了,因此,在我们的作者的著作里根本没有考虑这种发展。”(卡尔·阿伦德《与垄断精神及共产主义相对立的合乎自然的国民经济学,附与本书有关的资料的评述》1845年哈瑙版第477页)}\end{quote}

阿伦德先生把“资本的租金”理解为利息(同上,第123页)。如果有人不相信亚·斯密把国民财富归结为资本利息、地租和工资呢?(因为斯密正好相反,明确指出利润是资本的价值增殖,并且不止一次地明白指出,利息由于一般说来代表剩余价值,始终只是从利润中派生的形式。)在这种情况下,庸俗经济学家读到斯密所提到的源泉时就读出了直接与其含义相对立的东西。斯密写“利润”的地方,阿伦德读成“利息”。那末,他把亚·斯密的“利息”理解为什么呢?

正是这一位“我们的科学”的“谨慎的”发展者作出了以下有趣的发现:

\begin{quote}{“在财物生产的自然进程中,只有一个现象,在已经充分开发的国家,看来在一定程度内负有调节利息率的使命;那就是欧洲森林的树木总量由于树木的逐年增长而增加的比率。这种增长完全不以树木的交换价值为转移〈说树木的增长“不以树木的交换价值为转移”,这是多么滑稽啊!〉,而按每一百棵增加三棵到四棵的比率来进行。因此〈也就是因为,树木的交换价值虽然在很大程度上要取决于树木的增长,但树木的增长“完全不以树木的交换价值为转移”!〉,不能指望它〈利息率〉会下降到最富有货币的国家的现有水平以下。”(同上,第124—125页)}\end{quote}

这种利息率应当称为“原始的森林利息率”。这种利息率的发现者在所引著作中,又作为“犬税”\endnote{阿伦德在自己的著作中用专门的一节(第88节第420—421页)论证了犬税的正确性和合理性。——第561页。}哲学家在“我们的科学”领域里引人注目。

\centerbox{※     ※     ※}

{利润(其中也包括产业利润)和预付资本的量成比例;相反,产业资本家取得的“工资”和资本的量成反比:资本小的时候,它就大(因为在这里资本家是介于别人劳动的剥削者和靠自己劳动生活的劳动者之间的中间人物),资本大的时候,它就很微小,或者象在有经理的情况下,它就完全和利润分离。一部分管理劳动只是由资本和劳动之间的敌对性、由资本主义生产的对抗性引起的,它完全和流通过程引起的9/10的“劳动”一样,属于资本主义生产上的非生产费用\fnote{不直接参加生产过程,但在一定条件下又非有不可的辅助费用。——编者注}。一个乐队指挥完全不必就是乐队的乐器的所有者,用乐队队员的生活费用搞投机,也不是他这个乐队指挥职能范围以内的事情,他和他们的“工资”根本没有任何关系。非常奇怪,象约翰·斯图亚特·穆勒这样一些为了把“产业利润”变为监督劳动的工资而坚持“利息”、“产业利润”等形式的经济学家,却和斯密、李嘉图以及一切值得一提的经济学家一起,认为平均利率即平均利息率是由平均利润率决定的,照穆勒的说法,这种平均利润率和工资率成反比,因此它无非是无酬劳动,剩余劳动。

监督工资根本不加入平均利润率,以下两个事实是最好的证明:

[924](1)合作工厂\fnote{见本册第392和552页。——编者注}和其他一切工厂一样,那里的经理是有报酬的,并完成全部管理劳动,那里的监工本身只是劳动者,在这样的工厂里,利润率不是低于而是高于平均利润率;

(2)在某些特殊的、非垄断的行业,例如在小店主、农场主等等那里,利润经常大大高于平均利润率,对于这种情况,经济学家们公正地解释说,这是由于这些人自己给自己支付工资。如果这样的人独自一人劳动,他的利润就由(1)他的小额资本的利息、(2)他的工资、(3)由于他的资本而使他能够为自己而不是为别人劳动的那部分剩余时间,即已经不表现为利息的那部分剩余时间所构成。如果他雇用工人,那末其中便包括工人的剩余时间。

可尊敬的西尼耳(纳骚)自然也把产业利润变成监督工资。但是一当问题不涉及学理主义的语句而涉及工人和厂主之间的实际斗争时,他便忘记了这些诡辩。这时他就,例如,反对限制劳动时间,因为,照他的说法,例如工人每天在11+(1/2)小时内只为资本家劳动一小时,只有这一小时的产品构成资本家的利润(利息除外,照他的计算,工人还要为补偿利息劳动一小时)。因此,在这里产业利润突然变成不等于资本家的劳动在生产过程中加进商品的价值,而等于工人的无酬劳动时间加进商品的价值。如果产业利润是资本家自己劳动的产物,西尼耳就必然不会抱怨工人只白白地劳动一小时而不是两小时,而且更不会说,如果工人只劳动10+(1/2)小时而不是11+(1/2)小时,就完全不会有利润;他必然会说,如果工人只劳动10+(1/2)小时,而不是11+(1/2)小时,资本家得到的就只是10+(1/2)小时的监督工资,而不是11+(1/2)小时的监督工资,也就是说他丧失了一小时的监督工资,对于这一点工人会回答他说,如果对他们来说,10+(1/2)小时的普通工资就已经够了,那末对资本家来说,10+(1/2)小时的较高工资也应该够了。

很难理解,约翰·斯图亚特·穆勒这样一些属于李嘉图学派的经济学家,他们甚至把利润仅仅等于剩余价值即剩余劳动这一论点表述为:利润率和工资成反比,工资率决定利润率(这样说是不对的),可是,他们怎么竟突然把产业利润不是变成工人的剩余劳动,而是变成资本家自己的劳动,——只有他们把剥削别人劳动的职能称为劳动,那才的确会出现这样的结果:这种劳动的工资恰好等于被占有的别人劳动的量,或者说,这种劳动的工资直接取决于剥削程度,而不是取决于资本家为这种剥削所作出的那种努力的程度。(在资本主义生产中,这种剥削劳动的职能要求实际的劳动,就这方面说,这种职能表现为经理的工资。)我再说一遍,很难理解,这些经济学家,在他们(作为李嘉图学派)把利润归结为它的实际要素之后,怎么又由于把利息和产业利润对立起来而陷入谬误,产业利润只是利润的伪装形式,把产业利润理解为一种独立形式是由于对利润的实质无知。利润的一部分所以表现为产业利润,表现为从过程中的活动(从真正的活动过程,但其中同时也包括执行职能的资本家的活动)产生的,因而表现为资本家的劳动所应得的部分,只是因为另一部分即利息表现为资本作为与过程无关的、自动的、自行创造的物所应得的部分。也就是说,是因为资本和由其产生的剩余价值在利息的名称下被说成是某种神秘的东西。这种纯粹来自表象的、反映资本表面的最外表的形态的见解是和李嘉图的见解直接对立的,并且完全和他对价值的理解相矛盾。就资本是价值来说,资本的价值决定于早在这个资本加入过程以前就包含的劳动。就资本作为物加入过程来说,它是作为使用价值加入过程的,而作为使用价值,不管它的效用如何,它绝不能创造交换价值。由此可以看出,李嘉图学派对他们自己的老师的了解有多妙。同货币资本家相对来说,产业家是执行职能的资本家,因而是实际榨取剩余劳动的,他把这种剩余劳动的一部分装进自己的腰包,当然是完全正确的。同货币资本家相对来说,他是劳动者,不过是作为资本家的劳动者,即作为别人劳动的剥削者的劳动者。[925]同工人相对来说,这样一个论据,即认为剥削工人的劳动要花费资本家的劳动,因此工人还必须为这种剥削付给他工资,就是可笑的。这是奴隶监工用来对付奴隶的论据。}

\centerbox{※     ※     ※}

社会生产过程的任何前提同时也是它的结果,而它的任何结果同时又表现为前提。因此,生产过程借以运动的一切生产关系既是它的条件,同样也是它的产物。我们越是在这一过程的实际外部表现上来考察这一过程,它的形态就越是在条件的形式上固定下来,以致这些条件似乎是不取决于它但对它起决定作用的东西,而过程参加者们本身的关系对他们来说表现为物的条件、物的力量、物的规定性,尤其是在资本主义过程中,任何要素,甚至最简单的要素,例如商品,都已经是一种颠倒,并已使人与人之间的关系表现为物的属性,表现为人与这些物的社会属性的关系。

\begin{quote}{{“利息是对生产地使用积蓄的报酬;真正意义上的利润是对这种生产地使用期间进行的监督活动的报酬。”(《韦斯明斯特评论》\endnote{《韦斯明斯特评论》(《TheWestminsterReview》)——英国资产阶级自由派杂志,1824年至1914年在伦敦出版,每年出四期。——第564页。}1826年1月第107—108页)}\end{quote}

可见,在这里,利息是对货币等等作为资本使用的报酬;所以它来自资本本身,资本由于自己的资本属性而得到报酬。而产业利润是对“这种生产地使用期间”即生产过程本身中的资本或资本家的职能的报酬。}[925]

[925]利息只是产业的、执行职能的资本家付给资本所有者的一部分利润。因为前者只是由于有资本(货币、商品)等等才能占有剩余劳动,所以他支付一部分给向他提供这种手段的人。如果资本的所有者希望享受他的货币作为资本的利益而又不让他的货币执行资本的职能,那末他只有在满足于一部分利润的条件下才能这样做。他们实际上是伙伴:一个是法律上的资本所有者,另一个,当他使用资本的时候,是经济上的资本所有者。但是,因为利润只是来自生产过程,只是生产过程的结果,还有待生产出来,所以利息实际上不过是对于待完成的剩余劳动的一部分的要求权,对未来劳动的要求权,对还不存在的商品价值的一部分的要求权,因此,不过是在一段时间内(到这段时间终了,利息才能得到支付)所进行的生产过程的结果。

[926]资本在它被支付以前先被购买(即凭利息借入)。货币在这里象在购买劳动能力等等的情况下一样,执行支付手段的职能。因此,资本的价格(利息)加入产业家的预付(如果他用自己的资本经营,就是加入自己本身的预付),就象棉花的价格加入产业家的预付一样,棉花例如也是今天买进,要过比如说六个星期才得到支付。利率(货币的市场价格)的波动也和其他商品的市场价格的波动一样,在这里不会使事情发生变化。相反,货币的市场价格(这是作为货币资本的生息资本的名称)在货币市场上正象其他一切商品的市场价格一样,是由买者和卖者之间的竞争,是由需求和供给决定的。货币资本家和产业资本家之间的这种斗争只是分割利润的斗争,即在分割时双方为各自应得的份额而进行的斗争。关系本身(需求和供给)和它的两极中的任何一极一样,也是生产过程的结果,或者用普通的话来说,是由当时的营业状况,即再生产过程及其要素在当时所处的状况[决定的]。但是从形式上和从外部表现来看,早在资本加入再生产以前,这一斗争就已决定资本的价格(利息)。同时这种决定是在真正的生产过程以外进行的,由与这一过程无关的情况所调节,而且价格的这种决定表现为生产过程必须借以进行的条件之一。因此,这一斗争看来不仅确定对未来利润的一定部分的所有权,而且使这一部分本身不是作为结果从生产过程中产生出来,而是作为前提,作为资本的价格加入生产过程,完全和商品价格或工资作为前提加入生产过程一样,虽然它实际上不断——在再生产的过程中——从生产过程中产生出来。商品价格中作为预付出现并作为已经存在的商品价格加入生产价格的一切要素,在产业资本家看来已不再是剩余价值。因此,作为资本价格加入过程的那一部分利润列入预付的费用,不再表现为剩余价值,并从过程的产物变成它的既定的前提之一,变成生产条件,这种条件本身以独立的形式加入过程,并决定过程的结果。

(例如,如果利率下降,而市场状况要求把商品的价格降到它们的费用价格以下,那末,产业家就能够在不降低产业利润率的情况下降低商品价格;他甚至能够降低自己的商品价格并获得较高的产业利润,当然,在靠自有资本经营的人看来,这是利润率的下降,是总利润的下降。一切表现为既定的生产条件的东西,即商品、工资、资本的价格,也就是这些要素的市场价格,又会反过来对当时的商品市场价格产生决定性的影响,而单个商品的实际费用价格只是在市场价格的波动中为自己开辟道路,它只是这些市场价格的自行平均化,完全和商品的价值只是在所有各种不同商品的费用价格的平均化中为自己开辟道路一样。因此,庸俗观点的代表者——无论他是资本主义思想的理论家还是实践的资本家——的循环论证:商品价格决定工资、利息、利润和地租,反过来,劳动、利息、利润和地租等的价格又决定商品的价格,——只是一种循环运动的表现,在实际运动中和在现象的表面上普遍规律就是通过这种循环运动以矛盾的方式实现的。)

于是,剩余价值的一部分,利息,就表现为加入过程的资本的市场价格,因此它不是表现为剩余价值,而是表现为生产条件。因此,剩余价值在两类资本家(处在过程外的和处在过程内的)之间进行分割这种情况,表现为剩余价值的一部分应付给处在过程外的资本家,而另一部分则应付给处在过程内的资本家。分割的预先确定表现为一部分不依赖于另一部分;一部分不依赖于过程本身;最后,表现为某种物、货币、商品(不过这些物是作为资本)的内在属性,这又似乎不是某种关系的表现,而是这些货币、这种商品在工艺上是为劳动过程规定的;由于这种规定,它们就成为资本;有了这种规定,它们就是劳动过程本身的简单要素,[927]这些要素本身也就是资本。

商品的价值,部分分解为该商品所包含的各种商品的价值,部分分解为劳动的价值,即有酬劳动,部分分解为无酬的、然而是可出卖的劳动;商品中由无酬劳动构成的那一部分价值,即商品中包含的剩余价值,又分解为利息、产业利润和地租,就是说,这一总剩余价值的直接占有者和“生产者”不得不把总剩余价值中的一部分交给土地所有者,另一部分交给资本所有者,结果他给自己留下的总剩余价值中的第三部分,就在产业利润这个不同于利息和地租、也不同于剩余价值本身和利润本身的名称下留给了自己。以上这种情况是没有什么神秘的。剩余价值,即商品价值的一定部分,分解为这些特殊项目或类别是完全可以理解的,根本不会和价值本身的规律发生矛盾。但是,由于剩余价值的这些不同部分取得了独立的形式,由于它们归属于不同的人,由于对它们的要求权所依据的要素不同,最后,由于这些不同部分作为过程的条件借以和过程相对立的那种独立性,上述一切都被神秘化了。它们从价值可以被分解成的那些部分,变为构成价值的独立要素,变为构成要素。它们对市场价格说来就是这样。它们实际上成了市场价格的构成要素。它们作为过程条件的这种表面的独立性又怎样由内在的规律所调节,因而它们只是一种表面上独立的东西,——这一点在生产过程的任何时刻都不会明显地表现出来,也不会作为决定性的、有意识的动机起作用。正好相反。过程的结果借以表现为过程的独立条件的这种外观,当剩余价值的各部分(作为生产条件的价格)加入商品价格的时候,就获得了最大程度的固定性。

利息和地租的情况就是这样。它们属于工业资本家和租地农场主的预付。它们在这里似乎已经不再是无酬剩余劳动的表现,而是有酬剩余劳动即在生产过程中为其支付了等价物的那种剩余劳动的表现,诚然这种等价物不是支付给工人(这种剩余劳动就是工人的剩余劳动),而是支付给其他人——资本所有者和土地所有者。利息和地租就它们对工人的关系来说是剩余劳动,但是就它们对它们应被付给的[货币]资本家和土地所有者的关系来说是等价物。因此它们不是表现为剩余价值,更不是表现为剩余劳动,而是表现为“资本”这种商品和“土地”这种商品的价格,因为它们被付给只是作为商品所有者、只是作为这些商品的所有者和卖者的[货币]资本家和土地所有者。因此,商品价值中归结为利息的部分表现为为资本支付的价格的再生产,而归结为地租的部分则表现为为土地支付的价格的再生产。因此这些价格成了商品总价格的构成部分。这在产业资本家看来就不仅仅是如此;对他来说利息和地租确实构成他的预付的一部分,如果说一方面它们决定于他的商品的市场价格(通过这种市场价格,社会过程或它的结果表现为商品所固有的规定性,而这一过程的波动,它的运动,则表现为商品价格所固有的波动),那末另一方面市场价格则决定于它们,正象棉花的市场价格决定棉纱的市场价格,而棉纱的市场价格又决定对棉花的需求,从而决定棉花的市场价格一样。

由于剩余价值的两个部分,即利息和地租,作为商品(商品“土地”和商品“资本”)的价格加入生产过程,它们借以存在的形式就不仅掩盖了它们的实际来源,而且简直否定了这一来源。

剩余劳动,即无酬劳动,也和有酬劳动一样实?上加入资本主义生产过程这一情况,在这里表现为:与劳动不同的生产要素(土地和资本)必须得到报酬,或者说,与预付商品的价格和工资不同的费用加入商品的价格。剩余价值的两个部分(利息和地租)在这里表现为从事经营的资本家的费用即预付。

平均利润作为决定的因素加入商品的生产价格,因此,在这里剩余价值已经不是表现为结果,而是表现为条件;不是表现为商品价值分解成的那些部分中的一个部分,而是表现为商品价格的构成部分。但是平均利润也和生产价格本身一样不如说是观念上起决定作用的东西,它同时表现为超过预付的余额,[928]表现为不同于真正的生产费用的价格。在现存的市场价格情况下,即在过程的直接结果中,是否得到平均利润,得到的利润是大于还是小于平均利润,——这一点决定着再生产,或者更确切地说,决定着再生产的规模;决定着现有资本以怎样的量抽出或投入这一或那一生产领域,也决定着新积累的资本以怎样的比例流入这些不同领域,最后,决定着这些不同领域在什么程度上作为买者出现在货币市场上。相反,剩余价值中作为利息和地租的这些部分则以完全固定的形式,分别表现为单个生产价格的前提,并且是以预付形式预支的。

\centerbox{※     ※     ※}

{可以把预付,即资本家支付的东西叫作费用[Kosten]。按照这种说法,利润就表现为超过这些费用的余额。这与个别生产价格有关。而由预付决定的价格就可以叫作费用价格[Kosten-preise]\endnote{见注6和注18。这里马克思是在c+v的意义上使用“费用价格”这一术语的。——第570页。}。

由平均利润决定的价格,也就是由预付资本的价格加平均利润决定的价格,可以叫作生产费用[Produktionskosten],因为这一利润是再生产的条件,是在不同领域之间调节商品供给和资本分配的条件。这种价格是生产价格[Produktionspreise]。

最后,生产商品所花费的劳动(物化劳动和直接劳动)的实际量就是商品的价值。这一价值构成商品本身的实在的生产费用。与这一价值相适应的价格,只是以货币表现的价值。

“生产费用”这一术语交替地用在所有这三种意思上。}

\centerbox{※     ※     ※}

如果没有剩余价值再生产出来,那末,其中叫作利息的部分和叫作地租的部分自然也就会同剩余价值一起消失,这种剩余价值的预支,或者说,这种剩余价值作为商品价格加入生产费用这一事实,也会随着消失。加入生产的现有价值,那时就根本不会作为资本从生产中产生出来,因而也不可能作为资本加入再生产过程,或作为资本贷出。因此,正是同样一些关系——决定资本主义生产的关系——的不断再生产,使它们不仅表现为这一过程的社会形式和结果,而且同时表现为它的经常的前提。但是只有作为这一过程本身不断确定、创造、生产的前提,它们才是这样的前提。因此,这种再生产决不是有意识的,相反,它只是在作为前提和支配生产过程的条件的这些关系的经常存在中表现出来。例如,商品价值可能分解成的那些部分变成商品价值的构成部分,这些构成部分作为彼此独立的部分相对立,因而也作为独立的部分与它们的统一体相对立,而这个统一体反过来又表现为它们的结合。资产者看到产品经常成为生产的条件。但是他没有看到,生产关系本身,那些他借以进行生产并且在他看来是既定的自然关系的社会形式,是这一特殊社会生产方式经常的产物,并只是由此才成为经常的前提。不同的关系和因素不仅变成一种独立的东西,并取得一种奇异的、似乎彼此无关的存在方式,而且表现为物的直接属性,取得物的形态。

由此可见,资本主义生产的当事人是生活在一个由魔法控制的世界里,而他们本身的关系在他们看来是物的属性,是生产的物质要素的属性。但正是在最后的、最间接的形式上(同时在这些形式上中介过程不仅变得看不见了,而且甚至变成自己直接的对立面),资本的不同形态表现为生产的实际因素和直接承担者。生息资本在货币资本家身上人格化了,产业资本在产业资本家身上人格化了,提供地租的资本在作为土地所有者的地主身上人格化了,最后,劳动在雇佣工人身上人格化了。它们作为这样一些在独立的个人身上(这些个人同时只是表现为人格化的物的代表)人格化了的固定形态,加入竞争和实际生产过程。竞争以这种转化为前提。资本的这些固定形态,对于竞争来说,是合乎自然、在自然史意义上存在的形式,而竞争本身在自己的表面现象上[929]只是这一颠倒的世界的运动。就内在联系在这种运动中的实现来说,这种内在联系表现为一种神秘的规律。政治经济学本身,这门致力于重新揭示隐蔽的联系的科学,就是很好的证明。在竞争中一切都在这一最外表的最后的形式上表现出来。例如,市场价格在这里表现为一种占支配地位的东西,利率、地租、工资、产业利润表现为价值的构成要素,而土地价格和资本价格表现为既定的、从事经营必须计算的费用项目。

我们已经看到,亚·斯密起先把价值分解为工资、利润(利息)和地租,后来又反过来把它们说成是商品价格的独立的构成要素。\fnote{见本卷第1册第73—78页。——编者注}在前一种见解中,他说出了隐蔽的联系,在后一种见解中他说的是外部表现。

如果更接近现象的表面,那末除了平均利润率以外,利息,甚至地租也可以说成是商品价格(即市场价格)的构成部分。利息可以直接说成是这样的构成部分,因为它加入费用价格。地租(作为土地价格)虽然不能直接决定产品价格,但它决定生产方式:是把大量资本集中在少量土地上,还是把少量资本分配在大量土地上;是生产这种还是生产那种产品(牲畜还是谷物),其市场价格要最能抵补地租价格,因为地租必须在租约期满以前支付。因此,为了使地租不成为产业利润的扣除部分,牧场会变成耕地,耕地又会变成牧场,等等。可见地租不会直接地但会间接地决定单个产品的市场价格,即通过确定各种产品之间的比例,使需求和供给能够为每一种产品保证最好的价格,以便这种价格能够支付地租。如果说地租在这个意义上不直接决定例如谷物的市场价格,那末,它直接决定牲畜等等的市场价格,简单地说,它直接决定这样一些领域的产品的市场价格,在这些领域里,地租不是由本领域产品的市场价格决定,产品的市场价格却是由播种谷物的土地提供的地租率决定。例如肉类在工业发达的国家总是价钱很贵,即不仅大大高于它的生产价格,而且高于它的价值。因为它的价格不仅必须支付它的生产费用,而且必须支付土地提供的地租,如果在这块土地上种植谷物的话。否则,在大畜牧业的条件下,由于资本的有机构成非常接近[工业中的资本构成],或者不变资本对可变资本甚至占更大优势,肉类就只能支付很少的绝对地租,或者完全不能支付绝对地租。但是肉类支付的、直接加入肉类价格的地租,是由土地作为耕地时会支付的绝对地租和级差地租的总量决定的。这种级差地租在这里也大部分不存在。最好的证据是:在谷物不支付地租的土地上,肉类会支付地租。

因此,如果说利润作为决定的因素加入生产价格,那就可以说,工资、利息,以及在一定程度上地租,作为决定的因素加入市场价格,无疑也作为决定的因素加入生产价格。当然,因为整个说来利息的运动由利润决定,因为谷物地租部分地由利润率决定,部分地由农产品的价值以及由不同土地的产品的不同价值平均化为市场价值决定,而利润率部分地由工资决定,部分地由生产不变资本的生产领域的劳动生产率决定,从而由工资的高度和劳动生产率决定,而工资则归结为商品的一定部分的等价物(即工资等于商品包含的有酬劳动,而利润等于商品包含的无酬劳动);最后,因为劳动生产率的增长只能以两种方式影响商品价格:一是影响商品的价值,即降低其价值,二是影响商品的剩余价值,即提高其剩余价值,——所以全部问题最终可以归结为由劳动时间决定的价值。费用价格无非是预付资本的价值加预付资本所生产的剩余价值,这种剩余价值是在各个领域之间按照它们在总资本中所占的份额进行分配的。所以,如果考察的不是单个领域而是总资本,费用价格就归结为价值。另一方面,每个领域的市场价格,由于不同领域的资本的竞争,经常还原为费用价格。每一单个领域的资本家的竞争力求使商品的市场价格还原为它的市场价值。不同领域的资本家的竞争使市场价值还原为所有领域共同的费用价格。

斯密认为价值由其自身决定的价值各部分构成,李嘉图反对斯密的这种看法。但他不是前后一贯的。否则他就不可能和斯密争论:加入价格的,即作为构成部分加入价格的究竟是利润、工资和地租,还是象他所说的只是利润和工资。既然它们被支付,从分析来看它们是加入价格的。相反,他应当这样说:每一种商品的价格都可以分解为利润和工资,某些商品的价格(而且很多商品的价格是间接地)可以分解为利润、地租和工资;但是没有一种商品的价格是由它们构成的,[930]因为它们不会作为独立的自动的有一定大小的因素构成商品价值,而如果价值是既定的,它倒可以按极不相同的比例分解为上述各部分。并不是既定的因素(利润、工资和地租)通过相加或结合决定价值量,而是同一个价值量,即既定的价值量,分解为工资、利润和地租,并且是按不同的情况,以极不相同的方式在这三个范畴之间进行分配。

假定生产过程在条件不变的情况下不断重复,就是说,再生产和生产一样是在相同的条件下进行。这要有一个前提,就是劳动生产率不变,或至少生产率的变化不致改变生产当事人之间的关系,从而即使商品价值会由于生产力的变化而提高或降低,商品价值在生产当事人之间的分配仍旧不变。在这种情况下,虽然,说价值的不同部分决定整体的价值或价格,虽然在理论上是不确切的,但是,如果把构成理解为由各个部分相加而形成整体,那末,说价值的不同部分构成价值在实际上就是正确的。商品价值会照旧同地分为[预付资本的]价值和剩余价值,[新创造的]价值会同样地分解为工资和利润,而利润也会同样地分解为利息、产业利润和地租。因此似乎可以说:P,即商品价格,分解为工资、利润(利息)和地租,另一方面,工资、利润(利息)、地租则构成价值,或者更确切地说,构成价格。

但是这种再生产的均衡性或等同性(即生产在同样条件下反复进行)实际上是不存在的。生产率会变化并改变生产条件。条件也会从自己方面改变生产率。但是这种偏离部分地会表现在短期间内即可平均化的表面的波动上,部分地会表现在偏离的逐渐积累上,这种偏离或者是引起危机,即通过暴力在表面上回到原来的关系,或者是极缓慢地给自己打通道路,争取被承认为生产条件的改变。

在预支了剩余价值的利息和地租形式上,必须假定,再生产的一般性质保持不变。只要资本主义生产方式继续存在,情况就是这样。其次,甚至必须假定(情况或多或少也是这样),这一生产方式的一定生产关系在一定时间内保持不变。因此,生产的结果就作为牢固的、因而是充当前提的生产条件固定下来,并且作为物质生产条件的牢固的属性固定下来。生产过程所不断分解成的并不断再生产出来的各种不同要素的这种表面上的独立性,在危机到来时就会结束。

{对真正的经济学家来说是价值的东西,对实践的资本家来说就是市场价格,它总是整个运动的最初的东西。}

生息资本在信用上取得了资本主义生产所特有的并与它相适应的形式。信用是资本主义生产方式本身所创造的一种形式。(商业资本[对资本主义生产方式]的从属性实际上不要求这种新的创造,因为商品和货币,商品流通和货币流通,仍旧是资本主义生产的基本前提,只不过是变成了绝对的前提。因此,商业资本一方面是资本的一般形式,另一方面,就它代表执行一定职能的资本即专门在流通过程中执行职能的资本来说,它由生产资本决定这一点,并不会使它的形式有所改变。)

价值平均化为费用价格只有通过以下的方式来进行:单个资本作为整个阶级的总资本的一部分执行职能,另一方面,整个阶级的总资本根据生产的需要在不同的特殊领域之间进行分配。这是通过信用进行的。由于信用,不仅这种平均化成为可能并变得易于进行,而且资本的一部分(在货币资本的形式上)实际上表现为整个阶级用以从事经营的共同材料。这是信用的一种意义。另一种意义是资本总是力求缩短它在流通过程中必须完成的形态变化,总是力求预先实现流通时间,实现它向货币的转化等等,并[931]通过这种办法抵销自己的局限性。最后,积累的职能只要不是[收入]转化为资本,而是供给资本形式的剩余价值,它就会部分地被加在一个特殊的阶级身上,部分地表现为:社会的一切积累在这个意义上都变成资本的积累,并归产业资本家支配。在社会的无数个点上单独进行的这种活动被集中起来,汇集在一定的蓄水池里。由于商品在形态变化中凝结而闲置的货币,就通过这种途径转化为资本。

\centerbox{※     ※     ※}

“土地—地租”,“资本—利息”是一些不合理的说法,因为地租被固定为土地的价格,而利息被固定为资本的价格。在生息资本、提供地租的资本、提供利润的资本这些形式上还能够认出[所有这些不同收入的]共同来源,因为一般资本包括对剩余劳动的占有,也就是说,这些不同的形式只是表示,这种由资本产生的剩余劳动,就一般资本来说,是在两类资本家之间进行分配,而就农业资本来说,则是在资本家和土地所有者之间进行分配。

作为土地的(年)价格的地租和作为资本的价格的利息,就象一样地不合理。这后一种形式同数字的简单基本形式相矛盾,正象那些形式同资本的简单形式商品和货币相矛盾一样。地租和利息的不合理性表现在颠倒的意义上。“土地—地租”,即作为土地的价格的地租表示土地是商品,是具有价值的使用价值,这种价值的货币表现等于它的价格。但是一种不是劳动产品的使用价值不可能有价值,也就是说,它不能算作一定量社会劳动的物化,一定量劳动的社会表现。它不是这种东西。使用价值要表现为交换价值,要成为商品,就必须是具体劳动的产品。只有在这样的前提下,这种具体劳动才能表现为社会劳动,表现为价值。土地和价格是不可通约的量,不过它们彼此还是应当有一种关系。在这里,一个没有价值的物有着一个价格。

\todo{}

另一方面,作为资本价格的利息也表现出颠倒的不合理性。在这里,没有使用价值的商品有了双重价值,先是有价值,然后又有和这个价值不同的价格。因为资本首先不外是一个货币额或等于一定货币额的一定商品量。如果商品作为资本来贷放,这个商品就只是一个货币额的伪装形式。因为作为资本贷放的,并不是若干磅棉花,而是其价值存在于棉花中的若干货币。所以,资本的价格也和只是作为货币额——即表现为货币并存在于交换价值形式上的价值额——存在的资本有关。一个价值额怎么能够在那个要用它本身的货币形式来表示的价格之外,还有一个价格呢?价格是和商品的使用价值相区别的商品的价值。因此,价格作为一种和商品价值不同的东西,价格作为一个货币额的价值(因为价格只是价值在货币上的表现),是术语上的矛盾。

这种说法的不合理性(事物本身的不合理性是这样产生的:就利息来说,资本表现为前提,和它自己的、使它成为资本即自行增殖的价值的过程相分离,另一方面,提供地租的资本只是作为农业资本,只是作为特殊领域的资本才提供地租,也就是说,它借以表现的这种形式,被移到了使它根本区别于工业资本的要素上),被庸俗经济学家深深地感觉到了,于是他就假造了两种说法,以便使它们变得合理。他断言利息是为资本支付的,因为资本是使用价值,他因此也谈到产品或生产资料本身对再生产的有用性,谈到资本作为劳动过程的物质要素的有用性。

但是,资本的有用性,它的使用价值,本来已经存在于它的商品形式中,没有这种有用性,它就不会是商品,也不会有价值。在货币形式上,资本是商品价值的表现,并且[932]能够依照商品本身的价值转化为商品。但是,如果我把货币转化为机器、棉花等等,我就是把它们转化为具有同样价值的使用价值。这种转化只涉及价值的形式。在货币形式上,资本具有的使用价值使它能够转化为任何形式的、但具有同样价值的商品。通过这种形式变化,货币的价值并没有变,正象商品转化为货币时商品的价值不变一样。我能使货币转化成的那些商品的使用价值,不会给货币提供任何超过其价值、不同于其价值的价格。但是,如果我以这种转化为前提,并且说,价格是为商品的使用价值支付的,那末,商品的使用价值就根本得不到支付,或者只是在商品的交换价值被支付的情况下才得到支付。至于怎样利用买来的商品的使用价值,它进入个人消费还是进入生产消费,这绝对不会使它的交换价值有任何变化。由此而引起变化的只是谁购买商品:是产业资本家还是直接消费者。因此,商品在生产上的有用性可以说明商品一般具有交换价值,因为要使商品包含的劳动得到支付,它们必须具有使用价值,否则它们便不是商品,只有作为使用价值和交换价值的统一体,它们才是商品。但是这种使用价值决不能说明,商品作为交换价值或作为价格,还会有一个不同于这个价格的其他价格。

我们看到,庸俗经济学家在这里想通过试图把资本即货币或商品(就它们具有特殊的、不同于它们作为货币或商品的规定性来说)转化为简单的商品的办法来逃避困难,就是说,他要避开的正好是应当说明的那一特殊区别。他不想说,资本是剥削剩余劳动的手段,因此,资本是比它包含的价值更大的价值。他却说,资本所以具有比它的价值更大的价值,是因为它和其他任何商品一样,是普通的商品,也就是说,它具有使用价值。这里把资本和商品等同起来了,而需要说明的正是商品怎样能够表现为资本。

对于土地,庸俗经济学家所持的态度却相反,只要他不随声附和重农学派的话。在论述利息的时候,为了说明资本和商品之间的区别,为了说明商品向资本的转化,他把资本转化成商品。在这里,他把土地转化成资本,因为资本关系本身比土地价格更适合于他的观念。地租可以看作资本的利息。例如,如果地租是20,而利率等于5,那就可以说,这20便是资本400的利息。实际上土地就是按400出卖,这不过是出卖20年的地租。这种对预先实现的二十年地租的支付便是土地的价格。这样,土地就转化为资本。每年支付的20,只是为土地支付的资本的百分之五的利息。通过这种办法,“土地—地租”就变成“资本—利息”,而这又被幻想成对商品使用价值的支付,也就是说,被幻想成“使用价值—交换价值”这种关系。

庸俗经济学家中较有分析能力的人懂得,土地价格无非是地租资本化的表现,实际上是根据当时存在的利率决定的若干年地租的购买价格。他们懂得,地租的这种资本化以地租为自己的前提,所以地租不能反过来用它自己的资本化来解释。因此,当他们宣称地租是投在土地上的资本的利息时,他们也就否定了地租本身,而这并不妨碍他们承认,没有投入任何资本的土地也提供地租,也不妨碍他们承认,在肥力不同的土地上同量资本提供不同的地租,或者承认,在肥力不同的土地上不同量资本提供相同的地租。同样,这也不妨碍他们承认,投在土地上的资本(如果为土地支付的地租确实必须用它来解释的话)也许会提供比以固定资本形式投在工业上的同量资本提供的利息大四倍的利息,也就是大四倍的地租。

我们看到,在这里排除困难的办法总是:避开困难,而对应当解释的特殊区别则是用某种关系来代替,这种关系表明的却是和这种区别相反的东西,因而无论如何也不表明这种区别。[932]

\tchapternonum{[(6)庸俗社会主义反对利息的斗争(蒲鲁东)。不理解利息和雇佣劳动制度之间的内在联系]}

[935]蒲鲁东同巴师夏关于利息的论战是很有特色的,它既能说明庸俗经济学家是用什么样的方式来维护政治经济学的各种范畴的,也能说明肤浅的社会主义(蒲鲁东的论战未必配得上这个称号)是用什么样的方式来攻击这些范畴的。我们在论庸俗经济学家一节\endnote{在1863年1月拟定的《资本论》第三部分的计划中,倒数第二章即第十一章的标题是《庸俗政治经济学》(见本卷第1册第447页)。这个计划是在写完1861—1863年手稿第XV本中《收入及其源泉。庸俗政治经济学》这一部分之后一个半月到两个月拟定的。——第581页。}中将回过头来谈这个问题。这里只是预先说几点意见。

如果蒲鲁东对资本的运动一般有所了解,那末,[货币的]回流运动就不会作为[生息资本的]一种特性使他感到震惊了。回流总额中的价值余额也是同样情况。这正是资本主义生产的特征。

{但是在蒲鲁东那里,正如我们将要看到的,余额就是附加额。一般说来,他的批判是幼稚的,他甚至根本没有掌握他要批判的那门科学的基本要素。例如,他根本不懂得货币是商品的必要形式。(见第一部分\endnote{马克思指《政治经济学批判》第一分册。见《马克思恩格斯全集》中文版第13卷第45页和第76页。——第581页。})在这里他甚至把货币和资本混淆起来,因为借贷资本表现为货币形式上的货币资本。}

能够使他感到惊奇的不是不被支付任何等价物的余额,因为剩余价值(资本主义生产就是建立在它上面的)是不花费任何等价物的价值。这并不是生息资本的特征。生息资本的特征(就我们所考察的运动形式来说)只在于第一个要素,这和蒲鲁东设想的正好相反,那就是贷款人贷出货币,最初并没有为此得到等价物,因此,资本带着利息流回,就贷款人和借款人之间进行交易来说,和资本经过的形态变化[毫无共同之处],如果这些形态变化只是经济形式的形态变化,它们就表现为交换行为的序列,即商品转化为货币和货币转化为商品,如果它们是现实的形态变化,或者说是生产过程,它们就和生产消费相结合。(消费本身在这里构成经济形式的运动的一个要素。)

但是货币在贷款人手里没有做到的事情,在把它们实际用作资本的借款人手里做到了。它们在借款人手里完成了它们作为资本的现实运动。它们作为货币加利润,作为货币加1/x货币,流回他手里。贷款人和借款人之间的运动,只是表示资本的起点和终点。资本作为货币从A手里转到B手里。在B手里货币成为资本,它们作为资本经过一定的循环以后带着利润流回。这种中间行为,实际过程(包括流通过程和生产过程),完全与借款人和贷款人之间的交易无关。这一交易只是在货币作为资本已经得到实现以后才重新开始。现在货币带着一个余额回到贷款人手里,不过这个余额只是借款人实现的余额的一部分。借款人获得的等价物是产业利润,也就是这个余额中留给他的一部分,这一部分只是他靠借入的货币才占有的。所有这一切在借款人和贷款人之间的交易中都是看不见的。这种交易限于两种行为。货币从A到B的转手。货币在B手里的间歇。货币在间歇以后带着利息流回A手里。

因此,如果只是考察这种形式(A和B之间的这种交易),得到的就是没有中介过程的资本的单纯形式:货币,它以a额支出,经过一定时间,再以a+1/xa额流回,除了a额流出又以a+1/xa额流回这样一段时间以外,完全不存在任何中介过程。

蒲鲁东先生正是在这种没有概念的形式上(这种形式当然是作为独立的运动和资本的实际运动同时发生的,它使这个运动开始并结束)考察事物的,在这样的考察中,对他来说,一切都必然是不可理解的。如果这种代替买和卖的借贷形式不再存在,那末,在他看来余额也就不会再有了。其实,不会再有的只是余额在两类资本家之间的分割。但是这种分割能够而且必然不断重新发生,只要商品或货币能够转化为资本,而这一点在雇佣劳动的基础上总是能够做到的。如果商品和货币不能变成资本,从而也不能作为可能的资本贷出,它们就不能和雇佣劳动相对立。如果它们作为商品和货币不和雇佣劳动相对立,从而劳动本身也不成为商品,那末,这不过是意味着[936]回到资本主义生产以前的生产方式,在那里,劳动不转化为商品,而大量劳动还以农奴劳动或奴隶劳动形式出现。在以自由劳动为基础的条件下,这种情况只有在工人是自己的生产条件的所有者时才有可能。自由劳动在资本主义生产的范围内发展为社会劳动。因此,说工人是生产条件的所有者,就是说生产条件属于社会化的工人,工人作为社会化的工人进行生产,并把他们自己的生产作为社会化的生产从属于自己。但是象蒲鲁东那样,既要保存雇佣劳动,从而保存资本的基础,同时又想用否定资本的一种派生形式的办法来消除“弊端”,那就是幼稚。

《无息信贷。弗·巴师夏先生和蒲鲁东先生的辩论》1850年巴黎版。

在蒲鲁东看来,贷放是一件坏事,因为它不是出售。

\begin{quote}{取息的贷放“是这样一种能力,即人们可以不断重新出售同一物品,并且不断重新为此得到价格,但从来不出让对所售物品的所有权”。(《无息信贷》,《人民之声报》\endnote{《人民之声报》(《LaVoixduPeuple》)——蒲鲁东派的日报,1849年10月1日至1850年5月14日在巴黎出版。——第583页。}编者之一舍韦写的第一封信,第9页)}\end{quote}

使这封信的作者迷惑的是,这种“物品”(例如货币或房屋)不会变更所有者,这同在买和卖时不一样。不过他没有看到,当货币贷出时,并没有得到任何等价物作为报酬,而在实际过程中,在交换的形式和基础上不仅会得到一个等价物,而且会得到一个无酬的余额。在物品交换发生时,不会发生价值变动,同一个人仍然是同一价值的“所有者”。在产生剩余价值时,不会发生交换。当商品和货币的交换再开始时,剩余价值已经包含在商品中了。蒲鲁东不懂得利润,从而利息,怎样由价值的交换规律产生。因此,照他说来,“房屋”、“货币”等等就不应当作为“资本”,而应当作为“商品……按照成本”来交换(《无息信贷》第43—44页)。

\begin{quote}{“实际上,出售帽子的制帽业主……得到了帽子的价值,不多也不少。但借贷资本家……不仅一个不少地收回他的资本,而且他得到的,比这个资本,比他投入到交换中去的东西多;他除了这个资本还得到利息。”(同上,第69页)}\end{quote}

蒲鲁东先生的制帽业主看来不是资本家,而是手工业者、手艺人。

\begin{quote}{“因为在商业中,资本的利息加到工人的工资上,共同构成商品的价格,所以,工人要买回他自己的劳动的产品,就不可能了。自食其力的原则,在利息的支配下,包含着矛盾。”(第105页)}\end{quote}

在第九封信中(第144—152页),勇敢的蒲鲁东把作为流通手段的货币同作为资本的货币混淆起来,从而得出结论说:法国现存的“资本”会提供160%(即10亿资本——“法国流通中的现金总额”——在国债和抵押等等形式上的年利息为16亿)。

其次:

\begin{quote}{“货币资本从一次交换到另一次交换,通过利息的积累,总是不断回到它的出发点,由此可见,每次由同一个人的手重新把这些货币贷出,总能给同一个人带来利润。”(第154页)}\end{quote}

因为资本是在货币形式上贷出,所以蒲鲁东以为,货币资本即现金具有这种特殊的属性。在蒲鲁东看来,一切东西都应当出售,但任何东西也不应当贷放。换句话说,正象蒲鲁东想保存商品,但不想使商品变成“货币”一样,他在这里想保存商品和货币,但是它们不应当发展成资本。如果把一切空想的表达形式抛开,那就不过是说,不应当从小市民-农民的和手工业的小生产过渡到大工业。

\begin{quote}{“因为价值无非是一种比例,一切产品必然互成比例,所以由此可以得出结论,从社会的观点来看,产品总是价值并且是确定的价值。对社会来说,资本和产品之间的区别是不存在的。这种区别完全是主观的,只是对个人来说才是存在的。”(第250页)}\end{quote}

象“主观的”这样的德国哲学用语,落在蒲鲁东的手里是多么不幸!社会的、资产阶级的形式对他来说成了“主观的”。这个主观的,而且是错误的抽象使蒲鲁东断言,因为商品的交换价值表示商品之间的比例,所以它表示商品之间的任何比例,而不表示商品与之成某种比例的第三种东西,——这种错误的“主观的”抽象也就是一种[937]“社会的观点”,从这种观点来看,不仅商品和货币,甚至商品、货币和资本都是等同的。的确,从这种“社会的观点”来看,所有的猫都是灰色的。

最后,剩余价值还表现在道德的形式上:

\begin{quote}{“一切劳动都应当提供一个余额。”(第200页)}\end{quote}

这种道德训条自然是剩余价值的绝妙的定义。[937]

\tchapternonum{[(7)关于利息问题的历史。路德在进行反对利息的论战时胜过蒲鲁东。对利息的观点随着资本主义关系的发展而发生变化]}

[937]路德生活在中世纪市民社会瓦解为现代社会诸要素的时代(世界贸易和黄金新产地的发现加速了这个瓦解过程),所以,他当然只能从生息资本和商业资本这两个洪水期前的[形式]去认识资本。如果说已经站稳脚跟的资本主义生产在它的幼年时期力图迫使生息资本从属于产业资本(在资本主义生产以工场手工业和大商业的形式最先繁荣起来的荷兰,这一点事实上已经首先做到了,而在英国,这一点则是在十七世纪,并且部分地是以非常天真的形式,被宣告为资本主义生产的第一个条件),那末,在向资本主义生产方式过渡时,承认“高利贷”这个生息资本的古老形式是一个生产条件,是一种必要的生产关系,反而成为第一个步骤;正如后来,当产业资本征服了生息资本时(十八世纪,边沁\endnote{马克思指边沁的著作《为高利贷辩护》,1787年在伦敦出第一版,1790年出第二版,1816年出第三版。——第586、598页。}),它本身就承认了生息资本的合法性,承认了它们之间的血肉关系。

路德比蒲鲁东站得高。借贷和购买之间的区别没有把他弄糊涂;他认为在这两种情况下都同样有高利贷存在。在他进行的论战中,最有力的一点,是他把利息长在资本上当作主要的攻击点。

(I)《论商业与高利贷》(1524年),《可尊敬的马丁·路德博士先生的第六部著作》1589年维登堡版。

(这部著作是在农民战争前夜写成的。)

关于商业(商业资本):

\begin{quote}{“现在,商人对贵族或盗匪非常埋怨{由此可以看到,为什么商人和国君一起反对农民和骑士},因为他们经商必须冒巨大的危险,他们会遭到绑架、殴打、敲诈和抢劫等等。如果商人是为了正义而甘冒这种风险,那末他们当然就成了圣人了……但既然商人对全世界,甚至在他们自己中间,干下了这样多的不义行为和非基督教的盗窃抢劫行为,那末,上帝让这样多的不义之财重新失去或者被人抢走,甚至使他们自己遭到杀害,或者被绑架,又有什么奇怪呢?……国君应当对这种不义的交易给予应有的严惩,并保护他们的臣民,使之不再受商人如此无耻的掠夺。因为国君没有这么办,所以上帝就利用骑士和强盗,假手他们来惩罚商人的不义行为,他们应当成为上帝的魔鬼,就象上帝曾经用魔鬼来折磨或者用敌人来摧毁埃及和全世界一样。所以,他是用一个坏蛋来打击另一个坏蛋,不过在这样做的时候没有让人懂得,骑士是比商人小的强盗,因为一个骑士一年内只抢劫一两次,或者只抢劫一两个人,而商人每天都在抢劫全世界。”(第296页)“……以赛亚的预言正在应验:你的国君与盗贼作伴。因为他们把一个偷了一个古尔登或半个古尔登的人绞死,但是和那些掠夺全世界并比所有其他的人都更肆无忌惮地进行偷窃的人串通一气。大盗绞死[938]小偷这句谚语仍然是适用的。罗马元老卡托说得好:小偷坐监牢,戴镣铐,大盗戴金银,衣绸缎。但是对此上帝最后会说什么呢?他会象他通过以西结的口所说的那样去做,把国君和商人,一个盗贼和另一个盗贼熔化在一起,如同把铅和铜熔化在一起,就象一个城市被焚毁时出现的情形那样,既不留下国君,也不留下商人。我担心,这个日子已经不远了。”(第297页)}\end{quote}

关于高利贷,关于生息资本:

\begin{quote}{“有人对我说,现在每年在每一次莱比锡博览会上要收取10古尔登,就是说每一百收取30。\endnote{指100古尔登的贷款,条件是分三期在莱比锡博览会上支付利息。在莱比锡每年举行三次博览会:新年,复活节(春季),米迦勒节(秋季)。——第587页。}有人还加上瑙堡集市,因此,每一百要收取40,是否有比这更多的,我不知道。岂有此理,这样下去怎么得了?……现在,在莱比锡,一个有100佛罗伦的人,每年可以收取40,这等于每年吃掉一个农民或市民。如果他有1000佛罗伦,每年就会收取400,这等于每年吃掉一个骑士或一个富有的贵族。如果他有10000佛罗伦,每年就会收取4000,这等于每年吃掉一个富有的伯爵。如果他有100000佛罗伦(这是大商人必须具有的),每年就会收取40000,这等于每年吃掉一个富有的国君。如果他有1000000佛罗伦,每年就会收取400000,这等于每年吃掉一个大的国王。为此,他不必拿他的身体或商品去冒险,也不必劳动,只是坐在炉边,烤苹果吃。所以,一个强盗坐在家里,可以在十年内吃掉整个世界。”(第312—313页)\endnote{马克思以《关于高利贷,关于生息资本》为题的这段引文不是从路德的著作《论商业与高利贷》(1524年)中摘出的,而是从路德的另一部著作《给牧师们的谕示:讲道时要反对高利贷》(1540年)中摘出的,马克思在后面第III点考察了这一著作。——第588页。}}\end{quote}

{(II)《讲道:福音书中的富人和穷人拉撒路》1555年维登堡版。

\begin{quote}{“对于富人,我们不应当从外表举止上去判断,因为他身穿羊皮袄,生活阔绰,冠冕堂皇,巧妙地掩盖了狼子野心。因为福音书谴责他不是由于他犯了奸淫罪、杀人罪、抢劫罪、渎神罪,或者世人或理性会谴责的某种罪行。他的确和那个与众不同的、一礼拜禁食两次的法利赛人一样,过着端正的生活。”}}\end{quote}

在这里路德告诉我们,高利贷资本是怎样产生的:它是靠市民(小市民和农民)、骑士、贵族、国君的破产产生的。一方面,城关市民、农民、行会师傅、总之小商品生产者的剩余劳动以及劳动条件会流入高利贷者手里,因为他们在把自己的商品转化为货币以前就需要货币,例如用来支付,他们已经要由自己购买一部分劳动条件,等等。另一方面,地租所有者,即挥霍享乐的富有阶级的钱也流入高利贷者手里,高利贷者现在把地租据为己有。高利贷有两种作用:第一,总的说来,它形成独立的货币财产,第二,它把劳动条件占为己有,也就是说,使旧劳动条件的所有者破产;因此,它对形成产业资本的前提是一个有力的手段,对生产条件和生产者的分离是一个有力的因素。它完全和商人一样。二者有共同点:都会形成独立的货币财产,也就是说,以货币要求权的形式把年剩余劳动的一部分,劳动条件的一部分,以及年劳动积累的一部分,积累在自己手里。他们实际掌握的货币,只是既构成常年的和每年积累的贮藏货币的一小部分,又构成流动资本的一小部分。说在他们那里形成货币财产,就是说年产品和年收入的很大一部分会落到他们手里,并且不是以实物形式而是以货币这种转化形式支付给他们。因此,只要货币不是作为现金能动地流通,不是处于运动之中,货币就积累在他们手里,流通货币的蓄水池也部分地掌握在他们手里,对产品的要求权更是掌握在和积累在他们手里,不过这种要求权是对已转化为货币的商品的要求权,是货币要求权。[939]高利贷,一方面是封建财富和封建所有制的破坏者,另一方面是小资产阶级的、小农民的生产的破坏者,总之,是生产者仍然表现为自己的生产资料所有者的一切形式的破坏者。

在资本主义生产中,劳动者是生产条件(既包括他所耕种的土地,也包括他用来劳动的工具)的非所有者。但是,在这里,同生产条件的这种分离相适应,生产方式本身发生了真正的变化。工具变成了机器;劳动者在工厂劳动等等。生产方式本身不再容许生产工具处于那种和小所有制联系着的分散状态,也不再容许劳动者自己处于分散状态。在资本主义生产中,高利贷不能再使生产条件和劳动者、生产者分离,因为二者已经分离了。

高利贷只是在生产资料分散的地方,从而在劳动者作为小农、行会手工业者(小商人)等等或多或少独立地进行生产的地方,才会把财产集中起来,特别是以货币财产的形式集中起来。作为农民或手工业者,这个农民可以是农奴,也可以不是农奴,这个手工业者可以属于行会,也可以不属于行会。高利贷者在这里不仅把依附农本身支配的那一部分剩余劳动(在同自由农民等等打交道时,则把全部剩余劳动)占为己有,而且还把生产工具占为己有,虽然农民等等仍然是这些生产工具的名义上的所有者,并且在生产中,仍然作为所有者同这些生产工具发生关系。这种高利贷就是建立在这种基础上,建立在这种生产方式上,高利贷不改变这种生产方式,而是象奇生虫那样紧紧地吸在它身上,使它虚弱不堪。高利贷吮吸着它的脂膏,使它精疲力竭,并迫使再生产在每况愈下的条件下进行。由此产生了民众对高利贷的憎恶,在古代的关系下特别是这样,因为在这种关系下,生产条件为生产者所有这种生产性质,同时是政治关系即市民的独立地位的基础。一旦劳动者不再拥有生产条件,这种情况就终止了。高利贷的权力也就随之而告终。另一方面,在奴隶制占统治地位或者剩余劳动为封建主及其家臣所吞食的情况下,奴隶主或者封建主即使陷入高利贷之中,生产方式仍旧不变,只是它会更加残酷。负债的奴隶主或封建主会榨取得更厉害,因为他自己也被榨取了。或者,他最后让位给高利贷者,高利贷者本人象古罗马的骑士等等一样成为土地所有者等等。旧剥削者的剥削或多或少是政治权力的手段。现在代替旧剥削者出现的,则是粗暴的拚命要钱的暴发户了。但生产方式本身仍旧不变。

高利贷者在资本主义前的一切生产方式中所以只是在政治上有革命的作用,是因为他会破坏和瓦解这些所有制形式,而政治制度正是建立在这些所有制形式的牢固基础上,也就是建立在它们的同一形式的不断再生产上的。高利贷也有集中的作用,但只是在旧生产方式的基础上起集中的作用,结果是使除了奴隶、农奴等等以及他们的新主人以外的社会瓦解为平民。在亚洲的各种[社会]形式下,高利贷能够长期延续,这只是造成经济的衰落和政治的腐败,但没有造成[现存的生产方式]真正解体。只有在资本主义生产的其他条件——自由劳动,世界市场,旧的社会联系的瓦解,劳动在一定阶段上的发展,科学的发展等等——已经具备的时代,高利贷才表现为形成新生产方式的一种手段,同时又表现为使封建主,反资产阶级要素的支柱遭到毁灭,使小工业、小农业等等遭到破坏的手段,总之,表现为把作为资本的劳动条件集中起来的手段。

高利贷者、商人等等占有“货币财产”,这无非就是,把表现为商品和货币的国民财产集中在他们手里。

在高利贷者本人不是生产者的情况下,资本主义生产起初不得不同高利贷作斗争。到了资本主义生产已经确立的时候,高利贷对剩余劳动的支配权(这种支配权同旧生产方式的继续存在相联系)已经终止了。产业资本家以利润形式把剩余价值直接占为己有;他也已经部分地占有生产条件,并且直接占有一部分年积累。从这时起,特别是随着产业财产和商业财产的发展,高利贷者即贷款人,就只是一种由于分工而同产业资本家分离、但又从属于产业资本的角色。

[940](III)《给牧师们的谕示:讲道时要反对高利贷》1540年维登堡版(没有页码)。

商业(买、卖)和借贷(路德没有象蒲鲁东那样被这种形式上的差别弄糊涂)。

\begin{quote}{“十五年前我已经写过反对高利贷的文章,因为那时高利贷势力已经很大,我不抱任何改善的希望。从那时起,高利贷的身价高了,它已不愿被看作是丑恶、罪行或耻辱,而是让人作为纯粹的美德和荣誉来歌颂,好象它给了人民伟大的爱和基督教的服务似的。既然耻辱已经变为荣誉,丑恶已经变为美德,那还有什么办法呢?塞涅卡以自然的理性说道:如果恶习成自然,那就无可救药了。德国成了它应该成为的那个样子;卑鄙的贪婪和高利贷把它彻底毁灭了……首先谈借贷。当人们贷出货币,并为此要求或取得更多的或更好的东西时,这就是高利贷,它受到所有的法律的谴责。因此,所有那些从贷出的货币中每一百收取五、六或更多的人都是高利贷者,他们都懂得要照此行事,他们被称为崇拜贪婪或钱财的奴仆……因此,对于谷物,大麦和其他等等商品我们也应该这样说,如果为此要求更多的或更好的东西,这就是高利贷,就是偷盗和抢劫财物。因为借贷的意思是:我把我的钱财或用具在某人需用时,或者说在我能够并且愿意时,借给他使用,他过一定时间要按照我借给他的原样还给我。”“因此,从购买当中也能获得高利。但是现在要一口吃掉,那就太多了。现在必须先谈一种,即先谈放债的高利贷。等我们搞掉这个以后(在末日审判以后),我们再来谴责购买上的高利贷。”“高利贷者老爷会这样说:朋友,按照目前的情况,我按每一百收取五、六或十的利息借钱给我的邻人,那是对他提供很大的服务。他感谢我这种借贷,把它看成一种特别的行善。他再三向我请求,自愿地,毫不勉强地提议每一百古尔登送我五个、六个或十个古尔登。难道我就不能从高利贷问心无愧地得到这些古尔登吗?……你尽可以夸耀、粉饰和装扮……但是谁取得的更多或更好,那就是高利贷。也就是说,象偷盗和抢劫一样,他不是为邻人服务,而是损害邻人。一切名为对邻人服务和行善的事情,并非都是服务和行善。奸夫和淫妇也是互相提供重大的服务和互相满足的。骑士帮助罪犯拦路行抢,打家劫舍,也是对罪犯的重大服务。罗马教徒没有把我们全部淹死、烧死、杀死、囚死,而是让一些人活着,把他们驱逐,或者夺去他们所有的东西,也是对我们的重大服务。魔鬼对于侍奉他的人也提供重大的不可估量的服务……总之,世上到处都是重大的、卓越的、日常的服务和行善……诗人们记述,独眼巨人波利菲米斯答应乌利斯,要对他表示友好,那就是先吃掉他的同伴,最后再吃掉他。是啊,这也是一种服务和很大的行善!现在贵族和平民,农民和市民都在热心从事这种服务和行善;他们大量收购、囤积、造成物价昂贵,[941]抬高了谷物、大麦以及人们所必需的一切东西的价格,然后擦擦嘴巴说:是啊!人必须有他所必需的东西,我让它为人们服务,虽然我可以而且有权把它留归自己。于是,上帝也被巧妙地欺骗和愚弄了……现在,人都成了圣人……现在,没有人再会放高利贷,没有人会贪婪和为非作歹了,世界已经完全成了神圣的世界,每个人都为别人服务,没有人会去损害别人了……如果高利贷者这样来提供服务,那也是为害人的魔鬼服务,虽然一个贫困的人是需要这种服务的,而且他必须把他没有完全被吃掉看作是一种服务或行善……如果他想得到钱,他就会为你\fnote{高利贷者。——编者注},而且必须为你提供这样的服务{支付高利}。”}\end{quote}

{从上面的引文可以看到,在路德的时代,高利贷是极其盛行的,而且已被当作一种“服务”来加以辩护(萨伊——巴师夏\fnote{见本卷第1册第435页。——编者注})。已经出现了合作论或协调论:“每个人都为别人服务。”

在古代世界比较兴盛的时期,高利贷是被禁止的(即不允许收取利息)。后来它合法化了,并且盛行起来。在理论上则始终(如在亚里士多德的著作里\endnote{亚里士多德在他的著作《政治学》第一篇里讲过关于利息是一种违反自然的东西的观点,马克思在《资本论》第一卷第四章考察了这个观点(见《马克思恩格斯全集》中文版第23卷第187页)。——第593页。})认为高利贷本身是坏的。

在基督教的中世纪,高利贷被看成是一种“罪恶”,并为“教规”所禁止。

近代。路德。对高利贷还存在着天主教-异教的观点。高利贷广泛盛行(部分是由于政府需要货币,部分是由于商业和工场手工业的发展,部分是由于产品转化为货币的必要性)。但是它的公民权已被确认。

荷兰。对高利贷的最早的辩护。高利贷也是最早在那里现代化,从属于生产资本或商业资本。

英国。十七世纪。争论已不再是针对高利贷本身,而是针对利息的大小。高利贷对信贷的支配关系。要求创立信贷形式。强制的立法措施。

十八世纪。边沁。自由的高利贷被认为是资本主义生产的要素。}

[从路德的著作《给牧师们的谕示:讲道时要反对高利贷》中再摘引几段。]

利息作为对损失的赔偿:

\begin{quote}{“[可能发生而且会常常发生下述情况:我,汉斯,借给你,巴塔扎尔,100古尔登,条件是到米迦勒节时我必须收回来;如果你耽误了,我就会因此遭受损失。米迦勒节到了,你没有偿还我这100古尔登。我没有钱支付,法官就逮捕我,把我投入监狱,或使我遭受其他的不幸。我坐牢,我的营养,我的健康都遭受重大损失。这都是由于你的耽误,你以怨报德。这时我该怎么办呢?我继续遭受损失,是因为你耽误拖延,你多耽误拖延一天,我就多遭受一天损失。谁应该承担或赔偿这种损失呢?这种损失直到我死都会成为我家无法容忍的客人。]好吧,这里单从世俗和法律方面来谈谈这个问题(我们必须把神学放到以后谈),你巴塔扎尔应该在100古尔登之外赔偿我由此引起的全部损失以及一切费用。{路德在这里把费用理解为贷出者本人因不能偿付自己的债务而引起的诉讼费等等}……因此,就是从理性和自然法来看,你赔偿我的一切,即本金和我所受的损失,也是公平的……这种损失在法律书上,拉丁文叫interesse……除了这种损失还可能造成其他损失:如果你巴塔扎尔到米迦勒节不偿还我这100古尔登,而我本来有机会购买花园、田地、房屋或任何可为我和我的孩子提供很大好处或养活我们的东西,现在我就不得不放弃这种机会,由于你的耽误和拖延,你使我遭受到损失,并妨碍了我,使我再也不能买到这些东西……这样,我把它们贷放给你,你使我两头受损失:这里我不能支付,那里我不能购买,也就是我在两方面都不得不受到损失,这就叫作双重损失:既遭受损失,又丧失利益……他们听说汉斯贷放100古尔登受了损失,并要求适当的赔偿,就急忙趁此机会对每100古尔登都索取达双重损失的赔偿,即为不能支付的损失和失去购买花园的机会所受的损失要求赔偿,好象每100古尔登都自然会生出这样双重的损失一样。因此,只要他们有100古尔登,他们就会贷放出去,并按照他们实际上没有受到的这样双重的损失来要求赔偿……既然谁也没有使你受损失,并且你既不能证明,也不能计算这种损失,你却从邻人手里取得货币来赔偿你虚构的损失,因此,你就是高利贷者。法学家把这种损失不是叫作实际的损失,而是叫作幻想的损失。这是各个人为自己而想象出来的损失……因此,[942]说我可能会受损失,因为我可能既不能支付也不能购买,是不行的。如果这种说法能够成立,那就是从偶然生出必然,就是无中生有,就是从未必会有的东西生出确实会有的东西。这种高利贷,要不了几年,不就会把整个世界吞掉了吗?……贷出者必须得到赔偿,这种情况是一种偶然的不幸,是不以他本人的意志为转移的,但在这种交易中情况却不是这样,而是正好相反,人们总是费尽心机编造损失,让贫苦的邻人来赔偿,企图以此为生和发财致富,靠别人的劳动、忧患、危险和损失而使自己过着骄奢淫逸和荣华富贵的生活。我坐在火炉旁边,让我的100古尔登在国内为我搜集钱财。因为这是贷放出去的货币,所以终归要保存在我的钱袋里,没有任何危险,一点也不用担忧。朋友啊,谁不乐意这样做呢?贷款是如此,贷谷物,贷葡萄酒和类似的商品也是如此,都可能有这样双重的损失。但是这种损失不是从商品本身自然生出的,而可能是偶然生出的,因此,在这种损失真正发生并得到证实以前,就不能算作损失……高利贷必然会出现,不过,高利贷者是要倒霉的……所有明智的异教徒也都非常严厉地谴责高利贷。例如,亚里士多德在《政治学》中说:高利贷是违反自然的,因为它取得的总是比给予的多。这就废除了一切美德的手段和尺度,即所谓的对等交换,算术上的相等……但是,拿别人的东西,偷窃或抢劫别人,是一种无耻生涯,这种人,对不起,就叫作盗贼,通常要处以绞刑;而高利贷者是高尚的盗贼,坐在安乐椅上;因此,人们称他们为坐在安乐椅上的强盗……异教徒根据理性得出了高利贷者是四倍盗贼和杀人犯的结论。而我们基督教徒却非常尊敬他们,几乎要为了他们的货币而崇拜他们……凡是吸尽、抢劫和盗窃别人营养的人,就是犯了使人饿死,使人灭亡的杀人大罪(杀多少,由他决定)。高利贷者就是犯了这样的大罪,他照理应当上绞架,如果他身上的肉多得足供许多乌鸦啄而分食,那末,他盗窃了多少古尔登,就应该被多少乌鸦去吃。但是他们却泰然坐在安乐椅上……重利盘剥者和高利贷者会大喊大叫:要遵守契约,要遵守盖了章的东西!法学家对此立即作了很好的回答:那是邪恶的契约。神学家也说,给恶魔立的契约,就是用血签的字盖的章也无效。因为凡是违反上帝、法律和自然的东西都等于零。因此,国君(只有他能做到这一点)应该立刻干预这件事,毁掉印章和契约,而不考虑……所以,在世界上人类再没有比守财奴和高利贷者更大的敌人了(恶魔除外),因为他想成为支配一切人的上帝。土耳其人、武夫、暴君都是恶人,但他们仍不得不让人们生活,并自认是恶人和敌人。他们有时还会同情甚至不得不同情某些人。而高利贷者和贪财之徒却想竭尽全力使整个世界毁灭于饥渴贫苦之中(到什么程度,由他决定),从而使他能独占一切,人人都把他奉为上帝,去领受他的恩赐,[943]永远成为他的奴隶。这时,他的心在欢跳,血在畅流。他披上貂皮长外套,带上金链指环,穿着华丽的衣服,擦擦油嘴,让人看来俨如尊贵的虔诚者,比上帝自己还仁慈得多,比圣母和一切圣徒还友爱得多……异教徒描写了海格立斯的伟大功勋,描写他怎样打败了许多怪物和可怕的恶魔,拯救了国家和人民。高利贷者是一个庞大可怕的怪物,象一只蹂躏一切的恶狼,比任何卡库斯、格里昂或安泰都厉害。但他却装出一副虔诚的样子,想使人无法知道被他倒着牵回洞穴去的公牛究竟到什么地方去了。”}\end{quote}

{这是对一般资本家的绝妙写照,资本家装出一副样子,好象他从别人那里拖回他的洞里去的东西是从他那里出来的,因为他使这些东西倒着走,看起来好象是从他的洞里走出来的。}

\begin{quote}{“然而海格立斯必然会听到公牛的吼声和俘虏的叫声,甚至到悬崖峭壁中去搜寻卡库斯,把公牛从恶汉手中拯救出来。所谓卡库斯就是指盗窃、抢劫和吞食一切的虔诚的高利贷者这个恶汉。他不承认自己做了恶事,并且认为谁也不会找到他,因为公牛是倒着牵回他的洞里去的,从足迹看来公牛似乎是被放走了。高利贷者正是想这样愚弄整个世界,似乎他带来了利益,他把公牛给了世界,其实他夺取了公牛并把它独吞了……所以,高利贷者和守财奴绝不是正直的人,他也作恶多端,毫无人道,他必然是一只恶狼,比一切暴君、杀人犯和强盗还凶狠,几乎和魔鬼一样可恶,但却没有被当作敌人,而是被当作朋友和市民那样受到共同的保护和亲善,可是他进行抢劫和谋杀,比一切敌人、杀人放火犯还凶狠。既然对劫路人、杀人犯和强盗应处以磔车刑或斩首,那就更应该把一切高利贷者处以磔车刑和斩首,把一切守财奴驱逐,革出教门,或斩首……”}\end{quote}

这一切描写得绘声绘色,同时也确切地抓住了旧式高利贷和一般资本的性质,揭穿了这种“幻想的损失”、这种对货币和商品“自然生出的损失的赔偿”、高利贷者会带来好处这种普遍性论调以及高利贷者的这种“虔诚的”外表:他和“别人”不同,他装出一副样子,拿别人东西好象是给别人东西,牵回去好象是放出来,等等!

\centerbox{※     ※     ※}

\begin{quote}{“拥有金银有很大的优越性,因为它提供了选择有利的购买时机的可能,它逐渐导致银行家行业的产生……银行家和旧的高利贷者不同,他贷款给富人,很少或根本不贷款给贫民。因此,他贷款时冒的风险较小,贷款条件可以较低;由于这两个原因,他就避免了民众对高利贷者的那种憎恶。”(弗·威·纽曼《政治经济学讲演集》1851年伦敦版第44页)}\end{quote}

随着高利贷和货币的发展,封建土地所有权的强制性让渡也发展起来了:

\begin{quote}{“能购买一切东西的货币的采用,以及由此而来的对贷款给土地所有者的贷出者的利益的维护,引起了为偿还债务而使土地所有权合法让渡的必要性。”(约翰·达尔林普尔《大不列颠封建所有制通史概论》1759年伦敦第4版第124页)[944]“按照托马斯·卡耳佩珀(1641年)、约瑟亚·柴尔德(1670年)、帕特森(1694年)的看法,财富取决于金银的利息率的哪怕是强制性的降低。这种看法在英国占统治地位几乎达两个世纪。”(加尼耳[《论政治经济学的各种体系》1821年巴黎第2版第1卷第58—59页])}\end{quote}

休谟同洛克相反,当他说明利息率决定于利润率时,\fnote{见本卷第1册第400—404页。——编者注}他已经看到了资本更高得多的发展。当边沁在十八世纪末写他的为高利贷辩护的著作\endnote{马克思指边沁的著作《为高利贷辩护》,1787年在伦敦出第一版,1790年出第二版,1816年出第三版。——第586、598页。}时,更是如此。

从亨利八世到安女王,在法律上都规定了降低利息率。

\begin{quote}{“在中世纪,任何一个国家都没有一般的利息率。首先对牧师有严格的规定。法庭对于借贷很少给予保障。因此,在个别场合,利息率就更高。由于货币的流通量少,而在大多数支付上必须使用现金,而且票据业务还不发达。因此,利息相差很悬殊,关于高利贷的概念差别也很大。在查理大帝时代,收取100%的利息,被认为是高利贷。1344年,在博登湖畔的琳道,本地市民收取216+(2/3)%的利息。在苏黎世,评议会规定43+(1/3)%为法定利息。在意大利,有时必须支付40%的利息,虽然从十二世纪到十四世纪,普通的利息率不超过20%。维罗那规定12+(1/2)%为法定利息。弗里德里希二世在他的命令中规定10%的利息率,但只是给犹太人规定的。他是不屑替基督徒说话的。早在十三世纪,10%已经是德国莱茵区的普通利息率了。”(休耳曼《中世纪城市》1827年波恩版第2集第55—57页)}\end{quote}

中世纪的巨额利息(只要不是从封建贵族等那里收取来的),在城市大部分是以商人和城市手工业者从农村诈骗来的巨额“让渡利润”为基础的。

除了象雅典等工商业特别发达的商业城市以外,在罗马,象在整个古代世界一样,对大土地所有者来说,高利贷不仅是剥夺小私有者即平民的手段,而且是占有他们人身的手段。

\begin{quote}{高利贷在罗马最初是自由的。十二铜表法(罗马城建立后303年)“规定货币的年利息为1%(尼布尔说是10%。)……这些法令很快就被破坏了……杜伊利乌斯(罗马城建立后398年)重新把年利率限制为1%(增长额为一盎斯)。在408年,这一利率降到1/2%。在413年,护民官格努齐乌斯主持的全民投票绝对禁止了有息贷款……在一个禁止市民从事产业、批发商业和零售商业的共和国,也禁止从事货币贸易,那是不奇怪的。这种情况延续了三百年,直到迦太基陷落。[后来允许收取不超过]12%的年利率。普通年利率是6%……查士丁尼规定的利率为4%。在图拉真时期,五盎斯的利息就是5%的法定利息……公元前146年,埃及法定的商业利息是12%”。(杜罗·德·拉·马尔《罗马人的政治经济学》1840年巴黎版第2卷第259—263页)[944]}\end{quote}

\centerbox{※     ※     ※}

[950a]关于利息,詹·威·吉尔巴特在《银行业的历史和原理》(1834年伦敦版)一书中写道:

\begin{quote}{“一个用借款来牟利的人,应该把一部分利润付给贷款人,这是不言而喻的自然公道的原则。一个人通常是通过商业来牟利的。但是在中世纪,纯粹是农业人口。在这种人口中和在封建统治下,交易是很少的,利润也是很小的。因此,在中世纪,取缔高利贷的法律是有道理的。况且,在一个农业国,一个人很少需要借钱,除非他由于不幸而陷入贫穷困苦的境地。”(第163页)“亨利八世把利息限为10%,詹姆斯一世限为8%,查理二世限为6%,安女王限为5%。”(第164—165页)“那时候,贷款人虽不是合法的垄断者,却是事实上的垄断者,所以,必须限制他们,就象限制其他的垄断者一样。在我们现代,利息率是由利润率规定的;在那个时候,利润率却是由利息率规定的。如果贷款人要商人负担很高的利息率,那末,商人就不得不提高他的商品的利润率。这样,大量货币就从买者的口袋里转到贷款人的口袋里。附加在商品上的这种追加价格,使群众减少了购买这些商品的能力和兴趣。”(第165页)}\end{quote}

十七世纪,约瑟亚·柴尔德在1669年写的《论商业和论货币利息降低所产生的利益》(译自英文,1754年阿姆斯特丹和柏林版。该书附有托马斯·卡耳佩珀1621年写的《论反对高利贷》)一书中,反驳托马斯·曼利(反驳他的论文《对货币利息的错误看法》\endnote{托马斯·曼利不是1668年在伦敦匿名出版的论文《对货币利息的错误看法》的作者,而是另一篇论文的作者,这篇论文在内容上和前者很相似,1669年在伦敦出版,标题是《利息为百分之六的高利贷。研究证明托马斯·卡耳佩珀和约·柴尔德先生对百分之六的利率的指责是不公正的》。《对货币利息的错误看法》这篇论文的作者是谁,没有查明。——第599页。}),称他为“高利贷者的卫士”。正如十七世纪英国经济学家的所有推论一样,柴尔德的推论的出发点自然是荷兰的财富,而在荷兰,利息率是低的。柴尔德认为这种低利息率是财富的原因,曼利则断定低利息率不过是财富的结果。

\begin{quote}{“要知道一个国家是穷还是富,只要问:货币利息的价格怎样?”(第74页)“他作为一帮心惊胆战的高利贷者的卫士,把大炮台建筑在我认为最不坚固的地点上……他直截了当地否认低利息率是财富的原因,而硬说这只是财富的结果。”(第120页)“当利息降低时,那些收回他们的贷款的人就不得不去购买土地〈土地价格由于购买者人数的增加而上涨〉,或者把货币投入商业。”(第133页)“当利息是6%时,谁也不会为了仅仅得到8—9%的利润而去冒险从事海运业,而得到4%或3%的利息的荷兰人,对这个利润却非常满意。”(第134页)“低利率和土地的高价格迫使商人继续不断地从事商业。”(第140页)“利率的降低能使一个民族养成节约的习惯。”(第144页)“如果使一国富裕的是商业,而压低利息又使商业扩大,那末,压低利息或限制高利贷,无疑是足以使一国致富的根本的和主要的原因。同一件事[950b]可以在一种情况下是原因,同时在另一种情况下又是结果,这种说法决不是荒谬的。”(第155页)“鸡蛋是母鸡的原因,而母鸡又是鸡蛋的原因。利息降低,可以使财富增加,而财富增加,又可以使利息进一步大大降低。通过立法也能做到降低利息。”(第156页)“我是勤劳的辩护者,而我的反对者却为懒惰和游手好闲辩护。”(第179页)}\end{quote}

在这里,柴尔德直接充当了产业资本和商业资本的卫士。[XV—950b]




% --------------------------------------

\cleardoublepage

\pagestyle{endnotestyle}

\printendnotes

\addcontentsline{toc}{chapter}{注释}

% --------------------------------------

\end{document}

% ------------------------------------------------------------------------------
% EOF
% ------------------------------------------------------------------------------
