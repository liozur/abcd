\tpartnonum{《剩余价值理论》第三册}

\tchapternonum{[第十九章]托·罗·马尔萨斯}

\tchapternonum{[(1)马尔萨斯把商品和资本这两个范畴混淆起来]}

[\endnote{马克思在论马尔萨斯这一章中,考察了马尔萨斯在李嘉图的《原理》出版(1871年)以后所写的著作。在这些著作里,马尔萨斯企图用旨在维护统治阶级中最反动阶层的利益的庸俗辩护论来对抗李嘉图的劳动价值论,对抗李嘉图的千方百计发展生产力的要求,而照李嘉图的观点,这样发展生产力,就应当牺牲个人的甚至整个阶级的利益。关于作为“人口论”的鼓吹者的马尔萨斯,在本章中只是附带谈了一下。马克思在《剩余价值理论》第2册《对所谓李嘉图地租规律的发现史的评论》那一章中对马尔萨斯论人口的著作做了一般的评述(见本卷第2册第121、123、125—128、158页)。——第3页。}XIII—753]这里要考察的是马尔萨斯的下列著作:

(1)《价值尺度。说明和例证》1823年伦敦版。

(2)《政治经济学定义》1827年伦敦版(还要参看约翰·卡泽诺夫1853年在伦敦出版

的这一著作及其所附卡泽诺夫的《注释和补充评论》)。

(3)《政治经济学原理》1836年伦敦第2版(要参看1820年或1820年前后的第1版)。

(4)还要注意一个马尔萨斯主义者\endnote{后来查明,这一匿名著作的作者是约翰·卡泽诺夫。——第3页。}(一个反对李嘉图学派的马尔萨斯主义者)的下述著作:《政治经济学大纲》1832年伦敦版。

马尔萨斯在他《论谷物法的影响》(1814年)这一著作中谈到亚·斯密时还说:

\begin{quote}{“斯密博士作了这样一番论证[即断言谷物的实际价格永远不变],显然是由于他习惯于把劳动〈即劳动价值〉看作价值的标准尺度,而把谷物看作劳动的尺度……无论劳动或其他任何商品都不能成为实际交换价值的准确尺度,这在现在已被认为是政治经济学的最明白不过的原理之一。实际上这正是从交换价值的规定本身得出来的。”\authornote{本卷引文中凡是尖括号〈〉和花括号{}内的话都是马克思加的。——译者注}[第11—12页]}\end{quote}

但是,马尔萨斯在他1820年的著作《政治经济学原理》中,却借用斯密的这个“价值的标准尺度”来反对李嘉图,而斯密自己在他理论上真正有所发展的地方从来没有使用过它。\endnote{马克思在他的著作的前几章中批判了斯密把劳动价值看作价值的标准尺度的观点,并且证明这一观点与斯密对价值的其他更深刻的见解相矛盾。见本卷第1册第54—55和140页,第2册第457—459页。——第4页。}在上面引用的那本论谷物法的著作中,马尔萨斯自己采用了斯密的另一个价值规定:价值决定于生产某一物品所必需的资本(积累劳动)和劳动(直接劳动)的量。

总之,不能不承认,马尔萨斯的《原理》以及要在某些方面对《原理》作进一步发挥的上述另两部著作的产生,在很大程度上是由于马尔萨斯嫉妒李嘉图的著作\endnote{指李嘉图的主要著作《政治经济学和赋税原理》1817年伦敦版。——第4页。}所取得的成就,并且企图重新爬上他在李嘉图的著作问世前作为一个剽窃能手所骗取到的首席地位。此外,在李嘉图的著作中对价值所作的规定尽管还是抽象的,但它是反对地主及其仆从们的利益的,而马尔萨斯却维护这些人的利益,比维护工业资产阶级的利益更为直接。同时,不可否认,马尔萨斯有在理论上故弄玄虚的某种兴趣。不过,他所以能够反对李嘉图,以及能够以这种方式来反对,只是因为李嘉图有种种自相矛盾之处。

马尔萨斯在反对李嘉图时用来作为出发点的,一方面是剩余价值的产生问题\endnote{马克思在《剩余价值理论》第二册中指出,李嘉图没有分析剩余价值的产生,劳动与资本的交换问题按照李嘉图的提法无法解决。(见本卷第2册第449—454和459—474页)。——第4页。},另一方面是李嘉图把不同投资领域中费用价格\endnote{“费用价格”(“Kostenpreis”或“Kostpreis,”“costprice”)这一术语,马克思用在三种不同的意义上:(1)资本家的生产费用(c+v),(2)同商品价值一致的商品的“内在的生产费用”(c+v+m),(3)生产价格(c+v+平均利润)。这里正文中用的是第三种意义,即生产价格。在《剩余价值理论》第二册中,“费用价格”这一术语(以及“平均价格”和“生产价格”)总是指生产价格,而在本册中,它有时也用来指资本家的生产费用(c+v),在这种情况下,本册中都译为“生产费用”(例如,第38、39III40、410、516页)。《Kostenpreis》这一术语所以有三种用法,是由于《Kosten》(“费用”、“生产费用”)这个词在经济科学中被用在三种意义上,正如马克思在《剩余价值理论》第三册中(见本册第81—86页和第569—570页)特别指出的,这三种意义是:(1)资本家预付的东西,(2)预付资本的价格加平均利润,(3)商品本身的实在的(或内在的)生产费用。除了资产阶级政治经济学古典作家使用的这三种意义以外,“生产费用”这一术语还有第四种庸俗的意义,即让·巴·萨伊给“生产费用”下的定义:“生产费用是为劳动、资本和土地的生产性服务支付的东西。”(让·巴·萨伊《论政治经济学》1814年巴黎第2版第2卷第453页)马克思坚决否定了对“生产费用”的这种庸俗的理解(例如见本卷第2册第142、239和535—536页)。——第4页。}的平均化看作价值规律本身的变形的观点,以及他始终把利润和剩余价值混淆起来(把两者直接等同起来)的做法。马尔萨斯并没有解决这些矛盾和概念的混乱,而是从李嘉图那里把它们接受过来,以便依靠这种混乱去推翻李嘉图关于价值的基本规律等等,并作出使他的保护人乐于接受的结论。

马尔萨斯的上述三部著作的真正贡献在于,他强调了资本和雇佣劳动之间的不平等交换,而李嘉图实际上却没有阐明,按价值规律(按商品中所包含的劳动时间)进行的商品交换中,如何产生出资本和活劳动之间、一定量的积累劳动和一定量的直接劳动之间的不平等交换,也就是说,实际上没有说明剩余价值的起源(因为在李嘉图那里资本是直接和劳动相交换,而不是和劳动能力相交换)。[754]后来的为数不多的马尔萨斯信徒之一——卡泽诺夫,在为马尔萨斯的上述著作《政治经济学定义》所写的序言中觉察到了这一点,因此他说:

\begin{quote}{“商品的交换和商品的分配〈工资、地租、利润〉应当分开来考察……分配的规律不完全取决于同交换有关的那些规律。”(该书序言第VI、VII页)}\end{quote}

在这里这无非是说,工资和利润的相互关系,——资本和雇佣劳动,积累劳动和直接劳动的交换,——并不直接同商品交换的规律相符合。

如果考察货币或商品作为资本的价值增殖,也就是说,不是考察它们的价值,而是考察它们的资本主义的价值增殖,那末,很明显,剩余价值无非是资本——商品或货币——所支配的劳动超过商品本身所包含的劳动量的那个余额,即无酬劳动。除了商品本身所包含的劳动量(它等于商品中的生产要素所包含的劳动与加到这些要素上的直接劳动之和)以外,商品还买到商品中所不包含的劳动余额。这个余额构成剩余价值;资本价值增殖的程度取决于这个余额的大小。商品换得的这个活劳动余额是利润的源泉。利润(确切些说,剩余价值)并不是来源于似乎与等价物即等量活劳动相交换的物化劳动,而是来源于在这个交换中没有被支付等价物而占有的那部分活劳动,来源于资本在这个虚假的交换中占有的无酬劳动。因此,如果把这一过程的中介环节撇开不谈,——由于李嘉图著作中没有这种中介环节,马尔萨斯就更有权撇开不谈,——如果只考察这一过程的实际内容和结果,那末价值增殖,利润,货币或商品之转化为资本,都不是由于商品按价值规律进行交换,即与它们所花费的劳动时间成比例地进行交换而发生的,相反倒是由于商品或货币(物化劳动)同比它所包含的或者说耗费的劳动多的活劳动相交换的结果。

马尔萨斯在上述那些著作中的唯一贡献是强调指出了这一点,而李嘉图对这一点却说得不那么清楚,因为他始终是以在资本家和工人间分配的成品为前提,却不去考察导致这一分配的中介过程——交换。可是后来这一贡献却化为乌有了,因为马尔萨斯把作为资本的货币或商品的价值增殖,因而也就是把它们在执行资本的特殊职能时的价值,同商品本身的价值混淆起来。因此,我们将会看到,他在以后的论述中又退回到货币主义的荒谬概念——让渡利润,\endnote{“让渡利润”(“Profituponexpropriation”或“Profituponalie-nation”)是詹姆斯·斯图亚特的用语。马克思在《剩余价值理论》第一册引用和分析过这个用语(见本卷第1册第12—13页)。——第8页。}完全陷入最可悲的混乱之中。这样,马尔萨斯不但没有超过李嘉图,反而在他的论述中企图使政治经济学倒退到李嘉图以前,甚至倒退到斯密和重农学派以前。

\begin{quote}{“在同一国家和同一时间,只能分解为劳动和利润的那些商品的交换价值,是由生产这些商品实际耗费的积累劳动和直接劳动,加上以劳动表示的全部预付的不断变动的利润额而得出的那个劳动量来准确衡量的。但这必然和这些商品所能支配的劳动量相同。”(《价值尺度。说明和例证》1823年伦敦版第15—16页)“商品所能支配的劳动是价值的标准尺度。”(同上,第61页)“我在任何地方都没有看到过〈指马尔萨斯自己的著作《价值尺度》出版以前〉这样的表述:某一商品通常支配的劳动量,必定可以代表并衡量生产这一商品花费的劳动量加利润。”(《政治经济学定义》1827年伦敦版第196页)}\end{quote}

和李嘉图不同,马尔萨斯先生想一下子把“利润”包括在价值规定之中,以便使利润直接从这个规定得出。由此可见,马尔萨斯感到了困难之所在。

不过,他把商品的价值和商品作为资本的价值增殖等同起来,是极其荒谬的。当商品或货币(简单说,物化劳动)作为资本同活劳动相交换时,它们所换得[755]的劳动量总是比它们本身所包含的劳动量大;如果把交换前的商品同它与活劳动交换后所得到的产品二者加以比较,就会发现,商品所换得的,是商品本身的价值(等价物)加上超过商品本身价值的余额即剩余价值。但是,如果因此说商品的价值等于它的价值加超过这个价值的余额,那是荒谬的。所以,只要商品作为商品同另一商品相交换而不是作为资本同活劳动相交换,那末,由于这里它是同等价物相交换,它所换得的物化劳动量就和它自身所包含的物化劳动量相等。

可见,值得注意的只是,马尔萨斯认为利润已经直接地现成地包括在商品的价值之中,并且有一点对他来说是清楚的,这就是:商品所支配的劳动量始终大于它所包含的劳动量。

\begin{quote}{“正因为某一商品通常所支配的劳动,等于生产这一商品实际花费的劳动加利润,所以我们有理由认为它〈劳动〉是价值的尺度。因此,如果认为商品的一般价值决定于商品供给的自然的和必要的条件,那末毫无疑问,只有它通常所能支配的劳动才是这些条件的尺度。”(《政治经济学定义》1827年伦敦版第214页)“基本生产费用恰恰是商品供给条件的等价表现。”(卡泽诺夫出版的《政治经济学定义》1853年伦敦版第14页)“商品供给条件的尺度是商品在其自然和通常状况下所能交换的劳动量。”(同上)“某一商品所支配的劳动量,正好代表生产这一商品花费的劳动量加预付资本的利润,因此,它真正代表衡量商品供给的自然和必要的条件,代表那些决定价值的基本生产费用”。(同上,第125页)“对某一商品的需求虽然同买者愿意并能够用来和它交换的另一商品的量不相适应,但是它确实同买者为该商品付出的劳动量相适应;其原因是:某一商品通常所支配的劳动量正好代表对该商品的有效需求,因为该劳动量正好代表这种商品供给所必需的劳动和利润的意量;而商品在某一时间所支配的实际劳动量如果偏离了通常的数量,那就代表由于暂时原因而引起的需求过多或需求不足。”(同上,第135页)}\end{quote}

马尔萨斯在这里也是正确的。商品供给的条件,即在资本主义生产基础上的商品生产或者更确切说再生产的条件,就在于商品或它的价值(由商品转化成的货币)在它生产或再生产过程中换得比它本所身所包含的劳动量大的劳动量;因为生产商品仅仅是为了实现利润。

例如,一个棉织厂主卖掉了他的棉布。新棉布的供给条件是,在棉布的再生产过程中,厂主以货币(即棉布的交换价值)换得比棉布原来包含的或以货币表现的劳动量大的劳动量。因为棉织厂主是作为资本家生产棉布的。他要生产的不是棉布,而是利润。生产棉布只是生产利润的手段。由此应得出什么结果呢?所生产的棉布比所预付的棉布包含更多的劳动时间,更多的劳动。这种剩余劳动时间,剩余价值,也表现为剩余产品并不补偿用来交换劳动的棉布多的棉布。因此,一部分产品并不补偿用来交换劳动的那些棉布,而构成属于厂主的剩余产品。换句话说,如果我们考察全部产品,那末,每一码棉布中都有一定部分(或者说每码棉布的价值中都有一定部分)没有被支付任何等价物,它代表无酬劳动。可见,如果厂主将一码棉布按照它的价值出卖,就是说,用它同包含等量劳动时间的货币或商品相交换,那他就是实现了一定数额的货币或得到了一定数量的商品而未付任何代价。因为他出卖棉布不是按照他支付过报酬的劳动时间,而是按照这一码棉由里所包含的劳动时间,在这个劳动时间中有一部分[756]厂主并没有支付过报酬。棉布包含的劳动时间例如等于12先令。其中厂主只支付了8先令。如果他按棉布的价值出卖,卖得12先令,他就赚了4先令。

\tchapternonum{[(2)马尔萨斯所解释的庸俗的“让渡利润”见解。马尔萨斯对剩余价值的荒谬观点]}

至于说到买者,那末,根据假定,他在任何情况下都只支付棉布的价值,也就是说,他付出的货币额所包含的劳动时间和棉布所包含的一样多。这里可能有三种情况。(1)买者是资本家。他用来支付棉布的货币(即商品的价值)也包含一部分无酬劳动。因此,如果说一个出卖无酬劳动,那末另一个则用无酬劳动来购买。他们各自实现了无酬劳动,一个以卖者的身分实现,另一个以买者的身分实现。(2)或者买者是独立生产者。在这种情况下,他以等价物换取等价物。卖者以商品形式卖给他的劳动是否支付过报酬,与他根本无关。他得到的物化劳动和他付出的一样多。(3)最后,或者买者是雇佣工人。在这种情况下——假定商品按它的价值出卖,——他也和其他任何买者一样,用他的货币买得商品形式的等价物。他所得到的商品形式的物化劳动和他以货币形式付出的一样多。但是,他为换取构成他的工资的货币付出的劳动却比这些货币中包含的劳动要多。他补偿了货币中包含的劳动,还加上了他无偿地付出的剩余劳动。因此,他为货币支付的代价,超过了货币的价值,从而他为货币等价物(棉布等)支付的代价,也超过这种等价物的价值。可见,对他这个买者来说,费用要比对任何一个商品的卖者来说都大,尽管他在商品形式上为自己的货币取得等价物;但是,在货币形式上他却没有为他的劳动取得等价物,相反,他在劳动中付出的比等价物多。可见,工人是唯一高于商品价值来支付一切商品的买者,甚至在他按照商品价值购买商品时也是这样,因为他用超过货币价值的劳动量购买了一般等价物货币。对于卖商品给工人的人来说并没有因此得到任何好处。工人付给他的并不比其他任何买者付给的多,工人支付的是劳动创造的价值。资本家把工人生产的商品又卖给工人,他的确通过这种出卖实现了利润,但这只不过是他把商品卖给其他任何买者时所实现的那种利润。资本家把商品卖给这个工人时所得的利润,其来源并不是他高于商品价值把商品卖给工人,而是在此以前,事实上是在生产过程中,他低于商品的价值向工人购买了商品。

所以,马尔萨斯先生既然把商品作为资本的价值增殖变成商品的价值,也就前后一贯地把所有买者都变成雇佣工人,也就是说,他硬使所有买者不是用商品,而是用直接劳动同资本家相交换,硬使他们交回给资本家的劳动多于商品中包含的劳动,可是实际上,资本家的利润的产生却是由于他出卖的是商品中包含的全部劳动,而已经支付的只是商品中包含的一部分劳动。因此,如果说李嘉图的困难在于,商品交换规律无法直接解释资本和雇佣劳动之间的交换,反而似乎与这一交换相矛盾,那末,马尔萨斯却用把商品的购买(交换)变成资本和雇佣劳动之间的交换这样一个办法来解决这个困难。马尔萨斯所不理解的就是商品中包含的劳动总量和商品中包含的有酬劳动量之间的差额。正是这个差额构成利润的源泉。而马尔萨斯下一步就不可避免地这样得出利润:卖者出卖商品不仅高于他为商品所花费的(资本家正是这样做的),而且高于商品所值,这就是说,马尔萨斯回到“让渡利润”的庸俗观点,即认为剩余价值的产生是由于卖者高于商品价值出卖商品(也就是换得比商品中包含的劳动时间多的劳动时间)。这样一来,某人作为某一商品的卖者所赚得的,也就是他作为另一商品的买者所亏损的,因而完全不能理解,通过价格的这种普遍的名义上的提高,会有什么实际的“赢利”。[757]尤其不可理解的是,整个社会怎能由此而致富,真正的剩余价值或真正的剩余产品怎能由此而形成。这真是荒唐而愚蠢的见解。

我们已经看到\authornote{见本卷第1册第3章和第4章。——编者注},亚·斯密曾素朴地表述了一切相互矛盾的因素,因而他的学说成了截然相反的各种观点的源泉和出发点。马尔萨斯先生以亚·斯密的论点为依据,作了一种混乱的、然而是建立在正确地感觉和意识到有待克服的困难的基础上的尝试,企图用一种新的理论与李嘉图的理论相对抗,从而保持其“首席地位”。从这种尝试到荒谬的庸俗观点的过渡是这样实现的:

如果我们考察商品作为资本的价值增殖,即考察商品与活的生产劳动相交换,那末,商品除了它本身包含的劳动时间——即工人所再生产的等价物——之外,还支配形成利润源泉的剩余劳动时间。如果我们现在把商品的这种价值增殖变为商品的价值,那末商品的每一个买者都必须作为工人与商品发生关系,就是说,他在购买时除了商品中包含的劳动量之外,还要另外再付出一定数量的剩余劳动。既然除了工人以外其他买者都不是作为工人与商品发生关系{我们已经看到,即使工人单纯作为商品买者出现时,先前的、原有的差别仍然间接地存在},那末就必须假定:买者虽然不直接付出比商品中包含的更多的劳动量,但是——这其实是一回事——要付出一个包含更多劳动量的价值。上述的过渡就是靠这种“更多的劳动量或者说——这其实是一回事——包含更多劳动量的价值”实现的。总之,问题实?上可归结成这样:商品的价值就是买者为商品支付的价值,这个价值等于商品的等价物(价值)加超过这个价值的余额,即剩余价值。于是就得出这样一个庸俗观点:利润在于商品贱买贵卖。买者购买商品所花费的劳动或物化劳动多于卖者为商品所花费的。

但是,如果买者本身是资本家,是商品的卖者,而且他用来购买的货币只代表他所卖出的商品,结果就只能是:双方都过贵地出卖自己的商品,从而相互欺诈,而且只要双方都仅仅实现一般利润率,欺诈的程度也就相同。那末,应该到哪里去找付给资本家的劳动量等于资本家的商品中包含的劳动加资本家的利润的买者呢?举个例子。卖者为商品花费了10先令。他把商品卖了12先令。这样,他支配的劳动不只是10先令,而且多了2先令。但是买者同样把他值10先令的商品卖了12先令。这样,他们各自作为买者所亏损的正是他们作为卖者所赚得的。工人阶级是唯一的例外。因为,既然产品价格被提高到它的费用之上,工人就只能买回产品的一部分,这样,产品的另一部分,或该部分的价格,就构成资本家的利润。但是,既然利润恰恰是由于工人只能买回产品的一部分而获得的,那末资本家(资本家阶级)就决不能[仅仅]靠工人的需求来实现自己的利润,决不能靠全部产品同工资相交换来实现利润,相反只能靠工人的全部工资同仅仅一部分产品相交换来实现利润。可见,除了工人以外,还必须有其他需求和其他买者,否则就不会产生任何利润。这些买者从哪里来呢?如果他们本身是资本家,是卖者,那就会发生上述的资本家阶级的自相欺诈,因为他们互相在名义上提高他们商品的价格,他们各自作为卖者所赚得的正是他们作为买者所亏损的。因此,必须有不是卖者的买者,资本家才能实现他的利润,才能“按照商品的价值”出卖商品。所以就必须有地主、年金领取者、领干薪者、牧师等等以及他们的家仆和侍从。至于这些“买者”怎样占有[758]购买手段,也就是说,他们怎样必须不付等价物而先从资本家那里取得一部分产品,以便用这样取得的东西买回少于它的等价物的商品,马尔萨斯先生没有加以说明。不管怎样,由此就产生了他的为下述主张辩护的论据:尽可能多地增加非生产阶级,好让商品的卖者找到市场,为自己的供给找到需求。这样,再接下去,人口论小册子\endnote{这里指的是马尔萨斯的有名著作《人口原理》,该书第一版是1798年在伦敦匿名出版的。在这一著作中,马尔萨斯断言,劳动群众的贫困似乎是由于人口有按几何级数增加的趋势,而消费品的数量最多只能按算术级数增加。——第15页。}的作者就鼓吹,经常的消费过度和寄生者占有尽可能多的年产品是生产的条件。除了从他的理论必然产生的这一论据之外,还有一个进一步的辩护论据:资本代表对抽象财富的欲望,对价值增殖的欲望,但是,这种欲望只是由于有代表支出欲望、消费欲望、奢侈欲望的购买者阶级存在,也就是说,有那些是买者而不是卖者的非生产阶级存在才能实现。

\tchapternonum{[(3)十九世纪二十年代马尔萨斯主义者和李嘉图主义者之间的争论。他们在对待工人阶级的立场方面的共同点]}

在这个基础上,在二十年代(从1820年到1830年这段时间,总的说来,是英国政治经济学史上一个大的形而上学的时代),马尔萨斯主义者和李嘉图主义者之间发生了一场绝妙的争吵。李嘉图主义者象马尔萨斯主义者一样地认为,必须使工人自己不占有自己的产品,这个产品的一部分要归资本家所有,以便使他们(即工人)有生产的刺激,从而保证财富的增长。但是李嘉图主义者激烈地反对马尔萨斯主义者的下述观点:地主,国家和教会的领干薪者,以及一大帮游手好闲的仆从,必须首先占有资本家的一部分产品而不付任何等价物(正象资本家对工人那样),然后从资本家那里购买资本家自己的商品,并为他们提供利润,——虽然李嘉图主义者对于工人却持同样的主张。按照李嘉图主义者的学说,为了使积累从而也使对劳动的需求增加,工人必须把自己产品中尽可能大的一部分无偿地让给资本家,以便资本家把由此增加的纯收入再转化为资本。马尔萨斯主义者也是这样论证的。按照他们的意见,应该以地租、税收等等形式从产业资本家那里无偿地索取尽可能大的一部分,以便他们能把余下的部分卖给这些强加给他们的“分享者”而获得利润。工人不应占有自己的产品,这样才不致丧失劳动的刺激——李嘉图主义者和马尔萨斯主义者都这样说。[马尔萨斯主义者说]产业资本家必须把他的一部分产品让给只从事消费的阶级——“为享受果实而生的人们”\authornote{见贺雷西《书信集》。——编者注},——好让这些阶级再拿产业资本家让给他们的东西在对他们不利的条件下和产业资本家进行交换。否则资本家就要丧失生产的刺激,而这种刺激恰恰在于,资本家取得高额利润,大大高于商品价值出卖商品。以后我们还要回过来再谈这场滑稽的论战。

\tchapternonum{[(4)马尔萨斯片面地解释斯密的价值理论。他在同李嘉图论战中利用斯密的错误论点]}

现在我们首先来证明:马尔萨斯陷入了一种十分粗俗的观念。

\begin{quote}{“无论商品要通过多少次中间交换,无论生产者是把他们的商品运到中国,还是就在产地出卖,商品能否取得适当的市场价格的问题,完全取决于生产者能否补偿他们的资本并取得普通利润,从而能够顺利地继续他们的营业。但他们的资本是什么呢?正如亚·斯密指出的,资本是用来劳动的工具、被加工的材料、以及支配必要劳动量的手段。”}\end{quote}

(马尔萨斯认为,这也就是用在商品生产上的全部劳动。利润是超过这种用在商品生产上的劳动的余额。因而事实上这只不过是商品生产费用上的名义附加额。)为了使人们对他的看法不留下任何疑问,马尔萨斯还以赞同的态度引用了托伦斯上校的《论财富的生产》一书(第6章第349页)来证实他自己的观点。

\begin{quote}{“有效的需求在于,消费者{买者和卖者之间的对立在这里变成消费者和生产者之间的对立}[759]通过直接的或间接的交换能够和愿意付给商品的部分,大于生产它们时所耗费的资本的一切组成部分。”(卡泽诺夫出版的《政治经济学定义》第70—71页)}\end{quote}

而马尔萨斯的《定义》一书的出版者、辩护者和注释者卡泽诺夫先生自己则说:

\begin{quote}{“利润不取决于商品互相交换的比例}\end{quote}

{这就是说,如果考察的只是资本家之间的商品交换,那末,由于这里不存在资本家同工人之间的交换(工人除劳动之外没有任何其他商品可以交换),马尔萨斯的理论就会表现为这样一种谬论,即彼此单纯给商品价格加上一个附加额,一个名义上的附加额。因此,不得不撇开商品交换,而谈不生产任何商品的人之间的货币交换},

\begin{quote}{因为同一比例在任何利润高度上都能存在,而取决于对工资的比例,或者说对抵补原有费用所需的比例,这个比例在任何情况下都决定于买者为取得商品而作出的牺牲(或他付出的劳动的价值)超过生产者为使商品进入市场而作出的牺牲的程度。”(同上,第46页)}\end{quote}

为了获得这样奇妙的结果,马尔萨斯必须在理论上大耍花招。首先,在抓住亚·斯密学说的一个方面,即商品的价值等于商品支配的劳动量,或支配商品的劳动量,或商品交换的劳动量这一主张的同时,必须消除亚·斯密本人以及其后的经济学家,其中包括马尔萨斯,对商品的价值——价值——可以成为价值尺度这一论点所提出的异议。

马尔萨斯的《价值尺度。说明和例证》(1823年伦敦版)一书是愚蠢的真正典型,它用诡辩来自我陶醉,在自己内在的概念混乱中辗转迂回;它的晦涩、拙劣的叙述,给天真的、不内行的读者留下这样一个印象:如果读者弄不清楚这一团混乱,那末其困难不在于混乱与清楚之间的矛盾,而在于读者的理解力太差。

马尔萨斯首先必须把大卫·李嘉图在“劳动的价值”和“劳动量”之间所作的划分\endnote{关于李嘉图的“劳动的价值”和“劳动量”的概念,见本卷第2册第449—459页。——第18页。}重新抹掉,并把斯密的[不同价值规定的]并列归结到一个错误方面。

\begin{quote}{“一定的劳动量,必定具有同支配它或者它实际上交换的工资相等的价值。”(《价值尺度。说明和例证》1823年伦敦版第5页)}\end{quote}

这句话的目的就是把劳动量和劳动的价值这两个用语等同起来。

这句话本身纯粹是同义反复,是荒谬的陈辞滥调。既然工资或者说“它〈一定的劳动量〉所交换的”东西构成这个劳动量的价值,那末,说一定的劳动量的价值等于工资或等于这个劳动所交换的货币量或商品量,就是同义反复。换句话说,这不过意味着,一定的劳动量的交换价值等于这一劳动量的交换价值,或者叫作工资。但是,{且不说直接同工资相交换的不是劳动,而是劳动能力,正是这个混淆造成了谬误},决不能从上述同义反复中得出这样的结论:一定的劳动量等于工资中或者说构成工资的货币或商品中包含的劳动量。假定一个工人劳动12小时,得到6小时的产品作为工资,那末这6小时的劳动产品就构成12小时劳动的价值(因为它是用来换取12小时劳动的工资,商品)。不能由此推论说,6小时劳动等于12小时,或者代表6小时的商品等于代表12小时的商品;也不能说,工资的价值等于代表[同该工资相交换的]劳动的产品的价值。由此只能得出结论说,劳动的价值(因为它是用劳动能力的价值,而不是用劳动能力所完成的劳动来衡量的)、[760]一定劳动量的价值所包含的劳动,少于它所买到的劳动;因此,代表所买到的劳动的那些商品的价值和用来购买或支配这一定劳动量的那些商品的价值,是大不相同的。

马尔萨斯先生得出了直接相反的结论。因为一定劳动量的价值等于它的价值,按照马尔萨斯的意见,就可以得出结论:代表这个劳动量的价值等于工资的价值。按照马尔萨斯的意见,由此还可以得出结论:商品所吸收和包含的直接劳动(即扣除生产资料后剩下的劳动)创造的价值并不比为它支付的价值大;它只再生产工资的价值。所以不言而喻,如果商品的价值由商品所包含的劳动决定,利润就无法解释,而必须用别的源泉解释利润,——如果已经假定商品的价值必须包括它所实现的利润。因为用在商品生产上的劳动包括:(1)被磨损的因而再现于产品价值中的机器等等所包含的劳动;(2)使用的原料所包含的劳动。这两个要素自然不会因为它们成为新商品的生产要素而使它们在新商品生产前本来包含的劳动量有所增加。于是,剩下的是(3)包含在工资中的、与活劳动相交换的劳动。但是,按照马尔萨斯的意见,这种活劳动并不比它所交换的物化劳动多。因此,商品不包含任何无酬劳动部分,只包含补偿等价物的劳动。由此可以得出如下的结论:如果商品的价值由商品中所包含的劳动决定,它就不提供任何利润。如果它提供利润,那末,按照马尔萨斯的意见,这就是商品价格超过商品中所包含的劳动的余额。因而,为了使商品按照它的价值(包括利润在内的价值)出卖,商品必须支配这样一个劳动量,它等于用在商品生产上的劳动加上一个代表商品出卖时所实现的利润的劳动余额。

\tchapternonum{[(5)马尔萨斯对斯密关于不变的劳动价值这一论点的解释]}

其次,马尔萨斯为了证明劳动——不是生产所需要的劳动量,而是作为商品的劳动——是价值的尺度,他断言:

\begin{quote}{“劳动的价值是不变的。”(《价值尺度。说明和例证》第29页注)}\end{quote}

{这种说法决不是什么创见,而是亚·斯密《国富论》第1卷第5章(加尔涅的法译本,第1卷第65—66页)中下述论点的改写和进一步发挥:

\begin{quote}{“等量劳动,在任何时候和任何地方,对于完成这一劳动的工人必定具有相同的价值。在通常的健康、体力和精神状况下,在工人能够掌握通常的技能和技巧的条件下,他总要牺牲同样多的休息、自由和幸福。他所支付的价格总是不变的,不管他用这一价格换得的商品量有多少。诚然他用这个价格能买到的这些商品的量有时多有时少,但这里发生变化的是这些商品的价值,而不是购买商品的劳动的价值。在任何时候和任何地方,难于得到或者说要花费许多劳动才能得到的东西总是贵的,而容易得到或者说花费不多的劳动就能得到的东西总是便宜的。由此可见,劳动本身的价值永远不变,所以劳动是唯一真实的和最终的尺度,在任何时候和任何地方都可以用这个尺度来衡量和比较一切商品的价值。”}}\end{quote}

{其次,马尔萨斯如此引为骄傲的并扬言是他最早提出的一个发现(即价值等于商品中所包含的劳动量加代表利润的劳动量的论点),看来也只不过是把斯密以下两句话拼凑在一起(马尔萨斯始终不失为一个剽窃者):

\begin{quote}{“价格的各个不同构成部分的实际价值,是以每一构成部分所能购买或支配的劳动量来衡量的。劳动不仅衡量价格中归结为劳动的部分的价值,而且还衡量归结为地租的部分和归结为利润的部分的价值。”(第1卷第6章,加尔涅的译本,第1卷第100页)}}\end{quote}

[761]根据这一点马尔萨斯说:

\begin{quote}{“如果对劳动的需求增加了,那末,工人的较高工资就不是由劳动价值的提高,而是由劳动所交换的产品的价值的降低引起的。在劳动过剩的情况下,工人的低工资是由产品价值的提高,而不是由劳动价值的降低引起的。”(《价值尺度。说明和例证》第35页,并参看第33—34页)}\end{quote}

贝利很好地嘲笑了马尔萨斯对于劳动价值不变的论证(指马尔萨斯的进一步论证,而不是指斯密的论点;也指一般关于不变的劳动价值的论点):

\begin{quote}{“我们可以用同样的方法证明任何物品都具有不变的价值;可以以10码呢绒为例。因为不管我们对这10码呢绒付出5镑还是10镑,付出的金额在价值上总是等于用这笔钱换得的这块呢绒,或者换句话说,这个金额对这块呢绒来说具有不变的价值。但是,用来换取具有不变价值的物的东西,本身必须是不变的;所以这10码呢绒必须具有不变的价值……如果说,工资虽然在数量上有变化,但支配的劳动量不变,因此具有不变的价值,那末这种说法正同所谓买帽子付出的金额虽然时多时少,但总是买到一顶帽子,因此它具有不变的价值这种说法一样不足取。”(《对价值的本质、尺度和原因的批判研究,主要是论李嘉图先生及其信徒的著作》1825年伦敦版第145—147页)}\end{quote}

贝利在这同一本著作中,非常尖刻地嘲笑了马尔萨斯用来“说明”他的价值尺度的那些荒谬的、自以为高明的计算表格。

马尔萨斯在他的《政治经济学定义》(1827年伦敦版)中对贝利的讥讽大发雷霆,同时他试图这样来证明劳动价值不变:

\begin{quote}{“随着社会的进步,许多商品,如原产品,和劳动相比,价格上涨,而工业品的价格却下降。因此,差不多可以这样说:一定的劳动量在同一国家中支配的商品量,平均说来,在几百年的过程内不可能发生重大的变化。”(《定义》1827年伦敦版第206页)}\end{quote}

马尔萨斯还象证明“劳动价值不变”一样绝妙地证明:工资的货币价格的提高,必然引起商品的货币价格的普遍提高。

\begin{quote}{“如果货币工资普遍提高,货币的价值将相应地下降;而当货币的价值下降时……商品的价格总是上涨。”(同上,第34页)}\end{quote}

如果货币的价值同劳动相比降低了,那末恰恰需要证明:所有商品的价值同货币相比提高了,或者说,不是用劳动而是用其他商品计算的货币价值降低了。而马尔萨斯证明这一点的办法,却是事先就把这一点当作前提。

\tchapternonum{[(6)马尔萨斯利用李嘉图关于价值规律的变形的论点反对劳动价值论]}

马尔萨斯反对李嘉图的价值规定所持的论据,完全来自李嘉图本人最先提出的一个论点,即认为商品的交换价值的变动与生产商品时所花费的劳动量无关,它是因流通过程中产生的资本构成上的区别——流动资本和固定资本的比例不同,所用固定资本的耐久程度不同,流动资本的周转时间不同——引起的。简言之,马尔萨斯反对李嘉图所持的论据,来自李嘉图的费用价格和价值的混淆,因为李嘉图把不依各个生产领域使用的劳动量为转移的费用价格的平均化看作是价值本身的变形,从而把整个原理推翻了。马尔萨斯抓住李嘉图本人所强调的并且是他最先发现的那些违反价值决定于劳动时间这个规定的矛盾,不是为了解决矛盾,而是为了倒退到完全荒谬的观念上去,为了把说出互相矛盾的现象即用语言把这些现象表达出来,当成解决矛盾。我们在考察李嘉图学派的解体时,还会看到[詹姆斯·]穆勒和麦克库洛赫也使用了同样手法。他们试图用烦琐的荒谬的定义和区分,把与普遍规律相矛盾的现象胡说成直接同普遍规律一致,以便在自己的议论中避开这些现象,不过这样一来,基础本身也就不存在了。

在下面引用的马尔萨斯著作中的一段话里,马尔萨斯利用了李嘉图本人提供的违反价值规律的材料来反对李嘉图:

\begin{quote}{“斯密说过,谷物一年就可成熟,而肉用牲畜却需要喂养4—5年才能屠宰;因此,如果我们拿交换价值相等的一定数量的谷物和一定数量的肉相比较,那就可以肯定,不考虑其他因素,单是多出的3年或4年的利润(按生产肉类使用的资本15%计算)的差额,就会使一个少得多的劳动量[762]在价值上得到补偿。可见,两个商品的交换价值可以相等,而一个商品中的积累劳动和直接劳动却比另一个少40%或50%。对任何一个国家的大量最重要的商品来说,这是常见的事情;如果利润从15%降到8%,肉的价值和谷物相比就会降低20%以上。”(《价值尺度。说明和例证》第10—11页)}\end{quote}

既然资本由商品构成,并且加入资本或构成资本的商品有很大一部分具有这样一种价格(也就是普通意义上的交换价值),这种价格不仅包括积累劳动和直接劳动,而且——就我们考察的只是这种特殊商品来说,——还包括一个因加上平均利润而形成的纯粹名义上的价值附加额,所以马尔萨斯说:

\begin{quote}{“劳动不是用于生产资本的唯一要素。”(卡泽诺夫出版的《定义》第29页)“什么是生产费用呢?……就是生产商品所需要的和生产商品时消费的工具和材料中所包含的实物形式的劳动量,加上一个相当于预付资本在整个预付期间的普通利润的附加量。”(同上,第74—75页)“根据同样的理由,穆勒先生把资本叫作积累劳动是非常错误的。人们也许可以把资本叫作积累劳动加利润,但肯定不能单单叫作积累劳动,除非我们决定把利润叫作劳动。”(同上,第60—61页)“说商品的价值由生产商品所必需的劳动量和资本量来调节或决定,是完全错误的。说商品的价值由生产商品所必需的劳动量和利润量来调节,是完全正确的。”(同上,第129页)}\end{quote}

关于这一点,卡泽诺夫在第130页的注释中说:

\begin{quote}{“‘劳动和利润’的说法可能遭到这样的反驳,说这两者不是互相关连的概念,因为劳动是动因,利润是效果,一个是因,一个是果。因此,西尼耳先生用‘劳动和节欲’的说法取而代之〈按西尼耳的说法是:“谁把自己的收入转化为资本,谁就是节制了如果花费这笔资本就能获得的享受”〉……但是必须承认,利润的原因不在于节欲,而在于生产地使用资本。”}\end{quote}

绝妙的解释!商品的价值由包含在商品中的劳动加利润构成;由包含在商品中的劳动和不包含在商品中但购买商品时必须支付的劳动构成。

马尔萨斯继续反驳李嘉图说:

\begin{quote}{“李嘉图断言,利润随着工资价值的提高而按比例地下降,反之亦然。这种说法只有假定在其生产上耗费相等劳动量的商品始终具有相等价值的条件下才是正确的。而这种假定在五百次里难得有一次可以成立,而且必然如此,因为随着文明和技术的进步,使用的固定资本量不断增加,流动资本的周转时间则越来越不相同和不相等。”(《定义》1827年伦敦版第31—32页)〈在卡泽诺夫的版本第53—54页上,马尔萨斯的这段话,在文字上完全一样:“事物的自然状态”使李嘉图的价值尺度变了样,因为这种状态造成一种趋势:“随着文明和技术的进步,使用的固定资本量不断增加,流动资本的周转时间则越来越不相同和不相等。”〉“李嘉图先生自己也承认他的规则有相当多的例外;但是如果我们考察一下这些他所谓的例外的情况,即使用的固定资本量不同,耐久程度不同,使用的流动资本周转时间不同,那末我们就会发现,这些例外情况如此之多,以致规则可以看作例外,而例外可以看作规则。”(卡泽诺夫出版的《定义》第50页)}\end{quote}

\tchapternonum{[(7)马尔萨斯的庸俗的价值规定。把利润看成商品价值附加额。马尔萨斯对李嘉图相对工资见解的反驳]}

根据上面所说,马尔萨斯还提出了这样一个价值规定\endnote{这个价值规定是卡泽诺夫根据马尔萨斯和亚当·斯密的意见表述的,而马尔萨斯是从亚当·斯密那里借用了商品的价值决定于用这个商品可以买到的活劳动量这一规定。——第25页。}:

\begin{quote}{“价值是对商品的估价,这种估价的根据是买者为商品付出的费用,或者说买者为了得到它而必须作出的牺牲,这种牺牲用他为交换这一商品而付出的劳动量来衡量,或者也可以说用这一商品所支配的劳动来衡量。”(同上,第8—9页)}\end{quote}

卡泽诺夫还指出了马尔萨斯和李嘉图的区别:

\begin{quote}{[763]“李嘉图先生同亚·斯密一起,把劳动当作费用的真正尺度;但是他只是用它来衡量生产者的费用……它同样可以用作买者的费用的尺度。”(同上,第56—57页)}\end{quote}

换句话说:商品的价值等于买者所必须支付的货币额,这一货币额可以最准确地用它所能购买的普通劳动量来估量\authornote{马尔萨斯先假定利润的存在,然后就可以用一个外在尺度来衡量它的价值量。他没有涉及利润的产生和内在可能性的问题。}。但这一货币额又由什么决定,这一点当然没有说明。我们这里看到的是日常生活中人们对这种事情的十分粗俗的观念。用莫测高深的语言来表达的不过是肤浅的见解。换句话说,这无非是把费用价格和价值等同起来,——这种混同,在亚·斯密著作中,尤其是在李嘉图著作中是和他们的实际分析相矛盾的,而马尔萨斯却把它奉为规律。因此,这是沉湎于竞争、只看到竞争造成的表面现象的市侩所特有的价值观。费用价格究竟是由什么决定的呢?由预付资本的量加利润决定。而利润又是由什么决定的呢?利润的基金是从哪里来的呢?代表这一剩余价值的剩余产品是从哪里来的呢?如果问题只在于名义上提高货币价格,那末提高商品的价值是最容易的事了。预付资本的价值又由什么决定呢?马尔萨斯说,是由预付资本中包含的劳动的价值决定的。劳动的价值又由什么决定呢?是由花费工资购买的商品的价值决定的。而这些商品的价值又由什么决定呢?由劳动的价值加利润。这样,我们只好不断地在循环论证里兜圈子。假定付给工人的真是他的劳动的价值,也就是说,构成他的工资的那些商品(或货币额)等于他的劳动物化在其中的商品的价值(货币额),那末,他要是得到100塔勒工资,他加到原料等等上面的,简言之,加到预付[不变]资本上面的总共也就是100塔勒的价值。在这种情况下,利润无论如何只能由卖者在出卖商品时加在商品的实际价值上的附加额构成。所有的卖者都这样做。因此,只要是资本家彼此交换商品,那就谁也不能通过这种附加额实现任何东西,根本不能通过这种方法形成一个可供他们从中汲取收入的剩余基金。只有那些生产加入工人阶级的消费的商品的资本家,才能获得一个实际的、而不是虚构的利润,因为他们卖回给工人的商品比他们向工人购买的商品贵。他们用100塔勒从工人那里购买来的商品,又以110塔勒卖回给工人,也就是说,他们只把产品的10/11卖回给工人,而把1/11留给自己。但这仅仅意味着,例如工人做工11小时,只给他10小时的报酬,只给他10小时的产品,而1小时或者说1小时的产品,无代价地归资本家所有。而这也就意味着——就这里是同工人阶级发生关系而言——利润的产生是由于工人阶级把自己劳动的一部分白白送给资本家,因而“劳动量”和“劳动价值”不相等。但其他资本家却不会有这种出路,因而只能获得虚构的利润。

此外,下面这一段话令人信服地表明马尔萨斯多么不理解李嘉图的基本论点,他根本不懂得利润能够不通过价值附加额而通过别的办法获得。

\begin{quote}{“可以承认,直接制成并直接供人使用的最初商品是纯粹劳动的结果,因而它们的价值由这一劳动的量决定;但是这种商品作为资本来帮助生产其他商品时,资本家在一定时期内就必然不能使用他的预付,因之也就必然要求以利润形式取得报酬。在社会发展的早期阶段,用于生产商品的预付资本比较小,这种报酬是很高的,而且由于利润率高,这些商品的价值受到相当大的影响。在社会进一步发展的阶段,由于使用的固定资本量大大增加,由于很大一部分流动资本在资本家从卖得之款中得到补偿前的预付期加长,利润对资本和商品的价值也发生很大的影响。在这两种情况下,商品互相交换的比例都会受到不断变动的利润量的重大影响。”(卡泽诺夫出版的《定义》第60页)}\end{quote}

确立相对工资的概念是李嘉图的最大功绩之一。其要点就是:工资的价值(因而还有利润的价值)完全取决于工作日中工人为他自己劳动(为了生产或再生产他的工资)的那一部分和归资本家所有的那一部分劳动时间的比例。这一点在经济学上非常重要,事实上这只是对正确的剩余价值理论的另一种表达\endnote{关于李嘉图的“相对工资”的观念,见本卷第2册第475—483页。——第28页。}。其次,这一点对理解两个[764]阶级的社会关系是很重要的。马尔萨斯在这里嗅到了一些不大对头的味道,因而不得不提出这样的异议:

\begin{quote}{“在李嘉图先生以前我还没有见到有哪个著作家曾在比例的意义上使用工资或者实际工资这个术语。”}\end{quote}

(李嘉图谈的是工资的价值,它实际上也就表现为属于工人的那一部分产品\endnote{关于李嘉图的“实际工资”(“realwages”)的概念,见本卷第2册第456—457、459—460、474、482、497和636页。——第28、241页。}。)

\begin{quote}{“利润确实是指一种比例;而利润率始终被正确地表达为对预付资本的价值的百分比。”}\end{quote}

{要说清楚马尔萨斯所谓的预付资本的价值是指什么,那是很困难的,而要他本人说清楚甚至是不可能的。照马尔萨斯的说法,商品的价值等于商品中包含的预付资本加利润。既然预付资本中除了直接劳动外,还包括商品,所以预付资本的价值等于预付资本中包含的预付资本加利润。于是利润就等于预付资本的利润加利润的利润。如此等等,以至无穷。}

\begin{quote}{“至于工资,我们在考察它的增减时从来不是根据它对通过一定劳动量获得的全部产品的比例,而是根据工人所取得的某种产品量的多少,或者说根据这些产品支配必需品和舒适品的能力大小。”(《定义》1827年伦敦版第29—30页)}\end{quote}

因为在资本主义生产条件下,交换价值——交换价值的增殖——是直接目的,所以弄清怎样衡量交换价值是很重要的。由于预付资本的价值是用货币(实在货币或计算货币)表示的,所以这种增殖的幅度用资本本身的货币量来衡量,并以一定数量——100——的资本(货币额)作为标准。

\begin{quote}{马尔萨斯说:“资本的利润就是预付资本的价值和商品在出卖或被消费时所具有的价值之间的差额。”(《定义》1827年伦敦版第240—241页)}\end{quote}

\tchapternonum{[(8)马尔萨斯的生产劳动和积累的观点同他的人口论相抵触]}

\tsectionnonum{[(a)]生产劳动和非生产劳动}

\begin{quote}{“收入是用来直接维持生活和取得享受的,而资本是用来取得利润的。”(《定义》1827年伦敦版第86页)“工人和家仆是用于完全不同目的的两种工具,前者帮助获得财富,后者帮助消费财富。”\endnote{马尔萨斯的这一段话几乎是逐字重复亚当·斯密的论述。《剩余价值理论》第一册引用过亚当·斯密的这一论述(见本卷第1册第146页):“……制造业工人的劳动,通常把自己的生活费的价值和他的主人的利润,加到他所加工的材料的价值上。相反,家仆的劳动不能使价值有任何增加……一个人,要是雇用许多制造业工人,就会变富;要是维持许多家仆,就会变穷。”马克思用斯密所特有的术语“生产劳动和非生产劳动”作为这一节的标题,暗示马尔萨斯的这一观点是从斯密那里借用来的。——第29页。}(同上,第94页)}\end{quote}

下面这个对生产工人的定义倒是不错的:

\begin{quote}{“生产工人就是直接增加自己主人的财富的工人。”(《政治经济学原理》[第2版]第47页[注])}\end{quote}

此外,还可以引用下面一段话:

\begin{quote}{“唯一真正的生产消费,就是资本家为了再生产而对财富的消费和破坏……资本家使用的工人,自然把他不积蓄的那部分工资,作为用于维持生活和取得享受的收入来消费,而不是作为用于生产的资本来消费。他对于使用他的人、对于国家是生产的消费者,但严格说来,对自己本身就不是生产的消费者。”(卡泽诺夫出版的《定义》第30页)}\end{quote}

\tsectionnonum{[(b)]积累}

\begin{quote}{“现代任何政治经济学家都不能把积蓄看作只是货币贮藏;撇开这种做法的狭隘和无效不说,积蓄这个名词在涉及国民财富方面只能设想有一个用法,这个用法是从积蓄的不同用途中产生并以积蓄所维持的各种不同劳动的实际差别为基础的。”(《政治经济学原理》[第2版]第38—39页)“资本积累就是把收入的一部分当作资本使用。因此,现有的财产或财富不增加,资本也可能增加。”(卡泽诺夫出版的《定义》第11页)“在一个主要依靠工商业的国家里,如果在工人阶级中间盛行慎重地对待结婚的习惯,那对国家是有害的。”(《政治经济学原理》[第2版]第215页)}\end{quote}

这种话竟出自鼓吹制止人口过剩的人之口!

\begin{quote}{“生活必需品的缺乏,是刺激工人阶级生产奢侈品的主要原因;如果这个刺激消除或者大大削弱,以致花费很少劳动就能够获得生活必需品,那末我们就有充分理由认为,用来生产舒适品的时间将不会更多,而只会更少。”(《政治经济学原理》[第2版]第334页)}\end{quote}

但是,对于这个人口过剩论的说教者来说,最重要的是这样一段话:

\begin{quote}{“按人口的性质来说,即使遇到特殊需求,不经过16年或18年的时间,也不可能向市场供应追加工人。然而,收入通过节约转化为资本却可以快得多,一个国家用来维持劳动的基金比人口增长得快的情况,是经常有的。”(同上,第319—320页)}\end{quote}

[765]卡译诺夫正确地指出:

\begin{quote}{“当资本用于预付给工人的工资时,它丝毫不增加用来维持劳动的基金,而只不过是把这种已经存在的基金的一定部分用于生产的目的。”(卡泽诺夫出版的《政治经济学定义》第22页注)}\end{quote}

\tchapternonum{[(9)][马尔萨斯所理解的]不变资本和可变资本}

\begin{quote}{“积累劳动〈其实应当称为物质化劳动、物化劳动〉是花费在生产其他商品时使用的原料和工具上的劳动。”(卡泽诺夫出版的《政治经济学定义》第13页)“在谈到生产商品所花费的劳动时,应当把花费在生产商品所需的资本上的劳动称为积累劳动,以区别于最后的[即在生产商品最后阶段的]资本家所使用的直接劳动。”(同上,第28—29页)}\end{quote}

指出这种区别当然很重要。但是在马尔萨斯那里,这种区别却没有导致任何成果。

马尔萨斯试图把剩余价值或至少是剩余价值率(不过他总是把它们同利润和利润率混为一谈)解释为对可变资本之比,即对用在直接劳动上的那部分资本之比。但是在马尔萨斯那里,这一尝试是十分幼稚的,而且按照他的价值观,也只能是这样。他在他的《政治经济学原理》[第2版]中说:

\begin{quote}{“假定资本只用在工资上。如果100镑用在直接劳动上,年终收回110、120或130镑,显然,在任何一种情况下,利润决定于总产品价值中用来支付所使用的劳动的份额。如果在市场上产品的价值是110,那末用来支付工人的份额是产品价值的10/11,而利润就是10%。如果产品价值的120,那末支付劳动的份额是10/12,而利润是20%;如果产品价值是130,那末必须用来支付预付劳动的份额是10/13,而利润是30%。现在假定资本家预付的资本不单由劳动构成。资本家对于他所预付的资本的一切部分,都期望得到同样的利益。假定预付额的1/4,用于支付(直接)劳动,其余3/4则是积累劳动、利润以及因地租、赋税和其他支出而产生的利润的附加。在这种情况下,说资本家的利润将随着他产品的这1/4的价值与所使用的劳动量之比的变动而变动,这是完全正确的。例如,假定一个租地农场主在农业上花了2000镑,其中1500镑用于种籽、马饲料、固定资本的损耗、固定资本和流动资本的利息、地租、什一税、赋税等等,500镑用于直接劳动,而到年终收回2400镑。这个租地农场主的利润是由2000镑产生的400镑,即20%。同样明显的是,如果我们拿产品价值的1/4即600镑来同支付直接劳动的工资总额相比,结果得出的利润率完全一样。”(第267—268页)}\end{quote}

马尔萨斯在这里表现了邓德里厄里勋爵作风\endnote{邓德里厄里勋爵作风(或邓德里厄里作风)——指矫揉造作的浮华习气。邓德里厄里勋爵是英国作家汤姆·泰勒的喜剧《我们的美国亲戚》(《OurAmericanCousin》)里的人物,该剧于1858年首次上演。——第31页。}。他想(他模糊地感到,剩余价值,从而还有利润,与用在工资上的可变资本具有一定的关系)证明“利润决定于总产品价值中用来支付所使用的劳动的份额”。他最初说对了,因为他假定全部资本由可变资本即用于工资上的资本构成。在这种场合,利润和剩余价值确实是等同的。但是即使在这一场合,马尔萨斯也只是发表了一些十分荒唐的见解。如果所支出的资本是100,利润为10%,那末产品的价值等于110,利润占所支出的资本的1/10(即它的10%),占总产品价值(马尔萨斯已经把利润本身的价值也算在里面)的1/11。这样,利润是总产品价值的1/11,而预付资本是它的10/11。10%的利润同总产品价值的关系可以表述如下:总产品价值中不包括利润的那一部分等于总产品的10/11,或者说,有10%利润在内的、价值为110镑的产品,包含10/11的支出,利润就是由这些支出产生的。这个出色的数学推理使马尔萨斯感到如此有趣,以致他以利润为20%,30%等为例重复了同样的演算。到现在为止,我们看到的不过是同义反复。利润是对所支出的资本的百分比;总产品价值包含了利润的价值,而所支出的[766]资本是总产品的价值减去利润的价值。因而110—10=100。但100是110的10/11。让我们再看下去。

假定资本不仅由可变资本,而且由不变资本构成。“资本家对于他所预付的资本的一切部分,都期望得到同样的利益。”固然,这违背前面刚刚提出的论断,即利润(应该说是剩余价值)决定于用在工资上的资本的份额。但这有什么要紧呢?马尔萨斯这样的人是不会去违背“资本家”的“期望”或想法的。于是,他就大显身手。假定资本是2000镑,其中3/4即1500镑是不变资本,1/4即500镑是可变资本。利润是20%。这样利润就是400镑,产品的价值是2000十400=2400镑。\endnote{在手稿中有下面有三句话:“但是,600∶400=66+(2/3)%。总产品的价值=1000,其中用于工资的部分=6/10。而马尔萨斯先生的计算是怎样的呢?”最后一句话用以引出下文,但是马克思想用前两句话说明什么却不清楚。——第32页。}[马尔萨斯接着说,]拿总产品的1/4来看,这个1/4的价值等于600镑。所支出的资本的1/4等于500镑,即等于总预付资本中用于工资的部分;100镑则构成利润的1/4,等于总利润中分摊在资本家付出的工资总额上的部分。按照马尔萨斯的意见,这就能证明“资本家的利润将随着他产品的这1/4的价值与所使用的劳动量之比的变动而变动”。其实,这不过证明,一定的资本——例如4000镑——的一定比率的利润,例如20%的利润,形成这笔资本的每一个别部分的20%的利润,而这是同义反复。但这绝对证明不了这笔利润同用于工资的那部分资本之间存在某种确定的、独特的、特有的比例。如果我不象马尔萨斯先生那样取总产品的1/4而是取1/24即(2400镑中的)100镑为例,那末这100镑也包含了20%的利润,换句话说,其中的1/6是利润。在这种情况下,资本是83+(1/3)镑,利润是16+(2/3)镑。如果这83+(1/3)镑等于例如生产中使用的一匹马的价值,那末,按马尔萨斯的方法,就会证明利润随马的价值的变动而变动,即随总产品的28+(4/5)之一而变动。

马尔萨斯在他不能剽窃唐森、安德森或其他什么人而只好自己靠自己的时候,就表现得如此可怜。就实质而言(撇开此人的特点不谈),值得注意的倒是他的这种模糊猜测:剩余价值应按照用在工资上的那部分资本计算。

{在利润率已知的条件下,总利润即利润总额总是取决于预付资本量。而积累则是由利

润总额中再转化为资本的那部分来决定的。但是,因为这部分等于总利润减去资本家所消费的收入,所以它不仅取决于利润总额的价值,而且取决于资本家能用这笔利润总额购买的商品的低廉程度——一部分取决于加入他的消费,加入他的收入的商品的低廉程度,一部分取决于加入不变资本的商品的低廉程度。在这里,由于利润率假定为已知,所以工资同样假定为已知。}

\tchapternonum{[(10)马尔萨斯的价值理论[补充评论]}

在马尔萨斯看来,劳动的价值永远不会变动(这是从亚当·斯密那里继承来的),变动的只是我用劳动换得的商品的价值。\authornote{见本册第20—22页。——编者注}假定在一种情况下一个工作日的工资等于2先令,在另一种情况下等于1先令。资本家为同样的劳动时间付出的先令,在前一种情况下比在后一种情况下多一倍。但是工人为取得同样多的产品付出的劳动,在后一种情况下比在前一种情况下多一倍,因为在后一种情况下,他做完整个工作日才得到1先令,而在前一种情况下只要做半个工作日就可以得到1先令。马尔萨斯先生只看到资本家为换取同样多的劳动付出的先令有时多有时少。他没有看到工人为换取一定量产品付出的劳动也完全相应地有时多有时少。

\begin{quote}{“为一定量的劳动付出较多产品,或者用一定量的产品换取较多劳动,在他〈马尔萨斯〉看来都是一样。然而任何人都会认为这恰恰是相反的。”(《评政治经济学上若干用语的争论,特别是有关价值、供求的争论》1821年伦敦版第52页)}\end{quote}

这本书(《评政治经济学上若干用语的争论》1821年伦敦版)还非常正确地指出:劳动作为价值尺度,在马尔萨斯按亚·斯密的一种见解所理解的意义上,能完全和其他任何商品一样充当价值尺度,而它在这个意义上却不能成为象货币在实际上所充当的那样好的价值尺度。一般说来,这里只有在货币是价值尺度这个意义上才能谈价值尺度问题。

[767]一般说来,价值尺度(在货币的意义上)决不是使商品彼此可通约的东西,——参看我的著作的第一部分第45页\endnote{马克思指《政治经济学批判》第一分册。见《马克思恩格斯全集》中文版第13卷第57—58页。——第35页。}。

\begin{quote}{“相反,正是作为物化劳动时间的商品的可通约性使金成为货币。”}\end{quote}

各个商品作为价值是统一体,它们不过是同一统一体即社会劳动的表现。价值尺度(货币)先要有作为价值的商品为前提,而且只涉及这一价值的表现和数量。商品的价值尺度涉及的总是价值转化为价格,它已经把价值作为前提。

上面提到的《评政治经济学上若干用语的争论》中的那一段话是这样说的:

\begin{quote}{“马尔萨斯先生说:‘在同一地点和同一时间,不同的商品所能支配的不同的日劳动量,正好和这些商品的相对交换价值成比例,反过来也是一样。’\endnote{匿名著作《评政治经济学上若干用语的争论》的作者引用的是马尔萨斯《政治经济学原理》第1版(1820年伦敦版)第121页。——第35页。}如果这对劳动来说是正确的,那末对其他任何东西来说同样是正确的。”(《评政治经济学上若干用语的争论》第49页)“货币在同一时间和同一地点可以很好地执行价值尺度的职能……但是这〈指马尔萨斯的论点〉对劳动来说看来是不正确的。劳动甚至在同一时间和同一地点也不是尺度。我们拿一定量的谷物来说,假定它在同一时间和同一地点与一粒钻石在价值上相等;那末谷物和钻石,如果用它们的实物形式支付劳动,能否支配等量劳动呢?有人会说:不能,但是钻石可以购买货币,用货币便能支配等量的劳动……这种决定价值的方法是没有用的,因为这种方法如果不用似乎被它取代的另一种方法来校正,便不能采用。我们只能得出这样的结论:谷物和钻石所以能支配等量的劳动,是由于它们在货币形式上具有相同的价值。但是有人却要我们作出这样的结论:两者之所以具有相同的价值,是由于它们支配等量的劳动。”(同上,第49—50页)}\end{quote}

\tchapternonum{[(11)]生产过剩。“非生产消费者”等等。[马尔萨斯为“非生产消费者”的挥霍辩护,把它看成防止生产过剩的手段]}

从马尔萨斯的价值理论引出了他这个人口过剩(因生活资料不足而产生的人口过剩)论者如此狂热鼓吹的关于非生产消费必须不断增长的整个学说。商品的价值等于预付的材料、机器等等的价值加商品中包含的直接劳动量,而直接劳动量,照马尔萨斯的说法,则等于商品中包含的工资的价值加根据一般利润率的水平加在全部预付上的利润附加额。马尔萨斯认为,这一名义附加额构成利润,并且是商品供给即商品再生产的条件。这些要素构成不同于生产者价格的买者价格,而买者价格也就是商品的实际价值。现在要问,这一价格是怎样实现的呢?谁应该支付这一价格呢?这一价格应该从什么基金中支付呢?

在研究马尔萨斯的观点时我们必须先作如下的区分(这一点他没有做)。一部分资本家生产的商品直接加入工人的消费;另一部分资本家生产的商品或者只是间接地加入工人的消费(就这种商品作为原料和机器等等加入生产生活必需品所需的资本而言),或者根本不加入工人的消费,因为它们只加入非工人的收入。

我们首先来考察那些生产加入工人消费的物品的资本家。他们不只是工人的劳动的买者,而且是把工人生产的产品再卖给工人的卖者。如果工人加进的劳动量值100塔勒,那末资本家就付给他100塔勒。而[在马尔萨斯看来]这就是资本家所购买的劳动加在原料等等上的唯一的价值。因此工人得到了他的劳动的价值,他交给资本家的仅仅是这个价值的等价物。但是,工人虽然名义上取得了这个价值,他实际上得到的商品量却少于他所生产的商品量。实际上他只收回物化在产品中的自己劳动的一部分。为了简便起见,我们也象马尔萨斯本人经常做的那样,假定资本只由用于工资的资本构成。为了生产商品,预付给工人100塔勒(这100塔勒就是所购买的劳动的价值,而且是劳动加在产品上的唯一价值),可是,资本家却把这一商品卖110塔勒,而工人用100塔勒只能买回10/11的产品;1/11的产品,即10塔勒的价值或代表这10塔勒剩余价值的剩余产品量,则归资本家所有。如果资本家把商品卖120塔勒,工人得到的只有10/12,资本家则得到了2/12的产品及其价值。如果资本家把商品卖130塔勒(30%),工人就只得到10/13,资本家则得到10/13的产品。如果资本家加上50%的附加额,即把商品卖150塔勒,工人就只得到2/3的产品,[768]资本家则得到1/3的产品。资本家把商品的价格卖得越高,在产品的价值中,从而也在产品的量中,工人得到的份额就越小,资本家自己所占的份额就越大,工人用他的劳动的价值所能买回的那部分产品价值或产品本身就越少。即使在预付资本中除了可变资本外还有不变资本,例如除了100塔勒的工资外还有100塔勒的原料等等,事情也不会发生任何变化。在这种情况下,如果利润率为10%,资本家就不是把商品卖210塔勒,而是卖220塔勒(其中不变资本100塔勒,[直接]劳动的产品120塔勒)。

{西斯蒙第的《政治经济学新原理》,于1819年第一次出版。\authornote{见本册第51—52页。——编者注}}

在上例中,就生产直接加入工人消费的物品(生活必需品)的A类资本家来看,我们看到这样一种情况:通过名义附加额,即加在预付资本价格上的正常利润附加额,确实为资本家创造了一个剩余基金,因为资本家通过这种转弯抹角的方法只把工人产品的一部分还给工人,而把另一部分据为己有。但是,这种结果之所以产生,并不是由于资本家按照提高了的价值把全部产品卖给工人,而是由于产品价值的提高使工人没有可能用他的工资买回全部产品,而只能买回产品的一部分。因此非常明显,工人的需求在任何时候也不足以实现购买价格超过生产费用[costprice]\endnote{《costprice》这一术语,马克思在这里和有时在后面是用来指资本家的生产费用(c+v)。见注6。——第38页。}的余额,即不足以实现利润和商品的“价值”。相反,利润基金之所以存在,正是由于工人不能用他的工资买回他的全部产品,也就是他的需求和供给不相适应。于是A类资本家手里便有代表一定价值(在上例是20塔勒)的一定量商品,他不必用这些商品来补偿资本,而可以把其中的一部分当作收入花掉,把另一部分用于积累。注意:资本家手里有多少这样的基金,取决于他加在生产费用上的价值附加额,这一附加额决定资本家和工人分配总产品的比例。

现在我们来看B类资本家,他们给A类提供原料和机器等等,简言之,提供不变资本。B类只能把自己的产品卖给A类,因为他们既不能把他们自己的商品卖给与资本(原料、机器等等)毫无关系的工人,也不能卖给生产奢侈品(即不是生活必需品、不加入工人阶级日常消费的一切物品)的资本家,也不能卖给生产制造奢侈品所需的不变资本的资本家。

在上面我们已经看到,在A类预付的资本中有100塔勒用于不变资本。生产这一不变资本的工厂主,在利润率为10%时,用90+(10/11)塔勒的生产费用生产了它,但是卖100塔勒(90+(10/11)∶9+(1/11)=100:10)。于是他靠A类资本家得到自己的利润。也就是说,他从卖220塔勒的A类的产品中得到的是他的100塔勒,而不只是我们假定他用来购买直接劳动的90+(10/11)塔勒。B绝不是从他的工人那里得到利润的,他不能把价值90+(10/11)塔勒的工人的产品按100塔勒再卖给工人,因为工人根本不向他购买东西。然而,B的工人的情况仍然和A的工人一样。他们用90+(10/11)塔勒买到的只是名义上具有90+(10/11)塔勒的价值的商品量,因为A的产品的每一部分的价格都均等地提高了,或者说,它的价值的每一部分都由于相应的利润附加额而代表较少的一部分产品。

{但是,这种加价只能到一定程度,因为工人必须获得足够的商品来维持生活和再生产他的劳动能力。如果资本家A加上了100%,把他花费了200塔勒生产的商品卖400塔勒,工人就只能买回1/4的产品(假定工人得到100塔勒)。如果工人需要得到一半产品才能生活,资本家就必须付给他200塔勒。这样,资本家只留下100塔勒(不变资本100塔勒,工资200塔勒)。因此,结果和资本家按300塔勒出卖花费200塔勒生产的商品并付给工人100塔勒工资的情况是一样的。}

B类资本家不是(直接)靠他的工人,而是靠他把自己的商品卖给A类资本家获得利润基金的。A的产品不仅为实现B类资本家的利润服务,而且构成A自己的利润基金。可是,很清楚,资本家A靠工人获得的利润不可能通过把商品卖给资本家B来实现;资本家B也和资本家A自己的工人一样,不能对A的产品提出足够的需求(以保证产品按照它的价值出卖)。相反,这里出现了反作用。[769]资本家A加上的利润附加额越高,与他的工人相对比,他在总产品中所占有的、资本家B不能得到的那一部分就越大。

资本家B以和资本家A同样的幅度加上附加额。资本家B仍旧付给他的工人90+(10/11)塔勒,虽然工人用这笔钱只能买到较少的商品。但是,如果A取得20%,而不是10%,那末,B同样取得20%,而不是10%,并且按109+(1/11)塔勒,而不是按100塔勒出卖自己的商品。这样,A支出的这部分费用就增加了。

甚至完全可以把A和B作为一类来考察(B属于A的费用,资本家A从总产品中付给资本家B的部分越大,他自己剩下的部分就越小)。在290+(10/11)塔勒的资本中,B占90+(10/11),A占200。他们总共支出290+(10/11)塔勒,并取得29+(1/11)塔勒的利润。B从A买回的无论如何不可能多于100塔勒,而且其中还包括他的9+(1/11)塔勒的利润。正如刚才说过的,两者合在一起共有29+(1/11)塔勒的收入。

现在来谈C类和D类。C是生产为制造奢侈品所必需的不变资本的资本家,D是直接生产奢侈品的资本家。首先,这里很清楚,只有D向C提出直接需求,D类资本家是C类资本家的商品的买者。而资本家C只有把他的商品按照高价,通过把名义附加额加在商品生产费用上的办法卖给资本家D,才能实现利润。资本家D付给资本家C的东西,必须多于C为补偿他的商品的[生产费用的]一切组成部分所必需的东西。资本家D的利润则部分地加在资本家C为他生产的不变资本上,部分地加在D自己直接预付于工资的资本上。C可以用从D那里赚得的利润购买D的一部分商品,虽然他不能把自己的利润全都这样花掉,因为他还需要为自己取得生活必需品,而不仅是为工人(为他们,他是用在D那里实现的资本去交换的)。第一,C的商品的实现直接取决于能否把该商品卖给资本家D;第二,即使已经这样卖出,资本家C的利润所产生的需求也不能实现资本家D出卖的商品的全部价值,正象资本家B的需求不能实现A的商品的全部价值一样。因为资本家C的利润正是从资本家D那里赚来的,即使C把这个利润再用在D的商品上,而不用在其他商品上,他的需求也永远不会大于他从D那里赚得的利润。资本家C的利润永远比D的资本少得多,比D的全部需求少得多,并且C的利润永远不会构成D的利润源泉(顶多是D对C进行一些欺骗,把他卖回给C的商品加价),因为资本家C的利润是直接从资本家D的钱袋中来的。

其次,很清楚,如果在每一类(不管是C类还是D类)内部资本家互卖他们的商品,他们之中谁也不会因此赚得或实现任何利润。资本家m把只值100塔勒的商品按110塔勒卖给资本家n,而n对m也这样做。在交换之后,每个人都和交换前一样仍然拥有生产费用为100塔勒的商品量。每个人都用110塔勒只换得值100塔勒的商品。附加额并不使他拥有的对别人的商品的支配权大于附加额使别人拥有的对他的商品的支配权。至于说到价值,那末,即使m和n都不交换自己的商品而乐于把商品说成是值110塔勒而不是值100塔勒,其结果也会是完全一样的。

其次,很清楚,[按照马尔萨斯的观点]D类(因为C类包括在D类里)的名义剩余价值不代表实际的剩余产品。由于资本家A加上附加额,工人用100塔勒买到的生活必需品减少了,这一情况同资本家D没有直接关系。为了雇用一定数量的工人,资本家D必须照旧支付100塔勒。他付给工人的是工人劳动的价值;除此之外,工人[按照马尔萨斯的观点]没有把任何东西加在产品上,他们交给自己雇主的不过是他们所得工资的等价物。资本家D只有把品卖给第三者才能得到超过这个等价物的余额,因为他向第三者出卖商品的价格高于生产费用。

实际上,[生产奢侈品,例如]生产镜子的工厂主,在他的产品中有和租地农场主的一样的剩余价值和剩余产品。因为产品中包含着无酬劳动和有酬劳完全一样地表现在产品中。无酬劳动表现在剩余产品中。镜子的一部分,工厂主没有花费任何代价,但这一部分也有价值,因为这一部分,工厂主没有花费任何代价,但这一部分也有价值,因为这一部分完全和镜子的补偿预付资本的另一部分一样包含着劳动。剩余产品中的这种剩余价值在镜子出卖以前已经存在,并不是由于出卖才产生的。相反,如果工人在直接劳动中付出的只是他以工资形式取得的积累劳动的等价物,那就既不存在[770]剩余产品,也不存在与之相适应的剩余价值了。但是,在马尔萨斯看来,事情并不是这样,他认为工人还给资本家的只是工资的等价物。

很清楚,D类(包括C类)不能用A类的办法人为地为自己创造剩余基金,也就是说,不能像A类那样把自己的商品按照比向工人购买这种商品时贵的价格再卖给工人,从而在裣了支出的资本以后将总产品的一部分据为己有。因为工人不是D类的商品的买者。D类的剩余基金也不能通过D类资本家相互出卖自己的商品,换言之,通过D类内部进行商品交换而产生。因此,它只有通过D类把自己的产品卖给A类和B类才能实现。由于D类资本家把价值为100塔勒的商品按110塔勒出卖,A类资本家用100塔勒只能买到D类的10/11的产品,而D类资本家自己留下1/11,这一部分产品他们可以自己消费或者用来交换本类的其他商品。

一切不直接生产生活必需品、因而不是把自己的绝大部分或很在一部分商品再卖给工人的资本家,[按照马尔萨斯的观点]情况是这样的:

假定这种资本家的不变资本是100塔勒。其次,如果资本家在工资上付出100塔勒,他就把工人的劳动价值付给了工人。工人在100塔勒的价值上加上100塔勒,因此产品的总价值(生产费用)是200塔勒,而利润是从哪里来的呢?如果平均利润率是10%,资本家就把值200塔勒的商品卖220塔勒。如果他确实把商品卖220塔勒,那末很清楚,要再生产这种商品有200塔勒就够了:100塔勒用于原料等等,100塔勒用于工资;他把20塔勒装进自己的腰包。他可以把这20塔勒当作收入花掉,也可以用来积累资本。

但是,他以高于商品“生产价值”(按照马尔萨斯的说法,“生产价值”和“出售价值”或实际价值不同,所以实际上利润等于生产价值和出售价值之间的差额,或者说等于出售价值减生产价值)10%的价格把商品卖给谁呢?这些资本家靠彼此交换或出卖商品是不能实现丝毫利润的。如果A把价值为200塔勒的商品按220塔勒卖给B,那末B对A也会耍同样的花招。这些商品换一下手,既不改变它们的价值,也不改变它们的数量。以前在A手中的商品量,现在在B手中;反过来也是一样。以前用100表示的东西,现在用110表示,事情一点也没有变化。无论是A的购买力还是B的购买力,都没有任何变动。

但是,根据假定,这些资本家都不能把自己的商品卖给工人。

因此,他们必须把自己的商品卖给生产生活必需品的资本家。后者由于和工人进行交换,事实上拥有实际的剩余基金。名义剩余价值的形成,使他们实际上掌握了剩余产品。这是到目前为止存在的唯一的剩余基金。其他资本家的剩余基金,则只有靠他们把自己的商品以高于其生产价值的价格卖给这些剩余基金的占有者才会产生。

至于生产为制造生活必需品所需的不变资本的资本家,我们已经看到,生活必需品的生产者必须向他们购买。这些购买包括在他的生产费用之内。他的利润愈高,他所花费的附加上同样利润率的预付就愈贵。如果生活必需品的生产者按20%而不是10%的附加额出卖生活必需品,他的不变资本的生产者也会附加20%而不是10%。他要求用来换取90+(10/11)的不是100,而是109+(1/11),或者凑成整数就是110,因此现在产品的[生产]价值是210,这个数额的20%是42,因此整个产品的[出售]价值等于252。其中工人得到100。现在资本家取得的利润就超过总产品的1/11;他以前按220出卖产品时,只得到1/11。产品的量仍旧是那么多,但是归资本家支配的部分,在价值上和数量上都增加了。

至于既不生产生活必需品、也不生产加入这些生活必需品生产的资本的其他资本家,他们只有把自己的商品卖给前两类资本家,才[能]\authornote{手稿这一页缺左下角,因此,原文有几行缺少头几个字。所缺的字由编者根据意思补上并放在四角括号内。——编者注}取得利润。如果前两类资本家赚取20%,他们也赚取[这么多]。

但是,第一类资本家[同工人的交换]和两类资本家之间的交换有很大的区别。第一类资本家[由于]同工人进行[交换],形成了一个生活必需品的实际剩余基金,即[作为]资本的[增殖额]归他们支配的剩余产品,所以他们可以把其中一部分积累起来,一部分[作为收入花费],不管是用来购买他们自己的生活必需品,还是购买奢侈品。这里的剩余价值实际上[代表][XIV—771]剩余劳动和剩余产品,虽然这一结果[在马尔萨斯那里]是通过笨拙的、转弯抹角的办法,通过给商品价格加上附加额的办法得来的。假定生产生活必需品的工人的产品的价值实际上只等于100塔勒。但是,因为支付工资只要产品的10/11就够了,所以资本家只要花费90+(10/11)塔勒,他由此获得利润9+(1/11)塔勒。但是,如果他把劳动的价值和劳动的量设想为等同的,付给工人100塔勒,而把商品按110塔勒卖给他们,那末他仍旧会得到产品的1/11。产品的这个1/11现在不是值9+(1/11)塔勒而是值10塔勒,这对于资本家来说没有什么好处,因为他现在预付的资本已经不是90+(10/11)塔勒,而是100塔勒。

至于说到其他几类资本家,他们那里[按照马尔萨斯的观点]不存在任何现实的剩余产品,不存在任何代表剩余劳动时间的东西。他们把值100塔勒的劳动产品卖110,而只是由于这样给商品价格加上附加额,这笔资本才转化为资本加收入。

但是,从邓德里厄里勋爵\authornote{指马尔萨斯。——编者注}的观点来看,现在这两类资本家之间的情况又是怎样的呢?

生活必需品的生产者把价值100塔勒的新加劳动产品\endnote{手稿中这里用的是“剩余产品”(《Surplusprodukt》)一词,马克思在他的手稿第703页上曾专门谈到这个词的含义:“剩余产品在这里是产品中超过同不变资本相等的那部分产品的余额”(见本卷第2册第559页),也就是指新加劳动产品(v+m)。如果不变资本等于零,那末“新加劳动产品”就相当于产品的价值。——第45页。}卖110塔勒(因为他们支付的工资不是90+(10/11)塔勒,而是100塔勒)。但他们是唯一占有剩余产品的人。如果其他资本家同样把价值100塔勒的产品按110塔勒卖给他们,那末其他资本家实际上除补偿自己的资本外也会得到利润。为什么呢?因为他们有价值100塔勒的生活必需品就已经足以支付自己的工人,从而可以把10塔勒留给自己。或者更确切地说,因为他们实际上获得了价值100塔勒的生活必需品,而其中的10/11就足以支付自己的工人,因为他们这时所处的情况和A类、B类资本家是一样的。然而A类、B类资本家换回的仅仅是代表100塔勒价值的产品量。产品在名义上值110塔勒,并不能使他们多得分文,因为这些产品在数量上,作为使用价值,不能代表一个比100塔勒包含的劳动时间所提供的更大的产品量,[在价值上]他们也不能用这些产品在补偿100塔勒的资本之外再补偿10塔勒的资本。这种情况只有在他们再卖出商品的情况下才有可能。

虽然两类资本家都按110互相出卖值100的东西,但是在这种交换中只有在第二类手里的100才确实具有110的效用。第一类资本家用110的价值实际上只换得100的价值。他们按较高的价格出卖他们的剩余产品,仅仅是由于他们对加入他们收入的[奢侈]物品所支付的,高于这些物品的价值。但是,第二类资本家所实现的剩余价值,实际上仅仅限于他们从第一类所实现的剩余产品中分享的那一部分,因为他们自己并不创造任何剩余产品。

说到奢侈品的这种涨价,马尔萨斯非常及时地想起资本主义生产的直接目的不是挥霍,而是积累。因此,A类资本家会由于这种不合算的交易,——这种交易使他们从工人那里剥夺来的一部分果实重新失掉,——而减少他们对奢侈品的需求。但是,如果他们这样做了,并且更多地进行积累,那末,对他们所生产的生活必需品的有支付能力的需求即生活必需品的市场就会缩小,这个市场靠工人和不变资本生产者的需求是不可能充足地建立起来的。因此,生活必需品的价格会下降,然而A类资本家就是靠生活必需品价格的提高,靠加在价格上的名义附加额并且同这个附加额成比例地从工人身上榨取他们的剩余产品。如果生活必需品的价格从120降到110,他们的剩余产品和他们的剩余价值就会从2/12降到1/11。因而,奢侈品生产者的市场、对奢侈品的需求,还会按更大得多的比例缩小。

第一类资本家在他们的资本已得到补偿后同第二类资本家交换时出卖实际的剩余产品。相反,第二类只是出卖他们的资本,为的是通过这种交易使它从资本转化为资本加收入。这样,整个生产(特别是生产的增长)之所以能持续不断,全靠生活必需品的涨价,而同奢侈品的实际产量成反比的奢侈品价格,又应和这种涨价相适应。第二类资本家把值100的东西卖110,他们在这种交换中也不会得到任何利益,因为实际上他们换回的110也只值100。但是,这100(以生活必需品的形式)能补偿资本加上利润,而那100[以奢侈品的形式]只不过号称110罢了。所以结果就是,第一类资本家在这种交换中得到价值100的奢侈品。他们用110购买价值100的奢侈品。但是对第二类资本家来说,这110就有110的价值,因为他们用100支付劳动(补偿他们的资本),而把10作为余额留下来。

[772]交换双方按同样比率互相贵卖商品,在同样程度上互相欺骗,利润究竟怎么会由此而产生,这是难以理解的。

如果除了一类资本家同他们的工人交换以及各类资本家互相交换以外,还有第三类买者——从机器里出来的神\authornote{原文是与deusexmachina,直译是:“从机器里出来的神”(在古代的戏院里,扮演神的演员由特殊的机械装置送上舞台);转意是:突然出现以挽救危局的人。——编者注}——出现,那末困难就解决了。这第三类买者按照商品的名义价值付款,但他们自己不向任何人出卖商品,自己不以这一套骗人的把戏来回敬,就是说,这一类买者经历的过程是G—W,而不是G—W—G,他们购买商品不是为了补偿他们的资本并且得到利润,而是为了消费商品,他们买而不卖。在这种情况下,资本家实现利润不是靠相互交换商品,而是靠:(1)同工人交换,把总产品的一部分卖回给工人,卖得的货币等于先前用来向工人购买全部总产品(扣除不变资本之后)的货币;(2)把一部分生活必需品和奢侈品卖给第三类买者。因为这一类买者为100支付110,而不再把100按110出卖,所以资本家就会在事实上,而不仅仅在名义上实现10%的利润。利润可以由两种方法取得,即从总产品中尽可能少地卖回给工人,而尽可能多地卖给用现金支付、自己什么也不出卖、只是为消费而买的第三类买者。

但是,不兼卖者的买者,必须是不兼生产者的消费者,即必须是非生产消费者;正是这一类非生产消费者解决了马尔萨斯的矛盾。但是,这种非生产消费者必须同时是有支付能力的消费者,必须形成实际的需求,并且,他们所拥有的、每年支出的价值额,必须不仅足以支付他们购买和消费的商品的生产价值,而且除此以外还足以支付一个名义的利润附加额、剩余价值、出售价值和生产价值之间的差额。这一类买者在社会上代表为消费而消费,正象资本家阶级代表为生产而生产一样;前者代表“挥霍的热情”,后者代表“积累的热情”。(《政治经济学原理》[第2版]第326页)资本家阶级的积累欲望之所以能保持,是由于他们卖得之款经常大于他们的支出,而利润也就成为积累的刺激剂。尽管他们如此热中于积累,但不会弄到生产过剩的地步,或者说很难发生生产过剩,因为非生产消费者不仅是投入市场的产品的巨大排水渠,而且他们自己没有任何产品投入市场。所以,他们的人数不管怎样多,也不会造成对资本家的竞争;相反,他们所有的人都是只求不供的代表者,因此就会抵销资本家方面发生的供过于求。

但是,这一类买者每年的支付手段是从哪里来的呢?这里首先是土地所有者,他们在地租的名义下把很大一部分年产品价值据为己有,并把通过这种方法从资本家那里夺得的货币用于消费资本家生产的商品,他们在购买商品时受到资本家的欺骗。这些土地所有者自己不必从事生产,而且通常也的确不从事生产。根本的一点是:虽然他们花费货币购买劳动,他们雇用的不是生产工人,而只是那些帮助他们消耗财富的食客,家仆,这些人使生活必需品的价格保持在高水平上,因为他们只是购买生活必需品,而本身不会促使生活必需品或任何其他商品的供给有所增加。但是,这种地租所得者还不足以造成“足够的需求”。还必须求助于人为的手段。这就是征收高额的税,供养大批国家和教会的领干薪者,维持庞大的军队,支付大量年金,征收供养牧师的什一税,举借大量的国债,以及不时发动费用浩大的战争。这些就是马尔萨斯心目中的“灵丹妙药”。(《政治经济学原理》[第2版]第408页及以下各页)

总之,被马尔萨斯当作“灵丹妙药”的第三类买者——他们只买不卖,只消费不生产——先是不付代价地取得很大一部分年产品价值,并通过下述办法使生产者致富:生产者首先必须把购买他们商品所需的货币白白付给第三类买者,然后[773]再把这些货币取回,即把自己的商品按高于商品价值的价格卖给他们,或者说,以货币形式从他们那里收回的价值大于以商品形式向他们提供的价值。而这种交易是年年重复的。

\tchapternonum{[(12)马尔萨斯同李嘉图论战的社会实质。马尔萨斯歪曲西斯蒙第关于资产阶级生产的矛盾的观点。马尔萨斯对普遍生产过剩可能性的原理所作的解释的辩护论实质]}

马尔萨斯的结论是完全正确无误地从他的基本的价值理论中得出来的;不过,这个理论也十分明显地符合他的目的——为英国现状辩护,为大地主所有制、“国家和教会”、年金领取者、收税人、教会的什一税、国债、交易所经纪人、教区小吏、牧师和家仆(“国民支出”)辩护,而李嘉图学派恰好把这一切当作对资产阶级生产的无益的、陈腐的障碍,当作累赘来加以反对。李嘉图不顾一切地维护资产阶级生产,因为这种生产意味着尽可能无限制地扩大社会生产力,同时他不考虑生产承担者的命运,不管生产承担者是资本家还是工人。他承认这个发展阶段的历史的合理性和必然性。他完全生活在他那个时代的历史焦点上,就象他完全缺乏对过去的历史感一样。马尔萨斯也愿意资本主义生产尽可能自由地发展,只要这一生产的主要承担者即各劳动阶级的贫困是这一发展的条件;但是,这种生产同时应该适应贵族及其在国家和教会中的分支的“消费需要”,并且应该成为一种物质基础,以满足封建制度和君主专制制度遗留下来的利益的代表人物的过时要求。马尔萨斯愿意有资产阶级生产,只要这一生产不是革命的,只要这一生产不形成历史发展的因素,而只是为“旧”社会造成更广阔、更方便的物质基础。

因此,一方面,存在着工人阶级,由于人口规律的作用,他们同供他们使用的生活资料相比始终是过剩的,即由于生产不足而造成人口过剩;其次,存在着资本家阶级,由于这种人口规律的作用,他们始终能够把工人自己的产品按照这样的价格卖回给工人,使工人从中取回的仅仅能勉强维持他们的生存;最后,社会上还有很大一批奇生虫,一群专事享乐的雄蜂,他们一部分是老爷,一部分是仆役,他们部分地以地租的名义,部分地以政治的名义,无偿地从资本家阶级那里攫取一大批财富,但是,他们要用从资本家手里夺得的货币,按高于价值的价格支付向这些资本家购买的商品;资本家阶级受积累欲望的驱使从事生产,非生产者在经济上则只代表消费欲望,代表挥霍。而且这被描绘为避免生产过剩的唯一办法,而这种生产过剩又是和与生产相比的人口过剩同时存在的。处于生产之外的那些阶级的消费过度,被说成是医治这两种过剩的灵丹妙药。工人人口同生产之间的失调现象,通过根本不参加生产的游手好闲者吃掉一部分产品的办法得到消除。资本家引起的生产过剩的失调现象,则通过财富享受者的消费过度得到消除。

我们已经看到,当马尔萨斯企图根据亚·斯密观点的弱点建立一种对立的理论来反对李嘉图根据亚·斯密观点的优点建立的理论时,他显得多么幼稚、庸俗和浅薄。未必还有什么东西比马尔萨斯关于价值的著作所表现出的那种虚弱的挣扎更滑稽可笑的了。但是,一当他作出实际结论,从而重新进入他作为经济学方面的阿伯拉罕·圣克拉\endnote{阿伯拉罕·圣克拉是奥地利传教士和著作家乌尔利希·梅格尔勒(1644—1709年)的笔名,他力图用公众易懂的形式宣传天主教,并用所谓民间文体来进行“救人”的说教和写劝善的作品。——第51页。}从事活动的领域时,他又自由自在起来。不过,即使在这里,他也没有改变他那天生的剽窃者的本性。乍一看来,谁能相信马尔萨斯的《政治经济学原理》竟不过是西斯蒙第的《政治经济学新原理》一书的马尔萨斯化的译本呢?然而,事实就是如此。西斯蒙第的书于1819年出版。一年以后,马尔萨斯的拙劣的英文仿制品问世了。象过去剽窃唐森和安德森一样,他现在又在西斯蒙第那里为自己的一本厚厚的经济学论著找到了理论支柱,不过与此同时,他还利用了从李嘉图的《原理》一书中学来的新理论。

[774]如果说马尔萨斯攻击李嘉图的是李嘉图著作中对旧社会说来是革命的资本主义生产倾向,那末他凭着永无谬误的牧师本能从西斯蒙第的著作中抄来的,却只是对资本主义生产,对现代资产阶级社会说来是反动的东西。

在这里,我不把西斯蒙第列入我的历史述评之内,因为对于他的观点的批判,属于我写完这部著作以后才能着手的那一部分——资本的现实运动(竞争和信用)。

马尔萨斯利用西斯蒙第的观点来适应自己的目的,这从《政治经济学原理》的一章的标题就可以看出。这一章的标题是:

\begin{quote}{《生产力必须和分配手段相结合以保证财富的不断增长。》([第2版]第361页)}\end{quote}

[在这一章中写道:]

\begin{quote}{“只有生产力,还不能保证创造相应程度的财富。为了把生产力充分调动起来,还需要有一些别的东西。这就是对全部生产物的有效的和不受阻碍的需求。看来,最有助于达到这一目的的,是这样地分配产品并使这些产品这样地适应那些消费产品的人的需要,以致全部产品的交换价值能不断增加。”(《政治经济学原理》第2版第361页)}\end{quote}

其次,下面一段话同样是西斯蒙第式的和反对李嘉图的:

\begin{quote}{“一国的财富,部分地取决于靠本国的劳动所获得的产品的数量,部分地取决于这个数量与现有人口的需要和购买力的适应,这种适应要使它能具有价值。财富并不单单由这些因素中的一种因素决定,这是十分肯定无疑的。”(同上,第301页)“但是,财富和价值的最密切的联系,也许在于后者是前者的生产所必需的。”(同上)}\end{quote}

这段话是专门针对李嘉图,针对他的著作第二十章《价值和财富,它们的特性》的。李嘉图在那里说:

\begin{quote}{“因此,价值和财富在本质上是不同的,因为价值不取决于充裕程度,而取决于生产的困难或容易程度。”(李嘉图《原理》第3版第320页)}\end{quote}

{其实,价值也可能随着“生产的容易程度”的提高而增加。假定某一国家的人口从100万增加到600万。100万人过去每天工作12小时。600万人则把生产力发挥到每人工作6小时就能生产出以前用12小时生产的东西。那末,按照李嘉图本人的观点,财富就增加到六倍,价值增加到三倍。}

\begin{quote}{“富并不取决于价值。一个人的贫富取决于他所能支配的生活必需品和奢侈品的充裕程度……只是由于把价值的概念和财富即富的概念混淆起来,才会断言,减少商品的数量,即减少生活必需品、舒适品和享乐品的数量,可以增加财富。如果说价值是财富的尺度,那末这种说法是不能否定的,因为商品的价值会由于商品的稀少而增加;但是……如果财富是由生活必需品和奢侈品构成,它就不可能由于它们的数量的减少而增加。”(同上,第323—324页)}\end{quote}

换言之,李嘉图在这里是说,财富只是由使用价值构成。他把资产阶级生产变成单纯为使用价值而进行的生产,这对于交换价值占统治地位的生产方式是一种非常美妙的见解。他把资产阶级财富的特有形式只看成一种不触及这种财富内容的表面的东西。因此他也就否认在危机中爆发出来的资产阶级生产的矛盾。因此就产生了他对货币的完全错误的见解。因此他在考察资本的生产过程时也就完全不注意流通过程,——而流通过程却包括商品的形态变化,包括资本转化为货币的必然性。无论如何,没有一个人比李嘉图本人更好地、更明确地阐明了:资产阶级生产并不是为生产者(他不止一次地这样称呼工人)\endnote{李嘉图在其他一些地方使用的“生产者”(《producer》)一词是指“产业资本家”,马克思在《李嘉图的其他方面。约翰·巴顿》一章中指出了这一点(见本卷第2册第627页,并参看第480页和第487页马克思引自李嘉图《原理》一书的引文)。马克思在本卷第二册第478页和第528页上指出李嘉图把“生产者”和“工人”两个概念等同起来。在本卷第二册第527页和第622页马克思引用的引文中,李嘉图也是在上述意义上使用“生产者”一词的。——第54页。}生产财富,因此资产阶级财富的生产完全不是为“充裕”而生产,不是为生产生活必需品和奢侈品的人生产生活必需品和奢侈品,——如果生产只是满足生产者需要的一种手段,是一种仅仅由使用价值占统治地位的生产,那末情况本来应当是这样的。可是,同一个李嘉图说:

\begin{quote}{“如果我们生活在欧文先生的一个平行四边形\endnote{欧文在阐述他的空想的社会改革计划时证明:按平行四边形或正方形建立劳动公社新村,无论从经济上看,还是从组织家庭生活的观点看,都是最合适的。欧文在1817—1821年的一系列演说中都谈到了这些思想。——第54页。}里,共同享用我们的全部产品,那末谁也不会由于产品充裕而受害;但是,只要社会构成仍然象目前这样,充裕就往往对生产者有害,而匮乏倒对他们有利。”(《论农业的保护关税》1822年伦敦第4版第21页)}\end{quote}

[775]李嘉图把资产阶级的生产,确切些说,把资本主义的生产看作生产的绝对形式。这就是说,他认为,资本主义生产的生产关系的一定形式,在任何地方都不会同生产本身的目的即充裕发生矛盾或束缚这一目的,充裕既包括使用价值的量,也包括使用价值的多样性,这又决定作为生产者的人的高度发展,决定他的生产能力的全面发展。在这里李嘉图陷入了可笑的自相矛盾之中。当我们谈到价值和财富时,根据李嘉图的解释,我们只是指整个社会。而当谈到资本和劳动时,李嘉图认为“总收入”仅仅为了创造“纯收入”而存在,是不言而喻的事。实际上,他对资产阶级生产赞赏的,正是这种生产的一定形式同以前的各种生产形式相比能给生产力以自由发展的天地。当这种形式不再起这种作用的时候,或者当这种形式在其中起这种作用的那些矛盾显露出来的时候,李嘉图就否认矛盾,或者确切些说,他自己就以另一种形式表现矛盾,把财富本身,把使用价值总量本身说成是ultimaThule\authornote{最终目的,最终之物,极点,极限(直译是:极北的休里——古代人想象中的欧洲极北部的一个岛国)。——编者注},而不考虑生产者了。

西斯蒙第深刻地感觉到,资本主义生产是自相矛盾的;一方面,它的形式——它的生产关系——促使生产力和财富不受拘束地发展;另一方面,这种关系又受到一定条件的限制,生产力愈发展,这种关系所固有的使用价值和交换价值、商品和货币、买和卖、生产和消费、资本和雇佣劳动等等之间的矛盾就愈扩大。他特别感觉到了这样一个基本矛盾:一方面是生产力的无限制的发展和财富的增加——同时财富由商品构成并且必须转化为货币;另一方面,作为前一方面的基础,生产者群众却局限在生活必需品的范围内。因此,在西斯蒙第看来,危机并不象李嘉图所认为的那样是偶然的,而是内在矛盾的广泛的定期的根本爆发。他经常迟疑不决的是:国家应该控制生产力,使之适应生产关系呢,还是应该控制生产关系,使之适应生产力?在这方面,他常常求救于过去;他成为“过去时代的赞颂者”\authornote{见贺雷西《诗论》。——编者注},或者也企图通过别的调节收入和资本、分配和生产之间的关系的办法来制服矛盾,而不理解分配关系只不过是从另一个角度来看的生产关系。他中肯地批判了资产阶级生产的矛盾,但他不理解这些矛盾,因此也不理解解决这些矛盾的过程。不过,从他的论据的基础来看,他确实有这样一种模糊的猜测:对于在资本主义社会内部发展起来的生产力,对于创造财富的物质和社会条件,必须有占有这种财富的新形式与之适应;资产阶级形式只是暂时的、充满矛盾的形式,在这种形式中财富始终只是获得矛盾的存在,同时处处表现为它自己的对立面。这是始终以贫困为前提、并且只有靠发展贫困才能使自己得以发展的财富。

我们已经看到,马尔萨斯用多么巧妙的办法剽窃了西斯蒙第的观点。而马尔萨斯的理论又以夸张的、更丑恶得多的形式包含在托马斯·查默斯(神学教授)的《论政治经济学和社会的道德状况、道德远景的关系》(1832年伦敦第2版)中。在这里,不仅在理论上更明显地表现出牧师的成分,而且在实质上也更明显地表现出一个“从经济学方面”维护“法定教会”\endnote{“法定教会”(《EstablishedChurch》)是指英国国教会。——第56、344页。}的“尘世福祉”和“法定教会”与之共存亡的整套制度的“法定教会”教徒。

上面提到的马尔萨斯有关工人的论点如下:

\begin{quote}{“从事生产劳动的工人的消费和需求,决不能单独成为资本的积累和使用的动机。”(《政治经济学原理》[第2版]第315页)“如果租地农场主的全部产品在市场上出卖时卖价的增加部分正好等于他付给所雇10个追加工人的报酬,那末,就没有一个租地农场主会自找麻烦去监督这10个追加工人的劳动。在有关商品的过去的供求状况方面或在它的价格方面,必须——在新工人造成追加需求之前,因而与这种需求无关——出现某种东西证明雇用追加的工人来生产这种商品是合算的。”(同上,第312页)“由生产工人本身造成的需求,决不会是一种足够的需求,[776]因为这种需求不会达到同工人所生产的东西一样多的程度。如果达到这种程度,那就不会有什么利润,从而也就不会有使用工人的劳动的动机。任何商品的利润的存在本身,必须先有一种超过生产这种商品的工人的需求范围的需求。”(同上,第405页注)“工人阶级消费的剧增必然大大增加生产费用,因此,这一定会降低利润,削弱或破坏积累的动机。”(同上,第405页)“生活必需品的缺乏,是刺激工人阶级生产奢侈品的主要原因;如果这个刺激消除或者大大削弱,以致花费很少劳动就能够获得生活必需品,那末我们就有充分理由认为,用来生产舒适品的时间将不会更多,而只会更少。”(同上,第334页)}\end{quote}

马尔萨斯并不打算掩盖资产阶级生产的矛盾,相反,他是想要突出这些矛盾,以便一方面证明工人阶级的贫困是必要的(对这种生产方式说来,他们的贫困确实是必要的),另一方面向资本家证明,为了给他们出卖的商品创造足够的需求,养得脑满肠肥的僧侣和官吏是必不可少的。因此,马尔萨斯证明,要导致“财富的不断增长”,无论人口的增加,或资本的积累(同上,第319—320页),或“土地的肥力”(第331页及以下各页)、“节省劳动的发明”、“国外市场”的扩大(第352、359页),都是不够的。

\begin{quote}{“工人和资本同用它们获利的手段比较起来,都可能过剩。”(同上,第414页)}\end{quote}

因此,和李嘉图学派相反,马尔萨斯强调了普遍生产过剩的可能性(同上,第326页及其他各处)。

他在这方面提出的主要论点如下:

\begin{quote}{“需求总是由价值决定,而供给总是由数量决定。”(《政治经济学原理》[第2版]第316页)“商品不仅同商品相交换,而且也同生产劳动和个人服务相交换,而同这些东西相比,就象同货币相比一样,可能发生市场商品普遍充斥。”(同上)“供给必须始终同数量成比例,而需求必须始终同价值成比例。”(卡泽诺夫出版的《政治经济学定义》第65页)“詹姆斯·穆勒说:‘显然,一个人生产出来而不打算用于他自己消费的一切东西,就构成他可以用来交换其他商品的储备。因此,他的购买愿望和购买手段,换句话说,他的需求,正好等于他生产出来但不准备自己消费的东西的数量。’”\authornote{见本册第106页。——编者注}……[马尔萨斯反驳詹姆斯·穆勒说,]“很明显,他购买其他商品的手段,并不同他生产出来并打算销售的商品的数量成比例,而是同这些商品的交换价值成比例;除非某一商品的交换价值同该商品的数量成比例,否则说每一个个人的需求和供给永远相等,就不可能是正确的。”(同上,第64—65页)“如果每一个个人的需求都同他的确切意义上的供给相等,那末,这就证明,他永远能够按照生产费用(加上公平的利润)出卖他的商品;那时,甚至连市场商品的局部充斥也不可能有了。这种论点证明的东西太多了……供给必须始终同数量成比例,而需求必须始终同价值成比例。”(《政治经济学定义》1827年伦敦版第48页注)“在这里,穆勒把需求了解为他〈需求者〉的购买手段。但是,这种购买其他商品的手段,并不同他生产出来并打算销售的商品的数量成比例,而是同这些商品的交换价值成比例;除非某一商品的交换价值同该商品的数量成比例,否则说每一个个人的需求和供给永远相等,就不可能是正确的。”(同上,第48—49页)“托伦斯错误地认为,‘供给增长是有效需求增长的唯一原因’。\authornote{见本册第79页。——编者注}如果真是那样的话,人类在遇到食物和衣服暂时减少的情况时,将会多么难于恢复啊。但是,当食物和衣服的数量减少时,它们的价值会提高;剩下的食物和衣服的货币价格的增长程度,在一段时间内会超过它们的数量减少的程度,而劳动的货币价格可能保持不变。其必然结果是出现了推动比过去大的生产劳动量的力量。”(第59—60页)“一个国家的所有商品,同货币或劳动相比较,可能同时跌价。”(第64页及以下各页)“因此,市场商品普遍充斥是可能的。”(同上)“商品的价格可能全都跌到生产费用之下。”(同上)}\end{quote}

\centerbox{※     ※     ※}

[777]此外,还值得提到的只是马尔萨斯关于流通过程的观点:

\begin{quote}{“如果我们把所使用的固定资本的价值算作预付资本的一部分,我们就必须在年终时把这种资本的残余价值算作年收入的一部分……实际上,他〈资本家〉每年预付的资本只包括他的流动资本,他的固定资本的磨损,以及固定资本的利息和由用于按期支付各项年开支的货币构成的那一部分流动资本的利息。”(《政治经济学原理》[第2版]第269页)}\end{quote}

我认为,折旧基金,即补偿固定资本磨损的基金,同时也就是积累基金。

\tchapternonum{[(13)李嘉图学派对马尔萨斯关于“非生产消费者”的观点的批判]}

我还想从一本反对马尔萨斯理论的李嘉图学派的著作中摘引几段话。关于这本著作中从资本主义观点对马尔萨斯的全部非生产消费者特别是地主进行的抨击,我将在另外一个地方说明:这种抨击从工人的观点来看,也可以逐字逐句地用在资本家身上(这个说明要放在《对资本和雇佣劳动关系的辩护论的解释》那一篇\endnote{《对资本和雇佣劳动关系的辩护论的解释》这一篇,马克思没有写成。——第59页。})。

[这位匿名的李嘉图主义者写道:]

\begin{quote}{“马尔萨斯先生以及象他那样推论的人认为,除非能保证利润率等于或者大于以前的利润率,否则资本的使用就不可能增大,并且认为,单单是资本的增加本身并不能保证这样的利润率,而是适得其反,因此他们想找到一个不取决于生产本身并且处在生产之外的源泉,它能同资本一起不断增长,从它可以经常取得必要数量的超额利润。”(《论马尔萨斯先生近来提倡的关于需求的性质和消费的必要性的原理》1821年伦敦版第33—34页)在马尔萨斯看来,这种源泉就是“非生产消费者”。(同上,第35页)“马尔萨斯先生有时说什么存在着两类不同的基金:资本和收入,供给和需求,生产和消费,它们必须保持步调一致,不要互相超越。好象在生产出来的商品的总量之外还要有另外一个想必是从天上掉下来的总量,以便去购买这些生产出来的商品……马尔萨斯所要求的这种消费基金,只有牺牲生产才能取得。”(第49—50页)“他的〈马尔萨斯的〉论断使我们始终弄不清,究竟是应当增加生产还是限制生产。假如有人感到需求不足,那末马尔萨斯先生是否会劝他把钱付给别人,让别人用这笔钱购买他的商品呢?大概不会的。”(第55页)当然会的!“你出卖自己的商品,目的就是要得到一定数额的货币;如果你把这个数额的货币白送给另一个人,让他买你的商品,从而把这笔钱还给你,那就没有任何意义了。你还不如马上把你的商品烧掉,这样,你的情况也会是一样的。”(第63页)}\end{quote}

对马尔萨斯来说,匿名作者是正确的。但是,决不能从“生产出来的商品的总量”是同一基金——既是生产基金又是消费基金,既是供给基金又是需求基金,既是资本基金又是收入基金——这个论断中得出结论说,这个总基金在这些不同范畴之间怎样分配是无关紧要的。

这位匿名作者不理解,马尔萨斯所谓对于资本家来说工人的“需求”“不够”是指什么。

\begin{quote}{“至于来自劳动的需求,指的就是以劳动同商品交换,或者说……以将来在材料价值上追加的价值同现有的、现成的产品交换……这是实际的需求,它的增加对于生产者说来是十分重要的”……(同上,第57页)}\end{quote}

马尔萨斯指的不是劳动的供给(我们这位作者称之为“来自劳动的需求”),而是工人由于得到工资而能对商品提出的需求,就是工人作为买者在商品市场上出现时所拥有的货币。关于这种需求,马尔萨斯还正确地指出,对资本家的商品供给来说,它任何时候也不可能是足够的。否则,工人就能用自己的工资买回自己的全部产品。

[778]这位作者还说:

\begin{quote}{“他们〈工人〉[对工作的]需求的增加不过是表明他们甘愿自己拿走产品中更小的份额,而把其中更大的份额留给他们的雇主;要是有人说,这会由于消费减少而加剧市场商品充斥,那我只能回答说:市场商品充斥是高额利润的同义语。”(同上,第59页)}\end{quote}

按照作者的意思,这种说法象是开玩笑,但是实际上它包含着“市场商品充斥”的根本秘密。

关于马尔萨斯的《地租论》\endnote{这个匿名的李嘉图主义者所说的《地租论》,是指马尔萨斯的小册子《关于地租的本质和增长及其调整原则的研究》1815年伦敦版。——第61页。},这位作者说:

\begin{quote}{“马尔萨斯先生发表他的《地租论》,看来部分地是为了反对当时‘用红字写在墙上’的‘打倒地主!’的口号,为了起来保护这个阶级,并且证明他们与垄断者不同。他指出,地租不能废除,地租的增长通常是一种伴随财富和人口增长的自然现象;但是,‘打倒地主!’这个人民的呼声并不一定意味着不应该有象地租这样的东西,宁可说是意味着地租应该按照所谓‘斯宾斯计划’\endnote{“斯宾斯计划”是指英国空想社会主义者托马斯·斯宾斯从1775年开始鼓吹的土地国有化计划,他要求废除土地私有制,将地租(在扣除各种税款和公社的公用开支后)均等地分配给公社全体居民。——第61页。}在居民中间平均分配。但是当马尔萨斯先生着手为地主祛除垄断者这个可憎的名称和亚当·斯密关于‘他们喜欢在他们未曾播种的地方得到收获’的评语时,他似乎是在为一个名称而奋斗……在他的所有这些议论中,辩护士的气味太重了。”(同上,第108—109页)}\end{quote}

\tchapternonum{[(14)马尔萨斯著作的反动作用和剽窃性质。马尔萨斯为“上等”阶级和“下等”阶级的存在辩护]}

马尔萨斯的《人口原理》是一本攻击法国革命和与它同时的英国改革思想(葛德文等)的小册子。它对工人阶级的贫困进行辩解。理论是从唐森等人那里剽窃来的。

他的《地租论》是一本维护地主而反对产业资本的小册子。理论是从安德森那里剽窃来的。

他的《政治经济学原理》是一本维护资本家利益而反对工人,维护贵族、教会、食税者、谄媚者等等的利益而反对资本家的小册子。理论是从亚·斯密那里抄袭来的。至于他自己有所发明的地方,真是可怜之至。在进一步阐述理论时,西斯蒙第又成了依据。[XIV—778]

\centerbox{※     ※     ※}

[VIII—345]{马尔萨斯在他的《人口原理》(比·普雷沃从英文第5版译的法译本,1836年日内瓦第3版第4卷第104—105页)中以他惯用的“高深的哲理”发表了如下的见解,反对向英国茅舍贫农赠送乳牛的计划:

\begin{quote}{“有人指出有乳牛的茅舍贫农比没有乳牛的茅舍贫农更勤劳,生活更正规……现在大多数有乳牛的人,是用他们自己劳动所得购买乳牛的。所以,更正确地说,是劳动使他们得到乳牛,而不是乳牛使他们产生了对劳动的兴趣。”}\end{quote}

那末,还可以正确地说,勤劳(加上对别人劳动的剥削)使资产阶级暴发户得到乳牛,但是乳牛却使暴发户的子孙养成懒惰的习惯。如果去掉他们的乳牛支配别人无酬劳动的能力(不是产乳的能力),那末这对他们养成劳动的兴趣倒是十分有益的。

这位“高深的哲学家”说:

\begin{quote}{“很清楚,不能所有的人都属于中等阶级。有上等阶级和下等阶级是绝对必要的〈自然,没有两头就没有中间〉,而且有这两个阶级存在是非常有益的。如果在社会上人们不能指望上升,也不害怕下降,如果劳动没有奖赏,懒惰不受惩罚,人们就无法看到为改善自己的处境而表现出的那种勤奋和热情,而这是[346]社会幸福的极重要的动力。”(同上,第112页)}\end{quote}

必须有下等人,上等人才会害怕下降,必须有上等人,下等人才能指望上升。为了使懒惰受到惩罚,工人必须贫困,食利者和马尔萨斯十分心爱的土地所有者必须富有。可是马尔萨斯所谓的劳动的奖赏是什么呢?正如我们以后将要看到的\authornote{见本册第26—27、31、36—40等页。——编者注},马尔萨斯指的就是工人必须献出自己的一部分劳动而得不到任何等价物。如果成为刺激的是“奖赏”,而不是饥饿,那真是美妙的刺激了。上述一切最多不过是归结为:有的工人可以指望有朝一日也能剥削工人。

\begin{quote}{卢梭说:“垄断越扩大,被剥削者身上的锁链就越沉重。”\endnote{在卢梭的著作中没有找到这句话。——第63页。}}\end{quote}

“高深的思想家”马尔萨斯却不这样认为。他的最高希望是,中等阶级的人数将增加,无产阶级(有工作的无产阶级)在总人口中占的比例将相对地越来越小(虽然它的人数会绝对地增加)。马尔萨斯自己认为这种希望多少有点空想。然而实际上资产阶级社会的发展进程却正是这样。

\begin{quote}{马尔萨斯说:“我们应当抱这样的希望,近年来已有很大发展的节省劳动的方法,终将有一天能以比现在少的人类劳动满足最富裕的社会的一切需要;如果说,即使到那时工人还摆脱不了目前压在他们身上的重担〈他们仍将辛勤地做目前一样多的工作,并且,相对地说,为他人的越来越多,为自己的越来越少〉,那末,承担沉重的劳动的人数毕竟会减少。”(同上,第113页)}[VIII—346]}\end{quote}

\tchapternonum{[(15)匿名著作《政治经济学大纲》对马尔萨斯的经济理论原理的阐述]}

[XIV—778]1832年在伦敦匿名出版的《政治经济学大纲。略论财富的生产、分配和消费的规律》是一部阐述马尔萨斯原理的著作。

本书作者\authornote{即约翰·卡泽诺夫。——编者注}一开头就指出了马尔萨斯主义者反对价值由劳动时间决定的实际动机。

\begin{quote}{“关于劳动是财富的唯一源泉的学说,看来既是错误的,又是危险的,因为它不幸给一些人提供了把柄,他们可以断言一切财产都属于工人阶级,别人所得的部分仿佛都是从工人阶级那里抢来和骗来的。”(上述著作,第22页注)}\end{quote}

匿名作者下面说的话,比马尔萨斯著作更明显地表现出把商品的价值和商品或货币作为资本的价值增殖混为一谈。在后一意义上,它正确地表达了剩余价值的起源:

\begin{quote}{“资本的价值,即资本所值的劳动量或者说所能支配的劳动量,总是大于耗费在资本上的劳动量,这个差额就构成利润,或者说资本所有者的报酬。”(同上,第32页)}\end{quote}

从马尔萨斯那里吸收来的下述关于为什么在资本主义生产条件下利润应列入生产费用的论点,也是正确的:

\begin{quote}{“所使用的资本的利润{“如果得不到这种利润,那就没有生产商品的足够动机”}是供给的重要条件,而且它作为这样的条件成为生产费用的一个组成部分。”(同上,第33页)}\end{quote}

下面这段话,一方面包含着资本的利润直接产生于资本同劳动的交换这一正确思想,另一方面也阐述了马尔萨斯关于出卖中创造利润的学说:

\begin{quote}{“一个人的利润,不是取决于他对别人的劳动产品的支配,而是取决于他对这种劳动本身的支配。〈这里正确地区分了商品同商品的交换和作为资本的商品同劳动的交换。〉在工人的工资不变的情况下,如果他[779]〈在货币价值降低的情况下〉能以较高的价格出售他的商品,显然他就会从中获得利益,而不管其他商品是否涨价。他只要用他的产品的较小部分,就足以推动这种劳动,因而更大部分的产品就留给他自己了。”(同上,第49—50页)}\end{quote}

这种情况也会发生在下述场合:例如,一个资本家采用了新的机器、新的化学过程等等,生产出的商品低于原来的价值,而他却按照原来的价值出卖,或者至少高于现在降低了的个别价值出卖。在这种场合,当然工人不是直接为自己劳动了更短的时间,为资本家劳动了更长的时间。但是,在再生产过程中,“他只要用他的产品的较小部分,就足以推动这种劳动”。可见,工人实际上是用他的比过去更大的一部分直接劳动来换取他所得到的物化劳动。例如,他仍和以前一样得到10镑。但是,这10镑——尽管对社会来说代表同样多的劳动量——不再是以前同样多的劳动时间(也许少了一小时)的产品。因此,工人实际上为资本家劳动了更长的时间,为自己劳动了更短的时间。这就等于他现在总共只得到8镑,但是这8镑由于他的劳动生产率的提高所代表的使用价值量和原来10镑一样多。

对于上面提到的[詹姆斯·]穆勒关于需求和供给等同的论点\authornote{见本卷第2册第562—563、575—576页以及本册第57—58、106—109页。——编者注},匿名作者指出:

\begin{quote}{“每个人的供给,取决于他提供到市场上的数量;他对其他物品的需求,取决于他的供给的价值。供给是确定的,它取决于他自己;需求是不定的,它取决于别人。供给可能保持不变,需求则发生变化。某人向市场提供100夸特谷物,每夸特在一个时期可能值30先令,在另一时期可能值60先令。供给的数量在两种情况下是一样的,但是,这个人的需求,或者说,他购买其他物品的能力,在后一场合比前一场合大一倍。”(同上,第111—112页)}\end{quote}

关于劳动和机器的关系,匿名作者指出:

\begin{quote}{“当商品由于更合理的分工而增多时,无需比以前更大的需求,就可以维持先前使用的全部劳动}\end{quote}

(怎么会这样?如果分工更合理,那末,用同量的劳动就会生产更多的商品。因此,供给会增加,为了吸收供给,难道不要扩大需求吗?难道亚·斯密说的分工取决于市场规模不对吗?其实,谈到[必须增加]外来需求,那末,[实行更合理的分工和采用机器这两种情况]在这方面是没有差别的,只是在采用机器的情况下[需求的增加必须]有更大规模。不过,“更合理的分工”可能需要同以前一样多的甚至更多的工人,而采用机器,在任何情况下都会减少用在直接劳动上的那部分资本),

\begin{quote}{而当采用机器时,如果需求不增加,或者工资或利润不降低,那末部分工人无疑会失业。我们假定有价值1200镑的商品,其中1000镑是100个工人的工资(每人10镑),200镑(按利润率20%计算)为利润。现在假定,用50个使用机器的工人的劳动可以生产出同样的商品,机器的价值等于其余50个工人的劳动,并需要10个工人来维修;这时生产者可以把他所生产的商品的价格降低到800镑,而他的资本所得的报酬仍然不变。50个工人的工资…………………………500镑维修机器的10个工人的工资……………100镑500镑流动资本的20%利润200镑500镑固定资本的20%利润共计800镑”}\end{quote}

{(“维修机器的10个工人的工资”,在这里代表机器的年磨损。否则,这种算法就是错误的,因为维修机器的劳动应加在机器的最初生产费用上。)从前,企业主每年支出1000镑,但产品当时的价值是1200镑。现在,他一次就把500镑投在机器上,因而他用不着再以任何其他方式支出这笔钱了。他每年支出的,就是用于“维修机器”的100镑和用于工资的500镑(因为本例中没有原料一项)。他每年只须支出600镑,但是他的总资本照旧得到200镑利润。利润额和利润率都没有变。但是,他的年产品总共只有800镑。}

\begin{quote}{“从前为商品支付1200镑的人,现在可以节省400镑,这笔钱他可以花在别的方面,或者购买较多的同一种商品。如果这笔钱花在[780]直接劳动的产品上,只能给33.4个工人提供就业机会,但由于采用机器而失业的工人是40名。因为33.4个工人的工资(按每人10镑计算)…………………………334镑20%的利润………………………………………………………………66镑共计400镑”}\end{quote}

{换句话说,这就是:如果400镑花在作为直接劳动产品的商品上,而且每个工人的工资是10镑,那末价值400镑的商品应当是不到40个工人的劳动的产品。如果这些商品是40个工人的产品,那末,它们就只包含有酬劳动了。劳动的价值(或者说,物化在工资中的劳动量)就会等于产品的价值(物化在商品中的劳动量)。但是,400镑商品包含着正是构成利润的无酬劳动。所以,这些商品应当是不到40个工人的产品。如果利润是20%,那末只有5/6的产品可以由有酬劳动构成,约334镑,按每人10镑计算,相当于33.4个工人。而1/6的产品,约66镑,是无酬劳动。李嘉图完全以同样的方式证明:即使机器的货币价格同它所代替的直接劳动的价格一样高,机器在任何时候都不可能是同样多的劳动的产品。\authornote{参看本卷第2册第628—629页。——编者注}}

\begin{quote}{“如果这笔钱〈即上述的400镑〉用来购买更多的同一种商品,或者购买另一种用同样种类和同样数量的固定资本制造出来的商品,那末这笔钱只能给30个工人提供就业机会。因为25个工人的工资(每人按10镑计算)………………250镑维修机器的5个工人的工资……………………………50镑250镑流动资本和250镑固定资本的利润……………100镑共计400镑”}\end{quote}

{问题是这样的:在使用机器的情况下,生产价值800镑的商品要在机器上花费500镑;所以,生产价值400镑的商品在机器上只花费250镑。其次,操纵价值500镑的机器要50个工人;所以,操纵价值250镑的机器要25个工人(250镑)。再其次,“机器的维修”——价值500镑的机器的再生产——要10个工人;所以,价值250镑的机器的再生产只要5个工人(50镑)。这样一来,就是固定资本250镑和流动资本250镑,共计500镑。这笔资本的利润,按利润率20%计算,是100镑。于是,产品包含着300镑的工资和100镑的利润,共计400镑。在这种情况下,被雇的工人是30名。在这里无论如何应该假定,资本家(从事生产的)或者从消费者存在银行家那里的积蓄(400镑)中借用了资本,或者他自己除了等于消费者所积蓄的收入400镑之外还有资本。因为仅有400镑资本,他不可能在机器上花费250镑,又在工资上花费300镑。}

\begin{quote}{“当1200镑的总额花在直接劳动的产品上时,产品的价值分为工资1000镑和利润200镑〈工人人数为100名,工资为1000镑〉。当这笔钱一部分按一种方式用,一部分按另一种方式用时……产品的价值分为工资934镑和利润266镑〈即在使用机器的企业里的工人为60名,不使用机器劳动的工人为33.4名,工人总数为93.4名,他们总共得到934镑的工资〉。最后,按照第三种假定即全部款项花在机器与劳动的共同产品上,产品价值便分为900镑的工资〈因为在这种情况下工人人数为90名〉和300镑的利润。”(同上,第114—117页)[781]“资本家除非积累更多资本,否则在采用机器后他便不能使用和以前一样多的劳动。但是,这种物品的消费者在物品的价格下降后所积蓄的收入,将会增加他们对这种或其他某种物品的消费,从而能造成对一部分被机器排挤的劳动的需求,虽然不是对全部这种劳动的需求。”(同上,第119页)“麦克库洛赫先生认为,一个生产部门采用机器,必然会在其他某一生产部门造成同样大的或更大的对被解雇的工人的需求。为了证明这一点,他假定,到机器全部磨损以前为补偿机器价值所必需的年提成,每年都将造成越来越多的对劳动的需求。\endnote{匿名著作《政治经济学大纲》的作者(卡泽诺夫)在这里提到的麦克库洛赫的话,见约·雷·麦克库洛赫《政治经济学原理》1825年爱丁堡版第181—182页。参看本册第183页。——第68页。}但是,到一定时期末了,年提成加在一起只能等于机器的原有价值加机器使用期间的利息,因此很难理解,这种提成到底怎么会造成一个比不采用机器时更大的对劳动的需求。”(同上,第119—120页)}\end{quote}

当然,当机器的磨损只是在计算中而实际尚未发生作用的那段时间里,折旧基金本身也可以作积累之用。但是这样造成的对劳动的需求,无论如何也比全部投入机器的资本——而不只是补偿机器每年磨损所必须的年提成——用于工资时所产生的需求小得多。麦克彼得\authornote{把“麦克库洛赫”写作“麦克彼得”,含有嘲笑的意思。“彼得”一词,来自德文“dummerPeter”(直译是:“笨蛋彼得”,意为“蠢货”、“笨蛋”)。——编者注}始终是一头蠢驴。这段话所以值得注意,只是因为这里说出了折旧基金本身就是积累基金这个思想。

\tchapternonum{[第二十章]李嘉图学派的解体}

\tchapternonum{(1)罗·托伦斯}

\tsubsectionnonum{[(a)斯密和李嘉图论平均利润率和价值规律的关系]}

[782]罗·托伦斯《论财富的生产》1821年伦敦版。

对竞争——生产的外部表现——的考察表明,等量资本平均说来提供等量利润,或者说,如果平均利润率既定,利润量就取决于预付资本量(而平均利润率的含义也不过如此)。

亚·斯密记录了这个事实。关于这个事实同他提出的价值理论如何联系的问题,并没有引起他丝毫的内心不安;这个问题所以没有使他不安,尤其是因为除了他的所谓的内在理论以外,他还提出了其他各种各样的理论,并且可以随便采用其中这一种或那一种。这个情况使他产生的唯一反应,就是对那种试图把利润归结为监督劳动的工资的观点进行反驳,因为,撇开其他一切情况不谈,监督劳动并不是按生产规模扩大的程度增长的,而且生产规模不扩大,预付资本的价值也能增长(例如由于原料的涨价)\endnote{马克思在《剩余价值理论》第一册中引用并分析了亚当·斯密《国富论》中的这一段话(见本卷第1册第70—72页)。——第70页。}。在斯密那里没有决定平均利润和平均利润量本身的内在规律。他只限于说,竞争使这个x缩小。

李嘉图到处(除了少数的而且只是偶然的说明以外)都把利润和剩余价值直接等同起来。因此,在他看来,出卖商品之所以获得利润,并不是因为商品高于它的价值出卖,而是因为商品按照它的价值出卖。然而在考察价值方面(李嘉图的《原理》第一章),是他第一个一般地考虑到商品的价值规定同等量资本提供等量利润这一现象的关系。等量资本所以能够提供等量利润,只是因为它们生产的商品尽管不是按相同的价格出卖(然而可以说,如果把固定资本中没有被消费的部分的价值加到产品价值上,结果就会有相同的价格),但提供的剩余价值相同,提供的价格超过预付资本价格的余额相同。而且,李嘉图第一个注意到,同量资本决非具有相同的有机构成。他所理解的这种构成上的区别,是他从亚·斯密那里找到的区别即流动资本和固定资本,也就是说,他只看到从流通过程中产生的区别。

李嘉图根本没有直接说,有机构成不同从而推动的直接劳动量不同的各资本生产价值相同的商品并提供相同的剩余价值(他把剩余价值和利润等同起来)这一事实,同价值规律乍看起来是矛盾的。相反,他是以资本和一般利润率的存在为前提去研究价值的。他一开始就把费用价格和价值等同起来,而没有看到,这个前提一开始就同价值规律乍看起来是矛盾的。他只是根据这种包含着主要矛盾和基本困难的前提去考察个别的情况——工资的变动,即工资的提高或降低。为了使利润率保持不变,工资的提高或降低(与之相适应的是利润的下降或提高)必须对有机构成不同的资本发生不同的影响。如果工资提高,从而利润下降,那末用较大比例的固定资本生产的商品的价格就下降。反之,结果也相反。因此,各商品的“交换价值”在这种情况下不是由生产各该商品所需要的劳动时间决定。换句话说,有机构成不同的资本具有相同的利润率这个规定(不过,李嘉图只是在个别的情况下并且通过那样曲折的途径才得出这个结论),同价值规律是矛盾的,或者象李嘉图所说,成为价值规律的例外;对此马尔萨斯正确地指出,随着工业的发展[783],李嘉图的规则成了例外,而例外成了规则。\authornote{见本册第25页。——编者注}在李嘉图那里,矛盾本身没有表达清楚,即没有以下列形式表达:尽管一种商品比另一种商品包含的无酬劳动多,——因为在对工人的剥削率相同时,无酬劳动量取决于有酬劳动量,就是说,取决于所使用的直接劳动量,——但是它们提供的价值相同,或者说,提供的无酬劳动超过有酬劳动的余额相同。相反,矛盾在他那里只是以这种独特的形式出现:在某些情况下,工资——工资的变动——影响商品的费用价格(他说,影响交换价值)。

同样,资本的周转时间的区别,——资本不论是在生产过程中(即使不是在劳动过程中)\endnote{马克思在他的1857—1858年手稿中谈到关于特别是在农业中存在的生产时间和劳动时间的区别,以及与此有关的资本主义在农业中发展的特点(见卡·马克思《政治经济学批判大纲》1939年莫斯科版第560—562页)。生产期间(除了劳动时间以外,还包括劳动对象仅仅接受自然界的自然过程的作用的时间),这个概念马克思在《资本论》第二卷第二篇第十三章作了详细的阐述。参看本愿第2册第19页。——第72页。}还是在流通过程中停留时间较长,它为了本身的周转所需要的都不是更多的劳动,而是更多的时间,——对于利润的均等也毫无影响。这又和价值规律相矛盾,——照李嘉图说来,这又是价值规律的例外。

可见,李嘉图把问题阐述得非常片面。如果他以一般的形式来表达,他也就会使问题得到一般的解决。

但是,李嘉图仍然有很大的功绩:他觉察到价值和费用价格之间存在差别,并在一定的场合表述了(尽管只是作为规律的例外)这个矛盾:有机构成不同的资本,就是说,归根结蒂始终是那些使用不同量活劳动的资本,提供相同的剩余价值(利润),而且,——如果把一部分固定资本进入劳动过程而不进入价值形成过程这一情况撇开不谈,——提供相同的价值即具有相同价值(更确切地说是费用价格,但是李嘉图把它们混淆了)的商品。

\tsubsectionnonum{[(b)托伦斯在价值由劳动决定和利润源泉这两个问题上的混乱。局部地回到亚·斯密那里和回到“让渡利润”的见解]}

我们在前面已经看到\authornote{见本册第4页和第22—25页。——编者注},马尔萨斯利用这个[由大卫·李嘉图发现的关于价值规律和构成不同的资本有相同利润这一事实之间的矛盾]来否定李嘉图的价值规律。

托伦斯在他的著作一开头就从李嘉图的这个发现出发,但是决不是为了解决问题,而是为了把“现象”本身说成是现象的规律。

\begin{quote}{“假定所使用的是耐久程度不同的资本。如果一个毛织厂主和一个丝织厂主各使用2000镑资本,前者把1500镑花在耐用的机器上,500镑用在工资和材料上,而后者花在耐用的机器上的只有500镑,花在工资和材料上的是1500镑。假定这种固定资本每年消费1/10,利润率是10%。因为毛织厂主的2000镑资本必须有2200镑的进款,才给他提供10%的利润,又因为固定资本的价值经过生产过程从1500镑减少到1350镑,所以生产的商品必须卖850镑。同样,因为丝织厂主的固定资本经过生产过程减少了1/10,即由500镑减少到450镑,所以为了要给他的2000镑总资本提供普通利润率,所生产的丝就必须卖1750镑……如果所使用的是量相同而耐久程度不同的资本,那末,一个生产部门生产的商品连同资本余额,跟另一个生产部门生产的产品和资本余额,在交换价值上将是相等的。”(托伦斯《论财富的生产》1821年伦敦版第28—29页)}\end{quote}

这里只是指出了,记录了竞争中暴露出来的现象。同样,这里只是假定了一个“普通利润率”,而没有解释它从哪里来,甚至也没有觉察到必须加以解释。

\begin{quote}{“等量资本,或者换句话说,等量积累劳动,往往推动不等量的直接劳动;但是这丝毫不改变事情的本质”,(第29、30页)}\end{quote}

就是说,不改变下述情况:产品的价值加上没有被消费的资本余额提供相等的价值,或者同样可以说,提供相等的利润。

托伦斯这个论点的功绩不在于他在这里也只是再次把现象记录下来而不加解释,而是在于,他确定了资本之间的差别是等量资本推动不等量的活劳动,尽管他把这说成“特殊”情况而又把事情弄糟了。如果价值等于生产商品所花费的、物化在商品中的劳动,那就很清楚,在商品按它的价值出卖时,商品中包含的剩余价值只能等于其中包含的无酬劳动,或者说剩余劳动。但是在对工人的剥削率相同的情况下,这种剩余劳动量,对“推动不等量的直接劳动”的资本来说——不管这种不等是由直接的生产过程引起,还是由流通时间引起——是不可能相同的。因此,托伦斯的功绩就在于他作了这种表述。他由此作出什么结论呢?结论是,在这里,[784]在资本主义生产中,价值规律发生了一个突变,就是说,由资本主义生产中抽象出来的价值规律同资本主义生产的现象相矛盾。而他用什么来代替这个规律呢?什么也没有,他只不过对应该解释的现象作了粗浅的缺乏思考的文字上的表述。

\begin{quote}{“在社会发展的初期〈就是说,正好是交换价值——作为商品的产品——一般说来几乎没有发展,因而价值规律也没有发展的时期〉商品的相对价值是由花费在商品生产上的劳动(积累劳动和直接劳动)的总量决定的。但是一旦有了资本积累,并且有了资本家阶级和工人阶级的区别,一旦在某一工业部门作为企业主出现的人自己不劳动,而预付给别人生存资料和材料,商品的交换价值就由花费在生产上的资本量,或者说积累劳动量决定了。”(同上,第33—34页)“只要两笔资本相等,它们的产品的价值就相等,不管它们所推动的,或者说它们的产品所需要的直接劳动量如何不同。如果两笔资本不等,它们的产品的价值就不等,虽然花费在它们的产品上的劳动总量完全相同。”(第39页)“因此,在资本家和工人之间发生上述分离以后,交换价值就开始由资本量,由积累劳动量决定,而不象在这种分离以前那样,由花费在生产上的积累劳动和直接劳动的总量来决定了。”(同上,第39—40页)}\end{quote}

这里,又不过是确认了以下现象:等量资本提供等量利润,或者说,商品的费用价格等于预付资本的价格加平均利润;不过是暗示了,由于“等量资本推动不等量的直接劳动”,上述这种现象乍看起来同商品价值决定于商品中包含的劳动时间这一规定是不相容的。托伦斯说资本主义生产的这种现象,只有当资本存在——资本家阶级和工人阶级出现——时,当客观的劳动条件独立化为资本时才表现出来,这是同义反复。

但是,商品生产的[必要因素]——资本家和工人、资本和雇佣劳动——的分离是怎样推翻商品的价值规律的,这一点托伦斯只是从不理解的现象中“推论”出来的。

李嘉图试图证明,资本和雇佣劳动的分离丝毫没有改变——除了某些例外——商品的价值规定,托伦斯以李嘉图的例外为依据否定了规律本身。托伦斯回到了亚·斯密那里(李嘉图的论证是反对斯密的),按照斯密的看法,诚然,“在社会发展的初期”,当人们彼此还只是作为交换商品的商品所有者相对立时,商品的价值决定于商品中包含的劳动时间,但是资本和土地所有权一形成,就不是这样了。这就是说(正如我在第一部分\endnote{马克思指《政治经济学批判》第一分册。见《马克思恩格斯全集》中文版第13卷第49页。——第75页。}已经指出的),适用于作为商品的商品的规律,只要商品一被当作资本或当作资本的产品,只要一般说来一发生商品向资本的转变,就不适用于商品了。另一方面,只有整个产品全都转化为交换价值,产品生产的构成要素本身全都作为商品加入产品,产品才全面地具有商品的形式,就是说,只是随着资本主义生产的发展并在资本主义生产的基础上,产品才全面地成为商品。因此,商品的规律应该在不生产(或只是部分地生产)商品的生产中存在,而不应该在产品作为商品存在的那种生产中存在。这个规律本身,同作为产品的一般形式的商品一样,是由资本主义生产条件中抽象出来的,而它恰恰不适用于资本主义生产。

此外,关于“资本和劳动”的分离影响价值规定的议论——撇开所谓在资本还不存在的情况下资本不能决定价格这个同义反复不谈——又是对表现在资本主义生产表面的事实的非常肤浅的转述。只要每个人都用自己的工具劳动,都自己出卖自己生产的产品{但是实际上,产品按[785]全社会规模出卖的必然性,决不会同用自己的劳动条件进行的生产相一致},无论工具的费用或他自己从事的劳动的费用就都属于他的费用。资本家的费用是由预付资本,由他花费在生产上的价值总额构成,而不是由劳动构成,这种劳动是他没有从事过的而且他花费在这种劳动上的无非是他为它所支付的。从资本家的观点看来,这是一个很好的理由,可以用来说明他们不必按照一定资本所推动的直接劳动量,而应按照他们所预付的资本量彼此计算和分配(全社会的)剩余价值。但是这个理由决不能说明,这个应这样分配和被这样分配的剩余价值是从哪里来的。

托伦斯认为商品的价值由劳动量决定,就这一点来说,他还是坚持李嘉图理论的,但是他断言,只有花费在商品生产上的“积累劳动量”才能决定商品的价值。在这里,托伦斯又陷入极端混乱了。

因此,例如,呢绒的价值由织机、羊毛等等和工资中的积累劳动决定。这一切形成呢绒生产所需要的积累劳动的构成要素。在这里“积累劳动”一词不外是物化劳动,物化劳动时间。但是,当呢绒织成,生产结束的时候,花费在呢绒上的直接劳动也就转化为积累劳动,或者说物化劳动。因此,为什么织机和羊毛的价值应当由它们包含的物化劳动(这不外是物化在一个对象中,一个产品中,一个有用物中的直接劳动)决定,而呢绒的价值却不应当这样呢?如果呢绒也作为构成要素进入新的生产,例如进入染坊或缝纫工场,那它就是“积累劳动”,上衣的价值就由工人的工资、工人的工具和呢绒的价值决定,而呢绒的价值本身则由呢绒中的“积累劳动”决定。如果我把商品作为资本,就是说,在这里把它同时作为生产条件来考察,商品的价值就归结为直接劳动,这种直接劳动叫作“积累劳动”,因为它以物化的形式存在。相反,如果我把这同一商品作为商品来考察,作为产品和生产过程的结果来考察,这种商品的价值就不是由积累在商品本身的劳动决定,而是由积累在它的生产条件中的劳动决定了。

试图用资本的价值决定商品的价值,实际上是一个很妙的循环论证,因为资本的价值等于构成资本的那些商品的价值。詹姆斯·穆勒反驳这个家伙的话是对的,他说:

\begin{quote}{“资本就是商品,说商品的价值由资本的价值决定,就等于说,商品的价值由商品的价值决定。”\endnote{马克思引的是詹姆斯·穆勒的著作《政治经济学原理》。这段话见该书第1版第74页、第2版第94页。在这里马克思大概转引自赛米尔·贝利《对价值的本质、尺度和原因的批判研究》一书(第202页),在这本书中,这段话也被看作是反对托伦斯的。——第77页。}}\end{quote}

这里还要指出下面一点。因为[在托伦斯那里]商品的价值由生产商品的资本的价值决定,换句话说,由积累和物化在这个资本中的劳动量决定,所以只有两种情况是可能的。

[在托伦斯看来,]商品包含着:第一,消费掉的固定资本的价值;第二,原料的价值,换句话说,包含[消费掉的]固定资本和原料中所包含的劳动量;第三,还包含物化在用作工资的货币或商品中的劳动量。

这样,这里只有两种情况是可能的。

包含在固定资本和原料中的“积累”劳动量,在生产过程之后和在生产过程之前是一样的。至于预付的“积累劳动”的第三部分,工人则用他的直接劳动来补偿,就是说,在这一场合,加在原料等等上的“直接劳动”在商品中,在产品中所代表的积累劳动,正好和工资中所包含的一样多。或者,这种“直接劳动”代表更多的劳动量。如果它代表更多的劳动量,那末,商品就比预付资本包含更多的积累劳动。这样,利润就正好从商品包含的积累劳动超过预付资本包含的积累劳动的余额中产生。这样,商品的价值[786]照旧决定于商品中包含的劳动量(积累劳动加直接劳动,而后者现在在商品中也是作为积累劳动,而不再作为直接劳动存在了。它在生产过程中是直接劳动,在产品中是积累劳动)。

或者[也就是在第一种情况下],直接劳动代表的只是预付在工资中的劳动量,只是这个劳动量的等价物。(如果直接劳动比这个劳动量少,那末,要说明的就不是资本家为什么获利,而是资本家怎么不亏损的问题了。)在这种情况下,利润从哪里来呢?剩余价值,即商品价值超过商品生产的构成要素的价值,或者说超过预付资本的价值的余额,从哪里产生呢?它不是从生产过程本身产生(因此只有在交换或流通过程中实现),而是从交换,从流通过程产生了。这样,我们就回到马尔萨斯那里,回到粗浅的重商主义的“让渡利润”观念。托伦斯先生也前后一贯地得出了这种观念,虽然他又是那样前后不一贯,以致不是用一个无法解释的从天上掉下来的基金(这个基金不仅构成商品的等价物,而且构成超过这个等价物的余额,它由始终能够高于商品的价值支付商品而自己并不高于商品的价值出卖商品的这种买者的资金构成)来解释这个名义价值而把问题化为乌有。托伦斯不象马尔萨斯那样前后一贯地求助于这种虚构,相反地却认定“有效需求”即支付产品的价值额,仅仅由供给产生,因而也是商品;在这里,绝对无法理解,双方都既作为卖者又作为买者怎么能同样地相互欺诈。

\begin{quote}{“对某一商品的有效需求,总是由资本的组成部分,或者说,消费者为交换这个商品而能够和愿意提供的生产商品所必需的物品的量决定的,在利润率既定时,总是由这个量来衡量的。”(托伦斯,同上第344页)“供给增长是有效需求增长的唯一原因。”(同上,第348页)}\end{quote}

马尔萨斯从托伦斯的书中引了这句话,不无理由地对这种观点提出异议。(《政治经济学定义》1827年伦敦版第59页)\authornote{见本册第58页。——编者注}

下面托伦斯关于生产费用等的论述,表明他确实得出了上述荒谬的结论:

\begin{quote}{“市场价格〈马尔萨斯称之为“购买价值”〉总是包括某一时期的普通利润率。自然价格由生产费用构成,或者换句话说,由生产或制造商品时的资本支出构成,它不可能包括利润率。”(托伦斯,同上第51页)“一个租地农场主支出了100夸特谷物,而收回120夸特,这20夸特就是利润,把这个余额或者说利润,叫作他的支出的一部分,是荒谬的……同样,工厂主收回一定量成品。这些成品的交换价值高于材料等的价值。”(第51—53页)“有效的需求在于,消费者能够和愿意通过直接的或间接的交换付给商品的部分,大于生产商品时所耗费的资本的一切组成部分。”(第349页)}\end{quote}

120夸特谷物无疑比100夸特多。但是,如果人们象在这种场合一样,只考察使用价值和使用价值所经历的过程,其实也就是生长过程或生理[787]过程,那末,说它们(虽然不是说20夸特本身,但确实是说构成这20夸特的要素)不进入生产过程,就会是错误的。否则,这20夸特就不能从生产过程中出来。除了100夸特的谷物(种子\endnote{马克思在这里是从下述假定出发的:谷物的一切生产费用,即托伦斯提出的100夸特,都是种子的支出。实际上,生产120夸特谷物所花费的种子要少得多——比如说,20或30夸特。其余70或80夸特用于支付劳动工具、肥料、工人的工资等。但是,这种情况对于马克思的论证是毫无意义的。——第80页。})以外,进入使100夸特谷物转化为120夸特的过程的,还有由肥料提供的化学成分,土地内包含的盐,以及水、空气、阳光。这些要素、成分、条件(使100夸特转化为120夸特的自然界的支出)的转化和进入,是在生产过程本身进行的,而这20夸特的要素是作为生理的“支出”进入这个过程本身的,从100夸特转化为120夸特就是这个过程的结果。

单从使用价值的观点看,这20夸特不纯粹是利润。这不过是无机要素被有机部分同化并转化为有机物质罢了。没有作为生理支出的物质加入,无论如何决不会由100夸特变为120夸特。因此,事实上可以说(即使单从使用价值的观点来看,从谷物作为谷物来看),以无机的形式作为“支出”加入谷物的东西,则以有机的形式作为现有的结果20夸特,即收获的谷物超过播种的谷物的余额出现。

但是,这种考察方法本身,同利润问题毫无关系,就好比我们不能说,通过劳动过程把金属拔成一千倍长的金属丝,因为它的长度增加到一千倍,就代表一千倍的利润。就金属丝来说,长度增加了;就谷物来说,夸特数增加了。但是,只和交换价值有关的利润既不是由增加的长度也不是由增加的数量形成,虽然这个交换价值也表现为剩余产品。

至于交换价值,则无须再说明:90夸特谷物的价值可能丝毫不小于(甚至大于)100夸特的价值,100夸特的价值可能大于120夸特的价值,120夸特的价值可能大于500夸特的价值。

可见,托伦斯是根据一个同利润即同产品价值超过预付资本价值的余额毫无关系的例子得出关于利润的结论的。即使从生理方面,从使用价值的观点看,他的例子也是错误的,因为实际上在他那里,作为剩余产品出现的20夸特谷物已经以这种或那种方式(虽然是以另外的形式)存在于生产过程本身了。

不过,托伦斯最后还是脱口说出了利润是让渡利润这种陈旧的天才的观念。

\tsubsectionnonum{[(c)托伦斯和生产费用的概念]}

托伦斯一般地提出了什么是生产费用这个争论问题,这是他的功绩。李嘉图经常把商品的价值同生产费用(就它等于费用价格而言)混淆起来,因此,他看到萨伊虽然也认为生产费用决定价格但是得出不同的结论\authornote{见本卷第2册第535—536页。——编者注},就感到惊奇。马尔萨斯同李嘉图一样,认定商品的价格由生产费用决定,而且他同李嘉图一样把利润算在生产费用之内。但是他用完全不同的方式给价值下定义,就是说,他不用商品中包含的劳动量,而用商品能够支配的劳动量来决定价值。

生产费用这个概念的含混是由资本主义生产的性质本身引起的。

第一,对于资本家来说,他所生产的商品的费用,自然是他为商品所花费的东西。除了预付资本的价值以外,商品没有花费他任何东西,就是说,他没有在商品上支出其他任何价值。如果他为了生产商品,在原料、工具、工资等上面支出100镑,商品就花费他100镑,不会多些,也不会少些。除了包含在这笔预付中的劳动,就是说,除了包含在预付资本中的、决定为生产过程预付的商品的价值的积累劳动,商品不花费他任何劳动。他为直接劳动花费的,是他为直接劳动支付的工资。除了工资,直接劳动没有花费他任何东西,而除了直接劳动,他只预付不变资本的价值。

[788]托伦斯就是从这个意义上理解生产费用的,而每个资本家在计算利润时(不管利润率如何),也是从这个意义上理解生产费用的。

在这里,生产费用等于资本家的预付,等于预付资本的价值,等于为生产过程预付的商品中包含的劳动量。任何一个经济学家,李嘉图也在内,都是从预付、支出等意义上来使用生产费用的这个定义的。马尔萨斯把这叫作生产价格,而同购买价格相对立。剩余价值向利润形式的转化同预付的这个定义是相适应的。

第二,按第一个定义,生产费用是资本家在生产过程中为制造商品所支付的价格,因而是他为商品所花费的东西。但是,资本家为生产商品所花费的东西和商品生产本身所花费的东西是完全不同的两回事。资本家为生产商品支付过报酬的劳动(物化劳动和直接劳动)和生产商品所必要的劳动在量上是完全不同的。它们之间的差额也就形成预付价值和所得价值之间的差额,形成资本家购买商品的价格和商品出卖的价格(如果商品按照它的价值出卖)之间的差额。如果这个差额不存在,货币或商品就决不会转化为资本。随着剩余价值的消失,利润的源泉也就消失。商品生产本身的费用是由商品生产过程中消费的资本的价值,也就是由进入商品的物化劳动量加花费在商品生产上的直接劳动量构成的。在商品中消费的“物化劳动”和“直接劳动”的总量构成商品生产本身的费用。只有通过这个物化的和直接的劳动量的生产消费,商品才能制造出来。这也是商品作为产品,作为商品以及作为使用价值从生产过程中出来的必要条件。在现实的劳动过程的技术条件不变时,或者同样可以说,在劳动生产力的一定的发展水平毫无变化时,不管利润或工资怎样变动,商品的这个内在的生产费用保持不变。在这个意义上,商品的生产费用等于商品的价值。花费在商品上的活劳动和资本家支付过报酬的活劳动是不同的东西。所以,对于资本家来说的商品的生产费用(他的预付)从一开始就和商品生产本身的费用,和商品的价值不同。商品的价值(就是商品本身所花费的东西)超过预付资本的价值(即资本家为商品所花费的东西)的余额形成利润,因而,利润的产生不是由于商品高于它的价值出卖,而是由于商品高于资本家所支付的预付资本的价值出卖。

商品的生产费用,即商品的内在的生产费用等于商品的价值,也就是等于商品生产所必需的(物化的和直接的)劳动时间总量——这个定义表达了商品生产的基本条件,并且在劳动的生产力不变时保持不变。

第三,但是,我在前面已经指出\authornote{见本卷第2册第19—22、27、65—70、191—261页。——编者注},每一个别行业或个别生产部门的资本家决不是按照商品即特殊行业或特殊生产部门或特殊生产领域的产品本身所包含的价值出卖商品的,因此,这个资本家得到的利润量不会和剩余价值量,剩余劳动量,或者说物化在他所出卖的商品中的无酬劳动量相等。相反,资本家在他的商品中平均说来能实现的剩余价值,只是和这种商品作为社会资本一定部分的产品所分摊到的剩余价值相等。如果社会资本等于1000,某个[789]生产部门的资本等于100,如果剩余价值(从而剩余价值物化在其中的剩余产品)的总量等于200,即20%,那末,投入这个生产部门的资本100将按照120的价格出卖它的商品,而不管这个商品的价值是120,还是多于或少于120,就是说,不管这个商品中包含的无酬劳动是否等于预付在商品上的劳动的1/5。

这就是费用价格,如果谈到本来意义上的(经济学意义上的,资本主义意义上的)生产费用,那末这就是预付资本的价值加平均利润的价值。

很清楚,不管个别商品的这种费用价格怎样偏离商品的价值,它都是由社会资本的总产品的价值决定的。各个资本,由于它们利润的平均化,作为社会总资本的一定部分相互发生关系,并且作为这样的一定部分从剩余价值(剩余产品),剩余劳动或无酬劳动的总基金中获得股息。这丝毫没有改变商品的价值,丝毫没有改变下述情况:不管商品的费用价格等于、大于或小于商品的价值,只要商品的价值没有生产出来,就是说,只要生产商品所必要的物化劳动和直接劳动的总量没有花费在商品上,商品是决不能生产出来的。这个劳动(不仅是有酬劳动而且是无酬劳动)量必须花费在商品上;虽然有些生产部门的一部分无酬劳动由“资本家同伙”\endnote{马克思在《资本论》第三卷中论证了资本家们作为“同伙”的这个特点。在利润率平均化的过程中,“每一单个资本家,同每一个特殊生产部门的所有资本家总体一样,参与总资本对全体工人阶级的剥削,并参与决定这个剥削的程度”。(见马克思《资本论》第3卷第10章)马克思在研究了这个过程后写道:“……我们在这里得到一个象数学一样精确的证明:为什么资本家在他们的竞争中表现出彼此都是虚伪的兄弟,但面对着整个工人阶级却结成真正的共济会团体。”(同上)参看本卷第2册第21页。——第84页。}占有,而不是由推动这个特殊生产部门的劳动的资本家占有,但是这丝毫不改变资本和劳动之间的一般关系。其次,很清楚:不管商品的价值和费用价格之间是什么关系,费用价格总是随着价值的变动,也就是随着生产商品所必要的劳动量的变动而变动,而提高或降低。此外,很清楚:一部分利润始终必须代表剩余价值,代表物化在这个商品本身中的无酬劳动,因为在资本主义生产基础上,每个商品包含的在它上面花费的劳动,比推动这个劳动的资本家支付过报酬的劳动多。一部分利润可能由不是花费在一定行业所提供的或一定生产领域所生产的商品上的劳动构成;但是这样一来,就有其他某个生产领域生产的其他某个商品,它的费用价格降到它的价值以下,或者说,它的费用价格中计算和支付的无酬劳动,比它包含的无酬劳动少。

因此很清楚,虽然大多数商品的费用价格必定偏离它们的价值,就是说,虽然它们的“生产费用”必定偏离它们包含的劳动总量,但是,不仅这种生产费用和费用价格由商品的价值决定,并同价值规律相符合(而不是和它相矛盾),而且甚至生产费用和费用价格的存在本身,也只有在价值和价值规律的基础上才能理解,没有这个前提,它们的存在就是不可思议的和荒谬的。

同时我们也就可以理解,那些一方面看到了竞争中的实际现象,另一方面又不理解价值规律和费用价格规律之间的中介过程的经济学家,为什么求助于虚构,说是资本而不是劳动决定商品的价值,或者确切些说,价值根本不存在。

[790]利润加入商品的生产费用;亚·斯密正确地把利润作为构成要素包括在商品的“自然价格”中,因为在资本主义生产的基础上,如果商品不提供等于预付资本价值加平均利润的费用价格,它就——最终地、照例地——不会拿到市场上去。或如马尔萨斯所说(虽然他不理解利润的起源,不理解利润的真正原因),因为利润,从而包括利润在内的费用价格,是(资本主义生产的基础上的)商品供给的条件。商品要生产出来,要进入市场,它至少必须为卖者提供这个市场价格,这个费用价格,而不管它本身的价值比这个费用价格大还是小。对于资本家来说,只要他的商品的价格中包含的从无酬劳动或固定了无酬劳动的剩余产品的总基金中取得的量,同其他任何等量资本从这个总基金中获得的量相等就行了,至于他的商品比其他商品包含的无酬劳动多还是少,那是无关紧要的。在这个意义上,资本家是“共产主义者”。自然,在竞争中每个人都力求得到比平均利润多的利润,而这只有在别人得到的利润比平均利润少的情况下才是可能的。正是由于这种斗争,平均利润才得以形成。

以预付资本(不管是不是借的)的利息形式在利润中实现的一部分剩余价值,对资本家来说,也表现为支出,表现为他作为资本家的一项生产费用,就象利润根本就是资本主义生产的直接目的一样。而在利息上(特别是对借入的资本说)这一点也表现为资本家的生产活动的实际前提。

这一点同时表明了生产形式和分配形式的区别是怎么回事。利润、分配形式,在这里同时又是生产形式、生产条件、生产过程的必要的构成要素。因此,约·斯·穆勒等把资产阶级的生产形式看成绝对的,而把资产阶级的分配形式看成相对的,历史的,因而是暂时的,是多么愚蠢,——这一点以后还要回过头来谈。分配形式只不过是从另一个角度看的生产形式。构成资产阶级分配的界限的特征——也就是特殊的局限性——作为控制生产和支配生产的特定性质加入生产本身。但是,资产阶级的生产,由于它本身的内在规律,一方面不得不这样发展生产力,就好象它不是在一个有限的社会基础上的生产,另一方面它又毕竟只能在这种局限性的范围内发能生产力,——这种情况是危机的最深刻、最隐秘的原因,是资产阶级生产中种种尖锐矛盾的最深刻、最隐秘的原因,资产阶级的生产就是在这些矛盾中运动,这些矛盾,即使粗略地看,也表明资产阶级生产只是历史的过渡形式。

其次,这一点被例如西斯蒙第粗浅地但又相当正确地看成是为生产的生产同因此\authornote{即因为为生产的生产,而不是为工人生产者的生产。——编者注}而排除了生产率的绝对发展的分配之间的矛盾。

\tchapternonum{(2)詹姆斯·穆勒[解决李嘉图体系的矛盾的不成功的尝试]}

[791]詹姆斯·穆勒《政治经济学原理》1821年伦敦版(1824年伦敦第2版)。

穆勒是第一个系统地阐述李嘉图理论的人,虽然他的阐述只是一个相当抽象的轮廓。他力求做到的,是形式上的逻辑一贯性。“因此”,从他这里也就开始了李嘉图学派的解体。在老师[李嘉图]那里,新的和重要的东西,是在矛盾的“肥料”中,从矛盾的现象中强行推论出来的。作为他的理论基础的矛盾本身,证明理论借以曲折发展起来的活生生的根基是深厚的。而学生[穆勒]的情况却不是这样。他所加工的原料已不再是现实本身,而是现实经老师提炼后变成的新的理论形式了。一部分是新理论的反对者们的理论上的不同意见,一部分是这种理论同现实的往往是奇特的关系,促使他去进行把不同意见驳倒,把这种关系解释掉的尝试。在进行这种尝试时,他自己也陷入了矛盾,并且以他想解决这些矛盾的尝试表明,他教条式地维护的理论正在开始解体。穆勒一方面想把资产阶级生产说成是绝对的生产形式,并且从而试图证明,这种生产的真实矛盾不过是表面上的矛盾。另一方面,他力图把李嘉图的理论说成是这种生产方式的绝对的理论形式,并且同样用形式上的理由把有些已为别人所指出、有些是摆在他本人眼前的理论上的矛盾辩解掉。不过,穆勒在某种程度上也还是比李嘉图的观点前进了一步,越过了李嘉图本人阐述观点时所划的界限。穆勒还维护了李嘉图所维护的历史的利益——反对土地所有权的产业资本的利益,而且更加坚决地从理论中作出了实际结论,例如,他从地租理论做出了反对土地私有权存在的实际结论,他想或多或少直接地把土地私有变为国有。穆勒的这个结论和他这方面的观点,我们不打算在这里研究。

\tsectionnonum{[(a)把剩余价值同利润混淆起来。利润率平均化问题上的烦琐哲学。把对立的统一归结为对立的直接等同]}

在李嘉图的学生们那里,也象在李嘉图本人那里一样,看不到剩余价值和利润的区别。李嘉图本人只是在工资的变动可能对有机构成不同(这在李嘉图那里也只是涉及流通过程时谈到)的各资本产生的不同影响中,才注意到两者的区别。无论是李嘉图本人还是他的学生们都没有想到,即使我们考察的不是各个不同部门的资本,而是单独的每一笔资本,只要它不是仅仅由可变资本构成,不是只花在工资上的资本,那末,利润率和剩余价值率就有区别,因而利润就必然是剩余价值的一种进一步发展了的、发生了特殊变化的形式。只是在谈到不同生产领域的、由不同比例的固定部分和流动部分构成的各资本的相等的利润——平均利润率时,他们才注意到剩余价值和利润的区别。在这方面,穆勒只是把李嘉图在第一章《论价值》中陈述的东西重复一遍,加以通俗解释罢了。穆勒在这个问题上产生的唯一新的疑问是:

穆勒指出,“时间本身”(就是说,不是劳动时间,而是单纯时间)不生产任何东西,因此也不生产“价值”。在这种情况下,一笔资本,象李嘉图所说,由于需要更长的时间进行周转,和另一笔使用了更多的直接劳动但周转得更快的资本提供同样多的利润这个事实,怎么会同价值规律相符合呢?我们看到,穆勒在这里只是抓住了一个非常个别的情况,这个情况概括起来可以这样说:费用价格以及作为它的前提的[792]平均利润率(从而包含十分不同的劳动量的商品的相同价格),怎么会同利润无非是商品中包含的劳动时间的一部分,也就是资本家不付等价物而占有的一部分这种情况相符合呢?然而在考察平均利润率和费用价格时,提出的是同价值规定毫无关系的、纯粹外在的观点,例如这样的观点:如果有个资本家的资本因为要——譬如说——象葡萄酒那样较长久地停留在生产过程中(或者,在其他场合,较长久地停留在流通过程中),必须完成较长时间的周转,那末,这个资本家就应该得到不能使他的资本增殖的那段时间的补偿。但是,没有使价值增殖的时间怎么能创造价值呢?

穆勒关于“时间”的论点是:

\begin{quote}{“时间什么也做不出来……因此,它怎么能够增加价值呢?时间只不过是一个抽象的术语。它是一个词,一种音响。无论把一个抽象的单位说成是价值尺度,还是把时间说成是价值的创造者,在逻辑上都同样是荒谬的。”\endnote{詹姆斯·穆勒书中的这一段话,马克思是从贝利《对价值的本质、尺度和原因的批判研究》一书(第217页)转引的,这从该引文与穆勒书中的原文稍有出入可以看出来。——第89页。}(《政治经济学原理》第2版第99页)}\end{quote}

其实,在说明不同生产领域的资本之间的补偿理由时,问题并不涉及剩余价值的生产,却涉及剩余价值在不同类别的资本家之间的分配。因此,在这里有意义的是同价值规定本身绝对没有任何关系的观点。在这里,迫使某一特殊生产领域的资本放弃在其他领域可能生产更多剩余价值的条件的一切,都是补偿理由。例如,使用的固定资本多而流动资本少;使用的不变资本多于可变资本;资本必须较长久地停留在流通过程中;最后还有一种情况,就是资本必须较长久地停留在生产过程中而不经历劳动过程,这种情况,每逢生产过程按其工艺性质要求中断以便使制造中的产品经受自然力的作用时(例如,葡萄酒置于窖内),都会发生。在所有这些场合,——穆勒特别注意的是其中最后一种,可见,他把他所遇到的困难看得十分狭窄,只看作一种个别现象,——都会发生补偿。其他领域生产的剩余价值,有一部分会纯粹按照这些在直接剥削劳动方面条件比较不利的资本的数量转给这些资本(这种平均化是由竞争实现的,在平均化的条件下,每一笔个别资本都只是作为社会资本的一定部分出现)。只要理解剩余价值和利润的关系,其次理解利润平均化为一般利润率,这种现象是十分简单的。但是,如果想不经过任何中介过程就直接根据价值规律去理解这一现象,就是说,根据某一个别行业的个别资本所生产的商品中包含的剩余价值即无酬劳动(也就是根据直接物化在这些商品本身中的劳动)来解释这一资本所取得的利润,那末,这就是一个比用代数方法或许能求出的化圆为方问题更困难得多的问题。这简直就是企图把无说成有。但是,穆勒正是企图用这种直接的形式来解决问题。因此,这里实质上不可能解决问题,而只能口头上诡辩地把困难辩解掉,就是说,只能是烦琐哲学。穆勒开了这个头。而在麦克库洛赫这样一个无耻之徒那里,这种做法就具有故作高深的无耻性质了。

贝利的话最能说明穆勒提出的解决问题的办法:

\begin{quote}{“穆勒先生作了一次把时间的作用归结为耗费劳动的独特尝试。他说(《原理》1824年第2版第97页):‘如果窖中的葡萄酒因放置一年而价值增加1/10,那末,认为在葡萄酒上多耗费了1/10的劳动,是正确的。’……一件事只有当它确实发生了,[793]认为它已经发生才是正确的。在所举的例子中,根据假定,任何人都没有接近过葡萄酒,没有为它花费一刹那时间,或稍微动一动肌肉。”(《对价值的本质、尺度和原因的批判研究》1825年伦敦版第219—220页)}\end{quote}

在这里,一般规律同进一步发展了的具体关系之间的矛盾,不是想用寻找中介环节的办法来解决,而是想用把具体的东西直接列入抽象的东西,使具体的东西直接适应抽象的东西的办法来解决。而且是想靠捏造用语,靠改变事物的正确名称来达到这一点。(我们看到的确实是一场“用语的争论”\endnote{暗指反对“用语的争论”的匿名论战著作《评政治经济学上若干用语的争论,特别是有关价值、供求的争论》1821年伦敦版。马克思在后面(本章第3节)对这本匿名著作作了详细的评述。——第91页。},但是它之所以是“用语的”,是因为他们企图用空话来解决没有得到实际解决的实际矛盾。)这种手法在穆勒那里还只是处于萌芽状态;它比反对者的一切攻击更严重得多地破坏了李嘉图理论的整个基础,这一点在考察麦克库洛赫时可以看出来。

穆勒只是在他绝对找不到其他出路的时候,才求助于这种方法。但是,他的基本方法与此不同。在经济关系——因而表示经济关系的范畴——包含着对立的地方,在它是矛盾,也就是矛盾统一的地方,他就强调对立的统一因素,而否定对立。他把对立的统一变成了这些对立的直接等同。

例如,商品隐藏着使用价值和交换价值的对立。这种对立进一步发展,就表现为、实现为商品的二重化即分为商品和货币。商品的这种二重化作为过程出现在商品的形态变化中,在这种变化中,卖和买是一个过程的不同因素,但是这一过程的每一行为同时都包含着它的对立面。我在本书的第一部分\endnote{马克思指《政治经济学批判》第一分册。见《马克思恩格斯全集》中文版第13卷第87—88页。——第92页。}曾经指出,穆勒摆脱对立的办法是仅仅抓住买和卖的统一,从而把流通变成物物交换,又把从流通中搬来的范畴偷偷塞到物物交换里。还可参看我在那里关于他的货币理论所说的话,他在货币理论中对问题也采取了类似的态度。\endnote{同上,第169—172页。——第92页。}

在詹姆斯·穆勒那里有一些不适当的章节划分:《论生产》,《论分配》,《论交换》,《论消费》。

\tsectionnonum{[(b)穆勒使资本和劳动的交换同价值规律相符合的徒劳尝试。局部地回到供求论]}

关于工资,穆勒写道:

\begin{quote}{“人们发现,对工人说来,更加方便的是以预付的方式把工人的份额付给工人,而不是等到产品生产出来和产品的价值得到实现的时候。人们发现,适合于工人取得其份额的形式是工资。当工人以工资的形式完全得到了产品中他应得的份额时,这些产品便完全归资本家所有了,因为资本家事实上已经购买了工人的份额,并以预付的方式把这一份额支付给工人了。”(《政治经济学原理》,帕里佐的法译本,1823年巴黎版第33—34页)}\end{quote}

穆勒最大的特点是:正象货币在他看来只是为了方便而发明的一种手段一样,资本主义关系本身,按照他的意见,也是为了方便而想出来的。这种特殊的社会生产关系,是为了“方便”而发明出来的。商品和货币转化为资本,是由于工人不再以商品生产者和商品所有者的身分参加交换,相反,他们被迫不是出卖商品,而是把自己的劳动本身(直接把自己的劳动能力)当作商品卖给客观的劳动条件的所有者。工人与客观劳动条件的这种分裂,是资本和雇佣劳动的关系的前提,正象它是货币(或代表货币的商品)转化为资本的前提一样。穆勒以这种分离,这种分裂为前提,以资本家和雇佣工人的关系为前提,为的是以后再把以下的现象说成是方便的事情:工人不是出卖产品,不是出卖商品,而是出卖在他生产出产品之前他自己在产品(产品的生产丝毫不决定于他,生产的进行不由他作主)中所占的份额,[794]或者更确切地说,在资本家出卖、销出包含工人的份额的产品之前,工人在产品中所占的份额就已经由资本家支付了,就已经转化为货币了。

穆勒想通过这种对工资的观点来回避与这里所考察的关系的特殊形式相联系的特殊困难。对于认为工人是直接出卖自己的劳动(而不是出卖自己的劳动能力)的李嘉图体系来说,困难在于:既然商品的价值决定于生产该商品所耗费的劳动时间,那末在构成资本主义生产基础的、一切交换中最大的交换——资本家和雇佣工人之间的交换中,为什么这个价值规律不实现呢?为什么工人以工资形式取得的物化劳动量不等于他为换取工资而付出的直接劳动量呢?为了排除这个困难,穆勒把雇佣工人变成了商品所有者,说他向资本家出卖自己的产品,自己的商品,因为他在产品,商品中所占的份额是他的产品,他的商品,是他以特殊商品的形式生产出来的价值。他解决困难的方法是把包含着物化劳动和直接劳动的对立的、资本家和雇佣工人之间的交易,说成物化劳动的所有者之间、商品所有者之间的普通交易。

尽管穆勒由于耍了这样一个花招而使自己不可能理解资本家和雇佣工人之间发生的过程的特殊性质、特点,但是他决没有给自己减少困难,却增加了困难,因为,对于结果的特殊性,现在已经不可能依据工人所出卖的商品(它具有这样的特性:它的使用价值本身是形成交换价值的要素,因而这个商品的消费创造出比它本身所包含的更多的交换价值)的特殊性来理解了。

在穆勒看来,工人是和任何其他商品所有者一样的商品出卖者。例如,他生产6码麻布。这6码中2码所代表的价值等于他所加入的劳动。因此,他是把2码麻布卖给资本家的卖者。既然这时工人是和其他任何麻布所有者一样的麻布卖者,那他为什么不能象其他任何出卖2码麻布的卖者一样从资本家那里全部取得2码麻布的价值呢?恰恰相反,与价值规律的矛盾这时表现得尖锐得多。工人这时出卖的决不是和其他一切商品不同的特殊商品。他出卖的是物化在产品中的劳动,也就是这样的商品:它作为商品并没有由于具有某种特点而与其他任何商品不同。这样,如果1码麻布的价格——即代表1码麻布中包含的劳动时间的货币量——是2先令,那末为什么工人所得到的是1先令,而不是2先令呢?如果工人得到2先令,资本家就不能实现剩余价值,李嘉图体系也就全部被推翻了。我们也就会被迫退回到“让渡利润”。6码麻布使资本家花费的,等于它的价值,即12先令。但是资本家按照13先令出卖它。

或者事情是这样:当资本家出卖麻布的时候,它和其他一切商品一样按照自己的价值出卖,但是当工人卖它的时候,则低于它的价值出卖,这样,价值规律就被工人和资本家之间的交易破坏了。而穆勒所以求助于虚构,恰恰是为了逃避这一点。他想把工人和资本家之间的关系变成商品的卖者和买者之间的普通关系。那末,普通的商品价值规律为什么在这里不应决定这种交易呢?但是,据说工人的报酬是“以预付的方式”支付的。可见,这里我们所看到的毕竟不是普通的商品买卖关系。这种“预付”在这里应该是什么呢?(按照穆勒的假设和按照实际情况,)一个工人,譬如说按周领工资的工人,在他从资本家那里领到对一周产品中属于他的份额的“报酬”之前,就“预付了”自己的劳动,创造了这个份额,即已把自己一周的劳动物化在产品之中。资本家“预付了”原料和劳动工具,工人“预付了”劳动,而在周末支付工资的时候,工人把商品,自己的商品,即自己在全部商品中所占的份额卖给资本家。但是,穆勒会说,资本家还在他自己把6码麻布转化为货币,卖出去之前,就付款给工人了,也就是替工人把2码[795]麻布转化为银,转化为货币了!可是,如果资本家制造的是定货,如果他还在商品生产出来以前就已经把它卖出,那又当如何呢?更广泛地说,资本家向工人买这2码麻布是为了再卖,而不是为了自己消费,这对于工人——在这里是2码麻布的卖者,有什么关系呢?买者的动机同卖者有什么相干呢?买者的动机又怎么可能进而使价值规律也发生变化呢?如果前后一贯的话,那就必须承认,每个卖者都应当低于商品的价值出卖他的商品,因为他交给买者的是使用价值形式的产品,而买者交给他的是货币形式的价值,转化为银的产品形式。在这种情况下,麻织厂主也应当少付给麻纱商人、机器厂主、煤炭生产者等等。因为他们卖给他的商品,是他不过准备使之转化为货币的商品,而他不仅要在自己的商品卖出之前,而且要在自己的商品生产出来之前就把这个商品的组成部分的价值“以预付的方式”付给他们。工人供给他的是麻布,是制成的、适于出卖的形式的商品;相反,上述那些商品卖者供给他的是机器、原料等等,这些东西还必须经过一定的过程才能具有适于出卖的形式。对于穆勒这个极端的李嘉图主义者(在他看来,买和卖、供和求不过是等同的,货币则不过是一种形式)来说,妙不可言的是:商品转化为货币(在把2码麻布卖给资本家时,发生的无非就是这种转化)要以卖者不得不低于商品的价值出卖他的商品,而买者用自己的货币买到比货币价值大的价值为前提。

由此可见,穆勒把事情归结为这样一个荒谬的论断:在这笔交易中,买主购买是为了再卖并赚取利润,因此卖者不得不低于商品的价值出卖他的商品;这样,整个价值理论就被推翻了。穆勒为解决李嘉图的矛盾之一而作的这第二个尝试,实际上毁掉了李嘉图体系的整个基础,特别是毁掉了这个体系的优点:把资本和雇佣劳动的关系看作积累劳动和直接劳动之间的直接交换,即从这种关系的特定性质去考察它。

为了设法摆脱困难,穆勒必须再进一步说,这里所谈的不是单纯的商品买卖的交易;工人和资本家的关系不如说是出借货币或从事贴现业务的资本家(货币资本家)对产业资本家的关系,因为这里涉及的是支付,是把等于工人在总产品中所占份额的产品转化为货币。以生息资本(资本的特殊形式)的存在为前提来解释生产利润的资本(资本的一般形式);把剩余价值的派生形式(它已经以资本为前提)说成剩余价值产生的原因,——这倒是一种满不错的解释。此外,穆勒在这种情况下必须保持前后一贯,撇开李嘉图所发展的关于工资和工资水平的一切已确定的规律,而相反地从利息率中引伸出工资;在这种情况下,实际上却又不能说明利息率应该由什么决定,因为按照李嘉图学派和其他所有值得一提的经济学家的意见,利息率是由利润率决定的。

所谓工人在他自己产品中所占的“份额”这种空话,实质上是以下述情况为根据的:如果考察的不是资本家和工人之间的单独交易,而是他们在总的再生产过程中的交换,如果注意的是这个过程的实际内容,而不是它的表现形式,那末在实际上就会看到,资本家用来支付工人的东西(以及作为不变资本同工人对立的那部分资本),不外是工人本身的产品的一部分,并且不是尚须转化为货币的那部分产品,而是已经卖出、已经转化为货币的那部分产品,因为工资是以货币而不是以实物支付的。在奴隶制度等等的条件下,不存在由于花在工资上的那部分产品先要转化为货币而产生的假象,因此看得很清楚,奴隶作为工作报酬取得的东西,实际上不是奴隶主的“预付”,而只是奴隶的物化劳动中以生活资料的形式流回到奴隶手中的部分。在资本家那里情况也是如此。他只是在表面上“预付”。他作为工资预付给工人的,或者更确切地说,他付给工人的[796]报酬——因为他要到工作完成后才付给报酬——是工人已经生产出来而且已经转化为货币的产品的一部分。资本家所占有的、从工人那里夺来的工人产品,有一部分以工资形式,作为对新产品的预付——如果愿意用这个名称的话——流回到工人手里。

抓住资本家和工人之间的交易所特有的这种表面现象来解释交易本身,对于穆勒来说是完全不相称的(这种作法对于麦克库洛赫、萨伊或巴师夏来说倒是相称的)。资本家除了他以前从工人那里夺来的东西,也就是由其他人的劳动预付给他的东西之外,没有任何东西可以用来预付给工人。要知道,甚至马尔萨斯也说,资本家所预付的东西,不是“由呢绒”和“其他商品”,而是“由劳动”构成的\endnote{托·罗·马尔萨斯《价值尺度。说明和例证》1823年伦敦版第17—18页。——第97页。},也就是恰恰由资本家没有从事过的东西构成的。资本家预付给工人的是工人自己的劳动。

但是,所有这种代用语都丝毫不能帮穆勒的忙,就是说,丝毫不能帮助他回避解决这个问题:积累劳动和直接劳动之间的交换(李嘉图以及追随他的穆勒等人就是这样理解资本和劳动之间的交换过程的)如何同直接与它矛盾的价值规律相符合。从穆勒的以下论点中可以看出,上述代用语是丝毫帮不了他的忙的:

\begin{quote}{“产品按什么比例在工人和资本家之间进行分配,或者,工资水平按什么比例调节?”(穆勒《政治经济学原理》,帕里佐的法译本,第34页)“确定工人和资本家的份额,是他们之间的商业交易的对象,讨价还价的对象。一切自由的商业交易都由竞争来调节,讨价还价的条件随着供求关系的变化而变化。”(同上,第34—35页)}\end{quote}

付给工人的是工人在产品中所占的“份额”。穆勒这样说,是为了使工人在他同资本的相互关系中变成一个普通的商品(产品)卖者,是为了掩盖这种相互关系的特殊性质。工人在产品中所占的份额,可以认为是他的产品,即工人的新加劳动物化于其中的那部分产品。但是情况并不是这样。相反,现在我们问工人在产品中所占的“份额”是怎样的,即他的产品是怎样的(因为属于工人的那一部分产品,就是他所出卖的他的产品),这时,我们就听到说,他的产品和他的产品是两种完全不同的东西。我们还应该先搞清楚工人的产品(即他在产品中所占的份额,因而也就是属于他的那一部分产品)是什么。可见,所谓“他的产品”只是一句空话,因为工人从资本家那里得到的那一份价值,并不由他本身的产量决定。所以,穆勒只是把困难推远了一步。在解决困难方面,他仍在开始研究时的出发地点踏步不前。

这里表现了概念的混淆。如果认为资本和雇佣劳动之间的交换是不断的行为——凡是不把资本主义生产的个别行为、个别因素固定化、孤立化的人都会认为它是这样的行为,那末,工人所取得的就是他自己的产品的一部分价值,他已经补偿了这部分价值,还加上了他白白送给资本家的那部分价值。这是不断反复进行的。可见,实际上工人不断取得他自己的产品的价值的一部分,他所创造的价值的一个部分或份额。他的工资多少,不决定于他在产品中所占的份额,倒是他在产品中所占的份额决定于他的工资量。工人实际上取得产品价值中的一个份额。但是,他所取得的那个份额决定于劳动的价值,而不是劳动的价值反过来决定于他在产品中所占的份额。劳动的价值,即工人本身的再生产所需要的劳动时间,是一个已固定的量;这个量是由于工人的劳动能力出卖给资本家而固定下来的。实际上,工人在产品中所占的份额也是由此固定下来的。而不是相反,不是先把他在产品中所占的份额固定下来,然后由这个份额决定他的工资的水平或价值。其实,这也正是李嘉图的最重要的、最强调的论点之一,因为,不然的话,劳动的价格就会决定劳动所生产的商品的价格,而按照李嘉图的见解,劳动的价格只决定利润率。

那末,穆勒现在如何确定工人所取得的产品“份额”呢?他用供求,用工人和资本家之间的竞争来决定它。穆勒在这里提出的说法,可以适用于一切商品:

\begin{quote}{“确定工人和资本家〈卖者和买者〉的份额〈应读作:在商品价值中的份额〉,是他们[797]之间的商业交易的对象,讨价还价的对象。一切自由的商业交易都由竞争来调节,讨价还价的条件随着供求关系的变化而变化。”(同上,第34—35页)}\end{quote}

可见,问题就在这里!这就是穆勒说的话,他这个热诚的李嘉图主义者曾经证明:需求和供给固然能够决定市场价格在商品价值上下的波动,但是不能决定商品价值本身;需求和供给如果用来决定价值,就成了两个没有意义的字眼,因为它们本身的决定要以价值的决定为前提!而现在——萨伊早已在这一点上指责过李嘉图\endnote{马克思在《剩余价值理论》第二册(见本卷第2册第454和455页)提到萨伊的“幸灾乐祸”,说这是因为李嘉图在用维持工人生活所必需的生存资料决定“劳动价值”时,引证了供求规律。这里马克思引用的李嘉图著作是康斯坦西奥译、萨伊加注的法译本。马克思在这里是不确切的。萨伊在给李嘉图著作所加的注释中“幸灾乐祸”,是因为李嘉图用供给和需求来决定货币的价值。马克思在《哲学的贫困》(见《马克思恩格斯全集》中文版第4卷第126页)中曾引了萨伊注释中有关的这段话。这段话的出处是:大·李嘉图《政治经济学和赋税原理》,康斯坦西奥译自英文,附让·巴·萨伊的注释和评述,1835年巴黎版第二卷第206—207页。——第100页。}——穆勒为了决定劳动的价值,即一种商品的价值,竟求助于用需求和供给来决定它!

但是,问题还不止于此。

穆勒没有说——其实在这种情况下这是无关紧要的——双方当中哪一方代表供给,哪一方代表需求。但是,既然资本家付出货币,工人相反地提供某种东西来交换货币,我们就可以假定需求是在资本家方面,供给在工人方面。但是,这时工人“出卖”的是什么呢?他提供的是什么呢?是他在还不存在的产品中所占的“份额”吗?但是要知道,他在未来的产品中所占的“份额”恰恰还要由他和资本家之间的竞争,由“需求和供给”的关系来决定!这个关系的一个方面,即供给,不可能由本身不过是供求斗争结果的东西构成。那末,工人到底拿出什么来卖呢?自己的劳动吗?但是这样一来,穆勒就又遇到了他想回避的最初的困难,即积累劳动和直接劳动之间的交换。当他说这里发生的不是等价物的交换,或者说所卖的商品即劳动的价值不是用“劳动时间”本身来衡量,而是由竞争,由供求来决定的时候,他也就承认,李嘉图的理论遭到破产,而李嘉图的反对者是对的,后者认为商品价值决定于劳动时间的主张是错误的,因为最重要的一种商品即劳动本身的价值同商品价值的这个规律相矛盾。我们将在下文中看到,威克菲尔德就直接说过这样的话。\authornote{见本册第205页,并见本卷第2册第453—454页。——编者注}

穆勒愿意怎样打转转和兜圈子都可以,但是他找不到摆脱这个左右为难的窘境的出路。用他本人的表达方法来说,工人的竞争最多只能使他们按照这样的价格提供一定的劳动量,这个价格依据供求关系等于他们将要用这个劳动量生产出来的产品的一个较大的或较小的部分。但是,他们用自己的劳动换取的这个价格,这个货币量等于应当生产出来的产品的价值的一个较大的或较小的部分这一点,无论如何从一开始就不妨碍一定的活劳动(直接劳动)量在这里换得一个较大的或较小的货币(积累劳动,而且是以交换价值形式存在的劳动)量。从而,这也不妨碍不等量的劳动在这里相互交换,不妨碍较少的积累劳动换取较多的直接劳动。而这些恰恰是穆勒应该加以解释的现象,也是他为了避免违反价值规律而想用自己的解释搪塞过去的现象。这种现象丝毫也不会由于以下情况而改变它的性质并得到解释:工人用自己的直接劳动换得货币的那个比例,在生产过程结束后表现为付给工人的价值和工人所创造的产品的价值之间的比例。资本和劳动之间最初的不平等交换,在这里只是以另一种形式表现出来罢了。

穆勒进一步阐述的一段话,也可以表明他如何固执地回避劳动和资本之间的直接交换,而李嘉图还是毫无拘束地以此为出发点的。穆勒是这样说的:

\begin{quote}{[798]“假定有一定数目的资本家和一定数目的工人。假定他们分配产品的比例也通过某种方法确定了。如果工人人数增长了而资本量没有增加,增加的那一部分工人就会试图排挤原来在业的那一部分。他们只有按较低报酬提供自己的劳动,才能作到这一点。在这种情况下工资水平必然降低……在相反的情况下结果也相反……如果资本量同人口的比例不变,工资水平也就保持不变。”(同上,第35页及以下各页)}\end{quote}

穆勒应当确定的,恰恰也就是“他们〈资本家和工人〉分配产品的比例”。为了让竞争决定这个比例,穆勒就假定,这个比例已经“通过某种方法确定了”。为了让竞争决定工人的“份额”,他就假定,这个份额在竞争之前就已经“通过某种方法”确定了。这还不够。为了表明竞争如何改变已经“通过某种方法”确定了的产品分配,他还假定工人在他们的人数比资本量增加得快的时候,就“按较低报酬提供自己的劳动”。可见,穆勒在这里直接说出了工人的供给是由“劳动”构成的,工人提供这种劳动以换取“报酬”,即换取货币,换取一定量的“积累劳动”。为了避开劳动和资本之间的直接交换,避开直接出卖劳动,他求助于“产品分配”论。为了解释产品分配的比例,他又假定直接出卖劳动以换取货币,以致于资本和劳动之间的这种最初的交换后来就表现在工人在他的产品中所占的份额上,而不是这种最初的交换决定于工人在产品中所占的份额。最后,当工人人数和资本量不变时,“工资水平”也保持不变。但是,当需求和供给彼此适应时,这种工资水平是怎样的呢?这也正是应当说明的。说工资水平在这种供求的平衡遭到破坏时就会变动,在这里并不说明任何问题。穆勒的同义反复的说法只能证明,他在这里感到李嘉图的理论中有一种障碍,要克服这种障碍,只有根本脱离这个理论。

\centerbox{※     ※     ※}

反对马尔萨斯、托伦斯及其他人。穆勒反对商品价值决定于资本价值,他正确地指出:

\begin{quote}{“资本就是商品,说商品的价值由资本的价值决定,就等于说,商品的价值由商品的价值决定,商品的价值由它本身决定。”(《政治经济学原理》1821年伦敦第1版第74页)}\end{quote}

\centerbox{※     ※     ※}

{穆勒并不掩盖资本同劳动的对立。他说,为了使不依靠直接劳动的社会阶级壮大,利润率必须高;为此,工资也就应该相对地低。为了使人类的(社会的)能力能在那些把工人阶级只当作基础的阶级中自由地发展,工人群众就必须是自己的需要的奴隶,而不是自己的时代的主人。工人阶级必须代表不发展,好让其他阶级能够代表人类的发展。这实际上就是资产阶级[799]社会以及过去的一切社会所赖以发展的对立,是被宣扬为必然规律的对立,也就是被宣扬为绝对合理的现状。

\begin{quote}{“人类进步,即不断地从科学和幸福的一个阶段过渡到另一个更高的阶段的能力,看来在很大程度上取决于这样的人所组成的阶级:他们是自己的时代的主人,也就是说,他们相当富有,根本不必为取得过比较安乐的生活的资财而操心。科学的领域就是由这个阶级的人来培植和扩大的;他们传播光明;他们的子女受良好的教育,被培养出来去从事最重要、最高雅的社会职务;他们成为立法者、法官、行政官员、教师、各种技艺的发明家、人类赖以扩大对自然力的控制的一切巨大和有益的工程的领导者。”(《政治经济学原理》,帕里佐的法译本,1823年巴黎版第65页)“资本的收入应当大到足够使社会上很大一部分人能够享受余暇所提供的好处。”(同上,第67页)}}\end{quote}

\centerbox{※     ※     ※}

对以上所谈的再作一些补充。

在穆勒这个李嘉图主义者看来,劳动和资本之间的区别只是劳动的不同形式之间的区别:

\begin{quote}{“劳动和资本:前者是直接劳动,后者是积累劳动。”(《政治经济学原理》1821年伦敦英文第1版第75页)}\end{quote}

在另一个地方他说:

\begin{quote}{“关于这两种劳动应当指出,它们并不是始终按照同样比率取得报酬的。”(《政治经济学原理》,帕里佐的法译本,1823年巴黎版第100页)}\end{quote}

可见,他在这里接近了问题的实质。既然用来支付直接劳动的报酬的始终是积累劳动即资本,那末不按同样比率支付报酬,在这里只能意味着较多的直接劳动同较少的积累劳动交换,而且“始终”如此,因为,不然的话,积累劳动就不能作为“资本”同直接劳动交换,它不仅不能象穆勒所期望的那样提供足够大的收入,而且根本不会提供收入。因此,这里就承认了,——因为穆勒正象李嘉图一样,把资本同劳动的交换看作积累劳动同直接劳动的直接交换,——这两种劳动是按照不相等的比例进行交换的,而在这样交换的情况下,以等量劳动相互交换为内容的价值规律便遭到了破坏。

\tsectionnonum{[(C)穆勒不理解工业利润的调节作用]}

穆勒把李嘉图实际上为阐明自己的地租理论而假定的东西\authornote{见本卷第2册第532—533页。——编者注},作为一条基本规律提出来。

\begin{quote}{“农业的利润率调节其他利润率。”(《政治经济学原理》1824年伦敦第2版第78页)}\end{quote}

这是根本错误的,因为资本主义生产是在工业中,而不是在农业中开始的,而且是逐渐支配农业的;这样,农业利润只是随着资本主义生产的发展才和工业利润平均化,而且只是由于这种平均化才开始影响工业利润。所以,首先,上述论点从历史上看是错误的。其次,只要存在这种平均化,就是说,只要存在这样的农业发展状况,即资本随着利润率的高低从工业转入农业或从农业转入工业,那末,说农业利润从这时开始起调节作用,而不说这里是两种利润相互发生作用,同样是错误的。其实,李嘉图本人为了说明地租,是以相反的情况为前提的。谷物价格上涨,结果利润下降,但不是在农业中下降(在较坏土地或第二笔生产率较低的资本提供新的供给之前),——因为谷物价格的上涨绰绰有余地给租地农场主补偿谷物价格上涨所引起的工资的提高,——而是在工业中下降,因为这里不发生这种补偿或超额补偿。这样,工业利润率下降了,只得到这种较低利润率的资本就可以用于较坏土地。如果利润率不变,情况就不会这样。而且,只是由于工业利润下降对较坏土地上的农业利润的这种反作用,整个农业利润才下降,[800]较好土地上的一部分农业利润才以地租形式从利润中分出来。这就是李嘉图对这个过程的说明。可见,按照他的说明,是工业利润调节农业利润。

如果现在由于农业的改良,农业利润又提高了,那末工业利润也会提高。但是这决不能排除下述情况:正象最初工业利润的减少决定农业利润的减少一样,工业利润的提高也会引起农业利润的提高。每当工业利润的提高同谷物价格以及加入工人工资的其他农产必需品的价格无关的时候,也就是说,每当工业利润的提高是由于构成不变资本的商品等等价值降低的时候,情况就是这样。如果工业利润不调节农业利润,地租反而绝对不能得到解释。平均利润率在工业中是由于资本利润的平均化以及价值因此转化为费用价格而形成的。这种费用价格——预付资本的价值加平均利润——是农业从工业获得的前提,因为农业中由于土地所有权的存在,上述平均化是不可能发生的。如果农产品的价值因而高于由工业的平均利润决定的费用价格,那末这个价值超过费用价格的余额就形成绝对地租。但是,为了能对价值超过费用价格的这种余额进行衡量,费用价格应当是第一性的,也就是说,它应当作为规律由工业强加给农业。

\centerbox{※     ※     ※}

穆勒的下述论点值得注意:

\begin{quote}{“生产中消费的东西总是资本。这就是生产消费的一个特别值得注意的属性。生产中所消费的东西就是资本,并且通过消费才成为资本。”(《政治经济学原理》,帕里佐的法译本,1823年巴黎版第241—242页)}\end{quote}

\tsectionnonum{[(d)]需求,供给,生产过剩[直接把需求和供给等同起来的形而上学观点]}

\begin{quote}{“需求意味着购买愿望和购买手段……一个人所提供的等价物品〈购买手段〉就是需求的工具。他的需求量就是用这个等价物品的价值来衡量的。需求和等价物品是两个可以相互代替的用语……他的〈一个人的〉购买愿望和购买手段,换句话说,他的需求,正好等于他生产出来但不准备自己消费的东西的数量。”(同上,第252——253页)}\end{quote}

我们在这里看到,需求和供给的直接等同(从而市场商品普遍充斥的不可能性)是怎样被证明的。需求据说就是产品,而且这种需求的量是用这种产品的价值来衡量的。穆勒就是用这同样的抽象“证明方法”证明买和卖只是等同,而不是彼此相区别;他就是用这同样的同义反复证明价格取决于流通的货币量;他也就是用这同样的手法证明供给和需求(它们只是买者和卖者的关系的进一步发展的形式)必然是彼此平衡的。这还是同样的一套逻辑。如果某种关系包含着对立,那它就不仅是对立,而且是对立的统一。因此,它就是没有对立的统一。这就是穆勒用来消除“矛盾”的逻辑。

我们首先拿供给作出发点。我供给的是商品,是使用价值和交换价值的统一体,譬如说,一定量的铁,它等于3镑(这笔钱又等于一定量的劳动时间)。根据假设,我是一个制铁厂主。我供给一定的使用价值——铁,也供给一定的价值,即表现为铁的价格3镑的价值。但是,这里有下面一种小小的差别。一定量的铁确实是由我投入市场的。相反,铁的价值却只是作为铁的价格存在,这个价格还要由铁的买者去实现,买者对我来说代表对铁的需求。铁的卖者的需求,则是对铁的交换价值的需求,这种交换价值固然包含在铁里,但是还没有实现。同样大的交换价值可以表现在数量极不相同的铁上。由此可见,使用价值的供给和有待实现的价值的供给决不是等同的,因为数量完全不同的使用价值可以表现同量的交换价值。

[801]同样是3镑的价值,可能表现在1吨、3吨或10吨上。可见,我供给的铁(使用价值)的量和我供给的价值的量决不是互成比例的,因为无论前者怎样变化,后者的量可能始终不变。无论我供给的铁的量是大还是小,根据我们的假设,我始终要实现的是不以铁本身的量为转移,一般说来不以铁作为使用价值的存在为转移的价值。由此可见,我所供给的(但是还没有实现的)价值和我所供给的、已经实现的铁的量,决不是互成比例的。因此,没有丝毫理由认为一种商品按照自己的价值出卖的能力和我所供给的商品量是成比例的。对买者来说,我的商品首先是作为使用价值而存在的。买者把它作为使用价值来购买。但是他所需要的是一定量的铁。他对铁的需要并不决定于我所生产的铁的量,正象我的铁的价值本身不和这个量成比例一样。

当然,购买的人手中持有商品的转化形式,货币,即具有交换价值形式的商品,而且他之所以能作为买者出现,只是由于他或其他人曾经作为现在以货币形式存在的那个商品的卖者出现。但是,这决不能作为理由来说明他把自己的货币再转化为我的商品,或者说明他对我的商品的需要决定于我生产的商品量。就他对我的商品提出需求来说,他需要的量或者可能比我供给的量少,或者可能完全一样,但是要低于商品的价值。正象我供给的某种使用价值的量和我供给它时所依照的价值不是等同的一样,他的需求也可能和我的供给不相适应。

但是,全部关于供求的研究都不是这里所要涉及的。

既然我供给铁,我需求的就不是铁,而是货币。我供给的是某种特殊的使用价值,需求的是它的价值。因此,我的供给和需求,正象使用价值和交换价值一样,是不同的。既然我在铁上供给了某一价值,我需求的就是实现这一价值。可见,我的供给和需求就象观念和现实一样,是不同的。其次,我供给的量和它的价值绝不是互成比例的。而且,对我供给的某种使用价值的量的需求,不是决定于我想要实现的价值,而是决定于买者按照一定价格需要买到的量。

我们再引证穆勒如下的论点:

\begin{quote}{“显然,每个人加在产品总供给量上的,是他生产出来但不准备自己消费的一切东西的总量。无论年产品的一定部分以什么形式落到这个人的手里,只要他决定自己一点也不消费,他就希望把这一部分产品完全脱手;因此,这一部分产品就全部用于增加供给。如果他自己消费这个产品量的一部分,他就希望把余额全部脱手,这一余额就全部加在供给上。”(同上,第253页)}\end{quote}

换句话说,这无非是指,全部投入市场的商品构成供给。

\begin{quote}{“可见,因为每个人的需求等于他希望脱手的那一部分年产品,或者换一种说法,等于他希望脱手的那一部分财富}\end{quote}

{且慢!他的需求等于他希望脱手的那一部分产品的价值(在这个价值一旦实现时);他希望脱手的东西是一定量的使用价值,他希望取得的东西是这个使用价值的价值。这两种东西决不是等同的},

\begin{quote}{并且因为每个人的供给也完全与此相同}\end{quote}

{决不是这样,他的需求不是他希望脱手的东西,即不是产品,而是这种产品的价值;相反,他的供给现实地是这种产品,而这种产品的价值则只是在观念上被供给},

\begin{quote}{所以,每个人的供给和需求必然是相等的。”(第253—254页)}\end{quote}

{这就是说,他所供给的商品的价值和他以这个商品去要求、但并不拥有的那个价值是相等的。如果他按照商品的价值出卖商品,那末他供给的价值(以商品形式)和取得的价值(以货币形式)就是相等的。但是,不能由于他希望按照商品的价值出卖商品,就得出结论说实际发生的情况就是这样。一定量的商品由他供给,并出现在市场上。他想要得到他所供给的商品的价值。}

\begin{quote}{“供给和需求处于[802]一种特殊的相互关系之中。每一个被供给的、被运往市场的、被出卖的商品,始终同时又是需求的对象,而成为需求对象的商品,始终同时又是产品总供给量的一部分。每一个商品都始终同时是需求的对象和供给的对象。当两个人进行交换时,其中一个人不是为了仅仅创造供给而来,另一个人也不是为了仅仅创造需求而来;他的供给对象,供给品,必定给他带来他需求的对象,因此,他的需求和他的供给是完全相等的。但是如果每一个人的供给和需求始终相等,那末,一个国家的全体人员的供给和需求,总起来说,也是这样。因此,无论年产品总额如何巨大,它永远不会超过年需求总额。有多少人分配年产品,年产品总量就分成多少部分。需求的总量,等于所有这些部分的产品中所有者不留归自己消费的东西的总额。但是,所有这些部分的总量,恰恰等于全部年产品。”(同上,第254—255页)}\end{quote}

既然穆勒已经假定每个人的供给和需求相等,那末,说全体人员的供给和需求因而也彼此相等的全部冗长的高明议论,便是完全多余的了。

\centerbox{※     ※     ※}

和穆勒同时代的李嘉图学派是如何理解穆勒的,例如,从下面的引文就可以看出:

\begin{quote}{“可见,在这里{这是指穆勒关于劳动的价值规定}我们看到至少有这样一种情况:价格(劳动的价格)经常决定于供求关系。”(普雷沃《评李嘉图体系》,载于麦克库洛赫《论政治经济学》,普雷沃译自英文,1825年日内瓦—巴黎版第187页)}\end{quote}

麦克库洛赫在上述《论政治经济学》一书中说,穆勒的目的在于——

\begin{quote}{“对政治经济学原理进行逻辑演绎。”(第88页)“穆勒几乎阐述了所有成为讨论对象的问题。他善于解开和简化最复杂最困难的问题,并且把各种不同的科学原理纳入自然秩序。”(同上)}\end{quote}

从穆勒的逻辑中可以得出这样的结论:他把我们在前面\authornote{见本卷第2册,特别是第180—187页。——编者注}分析的李嘉图著作的十分不合逻辑的结构幼稚地当作“自然秩序”原封不动地保存下来了。

\tsectionnonum{[(e)]普雷沃[放弃李嘉图和詹姆斯·穆勒的某些结论。试图证明利润的不断减少不是不可避免的]}

至于上面提到的普雷沃(他把穆勒对李嘉图体系的说明作为他的《评李嘉图体系》一文的依据),他的某些反对意见是纯粹幼稚无知地误解李嘉图的结果。

但是,下面一段谈到地租的话值得注意:

\begin{quote}{“如果象应该做的那样,注意到较坏土地的相对数量,就会对这种土地在决定价格上所发生的影响提出怀疑。”(普雷沃,同上第177页)}\end{quote}

普雷沃引用了穆勒以下一段话,这段话对于我所作的分析也很重要,因为穆勒在这里为自己设想了一种情况,在这种情况下,级差地租之所以产生,是因为新的需求——追加的需求——通过向较好土地而不是向较坏土地推移,即按上升序列运动而得到满足:

\begin{quote}{“穆勒先生用了这样一个例子:‘假设某一国家的全部耕地质量相同,并且对投入土地的各资本提供同样多的产品,只有一英亩例外,它提供的产品六倍于其他任何土地。’(穆勒《政治经济学原理》第2版第71页)毫无疑问,正象穆勒先生所证明的那样,租用这一英亩土地的租地农场主并不能提高他的租地收入〈即不能比其他租地农场主取得更高的利润;普雷沃把这个思想表达得很拙劣\endnote{在普雷沃翻译的麦克库洛赫那本书所附的正误表上,这句话被改为:“租佃这最后一英亩土地的农场主不能逃避交纳相应的地租”。——第111页。}〉,六分之五的产品会属于土地所有者。”}\end{quote}

{可见,这里我们看到的是在利润率不降低和农产品价格不提高的条件下的级差地租。这种情况一定会更加常见,因为[803]不论自然肥力如何,土地的位置随着一国工业的发展、交通工具的发展和人口的增长必然不断改善,而位置(比较好的位置)是和自然肥力同样发生作用的。}

\begin{quote}{“但是,如果这位机灵的作者想到对相反的情况作同样的假设,他就会相信结果是不同的。我们假设全部土地具有相同的质量,只有一英亩较坏土地除外。在这唯一的一英亩土地上,所花费的资本的利润只等于其他任何一英亩土地的利润的六分之一。能不能设想,千百万英亩土地的利润会因此而降低到普通利润的六分之一呢?这唯一的一英亩土地想必根本不会对价格发生丝毫作用,因为,任何进入市场的产品(特别是谷物)不会由于如此微不足道的数量的竞争而受到显著的影响。因此我们说,对李嘉图拥护者关于较坏土地的影响的主张,应该考虑到不同肥力的土地的相对数量而加以修正。”(普雷沃,同上第177—178页)}\end{quote}

\centerbox{※     ※     ※}

{萨伊为康斯坦西奥翻译的李嘉图著作所加的注释,只有一个关于对外贸易的正确意见。\endnote{马克思指的是萨伊给李嘉图《政治经济学原理》第七章《论对外贸易》所加的注释。萨伊在这个注释中举了一个例子:法国从安的列斯群岛进口的糖在法国的价格,比法国本国生产的糖便宜。——第112页。}通过欺骗行为,由于一个人得到了另一个人失掉的东西,也可能获得利润。在一个国家内,亏损和盈利是平衡的。在不同国家的相互关系中,情况就不是这样。即使从李嘉图理论的角度来看,——这一点是萨伊没有注意到的,——一个国家的三个工作日也可能同另一个国家的一个工作日交换。价值规律在这里有了重大的变化。或者说,不同国家的工作日相互间的比例,可能象一个国家内熟练的、复杂的劳动同不熟练的、简单的劳动的比例一样。在这种情况下,比较富有的国家剥削比较贫穷的国家,甚至当后者象约·斯·穆勒在《略论政治经济学的某些有待解决的问题》一书中所指出的那样从交换中得到好处\endnote{约·斯·穆勒在他的《略论政治经济学的某些有待解决的问题》(1844年伦敦版)第一篇中考察了“各国相互交换的规律以及商业世界各国商业利益的分配”,并且指出:“我们通过对外国人的贸易取得他们的商品,而花费的劳动和资本,往往少于他们自己为这些商品所花费的。然而,这种贸易对外国人还是有利的,因为他们从我们这里换得的商品,如果他们自己去生产,就要花费较高的代价,尽管我们为它花费的代价较少。”(第2—3页)——第112页。}的时候,情况也是这样。}

\centerbox{※     ※     ※}

[关于农业利润和工业利润的相互关系问题,普雷沃说道:]

\begin{quote}{“我们承认,总的说来,农业利润率决定工业利润率。但是,我们同时应该注意到,后者必然也对前者发生反作用。如果谷物的价格达到一定的高度,工业资本就会流入农业,不可避免地使农业利润降低。”(普雷沃,同上第179页)}\end{quote}

反驳是正确的,但是提得十分狭隘。参看前面所说的\authornote{参看本册第104—105页。——编者注}。

李嘉图学派认为,只有工资增长,利润才会下降,因为随着人口的增长,生活必需品的价格提高,而这种情况又是资本积累的结果,因为随着资本的积累,较坏土地逐渐投入耕种。但是李嘉图本人承认,当资本增加得比人口快的时候,也就是当资本相互竞争使工资提高的时候,利润也会下降。这是亚·斯密的观点。普雷沃说:

\begin{quote}{“如果资本的需求的增长使工人的价格即工资提高,那末,认为这些资本的供给的增长会使资本的价格即利润降低,难道是不对的吗?”(同上,第188页)}\end{quote}

按照李嘉图的观点,利润降低只能由于剩余价值减少,也就是由于剩余劳动减少,也就是由于工人消费的生活必需品的价格上涨,也就是由于劳动价值提高,尽管工人得到的实际报酬这时不但不会提高,反而可能降低;普雷沃就以这种错误的观点为依据,试图证明利润的不断降低并不是不可避免的。

第一,他说:

\begin{quote}{“繁荣状态首先使利润提高}\end{quote}

(这里指的正是农业利润:随着繁荣状态的到来,人口增加,从而对农产品的需求也增加,从而租地农场主的超额利润也增加),

\begin{quote}{而且这是在新地投入耕种以前很久的时候,所以,当这种新地开始影响地租,使利润降低的时候,利润尽管马上降低下来,但依然和繁荣以前一样高……为什么在某个时候会转而耕种质量较坏的土地呢?这样做只是指望至少能获得等于普通利润的利润。那末,什么情况能使质量较坏的土地创造这种利润率呢?人口的[804]增长……增加的人口形成对现有的生存资料的压力,因而使食品(特别是谷物)价格上涨,结果是农业资本获得高额利润。其他资本流入农业:但是因为土地面积有限,所以这种竞争也有限度,最终结果是耕种较坏土地仍然获得比商业或工业更高的利润。从这时起(在这种较坏土地有足够数量的前提下)农业利润便不能不决定于投入土地的最后一批资本的利润。如果以财富开始增长时〈利润开始分为利润和地租时〉存在的利润率作为出发点,就会发现利润决没有降低的趋势。利润会和人口一起增长,直至农业利润增长到这样的程度,以致利润(由于耕种新的、较坏的土地)会显著下降,但是决不会降到低于它原来的比率,或者(更确切地说)降到低于各种不同的情况所决定的平均比率。”(同上,第190—192页)}\end{quote}

普雷沃显然错误地理解了李嘉图的观点。在普雷沃看来,由于繁荣,人口增加,这又使农产品的价格提高,从而使农业利润提高(尽管令人不解的是:如果农业利润的这种提高是经常性的,地租为什么在租佃期满后不会提高,这种农业上的超额利润为什么不会甚至在较坏土地投入耕种之前就以地租形式被人占有)。但是,促使农业利润提高的农产品价格的上涨,会使一切工业部门的工资提高,因而引起工业利润的下降。这样,工业中会形成一个新的利润率。即使质量较坏的土地在现行农产品市场价格下只提供这个降低了的利润率,资本也会投入质量较坏的土地。把资本吸引到这里来的,是高的农业利润和高的谷物市场价格。只要还没有足够多的资本转入农业,这些被吸引来的资本甚至还能象普雷沃所说的那样提供比已经降低的工业利润更高的利润。但是,一旦追加供给够了,市场价格就下降,因而较坏土地只能提供普通的工业利润。较好土地的产品所提供的超过这种利润的东西,就转化为地租。这就是李嘉图的观点,普雷沃接受了这个观点的基础,并且以这个观点作为自己立论的出发点。现在,谷物比农业利润提高之前贵。但是,它给租地农场主提供的超额利润则转化为地租。这样,较好土地的利润也降到因农产品涨价而下降的、较低的工业利润率水平。令人不解的是,如果没有某些其他情况出现并引起变化,利润为什么就不会因此降到低于它“原来的比率”。当然,其他情况是可能出现的。根据假定,不管怎样,农业利润在生活必需品涨价之后都要提高到工业利润以上。但是,如果这时工人消费的生活必需品中由工业生产的那一部分由于生产力的发展而降价,以致工人的工资(即使它按照它的平均价值支付)因农产品涨价而提高的程度,没有达到不存在这种起抵销作用的情况时所应达到的高度;其次,如果同样由于生产力的发展,采掘工业所提供的产品的价格以及不加入食物的农产原料的价格也降低了(固然,这种假定未必是现实的),那末,工业利润就可能不下降,尽管它还会低于农业利润。这时,因资本转移到农业中以及因形成地租而引起的农业利润的下降,[805]只会使利润率恢复原来的水平。

[第二,]普雷沃还试图用另一个论据:

\begin{quote}{“质量较坏的土地……只有当它提供的利润同工业资本的利润一样高或者更高时,才会被投入耕种。在这种情况下,尽管新地投入耕种,谷物和其他农产品的价格往往仍旧很高。这种高价格使工人人口陷于穷困,因为工资的提高赶不上雇佣工人消费的物品的价格的提高。农产品的高价格会成为全体居民或多或少的负担,因为工资的提高和生活必需品价格的提高几乎影响一切商品。这种普遍的穷困加上人口过多造成的死亡率的提高,引起雇佣工人人数的减少,并因而造成工资的提高和农业利润的下降。从这时起,进一步的发展方向便同以前相反。资本从较坏土地抽出,又流回工业。但是,人口规律很快又发生作用:一旦贫困消失,工人人数就增加,他们的工资就下降,因而利润就提高。这种向两个不同方向的波动会交替发生,但是并不触动平均利润。利润可能由于其他原因或者就由于这一原因而提高或下降,可能轮流地时而朝这一方向时而朝另一方向变动,但是不能认为利润的平均提高或下降是由于新地必须投入耕种造成的。人口是一个调节器,它可以恢复自然秩序并把利润控制在一定的范围内。”(同上,第194—196页)}\end{quote}

尽管叙述的方式非常混乱,但是从“人口规律”来看,这是正确的。不过这同下面的假定不一致:农业利润不断增长,直到与人口的增长相适应的追加供给创造出来为止。既然这里发生的是农产品价格的不断提高,那末由此得出的就不是人口减少,而是利润率普遍下降,这又引起积累的减少,只有这样,才造成人口的减少。根据李嘉图—马尔萨斯的观点,人口的增长是比较慢的。而普雷沃立论的基础是这样的思路:上述过程会使工资降到它的平均水平以下;随着工资的这种下降和工人的贫困会出现谷物价格的下跌,因而利润又会提高。

但是,这条思路会把我们引向与这里的问题无关的研究,因为我们在这里的前提是:劳动的价值总是被全部支付的,就是说,工人得到他本身的再生产所必需的生活资料。

普雷沃的这些论断之所以重要,是由于它们表明了,李嘉图的观点以及李嘉图所接受的马尔萨斯的观点虽然能够解释利润率的波动,但是不能解释利润率为什么不断下降而无回升:要知道,在谷物价格达到一定高度和利润降到一定程度时,工资就会降到它的水平以下,这又会造成强制性的人口减少,因此造成谷物价格和全部生活必需品价格下降,而这又必然会引起利润的提高。

\tchapternonum{(3)论战著作}

[806]从1820年到1830年这个时期是英国政治经济学史上形而上学方面最重要的时期。当时进行了一场拥护和反对李嘉图理论的理论斗争,出版了一系列匿名的论战著作;这里引用了其中一些最重要的著作,特别是只涉及那些和我们论题有关的论点的著作。不过,同时这些论战著作的特点也是,它们事实上都只是围绕价值概念的确定和价值对资本的关系进行论战的。

\tsectionnonum{(a)《评政治经济学上若干用语的争论》}

[政治经济学上的怀疑论;把理论的争论归结为用语的争论]

《评政治经济学上若干用语的争论,特别是有关价值、供求的争论》1821年伦敦版。

这部著作不无一定的尖锐性。书名很说明特点——《用语的争论》。

它部分是反对斯密、马尔萨斯的,但是也反对李嘉图。

这部著作的基本思想是:

\begin{quote}{“……争论的产生,完全是因为不同的人用语含义不同,是因为这些争论者象故事中的骑士那样从不同方面去看盾脾。”(第59—60页)}\end{quote}

这种怀疑论总是某种理论解体的预兆,也是某种无思想、无原则的适合家庭需要的折衷主义的先驱。

关于李嘉图的价值理论,匿名作者首先谈到:

\begin{quote}{“假设当我们谈到价值或者与名义价格相对立的实际价格的时候,我们指的是劳动,那就会出现一个明显的困难;因为我们常常要谈到劳动本身的价值或价格。如果我们把作为某物的实际价格的劳动理解为生产该物的劳动,那末就产生另一个困难;因为我们常常要谈到土地的价值或价格;但是土地不是由劳动生产出来的。因此,这个规定只适用于商品。”(同上,第8页)}\end{quote}

谈到劳动,这里对李嘉图的反驳是正确的,因为李嘉图认为资本直接购买劳动,也就是说,他直接谈论劳动的价值,而实际上,被买卖的是劳动能力——它本身是一种产品——的暂时使用权。匿名作者在这里并没有解决问题,而只是强调问题没有解决罢了。

说不是劳动产品的“土地的价值或价格”,表面看来直接同价值概念相矛盾,不能直接从其中得出来,这也是完全正确的。但是这句话用来反驳李嘉图就格外没有意义了,因为匿名作者并不反对李嘉图的地租理论,而李嘉图恰恰在那里阐明了怎样在资本主义生产的基础上形成土地的名义价值,以及土地的名义价值和价值规定并不矛盾。土地的价值不过是支付资本化的地租的价格。因此,这里假定的关系比从商品及其价值的简单考察中乍一看就得出的关系要深刻复杂得多;这正象虚拟资本\endnote{马克思在这里所说的虚拟资本是指国债资本,也就是说,国家(资产阶级的或地主资产阶级的)把贷款不是作为资本支出,而用从居民那里征收的税款来支付利息。参看马克思《资本论》第3卷第29章。——第118页。}(这种资本是交易所投机的对象,而且事实上不过是对部分年税的某种权利的买卖)不能用生产资本的简单概念去说明一样。

第二个反驳——说李嘉图把价值由某种相对的东西变为某种绝对的东西——在后来出版的另一部论战著作(赛米尔·贝利著)中,成了攻击李嘉图整个体系的出发点。我们在论述贝利的著作时还将提到《评政治经济学上若干用语的争论》中与此有关的观点。

关于支付劳动的资本产生的源泉,作者在一个顺便作出的评论中作了中肯的表述,但是他是不自觉的(相反,他是想借此证明下面那些我没有加上着重号的话,即劳动的供给本身阻碍劳动下降到它的自然价格的水平的趋势)。

\begin{quote}{“已经增长的劳动供给就属已经增长的用来购买劳动的东西的供给。因此,如果我们和李嘉图先生一起,说劳动总是具有下降到他称为劳动的自然价格的水平的趋势,那末我们就应该想到:引起这种趋势的劳动供给增长本身,就是阻碍这种趋势发生作用的对抗原因之一。”(同上,第72—73页)}\end{quote}

如果不从劳动的平均价格即劳动的价值出发,理论就不可能进一步展开;这就象不从一般商品价值出发,理论也不可能展开一样。只有那样,才能理解价格波动的实际现象。

\begin{quote}{[807]“这并不是说,他〈李嘉图〉主张,如果两类不同商品中分别取出的两件商品,例如,一顶帽子和一双鞋,是由等量劳动生产的,那末,这两件商品就能互相交换。这里所说的‘商品’,应该理解为‘一类商品’,而不是单独一顶帽子,一双鞋等等。英国生产所有帽子的全部劳动,为此必须看作是分配在所有帽子上面的。在我看来,这一点从一开始以及在李嘉图学说的一般阐述中都没有表示出来。”例如,李嘉图谈到,“机器制造工人的一部分劳动”,包含在例如一双袜子上。“可是制造每一双袜子的全部劳动,——如果我们说的是个别的一双袜子,——包含机器制造工人的全部劳动,而不只是他的一部分劳动;因为,虽然一台机器织出许多双袜子,但是缺少机器的任何一部分,连一双袜子也制造不出来。”(同上,第53—54页)}\end{quote}

后一段话是以误解为基础的。全部机器进入劳动过程,但只有一部分机器进入价值形成过程。

除此之外,这个评论中也有些正确的东西。

我们现在从作为资本主义生产的基础和前提的商品——产品的这个特殊的社会形式——出发。我们考察个别的产品,分析它们作为商品所具有的,也就是给它们打上商品烙印的形式规定性。在资本主义生产以前——在以前的生产方式下——很大一部分产品不进入流通,不投入市场,不作为商品生产出来,不成为商品。另一方面,在这个时期,加入生产的很大一部分产品不是商品,不作为商品进入过程。产品转化为商品,只发生在个别场合,只涉及产品的剩余部分等等,或只涉及个别生产领域(加工工业产品)等等。产品既不是全部作为交易品进入过程,也不是全部作为交易品从过程出来。但是产品发展为商品,一定范围的商品流通,因而一定范围的货币流通,也就是说,相当发达的贸易,是资本形成和资本主义生产的前提和起点。我们就是把商品看成这样的前提,因为我们是从商品出发,并把它作为资本主义生产的最简单的元素的。但是,另一方面,商品是资本主义生产的产物、结果。表现为资本主义生产元素的东西,后来表现为资本主义生产本身的产物。只有在资本主义生产的基础上,商品才成为产品的普遍形式,而且资本主义生产愈发展,具有商品形式的产品就愈作为组成部分进入资本主义生产过程。从资本主义生产中出来的商品,与我们据以出发的、作为资本主义生产元素的商品不同。在我们面前的已经不是个别的商品,个别的产品。个别的商品,个别的产品,不仅实在地作为产品,而且作为商品,表现为总产品的一个不仅是实在的、而且是观念的部分。每个个别的商品都表现为一定部分的资本和资本所创造的剩余价值的承担者。

在总产品例如1200码棉布的价值中,包含预付资本的价值加资本家占有的剩余劳动——譬如说120镑的价值(假设预付资本是100镑,剩余劳动等于20镑)。每码棉布等于120/1200镑,即1/10镑或2先令。作为过程的结果表现出来的,不是个别商品,而是商品总量,其中总资本的价值被再生产出来并加上了剩余价值。所生产的总价值除以产品数,决定个别产品的价值,而且个别产品只有作为总价值的这种相应部分才成为商品。现在决定个别产品的价值并使个别产品成为商品的,不再是花费在个别的特殊的商品上的劳动(这种劳动在大多数情况下根本无法计算出来;它在某一个商品中可以比在另外一个商品中多),而是总劳动,总劳动的相应部分,即总价值除以产品数得出的平均数。因此,为了补偿总资本连同剩余价值,商品总量中的每一商品也都必须按其由上述方式决定的价值出卖。如果1200码只卖出800码,资本就得不到补偿,更得不到利润。但是,每一码也都低于它的价值出卖,因为它的价值不是孤立地,而是作为总产品的一定部分决定的。

\begin{quote}{[808]“如果你们把劳动叫做商品,那末它也还是不同于一般商品。后者最初为交换的目的而生产,然后拿到市场上去,和同时在市场上出售的其他商品按照适当的比例相交换。劳动只有当它被带到市场上去的那一瞬间才被创造出来,或者不如说,劳动是在它被创造出来以前被带到市场上去的。”(同上,第75—76页)}\end{quote}

实际上,被带到市场上去的不是劳动,而是工人。工人卖给资本家的不是他的劳动,而是对他自身作为劳动力[workingpower]的暂时使用权。在资本家和工人订立的合同中,在他们商定的买卖中,这才是直接的对象。

如果实行计件工资制,工人按件得到报酬,而不是按劳动能力受资本家支配的时间得到报酬,那末,这只是决定这种时间的另一种方式。时间用产品来计量,在这里一定量的产品被看作社会必要劳动时间的表现。在伦敦许多盛行计件工资制的工业部门中,工资就是这样按[社会必要劳动时间的]小时支付的,但是这件或那件劳动产品是否代表“一个小时”,常常引起争执。

不管个别的工资形式如何,劳动能力虽然在被使用以前按一定条件出卖,却要在完成劳动以后,才得到报酬(无论是按日、按周等等),不仅在计件工资制下如此,而且普遍如此。在这里,货币先在观念上作为购买手段,然后成为支付手段,因为商品在名义上转移到买者那里和实际的转移是不同的。商品(劳动能力)的出卖,使用价值在法律上的转让和它在实际上的转让,在这里从时间上说是不一致的。因此,价格的实现迟于商品的出卖(见我的著作的第一部分,第122页)\endnote{马克思在这里引用了《政治经济学批判》第一分册。见《马克思恩格斯全集》中文版第13卷第132—134页。——第121页。}。这里也表明,不是资本家,而是工人在预付;正如出租房子,不是租赁人,而是出租人预付使用价值。诚然,工人在他生产的商品卖出以前得到工资(或者,至少可能得到工资,如果商品不是预先订购,等等)。但是,在他(工人)得到工资以前,他的商品,他的劳动能力,已经消费在生产上,已经转到买者(资本家)的手里。问题不在于,商品的买者打算怎样处理商品,不在于他购买商是为了把它保存下来当作使用价值,还是为了把它再出卖。问题在于第一个买者和卖者之间的直接交易。

[李嘉图在他的《原理》中说:]

\begin{quote}{“在不同的社会阶段,资本,或者说,使用劳动的手段的积累,速度有快有慢,它在所有情况下都必定取决于劳动生产力。一般说来,在存在着大量肥沃土地的地方,劳动生产力最大。”(李嘉图《政治经济学和赋税原理》1821年伦敦第3版第92页)}\end{quote}

关于李嘉图的这个论点,匿名著作的作者说:

\begin{quote}{“假如第一句话中的劳动生产力,是指每一产品中属于亲手生产该产品的人的那一部分很小,那末这句话就是同义反复,因为其余部分形成一个基金,只要它的所有者高兴,便可以用来积累资本。”}\end{quote}

(因此,就不言而喻地承认:从资本家的观点来看,“劳动生产力是指每一产品中属于亲手生产该产品的人的那一部分很小”。这句话非常好。)

\begin{quote}{“但是,在土地最肥沃的地方,大多不会有这种情况。”}\end{quote}

(这个反驳是愚蠢的。李嘉图是以资本主义生产为前提的。他不是研究,资本主义生产是在土地肥沃的地方容易发展,还是在土地相对来说不肥沃的地方容易发展。在资本主义生产存在的地方,资本主义生产在土地最肥沃的地方生产率最高。劳动的自然生产力,即劳动在无机界发现的生产力,和劳动的社会生产力一样,表现为资本的生产力。李嘉图本人在上面一段话中,把劳动生产力和生产资本的劳动——它生产的是支配劳动的财富、而不是归劳动所有的财富——等同起来,这是正确的。他的用语“资本,或者说,使用劳动的手段”,实际上是他把握资本的真正本质的唯一用语。他本人局限于[809]资本主义观点,以致对他来说这种颠倒,这种概念的混淆是不言而喻的。劳动的客观条件(而且是劳动本身创造的),原料和劳动工具,不是劳动作为自己的手段来使用的手段,相反,它们是使用劳动的手段。不是劳动使用它们,而是它们使用劳动。劳动是这些物作为资本进行积累的手段,而不是给工人提供产品、财富的手段。)

\begin{quote}{“在北美是这种情况,但这是一种人为的情况”}\end{quote}

(即资本主义的情况)。

\begin{quote}{“在墨西哥和新荷兰\authornote{澳大利亚的旧称。——编者注}不是这种情况。从另一种意义来说,在有许多肥沃土地的地方,劳动生产力确实最大,——这里是指人(只要他愿意)生产出与他所完成的总劳动相比是大量的原产品的能力。人能生产出超过维持现有人口生活所必需的最低限度的食物,这实际上是自然的赐予。”}\end{quote}

(这是重农学派学说的基础。这种“自然的赐予”是剩余价值的自然基础,它在农业劳动(最初几乎所有的需要都由农业劳动满足)中表现得最明显。在工业劳动中不是这样明显,因为工业劳动的产品首先必须作为商品出卖。最先分析剩余价值的重农学派,就是在剩余价值的实物形式上理解剩余价值的。)

\begin{quote}{“但是‘剩余产品’(李嘉图先生的用语,第93页)一般是指某物的全部价格超过生产该物的工人所得部分的余额}\end{quote}

(这个蠢驴没有看到,在土地肥沃,因而在产品价格中工人所得的份额虽然不大却能购买足够的生活必需品的地方,资本家所得的份额是最大的),

\begin{quote}{是指一种由人的协议确定而不是由自然规定的关系。”(《评政治经济学上若干用语的争论》第74—75页)}\end{quote}

如果最后结尾的这句话有什么意义,那就是,资本主义意义上的“剩余产品”应该同劳动生产率本身严格区别开来。劳动生产率只有当它对资本家来说作为利润实现时,才引起资本家的关心。这正是资本主义生产的局限性,资本主义生产的界限。

\begin{quote}{“如果对某一物品的需求超过了从供给的现状来看的有效需求,因而价格上涨,那末,或者,能够在生产费用的比率保持不变的情况下,增大供给的规模,——在这种情况下,供给的规模将一直增大到这种物品同其他物品按原先那样的比例进行交换为止。或者,第二,不能增大原来的供给规模,这样,上涨的价格将不下降,而是如斯密所说,将继续为生产这种物品所使用的特殊的土地、资本或劳动提供更多的地租,或利润,或工资(或所有三者)。或者,第三,供给的可能增加,相应地要求比原先供给的商品量的周期生产〈注意这个用语!〉有更多的土地,或资本,或劳动,或所有三者。这样,在需求增加到足以(1)按提高的价格支付追加供给;(2)按提高的价格支付原先的供给量以前,供给就不会增加。因为生产追加商品量的人,不会比生产原先商品量的人有更多的可能获得商品的高价……这样,这个行业就会得到超额利润……超额利润或者只落到一些特殊的生产者手里……或者在追加的产品和其余的产品不能区别时,由大家分享……人们为了加入能得到这种超额利润的行业,将付出一些东西……他们为此而付出的就是地租。”(同上,第79—81页)}\end{quote}

这里要注意的只是,在这一著作中地租第一次被看作固定的超额利润的一般形式。

\begin{quote}{[810]“‘收入转化为资本’这一用语,是这些用语争论的另一个根源。一个人以为这是指资本家把他的资本所赚得的一部分利润用于增加他的资本,而不是象他在另一种情况下可能做的那样,用于个人消费。另一个人则以为这是指某人作为资本支出的,决不是他作为他自己的资本的利润得到的,而是作为地租、工资、薪金得到的。”(同上,第83—84页)}\end{quote}

最后这些说法——“这些用语争论的另一个根源”,“一个人以为这是指”,“另一个人则以为这是指”——表明了这个自作聪明的拙劣作者的手法。

\tsectionnonum{(b)《论马尔萨斯先生近来提倡的关于需求的性质和消费的必要性的原理》[匿名作者的资产阶级的局限性。他对李嘉图的积累理论的解释。不理解引起危机的资本主义生产的矛盾]}

《论马尔萨斯先生近来提倡的关于需求的性质和消费的必要性的原理》1821年伦敦版。

李嘉图学派的著作。对马尔萨斯驳斥得好。表现出这些人的无限局限性。这里暴露出,当他们考察的不是土地所有权而是资本的时候,他们的敏锐性就变为无限的局限性了。虽然如此,这部著作还是上面提到的十年内最好的论战著作之一。

\begin{quote}{“如果用于刀的生产的资本增加1%,并且只能按同样的比例增加刀的生产,那末,假定其他物品的生产不增加,刀的生产者支配一般物品的可能性,将按较小的比例增加;正是这种可能性,而不是刀的数量的增加,构成企业主的利润,或增加他的财富。但是,如果其他所有行业的资本同时也增加1%,并且产品同样增加,那末结果就不同了,因为一种产品和另一种产品交换的比例不变,从而每种产品的一定部分所能支配的其他产品和以前一样多。”(上述著作,第9页)}\end{quote}

首先,如果象假定的那样,除了刀的生产,其他的生产(以及用于生产的资本)都不增加,那末,刀的生产者的收入将不是“按较小的比例”增加,而是根本不增加,甚至绝对亏损。这时,刀的生产者只有三条路可走。或者,他必须拿已增加的产品去交换,就象他拿较少量的产品去交换一样;这样,他的增产将造成真正的亏损。或者,他必须努力找到新的消费者;如果他限于原先的消费者范围,那末要做到这一点,就只有从其他行业把买主吸引过来,把自己的亏损转嫁到别人身上;或者,他必须超越原先的界限扩大他的市场,——但是,这两种办法都既不取决于他的美好愿望,也不仅仅取决于已增加的刀的数量的存在。或者,最后,他必须把他的产品的剩余部分转到下一年去,并相应地减少下一年的新的供给,这样,如果他的资本追加额不仅包括追加的工资,而且包括追加的固定资本,也会造成亏损。

其次,如果其他所有资本都按相同的比例积累,决不能由此得出结论说,它们的生产也按相同的比例增加。即使是这样,也不能由此得出结论说,它们需要多用1%的刀,因为它们对刀的需求,既同它们自己产品的增加没有什么联系,也同它们对刀的购买力的增长没有什么联系。这里只会得出同义反复:如果每一个别行业的资本的增加,与社会需要所造成的对每一个别商品的需求的增加成比例,那末,一种商品的增加就会为其他商品的增加供给提供市场。

因此,这里假定:(1)是资本主义生产,其中每一个别行业的生产以及这种生产的增加,都不是直接由社会需要调节,由社会需要[811]控制,而是由各个资本家离开社会需要而支配的生产力调节的;(2)尽管如此,生产却是这样按比例地进行,好象资本直接由社会根据其需要使用于各个不同的行业。

按照这个自相矛盾的假定,即假定资本主义生产完全是社会主义的生产,那末,实际上就不会发生生产过剩。

此外,在资本积累相等的不同行业内(说资本在不同行业按相等的比例积累,又是一个不妥当的假定),与所用资本的这

种增加相应的产品量,是极不相同的,因为不同行业的生产力,或者说,所生产的使用价值量与所使用的劳动之比,是大不相同的。这里和那里生产出相同的价值,但是同一价值表现出来的商品量却大不相同。因此,当A行业的价值增加1%,商品量增加20%,而B行业的价值同样增加1%,但商品量只增加5%时,就完全无法理解,为什么A的商品量必定在B行业找到市场。在这里忽视了使用价值和交换价值的区别。

萨伊的伟大发现——“商品只能用商品购买”,\endnote{在萨伊的著作(《论政治经济学》1814年巴黎第2版第2卷第382页)中说过:“产品只是用产品购买的”。对萨伊这个论点的批判见本卷第2册第563—564页和第569—574页。——第127页。}只不过是说,货币本身是商品的转化形式。这决不能证明,因为我只能用商品购买,所以我就能用我的商品购买,或者说,我的购买力和我所生产的商品量成比例。同一价值可以表现为极不相同的商品量。但是使用价值——消费——和产品价值无关,而和产品量有关。完全不能理解,为什么因为现在六把刀的价钱和以前一把刀一样,我就要买六把刀。且不说工人出卖的不是商品,而是劳动,而且有许多人不生产商品,但是用货币购买。商品的买者和卖者不是同一的。土地所有者和货币资本家等在货币形式上获得其他生产者的商品。他们是“商品”的买者,却不是“商品”的卖者。不仅产业资本家之间有买卖,而且他们还把自己的商品卖给工人和不是商品生产者的收入所有者。最后,他们作为资本家进行的买卖和他们花费自己的收入的购买,是大不相同的。

\begin{quote}{“李嘉图先生(第二版第359页)在引证了斯密关于利润下降的原因的观点之后,补充说:‘但是,萨伊先生曾经非常令人满意地说明:由于需求只受生产限制,所以任何数额的资本在一个国家都不会不加以使用。’”}\end{quote}

(多么聪明!当然,需求受生产限制。对那种不可能按定货生产的东西,或需求不能现成地在市场上找到的东西,是不可能产生需求的。但是,绝不能因为需求受生产限制就得出结论说,生产受需求限制或曾经受它限制,生产永远不能超过需求,特别是不能超过与当前市场价格适应的需求。这是萨伊式的敏锐思想。)

\begin{quote}{“‘在一个国家中,除非工资由于必需品的涨价而大大提高,因而剩下的资本利润极少,以致积累的动机消失,否则积累的资本不论多少,都不可能不生产地加以使用’〈匿名作者自己在括号内写道:〉(我认为,这是指“为所有者带来利润”)(同上,第360页)。”}\end{quote}

(在这里李嘉图把“生产地”和“有利润地”等同起来,而在资本主义生产中,只有“有利润地”才是“生产地”,这正是资本主义生产同绝对生产的区别,以及资本主义生产的界限。为了“生产地”进行生产,必须这样生产,即把大批生产者排除在对产品的一部分需求之外;必须在同这样一个阶级对抗中进行生产,[812]这个阶级的消费决不能同它的生产相比,——因为资本的利润正是由这个阶级的生产超过它的消费的余额构成的。另一方面,必须为那些只消费不生产的阶级生产。必须不仅仅使剩余产品具有成为这些阶级的需求对象的形式。另一方面,资本家本人,如果想要积累,也不应当对自己的产品按其生产的数量提出需求——就这些产品加入收入来说。否则他就不能积累。因此,马尔萨斯把那些任务不是积累而是消费的阶级同资本家对立起来。一方面假定所有这些矛盾是存在的,另一方面又假定,生产的进行完全没有冲突,好象这些矛盾都不存在。买和卖是分离的,商品和货币、使用价值和交换价值是分离的。可是又假定,这种分离是不存在的,存在的是物物交换。消费和生产是分离的;生产者不消费,消费者不生产。可是又假定,消费和生产是同一的。资本家进行生产是直接为了增加他的利润,为了交换价值,而不是为了享受。可是又假定,他进行生产是直接为了享受,而且仅仅为了享受。如果假定资本主义生产中存在的矛盾——这些矛盾诚然不断在平衡,但是这一平衡过程同时表现为危机,表现为互相分离、彼此对立、但又互相联系的各因素的通过暴力的结合——不存在,那末这些矛盾自然就不可能发生作用。在每个行业,每个资本家都按照他的资本进行生产,而不管社会需要,特别是不管同一行业其他资本的竞争性供给。可是又假定,他好象是按社会的定货进行生产的。如果没有对外贸易,据说,奢侈品就会不管生产费用多少而在国内生产。在这种情况下,除了生活必需品的生产以外,劳动就确实是非常不生产的了。因此,资本的积累也不多了。这样,每个国家就可以使用全部在国内积累的资本,因为按照假定,在国内只积累少量资本。)

\begin{quote}{“如果李嘉图前一句中的‘不会不加以使用’是指‘不可能不生产地加以使用’,或者更确切地说,‘不可能不有利润地加以使用’,那末后一句就把前一句限定了(不说同它矛盾)。如果它单单指‘加以使用’,这一论断就没有意义了,因为,我想,无论亚当·斯密或其他任何人都没有否认:如果不计较利润的多少,资本是能够‘加以使用’的。”(同上,第18—19页)}\end{quote}

实际上,李嘉图是说,在一个国家中,一切资本,无论是以什么样规模积累起来的,都能有利润地加以使用;另一方面,资本的积累又阻碍“有利润地”使用资本,因为资本的积累必定引起利润的减少,亦即积累率的缩减。

\begin{quote}{“他们〈工人〉[对工作的]需求的增加\authornote{见本册第60页。——编者注}不过是表明他们甘愿自己拿走产品中更小的份额,而把其中更大的份额留给他们的雇主;要是有人说,这会由于消费减少而加剧市场商品充斥,那我只能回答说:市场商品充斥是高额利润的同义语。”(同上,第59页)}\end{quote}

这的确是市场商品充斥的隐秘基础。

\begin{quote}{“只要由于使用机器而价格变得便宜的物品,不是工人因为便宜就能使用的东西,那末,作为消费者的工人在繁荣时期并不能(如萨伊先生在《论政治经济学》第四版第一卷第60页上所说的那样)从机器得到任何好处。从这方面来看,脱粒机和风磨,对于工人来说,可能是很重要的;但是截夹板机,滑轮制造机或花边织机的发明很少使他们的状况得到改善。”(同上,第74—75页)“在分工发达的地方,工人的技艺只能在他学得这种技艺的特殊领域应用;工人本身就是一种机器。在这种情况下,有一个很长的失业时期,就是说,一个失去劳动,即从根本上失去财富的时期。因此,象鹦鹉那样喋喋不休地说,事物都有找到自己的水准的趋势,是丝毫无济于事的。我们必须看看周围,我们会发现,事物[813]长时期都不能找到自己的水准;即使找到了,也比过程开始时的水准低。”(同上,第72页)}\end{quote}

这位李嘉图主义者,效法李嘉图,正确地承认了由商业途径的突然变化引起的危机\endnote{李嘉图的《原理》第十九章标题是《论商业途径的突然变化》,在这里,“商业”不仅指某个国家的商业,而且指某个国家的生产活动。参看本卷第2册第567—568页。——第130页。}。1815年战争以后英国的情况就是这样。因此,所有以后的经济学家每次都认为,每次危机的最明显的导火线就是引起每次危机的唯一可能的原因。

他也认为信用制度是危机的原因。(第81页及以下各页)(好象信用制度本身不是由“生产地”即“有利润地”使用资本的困难产生的。)例如,英国人为了开辟市场,不得不把他们自己的资本贷到国外去。在生产过剩、信用制度等上,资本主义生产力图突破它本身的界限,超过自己的限度进行生产。一方面,它有这种冲动。另一方面,它只能忍受与有利润地使用现有资本相适应的生产。由此就产生了危机,它同时不断驱使资本主义生产突破自己的界限,迫使资本主义生产飞速地达到——就生产力的发展来说——它在自己的界限内只能非常缓慢地达到的水平。

匿名作者对萨伊的评论非常正确。这在分析萨伊时应该引用。(见第VII本第134页\endnote{马克思指他的关于政治经济学的札记本之一。马克思在这里提到的第VII本的前63页是1857—1858年经济学手稿的结尾部分(见卡·马克思《政治经济学批判大纲》1939年莫斯科版第586—764页)。从第VII本第63a页起(马克思在这里注明:“从1859年2月28日开始”)。是路德、兰盖、加利阿尼、维里、帕奥累蒂、马尔萨斯、理查·琼斯以及其他作者的著作的摘录。在第VII本第134页马克思从《论马尔萨斯先生近来提倡的关于需求的性质和消费的必要性的原理》这一著作(第110和112页)中摘录了有关匿名作者批判和讽刺萨伊的段落。——第130页。})

\begin{quote}{“他〈工人〉同意用他的一部分时间为资本家劳动,或者——其结果一样——同意把生产出来并拿去交换的总产品的一部分归资本家所有。他不得不这样做,否则资本家将不给他提供帮助}\end{quote}

(即提供资本。妙极了,按照匿名作者的意见,不论资本家占有全部产品而以其中一部分作为工资付给工人,还是工人把自己的一部分产品留下给资本家,“其结果一样”)。

\begin{quote}{但是,因为资本家的动机是盈利,并且因为这种利益在一定程度上总是既取决于积蓄的能力,又取决于积蓄的意愿,所以资本家愿意提供这种帮助的追加量;而同时,因为他将发现,需要这种追加量的人比过去需要原有量的人少,所以他只能指望,归他自己的那一部分利益少些;他不得不同意把他的帮助所创造的利益的一部分作为(可以说是)礼物〈!!!〉送给工人,否则他就得不到另外一部分利益。这样,利润就由于竞争而降低了。”(同上,第102—103页)}\end{quote}

真妙极了!如果由于劳动生产力的发展,资本积累十分迅速,以致对劳动的需求使工资提高,工人白白为资本家劳动的时间少些,并在一定程度上分到自己生产率较高的劳动所创造的利益,那末这就是资本家送给工人“礼物”!

同一位作者详细地证明,高工资对工人是一种不良刺激,虽然在谈到土地所有者时,他认为低利润会使资本家心灰意懒。(见第XII本第13页\endnote{马克思指他的第XII本札记本。在这个札记本的封面上马克思亲笔写着:“1851年7月于伦敦”。在第13页上对匿名著作《论马尔萨斯先生近来提倡的关于需求的性质和消费的必要性的原理》第97、99、103—104、106—108和111页做了摘录。在马克思的第XII本札记本第12页上摘录了上述著作第54—55页,其中谈到土地所有者(土地所有者的地租减低资本家的利润)。——第131页。})

\begin{quote}{“亚·斯密认为,资本的一般积累或增加会降低一般利润率,其原理与每个个别行业的资本的增加会降低该行业的利润是一样的。但是实际上,个别行业的资本的这种增加,意味着这里资本增加的比例比其他行业同期内资本增加的比例大。”(同上,第9页)}\end{quote}

驳萨伊。(见第XII本第12页\endnote{马克思在他的第XII本札记本第12页上从《论马尔萨斯先生近来提倡的关于需求的性质和消费的必要性的原理》(第15页)摘录了匿名作者对萨伊关于英国生产过剩的原因是意大利生产不足的论断的批评意见。参看本卷第1册第237页,第2册第606—607页,第3册第277页。——第131页。})

\begin{quote}{“可以说,劳动是资本的直接市场或直接活动场所。在一定的时间,在一定的国家,或在全世界,能按照不低于既定利润率投放的资本量,看来主要取决于因支出该资本而可能推动当时实有人数去完成的劳动量。”(同上,第20页)[814]“利润不是取决于价格,而是取决于同费用比较而言的价格。”(同上,第28页)“萨伊先生的论点\endnote{在这以前匿名作者从萨伊的著作(《给马尔萨斯先生的信》1820年巴黎版第46页)引用了萨伊的论点:“产品只是用产品购买的”。这个论点在萨伊那里还有另外的说法:“产品总是为自己开辟市场”(《论马尔萨斯先生近来提倡的关于需求的性质和消费的必要性的原理》1821年伦敦版第13、110页)。——第132页。}决不证明,资本为自己开辟市场,而只证明,资本和劳动相互为对方开辟市场。”(同上,第111页)}\end{quote}

\tsectionnonum{(C)托马斯·德·昆西[无法克服李嘉图观点的真正缺陷]}

[托马斯·德·昆西]《三位法学家关于政治经济学的对话,主要是关于李嘉图先生的〈原理〉》(载于1824年《伦敦杂志》第9卷)。

试图反驳一切对李嘉图的攻击。从下面这句话可以看出,他是知道问题所在的:

\begin{quote}{“政治经济学的一切困难可以归结为:什么是交换价值的基础?”(上述著作,第347页)}\end{quote}

在这一著作中常常尖锐地揭露李嘉图观点的不充分,虽然在这样做的时候,其辩证法的深度与其说是真实的,不如说是矫揉造作的。真正的困难(这些困难不是由价值规定产生的,而是由于李嘉图在这个基础上所作的说明不充分,由于他强制地和直接地使比较具体的关系去适应简单的价值关系)根本没有解决,甚至根本没有觉察到。但是,这本著作具有它出版的那个时期的特征。可以看出,那时在政治经济学中人们对待前后一贯性和思维还是严肃的。

(同一作者后来一本较差的著作;《政治经济学逻辑》1844年爱丁堡版。)

德·昆西尖锐地表述了李嘉图观点和前人观点不同之处,并且没有象后来人们所作的那样,企图通过重新解释来削弱或抛弃问题中所有独特的东西,只在文句上加以保留,从而为悠闲的无原则的折衷主义敞开大门。

李嘉图学说中有一点德·昆西特别强调,在这里我们也必须指出,因为它在我们马上就要考察的同李嘉图的论战中起作用,这个论点就是:一种商品支配其他商品的能力(它的购买力;事实上就是它用其他商品表示的价值)和它的实际价值根本不同。

\begin{quote}{“如果得出结论说,实际价值大是因为它购买的量大,或者实际价值小是因为它购买的量小,那完全是错误的……如果商品A的价值增加一倍,它支配的商品B的量并不因此就比以前增加一倍。情况可能如此,但是也可能支配的量是500倍或只是1/500……谁也不否认,商品A由于本身的价值加倍,所支配的一切价值不变的物品的量也将加倍……但是,问题在于,是不是在一切情况下,商品A在它的价值加倍时所支配的量都将加倍。”(散见《三位法学家的对话》第552—554页)}\end{quote}

\tsectionnonum{(d)赛米尔·贝利}

\tsectionnonum{[(a)《评政治经济学上若干用语的争论》的作者和贝利在解释价值范畴中的肤浅的相对论。等价物问题。否认劳动价值论是政治经济学的基础]}

[寨米尔·贝利]《对价值的本质、尺度和原因的批判研究;主要是论李嘉图先生及其信徒的著作》,《略论意见的形成和发表》一书的作者著,1825年伦敦版。

这是反对李嘉图的主要著作(也反对马尔萨斯)。试图推翻学说的基础——价值。除了“价值尺度”的定义,或者更确切地说,具有这一职能的货币的定义以外,从积极方面来看,没有什么价值。(并参看同一作者的另一著作:《为〈韦斯明斯特评论〉杂志上一篇关于价值的论文给一位政治经济学家的信》1826年伦敦版)

因为正如前面讲的\authornote{见本册第118页。——编者注},这部著作的基本思想是赞同《评政治经济学上若干用语的争论》的,所以这里还要回头去谈后一著作并引用其中有关的地方。

《评政治经济学上若干用语的争论》的作者责备李嘉图,说他把价值由商品在其相互关系中的相对属性变成某种绝对的东西。

在这方面,李嘉图应该受责备的只是,他在阐述价值概念时没有把不同的因素,即没有把在商品交换过程中出现或者说表现出来的商品交换价值和商品作为价值的存在(这种存在与商品作为物、产品、使用价值的存在不同)严格区分开来。

[815]《评政治经济学上若干用语的争论》中谈到:

\begin{quote}{“如果用来生产大部分商品或除一种商品以外的所有商品的绝对劳动量增加了,那末能不能说,这一种商品的价值仍然不变?因为它将同较少量的其他各种商品相交换。如果实际上断定,应当把价值的增加或减少理解为生产这种商品的劳动量的增加或减少,那末,我刚才加以反驳的结论,可能在一定程度上是正确的。但是象李嘉图先生那样,说生产两种商品的相对劳动量是这两种商品相互交换的比例的原因,即两种商品的交换价值的原因,这就同所谓每种商品的交换价值表示生产该商品的劳动量,而完全与其他商品或其他商品的存在毫无关系的说法完全不一样。”(《评政治经济学上若干用语的争论》第13页)“李嘉图先生的确告诉我们,‘他希望引起读者注意的这个研究,涉及的是商品相对价值的变动的影响,而不是商品绝对价值的变动的影响’\authornote{见本卷第2册第189页。——编者注},在这里他好象认为,有一种是交换价值而又不是相对价值的东西。”(同上,第9—10页)“李嘉图先生离开了他对价值这个词的最初用法,使价值成为某种绝对的东西而不是相对的东西。这一点在他的《价值和财富,它们的特性》这一章中表现得尤其明显。那里讨论的问题,其他经济学家也曾讨论过,那纯粹是毫无益处的用语的争论。”(同上,第15—16页)}\end{quote}

我们在评论这个人以前,还要谈谈李嘉图。他在其《价值和财富》一章中证明,社会财富不取决于所生产的商品的价值,虽然后一点对于单个生产者来说具有决定性的意义。因此,他更应该理解,仅仅以剩余价值为目的即以生产者群众的相对贫困为基础的生产形式,绝不能象他一再说明的那样,是财富生产的绝对形式。

现在,我们来谈这位在“用语”上自作聪明的人是怎样“评”[注;讽刺性地暗喻这个作者的著作《评政治经济学上若干用语的争论》。——编者注]的。

如果除了一种商品以外,所有商品都因为比以前花费了更多的劳动时间而增加了价值,那末,劳动时间没有变动的这种商品,就同较少量的其他所有商品相交换。这种商品的交换价值(就它实现在其他商品上来说),即表现在其他所有商品的使用价值上的交换价值减少了。“然而能不能说这一种商品的交换价值仍然不变?”这只是提出了所谈的问题,这里既没有肯定的回答,也没有否定的回答。如果生产一种商品所需要的劳动时间减少了,而生产其他所有商品的劳动时间不变,结果仍旧一样:一定量的这种商品将同较少量的其他所有商品相交换。在这里,在两种情况下发生了同样的现象,虽然发生的原因是直接相反的。反之,如果生产商品A所需要的劳动时间不变,而生产其他所有商品的劳动时间减少了,那末,它将同较大量的其他所有商品相交换。由于相反的原因,即生产商品A所需要的劳动时间增加了,而生产其他所有商品的劳动时间不变,也会产生同样的结果。因此,在第一种情况下,商品A同较少量的其他所有商品相交换,而这可能是由于两种相反的原因。在第二种情况下,它同较大量的其他所有商品相交换,这也可能是由于两种相反的原因。但是请注意,按照假定,它每次都是按它的价值进行交换,因而是同等价物相交换。商品A每次都把它的价值实现在它所交换的一定量的其他使用价值上,而不管这些使用价值的量怎样变动。

由此显然可以得出结论:商品作为使用价值相互交换的量的比例,诚然是商品价值的表现,是商品的实现了的价值,但不是商品价值本身,因为同样的价值比例可以表现在完全不同的使用价值量上。商品作为价值的存在不表现在商品本身的使用价值上——不表现在商品作为使用价值的存在上。商品的价值是在商品用其他使用价值来表现时显现出来的,也就是在其他使用价值同这一商品相交换的比例中显现出来的。如果1盎斯金=1吨铁,也就是说,如果少量的金和大量的铁交换,难道表现在铁上的一盎斯金的价值因此就比表现在金上的铁的价值大吗?商品按它们所包含的劳动进行交换,也就是说,就它们代表等量劳动来说,它们是相等的,同一的。因而这也是说,每一商品,就本身来看,是和它[816]自己的使用价值,和它自己作为使用价值的存在不同的东西。

同一商品的价值,依照我把它表现在这种或那种商品的使用价值上,可以表现为极其不同的使用价值量,但是价值本身不变。这虽然使价值的表现改变了,但是没有使价值发生变动。同样,所有可以表现商品A的价值的不同使用价值的不同量,都是等价物,它们不仅作为价值,而且作为等量的价值互相发生关系,因此,当这些极不相同的使用价值量互相代替时,价值仍然不变,就象它没有在极不相同的使用价值上获得表现一样。

如果商品按照它们代表等量劳动时间的那种比例进行交换,那末它们作为物化劳动时间的存在,它们作为物体化劳动时间的存在,就是它们的统一体,它们的同一要素。作为这样的劳动产品,商品在质上是同一的,只是在量上根据它们代表的同一物即劳动时间的多少而有所不同。它们作为这个同一要素的表现,是价值,而就它们代表等量劳动时间来说,它们是相等的价值,是等价物。为了可以在量上把它们加以比较,它们必须首先是同名的量,是在质上同一的。

正是作为这种统一体的表现,这些不同的物是价值,并且作为价值互相发生关系,而它们的价值量的差别,它们的内在的价值尺度,也就由此得出来。而且只是因为如此,一种商品的价值,才能体现、表现在作为它的等价物的其他商品的使用价值上。因此,单个商品本身——完全撇开它的价值在其他商品上的表现不谈——作为价值,作为这个统一体的存在,也和它本身作为使用价值,作为物不同。作为劳动时间的存在,商品是价值;作为一定量的劳动时间的存在,它是一定的价值量。

因此,我们这位自作聪明的人的下面说法是很典型的:“如果我们是这样理解,那我们就不是这样理解”,反之亦然。我们的“理解”和我们所说的事情的本质特征没有一点关系。当我们说某物的交换价值时,我们当然首先把它理解为能够同某一种商品相交换的其他任何一种商品的相对量。但是,经过更进一步的考察,我们将发现:要使某物同其他根本不同的无数物品——即使它们之间有自然的或其他的相似之处,在交换时也不会加以注意——相交换的比例成为稳定的比例,所有这些不同的各种各样的物品都必须看成是同一的共同的统一体的相应表现,即看成与它们的自然存在或外表完全不同的要素的相应表现。其次,我们还将发现,如果我们的“理解”有一些意义,那末某一商品的价值就不仅是某种使该商品与其他商品不同或相同的东西,而且是某种使该商品与它本身作为物,作为使用价值的存在不同的质。

\begin{quote}{“A物的价值的提高,只是指用B、C等物衡量的价值,即同B、C等物交换时的价值。”(同上,第16页)}\end{quote}

为了A物(例如书)的价值可以用B物煤和C物葡萄酒来衡量,A、B、C作为价值必须是与它们作为书、煤或葡萄酒的存在不同的东西。为了A的价值可以用B来衡量,A必须具有不以B对这种价值的衡量为转移的价值,并且二者[在质上]都必须等于表现在二者上的第三物。

说商品的价值因此就由某种相对的东西变为某种绝对的东西,是完全错误的。正好相反。作为使用价值,商品表现为某种独立的东西。而作为价值,它仅仅表现为某种设定的东西\endnote{设定的东西——黑格尔哲学术语,是指和无条件的、原初的、第一性的东西相区别的某种受制约的东西,不以本身为根据而以他物为根据的某种东西。——第138页。},某种仅仅由它与社会必要的、同一的、简单的劳动时间的关系决定的东西。这是相对的,只要商品再生产所必要的劳动时间发生变动,它的价值也就变动,虽然它实际包含的劳动时间并未变动。

[817]我们这位自作聪明的人陷入了拜物教多深,以及他怎样把相对的东西变为某种肯定的东西,下面的话是最清楚的说明:

\begin{quote}{“价值是物的属性,财富是人的属性。从这个意义上说,价值必然包含交换,财富则不然。”(同上,第16页)}\end{quote}

在这里财富是使用价值。当然,使用价值对人来说是财富,但是一物之所以是使用价值,因而对人来说是财富的要素,正是由于它本身的属性。如果去掉使葡萄成为葡萄的那些属性,那末它作为葡萄对人的使用价值就消失了;它就不再(作为葡萄)是财富的要素了。作为与使用价值等同的东西的财富,它是人们所利用的并表现了对人的需要的关系的物的属性。相反,在我们的作者看来,“价值”竟是“物的属性”!

商品作为价值是社会的量,因而,和它们作为“物”的“属性”是绝对不同的。商品作为价值只是代表人们在其生产活动中的关系。价值确实包含交换,但是这种交换是人们之间物的交换;这种交换同物本身是绝对无关的。物不论是在A手中还是在B手中,都保持同样的“属性”。“价值”概念的确是以产品的“交换”为前提的。在共同劳动的条件下,人们在其社会生产中的关系就不表现为“物”的“价值”。产品作为商品的交换,是劳动的交换以及每个人的劳动对其他人的劳动的依存性的一定形式,是社会劳动或者说社会生产的一定方式。

我在我的著作的第一部分\endnote{指《政治经济学批判》第一分册。见《马克思恩格斯全集》中文版第13卷第22—23页,以及第37—39页。——第139页。}曾经谈到,以私人交换为基础的劳动的特征是:劳动的社会性质以歪曲的形式“表现”为物的“属性”;社会关系表现为物(产品,使用价值,商品)互相之间的关系。我们这位拜物教徒把这个假象看成为真实的东西,并且事实上相信物的交换价值是由它们作为物的属性决定的,完全是物的自然属性。直到目前为止,还没有一个自然科学家发现,鼻烟和油画由于什么自然属性而彼此按照一定比例成为“等价物”。

可见,正是他这位自作聪明的人,把价值变为某种绝对的东西,变为“物的属性”,而不是把它仅仅看成某种相对的东西,看成物和社会劳动的关系,看成物和以私人交换为基础的社会劳动的关系,在这种社会劳动中,物不是作为独立的东西,而只是作为社会生产的表现被规定的。

但是,“价值”不是绝对的东西,不能把它看成某种独立存在的东西,这跟下面一点完全不同:商品必然会使它的交换价值具有一种不同于它的使用价值,或者说不同于作为实际产品的存在并且不依赖于这种存在的独立的表现,也就是说,商品流通必然导致货币的形成。商品使它的交换价值在货币中,首先是在价格中具有这种表现,在价格中,所有商品都表现为同一劳动的物化,都不过是同一实体在量上的不同表现。商品的交换价值在货币上的独立化本身,是交换过程的产物,是商品中包含的使用价值和交换价值的矛盾以及商品中同样包含的矛盾——一定的、特殊的私人劳动必然表现为它的对立面,表现为同一的、必要的、一般的并且在这种形式上是社会的劳动——发展的结果。商品表现为货币,不仅包含这样的意思,即商品的不同价值量,是通过它们的价值在一种特殊商品的使用价值上的表现来衡量,同时也包含下面的意思:所有商品都表现为一种形式,在这种形式中,它们作为社会劳动的化身而存在,因而可以同其他任何商品交换,可以随意转化为任何使用价值。所以,它们表现为货币——价格——最初只是观念的,只有通过实际的出卖才能实现。

李嘉图的错误在于,他只考察了价值量,因而只注意[818]不同商品所代表的、它们作为价值所包含的物体化的相对劳动量。但是不同商品所包含的劳动,必须表现为社会的劳动,表现为异化的个人劳动。在价格上,这种表现是观念的。只有通过出卖才能实现。商品中包含的单个人的劳动转化为同一的社会劳动,从而转化为可以表现在所有使用价值上,可以同所有使用价值相交换的劳动——这种转化,交换价值的货币表现中包含的这个问题的质的方面,李嘉图没有加以阐述。商品中包含的劳动必须表现为同一的社会劳动即货币,这种情况被李嘉图忽视了。

资本的发展,从它自己这方面看,已经是以商品交换价值的充分发展,因而也是以商品交换价值在货币上的独立化为前提。在资本的生产过程和流通过程中,作为独立形式的价值是出发点;这个价值保存下来,得到增加,在它赖以表现的商品所经历的一切变化中通过与其原有量的比较来衡量自己的增加程度,并且更换着作为它的躯壳的商品,而不管价值本身表现在极不相同的使用价值上。作为生产的先决条件的价值和由生产中产生的价值二者之间的关系,——而作为先决条件的价值的资本是同利润相对立的资本,——构成整个资本主义生产过程的包罗一切的和决定性的因素。这里不仅是象货币形式那样的价值的独立表现,而且是处于运动过程中的价值,也就是在使用价值经历极不相同的形式的过程中保存下来的价值。因此,价值在资本上的独立化程度比在货币上要高得多。

从以上所说可以判断我们这位在“用语”上自作聪明的人是多么高明,他把交换价值的独立化看成是空洞的词句、表达的手法、经院式的虚构。

\begin{quote}{“‘价值’或法文的valeur,不仅被绝对地,而不是相对地当作物的属性,甚至被一些作者当作可衡量的商品。‘占有价值’,‘转移价值的一部分’〈固定资本的一个非常重要的因素〉,‘价值总额或总和’等等,我不知道这一切都说明什么。”(同上,第57页)}\end{quote}

因为货币本身是商品,从而具有可变的价值,所以独立化的价值本身,在货币上也只获得相对的表现,这一点丝毫不会使问题发生变化,而是一种缺陷,这种缺陷的产生是由于商品的性质,由于商品的交换价值必然具有和商品的使用价值不同的表现而产生的。我们的这位作者已经充分暴露了他的“我不知道”。这种情况从他的批判的全部性质可以看出。这种批判企图用空谈来回避事物本身的矛盾的规定性中包含的困难,并把困难说成是思考的产物或定义之争。

\begin{quote}{“‘两物的相对价值’可能有两种含义:指两物互相交换或将要互相交换的比例,或者指各自交换或将要交换的第三物的相对量。”(同上,第53页)}\end{quote}

不用说,这是个绝妙的定义。如果3磅咖啡今天或明天同1磅茶叶相交换,那绝不是说,这里是等价物相交换。照这种说法,每种商品能够永远只按它的价值进行交换,因为它的价值是它偶然交换的另一种商品的任何量。但是当人们说3磅咖啡同它们的等价物茶叶交换时,通常不是“指”这个意思。这里是假定,在交换后和交换前一样,每个交换者手里都有价值相等的商品。不是两种商品互相交换的比例决定它们的价值,而是它们的价值决定它们互相交换的比例。如果价值不过是商品A偶然交换的商品量,A的价值怎样表现在商品B、C等等上呢?因为[819]在这种情况下两种商品之间既然没有内在的尺度,在A同B交换之前,A的价值就不能表现在B上。

相对价值,第一,指价值量——不同于是价值这种质。因此,后者也不是某种绝对的东西。第二,指一种商品表现在另一种商品的使用价值上的价值。这只不过是它的价值的相对表现——即就价值和它借以表现的商品的关系而言。一磅咖啡的价值只是相对地表现在茶叶上。要绝对地表现一磅咖啡的价值,——即使以相对的形式,即不按它和劳动时间的关系,而按它和其他商品的关系,——就必须把它表现在它和其他所有商品的无限系列的等式上。这将是咖啡的相对价值[在相对形式上]的绝对表现。价值的绝对表现就是价值在劳动时间上的表现,通过这种绝对表现,价值就会表现为某种相对的东西,然而是在那种使价值成为价值的绝对关系中表现的。

\centerbox{※     ※     ※}

现在我们来谈贝利。

他的著作只有一个积极的贡献:他最早比较正确地阐明了价值尺度,实际上就是阐明了货币的一种职能,或者说,阐明了具有特殊的形式规定性的货币。为了衡量商品的价值——为了确定外在的价值尺度——不一定要使衡量其他商品的商品的价值不变。(相反,我在第一部分\endnote{指《政治经济学批判》第一分册。见《马克思恩格斯全集》中文版第13卷第56—58页。——第143页。}已经证明,它必定是可变的,因为价值尺度本身是商品,而且必须是商品,否则它和其他商品就没有共同的内在尺度了。)例如,货币的价值变了,那它的变动对其他所有商品来说都是相同的。因此,其他商品的相对价值就象货币保持不变一样正确地表现在货币上。

这样,就把寻求“不变的价值尺度”的问题排除了。但是,这个问题本身(把不同历史时期的商品价值加以比较的兴趣,实际上不是经济学本身的兴趣,只是纯学术的兴趣\authornote{见本册第166—167页。——编者注})是由误解产生的,它隐藏着一个深刻得多和重要得多的问题。“不变的价值尺度”首先是指一种本身价值不变的价值尺度,就是说,因为价值本身是商品的规定性,“不变的价值尺度”就是指价值不变的商品。例如,金和银或谷物,或劳动,是这种商品,那我们就可以通过同这种商品的比较,通过其他商品同这种商品交换的比例,用其他商品的金价格、银价格、谷物价格或它们和工资的比例,准确地衡量这些其他商品的价值的变动。因此,在这样提出问题时,一开始就假定,“价值尺度”只指其他所有商品赖以表现其价值的商品——不管这是指其他所有商品真正赖以表现其价值的商品即货币,具有货币职能的商品,还是指由于自己价值不变而成为理论家用于计算的货币的那种商品。但是,很显然,无论如何这里涉及的仅仅是作为价值尺度——在理论上或实际上——而本身价值不会变动的货币。

但是要使商品能把它们的交换价值独立地表现在货币上,表现在第三种商品,即特殊的商品上,其前提已经是存在商品价值。余下的问题只在于在量上比较它们。为了使它们的价值和价值差别能够得到这种表现,已经有了一个使它们成为相同的东西(价值),使它们作为价值在质上相同的统一体作为前提。例如,一切商品都用金表现它们的价值,那末它们在金上的这种表现,它们的金价格,它们和金的等式,就是可以说明并计算它们之间的价值比例的等式,因为现在它们表现为不等量的金,并且商品以这种方法在它们的价格中表现为[820]同名称的可以比较的量。

但是要这样表现商品,商品必须已经作为价值而成为同一的。如果商品和金,或任何两种商品,不能作为价值,作为同一的统一体的代表互相表现,那末,每种商品的价值都用金来表现的问题就不能解决。换句话说,问题本身已经包含了这个前提。在谈得上用一种特殊的商品来表现商品的价值以前,商品已经被假定为价值,被假定为和商品的使用价值不同的价值。为了使两个不同的使用价值量可以作为等价物彼此相等,已经假定,它们都等于第三物,它们在质上相同,并且都只是这个等质物的不同量的表现。

因此,“不变的价值尺度”的问题,实际上只是为探索价值本身的概念、性质,价值规定——它本身不再是价值,因此也就不会作为价值发生变动——所作的错误表达。这种价值规定就是劳动时间——在商品生产中特殊地表现出来的社会劳动。劳动量没有价值,不是商品,而是使商品转化为价值的东西,是商品中的统一体,而商品作为这个统一体的表现,在质上相同,只是在量上不同。商品是一定量的社会劳动时间的表现。

假定金具有不变的价值。这样,如果一切商品的价值用金表现,我就能够用商品的金价格来衡量商品价值的变动。但是要用金来表现商品的价值,商品和金作为价值必须是同一的。金和商品只有作为这个价值一定量的表现,作为一定的价值量,才能是同一的。金的不变价值和其余各种商品的可变价值,并不妨碍它们作为价值是同一的,由同一实体构成。在金的不变价值帮助我们哪怕前进一步以前,商品的价值首先必须用金表现,用金估计——就是说,把金和商品当作同一的统一体的表现,当作等价物。

{为了用商品中包含的劳动量衡量商品,——时间是劳动量的尺度,——商品中包含的不同种类的劳动就必须还原为相同的简单劳动,平均劳动,普通的非熟练劳动。只有这样,才能用时间,用一个相同的尺度衡量商品中包含的劳动量。这种劳动在质上必须相同,才能使它的差别成为纯粹量上的差别,纯粹大小的差别。但是,还原为简单的平均劳动,这不是这种劳动(一切商品的价值都还原为这种作为统一体的劳动)的质的唯一规定。某种商品所包含的劳动量是生产该商品的社会必要量,因而劳动时间是必要的劳动时间,这是一种只和价值量有关的规定。但是构成价值统一体的劳动不只是相同的简单的平均劳动。劳动是表现在一定产品中的私人劳动。可是,产品作为价值应该是社会劳动的化身,并且作为社会劳动的化身应该能够由一种使用价值直接转化为其他任何使用价值(劳动赖以直接表现的一定的使用价值,对劳动来说应该是无关紧要的,这样,产品就能够由使用价值的一种形式转化为使用价值的另一种形式)。因此,私人劳动应该直接表现为它的对立面,即社会劳动;这种转化了的劳动,作为私人劳动的直接对立面,是抽象的一般劳动,这种抽象的一般劳动因此也表现为某种一般等价物。个人劳动只有通过异化,才实际表现为它的对立面。但是,商品必须在它让渡以前具有这种一般的表现。个人劳动必然表现为一般劳动,就是商品必然表现为货币。就这些货币当作尺度,当作商品价值的价格表现来说,商品得到了这种表现。但是商品只有实际转化为货币,只有通过出卖,才作为交换价值得到自己的这种适当的表现。第一个转化只是理论的过程,第二个转化才是实际的过程。

[821]因此,谈到作为货币的商品的存在时,应该指出,不仅商品在货币形式上取得了衡量其价值量的一定尺度,——因为它们都把自己的价值表现在同一商品的使用价值上,——而且它们都表现为社会的抽象的一般劳动的存在。这是这样一种形式,通过这种形式,它们都取得相同的外形,它们都表现为社会劳动的直接化身;并且作为这样的化身它们都起着社会劳动的存在的作用,能够直接地——与它们的价值量成比例地——同其他一切商品相交换;其实,商品在已经把商品转化为货币的人手中,不是表现为具有特殊使用价值形式的交换价值的存在,而仅仅表现为作为交换价值承担者的使用价值(例如金)的存在。商品可以低于或高于它的价值出卖。这只和它的价值量有关。但是当它每一次出卖,转化为货币时,它的交换价值都具有一种独立的、和它的使用价值不同的存在。现在它只是作为一定量的社会劳动时间存在,它用来证明这一点的就是它能够直接同任何商品相交换,能够(按照它的量)转化为任何使用价值。在考察货币时,这一点就同商品中包含的劳动作为商品的价值要素所经历的形式上的转化一样,不能忽视。但是通过货币——通过作为货币的商品所具有的这种绝对可交换性,通过它作为交换价值的绝对效能(这和价值量毫无关系,这不是量的规定,而是质的规定)——可以看到:由于商品本身所经历的过程,它的交换价值独立化了,实在地表现在某种和商品的使用价值并列的、独立的形式中,就同曾经观念地表现在它的价格上一样。

这一切表明,《评政治经济学上若干用语的争论》的作者和贝利都不懂得价值和货币的本质,因为他们把价值的独立化看成是经济学家的一种经院式的虚构。价值的这种独立化在资本中表现得更加明显,资本在某种意义上,可以称为处于运动过程中的价值,——这样一来,因为价值只是在货币中独立地存在,——又可以称为处于运动过程中的货币,这种货币经历一系列过程,在其中保存下来,从自身出发并以加大的量回到自身。现实的怪异也表现为用语的怪异,它和人们的常识相矛盾,和庸俗经济学家所指的以及他们认为是他们所说的相矛盾,这是不言自明的。在商品生产的基础上,私人劳动表现为一般的社会劳动;人与人的关系表现为物与物的关系并表现为物——由此而产生的矛盾存在于事物本身,而不是存在于表达事物的用语中。}

看起来李嘉图经常认为,事实上有时也谈到,好象劳动量解答了错误的或者说被错误地理解的“不变的价值尺度”问题,就象从前把谷物、货币、工资等看作解决这个问题的秘方而提出来一样。在李嘉图那里所以会发生这种错误的假象,是因为确定价值量,对于他来说,是决定性的任务。因此,他不懂得劳动在其中成为价值要素的特殊形式,特别是不懂得,个别劳动表现为抽象的一般劳动,并以这个形式表现为社会劳动的必然性。因此,他不懂得货币的形成同价值的本质,同价值由劳动时间决定这一规定有什么联系。

贝利的著作有一些贡献,因为他通过自己的反驳,揭露了表现为货币——一种与其他商品并列的商品——的“价值尺度”同价值的内在尺度和实体的混淆。但是,如果他本人把货币作为“价值尺度”,不只是作为量的尺度,而且作为商品的质的转化来分析,那末他本人就会对价值作出正确的分析。他没有这样做,却满足于对已经以价值为前提的外在“价值尺度”作表面的考察,停留在毫无意义的议论上。

[822]但是在李嘉图著作中还是可以找到个别段落,在那里他直接强调,商品中包含的劳动量所以是衡量它们的价值量、它们的价值量的差别的内在尺度,只是因为劳动是使不同的商品成为相同的东西,是它们的统一体,它们的实体,它们的价值的内在基础。他只是忘掉去研究,劳动在什么样的一定形式上才是这种东西。

\begin{quote}{“如果我们把劳动作为商品价值的基础,把生产商品所必需的相对劳动量作为确定商品相互交换时各自必须付出的相应商品量的尺度,不要以为我们否定商品的实际价格或者说市场价格对商品的这种原始自然价格的偶然和暂时的偏离。”(李嘉图《政治经济学和赋税原理》1821年伦敦第3版第80页)“[德斯杜特·德·特拉西说:]‘衡量……就是找出它们〈被衡量的物〉包含……多少同类的单位。’如果法郎和要衡量的物不能还原为某个对两者共同的另一尺度,法郎就只是衡量铸成法郎的金属本身数量的价值尺度。我认为,它们是可以这样还原的,因为它们两者都是劳动的结果;并且,因此劳动〈因为劳动是它们的动因〉是共同的尺度,用这个尺度可以计量它们的实际价值和相对价值。”(李嘉图《原理》1821年伦敦第3版,第333—334页)}\end{quote}

一切商品都可以还原为劳动即它们的统一体。李嘉图没有研究的,是作为商品的统一体的劳动赖以表现的特殊形式。因此他不懂得货币。因此,在他那里,商品转化为货币,纯粹是形式的、没有深入到资本主义生产内部实质的东西。但是,他告诉我们一点:只因为劳动是商品的统一体,只因为一切商品都是同一统一体——劳动——的表现,所以劳动是商品的尺度。劳动是商品的尺度,不过因为劳动是作为价值的商品的实体。李嘉图对表现在使用价值上的劳动和表现在交换价值上的劳动没有加以应有的区别。作为价值基础的劳动不是特殊的劳动,不是具有特殊的质的劳动。在李嘉图那里,到处都把表现在使用价值上的劳动同表现在交换价值上的劳动混淆起来。诚然,后一种形式的劳动只是以抽象形式表现的前一种形式的劳动。

在上面引用的段落中,李嘉图所谓的实际价值是指作为一定劳动时间的体现的商品。他所谓的相对价值是指一种商品中所包含的劳动时间在其他商品的使用价值上的表现。

现在来谈贝利。

贝利紧紧抓住作为商品的商品的交换价值赖以体现、表现的形式。如果一种商品的交换价值表现在充当货币的第三种商品(其他一切商品同样把它们的价值表现在它上面)的使用价值上,即表现在商品的货币价格上,商品的交换价值就表现为一般的形式。如果我把任何一种商品的交换价值表现在其他任何一种商品的使用价值上,即表现为谷物价格,麻布价格等等,商品的交换价值就表现为特殊的形式。事实上,一种商品的交换价值,对其他商品来说,始终只表现为它们进行交换的量的关系。单个商品本身不能表现一般劳动时间,或者说,单个商品只能以它和充当货币的商品的等式,即以它的货币价格的形式,表现一般劳动时间。但是在这种情况下,商品A的价值始终表现为执行货币职能的商品G的一定量的使用价值。

这是直接的现象。贝利就是紧紧抓住了这种现象。交换价值表现为商品进行交换的量的关系这种最表面的形式,在贝利看来,就是商品的价值。从表面进入深处,是不允许的。贝利甚至忘记一个简单的道理:如果y码麻布=x磅麦秆,那末,不同物品即麻布和麦秆间的这个等式就使它们成为等量。它们作为相等的东西的这种存在,必须不同于它们作为麦秆和麻布的存在。[823]它们不是作为麦秆和麻布相等,而是作为等价物相等。因此,等式的一方必须表现和等式的另一方相同的价值。因此,麦秆和麻布的价值必须既不是麦秆,也不是麻布,而是二者共同的同时又跟二者作为麦秆和麻布不同的东西。这是什么呢?贝利没有回答这个问题。他没有这样做,而是把政治经济学的所有范畴逐一论述,以便不断重复千篇一律的老调:价值是商品的交换比例,因而不是别的什么东西。

\begin{quote}{“如果某个物品的价值就是它的购买能力,那末就必须有供购买的东西。因此,价值除了仅仅表示两个物品作为可交换的商品相互间的比例之外,不表示任何肯定的或内在的东西。”(《对价值的本质、尺度和原因的批判研究》第4—5页)}\end{quote}

事实上贝利的全部智慧已经包含在这段话里了。“如果价值无非是购买能力”(一个绝妙的定义,因为“购买”不仅以价值,而且以价值的货币表现为前提),“那它就表示”等等。但是,我们首先要从贝利这段话中去掉荒谬地偷运进来的东西。“购买”就是把货币转化为商品。货币已经以价值和价值的进一步发展为前提。因此,首先必须抛开“购买”这个用语。否则就是用价值解释价值。我们必须用“同其他物品交换”代替“购买”。“必须有供购买的东西”,是一个完全多余的说明。如果“物品”作为使用价值被它的生产者消费,如果它不是仅仅占有其他物品的手段,不是“商品”,那自然就谈不上价值。

贝利首先谈的是“物品”。但是接着,“两个物品相互间的”比例在他那里变成“两个物品作为可交换的商品相互间的比例”。其实这里所谈的物品相互间只处于交换关系中或者说可交换的物品的关系中。正因为如此,它们才是“商品”,是和“物品”不同的东西。另一方面,“可交换的商品的比例”,或者是废话(因为“不可交换的物品”不是商品),或者是贝利先生自相矛盾。物品不应随便按照什么比例进行交换,它们应该作为商品进行交换,也就是说,应该作为可交换的商品,作为各自具有价值并应按照自己的等价程度相交换的物品,互相发生关系。这样,贝利就承认了:它们交换的比例,因而每种商品购买其他商品的能力,是由它的价值决定的,而不是这种能力决定它的价值,这种能力只是价值的结果。

总之,如果我们从贝利这段话里,去掉一切错误的,偷运进来的或没有意义的东西,这段话就是下面这样。

且慢!我们还必须去掉另外的陷阱和废话。在我们面前有两种用语:一种是“物品的交换能力”等等(因为“购买”一词在没有货币概念的情况下是不成立的,没有意义的),另一种是“一个物品同其他物品交换的比例”。如果“能力”应该表示某种和“比例”不同的东西,那就不能说,“交换能力”“仅仅表示比例”,等等。如果两个用语应该表示同一个东西,那末同一个东西用两个彼此迥然不同的用语来表示,只能产生混乱。一物对另一物的比例是两物间的比例,不能说这个比例是属于其中某一物。相反,一物的能力是该物内在的东西,尽管它这个内在的属性只能[824]表现在它对其他物的关系上。例如,引力是物本身的能力,虽然这种能力在没有东西可以吸引时是“潜在的”。在这里贝利试图把“物品”的价值说成是它内在的,而同时又只是作为“比例”才存在的东西。因此他先用“能力”这个词,然后又用“比例”这个词。

因此,贝利思想的精确表达是这样的:

\begin{quote}{“如果某个物品的价值就是它同其他物品交换的比例,那末,因此〈即因为“如果”〉,价值除了表示两个物品作为可交换的物品相互间的比例之外,不表示任何东西。”}\end{quote}

这个同义反复谁也不会否认。不过由此可以得出结论:物品的“价值”“不表示任何东西”。例如,1磅咖啡=4磅棉花。在这里,什么是1磅咖啡的价值呢?4磅棉花。什么是4磅棉花的价值呢?1磅咖啡。既然1磅咖啡的价值是4磅棉花,而4磅棉花的价值=1磅咖啡,所以很清楚,1磅咖啡的价值=1磅咖啡(因为4磅棉花=1磅咖啡)。A=B,B=A;所以A=A。因此,从这种说明中可以得出以下的结论:某个使用价值的价值等于该使用价值的一定量。因此,1磅咖啡的价值不过是1磅咖啡。如果1磅咖啡=4磅棉花,那很清楚,1磅咖啡>3磅棉花,1磅咖啡<5磅棉花。1磅咖啡>3磅棉花以及<5磅棉花,也表示咖啡和棉花之间的比例,就同1磅咖啡=4磅棉花表示这种比例完全一样。这个=并不比>或<表示更多的比例,而只是表示另一种比例。为什么正是等号(=)关系把咖啡的价值表现在棉花上,把棉花的价值表现在咖啡上?难道这个等号纯粹是由于一般地说这些量相互交换而得出来的吗?这个=只是表示交换这个事实吗?不能否认,如果咖啡随便按照什么比例和棉花交换,那末它们就是相互交换,如果商品之间的比例只由交换这个事实来确定,那末咖啡无论是和2磅、3磅、4磅或5磅棉花交换,咖啡的价值同样都表现在棉花上。但是比例这个词是指什么呢?咖啡本身决不包含什么“内在的、肯定的东西”来决定它按什么比例同棉花交换。贝利所说的比例,不是由咖啡内在的并和实际交换不同的某种属性决定的。这样,比例这个词有什么用呢?贝利所说的比例是什么呢?就是同一定量咖啡交换的棉花量。严格地说,贝利没有理由说,咖啡按照什么比例进行交换,而只能说,它现在或过去是按照什么比例进行了交换的。因为如果比例的确定先于交换,那末交换就由“比例”决定,而不是比例由交换决定了。因此,我们也必须把作为某种超越于咖啡和棉花之外并和它们脱离的东西的比例抛开。

[这样,上面引证的贝利的话就具有以下的形式:]

\begin{quote}{“如果某个物品的价值就是同它交换的另一物品的量,那末,因此,价值除了表示同它交换的另一物品的量之外,不表示任何东西。”}\end{quote}

一种作为商品的商品,只能把它的价值表现在其他商品上,因为对于它作为[单个]商品来说,一般劳动时间是不存在的。[于是,贝利认为,]如果一种商品的价值表现在另一种商品上,这种商品的价值就无非是它和另一种商品的等式。贝利不知厌倦地到处玩弄他的聪明(在他的表述中,这就是同义反复,因为他[实质上是]说:如果一种商品的价值无非是它和另一种商品的交换比例,那末价值就无非是这个比例),这就格外使读者厌倦。

他用下面这段话表明了他的哲学的深奥:

\begin{quote}{“如果某物没有另一物同它存在距离的关系,我们就无法谈某物的距离,同样,如果某种商品没有另一种商品[825]同它相比较,我们也就无法谈某种商品的价值。一物如果不同另一物发生关系,其本身就不能有距离,同样,一物如果不同另一物〈同商品的价值有关的社会劳动不是另一物吗?〉发生关系,其本身就不能有价值。”(同上,第5页)}\end{quote}

一物和另一物有距离,这个距离的确是该物和另一物之间的关系;但是距离同时又是跟两物之间的这种关系不同的东西。这是空间的一维,一定的长度,它除了能够表示我们的例子中两物的距离外,同样能够表示其他两物的距离。但是还不止于此。当我们说距离是两物之间的关系时,我们是以物本身的某种“内在的”东西,某种能使物互相存在距离的“属性”为前提的。语音A和桌子之间有什么距离呢?这个问题是没有意义的。当我们说两物的距离时,我们说的是它们空间位置的差异。因此,我们假定,它们二者都存在于空间,是空间的两个点,也就是说,我们把它们统一为一个范畴,都作为空间的存在物,并且只有在空间的观点上把它们统一以后,才能把它们作为空间的不同点加以区别。它们同属于空间,这是它们的统一体。\authornote{[XV—887}{关于贝利的荒谬观点,还要指出:

当他说A物和B物有距离时,他并不是比较它们,不是把它们统一为一个范畴,而是在空间上区别它们。据说,它们不是占有同一空间。但是,关于二者,他[实质上是]说:它们是空间的并且作为空间的物而不同。可见,他已预先把它们统一为一个范畴,使它们有了同一的统一体。而这里讨论的正是纳入统一范畴的问题。

如果我说,三角形A的面积等于平行四边形B的面积,意思不只是说,三角形的面积表现在平行四边形上,平行四边形的面积表现在三角形上。而且是说,如果三角形的高=h,底=b,则A=h·b/2,这是它本身具有的一种属性,平行四边形也具有这种属性,它同样=h·b/2。在这里,三角形和平行四边形作为面积,是同一的,是相等物,虽然它们作为三角形和平行四边形是不同的。为了使这些不同的东西相等,每一个都必须独自表现同一的统一体。如果几何学,象贝利先生的政治经济学一样,只满足于说,三角形和平行四边形相等是指三角形表现在平行四边形上,平行四边形表现在三角形上,那几何学就不可能有什么成就了。}[XV—887]]

但是互相可以交换的物品的这个统一体是什么呢?这种交换不是物品作为自然物互相保持的关系。它也不是物品作为自然物同人的需要的关系,因为不是物品的效用程度决定物品互相交换的量。那末使它们能按照一定比例互相交换的同一性是什么呢?它们作为什么才变得能够互相交换呢?

事实上,贝利在这整个问题上都只是追随《评政治经济学上若干用语的争论》的作者。

\begin{quote}{“它〈价值〉不能对相比较的物品中的一个物品来说变动了,而对另一个物品来说又没有变动。”(同上,第5页)}\end{quote}

这仍然只是说:一种商品的价值在另一种商品上的表现只能作为这种表现发生变化;而这种表现本身不是以一种商品,而是以两种商品为前提的。

贝利先生认为,如果谈的只是在互相交换中的两种商品,那末人们自然而然地就会发现他所谓的价值的纯粹相对性。蠢驴!似乎在两种商品互相交换,两种产品作为商品互相发生关系时,就用不着象在千万种商品互相交换时那样,说明它们的同一性在什么地方。此外,在只有两种产品存在的地方,产品决不会发展成商品,因此商品的交换价值也决不会发展。包含在产品I中的劳动就没有必要表现为社会劳动。因为产品不是作为生产者的直接消费品生产出来,而只是作为价值的承担者,也可以说,是作为支取所有社会劳动体现物的一定量的凭证生产出来,所以一切产品作为价值都必须具有一种和它作为使用价值的存在不同的存在形式。正是它们中包含的劳动作为社会劳动的这种发展,它们的价值的发展,决定了货币的形成,决定了商品必须互相表现为货币,即表现为交换价值的独立的存在形式;产品所以能这样,那只是因为它们把一种商品从商品总额中分离出来,所有商品都用这种分离出来的商品的使用价值来衡量自己的价值,从而把这种特殊商品中包含的劳动直接转化为一般的社会劳动。

贝利先生用他那种只抓住现象表面的古怪的思维方法,得出了相反的结论:价值概念所以会形成,——这个概念把价值由商品进行交换的纯粹量的关系,变为某种同这种关系无关的东西(他认为,这是把商品的价值变为某种绝对的东西,变为一种和商品分离的、烦琐的本质)——只是因为在商品之外存在货币,使我们习惯于不是从商品的相互关系来考察商品的价值,而是把商品的价值看成和第三物的关系,看成一种[826]和商品相互的直接关系不同的第三种关系。在贝利看来,不是产品作为价值的规定性,导致货币的形成,并表现为货币,而是货币的存在导致价值概念的虚构。下面一点历史地看是完全正确的:对价值的研究最初是根据商品作为价值的可以看得见的表现,根据货币,因此,探索价值规定就(错误地)表现为探索“价值不变”的商品,或探索作为“不变的价值尺度”的商品。因为贝利先生证明,货币作为价值的外在尺度——和价值表现——虽然具有可变的价值,却执行着它的任务,所以他认为这样就排除了价值概念——它不受商品价值量的可变性的影响——的问题,并且事实上根本用不着再去考虑价值是什么了。因为商品的价值在货币上——在特殊的第三种商品上——的表现并不排除这第三种商品的价值的变动,因为“不变的价值尺度”的问题消失了,所以价值范畴本身的问题也就消失了。贝利非常得意地用成百页的篇幅写出这么一些空空洞洞的废话。

在下面一些段落中,他喋喋不休地重复着同样的意思,其中一部分是逐字逐句从《评政治经济学上若干用语的争论》上抄来的。

\begin{quote}{“假定只有两种商品,它们按照它们所包含的劳动量的比例互相交换。如果……在后来一个时期生产商品A需要的劳动量增加一倍,而生产商品B所需要的劳动量不变,商品A的价值就会比商品B增加一倍……但是,虽然商品B是用和过去一样多的劳动量生产的,它的价值却不会保持不变,因为它只和商品A——根据假定,它是商品B可以相比较的唯一商品——的半数相交换。”(同上,第6页)“当我们谈两种商品之间的比例时,经常同其他商品〈不是把价值仅仅看成两种商品之间的比例〉或同货币比较,这就产生了关于价值是某种内在的和绝对的东西的观念。”(第8页)“我的主张是:如果所有商品都是在完全相同的条件下生产出来的,例如都只是由劳动生产的,那末,始终需要花费同量劳动的商品的价值,在其他各种商品的价值都发生变动时,不会保持不变。{即该商品的价值在其他商品上的表现不会保持不变。这是同义反复。}”(同上,第20—21页)“价值决不是内在的和绝对的东西。”(同上,第23页)“除了通过一定量的另一种商品,就无法表示或表现一种商品的价值。”(同上,第26页)}\end{quote}

(同样,除了通过一定量的音节,就无法“表示”或“表现”一种思想。贝利由此得出结论:思想不过是音节。)

\begin{quote}{“他们〈李嘉图及其信徒〉不是把价值看成两个物之间的比例,而是把价值看成由一定量劳动生产出来的肯定的成果。”(同上,第30页)“因为按照他们的学说,商品A和商品B的价值相互之间是作为生产它们的劳动量发生关系,或者说……是由生产它们的劳动量决定的,所以,看来他们作出了结论:商品A的价值,撇开同其他任何东西的关系,等于生产它的劳动量。最后这个论断无疑是没有任何意义的。”(同上,第31—32页)李嘉图及其信徒“把价值说成是一种一般的和独立的属性”。(第35页)“商品的价值必定是它在某物上的价值。”(同上)}\end{quote}

我们看到,为什么把价值限定在两种商品上,把价值看成两种商品之间的关系,对贝利来说是如此重要。但是这里发生了困难:

\begin{quote}{“因为每种商品的价值都表示该商品和另一商品的交换比例}\end{quote}

(在这里“交换[827]比例”是什么意思呢?为什么不是商品的“交换”呢?但同时在交换中应该表现一定的比例,而不只是交换的事实。因此,价值=交换比例),

\begin{quote}{所以根据它用来比较的商品,我们可以称它的价值为货币价值、谷物价值、呢绒价值;因此有千万种价值,有多少种商品,就有多少种价值,它们都同样是现实的,又都同样是名义的。”(同上,第39页)}\end{quote}

原来如此!价值=价格。在它们之间没有区别。在货币价格和其他任何价格表现之间也没有“内在的”区别,虽然实际上正是货币价格,而不是呢绒价格等等,表现商品的名义价值,一般价值。

但是,虽然商品有千万种价值,或者说千万种价格,有多少种商品,就有多少种价值,这千万种表现都始终表示同一价值。最好的证明就是:所有这些不同表现都是等价物,它们不仅在表现上可以互相代替,而且在交换本身中也互相代替。我们谈到其价格的商品的这种关系可以表现为所有不同商品的千万种不同的“交换比例”,然而这里始终表现同一关系。因此,这种始终同一的关系和它的上千种不同的表现是不一样的,或者说,价值和价格是不一样的,价格只是价值的表现:货币价格是价值的一般表现,其他各种价格是特殊表现。但是,甚至这个简单的结论,贝利也没有得出。在这里,不是李嘉图是虚构家,而是贝利是拜物教徒,因为他即使没有把价值看成(被孤立地考察的)个别物的属性,毕竟把价值看成物和物之间的关系,而实际上价值只不过是人和人之间的关系、社会关系在物上的表现,它的物的表现,——人们同他们的相互生产活动的关系。

\tsectionnonum{[(β)劳动价值和资本家利润问题上的混乱。贝利把内在的价值尺度同价值在商品或货币上的表现混淆起来]}

[关于劳动的价值,贝利说:]

\begin{quote}{“李嘉图先生相当机智地避开了一个困难,这个困难乍看起来似乎会推翻他的关于价值取决于在生产中所使用的劳动量的学说。如果严格地坚持这个原则,就会得出结论说,劳动的价值取决于在劳动的生产中所使用的劳动量。这显然是荒谬的。因此,李嘉图先生用一个巧妙的手法,使劳动的价值取决于生产工资所需要的劳动量;或者用他自己的话来说,劳动的价值应当由生产工资所必需的劳动量来估量;他这里指的是为生产付给工人的货币或商品所必需的劳动量。那我们同样也可以说,呢绒的价值不应当由生产呢绒所花费的劳动量来估量,而应当由生产呢绒所换得的银所花费的劳动量来估量。”(同上,第50—51页)}\end{quote}

以上所述,对李嘉图关于资本直接和劳动相交换而不是和劳动能力相交换的错误观念来说,是正确的。这也就是我们以前在别的形式上听到的\authornote{见本册第117页。——编者注}那种指责。仅此而已。对劳动能力来说,贝利的类比是不适用的。他不应该拿呢绒,而应该拿一种生物产品例如羊肉来和活的劳动能力比较。生产家畜所必需的劳动,除了照料家畜所花费的劳动以及生产其生活资料所花费的劳动以外,不应指家畜本身花费在消费行为即饮食行为上,一句话,花费在消化这些产品或生活资料的行为上的“劳动”。劳动能力的情况也完全一样。生产劳动能力所花费的劳动是什么呢?除了在培养劳动能力、教育、学徒上花费的劳动——这在谈到非熟练劳动时几乎是用不着考虑的——以外,劳动能力的再生产所花费的,不过是工人消费的生活资料的再生产所花费的劳动。生活资料的消化并不是“劳动”,[828]正如呢绒中包含的劳动,除了织布工人的劳动和羊毛、染料等等包含的劳动以外,并不是还包含羊毛本身的化学作用或物理作用——由于这种作用,羊毛象工人或家畜吸收食物那样吸收染料等等。

其次,贝利企图推翻李嘉图关于劳动的价值同利润成反比的规律。而且他企图推翻的恰恰是这个规律的正确部分。问题在于,他和李嘉图一样,把剩余价值和利润等同起来。他没有提到这个规律的唯一可能的例外,那就是:工作日延长,工人和资本家均等地分得工作日延长的成果。但是即使在这种情况下,因为劳动力[workingpower]的价值将更快地(在更少的年份内)被消费掉,剩余价值也会靠牺牲工人的生命增长起来,工人的劳动力同它给资本家提供的剩余价值相比就贬值了。

贝利的论据极为肤浅。他是从他的价值概念出发的。在贝利看来,商品的价值是商品价值在一定量的其他使用价值(其他商品的使用价值)上的表现。因此,劳动的价值等于劳动所交换的其他商品(使用价值)量。{商品A的交换价值怎么能表现在商品B的使用价值上这个实际问题,他根本没有考虑。}这样一来,只要工人得到同量商品,劳动的价值就仍然不变,因为它仍旧表现在同量的其他有用物上。利润则表示对资本的比例,或者说,也是对总产品的比例。但是,虽然在劳动生产率提高时资本家所得的总产品的比例增大了,工人所得的产品份额却可能仍旧不变。既然资本家得到的东西的价值不是由比例决定,而是由“这一价值在其他商品上的表现”决定,那就不能理解,贝利在谈到资本时怎么突然得出一个比例,这个比例对资本家有什么用处。

事实上,这就是我们在考察马尔萨斯时已经谈到的那种妙论\authornote{见本册第27—29页。——编者注}。工资等于一定量的使用价值。而利润是价值的比例(但是贝利不得不回避这种说法)。如果我按使用价值来计量工资,而按交换价值来计量利润,那就很明显,在二者之间既不存在反比,也根本不存在任何比例,因为在这种情况下,我是拿两个不能相比的量,两个没有共同基础的物来互相比较了。

但是,贝利在这里所说的关于劳动价值的观点,按照他的原则,也适用于其他任何商品的价值。任何一种商品的价值无非是同它交换的其他物的一定量。如果我用1镑换得20磅棉纱,那末,在贝利看来,即使用来生产1磅棉纱的劳动这一次比另一次多一倍,这1镑的价值也始终是同一的,就是说始终得到支付的。一个最普通的商人也不会相信,如果在物价昂贵时和产品丰富时都用1镑买得1夸特谷物,他这1镑换得了相同的价值。在这里,任何价值概念都消失了。剩下的只是一个没有解释也无法解释的事实:若干量商品A和若干量商品B按照随便什么样的比例相交换。不管这个比例怎样,它总是表示等价物。这样,连贝利关于“表现在商品B上的商品A的价值”这一说法本身,也失掉了任何意义。如果商品A的价值表现在商品B上,那末就必须假定,同一价值一次表现在商品A上,另一次表现在商品B上,因而A的价值当它表现在B上时,仍和原先一样。但是照贝利看来,不存在可以表现在B上的A的价值,因为除了这种表现之外,无论A或B都没有价值。照贝利的看法,表现在B上的A的价值和表现在C上的A的价值,必定是完全不同的,就好象B和C是不同的一样。我们在这里看到的,不是在两种表现上等同的同一价值,而是A的两种比例,这两种比例彼此没有任何共同之处,而且要说它们是等价表现,那是荒谬的。

\begin{quote}{[829]“劳动价值的提高或降低,意味着用以交换劳动的商品量的增加或减少。”(同上,第62页)}\end{quote}

真是胡说![从贝利的观点看来]劳动的价值或其他任何物的价值都不可能提高或降低。我今天用1A换得3B,明天换得6B,后天换得2B。但是在这一切情况下,[照贝利看来]A的价值都无非是A所换得的B量。它以前是3B,现在是6B。贝利怎么能说A的价值提高或降低呢?表现在3B上的A,和表现在6B或2B上的A,有不同的价值。不过,在这种情况下,就不是同一个A在同一个时间换得3B或2B或6B了。同一个A在同一个时间总是表现在同量的B上。只有就不同的时间而言,才能说A的价值变动了。但是A只能和“同时存在的”商品相交换,并且只有和其他商品相交换这个事实(而不只是交换的可能性)[照贝利的看法]才使A成为价值。只有现实的“交换比例”形成A的价值,而现实的“交换比例”当然只有对同一个时间的同一个A才能发生。因此贝利宣称,把不同时期商品的价值加以比较,是荒谬的。\endnote{[赛·贝利]《对价值的本质、尺度和原因的批判研究》1825年伦敦版第71—93页;参看本卷第2册第565页,第3册第165—167、175—176页。——第163页。}但是,由此他本来应当宣称,价值的提高或降低也是荒谬的(既然商品在一个时间的价值同它在另一个时间的价值不能比较,价值就不可能有提高或降低),——因而“劳动价值的提高或降低”也是荒谬的。

\begin{quote}{“劳动是一种可交换的物,即在交换中支配其他物的物;‘利润’这一用语却只意味着商品的份额或比例,而不是一种可以同其他物品相交换的物品。我们问工资是否提高了,我们的意思是:一定量的劳动是否换得比以前更多的其他物。}\end{quote}

(因此,当谷物贵了,劳动的价值就是降低了,因为它换得的谷物少了;另一方面,如果与此同时呢绒贱了,劳动的价值同时就是提高了,因为它换得的呢绒多了。这样,劳动的价值在同一个时间又提高又降低;它的价值的两种表现——在谷物上的和在呢绒上的——不是等同的,不是等价的,因为它的提高了的价值不可能等于它的降低了的价值);

\begin{quote}{但是我们问利润是否提高了,我们指的是……资本家的收入对所使用的资本是否有更大的比例。”(同上,第62—63页)“劳动的价值不单取决于总产品中为换得工人的劳动而给予工人的那个份额,而且也取决于劳动生产率。”(同上,第63—64页)“劳动的价值提高时利润必定下降的论点,只有在这种提高不是由劳动生产力的增长引起的情况下,才是正确的。”(同上,第64页)“如果劳动生产力增长,就是说,如果同一劳动在同一时间生产更多的商品,那末,劳动的价值可能提高而利润不降低;后者甚至还可能提高。”(同上,第66页)}\end{quote}

(按照这个观点,对于其他任何商品也可以说,它的价值的提高不会引起和它交换的其他商品的价值的降低,甚至还会引起对方价值的提高。例如,假定同一劳动以前生产1夸特谷物,现在生产3夸特。以前生产1夸特花费1镑,现在生产3夸特也花费1镑。如果现在2夸特和1镑交换,货币的价值就提高了,因为它现在表现在2夸特上,不是表现在1夸特上。这样,谷物的买者就用他的货币换得更大的价值。但是,谷物的卖者,把他只花费2/3镑的东西卖1镑,赚了1/3镑。结果,他的谷物的价值就在谷物的货币价格降低的同时提高了。)

\begin{quote}{[830]“不管6个工人劳动的产品是多少,不管它是100夸特谷物还是200夸特或300夸特,只要资本家在产品中所占的比例始终是四分之一,这四分之一表现在劳动上就始终是不变的。”}\end{quote}

(归工人所得的3/4产品,如果把它们表现在劳动上,也可以这么说。)

\begin{quote}{“如果产品是100夸特,就会有75夸特付给6个工人,因而归资本家所得的25夸特将支配2个工人的劳动。”}\end{quote}

(而付给工人的75夸特将支配6个工人的劳动。)

\begin{quote}{“如果产品是300夸特,6个工人就会得到225夸特,归资本家所得的75夸特,仍然将仅仅支配2个工人的劳动,不会更多。”}\end{quote}

(同样,归6个工人所得的225夸特仍然将仅仅支配6个工人的劳动,不会更多。)(既然如此,为什么万能的贝利不许李嘉图把工人得到的产品份额,也象资本家得到的产品份额那样表现在劳动上,并且把表现在劳动上的这两份产品的价值互相比较呢?)

\begin{quote}{“归资本家所得的比例的这种增加,就是表现在劳动上的利润的价值的增长,}\end{quote}

(既然“利润意味着……不是一种可以同其他物品相交换的物品”(见上述),因而不是意味着“价值”,贝利怎么能说利润的价值和利润的价值的增长呢?另一方面,归工人所得的比例不减少,归资本家所得的比例难道能够增加吗?)

\begin{quote}{或者换句话说,也就是利润支配劳动的能力的增加。”(同上,第69页)}\end{quote}

(资本家占有别人劳动的能力的这种增加,和工人占有自己劳动的能力的减少,岂不是正好一致吗?)

\begin{quote}{“对于利润和劳动的价值同时增长的学说,如果有人反驳说,生产出来的商品是资本家和工人能够取得他们的报酬的唯一源泉,从而一方得到的必然是另一方失掉的,那末,对这种反驳的回答是明确的。当产品量保持不变时,这种反驳不可否认是正确的;但是同样不可否认,如果产品增加一倍,即使一方所得的比例减少而另一方所得的比例增加,归双方所得的产品份额也可能都增加。”}\end{quote}

(这正好是李嘉图所说的。双方的比例是不能同时增加的;即使归双方所得的产品份额同时增加,它们也不能按同一比例增加,因为不然的话,份额和比例就成了一回事了。一方比例的增加,只能靠另一方比例的减少。\endnote{在整个这一论断中,所谓归工人所有的(以及归资本家所有的)产品“份额”,是指新加劳动物化在其中的那部分产品的实物单位量;所谓“比例”,是指这种产品归一方或他方的百分比。例如,如果工人的新加劳动物化在100实物单位的产品中,其中工人所得的部分占60%,资本家所得的部分占40%,那末,在产品量增加一倍(由于劳动生产率的增长)而工人和资本家按原来的比例分配产品时,工人所得的“份额”就会增加60实物单位,而资本家所得的“份额”只增加40实物单位。但是,如果这时资本家的部分从40%增加到48%,那末,工人的部分就会从60%减少到52%,虽然他们所得的“份额”还是会增加44实物单位(资本家所得的“份额”同时增加56实物单位)。——第165页。}贝利先生把归工人所得的产品份额叫作“工资的价值”,而把资本家所得的比例叫作“利润”的价值,换句话说,他认为同一商品有两个价值——一个在工人手里,另一个在资本家手里,这是他自己的胡说。)

\begin{quote}{“当产品量保持不变时,这种反驳不可否认是正确的;但是同样不可否认,如果产品增加一倍,即使一方所得的比例减少而另一方所得的比例增加,归双方所得的产品份额也可能都增加。而正是归工人所得的产品份额的增加,形成工人劳动价值的增长}\end{quote}

(因为这里所说的价值是指一定量的物品),

\begin{quote}{然而,正是归资本家所得的比例的增加,形成资本家的利润的增长}\end{quote}

(因为这里所说的价值,是指不按量而按所花费的劳动来估量的同一些物品)。

\begin{quote}{由此}\end{quote}

(就是说,由荒谬的双重尺度:一次是物品,另一次是同一些物品的价值)

\begin{quote}{可以十分明确地得出结论说,关于二者同时增加的假定一点也没有矛盾的地方。”(同上,第70页)}\end{quote}

这个针对着李嘉图的荒谬论断完全没有击中[831]目标,因为李嘉图只是断言,两个份额的价值的提高和下降必定成反比。贝利却只是反来复去地说:价值是同某一物品相交换的物品量。他在考察利润时不可避免地会陷入困境,因为这里是资本的价值同产品的价值相比较。于是他就寻找一条出路:他在这里把价值理解为物品表现在劳动上的价值(照马尔萨斯的样子)。

\begin{quote}{“价值是同时存在的各商品之间的比例,因为只有这样的商品能够互相交换;而如果我们把商品在一个时间的价值同它在另一个时间的价值相比较,那末所比较的就只是该商品在这些不同时间内对其他某种商品的比例。”(同上,第72页)}\end{quote}

因此,如前面所说的,既没有价值的提高,也没有价值的降低,因为价值的提高和降低总是意味着商品在一个时间的价值同它在另一个时间的价值相比较。同样,商品既不能低于它的价值,也不能高于它的价值出卖,因为它的价值就是它卖得的东西。价值和市场价格是等同的。实质上,甚至不能说“同时存在的”商品,现在的价值,而只能说过去的价值。一夸特小麦的价值是什么呢?就是它昨天卖得的一镑。因为它的价值只能是它所换得的东西,在它没有被交换的时候,它“对货币的比例”不过是想象的比例。但是交换一经完成,我们持有的就不是一夸特谷物,而是一镑,因而已经不能再说这一夸特谷物的价值了。贝利在谈到把不同时期的价值相比较时,只是指,比如说,对十八世纪和十六世纪的商品的不同价值的学术研究,这里就产生了一个困难,因为价值的同一货币表现,由于货币本身价值的变动,在不同时间表示不同的价值。这里的困难就在于把货币价格还原为价值。但是贝利真是一头蠢驴!在资本的流通过程或再生产过程中,把一个时期的价值同另一个时期的价值相比较,难道不正是生产本身赖以进行的经常业务吗?

贝利先生根本不懂得“商品价值决定于劳动时间”和“商品价值决定于劳动价值”这两种说法是什么意思。他根本不懂得这两者之间的差别。

\begin{quote}{“请不要以为我不是主张商品价值相互之比等于生产这些商品所必要的劳动量相互之比,就是主张商品价值相互之比等于劳动价值相互之比。我只是主张:如果前一种说法是正确的,后一种说法也就不会是错误的。”(同上,第92页)}\end{quote}

各种商品的价值由一种商品的价值决定(如果它们由“劳动价值”决定,那末,它们就是由另一种商品决定;因为劳动价值是以劳动作为商品为前提的),和各种商品的价值由一种没有价值、本身不是商品而是价值实体并且最先使产品成为商品的第三物决定,——在贝利看来是一回事。可是,在第一种情况下,谈的是商品的一种价值尺度,实际上也就是货币,就是其他商品借以表现自己价值的商品。要使这一点成为可能,必须已经有各种商品的价值存在作为前提。无论是计量的商品还是被计量的商品,在第三物上必须已经是同一的。相反,在第二种情况下,最先确定了这种同一性本身,然后它表现在价格上,表现在货币价格或其他任何价格上。

贝利把“不变的价值尺度”和对内在的价值尺度的寻求,也就是和价值概念本身等同起来。只要把这两个东西混为一谈,寻求“不变的价值尺度”甚至就成为一种理性的本能。而可变性正是价值的特点。对“不变的东西”的寻求表达了这样的思想:内在的价值尺度本身不能也是商品,也是价值,相反,它必须是某种构成价值,因而形成内在的价值尺度的东西。贝利证明说,[832]商品价值可以获得货币表现,而且如果商品的价值比例是既定的,一切商品都可以把自己的价值表现在一种商品上,虽然这种商品的价值也会变动。但是这种商品的价值不管怎么变动,它在同一时间内对其他商品来说总是不变的,因为它是对一切商品同时发生变动的。贝利由此得出结论说,不需要什么商品之间的价值比例,因而也用不着去寻找它。因为他发现它已在货币表现上反映出来,所以他就用不着去“了解”这种表现为什么是可能的,它是怎么决定的,它在事实上表示什么。

一般说来,以上所述,既可以用来反驳马尔萨斯,也可以用来反驳贝利,因为贝利认为,无论以劳动量还是以劳动价值作为价值尺度,涉及的是同一个问题,同一回事。其实,在后一种情况下,价值已经作为前提,问题是要找到衡量这些已经作为前提的价值的尺度,找到它们的外在尺度,它们作为价值的表现。在前一种情况下,研究的是价值本身的发生和内在性质。在后一种情况下,研究的是商品到货币的发展,或交换价值在商品交换过程中取得的形式。在前一种情况下,研究的对象是价值,它不依这种表现为转移,相反地是这种表现的前提。贝利和其他蠢驴都认为,决定商品的价值就是指找到商品价值的货币表现,找到商品价值的外在尺度。但是,其他蠢驴出于理性的本能,说在这种情况下,这个尺度必须具有不变的价值,即在实际上它本身必须处于价值的范畴之外。贝利则说,这里没有什么需要进一步考虑的,因为我们在实践中已经找到了现成的价值表现,这种表现本身具有并且可以具有可变的价值而不损害它的职能。

[问题的一般提法就是这样。]特别是贝利本人在前面曾经告诉我们,6个工人劳动的产品,即同一劳动量的产品,可以是100夸特或200夸特或300夸特谷物,而“劳动价值”,在贝利看来,只是这6个工人从100、200或300夸特中得到的份额。这个份额可以是每个工人50、60或70夸特\endnote{“每一个工人50、60或70夸特”是随便举的数字,如果拿符合贝利上述例子(第163—165页)的数字来代替,那就是:“每一个工人12+(1/2)、25或37+(1/2)夸特”。——169页。}。这样,就是照贝利本人所举的例子看来,劳动量和这个劳动量的价值也是两种极不相同的表现。怎么能认为,价值表现在劳动量上,同表现在与劳动量根本不同的劳动量的价值上是一样的呢?如果同样的劳动以前提供3夸特谷物,现在提供1夸特,而同样的劳动以前提供20码麻布(或3夸特谷物),现在仍旧提供20码麻布,那末,用劳动时间来计量,1夸特谷物现在就等于20码麻布,或20码麻布等于1夸特谷物,而3夸特谷物就等于60码麻布而不是等于20码麻布。因此,1夸特谷物的价值和1码麻布的价值,相对地变动了。但是它们按“劳动价值”来说却丝毫没有变动,因为1夸特谷物和20码麻布仍旧是同以前一样的使用价值。并且很可能1夸特谷物现在支配的劳动量不比以前多。

如果拿单个商品来说,那末,贝利的论断是毫无意义的。如果生产一双长靴所必要的劳动时间减少十分之九,那末一双长靴的价值就减少十分之九;表现在其他一切商品上——生产这些商品所必要的劳动保持不变或不按同一程度减少——也相对地减少。但是劳动价值——例如制靴业以及其他一切生产部门的日工资——可能保持不变,甚至可能提高。现在在一双长靴中包含的劳动少了,因而包含的有酬劳动也少了。但是当谈到劳动价值时,这并不是说,对一小时劳动,一般地说对较小量的劳动,要比对较大量的劳动,支付较少的报酬。贝利的命题只有对资本的总产品来说才会有某种意义。假定200双长靴和以前100双长靴一样是同量资本(和同量劳动)的产品。在这种情况下,200双长靴的价值就和以前100双长靴的价值一样。于是这里可以说,200双长靴对1000码麻布(假定这是200镑资本的产品)之比,等于这两笔资本所推动的劳动的价值之比。在什么意义上呢?难道在一双长靴对一码麻布之比也[833]可以这样说的意义上吗?

劳动价值是商品包含的劳动时间中由工人自己占有的那一部分;是产品中体现属于工人自己的劳动时间的那一部分。所以,如果商品的全部价值分解为有酬劳动时间和无酬劳动时间,并且无酬劳动时间对有酬劳动时间之比是同一的,就是说,如果一切商品中的剩余价值在总价值中占有同一比例,那末很明显,既然各商品相互之比等于它们中包含的总劳动量之比,各商品相互之比同时必定等于这些总劳动量中相同比例部分之比,因而也必定等于一种商品中的有酬劳动时间对另一种商品中的有酬劳动时间之比。

W∶W′=GA∶G′A′,这里,GA表示总劳动时间。GA/x=W中的有酬劳动时间,G′A′/x=W′中的有酬劳动时间,因为我们假定,这两种商品中的有酬劳动时间在总劳动时间中占有相同的比例部分。

W∶W′=GA∶G′A′;

GA∶G′A′=GA/x:G′A′/x。

所以,

W∶W′=GA/x∶G′A′/x,

即各商品相互之比等于它们包含的有酬劳动时间之比,或者说,等于它们包含的劳动的价值之比。

但是在这种情况下,劳动价值并不是象贝利所希望的那样来决定,而是它本身由[商品中包含的]劳动时间决定。

其次,——撇开价值转化为生产价格不谈而只考察价值,——各个资本是由不同比例的可变资本部分和不变资本部分构成的。所以,在考察价值时可以看出,不同商品包含的剩余价值在量上是不相同的,或者说,有酬劳动在总预付劳动中所占的比例对各种商品来说是不相同的。

总之,工资——或者说,劳动价值——在这里是商品价值的指数,并不是因为它是价值,并不是因为工资会提高或降低,而是因为某种商品包含的表现在工资上的有酬劳动量,是该商品所包含的劳动总量的指数(与其他商品相比)。

一句话,全部问题归结为:既然商品价值相互之比等于A∶A′(即商品中包含的劳动时间量之比),那末,它们相互之比也等于A/x∶A′/x,即商品中包含的有酬劳动时间量之比,——如果一切商品中有酬劳动时间对无酬劳动时间之比是相同的,就是说,如果不管总劳动时间是多少,有酬劳动时间总是等于总劳动时间除以x。但是,这个“如果”是不符合实际情况的。即使假定各生产部门工人的剩余劳动时间是相同的,各生产部门中有酬劳动时间对所耗费的劳动时间之比也是不同的,因为耗费的直接劳动对耗费的积累劳动之比不同。比如有两笔资本:50v+50c和10v+90c。假定在这两种情况下无酬劳动都等于有酬直接劳动的十分之一。这样,在第一种商品中包含的价值是105,在第二种商品中是101。有酬劳动时间在第一种情况下占预付劳动的1/2,在第二种情况下只占1/10。

[834]贝利说:

\begin{quote}{“如果商品相互之比等于生产它们的劳动量之比,它们相互之比也必定等于这种劳动的价值之比;因为否则就必然含有这样的意思:两种商品A和B可能在价值上相等,虽然在一种商品上所耗费的劳动的价值比另一种商品上所耗费的劳动的价值大或小;或者说,商品A和B在价值上可能不等,虽然它们各自耗费的劳动在价值上是相等的。但是,由价值相等的劳动生产出来的两种商品在价值上的这个差别,就会和公认的利润的均等相矛盾,而利润的均等是李嘉图先生和其他作者一致承认的。”(同上,第79—80页)}\end{quote}

在最后一句话里,贝利无意中摸索到对李嘉图的正确的反驳,李嘉图是直接把利润和剩余价值等同起来,把价值和费用价格等同起来的。这个反驳的正确表述就是:如果商品按自己的价值出卖,它们就提供不同的利润,因为这时利润等于商品本身包含的剩余价值。这个反驳是正确的。但是它不是反对价值理论,而是反对李嘉图在应用这个理论方面的错误。

不过,在上面引用的话里,贝利本人对问题的理解多么不正确,可以从下面这段话看出来:

\begin{quote}{“相反,李嘉图认为,‘劳动在价值上可以提高或降低而不影响商品的价值’。这个论断和另一个论断显然有很大的不同。它是否正确,实际上要看另一个论断是否错误,或者说,要看相反的论断如何。”(同上,第81页)}\end{quote}

这个蠢驴自己以前说过,同量劳动的结果可以是100、200或300夸特谷物。这一点决定一夸特谷物对其他商品的比例,而不管劳动价值如何变动,就是说,不管100、200或300夸特中归工人自己所得的是多少。假如这个蠢驴要在某种程度上保持前后一贯,他就应该说:劳动价值可以提高或降低,但是商品价值相互之比仍然等于劳动价值之比,因为——按照错误的假定——工资的提高或降低是普遍的,而且工资的价值在所耗费的劳动总量中始终占相同的比例部分。

\tsectionnonum{[(γ)把价值同价格混淆起来。贝利的主观主义观点。关于费用价格和价值的差额问题]}

[贝利说:]

\begin{quote}{“表现商品价值的能力同商品价值的不变性没有关系}\end{quote}

(确实没有关系!但是它同表现价值以前首先找出价值大有关系;同找出彼此极不相同的使用价值怎样归入价值这个共同的范畴和共同的名称,从而使一物的价值可以由另一物表现,大有关系},

\begin{quote}{无论是把商品互相比较,还是把它们同所使用的尺度比较,都是如此。同样,把这些价值表现加以比较的能力也同价值的不变性没有关系。”}\end{quote}

{如果不同商品的价值都表现在同一的第三种商品上(不管后者的价值如何变动),那末,把这些已经具有共同名称的表现加以比较,当然是很容易的。}

\begin{quote}{“A值4B还是6B}\end{quote}

{困难是怎样使A和若干数量的B相等,这只有当A和B有一个共同的统一体,或者说,A和B是同一个统一体的不同体现物时,才有可能。如果所有商品都必须表现在金上,表现在货币上,困难仍然一样。在金和其他每种商品之间必须有一个共同的统一体},

\begin{quote}{以及C值8B还是12B,这对于把A和C的价值表现在B上的能力是无关紧要的,而且——既然A和C的价值都表现在第三种商品B上——对于把A和C的价值加以比较的能力,当然也是无关紧要的。”(同上,第104—105页)}\end{quote}

但是,A怎样表现在B或C上呢?必须把A、B、C看成某种和它们作为物、产品、使用价值不同的东西,才能使“它们”互相表现,换言之,才能把它们当作同一的统一体的等价表现。A=4B。因而,A的价值表现在4B上,而4B的价值表现在A上,结果等式的两方表现同一的东西。它们是等价物。它们两者都是价值的相等的表现。如果它们是不相等的表现,如A>4B或A<4B,也是一样。在所有这些情况下,只要[835]它们是价值,它们就只是在量上不同或相等,但是始终是同一个质的量。困难在于找到这个质。

\begin{quote}{“在对价值进行计量的过程中,必要条件是使被计量的商品具有共同的名称}\end{quote}

{例如,为了把三角形和其他一切多角形加以比较,只须把多角形化为三角形,把它们表现在三角形上。但是要这样做,三角形和多角形事实上就被看成等同的东西,看成同一个东西——空间——的不同表现形式},

\begin{quote}{这在任何时候都可以同样轻而易举地做到;或者更确切地说,这是现成的,因为这就是记录下来的商品价格,或者说商品对货币的价值比例。”(同上,第112页)“决定价值也就是表现价值。”(同上,第152页)}\end{quote}

这里我们把这个家伙弄清了。我们看到价值已用价格来计量和表现了。因此,[贝利认为]我们也就可以满足于不知道什么是价值。贝利把价值尺度到货币的发展,进而把货币作为价格标准的发展,同价值作为商品交换的内在尺度的发展中价值概念本身的确立混为一谈。他正确地认为,这种货币没有必要成为价值不变的商品;但是他由此得出结论:独立于商品本身之外、与商品本身不同的价值规定是没有必要的。

只要把商品的价值作为商品的共同的统一体,商品相对价值的计量和这种价值的表现就一致了。但是,在我们找到和商品的直接存在不同的统一体以前,我们将看不到这种表现。

就拿贝利关于物品A和B之间的距离的例子\authornote{见本册第154—155页。——编者注}来说,也可以看出这一点。当我们说它们之间有距离时,我们已经假定,它们二者是空间的点(或线)。如果把它们看成点,而且是同一线上的点,那它们的距离就可以用寸、尺等表示。A和B这两种商品的统一体,乍看起来,就是它们的可交换性。它们是“可交换的”物品。作为“可交换的”物品,它们是同一名称的量。但是,“它们”作为“可交换的”物品的存在必须和它们作为使用价值的存在不同。这种存在是什么呢?

货币本身已经是价值的表现,是以价值为前提的。货币作为价格标准,又已经以商品转化(理论上)为货币作为前提。如果所有商品的价值都表现为货币价格,我就可以比较它们;事实上它们已经被比较了。但是要把价值表现为价格,商品的价值必须先表现为货币。货币只不过是商品价值在流通过程中借以表现的形式。但是我怎样才能把x棉花表现在y货币上呢?这个问题可以归结为:一般地说,我怎样才能把一种商品表现在另一种商品上,或者说,把商品表现为等价物?只有离开一种商品在另一种商品上的表现去分析价值,才能回答这个问题。

\begin{quote}{“认为……不同时期的商品之间可以存在价值比例,是错误的,按照事物的性质,也是不可能的;而既然不存在这种比例,也就不能进行计量。”(同上,第113页)}\end{quote}

这种谬论前面已经有过\authornote{见本册第163、166—167页。——编者注}。在货币执行支付手段职能时,就已经存在“不同时期的商品之间的价值比例”。整个流通过程都是不同时期商品价值不断比较的过程。

\begin{quote}{“如果它〈货币〉不是不同时期商品进行比较的好手段,那末,这就意味着:它不能在不存在任何可以由它执行的职能的地方执行职能。”(同上,第118页)}\end{quote}

作为支付手段和贮藏货币,货币就是要执行这种比较不同时期商品价值的职能。

事实上,这全部谬论的秘密由下面一段话透露出来了,这段话完全是从《评政治经济学上若干用语的争论》的作者那里抄来的\authornote{同上,第138页。——编者注},它使我相信,贝利作为剽窃者利用了被他小心翼翼地隐瞒起来的《用语的争论》。

\begin{quote}{[836]“财富是人的属性,价值是商品的属性。人或共同体是富的;珍珠或金刚石是很有价值的。”(《对价值的本质、尺度和原因的批判研究》第165页)}\end{quote}

珍珠或金刚石所以有价值,是因为它们是珍珠或金刚石,也就是由于它们的属性,由于对人有使用价值,——也就是由于它们是财富。但是在珍珠或金刚石中没有什么东西可以确定它们和其他[使用价值]之间的交换比例。

贝利突然成了高深莫测的哲学家:

\begin{quote}{“在作为价值原因的劳动和作为价值尺度的劳动之间,总之,在价值的原因和尺度之间,是有区别的。”(第170页及以下各页)}\end{quote}

的确,在“价值尺度”(指货币而言)和“价值原因”之间有非常显著的区别(而且被贝利忽略了)。价值的“原因”把使用价值转化为价值。外在的价值尺度已经以价值的存在为前提。例如,金所以能成为棉花的价值尺度,只是因为金和棉花作为价值具有和二者都不相同的统一体。价值的“原因”是价值的实体,因而也是内在的价值尺度。

\begin{quote}{“一切在商品交换中间接或直接对人的意识起决定性影响的……情况,都可以看作价值的原因。”(第182—183页)}\end{quote}

实际上,这不过是说:那些使卖者或者使买者和卖者把某种东西看成商品的价值或等价物的情况,是商品价值的原因或两种商品等价的原因。把决定商品价值的“情况”,说成影响交换者的“意识”而本身又存在于交换者的意识中(也许不存在,也许以歪曲的形式存在),这样,就根本不能进一步认识它。

这些(虽然影响意识、但是独立于意识之外的)迫使生产者把他们的产品作为商品出卖的情况,——这些使一种社会生产形式区别于另一种社会生产形式的情况,——赋予他们的产品(也给他们的意识)一种与使用价值无关的交换价值。这些产品的生产者的“意识”可以完全不知道他们的商品的价值实际上是怎样决定的,或者说,是什么东西使他们的产品成为价值的,——对于意识来说,这甚至可能不存在。产品的生产者被置于决定他们的意识的条件下,而他们自己却不一定知道。每个人都可以把货币作为货币使用,而不知道货币是怎么一回事。经济范畴反映在意识中是大大经过歪曲的。贝利所以把问题转入意识领域,是因为他在理论上走进了死胡同。

贝利不说,他自己所理解的“价值”(或“价值的原因”)是什么,而对我们说,这是买者和卖者在交换活动中所想象的东西。

但是,实际上作为这个貌似哲理的词句的基础的是:

(1)市场价格是由表现在供求关系中的不同情况决定的,而这些情况本身影响市场上的交易者的“意识”。一个非常重要的发现!

(2)在商品价值转化为费用价格时要考虑到作为“补偿理由”影响意识或在意识中出现的“不同的情况”。但是,所有这些补偿理由只影响作为资本家的资本家的意识,并且它们是由资本主义生产本身的性质产生的,而不是由买者和卖者的主观理解产生的。在买者和卖者的头脑中,它们毋宁说是作为不言而喻的“永恒真理”存在的。

贝利和他的前辈一样,抓住李嘉图把价值和费用价格混淆起来这一点来证明,价值不是由劳动决定的,因为费用价格同价值相偏离。用这一点反对李嘉图的[把价值和费用价格]等同是完全正确的,但用来反对[价值决定于劳动的]论点本身则是不正确的。

在这方面,贝利首先引证了李嘉图本人谈到的商品相对价值[837]由于劳动价值提高而变动的论点\authornote{见本卷第2册第196—221页。——编者注}。其次,他引证了“时间的影响”(在不延长劳动时间情况下生产时间的差别),即已经引起穆勒疑问的同一情况\authornote{见本册第89—91页。——编者注}。贝利没有看到真正的普遍的矛盾——虽然资本构成不同,资本周转时间不同等等,却存在着平均利润率。他只是重复了李嘉图本人和后来的作者已经注意到的这个矛盾的个别表现形式。因此,他在这里不过是一个应声虫:他没有使批判前进一步。

其次,他强调生产费用是“价值”的主要原因,因而是价值的主要要素。但是,他象李嘉图以后的其他作者一样,正确地指出,生产费用概念本身有不同的含义。最后他本人宣称,他同意托伦斯的价值由预付资本决定的观点,这个观点对费用价格来说是正确的,但是如果它不从价值本身的发展得出,也就是说,如果想由更发达的关系,即由资本的价值得出商品的价值,而不是相反,那是毫无意义的。

他最后的一个反驳是:如果一个行业的劳动时间不等于另一个行业的劳动时间,以致例如体现工程师12小时劳动的商品的价值比体现农业工人12小时劳动的商品的价值大一倍,那末,商品的价值就不能用劳动时间计量。这可以归结为:例如,简单劳动日如果有其他劳动日作为复杂劳动日与之相比,就不是价值尺度。李嘉图已经证明,如果简单劳动和复杂劳动之比是既定的,上述事实并不妨碍用劳动时间计量商品。\endnote{大·李嘉图《政治经济学和赋税原理》1821年伦敦第3版第13—15页。——第179页。}诚然他没有说明,这种比例是怎样发展和决定的。这属于对工资问题的说明,这归根到底就是劳动能力本身的价值的差别,即劳动能力的生产费用(由劳动时间决定)的差别。

下面就是贝利对前面已经概括的观点加以表述的段落:

\begin{quote}{“事实上,说生产费用是决定这类商品〈不存在垄断,而且只要扩大生产就可以增加产量的商品〉交换量的主要情况,是不会引起异议的;但是什么是生产费用,我们最优秀的经济学家的理解是不完全一致的;有些人主张,耗费在商品生产上的劳动量构成它的费用;另外一些人则主张,应该把用在这上面的资本叫做生产费用。”(同上,第200页)“劳动者没有资本而生产的东西,花费他的是他的劳动;资本家生产的东西,花费他的是他的资本。”(第201页)}\end{quote}

(正是这个理由决定了托伦斯的观点。资本家使用的劳动,除了他用在工资上的资本以外,没有花费他任何东西。)

\begin{quote}{“大部分商品的价值是由用在商品上的资本决定的。”(第206页)}\end{quote}

贝利对商品价值仅仅决定于商品中包含的劳动量的论点提出了以下反驳意见:

\begin{quote}{“只要我们能够找到任何下面这类例子,这种看法就不可能是正确的。第一种情况是:由同量劳动生产的两种商品,卖得不同量的货币;第二种情况是:以前价值相等的两种商品,虽然使用的劳动量没有任何变动,但在价值上变得不等了。”(第209页)“如果我们和李嘉图先生一样,说‘对不同性质的劳动的估量,能够在市场上迅速地而且对所有实际目的都十分准确地确定’;或者和穆勒先生一样,说‘在估量等量劳动时当然要注意不同的繁重程度和不同的熟练程度’,这都不是〈对第一种情况的〉回答。这种例子完全破坏了规则的普遍适用性。”(第210页)“比较一个劳动量和另一个劳动量,只可能有两种方法;一种是按照耗费的时间,另一种是按照生产出来的结果〈这种方法用于计件工资制〉。前一种方法适用于一切种类的劳动;后一种方法只能用于比较耗费在同类物品上的劳动。因此,如果在估量两种不同劳动时,所耗费的时间不决定劳动量之间的[839]\endnote{马克思在编手稿页码时把“838”误写为“839”。——第180页。}比例,那末这种比例就必然始终是不确定的和无法确定的。”(第215页)“关于第二种情况:试举任何两种价值相等的商品A和B为例;一种是用固定资本生产的,另一种是不用机器由劳动生产的,并且假定,在固定资本或劳动量没有任何变动的情况下,劳动价值提高了。按照李嘉图先生自己的论据,A和B之间的价值比例马上会发生变化,就是说,它们的价值将变得不等了。”(第215—216页)“对这两种情况我们还可以加上时间对价值的影响。如果生产一种商品比生产另一种商品需要的时间多,那末,即使它不需要较多的资本和劳动,它的价值也较大。李嘉图先生承认这个原因的影响。但是穆勒先生主张”……(第217页)}\end{quote}

最后,贝利还谈到下面一点,这是他在这方面提出的唯一的新东西:

\begin{quote}{“上述三类商品{这一点,即这三类商品,又是从《评政治经济学上若干用语的争论》的作者那里抄来的}〈即(1)在绝对垄断下生产的商品;(2)在有限的垄断下生产的商品,如谷物;(3)在完全自由竞争下生产的商品〉不可能绝对分开。它们不仅毫无区别地互相交换,而且在生产过程中混合在一起。因此,一种商品的一部分价值可能由垄断造成,而另一部分价值则可能由那些确定非垄断产品价值的原因造成。例如,一种物品可以在最自由的竞争下用原料生产者依靠完全的垄断按照六倍于实际费用的价格出卖的原料生产出来。”(第223页)“在这种情况下很清楚,尽管可以正确地说,物品的价值由工厂主花费在它上面的资本量决定,但是任何分析也不能把这笔资本的价值归结为劳动量。”(第223—224页)}\end{quote}

这个意见是正确的。但是垄断在这里和我们没有关系,因为我们所涉及的只是两个范畴,即价值和费用价格。很明显,价值转化为费用价格有双重作用。第一,加到预付资本上的利润可以高于或低于商品本身包含的剩余价值,即利润所代表的无酬劳动可以大于或小于商品本身所包含的。这一点适用于商品中的可变资本部分及其再生产。但是,除此之外,不变资本——或者说,作为原料、辅助材料和劳动工具,总之作为劳动条件加入新生产的商品的价值的商品——的费用价格,同样可以高于或低于它们的价值。因此,加入新生产的商品的价值的,是偏离了价值的价格部分,这个价格部分不取决于新加劳动量,或者说,不取决于使这些具有一定费用价格的生产条件转化为新产品的劳动量。总之很清楚,对商品本身——作为生产过程的结果的商品——的费用价格和价值之间的差额适用的东西,同样适用于以不变资本的形式,作为组成部分,作为前提进入生产过程的商品。可变资本,无论它的价值和费用价格之间有多大差额,总是由构成新商品的价值组成部分的一定劳动量补偿的,至于新商品的价值是恰好表现在新商品的价格上,还是高于或低于价格,那是无关紧要的。相反,如果说的是不依赖新商品本身的生产过程而加入该商品的价格的价值组成要素,那末费用价格和价值的这种差额将作为先决要素转入新商品的价值。

因此,商品的费用价格和价值之间的差额是由双重原因产生的:一方面是那些构成新商品生产过程的前提的商品的费用价格和价值之间的差额,另一方面是实际加到生产费用上的剩余价值和[按预付资本]计算的利润之间的差额。但是,每一种作为不变资本加入另一种商品的商品本身都是作为结果,作为产品从另一个生产过程出来的。因此,一种商品交替地时而表现为其他商品的生产的前提,时而表现为生产过程的结果,在这个过程中其他商品的存在又是这种商品的生产的前提。在农业(畜牧业)中,同一商品时而表现为产品,时而表现为生产条件。

费用价格对价值的这种有重要意义的偏离——这种偏离是由资本主义生产决定的——丝毫没有改变费用价格照旧是由价值决定这个事实。

\tchapternonum{(4)麦克库洛赫}

\tsectionnonum{[(a)在彻底发展李嘉图理论的外表下使李嘉图理论庸俗化和完全解体。肆无忌惮地为资本主义生产辩护。无耻的折衷主义]}

[840]麦克库洛赫是李嘉图经济理论的庸俗化者,同时又是使这个经济理论解体的最可悲的样板。

他不仅是李嘉图的庸俗化者,而且是詹姆斯·穆勒的庸俗化者。

而且,他在一切方面都是庸俗经济学家,是现状的辩护士。使他担心到可笑地步的唯一事情,就是利润下降的趋势;他对工人的状况是完全满意的,总而言之,他对沉重地压在工人阶级身上的资产阶级经济的一切矛盾都是满意的。在这里,一切都生气勃勃。在这里,他甚至知道,

\begin{quote}{“一个生产部门采用机器,必然会在其他某一生产部门造成同样大的或更大的对被解雇的工人的需求”。\endnote{约·雷·麦克库洛赫《政治经济学原理》1825年爱丁堡版第181—182页。这段引文,马克思是从卡泽诺夫《政治经济学大纲》(1832年伦敦版)一书中转引来的。见本册第68页。——第183页。}}\end{quote}

在这个问题上他背离了李嘉图,正象他在后来的一些著作中开始对土地所有者大加奉承一样。但是,鉴于利润率下降的趋势,他把全部温情脉脉的关怀都倾注在可怜的资本家身上。

\begin{quote}{“麦克库洛赫先生看来和其他科学代表人物不同,他不是寻求具有特征的区别,而只是寻求类似之处;按照这个原则,他就把物质对象和非物质对象、生产劳动和非生产劳动、资本和收入、工人的食物和工人本身、生产和消费以及劳动和利润,统统混淆起来。”(马尔萨斯《政治经济学定义》1827年伦敦版第69—70页)“麦克库洛赫先生在他的《政治经济学原理》(1825年伦敦版)一书中,把价值区分为实际价值和相对价值即交换价值。他在第211和225页上说,前者‘取决于耗费在占有或生产商品上的劳动量,而后者取决于商品换得的劳动或其他任何商品的量’;而且,他说(第215页),在通常状况下,即当市场上的商品供给和对商品的有效的需求完全一致的时候,这两种价值是等同的。但如果它们是等同的,那末他谈的两个劳动量也应该是等同的。但是,他在第221页告诉我们,它们不是等同的,因为一个包括利润,另一个不包括利润。”([卡泽诺夫]《政治经济学大纲》1832年伦敦版第25页)}\end{quote}

麦克库洛赫在他的这本《政治经济学原理》第221页上是这样说的:

\begin{quote}{“事实上它〈商品〉换得的总是更多{即比生产该商品所用的劳动更多的劳动},而且正是这个余额构成利润。”}\end{quote}

这是这个苏格兰大骗子所用的手法的鲜明例证。

马尔萨斯、贝利等人的争论,迫使他把实际价值和交换价值即相对价值区别开来。但是他所作的这种区别实际上就是他在李嘉图那里发现的区别。实际价值,就是从生产商品所必需的劳动来看的商品;相对价值就是各种不同商品的比例,这些商品可以用同样的时间生产出来,因而它们是等价物,因此,其中一种商品的价值,可以用花费同样多劳动时间的另一种商品的使用价值量来表现。商品的相对价值,按李嘉图的这种见解,不过是它的实际价值的另一种表现,不过意味着各种商品按照它们包含的劳动时间进行交换,或者说,它们各自包含的劳动时间是相等的。因此,如果商品的市场价格等于它的交换价值(在需求和供给相符时就是如此),那末买进的商品包含的劳动就同卖出的商品包含的一样多。如果在交换时商品换回的和在商品中付出的劳动量相同,那末商品仅仅实现它的交换价值,或者说,商品不过按它的交换价值出卖。

这一切,库洛赫都加以确认,象鹦鹉学舌那样正确地加以重复。不过,他在这里走过了头,因为马尔萨斯的交换价值规定——交换价值是商品支配的雇佣劳动量——已经深入他的内心。他因此把相对价值规定为“商品换得的劳动或其他任何商品的量”。李·嘉图在考察相对价值时,始终只谈劳动以外的商品,因为在商品交换时,利润所以实现,仅仅因为商品同劳动交换并不是等量劳动相交换。李嘉图在其著作一开头就特别强调指出:商品价值决定于[841]商品中包含的劳动时间,和商品价值决定于商品可以买到的劳动量,这两者是根本不同的。\endnote{大·李嘉图《政治经济学和赋税原理》1821年伦敦第3版第1—12页。——第184页。}这样,他一方面把商品包含的劳动量同商品支配的劳动量区别开来;另一方面,他从商品的相对价值中排除了商品同劳动的交换。因为一种商品同另一种商品相交换,是等量劳动相交换。商品同劳动本身相交换,则是不等量劳动相交换,而资本主义生产正是以这种交换的不平等为基础的。李嘉图没有解释这个例外如何同价值概念相符合。李嘉图以后的经济学家们的争论就是由此产生的。但是,正确的本能使他看到了这种例外(事实上这根本不是例外,只是他把它理解为例外)。由此可见,库洛赫比李嘉图走得还远,表面上比李嘉图还彻底。

在他那里毫无破绽。一切完美无缺。无论商品同商品相交换,还是商品同劳动相交换,这种交换比例都同样是商品的相对价值。如果交换的商品按它们的价值出卖(也就是说,如果需求和供给相符),这种相对价值就始终是实际价值的表现,也就是说,在交换的两极有相同的劳动量。因此,“在通常状况下”,商品所交换的也仅仅是和该商品包含的劳动量相等的雇佣劳动量。工人以工资形式得到的物化劳动,恰好等于他在交换时以直接劳动的形式还给资本家的劳动。这样,剩余价值的源泉就消失了,李嘉图的整个理论也就瓦解了。

可见,库洛赫先生一开头是在使李嘉图理论贯彻到底的外表下破坏这个理论。

下一步怎么样呢?下一步,他无耻地从李嘉图投奔到马尔萨斯那里去了,——按照马尔萨斯的学说,商品的价值决定于商品买到的劳动量,这个劳动量必须始终大于商品包含的劳动量。麦克库洛赫和马尔萨斯的区别仅仅在于,马尔萨斯把这一点按其本来面目,即把它作为李嘉图的对立面说出来,而库洛赫先生却以一种使李嘉图理论失去意义的表面的彻底性(即彻底的浅薄无知)采用李嘉图的说法,然后又采用这个对立面。因此,李嘉图学说的最内部的核心——在商品按其价值进行交换的基础上利润如何实现——库洛赫是不理解的,而且对他来说这个核心是不存在的。既然交换价值——按照库洛赫的说法,交换价值“在市场的通常状况下”等于实际价值,但是“事实上”总是大于实际价值,因为利润就建立在这个余额上(借“事实上”一词作了一个出色的对比和出色的分析)——是商品换得的“劳动或其他任何商品的量”,所以,适用于·“劳动”的,也适用于“其他任何商品”。换句话说,商品不仅同比它包含的劳动量大的直接劳动量相交换,而且同比它包含的劳动量大的其他商品中的物化劳动量相交换;这就是说,利润是“让渡利润”,这样,我们就又回到重商主义者那里去了。马尔萨斯直截了当地作出了这个结论。在库洛赫那里,这个结论则是不言自明的,不过他却把这妄称为李嘉图体系的发展。

而李嘉图体系的这种完全解体(变成一堆废话)——被自夸为李嘉图体系的彻底发展的这种解体——却被那些无知之徒,尤其是大陆上的无知之徒(其中当然包括罗雪尔先生)当作从这个体系出发而得出的走得太远、走到极端的结论,他们因而相信库洛赫先生所学到的李嘉图的“咳嗽和吐痰”\endnote{暗指席勒的《华伦斯坦》(第六场)中华伦斯坦的一个士兵的话:“他怎样咳嗽,怎样吐痰,你学得满象!但他的天才,我是说他的精神,却没有办法模仿。”——第186页。}的姿态(库洛赫用这种姿态来掩盖自己的不可救药的、浅薄无知的和无耻的折衷主义),真的就是把李嘉图体系贯彻到底的科学尝试!

麦克库洛赫纯粹是一个想利用李嘉图的经济理论来发财的人,而他确实令人吃惊地做到了这一点。萨伊也曾经这样利用斯密的理论,不同的是,他至少还有点贡献:他使斯密的理论有一定的形式上的条理化,而且,除了误解的情况之外,有时他还敢于提出一些理论上的疑问。因为库洛赫起先是靠李嘉图的经济理论在伦敦登上教授的席位,所以他最初势必以李嘉图主义者的身分出现,特别是还要参加反对土地所有者的斗争。一旦他站住了脚,并踏着李嘉图[842]的肩膀获得了一定的地位,他就主要致力于在辉格主义范围内讲述政治经济学,特别是李嘉图的政治经济学,而把其中使辉格党人讨厌的一切结论全部剔除。他的最后论货币、税收等等的著作,不过是为当时的辉格党内阁作的辩护词而已。此人由此谋得了一个肥缺。他的统计著作纯粹是骗钱的东西。在这里,对理论的浅薄无知的糟蹋和庸俗化,也暴露出此人本身就是一个“庸夫俗子”,关于这一点,下面我们在结束有关这位苏格兰投机家的问题之前,还要谈到一些。

1828年麦克库洛赫出版了斯密的《国富论》。这个版本的第四卷包括麦克库洛赫本人所写的“注释和论述”,其中一部分是为了增加篇幅把他从前发表过的、与问题毫无关系的蹩脚文章,例如关于“长子继承制”等等的文章,重新刊印出来;另一部分几乎逐字逐句重复他的政治经济学史讲义,或者象他自己所说的,“有许多是从其中借用来的”;还有一部分则竭力把穆勒以及李嘉图的反对者在这段时间里提出的新东西按照自己的方式加以同化。

麦克库洛赫先生在他的《政治经济学原理》\endnote{马克思显然是指1830年出版的麦克库洛赫《政治经济学原理》一书第二版,因为马克思通常引用的该书第一版,是在1825年,即在附有麦克库洛赫的“注释和论述”的斯密《国富论》问世前发表的。——第187页。}一书中,只是把他的“注释”和“论述”抄了一遍,而这些“注释”和“论述”又是他本人从自己过去的“零散的著作”中抄下来的。不过在《原理》中情况更加糟糕一些,因为在“注释”中,前后矛盾的地方还比在所谓的系统叙述中容易过得去。所以,上面从麦克库洛赫的《原理》引述的一些论点,有一部分虽然是从“注释和论述”中一字不改地抄来的,但是它们在这些“注释”中毕竟不象在《原理》中那样显得前后矛盾。{此外,他的《原理》还包括从穆勒那里抄来并加上极其荒谬的解释的东西,以及重新刊印的论谷物贸易等等的文章;这些文章他大概已经用二十个不同的标题在各种不同的期刊上,甚至往往在不同时间的同一刊物上一字不改地一再发表过。}

麦克在上面提到的他出版的亚·斯密著作的第四卷(1828年伦敦版)中说(他在《政治经济学原理》中逐字逐句重复了这些话,但是他在“注释”中还认为是必要的那些区别却没有了):

\begin{quote}{“必须把商品或产品的交换价值和实际价值(即费用价值)区别开来。前者,即商品或产品的交换价值,是指它们交换其他商品或劳动的能力或可能性;后者,即商品的实际价值或费用价值,是指为生产或占有商品所必需的劳动量,更确切地说,是指在所考察的时间内生产或占有同种商品所必需的劳动量。”(麦克库洛赫出版的亚·斯密的《国富论》1828年伦敦版第4卷第85—86页)“用一定量劳动生产的商品{在商品的供给和有效的需求相等的情况下},始终将交换或者说购买用同样多的劳动生产的其他任何商品。但是,它决不会交换或者说购买和生产它所用的劳动正好同样多的劳动;但是,尽管它不会这样做,它交换或者说购买的劳动量,总是象其他任何在相同条件下(即用和它本身相同的劳动量)生产出来的商品交换或者说购买的一样多。”(同上,第96—97页)“事实上〈麦克库洛赫在《原理》中一字不差地重复了这个词,因为这个“事实上”事实上构成他的全部论据〉它〈商品〉换得的总是更多{即比生产该商品所用的劳动更多的劳动},而且正是这个余额构成利润。资本家不会有任何动机〈好象在进行商品交换和考察商品价值时,问题就在于买者的“动机”〉去用一定的已完成的劳动量的产品交换[843]相同的待完成的劳动量的产品。这就等于贷款{“交换”竟等于“贷款”!}而不收任何利息。(同上,第96页)}\end{quote}

让我们从末尾谈起。

如果资本家取回的劳动不比他在工资上预付的多,他就是“贷款”而没有“利润”。问题是要解释,如果商品(劳动或其他商品)都按照它们的价值进行交换,利润怎么可能产生。麦克库洛赫的解释是:如果是等价物进行交换,利润就不可能产生。起初假定资本家同工人进行“交换”。然后,为了解释利润,又假定他们“不是”进行交换,而是其中一方贷出(即付出商品),另一方借入,即在取得商品之后之才付出。或者,为了解释利润,说资本家如果没有利润,他就没有“任何利息”。在这里,问题本身的提法就是错误的。资本家用来支付工资的商品,与他作为劳动成果取回的商品,是不同的使用价值。因此,他取回的并不是他预付的东西,正象他用一种商品交换另一种商品时取回的不是原来那种商品一样。他是购买另一种商品还是购买为他生产这另一种商品的特殊[商品——]劳动,这都是一回事。正象在一切商品交换的情况下一样,他付出了一种使用价值,而换取了另一种使用价值。相反,如果考虑的只是商品的价值,那末,用“一定的已完成的劳动量”去交换“相同的待完成的劳动量”(尽管资本家实际上只是在劳动已经完成之后才支付的),就没有任何矛盾,正象用一定的已完成的劳动量去交换相同的已完成的劳动量是不矛盾的一样。后一种情况是毫无意义的同义反复。前一种情况的前提是:“待完成的劳动”物化于和已完成的劳动不同的使用价值之中。所以在这种场合,[交换对象之间]存在差别,因而也就存在由这种关系本身产生的交换动机;而在另一种场合就不存在这种动机,因为,在这种交换中问题[仅仅]在于劳动量,A只是同A相交换。因此,麦克先生求助于动机。资本家的动机,是要取回比他付出的更多的“劳动量”。利润的产生用资本家有赚取“利润”的动机来解释。但是,在商人出卖商品的情况下,在一切不以消费而以利润为目的出卖商品的情况下,也完全可以这样说:卖者没有用一定的已完成的劳动量去交换相同的已完成的劳动量的动机。他的动机是要换得比他付出的更多的已完成的劳动。因此,他必须以货币或商品形式取得比他以商品或货币形式付出的更多的已完成的劳动。从而,他必须贵卖贱买,贱买贵卖。这样,我们看到的便是“让渡利润”,其产生的原因,并不在于它符合价值规律,而在于买者和卖者据说都没有按照价值规律来买或卖的“动机”。这就是麦克的第一个“卓越的”发现,这在力图阐明价值规律如何不顾买者和卖者的“动机”而为自己开辟道路的李嘉图体系中真是个绝妙的发现。

[844]此外,麦克在“注释”中的叙述和他在《原理》中的叙述只有以下的不同:

在《原理》中,他区别了“实际价值”和“相对价值”,并且说,“在通常情况下”两者是相等的,但是“事实上”,如果必须取得利润,两者就不能相等。可见,他不过是说:“事实”和“原则”相矛盾。

在“注释”中,他区别了三种价值:“实际价值”,商品在同其他商品交换时的“相对价值”,同劳动交换的商品的“相对价值”。商品在同其他商品交换时的“相对价值”,是商品表现在其他商品即“等价物”上的实际价值。相反,商品在同劳动交换时的相对价值,则是商品表现在另一种实际价值上的实际价值,而这另一种实际价值比商品的实际价值本身大。这就是说,商品的价值是同一个更大的价值进行交换,同非等价物进行交换。如果商品同劳动等价物进行交换,那就不会有利润了。商品在它同劳动交换时的价值,是一个更大的价值。

问题:李嘉图的价值规定同商品和劳动的交换相矛盾。

麦克的解答:在商品同劳动交换时,不存在价值规律,存在的是它的对立面。否则就无法解释利润。[然而]对于他这个李嘉图主义者来说,利润是应该用价值规律来解释的。

解答:价值规律(在这个场合)就是利润。“事实上”,麦克所说的只是李嘉图理论的反对者所说的话:如果价值规律在资本和劳动的交换中起支配作用,那就不存在任何利润了。他们说,因此李嘉图的价值规律是错误的。他说,就这个场合而言(这个场合他本来是应该根据价值规律加以解释的),这个规律是不存在的,在这个场合“价值”“意味着”某种别的东西。

由此清楚地看出,麦克库洛赫对李嘉图的规律丝毫也不理解。不然的话他就应该说:在按照本身包含的劳动时间相交换的商品进行交换时产生利润,是由于商品中包含“无聊”劳动。因此,资本和劳动的不平等交换可以说明商品按其价值相交换和在这种商品交换中实现的利润。麦克库洛赫却不是这样,他说:包含同样多的劳动时间的商品,可以支配同样多的不包含在它们之中的劳动余额。他想用这个方法把李嘉图的论点和马尔萨斯的论点调和起来,硬把商品价值决定于劳动时间和商品价值决定于支配劳动的能力等同起来。但是,包含同样多的劳动时间的商品,可以支配同样多超过它们包含的劳动的劳动余额,这意味着什么呢?这仅仅意味着,包含一定的劳动时间的商品,可以支配一定量的超过它包含的劳动的剩余劳动。不仅包含x劳动时间的商品A是如此,而且同样包含x劳动时间的商品B也是如此,——这一点已经包含在马尔萨斯的公式的表述中了。

可见,矛盾在麦克那里是这样解决的:如果李嘉图的价值规律发生作用,就不可能有利润,也就是说,不可能有资本和资本主义生产。这正是李嘉图的反对者的论断。而麦克也正是用这一点来回答他们,反驳他们。在这里,他完全没有觉察到,对于同劳动[相交换]的交换价值的解释——价值就是同某种非价值的交换——是多么妙不可言。

\tsectionnonum{[(b)通过把劳动的概念扩展到自然过程而对劳动的概念进行歪曲。把交换价值和使用价值等同起来。把利润解释为“积累劳动的工资”的辩护论观点]}

[845]麦克先生在这样抛弃了李嘉图政治经济学的基础以后,还更进一步,破坏了这个基础的基础。

李嘉图体系的第一个困难是,资本和劳动的交换如何同“价值规律”相符合。

第二个困难是,等量资本,无论它们的有机构成如何,都提供相等的利润,或者说,提供一般利润率。实际上这是一个没有被意识到的问题:价值如何转化为费用价格。

困难是从这里产生的:具有不同构成(不管这是由不变资本和可变资本的比例不同或固定资本和流动资本的比例不同引起的,还是由周转时间不同引起的)的等量资本,推动不等量的直接劳动,从而也推动不等量的无酬劳动,所以,它们在生产过程中不可能占有相等的剩余价值或相等的剩余产品。因此,既然利润无非是按总预付资本的价值计算的剩余价值,它们就不能得到相等的利润。如果剩余价值是某种别的东西而不是劳动(无酬的),那末,劳动也就根本不是商品价值的“基础和尺度”。\endnote{李嘉图在他的《原理》中好几处(例如第三版第80页)把劳动称为“商品价值的基础”。在《原理》第三版第333—334页上李嘉图把劳动说成“价值的尺度”。参看本册第148—149页,马克思从李嘉图的《原理》中引了相应的话。——第192页。}

这里产生的困难,李嘉图自己已经发现(尽管不是在其一般形式上),并且把它们当做价值规则[即规律]的例外。马尔萨斯把这些例外连同规则一起抛弃,因为例外成了规则。也同李嘉图论战的托伦斯,至少在某种程度上表述了这个问题,说等量资本虽然推动不等量的劳动,但是仍然生产出“价值”相等的商品,因此,价值不是由劳动决定的。贝利等人也是这样。至于穆勒,则承认李嘉图所确认的例外是例外,而且这些例外,除了唯一的一个形式外,没有使他发生任何怀疑。他发现只有一个造成资本家利润平均化的理由是和规则相矛盾的。这个情况就是:某些商品停留在生产过程中(例如,葡萄酒置于窖内)而没有在它们上面花费任何劳动;这是一段使它们经受某种自然过程的作用的时期。(例如,在农业和制革业中,在开始采用某些新的化学药剂以前劳动长时间中断,就是这种情况,而这一点穆勒没有提到。)然而这段时间仍被算作提供利润的时间。商品不经受劳动过程的这段时间也被算作劳动时间。(在流通时间比较长的场合,情况也总是这样。)穆勒可以说是这样“摆脱了”困境:他说,例如葡萄酒置于窖内的时间,可以算作它吸收劳动的时间,尽管根据假定,实际上并非如此。不然,[穆勒指出,]就得说“时间”创造利润了,而时间本身,据说“不过是一种音响和烟云”\endnote{歌德的《浮士德》第一部第十六场(《玛尔特的花园》)中浮士德的话。马克思在前面第89页从詹姆斯·穆勒的书中引了相应的话。——第193页。}而已。库洛赫附和穆勒的这种胡说,更确切些说,他以其惯用的、矫揉造作的剽窃手法,以一般的形式重复了这种胡说,在这种形式下,隐蔽的荒谬思想就暴露出来了,李嘉图体系的以及整个经济思想的最后残余也就被顺利地抛弃了。

上述种种困难,加以进一步考察,可以归结为下面这个困难:

以商品的形式作为材料或工具进入生产过程的那部分资本加在产品上的价值,始终不会大于它在这个生产过程开始前所具有的价值。因为,这部分资本只是由于它是物体化的劳动才具有价值,而它包含的劳动并不由于它进入生产过程而发生任何变化。它根本不取决于它所进入的生产过程,而完全取决于生产它本身所需要的社会规定的劳动,因而,在再生产它所需要的劳动时间多于或少于它包含的劳动时间时,它本身的价值才发生变动。因此,这一部分资本作为价值,原封不动地进入生产过程,又原封不动地从生产过程中出来。如果说它毕竟实际进入生产过程并且发生了变动,那末,这是它的使用价值所经受的变动,是它本身作为使用价值所经受的变动。原料所经受的或者劳动工具所完成的一切操作,只不过是它们作为一定的原料和一定的劳动工具(纱锭等等)所经历的过程,是它们的使用价值的过程,这个过程本身同它们的交换价值毫不相干。交换价值在这个[846]变动中保持不变。全部情况就是这样。

同劳动能力交换的那部分资本,则不是这样。劳动能力的使用价值,是劳动,是创造交换价值的要素。因为劳动能力在它的生产消费中所完成的劳动,比劳动能力本身的再生产所需要的劳动多,比提供工资等价物的劳动多,所以,资本家以付给工人的工资从工人那里换得的价值,大于他为这个劳动支付的价格。因此,在劳动剥削率相同的前提下,可以得出这样的结论:两个等量资本中推动较少的活劳动的那个资本,——这无论是由于它的可变部分对不变部分的比例本来就小,还是由于它的流通时间,或者说它不同劳动交换、不接触劳动、不吸收劳动的生产时间[较长],——创造较少的剩余价值,并且一般说来创造价值较小的商品。在这种条件下,创造出来的价值怎么还会相等,而剩余价值怎么还会同预付资本成比例呢?李嘉图没有能够回答这个问题,因为这样提出的问题是荒谬的,实际上,这里既没有生产出相等的价值,也没有生产出相等的剩余价值。但是,李嘉图不理解一般利润率的起源,因而也不理解价值怎样转化为和它迥然不同的费用价格。

麦克依靠穆勒的荒谬的“遁词”排除了困难。排除困难的方法是,用空洞的辞句避开了困难所由产生的具有特征的区别。这个具有特征的区别是:劳动能力的使用价值就是劳动,因此它也创造交换价值。其他商品的使用价值,则是和交换价值不同的使用价值,因此,这种使用价值所经受的任何变动,都不影响商品的预先决定的交换价值。排除困难的方法是,把商品的使用价值称为交换价值,而把这些商品作为使用价值所经历的各种操作,把它们作为使用价值在生产中提供的各种服务,称为劳动。是啊,在日常生活中也确实谈到役畜劳动和机器劳动,而在诗的语言中还有这样的说法:铁在熊熊烈火中劳动,或者在锻锤的锤击下呻吟,劳动。甚至铁在呼号呢。可以最容易不过地证明,一切“操作”都是劳动,因为劳动是一种操作。同样可以证明,一切有形体的东西都有感觉,因为一切有感觉的东西都是有形体的东西。

\begin{quote}{“有充分理由可以把劳动下定义为任何一种旨在引起某一合乎愿望的结果的作用或操作,而不管它是由人,由动物,由机器还是由自然力完成的。”(麦克库洛赫《为斯密〈国富论〉写的注释和补充论述》第4卷第75页)}\end{quote}

而这决不[仅仅]适用于劳动工具。实质上,这同样适用于原料。羊毛在吸收染料时要经受物理的作用,即物理的操作。总而言之,对任何物施加物理的、机械的、化学的等等作用以“引起某一合乎愿望的结果”,物本身都必然发生反应。这就是说,它在经受加工的同时本身必然也在劳动。于是,一切进入生产过程的商品之所以增加价值,不仅因为它们本身的价值被保存下来,而且因为它们依靠本身“劳动”——不单单是作为物化劳动——而创造了新的价值。这样一来,当然一切困难都被排除了。实质上,这不过是萨伊的“资本的生产性服务”、“土地的生产性服务”等说法的改头换面;李嘉图始终反对这种说法,而麦克,说来奇怪,就在这同一“论述”或“注释”中也反对这种说法,他在这里傲慢地捧出了他从穆勒那里抄来并加以修饰的发现。在同萨伊的论战中,麦克库洛赫对李嘉图倒还念念不忘,他还记得这种“生产性服务”实际上只是作为使用价值的物在生产过程中表现出来的属性。但是,当他把“劳动”这个神圣的名称赋予这种“生产性服务”时,一切当然就完全改变了。

[847]在麦克顺利地把商品变为工人之后,不言而喻,这些“工人”也要取得工资,而且除了它们作为“积累劳动”具有的价值外,对它们的“操作”或者说“作用”也必须付给工资。商品的这种工资,资本家受权装入自己的腰包,它是“积累劳动的工资”,换句话说就是利润\authornote{见本册第202页。——编者注}。[按照麦克库洛赫的看法]这就证明,相等的资本提供相等的利润(不管这些资本推动的劳动多少),是直接从价值决定于劳动时间得出来的。

最令人吃惊的是,如上面已经指出的,麦克就在他从穆勒的理论出发剽窃萨伊的观点的同时,又用李嘉图的话去反对同一个萨伊。从下面李嘉图反驳萨伊的一些话里,可以最清楚不过地看到麦克是怎样逐字逐句地抄袭萨伊的,所不同的只是在萨伊谈到作用的地方,他把这种作用叫作劳动:

\begin{quote}{“萨伊先生……硬说他〈亚·斯密〉犯了一个错误,说‘他把生产价值的能力仅仅归于人的劳动。更正确的分析告诉我们,价值是由人的劳动的作用,确切地说,是由人的勤劳的作用,同自然所提供的各种因素的作用以及同资本的作用结合起来产生的。斯密不懂得这一原理,所以他就不能提出有关机器在财富生产中所发生的影响的正确理论’。\endnote{这段话引自萨伊《论政治经济学》1814年巴黎第2版第1卷第51—52页。——第197页。}同亚当·斯密的看法相反,萨伊先生……谈到了自然因素赋予商品的价值”等等……“但是,这些自然因素尽管能够大大增加使用价值,却从来不会给商品增加萨伊先生所说的交换价值。”(李嘉图《政治经济学原理》第3版第334—336页)“机器和自然因素能大大增加一国的财富……但是……它们不能给这种财富的价值增加任何东西。”(同上,第335页注)}\end{quote}

李嘉图,象所有值得提到的经济学家一样,象亚·斯密一样(虽然斯密有一次出于幽默把牛称为生产劳动者)\authornote{见本卷第1册第271页。——编者注},强调指出劳动是人的、而且是社会规定的人的活动,是价值的唯一源泉。李嘉图和其他经济学家不同的地方,恰恰在于他前后一贯地把商品的价值看作仅仅是社会规定的劳动的“体现”。所有这些经济学家都多少懂得(李嘉图更懂得)应该把物的交换价值看作仅仅是人的生产活动的表现,人的生产活动的特殊的社会形式,看作一种和物及其作为物在生产消费或非生产消费中的使用完全不同的东西。在他们后来,价值实际上不过是以物表现出来的、人的生产活动即人的各种劳动的相互关系。李嘉图引用德斯杜特·德·特拉西的下面一段话来反驳萨伊,这段话,正如他明确地声明的那样,也表达了他本人的见解:

\begin{quote}{“很清楚,我们的体力和智力是我们唯一的原始的财富,因此,这些能力〈人的能力〉的运用,某种劳动〈可见,劳动是人的能力的实现〉,是我们唯一的原始的财宝;凡是我们称为财富的东西,总是由这些能力的运用创造出来的……此外,这一切东西确实只代表创造它们的劳动,如果它们有价值,或者甚至有两种不同的价值,那也只能来源于……创造它们的劳动的价值。”(李嘉图,同上第334页)}\end{quote}

由此可见,商品所以有价值,一般说,物所以有价值,仅仅由于它们是人的[848]劳动的表现——不是因为它们本身是物,而是因为它们是社会劳动的化身。

可是有人竟敢于说可悲的麦克把李嘉图的观点发展到了极端。就是这个麦克,轻率地力图把李嘉图的理论同相反的见解折衷主义地混在一起加以“利用”,把李嘉图理论的原理和整个政治经济学的原理,把作为人的活动而且是社会规定的人的活动的劳动本身,与作为使用价值、作为物的商品所具有的物理等等的作用等同起来!就是他,把劳动的概念本身都丢掉了!

麦克库洛赫凭着穆勒的“遁词”而变得厚颜无耻,他抄袭萨伊的观点,同时又用李嘉图的话来反驳萨伊,而他抄袭萨伊的那些话,恰巧就是李嘉图在第二十章《价值和财富》中作为同他本人的观点以及斯密的观点根本对立的东西坚决加以驳斥的。(罗雪尔当然要重复说,麦克是发展到了极端的李嘉图。\endnote{威·罗雪尔《国民经济体系》,第1卷《国民经济学原理》1858年斯图加特和奥格斯堡第3版第82、191页。——第198页。})不过,麦克比萨伊更荒谬,因为萨伊并没有把火、机器等的“作用”称作劳动。而且麦克更加前后矛盾。在萨伊那里,风、火等可以创造“价值”,而麦克认为只有那些可以被独占的使用价值,物,才创造“价值”。风或蒸汽或水在不占有风磨、蒸汽机、水车的情况下,好象也可以被当作动力使用!占有和独占那些为使用自然力所必须占有的物的人,好象并没有把这些自然力也独占下来!空气、水等等,我要多少就能有多少。但是它们只有在我占有了能用来使它们起生产因素作用的那些商品、那些物的时候,对我来说才是生产因素!由此可见,麦克在这方面还比不上萨伊。

所以,在这样一些把李嘉图的观点庸俗化的言论中,我们看到了对李嘉图理论的最彻底、最无知的败坏。

\begin{quote}{“但是,既然这种结果〈由任何一种东西的作用或者说操作产生的结果〉是那些不能被一定数目的个人在排斥他人的情况下独占或占有的自然力的劳动或者说作用创造出来的,那末,这种结果就没有任何价值。这些自然力所完成的东西,是它们无代价地完成的。”(麦克库洛赫《为斯密〈国富论〉写的注释和补充论述》第4卷第75页)}\end{quote}

似乎棉花、羊毛、铁或机器所完成的东西,并不是同样“无代价地”完成的。机器有价值,机器的作用则不要付报酬。任何商品的使用价值在商品的交换价值被支付后,就什么也不值了。

\begin{quote}{“卖油的人并不要求为油的自然属性付任何费用。他在估计油的生产费用时考虑的是为获得油而使用的劳动的价值,这也就是油的价值。”(凯里《政治经济学原理》1837年费拉得尔菲亚版第1卷第47页)}\end{quote}

李嘉图在反驳萨伊时恰恰强调,例如机器,它的作用同风或水的作用一样什么也不值:

\begin{quote}{“自然力和机器为我们提供的服务……由于增加了使用价值,对我们是有用的;但是,由于它们做工不需要费用……它们为我们提供的帮助就不会使交换价值有丝毫增加。”(李嘉图,同上第336—337页)}\end{quote}

可见,麦克连李嘉图的最简单的原理都不懂。但是这个狡猾的家伙这样想:如果说棉花、机器等等的使用价值什么也不值,如果说除了它们的交换价值外,它们的使用价值不要另付报酬,那末,这种使用价值却会由使用棉花、机器等等的人出卖,——他们出卖对他们来说什么也不值的东西。

[849]这个家伙的极端浅薄无知从下面这一点可以看出:他接受了萨伊的“原理”,然后利用非常详细地从李嘉图那里抄来的东西大讲其地租理论。

因为土地是“一定数目的个人在排斥他人的情况下独占或占有的自然力”所以它的自然的生长作用或者说“劳动”,即它的生产力,具有价值,从而地租就可以象重农学派那样用土地的生产力来解释。这个例子清楚地说明了麦克把李嘉图观点庸俗化的手法。一方面,他抄袭了李嘉图的只有在李嘉图提出的前提基础上才有意义的论点,另一方面,他又接受了别人的(他自己保留的只是“名词术语”或者小小的更动)直接否定这些前提的东西。他想必会说:“地租是”被土地所有者装进腰包的“土地的工资”。

\begin{quote}{“如果一个资本家在支付工人工资、饲养马匹或租用机器上花费同样金额,又如果这些工人、马匹和机器能够完成同样的工作量,那末,工作无论是由工人、马匹还是机器来完成,它的价值显然都是相同的。”(麦克库洛赫《为斯密〈国富论〉写的注释和补充论述》第4卷第77页)}\end{quote}

换句话说,产品的价值与所花费的资本的价值相适应。这是有待解决的问题。照麦克看来,问题的提出“显然”就是问题的解决。但是,既然比如说机器完成的工作量比被它排挤的工人完成的工作量大,那末更加“显然”的是:机器产品的价值和“完成同样工作”的工人的产品的价值相比不会降低,而只会提高。因为机器在一个工人制造一件产品的时间里可能制造出一万件来,而且每件都具有相同的价值,所以机器的产品比工人的产品一定会贵一万倍。

麦克竭力要表示和萨伊不同,——他认为创造价值的不是自然力的作用,而只是被独占的或由劳动产生的力的作用;不过,他还是无法在用词上克制自己,又回到了李嘉图式的用语上去。例如,他写道:

\begin{quote}{“风的劳动对船产生了合乎愿望的作用,使船发生一定的变化。但是这种变化的价值不会由于有关的自然力的作用或者说劳动而增大,它根本不取决于它们,而取决于参与生产这一结果的资本量或者说过去劳动的产品,这正象小麦的磨粉费用不取决于推动磨的风或水的作用,而取决于在这种操作中所耗费的资本量一样。”(同上,第79页)}\end{quote}

这里,磨粉之所以增加小麦的价值,忽然又只是由于资本即“过去劳动的产品”在磨粉的操作中被“耗费”。这就是说,不是因为磨盘“劳动”了,而是因为在“耗费”磨盘的时候,也“耗费”了它所包含的价值,即物化在其中的劳动。

麦克在发表了这番堂皇的议论以后,把他从穆勒和萨伊那里借来的、他用以使价值概念同一切与之矛盾的现象调和起来的深奥道理归纳如下:

\begin{quote}{“在有关价值的一切讨论中……劳动一词表示……人的直接劳动或人所生产的资本的劳动,或兼指两者。”(同上,第84页)}\end{quote}

可见,劳动[850]应理解为人的劳动,其次应理解为人的积累劳动,最后还应理解为使用价值的有益利用,即使用价值在消费(生产消费)中表现出来的物理等等的属性。而离开了这些属性也就无所谓使用价值。使用价值只有在消费中才实际表现出来。这就是说,要我们把劳动产品的交换价值理解为这些产品的使用价值,因为这种使用价值仅仅在于它在消费(不管是生产消费或非生产消费)中的实际表现,或者如麦克所说,在消费中的“劳动”。但是,使用价值的“操作”、“作用”或“劳动”的种类以及它们的自然尺度,都象这些使用价值本身一样是各不相同的。那末,什么是我们能够用来把它们加以比较的统一依据即尺度呢?[在麦克库洛赫那里]这个统一依据是由一个共同的词“劳动”来造成的,在把劳动本身归结为“操作”或“作用”这些词之后,就用这个词暗中替换了使用价值的这些完全不同的表现。可见,对李嘉图观点的这种庸俗化的结果,就是把使用价值和交换价值等同起来,因此,我们必须把这种庸俗化看成是这个学派作为一个学派解体的最后的最丑恶的表现。

\begin{quote}{“资本的利润只是积累劳动的工资的别名”,(麦克库洛赫《政治经济学原理》1825年爱丁堡版第291页)}\end{quote}

也就是对商品作为使用价值在生产中提供服务而付给商品的工资的别名。

而且,这种“积累劳动的工资”在麦克库洛赫先生那里还有一种独特的奥妙的含义。我们已经提到过,除了他从李嘉图、穆勒、马尔萨斯和萨伊那里抄来的、构成他的著作的基本内容的那些东西以外,他自己还不断把他的“积累劳动”以不同的标题一再翻印出售,经常从他以前已经得过报酬的著作中“大量抄录”。对于这种赚取“积累劳动的工资”的手法,早在1826年就有一本专门著作进行过详细的分析,而从1826年到1862年,麦克库洛赫在赚取积累劳动的工资这方面又进一步取得了多么大的成就啊!\endnote{关于麦克库洛赫这一节,以及《李嘉图学派的解体》全章(除了约·斯·穆勒一节写于1862年春以外),是马克思于1862年10月写的(马克思自己在包括该章的第XIV本的封面上注明了这一点)。——第202页。}(作为修昔的底斯的罗雪尔也使用过“积累劳动的工资”这个可悲的词句。\endnote{威·罗雪尔《国民经济体系》,第1卷《国民经济学原理》1858年斯图加特和奥格斯堡增订第3版第353页。马克思用古希腊大历史学家修昔的底斯的名字来称呼罗雪尔,这是因为,如马克思在后面(见本册第558页)所说,“罗雪尔教授先生谦虚地宣称自己是政治经济学的修昔的底斯”。“修昔的底斯·罗雪尔”这个称呼具有辛辣的讽刺性:马克思在许多地方指出,罗雪尔既严重歪曲了经济关系的历史,又严重歪曲了经济理论的历史。参看本卷第2册第130—132页。——第202页。})

上面提到的著作叫作:莫迪凯·马利昂《对麦克库洛赫先生的〈政治经济学原理〉的若干说明》1826年爱丁堡版\endnote{这本小册子的真实作者是英国政论家约翰·威尔逊,他曾以不同的笔名发表著作。——第202页。}。这本著作说明我们这位骗子手是怎样成名的。他9/10是从亚·斯密、李嘉图和其他作者那里抄来的,其余1/10则是不断地从他自己的积累劳动中抄来的,“他最无耻最恶劣地一再重复这种积累劳动”。[第4页]例如,马利昂指出,麦克库洛赫不仅把同一些文章当作自己的“论述”,当作新的著作,轮流卖给《爱丁堡评论》\endnote{《爱丁堡评论,或批评杂志》是1802年至1929年发行的英国资产阶级的文学、政治杂志。在十九世纪二十年代和三十年代每三个月发行一期,是辉格党的机关报。这一时期发表的有关经济问题的文章大多数是麦克库洛赫写的。——第202页。}、《苏格兰人报》\endnote{《苏格兰人报,或爱丁堡政治文学报》是1817年开始发行的英国资产阶级报纸。十九世纪上半叶是辉格党的机关报。这个报纸从创刊到1827年发表了麦克库洛赫论述经济问题的文章。1818年至1820年麦克库洛赫是该报的编辑。——第202页。}、《英国百科全书》\endnote{《英国百科全书》是一部多卷的英国(现在是英美)百科词典。从1768年起不断以新版刊行。十九世纪末之前一直在爱丁堡出版。——第202页。},而且他比如说还在不同年份的《爱丁堡评论》杂志上把同一些文章一字不差地重新发表,只是多少颠倒一下次序,换上新的招牌。在这方面,马利昂是这样评论“这个最不可相信的修鞋匠”[第31页]、“这位所有经济学家中最经济的经济学家”[第66页]的:

\begin{quote}{“麦克库洛赫先生的文章不管和天体多么不一样,但是有一点却和星辰相似,就是它们总是定期再现。”(第21页)}\end{quote}

麦克库洛赫信仰“积累劳动的工资”,这是毫不奇怪的!

麦克先生获得的名声,说明这种骗子手的卑鄙行为可以有多么大的神通。

[850a]只要顺便看一下1824年3月《爱丁堡评论》(那篇拙劣文章的名称是《论资本积累》),就知道麦克库洛赫怎样利用李嘉图的某些论点来抬高自己。在那篇文章里,这位“积累劳动的工资”之友对利润率的下降发出了真正的哀鸣。

\begin{quote}{“作者……这样表达了他对利润下降的忧虑:‘英国所表现的繁荣外貌是虚假的;贫困的瘟疫悄悄地侵害着市民大众,国家富强的基础已被动摇……在象英国这样利息率低的地方,利润率也是低的,国家的繁荣已经越过了它的顶点。’这种论断不能不使每一个熟悉英国美好状况的人感到吃惊。”(普雷沃《评李嘉图体系》第197页)}\end{quote}

麦克先生不必对“土地”比“铁、砖等”得到优厚的“工资”感到不安。原因想必是土地“劳动”得更勤快。[XIV—850a]

\centerbox{※     ※     ※}

[XV—925]{瞎眼睛的猪有时也能找到橡实。麦克库洛赫有一次就是这样。但是即使如此,照他那样表达,这也不过是一些前后矛盾的说法,因为他没有把剩余价值和利润区别开来;其次,这是他的又一轻率的折衷主义的剽窃。照托伦斯之流看来(他们认为价值是由资本决定的),同样照贝利看来,利润应该从它对(预付)资本的比例加以考察。和李嘉图不同,他们不是把利润和剩余价值等同起来,但是这只是因为他们根本不感到需要在价值的基础上解释利润,因为他们把剩余价值借以表现出来的形式——作为剩余价值对预付资本的比例的利润——看作原始形式,实际上不过是把表现出来的形式用文字表达出来。

下面麦克著作中的两段话,说明(1)他是李嘉图主义者;(2)他直接抄袭李嘉图的反对者:

\begin{quote}{“利润只能由于工资降低而提高、只能由于工资提高而降低这一李嘉图的规律,只有在劳动生产率不变的情况下才是正确的”(麦克库洛赫《政治经济学原理》1825年爱丁堡版第373页)。这里是指提供不变资本的生产部门的劳动生产率。“利润取决于它对生产它的资本的比例,而不取决于它对工资的比例。如果所有生产部门的劳动生产率普遍提高了一倍,由此得到的额外产品在资本家和工人之间分配,那末,虽然按预付资本计算利润率提高了,资本家和工人之间的比例仍旧不变。”(同上,第373—374页)}\end{quote}

即使在这种情况下,正象麦克也指出的,可以说工资同产品相比也相对地降低了,因为利润提高了。(然而在这种情况下利润的提高正是工资降低的原因。)但是这种计算是以工资作为产品的一部分这种错误算法为依据的,我们在上面已经看到,约翰·斯图亚特·穆勒先生就企图用这种诡辩的办法把李嘉图的规律普遍化。\endnote{马克思指的是1861—1863年手稿第VII本和第VIII本(手稿第319—345页)中篇幅很长的关于约·斯·穆勒的插入部分。按照马克思在稿本封面上所编的《剩余价值理论》目录以及他在手稿第VII本正文中所作的指示,把关于约·斯·穆勒这一节移至本册(第208—258页)。关于“以工资作为产品的一部分这种错误算法”,马克思在后面第244—248页谈到过。——第204页。}}[XV—925]

\tchapternonum{(5)威克菲尔德[在“劳动价值”和地租问题上对李嘉图理论的局部反驳]}

[XIV—850a]威克菲尔德在理解资本上的真正功绩,已在前面《剩余价值转化为资本》这一节中阐明了\endnote{马克思在1862年初着手写作《剩余价值理论》时,打算把它作为关于资本生产过程的研究的第五节即最后一节,直接放在绝对剩余价值和相对剩余价值的结合这一节之后。但是马克思在写作《理论》的过程中认为有必要在第四节(《绝对剩余价值和相对剩余价值的结合》)和《剩余价值理论》之间再插进两节《剩余价值再转化为资本》和《生产过程的结果》(见本卷第1册第446页)。这一点也说明马克思为什么提到1862年10月还没有写的应对威克菲尔德的一些观点加以阐述的一节《剩余价值转化为资本》。马克思在《资本论》第一卷第二十二章即标题为《剩余价值转化为资本》的这一章的注22中,引了威克菲尔德的论点:“在资本使用劳动以前,劳动就已经创造了资本”(见《马克思恩格斯全集》中文版第23卷第639页注22)。——第205页。}。这里只涉及和“本题”直接有关的地方。

\begin{quote}{“如果把劳动看成一种商品,而把资本,劳动的产品,看成另一种商品,并且假定这两种商品的价值是由相同的劳动量来决定的,那末,在任何情况下,一定量的劳动就都会和同量劳动所生产的资本量相交换;过去的劳动就总会和同量的现在的劳动相交换。但是,劳动的价值同其他商品相比,至少在工资取决于[产品在资本家和工人之间的]分配的情况下,不是由同量劳动决定,而是由供给和需求的关系决定。”(威克菲尔德在他出版的亚·斯密《国富论》1835年伦敦版第1卷第230页上所加的注)}\end{quote}

因此,照威克菲尔德看来,如果劳动的价值被支付了,利润就无法解释。

威克菲尔德在上述他出版的斯密著作第二卷中指出:

\begin{quote}{“剩余产品\endnote{威克菲尔德所谓的剩余产品,是指产品中“补偿资本和普通利润”后余下的部分(威克菲尔德为亚·斯密《国富论》第2卷所加的注释,第215、217页)。——第205页。}总是形成地租。但不是由剩余产品构成的地租也还是可能被支付的。”(第216页)“如果象在爱尔兰那样,大多数人弄得只能吃马铃薯,住小屋,穿破衣,并且为了求得过上这种生活,必须把他们除了小屋、破衣和马铃薯之外所能生产的一切都交出来,那末,即使资本或劳动的产品照旧,他们赖以生活的土地的所有者得到的东西,也会随着他们聊以为生的东西的减少,而增加起来。贫困的佃户所交出的,都被土地所有者占有。所以,土地耕种者生活水平的下降是剩余产品的另一原因……当工资降低时,它对剩余产品的影响,和生活水平下降所起的影响是一样的:总产品不变,剩余部分增大了;生产者得到的更少,土地所有者得到的更多。”(第220—221页)}\end{quote}

在这种情况下利润称为地租,就同例如在印度,劳动者用资本家的贷款从事劳动(纵然劳动者在名义上是独立的),把全部剩余产品交给资本家时利润称为利息完全一样。

\tchapternonum{(6)斯特林[用供求关系对资本家的利润作庸俗解释]}

\begin{quote}{“每种商品的量必须这样调节,也就是使该商品的供给与商品的需求之比小于劳动的供给与劳动的需求之比。商品的价格或价值,同耗费在商品上的劳动的价格或价值之间的差额形成利润或余额,这种利润或余额是李嘉图根据他的理论所不能解释的。”(帕特里克·詹姆斯·斯特林《贸易的哲学》1846年爱丁堡版第72—73页)}\end{quote}

[851]同一个作者对我们说:

\begin{quote}{“如果商品的价值按照它们的生产费用互成比例,这就可以叫作价值水准。”(同上,第18页)}\end{quote}

因此,如果劳动的需求和供给相符,劳动就会按照它的价值出卖(不管斯特林如何理解这个价值)。如果劳动耗费在其上的商品的需求和供给相符,商品就会按照它的生产费用出卖,这个生产费用就是斯特林所谓的劳动的价值。于是,商品的价格等于耗费在商品上的劳动的价值。而劳动的价格又和劳动本身的价值处于同一水准。从而,商品的价格等于耗费在商品上的劳动的价格。因此在这种场合就不会有利润或余额。

于是,斯特林这样来解释利润或余额:

劳动的供给与劳动的需求之比必须大于劳动耗费在其上的商品的供给与这个商品的需求之比。必须设法使商品的卖价高于商品中包含的劳动的被支付的价格。

斯特林先生把这叫作对余额现象的解释,其实这不过是对必须解释的现象的另一种表达。进一步考察,只有三种情况是可能的。(1)劳动的价格合乎“价值水准”,就是说,劳动的需求和劳动的供给相符,劳动的价格等于劳动的价值。这时,商品必须高于它的价值出卖,就是说必须设法使商品的供给低于商品的需求。这是纯粹的“让渡利润”,不过加上了它得以实现的条件。(2)劳动的需求超过劳动的供给,劳动的价格高于劳动的价值。这时,资本家付给工人的多于这些工人所生产的商品的价值,买者必须付给资本家双重余额:第一,资本家起初付给工人的余额;第二,资本家的利润。(3)劳动的价格低于劳动的价值,劳动的供给超过劳动的需求。这时,余额的产生是由于劳动低于它的价值被支付,而[以商品的形式]按照它的价值或至少高于它的价格被出卖。

如果从斯特林的议论中把无稽之谈去掉,那末,在斯特林那里,余额的产生是由于劳动低于它的价值被资本家购买,而以商品形式高于它的价格再被出卖。

如果前面两种情况把所谓生产者必须“设法”使他的商品高于它的价值或高于“价值水准”出卖这一可笑的形式去掉,那无非是:如果一种商品的需求超过商品的供给,市场价格就提高到价值以上。这当然不是什么新的发现,它所解释的这种“余额”,无论对李嘉图还是对其他任何人从来没有造成丝毫困难。[XIV—851]

\tchapternonum{(7)约翰·斯图亚特·穆勒[直接从价值理论中得出李嘉图关于利润率和工资量成反比的原理的徒劳尝试]}

\tsectionnonum{[(a)把剩余价值率同利润率混淆起来。“让渡利润”见解的因素。关于资本家的“预付利润”的混乱见解]}

[VII—319]前面引证过的那本小册子\endnote{指约翰·斯图亚特·穆勒的著作《略论政治经济学的某些有待解决的问题》1844年伦敦版。马克思在《关于生产劳动和非生产劳动的理论》这一章中引用了该书(见本卷第1册第176页)。——第208页。},实际上包括了约翰·斯图亚特·穆勒先生关于政治经济学问题的全部创见(这与他的大部头的概论\endnote{马克思指约翰·斯图亚特·穆勒的著作《政治经济学原理及其对社会哲学的某些应用》,两卷集,1848年伦敦版。——第208页。}不同)。在这本小册子的第四篇题为《论利润和利息》的“论文”中写道:

\begin{quote}{“工具和原料象其他物一样,最初除劳动外并不花费别的任何东西……制造工具和原料所耗费的劳动,加上以后依靠工具加工原料所耗费的劳动,就是生产成品所耗费的劳动总量……因此,补偿资本无非是补偿所耗费的劳动的工资。”(约·斯·穆勒《略论政治经济学的某些有待解决的问题》1844年伦敦版第94页)}\end{quote}

这一点本身就是错误的,因为所耗费的劳动和所支付的工资决不是等同的。确切地说,所耗费的劳动等于工资和利润之和。补偿资本意味着既补偿有酬劳动(工资),也补偿资本家没有付酬但被他出卖的劳动(利润)。在这里,穆勒先生把“所耗费的劳动”和其中由使用该劳动的资本家付酬的那一部分混淆起来了。这种混淆本身,对于他理解他自称在传授的李嘉图理论,并不那么有利。

关于不变资本,还要顺便指出:尽管不变资本的每个部分都可以归结为过去劳动,因而可以设想它在某个时候曾经代表利润或工资,或者代表两者,但是,这个不变资本一旦形成,它的一部分(例如种子等等)就既不可能再归结为利润,也不可能再归结为工资。

穆勒没有把剩余价值同利润区别开来。因此他宣称,利润率(这对于已经转化为利润的剩余价值来说是正确的)等于产品的价格对花费在产品上的生产资料(包括劳动在内)的价格之比。(同上,第92—93页)同时,他又想直接从李嘉图关于“利润取决于工资,工资下降则利润提高,工资提高则利润下降”[同上,第94页]的规律得出利润率的规律,而李嘉图在他的这个原理中把剩余价值同利润混淆起来了。

穆勒先生本人甚至对于他试图解决的问题也不十分清楚。因此,我们在听取他的解答之前,先把他的问题简要地表述一下。利润率是剩余价值对预付资本总额(不变资本加可变资本)之比,而剩余价值本身则是工人所完成的劳动量超过以工资形式预付给他的劳动量的余额;就是说,剩余价值只是就它对可变资本或者说对花费在工资上的资本的关系,而不是就它对全部资本的关系来考察的。因此,剩余价值率和利润率是两种不同的比率,虽然利润本身不过是从一定角度来考察的剩余价值。就剩余价值率来讲,说它完全“取决于工资,工资下降则提高,工资提高则下降”,那是正确的。(就剩余价值量来讲,这样说就不对了,因为它在同一时间内不仅取决于单个工人的剩余劳动被占有的比率,而且取决于同时被剥削的工人人数。)既然利润率是剩余价值对预付资本总价值之比,它当然要受到剩余价值的下降或提高,也就是受到工资的提高或下降的影响,并由这种情况决定;但是,除了由这种情况决定之外,利润率还包括[320]不取决于工资的提高或下降并且不能直接归结为这种情况的其他因素。

约翰·斯图亚特·穆勒先生一方面同李嘉图一起把利润和剩余价值直接等同起来,另一方面(在同反李嘉图派的论战中)又不是在李嘉图的意义上,而是在利润率的真正意义上,把利润率理解为剩余价值对预付资本(可变资本加不变资本)总价值之比,因而煞费苦心地力图证明,利润率直接由决定剩余价值的规律决定,这个规律简单地归结为:工人在自己的工作日中占有的那部分越小,归资本家所有的那部分就越大,反之亦然。现在我们就来看一看他这种煞费苦心的努力,而在这番努力中最糟糕的是,他自己也不清楚他要解决的究竟是什么问题。如果他把问题本身正确地表述出来,那他就不会错误地以这种方法去解决了。

所以,他说:

\begin{quote}{“尽管工具、原料和建筑物本身都是劳动的产物,然而,它们的价值总量毕竟不能归结为生产它们的工人的工资。{他在上面说过,补偿资本就是补偿工资。}资本家因付出工资而取得的利润必须计算在内。生产成品的资本家,不仅应该用成品补偿他自己和工具生产者付出的工资,而且应该补偿他从自己的资本中预付给工具生产者的利润。”(同上,第98页)因此,“利润不单单代表[成品生产者]在补偿费用之后的余额;它还加入费用本身。[成品生产者的]资本一部分用于支付或补偿工资,一部分用于支付其他资本家的利润,这些资本家的协力是取得生产资料所必需的”。(第98—99页)“因此,一种物品可能是和以前同量的劳动的产品,而如果最后的生产者应付给先前那些生产者的利润的某一部分能够节约下来,物品的生产费用还是会减少的……然而,利润率的变动和工资的生产费用成反比这一点,仍然是正确的。”(第102—103页)}\end{quote}

在这里,我们当然始终是从商品的价格等于它的价值这个前提出发的。穆勒先生本人也是在这个基础上进行研究的。

首先必须指出,在刚才引证的穆勒的论述中,利润看上去和“让渡利润”十分相似。但是我们且不去说它。硬说一种物品(如果按照它的价值出卖)可能“是和以前同量的劳动的产品”,而同时由于某种情况,“物品的生产费用”会“减少”,这是再荒谬不过的了。{这一点只有在我最先提出的那个意义上才是可能的,也就是说把物品的[实际]生产费用和[它对]资本家[来说]的生产费用区别开来,因为这种生产费用中有一部分是资本家不支付的。\endnote{马克思在1861—1863手稿第II本的《货币转化为资本》这一节(手稿第88页)中,说明了这一区别:“资本家的生产费用只是他所预付的价值总额,因而产品的价值等于预付资本的价值。另一方面,产品的实际生产费用等于产品中包含的劳动时间的总和。但是产品中包含的劳动时间的总和大于资本家预付的或者说支付过代价的劳动时间的总和,产品价值中超过由资本家支付过代价的或预付的价值的这个余额,恰好形成剩余价值”。马克思在手稿第XIV本关于托伦斯的一节(见本册第81—86页)和第XV本关于庸俗政治经济学的一节中(见本册第569—570页),又回过来谈了这个问题。——第211页。}在这种情况下,说资本家靠自己工人的无酬剩余劳动获取利润,正象他可能通过对提供不变资本给他的资本家支付不足而获取利润,就是说,通过对这个资本家商品中包含的、未由这个资本家付酬的剩余劳动(这种劳动正是因此形成他的利润)的一部分不支付而获得利润一样,——这确实是对的。这始终归结为:他低于商品的价值对商品支付。利润率(即剩余价值对预付资本总价值之比)的提高,既可以由于同量预付资本在客观上变便宜了(生产不变资本的生产部门中劳动生产率提高的结果),也可以由于它对买者来说在主观上变便宜了,即买者低于它的价值支付。在这种情况下,对买者来说,它始终是较小劳动量的结果。}

[321]在上面引证的那段话里,穆勒首先表述的是这样一个思想:生产成品的资本家的不变资本,不仅分解为工资,而且分解为利润。在这里他的思路是这样的:

如果最后的资本家所预付的不变资本仅仅归结为工资,那末利润就是他在补偿构成预付资本总额的全部工资后剩下的余额{而预付在成品生产上的全部(支付的)费用就归结为工资}。预付资本总价值就等于产品中包含的全部工资的价值。利润就是超过这个总额的余额。既然利润率等于这个余额对预付资本总价值之比,那末,它的提高或降低显然就取决于预付资本总价值,即取决于全部构成预付资本的工资的价值。{如果考察的是利润和工资的一般关系,这个论据本身事实上就是荒诞无稽的。其实,穆勒先生只要把总产品中分解为利润的部分(不管这笔利润是付给最后的资本家,还是付给参与商品生产的先前的那些资本家,都无关紧要)放在一边,把分解为工资的部分放在另一边,那末,利润额就仍然会等于超过工资额的价值的余额,而李嘉图的“反比例”就可以直接适用于利润率了。但是,说预付资本总额分解为利润和工资,是不正确的。}然而,最后的资本家预付的资本不仅分解为工资,而且分解为预付利润。因此,最后的资本家的利润不仅是超过预付工资的余额,而且是超过预付利润的余额。可见,利润率不仅由超过工资的余额决定,而且由留在最后的资本家手里的超过工资和利润的总额(根据假定,这个总额构成全部预付资本)的余额决定。因此,这个比率显然不仅会因工资的提高或降低而变动,而且会因利润的提高或降低而变动。如果我们把利润率由于工资的提高或降低而发生的变动撇开不谈,如果我们假定,——其实在实践中时常会遇到这种情况,——工资的价值(即工资的生产费用,包含在工资中的劳动时间)保持不变,那末,跟着穆勒先生走下去,就会得出一条绝妙的规律:利润率的提高或降低取决于利润的提高或降低。

\begin{quote}{“如果最后的生产者应付给先前那些生产者的利润的某一部分能够节约下来,物品的生产费用还是会减少的。”}\end{quote}

这一点在事实上是很正确的。假定先前那些生产者的利润中没有任何一部分是纯粹的附加额,或如詹姆斯·斯图亚特所说的让渡利润,那末,“利润的”某一“部分”的节约{只要这种节约不是由于后来的生产者欺骗先前的生产者,也就是说不是由于没有把先前的生产者的商品中包含的价值全部付给他}都是商品生产所需的劳动量的节约。{在这里,我们把,比如说,为资本在生产期间闲置不用的时间等等而支付的利润撇开不谈。}举例来说,为了把原料,比如煤,从矿井运到工厂去,以前需要两天,而现在只要一天就够了,这样就“节约了”一个工作日;但是,这一点既与这个工作日中分解为工资的部分有关,也与其中分解为利润的部分有关。

穆勒先生在自己弄清楚了最后的资本家的余额的比率,或一般地说利润率,不仅取决于工资和利润的直接比例,而且取决于最后的利润或每个特定的资本家的利润对预付资本总价值,即(花费在工资上的)可变资本加不变资本的总和之比以后,换句话说,[322]在弄清楚了利润率不仅仅决定于利润对花费在工资上的资本部分之比,即不仅仅决定于工资的生产费用或者说工资的价值以后,又接着说:

\begin{quote}{“然而,利润率的变动和工资的生产费用成反比这一点,仍然是正确的。”}\end{quote}

尽管这是错误的,“然而……仍然是正确的”。

穆勒在这方面所作的例证,可以看成是政治经济学家所特有的例证方法的典范,而且由于这个例证的作者还写过一本逻辑学的书\endnote{马克思指约翰·斯图亚特·穆勒的著作《推论和归纳的逻辑体系,证明的原则与科学研究方法的关系》,两卷集,1843年伦敦版。——第213页。},这就更加令人惊异了。

\begin{quote}{“例如,假定有60个农业工人,他们领取60夸特谷物作为工资,他们用去价值也是60夸特的固定资本和种子;他们的劳作的产品是180夸特。假定利润率是50%,那末,生产180夸特谷物所用的种子和工具就必然是40个工人劳动的产品;因为这40个工人的工资连同他们的雇主的利润共60夸特。因此,如果产品是180夸特,那就是总共100个工人的劳动结果。现在再假定,仍旧是100个工人的劳动,但是由于某种发明,不需要任何固定资本和种子了。以前只有支出120夸特才能取得180夸特的结果;现在只支出100夸特就可以了。180夸特谷物仍然是和以前同量的劳动,即100个工人的劳动的结果。因此,一夸特谷物仍然是一个工人劳动的10/18的产品。因为作为一个工人的报酬的一夸特谷物,实际上是和以前同量的劳动的产品;然而,它的生产费用却减少了。现在一夸特谷物是一个工人劳动的10/18的产品,再没有别的;以前生产一夸特谷物,则需要这一劳动量同补偿[先前那个资本家的]利润的费用结合在一起,即多支出1/5。如果工资的生产费用仍然和以前一样,利润就不可能提高。每个工人以前取得1夸特谷物;但是以前1夸特谷物是现在1+(1/5)夸特的生产费用的结果。因此,为了使每个工人取得和以前一样多的生产费用,每人就应该有1夸特加1/5夸特的谷物。”(同上,第99—103页)“因此,假定付给工人的是工人自己生产的产品,那就很明显,当这种产品的生产费用有了某种节约而工人照旧得到以前的生产费用时,他得到的产品就必然会同资本的生产力的提高成比例地增加。但如果是这样,资本家的支出和他所得到的产品之间的比例,就会和以前完全一样,利润也不会提高。因此,利润率的变动和工资的生产费用的变动是同时发生的,是不可分割的。由此可见,如果李嘉图所说的低工资不仅指作为较小劳动量的产品的工资,而且指用较少的费用——包括劳动和先前的资本家的利润在内——生产的工资,那末他的意见就是完全正确的。”(同上,第104页)}\end{quote}

关于这个出色的例证,我们首先要指出:在这个例证中假定,由于某种发明,谷物不用种子(原料),不用固定资本就可以生产出来;不用原料,不用劳动工具,就是说,光凭两只手,就可以用空气、水和土地制造出来。[323]在这个荒谬的假定下掩盖的不外是这样一个假定:不用不变资本,就是说,只靠新加劳动,就可以生产产品。在这种情况下,当然可以证明原来应该证明的东西,即利润和剩余价值是等同的,从而利润率也仅仅取决于剩余劳动对必要劳动之比。可是,困难正好是由于下面这一点产生的:因为剩余价值[不仅和资本的可变部分,而且]和资本的不变部分发生比例关系,——这个比例关系我们称为利润率,——所以剩余价值率和利润率彼此是不同的。因此,如果我们假定不变资本等于零,那末,由于不变资本的存在而产生的困难,就被我们用撇开这个不变资本的存在的办法排除了。换句话说,我们用假定困难不存在的办法排除了困难。这倒是一个行之有效的办法。

现在我们把问题,或者说穆勒对问题的例证,正确地表述一下。

在第一个假定中,我们看到:

\todo{}

在这个例子中假定:加到不变资本上的劳动等于120夸特。因为每一夸特是一个工作日(或一个工人的年劳动,它可以看作由365工作日构成的一个工作日)的工资,所以180夸特总产品只包含60工作日[新加劳动],其中30补偿工人的工资,30构成利润。可见,我们实际上是假定,一个工作日物化在两夸特中;因此,60个工人把他们的60工作日物化在120夸特中,其中60是他们的工资,60形成利润。换句话说,工人用半个工作日为自己工作,补偿工资,用另外半个工作日为资本家工作,从而为资本家创造剩余价值。因此,剩余价值率在这里是100%,而不是50%。但是,因为可变资本只占全部预付资本的一半,所以利润率就不是表现为60夸特比60夸特,而是60夸特比120夸特,就是说,不是100%,而只是50%。如果资本的不变部分等于零,全部预付资本就仅仅由60夸特,仅仅由预付在工资上的等于30工作日的资本构成;利润和剩余价值,从而利润率和剩余价值率也就等同了。这样,利润就是100%,而不是50%。两夸特谷物是一个工作日的产品,120夸特谷物是60工作日的产品,尽管作为一个工作日的工资只有一夸特谷物,作为60工作日的工资只有60夸特谷物。换句话说,工人只得到自己的产品的一半(50%),而资本家得到的却是自己的费用的两倍,即得到100%的利润。

那60夸特不变资本的情况又是怎样的呢?它同样是30工作日的产品。假定在这笔不变资本中,它的生产要素相互间的比例和上面假定的一样,即1/3是不变资本,2/3是新加劳动;其次,假定剩余价值率和利润率也和上面假定的一样,我们就会得出如下的计算数字:

\todo{}

在这里,利润率又是50%,剩余价值率100%。总产品是[324]30工作日的产品,但是其中10工作日(=20夸特)是过去劳动(不变资本),20工作日是20个工人新加的劳动,但是他们每人只得到自己的产品的一半作为工资。两夸特仍然是一个工人劳动的产品,尽管其中一夸特仍然是一个工人劳动的工资,一夸特是资本家的利润——资本家占有了工人劳动的一半。

生产成品的资本家作为剩余价值取得的60夸特,形成50%的利润率,因为这60夸特的剩余价值不仅按预付在工资上的60夸特,而且按预付在种子和固定资本上的60夸特,即总共按120夸特计算。

由此可见,如果穆勒把生产种子和固定资本(总价值为60夸特)的资本家的利润也按50%计算,如果他还假定这里不变资本和可变资本的比例也同生产180夸特谷物时一样,那末他就应当说,不变资本生产者的利润等于20夸特,工资等于20夸特,不变资本等于20夸特。既然工资等于一夸特,60夸特就包含30工作日,正象120夸特包含60工作日一样。

但穆勒是怎样说的呢?

\begin{quote}{“假定利润率是50%,那末,生产180夸特谷物所用的种子和工具就必然是40个工人劳动的产品;因为这40个工人的工资连同他们的雇主的利润共60夸特。”}\end{quote}

第一个资本家雇用了60个工人,付给每人每天1夸特(就是说,这个资本家在工资上花费60夸特),另外,他还在不变资本上花费60夸特;在这个资本家那里,60工作日物化在120夸特中,但是工人从其中仅仅得到60夸特作为工资。换句话说,工资只占60个工人劳动产品的一半。因此,60夸特的不变资本总共只等于30个工人劳动的产品;如果这60夸特仅仅由利润和工资构成,那末工资占30夸特,利润占30夸特;因此,工资等于15个工人的劳动,利润也是如此。但是,如果利润只占50%,那是因为,按照假定,在60夸特不变资本所包含的30天中,10天是过去劳动(不变资本),分解为工资的只有10天。总之,10天包含在不变资本中,20天是新加的工作日,然而在这20天中,工人仅仅为自己劳动10天,另外10天则为资本家劳动。但是,穆勒先生却硬说这60夸特是40个工人劳动的产品,而以前120夸特却是60个工人劳动的产品。以前一夸特包含半个工作日(虽然一夸特是整个工作日的工资),而现在则是3/4夸特等于半个工作日。其实,花费在不变资本上的1/3产品(60夸特)具有的价值,即包含的劳动时间,同其他任何一个1/3的产品完全一样。即使穆勒先生想把60夸特的不变资本完全分解为工资和利润,那也丝毫改变不了其中包含的劳动时间量。其中包含的仍旧是30工作日;只不过利润和剩余价值一致了,因为这里没有不变资本需要补偿。因此,利润是100%,而不象以前那样是50%。以前剩余价值也是100%,但是利润只是50%,这正是由于在计算利润时把不变资本算进去了。

这样,我们在这里就看到穆勒先生的双重错误手法。

就前一个180夸特来说,困难在于剩余价值同利润不一致,因为60夸特的剩余价值不仅应当按60夸特(总产品中等于工资的部分)计算,而且[325]应当按120夸特即60夸特的不变资本加60夸特的工资计算。因此,剩余价值是100%,而利润只是50%。就构成不变资本的那60夸特来说,穆勒先生是这样排除上述困难的:他假定在这里全部产品是在资本家和工人之间分配的,也就是说,在创造由总价值60夸特的种子和劳动工具构成的不变资本时,没有任何不变资本参加。对资本I来说应该加以说明的情况,对资本II来说被事先假定不存在,从而问题本身就消失了。

其次,穆勒先假定在构成I的不变资本的60夸特的价值中只包括[直接]劳动,这里不存在过去劳动,不存在不变资本,从而利润同剩余价值——也就是说,还有利润率同剩余价值率——是一致的,它们之间没有任何区别;然后又反过来假定,象在I的场合一样,它们之间是有区别的,因此利润也象在I的场合一样只占50%。如果在I的场合产品中1/3不是由不变资本构成,那末利润同剩余价值就一致了。总产品只有120夸特,等于60工作日,其中工人取得30(=60夸特),资本家取得30(=60夸特)。利润率便等于剩余价值率——100%。实际上它等于50%,因为60夸特的剩余价值不是按60夸特(工资),而是按120夸特(工资、种子和固定资本)计算的。在II的场合,穆勒假定生产是不用任何不变资本的。他还假定这里的工资也没有变动,仍是一夸特。然而他还是设想利润和剩余价值在这里是彼此不同的,即利润只占50%,虽然剩余价值达100%。实际上穆勒所假定的是:占总产品1/3的这60夸特所包含的劳动时间,比总产品其他的1/3包含的多;这60夸特是40工作日的产品,而其余120夸特仅仅是60工作日的产品。

可是,实际上这里流露出关于“让渡利润”的陈旧谬见,这种利润同产品中包含的劳动时间以及同李嘉图的价值规定,是毫无关系的。这就是说,穆勒假定:一个工人的一个工作日的工资,等于他的一个工作日的产品,或者说,他劳动了多少时间,工资中就包含多少劳动时间。如果作为工资付出的是40夸特,利润等于20夸特,那末,作为工资付出的40夸特就包含40工作日。40工作日的报酬等于40工作日的产品。如果60夸特总产品有50%即20夸特的利润,那末,由此得出结论,40夸特等于40个工人的劳动的产品,因为根据上述假定,40夸特构成工资,而且1个工人1天领取1夸特。那末其余20夸特是从哪里来的呢?这40个工人劳动了40工作日,因为他们取得了40夸特。因而,1夸特是1工作日的产品。40工作日也就只生产40夸特,不会多生产1蒲式耳。那构成利润的20夸特又是从哪里来的呢?穆勒的这个例子所依据的,是关于“让渡利润”——产品价格超出产品价值之外的纯粹名义上的余额——的陈旧谬见。但是在这里,价值不是以货币表现,而是以产品本身的相应部分表现的,因此这种“让渡利润”是绝对荒谬的,是不可能的。如果40夸特谷物是40个工人的劳动的产品,他们每人每天(或每年)取得1夸特的工资,即取得自己的全部产品作为工资,如果以货币表示,1夸特谷物等于3镑,40夸特等于120镑,那末,资本家把这40夸特卖了180镑,就取得了60镑的利润,即50%的利润(60镑=20夸特谷物)——这样设想是最容易的。但是,如果硬说资本家从他的40个工人在40工作日中生产的、由他支付了40夸特工资的40夸特谷物中卖出了60夸特,那末上述设想本身就显得妄诞无稽了。他手里只有40夸特,可是他卖出了60夸特,也就是说,比他拥有供出卖的量多卖了20夸特。

[326]总之,穆勒首先是企图利用下列美妙的假定来证明李嘉图的规律(即李嘉图把剩余价值同利润混淆起来的错误规律):

(1)假定生产不变资本的资本家本身不需要任何不变资本;这样,穆勒就用他的这个假定把不变资本带来的困难全部排除了;

(2)假定在没有不变资本的情况下,不变资本带来的剩余价值和利润之间的区别仍然继续存在,尽管这里没有任何不变资本;

(3)假定生产40夸特小麦的资本家可以卖出60夸特小麦,因为他的总产品是作为不变资本卖给另一个资本家的,而那个资本家的不变资本等于60夸特;还因为资本家II用这60夸特取得一笔50%的利润。

后面这个谬论就是“让渡利润”的观念,它所以在这里表现得如此荒唐,只是因为应该构成利润的不是以货币表现的名义价值,而是所出卖的产品本身的一部分。这样,穆勒先生想为李嘉图辩护,却离开了李嘉图的根本观点,并远远落在李嘉图、亚·斯密以及重农学派的后面。

可见,穆勒为李嘉图学说作的第一个辩护,就是他从一开始就推翻了这个学说,也就是说推翻了它的这样一个根本原理:利润只是商品价值的一部分,就是说,只是商品所包含的劳动时间中由资本家随着他的产品出卖但没有给工人付报酬的那一部分。穆勒认为,资本家对工人的全部工作日付了报酬,但是仍然取得利润。

我们再看看穆勒接着是怎样做的。

他假定由于某种发明在生产谷物时不再需要使用种子和农具了;也就是说,根据他的这一假定,对最后的资本家来说,不再需要不变资本了,就象他已为用在种子和固定资本上的先前那60夸特的生产者作的假定一样。现在,穆勒本应这样推论下去:

资本家I现在就不必在种子和固定资本上花费60夸特,因为我们已经说过,他的不变资本等于零。因而,他只须在劳动60工作日的60个工人的工资上花费60夸特。这60工作日的产品等于120夸特。工人只得到60夸特。因此,资本家得到60夸特的利润,即100%。他的利润率恰恰等于剩余价值率,也就是恰恰等于工人不是为自己而是为资本家劳动的劳动时间[对他们为自己劳动的劳动时间之比]。他们一共劳动了60天。他们生产120夸特,得到60夸特作为工资。因而,尽管他们劳动了60天,可是他们作为工资得到的是30工作日的产品。2夸特所需的劳动时间量,仍然等于1工作日。由资本家付酬的工作日,仍然等于1夸特,即等于已完成的工作日的半数。产品减少了1/3,从180夸特减少到120夸特;但是利润率提高了50,即从50%增长到100%。为什么?以前180夸特有1/3是专门补偿不变资本的支出的,因此,它既不加入利润,也不加入工资。另一方面,工人为资本家生产的那60夸特(工人为资本家劳动的那30工作日),不是按花在工资上的60夸特(不是按工人为自己劳动的那30工作日),而是按花在工资、种子和固定资本上的120夸特(60工作日)计算的。因此,尽管工人在60天中为自己劳动了30天,为资本家劳动了30天,尽管资本家在工资上的60夸特支出为他提供了120夸特产品,他的利润率也不是100%,而只是50%,这是因为利润率的计算方法不同:在一种情况下按2×60计算,在另一种情况下则按60计算。剩余价值[327]在这两种情况下是相同的,但是利润率不同。

可是穆勒是怎样对待这个问题的呢?

他不是假定资本家[在采用了那个可以不用不变资本的发明之后]在支出60夸特时得到120夸特(从60工作日中占有30工作日);根据他的假定,资本家现在雇用100个工人,他们向他提供180夸特,而且始终假定1工作日的工资等于1夸特。这样,就得出如下的计算数字:

\todo{}

这样,资本家现在得到80%的利润。利润在这里等于剩余价值。因此,剩余价值率也仅仅等于80%;它以前等于100%,即比现在高20。于是我们在这里就看到这样一种现象:利润率提高了30,而剩余价值率降低了20。

如果资本家在工资上仍然只花费60夸特,那就会得出如下的计算数字:

100夸特提供80夸特的剩余价值

10夸特提供8夸特的剩余价值

60夸特提供48夸特的剩余价值

但是以前60夸特提供了60夸特的剩余价值(可见降低了20%)。或者用另一个方式来表示,以前:

\todo{}

由此可见,剩余价值(在这两种情况下,我们都应该按100夸特计算)从100夸特降低到80夸特,即降低了20%。

\todo{}

下面我们分析一下物化在一夸特中的劳动时间,即一夸特的价值。以前是2夸特等于1工作日,或者说1夸特等于半个工作日,即等于一个工人一天劳动的9/18。而现在,180夸特是100工作日的产品;因此,1夸特是100/180工作日的产品,即10/18工作日的产品。换句话说,产品贵了1/18工作日,或者说劳动的生产率降低了,因为以前一个工人生产1夸特,只需要9/18工作日,而现在需要10/18工作日。尽管剩余价值降低了,劳动生产率因而也降低了,也就是说,尽管工资的实际价值(生产费用)提高了1/18,即11+(1/9)%,但是利润率提高了。以前,180夸特是90工作日的产品(1夸特是90/180工作日即9/18或1/2工作日的产品)。而现在,180夸特则是100工作日的产品(1夸特是100/180即10/18工作日的产品)。假定1工作日等于12小时,或60×12=720分。[328]这样,1/18工作日就等于720/18分,即40分。从这720分中,工人在第一种场合提供给资本家720分的半数,即360分。因此,60个工人一共提供给他360×60分。在第二种场合,工人从720分中提供给资本家8/18,即仅仅320分。但是,第一个资本家雇用60个工人,从而占有360×60=21600分。第二个资本家雇用100个工人,从而占有320×100=32000分。于是第二个资本家的利润就多于第一个,因为100个工人每人每天提供320分,多于60个工人每人每天提供360分。可见,所以发生这种情况,仅仅因为第二个资本家多雇用了40个工人;但是他从每个工人身上取得的却少于第一个资本家。第二个资本家获得了更多的利润,尽管这里剩余价值率和劳动生产率降低了,实际工资的生产费用(即工资中包含的劳动量)提高了。然而穆勒先生却想证明与此截然相反的事情。\endnote{按照前面第213页所引的穆勒的错误论断,使用了不变资本和60个工人的资本家I,为了生产1夸特谷物(一个工人的工资),花费6/9夸特谷物(120/180=2/3=6/9),而不用不变资本、光使用100个工人的资本家II,在1夸特谷物上仅仅花费5/9夸特(100/180=5/9),就是说,资本家II的每个工人“工资的生产费用”低1/9夸特,换句话说,资本家I的这些“工资的生产费用”比资本家II的多五分之一(20%)。——第224页。}

假定没有采用这种使生产谷物可以不用种子和固定资本的“发明”的资本家I,也(象资本家II那样)使用了100工作日(而在上面的计算中他只使用了90工作日)。这样,他就要多使用10工作日,其中3+(1/3)用于他的不变资本(种子和固定资本),3+(1/3)用于工资。在生产保持原来的发展水平的条件下,这10工作日的产品等于20夸特,但是其中6+(2/3)夸特补偿不变资本,12+(4/3)夸特是6+(2/3)工作日的产品。在这些产品里,6+(2/3)夸特构成工资,6+(2/3)夸特构成剩余价值。

于是,我们就得出如下的计算数字:

不变资本

资本家I在100工作日的总产品中取得33+(1/3)工作日的利润,即在200夸特的总产品中取得66+(2/3)夸特的利润。或者说,如果我们把他支出的资本按夸特来计算,那末他就是用133+(1/3)夸特(66+(2/3)工作日的产品)取得66+(2/3)夸特的利润。然而资本家II支出100夸特取得80夸特的利润。可见,资本家II的利润多于资本家I的利润。但是,资本家I生产200夸特所用的劳动时间,与资本家II生产180夸特所用的劳动时间一样多。资本家I的一夸特等于半个工作日,而资本家II的一夸特等于10/18(即5/9)工作日,即多包含1/18的劳动时间,因此比前者贵1/18。所以,资本家I就必定会挤垮资本家II。后者则不得不放弃他的发明,安于象以前一样使用种子和固定资本来生产谷物。

资本家I用120夸特取得的利润是60夸特,即50%(这和用133+(1/3)夸特取得利润66+(2/3)夸特是一样的)。

资本家II用100夸特取得的利润则是80夸特,即80%。

II的利润:I的利润=80∶50=8∶5=1∶5/8。

与此相反,II的剩余价值:I的剩余价值=80∶100=8∶10=1∶10/8=1∶1+(2/8)=1∶1+(1/4)。

II的利润率比I的利润率高30。

II的剩余价值则比I的剩余价值低20。

资本家II多雇用了66+(2/3)%的工人,而资本家I则每个工作日从每个工人那里只多占了1/8(即12+(1/2)%)的劳动。

[329]总之,穆勒先生事实上证明了:资本家I——他共使用90工作日,其中1/3包含在不变资本(种子、机器等等)中,他雇用60个工人,但是只用30天的产品付报酬——用半个(即9/18)工作日生产1夸特谷物,用90工作日生产180夸特,其中60夸特补偿不变资本中包含的30工作日,60夸特补偿60工作日的工资(即30工作日的产品),60夸特补偿剩余价值(即30工作日的产品)。这个资本家I的剩余价值等于100%。他的利润则等于50%,因为60夸特剩余价值不是按60夸特,不是按花在工资上的那部分资本计算,而是按120夸特,也就是按多一倍的资本(可变资本加不变资本)计算的。

其次,穆勒证明了:资本家II——他使用100工作日,但是他(由于自己的发明)丝毫没有把工作日花费在不变资本上——生产180夸特的产品,因此1夸特等于10/18工作日,即比资本家I的夸特贵1/18工作日(40分)。他的工人的劳动生产率低1/18。既然工人仍旧一工作日取得1夸特作为报酬,那末他的工资就实际价值来说,即就生产它所需的劳动时间来说,提高了1/18。尽管现在工资的生产费用提高了1/18,尽管资本家II的总产品同所耗费的劳动时间相比减少了,而且他所生产的剩余价值只占80%,而资本家I的剩余价值是100%,但是资本家II的利润率还是80%,而资本家I的利润率只是50%。为什么呢?因为在资本家II那里,尽管工资的生产费用提高了,可是他雇用工人多了;还因为资本家II的剩余价值率同利润率相等,其所以如此,则是由于他所生产的剩余价值只按花在工资上的资本计算,而不变资本等于零。但是,穆勒恰恰与此相反,想要证明,利润率的提高,根据李嘉图的规律,是工资的生产费用减少的结果。我们已经看到,尽管工资的生产费用已经增加,利润率还是提高了;因此,如果把利润和剩余价值直接等同起来,而又把利润率理解为剩余价值或总利润(它等于剩余价值)对全部预付资本总价值之比,那末李嘉图规律就是错误的。

穆勒先生接着说:

\begin{quote}{“以前只有支出120夸特才能取得180夸特的结果;现在只支出100夸特就可以了。”}\end{quote}

穆勒先生忘了,在第一种场合,支出120夸特等于耗费60工作日,而在第二种场合,支出100夸特等于耗费55+(5/9)工作日(也就是说,在第一种场合,1夸特等于9/18工作日,而在第二种场合,等于10/18工作日)。

\begin{quote}{“180夸特谷物仍然是和以前同量的劳动,即100个工人的劳动的结果。”}\end{quote}

对不起!这180夸特,以前是90工作日的结果,而现在是100工作日的结果。

\begin{quote}{“一夸特谷物仍然是一个工人劳动的10/18的产品。”}\end{quote}

对不起!以前这是一个工人劳动的9/18的产品。

\begin{quote}{“因为作为一个工人的报酬的一夸特谷物,实际上是和以前同量的劳动的产品。”}\end{quote}

对不起!第一,1夸特谷物现在“实际上”是10/18工作日的“产品”,而以前是9/18工作日的产品;因此多耗费了1/18工作日的劳动;第二,一个工人的报酬,不管它耗费了9/18工作日还是10/18工作日,任何时候都不能同他的劳动的产品混为一谈;它始终只是这种产品的一部分。

\begin{quote}{“现在一夸特谷物是一个工人劳动的10/18的产品,再没有别的〈这是正确的!〉;以前生产一夸特谷物,则需要这一劳动量同补偿[先前那个资本家的]利润的费用结合在一起,即多支出1/5。”[同上,第100—103页]}\end{quote}

且慢!第一,[330]我们再说一遍,认为一夸特以前耗费10/18工作日,那是错误的;它只花费9/18工作日。更加错误的是(如果绝对错误的东西可以有程度差别的话),这9/18工作日再加上“利润的补偿,即多支出1/5”。90工作日(不变资本和可变资本计算在一起)生产了180夸特。180夸特=90工作日;1夸特=90/180=9/18=1/2工作日。可见,在这9/18工作日上面,即在I的场合1夸特所耗费的半个工作日上,没有任何“附加额”。

但是,我们在这里发现一种十足的谬论,所有上述一派胡言都是以隐蔽的形式围绕着它提出的。穆勒愚弄了自己,首先是因为他假定:如果说120夸特是60工作日的产品,并且这项产品在资本家和60个工人之间平分,那末,构成不变资本的那60夸特便是40工作日的产品。实际上,不管资本家和生产这60夸特的工人按照什么比例分配产品,它们只能是30工作日的产品。

但是,我们且不谈这一点。为了把穆勒的谬论彻底弄清楚,我们假定归结为利润的不是60夸特不变资本的1/3(即不是20夸特),而是全部60夸特。由于这种假定对穆勒有利而不是对我们有利,由于它使问题简化,我们就更可以作这样的假定。此外,与其认为穆勒的资本家有一种“发明”,使他不用种子和固定资本就能生产出180夸特谷物,还不如认为生产60夸特不变资本的资本家有一种发明,使他能够强迫30个工人白白地、不要任何报酬地(就象徭役劳动形式那样)劳动30工作日,生产出60夸特或60夸特的价值。总之,我们假定:上述60夸特仅仅包含给资本家I生产不变资本的资本家II的利润;资本家II应该卖出30工作日的产品,而对他的30个各劳动了一个工作日的工人不付分文。能不能说这笔仅仅归结为利润的60夸特加入资本家I的工资生产费用,并同他的工人所完成的劳动时间“结合在一起”呢?

当然,资本家I和他的工人如果没有那构成他们的不变资本并仅仅归结为利润的60夸特,是不能生产出120夸特的(根本连一夸特也生产不出来)。对于他们说来,这是必要的生产条件,而且是应该付酬的生产条件。可是,他们需要这60夸特,是为了生产出180夸特。在这180夸特中,有60夸特是用来补偿上述60夸特的。对不变资本的这种补偿,丝毫也不触动他们的120夸特,即他们的60工作日的产品。如果他们不用构成他们的不变资本的那60夸特也能生产出120夸特,那末他们的产品,即60工作日的产品,会照旧是那些;可是,总产品少了,这正是由于以前的那60夸特没有再生产出来。资本家的利润率提高了,因为在使他有可能取得60夸特剩余价值的生产条件上的支出,或者说费用,不加入他的生产费用。绝对利润照旧,即60夸特。但是这60夸特只要他花60夸特的支出。而现在他却为此支出120夸特。因而,这种花在不变资本上的支出,加入资本家的生产费用,而不加入工资的生产费用。

假定由于某种“发明”,资本家III也可以不给自己的工人付酬而用15[不是30]工作日生产出60夸特,这部分地是因为他采用了更好的机器等等。这个资本家III会把资本家II排挤出市场,并把资本家I变成自己的固定买者。这样,资本家I的支出现在便[331]由60工作日减少到45工作日。可是,工人为了用60夸特[不变资本]生产180夸特的产品,仍旧要花费60工作日。而且他们还是需要用30工作日来生产自己的工资。对他们来说,1夸特等于半个工作日。然而这180夸特要资本家花费的只是45工作日,而不是60工作日。但是,认为谷物在称为种子时比称为谷物时花费较少的劳动时间,那是荒谬的,因此,我们应该假定:在第一个60夸特中,种子花费的代价和以前是一样的,但是需用的种子少了;或者,作为固定资本包含在这60夸特中的那个价值组成部分便宜了。

\centerbox{※     ※     ※}

我们首先把迄今分析穆勒的“例证”得出的结果写下来。

第一,弄清楚了以下一点:

我们曾经假定,120夸特谷物是在没有任何不变资本的条件下生产出来的,并且仍旧是60工作日的产品,而以前,包括60夸特不变资本在内的180夸特,是90工作日的产品。在这种情况下,花在工资上的60夸特资本——它等于30工作日,但是支配了60工作日——仍旧提供同样多的产品,即120夸特。这项产品的价值也保持不变,即1夸特等于半个工作日。固然,产品在以前是180夸特,而不是现在的120夸特;但是这60夸特差额只是代表不变资本中包含的劳动时间。可见,工资的生产费用和工资本身——无论是它的使用价值,还是它的交换价值——都保持不变:1夸特谷物等于半个工作日。剩余价值也不变,即60夸特比60夸特,或1/2工作日比1/2工作日。剩余价值率在两种场合都是100%。但是利润率在第一种场合只是50%,而现在是100%。这仅仅是因为60∶60=100%,而60∶120=50%。在这里,利润率的提高,并不是由于工资的生产费用变动,而只是由于假定不变资本等于零。在下述情况下,结果也大致相同:不变资本的价值、从而预付资本总价值减少,剩余价值对资本之比因此提高,而这个比例也就是利润率。

作为利润率,剩余价值不仅按实际增加的、创造剩余价值的那部分资本计算,即不仅按花在工资上的那部分资本计算,而且按仅仅是重新出现在产品中的原料和机器的价值计算,并且,还要按全部机器设备的价值,即不仅按机器设备中确实进入价值形成过程、因而其磨损应该得到补偿的那部分价值,而且按仅仅进入劳动过程的那部分价值计算。

第二,

穆勒的第二个例子假定:资本I提供了180夸特的产品,这些产品等于90工作日(从而60夸特或30工作日构成不变资本;60夸特为可变资本,即60工作日的工资,从60工作日中工人得到30工作日的报酬;60夸特构成剩余价值),资本II也提供了180夸特,但是这180夸特等于100工作日,从而180夸特中100夸特为工资,80夸特为剩余价值。在这里,全部预付资本用于工资。这里的不变资本等于零;工资的实际价值提高了,尽管工人所取得的使用价值仍旧是1夸特;但是现在1夸特等于10/18工作日,而以前1夸特只等于9/18工作日。剩余价值从100%下降到80%,即降低了1/5或20%。利润率则从50%上升到80%,即提高了3/5或60%。由此可见,工资的实际生产费用在这种情况下甚至不是[象在“第一”的情况下那样]仅仅保持不变,而是增加了。劳动的生产率降低了,因而工人所完成的剩余劳动量减少了。虽然如此,利润率却提高了。为什么呢?第一,因为这里没有不变资本,利润率因而等于剩余价值率。在资本不是完全花在工资上——而在资本主义生产的条件下,资本完全花在工资上几乎是不可能的——的一切情况下,利润率始终应该低于剩余价值率,而它低于剩余价值率的程度,应该同预付资本总价值高于资本中花在工资上的那个组成部分的价值的程度相适应。第二,利润率提高了,是因为资本家II比资本家I雇用的工人多得多,比补偿这两个资本家各自使用的劳动在生产率上的差额所需的工人人数还多得多。

第三,如果从事情的一个方面来看,那末,上文中“第一”和“第二”两段所引证的事例就足以证明利润率的变动可以完全不取决于工资的生产费用的论点。因为“第一”这一段已经指出,尽管劳动的生产费用保持不变,利润率也会提高。“第二”这一段则指出,资本II与资本I相比,尽管劳动生产率降低,因而工资的生产费用提高,但是利润率还是提高了。因此,这种[VIII—332]情况本身向我们证明,如果我们反过来以资本I与资本II相比,那末,尽管剩余价值率提高,劳动生产率增长,因而工资的生产费用下降,但是利润率降低了。对I来说,[一夸特]工资的生产费用只是9/18工作日,对II来说,则是10/18工作日,尽管如此,II的利润率还是比I高60%。在所有这些场合,利润率的变动不仅不决定于工资生产费用的[方向相反的]变动,反而与它的方向一致。诚然,必须指出,不能因此认为一个运动是另一运动的原因(例如,不能认为利润率降低是由于工资的生产费用降低;也不能认为利润率提高是由于工资的生产费用提高),而只能认为其他情况抵销了这些变动的对立作用。但是无论如何,李嘉图的下述规律是错误的:利润率按照和工资变动相反的方向变动,一方因另一方降低而提高,或者反过来。这个规律只有对剩余价值率来说才是正确的。但是利润率与工资的价值不是按照相反的方向,而是按照相同的方向提高或降低,这里甚至有(固然不是永远有)某种必然的相互联系。哪里劳动的生产率较低,哪里就使用更多的手工劳动。而凡是劳动的生产率较高的地方,使用不变资本就更多。因此,在这里,引起剩余价值率提高或降低的同一些情况,必定引起利润率相反的变动,也就是说,必定在剩余价值率提高时引起利润率降低,等等。

\tsectionnonum{[(b)成品的生产和生产这个成品的不变资本的生产结合在一个资本家手里会不会影响利润率]}

我们现在且按照穆勒原来的设想讲一讲这个情况,尽管他没有正确表述它。这样同时也可以弄清他关于资本家的预付利润的种种议论的真正意思。

无论依靠什么样的“发明”和“结合”\authornote{见本册第214页:“……则需要这一劳动量同补偿利润的费用结合在一起……”——编者注},照穆勒那样举出的例子是站不住脚的,因为这个例子包含着绝对的矛盾和绝对的荒谬,它本身的各种前提是互相抵触的。

按照穆勒的意见,180夸特产品中有60夸特(种子和固定资本)等于利润20夸特和40工作日的工资40夸特。这样,如果20夸特的利润不存在,40工作日仍旧存在。根据这个假定,工人也就取得自己劳动的全部产品,这样就绝对无法理解,20夸特的利润及其价值是从哪里来的。如果假定它只不过是价格的名义附加额,也就是说,如果它不代表被资本家据为己有的劳动时间,那末它不存在,也应当盈利,完全象在60夸特中有20夸特算作没有作过工的工人的工资一样。其次:60夸特在这里只是不变资本的价值的表现。但是按照穆勒的意见,这是40工作日的产品。而在另一方面,其余120夸特又被假定为60工作日的产品。但是,这里应该把工作日理解为相等的平均劳动。因此,这个假定是荒谬的。

由此可见,首先必须假定180夸特的产品只包含90工作日,构成不变资本价值的60夸特只包含30工作日。假定等于20夸特即10工作日的利润可以不存在,也是荒谬的。因为,这等于假定用来生产不变资本的30个工人虽然不是为资本家工作,但是又甘愿只领取相当于他们劳动时间的一半的工资,而不把其余一半算在自己的商品内。一句话,他们要比价值低50%出卖自己的工作日。所以,这个假定同样是荒谬的。

但是我们假定,资本家I不是向资本家II购买自己所需的不变资本然后进行加工,而是在他自己的企业中把不变资本的生产和加工结合起来。这样,他就是自己向自己提供种子、农具等等。此外,我们也把那种可以不用种子和固定资本的“发明”撇开不谈。总之,假定这个资本家在生产他的不变资本所需的不变资本上花费20夸特(等于10工作日),在[每]10工作日的工资上花费10夸特,其中5日是工人白白工作的。

于是,就得出如下的计算数字:

[333]

\todo{}

工资的实际生产费用仍旧是那些,也就是说,劳动生产率也没有发生变化。总产品仍旧是那样多,即180夸特,并且和以前一样具有180夸特的价值。剩余价值率也没有变化:80夸特比80夸特。剩余价值的绝对额即绝对量则从60夸特增加到80夸特,即增加20夸特。预付资本从120夸特减少到100夸特。以前是以120夸特赚得60夸特,即利润率为50%。现在则是以100夸特赚得80夸特,即利润率为80%。预付资本总价值120夸特减少了20夸特,利润率则从50%提高到80%。撇开利润率不谈,利润本身现在是80夸特,而以前是60夸特;所以,它增加了20夸特,这就是说,其增加程度和剩余价值量(不是剩余价值率)是相同的。

因此,在这里,实际工资的生产费用没有任何变动,没有任何改变。这里利润率的增长是由于:

(1)尽管剩余价值率没有提高,但是剩余价值绝对量从60夸特增加到80夸特,即增加了1/3;它所以增加1/3或33+(1/3)%,是因为资本家不是象以前那样雇用60个工人,而是现在直接在他自己的企业里雇用80个工人,即多剥削1/3或33+(1/3)%的活劳动,此外,他是在剩余价值率和以前只雇用60个工人时相同的条件下雇用这80个工人的。

(2)当剩余价值绝对量(因此,还有总利润)这样提高33+(1/3)%即从60夸特增加到80夸特时,利润率则从50%增加到80%,增加了30,即提高了3/5(因为:50的1/5是10,50的3/5是30)或60%。问题在于,尽管资本中花在工资上的那个组成部分从60夸特增加到80夸特(从30工作日增加到40工作日),但是所花费的资本的价值却从120夸特减少到100夸特。资本的前一部分增加了10工作日(=20夸特)。相反,资本的不变部分却从60夸特减少到20夸特,或者说从30工作日减少到10工作日,即减少了20工作日。这样,如果我们从这20工作日中扣除所增加的用于工资的那一部分资本10工作日,结果,所花费的全部资本就减少10工作日(=20夸特)。它以前是120夸特(=60工作日),而现在总共只有夸特(=50工作日)。可见,它减少了1/6,即16+(2/3)%。

不过,利润率的这一切变动,仅仅是一种表面现象,只不过是从一本账簿转到另一本账簿上。资本家I现在取得了80夸特的利润,而不是60夸特,即多取得了20夸特的利润;但是这恰恰是不变资本的生产者以前得到而现在失去的利润,因为资本家I自己生产不变资本,而不再购买不变资本,就是说,不再[334]付给不变资本的生产者20夸特(10工作日)的剩余价值(这笔剩余价值是不变资本的生产者从他雇用的20个工人身上榨取来的),而把它装进自己的腰包。

180夸特象以前一样,有80夸特的利润,只不过以前这笔利润是在两个人中间分配的。利润率看起来高了,这是因为资本家I以前把上述60夸特仅仅看作不变资本,而且对于他来说,这60夸特也确实是不变资本,这样,他就没有把注意力集中在不变资本生产者所取得的利润上。利润率也象剩余价值或包括劳动生产率在内的任何生产条件一样,没有变动。以前[也是这样],[不变资本]生产者所花费的资本是40夸特(20工作日),而资本家I支出的[可变资本]是60夸特(30工作日),共计100夸特(50工作日)。前者的利润是20夸特,后者的利润是60夸特,共计80夸特(40工作日)。相当于90工作日(180夸特)的全部产品,给花在工资和不变资本上的100夸特提供80夸特的利润。对社会来说,这时仍象以前一样,来自利润的收入没有变化;剩余价值对工资之比,也没有变化。

差别是这样产生的:当资本家作为买者出现在商品市场上的时候,他只是一个商品所有者;他必须支付商品的全部价值,支付商品所包含的全部劳动时间,而不管资本家和工人过去或现在按照什么比例参加分配这些劳动时间的果实。可是当他作为买者出现在劳动市场上的时候,他实际上买到的劳动就多于他所支付的。由此可见,当他不再购买他所需要的原料和机器,而自己也生产这些东西的时候,他把否则就得向原料和机器的卖者支付的剩余劳动据为己有了。

对单个资本家来说,——不是对利润率来说,——是他自己取得一笔利润,还是他把这笔利润付给别人,当然不是没有差别的。(因此,在计算利润率因不变资本增加而降低的情况时,永远要采用整个社会的平均数,即社会在某一时间作为不变资本使用的价值总量,以及这个总价值对直接花在工资上的资本量之比。)但是,这个着眼点,甚至对于单个资本家实行联合化——譬如,往往同一个资本家兼营纺纱和织布,而且还自己烧制所需要的砖,等等——时也很少起(并且很少能起)决定性作用。在这里起决定性作用的,是运输时间的节约所造成的生产费用的实际节约,厂房、燃料、动力等的节约,以及对原料质量加强监督,等等。如果资本家想要自己制造所需要的机器,他就得象为了满足自己的需要或满足若干固定用户的个人需要而工作的小生产者那样,小规模地生产机器,而且他为机器花费的代价,比他向为市场生产的机器制造业者购买机器要高。如果他同时纺纱、织布和制造机器,不仅是为了自己,而且是为了供应市场,那他就需要一笔更大的资本,可是,他如果把这笔资本投入他原来的事业,也许更加合算(分工)。只有当资本家为自己建立了相当大的市场,以致能够以有利可图的规模自己生产自己的不变资本的时候,上述着眼点才具有决定性作用。为此,他本身对不变资本的需求应该是相当大的。在这种情况下,即使他的劳动的生产率低于专门从事这种不变资本的生产的人,他也把一部分否则就得向另一个资本家支付的剩余劳动据为己有。

我们看到,这和利润率没有任何关系。因此,如果象穆勒所举的例子那样,以前使用了90工作日和80个工人,那末,尽管产品中包含的40工作日(=80夸特)的剩余劳动以前由两个资本家占有,而现在由一个资本家独吞,生产费用也决不会因而有丝毫节约,20夸特的利润(10工作日)在一本账簿上消失,仅仅是为了在另一本账簿上重新出现。

因此,离开劳动时间的节约,从而离开工资的节约,这种先前利润的节约就不过是一种幻觉。\endnote{马克思就在这个第VIII本(其中结束了关于约·斯·穆勒的一节)的第368页手稿上(见本卷第1册第221页),以及在第X本中间部分关于洛贝尔图斯的一章的第461—464页手稿上(见本卷第2册第43—52页),回过来谈了成品的生产与生产这个成品的不变资本的生产结合在一个资本家手里,会不会影响利润率的问题。——第238页。}

\tsectionnonum{[(c)关于不变资本价值的变动对剩余价值、利润和工资的影响的问题]}

[335]可是第四,现在还剩下这样一种情况:不变资本的价值因劳动生产率提高而降低;在这里要研究,这种情况是否关系到以及在多大程度上关系到工资的实际生产费用,或者说劳动的价值。因此,问题在于:不变资本实际价值的变动会在多大程度上同时引起利润和工资之间的比例的变动?不变资本的价值——它的生产费用——可能保持不变,然而产品中包含的这种不变资本的量却可能有时较多,有时较少。甚至假定不变资本的价值是不变的,不变资本的总量也会随着劳动生产率和大规模生产的发展而增长。因此,所使用的不变资本的相对量在它的生产费用不变甚至增长的条件下发生的变化,即对利润率始终发生影响的变化,从一开始就被排除在这里研究的范围之外。

其次,产品既不直接、又不间接加入工人消费的一切生产部门,也被排除在对这一问题的考察的范围之外。但是,这一类生产部门中发生的实际利润率的变动(即这一类生产部门中实际生产的剩余价值对所花费的资本之比的变动),却同产品直接或间接加入工人消费的生产部门中的利润率的变动完全一样,对因利润平均化而形成的一般利润率发生影响。

此外,问题必须归结为:不变资本价值的变动怎样才会对剩余价值本身发生反作用?因为既然假定剩余价值是既定的,那末剩余劳动对必要劳动之比也就是既定的,从而工资的价值,即工资的生产费用也就是既定的。在这种情况下,不变资本的价值的任何变动,都根本不会触动工资的价值以及剩余劳动对必要劳动之比,尽管这种变动在任何情况下都会影响利润率,影响资本家的剩余价值的生产费用,而且在一定情况下(即当产品加入工人消费时)还会影响体现工资的使用价值的数量,虽然并不影响它的交换价值。

假定工资是既定的。譬如说,棉纺织厂中的工资等于10劳动小时,剩余价值等于2劳动小时。再假定子棉由于丰收而落价一半。同量棉花以前要工厂主花费100镑,现在只花费50镑。同量棉花所吸收的纺纱和织布的劳动量同以前一样。这样,资本家现在在棉花上只花费50镑,就能吸收同以前花费100镑时一样多的剩余劳动;或者,如果他还是在棉花上花费100镑,那他现在用同样的价格所换取的棉花数量,就可以使他多吸收一倍的剩余劳动。在这两种情况下,剩余价值率、即剩余价值对工资之比,依然保持不变;但是剩余价值量在后一种情况下增长了,因为,尽管剩余劳动率相同,所使用的劳动却增加了一倍。在这两种情况下,利润率都提高了,尽管这里工资的生产费用没有任何变动。利润率所以提高,是因为在利润率中,剩余价值按资本家所支付的生产费用,按他所花费的资本的总价值计算,是因为这种生产费用减少了。现在资本家花费比以前少的资本,就可以生产出象以前一样多的剩余价值。在后一种情况下,不仅利润率提高,而且利润量也增加,因为剩余价值本身由于使用较多劳动而增加了,可是原料费用这时并没有增加。而且,在这种情况下,利润率和利润量也是在劳动价值没有任何变动时提高的。

另一方面我们假定,棉花的价值由于歉收提高了一倍,因此,同样多的[336]棉花,以前值100镑,而现在值200镑。在这种情况下,利润率无论如何都会降低,而利润量,或者说利润的绝对量,在一定的条件下,也可能减少。如果资本家使用同以前一样多的工人,这些工人的劳动同以前一样多,而且条件同以前完全一样,那末,尽管剩余劳动对必要劳动之比,从而剩余价值率和剩余价值量都保持不变,资本家的利润率还是下降。利润率所以下降,是因为对资本家来说,剩余价值的生产费用增加了,也就是说,他必须在原料上多花费100镑,才能占有同以前一样多的别人的劳动时间。但是,如果资本家现在不得不把以前用在工资上的钱的一部分花在棉花上,譬如说,用150镑购买棉花,而其中50镑以前是加入工资的,那末,无论利润率或利润量都会下降,而利润量所以下降,是因为使用的劳动少了,尽管这时剩余价值率没有变动。如果由于歉收而没有足够的棉花用来吸收同以前一样多的活劳动,情形也会如此。在这两种情况下,利润量和利润率都下降,尽管劳动的价值,也就是说,剩余价值率,或资本家取得的无酬劳动量与他以工资付酬的劳动之比,都没有任何变动。

总之,在剩余价值率不变,也就是说,劳动的价值不变的条件下,不变资本价值的变动必然引起利润率的变动,而且可能伴随着利润量的变化。

另一方面,现在就工人来说,情况如下:

棉花价值降低了,因此有棉花加入的产品的价值也降低了,而工人仍象以前一样取得等于10劳动小时的工资。但是,他自己消费的那部分棉纺织品,现在可以按便宜的价格买到了,因此,他以前在棉纺织品上的支出,有一部分可以用在其他东西上。他能够得到的生活资料的数量,只能与此相应地增加,即与他在棉纺织品价格上的节约相应地增加。因为就其他方面来说,他现在用较多的棉纺织品换得的,并不比他以前用较少的棉纺织品换得的多。与棉纺织品价值的降低相应,其他商品的相对价值提高了。简言之,现在较多的棉纺织品所具有的价值,并不比以前较少的棉纺织品所具有的价值多。因而,在这种情况下,工资的价值仍会和以前一样,不过它代表较多的其他商品(使用价值)。但是利润率还是会提高,尽管剩余价值率在前提不变的情况下不可能提高。

在棉花涨价时,情况则相反。如果工人仍然一天劳动那样多时间,所取得的工资仍然等于10劳动小时,那末,他的劳动的价值就仍然是那些,可是[他的工资的]使用价值,就他自己消费棉纺织品来说,却减少了。在这种情况下,工资的使用价值减少,工资的价值保持不变,尽管利润率下降了。由此可见,虽然剩余价值和(实际)工资12的降低和提高彼此永远成反比例(工人参加分配靠劳动时间绝对延长而获得的果实这种情况例外;但是在这种情况下,工人的劳动能力会损耗得更快),利润率却可能在工资价值不变而其使用价值增长的情况下提高,在工资价值不变而其使用价值减少的情况下降低。

因此,利润率由于不变资本的价值降低而提高,这与工资的实际价值(工资中包含的劳动时间)的任何变动毫无直接关系。

这样,如果棉花的价值象上面假定的那样降低50%,那末,[穆勒著作中的]以下说法就是再错误不过的了:工资的生产费用在这种情况下降低了;或者说,既然以棉纺织品形式领取工资的工人取得和以前一样多的价值,也就是说,取得比以前更多数量的棉纺织品(因为10劳动小时仍和以前一样等于譬如说10先令,但由于子棉的价值降低了,现在我用这10先令就能够买到比以前多的棉纺织品),那末利润率就仍和以前一样。剩余价值率仍和以前一样,但[337]利润率提高了。产品的生产费用会减少,因为产品的一个组成部分,即产品中包含的原料,比以前耗费较少的劳动时间。工资的生产费用仍和以前一样,因为工人为自己劳动的时间和为资本家劳动的时间仍象以前一样多。(要知道,工资的生产费用并不取决于工人用来进行劳动的生产资料所耗费的劳动时间,而取决于他为补偿自己的工资而花费的劳动时间。在穆勒先生看来,由于工人加工的譬如说是铜而不是铁,或者是亚麻而不是棉花,他的工资的生产费用就会贵些;或者说,如果工人种的是亚麻种子而不是棉花种子,或者如果他劳动时使用昂贵的机器,而不是根本不用机器或只用简单的手工业工具,那末,他的工资的生产费用就会贵些。)利润的生产费用会减少,这是因为,为生产剩余价值而预付的资本的总量即总额会减少。剩余价值的费用,永远不会大于花在工资上的那部分资本的费用。相反,利润的费用则等于为创造这个剩余价值而预付的资本的费用总额。因而,利润的费用就不仅决定于花在工资上并创造剩余价值的资本组成部分的价值,而且决定于为推动和活劳动交换的资本组成部分所必需的那些资本组成部分的价值。穆勒先生把利润的生产费用和剩余价值的生产费用混淆起来了,也就是说,他把利润和剩余价值混淆起来了。

综上所述,可以看出原料的贵贱对于加工这种原料的工业的重要性(至于机器的相对跌价\authornote{所谓机器的相对跌价,我是指所使用的机器总量的绝对价值增加了,但是增加的程度不如这些机器的总量和工作效率增长的程度。}就不必说了),——甚至假定市场价格等于商品的价值,即商品的市场价格下降的程度同商品中包含的原料的价值下降程度完全一致,也是这样。

因此,托伦斯上校关于英国的说法是正确的:

\begin{quote}{“对于英国这样的国家来说,某个国外市场的重要性不应当根据它从这个国家得到的成品的数量,而应当根据它还给这个国家的再生产要素的数量来衡量。”(罗·托伦斯《就英国状况和消除灾难原因的手段致尊敬的从男爵、议员罗伯特·皮尔爵士的信》1849年伦敦第2版第275页)}\end{quote}

{但是托伦斯论证这一点所使用的方法是拙劣的。那是关于供求的老生常谈。在托伦斯那里,事情被归结为这样:从事例如棉花加工的英国资本的增长速度,高于从事棉花种植的例如美国资本,棉花的价格就会上涨。托伦斯说,这时,

\begin{quote}{“棉纺织品的价值和用于它的各要素的生产费用相比将会降低”。[同上,第240页]}\end{quote}

也就是说,当原料的价格由于英国的需求增长而提高的时候,因原料价格提高而变贵的棉纺织品的价格将会降低;例如我们在目前(1862年春)实际上就看到,棉纱比子棉贵不了多少,棉布比棉纱贵不了多少。然而托伦斯假定:棉花尽管贵,但是足够供英国工业消费。棉花的价格涨到它的价值之上。因此,如果说棉纺织品按照它的价值出卖,那末,这所以可能只是由于:棉花种植者从全部产品中取得多于他应得的剩余价值,实际上占有了应该属于棉纺织厂主的剩余价值的一部分。后者不能通过提高价格来为自己补偿这一部分剩余价值,因为价格提高,需求就会减少。由于需求减少,棉纺织厂主的利润甚至反而会下降到低于因棉花种植业者加价而应下降的程度。

对原料例如棉花的需求,每年不仅决定于实际的、当时存在的需求,而且决定于当年的平均需求,从而,它不仅决定于开工工厂的需求,而且决定于来年——根据现有的经验——开办的工厂所增加的需求,就是说,决定于一年内工厂的相对增加,或者说与工厂相对增加相适应的[338]追加需求。

相反,如果棉花等等的价格下跌了(例如,由于收成特别好),那末它多半会降低到棉花的价值之下——这依然是由于供求规律的作用。因此,利润率,有时如上所述还有利润量,不仅仅按照它们在降低的棉花价格等于棉花价值的情况下所应提高的程度而提高,它们还会由于以下原因而提高:成品价格的降低并未完全达到棉花种植业者出卖棉花时其价格低于价值的程度,也就是说,棉纺织厂主把应当归棉花种植业者的一部分剩余价值装进了自己的腰包。这不会减少对他的产品的需求,因为产品价格反正由于棉花价值降低而降低了。不过,它降价的程度并没有达到子棉价格降到子棉本身价值以下的程度。

此外,需求在这时还会由于以下原因而增长:工人的就业率和报酬都很高,以致他们本身也在很大程度上成为消费者,成为他们自己的产品的消费者。在原料价格的下降不是由于它的平均生产费用继续不断降低,而是由于收成特别好(天气条件)的情况下,工人的工资不会下降;相反,对工人的需求还会增加。这种需求造成的结果,不只是与它的增长成比例地发生作用。相反,如果产品价格突然上涨,一方面许多工人被解雇,另一方面工厂主设法把工资压低到它的正常水平以下,借以避免遭受损失。这样,工人的正常需求下降,而且,这就进一步加快已经发生的需求普遍下降,以及加强需求普遍下降对市场价格的影响。}

穆勒关于产品在工人和资本家之间分配的(李嘉图式的)观念,主要使他产生这样一个想法:不变资本价值的变动可以引起劳动的价值即劳动的生产费用的变动,因此,例如预付的不变资本价值降低,就会引起劳动的价值,劳动的生产费用的降低,就是说也会引起工资的降低。由于原料譬如说棉花的价值降低,棉纱的价值就降低。它的生产费用减少了。它包含的劳动时间减少了。举例来说,如果一磅棉纱是一个工人一个(12小时的)工作日的产品,而这一磅棉纱所包含的棉花的价值降低了,那末,这一磅棉纱的价值的减少数就正好等于用在棉纱上的棉花价值的减少数。例如,一磅二级品40支细纱的价格在1861年5月22日是12便士(1先令)。它在1858年5月22日是11便士(实际上是11+(6/8)便士,因为棉纱价格的降低并没有达到子棉价格降低的幅度)。但是在第一种情况下,一磅标准质量的子棉的价格为8便士(实际上是8+(1/8)便士),在第二种情况下为7便士(实际上是7+(3/8)便士)。可见,这里棉纱价值的降低数正好等于子棉价值即棉纱原料价值的降低数。因此,穆勒说,劳动仍和以前一样;如果以前劳动是12小时,那末产品仍然和以前一样是这12小时的结果;但是在第二种情况下,所加的过去劳动比在第一种情况下少1便士;劳动没有变化,可是劳动的生产费用减少了(即减少1便士)。

尽管一磅棉纱作为棉纱,作为使用价值,仍和以前一样是12小时劳动的产品,然而1磅棉纱的价值无论现在或过去都不[仅仅]是纺纱工人12小时劳动的产品。在第一种情况下,12便士中有2/3即8便士是子棉的价值,而不是纺纱工人的产品;在第二种情况下,11便士中有2/3即7便士不是纺纱工人的产品。在第一种情况下,作为12小时劳动的产品的是4便士,在第二种情况下同样是4便士。在这两种情况下,[纺纱工人的]劳动仅仅加进了棉纱价值的1/3。所以说,在第一种情况下,一磅棉纱中只有1/3磅是纺纱工人的产品(如果撇开机器不谈的话),在第二种情况下也是如此。工人和资本家应该象以前一样仅仅分享等于1/3磅棉纱的4便士。如果工人用4便士购买棉纱,那末他在第二种情况下得到的棉纱比第一种情况下多,但是现在较多的棉纱的价值,和过去较少的棉纱的价值完全相同。可是,4便士在资本家和工人中间的分配,仍和以前一样。如果工人花费在自己工资的再生产或生产上的时间是10小时,那末他的剩余劳动就等于2小时,这和以前的情况是一样的。工人仍和以前一样从4便士即1/3磅棉纱中取得5/6,而资本家从其中取得1/6。由此可见,产品即棉纱的分配没有发生任何[339]变化。虽然如此,利润率还是提高了,因为原料的价值降低了,因而剩余价值对总预付资本即对资本家的生产费用之比提高了。

如果我们为了使例子简化而抛开机器等等不谈,那末这两种情况就可以表示如下:

可见,这里的利润率提高了,尽管劳动的价值没有变动,而以棉纱表示的工资的使用价值增加了。利润率提高了(在工人自己占有的劳动时间没有发生任何变化的情况下),仅仅是由于棉花的价值、从而还有资本家的生产费用的总价值降低了。2/3便士对11+(1/3)便士的支出之比,当然小于2/3便士对10+(1/3)便士的支出之比。

\centerbox{※     ※     ※}

根据以上所说,可以看到,穆勒在结束他的例证时提出的以下论点\authornote{见本册第214页。——编者注}是错误的:

\begin{quote}{“如果工资的生产费用仍然和以前一样,利润就不可能提高。每个工人以前取得1夸特谷物;但是以前1夸特谷物是现在1+(1/5)夸特的生产费用的结果。因此,为了使每个工人取得和以前一样多的生产费用,每人就应该有1夸特加1/5夸特的谷物。”(同上,第103页)“因此,假定付给工人的是工人自己生产的产品,那就很明显,当这种产品的生产费用有了某种节约而工人照旧得到以前的生产费用时,他得到的产品就必然会同资本的生产力的提高成比例地增加。但如果是这样,资本家的支出和他所得到的产品之间的比例,就会和以前完全一样,利润也不会提高。〈这一点恰恰错了。〉因此,利润率的变动和工资的生产费用的变动是同时发生的,是不可分割的。由此可见,如果李嘉图所说的低工资不仅指作为较小劳动量的产品的工资,而且指用较少的费用——包括劳动和先前的资本家的利润在内——生产的工资,那末他的意见就是完全正确的。”(同上,第104页)}\end{quote}

由此可见,按照穆勒的例证,李嘉图的观点只有在这样的情况下才是完全正确的:所说的低工资(或一般说来,工资的生产费用)指的不仅同李嘉图所说的相反,而且简直是一种极端荒谬的东西,也就是说,工资的生产费用不是指工人用来补偿自己的工资的那一部分工作日,而且还是指工人加工的原料和使用的机器的生产费用,即工人既没有为自己也没有为资本家劳动过的劳动时间。

\centerbox{※     ※     ※}

第五,现在来谈谈本来的一个问题:不变资本价值的变动会对剩余价值发生什么影响?

如果我们说,平均日工资的价值等于10小时,换句话说,在工人劳动的整个工作日譬如说12小时中,要用10小时来生产和补偿他的工资,他在这10小时之外劳动的时间才是无酬的劳动时间,才构成资本家[340]没有付酬而取得的价值,那末,这无非是说,在工人所消费的生活资料总量中包含10小时的劳动时间。这10劳动小时表现为工人用来购买他所必需的生活资料的一定的货币额。

但是商品的价值决定于它所包含的劳动时间,而不问这种劳动时间是包含在原料中,在磨损的机器中,还是在工人利用机器新加到原料上的劳动中。因此,如果加入该商品的原料或机器的价值发生了永久性的(而不只是暂时性的)变动,——造成这种变动的原因,是生产这些原料和机器的劳动生产率,简言之,即生产商品中包含的不变资本的劳动生产率发生了变化——而且如果由于这种变动,现在要用比以前多的或比以前少的劳动时间来生产商品的各该组成部分,那末,商品本身就会因而变贵或变贱(在把原料变为产品的劳动的生产率不变,以及工作日的长度不变的条件下)。结果,劳动能力的生产费用,即劳动能力的价值,会提高或降低;从而,如果工人以前在12小时中为他自己劳动10小时,那末现在他就必须为自己劳动11小时,或者在相反的情况下只劳动9小时。在前一种情况下,他为资本家完成的工作,即剩余价值,将减少一半,从2小时减少到1小时;在后一种情况下,剩余价值将增加一半,从2小时增加到3小时。在后一种情况下,资本家的利润率和利润量都会增长:前者增长是由于不变资本的价值减少了,两者都增长是由于剩余价值率(及其绝对量)提高了。

不变资本的价值的变动,仅仅以这种方式影响劳动的价值,影响工资的生产费用,或者说影响工作日在资本家和工人之间的划分,从而也影响剩余价值。

但是这仅仅说明,对于资本家,例如纺纱的资本家来说,他自己的工人的必要劳动时间不仅决定于纺纱业中的劳动生产率,而且决定于棉花、机器等等生产部门中的劳动生产率,同样地还决定于所有这样一些生产部门的生产率:这些部门的产品虽然不作为不变资本——既不作为原料,也不作为机器等等——加入这个资本家的产品(根据假定,这种产品加入工人消费)即棉纱之内,但是构成花在工资上的流动资本的一部分;也就是说,同样地还决定于生产食品等等的劳动生产率。在某一生产部门中作为产品的东西,在另一部门会成为劳动材料或劳动资料;因此,某一生产部门的不变资本是由另一生产部门的产品构成的,它在另一生产部门不是不变资本,而是那个部门的生产的结果。对于单个资本家来说,劳动生产率的提高(从而还有劳动能力价值的下降)是发生在他自己那个生产部门,还是发生在那些为他的企业提供不变资本的部门,不是没有区别的。而对资本家阶级——对资本整体——来说则是一样的。

由此可见,这种情况{即不变资本的价值下降(或相反的变动)不是由于使用这种不变资本的生产部门扩大了生产规模,而是由于不变资本本身的生产费用变动了},完全没有超出就剩余价值所阐明的那些规律的范围。\endnote{马克思指他的1861—1863年手稿第I—V本,在这几本里马克思阐述了他的剩余价值理论(见《货币转化为资本》、《绝对剩余价值》、《相对剩余价值》各节)。——第249页。}

一般说来,当我们谈到利润和利润率的时候,总是假定剩余价值是既定的,因而影响剩余价值的一切因素都已经起过作用。这都是假定了的前提。

\centerbox{※     ※     ※}

第六,这里还可以再研究一下,不变资本对可变资本之比,因而还有利润率,如何由于剩余价值的特殊形式,即由于劳动时间延长到正常工作日以上而发生变动。[341]由于这种情况,不变资本的相对价值,或者说,不变资本在产品总价值中所占价值的比例部分减少了。不过我们把这一点留在第三章\endnote{马克思指他的研究中后来发展成为《资本论》第三卷的那一部分。——第250页。}里讲,因为这里所研究的东西,一般说来,大部分是属于那一章的。

\centerbox{※     ※     ※}

穆勒先生根据他所作的出色例证,提出了一个(李嘉图式的)一般原理:

\begin{quote}{“利润规律的唯一的表现……是利润取决于工资的生产费用。”(同上,第104—105页)}\end{quote}

应该说情况正好相反:利润率{穆勒所谈的也正是利润率}只是在唯一的一个情况下才仅仅取决于工资的生产费用,这个情况就是:剩余价值率和利润率等同。不过这只有在以下那种在资本主义生产中几乎不可能的情况下才是可能的:全部预付资本直接预付在工资上,任何不变资本——不论是原料还是机器、建筑物等等——都不加入产品;或者,原料等等虽然加入产品,但是它们本身都不是劳动的产品,都没有价值。只有在这种情况下,利润率的变动才和剩余价值率的变动等同,或者换句话说,才和工资的生产费用的变动等同。

但是一般说来(刚才谈到的例外情况也包括在内),利润率等于剩余价值对预付资本总价值之比。

如果我们用M代表剩余价值,用C代表预付资本的价值,那末利润率就是M∶C,或M/C。这个比例既取决于M的量{而决定M的量时,是把决定工资的生产费用的一切情况包括进去的},也取决于C的量。但是C,预付资本的总价值,是由不变资本c与(花在工资上的}可变资本v构成的。因此,利润率等于M/(v+c),即M/C。但是M本身,剩余价值,不仅决定于它本身的比率,即剩余劳动对必要劳动之比,或者说工作日在资本和劳动之间的划分,工作日划分为有酬劳动时间和无酬劳动时间。剩余价值量,即剩余价值的绝对量,还决定于同时被资本剥削的工作日数。而这个按一定无酬劳动率使用的劳动时间量,对一定的资本来说,则取决于产品在不再投入劳动或不再投入以前那样多劳动的情况下停留在生产过程本身(例如,在酒窖中酿制的葡萄酒,已经种在地里的谷物,在一定的时间内经受化学作用的皮革和其他材料,等等)的时间,也取决于商品流通时间的长短,取决于商品形态变化时间的长短,即从它作为产品被制成时到它作为商品投入再生产时这段时间的长短。多少工作日可以同时使用{在工资的价值从而剩余价值率既定的情况下},总的说来,取决于花在工资上的资本量。而刚刚谈到的情况总的来说可以改变一定量资本在一定时期譬如说一年内所能使用的活劳动时间的总量。这些情况也决定着一定资本能够使用的劳动时间的绝对量。但是这并没有改变以下事实:剩余价值仅仅决定于剩余价值率乘以同时使用的工作日数。上述那些情况只决定这两个因素中的后一个因素——所使用的劳动时间量。

剩余价值率等于剩余劳动在一个工作日中所占的比例,也就是等于一个工作日所生产的剩余价值。举例来说,如果一个工作日等于12小时,剩余劳动等于2小时,那末这2小时就等于12小时的1/6,或者更正确地说,我们应当拿这2小时按必要劳动(或者说按支付必要劳动的工资——同量的物化形式上的劳动时间)来计算,这样,剩余劳动所占的份额是1/5(10小时的1/5是2小时;1/5=20%)。在这里(一个工作日的)剩余价值量完全决定于剩余价值率。如果现在资本家使用100这样的[342]工作日,那末剩余价值(它的绝对量)就等于200劳动小时。剩余价值率仍和以前一样:200小时比1000小时必要劳动,即1/5或20%。如果剩余价值率是既定的,那末剩余价值量完全取决于所使用的工人人数,就是说取决于花在工资上的资本的绝对量,取决于可变资本。如果所使用的工人人数,即花在工资上的资本或可变资本的量是既定的,那末剩余价值量就完全取决于剩余价值率,即取决于剩余劳动对必要劳动之比,取决于工资的生产费用,取决于工作日在资本家和工人之间的划分。如果100个工人(他们一天劳动12小时)给我提供200劳动小时,那末剩余价值的绝对量就等于200小时,剩余价值率则是一个[有酬的]工作日的1/5,即2劳动小时。剩余价值在这里等于2小时乘以100。如果50个工人给我提供200劳动小时,那末剩余价值的绝对量就等于200劳动小时,剩余价值率则等于一个(有酬的)工作日的2/5,即4劳动小时。剩余价值在这里等于4小时乘以50。既然剩余价值的绝对量等于剩余价值率和工作日数相乘之积,那末,在两个因数按反比例变化时,它也会保持不变。

剩余价值率永远表现为剩余价值对可变资本之比。因为可变资本等于有酬劳动时间的绝对量,而剩余价值等于无酬劳动时间的绝对量。所以,剩余价值对可变资本之比,永远表示工作日的无酬部分对有酬部分之比。假定上例中的10小时的工资等于1塔勒,这里1塔勒就是包含10劳动小时的白银量。在这种情况下,100工作日的报酬是100塔勒。如果这里剩余价值等于20塔勒,那末剩余价值率就是20/100,即1/5或20%。换句话说,资本家用10劳动小时(等于1塔勒)取得2劳动小时,用10×100即1000劳动小时则取得200劳动小时,等于20塔勒。

总之,尽管剩余价值率仅仅决定于剩余劳动时间对必要劳动时间之比,换言之,决定于一个工作日中工人为生产他的工资所需的相应部分,即决定于工资的生产费用,但是剩余价值量此外还决定于工作日的数量,决定于按一定的剩余价值率使用的劳动时间的绝对量,也就是说,决定于用在工资上的资本的绝对量(如果剩余价值率是既定的话)。但是,既然利润是剩余价值的绝对量(而不是剩余价值率)对全部预付资本总价值之比,那末利润率显然就不仅仅决定于剩余价值率,而且决定于剩余价值的绝对量,而这个绝对量则取决于剩余价值率和工作日数的复比例,取决于用在工资上的资本量和工资的生产费用。

如果剩余价值率是既定的,那末剩余价值量就仅仅取决于(用在工资上的)预付资本量。平均工资到处都是相同的。换句话说,假定一切生产部门中的工人都领取譬如说10小时的工资。(在工资高于平均工资的部门中,这对于我们的研究以及对于问题本身来说,就好象资本家使用了较多的普通工人一样。)这样,假定剩余劳动到处都是相同的,就是说整个正常工作日也是相同的(不相同的程度,由于一小时的复杂劳动等于譬如两小时的简单劳动,得以部分地拉平),[343]那末,剩余价值量就仅仅取决于[花在工资上的]预付资本量。因此,可以说,剩余价值量的相互之比等于(花在工资上的)预付资本量的相互之比。但是对利润就不能这样说,因为利润是剩余价值对全部预付资本总价值之比,而在同量资本中,花在工资上的资本组成部分,或者说可变资本对全部资本之比,可能是而且常常是极不相同的。利润量在这里则取决于——不同资本中的——可变资本对全部资本之比,即取决于v/(c+v)。因此,如果剩余价值率是既定的——它始终表现为m/v,表现为剩余价值对可变资本之比,——那末利润率就仅仅决定于可变资本对全部资本之比。

总之,利润率首先决定于剩余价值率,或者说,决定于无酬劳动对有酬劳动之比;它随着剩余价值率的变动而变动——提高或降低(只要这种作用没有被其他决定因素的运动抵销)。剩余价值率的提高或降低则同劳动生产率成正比,而同工资的生产费用,或者说同必要劳动量成反比,也就是同劳动的价值成反比。

其次,利润率决定于可变资本对全部资本之比,决定于v/(c+v)。问题在于,剩余价值的绝对量在剩余价值率既定的条件下仅仅取决于可变资本量,而可变资本量,在我们假定的前提下,决定于或者说只是表示同时使用的工作日数,即所使用的劳动时间的绝对量。利润率则取决于这个由可变资本提供的剩余价值绝对量对全部资本之比,也就是取决于可变资本对全部资本之比,取决于v/(c+v)。既然在计算利润率时剩余价值M假定是既定的,因而v也假定是既定的,那末v/(c+v)的一切变动就只能由c,即不变资本量的变动产生。这是因为:如果v是既定的,那末c+v的总额,即C,只有当c变动时才能变动,而随着c的变动,v/(c+v),即v/C的比例也会变动。

如果v=100,c=400,那末v+c=500,而v/(c+v)=100/500=1/5=20%。可见,如果剩余价值率等于5/10,即1/2,剩余价值就等于50。但是,既然可变资本仅仅等于全部资本的1/5,那末利润就等于全部资本的1/5的1/2,即1/10。实际上,500的1/10,等于50。利润率就是10%。v/(v+c)这一比例随着c的每一次变动而变动,当然不是按相同数值变动。我们假定v和c最初都等于10,也就是说,总资本的半数是可变资本,半数是不变资本,那末v/(v+c)=10/(10+10)=10/20=1/2。这样,剩余率[dieMehrrate]如果等于v的1/2,那就等于C的1/4。或者说,如果剩余价值为50%,那末在可变资本为C/2的这一情况下,利润率就等于25%。现在假定不变资本增加一倍,从10增加到20,那末v/(c+v)=10/(20+10)=10/30=1/3。(以前剩余率是10的1/2,现在等于C的1/3的1/2,也就是说等于30的1/6,即等于5。而10的半数也等于5。5比10,得50%。5比30,得16+(2/3)%。而在以前,5比20,得1/4,即25%。)不变资本增加一倍,即从10增加到20;但是c+v的总额只增加一半,即从20增加到30。不变资本增加100%,c+v的总额只增加50%。v/(c+v)这一比例最初等于10/20,现在不过减少到10/30,从1/2减少到1/3,即从3/6减少到2/6,也就是只减少了1/6,而不变资本增加一倍。不变资本的增加或减少怎样影响v/(c+v)这一比例,显然取决于c和v在最初构成全部资本C(c+v)的两个部分时的比例。

[344]首先,不变资本(也就是它的价值)在所使用的原料、机器等等的数量保持不变的条件下可以增加(或减少)。因此,在这种情况下,不变资本的变动不是决定于它作为不变资本进入的那个生产过程的生产条件,而是与它无关。但是,无论不变资本价值的这种变动的原因是什么,它们总会影响利润率。在这种情况下,同量原料、机器等等的价值所以比以前多或少,是因为生产它们所需要的劳动时间比以前多或少。在这里,价值的变动是由不变资本的各组成部分作为产品从中出来的那些过程的生产条件决定的。我们在前面\authornote{见本册第238—246页。——编者注}已经考察了这种情况是如何影响利润率的。

但是,当同一个生产部门的不变资本(例如原料)的价值由于这种不变资本本身的生产变贵或变便宜而提高或降低时,这种情况对利润率的影响,同在下述情况下产生的影响完全一样:在某一生产部门(甚至是同一生产部门),在工资支出相同的条件下,一种商品比另一种商品采用了较贵的原料。

如果工资支出相同,而某一资本加工的原料(例如小麦)比另一资本加工的原料(例如燕麦)贵(或者是银和铜、羊毛和棉花,等等),这两笔资本的利润率都应该同原料价格的提高成反比。因此,如果这两个生产部门平均说来取得相同的利润,那末所以能够如此,仅仅是因为在资本家阶级内部,剩余价值在各个资本家之间的分配,不是根据每笔资本在它的特殊的生产领域内生产的剩余价值,而是根据所使用的资本的大小。这可能有以下两种情况。加工比较便宜的材料的A,按照商品的实际价值出卖他的商品,从而以货币形式取得他自己所生产的剩余价值。他的商品的价格等于商品的价值。加工比较贵的材料的B,高于商品的价值出卖他的商品,并规定这样的价格,[这个价格能使商品提供给他同样多的利润,]就象他加工比较便宜的材料一样。如果后来A同B交换他们的产品,那末,对A来说,这无异于他算在自己商品的价格里的剩余价值少于商品中实际包含的剩余价值。或者A和B两者事先按照所花费的资本量来规定利润率,也就是说,他们按照他们支出的资本量来分配总剩余价值,结果也一样。而这正是一般利润率的含义\endnote{马克思在这里第一次表述了他关于剩余价值转化为平均利润以及商品的价值变成与它不同的生产价格这一学说的基本思想。这一部分手稿写于1862年春(见正文第243页)。还可参看本卷第一册第76页,那里第一次出现“平均价格”这一术语,以表示与价值不同的生产价格。马克思在1862年6月(在关于洛贝尔图斯的一章中——见本卷第2册第19—22、25、63—70页)和1862年7—8月(在对李嘉图的经济观点体系的批判分析中——见本卷第2册第191—240页),更详尽地阐述了平均利润和生产价格的学说。——第256页。}。

十分明显,当某一资本的不变部分(例如原料)的价值由于收成好坏以及诸如此类的影响而有短时期的降低或提高的时候,是不会出现这种平均化的,虽然譬如说纺纱厂主在棉花收成特别好的年份取得的特别多的利润,毫无疑问会把大量新资本吸引进这个工业部门,促使建造大批新的工厂和棉纺织机。因此,如果继之而来的是棉花歉收的年景,[棉花价格突然提高所带来的]损失就会更大。

其次,在机器和原料——简言之,不变资本——的生产费用没有变动的条件下,可能需要更多的这种不变资本,从而它的价值会相应增加,其原因在于:不变资本的上述各组成部分作为生产资料进入的那些过程的生产条件改变了。在这种情况下,和在上述情况下一样,不变资本的价值的增加当然会引起利润率的下降;但是,从另一方面来看,生产条件的这种变化本身却证明,这里劳动具有更高的生产率,从而剩余价值率提高了。要知道,原来那样多的活劳动,现在消费比以前多的原料,只是因为它加工这些原料所需的时间少了;现在使用更多的机器,也只是因为机器的价值比它所代替的劳动的价值低。可见,利润率的降低在这里由于剩余价值率的提高,从而也由于剩余价值绝对量的提高,而多少得到补偿。

最后一点,引起不变资本价值变动的两种情况可能通过极不相同的结合共同发生作用。举例来说,[345]子棉的平均价值降低了,但是与此同时,在一定时间内加工的棉花总量的价值却更增加了。再举一个例子:一磅羊毛的价值和在一定时间内加工的羊毛总量的价值都增加了。第三个例子:更大量地使用机器,绝对地说变贵了,但是与其效率相对来说便宜了,等等。

到目前为止一直是假定可变资本保持不变。但是,可变资本本身也有可能不仅相对地——与不变资本量相比——减少,而且还绝对地减少,例如,在农业中就是这样。此外,可变资本也可能绝对地增加。但是,那时的结果仍旧同它保持不变时一样,只要不变资本由于上述原因以更大程度或以同样程度增加。

如果不变资本保持不变,那末它同可变资本相比每次增加或减少的原因,仅仅在于不变资本的这种相对增加或减少是可变资本绝对减少或增加的结果。

如果可变资本保持不变,那末不变资本每次增加或减少的原因,只在于它本身的绝对增加或减少。

如果两种资本同时发生变动,那末在除去这两者相同的变动之后,所得的结果仍旧和一种资本保持不变而另一种资本有所增加或减少一样。

但是,如果利润率是既定的,那末利润量就取决于所使用的资本量。利润率低的大量资本提供的利润,多于利润率高的小量资本。

\centerbox{※     ※     ※}

这个插入部分到此可以结束了。

此外,在约·斯·穆勒的著作中还应当注意的只有以下两个论点:

\begin{quote}{“严格说来,资本并不具有生产力。唯一的生产力是劳动力,当然,它要依靠工具并作用于原料。”(同上,第90页)}\end{quote}

严格说来,穆勒在这里把资本与构成资本的物质组成部分混为一谈了。可是,这个论点对于那些同样把两者混为一谈,但又认为资本有生产力的人来说,却是好的。当然,这里说穆勒的论断正确,也仅仅就所指的是价值的生产而论。要知道,如果指的只是使用价值,那自然界也是会生产的。

\begin{quote}{“资本的生产力不外是指资本家借助于他的资本所能支配的实际生产力的数量。”(同上,第91页)}\end{quote}

在这里,资本被正确地看作生产关系。[VIII—345]

\centerbox{※     ※     ※}

[XIV—85]在以前的一本稿本中,我曾经详细地分析了穆勒在《略论政治经济学的某些有待解决的问题》(1844年伦敦版)一书中,如何对剩余价值和利润不加区分,粗暴地试图直接从价值理论中得出李嘉图关于利润率(关于利润与工资成反比例)的规律。

\tchapternonum{[(8)结束语]}

以上关于李嘉图学派的全部叙述表明,这个学派的解体是在这样两点上:

(1)资本和劳动之间按照价值规律交换。

(2)一般利润率的形成。把剩余价值和利润等同起来。不理解价值和费用价格的关系。

\tchapternonum{[第二十一章]以李嘉图理论为依据反对政治经济学家的无产阶级反对派}

[852]在政治经济学上的李嘉图时期,同时也出现了[资产阶级政治经济学的]反对派——共产主义(欧文)和社会主义(傅立叶、圣西门)(社会主义还只是处在它的发展的最初阶段)。但是,依照我们的计划,这里要考察的只是本身从政治经济学家的前提出发的反对派。

从我们在下面引用的著作中可以看出,所有这些人实际上都是从李嘉图的形式出发的。

\tchapternonum{(1)小册子《国民困难的原因及其解决办法》}

\tsectionnonum{[(a)把利润、地租和利息看成工人的剩余劳动。资本的积累和所谓“劳动基金”之间的相互关系]}

《根据政治经济学基本原理得出的国民困难的原因及其解决办法。致约翰·罗素勋爵的一封信》1821年伦敦版(匿名)。

这本几乎没有人知道的小册子(约40页),是在“这个不可相信的修鞋匠”\endnote{“这个不可相信的修鞋匠”(《thisincrediblecobbler》)——《对麦克库洛赫先生的,〈政治经济学原理〉的若干说明》这一小册子的作者对麦克库洛赫的称呼。见前面正文第203页。——第260、294页。}麦克库洛赫开始被人注意的时候出现的,它包含一个超过李嘉图的本质上的进步。它直接把剩余价值,或李嘉图所说的“利润”(李嘉图常常也把它叫作“剩余产品”),或这本小册子作者所说的“利息”,看作“剩余劳动”,即工人无偿地从事的劳动,也就是工人除了补偿他的劳动能力价值的劳动量,即生产他的工资的等价物的劳动量以外而从事的劳动。把体现在剩余产品中的剩余价值归结为剩余劳动,同把价值归结为劳动是一样重要的。这一点其实亚·斯密已经说过\authornote{见本卷第1册第57—64页和第2册第461页。——编者注},并且成为李嘉图的阐述中的一个主要因素。但是,李嘉图从来没有以绝对的形式把它说出来并确定下来。

李嘉图和其他政治经济学家的兴趣仅仅在于理解资本主义生产关系,并把它说成是生产的绝对形式,而我们所考察的这本小册子以及要在这里考察的其他这一类著作,则是要掌握李嘉图和其他政治经济学家所揭露的资本主义生产的秘密,以便从工业无产阶级的立场出发来反对资本主义生产。

[小册子的作者说:]

\begin{quote}{“无论资本家得到的份额有多大〈从资本的立场出发〉,他总是只能占有工人的剩余劳动,因为工人必须生活。”(上述著作,第23页)}\end{quote}

这些必要的生活条件,工人能够维持生活所需要的这种最低限度,从而能够从工人身上榨取的剩余劳动量,的确都是相对的量。

\begin{quote}{“如果资本的价值\endnote{从马克思下面的说明中可以看出,小册子《根据政治经济学基本原理得出的国民困难的原因及其解决办法》的作者把“资本的价值”(“thevalueofcapital”)理解为“资本利息”率,即资本的所有者占有的剩余劳动量和他所使用的资本量之比(小册子的作者把“资本利息”理解为马克思叫作剩余价值的东西,但是小册子的作者在这里把剩余价值率同利润率混淆起来了:他把从工人身上榨取的剩余劳动直接和整个预付资本相比)。——第261页。}不按照资本量增加的比例而减少,资本家就会超过工人能够维持生活所需要的最低限度从工人那里榨取每一个劳动小时的产品。不管这种情况看起来多么可怕和多么令人讨厌,资本家最后还是可以把希望寄托在只须花费极少量劳动就能生产出来的那些食物上,并且最后可以对工人说:你不应当吃面包,因为大麦面更便宜;你不应当吃肉,因为吃甜菜和马铃薯也可以过活。我们已经到了这个地步。”(同上,第23—24页)“如果工人能够做到用马铃薯代替面包生活,那就毫无疑问,从他的劳动中可以榨取更多的东西。这就是说,如果靠面包生活,他要维持自己和他的家庭,他必须为自己保留星期一和星期二的劳动,如果靠马铃薯生活,他就只需要为自己保留星期一的一半。星期一的另一半和星期二的全部就可以游离出来,以使国家或资本家得利。”(同上,第26页)}\end{quote}

这里利润等等直接被归结为对工人没有得到任何等价物的那部分劳动时间的占有。

\begin{quote}{“谁都承认,支付给资本家的利息,无论是采取地租、借贷利息的性质,还是采取企业利润的性质,都是用别人的劳动来支付的。”(同上,第23页)}\end{quote}

由此可见,地租、借贷利息和企业利润都只是“资本利息”的不同形式,这种资本利息又归结为“工人的剩余劳动”。这种剩余劳动体现在剩余产品中。资本家是剩余劳动或剩余产品的所有者。剩余产品就是资本。

\begin{quote}{“假定……没有剩余劳动,因而也就没有什么东西可以作为资本积累起来。”(同上,第4页)}\end{quote}

他马上接着说:

\begin{quote}{“剩余产品的所有者,或者说,资本的所有者……”(同上)}\end{quote}

作者用与伤感的李嘉图学派截然不同的口气说:

\begin{quote}{“资本增加的自然和必然的结果是资本价值的减少。”(第22页)}\end{quote}

关于李嘉图,他说:

\begin{quote}{“既然已经发现,如果人口不是随着资本的增加而增加,工资就会由于资本和劳动之间的不平衡而提高,如果人口增加,工资就会由于得到食物的困难而提高,那末,为什么要竭力向我们证明,说因为只有工资的提高才能使利润降低,所以资本的任何积累都不会使利润降低呢?”(第23页)}\end{quote}

[853]如果“资本价值”,即“资本利息”,也就是资本所支配、占有的剩余劳动,不随资本量的增加而减少,那末复利就会按几何级数增长;这个级数用货币计算(见普莱斯),就要以不可能有的积累(不可能有的积累率)为前提,同样,如果把这个级数归结为它的真正要素,即归结为劳动,它就不仅会把剩余劳动,而且会把必要劳动作为资本“得到的份额”一齐吸收。(关于普莱斯的幻想,还要回过头来在收入及其源泉一节\endnote{《收入及其源泉》一节,马克思在1863年1月就已计划放在《资本论》第三部分(见本卷第1册第447页)。但是在1862年10月写的手稿第XIV本封面上,这一节附在《剩余价值理论》最后一章的《补充部分》(见本卷第1册第5页)。而实际上,在1862年10月和11月写成的手稿第XV本中,有一大节是探讨与批判庸俗政治经济学有关的收入及其源泉问题。但是那里根本没有谈到“普莱斯的幻想”。马克思在《资本论》第三卷第二十四章对这一幻想作了批判分析。——第263页。}中谈到。)

\begin{quote}{“如果能使资本不断增加,并使资本价值保持不变(其标志是借贷利息率不变),那末,为使用资本而支付的利息很快就会超过全部劳动产品……资本有快于算术级数增加资本的趋势。谁都承认,支付给资本家的利息,无论是采取地租、借贷利息的性质,还是采取企业利润的性质,都是用别人的劳动来支付的。因此,如果资本继续积累,在利息率保持不变的情况下,为使用资本而支付的劳动必然越来越增多,直到社会上全体工人的全部劳动都被资本家吸收为止。但这是不可能发生的;因为无论资本家得到的份额有多大,他总是只能占有工人的剩余劳动,因为工人必须生活。”(第23页)}\end{quote}

但是“资本的价值”怎样减少,小册子的作者是不清楚的。他自己说,照李嘉图的看法,这种情况之所以发生,或者是由于资本积累比人口增加快时工资提高了,或者是由于人口增加比资本积累快时(或甚至两者[同样地]同时增加时),工资的价值(但不是用生活资料表示的工资的量)因农业生产率降低而增加了。但是我们的匿名作者怎样来说明这一点呢?后一种说法他没有接受;照他的看法,工资会越来越降低,直至降到可能的最低限度。他说,[资本“利息”降低]之所以可能,只是由于虽然工人被剥削得更厉害,或者仍旧那样厉害,用来交换活劳动的那部分资本却相对减少。

不管怎样,匿名作者把利息按几何级数增长这句毫无意义的话还原为它的真正意义,即还原为毫无意义,这是他的功绩。\authornote{[XV—862a}由于剩余价值和剩余劳动的同一性,资本积累就有了质的界限,这种界限是由整个工作日的长度(24小时内劳动能力能起作用的时间)、当时生产力的发展程度以及能够限制同时遭受剥削的工人人数的人口数目决定的。相反,如果在不可理解的利息形式上来考察剩余价值,也就是把剩余价值看作是资本通过一种神秘的魔术而使自身增长的比例,那末,资本积累的界限就仅仅是量的,就绝对不能理解,为什么资本不天天早上把利息作为新的资本一次又一次地并入自身,从而创造出复利的无穷级数。[XV—862a]]

此外,匿名作者认为,有两种办法可以阻止资本在剩余产品或剩余劳动增加时把它掠夺来的赃物的越来越大的部分交还给工人。

第一种办法是把剩余产品转化为固定资本,这就可以阻止“劳动基金”,或者说,工人消费的那一部分产品必定随着资本的积累而增长。

第二种办法是对外贸易,它使资本家能够拿剩余产品去交换外国的奢侈品,从而自己把它消费掉。因此,即使是由必需品构成的那部分产品,也完全可以增加,而不必以工资的形式按其增加的某种比例流回给工人。

必须指出,第一种办法只是定期发生作用,而随后又失去作用(至少在固定资本由加入必需品生产的机器等等构成的情况下是这样),它以剩余产品转化为资本为条件,而第二种办法则以资本家消费剩余产品的部分越来越大,资本家的消费不断增加为条件,而不以剩余产品再转化为资本为条件。如果这种剩余产品以它直接存在的形式保留下来,那末其中就会有很大一部分必须作为可变资本同工人相交换,其结果就会提高工资和降低绝对或相对剩余价值。马尔萨斯宣扬“富人”必须增加消费,以便使那部分用来同劳动交换、转化为资本的产品具有很高的价值,带来很多的利润,吸收大量的剩余劳动,其真正的秘密也就在此。不过马尔萨斯不是让工业资本家本身增加消费,而是把这一职能给了土地所有者、领干薪者等等,因为积累的欲望和消费的欲望结合在一个人身上便会互相干扰。这里也暴露出巴顿、李嘉图等人观点中的错误。工资不是由产品总量中可能作为可变资本被消费,或者说,可能转化为可变资本的那一部分决定,而是由产品总量中实际转化为可变资本的部分决定。这些产品中有一部分甚至可能以实物形式被各种食客吃掉,另外一部分则可能通过对外贸易等等作为奢侈品消费掉。

我们这位小册子的作者忽略了以下两件事:

由于采用机器,大批工人经常失业,这就造成过剩人口;于是剩余产品找到了可以同它交换的现成的新劳动,而人口不必增加,绝对劳动时间无须延长。假定以前雇用500个工人,现在雇用300个工人,这300个工人提供相对来说更多的剩余劳动。只要剩余产品有足够的增加,其余的200个工人便可以用剩余产品来雇用。原有[可变]资本的一部分转化为固定资本,另一部分用来雇用较少量的工人,但是同他们的人数相比,却从他们身上榨取了更多的剩余价值,特别是榨取了奥多的剩余产品。其余的200个工人就是为了使新的剩余产品资本化而创造出来的材料。

[853a]正如这个小册子所说的,必需品通过对外贸易变成奢侈品,本身是很重要的:

(1)因为这种情况结束了这样一种谬论:似乎工资取决于所生产的必需品的量,似乎这些必需品必然以这种形式由它们的生产者或者甚至由从事生产的全体民众所消费,也就是说,必然再转化为可变资本,或者说,象巴顿和李嘉图所说的那样,再转化为“流动资本”;

(2)因为这种情况决定了某些同建立在资本主义生产基础上的世界市场有联系的落后国家——例如北美合众国奴隶占有制各州(见凯尔恩斯\endnote{马克思指的是当时刚出版的凯尔恩斯的著作《奴隶劳力:它的性质、经过及其可能的前途》(1862年伦敦版)。这本书他不止一次地在《资本论》第一卷和第三卷中引证过。——第266页。})或波兰等等——的整个社会形式(老毕希已经理解到了这一点,如果他不是从斯图亚特那里剽窃来的话)。无论这些国家从它们的奴隶的剩余劳动中榨取的简单形式即子棉或谷物形式的剩余产品量有多大,它们仍然能够保持这种简单的、没有等级差别的劳动,因为对外贸易使它们能够把这种简单的产品变成任何形式的使用价值。

说年产品中必须以工资形式花费的部分取决于“流动资本”量,这就等于说,当产品中有很大一部分由“建筑物”构成时,当与工人人口相比建筑了大量的工人住宅时,由于住宅的供给比对住宅的需求增加得快,工人一定会得到良好的和便宜的住宅。

相反,以下的说法是正确的:如果剩余产品很多,资本家又打算把其中很大一部分用作资本,那末(假定这么多剩余产品本身不是通过把大批工人抛向街头的办法取得的),对劳动的需求一定会增长,因而剩余产品中作为工资来交换的部分也必然会增长。无论如何,不是剩余产品(不管它以什么形式存在,甚至以必需品的形式存在)的绝对量迫使人们把剩余产品用作可变资本,因而使工资增加。而是除非机器经常造成过剩人口,除非资本的越来越大的部分(特别是也通过对外贸易)和资本交换而不是和劳动交换,资本化的欲望就会迫使人们把剩余产品的很大一部分用作可变资本,因而随着资本的积累,引起工资的增长。剩余产品中以只能用作资本的形式直接生产出来的部分,以及其中由于同外国交换而取得这种形式的部分,比其中必须和直接劳动相交换的部分增长得快。

工资取决于现有的资本,因而资本的迅速积累是引起工资提高的唯一手段,这一句话可归结如下:

一方面,如果把劳动条件表现为资本的形式撇开不谈,那就是这样一个同义反复:在工人生活条件不降低的情况下工人人数能够增加多快,取决于一定数量的工人所实现的劳动生产率。他们生产的原料、劳动工具和生活资料越多,他们就会有越多的钱,不仅用来抚养自己还不能工作的子女,而且用来实现新的正在成长的一代人的劳动,从而使人口的增长和生产的增长相一致,甚至使生产的发展超过人口的增长,因为随着人口的增长,工人的技能会提高,分工会增多,采用机器的可能性会增大,不变资本会增加,一句话,劳动生产率会提高。

如果人口的增长取决于劳动生产率,那末劳动生产率便取决于人口的增长。这里是互相发生作用。但是用资本主义的术语来表达,这就是说,工人人口的生活资料取决于资本的生产率,取决于工人的产品中尽可能大的一部分作为工人劳动的支配者同他们相对立。李嘉图本人正确地表达了这一点(我指的是同义反复),他认为工资取决于资本的生产率,而资本的生产率取决于劳动生产率。\authornote{见本卷第2册第618页和本册第121—123页。——编者注}

劳动取决于资本的增长这一点,一方面只不过是意味着如下的同义反复,[854]即工人人口的生活资料和就业手段的增长取决于他们自身的劳动生产率,第二,用资本主义的术语来表达,劳动取决于这样一种情况,即他们自己的产品作为别人的财产和他们相对立,因此,他们自己的生产率作为他们所创造的物的生产率和他们相对立。

这一点实际上意味着,工人在自己产品中占有的部分必须尽可能小,以便使他们的产品中作为资本和他们相对立的部分尽可能大;工人无偿地让给资本家的东西必须尽可能多,以便使资本家用来再购买工人劳动的资金(无偿地从工人那里榨取来的东西)尽可能多地增加。在这种场合,可能出现这样的情况:如果资本家让工人无代价地劳动得太多了,现在,为了换取这些没有付给等价物而得到的东西,他就可能让工人无代价地劳动得略微少一些。但是,因为这样做的结果正好妨碍资本家所追求的目的,即尽可能快地积累资本,所以工人必须在这样的条件下生活,以至于工人无酬劳动的减少又会因工人人口的增加(无论是由于采用机器而造成的相对增加,还是由于早婚而造成的绝对增加)而停止。(这也就是马尔萨斯主义者作为土地所有者和资本家之间的关系来宣扬,而为李嘉图学派所嘲笑的那样一种关系。)工人必须把自己产品中尽可能大的一部分无代价地交给资本,以便在较为有利的条件下用自己的新劳动买回这样让出的一部分产品。但是,由于这种有利的转变会同时消灭有利转变的条件,所以它只能是暂时的,它一定会再转化为它自己的对立面。

(3)适用于必需品通过对外贸易转化为奢侈品的,一般也适用于奢侈品的生产,但是,要使奢侈品花样繁多和增加,对外贸易的确是一个相当重要的条件。虽然从事奢侈品生产的工人为他们的雇主生产资本,但是他们的产品不能以实物形式再转化为资本,既不能再转化为不变资本,也不能再转化为可变资本。

如果把运往国外交换必需品(这些必需品全部或部分加入可变资本)的那部分奢侈品除外,那末,奢侈品所代表的只是一种剩余劳动,并且是直接以富人作为收入来消费的剩余产品形式出现的剩余劳动。诚然,奢侈品不只是代表生产它们的那些工人的剩余劳动。相反,这些工人完成的剩余劳动平均来说同其他生产部门的工人所完成的一样多。但是,正象可以把包含1/3剩余劳动的1/3产品看作这些剩余劳动的体现,而把产品的其余2/3看作预付资本的再生产一样,构成奢侈品生产者的工资的必需品生产者的剩余劳动,也可以体现为整个工人阶级的必要劳动。整个工人阶级的剩余劳动体现在:(1)资本家及其仆从所消费的那一部分必需品上;(2)全部奢侈品上。对单个资本家或单个生产部门来说,这就表现得不同了。对于单个资本家来说,他生产的奢侈品的一部分只是预付资本的等价物。

如果剩余劳动中直接表现为奢侈品形式的部分过大,那末,很明显,它一定会妨碍积累和扩大再生产,因为剩余产品中再转化为资本的部分太小。如果剩余劳动中表现为奢侈品形式的部分过小,那末,资本(即剩余产品中能够以实物形式再用作资本的部分)的积累将快于人口的增加,利润率将会下降,除非有必需品的国外市场存在。

\tsectionnonum{[(b)简单再生产条件下和资本积累条件下资本和收入的交换问题]}

我在解释资本和收入的交换时\authornote{见本卷第1册第233—258页。——编者注}把工资也看作收入,并且一般说来只考察了不变资本和收入的关系。工人的收入同时表现为可变资本,这一情况只有在如下的条件下才是重要的,那就是,在积累过程中(在新资本形成过程中),生产生活资料的资本家的由生活资料(必需品)构成的余额,能够同生产不变资本的资本家的由原料或工具构成的余额直接交换。在这里,一种形式的收入同另一种形式的收入交换,[855]这种交换一经完成,资本家A的收入就转化为资本家B的不变资本,而资本家B的收入就转化为资本家A的可变资本。

在考察资本的这种流通、再生产和相互补偿方式等等的时候,首先必须把对外贸易撇开不谈。

其次,必须区别以下两种现象:

(1)既定规模的再生产,

(2)扩大规模的再生产,或者说,积累——收入转化为资本。

关于(1)。

我曾经指出:

生活资料生产者必须补偿(1)他们的不变资本,(2)他们的可变资本。他们的产品中代表超过这两部分的余额的那一部分价值,构成剩余产品,构成剩余价值的物质存在,这种剩余价值又不过是剩余劳动的代表。

可变资本——生活资料生产者的产品中代表可变资本的部分——构成工资,构成工人的收入。这一部分在这里已经以实物形式存在,它以这种实物形式重新用作可变资本。这一部分,即工人再生产出来的等价物,被用来重新购买工人的劳动。这是资本和直接劳动之间的交换。工人以货币形式得到这一部分,他用这些货币买回他自己的产品或同一部类的其他产品。这是在工人以货币形式得到他应得的那一份产品的凭证以后可变资本各个不同组成部分相互之间的交换。这是同一部类(生活资料)内部新加劳动的一部分同另一部分的交换。

剩余产品(新加劳动)中由(生产生活资料的)资本家自己消费的部分,或者是被他们以实物形式消费,或者是在他们之间用一种可消费形式的剩余产品同另一种相交换。这是收入同收入的交换,同时这两种收入都归结为新加劳动。

上述交易其实不能说是收入同资本的交换。资本(必需品)是同劳动(劳动能力)交换。因此,这里不是收入和资本相交换。当然,工人一得到工资,就会把它消费掉。但是他用来同资本交换的不是他的收入,而是他的劳动。

[生活资料生产者的产品中的]第三部分[代表他们的]不变资本,它同生产不变资本的生产者的产品的一部分相交换,也就是同他们的产品中代表新加劳动的那一部分相交换。不变资本生产者的产品的这一部分,是由工资的等价物(也就是由[这些生产者的]可变资本)和剩余产品,剩余价值,即以只能用于生产消费而不能用于个人消费的形式存在的资本家的收入组成。所以,一方面,这是这些生产者的可变资本同生活资料中代表[生活资料生产者的]不变资本的部分相交换。实际上是不变资本生产者的产品中代表他们的可变资本而以不变资本形式存在的那一部分,同生活资料生产者的产品中代表不变资本而以可变资本形式存在的部分相交换。这里是新加劳动同不变资本的交换。

另一方面,不变资本生产者的产品中代表剩余产品而以不变资本形式存在的那一部分,同生活资料中代表生活资料生产者的不变资本的部分相交换。这里是收入同资本的交换。生产不变资本的资本家的收入,同生活资料相交换,并补偿生产生活资料的资本家的不变资本。

最后,生产不变资本的资本家的产品中本身代表不变资本的部分,部分地以实物形式得到补偿,部分地通过不变资本生产者之间的(被货币掩盖了的)实物交换得到补偿。

这一切都是在假定再生产规模和原有生产规模相同的情况下发生的。

如果我们现在要问,全部年产品中哪一部分代表新加劳动,那末,计算是非常简单的:

(A)[个人]消费品。分为三部分。[第一,]资本家的收入,等于一年内加进的剩余劳动。

第二,工资,即可变资本,等于工人用以再生产自己的工资的新加劳动。

最后,第三部分是原料、机器等等。这是不变资本,即产品价值中只被保存而不被生产的部分。因此,这不是一年内的新加劳动。

[856]如果我们用c′表示[这一部类的]不变资本,用v′表示可变资本,用r′表示剩余产品,表示收入,那末,这一部类就是由c′和v′十r′组成。

c′只是保存的价值,而不是新加劳动(这个c′代表产品的一部分);相反,v′十r′是一年内加进的劳动。

[A部类的]总产品(或它的价值)Pa扣除c′,就代表新加劳动。

因此,如果从A部类的产品中,即从Pa中扣除c′,我们便得出一年内的新加劳动。

(B)生产消费品。

v″+r″在这里也是代表新加劳动。在这一领域里执行职能的不变资本c″不代表新加劳动。

但是,v″+r″=它们所交换的c′。c′转化为B部类的可变资本和收入。另一方面,v″和r″转化为c′,转化为A部类的不变资本。

如果从B部类的产品中,即从Pb中扣除c″,我们便得出一年内的新加劳动。

但是,Pb-c″=c′。因为全部产品Pb扣除c″即B部类使用的不变资本后,同c′相交换。

在v″+r″同c′交换后,情况可以表述如下:

Pa只由新加劳动构成,新加劳动的产品分解为利润和工资,分解为必要劳动的等价物和剩余劳动的等价物。因为现在代替c′的v″+r″等于B部类的新加劳动。

因此,全部产品Pa,不论是它的剩余产品,还是它的可变资本和它的不变资本,都由一年内新加劳动的产品组成。

相反,全部产品Pb可以这样来看:它不代表新加劳动的任何部分,而只代表被保存的过去劳动。因为它的c″部分不代表任何新加劳动。同样,它的用v″+r″换得的c′部分也不代表新加劳动,因为这个c′在A部类代表预付不变资本,不代表新加劳动。

由此可见,年产品中所有作为可变资本构成工人收入,作为剩余产品构成资本家的消费基金的部分都归结为新加劳动,而产品中其余所有代表不变资本的部分只归结为被保存的过去劳动,仅仅补偿不变资本。

因此,那种把年产品中所有作为收入,作为工资和利润(包括利润的分枝——地租和利息等,也包括非生产劳动者的工资)消费的部分都归结为新加劳动的看法是正确的,而把全部年产品都归结为收入,归结为工资和利润,即只归结为新加劳动中某些部分的总和的看法却是错误的。年产品中有一部分归结为不变资本,它按价值来说不代表新加劳动,而作为使用价值,既不加入工资,也不加入利润。这部分产品,按其价值来说,代表真正意义上的积累劳动,按其使用价值来说,代表这种积累的过去劳动的消费。

另一方面,认为产品中归结为工资和利润的部分不能全部代表一年内加进的劳动,这种看法同样是正确的。因为这种工资和利润可以用来购买服务,即购买不加入代表工资和利润的产品的劳动。这种服务,这种劳动,是人们在消费产品的过程中使用的,它们不加入产品的直接生产。

[857]关于(2)。

关于积累,关于收入转化为资本,关于扩大规模的再生产(就这种再生产的发生不单单是由于更有效地使用原有资本而言),情况就不同了。在这里,全部新资本是由新加劳动构成的,而且是由利润等等形式的剩余劳动构成的。不过,说这里新生产的全部要素都是由新加劳动——工人的剩余劳动的一部分——构成和产生,虽然是正确的,但是象政治经济学家们又一次假定的那样,认为剩余劳动转化为资本时只归结为可变资本或工资,却是错误的。例如,假定租地农场主的一部分剩余产品同机器厂主的一部分剩余产品交换。这种交换使机器厂主能够直接或间接地把小麦转化为可变资本,雇用更多的工人。另一方面,由于这种交换,租地农场主就把他的一部分剩余产品转化为不变资本,由于这种转化,他可能不是雇用新的工人而是解雇一部分原有的工人。其次,租地农场主可能耕种更多的土地。那样一来,一部分小麦就将不转化为工资,而转化为不变资本,等等。

只有在进行这种积累时才能看出,所有的一切,不论是收入还是可变资本和不变资本,都是被占有的别人劳动,不论是工人赖以工作的劳动条件还是工人用自己的劳动换得的等价物,都是资本家不付等价物而得到的工人劳动。

甚至在原始积累的条件下也是这样。假定我从工资中节约500镑。那末,这500镑实际上所代表的不是单纯的积累劳动,而是和资本家的“积累劳动”不同的、我自己的、由我自己和为我积累的劳动。我把它转化为资本,购买原料等等和雇用工人。假定利润是20%即每年100镑。在五年中(如果始终没有新的积累,并且每年得到的100镑都被吃掉),我以收入的形式把我的资本“吃掉”。到第六年我的这500镑资本本身就代表不付等价物而占有的别人劳动了。如果我总是把我的利润的一半积累起来,那末[把我的原有资本吃掉的]过程就会慢一些,因为我不吃掉那么多,并且[占有别人劳动的过程]会快一些。

\todo{}

到第八年,虽然我吃掉的比原有资本多,我的资本却几乎增加了一倍。在972镑资本中,已经没有丝毫有酬劳动,或者说,我曾为之支付过等价物的劳动了。我以收入的形式把我的全部原有资本消费掉了。就是说,我得到了原有资本的等价物,我又把这个等价物消费掉了。新资本仅仅是由被占有的别人劳动所构成。

在考察剩余价值本身的时候,产品的实物形式,从而剩余产品的实物形式,是无关紧要的。在考察实际再生产过程的时候,它却具有重要意义,一方面是为了理解产品形式本身,另一方面是为了弄清楚奢侈品等等的生产对再生产过程的影响。这里我们又有了一个说明使用价值本身具有经济意义的例子。

\tsectionnonum{[(c)小册子作者的功绩及其观点在理论上的混乱。他提出的关于资本主义社会中的对外贸易的作用以及“自由时间”是真正的财富等问题的意义]}

[858]现在再回过头来谈我们的小册子。

[小册子的作者写道:]

\begin{quote}{“假定一个国家的全部劳动所生产的恰好足够维持全部人口的生活;在这种情况下,很明显,就没有剩余劳动,因而也就没有什么东西可以作为资本积累起来。假定一个国家的全部劳动一年中所生产的足够维持该国人口两年的生活;在这种情况下,很明显,或者是足够维持全部人口生活一年的消费资料必须毁掉,或者是人们必须停止一年的生产劳动。但是,剩余产品——或者说资本——的所有者在下一年中既不会让人们无事可做,也不会让这些产品毁掉;他们会把人们的劳动用于某种不是直接生产的工作,例如用来安装机器等等。但是到第三年,全部人口会重新从事直接生产劳动,并且,由于上一年安装的机器现在已经开始运转,所以很明显,这一年的产品将比第一年多,因为还要加上机器的产品。因此,这种剩余产品就更加要或者毁掉,或者象上面所说的那样被使用;而这种使用会重新增加社会的劳动生产力,直到人们必须停止一段时间生产劳动,否则他们的劳动产品就要毁掉。这便是最简单的社会状态下[资本积累]的明显后果。”(第4—5页)“其他国家的需求不仅受我们的生产能力的限制,而且受他们的生产能力的限制。”}\end{quote}

{这是对萨伊的论断的回答,萨伊认为,不是我们生产得太多,而是其他国家生产得太少\endnote{马克思指萨伊的下述论断(在他的《给马尔萨斯先生的信》1820年巴黎版第15页):例如,如果英国商品充斥意大利市场,那末,原因就在于能够同英国商品交换的意大利商品生产不足。萨伊的这些论断在匿名著作《论马尔萨斯先生近来提倡的关于需求的性质和消费的必要性的原理》(1821年伦敦版第15页)中引证过,在马克思的第XII本札记本第12页对这部著作所作的摘录中也有这些论断。参看本卷第1册第237页,第2册第607页和本册第131页。——第277页。}。他们的生产能力不一定和我们的生产能力相等。}

\begin{quote}{“因为,不管我们多么努力,在若干年内整个世界从我们这里拿走的未必会比我们从世界取得的多,所以我们备受赞扬的整个对外贸易从来没有、从来不能、也决不可能为我国的财富增加一先令或一文钱,因为每有一包丝绸、一箱茶叶、一桶酒进口,就有价值相等的某种东西出口,甚至我们的商人从他们的对外贸易中取得的利润,也由这里用出口换得的进口商品的消费者支付。”(第17—18页)“对外贸易只是为了资本家舒适和享乐而进行的一种商品交换:资本家没有一百个躯体和一百双脚,他不能以衣服和袜子的形式把国内生产的全部呢绒和棉针织品都消费掉,因此它们被用来交换酒和丝绸。但是这些酒和丝绸象那些呢绒和袜子一样代表我们本国人的剩余劳动;通过这种办法资本家的破坏力无限度地增大了:由于对外贸易,资本家得以巧胜自然,突破自然对他们的剥削要求和剥削愿望设置的成千的自然限制;现在无论对于他们的实力或者对于他们的愿望,都没有什么限制了。”(第18页)}\end{quote}

我们看到,小册子的作者接受了李嘉图的对外贸易学说。在李嘉图的著作中这一学说只是用来证明他的价值理论,或者说明这一学说和价值理论并不矛盾。在小册子里则着重指出,体现在对外贸易结果上的不仅是国民的劳动,而且是国民的剩余劳动。

如果剩余劳动和剩余价值只表现在国民的剩余产品中,那末,为了价值而增加价值,从而榨取剩余劳动,就会受到[国民]劳动所创造的价值借以表现的使用价值的局限性或狭隘范围的限制。但是只有对外贸易才使作为价值的剩余产品的真正性质显示出来,因为对外贸易使剩余产品中包含的劳动作为社会劳动发展起来,这种劳动表现在无限系列的不同的使用价值上,并且在实际上使抽象财富有了意义。

\begin{quote}{“只有需要和满足这些需要所必需的商品种类的无限多样性{因而还有生产这些不同种类的商品的具体劳动的无限多样性},才使对财富的贪欲{从而占有他人劳动的贪欲}成为无止境的和永远无法满足的。”(威克菲尔德在他出版的亚·斯密《国富论》1835年伦敦版第1卷第64页上所加的注)}\end{quote}

但是,只有对外贸易,只有市场发展为世界市场,才使货币发展为世界货币,抽象劳动发展为社会劳动。抽象财富、价值、货币、从而抽象劳动的发展程度怎样,要看具体劳动发展为包括世界市场的各种不同劳动方式的总体的程度怎样。资本主义生产建立在价值上,或者说,建立在包含在产品中的作为社会劳动的劳动的发展上。但是,这一点只有在对外贸易和世界市场的基础上[才有可能]。因此,对外贸易和世界市场既是资本主义生产的前提,又是它的结果。

[859]这本小册子不是理论性论著。它是对政治经济学家们为当时的贫困和“国民困难”所找到的虚假原因的抗议。因此,它在这里并没有奢望,而且也不能对它提出要求:把剩余价值理解为剩余劳动,就要对经济范畴的整个体系进行总的批判。相反,作者以李嘉图体系为依据,只是前后一贯地作出了这一体系本身中所包含的结论,并且为了工人阶级的利益而提出这一结论来反对资本。

可是,这位作者为既有的经济范畴所束缚。就象李嘉图由于把剩余价值同利润混淆起来而陷入令人不快的矛盾一样,他也由于把剩余价值命名为资本利息而陷入同样的矛盾。

诚然,他在以下方面超过了李嘉图:首先,他把一切剩余价值都归结为剩余劳动,其次,他虽然把剩余价值叫作资本利息,同时又着重指出,他把资本利息理解为剩余劳动的一般形式,而与剩余劳动的特殊形式即地租、借贷利息和企业利润相区别:

\begin{quote}{“支付给资本家的利息,无论是采取地租、借贷利息的性质〈应当说:形式〉,还是采取企业利润的性质……”(第23页)}\end{quote}

可见,小册子的作者把剩余劳动或剩余价值的一般形式和它们的特殊形式区别开来了,李嘉图和亚·斯密却没有做到这一点,至少是没有有意识地和前后一贯地做到这一点。但是,他还是把这些特殊形式之一的名称——利息,当作一般形式的名称。这就足以使他重新陷入经济学的费解的行话中。

\begin{quote}{“在一个巩固地建立起来的社会里,资本的不断增长会由借贷利息的下降表现出来,或者同样可以说,会由为使用资本而付出的他人的劳动量的减少表现出来。”(第6页)}\end{quote}

这有点象凯里的话。但是小册子的作者认为,不是工人使用资本,而是资本使用工人。既然他把利息理解为任何形式的剩余劳动,那末全部问题(即“解决我们的国民困难”)就归结为提高工资,因为利息的减少也就是剩余劳动的减少。但他的意思是:在劳动同资本交换的时候,对别人劳动的占有必须减少,或者说,工人从他自己的劳动中占有的必须多些,而资本占有的必须少些。

要求减少剩余劳动可能有两方面的意思:

(1)工人除了再生产劳动能力、创造工资的等价物所必需的时间以外,从事的劳动必须少些;

(2)在劳动总量中采取剩余劳动(即工人无代价地为资本家劳动的时间)的形式的部分必须少些;从而,在体现劳动的产品中采取剩余产品形式的部分必须少些;也就是说,工人从他自己的产品中得到的必须比以前多些,而资本家从这一产品中得到的必须比以前少些。

作者自己对这个问题是不清楚的,这一点也可以从下面一段话——其中实际上包含着他的著作中的结论性东西——看出来:

\begin{quote}{“一个国家只有在使用资本而不支付任何利息的时候,只有在劳动6小时而不是劳动12小时的时候,才是真正富裕的。财富就是可以自由支配的时间,如此而已。”(第6页)}\end{quote}

因为这里“利息”被理解为利润、地租、借贷利息,一句话,被理解为任何形式的剩余价值,因为在小册子的作者本人看来,资本只不过是劳动产品,是积累的劳动,用它来交换,不仅能够得到等量劳动,而且能够得到剩余劳动,所以在他看来,“资本不提供利息”这种说法的意思就是没有任何[860]资本存在。产品不转化为资本。既没有剩余产品,也没有剩余劳动。只有到那时国家才真正富裕。

但是,这一点的意思可能是:除了工人的再生产所需要的产品和劳动以外,既没有产品,也没有劳动。或者是:工人自己占有这个余额,无论是产品的余额,还是劳动的余额。

不过作者所指的不只是后面一点,这从以下的事实可以看出:他把“一个国家只有在劳动6小时而不是劳动12小时的时候,才是真正富裕的”,“财富就是可以自由支配的时间,如此而已”这两个论点和“使用资本而不支付任何利息”这一论点结合起来了。

这可能是这样的意思:

如果所有的人都必须劳动,如果过度劳动者和有闲者之间的对立消灭了,——而这一点无论如何只能是资本不再存在,产品不再提供占有别人剩余劳动的权利的结果,——如果把资本创造的生产力的发展也考虑在内,那末,社会在6小时内将生产出必要的丰富产品,这6小时生产的将比现在12小时生产的还多,同时所有的人都会有6小时“可以自由支配的时间”,也就是有真正的财富,这种时间不被直接生产劳动所吸收,而是用于娱乐和休息,从而为自由活动和发展开辟广阔天地。时间是发展才能等等的广阔天地。大家知道,政治经济学家们自己认为雇佣工人的奴隶劳动是合理的,说这种奴隶劳动为其他人,为社会的另一部分,从而也为[整个]雇佣工人的社会创造余暇,创造自由时间。

或者这一点也可能有这样的意思:

工人现在除了自己的再生产(现在)所需要的以外劳动6小时。(不过这大概不会是小册子作者的观点,因为他把工人现在所需要的说成是非人道的最低限度。)如果资本不再存在,那末工人将只劳动6小时,有闲者也必须劳动同样多的时间。这样,所有的人的物质财富都将降到工人的水平。但是所有的人都将有自由时间,都将有可供自己发展的时间。

显然,小册子的作者本人对这一点是不清楚的。不过下面这段话无论如何仍不失为一个精彩的命题:

\begin{quote}{“一个国家只有在劳动6小时而不是劳动12小时的时候,才是真正富裕的。财富就是可以自由支配的时间,如此而已。”}\end{quote}

李嘉图在《价值和财富,它们的特性》一章中也说,真正的财富在于用尽量少的价值创造出尽量多的使用价值,换句话说,就是在尽量少的劳动时间里创造出尽量丰富的物质财富。这里,“可以自由支配的时间”以及对别人劳动时间里创造出来的东西的享受,都表现为真正的财富,但是正象资本主义生产中的一切东西一样,因而正象资本主义生产的解释者所认为的那样,这是以对立的形式表现出来的。财富和价值的对立后来在李嘉图的著作里表现为这样的形式,即纯产品在总产品中占的比例应当尽量的大,而这(又是在这种对立的形式上)意味着,社会上那些虽然享受物质生产成果、但是其时间只有一部分被物质生产吸收或者完全不被物质生产吸收的阶级,与时间全部被物质生产吸收、因而其消费仅仅构成生产费用的一个项目、仅仅构成一种使其充当上层阶级的驮畜的条件的那些阶级比较起来,人数应当尽可能地多。这一点总是意味着期望社会上注定陷入劳动奴隶制即从事强制劳动的部分尽可能地小。这就是那些站在资本主义立场上的人所能达到的最高点。

小册子的作者批驳了这一点。即使交换价值消灭了,劳动时间也始终是财富的创造实体和生产财富所需要的费用的尺度。但是自由时间,可以支配的时间,就是财富本身:一部分用于消费产品,一部分用于从事自由活动,这种自由活动不象劳动那样是在必须实现的外在目的的压力下决定的,而这种外在目的的实现是自然的必然性,或者说社会义务——怎么说都行。

不言而喻,随着雇主和工人之间的社会对立的消灭等等,劳动时间本身——由于限制在正常长度之内,其次,由于不再用于别人而是用于我自己——将作为真正的社会劳动,最后,作为自由时间的基础,而取得完全不同的、更自由的性质,这种同时作为拥有自由时间的人的劳动时间,必将比役畜的劳动时间具有高得多的质量。

\tchapternonum{(2)莱文斯顿[把资本看成工人的剩余产品。把资本主义发展的对抗形式同资本主义发展的内容本身混淆起来。由此产生的对生产力的资本主义发展成果的否定态度]}

[861]皮尔西·莱文斯顿硕士《论公债制度及其影响》1824年伦敦版。

这是一部非常出色的著作。

小册子《国民困难的原因及其解决办法》的作者是就剩余价值的原始形式,即剩余劳动形式来考察剩余价值的。所以劳动时间的长短成了他的主要着眼点。他主要是就剩余劳动或剩余价值的绝对形式,即在工人本身的再生产所必要的劳动时间以外延长劳动时间的形式,而不是通过劳动生产力的发展缩减必要劳动的形式,来考察剩余劳动或剩余价值的。

缩减这种必要劳动是李嘉图的主要着眼点,但是,在资本主义生产的情况下,缩减必要劳动是延长属于资本所有的劳动时间的一种手段。与此相反,小册子的作者却把缩短生产者的劳动时间和停止为剩余产品的所有者劳动宣布为最终目的。

莱文斯顿似乎以工作日既定为前提。因此,他论述的主要对象——在这些论述中,也和小册子《国民困难的原因及其解决办法》的作者一样,只是附带涉及到一些理论问题——是相对剩余价值,或者说,由于劳动生产力的发展而归资本所有的剩余产品。就象抱这种观点的人一般所做的那样,这里多半是就剩余产品的形式来考察剩余劳动,而小册子的作者则多半是就剩余劳动的形式来考察剩余产品。

\begin{quote}{“教导人们说一国的富强取决于它的资本,就是要使劳动从属于财富,使人变成财产的奴仆。”(第7页)}\end{quote}

李嘉图的理论在它自己的前提基础上产生的对立面具有如下的特点:

政治经济学,随着它的不断发展,——这种发展,就基本原则来说,在李嘉图的著作里表现得最突出,——越来越明确地把劳动说成是价值的唯一要素和使用价值的唯一[积极的]创造者,把生产力的发展说成是实际增加财富的唯一手段,而把劳动生产力的尽可能快的发展说成是社会的经济基础。实际上,这也就是资本主义生产的基础。特别是李嘉图的著作,在它证明价值规律既不受土地所有权也不受资本积累等等的破坏的时候,其实只是企图把一切和这种见解矛盾或似乎矛盾的现象从理论中排除出去。但是,正象劳动被理解为交换价值的唯一源泉和使用价值的积极源泉一样,“资本”也被同一些政治经济学家,特别是大卫·李嘉图(在他以后,托伦斯、马尔萨斯、贝利等人更是这样)看作是生产的调节者、财富的源泉和生产的目的,而在他们的著作里,劳动表现为雇佣劳动,这种雇佣劳动的承担者和实际工具必然是赤贫者(而且这里还有马尔萨斯的人口论在起作用),他们只是生产费用的一个项目和单纯的生产工具,注定只能拿最低限度的工资,每当工人对资本来说成为“多余的”时候,还不得不降到这一最低限度以下。在这个矛盾中,政治经济学只是说出了资本主义生产的本质,或者也可以说,雇佣劳动,即从本身中异化出来的劳动的本质,这种劳动创造的财富作为别人的财富和它相对立,它自己的生产力作为它的产品的生产力和它相对立,它的致富过程作为自身的贫困化过程和它相对立,它的社会力量作为支配它的社会力量和它相对立。但是,这些政治经济学家把社会劳动在资本主义生产中表现出来的这种一定的、特殊的、历史的形式说成是一般的、永恒的形式,说成是自然的真理,而把这种生产关系说成是社会劳动的绝对(而不是历史地)必然的、自然的、合理的关系。由于受到资本主义生产视野的局限,他们把社会劳动在这里借以表现的对立形式说成和摆脱了上述对立的这一劳动本身一样是必然的。这样,他们一方面把绝对意义上的劳动(因为在他们看来,雇佣劳动和劳动是等同的),另一方面又把同样绝对意义上的资本,把工人的贫困和不劳动者的财富同时说成是财富的唯一源泉,他们不断地在绝对的矛盾中运动而毫不觉察。(西斯蒙第由于觉察到了这种矛盾而在政治经济学上开辟了一个时代。)“劳动,或者说,资本”——在李嘉图的这种说法\endnote{马克思在《剩余价值理论》第二册引用和分析了李嘉图著作中包含这一说法的引文(见本卷第2册第200、202和456页)。——第285页。}中,矛盾本身以及把这种矛盾当作等同的东西说出来的那种天真,明显地表现出来了。

但是很明显,既然使资产阶级政治经济学在理论上作了这种毫不留情的表述的那同一种现实的发展,又发展了现实本身所包含的实际矛盾,特别是发展了英国日益增长的“国民”财富和日益增长的工人贫困之间的对立,其次,既然这些矛盾在李嘉图以及其他政治经济学家的理论中得到了理论上中肯的、尽管是无意识的表现,那末,站到无产阶级方面来的思想家[XV—862]抓住了在理论上已经给他们准备好了的矛盾,是十分自然的。劳动是交换价值的唯一源泉和使用价值的唯一的积极的创造者。你们这样说。另一方面,你们说,资本就是一切,而工人算不了什么,或者说,工人仅仅是资本的生产费用的一个项目。你们自己驳倒了自己。资本不过是对工人的诈骗。劳动才是一切。

这实际上是从李嘉图的观点,从李嘉图自己的前提出发来维护无产阶级利益的一切著作的最后的话。李嘉图不懂得他的体系中所论述的资本和劳动的等同,同样,这些著作的作者也不懂得他们所论述的资本和劳动之间的矛盾。因此,即使是他们中间最出色的人物,如霍吉斯金,也把资本主义生产的一切经济前提看作是永恒的形式,他们所希望的只是消灭资本——这些前提的基础,同时也是必然结果。

莱文斯顿的主要思想是:

劳动生产力的发展创造了资本,或者说,财产,即为“有闲者”——游手好闲者、非劳动者——创造剩余产品,同时劳动还生出了它的寄生赘瘤;劳动生产力越发展,这个寄生赘瘤就越把劳动的骨髓吸尽。非劳动者获得占有这种剩余产品的权利,或者说,获得占有别人劳动产品的权力,是由于他已经拥有财富,还是由于他有土地、土地所有权,这并不会使事情发生变化。两者都是资本,即都是对别人劳动产品的支配权。财产——property——在莱文斯顿看来,只是对别人劳动产品的占有,而这一点只有在生产劳动发展的情况下才有可能,而且只可能是与生产劳动发展的程度相适应。莱文斯顿把生产劳动理解为生产必需品的劳动。非生产劳动,“消费劳动”\endnote{莱文斯顿所说的“消费劳动”(“industryofconsumption”),是指奢侈品的生产和为财产所有者进行的各种服务。——第286页。}是资本或财产发展的结果之一。莱文斯顿和小册子《国民困难的原因及其解决办法》的作者一样,表现为一个禁欲主义者。在这里,他本身又是为政治经济学家的概念所束缚。没有资本,没有财产,工人消费的必需品便会生产得极其丰富,但不会有奢侈品的生产。或者也可以说,既然在小册子的作者看来,资本生产维持工人生活所必需的劳动以外的剩余劳动,并且引起机器(小册子的作者称为“固定资本”)的制造以及对外贸易和世界市场的建立,部分是为了利用从工人那里榨取的剩余产品去增进生产力,部分是为了使这种剩余产品成为必需品以外的多种多样的使用价值,——既然如此,那末莱文斯顿同小册子的作者一样,是理解,或者至少是在实际上承认资本的历史必然性的。同样,在莱文斯顿看来,没有资本和财产,就既不会有“舒适品”、机器或奢侈品生产出来,也不会有自然科学的发展,也不会有靠余暇或靠富人从非劳动者那里取得自己“剩余产品”的等价物的欲望才能存在的精神产品。

小册子的作者和莱文斯顿说这些话并不是为资本辩护,而是以此作为攻击资本的出发点,因为所有这一切都纯粹是违背工人的利益而不是为了工人。但是他们这样实际上也就承认这是资本主义生产的结果,承认资本主义生产因而是社会发展的一种历史形式,尽管这种历史形式是和构成整个这一发展基础的那一部分人口的利益相矛盾的。在这方面他们(虽然是从相反的一极出发)也具有政治经济学家们的局限性,即把这一发展的对立形式和这一发展的内容本身混淆起来。一些人为了这种对立的成果而希望这种对立永世长存。另一些人则为了摆脱对立而决心牺牲在这种对立形式范围内产生的成果。这就使这种对于[资产阶级]政治经济学的反对不同于同一时期的欧文等人,另一方面也不同于为了摆脱尖锐形式的对立而想回到古老的对立形式的西斯蒙第。

[莱文斯顿写道:]

\begin{quote}{“穷人的贫困创造了他的〈富人的〉财富……如果一切人都是平等的,那末谁也不会为别人劳动。必需品将会有余,而奢侈品将会绝迹。”(第10页)“生产产品的劳动是财产的父亲,帮助别人消费产品的劳动是财产的孩子。”(第12页)“财产的增加,维持有闲者和非生产劳动的能力的增长,这就是政治经济学上称为资本的东西。”(第13页)“因为财产的使命就是花费,因为没有花费,财产对于它的所有者来说就完全是无用的东西,所以财产的存在是和消费劳动的存在[863]密切地联系在一起的。”(同上)“如果每个人的劳动刚够生产他自己的食物,那就不会有任何财产了,也就不会有任何一部分人民的劳动用来满足想象的需要了。”(第14—15页)“在社会发展的每一个阶段上,随满每个人的劳动生产率由于人口的增长和技术设备的改良而提高,劳动的人数会逐渐减少……财产由于生产资料的改良而增加;财产的唯一使命就是鼓励懒散。当每一个人的劳动勉强够维持他自己的生活的时候,因为不可能有财产,所以不会有有闲者。如果一个人的劳动能够养活五口人,那末一个从事生产的人就将负担四个有闲者的生活,因为只有这样产品才能消费掉……社会的目标就是牺牲勤劳者来抬高有闲者,从富裕中创造出实力。”(第11页)}\end{quote}

{莱文斯顿关于地租所说的话(不完全正确,因为正是在这里须要说明为什么地租落到土地所有者手里而不落到租地农场主,产业资本家手里)适用于因劳动生产力增长而发展的一般剩余价值:

\begin{quote}{“在社会发展的早期阶段,人们还没有人造的辅助手段来促进他们的劳动生产力,他们的收益中可以作为地租支出的部分是极小的;因为土地没有自然价值,它的全部产品都靠劳动。但是劳动技能每有提高,都会增加可以用来支付地租的那部分产品。在维持十个人的生活需要九个人的劳动的地方,总产品中只有1/10可以用作地租。在一个人的劳动足够维持五个人的生活的地方,就会有4/5的产品用作地租或用于国家的只能由劳动的剩余产品来满足的其他需要。前者似乎是英国被征服时期的情况,后者就象英国的现状,现在只有1/5的人口从事农业。”(第45—46页)“社会把每一种改良都只用来增加懒散,这种情况是千真万确的。”(第48页)}}\end{quote}

注。莱文斯顿的著作是独特的。它的直接的题目,正如书名写的,是现代的公债制度。其中莱文斯顿谈到:

\begin{quote}{“反对法国革命[后来反对拿破仑]的全部战争,除了把一些犹太人变成绅士和把一些笨蛋变成政治经济学家以外,没有作出任何高尚的事情。”(第66—67页)“公债制度也有一个好的结果:尽管它从国内老贵族那里夺取其一大部分财产,以便把这些财产转交给新出现的西班牙式绅士,作为对他们的欺骗和盗窃国库的高超手段的奖励……既然它鼓励欺骗和卑鄙行为,给招摇撞骗和自命不凡披上智慧的外衣,把全体人民变成从事证券投机的民族……既然它破坏了关于等级和门第的一切偏见,使货币成为人与人之间唯一的区别标志……它也就破坏了财产的永恒性。”(第51—52页)}\end{quote}

\tchapternonum{(3)霍吉斯金}

《保护劳动反对资本的要求,或资本非生产性的证明。关于当前雇佣工人的团结》,一个工人著,1825年伦敦版。

托马斯·霍吉斯金《通俗政治经济学。在伦敦技术学校的四次演讲》1827年伦敦版。

第一部匿名的著作也是霍吉斯金写的。如果说前面谈到的那些小册子以及其他许多类似的小册子都无声无息地过去了,那末霍吉斯金的这两部著作,特别是第一部著作,却引起了强烈的反应,至今仍然可算是(参看约翰·莱勒《货币和道德》1852年伦敦版[第XXIV页和第319—322页])英国政治经济学方面的重要著作。这里我们将依次考察这两部著作。

\tsectionnonum{[(a)资本的非生产性的论点是从李嘉图理论中得出的必然结论]}

在《保护劳动反对资本的要求,或资本非生产性的证明》这本小册子中,作者正象书名所表明的那样,想证明“资本的非生产性”。

李嘉图从来没有断言资本就生产价值的意义来说是生产的。李嘉图认为,资本加到产品上的只是它自己的价值,而它自己的价值则取决于它的再生产所需要的劳动时间。它只是作为“积累劳动”(更确切地说,作为[864]物化劳动)才具有价值,它只是把它的这一价值加到它所加入的产品中去。的确,李嘉图在一般利润率问题上犯了前后矛盾的错误。而这也正是他的反对者们用来抓住他的一个矛盾。

至于谈到生产使用价值意义上的资本的生产性,那末,在斯密、李嘉图等人看来,一般说,在政治经济学家看来,它不过是指过去的有用劳动的产品重新用作生产资料,即劳动对象、劳动工具和工人的生活资料。劳动的客观条件不是象在原始状态下那样表现为简单的自然物(作为简单的自然物,它们从来不会是资本),而是表现为已被人类的活动改造过的自然物。但是在这个意义上,“资本”这个词完全是多余的,是什么也说明不了的。小麦可以食用并不是因为它是资本,而是因为它是小麦。羊毛的使用价值是它作为羊毛所固有的,而不是作为资本所固有的。同样,蒸汽机的作用同它作为资本的存在毫无共同之处。如果蒸汽机不是“资本”,不属于工厂主而属于工人,它会提供同样的服务。在实际的劳动过程中,所有这些物之所以能提供服务,是因为它们作为使用价值,而不是作为交换价值,更不是作为资本,同加到它们身上的劳动发生关系。它们在这里是生产的,或者更确切地说,劳动生产率实现在作为自己的物质的它们身上,原因在于它们作为实际劳动的客观条件的属性,而不在于它们作为独立地同工人相对立、同工人相异化的条件、作为体现在资本家身上的活劳动支配者的那种社会存在。按照霍普金斯(不是我们的霍吉斯金)的正确说法\endnote{霍普金斯的著作《论地租及其对生存资料和人口的影响》(1828年伦敦版第126页)中有关的段落,马克思在《对所谓李嘉图地租规律的发现史的评论》一章中引用过(见本卷第2册第151页)。——第291页。},它们在这里是作为财富,而不是作为“纯”财富,是作为产品,而不是作为“纯”产品被消费和被使用的。固然,在政治经济学家的头脑里,也和在资本家的头脑里一样,这些物在同劳动的关系上所采取的一定的社会形式,是和它们作为劳动过程的因素的实际特性交织在一起,并且彼此不可分割地结合在一起的。但是,政治经济学家们一到着手分析劳动过程的时候,他们便不得不把“资本”这个用语完全抛开,而去谈论劳动材料、劳动资料和生活资料。但是在产品作为材料、工具和工人的生活资料这种特性中所反映的只是它们作为物的条件同劳动的关系;劳动本身在这里表现为支配它们的活动。在这方面绝对没有劳动和资本的关系,而只有人类合乎目的的活动在再生产过程中同它自己的产品的关系。它们仍然是劳动产品,仍然不过是劳动自由支配的对象。它们只是表示一种关系,在这种关系的范围内,劳动把它本身所创造的,至少是在这种形式上所创造的物的世界占为己有;但是,除了表示活动必须和它的材料相适应外,它们决不表示这些物对劳动的任何其他支配权,否则它就不是合乎目的的活动,就不是劳动了。

只有把资本看作一定的社会生产关系的表现,才能谈资本的生产性。但是如果这样来看资本,那末这种关系的历史暂时性质就会立刻显露出来,对这种关系的一般认识,是同它的继续不断的存在不相容的,这种关系本身为自己的灭亡创造了手段。

但是政治经济学家们没有把资本看成是这样一种关系,因为他们不敢承认它的相对性质,也不理解这种性质;相反,他们只是从理论上反映了为资本主义生产所束缚的、受资本主义生产支配的、同资本主义生产有利害关系的实际家们的观念。

霍吉斯金本人在[反对资产阶级政治经济学的]论战中是从政治经济学家们的狭隘的观念出发的。既然这些政治经济学家把资本描述成永恒的生产关系,他们就把资本归结为劳动对它的物质条件的一般关系,这种一般关系是一切生产方式所共有的,并不包含资本的特殊性质。在他们认为资本创造“价值”时,他们中间一些最优秀的人——[特别是]李嘉图——都承认,它除了创造它从劳动中过去得到以及现在不断得到的价值以外,并没有创造别的价值,因为一个产品所包含的价值是由再生产它所必需的劳动时间决定的,即由产品作为活的现在劳动的结果而不是过去劳动的结果这样一种关系决定的。而劳动生产率的增长,正象李嘉图着重指出的那样,恰好表现在过去劳动的产品的不断贬值上。另一方面,政治经济学家们经常把这些物借以表现为资本的一定的特殊形式同它们作为物以及作为一切劳动过程的简单因素的属性混为一谈。作为“劳动的使用者”\endnote{马克思指英国的流行的说法“capitalemployslabour”(“资本使用劳动”),这种说法反映了资本与雇佣劳动的关系的实质本身。马克思在1861—1863年手稿第XXI本上在揭示“资本的生产性”的含义时写道“……生产资料,劳动的物的条件——劳动材料、劳动资料(以及生活资料)——[在资本主义生产方式下]也不是从属于工人,相反,是工人从属于它们。不是工人使用它们,而是它们使用工人。正因为这样,它们才是资本”(见本卷第1册第419页;并参看第1册第73页、第2册第479页和本册第122—123页)。——第292页。}的资本所含的奥妙,他们却没有说明,他们只是不断无意识地把这种奥妙说成是某种同资本的物的性质不可分离的东西。

[867]\endnote{手稿这个地方的页码弄乱了:在第864页之后,正文转入第867页,接着是第868、869、870、870a页,然后是第865、866页,最后是第870b、871、872页等等。从不衔接的一页转到另一页,是马克思自己指出的。——第292页。}第一本小册子\authornote{指匿名小册子《根据政治经济学基本原理得出的国民困难的原因及其解决办法》。——编者注}从李嘉图的理论中作出了正确的结论,把剩余价值归结为剩余劳动。这一点同李嘉图的反对者和追随者所做的相反,因为他们死抱住李嘉图对剩余价值和利润的混淆不放。

第二本小册子\authornote{指莱文斯顿的小册子《论公债制度及其影响》。——编者注}也和李嘉图的反对者和追随者相反,更准确地规定了取决于劳动生产力发展程度的相对剩余价值。李嘉图也谈到了这一点,但是他避开了莱文斯顿所作的结论:劳动生产力的提高只是增加了别人的、支配劳动的财富即资本。

最后,第三本小册子\authornote{指霍吉斯金的匿名小册子《保护劳动反对资本的要求,或资本非生产性的证明》。——编者注}把作为李嘉图论述问题的必然结果的总的论点表达出来了:资本是非生产的。这一点是针对托伦斯、马尔萨斯等人的,因为这些人继续发展李嘉图学说的一个方面,把他的“劳动是价值的创造者”这个论点变成了“资本是价值的创造者”这样一个相反的论点。同时,在这本小册子里还反驳了劳动绝对取决于作为劳动存在条件的现有资本量这个论点,这个论点贯穿在从斯密到马尔萨斯的著作中,特别是被马尔萨斯(也被詹姆斯·穆勒)奉为绝对教条。

第一本小册子是以这样的命题结束的:

\begin{quote}{“财富就是可以自由支配的时间,如此而已。”\authornote{见本册第279—282页。——编者注}}\end{quote}

\tsectionnonum{[(b)反驳李嘉图的资本是积累劳动的定义。关于并存劳动的见解。对物化的过去劳动的意义估计不足。现存财富同生产运动的关系]}

霍吉斯金认为“流动资本”只不过是不同种类的社会劳动的并存(“并存劳动”),而积累只不过是社会劳动生产力的积累,所以工人本身的技能和知识(科学力量)的积累是主要的积累,比和它一同进行并且只是反映它的那种积累,即这种积累活动的现存客观条件的积累,重要得多,而这些客观条件会不断重新生产和重新消费,只是名义上进行积累:

\begin{quote}{“生产资本和熟练劳动……是一个东西……资本和工人人口完全是一个意思。”[《保护劳动反对资本的要求》第33页]}\end{quote}

这一切都只是加利阿尼的命题的进一步发展:

\begin{quote}{“真正的财富是……人”(《货币论》,库斯托第编,现代部分,第3卷第229页)。}\end{quote}

整个客观世界,“物质财富世界”,在这里不过是作为从事社会生产的人的因素,不过是作为从事社会生产的人的正在消失而又不断重新产生的实践活动而退居次要地位。请把这种“理想主义”同李嘉图的理论在“这个不可相信的修鞋匠”\endnote{“这个不可相信的修鞋匠”(《thisincrediblecobbler》)——《对麦克库洛赫先生的,〈政治经济学原理〉的若干说明》这一小册子的作者对麦克库洛赫的称呼。见前面正文第203页。——第260、294页。}麦克库洛赫的著作中变成的粗野的物质拜物教比较一下,在他的著作中,不仅人和动物的区别不见了,甚至连有生物和物之间的区别也不见了。让人们还去说什么在崇高的资产阶级政治经济学的唯灵论面前,无产阶级反对派所鼓吹的只是以满足鄙俗的需要为目的的粗野的唯物主义吧!

霍吉斯金的错误在于:在研究资本的生产性时,他没有区别在什么程度上涉及到使用价值的生产,在什么程度上涉及到交换价值的生产。

其次(但是从历史上说这是有它的理由的),他所考察的资本是他在政治经济学家们那里发现的资本。一方面,资本(在它加入实际劳动过程的情况下)被说成是劳动的单纯的物的条件,或者说只具有劳动的物质要素的意义;而且(在价值形成过程中)只不过是用时间来计算的一定的劳动量,也就是和这一劳动量本身没有什么区别的东西。另一方面,尽管资本在它出现在实际生产过程中的情况下,实际上不过是劳动本身的名称,是劳动的别名,它却被说成是支配劳动和决定劳动的力量,是劳动生产率的基础,是同劳动无关的财富。而这是没有任何中介过程的。这就是霍吉斯金在他的前辈那里发现的东西。他针对资产阶级关于经济发展的欺人之谈,阐述了这一发展的真实情况。

\begin{quote}{“资本是一种神秘的词,就象教会或者国家,或者由宰割其他人的人为了掩盖拿刀的手而发明的普通术语中的其他任何词一样。”(《保护劳动》第17页)}\end{quote}

其次,霍吉斯金按照他在政治经济学家们那里发现的传统,区分了流动资本和固定资本,而且他所理解的流动资本,主要是指流动资本中由工人的生活资料构成的或用作这种生活资料的部分。

\begin{quote}{“政治经济学家们断言,没有过去的资本积累,分工是不可能的”……但是“那些被认为由名叫流动资本的商品储备产生的结果,是由并存劳动引起的”。(第8—9页)}\end{quote}

面对着政治经济学家们的粗陋的理解,霍吉斯金有权说“流动资本”只是特殊“商品”的“储备”的“名称”。因为政治经济学家们没有说明在商品的形态变化中表现出来的特殊社会关系,所以他们只能从物质上去理解“流动”资本。从流通过程中产生的资本的一切[868]区别,——其实是资本流通过程本身,——事实上都不过是作为再生产过程因素的(由于同雇佣劳动的关系而取得资本性质的)商品的形态变化。

从某种意义上说,分工无非是并存劳动,即表现在不同种类的产品(或者更确切地说,商品)中的不同种类的劳动的并存。在资本主义的意义上,分工就是生产某种商品的特殊劳动分为一定数量的简单的、在不同工人之间分配而又相互联系的工序,它以行业划分这种社会内部即作坊外部的分工为前提。另一方面,作坊内部的分工又扩大了社会内部的分工。产品本身越片面,它所交换的商品越多样化,表现它的交换价值的使用价值的系列越大,它的市场越大,产品就越能在更充分的意义上作为商品来生产,它的交换价值就越不取决于它作为使用价值的直接存在,或者说,它的生产就越不取决于它的生产者对它的消费,越不取决于它作为它的生产者的使用价值的存在。情况越是这样,产品就越能作为商品来生产,因而也就越能大量地进行生产。产品的使用价值对产品生产者无关紧要这一事实,会在产品生产的总量中在量上表现出来,即使该产品的生产者同时又是他自己产品的消费者,这个总量同该产品生产者的消费需要也没有任何关系。但是,作坊内部的分工是这种大规模生产的方法之一,因而也是[作为商品的]产品的生产方法之一。因此,作坊内部的分工是以社会内部的行业划分为基础的。

市场的大小有两层意思:第一,消费者的数量,他们的人数;第二,也包括彼此独立的行业的数量。即使前者的数量不增加,后者的数量也可能增加。例如,当纺纱和织布从家庭工业和农业中分离出来时,所有土地耕种者就都成了纺纱者和织布者的市场。同样,后两者由于他们的行业划分现在也互为市场。社会内部分工的前提首先是不同种类劳动的相互独立,即它们的产品必须作为商品相互对立,并且通过交换,完成商品的形态变化,作为商品相互发生关系。(因此,在中世纪城市禁止农村从事尽可能多的职业。其目的不仅是为了排除竞争,——亚·斯密在这里只看到这一点,——而且是为了给自己开辟市场。)另一方面,社会内部的分工要得到适当的发展,就必须以一定的人口密度为前提。作坊内部分工的发展更是以这种人口密度为前提。以前一种分工的一定发展程度为前提的这后一种分工,又从自己这方面同前一种分工相互发生作用,并增进前一种分工,因为它把从前相互有联系的行业分为彼此独立的行业,增加和分化它们所间接需要的准备工作,同时,由于生产和人口的增加,资本和劳动的游离,它还创造出新的需要和满足这些需要的新的方法。

因此,霍吉斯金说“分工”不是被称为流动资本的商品储备的结果,而是“并存劳动”的结果,如果这里他所说的分工是指行业划分,那就是同义反复。这只是意味着分工是分工的原因或结果。因此,霍吉斯金所指的只能是:作坊内部的分工是以行业划分、社会分工为条件,并且在一定意义上是社会分工的结果。

不是“商品储备”造成这种行业划分,从而造成作坊内部的分工,而是上述行业划分(和分工)表现在商品的储备上,或者更确切地说,表现在产品的储备变为商品的储备这一点上。{但在政治经济学家们的著作中,总是不可避免地把资本主义生产方式的属性、特征,即资本本身的属性、特征(就资本表示生产者相互之间以及同自己产品之间的一定关系而言),说成是物的属性。}

[869]但是,如果在经济意义上(见杜尔哥、斯密等的著作)说“过去的资本积累”就是分工的条件,那末,这是指作为资本的商品储备事先在劳动的买者手里的积聚,因为作为分工的一个特点的这种协作形式,是以工人的集结,因而以他们在劳动期间所需的生活资料的积累为前提,以劳动生产率的提高,因而以劳动不断进行所必需的原料、工具和辅助材料数量的增加为前提(因为劳动不断需要大量这些东西),一句话,以大规模生产的客观条件为前提。

在这里,资本积累不可能指“作为分工条件的生活资料、原料和劳动工具数量的增加”,因为,如果把资本积累理解为这种资料的积累,那它就是分工的结果,而不是分工的前提。

在这里,资本积累也不可能意味着在新的生活资料生产出来以前,工人的生活资料一般就必须具备,或者说,工人已经生产出来的劳动产品必须用作新的生产的原料和劳动资料。因为这是一般的劳动条件,在分工发展以前也同样是如此。

一方面,从物质要素的观点来看,积累在这里无非是指:分工使生活资料和劳动资料的积聚成为必要,而这些生活资料和劳动资料在以前,当劳动者在各个行业中(在这种假定下,行业不可能是很多的)自己一个接一个地去完成生产一种或几种产品所需要的所有不同工序的时候,是零星分散的。这里的前提不是绝对的增加,而是积聚:把较大量的生活资料和劳动资料集结在一点,而且比集结在一起的工人人数相对地多。例如,在工场手工业中工人所需要的亚麻(与其人数相比)就比一切以副业方式纺麻的农民和农妇所需要的多。因此就有工人的集结以及原料、工具和生活资料的积聚。

另一方面:在产生这一过程的历史基础上(工场手工业,即以分工为特点的工业生产方式,就是从中发展起来的),这种积聚只能以这样的形式进行,即这些工人作为雇佣工人,作为被迫出卖自己劳动能力的工人集结在一起,因为他们的劳动条件作为别人的财产,作为别人的力量独立地同他们相对立,而这一点包括:这些劳动条件作为资本同他们相对立,也就是上述生活资料和劳动资料,或者同样可以说,依靠货币而拥有的对它们的支配权,掌握在单个的货币所有者或商品所有者手里,他们因此变成了资本家。劳动者丧失劳动条件表现为这些劳动条件作为资本离开劳动者而获得独立,或者说,表现为资本家支配这些劳动条件。

所以,象我指出的那样\endnote{马克思引用他在当时(1862年10月)还没有写成的关于原始积累的一节,按照马克思的计划(见本卷第1册第446页),这一节应当放在《剩余价值理论》一节之前。这一节的初稿包括在1857—1858年经济学手稿里(见卡·马克思《政治经济学批判大纲》1939年莫斯科版第363—374页)。——第299页。},原始积累无非是那些作为同劳动和工人对立的独立力量的劳动条件的分离。历史的过程使这种分离成为社会发展的因素。既然资本已经存在,那末,这种分离的保持和再生产就从资本主义生产方式本身中以越来越大的规模发展起来,直到发生历史变革。

使资本家成为资本家的不是对货币的占有。要使货币转化为资本,必须具备资本主义生产的前提,上述分离就是资本主义生产的第一个历史前提。在资本主义生产本身的范围内,这种分离,因而作为资本的劳动条件的存在,是既定的;这是生产本身的不断再生产出来和不断扩大的基础。

积累现在通过把利润,或者说剩余产品,再转化为资本而成为经常的过程,因此,数量已经增加了的、同时是劳动的客观条件、再生产条件的劳动产品,经常作为资本,作为从劳动异化出来的、支配劳动的和在资本家身上个性化了的力量同劳动相对立。但是这样一来,积累,即把一部分剩余产品再转化为劳动条件,就成了资本家的特殊职能。愚蠢的政治经济学家由此得出结论说:这种事情如果不在这种对抗的特殊形式上进行,就根本不可能进行。在他的脑子里,扩大规模的再生产是和这种再生产的资本主义形式——积累——分不开的。

[870]积累只是把原始积累中作为特殊的历史过程,作为资本产生的过程,作为从一种生产方式到另一种生产方式的过渡出现的东西表现为连续的过程。

政治经济学家们为资本主义生产代理人的观念所束缚,陷入了双重的、但是互为条件的概念的混淆。

一方面,他们把资本从一种关系变成一种物,变成“商品储备”(这时他们已经忘掉商品本身不单纯是物),这些商品由于被用作新劳动的生产条件而被称为资本,并按其再生产方式被称为流动资本。

另一方面,他们又把物变成资本,即把表现在物上并通过物表现的社会关系,看成物本身只要作为要素加入劳动过程或工艺过程就具有的属性。

因此,[一方面,]作为支配劳动的力量,作为分工的先决条件的原料和对生活资料的支配权在不劳动者手里的积聚(后来,分工不仅使积聚增多,而且由于劳动生产力的提高,也使被积聚的总量增多),就是说,作为分工条件的资本的预先积累,在政治经济学家们看来意味着生活资料和劳动资料的量的增加或积聚(他们没有区别这两者)。

另一方面,在他们看来,如果生活资料和劳动资料不具有成为资本的属性,如果构成劳动条件的劳动产品不消费劳动本身,如果过去劳动不消费活劳动,如果这些物属于工人而不属于自己本身或受委托的资本家,那末,这些生活资料和劳动资料就不会作为生产的客观条件起作用。

如果劳动条件属于联合起来的工人,如果这些工人同劳动条件的关系,就象同自然的劳动条件的关系一样,也就是象同他们自己的产品和他们自己活动的物的要素的关系一样,那末,分工似乎就不是同样可能的(虽然分工在历史上不可能从一开始就以它只有作为资本主义生产发展的结果才能表现出来的那种形式出现)。

其次,因为在资本主义生产的条件下,资本占有工人的剩余产品,因为资本已经占有的那些劳动产品现在因此而以资本的形式同工人相对立,所以很明显,剩余产品转化为劳动条件,只能从资本家那里开始,并且只能采取这样的形式,即资本家把不付等价物而占有的劳动产品变成获取新的不付等价物的劳动的生产资料。因此,扩大再生产就表现为利润转化为资本,表现为资本家的节约,资本家不是把他无代价地得到的剩余产品吃光,而是把它重新变为剥削劳动的手段,但是要做到这一点,只能通过把它重新转化为生产资本,其中也包括把剩余产品转化为劳动资料。因此,政治经济学家得出结论说:如果剩余产品事先不从工人的产品转化为他的雇主的财产,以便以后重新用作资本并重复过去的剥削过程,剩余产品就不能充当新的生产的要素。一些蹩脚的政治经济学家把贮藏和货币贮藏的观念也归入这一点。甚至一些优秀的政治经济学家,如李嘉图,也把关于禁欲的观念从货币贮藏者那里移到资本家身上。

政治经济学家们没有把资本看作是一种关系。他们不可能这样看待资本,因为他们没有同时把资本看作是历史上暂时的、相对的而不是绝对的生产形式。霍吉斯金本人也没有这样来理解资本。只要这样的理解是为资本辩护的话,它就不是为政治经济学家们对资本的辩护进行辩护,而是相反地否定他们的辩护。因此,霍吉斯金同对资本的这种看法没有任何关系。

就霍吉斯金和政治经济学家们之间存在的情况来看,他的论战的性质看来是预先确定了的并且是很简单的。霍吉斯金本来只是应该借助政治经济学家们“科学地”发展了的一个方面,来反对他们不加考虑地、无意识地和天真地从资本主义的思想方式接受来的拜物教观念,并且大致这样说:

如果工人想利用自己的产品来进行新的生产,那就必须把过去劳动的产品(一般说,劳动产品)当作材料、工具和生活资料来使用。他的产品的这种一定的消费方式是生产性的。但是,对工人的产品的这种使用,工人消费自己产品的这种方式,同这种产品对工人本身的支配,同这种产品作为资本的存在,同原料和生活资料的集中掌握[870a]在个别资本家手中,以及同工人被剥夺了对他们产品的所有权,究竟有什么关系呢?这同工人首先必须白白地把自己的产品交给第三者,以便后来用自己的劳动再从第三者那里把它赎回来,为此他不得不付给第三者比产品里包含的劳动更多的劳动来交换这一产品,并且这样来为资本家创造新的剩余产品,又有什么关系呢?

在这里,过去劳动表现在两种形式上。第一,表现为产品,使用价值。生产过程要求工人把这一产品的一部分[作为生活资料]消费,而把另一部分用作原料和劳动工具。这一点属于工艺过程,它只是表明,工人为了把他们的产品变成生产资料,他们在工业生产中应当怎样对待他们自己的劳动产品,怎样对待他们自己的产品。

第二,过去劳动表现为价值。这一点只是表明工人的新产品的价值不只是代表他们的现在劳动,而且代表他们的过去劳动,表明工人以自己的劳动扩大旧价值,同时正因为他们扩大了旧价值,于是就保存了旧价值。

资本家的要求同这一过程本身没有任何关系。当然,既然资本家占有劳动产品,占有过去劳动的产品,他就因此拥有占有新产品和活劳动的手段。但这正好是引起抗议的行动方式。“分工”所必需的预先的积聚和积累恰恰不一定表现为资本的积累。从它们是必需的这一点出发,决不能得出结论,说资本家必须支配那些由昨天的劳动为今天的劳动创造的条件。如果资本的积累[根据政治经济学家们的意见]无非就是劳动的积累,那末这决不包含它必须是别人劳动的积累这样一种意思。

但是霍吉斯金没有走这条简单的道路,初看起来这是很奇怪的。在反对资本的生产性(首先反对流动资本的生产性,但是更反对固定资本的生产性)的论战中,他好象是在反对或者否定过去劳动本身或它的产品作为新劳动的条件对再生产的重要性,也就是反对或者否定过去的、物化在产品中的劳动对于作为当前正在进行的活动的劳动的重要性。这样的转变是怎样引起的呢?

因为政治经济学家们把过去劳动同资本等同起来——过去劳动在这里既从具体的、物化在产品中的劳动的意义上来理解,也从社会劳动,即物化劳动时间的意义上来理解,——所以很明显,他们作为资本的品得\authornote{歌颂者、赞美者(品得是古希腊诗人)。——编者注},当然会把生产的物的要素提到首位,并且同主观要素即活的、直接的劳动相比,过高地估计物的要素的意义。在他们看来,只有当劳动成为资本,当它和自身相对立,当它的被动的一面和它的能动的一面相对立的时候,它才是适合的。因此,产品支配生产者,物支配主体,已实现的劳动支配正在实现的劳动,等等。在所有这些见解当中,过去劳动不是仅仅表现为活劳动的物的因素,从属于活劳动的物的因素,而是相反;不是表现为活劳动的权力要素,而是表现为支配这种劳动的权力。为了也从工艺上为特殊的社会形式即资本主义形式(在这种形式中,劳动和劳动条件的相互关系被颠倒了,以致不是工人使用这些条件,而是劳动条件使用工人)辩护,政治经济学家们赋予劳动的物的因素以一种和劳动本身相对立的虚假的重要性。正因为这样,霍吉斯金才相反地坚持认为,这种物的因素——从而一切物化财富——同活的生产过程比较起来,是极不重要的,它实际上只是作为活的生产过程的因素才具有价值,而它本身是没有任何价值的。这里,霍吉斯金有点低估过去劳动对现在劳动的意义,不过这一点在反对政治经济学家们的拜物教时是很自然的。

如果在资本主义生产中,从而在它的理论表现上,即在政治经济学上,过去劳动只表现为劳动本身给劳动创造的基础等等,那末这种争论便不可能发生。争论之所以存在,只是因为在资本主义生产的现实生活中,以及在它的理论中,物化劳动表现为同劳动本身的对立,同活劳动的对立。正象在受宗教束缚的思维过程中,思维的产品不仅要求支配思维本身,而且实现了这种支配一样。

[865]因此,霍吉斯金的命题

\begin{quote}{“那些被认为由名叫流动资本的商品储备产生的结果,是由并存劳动引起的”(第9页)}\end{quote}

其意思首先是说:

活劳动的同时并存,引起了大部分被认为由名叫流动资本的过去劳动产品产生的结果。

例如,流动资本的一部分是由生活资料的储备构成的,资本家积累这些生活资料,照政治经济学家们的说法,是为了在工人劳动时维持工人的生活。

因为在资本主义生产的条件下,生产和消费都最大,所以在市场上(在流通领域)商品量也最大,虽然如此,储备的形成却根本不是资本主义生产的特点。在资本家积累生活资料的储备这一见解中,仍然流露出对货币贮藏者所实现的积累即贮藏的回忆。

这里首先应当把消费基金撇开,因为这里谈的是资本和工业生产。一切属于个人消费范围的东西,无论它消费得较快还是较慢,都不再成为资本{虽然其中一部分可能再转化为资本,例如房屋、停车场、容器等等}。

\begin{quote}{“当时,欧洲所有的资本家是否都拥有供给他们所雇用的全部工人一个礼拜的食物和衣服呢?让我们首先来考察食物问题。人民的一部分食物是面包,它经常只是在食用以前几小时才烤出来……面包业主的产品不能贮藏。做面包的原料,无论是小麦还是面粉,没有不断的劳动就根本不能保存……纺纱工人确信在需要面包的时候就能得到面包,他的雇主确信他付给工人的钱能使工人买到面包,这些都不过是由以下事实产生的:在需要面包的时候,总是可以得到面包。”(第10页)“工人的另一种食物是牛奶,而牛奶的生产……一天两次。如果说乳牛已经有了,那末对于这一点应当这样来回答:它需要经常的照料和经常的劳动,它的饲料在一年的大部分时间里都是饲料作物每天生长的结果。它放牧的田野需要人手……肉类的情况也是一样。肉类不能贮藏,因为肉类刚一上市,就已经开始要坏。”(第10页)甚至拿衣服来说,由于怕虫蛀,“衣服的储备,同衣服的总消费比较起来,只是一个很小的数量”。(第11页)“穆勒说得对:‘一年内生产出来的东西一年内就被消费掉’,所以实际上不能积累起使人们能够完成持续一年以上的全部工作所需的商品储备。因此,从事这些工作的人不应当指望已经生产出来的商品,而应当指望由其他人劳动和生产出他们在完成自己产品的劳动期间为自己生存所必需的东西。所以,即使工人同意,为了在一年内完成的工作,必须积累一些流动资本……那末很明显,在进行持续一年以上的全部工作的过程中,工人不指望也不可能指望积累的资本。”(第12页)“如果我们适当地注意到那些创造财富的、不能在一年内完成的工作的数量和重要性,也注意到维持生存所必需的、无数的、每天劳动的产品,而这些产品在生产出来以后又立即被消费掉,那末,我们就会懂得,每一项不同种类劳动的成效和生产力取决于其他人的并存生产劳动的程度,总是比取决于流动资本的任何积累的程度大。”(第13页)“资本家能够养活,并因而雇用其他劳动者,不是由于他拥有商品储备,而是由于他有支配一些人的劳动的权力。”(第14页)“可以说储存起来和预先准备好的唯一的东西,就是工人的技能。”(第12页)“通常被认为是由流动资本的积累产生的一切结果,都是由于熟练劳动的积累和储存,这种最重要的工作,对大部分工人来说,不要任何流动资本也可以完成。”(第13页)“工人人数总是必须取决于流动资本的量,或者,照我的说法,取决于允许工人消费的并存劳动的产品的量。”(第20页)[866]“流动资本……只是为了消费才创造出来;固定资本……不是为了消费,却是为了帮助工人生产消费的物品而生产出来。”(第19页)}\end{quote}

所以,我们首先指出:

\begin{quote}{“每一项不同种类劳动的成效和生产力取决于其他人的并存生产劳动的程度,总是比取决于流动资本的任何积累(即“已经生产出来的商品”)的程度大。”这些“已经生产出来的商品”是和“并存劳动的产品”对立的。}\end{quote}

{在每一单个的生产部门内部,资本中归结为劳动工具和劳动材料的部分总是作为“已经生产出来的商品”而成为前提。不能纺还没有“生产出来的”棉花,不能使尚待制造的纱锭转动,不能烧还未从矿井里开采出来的煤。因此,它们总是作为过去劳动的存在形式加入[生产]过程。在这个意义上,现存的劳动取决于以前的劳动,而不只是取决于并存劳动,尽管这种以前的劳动,无论以劳动资料还是劳动材料的形式出现,总是只有作为活劳动的物的因素(仅仅作为生产消费即劳动消费的因素)同活劳动相接触,才具有某种用处(生产上的用处)。

但是在考察流通和再生产过程时,我们同时还看到,商品被制造出来并转化为货币以后,它之所以能再生产出来,只是因为它的一切要素被“并存劳动”同时生产和再生产出来\endnote{马克思在《剩余价值理论》的前几章批判地分析亚·斯密和大·李嘉图的观点时谈到了再生产过程的基本要素。特别提到某种商品的一切要素必须同时生产和再生产出来,(见本卷第1册第96—98和136—137页以及第2册第538—539、552和553页)。——第307页。}。

在生产中有两种运动。我们拿棉花作为例子。它从一个生产阶段转到另一个生产阶段。最初它作为子棉生产出来,然后经过许多道工序,直到适合出口,或者,如果是在本国进一步加工,它就要直接转到纺纱者手里。然后,它从纺纱者手里转到织布者手里,从织布者手里转到漂白者、染色者、整理者手里,从他们手里又转到各种各样为了专门目的而把它加工为衣服、床单等等的工厂。最后,如果不是作为劳动资料(不是材料)进入生产消费,它便从最后的生产者手里转到消费者手里,即转为个人消费。但是这样一来,无论是为了生产消费还是个人消费,棉花都取得了它的使用价值的最终形式。在这里作为产品从一个生产领域出来的东西,又作为生产条件进入另一个生产领域,这样经过连续的阶段,直到最后制成为使用价值。在这里过去劳动不断表现为现在正在进行的劳动的条件。

但是在产品这样地从一个阶段转到另一个阶段,在它完成这一现实的形态变化的同时,它又在每一个阶段上被生产出来。当织布者在加工纱,纺纱者在纺棉花的时候,新的子棉又处在自己的生产过程当中。

因为不断的、重新开始的生产过程就是再生产过程,所以它同样是由并存劳动决定的,当产品完成自己的形态变化,从一个阶段转到另一个阶段的时候,并存劳动就同时生产出产品的不同阶段。棉花、棉纱和布——所有这一切不只是一个在一个之后,一个由另一个生产出来,而且也是同时并行地生产出来和再生产出来。当我在考察单个商品的生产过程时,表现为以前的劳动的结果的东西,在我考察该商品的再生产过程时,也就是说,当我从该商品的不断进行的生产过程,从这个生产过程的条件的总和,而不只是从一个孤立的行为或有限的空间来考察该商品的生产过程时,就同时表现为并存劳动的结果。这不只是经过不同阶段的循环,而且是商品在其属于特殊生产领域和形成不同劳动部门的一切阶段上的并行生产。如果同一个农民先种亚麻,然后把它纺成纱,再把它织成布,那末这些工序就有连续性,但是没有同时性,而同时性则以建立在社会内部分工基础上的生产方式为前提。

如果从单个商品的生产过程中的某一阶段来考察单个商品的生产过程,那末以前的劳动,固然,只是由于它为之提供生产条件的活劳动才具有意义。但是另一方面,这些生产条件(没有它们,活劳动就不能实现)总是作为以前的劳动的已完成的结果加入这一过程。因此,提供生产条件的那些劳动部门的协作劳动总是表现为被动的,并且作为这种被动的因素而成为前提。政治经济学家们都强调这一方面。相反,在再生产和流通中,每一个特殊领域的商品生产过程所依靠的和作为其先决条件的社会中介劳动,则表现为现在的、并存的、同时的劳动。商品以它的最初形式和它的已完成形式或连续形式同时生产出来。没有这一点,商品在完成了它的现实的形态变化以后,就不能从货币再转化为它的生存条件。[870b]因此,商品只有同时表现为同时的活劳动的产品,它才是以前的劳动的产品。从这个意义上说,资本家所认定的全部物质财富只是包括流通过程在内的总生产源流中的一种迅速消逝的因素。}

\tsectionnonum{[(c)]所谓积累不过是一种流通现象(储备等是流通的蓄水池)}

霍吉斯金只是从流动资本的一个组成部分来考察流动资本。但是一部分流动资本会不断转化为固定资本和辅助材料,只有另一部分才转化为消费品。而且,即使那部分最终转化为供个人消费的商品的流动资本,除了它作为从终结阶段出来的最终产品所具有的最后形式外,在它较早的各个生产阶段,也一直同时以还不能进入消费的最初形式存在,也就是以在不同程度上有别于产品最终形式的原料或半成品的形式存在。

霍吉斯金所谈的问题是,工人现在给资本家提供的劳动与由工资转化成的物品(这些物品实际上就是构成可变资本的使用价值)所包含的劳动之间有什么关系。必须承认,如果没有这些供消费的物品,工人就无法劳动。所以政治经济学家们说,流动资本——过去劳动,资本家积累的已经生产出来的商品——是劳动的条件,其中也包括分工的条件。

谈到生产条件,特别是谈到霍吉斯金所说的流动资本,通常是说,在工人生产出新的商品之前,也就是在工人劳动期间,在工人自己生产的商品还处在形成状态的时候,资本家就应当积累起工人消费所需的生活资料。这里透露出一种看法,即认为资本家就象货币贮藏者那样从事积累,或者说,他就象蜜蜂采蜜那样收集生活资料的储备。

但是这只不过是一种说法而已。

首先,我们这里谈的不是做生活资料买卖的零售商。他们当然经常要有充足的商品储备。他们的栈房、店铺等只不过是蓄水池,商品在可以进入流通之后就分配在这里。这种积累不过是商品从流通转入消费之前所处的中间阶段。这是商品作为商品在市场上的存在。其实,它作为商品也只有以这种形式存在。至于它是不是已经不在第一个卖者(生产者)手里,而是在第三个或第四个卖者手里,它是不是最终转入把它卖给真正消费者的卖者手里,这对问题毫无影响。这只关系到:在中间阶段商品代表着资本(其实是资本加利润,因为生产者在商品中出卖的不仅是资本,而且还有他的资本所赚得的利润)同资本的交换,在最后阶段商品代表着资本同收入的交换(就是说,如果商品象在这里假设的那样预定不转入生产消费,而转入个人消费)。

已经最后成为使用价值并已进入可以出卖状态的商品,作为商品处于市场,处于流通阶段;一切商品,当它们必须完成它们的第一形态变化,即转化为货币时,都处于这个阶段。如果这叫作“积累”,那末积累就无非是商品作为商品的“流通”或存在。因此,这种“积累”就会同货币贮藏正好相反,因为货币贮藏是要使商品永远保持在这种可以流通的状态,而这也只有以货币的形式把商品从流通中抽出来才能办到。如果生产,从而还有消费,都是多种多样和大规模的,那末就会有大量的各种各样的商品经常处于这种停顿状态,即处于这种中间阶段,一句话,处于流通中,或者说,处于市场上。所以,如果从量的方面来考察,那末,大量的积累在这里无非是指大量的生产和大量的消费。

商品的停顿——商品停留在过程的这一时刻,它存在于市场而不存在于工厂或私人家里(作为消费品),即存在于商人的店铺、栈房中——只是[871]它生命过程中的一个很短暂的时刻。这种“财物世界”,“实物世界”的静止的、独立的存在只是一种表面现象。驿站始终客满,但始终都是新的旅客。同样的商品(同一种商品)不断地在生产领域中重新生产出来,出现在市场上并被消费掉。它们,不是同一些商品,而是同一种商品,始终同时存在于这三个阶段上。如果中间阶段延长,以致新商品从生产领域出来时,市场还是被旧商品占据着,那末就会产生停滞,阻塞;出现市场商品充斥,商品贬值;出现生产过剩。所以,流通的中间阶段在什么地方成为一种独立的存在,而不只是向前运动的源流中一个短暂的停留,以及商品在流通阶段的存在在什么地方表现为积累[Aufhaufung],这绝不是生产者的一种自由行动,绝不是生产的目的或者生产的内在的生命因素,正如血液涌向头部引起中风并不是血液循环的内在因素一样。资本作为商品资本(在这个流通阶段,在市场上,它就是以这种形式出现的)不应该停滞不动,而应该只是在运动进程中作短暂的停留。否则再生产过程就会遭到破坏。整个机构就会紊乱。所以,这种在个别点上以集中形式出现的物质财富同生产和消费的持续不断的源流相比,是微不足道的,也只能是微不足道的。因此,斯密也认为,财富是“年度的”再生产。所以,它所注明的日期不是什么遥远的过去,而只不过是昨天。另一方面,如果再生产由于受到某些干扰而停顿下来,那末仓库等等就会空起来,就会出现匮乏,就立刻会显示出:现存财富看起来所具有的那种经常性不过是它的更替、它的再生产的经常性,是社会劳动的不断的物化。

在商人那里也存在着W—G—W的过程。商人从中获取“利润”这一点,在这里和我们没有关系。他出卖商品,又购买同样的商品(同一种商品)。他把商品卖给消费者,又从生产者那里把商品买进来。同样的商品(同一种商品)在这里不断地转化为货币,货币又不断地再转化为同样的商品。但是这种运动只不过是不断的再生产,即不断的生产和消费;因为再生产包含着消费。(为了能够进行商品的再生产,商品就必须卖掉,必须加入消费。)商品必须用事实证明自己是使用价值。(因为对于卖者来说是W—G,对于买者来说就是G—W,也就是货币转化为作为使用价值的商品。)再生产过程既然是流通和生产的统一,它就包含着本身是流通因素的消费。消费本身就是再生产过程的因素和条件。如果就整个过程来考察,商人向生产者购买商品所支付的货币,实际上是消费者向商人购买商品所用的货币。对于生产者来说,商人代表消费者,而对于消费者来说,商人就代表生产者;他是同一商品的买者和卖者。他用来购买商品的货币,纯粹从形式上看,实际上就是消费者的商品的终结形态变化。消费者把他的货币转化为作为使用价值的商品。所以,货币转入商人之手就意味着商品的消费,或者从形式上看,意味着商品从流通转入消费。只要商人再用这些货币向生产者购买,这就是生产者的商品的第一形态变化,表示商品转入中间阶段,在这个阶段它作为商品停留在流通中。只要W—G—W这个过程是商品转化为消费者的货币,并且是现在为商人所有的货币再转化为同样的商品(同一种商品),那末这一过程就无非表示商品不断地转入消费,因为进入消费的商品所空出的位置为此就必须由从生产过程出来现在进入这一中间阶段的商品所填补。

[872]商品在流通中停留以及它被新商品所取代,当然还要取决于商品处在生产领域的时间的长度,因而取决于商品再生产时间的长度,随着这种时间长度的不同,商品停留的时间也不同。例如,谷物的再生产需要一年时间。例如,今年(1862年)秋季收获的谷物,只要不再用作种子,就必须足够供来年全年——直到1863年秋——的消费。它立即被投入流通(即使在农场主的粮仓里,它也是已经处于流通中了),在这里它被流通的各种蓄水池——仓库、谷物商、磨坊主等等——所吸收。这些蓄水池既是生产的排水渠,又是消费的引水渠。只要商品处于蓄水池中,它就是商品,因而就处于市场上,处于流通中。它只是点点滴滴地被年消费从流通中抽出。把它排挤出去的新商品所进行的补充,新商品的源流,只有在一年以后才会到来。因此,这些蓄水池也只是随着对已消费的商品的补充的到来而逐渐地变空。如果还有剩余,如果新的收成超过平均收成,那末就会发生阻塞。这种一定的商品在市场上占有的空间就会显得充斥。为了都能在市场上给自己找到位置,商品就会降低自己的市场价格,这样就会使它们重新运动起来。如果商品作为使用价值的量太大,那末它们就会通过降低自己的价格的办法来适应它们应占的空间。如果这个量太小,那末它们就会用提高自己的价格的办法来扩大自己。

另一方面,作为使用价值会迅速坏掉的那些商品,在流通的蓄水池中也只有瞬息间的停留。它们必须转化为货币和必须被再生产出来的时间,是由它们的使用价值的性质所规定的,这种使用价值如果不是每天或几乎每天被消费掉,就会坏掉,因而也就不再是商品。因为如果使用价值的消失本身不是生产行为,交换价值就会和它的承担者即使用价值一起消失。

一般说来很清楚,虽然聚集在流通蓄水池中的商品的绝对量会随着国民经济的发展而增长,但是由于生产和消费的增长,这个量同年生产和年消费的总量相比,还是会减少。商品从流通到消费的转移会加速,而且是由于一系列的原因。再生产的速度在下列场合会加快:

(1)商品迅速地通过它的各个生产阶段,生产过程在每个生产阶段缩短;这取决于商品在它的每一种形式上的生产所必需的劳动时间的缩短;所以,这是和分工、机器、化学过程的应用等等的发展同时发生的。{随着化学的发展,人为地加速了商品从一种聚集状态到另一种聚集状态的转变,加速了它和其他物体的结合,例如染色;加速了它和其他物质的分离,例如漂白,——一句话,无论是同一些物质的形式(它们的聚集状态)的变化,还是必然产生的物质变换,都人为地加速了;至于会给植物和动物提供较便宜的物质,即花费很少劳动时间的物质,以进行植物性的和有机的再生产,那就更不用说了。}

(2)部分地由于不同生产部门的联合,即由于形成了把一定生产部门联合起来的生产中心,[部分地]由于交通工具的发展,商品迅速地从一个生产阶段转到另一个生产阶段;换句话说,缩短了间歇期间,减少了商品在一个生产阶段和另一个生产阶段之间的中间阶段的停留时间,或者说,缩短了从一个生产阶段到另一个生产阶段的转移。

(3)所有这些发展——各个不同生产阶段的缩短以及从一个阶段到另一个阶段的转移的加快——都是以大规模生产,大量生产为前提,同时也以大量不变资本,特别是固定资本基础上的生产为前提;因而也以生产的不断进行为前提,所谓不断,不是指我们刚才考察这种不断进行时所说的不断,即不是指通过各个生产阶段的彼此接近和相互渗透而形成的不断,而是指在生产中不会发生有意的中断。这种中断在为订货而生产的情况下总是会发生的,就象在[873]手工业者那里出现的那样,在本来意义上的工场手工业中(只要工场手工业本身尚未被大工业改造)也还是那样。而现在生产是按资本所容许的规模进行的。这个过程并不等待需求,而是资本的一种职能。资本不断以同样的规模(且不说积累或扩大)进行工作,同时生产力不断发展和提高。因此,生产不仅进行得很快,使得商品很快就获得适合于流通的形式,而且是不断地进行。生产在这里仅仅表现为不断的再生产,同时也是大量的生产。

因此,如果商品长期滞留在流通的蓄水池中,如果商品积存在这里,那末,由于生产浪潮迅速地一个接着一个涌来,由于它们不断注入流通蓄水池大量材料,这些蓄水池很快就会充斥。例如,柯贝特正是在这个意义上说:“市场总是商品充斥。”\endnote{柯贝特关于市场总是商品充斥和关于供给总是超过需求的看法,在他的著作《个人致富的原因和方法的研究;或贸易和投机原理的解释》1841年伦敦版第115—117页上作了说明。——第315页。}可是造成再生产这样迅速、这样大量的这些情况,也会减少商品在这些蓄水池中聚集的必要性。就生产消费来说,这种情况已部分地包含在商品本身或其组成部分所必须通过的各个生产阶段的彼此接近中。如果煤炭每天大量生产,并且经由铁路、轮船等运送到工厂主的大门口,那末工厂主就不需要储备煤,或者只需要储备少量的煤,或者,如果有一个商人介入其中,情况也是一样,这个商人除了每天卖出和每天得到补给的以外,也只需要有很少的储备。纱、铁等的情况也是如此。可是,把生产消费(在生产消费领域中,商品储备,即商品各组成部分的储备,必然会这样减少)撇开不谈,[经营个人消费品的]商人也同样有:第一,迅速的交通工具,第二,可靠的、不断的、迅速的更新和供给。因此,虽然他的商品储备在数量上可能增加,但是这种储备的每一个要素存在于他的蓄水池,即存在于这种过渡状态的时间会缩短。同他出卖的全部商品量相比,也就是同生产量和消费量相比,他的仓库在每个一定时刻所保存的、聚集的商品储备是不大的。在生产比较不发达的阶段,情况就不一样了,在这些阶段再生产进行缓慢,——因而必然有较多的商品滞留在流通的蓄水池中,——交通工具缓慢,联络困难,因此储备的更新往往发生中断,从蓄水池变空到它重新装满,即商品储备的更新,这中间要经过很长的间歇期间。这时就会发生和下述产品类似的情况:这些产品由于其使用价值的性质,它们的再生产要经过一年或半年,总之,要经过比较长的期间才能实现。

{交通工具对于蓄水池变空所产生的影响,可以棉花为例来说明。由于利物浦和美国之间经常有船舶来往,——交通的迅速是一个因素,经常性是另一个因素,——所以用不着把全部棉花一下子运出去。棉花可以逐渐上市。(生产者也不希望商品一下子充斥市场。)棉花存在利物浦货栈内,诚然已经是在流通的蓄水池中,但是其数量——同这种商品的总消费量相比——已不象在船舶要经半年的航程、一年只从美国开来一两次时所需要的那样多了。曼彻斯特的工厂主等可以大致根据他直接消费的多少来充实他的仓库,因为有了电报和铁路,就有可能随时把棉花从利物浦运到曼彻斯特。}

流通蓄水池的特殊的充满现象(不是由于市场负担过重造成的充满,市场负担过重在这种情况下发生,要比在宗法式的生产速度缓慢的情况下容易得多)只是投机性的,只有在与价格的实际涨落或预料中的涨落有关的例外场合才会发生。

关于储备的这种相对减少,即处在流通中的商品量同生产和消费的总量相比而言的相对减少,见莱勒的著作、《经济学家》\endnote{《经济学家》(《TheEconomist》)是英国经济、政治问题周刊,1843年起在伦敦出版,大工业资产阶级的机关刊物。——第316页。}、柯贝特的著作(有关的引文放在霍吉斯金之后)。[874]西斯蒙第错误地认为这是值得遗憾的事(也请参阅他的著作)\endnote{西斯蒙第在他的《政治经济学概论》1837年布鲁塞尔版第一卷第49页及以下各页谈到随着贸易和交通工具的发展现有商品储备减少的问题。——第316页。}。

(诚然,另一方面,我们也会看到市场的不断扩大,随着商品在市场停留的间歇期间的缩短,空间的范围相应扩大,或者说,市场在空间上相应扩大,以商品生产领域为中心画出的圆的半径越来越大。)

“挣多少吃多少”的消费者,改变衣着就象改变意见一样迅速,而不是一件上衣等等一穿就是十年,这种情况和再生产的速度有关,或者说,不过是再生产速度的另一种表现。甚至那些不受使用价值的性质制约的物品的消费,也越来越在时间上和生产趋于一致,因而也越来越依附于现在劳动,并存劳动(因为实际上这里是并存劳动的交换),这一切都是同过去劳动越来越成为生产的重要因素的程度相适应的,虽然这种过去本身总是很近的,而且只是相对的。

(下面一个例子说明储备的建立同生产的不发展是多么紧密相联。在牲畜很难过冬的时候,冬天就没有鲜肉。一旦畜牧业克服了这一困难,由于必须以腌肉或熏肉代替鲜肉而产生的储备也就会自行停止。)

产品只有在它进入流通的场合,才成为商品。产品作为商品的生产,因而还有流通,会由于以下原因随着资本主义的生产而异常扩大:

(1)大规模的生产,量,大批,也就是同生产者[对他自己的产品]的需要在数量上丝毫没有关系的生产;事实上,他是不是哪怕在最小的程度上消费自己的产品,这纯粹是偶然的。生产者只有在他生产自己资本的一部分构成要素时,才会大量地消费自己的产品。相反,在社会发展的较早阶段,只是——或者主要是——超过自己需要的多余的产品才成为商品。

(2)同需要的日益增加的多样性成反比的产品的质的单一性。这一点会引起以前彼此联系着的生产部门较大程度的分离和独立,一句话,会使社会内部的分工增多,此外还会引起新的生产部门的建立和商品种类的多样性的增加。(最后,在论述霍吉斯金之后,还要列举威克菲尔德对这个问题的看法。)商品的这种多样化即分化,有两类。第一,同一产品的不同阶段,以及加在产品上的中间劳动(也就是生产它的构成要素等的劳动)分化为不同的彼此独立的劳动部门;换句话说,同一产品在它的不同阶段转化为不同种类的商品。第二,由于有劳动和资本(或者说,劳动和剩余产品)游离出来,另一方面,由于发现利用同一使用价值的新方法[从而出现新的种类的商品]。由于第一点中所谈到的那些变化,于是产生新的需要(例如,随着蒸气在工业中的利用,就出现对迅速和全面的交通工具的需要),因而也产生满足这些需要的新方法,——或者是发现利用同一使用价值的新方法,或者是发现新原料,或者是发现对旧原料进行不同处理的新加工方法(如电铸术),等等。

这一切归结为一个产品在其一个接一个的阶段或者说状态中转化为不同的商品,或者归结为创造作为商品的新产品或者说新的使用价值。

(3)以前以实物形式消费\authornote{这里指自然经济条件下的消费。——编者注}大量产品的人口中的大多数转化为雇佣工人。

(4)租地农民转化为产业资本家{地租随之转化为货币地租,总之,所有的实物交纳(赋税等,地租)转化为货币支付}。总之:土地以工业方式经营,因而它的化学和机械的生产条件,甚至种子等等,牲畜等等,肥料等等都要新陈代谢,而不象从前那样只限于使用自己的粪肥。

(5)大量以前“不可让渡的”财物的变卖使它们转化为商品,仅仅由流通券构成的财产形式被创造出来。一方面是地产的让渡(在广大群众变得连任何财产都没有的情况下,也出现了他们例如把自己的住房当作商品的现象)。另一方面是铁路股票,简言之,各种各样的股票。

\tsectionnonum{[(d)霍吉斯金对资本家为工人“积累”生活资料的见解的驳斥。霍吉斯金不了解资本拜物教化的真正原因]}

[875]现在我们回过头来谈霍吉斯金。

所谓资本家为工人“积累”[生活资料],当然不是指商品从生产转入消费时存在于流通蓄水池中,存在于流通中,存在于市场上这种情况。如果这样来解释这种“积累”,那就等于说,产品是为了工人而流通,为了工人而成为商品,总之,产品作为商品的生产是为了工人而进行的。

同其他任何人[商品所有者]一样,工人必须首先把他实际上(虽然不是在形式上)出卖的商品即他的劳动转化为货币,然后才能把这些货币再转化为供消费的商品。非常明显,如果没有消费品以及生产资料作为商品存在于市场,那末,分工(既然它以商品生产为基础),雇佣劳动,总之,资本主义生产,就不可能存在;如果没有商品流通,没有商品停留在流通蓄水池中,这种生产就不可能进行。因为真正说来,产品只有在流通中才是商品。工人必须在商品形式上取得他的生活资料,这对他来说,就象对其他任何人一样。

此外,工人与经营生活资料的商人相对立不是作为工人与资本家相对立,而是作为货币与商品,作为买者与卖者相对立。这里不存在雇佣劳动与资本的关系,除非是涉及商人自己的工人。然而即使是这样的工人,只要他们是向商人购买,他们就不是作为工人与商人相对立。只有在商人向他们购买时才会发生这种情况。所以我们要把这种流通的当事人撇开。

至于工业资本家,那末构成他的储备,即他的“积累”的是:

第一,他的固定资本——建筑物、机器等等,这些东西工人是不消费的,或者,他如果消费,那是在劳动过程中为资本家生产地消费;这些东西虽然是工人的劳动资料,但绝不是工人的生活资料。

第二,他的原料和辅助材料;不直接加入生产的那部分原料和辅助材料的储备,正如我们所看到的,有减少的趋势。这些东西也不是工人的生活资料。资本家为工人进行这种“积累”,不过是表示资本家为工人效劳,从工人那里夺走他的劳动条件的所有权,并把他的这些劳动资料(这些东西本身不过是他的劳动的转化了的产品)变为剥削劳动的手段。当工人把机器和原料当作劳动资料使用时,他无论如何不是靠它们生活的。

第三,他的进入流通以前存在于仓库、货栈中的商品。这些商品是劳动的产品,而不是在生产期间为维持劳动自身而积累的生活资料。

因此,资本家为工人“积累”生活资料,不过是表示资本家必须拥有足以支付工资的货币,工人用这些货币从流通蓄水池中取得自己的消费资料(如果就整个阶级来考察,就是买回工人自己的一部分产品)。但是这些货币不过是工人所出卖和提供的那种商品的转化形式。从这个意义上说,生活资料是为工人“积累”,如同生活资料是为他的资本家积累一样,因为资本家也用货币(同一商品的转化形式)购买消费资料等等。这些货币可以是单纯的价值符号;所以它们根本不一定是“过去劳动”的代表,而只是在每个人手中表示他所实现的价格——不是过去劳动(或以前的商品)的价格,而是这个人所出卖的同时劳动或商品的价格。是单纯的形式存在\endnote{这里马克思把货币描述为“单纯的形式存在”(《blossesFormdasein》)是在如下的意义上:货币的使用价值“虽然是实际存在的,但在[交换]过程本身中却表现为单纯的形式存在,它还需要通过转化为真正的使用价值才得到实现”(见《马克思恩格斯全集》中文版第13卷第37页)。——第321页。}。或者说,因为在以前的生产方式下工人也必须吃饭,而且不管他的产品的生产时间有多长,在生产时他总得消费生活资料,所以为工人“积累”生活资料,就是指工人必须首先把自己的劳动产品转化为资本家的产品,转化为资本,然后才能以货币的形式再拿回一部分这样的产品作为报酬。

[876]在这一过程中(对于这一过程本身说来,工人得到的是同时劳动的产品还是过去劳动的产品,是并行劳动的产品还是自己以前的产品,实际上完全是无关紧要的)使霍吉斯金感兴趣的是下面一点:

工人每天消费的产品(不管他自己的产品是否已经制成,他都必须消费)的一大部分以至绝大部分决不是以前的积累劳动。相反,这在很大程度上是工人在他生产自己的商品的同一天、同一周所生产的劳动产品。面包、肉、啤酒、牛奶、报纸等等就是这样。霍吉斯金也许还会说,其中有一部分是未来劳动的产品,因为工人要用六个月内积攒的工资来购买只是在这六个月的末尾才制成的上衣等等。(我们已经看到,全部生产都以加入其中的各组成部分和表现为原料、半成品等不同形式的产品的同时再生产为前提。一切固定资本则以未来劳动作为其再生产的前提,它也要以未来劳动作为再生产自己等价物的前提,没有这个等价物,它就不能进行再生产。)霍吉斯金说,在一年之内,工人(由于谷物的再生产的性质,由于植物性原料等的生产的性质)不得不在一定程度上“指靠”过去劳动。{例如关于住房就不能这样说。有的使用价值,因其性质,只是逐渐磨损,它不是一下子被消费掉,而只是被使用,在这种情况下,以前的劳动的这种产品存在于“市场”,决不是为工人而想出的某一特别行动的结果。工人在资本家为他“积累了”脏得要命的贫民窟之前,就早已“有了住房”。(关于这一点见兰格的著作\endnote{马克思指(小)赛米尔·兰格的著作《国家的贫困,贫困的原因及其防止办法》1844年伦敦版第149—154页。马克思在《资本论》第一卷第二十三章注115中引了这本书里描写资本主义大城市中工人居住条件极为恶劣的一段话。——第322页。}。)}(且不说特别对工人具有决定意义的大量日常需要,而工人是几乎只能满足自己的日常需要的,——我们已经看到,生产和消费一般说来在时间上越来越趋于一致,所以,如果就整个社会来考察,社会全体成员的消费就越来越依赖于他们的同时生产,或者更确切地说,依赖于同时生产的产品。)但是如果劳动操作延续若干年,工人就只得“指靠”自己的生产,“指靠”生产其他商品的工人的同时劳动和未来劳动。

工人总是必须在市场上取得作为商品的生活资料(因而他所购买的这些“服务”只是在它们被购买时才被创造出来),因此,这些生活资料相对地说是以前的劳动(即在它们作为产品存在之前就存在的劳动,但决不是在工人自己的劳动——即工人用其价格购买这些产品的劳动——之前存在的劳动)的产品。这些生活资料可能是与这种劳动在时间上一致的产品,对于“挣多少吃多少”的人来说,它们在大多数场合正是这样的产品。

如果考虑到这一切,那末资本家为工人“积累”生活资料可归纳为如下几点:

(1)商品生产的前提是,人们可以在市场上取得他们自己所不生产的作为商品的消费品,或者说,商品一般作为商品被生产出来。

(2)工人消费的绝大部分商品,在其作为商品同工人相对立的最后形式上,实际上是同时劳动的产品(因此,它们根本不是由资本家积累的)。

(3)在资本主义生产的条件下,工人自己生产的劳动资料和生活资料是同工人相对立的,前者作为不变资本,后者作为可变资本同他相对立;他的所有这些生产条件都表现为资本家的财产;而这些生产条件从工人手里转到资本家手里以及工人的产品或其产品的价值部分地流回到工人手里,就叫作为工人“积累”流动资本。工人在他的产品完成之前总是必须消费的这些生活资料所以成为“流动资本”,是因为工人不是以自己过去的产品的价值或未来的[877]产品来直接购买生活资料或者进行支付,而是必须先从资本家那里得到领取生活资料的凭证,即货币;资本家只是由于工人过去生产的、将来生产的或现在生产的产品才能够发给这种凭证。

霍吉斯金在这里力图证明工人是依靠其他工人的并存劳动,而不是依靠过去劳动,

(1)以便消除“积累这个用语”,

(2)因为“现在劳动”是同资本相对立的,而“过去劳动”则一直被政治经济学家们看作就是资本,是一种异化的、同劳动本身敌对的、独立的劳动形式。

但是对同时劳动,普遍地从它与过去劳动相对立的意义上来理解它,这本身就是一个十分重要的因素。

所以,霍吉斯金得出如下的结论:

资本或者仅仅是一种名称和托词,或者它表现的不是物,而是关系:一个人的劳动同其他人的并存劳动的社会关系,这种关系的后果,结果,被认为是由构成所谓流动资本的物造成的。商品在作为货币的一切存在上能否实现为使用价值,取决于同时劳动。(全年的[劳动]本身就是同时的[劳动]。)只有一小部分加入直接消费的商品是一年以上的产品,即使它们是这样的产品(例如牲畜等等),它们每年也需要新的劳动。所有需要一年以上时间的劳动操作都是建立在继续不断的年生产的基础上。

\begin{quote}{“资本家能够养活,并因而雇用其他劳动者,不是由于他拥有商品储备,而是由于他有支配一些人的劳动的权力。”(第14页)}\end{quote}

然而是货币给每个人以“权力”,去支配“一些人的劳动”,支配已经物化在他们的商品中的劳动,以及支配这种劳动的再生产——在这个限度内也就是支配劳动本身。

在霍吉斯金看来,真正“积累”起来的,但不是作为死的物质,而是作为活的东西“积累”起来的,是工人的技能,是劳动的发展程度。{诚然(霍吉斯金没有强调这一点,因为和政治经济学家们的粗陋见解相反,对他来说重要的是把重点放到与物相对立的主体上,也可以说放到主体中的主观方面),每一特定时刻所具有的、作为出发点的劳动生产力发展程度,不仅以工人的技能和能力的形式存在,而且同时存在于这种劳动为自己创造的、并且每天都在更新的物质工具之中。}这是形成出发点的真正的前提,而且这个前提是一定发展进程的结果。积累在这里就是把已承受下来的、被实现了的东西加以同化、继续保存并进行改造。正是在这个意义上,达尔文把通过一切有机体即植物和动物的遗传而进行的“积累”看作促使有机体形成的动因;这样,不同的有机体本身就是通过“积累”而形成,并且只是活的主体的“发明”,是活的主体的逐渐积累起来的发明。但是对生产来说,这并不是唯一的前提。对动物和植物来说,这种前提就是它们外部的自然界,——因而既包括无机的自然界,也包括它们同其他动植物的关系。在社会上从事生产的人,也同样遇到一个已经发生变化的自然界(特别是已经转化为他自己活动的工具的自然要素)以及生产者彼此间的一定关系。这种积累一部分是历史过程的结果,一部分就单个工人来说是技能的代代相传。霍吉斯金说,在这种积累的情况下,任何流动资本都不会对大多数工人有什么帮助。

霍吉斯金指出,“商品〈生活资料〉储备”同总消费和生产比较起来单是不大的。而现有人口的熟练程度却始终都是总生产的前提,因而是财富的主要积累,是以前劳动的被保存下来的最重要的结果,不过这种结果是存在于活劳动本身中的。

\begin{quote}{[878]“通常被认为是由流动资本的积累产生的一切结果,都是由于熟练劳动的积累和储存,这种最重要的工作,对大部分工人来说,不要任何流动资本也可以完成。”(第13页)}\end{quote}

政治经济学家们说,工人人数(从而现有工人人口的幸福或贫困)取决于现有的流动资本量,对于这种说法霍吉斯金正确地作了如下的评论:

\begin{quote}{“工人人数总是必须取决于流动资本的量,或者,照我的说法,取决于允许工人消费的并存劳动的产品的量。”(第20页)}\end{quote}

被认为由“流动资本”、由某种“商品储备”造成的东西,是“并存劳动”的结果。

所以,霍吉斯金用另外的话说:劳动的一定社会形式的作用被认为是由物,由这一劳动的产品造成的;关系本身被幻想为物的形式。我们已经看到,这是以商品生产,以交换价值为基础的劳动所固有的特点,这种混淆表现在商品上和货币上(霍吉斯金没有看到这一点),而且更多地表现在资本上。\endnote{关于商品、货币和资本的拜物教性质,马克思在《政治经济学批判》第一分册(见《马克思恩格斯全集》中文版第13卷第21—25、37—39和144—146页)中谈到过。——第326页。}物作为劳动过程的物的因素所产生的作用,被认为是由这些物在资本中造成的,就象这些物在自己的人格化中,在和劳动对立的自己的独立性中所具有的作用一样。假如它们不再以这种异化的形式和劳动相对立,它们[在政治经济学家们看来]就不再能够产生这种作用。资本家作为资本家只不过是资本的人格化,是具有自己的意志、个性并与劳动敌对的劳动产物。霍吉斯金认为这纯粹是主观的幻想,在这种幻想后面隐藏着剥削阶级的欺诈和利益。他没有看到这种表述方法是怎样从现实关系本身中产生的,没有看到后者不是前者的表现,而是相反。英国的社会主义者就是在这个意义上说:“我们需要的是资本,而不是资本家”。\endnote{大约在这些关于霍吉斯金的论述以前半年,马克思在未写完的关于英国社会主义者布雷的一节中顺便引用了布雷这样几句话:“对生产者的操作具有重大意义的不是资本家,而是资本。资本和资本家之间的区别就象船上装的货物和提货单之间的区别一样大。”(见本册第356页)。——第326页。}但是如果他们排除了资本家,他们也就使劳动条件丧失了资本性质。

\centerbox{※     ※     ※}

{《评政治经济学上若干用语的争论》一书的作者、贝利和其他人指出\authornote{见本册第138—139和176页。——编者注},“value,valeur”\authornote{价值。——编者注}这两个词表示物的一种属性。的确,它们最初无非是表示物对于人的使用价值,表示物的对人有用或使人愉快等等的属性。事实上,“value,valeur,Wert”\authornote{价值。——编者注}这些词在词源学上不可能有其他的来源。使用价值表示物和人之间的自然关系,实际上是表示物为人而存在。交换价值则代表由于创造交换价值的社会发展后来被加在Wert(=使用价值)这个词上的意义。这是物的社会存在。

\begin{quote}{“梵文Wer的意思是‘掩盖、保护’,由此有‘尊敬、敬仰’和‘喜爱、珍爱’的意思。从这个词派生的形容词Wertas是‘优秀的,可敬的’意思;哥特文wairth,古德文wert,盎格鲁撒克逊文weorth,vordh,wurth,英文worth,worthy,荷兰文waard,waardig,德文wert,立陶宛文wertas(“可敬的,有价值的,贵重的,受器重的”)。梵文Wertis,拉丁文virtus\authornote{力量,优点,优秀的品质。——编者注},哥特文wairthi,德文Wert。”[夏韦《试论哲学词源学》1844年布鲁塞尔版第176页]}\end{quote}

物的Wert\authornote{价值。——编者注}事实上是它自己的virtus\authornote{力量,优点,优秀的品质。——编者注},而它的交换价值却和它的物的属性完全无关。

\begin{quote}{“梵文Wal的意思是‘掩盖,加固’;[拉丁文]vallo\authornote{用堤围住,加固,保护。——编者注},valeo\authornote{成为有力的,坚固的,健康的。——编者注};val-lus\authornote{堤。——编者注}——起掩护和保护作用的东西;valor——是力量本身。”由此有[法文]valeur,[英文]value;“请把Wal同德文walle,walte\authornote{我支配,我照料,我管理。——编者注},英文wall\authornote{墙。--编者注},wield\authornote{掌握,拥有。--编者注}作一比较。”\endnote{马克思在1864年6月16的信里告诉恩格斯说,这些不同的印欧语词的对照,是从“一个比利时词源学家”那里引来的,而从这封信中可以看出,马克思自己不相信这些对照都有充分根据。“一个比利时词源学家”就是《试论哲学词源学》(1844年布鲁塞尔版)一书的作者奥诺莱·约瑟夫·夏韦。在引自夏韦著作的第二段引文中法文“valeur”和英文“value”是马克思自己加上去的。——第327页。}[夏韦《试论哲学词源学》1844年布鲁塞尔版第70页]}}\end{quote}

\centerbox{※     ※     ※}

接着霍吉斯金转到固定资本。这是被生产出来的生产力,是在大工业里,在这种资本的发展过程中,由社会劳动为自己创造的工具。

下面是关于固定资本的一段话:

\begin{quote}{“所有的工具和机器都是劳动产品……当它们只是过去劳动的结果而不由工人加以适当使用时,它们就不能补偿制造它们的费用……如果它们闲置不用,其中大部分就会失去价值……固定资本之所以有用不是由于过去劳动,而是由于现在劳动,它给自己的所有者提供利润不是因为它被积累,而是因为它是获得对劳动的支配权的手段。”(第14—15页)}\end{quote}

这里终于正确地抓住了资本的性质。

\begin{quote}{[879]“各种工具制成以后,它们本身能生产什么呢?什么也不生产。相反,如果它们不由劳动利用或使用,它们就会开始生锈和毁坏……是否应当把某一工具看成是生产资本,这完全要看它是否被某个生产工人所使用。”(第15—16页)“很容易理解,为什么……道路修建者应当得到一部分只有道路使用者才能从道路得到的利益;但是我不理解,为什么所有这些利益都应当属于道路本身,并且由那些既不修建道路也不使用道路的人以他们的资本的利润为名据为己有。”(第16页)“蒸汽机的巨大效用并不是取决于铁和木料的积累,而是取决于对自然力的实际的活的知识,这种知识使一些人能够制造机器,使另一些人能够操纵机器。”(第17页)“没有知识,它们〈机器〉就不可能发明,没有机器制造工人的灵巧和技能,它们就不可能制造出来,而没有技能和劳动,它们就不能在生产上使用。但是知识、技能和劳动却是资本家能够据以要求获得产品的一个份额的唯一因素。”(第18页)“当人们把若干代人的知识继承下来并且大群地生活在一起时,他们就有可能用他们的智力来完成自然界所做的事情。”(第18页)“一个国家的生产劳动不是取决于固定资本的量,而是取决于固定资本的质。”(第19页)“作为供养和维持人的生活的手段的固定资本,在其效率方面完全取决于工人的熟练程度,因此一个国家的生产劳动,就固定资本来说,是和人民的知识和技能成比例的。”(第20页)}\end{quote}

\tsectionnonum{[(e)]复利;根据复利说明利润率的下降}

\begin{quote}{“只要略微看一看,任何人都会相信,随着社会的发展,简单利润不会减少,只会增加,也就是说,同量劳动,前一时期生产100夸特小麦和100台蒸汽机,现在会生产更多一些……实际上我们看到,在我们国内现在靠利润过富裕生活的人比过去多得多。然而很清楚,任何劳动,任何生产力,任何发明才能,任何技术,都不能满足复利的压倒一切的要求。但一切积蓄都是从资本家的收入中来的〈也就是从“简单利润”中来的〉,因此,这些要求实际上不断地提出,而劳动生产力同样不断地拒绝满足它们。因此,不断有一种平衡创造出来。\endnote{在霍吉斯金的小册子里,紧接这段话的一个句子说明,霍吉斯金在这里说的“有一种平衡创造出来”,指的是:“资本家允许工人有生存资料,因为他们没有工人的劳动不行,而且他们宽宏大量地满足于占有产品中不是为实现这一目的”(即保证体力的最低工资)“所必需的每一个细小部分”。——第329页。}”(第23页)}\end{quote}

例如,如果利润不断重新积累起来,资本100,按10%计算,过20年后就是约673,因为小的差数在这里没有什么意义,所以我们也可以说是700。这样一来,资本在20年内就增加了六倍。照这样的规模,如果仅仅是单利,资本每年应该提供的就不是10%,而是30%,也就是说提供大两倍的利润,我们把年数增加得越多,在计算每年的单利的时候利息率或利润率就提高得越多,资本越大,这种提高也就总是越快。

但是,事实上资本主义积累无非是利息再转化为资本(因为这里对于我们的目的,即对于这种计算的目的来说,利息和利润被看作是等同的),——因而是复利。今天资本是100;它产生利润(或利息)10。把它加到资本上,得110,这就是现在的资本。因此,它提供的利息就不只是资本100的利息,而是(100K+10Z)的利息,即复利。这样,在第二年末就是(100K+10Z)+10Z+1Z=(100K+10Z)+11Z=121。现在这就是第三年开始时的资本。在第三年是:

(100K+10Z)+11Z+12.1Z,于是资本在第三年末便是133.1。

[880]我们在复利上加上一撇,就得出下表:

\todo{}

换句话说,在第九年就已经有一半以上的资本[这时资本等于214.358881]是由利息构成的,可见资本中由利息构成的部分是按几何级数增加的。

我们看到,二十年后资本就会增加六倍,然而即使按照马尔萨斯的“最极端的”假定,人口也只能在二十五年中增加一倍。但是我们且假定,人口在二十年中增加一倍,因而工人人口也增加一倍。如果算出每年的平均结果,那末利息应当是30%,比它原来大两倍。但是在剥削率不变的情况下,在二十年中已增加一倍的人口(在这二十年的很大一部分时间中,新的一代还不能劳动;尽管有儿童参加劳动,这新的一代在这个期间也几乎有一半时间不能劳动)只能比以前完成多一倍的劳动,因而也只能完成多一倍而不是多两倍的剩余劳动。

利润率(因而还有利息率)是这样决定的:

(1)假定剥削率不变,利润率决定于在业工人人数,决定于所使用的工人的绝对量,因而决定于人口的增长。虽然所使用的工人的绝对量增加了,但是随着资本的积累和工业的发展,它对所使用的资本的总额的比率却降低了(因此,在剥削率不变的情况下,利润率会下降)。同样,整个人口也绝对不会象复利那样按照几何级数增长。在工业发展的一定阶段,人口的增长可以说明剩余价值量和利润量的增加,但同时又可以说明利润率的下降。

(2)利润率决定于正常工作日的绝对量,即剩余价值率的提高。因此,利润率能够由于劳动时间超出正常工作日以外的延长而提高。但是这有它的身体界限和——不久以后——它的社会界限。随着工人推动更多的资本,同一资本会支配更大量的绝对劳动时间,——[881]这是没有疑问的。

(3)如果正常工作日不变,剩余劳动能够随着劳动生产力的发展,通过必要劳动时间的缩短和加入工人消费的生活资料的跌价而相对增加。但是劳动生产力的这种发展使可变资本和不变资本相比减少了。比方说用两个人代替20个人,不管绝对剩余劳动时间或相对剩余劳动时间怎样增加,要使这两个人的剩余劳动时间等于20个人的剩余劳动时间,这在体力上是不可能的。即使这20个人每天只完成两小时的剩余劳动,他们提供的剩余劳动就有40小时,而两个人一天生活的全部时间只有48小时。

劳动能力的价值不是按劳动或资本的生产力提高的比例降低的。生产力的这种提高也会在一切不(直接或者间接)生产必需品的部门提高不变资本对可变资本的比例,而不引起劳动价值的任何变化。生产力的发展是不平衡的。资本主义生产的性质的特点是,它发展工业比发展农业快。这并不是由于土地的性质,而是由于土地需要其他社会关系,以便按照它的性质实际加以利用。资本主义生产只是在它的影响使土地贫瘠并使土地的自然性质耗尽以后,才把注意力集中到土地上去。此外,由于存在土地所有权,农产品比其他商品贵,因为农产品是按其价值支付的,而不会降低到费用价格的水平。但是,农产品是必需品的主要组成部分。其次还有一点:由于竞争的规律,如果有1/10的土地在耕种时花费较贵,其余9/10的耕地也会“人为地”受到这种相对不肥沃的严重影响。

为了在资本积累时利润率保持不变,利润率实际上就必须提高。如果资本总是提供10%的剩余劳动,那末,在按照复利进行积累以及所使用的资本因而增加的情况下,同一个工人就必须按照复利增长的级数多提供两倍、三倍、四倍的剩余劳动,——这是荒谬的。

工人推动的、其价值通过工人的劳动保存和再生产的资本量,是和工人追加的价值即剩余价值完全不同的。如果资本量=1000,追加劳动=100,再生产出来的资本便是1100。如果资本量=100,追加劳动=20,再生产出来的资本便是120。利润率在前一场合=10%,在后一场合=20%。然而从100中可以比从20中积累得更多。因此,资本的源流{撇开资本由于生产力的提高而贬值的情况不谈}——或者说资本的“积累”——将比例于资本已有的量而不是比例于利润率的高度滚滚向前。这一点可以说明,尽管利润率下降,积累(按量来说)还是增加,至于在生产率不断提高而利润率即使降低的情况下,可能比在生产率低而利润率高的情况下积累更大一部分收入,那就更不用说了。高利润率(只要它以高剩余价值率为基础)在劳动生产率虽然不高但工作日很长的情况下是可能的。高利润率之所以可能,[还]因为劳动生产率虽然不高,但是工人的需要很小,因而工资的最低额也很小。与工资最低额的微小相适应的是劳动精力的缺乏。在这两种情况下,尽管利润率高,资本的积累却很慢。人口停滞,而生产产品所耗费的劳动时间很多,虽然支付给工人的工资很少。

[882]虽然剩余价值率不变甚至提高,利润率也会下降,对于这一点我曾这样解释过:可变资本同不变资本相比减少了,也就是说,活的现在劳动同所使用的和再生产出的过去劳动相比减少了。\authornote{见本卷第2册第498页和676页。——编者注}霍吉斯金和《国民困难的原因及其解决办法》小册子的作者则用工人不可能满足“复利”的要求,即不可能满足资本积累的要求来解释利润率的下降。

\begin{quote}{“任何劳动,任何生产力,任何发明才能,任何技术,都不能满足复利的压倒一切的要求。但一切积蓄都是从资本家的收入中来的〈也就是从“简单利润”中来的〉,因此,这些要求实际上不断地提出,而劳动生产力同样不断地拒绝满足它们。因此,不断有一种平衡创造出来。”\authornote{见本册第329页。——编者注}(同上,第23页)}\end{quote}

从总的意思来说这是一样的。我说,利润率会随着资本的积累而下降,因为不变资本同可变资本相比会增加,这就是说,如果撇开资本各部分的一定形式不谈,所使用的资本同所使用的劳动相比会增加。利润下降并不是因为工人被剥削得少了,而是因为同所使用的资本相比,所使用的劳动总的来说是少了。

例如,假定可变资本与不变资本之比=1∶1。在这种情况下,如果总资本=1000,那末,c=500,v=500;如果剩余价值率=50%,那末,500的50%=50×5,即250。因此,利润率将是1000分之250,即250/1000,或1/4,即25%。

如果总资本=1000,c=750,而v=250,那末,在剩余价值率为50%的情况下,250提供125。而利润率将是125/1000,即1/8,或12+(1/2)%

但是在第二种情况下使用的活劳动比第一种情况下[少]。如果我们假定,一个工人的工资一年等于25镑,那末在第一种情况下,工资为500镑时就雇用20个工人,在第二种情况下,工资为250镑时就雇用10个工人。同一笔资本1000镑在一种情况下雇用20个工人,在另一种情况下只雇用10个工人。在第一种情况下,资本总量和工作日数之比是1000∶20;在第二种情况下是1000∶10。在第一种情况下,20个工人中每个工人摊到所使用的资本(不变资本和可变资本)50镑(因为20×50=500×2=1000)。在第二种情况下,每个工人摊到所使用的资本100镑(因为100×10=1000)。与此相应,资本中用于一个工人的工资部分,在两种情况下却是一样的。

我提出的公式包含一个新的论据,它说明为什么在进行积累时,较少的工人会摊到同量的资本上,或者同样可以说,为什么较大量的资本会摊到同一劳动上。无论我是说,在第一种情况下,1个工人摊到的所使用的资本等于50,在另一种情况下,1个工人摊到100单位的资本,也就是只要1/2个工人就摊到50单位的资本;因此,无论我是说,在一种情况下1个工人摊到50单位的资本,在另一种情况下,1/2个工人摊到50单位的资本,还是说,在一种情况下50单位的资本摊到1个工人身上,而在另一种情况下,50×2单位资本摊到1个工人身上,这都是一回事。

霍吉斯金等人正是运用了这后一个公式。在他们看来,积累一般来说就是要求复利,就是说,有更多的资本摊到同一个工人身上,这个工人现在应当按照摊到他身上的资本量提供更多的剩余劳动。因为摊到一个工人身上的资本按复利增加了,而他的劳动时间却相反地具有十分明确的界限,“任何生产力”也不能把他的必要劳动时间缩短到符合这些复利所要求的程度,所以这里“经常会遇到一种平衡”。这时“简单利润”则保持不变或者甚至会增加(这种“简单利润”实际上是剩余劳动或剩余价值)。但是随着资本的积累,在单利形式的背后开始隐藏着复利。

[883]其次,很明显:如果复利=积累,那末,撇开积累的绝对界限不谈,利息的这种形成取决于积累过程本身的规模和强度等等,即取决于生产方式。要不然,复利就无非是以利息形式占有他人的资本(他人的财产),就象过去在罗马以及在一般放高利贷的情况下发生的一样。

霍吉斯金的看法是:原来摊到一个工人身上的资本比方说是50镑,同时假定工人要为50镑资本提供25镑利润。过了几年,由于一部分利息转化为资本,并且年年这样重复,摊到一个工人身上的资本已经是200镑了。如果每年的利息是50%,而且总是全部变成资本,那末这个过程不到四年就可以完成。工人象过去要为50镑资本提供25镑利润一样,现在应当为200镑资本提供100镑利润,即比过去提供的多三倍。但这是不可能的。为此他必须多劳动三倍的时间,就是说,如果以前他一天劳动12小时,现在就要劳动48小时,或者劳动价值必须由于劳动生产力的发展而减少四分之三。

如果工作日等于12小时,年工资是25镑,工人一年提供25镑利润,那末,他为资本家劳动的时间必须同为他自己劳动的时间相等,即6小时,或者说,半个工作日。如果工人必须提供100镑利润,那他就要在12小时当中为资本家劳动4×6小时,而这是荒谬的。假定工作日延长到15小时。即使是在这种情况下,工人也不能在15小时劳动当中提供24小时。他更不能在15小时的工作日中提供30小时,而这30小时是必需的,因为他要为资本家劳动24小时,为自己劳动6小时。如果他们自己的全部劳动时间都用来为资本家劳动,他也只能提供50镑,即只能使“利息”增加一倍——为200镑资本提供50镑利润,而他以前为50镑提供了25镑利润。以前利润率是50%,现在是25%。但是在资本为200镑的情况下,要得到25%是不可能的,因为工人还须生活。不论生产力怎样增长,如果12小时所创造的价值仍然如上例那样等于75镑,那末24小时所创造的价值等于2×75,即150镑。因为工人必须生活,所以他无论如何也不能提供150镑利润,更不用说提供200了。他的剩余劳动始终不过是他的工作日的一部分,但是绝不能由此得出结论,就象洛贝尔图斯先生那样\authornote{见本卷第2册第89—90页。——编者注},认为利润永远不可能等于100%。如果利润按整个工作日计算,利润便永远不能等于100%(因为在整个工作日中利润本身已计算在内),但是就工作日中被支付的部分来说,利润完全可能等于100%。

例如在上例中,利润是50%:

\todo{}

这里占工作日一半的利润等于全部产品的1/3。

[884]如果工人把3/4的工作日给资本家,那就是:

\todo{}

折算成100便得出:

\todo{}

现在我们更详细地来考察一下,在这种见解的背后会隐藏着什么东西,根据这种见解,利润下降是因为在积累进程中利润不是“简单利润”(因此,对工人的剥削率不会降低,却象霍吉斯金所说的会提高),而是“复合利润”,但劳动在任何情况下都赶不上复利的要求。

首先应当指出,这一点需要有进一步的规定才能一般具有意义。当作积累(即占有剩余劳动)的产物来看——这种看法就整个再生产来说是必要的——一切资本都是由利润构成(由“利息”构成,如果这个词被看作和利润等同而不是和“借贷利息”等同)。因此,如果利润率=10%,那末这就是“复利”,利润的利润。完全不能理解的是,在经济上10/100和11/110究竟有什么区别。这样就会得出结论:“简单利润”也是不可能的,或者至少是简单利润也应当下降,因为这种简单利润实际上同复合利润一样是复合的。如果把问题看得狭窄些,即仅仅指生息资本,那末,复利会吞没利润而且吞没的比利润还多;生产者(资本家或非资本家)必须付给放债人复利,这意味着他除利润外不得不逐渐把他的一部分资本也付给放债人。

所以,首先必须指出,霍吉斯金的见解只有在假定资本比人口,即比工人人口增长得快的情况下才有意义。(就是后一种增长也是相对的。资本的本性就是使一部分工人过度劳动,把他们弄得疲惫不堪,把另一部分变为赤贫者。)如果人口和资本增加的程度相同,那就没有任何理由可以说明,为什么我能用100镑从x个工人身上取得的剩余劳动,就不能用800镑从8x个工人身上取得。[885]8×100K对8x个工人提出的要求,不会比100K对x个工人提出的要求更多。因此,这里霍吉斯金的理由不能成立。(实际上完全不是这样。即使人口和资本增加的程度相同,资本主义的发展,由于不变资本靠减少可变资本而发展的结果,也会使一部分人口成为过剩人口。)

\begin{quote}{{“你分配它们〈商品〉是为了促使劳动的供给更多还是更少,你是在它们将成为劳动条件的地方分配它们,还是在它们将鼓励游手好闲的地方分配它们,这一点对劳动来说具有十分重大的意义。”(《论马尔萨斯先生近来提倡的关于需求的性质和消费的必要性的原理》1821年伦敦版第57页)“不断增加的人口的数目会促进这种劳动供给的增加。”(同上,第58页)“如果商品不能支配它以前所支配的那样多的劳动量,那末这只有在这一劳动生产的产品不比过去多的地方才有意义。如果劳动的生产率提高了,那末尽管现有的商品量现在支配的劳动量比过去少,生产也不会缩减。”(同上,第60页)}\end{quote}

这一点是针对马尔萨斯的。的确,生产不会缩减,但利润率会减低。“商品量支配劳动”这样一些厚颜无耻的说法包含了马尔萨斯的价值规定\authornote{见本册第8—10页和第24页。——编者注}中所具有的同样的厚颜无耻。“商品支配劳动”这种说法对资本的性质是极好的和充分的说明。

就是这个作者,对威斯特作了正确的评论:

\begin{quote}{“《论资本用于土地》的作者说,如果资本大量增加,劳动将获得较高的报酬,而这种情况……只有在资本利润很高的时候才会发生。他补充说:‘资本利润越多,劳动工资就越高。’这种说法的错误在于这里漏掉了几个字:‘已经得到的资本利润越多……劳动工资就越高’……高利润和高工资不会同时发生;它们不会在同一桩交易里发生;一个妨碍另一个,并降低其水平。同样可以这样来论述:‘商品的价格最高时,商品的供给也增加得最快,因此大量的供给和高的价格是一起前进的。’这是把因果混为一谈。”(同上,第100—101页)}}\end{quote}

因此,只有当(由于积累过程)同一个工人必须推动更多的资本,或者说,当资本同劳动相比增加了的时候,也就是当例如原来是100的资本由于积累变为110,而原来提供剩余价值10的同一个工人必须适应于资本的增长提供剩余价值11,即提供复利的时候,霍吉斯金的论点才有意义。所以,不仅工人过去推动的同一资本在它被再生产出来以后必须提供相同的利润(“简单利润”),而且这个资本已经由工人的剩余劳动增大起来,工人现在必须第一,为原有资本(或资本的价值)提供剩余劳动,第二,还要为他自己的已被积累起来的即资本化了的剩余劳动提供剩余劳动。既然这一笔资本每年都在增长,同一个工人就必须不断提供越来越多的劳动。

但是一般说来,要在同一工人身上摊到比过去更多的资本,只有[在下列两种情况下]才有可能:

第一种情况。如果劳动生产力不变,要在同一个工人身上摊到比过去更多的资本,就只有使工人延长他的绝对劳动时间,例如不是劳动12小时,而是劳动15小时,或者增加劳动强度,即在12劳动小时内完成15小时的劳动,在4小时内完成5小时的劳动,或者说,在4/5小时内完成5/5小时的劳动。因为工人在一定的时数里把自己的生活资料再生产出来,所以在这种情况下,资本家会得到3小时,就象劳动生产力提高了一样,而实际上这里是劳动增加了,而不是劳动生产力提高了。如果劳动的这种强化推广到一切劳动部门,那末商品的价值就必然按照它所化费的劳动时间的减少而下降。这种强度就会成为劳动的平均强度,成为劳动的自然性质。如果劳动的这种强度[886]只是发生在某些部门,这就等于复杂劳动,即自乘的简单劳动。于是较紧张的一小时劳动的某一部分就会等于较松弛的一小时劳动,它们创造同样的价值。例如,在上述情况下,4/5小时的较紧张劳动,就和5/5即1小时的较松弛劳动创造一样多的价值。

延长劳动时间和通过更大的劳动强度,即通过所谓的压缩劳动空隙来增加劳动,这两者都有其界限(尽管例如伦敦的面包工人通常都是劳动17小时,有时还更多),有十分明确的身体界限,而当达到界限的时候,复利,“复合利润”就会停止。

在这些界限内会出现下列情况:

如果资本家对劳动的延长或强化不予支付,他的剩余价值(利润也一样,如果不变资本的价值不变的话,因为我们假定生产方式不变)——(在上述情况下)他的利润——将比他的资本增长得快。他则不为增长的资本支付任何必要劳动。

如果他按照与过去相同的比例支付追加劳动,剩余价值就会和资本的增加成比例地增长。利润就会增长得更快。因为这里固定资本的周转更快;同时机器的磨损加快的程度不会象它的使用加快的程度一样。固定资本的支出会减少,因为同时劳动的200个工人比延长工作日的100个工人需要更多的机器、建筑物等等。同样,在后一种情况下所需的监工等等也较少。(这种情况给资本家造成了一种极其惬意的机会:他可以不再经过任何困难就能根据市场的情况扩大或缩减他的生产。此外,这种情况会增加他的权力,因为一部分工人劳动负担过重,相应地就会有失业的和半失业的后备军,因而工人间的竞争就会加剧。)

虽然在这种情况下必要劳动和剩余劳动之间的纯粹的算术比例没有被破坏,而且这里的唯一情况是它们两者都能以相同的程度增加,但是对劳动的剥削却增加了,——在工作日延长的情况下是这样,在工作日强化(紧张化)的情况下也是这样,只要在这种强化的同时工作日不缩短(如实行十小时工作日法案)。工人缩短了他的劳动能力的存在期限,在比他的工资的增加大得多的程度上消耗了他的劳动能力,而且更加变成一种单纯的工作机器。但是这后一种情况且撇开不说,如果工人在正常工作日的情况下,假定能活20年,而在工作日延长或强化的情况下只能活15年,那末,在一种情况下他是在15年内出卖他的劳动能力的价值,而在另一种情况下,他是在20年内出卖他的劳动能力的价值。在一种情况下劳动能力的价值必须在15年内被补偿,在另一种情况下则在20年内被补偿。

如果每年支付5%,价值100经过20年将得到补偿,因为5×20=100。如果每年支付6+(2/3)%,价值100经过15年将得到补偿。但是在上面所考察的情况下,工人从追加的3小时中得到的只是相当于按20年计算的他的劳动能力的一天的价值。假定他的劳动是8小时必要劳动和4小时剩余劳动,那末他从每1小时中得到2/3小时,因为12×2/3=8。与此相应,他从3小时的额外时间中得到2小时,或者说,从每1小时额外时间中得到2/3小时。但是只有在假定他的劳动能力存在20年的情况下,这才是他1小时劳动能力的价值。如果工人出卖劳动能力的时间只有15年,那末每小时的劳动能力的价值还要相应提高。

对未来的预支——真正的预支——一般说来在财富生产上只有对工人和对土地来说才有可能。由于过早的过度紧张和消耗,由于收支平衡的破坏,工人和土地的未来实际上可能被预支和被破坏。在资本主义生产条件下两者都会发生这种情况。至于所谓的预支,例如公债,那末关于对未来的这样一种预支,莱文斯顿正确地指出:

\begin{quote}{[887]“他们[公债制度的维护者们]宣称,他们打算把今天的开支转嫁到未来,并且坚决主张为了满足现在这一代人的需要可以加重后一代人的负担,这在实际上等于荒谬地认为,可以消费尚未存在的东西,可以在种子播入土地以前就吃粮食。”(莱文斯顿《论公债制度》第8页)“我们的政治家们的全部智慧就是大规模地把一批人的财产转到另一批人的手里,就是建立巨额基金以奖励投机和盗窃国库。”(同上,第9页)}\end{quote}

工人和土地的情况却不是这样。这里被支出的东西是作为力量而存在的,由于这种力量的加速支出,它的寿命就缩短了。

最后,如果资本家对额外时间,比对正常劳动时间不得不支付更多的报酬,那末照上面所说的,这绝不是工资的提高,而只是对额外时间的提高了的价值的补偿,而且追加的工资很少能达到为此所必需的高度。实际上,在工人进行额外劳动的时候,不仅额外时间应当有较好的报酬,而且每一个劳动小时都应当有较好的报酬,以便劳动能力的较快消耗也能多少得到点报酬。

因此,在所有情况下都是对劳动的更大剥削。同时在所有情况下,剩余价值都会随着资本的积累而[相对]减少,而且利润率也会下降,只要这一点不被不变资本的节约抵销的话。[887]

[887]因此,这就是随着资本积累——随着“复合利润”的出现——利润率必然下降的一种情况。如果资本300(第一笔)的利润率等于10%(因而利润是30),而追加资本100的利润率等于6%,那末资本400的全部利润便是36。因此总的来说,100的利润是9。利润率从10%降到9%。

但是,已经说过,在这个基础上(即在劳动生产率不变的情况下)达到一定点以后,追加资本的利润不仅一定会减少,而且会完全消失,于是以这种“复合利润”为基础的一切积累一定会停止。在这种场合,利润的减少是和对劳动的剥削的加重联系在一起的,利润在一定点上的消失,不是因为工人或其他什么人得到了自己的全部产品,而是因为在体力上劳动不可能超过一定量的劳动时间,也不可能把劳动强度增加到超过一定的程度。

第二种情况。在工人数量不变时,每个工人可以摊到比过去更多的资本,因而追加资本可以被用于、被花费于加强对同一数量工人的剥削的唯一的另外一种情况,[888]就是提高劳动生产率,改变生产方式。这种情况决定了不变资本和可变资本之间有机比例的改变。换句话说,这里资本同劳动相比的增加,和不变资本同可变资本以及一般来说同可变资本使用的活劳动量相比的增加,是等同的。

因此,这里霍吉斯金的见解可以归结为我所阐述的一般规律。

剩余价值即对工人的剥削增加了,但是同时利润率下降了,因为可变资本同不变资本相比减少了,活劳动量同推动它的资本相比,一般来说相对地减少了。在劳动的年产品中,一个较大的部分会在资本的名义下为资本家所占有,一个较小的部分会在利润的名义下为资本家所占有。

{这样就产生了查默斯牧师的幻想:年产品中资本家用作资本的量越小,他们吞掉的利润就越大;\endnote{指查默斯的著作《论政治经济学和社会的道德状况、道德远景的关系》1832年格拉斯哥、爱丁堡、都柏林和伦敦第2版第88—89页及其他各页。——第344页。}于是,“法定教会”\endnote{“法定教会”(《EstablishedChurch》)是指英国国教会。——第56、344页。}就来帮助他们,要他们把很大一部分剩余产品用于消费,而不要把它资本化。这个该死的牧师把原因和结果混淆了。而且利润量在利润率较低时也会随着所花费的资本量的增加而增加。此外,这一较小百分比代表的使用价值量增加了。但是,这同时需要资本的集中,因为现在各种生产条件都要求使用大量资本。这需要由大资本家吞并小资本家,使小资本家“丧失资本”。这不过又是劳动条件和劳动本身在另一种形式上的分离(因为小资本家还有较多的自己的劳动。总的说来资本家的劳动和他的资本量成反比,就是说,和他成为资本家的程度成反比。如果没有抵销这种向心力的离心倾向同向心力一起在经常起作用,那末这个过程很快就会使资本主义生产告终;对于这种离心倾向,这里我们不去考察,因为这是属于论资本的竞争那一章),——这种分离,形成资本和原始积累的概念,然后在资本的积累中表现为不断的过程,最后在这里表现为现有资本集中在少数人手中和许多人丧失资本。}

劳动生产率的提高不能完全补偿劳动量的(相对)减少,或者说,剩余劳动和所花费的资本的比例不是按照所使用的劳动的相对量减少的同一比例增长,这种情况之所以造成,部分地是由于:只有当一定的投资领域的劳动生产率有了发展,劳动价值,或者说,必要劳动量才会减少,即使在这些领域,劳动生产率的发展也是不平衡的,并且还会有各种抵销因素发生作用;例如,工人本身虽然不能阻止工资下降(就价值来说),但是他们不会容许工资绝对降到最低限度,反而会努力争取在量上分享一些增长的共同财富。

但是剩余劳动的这种增加也是相对的,并且只有在一定界限内才有可能。要使它适应复利的要求,必要劳动时间在这种情况下就必须等于零,就象在前面所考察的那种情况下[剩余劳动时间]必须无止境地延长那样。

利润率的提高或降低——由[劳动的]供求的变化,或者由必需品价格(同奢侈品相比)暂时的提高或降低(这种暂时的提高或降低又是由供求的这种变化和由此造成的工资的提高或降低引起的)造成的工资的提高或降低所决定的利润率的提高或降低——同利润率提高或降低的一般规律[889]没有任何关系,正象商品市场价格的提高或降低同商品价值的规定根本没有任何关系一样。这一点应当放在工资的现实运动那一章来考察。如果供求关系对工人有利,工人的工资提高,那末某些必需品的价格,特别是食品的价格就可能(但决不是必然)随之暂时提高。关于这一点,《论马尔萨斯先生近来提倡的关于需求的性质和消费的必要性的原理》一书的匿名作者正确地指出:

\begin{quote}{在这种情况下,“对必需品的需求同对非必需品的需求相比会增加,以致这两种需求之间的比例,同他行使这种权力〈即资本家支配商品的权力〉以获得物品供自己消费时的情况完全不同。必需品将因此同数量更多的一般物品交换……这些必需品至少有一部分会是食物”。(第21—22页)}\end{quote}

接着,匿名作者正确地发挥了李嘉图的见解:

\begin{quote}{“于是,不管怎样,谷物价格的提高并不是工资提高(工资提高使利润降低)的最初的原因,而是相反,首先工资的提高是谷物价格提高的原因,其次,土地的性质(由于这种性质,耕作加强时收成相对地越来越少)使一部分这种价格的提高成为永久性的,并阻止人口规律对已有的工资增加产生充分的反作用。”(第23页)}\end{quote}

霍吉斯金和《国民困难的原因及其解决办法》这一小册子的作者都是用活劳动不可能满足“复利”的要求来解释利润的下降,他们对这个问题虽然没有作更进一步的分析,但是比起斯密和李嘉图来,还是大大接近于真理,因为斯密和李嘉图是用工资上涨来解释利润下降的:一个是用实际工资和名义工资的上涨来解释,另一个是用名义工资的上涨,其实不如说是实际工资的降低来解释。霍吉斯金和所有这些[政治经济学家的]无产阶级反对派都以健全的理智指出了这样一个事实:随着资本的发展,靠利润过活的人数相对地增加了。

\tsectionnonum{[(f)霍吉斯金论劳动的社会性质以及资本与劳动之间的关系]}

现在还要从霍吉斯金的小册子《保护劳动反对资本的要求》中举几个结论性的论点。

对产品的交换价值,即对包含在商品里的作为社会劳动的劳动的论述:

\begin{quote}{“几乎每一个艺术和技能的产品都是联合劳动和结合劳动的结果。”}\end{quote}

(这是资本主义生产的结果。)

\begin{quote}{“人是依赖于人的,这种依赖性随着社会的发展而增长,以致任何个人的任何劳动如果不构成大的社会劳动的一部分,这种劳动就未必……会有丝毫价值。”}\end{quote}

{这段话可以用来说明下面这一论点:商品生产,或者说,作为商品的产品生产,只有在资本的基础上才具有包罗万象的性质,才触及产品的实质本身。}

\begin{quote}{“……在实行分工的地方,在工人能够实现他的收入以前,已有别人对这种或那种劳动的评价参加进来,再也没有什么东西可以叫做个人劳动的自然报酬。每个工人只生产整体的一个部分,由于每个部分单独就其本身来说没有任何价值或用处,因此没有东西工人可以拿来说:‘这是我的产品,我要留给我自己。’从某种联合操作例如制造呢绒的操作开始,直到在共同努力制造这一产品的各种不同的人之间分配其产品为止,这中间不止一次地有人对这种或那种劳动的评价参加进来,问题在于在这个共同产品之中有多少应当归于用联合劳动来生产它的每一个个人。[890]除了把这个问题交给工人自己去自由评价外,我不知道还有别的解决这一问题的办法。”(第25页)“我必须补充一点,未必有一种劳动会比别的劳动更有价值。毫无疑问,一切劳动都是同样必需的。”(第26页)}\end{quote}

最后,霍吉斯金谈到资本和劳动之间的关系:

\begin{quote}{“师傅和他们的帮工一样是工人。在这一点上,他们的利益和他们帮工的利益完全相同。但除此以外,他们还是资本家或是资本家的代理人,在这方面,他们的利益和他们工人的利益则截然相反。”(第27页)“这个国家的产业工人的教育已广为普及,这就使得几乎所有师傅和老板的劳动和技艺的价值日益降低,因为教育的广为普及,使拥有这种专门知识的人数增加了。”(第30页)“资本家是在各种工人之间从事压迫的中介人。”如果排除了资本家,那末“非常清楚,资本,或者说使用劳动的能力,和并存劳动就是一个东西;生产资本和熟练劳动也是一个东西。因此,资本和工人人口完全是一个意思。在自然界的体系中,口是同手和智慧结合在一起的”。(第33页)}\end{quote}

资本主义生产方式同社会劳动的不同因素在相互关系中所具有的并以资本为代表的异化形式一起消失。这便是霍吉斯金的结论。

\centerbox{※     ※     ※}

资本的原始积累。包括劳动条件的集中。它是劳动条件对工人和劳动本身的独立化。它的历史活动就是资本产生的历史活动——把劳动条件转化为资本、劳动转化为雇佣劳动的历史的分离过程。这样就提供了资本主义生产的基础。

在资本本身基础上、因而也是在资本和雇佣劳动关系基础上的资本积累。它以越来越大的规模再生产出物质财富同劳动的分离和独立。

资本的积聚。大资本通过消灭小资本而进行的积累。吸引力。资本和劳动的中间结合体的丧失资本。这不过是下述过程的最后一级和最后形式:把劳动条件转化为资本,然后把这种资本和某些资本以更大的规模再生产出来,最后把社会上许多地方形成的资本同它们的所有者分离开来,并把它们集中在大资本家手里。生产在取得这种对立和矛盾的极端形式的同时,转化为社会生产,尽管是以异化的形式。这就是社会劳动以及在实际劳动过程中生产工具的公共使用。资本家作为上述过程,即同时加速这一社会生产,从而加速生产力发展的过程的职能执行者,就依照他们以社会名义为自己刮取收入以及作为这一社会财富的所有者和社会劳动的指挥者而飞扬跋扈的程度日益成为多余的人。他们的情况也和封建主一样,封建主的要求连同他们的服务,就曾经随着资产阶级社会的产生而成为多余的东西,变成了纯粹是过时的和不适当的特权,从而迅速趋于消灭。[XV—890]

\tsectionnonum{[(g)霍吉斯金的基本论点在其《通俗政治经济学》一书中的表达]}

[XVIII—1084]托·霍吉斯金《通俗政治经济学。在伦敦技术学校的四次演讲》1827年伦敦版。

\begin{quote}{“容易的劳动只是留传下来的技能。”(第48页)“因为由分工产生的一切利益自然集中在工人那里并属于工人,如果工人被剥夺了这些利益,如果在社会发展的进程中由于工人的技能不断提高而发财致富的只是那些从来都不劳动的人,那末造成这种情况的原因一定是非正义的占有,是发财致富的人的篡夺和抢劫,是陷于贫困的人的同意俯首听命。”(第108—109页)[1085]“如果把工人的繁殖仅仅同资本家对他们的服务的需求相比较,那末工人确实是繁殖得太快了。”(第120页)“马尔萨斯指出,工人人数的增加对每个工人从年产品中得到的份额的减少有影响,其假定是:这一产品在工人中间进行分配的那一部分是一定的和固定的量,决不是由工人一年中所生产的东西调节的。”(第126页)“劳动是价值的唯一尺度;但是劳动,这个一切财富的创造者,不是商品。”(第186页)}\end{quote}

关于货币对财富增加的影响,霍吉斯金正确地指出:

\begin{quote}{“如果一个人能够用小量的容易毁坏的产品来换取某种不易毁坏的东西,那末他就不会被诱惑去抛弃那些容易毁坏的产品了。这样,货币的使用就会防止浪费,从而增加财富。”(第197页)“零售商业的主要好处是由这种情况决定的:商品最适于生产的量并不是商品最适于分配的量[对个人消费来说]。”(第146页)“关于资本的理论,以及使劳动停在除工人生活费用之外还能为资本家生产利润的那个点上的实践,看来,都是同调节生产的自然法相违背的。”(第238页)}\end{quote}

关于资本积累,霍吉斯金所说的同他在前一部著作中所说的几乎一样。但是为了完整起见,这里还是把那些主要段落引证如下:

\begin{quote}{“我们现在仅仅来考察一下例如固定资本——这是对那些断定资本有助于生产的人最有利的题目。必须区别资本积累的结果极不相同的下列三种情况:(1)生产资本和使用资本的是同一个人。非常明显,他生产和使用的工具在他手里的任何积累都会减轻他的劳动。工人生产和使用这种工具的能力就是这种积累的界限。(2)生产资本和使用资本的是按公平的比例分配共同劳动产品的不同的人。资本可以由一个工人生产,而由另一个工人使用;他们按照每一个人在生产商品时所花费的劳动的比例分配商品……但是这一事实我宁可这样来说明:社会的一部分生产工具,而另一部分却使用工具,这样便形成能够提高生产力和增长公共财富的一定的分工部门。只要这两类工人的产品在他们之间进行分配,他们生产和使用的工具的积累和增加,就会象生产工具和使用工具的是同一个人时一样地有益。(3)资本是既不生产它也不使用它的那一类人的财产……资本家只是工具的所有者,他本身不是劳动者。他无论怎样也不能促进生产。”}\end{quote}

{换句话说,促进生产的是工具,而不是某个A对这种工具所具有的所有者的头衔,不是工具属于非劳动者这种情况。}

\begin{quote}{“资本家占据一个工人的产品并把它转交给另一个工人——或者象多数种类的固定资本那样转交一个时期,或者象工资那样永远转交出去,——只要资本家认为这一产品的利用或消费可以给他带来好处,他就转交。资本家如果不是为了自己的利益,那他绝不容许落到他手里的一个工人的产品被另一个工人利用或消费。他使用或出借自己的财产,为的是在工人的产品或者说自然收入中得到一份;这种财产在他手里的任何积累,都不过是他支配劳动产品的权力的扩大,并且会阻碍国民财富的发展。目前的情况就是这样……因为资本家,整个产品的所有者,只要他除了维持工人生活的费用以外得不到利润,他就既不会允许工人生产工具,也不会允许工人使用工具,所以很明显,这里对生产劳动设置的界限比自然界规定的界限要狭窄得多。随着资本在第三者手里的积累,资本家所要求的全部利润额增加了,从而给生产和人口的增加制造了人为的障碍……在工人从来不是资本所有者的当前社会状况下,资本的任何积累都会使要求于工人的利润额增加,并且使仅能保证工人过舒适生活的一切劳动成为不可能……既然承认劳动生产一切,甚至生产资本,那末把生产力说成是由劳动所生产和使用的工具造成的,便是荒谬的了。”(第243—247页)“工资不会象工具那样使生产变得容易……劳动,而不是资本,支付一切工资。”(第247页)[1086]“资本家的大部分预付是由支付的诺言构成的……纸币的发明和使用显示了资本决不是积蓄的某种东西。只要资本家为了实现自己的财富或支配他人的劳动而不得不拥有真正积累起来的贵金属或商品,我们就可以认为资本的积累是真正积蓄的结果,认为社会的进步取决于资本的积累。但是,当发明了纸币和印在羊皮纸上的有价证券,当只有这么一张羊皮纸的人就能得到纸片形式的年收入,并且由于有了这些小纸片,他就可以得到供他使用或消费的一切所需的东西,而如果他不把所有这些小纸片都花光,他在年终就比年初更富,或者说,就有权在第二年得到更大量的纸片,于是他就有更大的支配劳动产品的权力,——这时就非常明显,资本不是积蓄的结果,单个资本家发财致富不是由于真正的物质的积蓄,而是由于他做了一件使他能够……从他人的劳动产品中得到更多东西的事情……呢绒厂主有用以支付工资的铸币或纸币。他的工人用这种工资去交换别的工人的产品,这种工资不论是铸币还是纸币,后者都不会保存起来;它又回到厂主那里,厂主又拿出他自己的工人制造好的呢绒来和工资交换。他又用返回来的铸币或纸币支付工资,这些铸币或纸币又进行同样的循环……把体现在机器上的知识和技能给予劳动的一切巨大帮助都只归功于他的〈资本家的〉财产,而不管这种财产是用来支付工资还是表现为有用的工具……矿工、熔炼工、锻工、机械工、司炉和无数其他的人的联合劳动,而不是死的机器,完成着蒸汽机所做的一切……按照通常的说法,工人的这种技能的生产力被认为是由它的有形产品即劳动工具造成的,既不生产工具,也不使用工具,而只是工具的所有者的人,却自认为是最生产的人。”(第248—251页)}\end{quote}

霍吉斯金对于“资本流往国外的危险”的议论的反驳,对于把资本利息看作发展生产的必要刺激的观点的反驳,或者说对于积蓄理论的反驳,见第IX本札记本第47页\endnote{马克思引的是他1851年的第IX本札记本。在这个札记本的第47页摘录了霍吉斯金《通俗政治经济学》一书第252—256页上的话。——第352页。},要在论庸俗经济学家一章中谈这一点。

\begin{quote}{“随着人口的增长,生产和消费两者也都增长,国民财富的积累或增长的概念无非就是这样。”(第257页)[XVIII—1086]}\end{quote}

\tsectionnonum{[(h)霍吉斯金论资本的权力以及论财产权利的变革]}

[XIII—670a][霍吉斯金]《财产的自然权利和人为权利的比较》1832年伦敦版。

\begin{quote}{“现在,社会的一切财富首先落入资本家手中,甚至大部分土地也被资本家买去。他对土地所有者支付地租,对工人支付工资,对赋税和什一税的征收者支付他们要求的东西,而留给自己的是年劳动产品的很大一部分,其实是最大的而且日益增长的一部分。现在,资本家可以看作是全部社会财富的最先所有者,虽然没有任何一项法律给予他这种所有权。”(第98页)“所有权方面的这种变化是由于资本的取息、由于复利的增长而产生的,同样值得注意的是,整个欧洲的立法者都想用取缔高利贷的法律来阻止这件事。”(第98页注)“资本家支配国家的全部财富的权力是所有权上的一种彻底的革命;然而这个革命是靠哪一项法律或者哪一套法律来实行的呢?”(第99页)[XIII—670a]}\end{quote}

\tchapternonum{[(4)]政治经济学家的反对派布雷}

[\endnote{关于布雷的一节马克思没有写完。实际上马克思只是收集了作为“政治经济学家的反对派”的布雷的一些最重要的见解。布雷关于“平等交换”是消除使工人阶级成为牺牲品的那种不公正现象的手段的空想学说,马克思早在《哲学的贫困》(1847年)第一章题为《构成价值或综合价值》的第二节(见《马克思恩格斯全集》中文版第4卷第88—117页)作了批判分析。关于布雷对货币的本质和作用的看法,见马克思的1847年手稿《工资》(《马克思恩格斯全集》中文版第6卷第641页);《政治经济学批判大纲》(1939年莫斯科版)第55、690、754页;1858年4月2日马克思给恩格斯的信;《政治经济学批判》(《马克思恩格斯全集》中文版第13卷第76页)。——第353页。}X—441]约·弗·布雷《对待劳动的不公正现象及其消涂办法》1839年里子版。

\begin{quote}{因为人的存在以劳动为条件,而劳动又以劳动资料为前提,所以“土地这个一切活动的巨大场所和一切财富的原料,必须是它的所有居民的共同财产”。(第28页)“生活有赖于食物,而食物有赖于劳动。这种依赖性是绝对的。因此,一个人要回避劳动,只有在其他大批人的劳动增加的情况下才有可能。”(第31页)“人们所加予别人的或自己遭受的一切不公正现象和痛苦,归根到底都是由于某些个人和阶级篡夺了土地的权利并剥夺了其他个人和其他阶级的这种权利……人们占有了土地所有权以后,下一步便是占有对人本身的所有权。”(第34页)}\end{quote}

布雷宣布自己的目的是:

\begin{quote}{“在政治经济学家们自己的基础上并用他们自己的武器来同他们进行斗争〈为了证明不是在任何社会制度下贫困都必然是工人的命运〉。政治经济学家们要推翻用这种方法作出的结论,就必须先否认或推翻他们自己的论点所依据的那些已确立的真理和原则。”(第41页)“根据政治经济学家们本身的意见,为了生产财富,必须有:(1)劳动,(2)过去劳动的积累,或者说资本,(3)交换……”这就是他们所认为的一般生产条件。“这些生产条件对整个社会都是有效的,它们的性质是:任何个人或任何阶级都不能不受它们的影响。”(第42页)“金科玉律:必须劳动!——对一切创造物来说是同样有约束力的……只有人能够回避这一规律;这一规律的性质是:一个人只有靠牺牲别人才能回避这一规律。”(第43页)“按照劳动和交换的真正性质来说,严格的公正态度要求{布雷在这里引用了政治经济学家们提出的商品交换价值的定义}交换双方的利益不仅是相互的,而且是相等的……在公正的交换制度下,一切商品的价值都会由它们的生产费用的总和来确定,并且相等的价值应该总是换得相等的价值……直到今天,工人们交给资本家一年的劳动,但只换得半年劳动的价值,现在在我们周围存在着的权力和财富的不平等就从这里产生。交换的不平等(按一种价格买进,按另一种价格卖出)的必然结果是:资本家继续是资本家,工人继续是工人,一个是暴君阶级,另一个是奴隶阶级。”(第47—49页)“在现在的制度下,交换不仅没有象政治经济学家们所断定的那样给交换双方的每一方提供相互的利益,而且可以有把握地说,在资本家和生产者之间的大多数交易中根本没有进行交换……工厂主或土地所有者用什么来支付工人的劳动呢?用劳动?不是,因为资本家不劳动。用资本?不是,因为他的财富的储备不断增长……因此,资本家不能用属于他自己的任何东西来交换。因此,整个这种交易明显地表明,资本家和土地所有者所做的只是:他们对工人一星期劳动的偿付,是他们上星期从工人那里取得的财富的一部分;而这一点正好说明他们同工人以无易有……资本家好象用来和工人的劳动相交换的财富既不是资本家的劳动创造的,也不是他的财富创造的,它最初由工人的劳动得来,并且通过欺骗性的不平等交换制度每天又从工人那里被夺走。生产者和资本家之间的全部交易是明显的欺骗,纯粹是一幕滑稽剧。”(第49—50页)“宣称‘必须积累!’的法律只有一半得到执行,它的提出有利于一个特殊阶级而有损于整个其余社会。”(第50页)“在现在的社会制度下,整个工人阶级在劳动资料方面依赖资本家或雇主;而在一个阶级由于自己的社会地位而在劳动资料方面依赖另一个阶级的地方,它在生活资料方面也同样依赖那个阶级。而这种状况同社会的目的本身是如此矛盾,并且是如此违背理性……因此一分钟也不能为它辩解,为它辩护。它赋予个别的人以任何一个凡人所不应有的权力。”(第52页)“我们的日常经验告诉我们,如果我们从一个面包上切下一块,这一块就不能再长大。一个面包只是若干块的总和,我们吃掉的块数越多,留下来吃的就越少。工人的面包的情况就是这样,[442]但是资本家的面包却不遵循这种规则。他的面包不是变小,而是不断增大;资本家不断地切,但面包总是在增大……如果交换是平等的,现在的资本家的财富就会逐渐地由他们那里转到工人阶级的手里;富人花掉的每一个先令都会使他的富有少一个先令。”(第54—55页)}\end{quote}

布雷在同一个地方还指出:

\begin{quote}{“一个资本家要从自己的属于工人阶级的祖先真正积累起来的劳动中继承哪怕是一千镑,也几乎是完全不可能的。”(第55页)“从政治经济学家们自己的学说中可以得出这样的结论:没有积累就不可能有交换,没有劳动就不可能有积累。”(第55页)“在现在的制度下,每个工人至少给予雇主六日劳动以换取一个只值四日或五日劳动的等价物,雇主的利益必然是工人的损失。”(第56页)“因此,不论怎样试图用赠予、个人积累、交换或继承来论证财富的起源,我们都会发现一个又一个的证据,说明在富人的所有权的这种论证方面有一个缺陷,这个缺陷使论证一下子便失去任何公正的外貌和任何意义……所有一切财富都是好多世纪以来在工人阶级的骨肉上生长起来的,并且是通过欺骗性的和奴役性的不平等交换制度从工人那里夺走的。”(第56—57页)“在现在的制度下,如果工人想成为富人,他就必须成为资本家,或者说,成为交换他人劳动的人,而不是交换自己的劳动,那时,他就会用别人掠夺他自己的办法,即通过不平等交换的办法来掠夺别人,从而有可能从别人的不大的损失中获得大量的利益。”(第57页)“政治经济学家们和资本家们写了并出版了很多书,目的是给工人灌输一种错误的观念,似乎‘资本家的利润并不是生产者的损失’。他们对我们说,劳动离了资本寸步难行,资本就象挖土工人手里的铁铲一样,资本对于生产就象劳动本身对于生产一样必要……资本和劳动的这种相互依赖性与资本家和工人的关系毫无共同之处,并不证明前者必须靠后者生活……对生产者的操作具有重大意义的不是资本家,而是资本。资本和资本家之间的区别就象船上装的货物和提货单之间的区别一样大。”(第59页)“从资本和劳动的相互关系中可以明显地看出,一个国家的资本越多,或者说,积累的产品越多,生产就越容易,为达到某一(一定的)结果所需要的劳动就越少。例如,不列颠人民利用他们现在的巨大的资本积累(他们的建筑物、机器、船舶、运河和铁路),在一个星期内所能生产的工业财富,比一千年前他们的祖先在半个世纪里所能生产的还多。使我们能够这样做的,不是我们的体力的优越,而是我们的资本。因为凡是资本缺乏的地方,生产就发展得缓慢而吃力,反过来也是一样。由此可以清楚地看出,资本的一切利益同样是劳动的利益,资本的任何增加都会减轻劳动的繁重程度,因此资本的一切损失必然也是劳动的损失。虽然这一真理早被政治经济学家们所发现,但是他们还从来没有作过公正的阐述。”}\end{quote}

{事实上,这些家伙是这样议论的:

积累的劳动产品——即未消费的产品——减轻劳动,并使劳动更有成效。因此,这种减轻等等的成果应当不是对劳动本身有利,而是对积累有利。因此,积累不应当是劳动的财产,而劳动却应当是积累的财产——劳动自己的产品的财产。因此,工人不应当为自己积累,而应当为别人积累,积累应当作为资本同他相对立。

在政治经济学家们那里,资本的物质要素和它的作为资本的社会的形式规定性(即和它的作为支配劳动的劳动产品的对抗性质)是如此地生长在一起,以致他们提出的任何一个论点都不能不自相矛盾。}

\begin{quote}{“政治经济学家们总是把资本和社会的一个阶级等同起来,把劳动和另一个阶级等同起来,虽然这两种力量都没有这种自然的联系,同样也不应当有这种人为的联系。政治经济学家们总是把事情说成这样:似乎工人的幸福,甚至工人的生存本身,只有在工人用自己的劳动来维持资本家的奢侈和懒散生活的情况下才有可能。他们不愿让工人吃饭,直到工人生产出两份饭——一份为自身,一份为他的老板,后者则是间接地即通过不平等交换得到自己的一份。”(第59—60页)“当工人生产出某种物品的时候,它已经不属于工人,而属于资本家,通过不平等交换的无形魔术,它从一个人手里转到另一个人手里。”(第61页)“在现在的制度下,资本和劳动,铁铲和挖土工人,是两种分离的和对抗的力量。”(第60页)[443]“但是,即使所有的土地、房屋和机器都属于资本家,而不存在工人阶级,资本家也不能回避伟大的条件‘必须劳动!’。尽管他们有一切财富,他们也只能在劳动和饿死之间进行选择。他们不能吃土地或房屋;没有人的劳动加进去,土地就不会长出食物,机器也不会做出衣服。因此,如果资本家和私有者说,工人阶级应该养活他们,那末他们实际上也就是说,生产者完全象土地和房屋一样属于他们,工人只是为了有钱人的需要才创造出来。”(第68页)“生产者用他所给予资本家的东西进行交换时,得到的不是资本家的劳动,也不是资本家的劳动产品,而是工作。在货币的帮助下,工人阶级不仅不得不完成为活命自然要完成的劳动,而且还得为其他阶级负担劳动。生产者从非生产阶级那里得到的是金银还是其他商品,那是无关紧要的;全部实质在于,工人阶级完成他自己的劳动并养活他自己,此外还要完成资本家的劳动和养活资本家。不论生产者从资本家那里得到的名义报酬是什么,他们的实际报酬却是:本来应当由资本家完成的劳动现在转到了他们身上。”(第153—154页)“我们假定联合王国的人口是2500万。假定他们的生活费平均每人每年至少15镑。联合王国全部人口的生活费的年价值总额是37500万镑。但是我们生产的不只是生活资料,因为我们的劳动也创造出许多不供个人消费的物品。我们增加了我们的房屋、船舶、工具、机器、道路和其他供未来的生产使用的设备,此外还修复了一切磨损了的东西,因此我们每年都在增加我们的积累储备,或者说资本。所以,虽然我们的生活费的价值象上面所说的每年只有37500万镑,但是人民所创造的财富总的年价值却不少于5亿镑……只有1/4的人口即大约600万从14岁至50岁的男人可以算作真正的生产者。可以说,在目前的情况下,这个数字中参加生产的恐怕还不到500万人〈布雷接着说,直接参加物质生产的只有400万人〉;因为成千上万有劳动能力的男人被迫坐着无事可做,而本应由他们去做的工作却由妇女和儿童去做;在爱尔兰就有几十万男人根本找不到工作。这样一来,不到500万男人连同几千名儿童和妇女却必须为2500万人进行生产……现有的工人人数如果不使用机器,便不能养活自己和养活现有的游惰者以及非生产劳动者。现在在农业和工业上使用的各种机器,据统计可以完成近1亿有劳动能力的男人的劳动……这些机器及其在现在的制度下的使用,产生了几十万现在压迫工人的游惰者和食利润者……机器使现在的社会制度富有成效,机器也将使它遭到破坏……机器本身是好的,没有机器不行;但是机器的使用,它们为个别人占有而不为整个国家占有这种情况却不好……现在参加生产的500万男人中,有些人一天只劳动5小时,而另一些人却劳动15小时;如果此外我们还注意到由于生意萧条时期大量工人被迫无事可做而造成的时间损失,我们就会发现我们的年产品是由社会上不到五分之一的每天平均劳动10小时的人创造和分配的……假定各种有钱的非生产者以及他们的家属和奴仆只有200万人,他们的生活费平均和工人的生活费一样多,即每人15镑,那末这200万人每年就将花掉工人阶级3000万镑……但是按照最低的估计,他们的生活费每人至少50镑。这样,作为社会上完全非生产的纯粹不劳而食者的生活费的年价值总额就是1亿镑……此外,还有各种小私有者阶级、产业家阶级和商人阶级以利润和利息的形式[444]获得的加倍的和四倍的收入。根据最低的估计,社会上这个人数众多的部分所消费的那部分财富一年不少于14000万镑,超过了相同人数的报酬最高的工人的平均消费量。这样,游惰者和食利润者这两个阶级(他们可能占总人口的1/4)连同他们的政府一起,每年就要吞食近3亿镑,即超过生产出来的全部财富的半数。这一点使帝国的每个工人平均一年损失50镑以上……剩下在国家其余3/4的人口中分配的大约平均每人每年至多11镑。根据1815年的统计,联合王国全体人民的年收入约为43000万镑,其中工人阶级得到99742547镑,而靠地租、年金和利润生活的阶级得到330778825镑。当时国内全部财产的价值约计30亿镑。”(第81—85页)}\end{quote}

参看金的图表\endnote{马克思指英国最初的统计学家之一格雷哥里·金所编的《1688年英格兰不同家庭的收支表》,这个图表被查理·戴韦南特收进他的著作《论使一国人民在贸易差额中成为得利者的可能的方法》(1699年伦敦版)。马克思在《剩余价值理论》第一册(见本卷第1册第171—172页)谈到这个图表。——第359页。}等等。

\begin{quote}{1844年英国的人口是:大小贵族——1181000人,商人、工业家、农场主等——4221000人(以上两类共5402000人),工人、贫民等——9567000人。(托·查·班菲尔德《产业组织》1848年伦敦第2版[第22—23页])[X—444]}\end{quote}

\tchapternonum{[第二十二章]拉姆赛}

\tchapternonum{[(1)区分不变资本和可变资本的尝试。关于资本是不重要的社会形式的观点]}

[XVIII—1086]乔治·拉姆赛(三一学院)《论财富的分配》1836年爱丁堡版。

说到拉姆赛,我们又回到政治经济学家们这里来了。

{为了把商业资本列入生产领域,拉姆赛把它称为“商品从一个地点向另一个地点的运输”。(拉姆赛,同上第19页)这样,他就把商业和运输业混淆了。}

拉姆赛的主要功绩在于:

首先,他事实上区分了不变资本和可变资本。诚然,这种区分是以如下的方式作出的:他把从流通过程得出的固定资本和流动资本的区别作为唯一的区别在名称上保留下来,但是对固定资本作了这样的解释,说它包括不变资本的一切要素。因此,他所理解的固定资本,不仅是机器和工具、劳动用或保存劳动成果用的建筑物、役畜和种畜,而且包括各种原料(半成品等等)、“土地耕种者的种子和制造业者的原料”。(第22—23页)此外,被拉姆赛列入固定资本的还有“各种肥料、农业用的篱笆和工厂中消费的燃料”。(第23页)

\begin{quote}{“流动资本只由在工人完成他们的劳动产品以前已经预付给工人的生活资料和其他必需品构成。”(同上)}\end{quote}

由此可见,他所谓的“流动资本”,无非是[1087]归结为工资的那部分资本,而固定资本则是归结为客观条件——劳动资料和劳动材料——的那部分资本。

当然,拉姆赛的错误在于,他把这种从直接生产过程得出的资本的划分与从流通过程中产生的区别等同起来。这是他墨守政治经济学传统的结果。

另一方面,拉姆赛又把按照上面那样解释的固定资本的纯粹物质构成和它作为“资本”的存在混淆起来。流动资本(即可变资本)不进入实际劳动过程;进入这个过程的,是用流动资本买来的东西,也就是用来代替它的东西——活劳动。除此之外,进入这个过程的还有不变资本,即物化在客观的劳动条件(劳动材料和劳动资料)中的劳动。因此,拉姆赛说道:

\begin{quote}{“严格地说,只有固定资本,而不是流动资本,才是国民财富的源泉。”(第23页)“劳动和固定资本是生产费用的所有要素。”(第28页)}\end{quote}

在生产商品时实际耗费的,是原料、机器等等以及推动它们的活劳动。

“流动”资本是多余的,它处在生产过程之外。

\begin{quote}{“如果我们假定工人不是在完成产品之前得到报酬,那就根本不需要流动资本。生产还会保持同样的规模。这证明,流动资本既不是生产的直接因素,甚至对生产也毫无重要意义,它只是由于人民群众可悲的贫困而成为必要的一个条件。”(第24页)“从国民的观点来看,只有固定资本才是生产费用的要素。”(第26页)}\end{quote}

换句话说,被拉姆赛称为“固定资本”的、物化在劳动条件(劳动材料和劳动资料)中的劳动以及活劳动,简言之,实现了的、物化了的劳动以及活劳动,是生产的必要条件,是国民财富的要素。反之,[在拉姆赛看来,]工人的生活资料一般地采取“流动资本”的形式,这纯粹是由“人民群众可悲的贫困”产生的“一个条件”。劳动是生产的条件,而雇佣劳动则不是;从而工人的生活资料作为“资本”,作为“资本家的预付”同工人相对立,这也不是生产的条件。拉姆赛忽视了这样一个情况:如果生活资料不作为“资本”(按他的说法,不作为“流动资本”)同工人相对立,客观的劳动条件也就同样不作为“资本”(按他的说法,不作为“固定资本”)同工人相对立。拉姆赛认真地,而不象其他政治经济学家那样只是在口头上,把资本归结为“国民财富中用于或预定用于促进再生产的部分”[第21页]。因此他宣称,雇佣劳动,从而资本——再生产资料在雇佣劳动的基础上取得的社会形式——是不重要的,它们只是由人民群众的贫困产生的。

总之,我们现在已经接近这样一点,即政治经济学本身根据它的分析宣布:生产的资本主义形式,从而资本,对生产来说并非绝对的条件,而只是“偶然的”、历史的条件。

然而,拉姆赛的分析还不够,还不足以从自己的前提中,从他给直接生产过程中的资本所下的新定义中,得出正确的结论。

\tchapternonum{[(2)拉姆赛关于剩余价值和价值的观点。剩余价值归结为利润。关于不变资本和可变资本的价值变动对利润率和利润量的影响问题的不能令人满意的说明。资本的有机构成,积累和工人阶级的状况]}

拉姆赛确实接近于正确地理解剩余价值。

\begin{quote}{“流动资本所使用的劳动,总是要多于先前用于它自身的劳动。因为,如果它使用的劳动不能多于先前用于它自身的劳动,那它的所有者把它作为流动资本使用,还能得到什么好处呢?”(第49页)“或许有人会说,任何一笔流动资本所能使用的劳动量,不过等于先前用于生产这笔资本的劳动。这就意味着,所花费的资本的价值等于产品的价值。”(第52页)}\end{quote}

因此,这就是说,资本家用较少的物化劳动同较多的活劳动相交换,这个无酬的活劳动余额,构成产品价值超过产品生产中消费掉的资本价值的余额,换句话说,构成剩余价值(利润等等)。如果资本家以工资支付的劳动量等于他在产品上从工人那里收回的劳动量,产品的价值就不会大于资本的价值,也就不会有利润了。尽管拉姆赛在这里如此接近于剩余价值的真正起源,然而他毕竟受政治经济学传统的束缚太甚,以致又立即走入歧途。首先,他对可变资本[1088]和劳动之间的这种交换的解释方法本身是模棱两可的。如果他对这种交换十分明确,就不可能产生进一步的误解。他说:

\begin{quote}{“一笔比如说由100个工人的劳动创造的流动资本,将推动150个工人。因此,在这种情况下,年终的产品将是150个工人劳动的结果。”(第50页)}\end{quote}

在什么条件下,100个工人的产品能够雇150个工人呢?

如果一个工人得到的12劳动小时的工资等于12劳动小时所创造的价值,那末,用他的劳动的产品只能重新买到一个工作日,用100工作日的产品只能买到100工作日。但是,如果他一天的劳动产品的价值等于12劳动小时,而他一天得到的工资的价值只等于8劳动小时,那末用他一天的产品的价值就可以支付(可以重新买到)1+(1/2)个工作日或1+(1/2)个工人。用100工作日的产品可以雇100(1+1/2个工人或工作日)=100+50=150个工人。由此可见,为使100个工人的劳动产品能推动150个工人,这100个工人中的每一个,或者总的说来,每一个工人都必须用相当于他为自己劳动的时间的一半白白为资本家劳动,或者说,他必须白白劳动1/3个工作日。在拉姆赛那里,这一点没有讲清楚。他的模棱两可表现在结论上:“因此,在这种情况下,年终的产品将是150个工人劳动的结果。”当然,它将是150个工人劳动的结果,正象100个工人的劳动产品曾经是100个工人劳动的结果一样。模棱两可(以及无疑由于拉姆赛或多或少地效法马尔萨斯而产生的含糊不清)在于,好象利润的产生只是由于现在使用的是150个工人,而不是100个工人。这就等于说,从150个工人那里获得利润,是由于现在用这150个工人的产品推动了225个工人(100∶150=150∶225;20∶30=30∶45;4∶6=6∶9)。但问题不在这里。

如果用x表示100个工人的全部工作日,他们提供的劳动量就是x。他们得到的工资则是2/3x。因此,他们的产品的价值=x,他们的工资的价值=x-1/3x,他们创造的剩余价值=1/3x。

如果把100个工人劳动的全部产品重新花在工资上,那末就可以雇150个工人,而这150个工人的产品又等于225个工人的工资。100个工人的劳动时间就是100个工人的劳动时间。但是,他们的有酬劳动却是66+(2/3)个工人劳动的产品,或者说仅仅等于100个工人产品中包含的价值的2/3。模棱两可是这样产生的:好象100个工人或100个工作日(按年或按日计算工作日都一样)提供了150工作日,即包含着150工作日创造的价值的产品;实际上则是100工作日创造的价值够支付150工作日的报酬。如果资本家仍旧使用100个工人,他的利润就仍旧那么多。他就仍旧用等于66+(2/3)个工人的劳动时间的产品支付100个工人,而把余下的部分装入腰包。如果他把100个工人的全部产品重新花在工资上,他就实现了积累,并且可以占有等于50工作日的剩余劳动,而以前占有的只是33+(1/3)工作日。

拉姆赛在这个问题上并没有搞清楚,这一点从下述事实立刻就可以看出来:他为了反对价值决定于劳动时间,又把那个否则就“无法解释”的现象——对于剥削不等量劳动的各资本,利润率是相同的——提了出来。

\begin{quote}{“固定资本的使用,大大改变了价值取决于劳动量的原则。因为一些耗费了等量劳动的商品,要成为可供消费的成品,却需要很不相同的时间。但是因为在这段时间里资本不带来收入,所以,为了使该项投资不比其他项投资——这些项投资的产品成为可供消费的成品所需的时间较短——获利少,当商品最后进入市场时,它必须提高价值,提高的数额相当于少得的利润。这一点表明,资本可以撇开劳动而调节价值。”(第43页)}\end{quote}

更确切地说,这一点表明资本撇开特殊产品的价值而调节平均价格\endnote{“平均价格”这一术语,马克思在这里是指“生产价格”,就是指生产费用(c+v)加平均利润。“平均价格”这一术语本身说明,这里所指的是“一个相当长的时期内的平均市场价格,或者说,市场价格所趋向的中心”(见本卷第2册第359页)。马克思用的这个术语最初见于本卷第1册第76页。在本卷第2册关于洛贝尔图斯和李嘉图的那几章,这个术语多次跟“费用价格”和“生产价格”同时并用。——第365页。},表明资本交换商品不是按照商品的价值,而是按照“使该项投资不比其他项投资[1089]获利少”的办法。拉姆赛也没有放弃重复自[詹姆斯·]穆勒以来就已经出名的例子“葡萄酒置于窖内”\authornote{见本册第89—91、193、251页。——编者注},因为在政治经济学中,不用脑子的传统比在其他任何一门科学中都更加顽固。于是,他得出结论:“资本是不取决于劳动的价值源泉”,(第55页)其实他至多可以作这样的结论:资本在某一特殊部门中实现的剩余价值,不取决于该特殊资本所使用的劳动量。[1089]

[1090]拉姆赛的错误观点在这里尤其令人奇怪,是因为他一方面理解到可以说是剩余价值的自然基础,而另一方面,在一个场合断定,剩余价值的分配——剩余价值平均化为一般利润率——并不增加剩余价值本身。

[首先,拉姆赛说道:]

\begin{quote}{“利润的存在决定于物质世界的规律,按照这个规律,自然界的恩惠,在得到人们的劳动和技艺的配合和指引时,就会充分报答国民的劳动,使国民劳动提供的产品,除了以实物形式补偿已消耗的固定资本并繁衍受雇的工人的种族所绝对必需的数额以外,还有一个余额……”[第205页]}\end{quote}

{“繁衍工人的种族”,这也[1091]是资本主义生产的美妙结果!当然,如果劳动生产率只够再生产劳动条件和维持工人的生活,就不可能有余额;从而也就没有利润,没有资本。但是,自然界同下述事实是毫不相干的:不管是否存在这个余额,工人的种族在繁衍着,而余额采取利润的形式并在这个基础上“繁衍”资本家的种族——这一点拉姆赛本人也承认了,因为他宣称,“流动资本”(在他那里是指工资,雇佣劳动)不是生产的本质条件,而只是由“人民群众的可悲的贫困”产生的。他并没有得出资本主义生产“繁衍”这种“可悲的贫困”的结论,虽然他说到资本主义生产“繁衍工人的种族”并给工人留下恰恰够这种繁衍所必需的数额时,也承认了这一点。在上述意义上可以说,剩余价值等等是以某种自然规律为基础的,是以与自然界相互作用的人类劳动的生产率为基础的。但是,拉姆赛自己指出劳动时间的绝对延长是剩余价值的源泉(第102页),也指出由于工业进步而提高的劳动生产率是源泉。}

\begin{quote}{“……只要总产品中除去用于上述目的所绝对必需的以外还有一点儿余额,就有可能从产品总量中分离出一种属于另外一个阶级的叫做利润的特殊收入。”(第205页)“资本主义企业主作为一个特殊阶级的存在本身是取决于劳动生产率的。”(第206页)}\end{quote}

其次,当谈到工资的[普遍]提高引起一些部门价格上涨而使利润率平均化时,拉姆赛说道:

\begin{quote}{“工资的普遍提高引起一些产业部门价格上涨,决不能使资本主义企业主避免利润的减少,甚至一点也不减轻他们的总的损失,而仅仅促使比较平均地把这种损失分配在这个阶级的不同阶层之间。”(第163页)}\end{quote}

如果有一个资本家,他的葡萄酒是100个工人的产品(拉姆赛举的例子),另一个资本家,他的商品是150个工人的产品;当前者的葡萄酒卖得的价格和后者的商品一样,以便“使该项投资不比其他项投资获利少”时,那末显而易见,在葡萄酒和另一种商品中包含的剩余价值并没有因此增多,而只是平均地分配在不同的资本家阶层中。[1091]

[1089]拉姆赛还再次援引了李嘉图[价值决定于劳动时间这一规定]的“例外”。这些例外将要在我们的正文中谈到价值转化为生产价格时加以探讨。\endnote{马克思指《资本和利润》这一篇,他是在1861—1863年手稿第XVI本上开始写的(该本注明:1862年12月,而包括论拉姆赛、舍尔比利埃、琼斯那几章的第XVIII本则注明:1863年1月)。马克思本想在《资本和利润》这一篇的第二章考察价值转化为生产价格的问题,这一点从写完论拉姆赛一章后不久写的第二章计划草稿中可以看出(见本卷第1册第447页)。《资本和利润》这一篇以后发展为《资本论》第三卷。关于大卫·李嘉图所表述的价值决定于劳动时间这一规定的“例外”,马克思在《剩余价值理论》第二册详细地谈到过(见本卷第2册第191—224页;并参看本册第72页)。——第367页。}就是说,这里只简单地谈一下。假定不同生产部门中的工作日的长度(在不被劳动的强度、劳动的不愉快等等抵销的情况下)相等,或者更确切些说,假定剩余劳动以及剥削率相等,那末,剩余价值率,只有在工资提高或降低时才可能发生变动。剩余价值率的这种变动,亦即工资的提高或降低,依资本有机构成的不同而对商品的生产价格发生不同的影响。可变部分大于不变部分的资本,当工资降低时,取得的剩余劳动多于不变部分大于可变部分的资本,而当工资提高时,占有的剩余劳动则少于后者。可见,工资的提高或降低对两个部门的利润率发生相反的作用,或者说,造成对一般利润率的两种相反的偏离。因此,为了维持一般利润率,在工资提高时,前一类商品的价格将上涨,后一类商品的价格则下降。(当然,各类资本只是按照它所使用的活劳动同所花费的全部资本量的比例直接受工资波动的影响。)相反,在工资下降时,前一类商品的价格将下降,后一类商品的价格则上涨。

严格说来,这一切在考察价值向生产价格的最初转化和一般利润率的最初形成时,是不必探讨的,因为更确切地说,这里涉及的问题是,工资的普遍提高或降低怎样影响受一般利润率调节的生产价格。

这种情况跟固定资本和流动资本之间的区别就更无关系了。银行家和商人几乎只使用流动资本,但很少使用可变资本,就是说,他们花在活劳动上的资本是比较少的。相反,矿主使用的固定资本比起缝纫业资本家不知要大多少倍。但是,他是否也按同样的比例使用活劳动,那就大成问题了。只是因为李嘉图把这个特殊的、比较不重要的情况作为区别生产价格和价值的唯一事例(或象他错误地表述的那样,作为价值决定于劳动时间这一规定的例外)提出来,并且是以固定资本和流动资本的区别的形式提出的,所以这一谬误才作为重要的教条——而且是以错误的形式——进入以后的全部政治经济学。(应该同矿主对比的不是缝纫业主,而是银行家和商人。)

[拉姆赛说道:]

\begin{quote}{“工资的提高受劳动生产率的限制。换句话说……一个工人劳动一天或一年所得到的,决不可能多于他依靠财富的其他任何源泉在这个时间里所能生产出来的……他的报酬必定低一些,因为总产品中的一部分始终要补偿固定资本〈按照拉姆赛的意思,就是不变资本:原料和机器等等〉及其利润。”(第119页)}\end{quote}

在这里,他把两个不同的东西混起来了。在日产品中包含的“固定资本”量,不是工人日劳动的产品,或者说,由一部分实物形式的产品代表的这部分产品价值,不是日劳动的产品。而利润倒确实是工人的这种日产品,或者说这种日产品的价值的扣除部分。

如果说,拉姆赛没有研究清楚剩余价值的本质,尤其是,他在价值同生产价格的关系上,在剩余价值转化为平均利润上,完全拘泥于旧的偏见,那末,他从自己对于固定资本和流动资本的理解中,反而得出了另外一个正确的[1090]结论。

在谈这一点之前,先再引证一段话:

\begin{quote}{“价值不仅要与实际消费掉的资本,而且要与还未变动的资本成比例,一句话,要与所使用的全部资本成比例。”(第74页)}\end{quote}

这段话的意思应该是:利润,从而生产价格,要与所使用的全部资本成比例,而价值则显然不能随着没有加入产品价值的资本部分发生变动。

[拉姆赛从自己对于固定资本和流动资本的理解中,得出如下的结论:]

随着社会的进步(即资本主义生产的进步),资本的固定部分靠缩小流动部分,即缩小用于劳动的部分而增大。因此,随着财富的增加或资本的积累,对劳动的需求相对地减少。在工业中,生产力的发展给工人带来的“祸害”是暂时的,但它们会一再重复。在农业中,特别是在耕地变为牧场时,这些祸害则是永久的。总的结果是:随着社会的进步,即随着资本的发展(也就是随着国民财富的发展),这种发展对工人状况的影响越来越小,换句话说,按照一般财富增长即资本积累的比例,或者同样可以说,按照再生产规模扩大的比例,工人状况相对恶化。我们看到,这个结论和亚·斯密的素朴见解或庸俗政治经济学的辩护论见解大不相同。在亚·斯密那里,资本的积累是和对劳动需求的增长,和工资的不断提高,从而和利润的下降等同的。在他那个时候,对劳动的需求,确实是至少和资本的积累按同样比例增长,因为当时工场手工业还占支配地位,而大工业则处在襁褓之中。

[拉姆赛说道:]

\begin{quote}{“对劳动的需求仅仅取决于〈直接地、不需任何媒介地〉流动资本量。”(第87页)〈这是拉姆赛的同义反复,因为在他那里流动资本等于花在工资上的资本。〉“随着文明的进步,国家的固定资本靠减少流动资本而增长。”(第89页)“因此,对劳动的需求,并不总是随资本的增长而增长,至少不是按同样的比例增长。”(第88页)“只有当流动资本由于新的发明而比原来数额增多时}\end{quote}

{在这里,再次流露出这样一个错误观点:似乎生活资料总量的增多和生活资料中供工人用的部分的增多是一回事},

\begin{quote}{对劳动的较大需求才会出现。那时需求会提高,但并不是同总资本的积累成比例地提高。在工业十分先进的国家,固定资本同流动资本相比总是越来越大。因此,在社会进步的过程中,用于再生产的国民资本的每次增加,对工人状况的影响会越来越小。”(第90—91页)“固定资本的每次增加,都是靠减少流动资本,亦即靠减少对劳动的需求来达到的。”(第91页)“机器的发明给工业中在业工人人口带来的祸害,可能只是暂时的,但是它们会经常重复发生,因为新的改进经常推动劳动的节约。”[第91页]}\end{quote}

在工业中这些祸害所以是暂时的,照拉姆赛看来(第91—92页),是由于以下原因:[第一,]使用新机器的资本家,得到超额利润;因此他们进行节约以及扩大资本的能力增长了。节约下来的一部分也作为流动资本使用。第二,工业品的价格按照生产费用减少的比例下跌;因此,消费者节约了,从而也促进了资本的积累,一部分资本可能进入商品生产费用减少的工业部门。第三,这些产品的价格的下跌,增大了对它们的需求。

\begin{quote}{“可见,尽管机器会使相当数量的人失业,然而,经过一段或长或短的时间,这些人,甚至更大数量的工人,可能重新被雇用。”(第92—93页)“在农业方面,情况完全不同。对原料的需求增长得不象对工业品的需求那么快……对农村人口来说,耕地变为牧场是最致命的……从前用以养活工人的基金,现在几乎全部用在牛、羊和固定资本的其他要素上。”(第93页)[1090]}\end{quote}

[1091]拉姆赛正确地指出:

\begin{quote}{“工资和利润一样,都应该看成成品的一部分,从国民的观点看来,这部分与成品的生产费用是完全不同的。”(第142页)“固定资本……撇开它被使用的结果不谈……是一种纯粹的损失……除所消费的固定资本外,只有劳动(不是工资,不是对劳动支付的东西)是生产费用的要素。劳动是一种牺牲。它在一个经济部门花费的越多,留给另一个经济部门的就越少。因此,如果把劳动用在无收益的事业中,国民就要由于主要的财富源泉的滥用而蒙受损失……劳动报酬并不构成费用要素。”(第142—143页)}\end{quote}

(把劳动,而不是把有酬劳动或者说工资作为价值要素,这是十分正确的。)

拉姆赛正确地描述了实际的再生产过程。

\begin{quote}{“怎样才能把产品和花费在产品上的资本加以比较呢?……如果指整个国民而言……那末很清楚,花费了的资本的各个不同要素应当在这个或那个经济部门再生产出来,否则国家的生产就不能继续以原有的规模进行。工业的原料,工业和农业中使用的工具,工业中无数复杂的机器,生产和贮存产品所必需的建筑物,这一切不仅应当成为一国所有资本主义企业主的全部预付的组成部分,而且应当成为该国总产品的组成部分。因此,总产品的量可以同全部预付的量相比较,因为每一项物品都可以看成是与同类的其他物品并列的。”(第137—139页)“至于单个资本家}\end{quote}

{这是错误的抽象。国民只是作为资本家阶级存在,而整个这一阶级的活动和单个资本家的活动完全一样。两种考察方式的区别仅仅在于,一种是把使用价值,另一种则把交换价值紧紧抓住并孤立起来},

\begin{quote}{由于他不是以实物来补偿自己的支出,他的支出的大部分必须通过交换来取得,而交换就需要一定份额的产品,由于这种情况,单个资本主义企业主不得不把更大的注意力放在自己产品的交换价值上,而不是放在产品的量上。”(第145—146页)[1092]“他的产品的价值愈高于预付资本的价值,他的利润就愈大。因此,资本家计算利润时,是拿价值同价值相比,而不是拿量同量相比。这一点是在国民和个人计算利润的方式上应该看到的第一个区别。”}\end{quote}

{假定国民跟全体资本家有所不同,国民在某种意义上也可以把价值同价值这样相比:国民可以计算用于补偿消费了的不变资本部分和加入个人消费的产品部分的全部劳动时间,以及花在创造用来扩大再生产规模的余额上的劳动时间。}

\begin{quote}{“第二个区别是,由于资本主义企业主总是向工人预付工资,而不是用成品支付工资,企业主就把这个预付看成和所消费的固定资本一样,是他的支出的一部分,虽然从国民的观点看来,工资并不是费用要素。”}\end{quote}

{事实上,对总再生产过程来说,这个区别也消失了。资本家总是用成品支付工资,就是说,他用工人昨天生产出来的商品支付工人明天的工资;换句话说,他以工资形式给工人的,事实上只是一种凭证,用来取得在未来制成或者是接近制成,就是说到被购买时要最后完成的产品。在再生产中,即在生产的连续过程中,仅仅作为表面现象的预付,也就消失了。}

\begin{quote}{“因此,资本主义企业主的利润率,取决于他的产品价值超过预付资本——固定资本和流动资本——的价值的余额。”(第146页)}\end{quote}

{从“国民的观点”来看也是这样。资本主义企业主的利润始终取决于他本人为产品所支付的,而不管他支付工资时产品是否制成。}

拉姆赛的功绩在于,首先,他反驳了自亚·斯密以来广为流行的错误观点,即认为总产品的价值分解为各种名称不同的收入;其次,他以双重方式,通过工资率即剩余价值率和通过不变资本的价值,决定利润率。但是,他犯了一个正好和李嘉图相反的过错。李嘉图想强行使剩余价值率同利润率相等。拉姆赛则相反,提出了利润率的二重性的规定:(1)利润率决定于剩余价值率(即工资率)和(2)利润率决定于这个剩余价值对总预付资本之比(就是说,事实上拉姆赛是以不变资本与总资本之比决定利润率的),——并且未弄清问题实质而把这二重性的规定看成是决定利润率的两个平行的情况。他没有看到剩余价值在成为利润之前所发生的转化。因此,如果说李嘉图为了贯彻价值理论,试图强行把利润率归结为剩余价值率,那末拉姆赛就是试图把剩余价值归结为利润。此后我们将看到,他叙述不变资本价值对利润率的影响所用的方法,是很不充分的,甚至是错误的。

[拉姆赛说:]

\begin{quote}{“利润的上升或下降,同总产品或它的价值中用来补偿必要预付的那个份额的下降或上升成比例……因此,利润率决定于以下两个因素:第一,全部产品中归工人所得的那个份额;第二,为了以实物形式或通过交换来补偿固定资本而必须储存的那个份额。”(第147—148页)}\end{quote}

因此,换句话说,利润率决定于产品价值超过流动资本和固定资本总额的余额;也就是说,决定于第一,流动资本和第二,固定资本在全部产品价值中所占的份额。如果我们知道这笔余额是哪里来的,问题就简单了。但是,如果我们只知道利润取决于余额对这些支出的比例,我们就可能得出关于这笔余额的来源的极其错误的看法,例如,就可能象拉姆赛那样,以为它部分地来源于固定(不变)资本。

\begin{quote}{[1093]“构成固定资本的各种物品在生产上变得容易,肯定会使这个份额\authornote{即总产品中用来补偿“固定资本”的份额。——编者注}减少从而提高利润率,就象在前一种场合,由于用以维持劳动的流动资本要素的再生产变得便宜而使利润率提高一样。”(第164页)}\end{quote}

例如,以租地农场主为例:

\begin{quote}{“无论总产品的数额是多少,其中用来补偿在生产中以不同形式消费了的全部东西的那个量,不应当有任何变动。只要生产以原有的规模进行,这个量就必须看成是不变的。因此,总产品越多,租地农场主必须为上述目的拨出的份额必然越小。”(第166页)“生产食物和诸如亚麻、大麻、木材之类的原料的租地农场主,把这些东西再生产出来越容易,他的利润提高得就越多。租地农场主的利润由于他的产品数量的增加而提高;产品的总价值保持不变,但是,租地农场主用来补偿他可以自己供应自己的各种固定资本要素在总产品中,从而在它的价值中的份额,比从前减少了。至于工业资本家则会由于他的产品具有较大的购买力而获利。”(第166—167页)}\end{quote}

假定收成等于100夸特,种子等于20夸特,即等于收成的1/5。再假定第二年收成增加一倍(支出同量的劳动);现在它等于200夸特。如果生产规模保持原有水平,种子就仍然等于20夸特,但现在这20夸特只占收成的1/10。然而必须考虑到,先前100夸特的价值等于现在200夸特的价值;也就是说,前一年收成的一夸特[按价值来说]等于后一年收成的两夸特。在前一场合,剩余80夸特,在后一场合,剩余180夸特。因为这里讨论的是不变资本价值的变动对利润率的影响问题,不涉及工资,所以就假定工资在价值上保持不变。这样一来,在前一场合工资等于20夸特,在后一场合等于40夸特。最后,再假定租地农场主不能以实物形式再生产的其他不变资本组成部分,在前一场合其价值等于20夸特,从而在后一场合等于40夸特。

于是,我们得出如下的计算数字:

(1)产品=100夸特,种子=20夸特。其他不变资本=20夸特,工资=20夸特,利润=40夸特。

(2)产品=200夸特,种子=20夸特。其他不变资本=40夸特,工资=40夸特,利润=100夸特。这100夸特在价值上等于(1)中的50夸特。因此,在这个场合有10夸特的超额利润。

可见,在这里,由于不变资本的价值变动,不[仅]利润率,而且利润本身也提高了。尽管在(1)、(2)两个场合工资是相同的,利润与工资之比,即剩余价值率,却提高了。然而这只是表面现象。在第二个场合,利润中首先有80夸特——这80夸特[按价值来说]等于(1)的40夸特——对工资之比与第一个场合相同;其次有20夸特——这20夸特只等于

(1)的10夸特——从不变资本转化为收入。

但是这个计算正确吗?我们必须假定,第二个场合[收成加倍]的结果属于下一年,尽管劳动是在和第一个场合相同的条件下进行的。为了更清楚起见,我们假定,在第一个场合1夸特等于2镑。在第二个场合,租地农场主为得到200夸特的收成,支出如下:种子20夸特(40镑),其他不变资本20夸特(40镑),工资20夸特(40镑)。总计120镑,而产品=200夸特。在第一个场合他也只是支出120镑(60夸特),而产品是100夸特,等于200镑。剩下的是利润80镑或40夸特。因为第二个场合的200夸特[和第一个场合的100夸特一样]是同量劳动的产品,所以也只值200镑。因此在第二个场合剩下的也只有80镑利润,但是这80镑现在等于140夸特\endnote{如果说,在最初计算中曾假定,在第二个场合,用于劳动工具和劳动力的生产费用,已经按每夸特谷物降低了一半的价值(由于收成增加一倍而造成)计算,那末,马克思现在注意到下面这种情况:每夸特谷物价值的这种降低,只是在第二年秋天才发生,而在秋天以前每夸特的价值要高一倍。因此,如果在最初计算中,第二个场合的生产费用以20c+40c+40v=100夸特的数额表示,那末,它现在就用和第一个场合相同的数额即20c+20c+20v=60夸特来表示。因为第二个场合的收成等于200夸特,所以剩下的是利润140夸特。——第376页。}。因此,每夸特[对租地农场主来说]仅仅值4/7镑而不是1镑。换句话说,每夸特的价值从2镑降到4/7镑,即减少1+(3/7)镑,而不是象上面第二个场合与第一个场合对比中所假定的那样,从2镑降到1镑,或者说只减少一半。

在第二个场合,租地农场主的全部产品等于200夸特,价值200镑。但其中120镑补偿他在生产上支出的60夸特,每夸特花费他2镑。因此,剩下的是80镑利润,等于剩下的140夸特。这是怎么回事呢?在第二个场合,每夸特值1镑,但是在生产上支出的60夸特,每夸特则值2镑。它们使租地农场主花费的,就等于他从新的收成中支出120夸特。这样,剩下的140夸特值80镑,或者说,具有的价值并不比第一场合剩下的40夸特多。诚然,租地农场主对这200夸特的每1夸特都是按1镑出卖的(假定他出卖自己的全部产品),这样他就卖得200镑。但是在这200夸特中,有60夸特,每夸特花费他2镑;因此,剩下的每1夸特只给他提供4/7镑[或者说,约1/2镑]。

如果他现在重新支出[种子]20夸特(10镑[按每夸特1/2镑计算])、工资40夸特(20镑)和其他不变资本40夸特(20镑),也就是总共支出100夸特以代替从前的60夸特,而得到180夸特的收成,那末这180夸特所具有的价值和从前100夸特所具有的价值[按每夸特1镑计算]是不相等的。诚然,他使用了同从前一样多的活劳动,从而[1094]可变资本的价值和从前相同,剩余产品的价值也和从前相同。但是,他支出的物化劳动却较少,因为同样的20夸特,从前值20镑,现在只不过值10镑。

因此,得出如下的计算数字:

\todo{}

第一个场合的产品等于100夸特(100镑)。

第二个场合的产品等于180夸特(90镑)。

然而[尽管产品的价值下降],利润率提高了:因为在第一个场合40镑利润是靠60镑支出获得的,而在第二个场合40镑利润是靠50镑支出获得的。前者为66+(2/3)%,后者为80%。

无论如何,利润率的提高不象拉姆赛假定的那样,是由于价值保持不变。因为在产品生产上支出的劳动的一部分,即不变资本(在这里是种子)中包含的那部分劳动减少了,所以,如果生产以原有的规模继续进行,产品的价值就要下降,正象100磅纱中包含的棉花降价时这些纱的价值也要下降一样。但是,可变资本对不变资本之比提高了(虽然可变资本的价值并没有提高)。换句话说,所花费的资本总额对剩余价值之比降低了。利润率的提高就是由此而来的。

如果拉姆赛所说的是正确的,也就是说,如果价值保持不变,那末利润,利润量从而还有利润率就会提高。利润率的单独提高是根本谈不上的。

但是,[不变资本的价值变动对利润率的影响]问题就特殊情况[一部分不变资本以实物形式得到补偿]来说,还没有解决。这种特殊情况在农业中的表现如下:

一定量的种子在收成中是按产品的原价计算的,而且这一部分以实物形式加入收成。其余的支出通过按原价出卖谷物而得到抵补。通过这些原有的支出,产品增加一倍。比如说,在前面讲到的场合,支出种子20夸特(40镑),其余支出等于40夸特(80镑),现在的收成是200夸特而不是原来的100夸特(200镑),在这100夸特中40夸特(80镑)是全部支出60夸特(120镑)的利润。这一次收成所支出的和上一次支出的绝对量相等,都是60夸特,价值120镑,但是现在的余额不是40夸特,而是140夸特。在这里,实物形式的余额大大增加。但是由于在两个场合支出的劳动是相同的,所以现在200夸特具有的价值并不比从前100夸特多。因此,这200夸特值200镑,即每夸特的价值从2镑降为1镑。但是,既然余额等于140夸特,那末看来它应该值140镑,因为其中每一夸特所值同别的一夸特是完全一样的。

如果我们先撇开再生产过程来观察问题,假定租地农场主不再经营,把全部产品出卖,那问题就会变得再简单不过了。那时,为了抵补(补偿)自己的120镑支出,他实际上应该卖出120夸特。这样预付资本就得到抵补。因此,余额是80夸特,而不是140夸特,并且因为这80夸特等于80镑,所以它所值和第一个场合的余额完全一样。

然而,再生产过程使问题多少起了变化。就是说,租地农场主从自己的产品中以实物形式补偿20夸特的种子。[按价值来说,]这20夸特使他得到40夸特产品的补偿。但是,在再生产过程中,他仍旧只须以20夸特的实物来支付。他的其余支出[以夸特表示]随着每夸特价值减少而相应增加(假定工资不降低)。为了补偿不变资本的其余部分,他现在需要40夸特,而不是原来的20夸特,为了补偿工资,也需要40夸特,而不是20夸特。从前他支出60夸特,现在一共要支出100夸特;但是,他不必按谷物减价所要求的那样,支出120夸特,因为他现在是用价值20镑的20夸特补偿(因为这里只涉及到这20夸特的使用价值)从前值40镑的20夸特[种子]。这样一来,他显然[1095]赚了现在值20镑的这20夸特。他的余额不是80镑,而是100镑,不是80夸特,而是100夸特。(如果按原有价值以夸特表示这个余额,它现在就不是40夸特,而是50。)这是一个不容置疑的事实,如果谷物的市场价格不因谷物丰足而下降,他就可以按新价值多出卖20夸特,赚得20镑。

他之所以通过再生产用相同的支出取得这20镑余额,是因为劳动的生产率高了,虽然剩余价值率在这里并没有提高,就是说,工人提供的剩余劳动并没有比从前多,或者他从产品的再生产部分(代表活劳动的部分)中得到的份额并不比从前少。相反,可以这样假定,工人在再生产中得到40夸特,而从前只得到20夸特。可见,这是一个独特的现象。这个现象没有再生产是不会发生的,但是,它的发生是同再生产联系着的,而且它之所以发生,是因为租地农场主以实物形式补偿自己的一部分预付。在这种情况下,不仅利润率会提高,而且利润也会增加。(至于再生产过程本身,租地农场主现在或者可以按原有规模继续进行再生产,这时,如果收成又是同样好,产品价格就下降——,因为一部分不变资本的价值减少了,——但是利润率会提高;或者他可以扩大生产规模,用相同的支出扩大播种,这时,利润和利润率都会提高。)

现在谈工厂主。假定他在棉纱生产上支出100镑,利润为20镑。因此产品等于120镑。假定在100镑中用于棉花的支出是80镑。如果现在棉花的价值下降一半,他就只须在棉花上支出40镑,其余一切则支出20镑,就是说,总共须支出60镑,而不是100镑。利润仍旧等于20镑。总产品等于80镑(假定他不扩大生产规模)。这样一来,40镑留在他的腰包里;他可以把这40镑自己花掉,或者作为追加资本投放。在后一种情况下,按照新的生产规模,他将在棉花上[追加]支出26+(2/3)镑,在劳动等等上[追加]支出13+(1/3)镑。[40镑追加支出的]利润是13+(1/3)镑。总产品现在=60+40+33+(1/3)=133+(1/3)镑。

可见,这里的问题不在于租地农场主以实物形式补偿自己的种子,因为工厂主所用的棉花是购买的,而不是用自己的产品补偿的。可见,这种现象可以归结如下:从前作为不变资本被束缚的那部分资本中,有一部分游离出来,或者说,一部分资本转化为收入。如果在再生产过程中花费的资本同从前正好一样多,其结果就会同在原有生产规模上使用追加资本完全一样。因此,这是一种由于提供资本各组成部分的生产部门的生产率提高而发生的积累。然而,原料价值的这种下降,如果是丰收造成,就会被歉收时原料的涨价抵销。因此,在一次或几次丰收时以上述方式游离出来的资本,在某种程度上成了预防歉收的准备资本。例如,某个工厂主的[固定资本]周转期为12年,他就必须这样安排,使他在这12年期间至少能够以同样的规模继续生产。因此,势必估计到,补偿[原料]时所支付的价格会发生波动,而且在比较长的年限内得到平衡。

资本各组成部分价格的上涨所起的作用,跟它们价格的下降所起的作用相反。(在这里,我们把可变资本撇开不谈,虽然工资降低时,必须支出的可变资本按价值来说减少了;而工资提高时必须支出的可变资本增多了。)现在为了能按原有规模继续生产,必须支出更多的资本。因此,撇开利润率下降不谈,这里必须使用准备资本,或者把一部分收入转化为资本,虽然它不是作为追加资本起作用。

在一种场合[在价格下降时]发生了积累,虽然预付资本的价值保持不变(但是它的物质组成部分增加了)。资本的价值增殖率和利润的绝对量增长了,因为这就同在原有生产规模上投入追加资本一样。在另一种场合[在价格上涨时],积累的发生是由于预付资本的价值,即总产品价值中执行资本职能的部分增长了。但是资本的物质组成部分没有增加。利润率下降了。(利润量只有在现在雇用的工人数量和从前不一样,或者工人的工资也提高的情况下才会减少。)

上述资本转化为收入的现象是值得注意的,因为它造成一种假象,似乎利润量的增加(或者反之——减少)不取决于剩余价值量。我们曾经看到\authornote{见本卷第2册第518—523页。——编者注},在[1096]一定的情况下,这种现象可以解释部分地租。

在前面考察的场合(等于20镑的20夸特余额不是立即重新用于扩大生产规模,也就是说,不是用于积累),一笔20镑的货币资本游离出来了。这是一个例子,说明尽管商品价值量保持不变,仍然可以有多余的货币资本从再生产中沉淀下来。这是由于一部分先前以固定(不变)资本形式存在的资本转化为货币资本而产生的。

前面讲到的现象[一部分资本转化为收入]和[拉姆赛的]利润率的规定是多么不相干,这一点,只要我们设想一个在新的生产条件下开始经营的租地农场主(或工厂主),就很清楚了。以前,为了开始经营需要120镑的资本:40镑用于购买20夸特种子,40镑用于其他不变资本要素,40镑用于支付工资。他的利润是80镑。80镑比120镑支出,即2比3,等于66+(2/3)%。

现在,租地农场主预付20镑来购买20夸特种子,40镑象以前一样购买其他不变资本,40镑支付工资,这样,他的资本支出为100镑。而80镑利润与100镑支出之比,为80%。利润量保持不变,但是利润率提高了20%。因此,我们看到,种子价值(或者说,补偿种子时所支付的价格)的下降本身同利润的增加毫无关系,而仅仅包含利润率的提高。

此外,租地农场主(或在另一场合,工厂主)本身也不把这件事看作是他的利润增加,而看作是一部分以前被束缚在生产中的资本游离出来。而且他这样考察问题是由于下面的简单计算。以前在生产上预付的资本等于120镑,现在等于100镑,而20镑则作为闲置资本、作为可以随意使用的货币留在租地农场主的腰包里。但是,在这两种场合,他的全部资本都只等于120镑,就是说,他的资本量没有增加。诚然,资本的1/6从那种被束缚于再生产过程的形式中游离出来,起着同追加资本一样的作用。

拉姆赛没有抓住这个问题的实质,因为他根本没有弄清价值、剩余价值和利润之间的关系。

\centerbox{※     ※     ※}

拉姆赛正确地阐述了机器等等怎样——在它们影响可变资本的范围内——对利润和利润率发生作用。就是说,它们通过降低劳动能力的价值,通过增加相对剩余劳动,或者——就总再生产过程来考察——通过减少总产品中用以补偿工资的份额发生作用。

\begin{quote}{“在那些不加入固定资本的商品的生产中,劳动生产率的提高或降低,不能对利润率产生任何影响,除非使总产品中用以维持劳动的份额发生变动。”(同上,第168页)“如果工厂主通过机器的改良使他的产品增加一倍,那末,他的商品的价值,归根到底,必然按商品数量增加的比例减少。”}\end{quote}

{这里假定,事实上,把机器的磨损计算在内,增加了一倍的产品数量所值并不比以前此数的一半多。不然的话,单位产品的价值会下降,但不是按产品数量增加的比例下降。产品数量可能增加一倍,而它的价值、单位商品的价值(在总产品价值提高的情况下),却可能不是从2降到1,而只是从2降到1+(1/4),等等。}

\begin{quote}{“……工厂主不过是由于他可以使工人的衣着更便宜,从而使工人在总收益中所得的份额更小,才会获利……租地农场主也不过是{由于工厂主那里的劳动生产率提高}在他的一部分支出用于供给工人衣着而现在他能够比较便宜地买到这一部分的情况下,才会获利,就是说,他用和工厂主相同的方式获利。”(第168—169页)}\end{quote}

不变资本各组成部分的价值的降低[或提高]所以影响利润率,是因为它影响剩余价值与所花费的资本总额的比例。至于工资的降低(或相反)影响利润率,则是因为它直接影响剩余价值率。

例如,假定在前面讲到的场合(假定租地农场主是亚麻种植业者),种子的价格保持不变,等于40镑(20夸特),花在其他不变资本上的仍旧是40镑(20夸特),但是工资——即同样工人人数的工资——从40镑降到20镑(从20夸特降到10夸特)。在这个场合,[新创造的]价值(工资加剩余价值)量保持不变。因为工人人数相同,他们的劳动仍旧体现在等于40镑+80镑=120镑的价值中。但是现在这120镑中,归工人的是20镑,属于剩余价值的是100镑。{换句话说,这里假定没有进行过任何能影响这个部门的在业工人人数的改良。}

现在预付资本是100镑,而不是120镑,这和种子价值降低一半的场合一样。但是现在利润是100镑,也就是说利润率是100%,而在另一场合[种子价值降低],所花费的资本也从120降为100镑,但利润率是80%。跟这另一场合一样,现在[1097]有20镑或1/6的资本游离出来。但是在这另一场合剩余价值保持不变,等于80镑(就是说,剩余价值率为200%,因为工资是40镑)。现在剩余价值量提高到100镑(就是说,剩余价值率提高到500%,因为工资是20镑)。

在这里,不仅利润率,而且利润本身也提高了,因为剩余价值率,从而剩余价值本身提高了。这就是现在这个场合不同于另一个场合的地方,而拉姆赛并没有看到这一点。只要利润的增长没有被由于不变资本的价值同时发生变动而造成的利润率相应降低所抵销,这种增长总是要出现的。例如在前面讲到的场合,所花费的资本是120镑,利润是80镑,即66+(2/3)%。在我们这个场合,所花费的资本等于1利润是100镑,即100%。但是,如果由于不变资本的价格发生变动,支出从100增加到150镑,那末,利润虽从增长到100镑,却仍旧只提供66+(2/3)%的利润率。

[拉姆赛继续说道:]

\begin{quote}{“既不加入固定资本也不加入流动资本的那些商品,不可能由于它们的生产率发生任何变化而影响利润。这类商品是各式各样的奢侈品。”(第169—170页)“资本主义企业主由于奢侈品充裕而得到好处,因为他们的利润将支配较大数量的奢侈品供他们个人消费;但是,这个利润的比率不会因这些商品的丰富或不足而受到任何影响。”(第171页)}\end{quote}

首先应该指出,一部分奢侈品可以作为不变资本要素进入生产过程。例如,葡萄进入葡萄酒的生产,金进入奢侈品的生产,金刚石用于磨玻璃,等等。但是,拉姆赛既然说“不加入固定资本的商品”,就把这种情况排除了。这样,他的下一句话:“这类商品是各式各样的奢侈品”,是错误的。

然而,说到奢侈品工业的劳动生产率,它增长的原因也只能和其他所有生产部门一样:要么由于取得奢侈品原料的自然仓库如矿山、土地等等的生产率提高了,或者发现较富饶的这类自然仓库;要么由于采用分工,或者特别是使用机器(以及改进的工具)和自然力。{工具的改进和工具的分化一样属于分工。}(化学过程也不应当忘记。)

现在假定,通过机器(或化学过程)缩短了奢侈品的生产时间;生产它们所需要的劳动减少了。这一点对于工资,对于劳动能力的价值不会有丝毫的影响,因为奢侈品不加入工人消费(至少从来不加入他们的决定其劳动能力价值的那部分消费)。{奢侈品生产时间的缩短对工人的市场价格可能产生影响,如果工人因此被抛到街头,从而使劳动市场上的供给增加的话。}因此,奢侈品生产时间的缩短,对剩余价值率不产生影响,从而在利润率决定于剩余价值率的情况下,对利润率也不产生影响。可是,只要它触动剩余价值量,或者触动可变资本对不变资本以及对总资本之比,它当然会对利润率产生影响。

例如,[在某奢侈品的生产中]如果从前雇用20个工人,现在使用机器只需要10个工人,那末剩余价值率实际上并没有受到任何影响。奢侈品变得便宜并不能使工人的生活费用变得便宜。为了再生产自己的劳动能力,他仍然需要和从前相同的劳动时间。

{因此,事实上,生产奢侈品的工厂主力图把劳动的报酬压到劳动的价值之下,压到它的最低限度之下,他所以能够做到这一点,是因为相对的人口过剩(例如刺绣女工的情况),而这种过剩又是由于其他生产部门劳动生产率的增长造成的。或者生产奢侈品的工厂主力图延长绝对劳动时间——在其他部门也是这样;在这种情况下,他实际上创造了绝对剩余价值。只有一点是正确的:奢侈品工业的劳动生产率的提高不能压低劳动能力的价值,不能创造相对剩余价值,总之,不能创造由劳动生产率本身的增长决定的剩余价值形式。}

但是,剩余价值量决定于两个因素:[第一,]剩余价值率,即单个工人的剩余劳动(绝对的或相对的);第二,同时使用的工人人数。因此,如果奢侈品工业的劳动生产率的增长使一定量资本所推动的工人人数减少,它就会使剩余价值量减少;从而在其他所有条件保持不变的情况下,它也会使利润率降低。如果工人人数减少了,或者虽然工人人数保持不变但用在机器和原料上的资本增加了,就是说,在可变资本与总资本相比出现任何减少,而这种减少在这里[根据假定]没有被工资的下降拉平或部分抵销时,利润率也会下降。但是,因为这个部门的利润率,和其他任何部门的利润率一样,也[1098]参加一般利润率的平均化,所以,奢侈品工业的劳动生产率的提高在这里会引起一般利润率的下降。

相反,如果劳动生产率的提高不是在奢侈品工业本身,而是在向它提供不变资本的那些部门,那末,奢侈品工业的利润率就会提高。

{剩余价值(也就是它的大小、它的量、它的总额)决定于剩余价值率乘在业工人人数。有些情况可能在同一个方向或者在相反的方向同时影响两个因素,也可能仅仅影响其中一个因素。撇开工作日的绝对延长不谈,奢侈品工业的劳动生产率的提高只影响在业工人人数。因此,其必然结果是剩余价值量减少,从而利润率下降——即使不变资本没有增加。如果不变资本增加了,那末,减少了的剩余价值则按照增大了的总资本来计算。}

\centerbox{※     ※     ※}

拉姆赛比其他人更接近于正确地理解利润率。因此,[传统观念的]缺陷在他那里也比在其他人那里表现得明显。他提出了所有的要点,但是提得片面,因而是错误的。

拉姆赛用以下的话总括了他对利润的观点:

\begin{quote}{“因此,单个资本家的利润率决定于下述因素:(1)生产工人衣食等等生活必需品的劳动的生产率;(2)生产加入固定资本的物品的劳动的生产率;(3)实际工资率{实际工资在这里应该是指工人得到的生活必需品等等的数量,而不管属于这些必需品的商品的价格如何}。上述第一个和第三个因素的变化,通过改变总产品中归工人的份额而影响利润。第二个因素的变化,则通过改变用于补偿——直接或经过交换——生产中消费了的固定资本的份额而影响利润;因为利润实质上是个份额问题。”(同上,第172页)}\end{quote}

拉姆赛公正地指责李嘉图(尽管他自己的说明也有缺陷):

\begin{quote}{“李嘉图忘记了,全部产品不仅分为工资和利润,而且还必须有一部分补偿固定资本。”(第174页注)}\end{quote}

\centerbox{※     ※     ※}

{只要对积累,即对剩余价值转化为资本进行初步考察,就可以看到,全部剩余劳动表现为资本(不变资本和可变资本)和剩余劳动(利润、利息、地租)。因为在剩余价值向资本的转化中显出:剩余劳动本身采取资本的形式,工人的无酬劳动作为客观的劳动条件的总和同工人相对立。在这种形式中,客观的劳动条件的总和作为他人的财产同工人相对立,以致作为工人劳动的前提的资本看来似乎和这种劳动无关。资本表现为现成的价值量,而工人只是必须增加它的价值。至于说到剥削,则不是指工人过去劳动的产品(也不是指以下任何情况,这种情况影响或提高过去劳动的[产品的]价值,而与这种过去劳动所进入的特殊劳动过程无关)或这种产品的补偿,而始终只是指工人现在劳动被剥削的方式和程度。只要单个资本家按原有的(或扩大的)规模继续生产,资本的补偿就好象是一种对工人没有影响的行为,因为即使劳动条件归工人所有,他自己也必须用总产品的一部分补偿这些劳动条件,以便按原有的规模继续再生产或者扩大再生产(而后者由于人口的自然增长也是必需的)。但是,资本的这种补偿在三个方面影响工人:(1)劳动条件作为不属于工人的财产,作为资本的永恒化,使工人作为雇佣工人的地位永恒化,从而使工人始终要用自己的一部分劳动时间白白为他人劳动的命运永恒化;(2)这些生产条件的扩大,换句话说,资本的积累,使得靠工人的剩余劳动为生的阶级的数量和人数增多;资本的积累通过使资本家及其同伙的相对财富增多而使工人的状况相对恶化,此外,还通过使工人的相对剩余劳动量增加(由于分工等等),使总产品中归结为工资的份额减少的办法使工人的状况恶化;(3)最后,由于劳动条件以愈来愈庞大的形式,愈来愈作为社会力量出现在单个工人面前,所以,对工人来说,象过去在小生产中那样,自己占有劳动条件的可能性已经不存在了。}

\tchapternonum{[(3)拉姆赛论“总利润”分为“纯利润”(利息)和“企业主利润”。在他关于“监督劳动”、“补偿风险的保险费”和“超额利润”等观点中的辩护论因素]}

[1099]拉姆赛把我仅仅称之为利润的东西称为总利润。他把这个总利润分为纯利润(利息)和企业主利润(企业主收入,产业利润)\authornote{[1130}{西尼耳先生的《大纲》和拉姆赛的《论财富的分配》是大致同时出版的,在后一著作[第二部分]第四章已经详尽地论述了利润分为“企业主利润”和“资本的纯利润”即“利息”;为什么这个在1821年和1822年已经是人所周知的利润划分却被认为是西尼耳先生发明的呢?——这一点只能这样来解释:西尼耳作为纯粹的现状辩护论者,从而作为庸俗经济学家,是深得罗雪尔先生同情的\endnote{马克思指德国庸俗经济学家罗雪尔的著作《国民经济学原理》1858年版第385页。罗雪尔在这里谈到利润分为企业主利润和利息时,引用了西尼耳的《大纲》。马克思指出利润分为“企业主利润”和“资本的纯利润”这一点早在1821和1822年就已经人所周知,这可能是指匿名著作《论马尔萨斯先生近来提倡的关于需求的性质和消费的必要性的原理》(1821年伦敦版〉第52—53页,以及霍普金斯的著作《关于调节地租、利润、工资和货币价值的规律的经济研究》(1822年伦敦版)第43—44页。——第389页。}。}[1130]]。

在一般利润率下降的问题上,拉姆赛同李嘉图一样,也和亚·斯密论战。他反驳亚·斯密说:

\begin{quote}{“诚然,资本主义企业主之间的竞争,可以使大大超过普通水平的利润平均化{这种平均化决不足以解释一般利润率的形成},但是,认为竞争会降低这个普通水平本身,则是错误的。”(第179—180页)“假定每一种商品(原料和成品)的价格,由于生产者之间的竞争而下降是可能的话,那末这一点决不会影响利润。每个资本主义企业主都会把他的产品卖较少的钱,但另一方面,他的每一项支出,不管它属于固定资本还是属于流动资本,都会相应地减少。”(第180—181页)}\end{quote}

拉姆赛也反对马尔萨斯:

\begin{quote}{“利润由消费者支付这种想法显然是十分荒谬的。消费者又是谁呢?他们必定是土地所有者、资本家、企业主、工人,或者领取薪金的人。”(第183页)“唯一能够影响一般总利润率的竞争,是资本主义企业主和工人之间的竞争。”(第206页)}\end{quote}

在最后这一句话里,表达了李嘉图的论点中正确的东西。利润率的下降可以不取决于资本和劳动之间的竞争,但是唯一能够使利润率下降的竞争,却是这种竞争。不过拉姆赛本人并没有给我们指出一般利润率具有下降趋势的原因。他所说的唯一东西是,——这一点是正确的,——利率的下降可以完全不取决于国内的总利润率。就是说:

\begin{quote}{“即使我们假定,借入资本除了用于生产之外,决不用于其他目的,那末,在总利润率没有任何变化的时候,利息仍然可能变化。因为,随着一个民族的财富不断增长,有一类人产生出来并不断增加,他们靠自己祖先的劳动{剥削,掠夺}占有一笔只凭利息就足以维持生活的基金。还有许多人,他们在青壮年时期积极经营,晚年退出,靠他们自己积蓄的钱的利息过安逸的生活。随着国家财富的增长,这两类人都有增加的趋势;这是因为那些在开始时已有相当资本的人,比那些在开始时只有少数资本的人,更容易获得独立的财产。因此,在老的富有的国家,不愿亲自使用资本的人所占有的国民资本部分,在社会全部生产资本中所占的比例,比新垦殖的贫穷的国家大。在英国,食利者阶级的人数是多么多啊!随着食利者阶级的增大,资本贷放者阶级也增大起来,因为他们是同一些人。只是由于这个缘故,利息在老的国家也必定有下降的趋势。”(第201—202页)}\end{quote}

关于纯利润(利息)率,拉姆赛说道:

\begin{quote}{“它部分地取决于总利润率,部分地取决于总利润分为利息和企业主利润的比例。这个比例取决于资本的贷出者和借入者之间的竞争。这种竞争受预期的总利润率的影响,但不是完全由它调节。竞争所以不是完全由它调节,一方面是因为有许多人借钱并不打算用在生产上,另一方面又因为全部国民资本中可贷出的份额,随着国家的财富而变化,不以总利润的任何变化为转移。”(第206—207页)“企业主利润取决于资本的纯利润,而不是后者取决于前者。”(第214页)}\end{quote}

[1100]除开前面提到的情况,拉姆赛还正确地指出:

\begin{quote}{“只有在文明程度已达到不必提出保证偿还贷款的要求的地方,借贷利息才是纯利润的尺度……例如在英国,目前我们不会考虑把承担风险的补偿加进利息中去,因为贷出的资金都有所谓良好的保证。”(第199页注)}\end{quote}

他在谈到他称之为资本主义企业主的产业资本家时,指出:

\begin{quote}{“资本主义企业主是财富的总分配者;他付给工人工资,付给[货币]资本家借贷利息,付给土地所有者地租。一方是企业主,另一方是工人、[货币]资本家和土地所有者。这两大类人的利益正好彼此相反。雇劳动、借资本和租土地的是企业主,他当然力图以尽可能低的报酬使用它们,而这些财富源泉的所有者则力求以尽可能高的价格出租它们。”(第218—219页)}\end{quote}

产业利润(监督劳动)。

总的说来,拉姆赛关于产业利润,特别是关于监督劳动的论述,是这部著作中提出的最合理的东西,尽管他的一部分论证是从施托尔希那里\endnote{马克思指施托尔希的著作《政治经济学教程》1823年巴黎版第1卷第3篇第13章。——第391页。}抄来的。

剥削劳动是要花费劳动的。就资本主义企业主所从事的劳动仅仅由于资本和劳动对立才成为必要这一点来说,这种劳动加入他的监工(工业军士)的费用,并且已经算在工资项下,这种情况跟奴隶监工和监工所用的鞭子的费用算在奴隶主的生产费用中完全一样。这种费用跟大部分商业费用完全一样,属于资本主义生产的非生产费用。凡是谈到一般利润率的地方,资本家之间的竞争以及他们的尔虞我诈所花费的劳动,也不在考察之列;同样,一个资本主义企业主同另一个相比,在花最少费用从自己工人身上榨取最大量剩余劳动并在流通过程中实现这种榨取来的剩余劳动方面,有多大技巧,也不在考察之列。对这一切的考察,属于对资本竞争的研究。这种研究,总的来说,涉及资本家之间以及他们为攫取最大数量的剩余劳动所作的斗争和努力,而且只涉及剩余劳动在不同的单个资本家间的分配,但同剩余劳动的来源及其一般大小无关。

对监督劳动来讲,只剩下组织分工和组织某些个人间的协作这种一般的职能。这种劳动在较大的资本主义企业里完全由经理的工资代表。它已经从可供形成一般利润率的东西中扣除了。英国的工人合作工厂\endnote{关于英国的工人合作工厂,见《资本论》第三卷第五章、第二十三章和第二十七章。并参看本册第552—553页。——第392页。}提供了最好的实际证明,因为这种工厂尽管支付较高的利息,提供的利润还是大于平均利润,即使在扣除了经理的工资——当然,它由这种劳动的市场价格决定——以后也是如此。本身就是经理的那些资本主义企业主,节省了一项生产费用,把工资支付给自己,从而取得高于平均利润率的利润率。如果辩护论者[关于企业主利润是监督工资]的这种说法,明天被认真地实现,如果资本主义企业主的利润只是管理和指挥的工资,那末占有他人剩余劳动并把这种剩余劳动转化为资本的资本主义生产,后天就完结了。

但是,即使我们把监督劳动[的报酬]看成是隐藏在一般利润率中的工资,拉姆赛[同上,第227—231页]以及其他经济学家阐述的规律在这里仍然适用。这个规律就是:在利润(企业主利润和总利润[包括利息])同所花费的资本量成比例时,监督劳动所占的份额同资本量成反比——资本大,这个份额就非常小;资本小,就是说,在资本主义生产仅仅名义上存在的地方,这个份额就非常大。一个几乎完全亲自从事企业中所需要的劳动的小资本家,拿他的资本来比,看来获得很高的利润率,而实际情况却是,他既然没有雇用什么工人,占有他们的剩余劳动,实际上就没有取得丝毫利润,而只是名义上从事资本主义生产(不管是工业还是商业)。这个“资本家”和雇佣工人的区别在于,他由于自己的名义上的资本,实际上是自己劳动条件的主人和所有者,因此没有主人压在头上,[1101]他的全部劳动时间都由他自己占有,而不是被他人占有。在这里,作为利润出现的,只是超过普通工资的余额,这个余额恰恰是由于[这个小所有者]占有自己的剩余劳动而造成的。不过,这种形式仅仅属于资本主义生产方式实际上还没有占支配地位的领域。

[拉姆赛说:]

\begin{quote}{“企业主利润可以分解为(1)企业主薪金;(2)补偿其风险的保险费;(3)他的超额利润。”(同上,第226页)}\end{quote}

至于(2),同这里丝毫没有关系。柯贝特(以及拉姆赛本人[同上,第222—225页])说过\endnote{马克思指柯贝特的著作《个人致富的原因和方法的研究;或贸易和投机原理的解释》1841年伦敦版第100—102页。——第393页。},补偿风险的保险费,只是把资本家的损失平均分摊,或者说,更普遍地在整个资本家阶级中分摊。从这个平均分摊的损失中,必须扣除保险公司的利润,即扣除投在保险事业中并担负这种平均分摊职能的资本的利润。这些保险公司以和商业资本家或货币资本家同样的方式取得一部分剩余价值,而不直接参加剩余价值的生产。这是一个如何在不同种类的资本家中间分配剩余价值以及因此对单个资本进行扣除的问题。它既同剩余价值的性质无关,也同剩余价值量无关。工人当然不可能提供比他的剩余劳动更多的东西。他不可能再另外付给资本家一笔钱,为资本家占有这种剩余劳动的果实保险。至多可以说,即使不谈资本主义的生产,生产者在这方面也会有一定的支出,就是说,他们必须支出自己的一部分劳动或者说一部分劳动产品,以防自己的产品、财富或财富的要素遇到意外等等。代替每个资本家自行保险的是,他用[总]资本的一定部分专门担负这项业务,这样,就更可靠、更便宜地取得相同的结果。保险费以一部分剩余价值支付;剩余价值在资本家之间的分配和剩余价值的保险,跟剩余价值的来源和数量无关。

因此,有待考察的是:第一,企业主的“薪金”;第二,“超额利润”——拉姆赛在这里用来称呼剩余价值的一部分,这部分属于和食利的资本家不同的产业资本家,因此它的绝对量也决定于利息与产业利润之比,即属于资本(不同于土地所有权)的那个剩余价值部分所分解成的两个部分之比。

至于说到企业主的“薪金”,那末首先不言而喻的是,在资本主义生产中,资本作为劳动的统治者的职能,落在资本家或者由资本家付酬的职员——资本家的代理人的身上。这种职能只要不是由合作劳动的性质,而是由劳动条件对劳动本身的统治产生的,它也就要随着资本主义生产一起消失。然而,拉姆赛本人把[企业主利润的]这个组成部分抛掉了,或者把它降低到不值一谈的地步:

\begin{quote}{“不管企业大小,企业主的薪金也和监督劳动一样,几乎是相同的。”(同上,第227页)“一个工人决不能说,他能够完成两个、三个或者更多象他一样的工人所能完成的工作。但是一个工业资本家或租地农场主却可以代替十个或更多象他一样的人。”(第255页)}\end{quote}

企业主利润的第三部分是“超额利润”(包括风险费,这种风险只是可能的,只是利润和资本的可能的损失,而实际上表现为保险费,从而表现为特定部门的一定资本在总剩余价值中所取得的份额)。

\begin{quote}{拉姆赛说道:“这种超额利润不折不扣地代表那种从支配资本使用权的权力中{换句话说,从支配他人劳动的权力中}产生的收入,不管这个资本是属于这个资本家本人还是从别人那里借来……纯利润〈利息〉完全随资本的大小而变化;反过来,资本越大,超额利润对所使用的资本之比也就越大。”(第230页)}\end{quote}

换句话说,这不过意味着,“企业主的薪金”与资本的大小成反比。资本活动的规模越大,生产方式越是资本主义的,产业利润中可以归结为“薪金”的组成部分就越微小,产业利润就越清楚地表现出它的真正性质:它是“超额利润”,即剩余价值,亦即无酬剩余劳动的一部分。

产业利润和利息的全部对立,只有基于食利者和产业资本家的对立,才有意义,但是它完全不涉及工人和资本的关系,不涉及资本的性质,也不涉及资本的利润的来源等等。

关于不是生产谷物、而是生产其他农产品的土地的地租,拉姆赛说:

\begin{quote}{“这样一来,为一种产品支付的地租,成了其他产品价值高的原因。”(同上,第279页)}\end{quote}

在最后一章[《论国民收入》]中,拉姆赛说:

\begin{quote}{“收入和年总产品的区别仅仅在于,收入中没有用于维持固定资本〈在拉姆赛那里就是指不变资本,即各生产阶段上的原料、辅助材料和机器等等〉的一切东西。”(第471页)}\end{quote}

[1102]拉姆赛在前面已经讲过\authornote{见本册第361—362页。——编者注}并且在最后一章再次讲到,

\begin{quote}{“流动资本”〈在他那里就是指花在工资上的资本〉是多余的,“它既不是生产的直接因素,甚至对生产也毫无重要意义”。(第468页)}\end{quote}

拉姆赛不过没有由此得出如下的明显结论:否定雇佣劳动和花在雇佣劳动上的资本,也就否定了整个资本主义生产的必要性,从而劳动条件就不再作为“资本”,或者用拉姆赛的术语,不再作为“固定资本”同工人相对立了。劳动条件的一部分之所以表现为固定资本,只是因为另一部分表现为流动资本。但是,一经把资本主义生产假定为事实,拉姆赛就宣布了工资和资本的总利润(包括产业利润,或按他的说法,企业主利润)是收入的必要形式(第475、478)。

当然,这两种形式的收入实际上把资本主义生产以及作为其基础的两个阶级的本质最简单最一般地概括起来了。可是,他却把地租,即土地所有权,说成对资本主义生产是多余的形式(第472页),他忘记了,地租是这种生产方式的必然产物。以上所述也适用于他的另一个论点,即“资本的纯利润”,或者说,利息,不是一种必要的形式:

\begin{quote}{“食利者[在总利润急剧下降的情况下]只好转变为产业资本家。这对于国民财富是无关紧要的……给资本的所有者和使用者提供各自的收入,无疑地不需要那么高的纯利润。”(第476—477页)}\end{quote}

在这里,拉姆赛又忘记了他自己说过的话:随着资本的发展,必然形成一个不断增大的食利者阶级。\authornote{见本册第390—391页。——编者注}

[拉姆赛说:]

\begin{quote}{“资本的总利润和企业主利润……对于生产的过程是必需的。”(第475页)}\end{quote}

当然。没有利润就没有资本,而没有资本就没有资本主义生产。

\centerbox{※     ※     ※}

总之,从拉姆赛那里得出的结论是,第一,建立在雇佣劳动基础上的资本主义生产方式,不是社会生产的必然的即绝对的形式。(拉姆赛本人仅仅以一种带局限性的说法来叙述这个观点,他说,如果不是人民大众那么穷,以致不得不在产品完成以前预支自己在产品中应得的份额,“流动资本”和“工资”就是多余的。)第二,与产业利润不同的利息,和地租(即由资本主义生产本身创造的土地所有权形式)一样,对资本主义生产来说,是不必要的,而且是可以被它扔掉的累赘。如果这种资产阶级的理想真正实现的话,结果只能是,全部剩余价值直接落在产业资本家手中,社会(在经济方面)就会归结为资本与雇佣劳动的简单对立——这种简化无疑会加速这种生产方式的灭亡。[1102]

\centerbox{※     ※     ※}

[1102]{在1862年12月1日的《晨星报》\endnote{《晨星报》(《ThemorningStar》)是英国的一家日报,自由贸易派的机关报;1856年到1869年在伦敦出版。——第397页。}上,一个工厂主抱怨说:

\begin{quote}{“从总产品中扣除工资、地租、资本利息、原料费用以及经纪人、商人的赢利,剩下的才是工厂主、郎卡郡居民、业主的利润,而且他们还得为这么多参与总产品分配的人负担工人的生活费。”}\end{quote}

如果把价值放在一边,来考察实物形式的总产品,那就很明显,在补偿了不变资本和花在工资上的资本之后,剩下的是代表剩余价值的产品部分。但是,从这个剩余中,要扣除一部分作为地租和经纪人、商人的赢利,不论这些人是否使用自己的资本——这一切都取自总产品中代表剩余价值的部分。因此,这一切对工厂主来说是一种扣除。如果工厂主的资本是借来的,那他的利润本身也要分成产业利润和利息。}

{关于级差地租:在比较肥沃的土地上劳动的工人,比起在比较不肥沃的土地上劳动的工人,劳动生产率要高些。因此,如果前一个工人以实物形式得到报酬,那末他在总产品中取得的份额就小于在比较不肥沃的土地上劳动的工人。或者同样可以说,尽管他每天劳动的时数相同,他的相对剩余劳动却大于另一个工人。但是,他的工资和另一个工人的工资的价值是相同的。因此他的雇主的利润也并不比另一个雇主的利润大。在他的产品的超额部分中包含的剩余价值,他的较高的相对劳动生产率,或者说,他的级差剩余劳动,被土地所有者装进了腰包。}[1102]

\tchapternonum{[第二十三章]舍尔比利埃}

[1102]舍尔比利埃《富或贫》1841年巴黎版(从日内瓦版翻印)。

(我们是把这个人专门归入[政治经济学家]这个行列呢,——因为他的观点大部分是西斯蒙第的,——还是把他的比较中肯的见解在适当场合作为引文列举出来,这还是一个大问题。\endnote{这里指《剩余价值理论》的计划。马克思在手稿第XIV本的封面所写的《理论》最后几章的计划包括《(n)舍尔比利埃》(见本卷第1册第5页),本章就是根据这一点写成的。至于西斯蒙第,马克思不打算在《剩余价值理论》中研究他的观点,而打算在自己著作的下一部分,即阐述《资本的竞争和信用》的部分来研究(见本册第52页〉。——第399页。})[1102]

\tchapternonum{[(1)把资本区分为两部分:由机器和原料构成的部分以及由工人的“生活资料基金”构成的部分]}

\begin{quote}{[1103]舍尔比利埃说:“资本就是原料、工具、生活资料基金[approvisionnement]。”(第16页)“资本同财富的其他任何部分之间没有任何区别。只是由于特殊的使用方式,物才成为资本,就是说,只有它被当作原料、工具或生活资料基金在生产行为中加以使用,它才成为资本。”(第18页)}\end{quote}

可见,这是一个普通的方法,即把资本归结为它在劳动过程中所表现的物质要素:劳动资料和生活资料。而且,把资本归结为生活资料的说法是不确切的,因为生活资料虽然是生产者在生产中生存的条件、前提,但是并不加入劳动过程本身;加入劳动过程的,只有劳动对象、劳动资料和劳动本身。因此,劳动过程的客观因素——它们对一切生产形式都是共同的——在这里被称为资本,虽然“生活资料基金”(工资已包含在内)默默地以这些劳动条件的资本主义形式为前提。

舍尔比利埃和拉姆赛完全一样,认为“生活资料基金”——拉姆赛称之为流动资本——会减少(同资本总量相比至少会相对减少,在机器不断排挤工人的情况下则会绝对减少)。但是,他和拉姆赛似乎都认为,可以作为生产资本使用的生活资料即生活必需品的量必然会减少。情况完全不是这样。在这里,总产品中补偿资本并当作资本使用的部分和代表剩余产品的部分,总是被混为一谈。“生活资料基金”会减少,是因为资本(即总产品中当作资本使用的部分)中有一大部分已经不是以可变资本的形式,而是以不变资本的形式被再生产出来。[另一方面,]较大一部分由生活资料构成的剩余产品,则被非生产劳动者和完全不劳动的人吃掉,或者用来交换奢侈品。如此而已。

当然,总资本中转化为可变资本的部分越来越小这个事实,也可以用另外的方式来表示。资本中由可变资本组成的部分,等于总产品中工人自己占有、为自己生产的部分。因此,这部分越小,再生产它所需的工人人数在工人总数中占的比例也就越小(单个工人情况也是这样:他为自己劳动的劳动时间也就越少)。同总劳动一样,工人的总产品也分解为两部分。一部分是工人为自己生产的,另一部分是为资本家生产的。同单个工人的劳动时间可以分成两部分一样,整个工人阶级的劳动时间也可以分成两部分。如果剩余劳动等于半个工作日,那末这就象是工人阶级中有一半为工人阶级生产生存资料,而另一半则为资本家——他们一方面作为生产者,一方面作为消费者——生产原料、机器和成品。

可笑的是,舍尔比利埃和拉姆赛都以为,总产品中能够由工人消费、能够以实物形式加入工人消费的部分必然会减少,或者一般说来会减少。会减少的只是以这种形式,即作为可变资本被消费的部分。相反,被仆人、士兵等等吃掉,或者被输出国外换取更讲究的生活资料的那一部分则会更大。

在拉姆赛和舍尔比利埃的著作中只有一点是重要的,即他们实际上把可变资本和不变资本相对立,而不是停留在从流通中得出的固定资本和流动资本的区分上。因为,舍尔比利埃把资本中归结为“生活资料基金”的部分,同由原料、辅助材料和劳动资料(工具、机器)组成的部分对立起来。不过,不变资本中的两个组成部分——原料和辅助材料,就它们的流通形式来说,都属于流动资本。

在资本组成部分的变动中,重要的不是生产原料和机器的工人相对来说多于直接生产生活资料的工人(这只是分工而已),重要的是,产品应按什么比例补偿过去劳动(即不变资本)和支付活劳动。资本主义生产的规模越大,——从而积累资本越大,——用来生产机器和原料的资本所转化成的机器和原料,在[总]产品价值中占的份额也就越大。因此,以实物形式,或者通过不变资本各不同部分的生产者之间的交换,必然返回生产的产品部分就越大。属于生产的产品部分的比例也就更大,代表活劳动,新加劳动的部分相对来说也就更小。当然,这后一部分表现在商品上,表现在使用价值上,也会增加,因为上述事实和劳动生产率的提高有相同的意义。但是,相对来说,这部分中归工人所有的部分,还会更加减少。而且这同一过程会引起工人人口经常的相对过剩。

\tchapternonum{[(2)关于工人人数同不变资本量相比不断减少的问题]}

[1104]{随着资本主义生产的发展,投在机器和原料上的资本部分在增加,花在工资上的资本部分在减少,这是不容争辩的事实。这是使拉姆赛和舍尔比利埃唯一感到兴趣的问题。但对我们来说,主要的是:这个事实是否说明利润率降低(而且这种降低远不象所说的那么厉害)?同时这里问题不仅涉及量的比例,而且涉及价值比例。

如果现在一个工人纺的棉花能和过去100个工人纺的一样多,原料就必须增加到100倍,此外,这个过程只有依靠一台能使一个工人看管100枚纱锭的纺纱机才能实现。但是,如果与此同时一个农业工人现在生产的棉花和过去100个工人生产的一样多,一个机器制造业的工人现在是生产一整台纺纱机,而不是生产一枚纱锭,那末价值比例保持不变,也就是说现在花在纺纱、棉花和纺纱机上的劳动和过去花在纺纱、棉花和纱锭上的劳动完全相同。

至于机器,它的费用不象它所代替的劳动的费用那么大,虽然纺纱机比纱锭贵得多。有一台纺纱机的单个资本家,所拥有的资本必然大于购买一架纺车的单个纺纱者。但是,如果把纺纱机所需的工人人数考虑在内,使用纺纱机就比使用纺车便宜。否则,纺纱机就排挤不了纺车。资本家取代了纺纱者。但是,纺纱者花在纺车上的资本,与产品量相对来说,大于资本家花在纺纱机上的资本。}

劳动生产率的增长(由于使用机器)和工人人数的减少(同使用的机器的数量与功率相比),二者是一回事。代替简单而便宜的工具的是这类工具(虽然形式有了改变)的组合,此外,还加上由发动装置和传动装置组成的整套机器;然后是用来生产动力(如蒸汽)的材料(如煤等等),最后是建筑物。如果一个工人看管1800枚纱锭,而不是转动一架纺车,那末如果问,为什么这1800枚纱锭不象一架纺车那样便宜,那就是再荒唐不过了。在这里,生产率正是取决于以机器形式使用的资本量。机器磨损的比例只和商品有关;工人同全部机器相对立,因而花在劳动上的资本的价值也同花在机器上的资本的价值相对立。

毫无疑问,机器变得便宜是由于两个原因:由于制造机器的原料是用机器生产的;由于在把这种原料变成机器时使用机器。但是这样说包含着两层意思:第一,在这两个部门,拿它们采用的机器和工场手工业生产中使用的工具相比,花在机器上的资本同花在劳动上的资本相对来说,在价值上增加了。第二,单个机器和它的组成部分变得便宜了,但是发展起一个机器体系:代替工具出现的不仅是单个机器,而且是整个体系,从前可能是起主要作用的工具,例如(织袜机或类似的机器上的)织针,现在是成千上万地结合在一起。同工人相对立的每一台机器,都是工人从前一个个单独使用的工具的庞大组合,例如1800枚纱锭代替了一枚纱锭。但是除此以外,机器还包含旧工具所没有的要素等等。尽管各单个要素便宜了,机器的总体在价格上却大大提高了,生产率的增长就是由于这个总体的不断扩大。

其次,机器所以变得便宜,除了由于它的组成部分变得便宜以外,还由于动力源泉(例如蒸汽锅炉)和传动装置也变得便宜了。动力节约了。但是所以能够节约,正是因为同一个发动机,由于规模的不断增大,能够推动更大的机器体系。发动机相对地说变得便宜了,或者说,它的费用不是和用它来推动的机器体系的扩大成比例地增长;它本身随着自己的规模的增大而变贵,但它的价格不是和它的规模的增大成比例地提高;即使它的费用绝对地说是增长了,但是相对地说还是减少了。因此,撇开单个机器的价格不谈,这是一个使得与劳动对立的机器资本增大的新的因素。机器运转速度增加,会大大提高生产力,但是同机器价值本身毫无关系。

因此,机器价值的增长(与使用的劳动量相比较,因而也是与劳动价值,可变资本相比较),同机器引起的劳动生产率的增长相适应,这种说法,是不言而喻的,或者说是同义反复。

[1105]造成商品由于使用机器而降价的一切情况,首先可以归结为单位商品吸收的劳动量的减少,其次可以归结为机器磨损(这种磨损的价值加入单位商品)的减少。机器磨损得越慢,再生产它们所需的劳动就越少。这就使得由机器构成的资本的量和价值,同以劳动形式存在的资本相比,有了增加。

这样一来,剩下的就只是原料问题。很明显,原料量必须同劳动生产率成比例地增长,也就是说,原料量必须同劳动量成比例。这个比例比它表面上看起来要大。

例如,假定每个星期消费10000磅棉花。一年按50个星期计算,全年共消费10000×50=500000磅。假定全年的工资=5000镑。每磅棉花比如值6便士,全年就是250000先令=12500镑。假定资本每年周转5次。这样,全年的1/5就消费100000磅棉花,价值2500镑。在这1/5年中,支出工资1000镑,即等于投在棉花上的资本价值的1/3强。但是,这丝毫不影响原料量同劳动量的比例。如果现在每1/5年中所消费的棉花的价值等于10000镑,劳动的价值等于1000镑,那它们的比例就是1∶10。(如果考察全年的产品,即一方面是50000镑,另一方面是5000镑,这个比例同样是1∶10。)

{商品的价值就其与机器有关来说,决定于加入商品的机器磨损;因此,只是在机器的价值本身加入价值形成过程,即机器的价值在劳动过程中被消费的情况下,商品的价值才决定于机器的价值。相反,利润却决定于(撇开原料不谈)进入劳动过程的全部机器的价值,而不管这个价值被消费的程度如何。因此,利润必然随着[活]劳动总量的减少(同花在机器上的资本部分相对来说)而下降。利润并不是按照相同的比例下降,因为剩余劳动在增加。}

在原料方面,可以提出这样一个问题:假如纺纱业的生产力提高十倍,也就是一个工人现在纺的纱和过去十个工人纺的纱一样多,那末,为什么一个黑人现在生产的棉花不可以和过去十个黑人生产的棉花一样多,也就是说,为什么不可以使价值比例在这里保持不变呢?纺纱者在同一时间里纺掉十倍的棉花,但是黑人在同一时间里也生产十倍的棉花。因此,十倍的棉花量,并不比以前等于它的1/10的棉花量贵。所以,尽管原料量增加了,它对可变资本的价值比例却可以保持不变。实际上,这个工业部门一般说来所以能够这样发展起来,完全是棉花大降价的结果。\authornote{[1105}{如果明天棉花降价90%,那末后天,纺纱业就会发展得更快,等等。}[1105]]材料(例如金和银)越贵,用它来制作奢侈品时使用机器和分工就越少。这是因为在原料上资本支出太大,而且由于原料昂贵,对这些产品的需求有限。

对上面提出的问题,可以非常简单地回答如下:一部分原料,如毛、丝、皮革,是通过动物性有机过程生产出来的,而棉、麻之类是通过植物性有机过程生产出来的;资本主义生产至今不能,并且永远不能象掌握纯机械方法或无机化学过程那样来掌握这些过程。象皮革等等以及动物的其他组成部分这类原料所以变得昂贵,部分原因就在于不合理的地租规律随着文明的进步使这些产品的价值提高了。至于煤和金属(以及木材),它们随着生产的发展已变得非常便宜;然而在矿源枯竭时,金属的开采也会成为比较困难的事情,等等。

{关于谷物地租和矿山地租,如果可以说,它们并没有提高产品的价值(只是提高了它的市场价格),它们只不过是产品价值的表现(产品价值超过生产价格的余额),那末相反,毫无疑问的是,牲畜租、房租等等就不是这些产品价值提高的结果,而是原因。}

原料、辅助材料等的降价,使资本的这个部分的价值增长变慢,但没有使增长停止。这种降价在一定程度上抑制了利润率的下降。

这个讨厌的问题到此结束。[1105]

[1105]{在考察利润时,假定剩余价值是既定的。只考察不变资本的变动对利润率的影响。只有一种方法使剩余价值直接影响不变资本,那就是通过绝对剩余劳动,通过延长工作日,使不变资本在[产品]价值中占的比例减小。相对剩余劳动——在工作日保持不变的场合(撇开劳动强度增大不谈)——通过剩余价值本身的提高使利润对总资本的价值比例增大。绝对剩余劳动时间则使不变资本的费用相对减少。}

\tchapternonum{[(3)舍尔比利埃关于利润率取决于资本有机构成的猜测;他在这个问题上的混乱。舍尔比利埃论资本主义条件下的“占有规律”]}

[1106]现在回过头来谈舍尔比利埃。

他提出的利润率的公式,或者说是用数学来表示通常所理解的利润,本身并不包含任何规律;或者说甚至是绝对错误的,尽管他对这个事物有某种模糊的概念,接近于对它的了解。

\begin{quote}{“商业利润\endnote{“商业利润”(《profitmercantile》)是舍尔比利埃对单个资本家的利润的称呼,以区别于整个社会的利润。--第407页。}决定于同生产资本各不同要素相比的产品价值。”[同上,第70页]}\end{quote}

{实际上,利润是产品的剩余价值与总预付资本价值之比,而与资本各要素的区别无关。但是剩余价值本身决定于可变资本的量和它的价值增殖率;而这个剩余价值与总资本之比,又决定于可变资本与不变资本之比,也决定于不变资本的价值变动。}

\begin{quote}{“这种规定的两个主要要素,显然是原料价格和加工这些原料所必需的生活资料基金的数量。社会的经济进步,以相反的方向作用于这两个要素。这种进步具有使原料变贵的倾向,因为它使在面积有限的私人土地上经营的采掘业\endnote{舍尔比利埃所说的“采掘业”(《industriesextractives》)不仅指采矿、伐木、捕鱼、狩猎,而且指生产农产原料的各种农业生产。——第407页。}的一切产品的价值提高。”相反,生活资料基金却随着社会进步而(相对)减少,关于这一点我们以后再谈。“产品总量减去为获得这些产品而消费的资本总量,就得出一定时期内的利润总量。产品总量同使用的资本,而不是同已消费的资本成比例地增长。因此,利润率,或者说,利润与资本之比,是另外两个比——使用的资本与已消费的资本之比以及已消费的资本与产品之比——结合的结果。”(同上,第70页)}\end{quote}

舍尔比利埃一开始说对了,利润决定于同生产资本“各不同要素”相比的产品价值。突然他又跳到产品本身,跳到产品量上去了。但是第一,产品量的价值不增加,产品量也可以增加。第二,在把产品量同构成已消费的和未消费的资本的产品量进行比较时,最多也只能按照拉姆赛的办法去做,这就是,把总国民产品同它的以实物形式耗费的构成要素相比较。\authornote{见本册第371—372页。——编者注}但是,就每一单个生产领域的资本来说,产品的形式和它的构成要素是不同的(即使在农业这一类生产部门里也是如此,在那里,一部分产品以实物形式构成该产品的生产要素)。为什么舍尔比利埃会走上这条歧路呢?因为,尽管他猜测到资本的有机构成对利润率有决定意义,但是他完全没有利用他探索到的不变资本和资本的另一组成部分之间的对立来说明剩余价值;正如他没有说明价值本身一样,他也根本没有说明剩余价值。他没有指出剩余价值是从哪里来的,所以就去求助于剩余产品,即求助于使用价值。

虽然任何剩余价值都表现为某种剩余产品,但是剩余产品本身不代表剩余价值。{假定产品根本不包含剩余价值,例如,一个农民有自己的工具(再加上自己的土地),他劳动的时间正好只是一个雇佣工人为补偿自己的工资而劳动的时间,比方说6小时。如果是丰年,他的产品可能加倍,但是全部产品的价值仍然和过去一样。在这种情况下,虽然有剩余产品,却没有剩余价值。}

舍尔比利埃用“生活资料基金”这种“被动的”、纯物质的形式,即用可变资本在工人手中转化成的使用价值形式来表示可变资本,这本身就已经是错误的。相反,如果他按照可变资本的实际表现形式来看待可变资本,就是说把它看作货币(交换价值的即一定量社会劳动时间本身的存在),那末对资本家来说,可变资本就会转化为他用可变资本交换得来的劳动(而在物化劳动同活劳动的这种交换中,可变资本会发生变动,它会增长);可变资本是作为劳动,不是作为“生活资料基金”,而成为生产资本的要素的。而“生活资料基金”是使用价值,是可变资本借以实现为工人的收入的使用价值的物质存在。所以,作为“生活资料基金”,可变资本完全同舍尔比利埃称为“被动的”要素的另外两个资本部分\authornote{[1110}在第59页上,舍尔比利埃把原料和机器称为“资本的两个被动的要素”同“生活资料基金”相对立。[1110]]一样是“被动的”要素。

同一个错误观念妨碍舍尔比利埃通过这个主动要素与被动要素之比,去说明利润率和随着社会发展而出现的利润率的下降。事实上,他得出的无非是这样一个结论:“生活资料基金”[1107]由于生产力的发展而减少,同时工人人口却在增长;因此,工资由于人口过剩而降到它的价值以下。他没有在价值交换的基础上,即在劳动能力按价值支付的基础上,说明任何问题,这样,利润实际上(尽管他没有说出来)就表现为工资的扣除部分;当然,实际的利润有时也可能包含这个扣除部分,但是后者永远不可能成为利润范畴的根据。

首先,让我们把舍尔比利埃的第一个论点还原为它的正确的表达:

\begin{quote}{“产品总量的价值减去为获得〈生产〉这些产品而消费的资本总量的价值,就得出一定时期内的利润总量。”}\end{quote}

这就是利润的第一个(通常的)表现形式,对资本主义意识来说也是这样。换句话说:利润是在一定时期内获得的产品价值超过已消费的资本价值的余额。或者说:是产品价值超过产品的生产费用的余额。甚至这个“一定时期”在舍尔比利埃那里也是突如其来的,因为他没有向我们说明资本的流通过程。因此,他的第一个论点不外是利润的普通定义,是利润的直接表现形式。

舍尔比利埃的第二个论点:

\begin{quote}{“产品总量同使用的资本,而不是同已消费的资本成比例地增长。”}\end{quote}

换句话说,这依然是:

\begin{quote}{“产品总量的价值同预付资本〈不管它是否已经消费〉成比例地增长。”}\end{quote}

这里的目的只是想用狡猾的手法得出利润量决定于使用的资本量这样一个完全未经证明,而且在直接表述上也是错误的论点(因为这个论点已经把[个别利润]平均化为一般利润率这一点当作前提)。但是,由于“产品总量同使用的资本,而不是同已消费的资本成比例地增长”,就必然形成一种表面的因果关系。

让我们看一下这个论点的两种表述,一种是舍尔比利埃的表述,一种是它应该有的表述。从它的上下文来看,并且根据它被当作:mediusterminus\authornote{(推理的)中词。这里是第二个前提的意思。——编者注}而推出的结论来看,它应该是:

\begin{quote}{“产品总量的价值同使用的资本,而不是同已消费的资本成比例地增长。”}\end{quote}

这里显然是想用下列说法来巧妙地解释剩余价值:使用的资本超过已消费的资本的余额构成产品的价值余额。但是,未消费的资本(机器等等)保存着自己的价值(因为,“未消费”,正是说它的价值未消费),它在生产过程结束后,仍然保存着它在生产过程开始前具有的价值。如果发生了价值变动,这种变动也只能发生在已消费的,从而已加入价值形成过程的资本部分。舍尔比利埃解释利润的方法,从下面这一点来看实际上也是错误的:例如一笔三分之一未消费、三分之二在生产中已消费的资本,在劳动剥削率相等(撇开利润率的平均化不谈)时,必然比另一笔三分之二未消费、三分之一已消费的资本提供更多的利润。因为后一笔资本包含的机器等等和其他的不变资本显然多一些,而前一笔资本包含的这个要素少一些,推动的活劳动量多一些,从而推动的剩余劳动也多一些。

如果我们看一下舍尔比利埃本人对自己的论点所作的表述,那末首先应该指出,这种表述不会给他带来什么好处,因为产品量或者说使用价值量本身,不论是对于价值、剩余价值,还是对于利润,都根本不起决定作用。然而在这一切的背后隐藏的究竟是什么呢?由机器等等组成的不变资本部分,加入劳动过程,不加入价值形成过程,因而它有助于产品量的增加,却不在其价值上附加任何东西。(因为在它通过本身的磨损给产品附加价值的情况下,它自己也就属于已消费的资本,不属于同已消费的资本相区别的使用的资本。)但是不变资本的这个未消费的部分本身并不造成产品量的增长。它有助于在一定的劳动时间内创造更大量的产品。因此,如果劳动只是在“生活资料基金”所包含的那么多劳动时间内进行,产品量就会保持不变。所以,产品的余额不是由使用的资本超过已消费的资本的余额构成,而是由这个已消费的资本部分发生的变动造成的(前提是,这里讲的不是象农业这一类生产部门,在这类部门中,产品量不取决于或者可以不取决于所支出的资本量,劳动生产率部分地取决于无法控制的自然条件)。

如果舍尔比利埃把不变资本——已消费的或未消费的——看成同劳动时间[长度]无关,同可变资本在价值增殖过程中发生的变动无关,他同样可以说:

\begin{quote}{“产品总量[1108]的增长〈至少在加工工业中〉,同已消费的资本中由原料组成的部分的增长成比例。”}\end{quote}

因为产品的增长和资本的这个部分的增长在物质上是等同的。另一方面,在农业中(在采掘业中也是一样),土地比较肥沃时,在未消费的资本(即不变资本)用得少、已消费的资本(例如工资)用得比较多的地方,产品量可能比先进的国家大得多,在先进的国家,使用的资本与已消费的资本之比要高得多。

这样,舍尔比利埃的第二个论点就是企图用巧妙的手法偷运剩余价值(利润的必要基础)。

[舍尔比利埃得出的结论是:]

\begin{quote}{“因此,利润率,或者说,利润与资本之比,是另外两个比例——使用的资本和已消费的资本之间的比例以及已消费的资本和产品之间的比例——结合的结果。”(第70页)}\end{quote}

应当先说明利润。但是,代替这个说明的只是对利润下了这样一个定义,这个定义只是表示了利润的表现方式,只是表示了利润是总产品的价值超过产品的生产费用,或者说超过已消费的资本价值的余额这样一个事实,就是说,对利润下了一个平平常常的定义。

现在应当说明利润率。但是,又只是下了一个平平常常的定义:利润率是利润与总资本之比,或者也可以说是产品价值超过它的生产费用的余额与预付在生产上的总资本之比。可见,对于资本要素的近似正确的区分加以歪曲的理解和拙劣的运用,以及对于利润和利润率同这些要素的比例的直接关系所作的猜测,只是使舍尔比利埃以更加明显的学理主义的形式重复那些人所共知的词句,事实上,这些词句只是确认了利润和利润率的存在,关于它们的本质却什么也没有谈到。

舍尔比利埃用代数的方法来表示他的学理主义的公式,这也无济于事:

\begin{quote}{“用P表示一定时期的总产品,C表示使用的资本,π表示利润,r表示利润和资本的比例(利润率),c表示已消费的资本。这样,P-c=π,r=π/C,即Cr=π。因此P-c=Cr;r=(P-c)/C。”(第70页注)}\end{quote}

这一切只是表示,利润率等于利润与资本之比,而利润等于产品价值超过产品生产费用的余额。

一般说来,当舍尔比利埃说到已消费的资本和未消费的资本时,他脑子里想的是固定资本和流动资本的区别,而并不坚持他自己所确认的、与这种区别不同的从生产过程产生的资本的区别。剩余价值在流通之前就已被假定了;不管从流通中产生的区别怎样影响利润率,这些区别与利润的来源都毫无关系。

\begin{quote}{“生产资本由可消费的部分[生活资料基金、原料、辅助材料]和不可消费的部分[工具、器具、机器]组成。随着财富和人口的增长,可消费的部分有增长的趋势,因为采掘业需要越来越大的劳动量。另一方面,这同一个发展又使得使用的资本量的增长程度大大高于已消费的资本量的增长程度。因此,虽然已消费的资本总量有增长的趋势,但这一过程的影响会受到抑制,因为产品量会以更快的速度增长,并且必须承认,利润总量至少是和使用的资本总量一样快地增长的。”(第71页)“增长的是利润量,而不是利润率即这个量与使用的资本之比,r=(P-c)/C。显然,如果C比P-c增长得快,即使r下降,P-c或者说利润(因为P-c=π)也可能增长。”(第71页注)}\end{quote}

在这里,还算在某种程度上接触到了利润率下降的原因;但是,有了先前的歪曲以后,这只能导致混乱和自相抵触的矛盾。起先是已消费的资本量增长,但是产品量增长得更快(就是说,在这里产品价值超过产品生产费用的余额增长得更快),因为产品量同使用的资本成比例地增长,而后者比已消费的资本增长得快。为什么固定资本例如比原料量增长得快,这一点在任何地方都没有说明。但是且不管它。利润量同使用的资本,同总资本成比例地增长,可是[1109]利润率据说还是要下降,因为总资本比产品量,或者不如说,比利润量增长得快。

舍尔比利埃先是说利润量至少是和“使用的资本总量”按同样的比例增长,可是后来又说利润率下降,因为使用的资本总量比利润量增长得快。起先是P-c“至少是”和C成比例地增长,后来又是(P-c)/C下降,因为C比“至少是和C一样快地增长的”P-c增长得更快。如果去掉这一切混乱,剩下的就只是如下的同义反复:即使P-c增长,(P-c)/C也可能下降,也就是说,如果利润率下降,即使利润增长,利润率也可能下降。利润率只是指P-c与C之比,如果资本比利润量增长得快,[这个比例就下降]。

于是便得出如下的聪明的结论:如果资本比利润量增长得快,或者说,如果利润量尽管绝对增长,但是和资本相比却相对减少,那末,利润率可能下降,即增长的利润量与资本之比可能下降。这无非是利润率下降的另一种表现。对这种现象的可能性,甚至对它的存在,从来都没有人怀疑过。这里涉及的唯一的问题恰恰是要说明这种现象的原因,而舍尔比利埃却用利润量的增长至少是同资本的增长成比例,来说明利润率的下降,说明与总资本相比利润量的下降!他显然模糊地猜测到,使用的活劳动量,尽管绝对地说增加了,但是与过去劳动相比还是相对减少;因此利润率必然下降。但是他的这种猜测还不是清醒的理解。越接近入门,而实际上并未入门,表述上的歪曲程度就越大,并且认为已经入门的错觉就越大。

相反,舍尔比利埃关于一般利润率的平均化所说的,倒很中肯。\endnote{在手稿中接下去是《资本论》第三部分第二章的计划草稿,作为插入部分放在方括号内。在这里,马克思打算研究一般利润率的形成问题。本版把这个计划收入本卷第1册的《附录》(第447—448页)。——第415页。}[1109]

\begin{quote}{[1109]“扣除地租之后,利润量——即产品超过已消费的资本的余额——的剩余部分,在资本主义生产者之间,按照他们每人使用的资本的比例进行分配,而与已消费的资本相应的并确定用来补偿它的那一部分产品,则按照他们实际消费掉的资本的比例进行分配。这种二重分配规律由于力图把各方面使用的资本的收益平均化的竞争的作用而得以实现。这种二重分配规律最终决定不同种类产品的相应的价值和价格。”(第71—72页)}\end{quote}

这一段很好。只是最后一句话,即一般利润率的这种形成决定商品的价值和价格(应该说生产价格)是错误的。相反,价值规定是第一性的,是利润率的前提,也是生产价格形成的前提。“利润量”——即剩余价值,[1110]它本身只是商品总价值的一部分——的某种分配,又怎么能够决定这个“利润量”,因而决定剩余价值,因而也决定商品价值本身呢?只有把商品的相对价值理解为商品的生产价格,舍尔比利埃的说法才是正确的。舍尔比利埃的全部错误都是由于他没有独立地考察价值和剩余价值的起源和规律。

此外,对雇佣劳动和资本的关系,他的理解在一定程度上也是正确的:

\begin{quote}{“没有通过让渡〈合法转让财产、继承等等〉得到什么东西,也没有什么东西可以拿去进行交换的人,只有向资本家提供自己的劳动,才能得到他们所需要的东西。他们只有权得到作为劳动价格付给他们的东西,而无权得到这种劳动的产品以及他们附加在产品上的价值。”(第55—56页)“无产者为换取一定量的生活资料出卖自己的劳动,也就完全放弃了对资本其他部分的任何权利。这些产品的占有还是和以前一样;并不因上述的[无产者和资本家之间的]契约而发生变化。产品完全归提供原料和生活资料的资本家所有。这是占有规律的严格结果,相反地,这个规律的基本原则却是每个劳动者对自己的劳动产品拥有专门的权利。”(第58页)}\end{quote}

照舍尔比利埃的说法,这个基本原则就是:

\begin{quote}{“劳动者对于作为自己劳动的结果的价值,拥有专门的权利。”(第48页)}\end{quote}

由于商品规律,商品形成等价物,并按照它们的价值,即按照它们包含的劳动时间彼此交换。这个规律怎么一下子变了样子,以致资本主义生产(对于产品来说,只有在资本主义生产的基础上,作为商品来进行生产,才具有本质的意义)竟然反过来建立在一部分劳动不经交换就被占有的基础上,——这一点舍尔比利埃既不理解,也没有加以说明。他只是感到,这里发生了某种转变。

舍尔比利埃所说的“基本原则”纯粹是一种虚构。它是由商品流通造成的假象产生的。商品按照它们的价值,即按照它们包含的劳动彼此交换。单个人在这里只是作为商品所有者互相对立,所以,只有让出自己的商品,才能占有别人的商品。因此形成一种似乎他们能交换的只是自己的劳动的假象,因为包含别人劳动的商品的交换,在这些商品本身又不是用自己的商品换得的情况下,是以与[简单]商品所有者即买者和卖者的关系不同的另一种人与人之间的关系为前提的。在资本主义生产当中,资本主义生产表面上反映出来的这种假象消失了。但是有一种错觉并没有消失:似乎最初人们只是作为商品所有者互相对立,因而每个人只有在他是劳动者的情况下才是所有者。如上所述,这“最初”就是由资本主义生产的假象产生的错觉,——这种现象在历史上从来不曾有过。一般说来,人(不论是孤立的还是社会的)在作为劳动者出现以前,总是作为所有者出现,即使所有物只是他从周围的自然界中获得的东西(或者他作为家庭、氏族或公社的成员,部分地从周围的自然界中获得,部分地从公共的、已经生产出来的生产资料中获得)。最初的动物状态一终止,人对他周围的自然界的所有权,就总是事先通过他作为公社、家庭、氏族等等的成员的存在,通过他与其他人的关系(这种关系决定他和自然界的关系)间接地表现出来。“没有所有权的劳动者”作为“基本原则”,倒不如说只是文明的产物,而且是“资本主义生产”这个一定的历史阶段上的产物。这是“剥夺”规律,不是“占有”规律,至少不是舍尔比利埃所想象的一般占有规律,而是和一定的、特殊的生产方式相适应的占有规律。[1110]

[1111]舍尔比利埃说:

\begin{quote}{“产品在转化为资本以前就被占有了;这种转化并没有使它们摆脱那种占有。”(第54页)}\end{quote}

这句话不仅适用于产品,而且适用于劳动。原料等等和劳动资料属于资本家;它们是他的货币的转化形式。另一方面,如果资本家用等于6劳动小时的产品的货币额,购买劳动能力或一天(例如12小时)的劳动能力的使用权,那末,这12小时的劳动就属于资本家,这个劳动在实现以前就已被资本家占有。它通过生产过程本身转化为资本。不过,这种转化是在它被占有以后发生的行为。

“产品”转化为资本:如果产品在劳动过程中执行劳动条件、生产条件(劳动对象和劳动资料)的职能,就是在物质上转化;如果不仅产品的价值被保存,而且产品本身还成了吸收劳动和剩余劳动的手段,也就是说,产品实际上执行劳动吸收器的职能,那就是在形式上转化。[1112]另一方面,在生产过程之前被占有的劳动能力,在生产过程中会直接转化为资本,因为它转化成了劳动条件和剩余价值,因为它物化为产品时既保存不变资本,又补偿可变资本并附加剩余价值。[1112]

\tchapternonum{[(4)关于作为扩大再生产的积累问题]}

[舍尔比利埃说:]

\begin{quote}{[1110]“财富的任何积累,都为加速进一步的积累提供手段。”(同上,第29页)}\end{quote}

{李嘉图(从斯密那里继承下来的)关于任何积累都归结为工资的支出的观点,即使在积累的任何部分都不是以实物形式(例如,租地农场主播下更多的种子,畜牧业者增加种畜或肥育牲畜的头数,机器制造业者在机器制造机上占有一部分剩余价值)实现的情况下也是错误的。这个观点即使在下述情况下也是错误的:即使不存在这样一种现象,即生产某个资本部分的构成要素的所有生产者,由于考虑到年积累的事实,即考虑到一般生产规模的扩大,都是经常地进行超额生产。此外,土地耕种者可以用他的一部分剩余谷物和畜牧业者进行交换,畜牧业者可以把这部分谷物转化为可变资本,而土地耕种者则[通过这种交换]把自己的谷物转化为不变资本。亚麻种植业者[1111]出卖他的一部分剩余产品给纺纱业者,纺纱业者把它转化为不变资本;亚麻种植业者可以用这笔货币购买工具,而工具生产者又可以用这笔货币购买铁等等,这样一来,所有这些要素都直接成了不变资本。但是,撇开这一点不谈。

假定机器厂主想把一笔1000镑的追加资本转化为生产要素。在这种情况下,他当然要把其中的一部分花在工资上,比如说,200镑。他用其余的800镑购买铁、煤等等。假定这些铁、煤还有待于生产。如果制铁业者或煤炭业者在这个时候既没有剩余的(积累的)商品储备,又没有追加的机器,而且也不能直接购买机器(因为在这种情况下,又会发生不变资本同不变资本的交换),那末,制铁业者和煤炭业者只有使他们的旧机器延长工作时间,才能为机器厂主生产铁和煤的追加量。于是旧机器就要加速补偿,但是它们的一部分价值会加入新产品。不过这一点也撇开不谈。制铁业者无论如何也需要更多的煤;因此,这里他必须至少把800镑中属于他的份额的一部分直接转化为不变资本。但是他们两人——煤炭业者和制铁业者——出卖他们的煤和铁时,也使其中包含了无酬剩余劳动。如果这种劳动占1/4,在800镑中就已经有200镑不归结为工资,更不用说产品价值中归结为旧机器磨损的那一部分了。

剩余产品总是由各种特定资本生产的实物形式的东西如煤、铁等等组成。有些生产者的产品互为生产要素,如果他们互相交换这些产品,那末一部分剩余产品就直接转化为不变资本。而同生活资料生产者生产的产品交换并补偿其不变资本的那一部分,则形成必要的可变资本。有些生活资料已不能作为要素(除了作为可变资本)加入自身的生产,这些生活资料的生产者,正是通过其他生产者借以获得追加的可变资本的同一过程来获得追加的不变资本。

再生产——就它是积累而言——和简单再生产的区别如下:

第一,积累的生产要素——它们的属于可变资本的部分和属于不变资本的部分——由新加劳动构成;它们不完全转化为收入,虽然它们是由利润产生的;利润,或者说剩余劳动,转化为所有这些生产要素。而在简单再生产中,一部分产品代表过去劳动(也就是说,这里指的不是当年完成的劳动)。

第二,不言而喻,如果某些生产部门劳动时间延长了,就是说,在那里没有使用追加的工具或机器,那末新产品就要部分地支付旧的工具或机器的更快的磨损,而旧的不变资本的这种加速消费也是积累的因素。

第三,在[扩大]再生产过程中,部分地由于资本的游离,部分地由于一部分产品转化为货币,部分地只是由于生产者[用自己的追加商品]收回货币而使对其他人——例如对出卖奢侈品的人——的商品的需求减少,从而形成追加货币资本;因为有了这种资本,就完全没有必要象在简单再生产中那样系统地补偿生产要素。

每个人都可以用剩余的货币购买产品或支配产品,尽管他向之购买产品的生产者既不把自己的收入花在买者的产品上,也不用这种产品补偿自己的资本。}{每当追加资本(可变的或不变的)不是相互补充的时候,它必然在某一方面作为货币资本沉淀下来,即使这种货币资本只是以债权形式存在。}

\tchapternonum{[(5)舍尔比利埃的西斯蒙第主义因素。关于资本有机构成问题。比较发达的资本主义生产部门的可变资本绝对减少。在资本有机构成保持不变情况下不变资本和可变资本的价值比例的变动。资本的有机构成以及固定资本和流动资本之间的不同比例。资本周转的差别及其对利润的影响]}

在其他方面,舍尔比利埃的观点是西斯蒙第和李嘉图的互相排斥的见解的奇怪混合物。[1111]

[1112]下面的话是西斯蒙第的东西:

\begin{quote}{“关于资本不同要素之间的比例不变的假设,在社会经济发展的任何阶段都不会实现。它们之间的比例实质上是可变的,而且是由于两个原因:(1)分工;(2)人力由自然力代替。这两个原因使生活资料基金与资本的另外两个要素之比有下降的趋势。”(第61—62页)“在这种状况下,生产资本的增加,不一定会引起用来形成劳动价格的生活资料基金的增长;在这种增加的同时,资本的这个要素至少是暂时地会绝对减少,从而劳动价格会下降。”(第63页)}\end{quote}

{这是西斯蒙第的东西;这种[生活资料基金的减少]对工资高度的影响,是舍尔比利埃的唯一着眼点。如果研究是以劳动按其价值支付为前提,而劳动的市场价格在这一点(价值)的上下波动则不考虑在内,这个着眼点也就完全失去意义。}

\begin{quote}{“一个生产者想要在自己的企业中采用新的分工或者利用某种自然力,他不会等到积累的资本足以在这些条件下使用他以前所需要的全部工人时才这样做;在分工的场合,他也许会满足于用五个工人来生产他以前用十个工人生产的东西;在使用自然力的场合,他也许只要使用一台机器和两个工人。因此,生活资料基金[以前等于3000],在第一种场合将减少到1500,在第二种场合将减少到600。但是因为工人现有人数保持不变,所以他们的竞争会很快使劳动价格降到它原来的水平以下。这是占有规律的极其惊人的结果之一。财富即劳动产品的绝对增多,并没有引起工人生活资料基金的相应增多,甚至还能引起这一基金的减少,使各种产品中应归于工人的份额减少。”(第63—64页)“决定劳动价格〈这里始终只是指劳动的市场价格〉的原因,是生产资本的绝对量以及资本不同要素之间的比例,这是工人的意志不能给予任何影响的两个社会事实。”(第64页)“一切机会几乎都是对工人不利的。”(同上)}\end{quote}

生产资本不同要素之间的比例,是由两种方式决定的。

第一,生产资本的有机构成。我们指的是技术构成。在劳动生产力既定的情况下,——只要不发生什么变化,就可以假定它是不变的,——在每个生产领域中,原料和劳动资料的量,也就是与一定的活劳动量(有酬的和无酬的),即一定的可变资本的物质要素量相应的、表现为物质要素的不变资本量,是一个确定的量。

如果与使用的活劳动相比,物化劳动小,代表活劳动的产品份额就大,而不管这个产品部分在资本家和工人之间怎样分配。反之则相反。因此,如果劳动剥削率是既定的,剩余劳动在前一场合也就大,在后一场合也就小。只是由于生产方式发生变化,而这种变化又改变资本两个部分的技术比例,这里才能发生变化。即使在这种场合,如果各资本的量不同,使用较多不变资本的资本所使用的活劳动的绝对量也可能相同,或者甚至更大。但是相对来说,它必然要小一些。对于等量的资本来说,或者以总资本的一定的相应部分(例如100)来计算,不论从绝对还是相对来说,它都必然要小一些。由于劳动生产力的发展(不是下降)而出现的一切变化,都使代表活劳动的产品部分减少,使可变资本减少。因为假定工资到处相同,所以在考察不同生产部门的资本[1113]时,我们可以说,上述变化会使处在较高生产发展阶段的部门的可变资本绝对减少。

这就是由生产方式的变化产生的变化。

但是第二,如果把资本的有机构成和由资本有机构成的差别产生的资本之间的差别假定为既定的,那末尽管技术构成保持不变,[不变资本和可变资本之间的]价值比例也能发生变动。这里可能有以下几种情况:(a)不变资本的价值发生变动;(b)可变资本的价值发生变动;(c)二者按相同的或不同的比例同时变动。

(a)如果技术构成保持不变,不变资本的价值发生变动,那末,这个价值或者下降或者提高。如果它下降,并且只使用原有的活劳动量,就是说,如果生产的阶段或规模保持不变,从而照旧使用例如100个工人,那末,在物质上就照旧需要同量的原料和劳动资料。但是,剩余劳动与总预付资本之比将比以前大。利润率会提高。在相反的场合利润率就下降。在前一场合,对于某个生产领域已经使用的资本来说(不是指那些在不变资本要素的价值发生变动以后新投入该领域的资本),使用的资本总量会减少,或者说,这个资本的某一部分会游离出来,尽管生产继续以原有的规模进行;或者这样游离出来的资本会追加投入生产,起着资本积累的作用。生产规模会扩大,剩余劳动的绝对量会相应地增长。在这种生产方式下,不管剩余价值率如何,任何的资本积累都会导致剩余价值总量的增长。

相反,如果不变资本要素的价值提高,那末,或者生产规模(从而总预付资本量)必须扩大,才能使用和以前同量的劳动(价值没有变动的同一可变资本);这时,尽管剩余价值的绝对量和剩余价值率保持不变,剩余价值与总预付资本之比却减少,因而利润率下降。或者生产规模和预付资本总量不扩大。在这种场合,可变资本在任何情况下都必然会减少。

如果花在[变贵了的]不变资本上的数额和以前一样,这个数额就代表不变资本的较少量的物质要素,因为技术比例保持不变,所以使用的劳动必须减少。这样,总预付资本中就减少了一个游离出来的劳动量;预付资本的总价值减少了;但是在这个减少了的资本中,不变资本占的比例(从价值上说)比以前大。剩余价值绝对减少,因为使用的劳动减少了,剩下来的剩余价值与总预付资本之比下降,因为和不变资本相比可变资本减少了。

另一方面,如果使用的总资本和以前一样,——减少了的可变资本价值(代表减少了的使用的活劳动总量)被增大了的不变资本价值吸收(前者减少的比例和后者增大的比例相同),那末,剩余价值的绝对量会减少,因为使用的劳动减少了,同时,这个剩余价值与总预付资本之比也会下降。因此,利润率下降在这里是由于两个原因:剩余劳动量减少和这个剩余劳动与总预付资本之比下降。

在第一种场合(在不变资本要素的价值下降时),利润率不管怎样都会提高,要使利润额增加,生产规模就必须扩大。假定资本等于600,其中一半是不变资本,一半是可变资本。如果不变部分价值下降一半,那末,可变资本仍旧是300,而不变资本只是150。使用的总资本就只有450,150就会游离出来。如果把这150再加入资本,那末其中有100现在将会作为可变[1114]资本支出。因此,在这里,如果在生产中继续使用和以前相同的资本,生产规模就要扩大,使用的劳动量就要增多。

在相反的场合(在不变资本要素的价值提高时),利润率不管怎样都会下降,要使利润额不减少和使用的劳动量(从而剩余价值量)保持不变,生产规模就必须扩大,也就是说,预付资本必须增多。如果生产规模不扩大,如果预付的资本只是和以前一样多,或者甚至比以前还少,那末不仅利润率要下降,而且利润量也要减少。

在以上两种场合,剩余价值率都保持不变;而在资本的技术构成发生变化时,它就会变化:不变资本增加时,它会提高(因为这时劳动生产率提高了),不变资本减少时,它会下降(因为这时劳动生产率降低了)。

(b)如果可变资本价值的变动与有机构成无关,那末这种情况之所以能发生,仅仅因为不是这个生产领域生产的、而是作为商品从外部进入该领域的生活资料在价格上下降或提高了。

如果可变资本的价值下降,那末这个可变资本仍旧代表相同的活劳动量,只是这个活劳动量的所值现在减少了。因此,如果生产规模保持不变(因为不变资本的价值没有变),总资本中预付在购买劳动上的部分就减少。现在只需花费较少的资本,就可支付同样数量的工人。可见,在这里,在生产规模保持不变的情况下,所花费的资本额会减少。利润率会提高,这是由两种原因产生的。剩余价值增大了;活劳动与物化劳动之比保持不变,但是增大的剩余价值却是与减少的总资本相应的。如果把游离出来的部分附加在资本上,这就等于积累。

如果可变资本的价值提高,那末,为了使用原有数量的工人,就必须花费更多的总资本,因为不变资本的价值保持不变,而可变资本的价值提高了。使用的劳动量保持不变,但是剩余劳动在使用的劳动总量中占的部分比以前小了,而这个较小的部分是与一个比以前大的资本相应的。这是在生产规模保持不变而总资本价值提高的场合下发生的。如果总资本价值不提高,那末生产规模就必须缩减。使用的劳动量减少了,在这个减少了的劳动量中,剩余劳动部分比以前小,它与总预付资本之比也小了。

有机变化和由价值变动引起的变化,在某种情况下,能够对利润率产生相同的影响。但是,它们之间有如下的区别:如果价值变动不单是由市场价格的波动引起,就是说,如果它们不是暂时的,那末它们就始终必然是由提供不变资本或可变资本要素的领域发生的有机变化引起的。

(c)对第三种情况这里不需要作进一步的考察。

在不同生产领域的资本相等的情况下,——或者按总资本的等量部分计算,例如都按100计算,——有机构成可能相同,虽然不变资本和可变资本的价值比例将随着使用的辅助材料和原料量的价值不同而不同。例如,铜代替铁、铁代替铅、羊毛代替棉花等等。

另一方面,如果价值比例相同,有机构成可能不同吗?

[不同生产领域的两笔资本]有机构成相同时,每100单位的资本中不变资本和活劳动的相对量也相同——它们之间的量的比例也相同。很可能是,不变资本的价值相同,虽然被推动的相对劳动量不同。如果机器或原料在一种场合比另一种场合贵(或者相反),那末所需要的劳动,比如说,可能减少;但是那时可变资本的价值也将相对减少(或者相反)。

[1115]举资本A和B为例。假定c′和v′是资本A的组成部分(按价值来说),c和v是资本B的组成部分(按价值来说)。如果c′∶v′=c∶v,那末,c′v=v′c。因此c′/c=v′/v。

在不变资本和可变资本之间的价值比例相同时,只可能出现下述情况。如果一个领域比另一领域完成了更多的剩余劳动{例如农业中就不能打夜工,虽然单个农业工人可能被迫过度劳动,但是在地块等等的大小既定的条件下所能使用的劳动总量,却受到需要生产的对象(谷物)的限制,可是,在工厂的大小既定的条件下,生产的产品量(有可能)取决于劳动小时的数量,——就是说,由于生产方式不同,在生产规模既定时,一个领域可以比另一领域使用更多的剩余劳动},那末不变资本和可变资本之间的价值比例可能相同,但是使用的劳动量与总资本之比会不同。

或者假定,材料和劳动(由于它属于较高级的劳动)按同一比例变贵。在这种情况下,资本家A在资本家B使用25个工人的地方使用5个工人,这5个工人的费用是100镑,和那25个工人的费用一样,因为他们的劳动变贵了(从而他们的剩余劳动的价值也变贵了)。同时,这5个工人加工价值500镑的原料y100磅,而资本家B的工人加工价值500镑的原料x1000磅,因为在资本家A那里材料较贵,劳动生产力较不发达。这里,在两种场合可变资本和不变资本的价值比例都是100镑比500镑,但是资本A和B的有机构成不同。

价值比例相同:资本家A的不变资本的价值等于资本家B的不变资本的价值,A花在工资上的资本,和B一样多。但是他的产品量较少。虽然他需要的劳动量绝对地说和资本家B需要的一样,但是相对地说他需要的却多一些,因为他的不变资本贵一些。在同一时间里,A加工的原料等等较少,但是这个较少量和B的较多量的原料的价值相同。在这种场合价值比例相同,有机构成则不同。在另一种场合,如果价值比例相同,这种情况只有在剩余劳动量不同或各种劳动的价值不同时才有可能发生。

资本有机构成的概念可以这样表述:这是在不同生产领域为吸收同量劳动而必须花费的不变资本的不同比例。同量劳动与劳动对象的结合,在一种场合比在另一种场合需要更多的原料和机器设备,或者只是其中之一。

{在固定资本和流动资本之间的比例很不相同的情况下,不变资本和可变资本之间的比例可能相同,从而剩余价值也可能相同,虽然一年内生产的价值必然不同。假定在不使用任何原料(辅助材料撇开不谈)的煤炭工业中,固定资本占总资本的一半,可变资本占另一半。假定在裁缝业中固定资本等于零(和上一场合一样,辅助材料撇开不谈),但是原料等于一半,可变资本和上一场合一样也等于总资本的一半。这样,两笔资本(在对劳动的剥削相同的情况下)将实现相同的剩余价值,因为按100单位的资本计算,它们使用的劳动量相同。但是,假定煤炭工业中的固定资本十年周转一次,而两种场合的流动资本的周转毫无差别。如果剩余价值等于50,那末,裁缝业主到年终生产的总价值(假定两种场合的可变资本都是一年周转一次)将等于150。相反,煤炭业者到第一年年终生产的价值等于105(即固定资本5,可变资本50,剩余劳动50)。他的产品的总价值加固定资本等于150(也就是说,产品等于105,剩下的固定资本等于45),和裁缝业主那里的情况一样。可见,生产的价值量不同,并不排除生产的剩余价值相同。

第二年,煤炭业者的固定资本将等于45,可变资本等于50,剩余价值等于50。因此,预付资本将等于95,利润等于50。利润率会提高,因为固定[1116]资本的价值由于固定资本在第一年磨损十分之一而减少。因此,毫无疑问,对所有使用很多固定资本的资本来说,——在生产规模不变的情况下,——利润率必然提高,其提高的程度等于机器即固定资本的价值由于已经补偿的磨损而每年下降的程度。如果煤炭业者在十年内总是按同一价格出卖自己的产品,那末,他在第二年得到的利润率必然高于第一年,依此类推。或者必须假定,维修工作等等同磨损成正比,以致在固定资本项目下每年预付的资本部分的总额保持不变。上述超额利润也能得到平衡,因为固定资本的价值(与磨损无关),由于旧机器必须同较完善的、较晚发明的新机器相竞争而会逐渐下降。但是另一方面,这种由于磨损,由于固定资本价值的减少而自然产生的不断提高的利润率,使旧机器能够同较完善的新机器竞争,因为新机器还要按全部价值进行计算。最后,如果煤炭业者[在第二年年终]卖得便宜些,就是说这样计算:50比预付资本100得50%利润,95乘50%得47+(1/2);即如果他出卖同量产品得到[不是105,而是]102+(1/2),那末,和比如说还只是第一年把机器投入生产的人相比,他就卖得便宜些。固定资本的大量投入以拥有大资本为前提。因为这些大资本所有者控制着市场,所以看来他们只是由于上述原因才获得超额利润(租)。这种租在农业中是因为在相对来说比较肥沃的土地上劳动而获得的,在这里,则是因为利用相对来说比较便宜的机器进行劳动而获得的。}

{许多被说成是由固定资本和流动资本之间的比例造成的情况,实际上是与可变资本和不变资本之间的差别有关。第一,[某些生产部门的]不变资本和可变资本之间的比例可能相同,尽管固定资本和流动资本之间的比例会不同。第二,当我们说到不变资本和可变资本时,指的是资本最初的划分为活劳动和物化劳动,而不是流通过程或流通过程对再生产的影响所引起的这种比例的变化。

首先,显而易见,固定资本和流动资本之间的差别只有在它影响总资本的周转时,才能影响剩余价值(撇开同可变资本与不变资本之比有关的在所用活劳动量上的差别不谈)。因此,必须研究资本周转怎样影响剩余价值。显然有两种情况同这个问题密切相关:(1)剩余价值不能那么快(那么经常)地积累起来,再转化为资本;(2)预付资本必须增长,既为了继续推动同一数量的工人,等等,也由于资本家对本身消费不得不作的预付的时间延长。这两种情况在考察利润时很重要。但是这里首先应该考察的,只是它们怎样影响剩余价值的问题。而这两种情况始终必须清楚地区别开来。}

{凡是使预付增长而没有使剩余价值相应增长的情况,都会使利润率下降,即使剩余价值保持不变;凡是使预付减少的情况,其作用则相反。因此,只要同流动资本相比的较大固定资本量——或资本的不同周转——影响预付量,它也就影响利润率,即使它丝毫不影响剩余价值。}

{利润率不是单纯地按预付资本计算的剩余价值,而是在既定的期间,即在一定的流通时间实现的剩余价值量。因此,只要固定资本和流动资本的差别影响了一定的资本在既定的期间实现的剩余价值量,它也就影响利润率。这里有两个因素:第一,(同实现的剩余价值相比的)预付的量的差别;第二,在这些预付连同剩余价值流回以前,生产这些预付所必需的时间的长度的差别。}

[1117]{实质上影响再生产时间,或者更确切地说,影响一定时间内的再生产次数的,有两种情况:

(1)产品在生产领域本身停留的时间较长。

第一,一件产品本身所需要的生产时间比另一件产品所需要的可能长一些,可能需要一年中较长的一段时间,也可能是整整一年,或者是一年以上。(例如,在建筑业、畜牧业和某些奢侈品的生产中就需要一年以上。)在这种情况下,按照生产资本的构成,即按照它的不变资本和可变资本的划分,产品不断吸收劳动,同不变资本相比,往往(例如在奢侈品生产中、在建筑业中)吸收很多的劳动。这样,随着产品生产时间的延长(然而这种生产也是劳动过程的均衡的持续),劳动和剩余劳动就不断地被吸收。例如,在畜牧业或建筑业中,比如说建筑业需要一年以上的时间。产品只有在完成以后才能进入流通,也就是说出卖,投入市场。第一年的剩余劳动,和其他劳动一起,在第一年的未完成产品中客体化了。它既不小于也不大于具有同样的不变资本和可变资本比例的其他生产部门的剩余劳动。但是,这个产品的价值不能实现,就是说不能转化为货币,从而剩余价值也不能实现。因此,这个剩余价值既不能作为资本积累起来,也不能用于消费。预付资本和剩余价值可以说都成了进一步生产的基础。它们是进一步生产的前提,它们在某种程度上作为半成品,以某种方式作为原料,加入第二年的生产。

假定预付资本等于500,劳动等于100,剩余价值等于50,这样,第二年用于生产的预付资本就等于550加上第二年追加的预付500。剩余价值仍然等于50。这样,到第二年年终产品的价值就等于1100镑,其中100是剩余价值。在这种场合,剩余价值就象在第一年资本全部再生产出来、第二年又重新投入500镑的场合一样。可变资本在第一年和第二年都是100,剩余价值都是50。但是利润率不同。第一年利润率是50/500,即10%。但是第二年预付是550+500=1050,这个数额的十分之一等于105。因此,如果假定利润率相同,产品的价值就是:第一年550,第二年550+500+55+50=1155。产品的价值在第二年年终就等于1155。在另一种情况下它只等于1100。在这里,利润大于生产出来的只是100的剩余价值。如果把资本家在两年内必须预付的自己消费的费用也计算在内,那末支出的资本同剩余价值相比就更大了。不过第一年的全部剩余价值确实在第二年也都已转化为资本。此外,花在工资上的资本增大了,因为第一年预付的100镑到第一年年终没有再生产出来,因此在第二年必须为同样的劳动预付200镑,否则,用第一年再生产出来的100镑就够了。

第二,在劳动过程结束之后,产品可能还必须停留在生产领域里,以便经受自然过程的作用,这些过程不需要任何劳动或者只需要相对来说非常少的劳动,例如葡萄酒置于窖内。只有这个时期过去以后,资本才能再生产出来。显然,这里不管可变资本和不变资本的比例如何,得出的结果都和支出较多不变资本和较少可变资本时的情况一样。剩余劳动和这里在一定期间使用的全部劳动一样,量比较小。如果利润率相同,那末,这是由于利润的平均化,而不是由于这个领域里生产的剩余价值。为了维持再生产过程——生产的连续性——必须事先预付较多的资本。又是由于这个原因,在这里剩余价值与预付资本之比会下降。

第三,当产品还处在生产过程中的时候,劳动过程可能中断,例如农业中就有这种情形,还有象制革之类的过程也是这样,在那里,在产品能够从一个生产阶段转到下一个较高的生产阶段以前,化学过程造成劳动过程的中断。在这种场合,如果有化学方面的发现来缩短这种中断的持续时间,那末劳动生产率就会提高,剩余价值就会增大,向生产过程预付物化劳动的时间就会缩短。在劳动过程发生中断的所有场合,剩余价值减少,预付资本增大。

(2)当某个流动资本的周转速度由于离市场远而比平均的周转速度慢时,也会发生同样的情况。在这里,资本预付也增大,剩余价值也减少,剩余价值与预付资本之比也下降。}{在这一场合,资本在流通领域里停滞的时间较长,在前面所说的场合,则在生产领域里停滞的时间较长。}

[1118]{假定在运输业的某个部门里,预付资本等于1000;固定资本等于500,五年耗损完;可变资本等于500,每年周转四次。这样,年产品的价值就是100+2000+100(如果[年]剩余价值率等于20%),总计是2200。另一方面,假定在裁缝业的某个部门里,不变流动资本等于500(固定资本等于0),可变资本等于500,剩余价值等于100。假定资本每年周转四次。这样,年产品的价值是4(500+500)+100=4100。两种场合的剩余价值相同。后一笔资本一年全部周转四次,或者说,每季度周转一次。在第一笔资本中,每年周转的有600[其中有500每年周转四次]。因此每季度有500+100/4,即525在周转。因此一个月有175,两个月有350,8个月有1400。总资本(1000)周转一次需要5+(5/7)个月。它一年只周转2+(1/10)/次。

有人会说,第一笔资本,为取得10%的利润,每个季度加在价值1000上的附加额少于第二笔资本。但是这里的问题不在于附加额。一个资本家获得较多的剩余价值,是靠他已消费的资本,而不是靠他使用的资本。这里的差别来自剩余价值[相对量],不是来自利润附加额。这里差别在于价值,不在于剩余价值。两笔资本中的可变资本500,每年都周转四次。两笔资本每年获得的剩余价值都等于100,[年]剩余价值率都等于20%。但是每季度的剩余价值是25镑,——是不是说,百分比较高呢?每季度25比500等于每季度5%,因此全年就是20%。

第一个资本家有一半资本一年周转四次,另一半一年只有五分之一在周转。四次的一半是二次。因此,他的资本一年周转2+(1/10)次。第二个资本家的全部资本一年周转四次。但是这绝对改变不了剩余价值。如果第二个资本家不间断地继续再生产过程,那他就必须不断把500转化成原料等等,而他用于支付劳动的始终只是500,但是第一个资本家用于支付劳动的也是500,其余的500却一劳永逸地(即为期5年)被赋予一种不需要他再来转化的形式。但是这一点[剩余价值的均等]只有在尽管固定资本和流动资本的量有差别,但是[两笔资本的]可变资本和不变资本的比例却相同的情况下才会出现。

如果在两笔资本中,都是一半为不变资本,一半为可变资本,那末[第一笔资本]有一半可能只由固定资本构成,如果流动不变资本等于零;而[第二笔资本]有一半可能只由流动不变资本构成,如果固定资本等于零。应当看到,虽然流动不变资本可能等于零,例如在采掘业和运输业中(在那里,不过是辅助材料代替原料构成流动不变资本),然而固定资本(除了在银行家等等那里)永远不会等于零。但是,如果不变资本在两种场合都和可变资本处于相同的比例,尽管不变资本在一种场合包含较多的固定资本和较少的流动不变资本,在另一种场合则包含较少的固定资本和较多的流动不变资本,情况也不会有所改变。这里有的只是一半资本的再生产时间上的差别和总资本的再生产时间上的差别。一个资本家必须在他的500镑流回之前把它预付5年,另一个则预付一个季度或一年。对资本的支配能力不同。预付没有什么不同,但是预付的时间不同。这种差别在这里和我们无关。如果就全部预付资本来考察,剩余价值和利润在这里是相同的:第一年预付1000镑得100镑。第二年不如说是在固定资本方面出现较高的利润率,因为可变资本保持不变,而固定资本的价值减少了。第一个资本家第二年只预付400固定资本和500可变资本,照旧获得100镑利润。但是100比900等于11+(1/9)%,而第二个资本家继续进行再生产,照旧预付1000,获得100镑利润,等于10%。

当然,如果同可变资本相比,整个不变资本随同固定资本一起增加了,或者说,如果为了推动同量劳动,而不得不总的来说预付较多的资本,那末情况就会改变。在上面举的例子中,问题不在于总资本周转有多快或预付有多大,问题在于那部分足以推动同量生产劳动而在另一种场合足以更新生产过程的资本周转有多快。但是,如果在上述例子中,固定资本[不是等于500,而是]等于1000,而流动资本[仍旧]只等于500,那末情况就会改变。但是发生这种变化,并不因为是固定资本。要知道,如果第二种场合的流动不变资本(例如由于材料变贵)值1000[而不是500],那末情况也会同样改变。因为在[两个例子中的]第一种场合,固定资本越大,总预付资本同可变资本相比的相对量也就越大,二者也就会混淆起来。此外,周转这种事情,实际上来源于商业资本,在那里这由其他规律决定:在商业资本中,如我指出的那样\endnote{马克思指他在1861—1863年手稿第XV和XVII本中,特别是在第964页(第XV本)和1030页(第XVII本)对商业资本的研究。——第435页。},利润率实际上由周转的平均数决定,而不管该资本的构成如何,不过它主要是由流动资本构成。因为商业资本的利润决定于一般利润率。}

[1119]{整个情况如下。

假定固定资本等于x。如果它15年只周转一次,那末,它一年有1/15在周转,但是每年需要补偿的也只是这笔资本的1/15。如果它一年中补偿了15次,那末情况并不会因此而有丝毫改变。它的量仍旧保持不变。但是产品因此而变贵了。当然,比起以流动资本形式预付的同量资本来,对资本的支配能力就较小,贬值的风险就较大。但是这丝毫不影响剩余价值,虽然资本家先生们在计算利润率时把这一点也计算在内,因为在计算磨损时这种贬值的风险是被计算在内的。

至于资本的另一部分,那末假定不变资本的流动部分(原料和辅助材料)一年等于25000镑,工资等于5000镑。这样,如果这笔资本一年只周转一次,全年就必须预付30000镑;如果剩余价值等于100%,即5000镑,那末到年终利润就等于5000镑比30000,即1/6,或16+(2/3)%。

但是如果这笔资本每1/5年周转一次,那末在不变流动资本上只须预付5000镑,在工资上只须预付1000镑。利润——1000镑,5/5年——5000镑。但是这个剩余价值是用6000镑资本获得的,因为决不会预付大于这个数额的资本。因此,利润是5000比6000,即5/6,也就是第一种场合的5倍:83+(1/3)%(撇开固定资本不谈)。于是,这里就出现利润率的极大差别,因为实际上5000镑的劳动是用1000镑的资本买来的,而25000镑的原料等等是用5000镑的资本买来的。如果在这种周转率不同的情况下资本的量相同,那末在第一种场合就只能预付6000镑。或者说每月只预付500镑,其中5/6由不变资本构成,1/6由可变资本构成。这1/6=83+(1/3)镑,用它获得100%的剩余价值是83+(1/3)镑,全年是(83+1/3)12=12/3(或4)+996=1000。但是1000比6000等于16+(2/3)%。}

\tchapternonum{[(6)李嘉图和西斯蒙第的互相排斥的见解在舍尔比利埃著作中的折衷主义的结合]}

现在回过头来谈舍尔比利埃。

下面的论述是西斯蒙第的见解:

\begin{quote}{“社会的经济进步只要是以生产资本的绝对增长和这一资本不同要素之间的比例发生变化为特征,它就会给工人提供若干好处:(1)劳动生产率的提高——特别是由于使用机器——引起生产资本非常迅速的增长,以致尽管生活资料基金和这一资本的其他要素之间的比例发生了变化,但这个基金还是有了绝对的增长,这样就不仅允许使用原有数量的工人,而且允许雇用追加数量的工人,所以,进步的结果,如果撇开一些暂时的中断不谈,对工人来说就意味着生产资本的扩大和对劳动的需求的增长。(2)资本生产率的提高有着使一系列产品的价值大大下降的趋势,从而使它们成为工人可以获得的东西,工人的消费范围由此而得到扩大……但是:(1)构成劳动价格的生活资料基金有时会减少,即使这种减少是短暂的、局部的,它也会对工人造成极其有害的后果。(2)促使某一社会的经济进步的情况大部分是偶然的,是不以生产资本家的意志为转移的。因此,这些原因的作用不是经常不变的”……。“(3)使工人的状况变得幸福或不幸福的,与其说是工人的绝对消费,不如说是工人的相对消费。如果工人无法获得的产品的数量以更大的比例增加了,如果把他和资本家隔开的距离只是增大了,如果他的社会地位变得更低和更不利了,那末,对他来说,能够获得一些他们这样的人以前无法获得的产品又有什么意义呢?除了维持体力所绝对必需的消费品以外,由我们消费的消费品的价值完全是相对的。人们忘记了,雇佣工人是有思想的人,是赋有和劳动资本家同样的才能、被同样的动机推动的人。”(第65—67页)[1120]“社会财富的迅速增长,不管能给雇佣工人带来怎样的好处,也消除不了他们贫困的原因……他们照旧被剥夺了对资本的任何权利,因而不得不出卖自己的劳动,并且放弃对这种劳动的产品的任何要求。”(第68页)“这是占有规律的根本缺陷……弊病在于雇佣工人和由他的劳动推动的资本之间这种完全缺乏联结的链条。”(第68—69页)}\end{quote}

这最后一句关于“联结的链条”的话纯粹是西斯蒙第的见解,同时是荒谬的。

关于标准人即资本家,等等——见同书第74—76页。

关于资本积聚和排除小资本家——第85—88页。

\begin{quote}{“如果在目前状况下实际利润是从资本家的节约中得到的,那末[在其他的分配制度下]它同样也可以从雇佣工人的节约中得到。”(第88—89页)}\end{quote}

[另一方面,]舍尔比利埃

(1)赞同[詹姆斯·]穆勒关于一切赋税都应从地租征收的观点\endnote{马克思指詹姆斯·穆勒在他的《政治经济学原理》(1821年伦敦版第4章第5节《地租税》)中的论断。穆勒在书中证明,国家的全部费用在土地还不是私有财产的情况下用全部地租来支付,在土地已经成了私有财产以及地租同原来的水平相比有所增加的情况下用地租的增长额来支付,是合理的。——第438页。}(第128页),但是因为不可能“规定一种真正从地租征收,而且只触及地租的赋税”,又因为很难把利润和地租区分开,——如果土地所有者本人耕种土地,就根本不可能区分开,——所以舍尔比利埃就

(2)继续前进,接近了李嘉图学说的正确结论:

\begin{quote}{“为什么不再前进一步,废除土地私有制呢?”(第129页)“土地所有者是有闲者,他们靠公众的费用来养活自己,对生产或社会的一般福利毫无益处。”“使土地具有生产能力的,是使用在农业上的资本。土地所有者对此毫无贡献。他的存在只是为了收取地租,而地租并不构成他的资本的利润的一部分,它既不是劳动的产物,也不是土地生产力的产物,而是由消费者的竞争抬高的农产品价格的结果”……(第129页)“因为废除土地私有制丝毫不会改变产生地租的原因,所以地租还会继续存在;不过地租将由国家征收,因为全部土地将属于国家,国家将把可耕种的地块租给拥有足够的资本来经营这些地块的私人。”(第130页)地租将取代国家的全部收入。“最后,获得解放的、摆脱了一切枷锁的工业将得到空前的发展”……(第130页)}\end{quote}

但是,怎样使这个李嘉图式的结论和西斯蒙第的给资本和资本主义生产拴上“链条”的虔诚愿望协调起来呢?怎样使它和下面这种悲叹协调起来呢:

\begin{quote}{“如果没有一场变革来阻止我们的社会在占有规律的[目前]统治下实现的发展进程,资本最终将成为世界的主宰。”(第152页)“资本将到处消灭旧的社会差别,以便用一种简单的人类划分来代替它,这就是把人类划分为富人和穷人,富人享乐和统治,穷人劳动和服从。”(第153页)“生产基金和产品的普遍占有,向来是把人数众多的无产者阶级降低到屈从和政治上无权力的状态,但是这种占有曾经和整套限制性法律结合在一起;这些法律阻碍产业的发展和资本的积累,[1121]限制被剥夺继承权阶级的增长,把他们的公民自由约束在狭小的范围内,从而用各种方法使这个阶级无能为害。今天,资本已把这些枷锁的一部分打碎了。它正在准备把它们全部打碎。”(第155—156页)“无产者的堕落是[目前]财富分配的第二个后果。”\endnote{舍尔比利埃把富人和穷人之间的不平等称为“目前财富分配的第一个后果”。——第438页。}(第156页)}\end{quote}

\tchapternonum{[第二十四章]理查·琼斯}

\tchapternonum{(1)理·琼斯《论财富的分配和税收的源泉》,第一部分:《地租》,1831年伦敦版。[地租历史观的因素。琼斯在地租理论的个别问题上胜过李嘉图之处以及他在这方面的错误]}

琼斯的这第一部论地租的著作就已经有一个特点,那是詹姆斯·斯图亚特爵士以来一切英国经济学家所没有的,这就是:对各种生产方式的历史区别有了一些理解。(对各种历史形式所作的这种正确的区分,总的说来同已被指出的琼斯的考古学的、语言学的和历史学的非常重要的错误并不矛盾。例如,见《爱丁堡评论》第54卷第4篇文章。\endnote{马克思指的是该杂志第54卷(1831年8月至12月)上刊登的一篇没有署名的书评,评论当时刚出版的琼斯的著作《论财富的分配》。——第439页。})

琼斯在李嘉图以后的现代经济学家的著作中发现,地租被规定为超额利润,这一规定的前提是:农场主是资本家(或者说,农业资本家经营土地),他期望从资本的这种特殊使用中得到平均利润;农业本身从属于资本主义生产方式。简言之,这里所考察的仅仅是土地所有权的改变了的形式,即资本这一占统治地位的社会生产关系赋予它的那种形式,亦即它的现代资产阶级形式。琼斯完全没有资本自有世界以来就已存在这样一种错觉。

琼斯关于地租的起源的见解,一般说来概括在以下的论述中:

\begin{quote}{“甚至在人们从事最原始的劳动时,土地也有能力提供多于土地耕种者维持生活所必需的东西,这样就使他有可能向土地所有者交纳贡物,这就是地租的起源。”(第4页)“由此可见,地租起源于那样一个时代的土地占有,在那个时代,大多数居民或者不得不在所能得到的任何条件下耕种土地,或者饿死,而且那时这些人的微薄的资本,如工具、种子等,由于不可克服的必然性而同他们一起被束缚在土地上,因为,他们若不从事农业,而去从事任何别的,他们的资本就根本不够维持他们的生活。”[第11页]}\end{quote}

琼斯研究了地租的一切变化:从最原始的徭役劳动形式到现代的租地农场主地租。他到处都发现,地租的一定形式,即土地所有权的一定形式,与劳动和劳动条件的一定形式相适应。所以他依次考察了劳役地租或农奴地租,考察了劳役地租向实物地租的转化,考察了分成制地租、莱特\endnote{莱特(Ryot)——印度农民。琼斯用这个术语来称呼印度和亚洲其他国家的这样一些农民,他们向君主,即向被认为是全部土地的最高所有者缴纳实物租税。——第440页。}地租等,他的这种研究的细节,我们在这里不感兴趣。在一切较早的地租形式中,直接占有别人剩余劳动的人不是资本家,而是土地所有者。地租(正如重农学派根据回忆所理解的那样)在历史上(在亚洲各民族中还是在最大范围内)表现为剩余劳动的普遍形式,即无偿地完成的劳动的普遍形式。与资本主义关系不同,在这里,对这种剩余劳动的占有不是以交换为中介,而是以社会的一部分人对另一部分人的暴力统治为基础。(由此而来的还有直接的奴隶制、农奴制或政治的依附关系。)

因为我们在这里考察土地所有权只是由于理解土地所有权是理解资本的先决条件,所以我们也就不去详细叙述琼斯的论证,而立即转到那个十分有利于把他同所有他的前辈们区别开来的结果上去。

但是,在此之前还要附带谈几点意见。

琼斯在谈到徭役劳动——以及或多或少与此相适应的农奴制(或奴隶制)的各种形式——的时候,[1122]无意中突出了任何剩余价值(任何剩余劳动)都可以归结成的两种形式。总的说来值得注意的是:本来意义的徭役劳动在其最粗野的形式中最鲜明不过地显示了雇佣劳动的本质。

\begin{quote}{“地租〈在有徭役劳动的地方〉在这样的情况下只有用以下两种办法才能增加:或者是更巧妙更有效地使用农奴的劳动〈这是相对剩余劳动〉,然而这将由于土地所有者这个阶级无力发展农业科学而受到阻碍;或者是增加从农奴身上榨取的劳动的量,在这种情况下,如果所有者的土地会耕种得好一些,那末,农奴的土地就会因劳动被夺去而耕种得坏一些。”(同上,第II章[第61页])}\end{quote}

琼斯的这本论地租的书同我们将在第二节中加以考察的他的《大纲》有以下区别:在第一本著作中,琼斯把土地所有权的各种不同形式当作某种既定的东西,并以这些形式为出发点,而在第二本著作中,他以土地所有权的各种不同形式与之相适应的劳动的各种形式为出发点。

琼斯还指出,劳动的社会生产力的不同发展程度怎样和这些不同的生产关系相适应。

徭役劳动(奴隶劳动也完全一样),就地租这一点来说,同雇佣劳动有一个共同点,那就是,地租是用劳动支付,不是用实物支付,更不是用货币支付。

\begin{quote}{在分成制地租的情况下,“资本由土地所有者预付,并让实际劳动者自由地耕种土地,表明这里依然没有起中介作用的资本家阶级”。(第74页)“莱特地租是从土地取得工资的劳动者向作为土地所有者的君主交纳的实物地租。”(第IV章[第109页])(这种地租主要见于亚洲。)“莱特地租往往和劳役地租及分成制地租结合在一起。”(第136页及以下各页)在这里主要的土地所有者是君主。“在亚洲,城市的繁荣,或者更确切地说,城市的存在,完全依赖于政府的地方性开支。”(第138页)“茅舍贫农\endnote{琼斯所理解的“茅舍贫农”(《Cottier》)是爱尔兰的无地农民,他们从地主那里租一小块耕地,交一定的货币地租。——第442页。}地租……这就是从土地取得生存资料的佃农按照契约以货币形式支付的地租。”(第143页)(爱尔兰。)“在地球的大部分地区不存在货币地租。”[同上]“所有这些形式〈劳役地租、莱特地租、分成制地租、茅舍贫农地租等等,一句话,农民地租的一切形式〉都阻碍土地生产力的充分发展。”[第157页]“不同的人的劳动生产率的差别取决于以下两点:第一,在使用手工劳动的情况下,利用发明的程度,第二,人的纯体力活动在多大程度上得到过去劳动的积累结果的帮助,也就是说,这种差别取决于生产中使用的技能、知识和资本的差别。”[第157—158页]“非农业阶级的人数不多。显然,不从事农业劳动而能生活的人的相对数,完全取决于土地耕种者的劳动生产率。”(第VI章[第159—160页])“在英国农村,在农奴劳动废止以后,出现了在土地所有者的领地上从事耕作的租佃者。那是自由民。”(同上[第166页])}\end{quote}

最后,我们要谈到这里使我们最感兴趣的一点,即租地农场主地租。正是在这里,琼斯的优越之处突出地显示了出来:他证明,李嘉图等人看作是土地所有权的永恒形式的东西,却是土地所有权的资产阶级形式,这种形式一般只在以下情况才出现,第一,土地所有权不再是支配生产从而支配社会的关系;第二,农业本身以资本主义方式经营,而这又是以城市的大工业(至少是工场手工业)的发展为前提。琼斯指出,李嘉图所说的地租只存在于[1123]以资本主义生产方式为基础的社会。随着地租转化为超额利润,土地所有权对工资的直接影响也就终止,换句话说,这只是意味着,今后直接占有剩余劳动的人不是土地所有者,而是资本家。地租的相对量现在仅仅取决于剩余价值在资本家和土地所有者之间的分配,而不取决于对这种剩余劳动的榨取本身了。这层意思实际上在琼斯那里已经有了,尽管他没有明确地把它说出来。

同李嘉图相比,琼斯不论在历史地解释现象方面,还是在经济学的细节问题上,都向前迈出了重要的一步。我们将逐步考察他的理论。当然他的理论中也有错误。

琼斯在下面的论述中正确地说明,在什么历史条件和经济条件下地租是超额利润,或者说,是现代土地所有权的表现。

\begin{quote}{“只有当社会各阶级的最重要的相互关系不再从土地所有权和土地占有产生的时候,租地农场主地租才能存在。”(第185页)}\end{quote}

资本主义生产方式开始于工业,只是到后来才使农业从属于自己。

\begin{quote}{“最先受资本家支配的是手艺人和手工业者。”(第187页)“这种制度的直接结果之一,就是有可能随意把用于农业的劳动和资本转移到其他行业中去。”}\end{quote}

{只有具备了这种可能性,才谈得上农业利润和工业利润的平均化。}

\begin{quote}{“当租佃者自己是劳动农民,由于缺少其他生存资料而被迫从土地获取这些资料时,他被穷困束缚在这块土地上;他可能拥有的少量资本实际上也同它的所有者一起被束缚在土地上,因为这笔资本如果不完全用来耕种土地,就不够维持他的生活。这种对土地的依赖性随着资本主义企业主的出现而终止了,如果在农业中使用工人所能赚得的,不如在那种社会情况下的其他各种行业里从工人劳动中赚得的多,就会停止经营农业。在这种情况下,地租必然完全由超额利润构成。”(第188页)“地租不再对工资发生影响了。”[同上]“当一个劳动者为资本家雇用时,他对土地所有者的依赖性就终止了。”(第189页)}\end{quote}

下面我们将看到,琼斯并没有真正说明超额利润是怎样产生的,或者更确切地说,他不过是按李嘉图的方式去说明,也就是用各种土地的自然肥力的差别去说明。

\begin{quote}{“当地租由超额利润构成时,特定的一块土地的地租可能由于以下三种原因而增加:(1)由于在生产中使用更多的积累资本而使产品增加;(2)更有效地使用已有投资;(3)在资本和产品保持不变的情况下,各生产阶级所得的产品份额减少,而土地所有者的份额相应增加。这些原因也可以按不同的比例结合起来发生作用。”(第189页)}\end{quote}

现在让我们来看看这几种原因都是些什么情况。首先,它们都是以地租来自超额利润为前提。其次,毫无疑问,这些原因中的第一个原因是完全正确的,这个原因李嘉图只有一次顺便提到过\authornote{见本卷第2册第112—113、138、358页。——编者注}。如果农业上使用的资本增加,地租的量也就增加,尽管谷物等等的价格不提高,并且一般说来也不发生其他任何变动。显然,在这种情况下土地价格也会提高,尽管谷物价格不提高,并且一般说来在谷物价格方面也不发生其他任何变动。

琼斯把最坏的土地的地租解释为垄断价格。所以他把地租的真正源泉归结为:或者是垄断价格(如布坎南、西斯蒙第、霍普金斯等人的主张),如果存在(不是由各种土地的肥力的差别产生的)绝对地租的话;或者是级差地租(如李嘉图的主张)。

{关于绝对地租。拿金矿为例。假定使用的资本等于100镑,平均利润等于10镑,地租等于10镑。再假定资本的半数由不变资本(在这一场合是机器和辅助材料)构成,半数由可变资本构成。50镑不变资本只是表示,它所包含的劳动时间[1124]和50镑金包含的劳动时间相等。所以与50镑相等的那一部分产品将补偿已消费的不变资本。如果剩下来的产品等于70镑,并且用50镑可变资本去推动50个工人,那末50个工人[的劳动](假定工作日等于12小时)就必须表现为70镑金,其中50镑支付工资,20镑体现无酬劳动。在这种情况下,有机构成相同的所有资本的产品价值都等于120镑。于是产品就等于50c+70,而后面这70镑代表50个工作日,并且等于50v+20m。一笔100镑的资本,如果它使用的不变资本较多,使用的工人人数较少,它生产出来的将是价值较小的产品。但是,一切普通的产业资本,即使它们的产品价值在这种情况下等于120镑,也只会按产品的生产价格110镑来出卖产品。但是,对金矿来说,即使撇开土地所有权不谈,这也是不可能的,因为在这里价值表现在产品的实物形式上。因此,在这里必然会产生10镑地租。}

\begin{quote}{“谷物能够按照垄断价格(即按照超出在最不利条件下生产谷物的人的费用和利润的价格)出卖,或者按照仅仅支付普通利润的价格出卖。如果是第一种情况,那末,撇开耕地肥力的一切差别不说,由于资本增加而达到的产品增加(在价格不变的情况下),就能使地租同所花费的资本的增加成比例地提高。例如,假定普通利润率为10%。如果花100镑生产出的谷物能卖115镑,那末地租就是5镑。如果由于耕作水平的提高,在这同一块土地上使用的资本增加一倍,并且产品也增加一倍,那末200镑的资本就会提供230镑的产品,地租将是10镑,就是说也增加一倍。”(第191页)}\end{quote}

{对于绝对地租是这样,对于级差地租也是这样。}

\begin{quote}{“在小的社会内,谷物总是能够按照垄断价格出卖……在比较大的国家,如果人口增长的速度总是比农产品增长的速度快,这种情况也是可能的。但是对于土地异常辽阔而又多种多样的国家来说,谷物的垄断价格则是极不平常的现象。如果谷物价格显著提高,就会有更多的土地被耕种,或者有更多的资本投到原有的耕地上去,直到价格所提供的利润不再多于所花费用的普通利润为止。那时农业的发展就会停下来,在这样的国家,谷物通常出售的价格只够补偿在最不利条件下使用的资本,并得到该资本的普通利润率;而比较肥沃的土地所支付的地租,则按这些土地的产品超过花费同样资本耕种的最坏土地的产品的余额来计算。”(第192页)“如果一个国家拥有各种质量的土地,要在这个国家的整个土地面积上增加地租,所必需的条件就是:较好的土地必须给随着农业的发展而投到它上面的追加资本提供多于显然较坏的土地所提供的产品;因为,在能够找到办法把新资本按普通利润率使用在A和Z\authornote{A和Z是拉丁字母的第一个和最后一个,这里用来表示这个国家的最坏的和最好的土地。——编者注}之间的任何一块土地上的时候,凡是质量比这块特定的土地好的土地的地租都会增长。”(第195页)“如果经营土地A花费100镑,每年获得110镑,——其中普通利润是10镑,——土地B花费100镑,获得115镑,土地C花费100镑,获得120镑,依此类推,直到土地Z,那末,土地B就提供地租5镑,土地C则提供地租10镑。现在假定,这些土地中的每一块土地都花费200镑来经营。这时A将提供220镑,B—230镑,C—240镑,依此类推,于是地租在土地B就是10镑,在土地C是20镑,等等。”(第193页)“在农业中使用的资本的一般积累,会使一切等级的土地的产品都或多或少地与这些土地的原有质量相应地增长,同时它本身也必定会提高地租,而不管从使用的劳动和资本中得到的收益怎样日益减少,而且事实上也与其他任何原因毫无关系。”(第195页)}\end{quote}

琼斯的功绩在于他最先明确地指出,既然已经假定地租是存在的,那末一般说来它就会{始终要假定生产方式不发生任何变革}由于农业资本,即使用在土地上的资本的增加而增加。这种情况不仅在价格保持不变时可能发生,而且甚至在价格下跌到原来水平以下时也可能发生。

[1125]对于[农业]生产率递减的论断,琼斯反驳说:

\begin{quote}{“英国谷物的平均收获从前每英亩不超过12蒲式耳。现在增加了将近一倍。”(第199页)“依次投入土地的资本和劳动,都会比前一次使用得更经济和更有效。”(第199—200页)“当投在原来那块土地上的资本增加一倍、两倍、三倍时,在收益不减少,耕地的相对肥力不发生变化的情况下,地租也会增加一倍、两倍、三倍等等。”(第204页)}\end{quote}

这就是琼斯胜过李嘉图的第一点。地租既然已经存在,它就能够由于在土地上使用的资本的单纯增加而增加,既不管各种土地的相对肥力怎样变化,也不管相继使用的各笔资本的收益怎样变化,也不管农产品价格怎样变化。

琼斯的另外一点是:

\begin{quote}{“对于地租的增长来说,各种土地肥力的比例完全不变,并不是绝对必要的。”(第205页)}\end{quote}

{琼斯在这里没有看到:正好相反,甚至在全部农业资本使用得更有效时,土地肥力的差别增大也必定会,而且确实会使级差地租的量增大。反之,土地肥力的差别缩小,必定会使级差地租即从这些差别产生的地租减少。去掉原因,也就去掉了结果。然而地租(撇开绝对地租不说)还是能够增长,但那仅仅是由于农业上使用的资本增加了。}

\begin{quote}{“李嘉图没有看到,追加资本在肥力不同的土地上必然带来不同的结果。”(同上)}\end{quote}

(可见,这无非是说,追加资本的使用,会扩大土地的相对肥力的差别,从而使级差地租提高。)

\begin{quote}{“如果用同一个数去乘几个相互之间有一定比例的数,各乘积之间的比例仍将和原数之间的比例相同,但是各个乘积在量上的差额将逐次增大。如果10、15、20各乘以2或4,得数将是20、30、40或40、60、80,它们的相互比例并没有破坏;80和60同40的比例,与20和15同10的比例一样,但是它们的乘积在量上的差额每次都将增大:最初差额是5和10,后来差额是10和20,而最后差额是20和40。”(第206—207页)}\end{quote}

这个规律可简单表述如下:

\todo{}

各项之间的差额在(2)是两倍,在(3)是四倍。差额总和,也是在(2)是两倍,在(3)是四倍,等等。

这就是第二个规律。

第一个规律(琼斯只把它用在级差地租上)是:地租量和使用的资本量一同增加。如果资本为100时地租量等于5,那末资本为200时地租量就等于10。

[1126]第二个规律:如果所有其他情况保持不变,在各种土地上使用的资本[的收益]的差额的比例保持不变,那末,这些差额的总量,从而总地租量或这些差额总和,就会和由于使用的资本增加而引起的这些差额的绝对量的增长一同增长。所以,第二个规律是:在各种土地的肥力的比例不变,但是使用在这些土地上的资本以同等程度增加的情况下,级差地租的量同这些土地上的产品的差额的增长成比例地增长。

\begin{quote}{再往下看:“如果在A、B、C三个等级的土地上各使用100镑,所得产品分别为110、115和120镑,而后来使用200镑,总收入为220、228和235镑,那末产品的相对差额减小了,而且这些土地在肥力上相互接近了。然而,它们的产品量的差额还是从5和10增加到8和15,地租也因而提高了。由此可见,具有使耕地肥沃程度互相接近趋势的那些改良,即使没有其他任何原因起促进作用,也完全能够使地租提高。”(第208页)“种植芜菁和饲养羊,以及在这方面使用的新资本,给较坏的土地的肥力带来的变化,比给较好的土地的肥力带来的变化要大。但是这使较坏的土地和较好的土地的绝对产量增加了,因而也使地租提高了,尽管这时耕地肥力的差别缩小了。”(同上)“至于李嘉图的看法,即改良能够引起地租下降,那末这里就应当想起,农业改良的发现、完善和推广实际上是非常缓慢的。”(第211页)}\end{quote}

{最后这句话只有实践的意义,没有涉及事情的本质;它仅仅指出这些改良进行得不够快,以致不能使供给较之需求有很大的增加,不能使市场价格因而下降。}

最初我们看到:

(1)ABC

10,15,20。

每一个等级使用的资本都等于100。产品等于110、115、120。差额是5+10=15。

由于进行了改良,现在使用的资本增加了一倍,在A、B、C三个等级的每一个等级使用的都不是100,而是200,但是这个资本在不同等级的土地上起着不同的作用,得到的产品等于220(即A的产品的一倍)、228和235。

因此得出:

(2)ABC

20,28,35。

每一个等级使用的资本现在都等于200。产品等于220、228和235。差额是8+15=23。但是这个差额的比率减小了。5∶10(即在第一种情况下B—A[的差额同A的比例])=1/2,10∶10=1,而8∶20仅仅等于2/5,15∶20=3/4。差额的比率减小了,但是差额本身的量增大了。然而,这并不会构成任何新规律,而不过象在第一个规律中那样,证明地租随着使用的资本的增加而增加,虽然产品的增加,在A、B、C,都不是与这些土地肥力的原有差别成比例的。如果由于肥力的这种提高(然而,对于B和C来说,这意味着肥力的[相对]减小,因为不然的话,它们的产品就应当等于230和240),价格会下降,那末,地租提高或者仅仅保持不变,就决不是必然的。

[1127]作为第二个规律的结果,作为这一规律的进一步运用,得出了

第三个规律:如果“那些提高农业上所用资本的效率的改良”,会增加某些地段上获得的超额利润,那末这些改良也会增加地租。

与此有关的有琼斯以下的(以及前面的)论述:

\begin{quote}{“因此,租地农场主地租提高的第一个源泉,是不断增长的积累以及资本在不同土地上产生的不同效果。”(第234页)}\end{quote}

{但是,这里所说的只能是那些直接影响土地肥力的改良,如肥料、轮作制等等。}

\begin{quote}{“那些提高农业上所用资本的效率的改良,会使地租提高,因为这些改良会增加某些地块上获得的超额利润。除非这些改良使土地的产品量增加得那样快,以致超过了需求的增长,它们就总是会引起超额利润的这种增加。提高所用资本的效率的这些改良,通常是随着农业技术的进步和较大量辅助资本〈不变资本〉的积累出现的。地租由于这种原因而提高,随着地租的这种提高而来的通常是把耕作扩展到较坏的土地上去,但一点也不减少在最坏的耕地上使用的农业资本的收益。”(第244页)}\end{quote}

{琼斯十分正确地指出,利润的下降并不证明农业生产率降低。但他本人对于利润下降的可能性解释得非常不完善。他说,或者是产品的数量可能发生变动,或者是产品在工人和资本家之间的分配可能发生变动。在这里对于利润率下降的真正规律还毫无所知。

\begin{quote}{“利润的下降不是农业生产率降低的证据。”(第257页)“利润部分地取决于劳动产品量,部分地取决于劳动产品在工人和资本家之间的分配;所以利润量能够由于这两个因素中的任何一个因素发生变动而变动。”(第260页)}\end{quote}

由此也就产生了琼斯所表述的一条错误规律:

\begin{quote}{“撇开课税的影响不谈,如果所有的生产阶级合起来看,其收入有了明显的减少〈这里没有说,什么是收入,是使用价值还是交换价值,是指利润量还是利润率〉,如果出现利润率下降而没有通过提高工资得到补偿,或者出现工资下降而没有通过提高利润率得到补偿〈这正是错误的李嘉图规律〉,那就可以得出结论说,劳动和资本的生产力已经有些减低。”(第273页)}}\end{quote}

琼斯正确地理解到,尽管绝对地说农业实际上是在不断进步,但是随着社会的发展,农产品价值同工业品相比会有相对的增长:

\begin{quote}{“在一个国家的发展过程中通常可以看到:在人口不断增长的情况下,工业的能力和技能的增长程度大于可以期待于农业的增长程度。这是不容争辩的真理。所以,随着国家的进步,没有农业生产率的任何绝对的下降,也能期待农产品的相对价值增长。”(第265页)}\end{quote}

但这并不说明农产品的货币价格的绝对的上涨,除非金的价值下降,而这种下降在工业中由于工业品价格的更大下降而得到平衡和超过平衡,但是这样的平衡在农业中不会发生。甚至[1128]在不发生金(货币)的价值普遍下降的时候,上述情况也会出现,例如某个国家用自己的日劳动换取的货币多于同它竞争的国家时就是如此。

琼斯不相信李嘉图规律在英国的作用,但承认这一规律的抽象的可能性,理由如下:

\begin{quote}{“如果地租增长仅仅是由于李嘉图提出的那个原因,也就是说,由于‘使用追加劳动量带来比较少的收益’,因而较好土地的部分产品会转到土地所有者手里,那末,总产品中被土地所有者当作地租拿去的平均份额就必然要增长。第二,在这种情况下,就必然会有更大部分人口的劳动使用在农业上。”(第280—281页)}\end{quote}

(后面一点是不确切的。有可能:更多的间接[secondary]劳动被使用,即更多的由工商业提供的商品加入了农业过程,可是总产品并不相应增加,使用的直接[农业]劳动量也不增多,甚至会更少。)

\begin{quote}{“我们在英国的统计中发现三个事实:随着耕地面积的扩大,全国的地租总额增加了;从事农业的那部分人口减少了;土地所有者从产品中得到的份额减少了。”(第282页)}\end{quote}

(最后一点,完全可以和利润率下降一样,用补偿不变资本的那部分产品增加来解释。这时地租在量和价值上都可能增长。)

\begin{quote}{“亚·斯密说:‘随着农业改良的发展,地租同耕地面积相比虽然增加了,同土地产品相比却减少了。’\endnote{亚·斯密的这段话载于他的《国富论》第二篇第三章。——第452页。}”(第284页)}\end{quote}

琼斯把不变资本叫作“辅助[auxiliary]资本”。

\begin{quote}{“从农业部在不同时期提出的各种报告可以看出,在英国,使用在农业上的全部资本同用于维持工人的资本之比是5∶1,即所使用的辅助资本,比用来维持直接使用在农业上的劳动的资本多三倍。在法国这种比是2∶1。”(第223页)“如果有一定量的追加资本,以过去劳动的结果的形式被使用,以便促进当前使用的工人的劳动,那末,要使这种资本的使用有利可图,因而成为经常可行的,只要有较少的年收益就够了,可是,如果用同量的新资本来维持追加工人,那就需要有较多的年收益。”(第224页)“假定在土地上花100镑来维持三个工人,他们生产自己的工资和10%的利润,即总共110镑。假定花费的资本数量增加一倍。起初有三个新工人被使用。增加的产品应当等于110镑,即三个追加工人的工资加10镑利润。现在假定,追加的100镑以工具、肥料的形式,或者以过去劳动的其他任何结果的形式被使用,而所使用的工人人数保持不变。就算这笔辅助资本平均够用五年。在这种情况下,资本家的年收益必须能够支付[追加资本的]10%的利润,并且用20镑抵补这笔资本的年损耗;因此,要使第二个100镑的继续使用有利可图,所必需的年收益是30镑,而用这100镑来使用直接劳动所必需的金额则是110镑。所以,很明显,在不再能用同量资本来维持追加劳动的时候,农业上的辅助资本的积累也是可能的,而且农业上的这种资本的积累能够在无限长的时期内继续下去。”(第224—225页)“可见,辅助资本的增长,一方面,在直接或间接地花费在土地耕种上的劳动量[1129]相同的情况下,会提高人对地力的支配权,另一方面会减少使进一步使用一定量新资本能获利所必需的年收益。”(第227页)“我们假定,例如有100镑的农业资本,全用来支付工资,并提供10%的利润。在这种情况下,租地农场主的收入等于工人收入的十分之一。如果这种资本增加一倍、两倍等等,那末租地农场主的收入将同工人的收入保持原有的比例。但是,如果工人人数保持不变,而资本量增加一倍,那末利润就变成20镑,或者说,是工人收入的五分之一。如果资本增加三倍,利润就变成40镑,或者说,是工人收入的五分之二。如果资本增加到500镑,利润就变成50镑,或者说,是工人收入的一半。资本家在社会上的财富、影响,也许在某种程度上还有他们的人数,都会与此相应地增长……随着资本的增长,一定数量追加直接劳动的使用,往往也成为必要。然而这种情况并不妨碍辅助资本的连续不断的、相对的增长。”(第231—232页)}\end{quote}

在这段话中首先有一点是重要的,即随着资本的增长,“辅助资本”同可变资本相比会增加,换句话说,可变资本同不变资本相比会相对地减少。

当“辅助资本”中由固定资本构成的部分,即不变资本中其周转历时数年,其价值仅仅以损耗的形式逐年加入产品的那一部分增长的时候,年收益同预付资本相比减少的现象到处都会发生,而不仅仅是在农业中。诚然,在工业中一年内加工的原料量的增加,要比固定资本量的增加快得多(例如,试把一台纺纱机每周以及每年用掉的棉花量,同一架纺车用掉的棉花量比较一下)。但是如果假定,例如(大规模的)裁缝业加工的原料的价值,和纺纱业加工的一样多(裁缝业加工的原料的量虽然少,但是比纺纱业的原料贵),那末裁缝业的年收益就应当比纺纱业的年收益大得多,因为在纺纱业中有较大部分已耗费的资本(固定资本)只是以年折旧的形式加入产品。

在资本增长,但增长的只是不变资本而不是可变资本的时候,农业上(这里可以被看作原料的东西,如种子,不会和不变资本的其他部分,特别是固定资本按同一比例增长)的年收益的价值自然会减少。因为,可变资本必须在产品中全部得到补偿,而固定资本只是以年折旧的形式,按照它每年消费的程度得到补偿。假定谷物价格既定,如果一夸特等于1/2镑,那末要在利润是10%的时候补偿100镑可变资本,就需要220夸特,而补偿20镑的磨损和10镑的利润,只需要60夸特(=30镑)。较少的绝对收益(在这里,和在类似条件下的工业中的情况一样)提供同样的利润。但是在这里琼斯毕竟有种种错误。

首先,不能说(在既定的前提下)土地的生产力增长了。它们增长了,是同直接使用的劳动相比,而不是同使用的全部资本相比而言。只能说,现在需要较少的总产品,就可以提供与以前相同的纯产品,即相同的利润。

[1130]其次,在这种特殊的领域内,同工人的收入相比,租地农场主收入的增长,在这里的总产品中转化为利润的部分同工人所得到的部分相比日益增加的情况下,确实有重要的意义。这样,农场主-资本家的“财富和影响”,同他的工人的“财富和影响”相比,才会不断地增长和扩大。但是,琼斯却似乎是这样计算的:10比100是1/10;20镑比120(即100花在劳动上,20表示损耗)是1/6,而这20镑等于付给工人的工资的1/5,等等。但是笼统地说,花在劳动上的资本减少时利润率会提高,那是再错没有了。恰好相反。在这种情况下剩余价值会相对减少,因而利润率会降低。至于特殊的单个租地农场主(每个单个企业也一样),利润率却能够保持不变,不管在那里200镑资本是使用三个工人,还是使用六个工人。

为了使地租等于超额利润即超过平均利润的余额,前提是,不仅农业要在形式上从属于资本主义生产,而且利润率在各个生产部门中,特别是在农业和工业之间,要平均化。否则地租就会等于超过工资的余额(也就是利润)。地租也可以代表利润的一部分,或者甚至是工资的扣除部分。

\tchapternonum{(2)理·琼斯《1833年2月27日在伦敦皇家学院讲述的政治经济学绪论。附工资讲座大纲》1833年伦敦版。[“国家的经济结构”的概念以及用它来说明社会制度的不同类型的尝试。关于“劳动基金”的混乱思想]}

[琼斯在《绪论》中写道:]

\begin{quote}{“在一个民族的一定历史时期,土地所有权几乎普遍都是或者掌握在国家政府的手里,或者掌握在从政府得到权利的人的手里。”(第14页)“我所说的国家的经济结构,是指各不同阶级之间的关系,这些关系最初由于土地所有权的制定和土地剩余产品的分配而建立起来,后来由于资本家的出现而(在或大或小的程度上)发生了变化和变动,资本家则是作为从事财富的生产和交换并向工人人口提供食物和工作的当事人出现的。”(第21—22页)[1130]\endnote{手稿中接着有一段属于论拉姆赛那一章的简短插话,这段插话以脚注形式放在本册第389页上。——第456页。}}\end{quote}

[1130]琼斯所说的“劳动基金”[《LaborFund》]是指

\begin{quote}{“劳动者所消费的收入总量,不管这些收入的源泉是什么”。(《大纲》第44页)}\end{quote}

琼斯的主要论点(“劳动基金”这个术语也许是属于马尔萨斯的?)\endnote{琼斯称为“劳动基金”(《LaborFund》)的东西,在马尔萨斯那里叫作“用来维持劳动的基金”(《fundsforthemaintenanceoflabour》)。这个术语已经多次出现在马尔萨斯的《人口原理》第一版中(1798年伦敦版第303、305、306、307、312、313页及其他各页)。在第五版(1817年伦敦版)中,它主要出现在第三篇第五章和第六章。在马尔萨斯的《政治经济学原理》中也出现过这个术语,例如在本册第30页所引用的地方。——第456页。}如下:整个社会经济结构是围绕着劳动形式旋转的,也就是说,是围绕着劳动者借以占有自己的生活资料,或者说,占有其产品中他赖以生存的那一部分产品的形式旋转的;这个“劳动基金”有各种不同的形式,资本仅仅是其中的一种形式,是历史上出现较晚的一种形式。亚·斯密提出的那个重大区别——劳动是由资本支付还是直接由收入支付——只有在琼斯那里才得到它能够得到的充分阐明,并且成为理解社会上各种经济结构的一个重要关键。与此同时,这样一种荒诞的观念也因此消失了:似乎因为在资本中工人的收入一开始就以资本家的占有物或积蓄物的形式出现,所以这就不仅仅是一种形式上的区别。

\begin{quote}{“甚至在西欧各国我们还能发现由它们的土地和劳动的产品的特殊分配方式所产生的社会制度的影响,这种分配方式是[1131]在它们作为农业国存在的初期形成的〈也就是说,分配是在下面几个阶级之间进行:(1)农业劳动者阶级,(2)土地所有者阶级,(3)仆人、侍从和手工业者,这些人直接或间接地分享土地所有者的收入〉。”(《绪论》第16页)“这种经济结构经受的变动,其基本因素和动力是资本,即为了赚取利润而使用的积累财富……在一切国家中,这里所指的财富的特殊部分,对于社会各个不同阶级之间的联系的变化起着极大的作用,并且对于这些阶级的生产力发生决定的影响……在亚洲以及在欧洲的一部分(以前是在全欧洲)非农业阶级几乎完全靠其他阶级的收入,主要是靠土地所有者的收入维持生活。如果你需要一个手工业者的劳动,你就供给他材料;他到你家里来,你管他饭,并且付给他工资。过了一段时间,出现了资本家;他备置材料,预付工人的工资,成为工人的雇主,并且是生产出来的产品的所有者,他用这种产品交换你的货币……这样,在土地所有者和一部分非农业劳动者之间就有了一个中间阶级,这些非农业劳动者现在要靠这个中间阶级来得到工作和生存资料了。以前联结社会的纽带现在削弱和瓦解了;另外的联系,另外的相互依赖原则现在联结着社会的各个不同阶级,新的经济关系出现了”……“这里,在英国,不仅绝大多数非农业劳动者几乎完全靠资本家雇用,而且农业劳动者也成了资本家的仆人。”(同上,第16页及以下各页)}\end{quote}

琼斯的《工资讲座大纲》和他的论地租的书有以下区别。在论地租的书中考察的是土地所有权的各种不同形式,和这些形式相适应的则是劳动的各种不同社会形式。在《大纲》中,琼斯从劳动的这些不同形式出发,并且把土地所有权的各种不同形式和资本当作它们的产物来考察。劳动者的劳动的社会规定性,和劳动条件——特别是土地、自然界,因为这个关系包括其他一切关系——对劳动者所采取的形式相适应。但是,实际上劳动者的劳动的这个社会规定性只是在上述形式中得到自己的客观表现。

因此,我们将看到,“劳动基金”的各种不同形式,是和劳动者同他自己的生产条件发生关系的不同方式相适应的。他以什么方式占有自己的产品(或产品的一部分),要看他同他的生产条件发生什么关系。

\begin{quote}{琼斯说:“劳动基金可以分为三类:(1)由劳动者自己生产并由他们自己消费的收入,这些收入决不属于其他任何人。{在这种情况下,劳动者实际上必然是他自己的生产工具的所有者,不管他的收入具有什么样的特殊形式。}(2)属于和劳动者不同的那些阶级的收入,这些阶级花费这些收入来直接维持劳动。(3)真正的资本。劳动基金的所有这些不同的种类都可以在我们本国观察到;但是,如果我们看看其他国家,我们就会发现,这个基金的某些部分在我们这里极为有限,在其他一些国家却是居民生存的主要源泉,并且决定着多数国民的性格和状况。”(《大纲》第45—46页)关于第一点。“农业劳动者,或者说,占有土地的农民的工资……这些农业劳动者,或者说农民,是份地的继承者、私有者、佃农。佃农就是农奴、分成制佃农、茅舍贫农。后者是爱尔兰特有的。所有这几种土地耕种者-农民的收入中往往混有地租或利润之类的东西,但是,如果他们主要是依靠自己体力劳动的报酬生活,他们就应当被看作工资劳动者[wageslabourers]。因此,劳动农民中包括:(α)份地的继承者,他们是农业[1132]劳动者。古代的希腊。现今的亚洲,尤其是印度。(β)农民-私有者。法国、德国、美国、澳大利亚、古代的巴勒斯坦。(γ)茅舍贫农。”(第46—48页)}\end{quote}

这里值得注意的是,劳动者为自己再生产“劳动基金”。这种“劳动基金”不转化为资本。劳动者直接生产自己的“劳动基金”,他也直接占有它,尽管他的剩余劳动,按照他和他自己的生产条件发生关系的特殊形式,由他自己占有全部或一部分,或者全部由其他阶级占有。琼斯把这类劳动者叫作“工资劳动者”,这纯粹是经济学上的偏见。他们并没有工资劳动者即雇佣工人的任何特征。既然在资本统治下归工人自己占有的那部分产品是工资,那末归任何一个劳动者自己消费的那部分产品也就必然是工资,——这是一种资产阶级政治经济学的美妙的概念。

\begin{quote}{关于第二点。[靠这种劳动基金维持生活的人口,]“在英国限于家仆、士兵、水手和少数独立从事劳动并从其雇主的收入中得到支付的手工业者。在地球上相当大的地区,这种劳动基金维持着几乎所有的非农业劳动者。以前这种基金在英国占优势。沃里克——国王制造者\endnote{琼斯在《关于劳动和资本的讲义》中关于沃里克是这样说的:“据说,著名的沃里克伯爵,‘国王制造者’,不得不每天在自己的城堡和家里供养四万人。”——第459页。}。英国的贵族。如今这种基金在东方占优势。手工业者、仆人。靠这种基金维持的庞大军队。在整个亚洲这种基金集中于君主之手所产生的后果。某些城市的突然兴盛。突然衰落。撒马尔汗、坎大哈等地”。(第48—49页)}\end{quote}

琼斯忽略了两个主要的形式:第一,具有农业和工业合一特点的亚洲村社;第二,中世纪的城市行会制度,这种制度部分地在古代世界也存在过。

\begin{quote}{关于第三点。“资本决不应和世界上的一般劳动基金混淆起来,劳动基金的大部分是由收入构成的。国家的各种收入……都参加资本赖以形成的积累。在不同的国家和不同的社会发展阶段,它们以不同的程度参加这种积累。例如有这样的情况,即主要依靠工资和地租进行积累。”(第50页)}\end{quote}

剩余劳动转化为资本(而不是直接作为收入同劳动交换),这就造成一种印象,似乎资本是收入的积蓄。这就是琼斯的主要观点。确实,在社会发展过程中,资本量是由再转化为资本的收入构成的。但是在资本主义生产中,连最初的“劳动基金”本身也表现为资本家的积蓄。再生产出来的“劳动基金”本身,不象在第一种情况下那样为劳动者所占有,而是表现为资本家的财产,表现为对工人来说是别人的财产。而这一点是琼斯没有阐明的。

琼斯在这个教学大纲里关于利润率及其对积累的影响的论述是软弱无力的:

\begin{quote}{“在其他一切条件相同的情况下,一个国家从它的利润中进行积蓄的能力,随着利润率的变化而变化:这种能力在利润率高时就大,在利润率低时就小;但是在利润率下降时,其他一切条件就不会保持不变。使用的资本量和人口数目相比可能增加。”}\end{quote}

(琼斯不懂得,怎样由于使用的资本“可能”增加而发生如下情况:正是因为“使用的资本量和人口数目相比增加了”,利润率才下降。但是他正在接近于正确的观点。}

\begin{quote}{“积累的动因和便利条件可能增加……在利润率低时积累的速度通常会比人口的增加快,例如在英国;在利润率高时积累的速度通常会比人口的增加慢,[1133]例如在波兰、俄国、印度等地。”(第50—51页)}\end{quote}

在利润率高的地方(撇开北美的情况不谈,在那里,一方面,资本主义生产占统治地位,另一方面,一切农产品的价值都低),利润率高通常是由于,资本主要由可变资本构成,即直接劳动占优势。假定资本是100,其中五分之一是可变资本,并且假定剩余劳动是三分之一工作日。在这种情况下利润等于10%。现在假定五分之四的资本由可变资本构成,而剩余劳动是六分之一工作日。在这种情况下利润等于16%。

\begin{quote}{“有一种理论错误地以为,在利润率随着国家的发展而降低的地方,日益增长的人口的生存资料必定减少。这种错误的基础是:(1)错误的概念,即利润的积累在利润率低的地方必定会慢,在利润率高的地方必定会快;(2)错误的假定,即利润是积累的唯一源泉;(3)错误的假定,即地球上所有劳动者的生活都是依靠积累和收入的积蓄,而决不是依靠收入本身。”(第51页)}\end{quote}

[琼斯指出]

\begin{quote}{“当资本负起预付工资的责任时,国家的经济结构中就会发生变动”。[1133]}\end{quote}

[1157]理·琼斯在下面的论述中作了正确的概括:[1157]

\begin{quote}{[1133]“用于维持劳动的资本量可以独自发生变化,而不管资本总量的变化如何〈这是一个重要的论点〉……有时可以看到,当资本本身变得更加充裕的时候,在业人数的大变动,以及由此而来的大灾难,就会变得更加频繁。”(第52页)\endnote{这段引文马克思在手稿第XVIII本的最后一页(手稿第1157页)又引了一次,并且增加了这里所引用的马克思的补充评论。——第460页。}[1133]}\end{quote}

[1157]总资本可以保持不变,但是可变资本可以发生变化(特别是减少)。资本两个组成部分的比例的变化,不一定意味着总资本量方面的变化。

另一方面,总资本的增长不仅可以和可变资本的相对减少有关,而且可以和它的绝对减少有关;总资本的增长总是和可变资本的剧烈变动有关,因此,也和“在业人数的变动”有关。[1157]

[接着,琼斯在这个关于工资的教学大纲中写道:]

\begin{quote}{[1133]“劳动者从依赖一种基金逐渐向依赖另一种基金过渡的各个时期……农业劳动者向着由资本家支付报酬转变……非农业阶级向着受资本家雇用转变。”(第52—53页)}\end{quote}

琼斯在这里所说的“转变”,就是我所说的“原始积累”。只有形式上的区别。它也是和庸俗的“积蓄”观点对立的。

\centerbox{※     ※     ※}

\begin{quote}{“奴隶制。奴隶可以划分为牧羊奴隶、耕作奴隶、家庭奴隶,最后,还有一种既是耕作奴隶又是家庭奴隶的混合型奴隶。我们看到,有的奴隶是耕种土地的农民,有的是靠富人的收入维持生活的仆人或手工业者,有的是靠资本维持生活的工人。”(第58—59页)}\end{quote}

但是只要奴隶制占统治地位,资本主义关系就每次只能偶然地作为从属关系出现,决不能作为统治的关系出现。

\tchapternonum{(3)理·琼斯《国民政治经济学教程》1852年哈特福版}

\tsectionnonum{[(a)资本主义生产方式的历史观的萌芽同关于资本只是“积累的储备”的资产阶级拜物教观点的结合。生产劳动和非生产劳动问题]}

[琼斯在《国民政治经济学教程》中写道:]

\begin{quote}{“国民的劳动生产率实际上取决于两种情况。第一,取决于他们所生产的财富的原始源泉〈土地和水〉是富饶还是贫乏。第二,取决于在利用这些源泉或者在对取自这些源泉的商品进行加工时,他们所使用的劳动的效率如何。”(第4页)“人的劳动效率取决于:(1)劳动的连续性;(2)劳动用来实现生产者的目的所具备的知识和技能;(3)帮助劳动的机械力。”(第6页)“劳动者在生产财富时所使用的力……可以由以下几点来增强:(1)让比他们自身的动力大的动力为他们服务……(2)用更能发挥机械效益的方法去使用他们所拥有的某一数量或某一种类的动[1134]力。例如,40马力的蒸汽机在铁路上发挥的效力,和在公路上发挥的就不同。”(第8页)“用两匹马拉一种较好的犁,可以和用四匹马拉一种较坏的犁完成同样多和同样好的劳动。”(第9页)“蒸汽机不是一种单纯的工具;它能提供追加动力,而不单是提供更能发挥机械效益地使用工人已经拥有的力的手段。”(第10页注)}\end{quote}

可见,在琼斯看来,工具和机器的差别就是如此。工具向工人提供更能发挥机械效益地使用他已经拥有的力的手段;机器能使动力增加。(?)

\begin{quote}{“资本……就是由收入中积蓄起来并用来获取利润的财富所构成的。”(第16页)“资本的可能的源泉……显然是构成社会的所有个人的所有可以积蓄起来的收入。最有助于国民资本进步的几种特殊收入,在它们各个不同发展阶段上是不同的,因此它们在处于这种发展的不同阶段的各个国家里也是截然不同的。”(同上)“因此,利润决不是资本形成和增加的唯一源泉,而在社会的初期阶段,同工资和地租相比,利润甚至是一个不重要的积累源泉。”(第20页)“当国民劳动的力量真正得到显著发展时,利润作为积累的源泉就相当重要了。”(第21页)}\end{quote}

按照这种说法,资本是构成收入的那种财富的一部分,这一部分不是作为收入被消耗,而是用来生产利润。利润已经是剩余价值的一种形式,这种形式专门以资本为前提。如果以资本主义生产方式的存在即资本的存在为前提,那末琼斯的解释是对的。换句话说,在应当解释的东西被当作前提的时候,他的这个解释是对的。但是琼斯在这里所指的是一切不作为收入被消耗的收入,而是为达到致富目的即在生产上被消耗的收入。

不过,在这里有两点是重要的。

第一,在经济发展的一切阶段上都有一定的财富积累,也就是说,一部分采取扩大生产规模的形式,一部分采取货币贮藏之类的形式。当“工资”和地租在社会上占优势的时候,——也就是按照上面所说,这时,一般不归劳动者自己所有的剩余劳动和剩余产品的大部分归土地所有者所有(在亚洲,归国家所有),另一方面,劳动者则自己再生产自己的“劳动基金”,不仅自己生产自己的“工资”,而且自己把它付给自己,并且在大多数情况下(在这种社会状况下,几乎是经常)他至少能够使自己得到自己的剩余劳动和剩余产品的一部分,——在这样的社会状况下,“工资”和地租也是积累的主要源泉。(在这里利润只限于商人等等的利润。)只有当资本主义生产占统治地位,当它不只是偶尔存在,而是使社会的生产方式从属于它时;当资本家实际上把全部剩余劳动和全部剩余产品首先直接占为己有,尽管他不得不把其中的一部分付给土地所有者等等时,——只是从这时候起,利润才成为资本的主要源泉,积累的主要源泉,由收入中积蓄起来并用来获取利润的财富的主要源泉。同时这要有一个先决条件(在资本主义生产方式占统治地位的情况下,这是不言而喻的),即“国民劳动的力量真正得到显著发展”。

有一些蠢驴,他们以为,没有资本的利润,就不会有积累,或者他们这样来为利润辩护,说资本家做出了牺牲,为了生产的目的,由自己的收入中进行积蓄;对于这些人,琼斯回答说,“积累”这个职能,正是在这种特殊的生产方式(资本主义生产方式)下,才主要由资本家承担,而在以前的生产方式下,在这个过程中起主要作用的是劳动者自己,部分是土地所有者,利润在那里几乎不起任何作用。

当然,积累的职能总是会转到这样的人的身上:(1)占有剩余价值,(2)特别是占有剩余价值,同时又是生产本身的当事人。因此,如果有人说,[1135]因为资本家通过由利润中进行“积蓄”来积攒自己的资本,因为他执行积累的职能,所以利润是合理的,那末,这只是说,资本主义生产方式因为事实上是存在的,所以是合理的,这种说法对过去的和以后的生产方式也同样适用。如果有人说,用其他办法不可能进行积累,那就是忘记了,这个特定的积累方法——通过资本家进行积累——有其发生的特定的历史日期,并且会朝着其消亡的(也是历史的)日期走去。

第二,既然有那么多积累的财富通过各种手段转到资本家手中,以致他们能够支配生产,那末最大量的现有资本——经过一定期间——可以被认为完全是由利润(收入)产生的,也就是由资本化的剩余价值产生的。

有一点琼斯提得不够,说实在的,他只是作了暗示,这就是:如果劳动生产者自己付给自己“工资”,并且他的产品不是先采取由他人的收入中“积蓄起来”的形式,然后再由他人付还给劳动者,那末劳动者就必须占有自己的生产条件(无论是作为私有者,还是作为佃农或份地的继承者等等)。要使他的“工资”(以及“劳动基金”)作为别人的资本同他相对立,就必须预先使他丧失这些生产条件,而这些生产条件必须采取别人的财产的形式。只有在劳动者的“劳动基金”连同他的生产条件被夺去,并且作为与工人对立的资本独立出来以后,进一步的过程(这个过程涉及的不是这些原有条件的单纯的再生产,而是它们的进一步发展)才会开始,以致生产条件和“劳动基金”都是作为由他人的收入中“积蓄起来”以便转化为资本的东西出现在工人面前。工人丧失了对自己的生产条件的占有,因而也丧失了对自己的“劳动基金”的占有以后,也就丧失了积累的职能,他在财富上所追加的一切,都表现为他人的收入,而这种收入必须预先被这些人“积蓄起来”,即不应作为收入消耗掉,才能执行资本的职能以及工人的“劳动基金”的职能。

因为琼斯本人所叙述的是这样的社会状况,在这种社会状况下,事情还不是这个样子,当时[劳动者和劳动条件之间]还存在着统一,所以他自然必须把上述的“分离”当作资本真正的形成过程提出来。一旦有了这种“分离”,资本的形成过程自然就会发生,——它将继续并且扩大,——因为工人的剩余劳动现在总是作为别人的收入同工人相对立,也只有通过这种收入的“积蓄”才能发生财富的积累和生产规模的扩大。

收入再转化为资本。如果资本{即生产条件和劳动者相分离)是利润的源泉(也就是说,剩余劳动表现为资本的收入,而不表现为劳动的收入},那末现在利润又成了资本的源泉,成了新资本形成的源泉,也就是说,追加的生产条件作为资本同工人相对立,作为手段来保持工人的工人身分并一再占有工人的剩余劳动。劳动者和劳动条件之间原有的统一(我们不谈奴隶关系,因为当时劳动者自身属于客观的劳动条件}有两种主要形式:亚洲村社(原始共产主义)和这种或那种类型的小家庭农业(与此相结合的是家庭工业)。这两种形式都是幼稚的形式,都同样不适合于把劳动发展为社会劳动,不适合于提高社会劳动的生产力。因此,劳动和所有权(后者应理解为对于生产条件的所有权)之间的分离、破裂和对立就成为必要的了。这种破裂的最极端的形式(在这种形式下社会劳动的生产力同时会得到最有力的发展)就是资本的形式。原有的统一的恢复,只有在资本创造的物质基础上,并且只有通过工人阶级和整个社会在这个创造过程中经历的革命,才有可能实现。

琼斯提得不够的还有下面一点:

直接作为收入同劳动交换的那种收入,只要它不是雇用次要劳动者的独立劳动者的收入,那就是土地所有者的收入,这种收入来源于独立劳动者付给他的地租,他同他的仆人和侍从没有以实物形式把这种地租完全消费掉,而用其中的一部分来购买次要劳动者的产品和服务。因此,收入同劳动的这种交换总是以第一种关系[土地所有者和向土地所有者交付地租的独立劳动者之间的关系]为前提。

[1136]{即使产业资本家使用他自己的资本,也有一部分利润被看作是利息,这仅仅是因为,这种收入具有单独的存在形式,同样,在资本主义生产方式的基础上,即使劳动者拥有自己的生产资料,并且不雇用其他任何劳动者,这些生产资料仍被看作是资本,而劳动者自己的在普通工资以外实现的那部分劳动,表现为由他的资本产生的利润。在这种情况下,劳动者本身将分解为不同的经济身分。他作为他自己的工人得到自己的工资,又作为资本家得到自己的利润。这个评论属于《收入及其源泉》\endnote{关于收入及其源泉,马克思在1861—1863年手稿第XV本后半部分作了论述,他在这方面揭示了庸俗政治经济学的阶级根源和认识论根源(见本册第499—600页)。这个“补充部分”(这是马克思在手稿第XIV本封面上对它的称呼),也就是对《剩余价值理论》正文的补充,马克思后来决定放在《资本论》第三部分,这从他在1863年1月拟定的这一部分的计划可以看出;按照这一计划,第九章的标题应该是《收入及其源泉》(见本卷第1册第447页)。——第466页。}那一章。}

\begin{quote}{“有两种财富,它们对于国民生产力的影响是有区别的,一种财富是积蓄起来,并且作为工资支出来获取利润,一种财富是从收入中预付来维持劳动。说到这种区别时,我用资本这个词,只是为了表示由收入中积蓄起来并用来获取利润的那部分财富。”(第36—37页)“我们也许可以……把资本理解为一切被用来维持劳动的财富,而不问这种财富是否经过了预先的积蓄过程……在这种情况下,如果我们仔细研究一下不同国家中和不同条件下的劳动阶级以及向劳动阶级支付的人的状况,我们就必须把积蓄起来的资本和没有经过积累过程的资本加以区别;一句话,把是收入的资本和不是收入的资本加以区别。”(第36页)“除了英国和荷兰,在旧大陆的一切国家中,农业劳动者的工资不是从由收入积蓄和积累的基金中预付,而是由劳动者自己生产出来,并且除了以供他们自己直接消费的储备形式存在之外,从来不以其他任何形式存在。”(第37页)}\end{quote}

琼斯和其他政治经济学家(也许西斯蒙第除外)不同的地方是,他把资本的社会的形式规定性作为本质的东西强调出来,并把资本主义生产方式和其他生产方式之间的一切区别归结为这个形式规定性。资本的这个社会的形式规定性就是,劳动直接转化为资本,另一方面,这个资本购买劳动不是为了它的使用价值,而是为了增加自己本身的价值,为了创造剩余价值(更高的交换价值),“用来获取利润”。

但是,这里同时也表明,为使收入转化为资本而进行的“收入的积蓄”本身,以及“积累”本身,仅仅在形式上不同于使“财富被用来维持劳动”的其他条件。领取由资本“预付的”工资的英国和荷兰的农业工人,如同法国农民或俄国的独立经营的农奴一样,也是“自己生产自己的工资”。如果从生产过程的连续性来考察生产过程,那末资本家今天作为“工资”预付给工人的,不过是这个工人昨天“生产出来的”产品的一部分。因此,[资本主义生产方式和其他生产方式之间的]区别并不在于,在一种场合工人生产他自己的工资,而在另一种场合不生产。区别在于,他的这个产品[在一种场合]表现为工资;还在于,在一种场合[在资本主义生产方式的条件下]工人的产品(工人的产品中构成“劳动基金”的那一部分),第一,表现为别人的收入;但是,第二,不当作收入来花费,也不花费在收入借以直接消费掉的劳动上面,第三,作为资本和工人相对立,资本把产品的这一部分归还给工人,不是简单地去交换等价物,而是去交换比物化在给工人的产品中的劳动量更大的劳动量。因此,工人的产品表现为,第一,别人的收入,第二,收入的“积蓄”,用来购买劳动以便获得利润,也就是说,它表现为资本。

工人自己的产品作为资本和他相对立的这个过程,也就是琼斯所说的:“劳动基金”“经过预先的积蓄过程”,“经过积累过程”,即在再转化为工人的生存资料以前,“以另一种形式存在”(琼斯在这里直接谈判的只是形式变化),而不是“以供劳动者直接消费的储备形式存在”。全部区别在于工人所生产的“劳动基金”在以工资形式重新回到工人手里以前所经受的那个形式转化。因此,在独立农民或独立手工业者那里,“劳动基金”决不会采取“工资”的形式。

[1137]就“劳动基金”来说,“积蓄”和“积累”在这里不过是劳动者的产品经受的那些形式转化的名称。独立劳动者和雇佣工人一样,消费自己的产品,或者更确切地说,后者和前者一样,消费自己的产品。区别只在于,在雇佣工人那里,他的产品表现为由另一个人——资本家——的收入中积蓄起来,或者说,积累起来的东西。实际情况则是,这个过程使资本家能够为自己“积蓄”,或者说,“积累”工人的剩余劳动;因此,琼斯也非常有力地强调了这种情况:在非资本主义生产方式下,积累不是来源于利润,而是来源于“工资”,即来源于独立农业劳动者的收入或用自己的劳动直接同收入交换的手工业者的收入(否则怎样会从这些独立农业劳动者和独立手工业者当中产生出资产者呢?),并且来源于土地所有者得到的地租。但是为了使“劳动基金”经过这些转化,还必须使生产条件象资本那样和劳动者相对立,在其他形式中情况就不是这样。在这一场合,财富的增加,不表现为来源于工人,而表现为通过积蓄,通过剩余价值再转化为资本而从利润得来,——正如“劳动基金”本身(在由于新的积累而增加以前)作为资本和工人相对立一样。

“积蓄”,就这个词的本来意义讲,只有在谈到那种把自己的收入资本化的资本家对那种把自己的收入作为收入吃光花光的资本家的关系时才有意义,它对于说明资本家和工人之间的关系则毫无意义。

有两个主要事实可以说明资本主义生产的特点:

第一,生产资料积聚在少数人手中,因此不再表现为单个劳动者的直接财产,而表现为社会的生产能力,尽管首先表现为不劳动的资本家的财产。在资产阶级社会里,这些资本家是生产资料的受托人,并享受从这种委托中得到的全部果实。

第二,劳动本身由于协作、分工以及劳动同社会对自然力支配的结果相结合,而组织成为社会的劳动。

从这两方面,资本主义生产把私有财产和私人劳动取消了,虽然还是处在对抗的形式中。

在亚·斯密看来,生产劳动和非生产劳动的主要区别是,前者直接同资本交换,后者直接同收入交换,——这一区别的意义,只是在琼斯那里才得到充分的阐明。这里指出,第一种劳动说明资本主义生产方式的特征;第二种劳动,在它占统治的地方,属于以前的各种生产方式,而在它只是间或出现的地方,则限于(或者应当限于)那些不直接生产财富的领域。

\begin{quote}{“资本是一种工具,借助这种工具,可以把能够提高人的劳动效率和国民生产力的一切因素推动起来……资本是过去劳动的积累起来的结果,这个结果被用来在财富生产工作的某个部分获得某种效果。”(第35页)}\end{quote}

(琼斯在对这段话作的注释中说:

\begin{quote}{“我们可以并且有理由认为,在所生产的商品没有到达要消费它的人手中以前,生产行为并没有结束;在此以前所做的一切,都是为着这一目的。给我们把茶从哈特福运到我们学院来的小店主的马和车,对于我们要得到茶以满足消费的目的来说,是同采茶和焙茶的中国人的劳动一样必要的。”)“但是……这个资本……并不是在每个社会中都完成它所能完成的一切任务。在一切场合,它都是逐渐地、依次地着手完成这些任务,一个值得注意并且极端重要的事实是,有一种特殊职能,它的执行对于资本的力量在资本的其他一切职能中的重大发展是非常重要的,这种特殊职能正是资本对地球上大部分劳动者还从来没有执行过的那种职能。”(第35—36页)“我指的是工资的预付。”(第36页)“由资本家预付工资的劳动者,在地球上还不到四分之一……这一事实……在说明各国的进步程度时有头等重要意义。”(同上)[1138]“资本,或者说,积累的储备,只是后来,当它在财富的生产中执行了其他各种职能以后,才担负起向劳动者预付工资的职能。”(第79页)}\end{quote}

在琼斯的最后这句话(第79页)中,资本实际上被说成是“关系”,被说成不仅是“积累的储备”,而且是完全确定的生产关系。“储备”不可能“担负起预付工资的职能”。琼斯着重指出,资本的基本形式,就是资本和雇佣劳动相对立,并支付工资;资本的这种形式赋予整个社会生产过程一种特征,支配它,使社会劳动生产力达到全新的发展,同时使一切社会的和政治的关系革命化。琼斯着重指出,资本在担负起这个具有决定意义的职能以前,执行了其他的职能,表现为其他一些从属的但在历史上是较早的形式,但是只有随着它作为产业资本出现,它的力量在它的一切职能中才得到充分的发挥。另一方面,在他的第三篇讲义《论资本或者资本家{在这里,问题就在这个“或者”上:只是由于这种人格化,积累的储备才成为资本}如何逐渐地担负起财富生产中的一连串职能》中,琼斯并没有告诉我们,这些较早的职能是什么。实际上,这只能是商业资本或货币经营资本的职能。但是,尽管琼斯如此接近了正确的观点,并在一定形式上说出了这种观点,另一方面,他作为政治经济学家,却仍然在很大程度上为资产阶级拜物教所束缚,以致连魔鬼也不能担保,他是否理解“积累的储备”本身所能完成的各种职能。

琼斯说:

\begin{quote}{“资本,或者说,积累的储备,只是后来,当它在财富的生产中执行了其他各种职能以后,才担负起向劳动者预付工资的职能。”(第79页)}\end{quote}

这句话充分表现出这样一种矛盾:一方面,琼斯对资本有正确的历史的理解,另一方面,这种理解又被政治经济学家所固有的狭隘见解即“储备”本身就是“资本”弄得模糊不清。因此,“积累的储备”在琼斯那里成为一个“担负起”向劳动者“预付工资的职能”的人。琼斯在破除政治经济学家所固有的偏见时自己还是为这种偏见所束缚,——既然资本主义生产方式被看作历史上一定的生产方式,而不再是生产的永恒的自然关系,那末破除这种偏见就是必要的。

我们看到,从拉姆赛到琼斯有了多么大的飞跃。正是资本的那个使资本成为资本的职能——预付工资——被拉姆赛说成是偶然的,只是由大多数人的贫困引起的,对于生产过程本身是无关紧要的。拉姆赛用这种狭隘的形式否认资本主义生产方式的必然性。而琼斯{奇怪的是,他们两人都是英国国教会的牧师\endnote{这里提到的两个英国经济学家当中,只有琼斯是牧师。——第472页。}。看来英国的牧师比大陆上的还是考虑得多些}指出,正是这个职能使资本成为资本,并且决定资本主义生产方式的特征。琼斯指出,这个形式怎样在生产力发展的一定阶段才会发生,并且那时就会创造出全新的物质基础。但是琼斯因此也以比拉姆赛更深刻的完全不同的方法理解这个形式的“可废止性”,理解它的仅仅是历史上暂时的必然性。琼斯决没有把资本主义的关系看作永恒的关系。

\begin{quote}{“将来可能出现这样一种情况,——世界各大洲可能会逐渐接近这样一种情况,——在这种情况下,劳动者和积累的储备的所有者将是同一的;但是在各国的发展中……这种情况至今还从未有过,为了探索和理解这种发展,我们必须考察劳动者怎样逐渐地从用自己的收入支付劳动者报酬的主顾的支配下,转到用资本(它的所有者指望从它的总产品中为自己实现一种特殊的收入)的预付支付劳动者报酬的企业主的支配下。也许,这种情况同劳动者和资本家是同一个人的情况相比,还不是那么令人满意;但是我们必须仍然把它看作生产发展进程中的一定阶段,这个阶段直到现在还是先进国家的发展的特征。亚洲的居民还没有达到这个阶段。”(第73页)}\end{quote}

[1139]在这里琼斯直截了当地宣称,他把资本和资本主义生产方式只“看作”社会生产发展中的一个过渡阶段,从社会劳动生产力的发展来看,这个阶段同一切过去的形式相比是一个巨大的进步,但是这个阶段决不是最终的结果,而是相反,在它固有的对抗形式中,即在“积累的财富的所有者”和“实际的劳动者”之间的对抗形式中,包含着它灭亡的必然性。

琼斯曾在海利贝里任政治经济学教授,是马尔萨斯的继任者。在这里我们看到,政治经济学这门实际科学是怎样结束的:资产阶级生产关系被看作仅仅是历史的关系,它们将导致更高级的关系,在那里,那种成为资产阶级生产关系的基础的对抗就会消失。政治经济学以自己的分析破坏了财富借以表现的那些表面上相互独立的形式。它的分析(甚至在李嘉图那里就已经)进行得如此远了:

(1)财富的独立的物质形式趋于消灭,财富不过表现为人的活动。凡不是人的活动的结果,不是劳动的结果的东西,都是自然,而作为自然,就不是社会的财富。财物世界的幻影消逝了,这个世界不过表现为不断消失又不断重新产生的人类劳动的客体化。任何物质上持久的财富都只是这个社会劳动的转瞬即逝的物化,是生产过程的结晶化,生产过程的尺度是时间,即运动本身的尺度。

(2)财富的不同组成部分,通过各种各样的形式流入社会的不同部分,这些形式正在丧失自己的表面的独立性。利息,仅仅是利润的一部分,地租,仅仅是超额利润。因此,不论利息还是地租都溶解在利润里面,而利润本身则归结为剩余价值即无酬劳动。但是商品价值本身只归结为劳动时间。李嘉图学派甚至走得这样远,以至把这个剩余价值的占有形式之一——土地所有权(地租)——当作无用的形式加以否定,只要得到它的是私人[而不是国家]。李嘉图学派不承认土地所有者是资本主义生产的职能执行者。这样,对抗就归结为资本家和雇佣工人之间的对抗。但是李嘉图学派的政治经济学把资本家和雇佣工人之间的这种关系看作某种既定的东西,看作生产过程本身所依据的自然规律。后来的经济学家,象琼斯这样的经济学家,超过了这一点,他们只承认这种关系的历史的合理性。但是,自从资产阶级生产方式以及与它相适应的生产关系和分配关系被认为是历史的以来,那种把资产阶级生产方式看作生产的自然规律的谬论就宣告破产了,并且开辟了新社会的远景,开辟了新的经济社会形态的远景,而资产阶级生产方式只构成向这个形态的过渡。\endnote{在手稿中接下去是《资本论》第三部分或第三篇的计划草稿——《资本和利润》,作为插入部分放在方括号内。本版把这个计划收入本卷第1册《附录》(第447页)。——第474页。}[1139]

[1139]关于琼斯,我们还要考察几个问题:

(1)资本主义生产方式——由资本预付工资——究竟怎样改变[生产]形式和生产力。(2)琼斯关于积累和利润率的论断。

但是这里首先还要指出下面一点:

\begin{quote}{[1140]“资本家只是一种中介人,他使劳动者在新的形式和新的情况下从周围的主顾所支出的收入中得到利益。”(第79页)}\end{quote}

这里说的是过去直接靠土地所有者等等的收入生活的非农业劳动者。现在不是这些劳动者用自己的劳动(或自己劳动的产品)直接同这种收入交换,而是资本家用这些劳动者的劳动产品——收集和集中在他手中的——同这种收入交换,或者说,收入转化为资本,同资本交换,同时构成资本的收益。这种收入现在不是构成劳动的直接收益,而是构成使用工人的资本的直接收益。\endnote{在手稿(第1140—1144页)中接下去是:《资本论》第一部分或第一篇的计划草稿——《资本的生产过程》,本版把这个计划收入本卷第1册《附录》(第446页),再往下是从报刊杂志和书籍上摘录的关于利率的高度、资本家对工人阶级的剥削、不变资本和可变资本之间的各种比例等问题的材料。马克思在《资本论》第一卷和第三卷中引用了这里的某些摘录。第1142页上有关庸俗政治经济学把利润看作资本家的“工资”的辩护论观点的一小段话,收入本册《附录》(第553页)。——第474页。}[1140]

[1144]琼斯把资本作为特殊的生产关系来描述,认为这种生产关系的主要特征是:积累的财富表现为预付的工资,“劳动基金”本身则表现为“由收入中积蓄起来并用来获取利润的财富”,然后,他就从生产力的发展中考察这一生产方式所特有的变化。琼斯很好地论述了,怎样随着物质生产力的变化,经济关系以及与此相连的国民的社会状况、道德状况和政治状况,也都在发生变化:

\begin{quote}{“随着各社会改变自己的生产力,它们也必然改变自己的习俗。社会上所有各个不同的阶级在其发展进程中都会发觉,新的关系已把它们同其他阶级联系起来,它们处在新的地位,并被新的道德的和社会的危险所包围,被社会进步和政治进步的新条件所包围。”(第48页)}\end{quote}

在考察琼斯怎样说明资本主义生产形式对生产力发展的影响之前,还要引几段同我们上面所引的有联系的话。

\begin{quote}{“随着社会的经济组织以及生产任务借以完成的因素和手段(丰富的或贫乏的)的变化,会发生大的政治的、社会的、道德的和精神的变化。这些变化发生在居民当中,必然对居民的各种政治要素和社会要素产生决定性的影响;这种影响将涉及国民的精神面貌、习惯、风俗、道德和幸福。”(第45页)“英国是这样一个唯一的大国,它……作为一个生产的机构,在向着完善前进的过程中迈出了第一步;只有这个国家,它的居民,不论是农业的还是非农业的,都受资本家的指挥,在这个国家里,不仅在它的财富的巨大增长中,而且在它的居民的一切经济关系和地位中,处处可以感觉到资本家所拥有的手段和唯有资本家才能执行的那些特殊职能所产生的影响。但是英国——我遗憾地但毫不犹豫地说——对于用这种方法发展自己的生产力的人民的经历[1145]来说,决不是一个成功的范例。”(第48—49页)“一般劳动基金由以下几部分构成:(1)劳动者自己生产的工资;(2)其他阶级用于维持劳动的收入;(3)资本,或者说,由收入中积蓄起来并用来预付工资以便获取利润的财富。我们把靠第一部分劳动基金维持生活的人叫作非雇佣劳动者。靠第二部分劳动基金维持生活的人叫作领薪金的服务人员。靠第三部分劳动基金维持生活的人叫作雇佣工人。从这三部分劳动基金中的哪一部分领取工资,这决定劳动者和社会其他阶级的相互关系,并从而决定——有时直接地,有时或多或少间接地——完成生产任务的连续性、技能和力量的程度。”(第51—52页)“世界上的劳动人口有半数以上,也许甚至三分之二以上,是靠第一部分劳动基金,即劳动者自己生产的工资维持生活的。这些劳动者到处都由占有并耕种土地的农民构成……第二部分劳动基金,即用于维持劳动的收入,维持着东方绝大部分非农业生产劳动者。这一部分劳动基金在欧洲大陆有一定的重要性,但在英国它只包括人数不多的做零工的手工业者,他们是一个人数众多的阶级的残余……第三部分劳动基金,即资本,在英国雇用了大多数劳动者,然而在亚洲它只维持着不多的人,在欧洲大陆这部分基金只维持着非农业劳动者,他们总共也许不到全部生产人口的四分之一。”(第52页)“我没有把奴隶劳动作为一个特别的范畴……劳动者的公民权对于他们的经济地位不发生影响。可以看到,奴隶象自由民一样,靠某种形式的劳动基金维持生活。”(第53页)}\end{quote}

但是,如果说劳动者的“公民权”对于“他们的经济地位”不发生影响,那末他们的经济地位对于他们的公民权却发生影响。只有在工人有人身自由的地方,国家范围内的雇佣劳动,从而还有资本主义生产方式,才是可能的。它是建立在工人的人身自由之上的。

琼斯正确地把斯密的生产劳动和非生产劳动还原为它们的本质,即还原为资本主义劳动和非资本主义劳动,因为他正确地运用了斯密关于由资本支付的劳动者和由收入支付的劳动者的区分。但是琼斯自己把生产劳动和非生产劳动显然理解为加入物质[财富]生产的劳动和不加入这种生产的劳动。这是根据[上面引用过的]那段话[第52页]得出的,琼斯在那里谈到依靠别人花费的收入维持生活的生产劳动者。还根据以下的话:

\begin{quote}{“社会上不生产物质财富的那一部分,可能是有用的,也可能是无用的。”(第42页)“我们可以并且有理由认为,在所生产的商品没有到达要消费它的人手中以前,生产行为并没有结束。”(第35页注)}\end{quote}

靠资本生活的劳动者和靠收入生活的劳动者之间的区别,同劳动的形式有关。资本主义生产方式和非资本主义生产方式的全部区别就在这里。相反,如果从较狭窄的意义上来理解生产劳动者和非生产劳动者,那末生产劳动就是一切加入商品生产的劳动(这里所说的生产,包括商品从首要生产者到消费者所必须经过的一切行为),不管这个劳动是体力劳动还是非体力劳动(科学方面的劳动);而非生产劳动就是不加入商品生产的劳动,是不以生产商品为目的的劳动。这种区分决不可忽视,而这样一种情况,即其他一切种类的活动都对物质生产发生影响,物质生产也对其他一切种类的活动发生影响,——也丝毫不能改变这种区分的必要性。

\tsectionnonum{[(b)琼斯论资本主义生产形式对生产力发展的影响。关于追加固定资本的使用条件问题]}

[1146]现在我们来谈资本主义生产方式影响下的生产力的发展问题。

[琼斯说:]

\begin{quote}{“这里最好指出这样一点,即这个事实{由资本预付工资}怎样影响劳动者的生产力,或者说,怎样影响劳动的连续性、知识和力量……向工人支付工资的资本家,能够促进工人劳动的连续性。第一,他使这种连续性成为可能,第二,他对此进行监督和强制。世界上有人数很多的各种各样的劳动者,他们经常徘徊街头,寻找主顾,他们的工资取决于人们的偶然需要,就是说有人恰好在这个时候需要他们的服务,或者需要他们所制造的物品。最早的传教士在中国看到过这样的情况:‘那里的手工业者从早到晚在城里到处奔走,寻找主顾。大部分中国工人都是在私人家里劳动。例如,你需要衣服吗?裁缝便从早上到你家里来,到晚上才回家。其他一切手工业者的情况也是这样。他们经常为了寻找工作而走街串巷,甚至铁匠也担着他的锤子和炉子沿街寻找普通的零活。理发匠也是……肩上扛着靠椅,手里提着盆子和烧热水的小炉子走街串巷。’\endnote{琼斯在这里引用的是重农学派的月刊《公民历书》(《EphemeridesduCitoyen》)1767年第三卷第56页。——第477页。}这种情况至今在整个东方仍然是常见的现象,在西方世界也有一部分是这样。所以说,这种劳动者不可能在任何长时间内连续地劳动。他们必须象出租马车那样在街头招揽主顾,如果找不到主顾,他们就不得不闲起来。如果经过一段时间他们的经济地位发生了变化,并且成了资本家的工人,由资本家预付给他们工资,那就会产生两种结果:第一,他们现在能够连续地劳动;第二,出现了这样一种代理人,他的职能和利益就是迫使工人真正连续地劳动……资本家……所拥有的资财允许他等待主顾……因此,所有这类人的劳动就有了更大的连续性。他们每天从早到晚地劳动,他们的劳动不致因为等待或寻找那个必须消费他们所制造的物品的主顾而中断。但是,因此就成为可能的、工人劳动的连续性由于资本家的监督而得到了保障和增加。他预付他们的工资,他应当得到他们劳动的产品。他的利益和他的特权就是留心监视,不让他们工作中断或懈怠。既然劳动的连续性因此得到了保障,那末单是这个变化对劳动生产力的影响就非常大……生产力增加一倍。两个从早到晚连续劳动一年的工人所生产的东西,可能多于四个没有固定工作的工人所生产的,因为后者要把很多时间消耗在寻求主顾和恢复中断了的工作上面。”(第37—38页)}\end{quote}

[关于琼斯在这里所说的,必须指出:]

第一,关于从做临时活(如在土地所有者家中缝衣等等)的劳动者转变成受资本雇用的工人的事,杜尔哥已经作了很好的阐述。

第二,劳动的这种连续性虽然把资本主义劳动和琼斯所描述的劳动形式很好地区别开来,但是没有把它和大规模的奴隶劳动区别开来。

第三,把由于劳动持续时间的增加和工作中断现象的消除而引起的劳动本身的增加,叫作劳动生产力的增加是不正确的。劳动生产力的增加,只有在劳动的连续性提高工人个人技能的限度内才会发生。我们所理解的劳动生产力[增加],是指使用一定量劳动时具有更大的效率,而不是指使用的劳动的量的任何变化。琼斯所说的那种情况,宁可说是劳动对资本的形式上的隶属,这种情况只有随着固定资本的发展才获得充分的发展。(关于这一点,马上就要谈。)

琼斯正确地着重指出:资本家把劳动视为自己的财产,丝毫也不让它白白耗费。至于直接依赖收入的劳动,那末说的只是劳动的使用价值。

[1147]琼斯继而完全正确地指出,非农业工人从早到晚继续不断地埋头劳动,决不是天然如此,这种劳动本身是经济发展的产物。中世纪的城市劳动,与亚洲的劳动形式和西方的农村劳动形式(以前占统治地位,现在还部分地可以看到)不同,它已经前进了一大步,并且对于资本主义生产方式,对于劳动的连续性和经常性来说,是一所预备学校。

{关于劳动的这种连续性,在1821年伦敦出版的匿名小册子《论马尔萨斯先生近来提倡的关于需求的性质和消费的必要性的原理》中写道:

\begin{quote}{“资本家好象还掌管着一个劳动介绍所;他保险劳动不会没有把握找到销路。如果没有资本家,这种没有把握的事,就会使劳动在很多情况下得不到雇用。由于他的资本,那些寻找买者和奔跑市场的麻烦事就比较少了。”(第102页)}\end{quote}

在这本小册子里我们还读到:

\begin{quote}{“在资本在很大程度上由固定资本构成的地方,或者在资本投于土地的地方……企业主在更大得多的程度上(和使用较少固定资本时相比)不得不继续使用和过去几乎同样多的流动资本,以便不致于失去固定资本部分的任何利润。”(第73页}}\end{quote}

{[琼斯还说:]

\begin{quote}{“关于在中国由于劳动者依赖他们的主顾的收入而造成的状况,你大概可以在一个由美国人举办的中国展览会上看到一幅极其引人注目的图景,这个展览会在伦敦展出了很久。展览会充满了对手工业者的描绘,他们携带着自己的一套简单工具到处寻找主顾,如果找不到主顾,就得闲起来。这里明显地呈现出,在他们的情况下,必然没有作为劳动生产率的三大要素之一的劳动的连续性;任何一个有见识的观众都能看出,这里也缺乏固定资本和机器,它们不见得是劳动生产率的次要的要素。”(第73页)“类似的情景在印度的城市也能看到,欧洲人的出现并没有改变这种状况。不过,在印度的农村地区,手工业者是靠特别的方法维持的……确实为某个村庄所需要的手工业者和其他非农业劳动者,靠这个村庄居民的公共收入的一部分来维持生活。在全国范围内有一大批世代相传的劳动者靠这种基金生活,他们的劳动满足了农业劳动者用自己的劳动满足不了的简单的需要和嗜好。这些农村手工业者的地位和权利,象东方的一切权利一样,很快就成为世代继承的了。手工业者在别的农村居民那里找到主顾。农村居民是定居的、不变动的,为他们服务的手工业者也是这样……城市手工业者过去和现在都是处于完全不同的地位。他们实质上是从同一个基金——土地的多余收入——获得自己的工资,但是在这里,基金的分配方法以及分配者是不同的,因此,手工业者不再能是永久定居的了,他们不得不进行频繁的和往往是灾难性的迁移……这样的手工业者不会由于依赖大量的固定资本而被限定在一个地方。(例如,在欧洲,棉纺织业和其他企业被限定在富有水力,或者富有生产蒸汽的燃料的地方,欧洲已有大量的财富转化为建筑物、机器等等。)……如果劳动者完全[1148]依靠从那些消费他所生产的商品的人的收入中直接领取一部分来维持生活,那末情况就不同了……这种劳动者不会被限定在有任何固定资本的地方。如果他们的主顾在较长时期内,有时甚至在短时期内,迁移了自己的住地,非农业劳动者为了不致饿死,就不得不跟随他们一起去。”(第73—74页)“为手工业者预备的这种基金的大部分,在亚洲由国家及其官吏来分配。分配的主要中心自然是首都。”(第75页)“由撒马尔汗往南到比贾普尔和塞林格帕塔姆,我们可以看到一些消失了的首都的遗迹,只要国王的收入(也就是土地的全部多余收入)的新的分配中心一形成,这些首都就被它们的居民突然舍弃了(而不象在其他国家那样是由于逐渐的衰落)。”(第76页)}\end{quote}

请看一看贝尔尼埃博士的书\endnote{马克思指的是法国医生和旅行家弗朗斯瓦·贝尔尼埃的书《大莫卧儿等国游记》,1670—1671年在巴黎初次出版,后来曾多次再版。马克思在1853年6月2日致恩格斯的信中从贝尔尼埃的这本书中引了很长的两段,其中包括把印度的城市比作军营那句话(见《马克思恩格斯全集》中文版第28卷第256页)。——第480页。},他把印度的城市比作军营。可见,这是以亚洲的土地所有制形式为基础的。}

\centerbox{※     ※     ※}

现在我们要从劳动的连续性转到分工、知识的发展、机器的使用等等。

[琼斯写道:]

\begin{quote}{“支付劳动者的雇主的变动对劳动的连续性产生的影响,决不限于以上所说。现在可以对生产上的各种工作作进一步的划分……如果他〈资本家〉使用的不是一个人,而是几个人,那末他就能在他们当中分配工作;他就能使每个工人固定地去完成整个工作中他完成得最好的那一部分……如果资本家是富有的,并且雇用了足够数量的工人,那末,只要工作还能细分,它就会尽量细分下去。这时,劳动的连续性也就达到了完善的程度……资本取得了预付工资的职能以后,现在已逐步地使劳动的连续性趋于完善。同时,资本也使这种劳动为产生一定效果而应用的知识和技能增加了。资本家阶级最初部分地摆脱了体力劳动的必要性,最后完全摆脱了体力劳动的必要性。他们的利益要求他们使用的工人的生产力尽可能地大。所以他们的注意力放在,而且几乎完全放在这种力量的增加上面。思想越来越集中于寻找最好的手段以达到人类劳动的一切目的;知识扩大了,增大了它的应用范围,并且几乎在所有生产部门中协助了劳动……但是我们再往下看一看机械力。不是用来支付劳动,而是用来协助劳动的资本,我们要称为辅助资本。”}\end{quote}

{可见,琼斯所理解的“辅助资本”,是不变资本中那个不是由原料构成的部分。}

\begin{quote}{“一个国家的辅助资本量,在具备一定条件时,能够无限地增加,即使工人人数保持不变。在这方面每前进一步,人类劳动效率的第三个要素,即它的机械力,都会增大……因而,辅助资本量同人口相比会增加……必须具备什么样的条件,才能使用于协助他们{资本家雇用的工人}的辅助资本量增加呢?必须同时具备三个条件:(1)积蓄追加资本的手段;(2)积蓄追加资本的愿望;(3)某种发明,由此有可能通过使用辅助资本来提高劳动生产力,而且提高到这样的程度,以致劳动在它以前生产的财富之外,还把使用的追加辅助资本按其消费的程度,连同其利润再生产出来……如果在现有的知识状况下能有利地加以使用的整个辅助资本量已经具备……那末只有知识水平的提高才能指出使用更大量资本的手段。其次,这种使用只在如下场合才有可能,即所发明的手段要把劳动力提高到能够把追加资本在它被消耗期间再生产出来。如果不是这样,资本家就定会损失自己的财富……但是,除此以外,工人的提高了的劳动效率应当还能生产一些利润,否则资本家把自己的资本用于生产的动机就完全没有了……只要通过使用新的辅助资本量能够达到这两个目的,对于进一步使用这种新的资本量就不会有固定的和最终的界限。资本的增长能够和知识的增长一起前进。但是知识永远不会停滞不前。由于知识每时每刻都在各个方面向前发展,所以每时每刻都能出现新工具、新机器、新动力,这就使社会能够有利可图地追加一些协助劳动的辅助资本量,并以此来扩大它的劳动生产率同那些较贫穷的、技能较差的国家的劳动生产率之间的差别。”(同上[第38—41页])}\end{quote}

[1149]首先,我们来看看琼斯的意见,他认为新的发明、装置或设备必须能够“把劳动生产力提高到这样的程度,以致劳动在它以前生产的财富之外,还把使用的追加辅助资本按其消费的程度再生产出来”,或者说,使劳动“把追加资本在它被消耗期间再生产出来”。可见,这仅仅意味着,磨损是按照磨损的程度得到补偿的,或者说,追加资本在它被消费期间平均得到补偿。产品价值的一部分——或者也可以说,产品的一部分——必须补偿已消费的“辅助资本”,而且必须在这样的期限内补偿,也就是说,在它完全被消费掉时,它就能完全被再生产出来,或者说,同一种新资本就能代替已消费的资本。但是做到这一点的条件是什么呢?劳动生产率必须由于使用追加的“辅助资本”而提高到这样的程度,以致产品的一部分能够分出来,或者以实物形式,或者通过交换,来补偿这个组成部分。

如果劳动生产率有这样高,也就是说,如果在同样一个工作日生产出的产品量增加了这样多,以致单位商品比原来生产过程中的单位商品便宜,即使这时商品总额要用自己的总价格来抵补机器的(比如说)年磨损,但摊在单位商品上的相应的磨损部分极小,那末“辅助资本”也会被再生产出来。如果我们从总产品中扣除,第一,补偿磨损的部分,第二,补偿原料价值的部分,那末剩下的就是支付工资的部分,以及抵补利润的部分,这个部分甚至会在单位商品价格不变的情况下提供更多的剩余价值。

不具备这种条件,产品也有可能增加。例如,如果棉纱的磅数[仅仅]增加到10倍(而不是100倍,等等),而摊到单位产品价值上的补偿机器磨损的附加额,从1/6减少到1/10,那末用机器生产的棉纱就会比用纺车生产的棉纱贵。\endnote{马克思在这里考察的是使用新固定资本的盈利性问题。只有在补偿磨损的补充费用,因产品数量增多引起单位产品成本降低而得到补偿的情况下,资本家才会使用追加固定资本。马克思的意思可以用下面的例子来说明。假定用手工纺纱生产的10磅棉纱总价值为10镑,其中8镑用在原料上,2镑用在劳动力上(马克思在这里撇开利润不谈)。因此,用手工纺的1磅棉纱的价值等于1镑。又假定,由于使用了机器,所生产的棉纱数量增加到100倍(是1000磅而不是10磅),原料的花费同样也增加到100倍,而劳动力的花费增加较少,例如增加到10倍。在这种情况下,1000磅棉纱的价值等于800镑(原料的花费)+20镑(劳动力的花费)+164镑(按马克思的假设,固定资本的损耗在这里是棉纱价值的1/6),也就是说等于984镑。在这种情况下,1磅棉纱的价值大致是9/10镑,也就是说,同手工纺纱相比棉纱落价了。这表明在这里机器的使用是有利的。假如棉纱数量只增加到10倍(是100磅而不是10磅),那末棉纱的价值等于80镑(原料的花费)+12镑(劳动力的花费假定增加到6倍)+10+(2/9)镑(按马克思的假设,固定资本的损耗量现在减少到棉纱价值的1/10),也就是说,等于102+(2/9)镑。这样,1磅棉纱的价值就超过了1镑。这表明:尽管用以补偿固定资本损耗的费用相对减少了(从1/6减少到1/10),在这种情况下机器生产的棉纱还是比手工生产的棉纱贵。因此,在这里机器的使用对资本家来说是不利的。——第483页。}如果用100镑追加资本购买鸟粪投在农业上,如果这些鸟粪必须在一年内被补偿,而一夸特产品的价值(在旧的生产方法下)等于2镑,那末,仅仅为了补偿损耗\endnote{马克思在《资本论》第二卷中指出,为改良土壤而投下的物质的一部分,在较长的时期内“继续作为生产资料存在,因而取得固定资本的形式”(见《马克思恩格斯全集》中文版第24卷第179页)。马克思正是在这个意义上在正文中谈到对投在土地上的鸟粪的“补偿损耗”问题。——第483页。},就必须生产50追加夸特。否则这笔追加资本就不会被使用(这里我们撇开利润不谈)。

琼斯认为,追加资本必须“在它被消耗期间再生产出来”(当然,通过出卖产品或者以实物形式),他的这种意见仅仅意味着,商品必须补偿它所包含的损耗。为了重新开始再生产,商品所包含的一切价值要素,都必须在商品的再生产重新开始时就得到补偿。在农业上这种再生产时间是由自然条件决定的,而在什么时间内必须补偿损耗,也完全和在什么时间内必须补偿比如说谷物的其他一切价值要素一样,在这里是决定了的。

为了使再生产过程能够开始,也就是说,为了使本来的生产过程能够得到更新,必须经过流通过程,即商品必须出卖(除非它是以实物形式自己补偿自己,就象种子那样),而商品卖得的货币必须重新转化为生产要素。就谷物和其他农产品来说,对于这种再生产,存在着一定的、由四季更替所规定的期限,因此,对于流通过程的持续时间,也存在着最终的界限,即肯定的界限。[这是第一点。]

第二,流通过程的这种肯定的界限,一般来自作为使用价值的商品的性质。所有的商品都会在一定的时间内变坏,尽管它们存在的ultimaThule\authornote{极限,极点,最终之物,最终目的(直译是:极北的休里——古代人想象中的欧洲极北部的一个岛国)。——编者注}各不相同。如果人不消费它们(为了生产或者为了个人消费),天然的自然力就会消费它们。它们会逐渐变质,最后完全毁坏。如果商品的使用价值失去了,它的交换价值也就见鬼去了,它的再生产也就停止了。因此,商品流通时间的最终的界限决定于作为使用价值的商品所固有的再生产时间的自然期限。

第三,为了使商品的生产过程连续不断,也就是说,为了使资本的一[1150]部分不间断地处在生产过程中,另一部分不间断地处在流通过程中,就必须按照再生产时间的自然界限,按照各种不同使用价值存在的界限,或者按照资本的各种不同的作用领域,对资本进行极不相同的划分。

第四,上述一切同时适用于商品的所有价值要素。但是,对于那些有很多固定资本参与生产的商品来说,除了由商品本身的使用价值给流通时间规定的界限外,固定资本的使用价值也具有决定的作用。固定资本在一定时间内被损耗,因此,它必须在一定期间再生产出来。比方说,一只船在10年内用坏,或者一架纺纱机在12年内用坏。在这10年内所获得的运费,或者在这12年内卖的纱,必须足以在10年后用一只新船来代替旧船,或者在12年后用一架新纺纱机来代替旧纺纱机。如果固定资本在半年内消费掉,产品就必须在半年内从流通中返回。

因此,除了作为使用价值的商品的自然毁灭期限(这个期限对于不同的使用价值是极不相同的),除了生产过程连续性的要求(由于这种要求,根据商品必须在生产领域停留时间的长短,根据商品能够在流通领域停留时间的长短,又有流通时间的各种不同的最终界限),还要加上第三点,即加入商品生产的“辅助资本”的各种不同的毁灭期限以及由此引起的再生产的必要性。

琼斯认为[使用“辅助资本”的]第二个条件是“辅助资本”必须“生产”的“利润”,而这是任何资本主义生产的必要条件,不论使用的资本有怎样特殊的形式规定性。其实,琼斯在任何地方都没有向我们说明,他对这种利润的产生是怎样理解的。但是,因为他只从“劳动”中引出这种利润,只从提高了的工人劳动效率中引出“辅助资本”所提供的利润,所以在琼斯那里,任何利润都必然归结为绝对的或相对的剩余劳动。一般说利润是这样产生的:产品的一部分以实物形式或通过交换去补偿资本中那些由原料和劳动资料构成的部分,资本家在扣除这一部分产品之后,第一,由余下的产品部分中支付工资,第二,把一部分产品作为剩余产品占为己有,他出卖这部分产品,或者以实物形式消费它。(后一种情况,在资本主义生产条件下不必考虑,只是少数直接生产必要生活资料的资本家除外。)可是这个剩余产品正象产品的其他部分一样,是工人的物化劳动,不过是无酬劳动,是资本家不付等价而占有的劳动产品。

在琼斯对问题的论述中,有一点是新的,即他指出“辅助资本”在一定限度以上的增加取决于知识的增加。琼斯说,要使“辅助资本”增加,必须有:(1)积蓄追加资本的手段,(2)积蓄追加资本的愿望,(3)某种发明,由此有可能把劳动生产力提高到这样的程度,以致能够再生产出追加资本,并且生产出追加资本的利润。

在这里首先必需的是剩余产品的存在,不管它是以实物形式存在,还是已经转化为货币。

以棉花生产为例,曾经有过这样的时候,那时在美国(如同现在在印度)种植场主能够种植大面积的棉花,但是他们没有办法通过清棉及时把子棉变成棉纤维。一部分长好的棉花便在地里腐烂。轧棉机的发明结束了这种情况。现在一部分产品转化为轧棉机,但是轧棉机不仅能补偿自己的费用,而且能增加剩余产品。新市场的出现也有同样的作用,例如,它能把皮革转化为货币。(运输工具的改良也有同样的作用。)

每一种消费煤的新机器,都是一种把以煤的形式存在的剩余产品转化为资本的手段。把一部分剩余产品转化为“辅助资本”,可以通过两种方式:[第一,]通过现有“辅助资本”的增加,也就是它的扩大规模的再生产;[第二,]通过新使用价值的发现,或者通过旧使用价值的新应用,以及通过新机器或动力的发明,从而创造出新的种类的“辅助资本”。在这里,知识的扩大当然是“辅助资本”增加的条件之一,或者同样可以说,是剩余产品或剩余货币转化为(在这里对外贸易具有重要意义)追加的“辅助资本”的条件之一。例如,电报的发明为投入“辅助资本”开辟了完全新的范围,铁路等等也是这样,古塔波胶或印度胶的整个生产也是这样。

[1151]关于知识的扩大这一点是重要的。

积累完全不一定要直接推动新劳动,它可以仅限于给旧劳动提供新方向。例如,同一个机械厂,过去生产[手工]织布机,现在制造机械织布机,一部分[手工]织布工人转到这个改变了的生产上来,其余部分则被抛弃街头。

当一种机器代替劳动的时候,它(为了它本身的生产)所需要的新劳动不管怎样都少于它所代替的劳动。也许仅仅给旧劳动提供新方向。不管怎样,都会有劳动游离出来,这种劳动经过或多或少的流浪和苦难之后可能被用到其他的方向。这样就为新生产领域提供了人身材料。至于资本的直接游离,这里游离出来的不是购买机器的资本,因为它就是投到机器上的。即使假定,机器比被它排挤的工人的工资便宜,那也会需要更多的原料等等。如果被解雇的工人一年花费500镑,新机器也值500镑,那末资本家以前每年都必须花费500镑,而现在机器也许能用10年,资本家实际上每年只花费50镑。但是不管怎样,游离出来的(扣除在机器生产及其辅助材料例如煤的生产上使用的追加工人的费用以后)或者是曾构成[被解雇]工人的收入的资本,或者是工人曾用自己的工资与之交换的资本。这种资本依然存在。如果工人仅仅作为动力被代替,而机器本身没有什么显著的变化,举例说,如果现在机器用水或风推动,而过去是用工人推动,那末就有双重资本游离出来:一种是以前用于支付工人的资本,一种是工人曾用自己的货币收入与之交换的资本。这样的例子李嘉图已经用过\authornote{见本卷第2册第630—633页。——编者注}。

但是,一部分以前转化为工资的产品,现在总是被作为“辅助资本”再生产出来。

一大部分以前直接用于生活资料生产的劳动,现在用于“辅助资本”的生产。这和亚·斯密的观点也是相矛盾的,按照斯密的观点,资本的积累等于使用要多的生产劳动。除开上面所说的,这里发生的只能是劳动使用的改变,以及劳动由直接生产生活资料转移到生产生产资料,即铁路,桥梁、机器、运河等等。

\centerbox{※     ※     ※}

{现有的生产资料量和现有的生产规模对于积累是多么重要,这从下面一段引文可以看出:

\begin{quote}{“在郎卡郡能用如此惊人的速度建造起一座包括纺纱间和织布间的大棉纺织厂,这是由于,在工程师、设计师、机器制造者那里大量搜集有各种模型,从巨大的蒸汽机、水车、铁梁、铁柱,直到翼锭精纺机或织机的最小零件。在最近一年内,费尔贝恩先生在他的一个机器制造厂中(不依赖他的大的机器制造厂和蒸汽锅炉制造厂)就建造了700马力的水车和400马力的蒸汽机。每当增大的商品需求吸引新资本的时候,有利地使用新资本的手段就如此迅速地制造出来,以致在法国、比利时或德国的同类工厂能够开工以前,新资本就能实现同它自身价值相等的利润。”(安·尤尔《工厂哲学》1836年巴黎版第1卷第61—62页)}\end{quote}

[1152]工业的发展导致机器降价,部分是相对降价(同机器的功率相比),部分是绝对降价;但同时与此相联的是在一个工厂里集中有大量的机器,因此机器设备的价值同被使用的活劳动相比增大了,虽然它的个别组成部分的价值减少了。

动力——生产动力的机器——随着动力传送机械和工作机的改进,即随着磨擦力的减少等等而逐渐降价。

\begin{quote}{“使用自动工具所带来的优越性,不仅改进了工厂的机器设备的精度,加速了它的制造,而且在很大程度上降低了它的价格,增大了它的灵便性。现在可以买到最好的翼锭精纺机,每枚纱锭9先令6便士;自动走锭精纺机也可以买到,每枚纱锭约8先令,包括它的专利税在内。棉纺织厂的纱锭运行时磨擦很小,以致一马力就能带动精纺机的500枚纱锭,自动走锭精纺机的300枚纱锭,翼锭精纺机的180枚纱锭;这一马力还带动一切准备机器,即梳棉机、粗纺机等等。三马力足以带动30台大织机连同它们的浆纱机。”(安·尤尔《工厂哲学》1836年巴黎版第1卷第62—63页)}}\end{quote}

\centerbox{※     ※     ※}

[琼斯进一步指出:]

\begin{quote}{“在地球上绝大部分地区,劳动阶级的大多数还根本不是从资本家那里得到自己的工资;他们或者自己生产它,或者从自己的主顾的收入中得到它。在这里,保证他们的劳动连续性的第一个大步骤还没有完成。在劳动中帮助他们的,只是为了生计而用自己的双手劳动的人所能掌握的那种知识和那样一种数量的机械力。较发达的国家的技能和科学,巨大的动力,这种动力所能带动的积累的工具和机器,在那些仅有这种劳动者参加的劳动中是没有的。”(第43页)}\end{quote}

{甚至在英国:

\begin{quote}{“以农业为例……很好地经营农业所必需的知识,在全国传播得少而且不普遍。非常小的一部分农业人口享用着……能够在国民劳动的这个部门使用的全部资本……在我们的非农业劳动者中,只有很小一部分人在大工厂中工作。在农村作坊中,在那些通过小的组合完成自己的单项工作的手工业者和手艺人那里,分工是不充分的,因而劳动的连续性也是不完善的……走出大城市的圈子,看看国家的广阔原野,那就可以看到,国民劳动的很大一部分,无论在劳动的连续性方面,还是在劳动的技能和力量方面,都距离完善还很远很远。”(第44页)}}\end{quote}

资本主义生产的发展势必引起科学和劳动的分离,同时使科学本身被应用到物质生产上去。

\centerbox{※     ※     ※}

关于地租,琼斯正确地指出:

完全依赖于利润的现代意义上的地租的前提是:

\begin{quote}{“资本和劳动从一个生产部门转移到另一个生产部门的可能性……资本和劳动的灵活性,在农业资本和农业劳动没有这种灵活性的那些国家……我们根本不能指望看到在英国看到的那些纯粹由这种灵活性产生的结果。”(第59页)}\end{quote}

这种“资本和劳动的灵活性”,一般说来是形成一般利润率的现实前提。这种灵活性以劳动的确定形式无关紧要为前提。在这里,实际上发生了(靠损害工人阶级的利益)以下两种情况之间的磨擦:一方面,分工和机器赋予劳动能力以片面性,另一方面,这种劳动能力只作为任何一种劳动的现实的可能性和资本相对立{这就使资本同它在行会工业中的不发达形式有了区别},劳动投向这个方向还是投向另一个方向,要看在这个或那个生产领域里能获得什么样的利润,因此,各种不同的劳动量能够从一个领域转到另一个领域。

\begin{quote}{在亚洲等地,“主要的人口由劳动农民构成。他们所采用的落后的耕作制[1153]提供了长的闲暇时间。农民正如生产自己的食物一样……也生产大部分自己消费的其他生活必需品——自己的衣服,自己的工具,自己的家具,甚至自己的房屋,因为在这个阶级中只有很少的行业划分。这些人的风俗习惯是不变的;它们从父母传到子女;没有任何东西能改变或破坏它们”。(第97页)}\end{quote}

相反,资本主义生产的特征是,资本和劳动的灵活性,生产方式的不断变革,从而,生产关系、交往关系和生活方式等方面的不断变革,与此同时,在国民的风俗习惯和思想方式等等方面也出现了很大的灵活性。

让我们把刚才引用的关于“落后的耕作制”条件下的“闲暇时间”那段话,同下面两段话比较一下:

\begin{quote}{(1)“如果在农场使用蒸汽机,那它就会构成在农业中使用最多的工人的体系的一部分,并且不管怎样,马的数目必定会减少。”(《论农业中使用的动力》,约翰·查默斯·摩尔顿先生1859年12月7日在艺术和手工业协会所作的报告\endnote{艺术和手工业协会(SocietyofArtsandTrades)是资产阶级教育性质和慈善性质的团体,于1754年在伦敦成立。摩尔顿的报告发表在该协会的周刊《艺术协会杂志》(《JournaloftheSocieyofArts》)1859年12月9日那一期上。马克思引用的那段话在该期第56页上。——第490页。})(2)“生产农产品和生产其他劳动部门的产品所需要的时间是有差别的,这种差别就是农民具有很大依赖性的主要原因。他们不能在不满一年的时间内就把商品送到市场上去。在这整个期间内,他们不得不向鞋匠、裁缝、铁匠、马车制造匠以及其他各种生产者,赊购他们所需要的、可以在几天或几周内完成的各种产品。由于这种自然的情况,并且由于其他劳动部门的财富的增长比农业快得多,那些垄断了全国土地的土地所有者,尽管还垄断了立法权,但仍旧不能使他们自己和他们的奴仆即租地农民摆脱成为国内依赖性最强的人的命运。”(霍吉斯金《通俗政治经济学》1827年伦敦版第147页注)}\end{quote}

资本家和资本的区别在于,资本家必须生活,也就是说,必须每日每时把剩余价值的一部分作为收入来消费。因此,在资本家能把自己的商品运到市场以前,生产所经历的时间越长,或者说,他从市场得到出卖商品的收益所需要的时间越长,资本家就越是不得不在这一段时间靠借债生活(这一点我们在这里不必去考察),或者说,他就越是必须积累有更多的作为收入花费的货币储备。他就越是必须在较长的时间内向自己预付自己的收入。他的资本就必须越多。他不得不把自己的一部分资本经常闲置不用,以便作为消费基金。

{所以在小农业中,家庭工业是和农业结合在一起的;必须有一年的储备等等。}

\tsectionnonum{[(c)琼斯论积累和利润率。关于剩余价值的源泉问题]}

现在我们转到琼斯的积累学说。在此以前只是指出琼斯对积累的看法中有两个特点:第一,积累的源泉完全不一定是利润;第二,“辅助资本”的积累取决于知识的进步。琼斯把这种进步限于新的机器设备、动力等等的发明。但是这具有一般的意义。例如,如果把谷物用作制烧酒的原料,就会产生一个新的积累源泉,因为在这种情况下,剩余产品能被转化为新的形式,能用来满足新的需要,并且能作为生产要素进入新的生产领域。用谷物制造淀粉等等的情况也是如此。这些商品以及一切商品的交换领域因而扩大了。如果煤炭被用于照明等等,也会发生同样的情况。

当然,对外贸易——通过增加使用价值的多样化和商品量——也是积累过程中的重要因素。

琼斯这里所说的积累首先是指积累和利润率之间的关系(关于利润率的产生,他远远没有弄清楚):

\begin{quote}{“国家由利润积累资本的能力,不是随着利润率的变化而变化……相反,由利润积累资本的能力通常是按照同利润率相反的方向发生变化,即利润率低的地方,积累能力大,利润率高的地方,积累能力小。亚·斯密说:[1154]‘居民由利润得来的那部分收入,在富国总是比在贫国大得多,这是因为富国的资本大得多;但利润同资本相比,富国的利润通常又低得多’(《国富论》第2篇第3章)在英国和荷兰,利润率比欧洲其他任何地方都低。”(第21页)“在它的〈英国的〉财富和资本增长最快的时期,利润率逐渐下降。”(第21—22页)“所生产的利润的相对量……不是仅仅取决于利润率……而是取决于与使用的资本相对量结合起来考察的利润率。”(第22页)“较富国家的资本量的增长……通常还引起利润率下降,或者说,从使用的资本得到的年收入同这个资本总量之间的比例下降。”(同上)“如果有人说,在其他条件相等的情况下,由利润进行积累的能力决定于利润率,那末对这种说法的回答应当是,这种情形,即使实际上有可能发生,也是非常少见,因而不值得考虑。我们从观察中知道,利润率的下降是这样一种现象,它通常是由各国使用的资本量差额的增长引起的,因此,在较富国家中利润率下降时,所有其他条件并不是相等的。如果有人断言,利润可能下降得非常厉害,以至完全不可能由利润积累,那末对这种说法的回答应当是,从利润会如此下降的假定出发来加以论证是荒谬的,因为在利润率达到这个水平之前很久,资本就已向国外流走,以便在别的国家得到更高的利润,而资本输出的可能性总是会确立某种界限,只要还存在利润率较高的其他国家,任何一个国家的利润都不会下降到低于这个界限。”(第22—23页)“除了积累的原始源泉……还有派生源泉,例如,公债券所有者、官吏等等的收入。”(第23页)}\end{quote}

所有这些都很好。说[利润的]积累量决不是仅仅取决于利润率,而是取决于乘以使用的资本的利润率,也就是说,同样取决于使用的资本量,那是完全正确的。如果使用的资本=C,利润率=r,那末[最大限度的]积累=Cr,很明显,如果乘数C的增加比乘数r的减少迅速,这个乘积就会增加。通过观察所确定的事实的确就是这样。但是我们对于这一事实的原因还是一无所知。不过琼斯本人已经非常接近于这个原因,因为他已观察到,“辅助资本”和推动它的工人人口相比是在不断增长。

如果利润的下降是由于李嘉图所说的原因,即由于地租的增加,那末总剩余价值对使用的资本的比例会保持不变。区别只在于,总剩余价值的一部分——地租——靠牺牲另一部分即利润而增长,这就使得总剩余价值[对总资本]的比例保持不变,因为利润、利息和地租只是总剩余价值的单个范畴。可见,李嘉图实际上否定了这种现象。

另一方面,单是利息率的下降不能说明任何问题,正如它的上升不能说明任何问题一样,尽管利息率自然始终是一个最低比率的指标,利润不能低于这个最低比率。因为利润必须始终大于平均利息率。

[1155]撇开利润率下降规律使政治经济学家感到恐惧这点不谈,它的最重要的后果就是它以不断增长的资本积聚为前提,因而以较小的资本家日益丧失资本为前提。一般说来,这是资本主义生产的所有规律的结果。如果我们摘除这个事实的对抗性质,即摘除它在资本主义生产基础上所具有的那个特点,那末这个事实,即这个不断向前发展的集中化过程,将表明什么呢?不外是,生产丧失自己的私有性质并成为社会过程,并且这是现实的,而不只是形式上的,即不象在任何交换中那样,生产具有社会性是由于生产者的绝对的相互依赖性,由于他们必须把自己的劳动表现为抽象的社会劳动(货币)。因为生产资料现在是作为公共的生产资料被使用,因而——不是由于它们是单个人的财产,而是由于它们对生产的关系——作为社会的生产资料被使用;各个企业的劳动现在同样也是以社会规模来完成。

琼斯书中专门有一节,标题是《决定积累倾向的各种原因》。[琼斯把这些原因归结为以下五点:]

\begin{quote}{“(1)民族的气质和意向方面的差别;(2)国民收入在各居民阶级之间的分配有差别;(3)可靠地使用积蓄起来的资本的保障程度有差别;(4)有利而可靠地用连续的积蓄进行投资的难易程度有差别;(5)不同居民阶层通过积蓄改善自己地位的可能性有差别。”(第24页)}\end{quote}

这五点原因实质上可以归结为,积累取决于某一特定国家所达到的资本主义生产方式的发展阶段。

我们首先看一看第(2)点。资本主义生产发达的地方,利润是积累的主要源泉,即资本家把最大部分国民收入集中在自己手里,甚至一部分土地所有者也力图把自己的收入资本化。

第(3)点。资本家越是把管理国家的权力抓到自己手里,法律的和警察的保障就越增加。

第(4)点。随着资本的发展,一方面,生产领域会增大,另一方面,信用组织会发展,它使贷款人(银行家)能够把积蓄的每一文钱都集中在自己手里。

第(5)点。在资本主义生产条件下,人的社会地位的改善仅仅取决于金钱,而每个人都能幻想他有一天会成为路特希尔德。

还有第(1)点。并非一切民族都有相同的从事资本主义生产的才能。某些原始民族,例如土耳其人,既没有这方面的气质,也没有这方面的意向。但这是例外。随着资本主义生产的发展,会形成资产阶级社会的平均水平,与此同时,也会在极不相同的民族之间形成气质和意向的平均水平。资本主义生产,象基督教一样,本质上是世界主义的。所以,基督教也是资本所特有的宗教。在这两个方面只有人是重要的。一个人就其自身来说,他的价值不比别人大,也不比别人小。对于基督教来说,一切取决于人有没有信仰,而对于资本来说,一切取决于他有没有信用。此外,当然在第一种场合还要附加上天命,而在第二种场合要附加上一个偶然因素,即他是否生下来就有钱。

\centerbox{※     ※     ※}

剩余价值的源泉和最初的地租:

\begin{quote}{“当土地被占有并被耕种以后,它向花费在它上面的劳动所提供的,几乎总是多于用旧方法继续耕种它所必需的。土地在此以外所生产的[1156]一切,我们将称为它的剩余产品。这个剩余产品就是最初的地租的源泉,并由它来规定土地的所有者(他们不同于租种土地的人)经常能从土地上获得的那些收入的界限。”(第19页)}\end{quote}

这些最初的地租是剩余价值借以表现的最早的社会形式,这个隐秘的观点是重农学派学说的基础。

绝对剩余价值和相对剩余价值之间有一个共同点,即二者都以劳动生产力的一定发展程度为前提。如果一个人的(每一个人的)整个工作日(可支配的劳动时间)只够养活他自己(至多还有他的一家),那末也就不再有剩余劳动、剩余产品和剩余价值了。劳动生产力的一定发展程度这个前提,是以财富的自然源泉(土地和水)的天然富饶程度为基础的,而这种天然富饶程度在不同的国家等等是不同的。起初,需要是简单的,原始的,因而必须用来维持生产者本身生存的产品最低量也是很少的。这里的剩余产品同样是很少的。另一方面,在这样的条件下,靠剩余产品为生的人数也很少,因此,剩余产品在这里是人数较多的生产者的较少的剩余产品的总额。

绝对剩余价值的基础,即它赖以存在的现实条件,是土地(即自然)的天然富饶程度,而相对剩余价值则以社会生产力的发展为基础。

\tchapternonum{收入及其源泉。庸俗政治经济学}

\tchapternonum{[(1)]生息资本在资本主义生产基础上的发展[资本主义生产方式的关系的拜物教化。生息资本是这种拜物教的最充分的表现。庸俗经济学家和庸俗社会主义者论资本利息]}

[\endnote{在手稿第XIV本封面上草拟的《剩余价值理论》最后几章的计划中,紧接着《理查·琼斯》(这第五部分结束)这一章之后,马克思写了:《补充部分:收入及其源泉》(见本卷第1册第5页),而在第XV本封面上草拟的这一本的目录中,有《庸俗政治经济学》这一题目(同上,第6页)。这两个题目(《收入及其源泉》和《庸俗政治经济学》)占手稿第XV本很大一部分篇幅,而且二者是紧密地交错在一起加以论述的。在这一本(写于1862年10月—11月)里,马克思在手稿第890页上中断了对霍吉斯金观点的分析,转而写关于收入及其源泉和关于庸俗政治经济学的《补充部分》,庸俗政治经济学抓住收入及其源泉的拜物教形式的外表,在这个基础上建立了自己的辩护士理论。在进一步阐述有关问题(也在第XV本)的过程中,这个《补充部分》首先转到对借贷资本的分析,这种分析同对于庸俗政治经济学的批判紧密地结合在一起,然后又转到对商业资本这种不创造剩余价值而只分配剩余价值的资本主义经济领域的分析。这样一来,马克思就逐渐地越出了作为自己著作的历史批判部分的《剩余价值理论》的直接对象的范围。对商业资本的研究马克思一直继续到第XV本结尾。但下一本手稿即第XVI本(1862年12月)却以《第三章。资本和利润》这个标题开始。这一本的主要内容,是马克思在1865年写《资本论》第三卷第一、二两篇时广为利用的、对于剩余价值转化为利润和剩余价值率转化为利润率以及利润转化为平均利润的研究。在第XVI本结尾,马克思转到——如他自己所说——“这一篇最重要的问题”,即分析利润率随着资本主义生产方式的发展而下降的原因。马克思后来把这个阐述加以改写,用于《资本论》第三卷第三篇(《利润率趋向下降的规律》),这个阐述是到下一本手稿即第XVII本(1862年12月—1863年1月初)的开头才完成的。在第XVII本(从手稿第1029页开始),马克思重新回过头来分析商业资本,继续手稿第XV本的正文。但是在这里马克思也中断了对商业资本问题的阐述,这一次是为了写题为《资本主义再生产过程中货币的回流运动》的《补充部分》。这个篇幅相当大的《补充部分》,只是到第XVIII本(1863年1月)的开头,马克思才用下面的话作了结束:“对这个问题的进一步考察应该推后”。接着,马克思又重新(在手稿第1075页)回过头来研究商业资本,这一次他考察了不同的作者对这个问题的观点。对商业资本的所有这些研究包括在手稿第XV、XVII和XVIII本中;马克思在1865年写作《资本论》第三卷第四篇时在很大程度上利用了这些研究材料。结束了对商业资本的研究之后,马克思(在手稿第1084页)又回到《剩余价值理论》,回到在第XV本中断的关于霍吉斯金那一节。从这里列举的1861—1863年手稿第XV—XVIII本内容丰富的材料中,按照马克思本人的计划,只有包含在第XV本中(第891—950页)的《收入及其源泉。庸俗政治经济学》这一部分,作为附录收入本版《剩余价值理论》。马克思著作的全部历史批判部分以此结束。——第499页。}XV—891]收入的形式和收入的源泉以最富有拜物教性质的形式表现了资本主义生产关系。这是资本主义生产关系从外表上表现出来的存在,它同潜在的联系以及中介环节是分离的。于是,土地成了地租的源泉,资本成了利润的源泉,劳动成了工资的源泉。现实的颠倒借以表现的歪曲形式,自然会在这种生产方式的当事人的观念中再现出来。这是一种没有想象力的虚构方式,是庸人的宗教。庸俗经济学家——应该把他们同我们所批判的经济学研究者严格区别开来——实际上只是[用政治经济学的语言]翻译了受资本主义生产束缚的资本主义生产承担者的观念、动机等等,在这些观念和动机中,资本主义生产仅仅在其外观上反映出来。他们把这些观念、动机翻译成学理主义的语言,但是他们是从[社会的]统治部分即资本家的立场出发的,因此他们的论述不是素朴的和客观的,而是辩护论的。对必然在这种生产方式的承担者那里产生的庸俗观念的偏狭的和学理主义的表述,同诸如重农学派、亚·斯密、李嘉图这样的政治经济学家渴求理解现象的内部联系的愿望,是极不相同的。

然而,在所有这些形式中,最完善的物神是生息资本。在这里,我们看到的是资本的最初起点——货币,以及G—W—G′这个公式,而这个公式已被归结为它的两极G—G′。创造更多货币的货币。这是被缩简成了没有意义的简化式的资本最初的一般公式。

土地,或者说自然,是地租即土地所有权的源泉,——这具有充分的拜物教性质。但是,由于把使用价值和交换价值随意地混淆起来,通常的观念就还有可能求助于自然本身的生产力[来解释地租],而这种生产力借助某种魔术在土地所有者身上人格化了。

劳动是工资(即工人在他的产品中所占有的由劳动的特殊社会形式决定的份额)的源泉;劳动是下述事实的源泉:工人用自己的劳动从产品(即从物质上考察的资本)中为自己购买从事生产的许可权,并在劳动中占有一个源泉,由于有了这个源泉,他的一部分产品才以报酬的形式从这个作为雇主的产品中流回他那里,——这种说法也是够妙的。但是,在这里,通常的观念在如下的限度内还算是符合事实的,即尽管它把劳动同雇佣劳动混淆起来,从而把雇佣劳动的产品即工资同劳动的产品混淆起来,然而对健全的人类理智来说,有一点仍然是清楚的,这就是劳动本身创造它的工资。

至于资本,如果就生产过程来进行考察,那末认为它是猎取别人劳动的工具这样一种观念,总是或多或少地保存着。无论把这一点看作是“合理的”还是“不合理的”,有根据的还是无根据的,——这里总是以资本家和工人的关系为前提,总是指的这种关系。

就资本出现在流通过程来说,通常的看法所特别注意的是,它表现为商人资本,这是一种仅仅从事这种业务的资本,所以利润在这里有时用普遍欺诈这个含糊的观念来说明,有时用比较明确的观念来说明,即:商人欺诈产业资本家,就象产业资本家欺诈工人那样,或者说,商人欺诈消费者,就象生产者相互欺诈那样。不管怎样,这里利润是用交换,就是说,用社会关系而不是用物来解释的。

相反,在生息资本上物神达到了完善的程度。这是一个已经完成的资本,——因而是生产过程和流通过程的统一,——因此,它在一定的期间提供一定的利润。在生息资本的形式上,只剩下了这种规定性,而没有生产过程和流通过程作媒介。在资本和利润中,还存在着对过去的回忆,尽管由于利润和剩余价值的不同,由于所有资本具有形式单一的利润——一般利润率,资本已经[892]非常模糊不清了,已经变得难以理解和神秘莫测了。

在生息资本上,这个自动的物神,自行增殖的价值,创造货币的货币,达到了完善的程度,并且在这个形式上再也看不到它的起源的任何痕迹了。社会关系最终成为物(货币、商品)同它自身的关系。

对于利息以及利息与利润的关系,这里不作进一步的研究,对于利润按怎样的比例分为产业利润和利息,这里也不研究。有一点是清楚的,这就是:在资本和利息上,资本作为利息的神秘的、自行创造的源泉,即作为资本自行增长的源泉已达到了完善的程度。正因为如此,照[通常的]观念看来,资本主要存在于这种形式中。这就是真正意义上的资本。

既然在资本主义生产的基础上,体现在货币或商品中——真正说来是体现在货币即商品的转化形式中——的一定价值额提供了一种权力,使人有可能白白地从工人身上榨取一定量的劳动,占有一定的剩余价值,剩余劳动,剩余产品,那末很清楚,货币本身可以作为资本,作为特殊种类的商品出卖,或者说,资本可以在商品或货币的形式上被购买。

资本可以作为利润的源泉出卖。通过货币等等,我使另一个人能够占有剩余价值。因此,我取得这个剩余价值的一部分,是很自然的。土地具有价值,是由于它使我能够获得一部分剩余价值,因此我在土地上不过是为借助于土地所获得的这部分剩余价值而支付;同样,我在资本上不过是为借助于资本所创造的剩余价值而支付。因为在资本主义生产过程中,除了实现剩余价值外,资本的价值还会永恒化,会再生产出来,所以自然而然,货币或商品作为资本出卖时,会在一定时期之后又流回卖者手中,卖者永远不会象转让商品那样转让货币,而是保留自己对货币的所有权。在这种场合,货币或商品不是作为货币或商品出卖,而是作为它的二次方,作为资本,作为自行增殖的货币或商品价值来出卖了。它不仅会自行增长,而且会在总生产过程中把自己保存下来。因此,对于卖者来说,它仍旧是资本,会流回卖者手中。在这里,出卖就在于:一个把它作为生产资本使用的第三者,必须从他只是因有这笔资本而获取的利润中,支付一定的部分给资本所有者。象土地一样,货币是作为创造价值的物贷出的,这个物在这个创造价值的过程中被保存下来,不断地流回,因而也可能流回最初的卖者手中。只是由于流回最初的卖者手中,货币才成为资本。否则,他就是把它作为商品来卖,或是用它作为货币来买了。

但是不管怎样,形式就其本身来考察(实际上,货币作为榨取劳动的手段,作为获得剩余价值的手段,是定期转让的)是这样的:物现在表现为资本,资本也表现为单纯的物,资本主义生产过程和流通过程的全部结果则表现为物所固有的一种属性;究竟是把货币作为货币支出,还是把货币作为资本贷出,取决于货币所有者,即处在随时可以进行交换的形式上的商品的所有者。

这里我们看到的是作为本金的资本和作为果实的资本的关系,资本提供的利润由资本自己的价值来决定,并且资本本身不会因这个过程而消失(这是符合资本的性质的)。

由此可以明白,为什么肤浅的批判完全象它想要保存商品而反对货币那样,现在却要用它那改良派的智慧去反对生息资本,同时毫不触动现实的资本主义生产,而只是攻击这种生产的一个结果。这种从资本主义生产的立场出发对于生息资本的反驳,今天竟自诩为“社会主义”,其实这种反驳,作为资本本身的发展因素,例如在十七世纪就已出现,那时,产业资本家还必须首先同当时还比自己强大的旧式高利贷者进行斗争,以夺取自己的地位。

[893]作为生息资本的资本,它的充分的物化、颠倒和疯狂,——不过,在生息资本上,资本主义生产的内在本性,它的疯狂性,只是以最明显的形式表现出来,——就是生“复利”的资本,在这里,资本好象一个摩洛赫,他要求整个世界成为献给他的祭品,然而由于某种神秘的命运,他永远满足不了自己理所当然的、从他的本性产生的要求,总是到处碰壁。

货币或商品流回它们的起点即资本家手中,是资本在生产过程和流通过程中具有特征的运动,这一方面表示现实的形态变化,即商品转化为它的生产条件,生产条件再转化为商品形式:再生产;另一方面又表示形式上的形态变化,即商品转化为货币,货币再转化为商品。最后,这还表示价值的增长,G—W—G′。原有的、但是已在过程中增大了的价值始终保留在同一个资本家手中。改变的只是资本家占有这个价值的形式——或者是货币形式,或者是商品形式,或者是生产过程本身的形式。

资本流回到它的起点,在生息资本的场合,取得了一个完全表面的、同现实运动(资本的回流就是这种运动的形式)相分离的形态。A把他的货币不是作为货币,而是作为资本支出。在这里,货币没有发生任何变化。它不过转手而已。它只是在B手中才实际转化为资本。但对A来说,货币变成资本是由于它从A手中转到了B手中。资本由生产过程和流通过程实际流回的现象,是对B来说的。而对A来说,流回是在和让渡相同的形式上进行的。货币由B手中再回到A手中。A是贷出货币,而不是支出货币。

货币在资本的实际生产过程中的每一次换位,都表示再生产的一个要素:或者是货币转化为劳动,或者是完成的商品转化为货币(生产行为的结束),或者是货币再转化为商品(生产过程的更新,再生产的重复)。在货币作为资本贷出时,就是说,在它不是转化为资本,而是作为资本进入流通时,货币的换位不过表示货币本身的转手。所有权留在贷款人手中,而对货币的支配则转到产业资本家手中。但对贷款人来说,货币转化为资本是从他把货币不是作为货币而是作为资本支出时开始的,即从他把货币交到产业资本家手中开始的。(对贷款人来说,即使他把货币不是贷给产业家,而是贷给浪费者,或者贷给交不起房租的工人,货币也仍然是资本。全部典当业就是建立在这个基础上的。)诚然,另外一个人把货币转化为资本,但这个行为是在贷款人与借款人发生的行为之外完成的。在这个行为中,这种中介过程消失了,看不见了,不直接包含在内了。这里表现出来的不是货币向资本的实际转化,只是这种转化的毫无内容的形式。正如劳动能力的情况一样,在这里,货币的使用价值就是:货币创造交换价值,创造比它本身所包含的更大的交换价值。货币作为自行增殖的价值贷出,作为商品贷出,不过是作为这样一种商品,它恰恰由于自己的这种属性而同商品本身相区别,从而也具有特殊的让渡形式。

资本的起点是商品所有者,货币所有者,简单地说,是资本家。因为资本的起点和终点是一致的,所以资本又流回到资本家手中。但是在这里,资本家是以双重身分存在的:既作为资本所有者,又作为把货币实际转化为资本的产业资本家。事实上,[894]资本是从产业资本家那里流出,然后又流回到他那里,但他仅仅是暂时的所有者。资本家是以双重身分存在的:法律上的和经济上的。因此,资本作为所有物,也就回到法律上的资本家那里,回到非正式的丈夫那里。然而资本的回流(这种回流包含着资本价值的保存,它使资本成为自行保存的和永久化的价值)只是对资本家II起中介作用,而不是对资本家I起中介作用。因此,资本的回流在这里也不是表现为一系列经济过程的归宿和结果,而是表现为买者和卖者之间的特殊的法律上的交易的结果,这就是,资本在这种场合是被贷出,不是被卖出,即只是暂时让渡。事实上,被卖出的只是它的使用价值,使用价值在这里就在于生产交换价值,提供利润,生产比它本身所包含的价值更多的价值。作为货币,资本并不由于使用而改变。但它是作为货币被付出,也是作为货币再流回。

资本流回的形式,取决于它的再生产方式。如果资本作为货币贷出,它就以流动资本的形式流回,流回的量等于它的全部价值加剩余价值(在这个场合就是剩余价值或利润中归结为利息的部分),即贷出的货币额加由它产生的增长额。

如果资本以机器、建筑物等形式贷出,简单地说,以资本在生产过程中必须借以执行固定资本职能的物质形式贷出,那末,它就以固定资本的形式,例如作为年支付流回(这个年支付等于对损耗的补偿额,即固定资本中进入流通的价值部分),再加上剩余价值中算作固定资本(不是因为它是固定资本,而是因为它是一定量的资本一般)利润的部分(在这里是利润的一部分,即利息)。

在利润本身,剩余价值,从而利润的真正源泉,已经模糊不清了,神秘化了:

(1)因为,从形式上考察,利润是以全部预付资本计算的剩余价值,因此资本的每个部分,不管是固定资本还是流动资本,是花在原料、机器上还是花在劳动上,都提供相同的利润;

(2)因为,某一单个的已知资本,例如500,如果它的剩余价值等于50,资本的每个部分,例如每五分之一,就都提供10%,这样,由于一般利润率的确立,现在每个500或100的资本,不管它用于哪个领域,不管其中可变资本和不变资本的比例如何,也不管它的周转时间如何不同等等,它同其他任何一个在完全不同的有机条件下活动的资本一样,在相同的期间,总要提供相同的平均利润,例如10%。这就是说,因为孤立地加以考察的各个单个资本的利润和由这些资本本身在其各自的生产领域创造的剩余价值,实际上是不等的量。

其实,第二点只是把第一点已经包括的东西作了进一步的阐述。

不过,作为利息的基础的,正是剩余价值的这种外表化的形式,也就是剩余价值作为利润而存在的形式,这种形式不同于它的最初的简单形态(这时它还带着出生的脐带)而且绝非一眼就可以辨认出来。利息不是直接以剩余价值为前提,而是直接以利润为前提,利息本身只是被归入特殊范畴、特殊项目内的一部分利润。因此,在利息上比在利润上识别剩余价值要困难得多,因为只有当剩余价值以利润形式出现时,利息才同它直接发生关系。

资本回流的时间取决于实际的生产过程;就生息资本来说,它作为资本的回流看来仅仅取决于贷款人和借款人之间的契约。所以,就这种交易来看,资本的回流不再表现为由生产过程决定的结果,而是表现为资本似乎一刻也没有丧失货币的形式。当然,这些交易是由资本的实际的回流决定的,但是这一点不会在交易本身中表现出来。

[895]利息和利润不同,它代表单纯的资本所有权的价值,就是说,它使货币(价值额,任何形式的商品)所有权潜在地成为资本所有权,从而使商品或货币本身成为自行增殖的价值。当然,劳动条件只有当它们作为工人的非所有物,从而作为别人的所有物同工人相对立来执行职能的时候,才是资本。但是只有同劳动相对立,它们才能作为别人的所有物执行职能。这些劳动条件和劳动的对立存在,使它们的所有者成为资本家,使资本家占有的这些劳动条件成为资本。但是,在货币资本家A手中,资本不具有这种使自己成为资本,从而也使货币所有权表现为资本所有权的对立性质。货币或商品借以成为资本的现实的形式规定性消失了。货币资本家A决不是同工人相对立,他只是同另一个资本家B相对立。他卖给B的,事实上只是货币的“使用权”,是货币转化为生产资本时将会产生的结果。但是他直接出卖的事实上并不是使用权。如果我出卖商品,我就是出卖一定的使用价值。如果我用商品购买货币,那我就是购买了作为商品的转化形式的货币所具有的执行职能的使用价值。我不是在出卖商品的交换价值的同时出卖商品的使用价值,我也不是在购买货币本身的同时购买货币的特殊的使用价值。但是,作为货币的货币,在转化为资本并执行资本的职能(货币在贷款人手中没有执行这种职能)之前,它所具有的使用价值,不外是它作为商品(金、银——货币的物质实体)或作为货币,作为商品的转化形式所具有的使用价值。事实上,贷款人卖给产业资本家的,即在这次交易中发生的,不过是贷款人把货币所有权让给产业资本家一段时间。他在一定期间让渡自己的所有权,也就是产业资本家在一定期间购买这个所有权。因此,贷款人的货币在被让渡之前就已经作为资本出现:同资本主义生产过程分离的、单纯的货币或商品所有权就已经作为资本出现。

货币只有在让渡之后才表现为资本,这对于事情本身毫无影响,正象棉花的使用价值实际上只有在棉花让渡给纺纱业者之后才表现出来,或者肉的使用价值实际上只有在肉从肉铺里转到消费者的餐桌上才表现出来,并不改变棉花或肉的使用价值一样。货币一旦不用于消费,商品一旦不再为它的所有者的消费服务,它们就会使它们的所有者成为资本家,而它们自己——同资本主义生产过程相分离,并且在转化为“生产”资本之前——就作为资本出现,也就是说,作为自行增殖、自行保存、自行增长的价值出现。创造价值、提供利息是它们内在的属性,就象梨树的属性是结梨子一样。贷款人就是把自己的货币作为这种生息的东西出卖给产业资本家的。因为货币会自行保存,是自行保存的价值,所以产业资本家能够按照随意约定的期限把它归还。因为货币每年创造一定的剩余价值,一定的利息,确切些说,因为在每一段时间内它的价值都在增长,所以,产业资本家也能够每年或在契约规定的其他任何期限内把这个剩余价值支付给贷款人。要知道,作为资本的货币,和雇佣劳动完全一样,每天都提供剩余价值。利息虽然只是利润中固定在特殊名称下的部分,它在这里却表现为这样一种剩余价值的创造,这种剩余价值的创造是资本本身所固有的,同生产过程是分离的,因而是单纯的资本所有权即货币和商品的所有权所固有的,同造成这种所有权和劳动之间的对立从而使这种所有权具有资本主义所有权性质的那些关系是分离的,——这种剩余价值的创造是单纯的资本所有权所固有的,因而是本来意义上的资本所固有的。相反,产业利润在这里只不过表现为利息的附加额,这个附加额是借款人把借来的资本用在生产上,即用这笔资本对工人进行剥削而挣得的(或者象人们所说的:通过自己作为资本家的劳动;资本家的职能在这里被说成等于劳动,甚至被说成和雇佣劳动等同,因为[896]真正在生产过程中执行职能的产业资本家,事实上是作为从事活动的生产当事人,作为劳动者而与游手好闲、无所事事的贷款人相区别,贷款人同生产过程相分离并且处在这个过程之外执行所有者的职能)。

这样,利息,而不是利润,表现为从资本本身,因而从单纯的资本所有权中产生的资本的价值创造;因此利息表现为由资本本能地创造出来的收入。庸俗经济学家就是在这种形式上理解利息的。在这种形式上,一切中介过程都消失了,资本的物神的形态也象资本物神的观念一样已经完成。这种形态之所以必然产生,是由于资本的法律上的所有权同它的经济上的所有权分离,由于一部分利润在利息的名义下被完全离开生产过程的资本自身或资本所有者所占有。

对于要把资本说成是价值和价值创造的独立源泉的庸俗经济学家来说,这个形式自然是他们求之不得的,在这个形式上,利润的源泉再也看不出来了,资本主义过程的结果也离开过程本身而取得了独立的存在。在G—W—G′中,还包含有中介过程。在G—G′中,我们看到了资本的没有概念的形式,看到了生产关系的最高度的颠倒和物化。

一般利息率,或者说,一般利率当然是和一般利润率相适应的。在这里,我们不打算进一步阐明这个问题,因为对生息资本的分析不属于概论这一篇,而属于论信用那一篇。\endnote{马克思说的“概论这一篇”是指《资本一般》(《dasKapitalimallge-meinen》)那一篇。按照1858—1862年计划,这一篇应由三部分组成(《资本的生产过程》、《资本的流通过程》和《二者的统一,或资本和利润》),紧接着这一篇之后应该是具有更专门性质的三篇:《资本的竞争》、《信用》和《股份资本》。马克思在后来写作《资本论》的过程中,还把很多按照原来计划不属于《资本一般》这一篇问题范围的东西逐渐地收入《资本的生产过程》、《资本的流通过程》和《二者的统一,或资本和利润》这几部分中。其中包括有很多与信用和信用制度有关的问题被收入由《资本和利润》这一篇发展而来的《资本论》第三卷。——第512页。}但是,为了完全弄清楚资本的这个表现形式,指出如下一点是重要的:一般利润率远远不象利息率,或者说利率那样表现为可以捉摸的、明确的事实。诚然,利率在不断地波动。今天(在向产业资本家贷款的货币市场上,我们所谈的只是这个方面)利率是2%,明天是3%,后天又是5%。但这个2%,3%或5%的利率是适用于所有贷款人的。提供2%,3%,5%,是任何一个100镑货币额的一般比率,同一个实际执行资本职能的价值额,在不同的生产领域所提供的实际利润却很不相同,这些实际利润对利润的观念上的平均水平的偏离,使利润的这个平均水平始终只有通过某种过程,通过某种反作用才能确立下来,而这一点又始终只有在较长的资本流通期间才能做到。在若干年间,一定领域的利润率较高,而在以后若干年间则较低。把这若干年或一系列这样的演变综合在一起,平均起来就得出平均利润。但这样一来,平均利润就不表现为直接既定的东西,而只表现为各种相互矛盾的波动的平均结果。利率却不是这样。它普遍地是每天确定的事实,这个事实对产业资本家来说,甚至是他们从事活动时计算上的前提和项目。一般利润率在用来估计实际利润时,事实上仅仅作为观念上的平均数存在;在它被固定为现成的、确定的、既定的东西时,它仅仅作为平均数、作为抽象物存在;在现实生活中,它仅仅作为在各种不同的实际利润率的平均化运动中起决定作用的趋势存在,而不管这些利润率是属于同一领域的单个资本,还是属于不同生产领域的不同资本。

[897]贷款人向资本家要求的,是根据一般利润率(平均利润率)计算,而不是根据单个资本家那里出现的对一般利润率的个别偏离。平均数在这里成了前提。利息率本身在变动,但是这种变动对所有的贷款人都适用。

相反,确定的、相同的利息率不仅按平均数来说是存在的,而且事实上也是存在的(虽然它根据借款人是否被认为第一流的债务人而在最低限度和最高限度之间变动),对它的偏离宁可说是由特殊情况所造成的例外。同记载气压状况的气象报告相比,这种不是为这个或那个资本编制,而是为货币市场上现有的资本即借贷资本编制的记载利息率状况的证券交易报告,其准确性毫不逊色。

借贷资本的利息率具有较大的固定性和等同性,它和一般利润率的较难捉摸的形式不同,而且相反。这种情况从何而来,这里不去阐述。这样的阐述属于论信用那一篇。不过有一点是明显的:每个领域内的利润率的波动,——同一个生产领域内的单个资本家享有的特殊利益完全撇开不谈,——都取决于当时市场价格的状况和市场价格围绕费用价格的波动。不同领域的利润率的差别,只有通过不同领域的市场价格即不同商品的市场价格和这些商品的费用价格的比较才能知道。某个特殊领域的利润率下降到观念上的平均水平之下,如果时间拖得很久,就足以使资本离开这个领域,或者使新资本不可能按平均规模流入这个领域。因为,新的追加资本的流入,同已经投入的资本的再分配相比,更能使资本在各特殊领域的分配平均化。而特殊领域的超额利润只有通过市场价格和费用价格的比较才能知道。只要差别以某种方式表现出来,资本就开始从一些生产领域流出而流入另一些生产领域。撇开这种平均化行为需要时间这一点不谈,每个特殊领域的平均利润本身,也只有根据资本的性质,通过例如在7年等等的周期内所实现的利润率的平均数表现出来。因此,单是上下波动,如果不超过平均程度,不采取异常的形式,就不足以引起资本的转移,何况固定资本还会给资本的转移带来困难。一时的行情只能在有限的程度上产生影响,而且它对追加资本的流人或流出的影响,要大于对已经投入不同领域的资本的再分配的影响。

我们看到,所有这一切是一个非常复杂的运动,这里要考察的,不仅有每个特殊领域的市场价格、不同商品的比较费用价格、每个领域的供求状况,而且有不同领域的资本家的竞争!此外,平均化的快慢在这里取决于资本的特殊有机构成(例如,固定资本多还是流动资本多)和它们的商品的特殊性质,就是说,要看商品作为使用价值的性质是否易于允许按照市场价格的状况把它们较快地撤出市场、减少或增加它们的供给。

货币资本的情况则完全不同。在货币市场上,互相对立的只是两个范畴:买者和卖者,需求和供给。一方面是借款的资本家阶级,另一方面是贷款的资本家阶级。商品具有同一形式——货币。资本因投在特殊生产领域或流通领域而具有的一切特殊形态,在这里都消失了。在这里,资本是存在于独立的交换价值即货币的没有差别的彼此等同的形态上。特殊领域之间的竞争在这里停止了;它们全体一起作为借款人出现,资本则以这样一个形式与它们全体相对立,在这个形式上,按怎样的方式使用的问题对资本来说还是无关紧要的事。如果说生产资本[898]只是在特殊领域之间的运动和竞争中把自己表现为整个阶级共有的资本,那末,资本在这里现实地有力地在对资本的需求中表现为整个阶级共有的资本。另一方面,货币资本(货币市场上的资本)也实际具有这样一个形态,在这个形态上,它是作为共同的要素,而不问它的特殊使用方式如何,根据每个特殊领域的生产需要,被分配在不同领域之间,被分配在资本家阶级之间。并且,随着大工业的发展,出现在市场上的货币资本,会越来越不由个别的资本家来代表,即越来越不由市场上现有资本的这个部分或那个部分的所有者来代表,而是由把它集中起来,组织起来,并且以完全不同于实际生产的方式把它控制起来的银行家来代表。因此,就需求的形式来说,和货币资本相对立的是整个阶级的力量;但就供给来说,这个资本整个地表现为借贷资本,表现为集中在少数蓄水池里的全社会的借贷资本。

这就是为什么一般利润率同固定的利息率相比,表现为模糊不清的景象的一些理由;利息率的大小固然也会变动,但并不妨碍它对所有借款人来说都一样地发生变动,所以它在他们面前总是表现为固定的、既定的量,象货币的价值的变动并不妨碍它对一切商品来说都具有相同的价值一样;象商品的市场价格虽然每天发生波动,但并不妨碍它逐日都有牌价一样,利息率的变动也不妨碍它作为货币的价格有规则地在牌价中标示出来。这是因为资本本身在这里是作为一种特殊的商品——货币——提供的;因此,它的价格的确定,和其他一切商品的情形一样,就是它的市场价格的确定;因此,利息率总是表现为一般利息率,表现为这样多的货币取得这样多的利息。而利润率甚至在同一个领域内,在商品市场价格相等的情况下,也可能不同(因为各单个资本生产相同的商品时的条件不同;因为特殊利润率不是由商品的市场价格决定的,而是由市场价格和生产费用之间的差额决定的),而不同领域的利润率,只是在过程中通过不断的波动才能达到平均化。一句话:只是在货币资本上,在借贷货币资本上,资本才成为商品,这种商品的自行增殖的属性具有一个固定的价格,由当时的利息表示出来。

因此,资本作为生息资本,而且正是在它作为生息货币资本的直接形式上(生息资本的其他形式在这里与我们无关,这些其他形式也是由这个形式派生出来的,并以这个形式为前提),取得了它的纯粹的拜物教形式。第一,这是由于资本作为货币的不断存在;在这样的形式上,资本的一切规定性都已经消失,它的现实要素也看不出来;它仅仅作为独立的交换价值、作为获得独立存在的价值而存在。在资本的现实过程中,货币形式是一个转瞬即逝的形式。在货币市场上,资本总是以这个形式存在。第二,资本所产生的剩余价值,又是在货币形式上,表现为资本本身应得的东西,表现为货币资本,即同它完成的过程相脱离的资本的单纯所有者应得的东西。G—W—G′在这里成了G—G′,而且,正象资本形式在这里是没有差别的货币形式一样,——因为货币正好是这样一个形式,在这个形式上,商品作为使用价值的差别消失了,从而由这些商品的存在条件构成的生产资本的差别,生产资本本身的特殊形式的差别也消失了,——货币资本所产生的剩余价值,它所转化成或表现出来的剩余货币,也表现为根据货币额本身的量来计算的一定的比率。利息率是5%时,作为资本的100镑就等于105镑。这样就得出一个自行增殖的价值的,或者说,创造货币的货币的十分明显的形式。它同时又是毫无内容的形式,不可理解的、神秘的形式。我们在分析资本时是从G—W—G出发的,G—G′不过是它的结果而已。\endnote{马克思指他的1861—1863年手稿第I本,这一本从《货币转化为资本》这一节开始。该节第一小节是《G—W—G。资本的最一般形式》(1861—1863年手稿第1—6页)。——第517页。}现在我们发现G—G′作为主体。正象生长是树木固有的属性一样,生出货币(τοχοs)\authornote{产物,利息。——编者注}是资本在其作为货币的纯粹的形式上固有的属性。我们在外表上发现的、因而曾经作为我们分析的出发点的这个不可理解的形式,现在又作为一个过程的结果被我们碰到了,在这个过程中,资本的形态越来越和它的内在本质相异化,并且越来越与之失去联系。

[899]我们从作为商品的转化形式的货币出发。现在我们到达作为资本的转化形式的货币。这和我们曾经把商品看成是资本生产过程的前提和结果完全一样。

资本在自己这种最奇特同时又和普通观念最接近的形态上,既是庸俗经济学家的“基本形式”,又是肤浅的批判的最直接的攻击点。就前者来说,部分地是因为内在联系在这里最少表现出来,而且资本是以一种好象是价值的独立源泉的形式出现;部分地是因为在这种形式上资本的对立性质完全被掩盖了,被抹杀了,资本和劳动的对立不见了。另一方面,这种形式的资本所以受到攻击,是因为它在这里以最不合理的形式表现出来,给庸俗社会主义者提供了最容易突破的攻击点。

十七世纪资产阶级经济学家(柴尔德、卡耳佩珀等人)反对把利息看作剩余价值的独立形式,这种论战只是新兴的产业资产阶级反对旧式高利贷者——当时货币财富的垄断者——的斗争。在这里,生息资本还是一种洪水期前的资本形式,这种形式只是刚刚不得不从属于产业资本,处于依附产业资本的地位,这是生息资本在资本主义生产基础上从理论和实践上都必须占有的地位。资产阶级在这里也象在其他场合一样,毫不迟疑地去求助于国家,使现存的、旧时遗留下来的生产关系适合于它自己的需要。

显然,按另一种办法在不同种类的资本家之间分配利润,即靠降低利率来提高产业利润或者相反,都绝不会触动资本主义生产的本质。因此,把生息资本当作资本的“基本形式”来反对的社会主义,就不仅是本身完全局限于资产阶级视野的问题。而且,就它的论战并非一种出于误解的、盲目向资本本身发起的攻击和批判来说(不过,在这里把资本和资本的一种派生形式等同起来了),它无非是一种披着社会主义外衣的、要求发展资产阶级信用的愿望,因此,它只是表示,在这种论战披上社会主义外衣的国家里,现存关系是不发达的。这种社会主义本身只是资本主义发展的一个理论上的征兆,尽管这种资产阶级的努力可能采取非常惊人的形式,例如“无息信贷”的形式\endnote{指蒲鲁东在他同巴师夏的论战(1849—1850年)中为“无息信贷”辩护。马克思在后面,在他的手稿第935—937页(见本册第581—585页)对蒲鲁东的这个观点进行了批判。——第518页。}。圣西门主义及其对于银行制度的赞美就是属于这一类(以后又出现过“动产信用公司”\endnote{动产信用公司(CreditMobilier,全称SocietegeneraleduCreditMobilier)是法国的一家大股份银行,创办于1852年。该银行以进行金融投机活动著称,最后于1867年破产。1856—1857年,马克思为伦敦的宪章派报纸《人民报》(《ThePeople’sPaper》)和美国的报纸《纽约每日论坛报》(《New-YorkDailyTribune》)写了六篇文章评论该银行的投机活动(见《马克思恩格斯全集》中文版第12卷第23—40、218、227、313—317页以及第13卷第85—86、186页)。——第518页。})。

\tchapternonum{[(2)]生息资本和商业资本同产业资本的关系。更为古老的形式。派生的形式}

商业形式和利息形式比资本主义生产的形式即产业资本更古老。产业资本是在资产阶级社会占统治地位的资本主义关系的基本形式,其他一切形式都不过是从这个基本形式派生的,或者与它相比是次要的,——派生的,如生息资本;次要的,也就是执行某种特殊职能(属于资本的流通过程)的资本,如商业资本。所以,产业资本在它的产生过程中还必须使这些形式从属于自己,并把它们转化为它自己的派生的或特殊的职能。产业资本在它形成和产生的时期碰到了这些更为古老的形式。产业资本碰到它们时把它们作为前提,但不是作为由它本身确立的前提,不是作为它自己生活过程的形式。这如同它最初碰到了商品,但不是作为它自己的产品,它碰到了货币流通,但不是作为它自己的再生产要素。一旦资本主义生产在它的所有形式上发展起来,成了占统治地位的生产方式,生息资本就会从属于产业资本,商业资本就会仅仅成为产业资本本身的一种从流通过程派生的形式。但是,作为独立形式的[900]生息资本和商业资本必须先被摧毁并从属于产业资本。对生息资本使用行政权力(国家),强行降低利率,使生息资本再也不能把条件强加于产业资本。但是,这是资本主义生产一些最不发达的阶段所特有的形式。产业资本为了使生息资本从属于自己而使用的真正方式,是创造一种产业资本所特有的形式——信用制度。强行降低利率还是产业资本本身从以前的生产方式的方法中借用来的形式,一旦产业资本强大了,夺取了地盘,它就把这个形式当作无用的、不合目的的东西扔掉。信用制度是它自己的创造,信用制度本身是产业资本的一种形式,它开始于工场手工业,随着大工业而进一步发展起来。信用制度最初是反对旧式高利贷者(英国的金匠、犹太人、伦巴第人等等)的论战形式。十七世纪揭示信用制度的最初秘密的著作,全是以这种论战形式写成的。

至于商业资本,它以各种不同的形式从属于产业资本,或者也可以说,它成了后者的职能,成了执行某种特殊职能的产业资本。商人不是购买商品,而是购买雇佣劳动,用以生产供他进行商业销售的商品。但是,这样一来,商业资本本身就失去了它和生产相比所具有的固定形式。工场手工业通过商人向商品生产者的这种转化来反对中世纪的行会,并把手工业限制在比较狭小的范围。在中世纪,商人(意大利、西班牙等国工场手工业发达的个别分散的点除外)不过是城市行会或农民所生产的商品的包买商。\endnote{马克思在《资本论》第三卷第二十章中指出:波佩认为,中世纪的商人不过是行会手工业者或农民所生产的商品的“包买商”。见约·亨·摩·波佩的著作《从科学复兴至十八世纪末的工艺学历史》1807年哥丁根版第1卷第70页。——第520页。}

商人向产业资本家的这种转化,同时也是商业资本向单纯的产业资本形式的转化。另一方面,生产者成了商人。例如,呢绒生产者不是逐渐地一小批一小批地从商人那里获得自己所需要的材料并为之加工,而是自己按照自己资本的大小去购买材料等等。各种生产条件都作为他自己买来的商品进入生产过程。呢绒生产者现在已经不是为个别商人或某些顾客生产,而是为商业界生产了。

在第一种形式上,商人统治着生产,商业资本统治着由它推动的城市手工业者的劳动和农民家庭手工业。手工业和家庭手工业是从属于它的。在第二种形式上,生产转化为资本主义生产。生产者自己就是商人;商业资本在这里只是在流通过程中起中介作用,在资本的再生产过程中执行一定的职能。这是两种形式。商人作为商人成为生产者、产业家。产业家、生产者成为商人。

起初,商业是行会的、农村家庭的和封建的农业生产转化为资本主义生产的前提。它使产品发展成为商品,这部分地是因为它为产品创造了一个市场,部分地是因为它提供了新的商品等价物,部分地是因为为生产提供了新的材料,并由此开创了一些生产部门,它们一开始就以商业为基础:既以替市场生产为基础,也以世界市场造成的生产要素为基础。

一旦工场手工业(尤其大工业)相当巩固了,它就又为自己创造市场,夺取市场,一部分是采用暴力手段来开辟市场,但市场是它用自己的商品本身来夺取的。以后,商业就只不过是工业生产的奴仆,而对工业生产来说,市场的不断扩大则是它的生活条件,因为不断扩大的大量生产不受现有的商业界限(就商业仅仅反映现有需求而言)的限制,而是仅仅受现有的资本量和劳动生产力发展水平的限制,它不断地使现有市场商品充斥,从而不断地促使市场界限扩大和改变。在这里,商业是产业资本的奴仆,它执行从产业资本的生产条件中产生的一项职能。

产业资本在其发展的初期,试图用殖民制度(同时用禁止性关税制度)以暴力手段为自己确保一个市场和若干市场。产业资本家面对着世界市场;因此,他要[901]把自己的生产费用不仅同国内的市场价格相比较,而且同整个世界市场的市场价格相比较,同时必须经常这样做。他在生产时总是要考虑世界市场的市场价格。以前,这种比较只是商人的事,这样就保证了商业资本对生产资本的统治。[901]

\centerbox{※     ※     ※}

[902]可见,利息无非是利润的一部分(利润本身又无非是剩余价值,无酬劳动),它是由完全地或部分地借助别人的资本从事“劳动”的产业资本家支付给这笔资本的所有者的。利息是利润——剩余价值——的一部分,这一部分作为一种特殊的范畴被固定下来,以特有的名称和总利润相分离;这种分离和利息的起源毫无关系,只和它的支付或占有的方式有关。尽管产业资本家直接掌握全部剩余价值,不管剩余价值以地租、产业利润和利息的名义在产业资本家和其他人之间怎样进行分配,产业资本家总不是自己占有这部分利润,而是把它从自己的收入中扣除,支付给资本所有者。

如果利润率是既定的,利息率的相对高度就取决于利润分割为利息和产业利润的比例;如果这种分割的比例是既定的,利息率的绝对高度(即利息对资本的比例)就取决于利润率。这种分割比例是怎样确定的,这里不打算研究。这是属于对资本的现实运动,亦即对各个资本的现实运动的考察问题,而我们这里涉及的是资本的一般形式。

生息资本的形成,它和产业资本的分离,是产业资本本身的发展、资本主义生产方式本身的发展的必然产物。货币(即总是可以转化为生产条件的价值额)——或生产条件(货币随时都可以转化为生产条件,货币不过是生产条件的转化形式)——作为资本来使用,就可以支配一定量的别人劳动,支配比它本身所包含的更多的劳动。货币在同劳动交换时不仅保存自己的价值,而且增加自己的价值,产生剩余价值。作为资本的货币或商品,其价值不是由它们作为货币或商品所具有的价值来决定,而是由它们为自己的所有者“生产”的剩余价值的量来决定。资本的产品是利润。在资本主义生产的基础上,货币是作为货币支出还是作为资本支出,只是货币的不同的用途。在资本主义生产的基础上,货币(商品)从可能性来说是资本(正象劳动能力从可能性来说是劳动完全一样),因为,第一,它可以转化为生产条件,而且实际上也只是这些生产条件的抽象表现,是它们作为价值的存在,第二,财富的物质要素从可能性来说具有成为资本的属性,因为这些要素的对立面——雇佣劳动,也就是使它们成为资本的东西——是作为社会生产的基础存在的。

地租也只是产业资本家必须支付给另一个人的一部分剩余价值的名称,正如利息是由产业资本家虽然收进来(和地租一样)但是必须支付给第三者的另一部分剩余价值一样。然而,这里有很大的区别。土地所有者利用土地所有权阻止资本按照农产品的费用价格使它们的价值平均化。对土地所有权的垄断使他有可能这样做。它使他有可能把价值和费用价格间的差额装进自己的腰包。另一方面——在涉及级差地租的情况下——这种垄断还使他有可能拿走产品的市场价值超过一定土地上产品的个别价值的余额,而不是象在其他部门那样,这个差额作为超额利润落入在比平均条件有利的条件下从事经营的资本家的腰包,因为平均条件满足需求的基本量,决定生产的主要量,从而调节每个特殊生产领域的市场价值。

土地所有权是夺取产业资本生产的一部分剩余价值的手段。相反,贷出的资本——在资本家用借来的资本从事经营的情况下——是生产全部[903]剩余价值本身的手段。货币(商品)可以作为资本贷出这种情况,不过意味着,货币从可能性来说是资本。李嘉图所说的废除土地所有权,即把土地所有权变为国家所有权,把地租交付给国家而不是交给地主,是一种理想,是资本从它最内在的本质中产生的内心愿望。资本不可能废除土地所有权。但是,通过把土地所有权转化为[交给国家的]地租,资本作为阶级占有了地租,以抵补自己的国库开支,就是说,资本通过迂回的办法占有了它不能直接拿到手的东西。可是废除利息和生息资本,就是废除资本和资本主义生产本身。只要货币(商品)可以用作资本,它就可以作为资本出卖。因此,那些要商品而不要货币、要产业资本而不要生息资本、要利润而不要利息的人,真不愧为小资产阶级空想主义者。

生息资本和提供利润的资本——这并不是两种不同的资本,而是同一个资本,它在生产过程中执行资本的职能,提供利润,利润在两种不同的资本家之间进行分配:一种是处在生产过程之外、作为所有者代表资本自身的资本家{不过,资本由私有者代表是资本的基本条件;不然的话,它就不成其为与雇佣劳动相对立的资本了},一种是代表执行职能的资本即处于生产过程中的资本的资本家。

\tchapternonum{[(3)剩余价值的各个部分独立化为不同形式的收入。利息和产业利润之间的比例。收入的拜物教形式的不合理性]}

利润分割的进一步“硬化”或独立化是这样表现出来的:每个资本的利润,从而以资本互相平均化为基础的平均利润,都分成或被割裂成两个互不依赖或互相独立的部分,即利息和产业利润,后者现在往往也被简单地称为利润或取得“监督劳动的工资”这样的新教名。如果利润率(平均利润)等于15%,利息率(如我们已看到的那样,它总是固定在一般的形式上)等于5%(一般利息率在货币市场上总是作为货币的“价值”或“价格”标示出来),那末资本家——即使他是资本的所有者,资本的任何部分都不是借来的,从而利润不必在两种资本家中间进行分配——就会这样来考虑问题:在这15%当中,5%代表他的资本的利息,只有10%代表他把资本用于生产而取得的利润。这5%的利息是他作为“产业资本家”对作为资本“所有者”的他自己所负的债;这个利息应付给他的资本自身,因而也应付给作为资本自身的所有者的他(这个资本自身同时也是资本的自为的存在,或者说资本作为资本家的存在,作为从自身中排他的所有权的存在),这个抽掉了生产过程的资本与执行职能的资本、处于生产过程中的资本不同,与代表这种执行职能的、“劳动的”资本的“产业资本家”不同。“利息”是资本的果实,即不进行“劳动”、不执行职能的资本的果实,而利润则是“劳动的”、执行职能的资本的果实。这和下述情况相类似:农场主-资本家——他同时又是土地所有者,用资本主义方式经营土地的土地所有者——把他的利润中形成地租的那一部分,把这个超额利润,不是归于作为资本家的他自己,而是归于作为土地所有者的他自己,不是归于资本,而是归于土地所有权,也就是说,他作为资本家欠了作为土地所有者的他自己的“地租”。这样一来,表现在一种规定性上的[生息的]资本,就和表现在另一种规定性上的[提供产业利润的]同一个资本,在固定的形式上对立起来,就象土地所有权和资本相互对立一样,事实上,土地所有权和资本是以两种本?上不同的生产资料为基础的、占有别人劳动的权利。

如果在一种场合有5个股东经营一家棉纺厂,棉纺厂代表100000镑资本,提供10%的利润,即10000镑,那末,每个股东各得利润的1/5,即2000镑。如果在另一种场合单独一个资本家把同量资本投入工厂,获得同量利润10000镑,那末,这个资本家不会认为:他得到的2000镑是一个股东的利润,得到的8000镑是四个不存在的股东的合伙利润。因此,在不同的[904]资本家——他们对同一笔资本具有不同的法律权利,以这种或那种形式表现为同一笔资本的共有者——之间进行的单纯的利润分割本身,绝不会给利润的这些部分建立不同的范畴。那末,为什么利润在资本贷出者和资本借入者之间的偶然的分割会建立这些范畴呢?

乍看起来,这里谈的只是利润的这样一种分割,在进行这种分割时存在两个资本所有者,两个具有不同权利的所有者,而这一点乍看起来是个法律因素而不是经济因素。资本家是用自有资本还是用别人的资本从事生产,或者他是按什么比例用自有资本和别人的资本从事生产,这个问题本身是完全无关紧要的。可是,利润分为[产业]利润和利息的这种分割并不表现为偶然的分割,不是取决于偶然的情况(即一个资本家是否实际上要同另一个资本家分割利润,他在这一场合是用自有资本经营还是用别人的资本经营),而是相反,即使他仅仅用自有资本进行生产,他也无论如何都要分裂为资本的单纯所有者和资本的使用者,分裂为生产过程外的资本和生产过程内的资本,分裂为自身提供利息的资本和作为处于[生产]过程中的、提供利润的资本,——这种情况是怎样产生的呢?

在这里,有一个现实的因素作为基础。货币(作为商品一般的价值表现)在生产过程中所以能占有剩余价值(不管它叫什么,不管它分解成哪些部分),只是因为在生产过程之前货币就已经被假定为资本。在生产过程中,货币作为资本把自己保存、生产和再生产出来,而且是在不断扩大的规模上这样做。但是,如果资本主义生产方式已经存在,如果生产是在这种生产方式的基础上并且在与它相适应的社会关系的范围内进行,就是说,如果所涉及的问题不单单是资本的形成过程,那末,早在生产过程之前,货币按其性质来说就已经作为资本自身存在了,尽管这种性质只是在过程中才实现,而且一般说来只有在过程本身中才具有现实性。如果货币不是作为资本进入过程,它也就不会作为资本,就是说,不会作为提供利润的货币,作为自行增殖的价值,作为生产剩余价值的价值从过程中出来。

这里的情况和货币的情况一样。例如,铸币无非是一块金属。它成为货币只是由于它在流通过程中的职能。但是,商品的流通过程一旦作为前提,铸币就不仅执行货币的职能,而且在它进入流通过程之前,就作为货币在每个单独的场合充当这个过程的前提了。

资本不仅是资本主义生产的结果,而且是它的前提。因此,货币和商品就其自身来说,潜在地是资本,在可能性上是资本:一切商品就其可能转化为货币而言,货币就其可能转化为形成资本主义生产过程要素的商品而言,都是这样的资本。可见,货币,——作为商品和劳动条件的纯粹的价值表现,——自身作为资本,是资本主义生产的前提。不作为过程的结果,而作为过程的前提来考察的资本是什么呢?是什么使它在进入过程之前就成为资本,从而过程只是使它的内在性质得到发展呢?是它借以存在的社会规定性:过去劳动同活劳动相对立,产品同活动相对立,物同人相对立,劳动本身的物的条件作为别人的、独立的、自我孤立的主体或人格化,一句话,作为别人的所有物,而且在这个形式上作为劳动本身的“使用者”和“支配者”(它们占有劳动而不是被劳动占有)同劳动相对立。价值(无论它是作为货币还是作为商品而存在),而在进一步的发展中则是劳动条件,作为别人的所有物,作为自我的所有物,同劳动者相对立,这无非是说,它们是作为非劳动者的所有物同劳动者相对立,或者至少是说,在劳动条件的所有者是资本家的情况下,他也不是作为劳动者,而是作为价值等等的所有者,作为主体(这些物就是在这个主体上具有自己的意志,自己属于自己,人格化为独立的力量)同这些劳动条件相对立。资本,作为生产的前提,资本,在它不是从生产过程中出来,而是在它进入生产过程之前的形式上,是一种对立性,在这种对立性中,劳动作为别人的劳动同资本相对立,资本本身作为别人的所有物同劳动相对立。在这里表现出来的,是离开过程本身的、已表现为资本所有权本身的那种对立的社会性质。

[905]这个因素是资本主义生产过程不断产生的结果,并且作为这样的结果又是它的不断需要的前提;这个因素离开资本主义生产过程本身,现在表现在这样的事实上:货币和商品就其自身来说,潜在地是资本;它们能够作为资本出售;在这个形式上,它们代表单纯的资本所有权,代表作为单纯所有者的资本家(撇开他的资本主义职能不谈);就它们本身来考察,它们是对别人劳动的支配权,因而是自行增殖的价值,并且提出占有别人劳动的要求。

这里也清楚地表明了:占有别人劳动的根据和手段,就是这种关系,而不是资本家方面提供的任何劳动或对等价值。

因此,利息表现为由作为资本的资本,由单纯的资本所有权产生的剩余价值,资本之所以从生产过程中得到这个剩余价值,是因为资本作为资本进入生产过程,也就是说,这个剩余价值属于资本本身,而不以生产过程为转移,——尽管剩余价值只是在生产过程中才出现,——因此,这个剩余价值是资本作为资本就已经潜在地包含着的。相反,产业利润则表现为不是属于作为资本所有者的资本家,而是属于作为执行职能的所有者的资本家,即属于执行职能的资本的那一部分剩余价值。就象在这种生产方式中一切看来都是颠倒的那样,在利息和利润的关系上的这种最后的颠倒也终于出现了,以致利润中划为特殊项目[利息]的部分反而表现为专门属于资本的产物,而产业利润却不过是在这个部分上增长起来的追加额。

因为货币资本家处于生产过程本身之外,实际上仅仅作为资本所有者取得剩余价值中属于他自己的一份;因为资本的价格,即单纯的资本所有权的价格会在货币市场上以利息率表示出来,就象其他任何商品的市场价格会在市场上表示出来一样;因为剩余价值中由资本自身即单纯的资本所有权决定的份额由于这一点而是一个既定的量,而利润率却是波动的,在不同的生产领域每时每刻都不同,在每个领域内在各个资本家之间也不同(部分是由于他们进行生产的有利条件不同,部分是由于他们对劳动实行资本主义剥削的本领和能力不同,部分是由于他们在欺骗商品的买者或卖者时走运和狡黠的程度——让渡利润——不同),——因为上述种种原因,所以在资本家看来,不管他们是不是处于过程中的资本的所有者,利息自然是由资本本身,由资本所有权,由资本所有者(不管资本所有者是他们自己还是第三者)产生的;相反,产业利润在他们看来则是他们的劳动的产物。他们是作为执行职能的资本家——资本主义生产的实际当事人——同作为资本的单纯的、不活动的存在的他们自己或第三者相对立,从而作为劳动者同作为所有者的他们自己或其他人相对立。既然他们是劳动者,他们实际上就是雇佣工人,而且由于他们的特殊的优越地位,他们不过是报酬较高的雇佣工人,他们所以能这样,部分地应归功于他们是自己给自己支付工资。

因此,当利息和作为生息资本的资本表示物质财富同劳动的单纯对立,因而表示物质财富作为资本的存在时,在通常的观念中这一点恰恰被颠倒过来了,因为在表面现象上货币资本家乍看起来同雇佣工人毫无关系,而只同另一个资本家发生关系,这另一个资本家又不是同雇佣劳动相对立,而是自己作为劳动者同作为资本的单纯存在、单纯所有者的他自己或另一个资本家相对立。此外,单个资本家可以把他的货币作为资本贷出,也可以自己把它作为资本来使用。在他从自己的货币中取得利息时,他得到的只是货币的价格,这个价格,在他不是作为资本家“执行职能”,在他不“劳动”时,也是可以得到的。因此很明显,就他实际上从生产过程中得到的只是利息来说,他只应归功于资本,而不应归功于生产过程本身和[906]代表执行职能的资本的他自己。

由此也产生了某些庸俗经济学家的绝妙的论调:如果产业资本家除利息外得不到其他利润,他就会用他的资本去生息而过食利者的生活。这样一来,所有的资本家就会停止生产,所有的资本就会停止执行资本的职能,但是他们仍然可以靠它的利息为生!杜尔哥就已经有类似的议论:如果资本家得不到利息,他就会购买土地(资本化的地租),就会靠地租为生。\endnote{安·罗·雅·杜尔哥《关于财富的形成和分配的考察》1766年版第73节和第85节。——第530页。}但是,因为地租在重农学派那里代表真正的剩余价值,所以在这里利息仍然是从剩余价值来的;而在上述那些庸俗见解中,真实关系却被颠倒过来了。

还有另一种情况必须指出:对于借钱的产业资本家来说,利息加入费用,在这里费用是指预付的价值。一笔例如1000镑的资本,不是作为价值1000镑的商品,而是作为资本加入他的生产;因此,如果1000镑资本每年提供10%的利息,那末它就作为1100镑的价值加入年产品。所以,这里清楚地表明,价值额(以及体现它的商品)不是在生产过程中才成为资本,而是作为资本形成生产过程的前提,从而它自身已经包含了属于它(单纯的资本)的剩余价值。对于用借来的资本经营的产业家来说,利息,或者说,作为资本的资本,加入他的费用,它所以表现为这样的资本,只是因为它产生剩余价值(所以,它作为商品,例如值1000,作为资本就值1100,即1000+1000/10,C+C/x)。如果在产品中只得到利息,那末,它虽然是超过作为单纯商品计算的预付资本的价值的余额,但不是超过作为资本计算的商品的价值的余额;产业家必须把这个剩余价值付给别人,它属于他的预付,属于他生产商品时的支出。

至于用自有资本经营的产业家,他必须把资本的利息支付给自己,并把它看成是预付。实际上,他预付的不仅是例如价值为1000镑的资本,而且是作为资本的1000镑的价值,如果利息等于5%,这个价值就是1050镑。这对他来说也决不是什么空想。因为,如果他不把这1000镑用于生产而把它借出,它作为资本就会给他带来1050镑。因此,只要他把这1000镑作为资本预付给自己,他就是给自己预付1050镑。总得在某一个人身上弥补自己的损失,哪怕在自己身上!

价值为1000镑的商品作为资本具有1050镑的价值。就是说,资本不是简单的数字:它不是简单的商品,而是自乘的商品;不是简单的量,而是量的比例。资本是它作为本金、作为既定的价值同作为[生产了]剩余价值[的本金]的它自己的比例。资本C的价值(按一年计算)等于C(1+1/x)\endnote{马克思在论马尔萨斯这一章中,考察了马尔萨斯在李嘉图的《原理》出版(1871年)以后所写的著作。在这些著作里,马尔萨斯企图用旨在维护统治阶级中最反动阶层的利益的庸俗辩护论来对抗李嘉图的劳动价值论,对抗李嘉图的千方百计发展生产力的要求,而照李嘉图的观点,这样发展生产力,就应当牺牲个人的甚至整个阶级的利益。关于作为“人口论”的鼓吹者的马尔萨斯,在本章中只是附带谈了一下。马克思在《剩余价值理论》第2册《对所谓李嘉图地租规律的发现史的评论》那一章中对马尔萨斯论人口的著作做了一般的评述(见本卷第2册第121、123、125—128、158页)。——第3页。}或C+C/x。正象用简单的计算方法不可能理解或推算出等式ax=n中的x一样,从基本的概念中也无法理解或推算出自乘的商品,自乘的货币,资本。

同利润的一部分,即资本所创造的剩余价值的一部分以利息的形式表现为资本家的预付完全一样,在农业生产中,另外一部分——地租,也表现为资本家的预付。但是在这里,这种看法的不合理性却不那么引人注目,因为地租在这里表现为土地的年价格,土地因此作为商品加入生产。“土地价格”固然比资本价格包含的不合理性更大,但这种不合理性不是包含在形式本身之中。因为土地在这里表现为一种商品的使用价值,而地租则表现为这种商品的价格。(不合理性在于,不是劳动产品的东西——土地,却有了价格,即表现在货币上的价值,也就是说,有了价值,因而,应该被看作物化社会劳动。)因此,就外在的形式来说,我们在这里也和在任何其他商品那里一样,看到了双重表现:表现为使用价值和交换价值,交换价值观念地表现为价格,表现为某种绝对不同于商品使用价值的东西。而在“1000镑=1050镑”或“50镑是1000镑的年价格”这种说法中,却是相同的东西即交换价值和交换价值发生关系,而且交换价值要作为和自身不同的东西成为它自身的价格,即表现在货币上的交换价值本身。

[907]所以,这里有两种形式的剩余价值——利息和地租,资本主义生产的结果——作为前提,作为预付加入生产;这种预付是资本家本人投入的,所以它对于资本家来说决不代表剩余价值,即超过他的预付价值的余额。就剩余价值的这些形式而言,在单个资本家本人看来,剩余价值的生产属于资本主义生产的生产费用,而占有别人的劳动和占有超过在过程中消费掉的商品(不论这些商品是加入不变资本还是加入可变资本)价值的余额,是这种生产方式的必要条件。当然,这一点也表现在:平均利润形成商品费用价格的一个要素,因而形成商品供给的条件,形成商品生产本身的条件。然而,产业资本家把这个余额,这部分剩余价值,——尽管它形成生产本身的一个要素,——公正地看成超过自己的费用的余额,不认为它象利息和地租那样属于自己的预付。事实上,在危机时刻,当价格的下跌使产业利润消失或显著减少,因而生产缩减或停顿的时候,利润也会作为生产条件同他本人相对立。从这里可以看出那些把剩余价值的不同形式单纯看作分配形式的人的愚钝。它们同样是生产形式。[907]

\centerbox{※     ※     ※}

[937]在“土地—地租,资本—利润(利息),劳动—工资”这个三位一体的公式中,可能看起来最后一个环节还是最合理的:这里至少说出了工资产生的源泉。但是实际上正相反,最后这个形式却是最不合理的,它是另外两个形式的基础,就象雇佣劳动一般说来以作为土地所有权的土地和作为资本的产品为前提一样。只有当劳动条件以这种形式同劳动对立的时候,劳动才是雇佣劳动。但是在“劳动—工资”这个公式中,劳动正是作为雇佣劳动来表现的。因为工资在这里表现为劳动的特殊产品,表现为劳动的唯一产品(工资对于雇佣工人来说确实也是劳动的唯一产品),所以价值的其他部分——地租、利润(利息)——也必然表现为是从其他的特殊源泉产生的;正象产品中归结为工资的价值部分必须理解为劳动的特殊产品一样,归结为地租和利润的价值部分也必须理解为它们为之存在并归其所有的那些因素的特殊结果,就是说,这些价值部分必须理解为土地和资本各自的果实。[937]

\tchapternonum{[(4)剩余价值的转化形式的硬化过程以及这些形式同它们的内在实质即剩余劳动日益}

分离的过程。生息资本是这个过程的最终阶段。把产业利润看成“资本家的工资”的辩护论观点]

[910]我们来考察一下资本在以生息资本形式出现之前所经历的道路。

在直接的生产过程中,情况还简单。剩余价值除了剩余价值本身这种形式外,还没有取得特殊的形式,剩余价值本身这种形式,只不过使剩余价值有别于产品的另一部分价值,即构成产品中再生产出来的价值等价物的那一部分价值。正如一般价值归结为劳动一样,剩余价值归结为剩余劳动,即无酬劳动。因此,剩余价值也只是以实际会改变自己价值的那部分资本——可变资本,花在工资上的资本——来计量的。不变资本不过是使资本的可变部分能够发生作用的条件。情况很简单,如果用等于10个人的劳动的100镑购买20个人的劳动(即包含20个人的劳动的商品),产品价值就等于200镑,而100镑剩余价值就是10个人的无酬劳动。或者说,如果有20个人劳动,那末每人只有半天为自己劳动,另外半天为资本家劳动。20个半天等于10天。这等于是,只有10个人的劳动得到报酬,而另外10个人则是白白为资本家劳动。

这里,在这种胚胎状态中,关系还是很清楚的,或者更确切地说,完全不会误解。这里的困难只是在于说明这种不付等价物便能占有劳动是怎样由商品交换规律(即商品按其包含的劳动时间互相交换)产生的,首先是,怎样与商品交换规律不发生矛盾。

[911]流通过程已经抹掉了、已经掩盖了实际存在的联系。因为剩余价值量在这里同时还决定于资本的流通时间,所以看起来,这里还加进了一种与劳动时间不同的要素。

最后,如果考察一下完成了的资本(它表现为一个整体,表现为流通过程和生产过程的统一,它是再生产过程的表现,即一定的价值额,这个价值额在一定期间,在一定的流通阶段,生产出一定的利润即剩余价值),那末,在这种形态上,生产过程和流通过程还只是作为一种回忆和作为在同等程度上决定剩余价值的因素而存在,因此,剩余价值的单纯性质就被模糊了。剩余价值现在表现为利润。这里必须注意以下几点:(1)这种利润与不同于劳动时间的资本的一定流通阶段有关;(2)剩余价值在计算时,不是同直接产生它的那部分资本相比,而是不加区分地同整个总资本相比;这样一来,剩余价值的源泉完全看不见了;(3)虽然在利润的这种最初形式上,利润量在数量上还与单个资本生产的剩余价值量相等,但是利润率从一开始就不同于剩余价值率,因为剩余价值率等于,m/v,而利润率等于m/(c+v);(4)在剩余价值率既定的情况下,利润率可能提高或降低;利润率甚至可能朝着与剩余价值率变动相反的方向变动。

可见,剩余价值在利润的最初形式上已经具有这样一种形式,这种形式不仅使人不能直接辨认它与剩余价值、剩余劳动的同一性,而且好象是直接与这种同一性相矛盾的。

其次,由于利润转化为平均利润,由于一般利润率的形成,以及与此有关的或由此决定的价值转化为费用价格,单个资本的利润,不仅在表现上(即在利润率和剩余价值率的区别上),而且在实体上(这里也就是在数量上)都和单个资本在其特殊生产领域里所生产的剩余价值本身不同。如果我们考察单个资本,而且也考察某个特殊领域的总资本,那末,利润现在就不仅在表面上,而且在实际上都和剩余价值不同。等量资本提供等量利润,或者说,利润与资本的量成比例。或者说,利润由预付资本的价值决定。在所有这些表现上,利润同资本有机构成的关系完全被掩盖了;这种关系在这里已经无法辨认了。相反,一眼就能看到的是,等量资本推动的劳动量极不相同,从而支配的剩余劳动极不相同,创造的剩余价值量极不相同,但是提供的利润量相同。由于价值转化为费用价格,商品价值决定于商品中包含的劳动时间这种基础本身,似乎也被取消了。

正是在利润的这种完全异化的形式上以及在利润的形式愈来愈掩盖自己的内核的情况下,资本愈来愈具有物的形态,愈来愈由一种关系转化为一种物,不过这种物是包含和吸收了社会关系的物,是获得了虚假生命和独立性而与自身发生关系的物,是一个可感觉而又超感觉的存在物;而且在资本和利润的这种形式上,资本表面上是作为现成的前提出现的。这就是资本的现实性的形式,或者更确切地说,是资本的现实存在的形式。资本也正是以这种形式存在于其承担者即资本家的意识中,反映在他们的观念中。

这种固定的和硬化的(变了形的)利润形式(从而也是利润的创造者即资本的形式,因为资本是根据,利润是归结,资本是原因,利润是结果,资本是实体,利润是偶性;资本所以成为资本,只是因为它生产利润,只是因为它是创造利润即创造追加价值的价值)——从而也是作为利润根据的资本的形式,作为资本保存下来并通过利润来增殖的资本的形式——更加固定在它的外表性上了,因为赋予利润以这种平均利润形式的资本的平均化过程,使一部分独立存在的、似乎是在其他基础上(即在土地上)生长出来的利润,在地租形式上同利润脱离了。诚然,地租起初是作为租地农场主向土地所有者支付的一部分利润出现的。但是,因为他(租地农场主)既不能把这种超额利润装入自己的腰包,而他所使用的资本,作为资本来说,又与其他资本不论在哪一点上都毫无区别(租地农场主之所以把超额利润交给土地所有者,是因为他并不认为作为资本的资本是超额利润的源泉),所以土地本身在这里就表现为商品的这一部分价值(它的这一部分剩余价值)的源泉,而土地所有者不过是[912]土地在法律上的人格化。

如果地租按预付资本计算,那还有一点线索,可以使人想起地租的来源就是利润即一般剩余价值的一个特殊部分。(当然,在土地所有权直接剥削劳动的社会制度下,情况不是这样。在那里,要认识剩余财富的源泉,那是毫无困难的。)但是地租是按一定量的土地支付的;地租会资本化为土地价值;这个价值会比例于地租的涨落而涨落;地租则比例于保持不变的土地面积(可是用在土地上的资本的量却会变动)而涨落;土地等级的差别在必须按一定单位面积支付的地租的高度上表现出来;地租总额按总面积计算,这就能确定出比如说每平方尺的平均地租;如同资本主义生产所创造的这一生产的其他一切形式一样,地租也表现为一种固定的、既定的、任何时候都存在的、从而对个人来说是独立存在的前提。租地农场主必须支付地租,并且要按照土地的单位面积,根据土地的质量来支付一定的数额。如果土地质量有了提高或降低,他为若干英亩土地必须支付的地租也要提高或降低,而不管他在土地上使用了多少资本。正如他必须支付利息,而不管他获得了多少利润一样。

按产业资本来计算地租,是政治经济学的又一个批判性的公式,这个公式保持了地租同作为产生地租的基础的利润之间的内在联系。但是在现实中这种联系是不表露在外的;相反,地租在这里是以实际的土地来计量,——因此,一切中介过程都被砍去了,而地租的纯粹外表的独立形态却完成了。地租只有在这种外表化上,在完全脱离它的中介过程的情况下,才是独立的形态。多少平方尺的土地就提供多少地租。在这种说法中,剩余价值的一部分——地租——表现为同某种特殊的自然要素的关系,而与人的劳动无关,在这里,不仅剩余价值的性质完全被掩盖了(因为价值本身的性质被掩盖了),而且利润本身的存在现在也要归功于作为一种特殊的、物的生产工具的资本,正如地租的存在要归功于土地一样。土地作为自然界的一部分而存在,并提供地租。资本由人们生产的产品构成,这些产品提供利润。一种由人们生产的使用价值提供利润,另一种不是由人们生产的使用价值提供地租,——这只是物创造价值的两种不同的形式,前者与后者一样都是既可理解又不可理解。

显然,只要剩余价值分解成各个不同的特殊部分,而这些部分又与各种不同的、只是在物?上不同的生产要素——自然界、劳动产品、劳动——发生关系,只要剩余价值一般获得特殊的、彼此无关、互不依赖、由各种不同的规律调节的形态,那末,剩余价值所有这些形态的共同的统一体(即剩余价值本身),从而这个共同的统一体的性质,也就愈来愈无法辨认,不再通过现象表示自己,而必须当作某种隐藏的秘密来发现了。剩余价值各个特殊部分的形态的这种独立化,它们作为独立形态的相互对立,由于以下的事实而完成了:这些部分中的每一部分都可以归结为作为其尺度和特殊源泉的某种特殊要素,或者说,剩余价值的每一部分都表现为某种特殊原因的结果,某种特殊实体的偶性。这就是:利润—资本,地租—土地,工资—劳动。

就是这些完成了的关系和形式,在实际生产中表现为前提,因为资本主义生产方式是通过它本身所创造的各种形态运动的,这些形态即它的结果,又同样地在再生产过程中作为完成了的前提同它相对立。它们就是以这样的身分实际上决定着单个资本家等等的行动,成为他们的动机,并作为这样的动机反映在他们的意识之中。庸俗政治经济学无非是以学理主义的形式来表达这种在其动机和观念上都囿于资本主义生产方式的外在表现的意识。而庸俗政治经济学愈是肤浅地抓住现象的表面,仅仅用一定的方式把这种现象的表面复制出来,它就愈觉得自己“合乎自然”,而与任何抽象的空想无关。

[913]前面我们谈到流通过程的地方\authornote{见本册第534—535页。——编者注},还应当指出一点,由流通过程产生的一些规定,结晶为一定种类的资本(固定资本、流动资本等等)的属性,这样也就表现为一定商品在物质上所固有的既定的属性。

如果说在利润表现为资本主义生产的既定前提的这种最终形态中,利润所经历的许多转化和中介过程都消失了,无法认识了,从而资本的性质也消失了,无法认识了;如果说这种形态由于以下的事实而更加固定化:使它得以完成的同一过程,会使利润的一部分作为地租同它相对立,从而使它成为剩余价值的一种特殊形式而同作为特殊物质生产工具的资本发生关系,完全象地租同土地发生关系一样;那末,这种由于大量看不见的中间环节而与自己的内在实质相分离的形态,就会获得更加外表化的形式,或者不如说,就会在生息资本上,在利润和利息的划分上,在作为资本的简单形态(这种形态使资本成为它自己的再生产过程的前提)的生息资本上获得绝对外表化的形式。一方面,这里表现出资本的绝对形式:G—G′,自行增殖的价值。另一方面,甚至在纯粹商业资本中也存在的中间环节,即G—W—G′公式中的W在这里消失了。在G—G′公式中,只有G同它自身的关系,这种关系是用它自身来衡量的。这是绝对地从过程中抽出、脱离过程、处于过程之外的资本,它是这样一个过程的前提,对这个过程来说它又是结果,它只有在这一过程之中并通过这一过程,才成为资本。

{我们把利息可能是单纯的财产转移而不一定表示实际的剩余价值这一点撇开不谈。譬如说,当货币贷给“挥霍者”,也就是说当它用于消费时,利息就不表示实际的剩余价值。但是,当货币被借来用于支付时,情况也会是这样。在这两种场合,货币都是作为货币而不是作为资本贷出的,但是对于货币所有者来说,仅仅由于贷放行为,货币才成为资本。在第二种场合,当票据贴现或以当时卖不出去的商品作抵押进行贷款时,借来供支付用的货币,就能够同资本的流通过程,同商品资本向货币资本的必要的转化过程发生关系。只要这种转化过程的加快——这是在信贷中按照信贷的一般性质会发生的情况——能使再生产的速度加快,也就是使剩余价值的生产速度加快,借入的货币就是资本。但是,如果借入的货币仅仅是用来偿还债务,并不加速再生产过程,甚至可能使再生产过程无法进行或者缩小其规模,那末,这笔货币就只是支付手段,对借款人来说只是货币,而对贷款人来说,则是在实际上不依赖资本过程的资本。在这种场合,利息同“让渡利润”一样,是不依赖资本主义生产本身——不依赖剩余价值的创造——的事实。在货币的这两种形式,即作为获得商品以供消费的购买手段和作为偿还债务的支付手段的形式上,利息完全同“让渡利润”一样,表现为这样一种形式:它虽然是在资本主义生产中再生产出来,却不依赖资本主义生产,属于更早的生产方式。但是资本主义生产的性质就包括这样一点:货币(或商品)能够在生产过程之外成为资本并作为资本出卖,这种情形在货币不转化为资本而只起货币作用的更古老的形式上也会发生。

生息资本的第三种更古老的形式以这样的事实为基础:资本主义生产还不存在,而利润还是以利息的形式被占有,资本家纯粹以高利贷者的身分出现。这包括以下两点:(1)生产者还是独立地利用自己的生产资料进行劳动,而不是生产资料利用生产者进行劳动(虽然奴隶也属于这种生产资料,但是奴隶在这里也同役畜一样,并不形成特殊的经济范畴,或者,最多也只是存在物?上的差别:不会说话的工具;有感觉的、会说话的工具);(2)生产资料只是在名义上属于生产者,也就是说,生产者会由于某些偶然情况而不能用出卖自己商品的所得来再生产这些生产资料。因此,生息资本的这些形式在存在商品流通和货币流通的一切社会形式中都会出现,而不管其中占统治地位的是奴隶劳动、农奴劳动,还是自由劳动。在上述形式的最后一种形式中,生产者以利息的形式向资本家支付自己的剩余劳动,因而这种利息也包含着利润。在这里,有了整个[914]资本主义生产,却没有它的优越性,即没有劳动的社会形式的发展和由这些形式中产生的劳动生产力的发展。这种形式在农民中占决定性的优势,他们的一部分生活资料和生产工具已经必须作为商品来购买,也就是说,除他们外已经有独立的城市工业,此外,他们还必须用货币纳税、交付地租等等。}

生息资本只有在借贷货币实际转化为资本并生产一个余额(利息是其中的一部分)时,才成为生息资本。但这一点并不能排除:利息和生息这种属性,不管有没有[生产]过程,都同生息资本长在一起。同样,下面这样一个事实,即为了实际证明棉花的有用属性,必须把棉花纺成纱或进行其他某种加工,也不能排除棉花作为棉花的使用价值。资本也是这样,只有转入生产过程,才能实际证明自己的生息能力。而劳动能力,也只有当它在过程中作为劳动被使用,被实现时,才表明它有创造价值的能力。这一点并不能排除:劳动能力自身作为一种能力,是创造价值的活动,并且作为这样的活动,它不是从过程中才产生的,而相反地是过程的前提。它是作为创造价值的能力被人购买的。购买它的人也可以不让它去从事劳动(例如,剧院经理有时购买一个演员,并不是为了要他演戏,而是为了使他不能再为自己的竞争者的剧院演戏)。购买劳动能力的人是否利用他支付过报酬的劳动能力的属性,即它创造价值的属性,这与卖者或所卖商品无关,正如购买资本的人是否把这些资本作为资本来使用,也就是说,他是否在过程中使这种资本所固有的创造价值的属性发挥作用,这与卖者或所卖商品无关一样。在这两种场合,他为之支付的东西,是那个就自身来说,在可能性上,就所买商品(在一种场合是劳动能力,另一种场合是资本)的性质来说,已经包含在这两种商品中的剩余价值和它们保存它们自身的价值的能力。因此,用自有资本经营的资本家也把剩余价值的一部分看成利息,即看成这样的剩余价值,它之所以从生产过程产生出来,是因为资本把它带进了生产过程而与这个过程无关。

地租和“土地—地租”关系,可以表现为比利息和“资本—利息”关系更加神秘的形式。但是地租形式上的不合理性,也并不在于表示资本本身的关系。因为土地本身是生产的(就使用价值来说),本身是活的生产力(具有使用价值或用来生产使用价值),所以这里可能有两种见解:或者是迷信地把使用价值同交换价值混淆起来,把物同产品中包含的劳动的某种特殊社会形式混淆起来(于是,不合理性就在自身中为自己找到理由,因为这里地租作为某种特殊的东西同资本主义过程本身没有任何关系),或者是“启蒙的”政治经济学的见解:认为既然地租与劳动无关,也与资本无关,那末地租就根本不是剩余价值的一种形式,而只是价格的附加额,是土地占有的垄断使土地所有者可能获得的附加额。

生息资本的情况却不同。这里涉及的不是某种与资本无关的关系,而是资本主义关系本身,是由资本主义生产产生的、它所特有的、反映资本实质本身的关系,是资本借以表现为资本的那种资本形态。利润仍然包含着对处于过程中的资本的关系,对生产剩余价值、生产利润本身的过程的关系。生息资本的情况与利润不同,在利润上,剩余价值的形态成了某种异化的、离奇的东西,使人不能直接认清剩余价值的简单形态,从而不能认清它的实体和产生的原因;相反,在利息上,这种异化形式却明显地作为本质的东西出现、存在和表现。这种形式作为某种同剩余价值的实际性质相对立的东西独立化并固定化了。在生息资本上,资本同劳动的关系消失了。实际上利息是以利润为前提的,利息只是利润的一部分,剩余价值[915]怎样在利息和利润之间、在不同种类的资本家之间进行分配,这实际上与雇佣工人完全无关。

利息明确地表现为离开资本主义过程本身的、独立于过程的、处于过程之外的资本的果实。它应付给作为资本的资本。它进入生产过程,因而也从生产过程中出来。资本孕育着利息。资本不是从生产过程中得出利息,而是把利息带进生产过程。因此,利润中超过利息的余额,即资本只是靠生产过程得到的、只是作为执行职能的资本生产出来的那个剩余价值量,就获得一种不同于利息(即资本自身、资本本身、作为资本的资本所固有的价值创造)的产业利润这样一种特殊形式(即企业利润,至于是产业利润还是商业利润,那要看重点是在生产过程上还是在流通过程上)。这样一来,连剩余价值的最后一种形式,即在一定程度上还能使人想起其起源的形式,也分离为并被理解为不仅是异化的形式,而且是直接同剩余价值本身相对立的形式,因此,资本和剩余价值的性质,也和一般资本主义生产的性质一样,终于被完全神秘化了。

产业利润,与利息相对立,代表着过程中的资本而与过程外的资本相对立,代表着作为过程的资本而与作为所有权的资本相对立,——因而代表着作为执行职能的资本家、作为劳动资本的代表者的资本家而与只是作为资本的人格化、只是作为资本的所有者的资本家相对立。这样,它就作为劳动资本家而与作为资本家的自身相对立;进而作为劳动者而与只是作为所有者的自己相对立。因此,如果说这里还保存着剩余价值同过程的关系,那末,这恰好是以剩余价值概念本身被否定的形式表现出来的。产业利润被归结为劳动,但不是归结为别人的无酬劳动,而是归结为雇佣劳动,即归结为付给资本家的工资,这样一来,资本家就同雇佣工人落入一个范畴,就不过是一种报酬较高的雇佣工人,正如工资一般就存在着各种差别一样。

实际上,货币转化为资本,并不是由于货币同商品的物质生产条件相交换,也不是由于这些生产条件——劳动材料、劳动资料、劳动——在劳动过程中进入发酵状态、相互作用、相互结合,即进入某种化学过程,并把商品作为这个过程的结晶沉淀下来。如果情况仅仅是这样,那末我们就决不会有资本,决不会有剩余价值。劳动过程的这种抽象形式是一切生产方式所共同的,而不管它们的社会形态和历史规定性如何。这种过程成为资本主义过程,货币转化为资本,只是由于:(1)商品生产,即作为商品的产品生产,是生产的普遍形式;(2)商品(货币)同作为商品的劳动能力(即实际上同劳动)相交换,因而劳动是雇佣劳动;(3)但是后者只有在下述情况下才会发生:客观条件,也就是(就整个生产过程来考察)产品本身,作为独立的力量,作为不是劳动的财产,作为他人的财产,因而按形式来说是作为资本,同劳动相对立。

作为雇佣劳动的劳动和作为资本的劳动条件(从而作为资本家的所有权:它们人格化为资本家,在资本家身上,它们表现为它们本身的所有者,它们代表着资本家对它们的所有权,即它们对本身的所有权而与劳动相对立),是同一种关系的表现,不过是从这种关系的不同的两极出发而已。这种资本主义生产的条件,是资本主义生产的经常的结果。这是资本主义生产本身为它自己提供的前提:资本主义生产本身就是它自己的前提,也就是说,一当它发展起来并在与它相适应的关系中发挥作用时,它就连同它的条件一起被作为前提。但是资本主义生产过程不是一般生产过程;它的各个要素的对抗的社会规定性,只有在过程本身中才能发展和实现,这种规定性是该过程的贯彻始终的特征,并使该过程正好成为这种社会规定的生产方式即资本主义生产过程。

[916]当资本——不是某种特定的资本,而是一般资本——刚一开始形成,它的形成过程就是在它之前的社会生产方式的解体过程和这一生产方式瓦解的产物。因而,这是一个历史过程和属于一定的历史时期的过程。这是资本的历史创始时期。(例如,人的存在是有机生命所经历的前一个过程的结果。只是在这个过程的一定阶段上,人才成为人。但是一旦人已经存在,人,作为人类历史的经常前提,也是人类历史的经常的产物和结果,而人只有作为自己本身的产物和结果才成为前提。)在这里,劳动还只是必须同旧形式的劳动条件分离,而在旧形式下,劳动和劳动条件是一个统一的整体。只有这样,劳动才成为自由劳动,只有这样,劳动条件才转化为与劳动相对立的资本。资本成为资本的过程,或者说,资本在资本主义生产过程本身出现之前的发展过程,和资本在这个过程中的实现,在这里是属于历史上两个不同的时期。在后一时期,资本是前提,它的存在是作为一种自行起作用的东西而成为前提。在前一时期,资本是另一个社会形式解体过程的沉淀物。这里资本是另一个形式的产物,而不是象后来那样,它是它自己再生产的产物。资本主义生产是在雇佣劳动这个资本主义的、现存的、但是同时又是被它不断再生产出来的基础上进行的。因而资本主义生产也是在作为劳动条件的形式、作为资本主义生产的既定前提的资本这个基础上进行的,但是这种前提,也象雇佣劳动一样,是资本主义生产的经常的创造,是它的经常的产物。

在这个基础上例如货币自身就是资本,因为生产条件自身具有与劳动相对立的异化形式,表现为他人的所有权而与劳动相对立,并作为这样的所有权对劳动进行统治。这时资本也可以作为具有这种属性的商品出卖,也就是资本可以作为资本出卖,当资本作为有息贷款贷放时就是这样。

但是,因为资本和资本主义生产的独特的社会规定性的因素——这种独特的社会规定性在法律上通过资本表现为一种所有权,通过资本所有权表现为一种独特的所有权形式——已经固定下来,利息又因此表现为资本在这种规定性上(与作为一般生产过程的规定性的这种规定性无关)生出的剩余价值的一部分,所以很明显,剩余价值的另一部分,即利润中超过利息的余额,即产业利润,就必然表现为这样一种价值,这种价值不是由作为资本的资本生出的,而是由同它的、已经以“资本利息”这个名称取得独特存在方式的社会规定性相分离的生产过程生出的。但是,生产过程同资本相分离,就是一般的劳动过程。因此,同作为资本家的本身相区别的产业资本家,同作为资本家即资本所有者的本身相区别的产业家,不过是劳动过程中单纯的职能执行者,不是执行职能的资本,而是与资本无关的职能执行者,即一般劳动过程的特殊承担者,即劳动者。这样,产业利润就顺利地转化为工资,同普通的工资落入同一个范畴,不同于普通工资的只是数量和支付的特殊形式,也就是资本家自己给自己支付工资,而不是由别人给他支付工资。

在利润分为利息和产业利润这最后一次分裂中,剩余价值的性质(从而资本的性质)不仅完全消失了,而且显然表现为一种完全不同的东西。

利息表示剩余价值的一部分,这不过是在特殊名称下从利润中分出的一个份额,这个份额是给资本的单纯所有者的,是由他夺去的。但是这个单纯的量的分割转化成了质的分割,这种质的分割赋予两个部分一种转化形式,在这种转化形式上,它们的原始实质的痕迹已经看不见了。[917]这种情况得以固定下来,首先是因为利息不是表现为同生产无关的、仅仅在产业家用别人的资本从事经营时才“偶然”发生的分割。即使产业家用自有的资本从事经营,他的利润也会分为利息和产业利润,因此,不管产业家是不是他的资本的所有者这种偶然情况,单纯量的分割已经固定化为质的分割,固定化为由资本本身和资本主义生产本身的性质产生的质的分割。这不仅是在不同的人之间进行分配的利润的两个部分,而且还是利润的两种特殊范畴,它们和资本有不同的关系,也就是说,和资本的不同规定性有关。利润的各部分以独立范畴出现的这种独立化,比较容易地固定下来,原因是(撇开以前已阐明的原因不谈)生息资本作为一种历史形式是出现在产业资本之前,并在它的旧形式上继续同产业资本并存,只是在产业资本的发展进程中才被产业资本作为它本身的一种特殊形式置于从属资本主义生产的地位。

这样,从单纯的量的分割中就产生了质的分裂。资本本身被分裂。只要它是资本主义生产的前提,从而,只要它表示劳动条件的异化形式,表示某种特殊的社会关系,它就在利息上得到实现。它在利息上实现它的作为资本的性质。另一方面,只要它在过程中执行职能,这个过程就表现为脱离自己的特殊资本主义性质,脱离自己的特殊社会规定性的过程——表现为单纯的一般劳动过程。因此,只要资本家参加劳动过程,他就不是作为资本家来参加(因为他的这个性质体现在利息中),而是作为一般劳动过程的职能执行者,作为劳动者来参加,他的工资就表现为产业利润。这是一种特殊的劳动方式——管理的劳动,而劳动方式一般来说是彼此各不相同的。

这样,在这两种剩余价值形式上,剩余价值的性质、资本的实质以及资本主义生产的性质,不仅完全消失了,而且转到了自己的反面。但是,由于物的主体化、主体的物化、因果的颠倒、宗教般的概念混淆、资本的单纯形式G—G′在这里被荒诞地、不经过任何中介过程地展示和表现出来,资本的性质和形态也就完成了。同样,各种关系的硬化以及它们表现为人同具有一定社会性质的物的关系,在这里也以完全不同于商品的简单神秘化和货币的已经比较复杂的神秘化的方式表达出来了。变体和拜物教在这里彻底完成了。

利息自身正好表现出,劳动条件作为资本而存在,同劳动处于社会对立中,并且转化为同劳动相对立并且支配着劳动的私人权力。利息概括了劳动条件对主体活动的关系上的异化性质。利息把资本的所有权,或者说单纯的资本所有权,表现为占有别人劳动产品的手段,表现为支配别人劳动的权力。但是,它是把资本的这种性质表现为某种在生产过程本身之外属于资本的东西,而不是表现为这个生产过程本身的独特的规定性的结果。它不是把资本的这种性质表现为同劳动对立,而是相反地同劳动无关,只是表现为一个资本家对另一个资本家的关系,也就是说,表现为一种存在于资本对劳动本身的关系之外的、与这种关系无关的规定性。利润在资本家之间的分配,与工人本身毫无关系。因此,在利息上,在利润的这个形态上,资本的对立性质固然得到了特殊的表现,但是表现成这样:这种对立在其中已经完全消失,而且明显地被抽掉了。利息除了表现货币、商品等等增殖自己价值的能力以外,还把剩余价值表现为从货币和商品中生长出来的某种东西,表现为它们的自然果实,也就是说,利息不过是资本的神秘化在最极端的形式上的表现,——只要它一般表现社会关系本身,它表现的[918]就只是资本家之间的关系,而决不是资本与劳动之间的关系。

另一方面,这个利息形式又使利润的另一部分取得产业利润这种质的形式,即产业资本家——不是作为资本家而是作为劳动者(产业家)——的劳动工资形式。资本家作为资本家在劳动过程中所要完成的、恰好使他同工人相区别的特殊职能,被表现为单纯的劳动职能。他创造剩余价值,不是因为他作为资本家进行劳动,而是因为他,即资本家,也进行劳动。这好比一个国王,他作为国王在名义上指挥着军队,于是有人就说,不是因为他作为王位所有者进行指挥,起着统帅的作用,他才指挥军队,而是因为他指挥军队,执行统帅的职能,所以他才是国王。因此,如果说,剩余价值的一部分在利息的形式上完全同剥削过程相分离,那末另一部分在产业利润的形式上就表现为剥削过程的直接的对立面,即不是对别人劳动的占有,而是自己劳动的价值创造。因此,剩余价值的这一部分也就不再是剩余价值,而是一种和剩余价值相反的东西,是所完成的劳动的等价物。因为资本的异化性质,它同劳动的对立,处于剥削过程之外,处于这种异化的现实行动范围之外,所以一切对立性质也就从这个过程本身排除了。因此,现实的剥削,即实现并实际表现对立性质的东西,就表现为它的直接对立面,表现为一种在物质上是独特的、但是属于劳动的同一社会规定性的劳动,即雇佣劳动,即属于同一劳动范畴的劳动。在这里剥削的劳动和被剥削的劳动被等同起来了。

利润的一部分转化为产业利润,正如我们看到的,是由利润的另一部分转化为利息引起的。与利润的一部分相适应的是资本的社会形式,即资本是所有权;与利润的另一部分相适应的是资本的经济职能,即资本在劳动过程中的职能,不过这种职能已摆脱并抽掉了使资本得以执行这种职能的社会形式,即对立形式。至于有人怎样用聪明的理由进一步为这一点作辩护,我们将在分析把利润解释为“监督劳动”的报酬的辩护论观点时作更详细的考察。在这里人们把资本家和他的经理混同起来了,这一点斯密已经指出过\endnote{马克思指的是斯密《国富论》第一篇第六章。——第550页。}。

当然,产业利润中也包含一点属于工资的东西(在不存在领取这种工资的经理的地方)。资本家在生产过程中是作为劳动的管理者和指挥者(captainofindustry)出现的,在这个意义上说,资本家在劳动过程本身中起着积极作用。但是只要这些职能是产生于资本主义生产的特殊形式,(也就是说,产生于资本对作为它的劳动的劳动的统治,从而对作为它的工具的工人的统治;产生于作为社会的统一体,作为在资本上人格化为支配劳动的权力的社会劳动形式的主体而表现出来的资本的性质),那末,这种与剥削相结合的劳动(这种劳动也可以转给经理)当然就与雇佣工人的劳动一样,是一种加入产品价值的劳动,正如在奴隶制下奴隶监工的劳动,也必须和劳动者本人的劳动一样给予报酬。如果一个人把他对自己的本性、对外部自然界以及对其他人的关系以宗教形式想象成某些独立存在的力量,以致被这些想象所统治,那末,他就需要祭司和祭司的劳动。但是随着意识的宗教形式以及与此相联系的关系的消失,这种祭司的劳动也就不再进入社会生产过程。祭司的劳动与祭司一起消失了,而资本家作为资本家所完成的或他委托别人完成的劳动,也会与资本家一起同样消失。(奴隶制的例子用几段引文加以说明。\endnote{马克思在两三年后写的《资本论》第三卷第二十三章里列举了关于奴隶监工的引文。——第551页。})

可是,把利润归结为作为监督劳动的报酬的工资这一辩护论观点,本身又转过来反对辩护士;因为英国[919]社会主义者曾以充分的理由回答说:很好,以后你们就只应拿普通经理的工资;你们的产业利润不仅在口头上而且在实际上都应归结为监督或管理劳动的工资。

{自然,不可能详细地研究辩护士的这些愚蠢的废话及其种种矛盾。例如,产业利润的提高和下降不论同利息还是同地租都成反比。但是对劳动的监督,即资本家实际完成的一定量劳动,却与此无关,就象与工资的下降无关一样。这种工资的特点正是:它的下降和提高同实际的工资成反比(在利润率由剩余价值率决定的情况下,如果全部生产条件保持不变,它就完全由剩余价值率决定)。但是诸如此类的“小矛盾”并没有消除持辩护论观点的庸俗经济学家头脑中的等同性。不管资本家付出的工资是少还是多,不管工人得到的工资是较高还是较低,资本家完成的劳动都绝对地保持不变。(正如按一个工作日支付的工资丝毫不改变劳动本身的量一样。)不仅如此。工人的工资较高时,他的劳动强度就较大。相反,资本家的劳动则是个完全确定的东西:它在质上和量上都是由资本家应管理的劳动的量决定,而不是由这一劳动量的报酬决定。资本家不会强化自己的劳动,正如工人不会加工出多于他在工厂中得到的棉花一样。}

英国社会主义者接着还说:管理劳动和监督劳动也同其他任何劳动能力一样,现在可以在市场上购买,并且可以同样比较便宜地生产出来,因而可以同样比较便宜地买到。资本主义生产本身已经使那种完全同资本所有权(不管是自有的资本还是别人的资本)分离的管理劳动比比皆是。因此,这种管理劳动就完全无需资本家亲自担任了。这种劳动实际上是同资本分离而存在的,但这不是表现在产业资本家同货币资本家那种表面上的分离上,而是表现在产业管理人员等等同各种资本家的分离上。最好的证明就是:第一,工人们自己创办的合作工厂\authornote{参看本册第392页。——编者注}。它们提供了一个实例,证明资本家作为生产上的职能执行者对工人来说已经成为多余的了,就象在资本家本人看来,土地所有者的职能对资产阶级的生产是多余的一样。第二,只要资本家的劳动不是由作为资本主义过程的那种[生产]过程引起,因而这种劳动并不随着资本的消失而自行消失;只要这种劳动不是剥削别人劳动的职能的名称,也就是说,只要这种劳动是由劳动的社会形式(协作、分工等等)引起,它就同资本完全无关,就象这个形式本身一旦把资本主义的外壳剥去,就同资本完全无关一样。说这种劳动作为资本主义的劳动,作为资本家的职能是必要的,这无非就是说,庸俗经济学家不能设想在资本内部发展起来的劳动的社会生产力和劳动的社会性质,能够脱离它们的这种资本主义形式,脱离它们的各因素的异化、对立和矛盾的形式,脱离它们的颠倒和混乱。而这正是我们所要坚持的。[XV—919]

\centerbox{※     ※     ※}

[XVIII—1142]{资本家的实际利润,有很大一部分是“让渡利润”,而且资本家的“个人劳动”在不是涉及剩余价值的创造,而是涉及整个资本家阶级的总利润通过商业途径在其各个成员之间进行分配的场合,有着特别广阔的活动余地。这一点在这里与我们无关。某些种类的利润——例如,以投机为基础的利润——只有在这种场所才能获得。因此,这里就不去考察这些利润了。庸俗政治经济学(特别是为了把利润说成是“工资”)把这种“让渡利润”同来源于剩余价值的创造的利润混为一谈,这表明庸俗政治经济学象畜生一样愚蠢。例如,请看看可敬的罗雪尔。因此,对于这类蠢驴来说,他们把分配整个资本家阶级的总利润时不同生产领域的资本家在计算上的考虑和补偿的理由,同资本家剥削工人的理由,同所谓的利润本身的来由混为一谈,这也是十分自然的。}[XVIII—1142]

\tchapternonum{[(5)古典政治经济学和庸俗政治经济学的本质区别。利息和地租是商品市场价格的构成要素。庸俗经济学家企图赋予利息和地租的不合理形式以合理的外观]}

[XV—919]在生息资本上,——由于利润分为利息和[产业]利润,——资本取得了它的最彻底的物的形式,它的纯粹的拜物教形式,剩余价值的性质表现为一种完全丧失了它自身的东西。正象土地表现为地租的源泉,劳动表现为工资(部分是真正的工资,部分是产业利润)的源泉一样,资本——作为物——在这里表现为价值的独立的源泉,表现为价值的创造者。诚然,这种观点的代表者始终认为,商品的价格应当支付工资、利息和地租,但它支付它们是因为加入商品的土地创造地租,加入商品的资本创造利息,加入商品的劳动创造工资;是因为它们创造了落入它们各自的所有者或代表[920]——土地所有者、资本家和劳动者(雇佣工人和产业家)——手里的这几部分价值。因此,从这个观点来看,说一方面商品的价格决定工资、地租和利息,另一方面利息、地租和工资的价格决定商品的价格,在理论上也没有什么矛盾,或者说,如果有矛盾,那也是价格的实际运动的矛盾或循环论证。

不错,利率会波动,但它只是和其他任何商品市场价格的波动一样,取决于供求关系。这不会使利息不再成为资本内在的东西,就象商品价格的波动不会使价格不再成为商品固有的规定一样。

因此,一方面,只要土地、资本和劳动被看作地租、利息和工资的源泉,而地租、利息和工资被看作商品价格的构成要素,土地、资本和劳动就表现为创造价值的要素;另一方面,只要它们归于每一种生产价值的工具的所有者,并把它们创造的那部分产品价值归于他,它们就表现为收入的源泉,而地租、利息和工资的形式则表现为分配形式。(庸俗经济学家把分配形式实际上只当作从另一角度看的生产形式,而批判的经济学家却把它们彼此分开,并且否认它们的同一性,这一点表明,正如我们以后将看到的,和批判的政治经济学比较起来,庸俗经济学家真是愚蠢透顶。)

在生息资本上,资本表现为它作为货币或商品所具有的价值或剩余价值的独立源泉。而且它是在本身,在自己的物的形式上成为这样的源泉的。诚然,资本为了实现它的这种属性必须加入生产过程,但是土地或劳动也必须这样做。

因此,很明显,为什么庸俗政治经济学宁愿采取“土地—地租,资本—利息,劳动—工资”这样的公式,而不愿采取斯密等人用来说明价格要素(更确切地说,价格分解成的各部分)的公式,在这一公式里出现的是“资本—利润”的关系,所有的古典经济学家一般都用这种关系来说明资本关系本身。在利润中还包含着同[生产]过程的[使庸俗政治经济学]感到为难的联系,剩余价值和资本主义生产的真正性质(和它们的外部表现不同)还多少可以辨认。当利息被说成是资本的真正产物,从而剩余价值的另一部分即产业利润完全消失并归入工资范畴时,情况就不再是如此了。

古典政治经济学力求通过分析,把各种固定的和彼此异化的财富形式还原为它们的内在的统一性,并从它们身上剥去那种使它们漠不相关地相互并存的形式;它想了解与表现形式的多样性不同的内在联系。因此,它把地租还原为超额利润,这样,地租就不再作为特殊的,独立的形式而存在,就和它的虚假的源泉即土地分离开来。它同样剥去了利息的独立形式,证明它是利润的一部分。于是,它把非劳动者借以从商品价值中获取份额的一切收入形式,一切独立的形式或名义都还原为利润这一种形式。但是利润归结为剩余价值,因为全部商品的价值都归结为劳动;商品中包含的有酬劳动量归结为工资;因此,超过这一数量的余额归结为无酬劳动,归结为在各种名义下被无偿地占有的、然而是由资本引起的剩余劳动。在进行这种分析的时候,古典政治经济学有时也陷入矛盾;它往往试图不揭示中介环节就直接进行这种还原和证明不同形式的源泉的同一性。但这是它的分析方法的必然结果,[921]批判和理解必须从这一方法开始。它感兴趣的不是从起源来说明各种不同的形式,而是通过分析来把它们还原为它们的统一性,因为它是从把它们作为已知的前提出发的。但是,分析是说明起源,理解实际形成过程的不同阶段的必要前提。最后,古典政治经济学的缺点和错误是:它把资本的基本形式,即以占有别人劳动为目的的生产,不是解释为社会生产的历史形式,而是解释为社会生产的自然形式,不过它自己已通过它的分析开辟了一条消除这种解释的道路。

庸俗政治经济学的情况就完全不同了,正当政治经济学本身由于它的分析而使它自己的前提瓦解、动摇的时候,正当政治经济学的对立面也已经因此而多少以经济的、空想的、批判的和革命的形式存在的时候,庸俗政治经济学开始嚣张起来。因为政治经济学和由它自身产生的对立面的发展,是同资本主义生产固有的社会矛盾以及阶级斗争的现实发展齐头并进的。只是在政治经济学达到一定的发展程度(即在亚·斯密以后)和形成稳固的形式时,政治经济学中的一个因素,即作为现象观念的单纯的现象复写,即它的庸俗因素,才作为政治经济学的特殊表现形式从中分离出来。例如萨伊就把亚·斯密著作中这里或那里渗透的庸俗观念分离出来,并作为特殊的结晶和亚·斯密并存。随着李嘉图的出现和由他引起的政治经济学的进一步发展,庸俗经济学家也得到了新的营养(因为他自己什么也不生产),政治经济学越是接近它的完成,也就是说它越是走向深入和发展成为对立的体系,它自身的庸俗因素,由于用它按照自己的方法准备的材料把自己充实起来,就越是独立地和它相对立,直到最后在学术上的混合主义和无原则的折衷主义的编纂中找到了自己至上的表现。

随着政治经济学的深入发展,它不仅自己表现出矛盾和对立,而且它自身的对立面,也随着社会经济生活中的现实矛盾的发展而出现在它的面前。与这种情况相适应,庸俗政治经济学也就有意识地越来越成为辩护论的经济学,并且千方百计力图通过空谈来摆脱反映矛盾的思想。因此,萨伊同例如巴师夏比较起来还算是一个批评家,还算无所偏袒,因为他在斯密的著作里发现的矛盾相对说来还是未发展的,而巴师夏却是一个职业的调和论者和辩护论者,虽然他不仅在李嘉图的政治经济学中发现了经济学本身在内部已经形成的矛盾,而且发现了在社会主义和当时日常的阶级斗争中正在形成的矛盾。再加上,庸俗政治经济学在其较早的发展阶段,找到的材料还没有完全加工好,因此它本身在参与解决经济问题的时候还或多或少地从政治经济学的观点出发,例如萨伊就是这样,而那位巴师夏却只有剽窃,并且力图用自己的论据把古典政治经济学中不合口味的方面消除掉。

但巴师夏还不代表最后的阶段。他还有一个特点,这就是学识贫乏,对于他为了统治阶级的利益而加以粉饰的那门科学的认识十分肤浅。他搞辩护论还是很热情的,这是他的真正的工作,因为政治经济学的内容,只要是合他心意的,他可以从别人那里取来。最后的形式是教授形式,这种形式是“从历史的角度”进行工作的,并且以明智的中庸态度到处搜集“最好的东西”,如果得到的结果是矛盾,这对它说来并不重要,只有完备才是重要的。这就是阉割[922]一切体系,抹去它们的一切棱角,使它们在一本摘录集里和平相处。在这里,辩护论的热忱被渊博的学问所抑制,这种渊博的学问宽厚地俯视着经济思想家的夸张的议论,而只是让这些议论作为稀罕的奇物漂浮在它的内容贫乏的稀粥里。因为这类著作只有在政治经济学作为科学已走完了它的道路的时候才会出现,所以它们同时也就是这门科学的坟墓。(至于它们完全以同样的方式超然耸立于社会主义者的空想之上,那就不用说了。)甚至斯密、李嘉图和其他人的真正的思想(不仅是他们本身的庸俗因素)在这里也好象是毫无内容,变成了庸俗的东西。罗雪尔教授先生就是这样的大师,他谦虚地宣称自己是政治经济学的修昔的底斯。\endnote{罗雪尔在他的著作《国民经济学原理》(1854年)第一版序言中,不知羞耻地引证了修昔的底斯。——第558页。}他把自己比作修昔的底斯,可能是因为他对修昔的底斯有这样一种看法,即修昔的底斯似乎经常把原因和结果相混淆。

诚然,资本不花费任何劳动就占有别人的劳动成果这一事实,非常明显地表现在生息资本的形式上:因为在这里资本以它借以与生产过程本身脱离的形式表现出来。但是在这个形式上,资本所以能够这样,只是因为它本身实际上并不花费任何劳动,而是作为自行创造价值的、成为价值源泉的要素加入劳动过程。如果说生息资本不花费任何劳动便占有一部分产品价值,那末它不花费任何劳动也创造了这部分价值,由自身、由自身内部创造了这部分价值。

异化形式使古典的,因而也使批判的政治经济学家感到困难,他们试图通过分析来剥去这种形式,可是庸俗政治经济学却正好是在产品价值的各个不同部分相互对立的异化中第一次感到十分自在:正如一个经院哲学家在谈到“圣父、圣子和圣灵”这一公式时感到十分自在一样,庸俗经济学家在谈到“土地—地租,资本—利息,劳动—工资”这一公式时也感到十分自在。因为这正是这样一种形式,在这种形式中,这些关系在现象上似乎直接相互联系着,因而也在受这种生产方式束缚的资本主义生产当事人的观念和意识中存在着。庸俗政治经济学认为它越是实际上仅仅从事于把普通观念译成学理主义的语言,它就越是单纯、合乎自然和对公众有益,就和一切理论上的吹毛求疵离得越远。因此,它越是在异化的形式上来认识资本主义生产的各种形态,它就越是接近于普通观念的要素,也就是越在它自己的自然要素中浮游。

此外,这给辩护论帮了很大的忙。因为,例如在“土地—地租,资本—利息,劳动—工资”这一公式中,剩余价值的各种不同形式和资本主义生产的各种不同形态,不是作为异化形式相互对立,而是作为相异的和彼此无关的形式、作为只是彼此不同但无对抗性的形式相互对立。不同的收入来自完全不同的源泉,一个来自土地,另一个来自资本,第三个来自劳动。因此,它们不是处于相互敌对的关系,因为它们根本没有任何内在联系。如果说它们还是在生产上共同起作用,那末,这是一种协调的动作,是协调的表现;这好比农民、牛、犁和土地,尽管它们彼此不同,但它们却在农业中,在实际的劳动过程中协调地共同劳动。如果它们之间发生了对抗,那末,这种对抗只是由于生产当事人中谁应当从产品,从它们共同创造的价值中多占一些而引起的竞争造成的。如果有时会发展到冲突,那末,土地、资本和劳动之间这一竞争的最后结果终归还是这样:在它们[923]对分割的争执过程中,它们由于竞争而大大增加了产品的价值,以致每一个都获得了更大的一份,所以它们的竞争本身只是刺激所有生产当事人的协调的表现。

例如阿伦德先生批评劳说:

\begin{quote}{“作者受他的某些前辈的影响,把企业主的收入作为第四种要素和国民财富的三种要素(工资、资本的租金和地租)并列;这样,由亚·斯密如此谨慎地建立起来的、我们的科学〈!〉的任何进一步发展的整个基础被破坏了,因此,在我们的作者的著作里根本没有考虑这种发展。”(卡尔·阿伦德《与垄断精神及共产主义相对立的合乎自然的国民经济学,附与本书有关的资料的评述》1845年哈瑙版第477页)}\end{quote}

阿伦德先生把“资本的租金”理解为利息(同上,第123页)。如果有人不相信亚·斯密把国民财富归结为资本利息、地租和工资呢?(因为斯密正好相反,明确指出利润是资本的价值增殖,并且不止一次地明白指出,利息由于一般说来代表剩余价值,始终只是从利润中派生的形式。)在这种情况下,庸俗经济学家读到斯密所提到的源泉时就读出了直接与其含义相对立的东西。斯密写“利润”的地方,阿伦德读成“利息”。那末,他把亚·斯密的“利息”理解为什么呢?

正是这一位“我们的科学”的“谨慎的”发展者作出了以下有趣的发现:

\begin{quote}{“在财物生产的自然进程中,只有一个现象,在已经充分开发的国家,看来在一定程度内负有调节利息率的使命;那就是欧洲森林的树木总量由于树木的逐年增长而增加的比率。这种增长完全不以树木的交换价值为转移〈说树木的增长“不以树木的交换价值为转移”,这是多么滑稽啊!〉,而按每一百棵增加三棵到四棵的比率来进行。因此〈也就是因为,树木的交换价值虽然在很大程度上要取决于树木的增长,但树木的增长“完全不以树木的交换价值为转移”!〉,不能指望它〈利息率〉会下降到最富有货币的国家的现有水平以下。”(同上,第124—125页)}\end{quote}

这种利息率应当称为“原始的森林利息率”。这种利息率的发现者在所引著作中,又作为“犬税”\endnote{阿伦德在自己的著作中用专门的一节(第88节第420—421页)论证了犬税的正确性和合理性。——第561页。}哲学家在“我们的科学”领域里引人注目。

\centerbox{※     ※     ※}

{利润(其中也包括产业利润)和预付资本的量成比例;相反,产业资本家取得的“工资”和资本的量成反比:资本小的时候,它就大(因为在这里资本家是介于别人劳动的剥削者和靠自己劳动生活的劳动者之间的中间人物),资本大的时候,它就很微小,或者象在有经理的情况下,它就完全和利润分离。一部分管理劳动只是由资本和劳动之间的敌对性、由资本主义生产的对抗性引起的,它完全和流通过程引起的9/10的“劳动”一样,属于资本主义生产上的非生产费用\authornote{不直接参加生产过程,但在一定条件下又非有不可的辅助费用。——编者注}。一个乐队指挥完全不必就是乐队的乐器的所有者,用乐队队员的生活费用搞投机,也不是他这个乐队指挥职能范围以内的事情,他和他们的“工资”根本没有任何关系。非常奇怪,象约翰·斯图亚特·穆勒这样一些为了把“产业利润”变为监督劳动的工资而坚持“利息”、“产业利润”等形式的经济学家,却和斯密、李嘉图以及一切值得一提的经济学家一起,认为平均利率即平均利息率是由平均利润率决定的,照穆勒的说法,这种平均利润率和工资率成反比,因此它无非是无酬劳动,剩余劳动。

监督工资根本不加入平均利润率,以下两个事实是最好的证明:

[924](1)合作工厂\authornote{见本册第392和552页。——编者注}和其他一切工厂一样,那里的经理是有报酬的,并完成全部管理劳动,那里的监工本身只是劳动者,在这样的工厂里,利润率不是低于而是高于平均利润率;

(2)在某些特殊的、非垄断的行业,例如在小店主、农场主等等那里,利润经常大大高于平均利润率,对于这种情况,经济学家们公正地解释说,这是由于这些人自己给自己支付工资。如果这样的人独自一人劳动,他的利润就由(1)他的小额资本的利息、(2)他的工资、(3)由于他的资本而使他能够为自己而不是为别人劳动的那部分剩余时间,即已经不表现为利息的那部分剩余时间所构成。如果他雇用工人,那末其中便包括工人的剩余时间。

可尊敬的西尼耳(纳骚)自然也把产业利润变成监督工资。但是一当问题不涉及学理主义的语句而涉及工人和厂主之间的实际斗争时,他便忘记了这些诡辩。这时他就,例如,反对限制劳动时间,因为,照他的说法,例如工人每天在11+(1/2)小时内只为资本家劳动一小时,只有这一小时的产品构成资本家的利润(利息除外,照他的计算,工人还要为补偿利息劳动一小时)。因此,在这里产业利润突然变成不等于资本家的劳动在生产过程中加进商品的价值,而等于工人的无酬劳动时间加进商品的价值。如果产业利润是资本家自己劳动的产物,西尼耳就必然不会抱怨工人只白白地劳动一小时而不是两小时,而且更不会说,如果工人只劳动10+(1/2)小时而不是11+(1/2)小时,就完全不会有利润;他必然会说,如果工人只劳动10+(1/2)小时,而不是11+(1/2)小时,资本家得到的就只是10+(1/2)小时的监督工资,而不是11+(1/2)小时的监督工资,也就是说他丧失了一小时的监督工资,对于这一点工人会回答他说,如果对他们来说,10+(1/2)小时的普通工资就已经够了,那末对资本家来说,10+(1/2)小时的较高工资也应该够了。

很难理解,约翰·斯图亚特·穆勒这样一些属于李嘉图学派的经济学家,他们甚至把利润仅仅等于剩余价值即剩余劳动这一论点表述为:利润率和工资成反比,工资率决定利润率(这样说是不对的),可是,他们怎么竟突然把产业利润不是变成工人的剩余劳动,而是变成资本家自己的劳动,——只有他们把剥削别人劳动的职能称为劳动,那才的确会出现这样的结果:这种劳动的工资恰好等于被占有的别人劳动的量,或者说,这种劳动的工资直接取决于剥削程度,而不是取决于资本家为这种剥削所作出的那种努力的程度。(在资本主义生产中,这种剥削劳动的职能要求实际的劳动,就这方面说,这种职能表现为经理的工资。)我再说一遍,很难理解,这些经济学家,在他们(作为李嘉图学派)把利润归结为它的实际要素之后,怎么又由于把利息和产业利润对立起来而陷入谬误,产业利润只是利润的伪装形式,把产业利润理解为一种独立形式是由于对利润的实质无知。利润的一部分所以表现为产业利润,表现为从过程中的活动(从真正的活动过程,但其中同时也包括执行职能的资本家的活动)产生的,因而表现为资本家的劳动所应得的部分,只是因为另一部分即利息表现为资本作为与过程无关的、自动的、自行创造的物所应得的部分。也就是说,是因为资本和由其产生的剩余价值在利息的名称下被说成是某种神秘的东西。这种纯粹来自表象的、反映资本表面的最外表的形态的见解是和李嘉图的见解直接对立的,并且完全和他对价值的理解相矛盾。就资本是价值来说,资本的价值决定于早在这个资本加入过程以前就包含的劳动。就资本作为物加入过程来说,它是作为使用价值加入过程的,而作为使用价值,不管它的效用如何,它绝不能创造交换价值。由此可以看出,李嘉图学派对他们自己的老师的了解有多妙。同货币资本家相对来说,产业家是执行职能的资本家,因而是实际榨取剩余劳动的,他把这种剩余劳动的一部分装进自己的腰包,当然是完全正确的。同货币资本家相对来说,他是劳动者,不过是作为资本家的劳动者,即作为别人劳动的剥削者的劳动者。[925]同工人相对来说,这样一个论据,即认为剥削工人的劳动要花费资本家的劳动,因此工人还必须为这种剥削付给他工资,就是可笑的。这是奴隶监工用来对付奴隶的论据。}

\centerbox{※     ※     ※}

社会生产过程的任何前提同时也是它的结果,而它的任何结果同时又表现为前提。因此,生产过程借以运动的一切生产关系既是它的条件,同样也是它的产物。我们越是在这一过程的实际外部表现上来考察这一过程,它的形态就越是在条件的形式上固定下来,以致这些条件似乎是不取决于它但对它起决定作用的东西,而过程参加者们本身的关系对他们来说表现为物的条件、物的力量、物的规定性,尤其是在资本主义过程中,任何要素,甚至最简单的要素,例如商品,都已经是一种颠倒,并已使人与人之间的关系表现为物的属性,表现为人与这些物的社会属性的关系。

\begin{quote}{{“利息是对生产地使用积蓄的报酬;真正意义上的利润是对这种生产地使用期间进行的监督活动的报酬。”(《韦斯明斯特评论》\endnote{《韦斯明斯特评论》(《TheWestminsterReview》)——英国资产阶级自由派杂志,1824年至1914年在伦敦出版,每年出四期。——第564页。}1826年1月第107—108页)}\end{quote}

可见,在这里,利息是对货币等等作为资本使用的报酬;所以它来自资本本身,资本由于自己的资本属性而得到报酬。而产业利润是对“这种生产地使用期间”即生产过程本身中的资本或资本家的职能的报酬。}[925]

[925]利息只是产业的、执行职能的资本家付给资本所有者的一部分利润。因为前者只是由于有资本(货币、商品)等等才能占有剩余劳动,所以他支付一部分给向他提供这种手段的人。如果资本的所有者希望享受他的货币作为资本的利益而又不让他的货币执行资本的职能,那末他只有在满足于一部分利润的条件下才能这样做。他们实际上是伙伴:一个是法律上的资本所有者,另一个,当他使用资本的时候,是经济上的资本所有者。但是,因为利润只是来自生产过程,只是生产过程的结果,还有待生产出来,所以利息实际上不过是对于待完成的剩余劳动的一部分的要求权,对未来劳动的要求权,对还不存在的商品价值的一部分的要求权,因此,不过是在一段时间内(到这段时间终了,利息才能得到支付)所进行的生产过程的结果。

[926]资本在它被支付以前先被购买(即凭利息借入)。货币在这里象在购买劳动能力等等的情况下一样,执行支付手段的职能。因此,资本的价格(利息)加入产业家的预付(如果他用自己的资本经营,就是加入自己本身的预付),就象棉花的价格加入产业家的预付一样,棉花例如也是今天买进,要过比如说六个星期才得到支付。利率(货币的市场价格)的波动也和其他商品的市场价格的波动一样,在这里不会使事情发生变化。相反,货币的市场价格(这是作为货币资本的生息资本的名称)在货币市场上正象其他一切商品的市场价格一样,是由买者和卖者之间的竞争,是由需求和供给决定的。货币资本家和产业资本家之间的这种斗争只是分割利润的斗争,即在分割时双方为各自应得的份额而进行的斗争。关系本身(需求和供给)和它的两极中的任何一极一样,也是生产过程的结果,或者用普通的话来说,是由当时的营业状况,即再生产过程及其要素在当时所处的状况[决定的]。但是从形式上和从外部表现来看,早在资本加入再生产以前,这一斗争就已决定资本的价格(利息)。同时这种决定是在真正的生产过程以外进行的,由与这一过程无关的情况所调节,而且价格的这种决定表现为生产过程必须借以进行的条件之一。因此,这一斗争看来不仅确定对未来利润的一定部分的所有权,而且使这一部分本身不是作为结果从生产过程中产生出来,而是作为前提,作为资本的价格加入生产过程,完全和商品价格或工资作为前提加入生产过程一样,虽然它实际上不断——在再生产的过程中——从生产过程中产生出来。商品价格中作为预付出现并作为已经存在的商品价格加入生产价格的一切要素,在产业资本家看来已不再是剩余价值。因此,作为资本价格加入过程的那一部分利润列入预付的费用,不再表现为剩余价值,并从过程的产物变成它的既定的前提之一,变成生产条件,这种条件本身以独立的形式加入过程,并决定过程的结果。

(例如,如果利率下降,而市场状况要求把商品的价格降到它们的费用价格以下,那末,产业家就能够在不降低产业利润率的情况下降低商品价格;他甚至能够降低自己的商品价格并获得较高的产业利润,当然,在靠自有资本经营的人看来,这是利润率的下降,是总利润的下降。一切表现为既定的生产条件的东西,即商品、工资、资本的价格,也就是这些要素的市场价格,又会反过来对当时的商品市场价格产生决定性的影响,而单个商品的实际费用价格只是在市场价格的波动中为自己开辟道路,它只是这些市场价格的自行平均化,完全和商品的价值只是在所有各种不同商品的费用价格的平均化中为自己开辟道路一样。因此,庸俗观点的代表者——无论他是资本主义思想的理论家还是实践的资本家——的循环论证:商品价格决定工资、利息、利润和地租,反过来,劳动、利息、利润和地租等的价格又决定商品的价格,——只是一种循环运动的表现,在实际运动中和在现象的表面上普遍规律就是通过这种循环运动以矛盾的方式实现的。)

于是,剩余价值的一部分,利息,就表现为加入过程的资本的市场价格,因此它不是表现为剩余价值,而是表现为生产条件。因此,剩余价值在两类资本家(处在过程外的和处在过程内的)之间进行分割这种情况,表现为剩余价值的一部分应付给处在过程外的资本家,而另一部分则应付给处在过程内的资本家。分割的预先确定表现为一部分不依赖于另一部分;一部分不依赖于过程本身;最后,表现为某种物、货币、商品(不过这些物是作为资本)的内在属性,这又似乎不是某种关系的表现,而是这些货币、这种商品在工艺上是为劳动过程规定的;由于这种规定,它们就成为资本;有了这种规定,它们就是劳动过程本身的简单要素,[927]这些要素本身也就是资本。

商品的价值,部分分解为该商品所包含的各种商品的价值,部分分解为劳动的价值,即有酬劳动,部分分解为无酬的、然而是可出卖的劳动;商品中由无酬劳动构成的那一部分价值,即商品中包含的剩余价值,又分解为利息、产业利润和地租,就是说,这一总剩余价值的直接占有者和“生产者”不得不把总剩余价值中的一部分交给土地所有者,另一部分交给资本所有者,结果他给自己留下的总剩余价值中的第三部分,就在产业利润这个不同于利息和地租、也不同于剩余价值本身和利润本身的名称下留给了自己。以上这种情况是没有什么神秘的。剩余价值,即商品价值的一定部分,分解为这些特殊项目或类别是完全可以理解的,根本不会和价值本身的规律发生矛盾。但是,由于剩余价值的这些不同部分取得了独立的形式,由于它们归属于不同的人,由于对它们的要求权所依据的要素不同,最后,由于这些不同部分作为过程的条件借以和过程相对立的那种独立性,上述一切都被神秘化了。它们从价值可以被分解成的那些部分,变为构成价值的独立要素,变为构成要素。它们对市场价格说来就是这样。它们实际上成了市场价格的构成要素。它们作为过程条件的这种表面的独立性又怎样由内在的规律所调节,因而它们只是一种表面上独立的东西,——这一点在生产过程的任何时刻都不会明显地表现出来,也不会作为决定性的、有意识的动机起作用。正好相反。过程的结果借以表现为过程的独立条件的这种外观,当剩余价值的各部分(作为生产条件的价格)加入商品价格的时候,就获得了最大程度的固定性。

利息和地租的情况就是这样。它们属于工业资本家和租地农场主的预付。它们在这里似乎已经不再是无酬剩余劳动的表现,而是有酬剩余劳动即在生产过程中为其支付了等价物的那种剩余劳动的表现,诚然这种等价物不是支付给工人(这种剩余劳动就是工人的剩余劳动),而是支付给其他人——资本所有者和土地所有者。利息和地租就它们对工人的关系来说是剩余劳动,但是就它们对它们应被付给的[货币]资本家和土地所有者的关系来说是等价物。因此它们不是表现为剩余价值,更不是表现为剩余劳动,而是表现为“资本”这种商品和“土地”这种商品的价格,因为它们被付给只是作为商品所有者、只是作为这些商品的所有者和卖者的[货币]资本家和土地所有者。因此,商品价值中归结为利息的部分表现为为资本支付的价格的再生产,而归结为地租的部分则表现为为土地支付的价格的再生产。因此这些价格成了商品总价格的构成部分。这在产业资本家看来就不仅仅是如此;对他来说利息和地租确实构成他的预付的一部分,如果说一方面它们决定于他的商品的市场价格(通过这种市场价格,社会过程或它的结果表现为商品所固有的规定性,而这一过程的波动,它的运动,则表现为商品价格所固有的波动),那末另一方面市场价格则决定于它们,正象棉花的市场价格决定棉纱的市场价格,而棉纱的市场价格又决定对棉花的需求,从而决定棉花的市场价格一样。

由于剩余价值的两个部分,即利息和地租,作为商品(商品“土地”和商品“资本”)的价格加入生产过程,它们借以存在的形式就不仅掩盖了它们的实际来源,而且简直否定了这一来源。

剩余劳动,即无酬劳动,也和有酬劳动一样实?上加入资本主义生产过程这一情况,在这里表现为:与劳动不同的生产要素(土地和资本)必须得到报酬,或者说,与预付商品的价格和工资不同的费用加入商品的价格。剩余价值的两个部分(利息和地租)在这里表现为从事经营的资本家的费用即预付。

平均利润作为决定的因素加入商品的生产价格,因此,在这里剩余价值已经不是表现为结果,而是表现为条件;不是表现为商品价值分解成的那些部分中的一个部分,而是表现为商品价格的构成部分。但是平均利润也和生产价格本身一样不如说是观念上起决定作用的东西,它同时表现为超过预付的余额,[928]表现为不同于真正的生产费用的价格。在现存的市场价格情况下,即在过程的直接结果中,是否得到平均利润,得到的利润是大于还是小于平均利润,——这一点决定着再生产,或者更确切地说,决定着再生产的规模;决定着现有资本以怎样的量抽出或投入这一或那一生产领域,也决定着新积累的资本以怎样的比例流入这些不同领域,最后,决定着这些不同领域在什么程度上作为买者出现在货币市场上。相反,剩余价值中作为利息和地租的这些部分则以完全固定的形式,分别表现为单个生产价格的前提,并且是以预付形式预支的。

\centerbox{※     ※     ※}

{可以把预付,即资本家支付的东西叫作费用[Kosten]。按照这种说法,利润就表现为超过这些费用的余额。这与个别生产价格有关。而由预付决定的价格就可以叫作费用价格[Kosten-preise]\endnote{见注6和注18。这里马克思是在c+v的意义上使用“费用价格”这一术语的。——第570页。}。

由平均利润决定的价格,也就是由预付资本的价格加平均利润决定的价格,可以叫作生产费用[Produktionskosten],因为这一利润是再生产的条件,是在不同领域之间调节商品供给和资本分配的条件。这种价格是生产价格[Produktionspreise]。

最后,生产商品所花费的劳动(物化劳动和直接劳动)的实际量就是商品的价值。这一价值构成商品本身的实在的生产费用。与这一价值相适应的价格,只是以货币表现的价值。

“生产费用”这一术语交替地用在所有这三种意思上。}

\centerbox{※     ※     ※}

如果没有剩余价值再生产出来,那末,其中叫作利息的部分和叫作地租的部分自然也就会同剩余价值一起消失,这种剩余价值的预支,或者说,这种剩余价值作为商品价格加入生产费用这一事实,也会随着消失。加入生产的现有价值,那时就根本不会作为资本从生产中产生出来,因而也不可能作为资本加入再生产过程,或作为资本贷出。因此,正是同样一些关系——决定资本主义生产的关系——的不断再生产,使它们不仅表现为这一过程的社会形式和结果,而且同时表现为它的经常的前提。但是只有作为这一过程本身不断确定、创造、生产的前提,它们才是这样的前提。因此,这种再生产决不是有意识的,相反,它只是在作为前提和支配生产过程的条件的这些关系的经常存在中表现出来。例如,商品价值可能分解成的那些部分变成商品价值的构成部分,这些构成部分作为彼此独立的部分相对立,因而也作为独立的部分与它们的统一体相对立,而这个统一体反过来又表现为它们的结合。资产者看到产品经常成为生产的条件。但是他没有看到,生产关系本身,那些他借以进行生产并且在他看来是既定的自然关系的社会形式,是这一特殊社会生产方式经常的产物,并只是由此才成为经常的前提。不同的关系和因素不仅变成一种独立的东西,并取得一种奇异的、似乎彼此无关的存在方式,而且表现为物的直接属性,取得物的形态。

由此可见,资本主义生产的当事人是生活在一个由魔法控制的世界里,而他们本身的关系在他们看来是物的属性,是生产的物质要素的属性。但正是在最后的、最间接的形式上(同时在这些形式上中介过程不仅变得看不见了,而且甚至变成自己直接的对立面),资本的不同形态表现为生产的实际因素和直接承担者。生息资本在货币资本家身上人格化了,产业资本在产业资本家身上人格化了,提供地租的资本在作为土地所有者的地主身上人格化了,最后,劳动在雇佣工人身上人格化了。它们作为这样一些在独立的个人身上(这些个人同时只是表现为人格化的物的代表)人格化了的固定形态,加入竞争和实际生产过程。竞争以这种转化为前提。资本的这些固定形态,对于竞争来说,是合乎自然、在自然史意义上存在的形式,而竞争本身在自己的表面现象上[929]只是这一颠倒的世界的运动。就内在联系在这种运动中的实现来说,这种内在联系表现为一种神秘的规律。政治经济学本身,这门致力于重新揭示隐蔽的联系的科学,就是很好的证明。在竞争中一切都在这一最外表的最后的形式上表现出来。例如,市场价格在这里表现为一种占支配地位的东西,利率、地租、工资、产业利润表现为价值的构成要素,而土地价格和资本价格表现为既定的、从事经营必须计算的费用项目。

我们已经看到,亚·斯密起先把价值分解为工资、利润(利息)和地租,后来又反过来把它们说成是商品价格的独立的构成要素。\authornote{见本卷第1册第73—78页。——编者注}在前一种见解中,他说出了隐蔽的联系,在后一种见解中他说的是外部表现。

如果更接近现象的表面,那末除了平均利润率以外,利息,甚至地租也可以说成是商品价格(即市场价格)的构成部分。利息可以直接说成是这样的构成部分,因为它加入费用价格。地租(作为土地价格)虽然不能直接决定产品价格,但它决定生产方式:是把大量资本集中在少量土地上,还是把少量资本分配在大量土地上;是生产这种还是生产那种产品(牲畜还是谷物),其市场价格要最能抵补地租价格,因为地租必须在租约期满以前支付。因此,为了使地租不成为产业利润的扣除部分,牧场会变成耕地,耕地又会变成牧场,等等。可见地租不会直接地但会间接地决定单个产品的市场价格,即通过确定各种产品之间的比例,使需求和供给能够为每一种产品保证最好的价格,以便这种价格能够支付地租。如果说地租在这个意义上不直接决定例如谷物的市场价格,那末,它直接决定牲畜等等的市场价格,简单地说,它直接决定这样一些领域的产品的市场价格,在这些领域里,地租不是由本领域产品的市场价格决定,产品的市场价格却是由播种谷物的土地提供的地租率决定。例如肉类在工业发达的国家总是价钱很贵,即不仅大大高于它的生产价格,而且高于它的价值。因为它的价格不仅必须支付它的生产费用,而且必须支付土地提供的地租,如果在这块土地上种植谷物的话。否则,在大畜牧业的条件下,由于资本的有机构成非常接近[工业中的资本构成],或者不变资本对可变资本甚至占更大优势,肉类就只能支付很少的绝对地租,或者完全不能支付绝对地租。但是肉类支付的、直接加入肉类价格的地租,是由土地作为耕地时会支付的绝对地租和级差地租的总量决定的。这种级差地租在这里也大部分不存在。最好的证据是:在谷物不支付地租的土地上,肉类会支付地租。

因此,如果说利润作为决定的因素加入生产价格,那就可以说,工资、利息,以及在一定程度上地租,作为决定的因素加入市场价格,无疑也作为决定的因素加入生产价格。当然,因为整个说来利息的运动由利润决定,因为谷物地租部分地由利润率决定,部分地由农产品的价值以及由不同土地的产品的不同价值平均化为市场价值决定,而利润率部分地由工资决定,部分地由生产不变资本的生产领域的劳动生产率决定,从而由工资的高度和劳动生产率决定,而工资则归结为商品的一定部分的等价物(即工资等于商品包含的有酬劳动,而利润等于商品包含的无酬劳动);最后,因为劳动生产率的增长只能以两种方式影响商品价格:一是影响商品的价值,即降低其价值,二是影响商品的剩余价值,即提高其剩余价值,——所以全部问题最终可以归结为由劳动时间决定的价值。费用价格无非是预付资本的价值加预付资本所生产的剩余价值,这种剩余价值是在各个领域之间按照它们在总资本中所占的份额进行分配的。所以,如果考察的不是单个领域而是总资本,费用价格就归结为价值。另一方面,每个领域的市场价格,由于不同领域的资本的竞争,经常还原为费用价格。每一单个领域的资本家的竞争力求使商品的市场价格还原为它的市场价值。不同领域的资本家的竞争使市场价值还原为所有领域共同的费用价格。

斯密认为价值由其自身决定的价值各部分构成,李嘉图反对斯密的这种看法。但他不是前后一贯的。否则他就不可能和斯密争论:加入价格的,即作为构成部分加入价格的究竟是利润、工资和地租,还是象他所说的只是利润和工资。既然它们被支付,从分析来看它们是加入价格的。相反,他应当这样说:每一种商品的价格都可以分解为利润和工资,某些商品的价格(而且很多商品的价格是间接地)可以分解为利润、地租和工资;但是没有一种商品的价格是由它们构成的,[930]因为它们不会作为独立的自动的有一定大小的因素构成商品价值,而如果价值是既定的,它倒可以按极不相同的比例分解为上述各部分。并不是既定的因素(利润、工资和地租)通过相加或结合决定价值量,而是同一个价值量,即既定的价值量,分解为工资、利润和地租,并且是按不同的情况,以极不相同的方式在这三个范畴之间进行分配。

假定生产过程在条件不变的情况下不断重复,就是说,再生产和生产一样是在相同的条件下进行。这要有一个前提,就是劳动生产率不变,或至少生产率的变化不致改变生产当事人之间的关系,从而即使商品价值会由于生产力的变化而提高或降低,商品价值在生产当事人之间的分配仍旧不变。在这种情况下,虽然,说价值的不同部分决定整体的价值或价格,虽然在理论上是不确切的,但是,如果把构成理解为由各个部分相加而形成整体,那末,说价值的不同部分构成价值在实际上就是正确的。商品价值会照旧同地分为[预付资本的]价值和剩余价值,[新创造的]价值会同样地分解为工资和利润,而利润也会同样地分解为利息、产业利润和地租。因此似乎可以说:P,即商品价格,分解为工资、利润(利息)和地租,另一方面,工资、利润(利息)、地租则构成价值,或者更确切地说,构成价格。

但是这种再生产的均衡性或等同性(即生产在同样条件下反复进行)实际上是不存在的。生产率会变化并改变生产条件。条件也会从自己方面改变生产率。但是这种偏离部分地会表现在短期间内即可平均化的表面的波动上,部分地会表现在偏离的逐渐积累上,这种偏离或者是引起危机,即通过暴力在表面上回到原来的关系,或者是极缓慢地给自己打通道路,争取被承认为生产条件的改变。

在预支了剩余价值的利息和地租形式上,必须假定,再生产的一般性质保持不变。只要资本主义生产方式继续存在,情况就是这样。其次,甚至必须假定(情况或多或少也是这样),这一生产方式的一定生产关系在一定时间内保持不变。因此,生产的结果就作为牢固的、因而是充当前提的生产条件固定下来,并且作为物质生产条件的牢固的属性固定下来。生产过程所不断分解成的并不断再生产出来的各种不同要素的这种表面上的独立性,在危机到来时就会结束。

{对真正的经济学家来说是价值的东西,对实践的资本家来说就是市场价格,它总是整个运动的最初的东西。}

生息资本在信用上取得了资本主义生产所特有的并与它相适应的形式。信用是资本主义生产方式本身所创造的一种形式。(商业资本[对资本主义生产方式]的从属性实际上不要求这种新的创造,因为商品和货币,商品流通和货币流通,仍旧是资本主义生产的基本前提,只不过是变成了绝对的前提。因此,商业资本一方面是资本的一般形式,另一方面,就它代表执行一定职能的资本即专门在流通过程中执行职能的资本来说,它由生产资本决定这一点,并不会使它的形式有所改变。)

价值平均化为费用价格只有通过以下的方式来进行:单个资本作为整个阶级的总资本的一部分执行职能,另一方面,整个阶级的总资本根据生产的需要在不同的特殊领域之间进行分配。这是通过信用进行的。由于信用,不仅这种平均化成为可能并变得易于进行,而且资本的一部分(在货币资本的形式上)实际上表现为整个阶级用以从事经营的共同材料。这是信用的一种意义。另一种意义是资本总是力求缩短它在流通过程中必须完成的形态变化,总是力求预先实现流通时间,实现它向货币的转化等等,并[931]通过这种办法抵销自己的局限性。最后,积累的职能只要不是[收入]转化为资本,而是供给资本形式的剩余价值,它就会部分地被加在一个特殊的阶级身上,部分地表现为:社会的一切积累在这个意义上都变成资本的积累,并归产业资本家支配。在社会的无数个点上单独进行的这种活动被集中起来,汇集在一定的蓄水池里。由于商品在形态变化中凝结而闲置的货币,就通过这种途径转化为资本。

\centerbox{※     ※     ※}

“土地—地租”,“资本—利息”是一些不合理的说法,因为地租被固定为土地的价格,而利息被固定为资本的价格。在生息资本、提供地租的资本、提供利润的资本这些形式上还能够认出[所有这些不同收入的]共同来源,因为一般资本包括对剩余劳动的占有,也就是说,这些不同的形式只是表示,这种由资本产生的剩余劳动,就一般资本来说,是在两类资本家之间进行分配,而就农业资本来说,则是在资本家和土地所有者之间进行分配。

作为土地的(年)价格的地租和作为资本的价格的利息,就象一样地不合理。这后一种形式同数字的简单基本形式相矛盾,正象那些形式同资本的简单形式商品和货币相矛盾一样。地租和利息的不合理性表现在颠倒的意义上。“土地—地租”,即作为土地的价格的地租表示土地是商品,是具有价值的使用价值,这种价值的货币表现等于它的价格。但是一种不是劳动产品的使用价值不可能有价值,也就是说,它不能算作一定量社会劳动的物化,一定量劳动的社会表现。它不是这种东西。使用价值要表现为交换价值,要成为商品,就必须是具体劳动的产品。只有在这样的前提下,这种具体劳动才能表现为社会劳动,表现为价值。土地和价格是不可通约的量,不过它们彼此还是应当有一种关系。在这里,一个没有价值的物有着一个价格。

\todo{}

另一方面,作为资本价格的利息也表现出颠倒的不合理性。在这里,没有使用价值的商品有了双重价值,先是有价值,然后又有和这个价值不同的价格。因为资本首先不外是一个货币额或等于一定货币额的一定商品量。如果商品作为资本来贷放,这个商品就只是一个货币额的伪装形式。因为作为资本贷放的,并不是若干磅棉花,而是其价值存在于棉花中的若干货币。所以,资本的价格也和只是作为货币额——即表现为货币并存在于交换价值形式上的价值额——存在的资本有关。一个价值额怎么能够在那个要用它本身的货币形式来表示的价格之外,还有一个价格呢?价格是和商品的使用价值相区别的商品的价值。因此,价格作为一种和商品价值不同的东西,价格作为一个货币额的价值(因为价格只是价值在货币上的表现),是术语上的矛盾。

这种说法的不合理性(事物本身的不合理性是这样产生的:就利息来说,资本表现为前提,和它自己的、使它成为资本即自行增殖的价值的过程相分离,另一方面,提供地租的资本只是作为农业资本,只是作为特殊领域的资本才提供地租,也就是说,它借以表现的这种形式,被移到了使它根本区别于工业资本的要素上),被庸俗经济学家深深地感觉到了,于是他就假造了两种说法,以便使它们变得合理。他断言利息是为资本支付的,因为资本是使用价值,他因此也谈到产品或生产资料本身对再生产的有用性,谈到资本作为劳动过程的物质要素的有用性。

但是,资本的有用性,它的使用价值,本来已经存在于它的商品形式中,没有这种有用性,它就不会是商品,也不会有价值。在货币形式上,资本是商品价值的表现,并且[932]能够依照商品本身的价值转化为商品。但是,如果我把货币转化为机器、棉花等等,我就是把它们转化为具有同样价值的使用价值。这种转化只涉及价值的形式。在货币形式上,资本具有的使用价值使它能够转化为任何形式的、但具有同样价值的商品。通过这种形式变化,货币的价值并没有变,正象商品转化为货币时商品的价值不变一样。我能使货币转化成的那些商品的使用价值,不会给货币提供任何超过其价值、不同于其价值的价格。但是,如果我以这种转化为前提,并且说,价格是为商品的使用价值支付的,那末,商品的使用价值就根本得不到支付,或者只是在商品的交换价值被支付的情况下才得到支付。至于怎样利用买来的商品的使用价值,它进入个人消费还是进入生产消费,这绝对不会使它的交换价值有任何变化。由此而引起变化的只是谁购买商品:是产业资本家还是直接消费者。因此,商品在生产上的有用性可以说明商品一般具有交换价值,因为要使商品包含的劳动得到支付,它们必须具有使用价值,否则它们便不是商品,只有作为使用价值和交换价值的统一体,它们才是商品。但是这种使用价值决不能说明,商品作为交换价值或作为价格,还会有一个不同于这个价格的其他价格。

我们看到,庸俗经济学家在这里想通过试图把资本即货币或商品(就它们具有特殊的、不同于它们作为货币或商品的规定性来说)转化为简单的商品的办法来逃避困难,就是说,他要避开的正好是应当说明的那一特殊区别。他不想说,资本是剥削剩余劳动的手段,因此,资本是比它包含的价值更大的价值。他却说,资本所以具有比它的价值更大的价值,是因为它和其他任何商品一样,是普通的商品,也就是说,它具有使用价值。这里把资本和商品等同起来了,而需要说明的正是商品怎样能够表现为资本。

对于土地,庸俗经济学家所持的态度却相反,只要他不随声附和重农学派的话。在论述利息的时候,为了说明资本和商品之间的区别,为了说明商品向资本的转化,他把资本转化成商品。在这里,他把土地转化成资本,因为资本关系本身比土地价格更适合于他的观念。地租可以看作资本的利息。例如,如果地租是20,而利率等于5,那就可以说,这20便是资本400的利息。实际上土地就是按400出卖,这不过是出卖20年的地租。这种对预先实现的二十年地租的支付便是土地的价格。这样,土地就转化为资本。每年支付的20,只是为土地支付的资本的百分之五的利息。通过这种办法,“土地—地租”就变成“资本—利息”,而这又被幻想成对商品使用价值的支付,也就是说,被幻想成“使用价值—交换价值”这种关系。

庸俗经济学家中较有分析能力的人懂得,土地价格无非是地租资本化的表现,实际上是根据当时存在的利率决定的若干年地租的购买价格。他们懂得,地租的这种资本化以地租为自己的前提,所以地租不能反过来用它自己的资本化来解释。因此,当他们宣称地租是投在土地上的资本的利息时,他们也就否定了地租本身,而这并不妨碍他们承认,没有投入任何资本的土地也提供地租,也不妨碍他们承认,在肥力不同的土地上同量资本提供不同的地租,或者承认,在肥力不同的土地上不同量资本提供相同的地租。同样,这也不妨碍他们承认,投在土地上的资本(如果为土地支付的地租确实必须用它来解释的话)也许会提供比以固定资本形式投在工业上的同量资本提供的利息大四倍的利息,也就是大四倍的地租。

我们看到,在这里排除困难的办法总是:避开困难,而对应当解释的特殊区别则是用某种关系来代替,这种关系表明的却是和这种区别相反的东西,因而无论如何也不表明这种区别。[932]

\tchapternonum{[(6)庸俗社会主义反对利息的斗争(蒲鲁东)。不理解利息和雇佣劳动制度之间的内在联系]}

[935]蒲鲁东同巴师夏关于利息的论战是很有特色的,它既能说明庸俗经济学家是用什么样的方式来维护政治经济学的各种范畴的,也能说明肤浅的社会主义(蒲鲁东的论战未必配得上这个称号)是用什么样的方式来攻击这些范畴的。我们在论庸俗经济学家一节\endnote{在1863年1月拟定的《资本论》第三部分的计划中,倒数第二章即第十一章的标题是《庸俗政治经济学》(见本卷第1册第447页)。这个计划是在写完1861—1863年手稿第XV本中《收入及其源泉。庸俗政治经济学》这一部分之后一个半月到两个月拟定的。——第581页。}中将回过头来谈这个问题。这里只是预先说几点意见。

如果蒲鲁东对资本的运动一般有所了解,那末,[货币的]回流运动就不会作为[生息资本的]一种特性使他感到震惊了。回流总额中的价值余额也是同样情况。这正是资本主义生产的特征。

{但是在蒲鲁东那里,正如我们将要看到的,余额就是附加额。一般说来,他的批判是幼稚的,他甚至根本没有掌握他要批判的那门科学的基本要素。例如,他根本不懂得货币是商品的必要形式。(见第一部分\endnote{马克思指《政治经济学批判》第一分册。见《马克思恩格斯全集》中文版第13卷第45页和第76页。——第581页。})在这里他甚至把货币和资本混淆起来,因为借贷资本表现为货币形式上的货币资本。}

能够使他感到惊奇的不是不被支付任何等价物的余额,因为剩余价值(资本主义生产就是建立在它上面的)是不花费任何等价物的价值。这并不是生息资本的特征。生息资本的特征(就我们所考察的运动形式来说)只在于第一个要素,这和蒲鲁东设想的正好相反,那就是贷款人贷出货币,最初并没有为此得到等价物,因此,资本带着利息流回,就贷款人和借款人之间进行交易来说,和资本经过的形态变化[毫无共同之处],如果这些形态变化只是经济形式的形态变化,它们就表现为交换行为的序列,即商品转化为货币和货币转化为商品,如果它们是现实的形态变化,或者说是生产过程,它们就和生产消费相结合。(消费本身在这里构成经济形式的运动的一个要素。)

但是货币在贷款人手里没有做到的事情,在把它们实际用作资本的借款人手里做到了。它们在借款人手里完成了它们作为资本的现实运动。它们作为货币加利润,作为货币加1/x货币,流回他手里。贷款人和借款人之间的运动,只是表示资本的起点和终点。资本作为货币从A手里转到B手里。在B手里货币成为资本,它们作为资本经过一定的循环以后带着利润流回。这种中间行为,实际过程(包括流通过程和生产过程),完全与借款人和贷款人之间的交易无关。这一交易只是在货币作为资本已经得到实现以后才重新开始。现在货币带着一个余额回到贷款人手里,不过这个余额只是借款人实现的余额的一部分。借款人获得的等价物是产业利润,也就是这个余额中留给他的一部分,这一部分只是他靠借入的货币才占有的。所有这一切在借款人和贷款人之间的交易中都是看不见的。这种交易限于两种行为。货币从A到B的转手。货币在B手里的间歇。货币在间歇以后带着利息流回A手里。

因此,如果只是考察这种形式(A和B之间的这种交易),得到的就是没有中介过程的资本的单纯形式:货币,它以a额支出,经过一定时间,再以a+1/xa额流回,除了a额流出又以a+1/xa额流回这样一段时间以外,完全不存在任何中介过程。

蒲鲁东先生正是在这种没有概念的形式上(这种形式当然是作为独立的运动和资本的实际运动同时发生的,它使这个运动开始并结束)考察事物的,在这样的考察中,对他来说,一切都必然是不可理解的。如果这种代替买和卖的借贷形式不再存在,那末,在他看来余额也就不会再有了。其实,不会再有的只是余额在两类资本家之间的分割。但是这种分割能够而且必然不断重新发生,只要商品或货币能够转化为资本,而这一点在雇佣劳动的基础上总是能够做到的。如果商品和货币不能变成资本,从而也不能作为可能的资本贷出,它们就不能和雇佣劳动相对立。如果它们作为商品和货币不和雇佣劳动相对立,从而劳动本身也不成为商品,那末,这不过是意味着[936]回到资本主义生产以前的生产方式,在那里,劳动不转化为商品,而大量劳动还以农奴劳动或奴隶劳动形式出现。在以自由劳动为基础的条件下,这种情况只有在工人是自己的生产条件的所有者时才有可能。自由劳动在资本主义生产的范围内发展为社会劳动。因此,说工人是生产条件的所有者,就是说生产条件属于社会化的工人,工人作为社会化的工人进行生产,并把他们自己的生产作为社会化的生产从属于自己。但是象蒲鲁东那样,既要保存雇佣劳动,从而保存资本的基础,同时又想用否定资本的一种派生形式的办法来消除“弊端”,那就是幼稚。

《无息信贷。弗·巴师夏先生和蒲鲁东先生的辩论》1850年巴黎版。

在蒲鲁东看来,贷放是一件坏事,因为它不是出售。

\begin{quote}{取息的贷放“是这样一种能力,即人们可以不断重新出售同一物品,并且不断重新为此得到价格,但从来不出让对所售物品的所有权”。(《无息信贷》,《人民之声报》\endnote{《人民之声报》(《LaVoixduPeuple》)——蒲鲁东派的日报,1849年10月1日至1850年5月14日在巴黎出版。——第583页。}编者之一舍韦写的第一封信,第9页)}\end{quote}

使这封信的作者迷惑的是,这种“物品”(例如货币或房屋)不会变更所有者,这同在买和卖时不一样。不过他没有看到,当货币贷出时,并没有得到任何等价物作为报酬,而在实际过程中,在交换的形式和基础上不仅会得到一个等价物,而且会得到一个无酬的余额。在物品交换发生时,不会发生价值变动,同一个人仍然是同一价值的“所有者”。在产生剩余价值时,不会发生交换。当商品和货币的交换再开始时,剩余价值已经包含在商品中了。蒲鲁东不懂得利润,从而利息,怎样由价值的交换规律产生。因此,照他说来,“房屋”、“货币”等等就不应当作为“资本”,而应当作为“商品……按照成本”来交换(《无息信贷》第43—44页)。

\begin{quote}{“实际上,出售帽子的制帽业主……得到了帽子的价值,不多也不少。但借贷资本家……不仅一个不少地收回他的资本,而且他得到的,比这个资本,比他投入到交换中去的东西多;他除了这个资本还得到利息。”(同上,第69页)}\end{quote}

蒲鲁东先生的制帽业主看来不是资本家,而是手工业者、手艺人。

\begin{quote}{“因为在商业中,资本的利息加到工人的工资上,共同构成商品的价格,所以,工人要买回他自己的劳动的产品,就不可能了。自食其力的原则,在利息的支配下,包含着矛盾。”(第105页)}\end{quote}

在第九封信中(第144—152页),勇敢的蒲鲁东把作为流通手段的货币同作为资本的货币混淆起来,从而得出结论说:法国现存的“资本”会提供160%(即10亿资本——“法国流通中的现金总额”——在国债和抵押等等形式上的年利息为16亿)。

其次:

\begin{quote}{“货币资本从一次交换到另一次交换,通过利息的积累,总是不断回到它的出发点,由此可见,每次由同一个人的手重新把这些货币贷出,总能给同一个人带来利润。”(第154页)}\end{quote}

因为资本是在货币形式上贷出,所以蒲鲁东以为,货币资本即现金具有这种特殊的属性。在蒲鲁东看来,一切东西都应当出售,但任何东西也不应当贷放。换句话说,正象蒲鲁东想保存商品,但不想使商品变成“货币”一样,他在这里想保存商品和货币,但是它们不应当发展成资本。如果把一切空想的表达形式抛开,那就不过是说,不应当从小市民-农民的和手工业的小生产过渡到大工业。

\begin{quote}{“因为价值无非是一种比例,一切产品必然互成比例,所以由此可以得出结论,从社会的观点来看,产品总是价值并且是确定的价值。对社会来说,资本和产品之间的区别是不存在的。这种区别完全是主观的,只是对个人来说才是存在的。”(第250页)}\end{quote}

象“主观的”这样的德国哲学用语,落在蒲鲁东的手里是多么不幸!社会的、资产阶级的形式对他来说成了“主观的”。这个主观的,而且是错误的抽象使蒲鲁东断言,因为商品的交换价值表示商品之间的比例,所以它表示商品之间的任何比例,而不表示商品与之成某种比例的第三种东西,——这种错误的“主观的”抽象也就是一种[937]“社会的观点”,从这种观点来看,不仅商品和货币,甚至商品、货币和资本都是等同的。的确,从这种“社会的观点”来看,所有的猫都是灰色的。

最后,剩余价值还表现在道德的形式上:

\begin{quote}{“一切劳动都应当提供一个余额。”(第200页)}\end{quote}

这种道德训条自然是剩余价值的绝妙的定义。[937]

\tchapternonum{[(7)关于利息问题的历史。路德在进行反对利息的论战时胜过蒲鲁东。对利息的观点随着资本主义关系的发展而发生变化]}

[937]路德生活在中世纪市民社会瓦解为现代社会诸要素的时代(世界贸易和黄金新产地的发现加速了这个瓦解过程),所以,他当然只能从生息资本和商业资本这两个洪水期前的[形式]去认识资本。如果说已经站稳脚跟的资本主义生产在它的幼年时期力图迫使生息资本从属于产业资本(在资本主义生产以工场手工业和大商业的形式最先繁荣起来的荷兰,这一点事实上已经首先做到了,而在英国,这一点则是在十七世纪,并且部分地是以非常天真的形式,被宣告为资本主义生产的第一个条件),那末,在向资本主义生产方式过渡时,承认“高利贷”这个生息资本的古老形式是一个生产条件,是一种必要的生产关系,反而成为第一个步骤;正如后来,当产业资本征服了生息资本时(十八世纪,边沁\endnote{马克思指边沁的著作《为高利贷辩护》,1787年在伦敦出第一版,1790年出第二版,1816年出第三版。——第586、598页。}),它本身就承认了生息资本的合法性,承认了它们之间的血肉关系。

路德比蒲鲁东站得高。借贷和购买之间的区别没有把他弄糊涂;他认为在这两种情况下都同样有高利贷存在。在他进行的论战中,最有力的一点,是他把利息长在资本上当作主要的攻击点。

(I)《论商业与高利贷》(1524年),《可尊敬的马丁·路德博士先生的第六部著作》1589年维登堡版。

(这部著作是在农民战争前夜写成的。)

关于商业(商业资本):

\begin{quote}{“现在,商人对贵族或盗匪非常埋怨{由此可以看到,为什么商人和国君一起反对农民和骑士},因为他们经商必须冒巨大的危险,他们会遭到绑架、殴打、敲诈和抢劫等等。如果商人是为了正义而甘冒这种风险,那末他们当然就成了圣人了……但既然商人对全世界,甚至在他们自己中间,干下了这样多的不义行为和非基督教的盗窃抢劫行为,那末,上帝让这样多的不义之财重新失去或者被人抢走,甚至使他们自己遭到杀害,或者被绑架,又有什么奇怪呢?……国君应当对这种不义的交易给予应有的严惩,并保护他们的臣民,使之不再受商人如此无耻的掠夺。因为国君没有这么办,所以上帝就利用骑士和强盗,假手他们来惩罚商人的不义行为,他们应当成为上帝的魔鬼,就象上帝曾经用魔鬼来折磨或者用敌人来摧毁埃及和全世界一样。所以,他是用一个坏蛋来打击另一个坏蛋,不过在这样做的时候没有让人懂得,骑士是比商人小的强盗,因为一个骑士一年内只抢劫一两次,或者只抢劫一两个人,而商人每天都在抢劫全世界。”(第296页)“……以赛亚的预言正在应验:你的国君与盗贼作伴。因为他们把一个偷了一个古尔登或半个古尔登的人绞死,但是和那些掠夺全世界并比所有其他的人都更肆无忌惮地进行偷窃的人串通一气。大盗绞死[938]小偷这句谚语仍然是适用的。罗马元老卡托说得好:小偷坐监牢,戴镣铐,大盗戴金银,衣绸缎。但是对此上帝最后会说什么呢?他会象他通过以西结的口所说的那样去做,把国君和商人,一个盗贼和另一个盗贼熔化在一起,如同把铅和铜熔化在一起,就象一个城市被焚毁时出现的情形那样,既不留下国君,也不留下商人。我担心,这个日子已经不远了。”(第297页)}\end{quote}

关于高利贷,关于生息资本:

\begin{quote}{“有人对我说,现在每年在每一次莱比锡博览会上要收取10古尔登,就是说每一百收取30。\endnote{指100古尔登的贷款,条件是分三期在莱比锡博览会上支付利息。在莱比锡每年举行三次博览会:新年,复活节(春季),米迦勒节(秋季)。——第587页。}有人还加上瑙堡集市,因此,每一百要收取40,是否有比这更多的,我不知道。岂有此理,这样下去怎么得了?……现在,在莱比锡,一个有100佛罗伦的人,每年可以收取40,这等于每年吃掉一个农民或市民。如果他有1000佛罗伦,每年就会收取400,这等于每年吃掉一个骑士或一个富有的贵族。如果他有10000佛罗伦,每年就会收取4000,这等于每年吃掉一个富有的伯爵。如果他有100000佛罗伦(这是大商人必须具有的),每年就会收取40000,这等于每年吃掉一个富有的国君。如果他有1000000佛罗伦,每年就会收取400000,这等于每年吃掉一个大的国王。为此,他不必拿他的身体或商品去冒险,也不必劳动,只是坐在炉边,烤苹果吃。所以,一个强盗坐在家里,可以在十年内吃掉整个世界。”(第312—313页)\endnote{马克思以《关于高利贷,关于生息资本》为题的这段引文不是从路德的著作《论商业与高利贷》(1524年)中摘出的,而是从路德的另一部著作《给牧师们的谕示:讲道时要反对高利贷》(1540年)中摘出的,马克思在后面第III点考察了这一著作。——第588页。}}\end{quote}

{(II)《讲道:福音书中的富人和穷人拉撒路》1555年维登堡版。

\begin{quote}{“对于富人,我们不应当从外表举止上去判断,因为他身穿羊皮袄,生活阔绰,冠冕堂皇,巧妙地掩盖了狼子野心。因为福音书谴责他不是由于他犯了奸淫罪、杀人罪、抢劫罪、渎神罪,或者世人或理性会谴责的某种罪行。他的确和那个与众不同的、一礼拜禁食两次的法利赛人一样,过着端正的生活。”}}\end{quote}

在这里路德告诉我们,高利贷资本是怎样产生的:它是靠市民(小市民和农民)、骑士、贵族、国君的破产产生的。一方面,城关市民、农民、行会师傅、总之小商品生产者的剩余劳动以及劳动条件会流入高利贷者手里,因为他们在把自己的商品转化为货币以前就需要货币,例如用来支付,他们已经要由自己购买一部分劳动条件,等等。另一方面,地租所有者,即挥霍享乐的富有阶级的钱也流入高利贷者手里,高利贷者现在把地租据为己有。高利贷有两种作用:第一,总的说来,它形成独立的货币财产,第二,它把劳动条件占为己有,也就是说,使旧劳动条件的所有者破产;因此,它对形成产业资本的前提是一个有力的手段,对生产条件和生产者的分离是一个有力的因素。它完全和商人一样。二者有共同点:都会形成独立的货币财产,也就是说,以货币要求权的形式把年剩余劳动的一部分,劳动条件的一部分,以及年劳动积累的一部分,积累在自己手里。他们实际掌握的货币,只是既构成常年的和每年积累的贮藏货币的一小部分,又构成流动资本的一小部分。说在他们那里形成货币财产,就是说年产品和年收入的很大一部分会落到他们手里,并且不是以实物形式而是以货币这种转化形式支付给他们。因此,只要货币不是作为现金能动地流通,不是处于运动之中,货币就积累在他们手里,流通货币的蓄水池也部分地掌握在他们手里,对产品的要求权更是掌握在和积累在他们手里,不过这种要求权是对已转化为货币的商品的要求权,是货币要求权。[939]高利贷,一方面是封建财富和封建所有制的破坏者,另一方面是小资产阶级的、小农民的生产的破坏者,总之,是生产者仍然表现为自己的生产资料所有者的一切形式的破坏者。

在资本主义生产中,劳动者是生产条件(既包括他所耕种的土地,也包括他用来劳动的工具)的非所有者。但是,在这里,同生产条件的这种分离相适应,生产方式本身发生了真正的变化。工具变成了机器;劳动者在工厂劳动等等。生产方式本身不再容许生产工具处于那种和小所有制联系着的分散状态,也不再容许劳动者自己处于分散状态。在资本主义生产中,高利贷不能再使生产条件和劳动者、生产者分离,因为二者已经分离了。

高利贷只是在生产资料分散的地方,从而在劳动者作为小农、行会手工业者(小商人)等等或多或少独立地进行生产的地方,才会把财产集中起来,特别是以货币财产的形式集中起来。作为农民或手工业者,这个农民可以是农奴,也可以不是农奴,这个手工业者可以属于行会,也可以不属于行会。高利贷者在这里不仅把依附农本身支配的那一部分剩余劳动(在同自由农民等等打交道时,则把全部剩余劳动)占为己有,而且还把生产工具占为己有,虽然农民等等仍然是这些生产工具的名义上的所有者,并且在生产中,仍然作为所有者同这些生产工具发生关系。这种高利贷就是建立在这种基础上,建立在这种生产方式上,高利贷不改变这种生产方式,而是象奇生虫那样紧紧地吸在它身上,使它虚弱不堪。高利贷吮吸着它的脂膏,使它精疲力竭,并迫使再生产在每况愈下的条件下进行。由此产生了民众对高利贷的憎恶,在古代的关系下特别是这样,因为在这种关系下,生产条件为生产者所有这种生产性质,同时是政治关系即市民的独立地位的基础。一旦劳动者不再拥有生产条件,这种情况就终止了。高利贷的权力也就随之而告终。另一方面,在奴隶制占统治地位或者剩余劳动为封建主及其家臣所吞食的情况下,奴隶主或者封建主即使陷入高利贷之中,生产方式仍旧不变,只是它会更加残酷。负债的奴隶主或封建主会榨取得更厉害,因为他自己也被榨取了。或者,他最后让位给高利贷者,高利贷者本人象古罗马的骑士等等一样成为土地所有者等等。旧剥削者的剥削或多或少是政治权力的手段。现在代替旧剥削者出现的,则是粗暴的拚命要钱的暴发户了。但生产方式本身仍旧不变。

高利贷者在资本主义前的一切生产方式中所以只是在政治上有革命的作用,是因为他会破坏和瓦解这些所有制形式,而政治制度正是建立在这些所有制形式的牢固基础上,也就是建立在它们的同一形式的不断再生产上的。高利贷也有集中的作用,但只是在旧生产方式的基础上起集中的作用,结果是使除了奴隶、农奴等等以及他们的新主人以外的社会瓦解为平民。在亚洲的各种[社会]形式下,高利贷能够长期延续,这只是造成经济的衰落和政治的腐败,但没有造成[现存的生产方式]真正解体。只有在资本主义生产的其他条件——自由劳动,世界市场,旧的社会联系的瓦解,劳动在一定阶段上的发展,科学的发展等等——已经具备的时代,高利贷才表现为形成新生产方式的一种手段,同时又表现为使封建主,反资产阶级要素的支柱遭到毁灭,使小工业、小农业等等遭到破坏的手段,总之,表现为把作为资本的劳动条件集中起来的手段。

高利贷者、商人等等占有“货币财产”,这无非就是,把表现为商品和货币的国民财产集中在他们手里。

在高利贷者本人不是生产者的情况下,资本主义生产起初不得不同高利贷作斗争。到了资本主义生产已经确立的时候,高利贷对剩余劳动的支配权(这种支配权同旧生产方式的继续存在相联系)已经终止了。产业资本家以利润形式把剩余价值直接占为己有;他也已经部分地占有生产条件,并且直接占有一部分年积累。从这时起,特别是随着产业财产和商业财产的发展,高利贷者即贷款人,就只是一种由于分工而同产业资本家分离、但又从属于产业资本的角色。

[940](III)《给牧师们的谕示:讲道时要反对高利贷》1540年维登堡版(没有页码)。

商业(买、卖)和借贷(路德没有象蒲鲁东那样被这种形式上的差别弄糊涂)。

\begin{quote}{“十五年前我已经写过反对高利贷的文章,因为那时高利贷势力已经很大,我不抱任何改善的希望。从那时起,高利贷的身价高了,它已不愿被看作是丑恶、罪行或耻辱,而是让人作为纯粹的美德和荣誉来歌颂,好象它给了人民伟大的爱和基督教的服务似的。既然耻辱已经变为荣誉,丑恶已经变为美德,那还有什么办法呢?塞涅卡以自然的理性说道:如果恶习成自然,那就无可救药了。德国成了它应该成为的那个样子;卑鄙的贪婪和高利贷把它彻底毁灭了……首先谈借贷。当人们贷出货币,并为此要求或取得更多的或更好的东西时,这就是高利贷,它受到所有的法律的谴责。因此,所有那些从贷出的货币中每一百收取五、六或更多的人都是高利贷者,他们都懂得要照此行事,他们被称为崇拜贪婪或钱财的奴仆……因此,对于谷物,大麦和其他等等商品我们也应该这样说,如果为此要求更多的或更好的东西,这就是高利贷,就是偷盗和抢劫财物。因为借贷的意思是:我把我的钱财或用具在某人需用时,或者说在我能够并且愿意时,借给他使用,他过一定时间要按照我借给他的原样还给我。”“因此,从购买当中也能获得高利。但是现在要一口吃掉,那就太多了。现在必须先谈一种,即先谈放债的高利贷。等我们搞掉这个以后(在末日审判以后),我们再来谴责购买上的高利贷。”“高利贷者老爷会这样说:朋友,按照目前的情况,我按每一百收取五、六或十的利息借钱给我的邻人,那是对他提供很大的服务。他感谢我这种借贷,把它看成一种特别的行善。他再三向我请求,自愿地,毫不勉强地提议每一百古尔登送我五个、六个或十个古尔登。难道我就不能从高利贷问心无愧地得到这些古尔登吗?……你尽可以夸耀、粉饰和装扮……但是谁取得的更多或更好,那就是高利贷。也就是说,象偷盗和抢劫一样,他不是为邻人服务,而是损害邻人。一切名为对邻人服务和行善的事情,并非都是服务和行善。奸夫和淫妇也是互相提供重大的服务和互相满足的。骑士帮助罪犯拦路行抢,打家劫舍,也是对罪犯的重大服务。罗马教徒没有把我们全部淹死、烧死、杀死、囚死,而是让一些人活着,把他们驱逐,或者夺去他们所有的东西,也是对我们的重大服务。魔鬼对于侍奉他的人也提供重大的不可估量的服务……总之,世上到处都是重大的、卓越的、日常的服务和行善……诗人们记述,独眼巨人波利菲米斯答应乌利斯,要对他表示友好,那就是先吃掉他的同伴,最后再吃掉他。是啊,这也是一种服务和很大的行善!现在贵族和平民,农民和市民都在热心从事这种服务和行善;他们大量收购、囤积、造成物价昂贵,[941]抬高了谷物、大麦以及人们所必需的一切东西的价格,然后擦擦嘴巴说:是啊!人必须有他所必需的东西,我让它为人们服务,虽然我可以而且有权把它留归自己。于是,上帝也被巧妙地欺骗和愚弄了……现在,人都成了圣人……现在,没有人再会放高利贷,没有人会贪婪和为非作歹了,世界已经完全成了神圣的世界,每个人都为别人服务,没有人会去损害别人了……如果高利贷者这样来提供服务,那也是为害人的魔鬼服务,虽然一个贫困的人是需要这种服务的,而且他必须把他没有完全被吃掉看作是一种服务或行善……如果他想得到钱,他就会为你\authornote{高利贷者。——编者注},而且必须为你提供这样的服务{支付高利}。”}\end{quote}

{从上面的引文可以看到,在路德的时代,高利贷是极其盛行的,而且已被当作一种“服务”来加以辩护(萨伊——巴师夏\authornote{见本卷第1册第435页。——编者注})。已经出现了合作论或协调论:“每个人都为别人服务。”

在古代世界比较兴盛的时期,高利贷是被禁止的(即不允许收取利息)。后来它合法化了,并且盛行起来。在理论上则始终(如在亚里士多德的著作里\endnote{亚里士多德在他的著作《政治学》第一篇里讲过关于利息是一种违反自然的东西的观点,马克思在《资本论》第一卷第四章考察了这个观点(见《马克思恩格斯全集》中文版第23卷第187页)。——第593页。})认为高利贷本身是坏的。

在基督教的中世纪,高利贷被看成是一种“罪恶”,并为“教规”所禁止。

近代。路德。对高利贷还存在着天主教-异教的观点。高利贷广泛盛行(部分是由于政府需要货币,部分是由于商业和工场手工业的发展,部分是由于产品转化为货币的必要性)。但是它的公民权已被确认。

荷兰。对高利贷的最早的辩护。高利贷也是最早在那里现代化,从属于生产资本或商业资本。

英国。十七世纪。争论已不再是针对高利贷本身,而是针对利息的大小。高利贷对信贷的支配关系。要求创立信贷形式。强制的立法措施。

十八世纪。边沁。自由的高利贷被认为是资本主义生产的要素。}

[从路德的著作《给牧师们的谕示:讲道时要反对高利贷》中再摘引几段。]

利息作为对损失的赔偿:

\begin{quote}{“[可能发生而且会常常发生下述情况:我,汉斯,借给你,巴塔扎尔,100古尔登,条件是到米迦勒节时我必须收回来;如果你耽误了,我就会因此遭受损失。米迦勒节到了,你没有偿还我这100古尔登。我没有钱支付,法官就逮捕我,把我投入监狱,或使我遭受其他的不幸。我坐牢,我的营养,我的健康都遭受重大损失。这都是由于你的耽误,你以怨报德。这时我该怎么办呢?我继续遭受损失,是因为你耽误拖延,你多耽误拖延一天,我就多遭受一天损失。谁应该承担或赔偿这种损失呢?这种损失直到我死都会成为我家无法容忍的客人。]好吧,这里单从世俗和法律方面来谈谈这个问题(我们必须把神学放到以后谈),你巴塔扎尔应该在100古尔登之外赔偿我由此引起的全部损失以及一切费用。{路德在这里把费用理解为贷出者本人因不能偿付自己的债务而引起的诉讼费等等}……因此,就是从理性和自然法来看,你赔偿我的一切,即本金和我所受的损失,也是公平的……这种损失在法律书上,拉丁文叫interesse……除了这种损失还可能造成其他损失:如果你巴塔扎尔到米迦勒节不偿还我这100古尔登,而我本来有机会购买花园、田地、房屋或任何可为我和我的孩子提供很大好处或养活我们的东西,现在我就不得不放弃这种机会,由于你的耽误和拖延,你使我遭受到损失,并妨碍了我,使我再也不能买到这些东西……这样,我把它们贷放给你,你使我两头受损失:这里我不能支付,那里我不能购买,也就是我在两方面都不得不受到损失,这就叫作双重损失:既遭受损失,又丧失利益……他们听说汉斯贷放100古尔登受了损失,并要求适当的赔偿,就急忙趁此机会对每100古尔登都索取达双重损失的赔偿,即为不能支付的损失和失去购买花园的机会所受的损失要求赔偿,好象每100古尔登都自然会生出这样双重的损失一样。因此,只要他们有100古尔登,他们就会贷放出去,并按照他们实际上没有受到的这样双重的损失来要求赔偿……既然谁也没有使你受损失,并且你既不能证明,也不能计算这种损失,你却从邻人手里取得货币来赔偿你虚构的损失,因此,你就是高利贷者。法学家把这种损失不是叫作实际的损失,而是叫作幻想的损失。这是各个人为自己而想象出来的损失……因此,[942]说我可能会受损失,因为我可能既不能支付也不能购买,是不行的。如果这种说法能够成立,那就是从偶然生出必然,就是无中生有,就是从未必会有的东西生出确实会有的东西。这种高利贷,要不了几年,不就会把整个世界吞掉了吗?……贷出者必须得到赔偿,这种情况是一种偶然的不幸,是不以他本人的意志为转移的,但在这种交易中情况却不是这样,而是正好相反,人们总是费尽心机编造损失,让贫苦的邻人来赔偿,企图以此为生和发财致富,靠别人的劳动、忧患、危险和损失而使自己过着骄奢淫逸和荣华富贵的生活。我坐在火炉旁边,让我的100古尔登在国内为我搜集钱财。因为这是贷放出去的货币,所以终归要保存在我的钱袋里,没有任何危险,一点也不用担忧。朋友啊,谁不乐意这样做呢?贷款是如此,贷谷物,贷葡萄酒和类似的商品也是如此,都可能有这样双重的损失。但是这种损失不是从商品本身自然生出的,而可能是偶然生出的,因此,在这种损失真正发生并得到证实以前,就不能算作损失……高利贷必然会出现,不过,高利贷者是要倒霉的……所有明智的异教徒也都非常严厉地谴责高利贷。例如,亚里士多德在《政治学》中说:高利贷是违反自然的,因为它取得的总是比给予的多。这就废除了一切美德的手段和尺度,即所谓的对等交换,算术上的相等……但是,拿别人的东西,偷窃或抢劫别人,是一种无耻生涯,这种人,对不起,就叫作盗贼,通常要处以绞刑;而高利贷者是高尚的盗贼,坐在安乐椅上;因此,人们称他们为坐在安乐椅上的强盗……异教徒根据理性得出了高利贷者是四倍盗贼和杀人犯的结论。而我们基督教徒却非常尊敬他们,几乎要为了他们的货币而崇拜他们……凡是吸尽、抢劫和盗窃别人营养的人,就是犯了使人饿死,使人灭亡的杀人大罪(杀多少,由他决定)。高利贷者就是犯了这样的大罪,他照理应当上绞架,如果他身上的肉多得足供许多乌鸦啄而分食,那末,他盗窃了多少古尔登,就应该被多少乌鸦去吃。但是他们却泰然坐在安乐椅上……重利盘剥者和高利贷者会大喊大叫:要遵守契约,要遵守盖了章的东西!法学家对此立即作了很好的回答:那是邪恶的契约。神学家也说,给恶魔立的契约,就是用血签的字盖的章也无效。因为凡是违反上帝、法律和自然的东西都等于零。因此,国君(只有他能做到这一点)应该立刻干预这件事,毁掉印章和契约,而不考虑……所以,在世界上人类再没有比守财奴和高利贷者更大的敌人了(恶魔除外),因为他想成为支配一切人的上帝。土耳其人、武夫、暴君都是恶人,但他们仍不得不让人们生活,并自认是恶人和敌人。他们有时还会同情甚至不得不同情某些人。而高利贷者和贪财之徒却想竭尽全力使整个世界毁灭于饥渴贫苦之中(到什么程度,由他决定),从而使他能独占一切,人人都把他奉为上帝,去领受他的恩赐,[943]永远成为他的奴隶。这时,他的心在欢跳,血在畅流。他披上貂皮长外套,带上金链指环,穿着华丽的衣服,擦擦油嘴,让人看来俨如尊贵的虔诚者,比上帝自己还仁慈得多,比圣母和一切圣徒还友爱得多……异教徒描写了海格立斯的伟大功勋,描写他怎样打败了许多怪物和可怕的恶魔,拯救了国家和人民。高利贷者是一个庞大可怕的怪物,象一只蹂躏一切的恶狼,比任何卡库斯、格里昂或安泰都厉害。但他却装出一副虔诚的样子,想使人无法知道被他倒着牵回洞穴去的公牛究竟到什么地方去了。”}\end{quote}

{这是对一般资本家的绝妙写照,资本家装出一副样子,好象他从别人那里拖回他的洞里去的东西是从他那里出来的,因为他使这些东西倒着走,看起来好象是从他的洞里走出来的。}

\begin{quote}{“然而海格立斯必然会听到公牛的吼声和俘虏的叫声,甚至到悬崖峭壁中去搜寻卡库斯,把公牛从恶汉手中拯救出来。所谓卡库斯就是指盗窃、抢劫和吞食一切的虔诚的高利贷者这个恶汉。他不承认自己做了恶事,并且认为谁也不会找到他,因为公牛是倒着牵回他的洞里去的,从足迹看来公牛似乎是被放走了。高利贷者正是想这样愚弄整个世界,似乎他带来了利益,他把公牛给了世界,其实他夺取了公牛并把它独吞了……所以,高利贷者和守财奴绝不是正直的人,他也作恶多端,毫无人道,他必然是一只恶狼,比一切暴君、杀人犯和强盗还凶狠,几乎和魔鬼一样可恶,但却没有被当作敌人,而是被当作朋友和市民那样受到共同的保护和亲善,可是他进行抢劫和谋杀,比一切敌人、杀人放火犯还凶狠。既然对劫路人、杀人犯和强盗应处以磔车刑或斩首,那就更应该把一切高利贷者处以磔车刑和斩首,把一切守财奴驱逐,革出教门,或斩首……”}\end{quote}

这一切描写得绘声绘色,同时也确切地抓住了旧式高利贷和一般资本的性质,揭穿了这种“幻想的损失”、这种对货币和商品“自然生出的损失的赔偿”、高利贷者会带来好处这种普遍性论调以及高利贷者的这种“虔诚的”外表:他和“别人”不同,他装出一副样子,拿别人东西好象是给别人东西,牵回去好象是放出来,等等!

\centerbox{※     ※     ※}

\begin{quote}{“拥有金银有很大的优越性,因为它提供了选择有利的购买时机的可能,它逐渐导致银行家行业的产生……银行家和旧的高利贷者不同,他贷款给富人,很少或根本不贷款给贫民。因此,他贷款时冒的风险较小,贷款条件可以较低;由于这两个原因,他就避免了民众对高利贷者的那种憎恶。”(弗·威·纽曼《政治经济学讲演集》1851年伦敦版第44页)}\end{quote}

随着高利贷和货币的发展,封建土地所有权的强制性让渡也发展起来了:

\begin{quote}{“能购买一切东西的货币的采用,以及由此而来的对贷款给土地所有者的贷出者的利益的维护,引起了为偿还债务而使土地所有权合法让渡的必要性。”(约翰·达尔林普尔《大不列颠封建所有制通史概论》1759年伦敦第4版第124页)[944]“按照托马斯·卡耳佩珀(1641年)、约瑟亚·柴尔德(1670年)、帕特森(1694年)的看法,财富取决于金银的利息率的哪怕是强制性的降低。这种看法在英国占统治地位几乎达两个世纪。”(加尼耳[《论政治经济学的各种体系》1821年巴黎第2版第1卷第58—59页])}\end{quote}

休谟同洛克相反,当他说明利息率决定于利润率时,\authornote{见本卷第1册第400—404页。——编者注}他已经看到了资本更高得多的发展。当边沁在十八世纪末写他的为高利贷辩护的著作\endnote{马克思指边沁的著作《为高利贷辩护》,1787年在伦敦出第一版,1790年出第二版,1816年出第三版。——第586、598页。}时,更是如此。

从亨利八世到安女王,在法律上都规定了降低利息率。

\begin{quote}{“在中世纪,任何一个国家都没有一般的利息率。首先对牧师有严格的规定。法庭对于借贷很少给予保障。因此,在个别场合,利息率就更高。由于货币的流通量少,而在大多数支付上必须使用现金,而且票据业务还不发达。因此,利息相差很悬殊,关于高利贷的概念差别也很大。在查理大帝时代,收取100%的利息,被认为是高利贷。1344年,在博登湖畔的琳道,本地市民收取216+(2/3)%的利息。在苏黎世,评议会规定43+(1/3)%为法定利息。在意大利,有时必须支付40%的利息,虽然从十二世纪到十四世纪,普通的利息率不超过20%。维罗那规定12+(1/2)%为法定利息。弗里德里希二世在他的命令中规定10%的利息率,但只是给犹太人规定的。他是不屑替基督徒说话的。早在十三世纪,10%已经是德国莱茵区的普通利息率了。”(休耳曼《中世纪城市》1827年波恩版第2集第55—57页)}\end{quote}

中世纪的巨额利息(只要不是从封建贵族等那里收取来的),在城市大部分是以商人和城市手工业者从农村诈骗来的巨额“让渡利润”为基础的。

除了象雅典等工商业特别发达的商业城市以外,在罗马,象在整个古代世界一样,对大土地所有者来说,高利贷不仅是剥夺小私有者即平民的手段,而且是占有他们人身的手段。

\begin{quote}{高利贷在罗马最初是自由的。十二铜表法(罗马城建立后303年)“规定货币的年利息为1%(尼布尔说是10%。)……这些法令很快就被破坏了……杜伊利乌斯(罗马城建立后398年)重新把年利率限制为1%(增长额为一盎斯)。在408年,这一利率降到1/2%。在413年,护民官格努齐乌斯主持的全民投票绝对禁止了有息贷款……在一个禁止市民从事产业、批发商业和零售商业的共和国,也禁止从事货币贸易,那是不奇怪的。这种情况延续了三百年,直到迦太基陷落。[后来允许收取不超过]12%的年利率。普通年利率是6%……查士丁尼规定的利率为4%。在图拉真时期,五盎斯的利息就是5%的法定利息……公元前146年,埃及法定的商业利息是12%”。(杜罗·德·拉·马尔《罗马人的政治经济学》1840年巴黎版第2卷第259—263页)[944]}\end{quote}

\centerbox{※     ※     ※}

[950a]关于利息,詹·威·吉尔巴特在《银行业的历史和原理》(1834年伦敦版)一书中写道:

\begin{quote}{“一个用借款来牟利的人,应该把一部分利润付给贷款人,这是不言而喻的自然公道的原则。一个人通常是通过商业来牟利的。但是在中世纪,纯粹是农业人口。在这种人口中和在封建统治下,交易是很少的,利润也是很小的。因此,在中世纪,取缔高利贷的法律是有道理的。况且,在一个农业国,一个人很少需要借钱,除非他由于不幸而陷入贫穷困苦的境地。”(第163页)“亨利八世把利息限为10%,詹姆斯一世限为8%,查理二世限为6%,安女王限为5%。”(第164—165页)“那时候,贷款人虽不是合法的垄断者,却是事实上的垄断者,所以,必须限制他们,就象限制其他的垄断者一样。在我们现代,利息率是由利润率规定的;在那个时候,利润率却是由利息率规定的。如果贷款人要商人负担很高的利息率,那末,商人就不得不提高他的商品的利润率。这样,大量货币就从买者的口袋里转到贷款人的口袋里。附加在商品上的这种追加价格,使群众减少了购买这些商品的能力和兴趣。”(第165页)}\end{quote}

十七世纪,约瑟亚·柴尔德在1669年写的《论商业和论货币利息降低所产生的利益》(译自英文,1754年阿姆斯特丹和柏林版。该书附有托马斯·卡耳佩珀1621年写的《论反对高利贷》)一书中,反驳托马斯·曼利(反驳他的论文《对货币利息的错误看法》\endnote{托马斯·曼利不是1668年在伦敦匿名出版的论文《对货币利息的错误看法》的作者,而是另一篇论文的作者,这篇论文在内容上和前者很相似,1669年在伦敦出版,标题是《利息为百分之六的高利贷。研究证明托马斯·卡耳佩珀和约·柴尔德先生对百分之六的利率的指责是不公正的》。《对货币利息的错误看法》这篇论文的作者是谁,没有查明。——第599页。}),称他为“高利贷者的卫士”。正如十七世纪英国经济学家的所有推论一样,柴尔德的推论的出发点自然是荷兰的财富,而在荷兰,利息率是低的。柴尔德认为这种低利息率是财富的原因,曼利则断定低利息率不过是财富的结果。

\begin{quote}{“要知道一个国家是穷还是富,只要问:货币利息的价格怎样?”(第74页)“他作为一帮心惊胆战的高利贷者的卫士,把大炮台建筑在我认为最不坚固的地点上……他直截了当地否认低利息率是财富的原因,而硬说这只是财富的结果。”(第120页)“当利息降低时,那些收回他们的贷款的人就不得不去购买土地〈土地价格由于购买者人数的增加而上涨〉,或者把货币投入商业。”(第133页)“当利息是6%时,谁也不会为了仅仅得到8—9%的利润而去冒险从事海运业,而得到4%或3%的利息的荷兰人,对这个利润却非常满意。”(第134页)“低利率和土地的高价格迫使商人继续不断地从事商业。”(第140页)“利率的降低能使一个民族养成节约的习惯。”(第144页)“如果使一国富裕的是商业,而压低利息又使商业扩大,那末,压低利息或限制高利贷,无疑是足以使一国致富的根本的和主要的原因。同一件事[950b]可以在一种情况下是原因,同时在另一种情况下又是结果,这种说法决不是荒谬的。”(第155页)“鸡蛋是母鸡的原因,而母鸡又是鸡蛋的原因。利息降低,可以使财富增加,而财富增加,又可以使利息进一步大大降低。通过立法也能做到降低利息。”(第156页)“我是勤劳的辩护者,而我的反对者却为懒惰和游手好闲辩护。”(第179页)}\end{quote}

在这里,柴尔德直接充当了产业资本和商业资本的卫士。[XV—950b]

