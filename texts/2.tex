\tpartnonum{《剩余价值理论》第二册}

\tchapternonum{[第八章]洛贝尔图斯先生。新的地租理论(插入部分)}

\tsectionnonum{[(1)农业中的超额剩余价值。在资本主义条件下,农业的发展比工业慢]}

[\endnote{马克思结束了属于斯密部分的篇幅很大的一章《关于生产劳动和非生产劳动的理论》,并写完了对重农学派部分具有补充性质的三章(论奈克尔,论魁奈的《经济表》和论兰盖)之后,按照自己的计划,应该着手写李嘉图部分。但是,马克思没有立即着手进行这一工作。在论兰盖这一章之后,他开始写论布雷的一章,——显然,这与马克思在《兰盖》那一章提到他打算“在以后”论述“少数几个社会主义著作家”(见本卷第1册第367页)有关。按照这个意图,马克思在手稿第X本的目录计划中,把原来写的《f》这一章的标题(这个标题紧接《(e)兰盖》一章的标题)《李嘉图》删去,改为《布雷》(见同上,第4页)。但是,论布雷的一章没有写完。后来,马克思决定把对布雷观点的分析移到论政治经济学家的无产阶级反对派这一章中去(见本卷第3册)。当马克思着手写论布雷一章时,他想从下一章《g》起开始写论李嘉图部分。但是,这一次李嘉图的名字又从目录中删掉了,作为《g》章出现的是《插入部分》,标题是《洛贝尔图斯先生。新的地租理论》。马克思是在1862年6月写作论洛贝尔图斯一章的。拉萨尔在1862年6月9日给马克思的信里,请马克思在最近把洛贝尔图斯论地租的书还给他。这显然成了马克思立即写作论洛贝尔图斯这一章的外因。但是,还有必须首先分析批判洛贝尔图斯的地租理论的重要内因。从马克思的书信中可以看出,马克思在这个时候已经把李嘉图的地租理论究竟错在哪里的问题完全弄清楚了。马克思认为,李嘉图地租理论的根本缺点之一是其中没有绝对地租这个概念。洛贝尔图斯在他给冯·基尔希曼的第三封《社会问题书简》中企图阐明这个概念。马克思在着手专门分析李嘉图的地租理论之前,在这篇《插入部分》中对洛贝尔图斯的这个企图作了详尽的分析批判。——第3页。}X—445]洛贝尔图斯先生。洛贝尔图斯《给冯·基尔希曼的第三封信:驳李嘉图的地租学说,对新的地租理论的论证》1851年柏林版。

事先应该作以下说明。如果我们说,必要工资等于10小时,那末,这句话最简单的解释就是这样:如果,平均地说,10小时劳动(也就是等于10小时的货币额)使农业短工能够购买他所必需的一切生活资料——农产品、工业品等等,那末,10小时劳动也就是非熟练劳动的平均工资。因而,这里所说的是工人一日产品中必须归他的那一部分产品的价值。这个价值最初以他所生产的商品形式,也就是作为这种商品的一定量存在,他可以用这个量——在扣除了他自己消费的那一部分(如果他消费这个商品的话)之后——换得他所需要的生活资料。因此,在这里,对于他的必要“收入”说来,工业、农业等等也具有意义,而不只是他自己生产的使用价值才具有意义。但是,商品的概念本身就包含了这一点。工人生产的是商品,不简单是产品。所以,关于这一点就不用多说了。

洛贝尔图斯先生首先研究在土地占有和资本占有还没有分离的国家中是什么情况,并且在这里得出重要的结论说:租(他所谓租,是指全部剩余价值)只等于无酬劳动,或无酬劳动借以表现的产品量。

首先要注意,洛贝尔图斯所指的只是相对剩余价值的增长,就是说,只是由于劳动生产率提高而产生的剩余价值的增长,而不是由于工作日本身延长而产生的剩余价值的增长。自然,任何绝对剩余价值,从一定意义上来说,都是相对的。劳动必须有足够的生产率,使工人不致为了维持自己的生活而用去全部时间。但是,区别也就从这里开始了。并且,如果说,最初劳动生产率很低,那末,需要也非常简单(如奴隶的情况),主人自己的生活比仆人的生活好得不太多。一个食利的寄生者出现所必需的相对劳动生产率是很低的。如果我们在劳动生产率还很低、机器和分工等等还没有采用的地方,看到有高的利润率,那末,这只有用以下几种情况来解释:或者,象在印度那样,工人的需要极低,而工人本人甚至还被压到这个极低的需要水平以下,另一方面,劳动生产率低,也就是固定资本对花费在工资上的那部分资本的比例小,换句话说,花费在劳动上的那部分资本对总资本的比例大;或者,劳动时间极度延长。后面这种情况,则发生在那些已经存在着资本主义生产方式,但是要同发达得多的国家竞争的国家(例如,奥地利等)。在这里,工资可能很低——部分是因为工人的需要比较不发展,部分是因为农产品按比较便宜的价格出卖,或者对于资本家同样可以说农产品的货币价值比较小。在劳动生产率低的情况下,在例如10小时必要劳动时间中生产出来的、用于支付工人工资的产品量也是小的。但是,如果工人不是工作12小时而是工作17小时,这就可以[为资本家]弥补低的劳动生产率。总之,不应该因为在某个国家中劳动的相对价值随该国劳动生产率的增长而下降,就认为在不同国家中工资与劳动生产率成反比。情况恰恰相反。在世界市场上一个国家同其他国家相比,生产率越高,它的工资也就越高。在英国,不仅名义工资比大陆高,实际工资也比大陆高。工人吃较多的肉,满足较多的需要。可是,这只适用于工业工人,不适用于农业工人。不过,英国的工资高的程度,没有达到英国工人的生产率超过其他国家工人的生产率的程度。

由于农业工人的平均工资低于工业工人的平均工资,地租(也就是说,土地所有权的现代形式)已经成为可能,这是撇开由土地肥力不同引起的地租差别而单单就地租的存在本身说的。因为,在这里,资本家起初按照传统(因为旧时代的租地农民变成资本家早于资本家变成租地农场主),一开始就从他的收入中拿出一部分来交给土地所有者,所以他就把工资压到水平以下,来弥补自己的损失。随着工人从农村外逃,工资必然上涨,实际上也上涨了。但是,当这种压力几乎还没有感觉出来的时候,机器等等就被采用了,农村中又形成了(相对的)人口过剩(请看英国)。尽管劳动时间没有延长,劳动生产力也没有发展,剩余价值可以由于工资压到传统水平以下而增加。凡是以资本主义方式经营农业生产的地方,实际上都是这种情况。在这一点不能靠机器做到的地方,就靠把耕地变为牧羊场来做到。因此,这里已经有了地租的可能性,[446]因为实际上农业工人的工资不等于平均工资。地租存在的这种可能性,完全不取决于产品价格——假定它等于产品价值。

地租的第二种增加,即价格不变,地租由于产品增加而增加,李嘉图也知道,但是没有加以考虑,因为他是按每夸特,而不是按每英亩计算地租的。他不会因为每夸特2先令的20夸特比每夸特2先令的10夸特多,或者比每夸特3先令的10夸特多,就说地租增加了(按这种方式,即使价格下降,地租也可以增加)。

此外,不论怎样解释地租本身,农业同工业比起来,仍然存在着重大差别:超额剩余价值的产生,在工业中是由于产品的生产较便宜,而在农业中是由于产品的生产较贵。如果1磅棉纱的平均价格等于2先令,而我能够用1先令把它生产出来,那末,为了争取销路,我一定会把它按1+(1/2)先令,或者至少低于2先令出卖。这样做甚至是绝对必需的。因为较便宜的生产是以较大规模的生产为前提的。这样一来,我就使市场商品充斥(同以前相比)。我必须出卖比以前多的东西。如果1磅棉纱只花费我1先令,那末,这是由于我现在生产比如说10000磅,而以前只生产8000磅。其所以便宜,只是因为固定资本分摊到10000磅上去了。如果我只出卖8000磅,那末,机器的损耗就会使每1磅的价格提高1/5,即20%。因此,我为了能够出卖10000磅,就以低于2先令的价格[比如说,按1+(1/2)先令]出卖。这样我仍然得到1/2先令的超额利润,也就是我的产品价值1先令(已经包括普通利润在内)的50%。无论如何,这样一来,我迫使市场价格下降,结果消费者一般按较便宜的价格得到产品。而在农业中,在类似的情况下,我按2先令出卖,因为,假定我的肥沃土地够用了,比较不肥沃的土地就不会去耕种。如果肥沃土地的数量增加了,或者较坏土地的肥力提高了,使我能够满足需求,那末,问题就不存在了。李嘉图不仅不否认这一点,并且十分明确地强调这一点。

因此,即使我们承认,土地肥力的不同不能解释地租本身,而只能解释地租的差别,下面这一规律仍然存在:在工业中,超额利润的获得通常是由于产品变得便宜,在农业中,地租的相对量的产生则不仅由于产品相对变贵(肥沃土地的产品的价格提到它的价值之上),并且由于便宜产品按较贵产品的生产费用出卖。但是,我曾经指出过(蒲鲁东)\endnote{马克思指他的反对蒲鲁东的著作《哲学的贫困》(第2章第4节,标题是《土地所有权或地租》)。见《马克思恩格斯全集》中文版第4卷第180—191页,特别是第183—186和190页。——第7页。},这仅仅是竞争的规律,它不是从“土地”产生,而是从“资本主义生产”本身产生的。

其次,李嘉图在另外一点上也是正确的,不过他按照经济学家的习惯,把历史现象变成永恒的规律。这个历史现象就是工业(真正资产阶级的生产部门)比农业发展快。农业生产率提高了,但是比不上工业生产率提高的程度。在工业生产率提高到10倍的地方,农业生产率或许提高到2倍。因此,农业生产率,尽管绝对地说提高了,相对地说却降低了。这一点仅仅证明资产阶级生产的极其古怪的发展和它所固有的矛盾,但是并不妨碍下述论点的正确性:农业生产率在相对地降低,因而同工业品相比,农产品的价值以及地租都在提高。随着资本主义生产的发展,农业劳动同工业劳动相比生产率相对地降低,这只是意味着,农业生产率不是以同样速度和同样程度发展。

假定生产部门A与生产部门B之比是1∶1。最初,农业生产率较高,因为在农业中,参加生产的不仅有自然力,而且有自然本身创造的机器;单个劳动者一开始就用这种机器进行劳动。因此,在古代和中世纪,农产品比工业品相对地说便宜得多,这一点从两种产品在平均工资中所占比例已经可以看出来(见威德的著作)\endnote{指约·威德《中等阶级和工人阶级的历史》1833年伦敦版。——第7页。}。

假定1∶1还表示两个生产部门的生产率。如果现在生产部门A=10,也就是说,它的生产率增大到10倍,而生产部门B=3,也就是说,只增大到3倍,那末,两个生产部门之比,以前是1∶1,现在是10∶3或1∶(3/10)。相对地说,生产部门B的生产率降低了7/10,虽然绝对地说它增加到3倍。对于最高的地租来说,这——同工业对比——就好比它由于最坏土地的肥力减低了7/10而提高一样。

诚然,从这里决不能象李嘉图所想的那样得出结论说,利润率下降是因为工资由于农产品相对变贵[447]而提高了,——要知道,平均工资不决定于加入该工资的产品的相对价值,而决定于这些产品的绝对价值。但是从这里确实可以得出结论说,利润率(其实,是剩余价值率)不是按加工工业生产力提高的比例提高的,并且这是由于农业(不是土地)的生产率比较低的缘故。这是无庸置疑的。必要劳动时间的减少,同工业的进步相比,是微不足道的。这从俄国等这样的国家竟能在农产品市场上打击英国这一点就表现出来了。较富国家的货币价值较小(就是说,对于较富国家说来,货币的生产费用相对地小),在这里不起任何作用。因为问题恰恰在于,为什么这种情况在较富国家同较穷国家的竞争中,不影响它们的工业品,而只影响它们的农产品。(可是,这并不证明穷国生产比较便宜,它们的农业劳动生产率高。即使是美国,不久以前统计材料证明,按既定价格出卖的小麦总量的确增加了,但这不是因为每一英亩出产小麦多了,而是因为种小麦的亩数多了。有些国家拥有大量土地,在大地段上进行粗放耕作,用同量劳动提供的产品,就绝对量来说,大于比较发达的国家在小得多的地段上提供的产品,但是,不能说,前者的土地的生产率高于后者。)

转而耕种生产率较低的土地,不一定证明农业生产率下降。相反,它可以证明农业生产率提高;耕种贫瘠土地,不仅因为农产品的价格已经提高到足以补偿投入土地的资本的程度,而且因为生产资料的发展已经达到使不生产的土地变成“生产的”土地,使它能够不仅支付普通利润,并且还支付地租的地步。对于生产力的一定发展阶段说来是肥沃的土地,对于生产力的较低发展阶段说来,就是贫瘠的土地。

在农业中,绝对延长劳动时间,也就是说,增加绝对剩余价值,只有很小的可能。在农业中,劳动不能借瓦斯照明等等。当然,在夏季和春季可以早起。但是,这一点,由于冬季昼短,一般干活较少,就被抵销了。因此,就这方面说,工业中绝对剩余价值比较大,除非法律上强制规定正常工作日。农业中创造的剩余价值量比较小的第二个原因是,农产品长时间滞留在生产过程中而没有新的劳动加在它上面。但是,从另一方面来说,除了如畜牧业、养羊业等绝对排挤人口的一些农业部门以外,甚至在最先进的大农业中,使用的人数对使用的不变资本的比例,总是比工业,至少比主要工业部门大得多。因此,从这一方面说,即使由于上述原因,农业中的剩余价值量小于工业中使用相同人数时得到的剩余价值量(这种情况又部分地由于农业工人的工资降到平均水平之下而被抵销),农业的利润率仍可能大于工业的利润率。但是,如果说在农业中存在着提高利润率(不是暂时提高,而是同工业相比,平均地提高)的某些原因(上述这些,我们只是大略谈了一下),那末,单单土地所有者的存在这一事实本身,就使这种超额利润不是进入一般利润率的平均化过程,而是固定下来,落到土地所有者手中。

\tsectionnonum{[(2)利润率和剩余价值率的关系。作为农业中的不变资本要素的农业原料价值]}

考察洛贝尔图斯的理论时要回答的问题,总的说来,归结如下。

预付资本的一般形式是:

\todo{}

不变资本的两个要素,一般地说,就是劳动资料和劳动对象。劳动对象不一定是商品,不一定是劳动产品。因此,劳动对象作为劳动过程的要素虽然永远存在,但作为资本的要素可能不存在。土地是土地耕种者的劳动对象\endnote{在手稿中这里是“Rohmaterial”(“原料”)一词。马克思在他的1861—1863年手稿的所有其他地方都是在较狭义上应用“原料”这个术语,象《资本论》第一卷第五章所表述的那样,是指“已经通过劳动而发生变化”的劳动对象(见《马克思恩格斯全集》中文版第23卷第203页)。而在这里,从前后文看来,谈的是最广义的劳动对象,即首先指的不是劳动产品而是自然给予的劳动对象,所以可以设想,“原料”一词是笔误,应当用“Arbeitsgegenstand”(“劳动对象”)代替它。因此,这里就改译为“劳动对象”。——第12页。},煤矿是煤炭业者的劳动对象,水域是渔夫的劳动对象,森林是猎人的劳动对象。但是,当上述劳动过程三要素也作为资本三要素出现,就是说,它们三者都是商品,都是一种具有交换价值并且表现为劳动产品的使用价值的时候,资本具有最完整的形式。在这种情况下,这三个要素也就都进入价值形成过程,虽然机器不是按它进入劳动过程多少,而只是按它被劳动过程消耗多少进入价值形成过程。

现在的问题是:缺少其中一个要素,能否使缺少这个要素的生产部门的利润率(不是剩余价值率)提高呢?一般地说,这个问题可由下面的公式本身来回答:

利润率等于剩余价值和预付资本总额之比。

全部研究是在这种假定下进行的:剩余价值率不变,就是说,产品价值在资本家和雇佣工人之间的分配不变。

[448]剩余价值率=m/v;利润率=m/(c+v)。因为m’即剩余价值率是既定的,v也就既定,而且m/v被假定为常量。所以,m/(c+v)的量只有在c+v变化时才变化,又因为v是既定的,所以m/(c+v)只有在c减小时增大,或者在c增大时减小。并且,m/(c+v)的增大或减小,不是同c∶v成比例,而是同c∶(c+v)之和成比例。假设c=零,那末m/(c+v)=m/v。换句话说,在这种情况下,利润率等于剩余价值率,而这就是利润率不能超越的极限,因为任何计算方法都不能改变m和v的量。如果v=100,m=50,那末m/v=50/100=1/2=50%。现在如果加上不变资本100,那末利润率=50/100+100=50/200=1/4=25%。利润率减了一半。如果把150加到100上,那末利润率=50/(150+100)=50/250=1/5=20%。在第一种情况下,总资本=v=可变资本,因而利润率=剩余价值率。在第二种情况下,总资本=2×v,因而利润率只有剩余价值率的一半。在第三种情况下,总资本=[2+(1/2)]×100=[2+(1/2)]×v=(5/2)×v。在这里,v只是总资本的2/5。剩余价值=v的1/2,100的1/2,因此只是总资本的2/5的1/2,也就是说,只是总资本的2/10。(250/10=25,而250的2/10=50。)而2/10就是20%[也就是说,利润率是剩余价值率的2/5]。

因此,这些是一开始就确定了的。如果v和m/v不变,那末c这个量究竟由哪些部分构成,是完全没有关系的。如果c是一定量,例如等于100,那末,不论c分成50是原料和50是机器,或者10是原料和90是机器,或者0是原料和100是机器,或者反过来,都完全没有关系,因为决定利润率的是m/(c+v)这个比例;构成c的各个生产要素,作为价值部分,同整个c之比究竟如何,在这里是没有关系的。例如,在煤的生产中,可以把原料(本身又用作辅助材料的煤除外)看作零,而假定全部不变资本都是由机器(包括建筑物、劳动工具在内)构成。另一方面,在缝纫业者那里,可以假定机器等于零(就是说,在大缝纫业者还没有应用缝纫机的地方,另一方面,象目前伦敦有一部分做法那样,甚至把建筑物都节省掉,让自己的工人在家里劳动;这是件新鲜事:第二种分工又以第一种分工形式出现\endnote{马克思在1861—1863年手稿第IV本中(第149页及以下各页),把社会上彼此独立的商品生产者之间的分工称为“第一种分工”,把资本主义企业内部,特别是手工业工场内部的分工称为“第二种分工”。参看《马克思恩格斯全集》中文版第23卷第389—398页。——第14页。}),于是,在这个缝纫业者那里,全部不变资本都归结为原料。如果煤炭业者把1000花费在机器上,把1000花费在雇佣劳动上,缝纫业者则把1000花费在原料上,把1000花费在雇佣劳动上,那末,在剩余价值率相等时,这两种情况下的利润率也相等。我们假定,剩余价值=20%,在这两种情况下,利润率就都=10%,即200/2000=2/20=1/10=10%。因此,如果说c的组成部分即原料和机器之间的比例,对利润率有影响,那只有在下列两种情况下才可能:第一,如果c的绝对量由于这个比例发生变化而有了变化;第二,如果v的量由于c的组成部分之间的这个比例而有了变化。这里,必定是生产本身发生了有机变化,而不能归结为这样一个简单的同义反复:如果c的一定部分在总数中占较小的部分,那末c的另一部分在总数中一定占较大的部分。

在一个英国租地农场主的实际开支中,工资=1690镑,肥料=686镑,种子=150镑,牛饲料=100镑。因而用于“原料”的是936镑,比工资的一半还多。(见弗·威·纽曼《政治经济学讲演集》1851年伦敦版第166页)

\begin{quote}{“在弗兰德〈比利时〉,这一带从荷兰进口肥料和干草〈用于种植亚麻等。作为交换,这一带出口亚麻和亚麻籽等〉……荷兰各城市的垃圾成了交易品,经常以高价卖给比利时……从安特卫普溯些耳德河而上约20英里,就可以看到从荷兰运来的肥料的堆栈。肥料贸易由一帮资本家用荷兰船只经营”等等。\authornote{本卷引文中凡是尖括号〈〉和花括号{}内的话都是马克思加的。——译者注}(班菲尔德的著作)\endnote{马克思在这里引用的书是:托·查·班菲尔德的《产业组织》1848年伦敦第2版第40、42页。——第15页。}}\end{quote}

既然连普通粪便这样的肥料都成了交易品,骨粉、鸟粪、炭酸钾等就更不用说了。这里,生产要素用货币来估价,不只是生产中的形式上的变化。为了提高生产率,把新的物质送到地里,而把地里旧的物质卖出去。这也不单纯是资本主义生产方式和它以前的生产方式之间的形式上的差别。随着人们认识到换种的重要性,连种子交易也越来越重要了。因此,就真正的农业来说,如果说没有“原料”——并且是作为商品的原料——加入农业(不论是农业自己把它再生产出来,还是把它作为商品买进、从外面取得,都一样),那是可笑的。如果说机器制造业者[449]自己使用的机器不作为价值要素加入他的资本,那是同样可笑的。

一个年年自己生产自己的生产要素(种子、肥料等等),并且自己全家吃掉自己的一部分谷物的德国农民,只是为了购置少数农具和支付工资才(为生产本身)支出货币。假定他的全部支出的价值等于100[其中50用货币支付]。他以实物形式消费产品的一半([这里也包括实物形式的]生产费用)。他把另一半出卖,比如说得到100。在这种情况下,他的总的[货币]收入等于100。如果他按资本50来计算[他的货币形式的纯收入],那就是100%[利润]。如果现在[作为利润得到的]50中有1/3交地租,1/3交税款(合计33+(1/3)),他自己留下16+(2/3),按50计算,就是[33+(1/3)]%。但是,实际上他只得到[所支出的100的][16+(2/3)]%。这个农民完全算错了,自己骗了自己。在资本主义租地农场主那里是不会有这种错误计算的。

马蒂约·德·东巴尔《农业年鉴》1828年巴黎版第四分册说到,按照对分制租佃契约(例如贝里省),

\begin{quote}{“土地所有者提供土地、建筑物,通常还提供全部或一部分牲畜和生产所必需的农具;租地农民方面提供自己的劳动,此外不提供或几乎不提供什么。土地的产品拿来对分。”(第301页)“对分制租地农民通常是贫困不堪的人。”(第302页)“如果对分制租地农民预付1000法郎,增加总产品1500法郎〈即总利润500法郎〉,他必须同土地所有者对分,因而只得到750法郎,也就是说,自己的预付资本损失250法郎。”(第304页)“在以前的耕作制下,生产支出即生产费用几乎完全以实物形式从产品本身取得,以供饲养牲畜并供土地耕种者和他的家庭消费;几乎完全没有现金支出。只有这种情况才会使人相信,土地所有者和租地农民可以分享没有在生产中消费掉的全部收成;但是,这种做法只适用于这种农业,即处于低水平的农业;人们一旦想要在农业中实行某种改良,就会立即发觉,只有预先付出一笔款项才可能做到,而这笔款项必须从总产品中扣出,供下年生产之用。因此,土地所有者和租地农民对总产品的任何分成,对任何改良都是不可克服的障碍。”(第307页)}\end{quote}

\tsectionnonum{[(3)农业中的价值和平均价格。绝对地租]}

\tsubsectionnonum{[(a)工业中利润率的平均化]}

洛\endnote{“平均价格”(Durchschnittspreis)这一术语,马克思这里是指“生产价格”,即生产费用(c+v)加平均利润。“平均价格”这一术语本身说明,这里所指的,正如马克思后来在本册第359页上所解释的那样,是“一个相当长的时期内的平均市场价格,或者说,市场价格所趋向的中心”。马克思用的这个术语最初见于本卷第一册第76页。——第16页。}贝尔图斯先生对于竞争调节正常利润,或平均利润,或一般利润率,总的说来,似乎是这样想的:竞争使商品还原为它们的实际价值,就是说,竞争调节商品价格之间的比例,使物化在各种商品中的劳动时间的相当量,以货币或其他某种价值尺度表现出来。当然,这种调节,不是使这种或者那种商品的价格在任何时候、任何一定时刻都等于或都必定等于它的价值。[照洛贝尔图斯的想法,这种调节是这样进行的。]例如,商品A的价格提高到它的价值以上,并且,这种价格在一定时间内保持这个高度或者甚至继续提高。资本家A的利润因而提高到平均利润以上,因为他不仅占有他自己的“无酬”劳动时间,而且占有其他资本家“生产”的无酬劳动时间的一部分。与此相应——在其他商品的货币价格不变的情况下——必然有这个或那个生产领域的利润下降。如果该商品作为一般生活资料加入工人的消费,这就会使其他一切部门的利润率下降;如果该商品成为不变资本的组成部分,这就会使那些以该商品作为不变资本要素的生产部门的利润率下降。

最后,可能还有一种情况,即这种商品既不作为要素加入任何不变资本,也不是工人的必要生活资料(因为,工人随自己的意可买可不买的那些商品,工人是作为一般消费者而不是作为工人去消费的),而是消费品,一般个人消费品。如果这种商品作为消费品加入工业资本家本人的消费,那末,它的价格提高决不影响剩余价值量或剩余价值率。但是,如果资本家想要保持他原来的消费水平,那末,利润(剩余价值)中被他用于个人消费的部分,同被他用于工业再生产的部分相比就会增加。这样,用于再生产的部分就会减少。因此,由于A的价格提高,或者说,A的利润提高到平均利润率以上,经过一定时期(这个时间也是由再生产决定的),B、C等的利润量就会降低。如果商品A完全加入非工业资本家的消费,那末,同以前相比,这些非工业资本家消费商品A多了,而消费商品B、C等少了。对商品B、C等的需求会减少;它们的价格将下降,而在这种情况下,A的价格的提高,或者说,A的利润提高到平均利润率以上,会通过压低B、C等的货币价格,使B、C等的利润降到平均利润率以下(这同前面所举的情况不同,在那里,B、C等的货币价格是[450]保持不变的)。利润率降到普通水平以下的B、C等领域的资本,将离开它们自己的生产领域,转入A生产领域;在市场上不断重新出现的一部分资本,尤其是这样,这种资本当然会力求挤进更加有利可图的A生产领域。由于这个原因,商品A的价格,在若干时间以后,将会降到它的价值以下,并且在或长或短的一段时间内继续下降,直到相反的运动重新开始为止。在B、C等领域中,将发生相反的现象,部分由于商品B、C等的供给因资本流出而减少,就是说,由于这些领域本身发生了有机变化,部分则由于A领域中过去发生的变化现在以相反方向作用于B、C等领域。

顺便指出:在刚才描述的运动中,虽然商品B、C等的货币价格(假定货币的价值不变)提高到商品B、C等的价值以上,因而B、C等的利润率也提高到一般利润率以上,但是,商品B、C等的货币价格,有可能再也达不到它们原来的水平。改良、发明、生产资料的更大节约等等,不是在价格提高到自己的平均水平以上的时候运用,而是在价格降到这个水平以下、因而利润降到普通利润率以下的时候运用。因此,在商品B、C等的价格下降的时期,它们的实际价值可能下降,换句话说,为生产这些商品所需的最低限度的劳动时间可能下降。在这种情况下,只有当商品的价格超过它的价值的程度,等于表现它的新价值的价格和表现它的较高的原有价值的价格之间的差额的时候,商品才能恢复它以前的货币价格。在这种情况下,商品价格将会通过影响供给、影响生产费用来改变商品的价值。

上述运动的结果就是这样:如果就商品价格在商品价值上下波动的平均数来看,或者说,如果就上下波动平均化的时期——不断反复出现的时期——来看,那末平均价格等于价值,因而一定生产领域的平均利润也等于一般利润率;因为,在这个领域中,虽然随着价格的涨落,或者,在价格不变的情况下,随着生产费用的增减,利润提高到原来的利润率以上或降到原来的利润率以下,但是就一个时期平均起来,商品是按自己的价值出卖的,因而,赚到的利润等于一般利润率。这就是亚·斯密的观点,尤其是李嘉图的观点,因为后者更明确地坚持真正的价值概念。洛贝尔图斯先生也从他们那里接受了这个观点。可是,这个观点是错误的。

资本的竞争究竟产生什么结果呢?在任何一个平均化的时期中,商品的平均价格是这样的:这种价格向每个领域的商品生产者提供同样的利润率,譬如10%。这又是什么意思呢?这就是说,每种商品的价格,比这种商品使资本家花费的、资本家为生产它而支出的生产费用,高出十分之一。一般说来,这不过是说,等量资本提供等量利润,每种商品的价格,比在这种商品上预付、消费或者体现的资本的价格,高出十分之一。但是,如果以为资本按照自己的大小,在不同的领域中生产相同的剩余价值,那是完全错误的。{这里我们完全不考虑,一个资本家是否比另一个资本家强迫工人劳动更长的时间;我们在这里假定,在一切领域中,绝对工作日是一样的。绝对工作日的差别,一部分由不同长度的工作日的劳动强度等抵销了,一部分不过表现为强求的超额利润、例外等。}即使假定绝对工作日在一切领域中是一样的,就是说,假定剩余价值率是既定的,这种说法也是错误的。

在资本量相等的情况下,——并且在上述假定的条件下,——这些资本所生产的剩余价值量依下述情况不同而不同:第一,资本的有机组成部分即可变资本和不变资本之间的比例;第二,资本的周转时间,因为这个时间取决于固定资本和流动资本之间的比例,以及不同种类固定资本的不同的再生产期间;第三,和劳动时间本身长度不同的、真正生产期间的长度,\endnote{马克思在他的1857—1858年手稿中谈到关于特别是在农业中存在的生产时间和劳动时间的区别,以及与此有关的资本主义在农业中发展的特点(见卡·马克思《政治经济学批判大纲》1939年莫斯科版第560—562页)。生产期间即生产时间(除了劳动时间以外,还包括劳动对象仅仅接受自然界的自然过程的作用的时间),这个概念马克思在《资本论》第二卷第二篇第十三章作了详细的阐述(见《马克思恩格斯全集》中文版第24卷第2篇第13章)。——第19页。}这个长度也决定生产期间和流通期间的比例的重大差别。(上述第一个比例,即不变资本和可变资本之间的比例本身,可以由非常不同的原因产生。例如,它可以仅仅是形式上的,——当一个生产领域加工的原料比另一个生产领域加工的原料贵的时候,就是这样,——或者,它可以由不同的劳动生产率产生,等等。)

因此,如果商品按其价值出卖,或者说,如果商品的平均价格等于其价值,那末,利润率在不同的生产领域中必定是完全不同的;在一种情况下,它会是50%,在其他情况下,它会是40%、30%、20%、10%等。例如,拿A领域一年的商品总量来看,它的价值等于预付在它上面的资本加上它所包含的无酬劳动。在B、C领域中也是一样。但是因为A、B、C包含的无酬劳动量不同,例如,A包含的大于B包含的,B包含的大于C包含的,商品A给自己的生产者提供比方说3M(M是剩余价值),商品B提供2M,商品C提供M。因为利润率决定于剩余价值和预付资本之比,而预付资本,根据假定,在A、B、C等领域中是一样的,所以,[451]如果C代表预付资本,那末,这些领域的不同的利润率就等于(3M)/C、(2M)/C、M/C。因此,资本的竞争要使利润率平均化,在上述例子中,只有使A、B、C领域的利润率等于(2M)/C、(2M)/C、(2M)/C。这样,A将会把它的商品卖得比它的价值便宜1M,而C把它的商品卖得比它的价值贵1M。A领域的平均价格将低于商品A的价值,C领域的平均价格将高于商品C的价值。

B的情况说明,商品的平均价格同价值一致,确实可能发生。这发生在B领域本身生产的剩余价值等于平均利润的时候,也就是说,这时候,在这个领域中,资本的不同部分的相互比例,等于(如果把资本的总额,资本家阶级的全部资本,当作一个量,按这个量来计算全部剩余价值,不问这些剩余价值由总资本的哪个领域生产出来)总资本不同部分的相互比例。在这个总资本中,周转时间等等也平均化了;这整个资本按例如一年周转一次计算,等等。于是,这个总资本的每个部分,实际上就会根据自己的大小,按比例来瓜分全部剩余价值,各自取得全部剩余价值的相应部分。既然每一单个资本被看作这个总资本的股东,那末由此可以得出结论:第一,单个资本的利润率同其他任何资本的利润率是一样的,等量资本提供等量利润;第二,这是从第一点自然得出的,就是,利润量取决于资本的大小,取决于资本家在这个总资本中拥有的股数。资本的竞争力图把每个资本作为总资本的一部分来对待,并且根据这一点来调节每个资本取得剩余价值的份额,也就是说,调节利润。竞争通过它的平均化作用或多或少达到了这个目的。(竞争在个别领域中遇到特殊障碍的原因不应在这里研究。)直截了当地说,这无非是资本家们努力(而这种努力就是竞争)把他们从工人阶级身上榨取的全部无酬劳动量(或这个劳动量的产品)在他们相互之间进行分配,而且这种分配不是根据每一个特殊资本直接生产多少剩余劳动,而是根据:第一,这个特殊资本在总资本中占多大部分;第二,总资本本身生产的剩余劳动总量。资本家们既作为同伙又作为敌手来瓜分赃物——他们所占有的别人劳动,于是他们每个人占有的无酬劳动,平均说来,同其他任何一个资本家占有的一样多。\endnote{马克思在《资本论》第三卷中论证了资本家们既作为竞争的敌手又作为“同伙”这个特点。在利润率平均化的过程中,“每一单个资本家,同每一个特殊生产部门的所有资本家总体一样,参与总资本对全体工人阶级的剥削,并参与决定这个剥削的程度”。(见马克思《资本论》第3卷第10章)马克思在研究了这个过程后写道:“……我们在这里得到一个象数学一样精确的证明:为什么资本家在他们的竞争中表现出彼此都是虚伪的兄弟,但面对着整个工人阶级却结成真正的共济会团体。”(同上)——第21页。}

竞争是通过调节平均价格来实现这种平均化的。但是,这种平均价格本身,使商品高于或低于它的价值,以致该商品不能比其他任何商品提供较大的利润率。因此,认为资本竞争是通过使商品价格等于价值来确立一般利润率的说法,是错误的。相反,竞争正是通过以下途径来确立一般利润率的:它把商品的价值转化为平均价格,在平均价格中,一种商品的剩余价值的一部分转到另一种商品上,等等。商品的价值等于商品包含的有酬劳动和无酬劳动的量。商品的平均价格等于商品包含的有酬劳动(物化劳动或活劳动)量加无酬劳动的平均份额,这个平均份额不取决于它原来是否如数包含在这种商品本身,换句话说,不取决于原来包含在该商品的价值中的无酬劳动是大还是小。

\tsubsectionnonum{[(b)地租问题的提法]}

可能,——这一点我留到以后研究,不属于本册\endnote{“本册”,马克思是指《论资本》那一册。他在1858—1862年间想把他的全部经济著作分为六册,《论资本》是其中第一册也是最基本的一册(见《马克思恩格斯全集》中文版第13卷第7页)。——第22页。}研究范围,——某些生产领域是在这样的环境下工作的,这种环境阻碍它们的价值转化为上述意义的平均价格,也就是说,不让竞争取得这种胜利。如果,比如说,农业地租或矿山地租就是这种情况(有一些租,完全只能用垄断来说明,例如伦巴第和亚洲某些地区的水租;又如实际是地产租的房租),那末,从这里得出的结论是,当所有工业资本的产品的价格提高或者降低到平均价格的水平的时候,农产品的价格却始终等于自己的价值,而这个价值将高于平均价格。这里是否存在着一种障碍,使这个生产领域生产的剩余价值中被当作本领域财产来占有的部分,大于按照竞争规律应得的部分,大于按照投在这个生产部门的资本的份额应得的部分?

我们假定有这样一些工业资本,它们不是暂时地,而是由于它们的生产领域的性质,[452]比其他生产领域中同量工业资本多生产10%,或20%,或30%的剩余价值。我说,如果这些资本能够在竞争面前保住这种超额剩余价值,不让它参加决定一般利润率的总计算(分配),那末,在这种情况下,在这些资本发挥作用的各个生产领域中,就会有两个不同的获利者,一个取得一般利润率,另一个取得该领域所独有的超额部分。每一个资本家,为了有可能把他的资本投入该领域,就要对这个享受特权的人支付、交付这个超额部分,而他自己同其他任何一个在相同条件下经营的资本家一样,为自己保住一般利润率。既然农业中的情况是这样,那末,这里,剩余价值分解为利润和地租,完全不是表明劳动本身在这里比在加工工业中“具有更高的生产率”(从生产剩余价值的意义上来说);因此,把任何创造奇迹的力量归于土地是毫无理由的,并且,这本身就是可笑的,因为价值等于劳动,从而,剩余价值决不能等于土地。(诚然,相对剩余价值可能取决于土地的自然肥力,但是,无论如何不能由此得出土地产品价格较高的结论。倒是恰恰相反。)也不必找李嘉图的理论帮忙,这个理论本身令人讨厌地同马尔萨斯废话联结在一起,得出可鄙的结论,特别是,这个理论同我的相对剩余价值学说,即使在理论上不是对立的,在实践上也把它的意义抹去了一大部分。

在李嘉图那里,问题的全部要点如下:

地租(例如在农业中)——照他的假定——在农业以资本主义方式经营、有租地农场主存在的地方,只能是超过一般利润的余额。土地所有者取得的地租是否真正是这种资产阶级经济学意义上的地租,是完全没有关系的。它可能纯粹是工资的扣除部分(参看爱尔兰的情况),也可能部分地靠租地农场主的利润被压到利润的平均水平以下而得到。这一切可能的情况在这里是绝对无关紧要的。地租之所以在资本主义制度下成为剩余价值的一种特殊的、具有特征的形式,只在于它是超过(一般)利润的余额。

但是,这怎么可能呢?商品小麦,同其他任何商品一样,按它的价值出卖,就是说,按照它所包含的劳动时间同其他商品交换。{这是第一个错误的前提,它人为地使问题变得更加困难了。商品按其价值交换只是例外。商品的平均价格是按另外的方式决定的。见上述。}种植小麦的租地农场主同其他所有资本家一样,赚得同样的利润。这证明,他同其他所有资本家一样,占有自己工人的无酬劳动时间。在这种情况下,究竟还从哪里产生地租呢?地租无非代表劳动时间。为什么剩余劳动在工业中只等于利润,而在农业中却应该分解为利润和地租呢?如果农业中的利润等于其他各个生产领域的利润,这怎么可能呢?{李嘉图的不正确的利润观点,以及他把利润和剩余价值直接混淆起来,在这里也是有害的。这些使他考察问题更困难了。}

李嘉图解决这个困难的办法是:假定困难在原则上是不存在的。{确实,这是在原则上解决困难的唯一方式。不过,这可以有两种办法。或者证明,与一定原则矛盾的现象只是某种表面的东西,只是从事物本身发展中产生出来的假象。或者象李嘉图所作的那样,在某一点上抛开困难,然后把这一点作为出发点,从这里出发,可以说明造成困难的现象在另一点上存在。}

李嘉图假定这样一种情况,那就是,租地农场主的资本{不论是指个别农场的不付地租的那部分资本,或者是指农场的不付地租的那部分土地;总之,这里是指投入农业而不付地租的资本}同其他任何一个资本家的资本一样,只提供利润。这个假定甚至是李嘉图的出发点,它也可以这样表达:

最初,租地农场主的资本只提供利润{但是,这个伪历史形式是无关重要的,它是所有资产阶级经济学家在编造其他类似“规律”时所共有的},这笔资本不支付地租。租地农场主的资本同其他任何产业资本没有区别。只因为对于谷物的需求增加了,结果,和其他生产部门不同,不得不向“比较不”肥沃的土地找出路,这才产生地租。由于生活资料涨价,租地农场主(假定的最初的租地农场主)同其他任何产业资本家一样受损失,因为租地农场主也不得不给自己的工人多支付报酬。但是,租地农场主由于自己的商品的价格提高到它的价值以上,占了便宜。他所以占便宜,第一,因为加入他的不变资本的其他商品,同他的商品比起来,相对价值下降了,于是他按比较便宜的价格购买这些商品;第二,因为他以较贵的商品形式占有他的剩余价值。这样一来,这个租地农场主的利润就提高到已经降低的平均利润率以上。于是,另一个资本家去经营较坏的II等地,这块土地,在这个利润率较低的情况下,能够按I的产品的价格提供产品,甚至还稍便宜一些。不管怎样,我们现在[453]在II等地上又有了使剩余价值仅仅归结为利润的正常关系,然而因此我们已经把I的地租解释了,也就是说,因为存在着两种生产价格,而II的生产价格同时就是I的市场价格。这就完全象在比较有利的条件下生产出来的工业品提供暂时的超额利润一样。除利润外还包含地租的小麦价格,虽然也是仅仅由物化劳动构成,虽然也等于小麦的价值,但是,它不等于小麦本身包含的价值,而等于II上种植的小麦的价值。两种市场价格并存是不可能的。{李嘉图因为利润率下降而引进租地农场主II,斯特林则由于谷物价格使工资下降而不是提高,让租地农场主II登场。这种下降的工资使租地农场主II能够以原来的利润率经营II等地,虽然这块土地比较不肥沃。\endnote{帕·詹·斯特林《贸易的哲学,或利润和价格理论概要》1846年爱丁堡版第209—210页。——第25、525页。}}地租的存在既然这样来解释,其余也就不难推论了。地租的差别同肥力的差别相适应等等,自然还是正确的。但是,肥力的差别本身并不证明必须去耕种越来越坏的土地。

因此,李嘉图的理论就是这样。因为给租地农场主I提供超额利润的上涨了的小麦价格,给租地农场主II提供的甚至不是原来的利润率,而是较低的利润率,所以,很清楚,II的产品包含的价值大于I的产品,或者说,II的产品是较多劳动时间的产品,它包含较多的劳动量;因此,为了生产同样多的产品,例如一夸特小麦,就要花费较多的劳动时间。地租的增长,将同土地肥力的这种不断降低的情况相适应,或者说,将同生产例如一夸特小麦所必需花费的劳动量的增加相适应。当然,如果增加的只是支付地租的夸特数,李嘉图是不会说地租“增长”的,在李嘉图看来,只有同样一夸特的价格增长,例如从30先令涨到60先令,地租才是增长了。诚然,李嘉图有时忘记了,地租的绝对量在地租率下降的情况下可能增长,正如利润的绝对量在利润率下降的情况下可能增长一样。

另外一些人(例如凯里)想绕过这个困难,他们干脆用另一种方式否认这个困难的存在。据说,地租只是以前投入土地的资本的利息。\endnote{马克思在后面(见本册第152—153、157和179页)以及《资本论》第三卷(第37章和第46章)中,都谈到凯里把地租看成投入农业的资本的利息这种庸俗见解。——第26页。}所以,地租也只是利润的一种形式。因此,这里,地租的存在被否定了,从而地租实际上就被解释掉了。

另外一些人,例如布坎南,把地租看成纯粹是垄断的后果。再看霍普金斯的著作。\endnote{马克思在后面,在本册第178、377、439—440页谈到布坎南关于农产品垄断价格的见解。马克思在本册第147—153页分析了霍普金斯的地租观点。——第26页。}这里,地租完全归结为超过价值的附加部分。

在奥普戴克先生那里,土地所有权或地租是“资本价值的合法反映”\endnote{马克思引用的著作是:乔·奥普戴克《论政治经济学》1851年纽约版第60页。——第26页。},这是美国佬所特有的。\authornote{[486}{奥普戴克把土地所有权称为“资本价值的合法反映”,那末,资本同样是“别人劳动的合法反映”。}[486]]

在李嘉图那里,由于两个错误的假定,增加了研究的困难。{确实,李嘉图不是地租理论的发明者。威斯特和马尔萨斯在李嘉图之前已经出版了自己关于地租理论的著作。然而,来源是安德森。但是,李嘉图的特点是他的地租理论和他的价值理论的相互联系(虽然在威斯特的著作中也不是完全没有真实联系)。马尔萨斯后来同李嘉图在地租问题上的论战证明,马尔萨斯甚至并不理解他从安德森那里借用的理论。}如果从商品价值决定于生产商品所必需的劳动时间(以及价值无非是物化了的社会劳动时间)这个正确的原则出发,那末,自然得出结论说,商品的平均价格决定于生产商品所必需的劳动时间。如果平均价格等于价值这一点得到证明,这个结论就会是正确的。可是,我证明情况恰好相反:正因为商品价值决定于劳动时间,商品平均价格决不能等于商品价值(只有一个情况除外,就是某一个生产领域的所谓个别利润率,即由这个生产领域本身生产出来的剩余价值决定的利润,等于总资本的平均利润率),虽然平均价格这个规定只是从由劳动时间决定的价值引伸出来的。

由此首先得出一个结论:即使有些商品的平均价格(如果撇开不变资本的价值不说)只分解为工资和利润,而工资和利润又处于正常水平,是平均工资和平均利润,这种商品,也可能高于或者低于它们自己的价值出卖。因此,一种商品的剩余价值只表现为正常利润的项目这个情况并不足以证明,这种商品就是按它的价值出卖,同样,商品除利润外[454]还提供地租这个情况也不足以证明,这种商品是高于它的内在价值出卖的。既然确定,一种商品所实现的资本的平均利润率即一般利润率,可能低于商品自己的、由商品中实际包含的剩余价值决定的利润率,那就可以由此得出结论:如果一个特殊生产领域的商品,除了提供这种平均利润率以外,还提供第二个剩余价值量,这种剩余价值量具有特殊的名称,比如叫作地租,那末,这并不使利润加地租,即利润与地租之和,一定要大于这个商品本身所包含的剩余价值。既然[资本家所得的]利润可能小于该商品的内在剩余价值,也就是说,小于该商品所包含的无酬劳动量,那末,利润加地租也就不一定要大于商品的内在剩余价值。

的确,剩下还要说明的是,为什么这类现象发生在一个不同于其他生产领域的特殊生产领域。但是,解决这个问题已经非常容易了。提供地租的这种商品和其他一切商品的不同之处在于一部分其他商品的平均价格高于它们的内在价值,但其程度只是使它们的利润率提到一般利润率水平;而另一部分其他商品的平均价格低于它们的内在价值,但其程度只是使它们的利润率降到一般利润率水平;最后,第三部分其他商品的平均价格等于它们的内在价值,但这只是因为它们在按它们的内在价值出卖时提供一般利润率。提供地租的商品同所有这三种情况都不相同。在任何情况下,这种商品出卖的价格都是这样的:这种商品所提供的利润,大于由资本的一般利润率决定的平均利润。

现在产生的问题是:在这里,上述三种情况中哪一种情况或者其中哪几种情况可能发生?提供地租的商品所包含的全部剩余价值在该商品的价格中是否得到实现?如果是这样,上述第三种情况就被排除了,在第三种情况下,商品的全部剩余价值之所以在它们的平均价格中得到实现,是因为它们只有这样才提供普通利润。因此这种情况不属于考察的范围。同样,按照这个假定,第一种情况,就是在商品的价格中实现的剩余价值高于它的内在剩余价值的情况,也不属于考察的范围。因为我们恰恰假定,在提供地租的商品的价格中“实现了它所包含的剩余价值”。因此,这种情况同第二种情况相类似,在第二种情况下,商品的内在剩余价值高于在它们的平均价格中实现的剩余价值。同这第二种情况下的商品一样,特殊生产领域的商品的内在剩余价值——以利润的形式出现,并降低到一般利润率的水平,——在这里形成所花费的资本的利润。但是,和第二种情况下的商品不同,在我们所考察的这些例外的商品的价格中,也实现了商品的内在剩余价值超过这个利润的余额,但是这个余额不是落到资本所有者手里,而是落到别的所有者手里,就是说,落到土地、自然因素、矿山等等的所有者手里。

也许这些商品的价格被哄抬到足以提供多于平均利润率的东西吧?例如,在(真正的)垄断价格的情况下就是这样。这个假定——对于每一个可以自由使用资本和劳动,而生产就使用的资本量来说已经服从于一般规律的生产领域——不仅是petitioprincipii〔本身尚待证明的论据〕,并且是同科学和资本主义生产(前者仅仅是后者的理论表现)的基础直接矛盾的。因为,这种假设的前提恰恰是需要加以说明的东西,即在一个特殊生产领域中,商品的价格所提供的必然要比一般利润率,比平均利润多,为此,商品必然要高于它的价值出卖。因而,它的前提是,农产品不受商品价值和资本主义生产的一般规律影响。并且,所以以此为前提,是因为初看起来,利润之外还特别存在地租,造成了这种假象。所以,上述假设是荒谬的。

因此,唯一的办法就是,假设在这个特殊生产领域存在着特殊的条件,存在着某种影响,使商品的价格实现了商品的[全部]内在剩余价值,而不是象第二种情况下的商品,其价格只在一般利润率所提供的利润的限度内实现其剩余价值。在所考察的特殊生产领域中,商品的平均价格并没有降到商品的剩余价值以下,以致它们只提供一般利润率,或者说,以致它们的平均利润不大于其他一切使用资本的生产领域。

这样一来,问题已经大大简化了。问题已经不是要说明,一种商品的价格,怎么除了提供利润之外还提供地租,——因而,它表面上看来,违背了一般价值规律,并且通过把它的价格提到高于它的内在剩余价值,而给一定量资本提供了大于一般利润率所能提供的东西。相反,问题是要说明:这种商品在商品价格平均化而导致平均价格的过程中,怎么不把它的内在剩余价值让一些给其他商品,使它只留下平均利润;这种商品怎样把自己剩余价值中构成超过平均利润的余额的那部分也加以实现。因此,问题在于一个在该生产领域投资的租地农场主,他出卖商品的价格,怎么会使这种商品除了给他提供普通利润外,同时还使他能够把实现的商品剩余价值超过这个利润的余额,付给第三者即土地所有者。

[455]这样提出问题的提法本身,就已经包含问题的解答。

\tsubsectionnonum{[(c)土地私有权是绝对地租存在的必要条件。农业中剩余价值分解为利润和地租]}

十分简单:一定的人们对土地、矿山和水域等的私有权,使他们能够攫取、拦截和扣留在这个特殊生产领域即这个特殊投资领域的商品中包含的剩余价值超过利润(平均利润,由一般利润率决定的利润)的余额,并且阻止这个余额进入形成一般利润率的总过程。这部分剩余价值,甚至在一切工业企业中也被拦截,因为不论什么地方,都要为使用地皮(工厂建筑物、作坊等所占的地皮)付地租,因为即使在可以完全自由占用土地的地方,也只有在多少是人口稠密和交通发达的地点才建立工厂。

如果在最坏的土地上得到的商品,属于平均价格等于价值的第三类商品,就是说,属于这样一类商品,它们的全部内在剩余价值在它们的价格中得到实现,因为它们只有这样才提供普通利润,——那末,这块土地就不付任何地租,土地所有权在这里就只是名义上的。假如这里付一笔租金,那末,这不过证明小资本家们满足于赚取低于平均利润的利润,在英国有一部分就是这种情况(见纽曼的著作)\endnote{指弗·威·纽曼《政治经济学讲演集》1851年伦敦版第155页。——第31页。}。当地租率大于商品的内在剩余价值和平均利润的差额的时候,总是这种情况。甚至有的土地,耕种它至多只够补偿工资,因为,虽然劳动者在这里用他的整个工作日为自己劳动,但是他的劳动时间超过社会必要劳动时间。他的劳动生产率低于这个劳动部门中占统治地位的生产率,虽然他用12小时为自己劳动,他生产的产品几乎没有工人在比较有利的生产条件下用8小时生产的多。这就好比与机器织机竞争的手工织工的情况一样。这个手工织工的产品,的确包含12劳动小时,但是它只等于8小时或者还不到8小时的社会必要劳动,因此,只有8个必要劳动小时的价值。如果在类似情况下,一个茅舍贫农支付租金,那末这笔租金纯粹是他的必要工资的扣除部分,不代表任何剩余价值,更不代表任何超过平均利润的余额。

假定在某一国家,例如美国,进行竞争的租地农场主的人数还很少,土地占有还不过是形式,每一个人都可以找到空闲的土地来投资耕种,而不必经过在他以前已经经营土地的所有者或租地农场主的许可。在这种情况下,除了因位于人口稠密的地带而被垄断的土地以外,在一个较长的时期内,租地农场主生产的超过平均利润的剩余价值,在他的产品的价格中可能得不到实现;他会被迫把他所得到的剩余价值与资本家同伙瓜分,这正象有些商品的剩余价值一样,它们包含的全部剩余价值如果在商品的价格中得到实现,就会提供超额利润,也就是提供超过一般利润率的利润。在这种情况下,一般利润率就会提高,因为小麦等,将同其他工业品一样,低于它的价值出卖。这种低于价值出卖的情况不会成为例外,相反,倒会阻止小麦成为其他同类商品中的例外。

第二,假定某一国家的全部土地都是一种质量,但是属于这样一种质量:如果商品包含的全部剩余价值都在商品的价格中得到实现,商品就会给资本提供普通利润。在这种情况下,不支付任何地租。地租的消失,丝毫不影响一般利润率,既不会使它提高也不会使它降低,正如其他非农产品属于这一类并不影响利润率一样。这些商品之所以属于这一类,正是因为它们的内在剩余价值等于平均利润;因此,它们不能改变这种利润的高度,相反,它们适应于这种利润而完全不影响这种利润,尽管这种利润影响它们。

第三,假定全国土地都是一类,而且这些土地如此贫瘠,投在它上面的资本的生产率如此低,以致它的产品属于剩余价值低于平均利润的一类商品。自然,在这种情况下(因为工资由于农业生产率低而普遍提高),只有在绝对劳动时间可以延长,原料(如铁等)不是农产品,或者原料(如棉花、丝等)是进口物和比较肥沃土地的产品的地方,剩余价值才处于较高的水平。在这种情况下,[农业]商品的价格包含的剩余价值必须高于它们的内在剩余价值,才能提供普通利润。一般利润率将因此降低,虽然地租并不存在。

或者,我们假定在第二种情况下土地的生产率非常低。那末,这种农产品的剩余价值等于平均利润,说明这里的平均利润本来就低,因为在农业中,12劳动小时里面,单单用来生产工资,或许就要11劳动小时,而剩余价值只有1小时或者更少。

[456]这几种不同的情况说明:

在第一种情况下,地租的消失或不存在,是同一个与地租已经发展的其他国家相比提高了的利润率联系着、并存着的。

在第二种情况下,地租的消失或不存在丝毫不影响利润率。

在第三种情况下,地租的消失或不存在——与有地租存在的其他国家相比——是同一个低的、较低的一般利润率联系着的,并且是一般利润率水平低的标志。

由此可见,一个特殊的地租的发展,就其本身来说,同农业劳动的生产率是绝对无关的,因为地租的不存在或者消失既可以同一个提高的利润率联系着,也可以同一个保持不变的利润率联系着,也可以同一个下降的利润率联系着。

这里的问题不在于为什么在农业等部门剩余价值超过平均利润的余额被扣留下来;相反,问题倒在于:由于什么原因这里竟要发生相反的现象?

剩余价值无非是无酬劳动;平均利润,或者说正常利润,无非是假定由每一个一定量的资本实现的无酬劳动量。如果说平均利润是10%,那末这不过是说,一个100单位的资本摊到10单位无酬劳动;或者说,等于100的物化劳动支配相当于本身数额的1/10的无酬劳动。因此,剩余价值超过平均利润的余额是指:商品中(商品的价格中,或者说,由剩余价值构成的那部分商品价格中)包含的无酬劳动量,大于形成平均利润的无酬劳动量,因而大于商品的平均价格中构成商品价格超过商品生产费用的余额的无酬劳动量。在单个商品中,生产费用代表预付资本,超过这个生产费用的余额代表预付资本所支配的无酬劳动;因此,这个价格余额与生产费用之比,代表用于商品生产过程的一定量资本支配无酬劳动的比率,而不管该特殊生产领域的商品所包含的无酬劳动是否等于这个比率。

那究竟是什么东西迫使单个资本家例如按照平均价格出卖他的商品?(这个平均价格作为某种已经形成的东西强加于资本家,这决不是他的自由行动。他是更愿意高于商品价值出卖商品的。)究竟是什么东西迫使资本家按照这种只向他提供平均利润,使他实现的无酬劳动小于他商品中实际包含的无酬劳动的价格出卖他的商品呢?迫使他这样做的,是其他资本通过竞争所施加的压力。如果A生产部门的无酬劳动对预付资本(例如100镑)之比大于B、C等生产领域(B、C等生产领域的产品,完全同A生产领域的商品一样,以其使用价值满足某种社会需要),任何同量资本也就会涌向A生产部门。

因此,如果存在这样一些生产领域,那里的某些自然生产条件,如耕地、煤层、铁矿、瀑布等,——没有这些条件,生产过程就无从进行,这些领域的商品就不能生产,——不是掌握在物化劳动的所有者或占有者资本家的手里,而是掌握在其他人的手里,那末这第二类的生产条件所有者就对资本家说:

如果我让你使用这些生产条件,那你将赚你的平均利润,占有正常的无酬劳动量。但是你的生产提供一个超过利润率的剩余价值余额,即无酬劳动余额。这个余额,你不应象你们资本家们通常做的那样,投进总库。这个余额我来占有,它是属于我的。这种交易会使你完全满意,因为你的资本在这个生产领域给你提供的,同在其他任何领域一样多,并且,这是一个十分稳定的生产部门。你的资本在这里除了给你提供构成平均利润的那10%的无酬劳动以外,还给你提供20%的超额无酬劳动。你要把这个付给我,为了能够这样做,你要把这20%的无酬劳动加在商品的价格上,但是不要把它算入你和其他资本家的总账。你对一种劳动条件——资本,物化劳动——的所有权,使你能够占有工人的一定数量的无酬劳动,同样,我对另一种生产条件——土地等等——的所有权,使我能够从你和整个资本家阶级那里扣下无酬劳动中超过你的平均利润的那个余额。你们的规律要求在正常情况下等量资本占有等量无酬劳动,你们资本家可以[457]通过竞争彼此强制做到这一点。好吧!我正要把这个规律应用到你的身上。你从你的工人的无酬劳动中占有的,不要多于你用同一笔资本在其他任何生产领域所能占有的。但是,这个规律同你“生产”的那个超过无酬劳动正常量的余额是毫无关系的。谁能阻止我占有这个“余额”呢?我为什么要象你们那样,把它投入资本的大锅,供资本家阶级内部分配,使每个人按他在总资本中拥有的股份取得这个余额的一定部分呢?我不是资本家。我让你使用的生产条件不是物化劳动,而是自然的赐予。你们能制造土地、水、矿山或者煤层么?不能。因此,可以用到你身上、使你把你自己侵吞的剩余劳动吐出一部分来的那种强制手段,对我来说是不存在的!所以,拿来吧!你的资本家同伙能做的唯一事情,不是同我竞争,而是同你竞争。如果你付给我的超额利润,小于你占有的剩余劳动时间与依照资本的规律你应得的那份剩余劳动之间的差额,你的资本家同伙就会出面,通过竞争,逼你把我能从你那里挤出的全部数额老老实实支付给我。

现在本来应该研究:(1)从封建土地所有制到另一种由资本主义生产调节的商业地租的过渡;或者,另一方面,从这种封建土地所有制到自由的农民土地所有制的过渡;(2)在土地最初不是私有财产而资产阶级生产方式至少在形式上一开始就占统治地位的一些国家,如美国,地租是怎样产生的;(3)仍然存在着的土地所有制的亚洲形式。但是这一切都不是这里要谈的。

这样,按照我们所谈的理论,对于自然对象如土地、水、矿山等的私有权,对于这些生产条件,对于自然所提供的这种或那种生产条件的所有权,不是价值的源泉,因为价值只等于物化劳动时间;这种所有权也不是超额剩余价值即无酬劳动中超过利润所包含的无酬劳动的余额的源泉。但是,这种所有权是收入的一个源泉。它是一种权利,一种手段,使这一生产条件的所有者能够在他的所有物作为生产条件加入的生产领域中占有被资本家榨取的无酬劳动的一部分,否则这一部分会作为超过普通利润的余额被投进资本总库中去。这种所有权是一种手段,它能阻止在其余资本主义生产领域发生的上述过程发生,并且把这个特殊生产领域所生产的剩余价值扣留在这个领域中,于是剩余价值现在就在资本家和土地所有者之间进行分配。因此,土地所有权,就象资本一样,变成了支取无酬劳动、无代价劳动的凭证。在资本上,工人的物化劳动表现为统治工人的权力,同样,在土地所有权上,土地所有权使土地所有者能从资本家那里扣下一部分无酬劳动的这种情况,表现为土地所有权似乎是价值的一个源泉。

这就说明了现代地租,说明了它的存在。在投资相等的条件下,地租量不等,只能用土地的肥力不同来说明。在肥力相等的条件下,地租量不等,只能用投资量不等来说明。在前一种情况下,地租增加是因为地租对所投资本(也对土地面积)的比率提高了。在后一种情况下,地租增加是因为在同一比率下,甚至在不同比率下(如果第二笔投入土地的资本的生产率较低的话)地租量增加。

按照这个理论,最坏的土地无论完全不提供地租,或者提供地租,都不是必然的。其次,完全没有必要假定农业生产率减低,虽然,生产率的差异,如果不是人为地加以排除(这是可能的),在农业中比在同一工业生产领域内要大得多。我们谈生产率的高低,指的总是同种产品。至于不同产品之间的关系,那是另外一个问题。

按土地本身计算的地租是地租总额,地租量。地租率不提高,地租也可能增加。如果货币价值不变,农产品的相对价值可能提高,但不是因为农业生产率降低,而是因为农业生产率虽然提高,但是提高的程度不如工业。相反,如果货币价值不变,农产品货币价格的提高,只有在农产品价值本身提高的时候,也就是说,只有在农业生产率降低的时候,才有可能(这里不谈需求对于供给的暂时压力,象其他商品经常发生的情况那样)。

在棉纺织工业中,原料价格随着工业本身的发展不断下降;在制铁、煤炭等工业中情况也是一样。这里,地租的增加只可能由于使用了更多的资本,而不是由于地租率提高。

李嘉图认为:空气、光、电、蒸汽、水这些自然力是白白取得的,土地就不是这样,因为土地是有限的。因此,照李嘉图看来,仅仅由于这一点,农业的生产率已经不如其他生产部门。如果土地象其他自然要素和自然力一样,属于大家而不被占有,要多少有多少,那末,照李嘉图看来,生产率就会高得多。

[458]首先必须指出,假如土地作为自然要素供每个人自由支配,那末,资本的形成就缺少一个主要要素。一个最重要的生产条件,而且是——如果不算人本身和人的劳动——唯一原始的生产条件就不能转让、占有,因而不能作为别人的财产同劳动者对立并因此把他变成雇佣工人。这样一来,李嘉图意义上的即资本主义意义上的劳动生产率,无酬的别人劳动的“生产”,就不可能了。这样一来,资本主义生产就根本完结了。

至于李嘉图列举的那些自然力,一部分的确可以白白取得,它们不要资本家花费什么。煤使资本家花了费用,但是如果资本家白白取得水,蒸汽就不要他花费什么。但是现在我们以蒸汽为例。蒸汽的属性是永远存在的。生产上利用蒸汽,是一个已被资本家据为己有的新的科学发现。由于这个发现,劳动生产率提高了,从而相对剩余价值也提高了。这就是说,资本家从一个工作日中占有的无酬劳动量由于利用蒸汽而增加了。因此,蒸汽的生产力同土地的生产力之间的差别,仅仅在于前者给资本家带来无酬劳动,后者则给土地所有者带来无酬劳动,这种无酬劳动,土地所有者不是[直接]从工人手上而是从资本家手上取去的。因此,资本家也就热中于“废除”这个自然要素的“所有权”。

李嘉图对问题的提法中只有下面一点是正确的:

在资本主义生产方式下,资本家不仅是一个必要的生产当事人,而且是占统治地位的生产当事人。相反,土地所有者在这种生产方式下却完全是多余的。资本主义生产方式所需要的只是:土地不是公共所有,土地作为不属于工人阶级的生产条件同工人阶级对立。如果土地国有,因而国家收地租,这个目的就完全达到了。土地所有者,在古代世界和中世纪世界是那么重要的生产当事人,在工业世界中却是无用的赘疣。因此,激进的资产者在理论上发展到否定土地私有权(而且还打算废止其他一切租税),想把土地私有权以国有的形式变成资产阶级的、资本的公共所有。然而,他们在实践上却缺乏勇气,因为对一种所有制形式——一种劳动条件私有制形式——的攻击,对于另一种私有制形式也是十分危险的。况且,资产者自己已经弄到土地了。

\tsectionnonum{[(4)洛贝尔图斯关于农业中不存在原料价值的论点是站不住脚的]}

现在谈谈洛贝尔图斯先生。

按照洛贝尔图斯的意见,在农业中是根本不计算原料的,因为,洛贝尔图斯肯定说,德国农民不把种子、饲料等算作自己的支出,不计算这些生产费用,也就是说,计算错误。这样说来,在租地农场主进行正确计算已有150年以上的英国,就不应该存在任何地租。因此,洛贝尔图斯从这里得出的不应该是这个结论:租地农场主支付地租是因为他的利润率比工业中的利润率高;而应该完全是另一个结论:租地农场主支付地租是因为他由于计算错误而满足于较低的利润率。本人是租地农场主的儿子,并且十分熟悉法国租佃关系的魁奈医生,是不会欣然同意洛贝尔图斯的。魁奈在“预付”项目下,在“年预付”中,把租地农场主所使用的“原料”价值计算为10亿,尽管租地农场主会把这个原料以实物形式再生产出来。

在一部分工业中几乎完全不存在固定资本,或者说,机器设备,而在另一部分工业中,在整个运输业中,即在(用马车、铁道、船舶等)使位置发生变化的工业中,则根本不使用原料而只使用生产工具。这些工业部门除了利润之外是否还提供地租呢?这种工业部门同例如采矿工业有什么区别呢?在这两种场合,都只有机器设备和辅助材料,例如轮船、火车头和矿山所用的煤,马的饲料等。为什么利润率的计算在一种生产部门中要不同于另一种生产部门呢?假定农民用在生产上的实物形式的预付占他的全部预付资本的1/5,另外,用于购买机器和支付工资的预付占4/5,而全部支出[按价值]等于150夸特。其次,如果农民得到10%的利润,那末利润就是15夸特。因此,总产品等于165夸特。如果农民从他的预付资本中扣除1/5,即30夸特,而15夸特只用120夸特来除,他的利润就等于[12+(1/2)]%。

我们还可以这样来说明。农民的产品价值或他的产品等于165夸特(330镑)。他把自己的预付计算为120夸特(240镑)。这笔预付的10%就是12夸特(24镑)。但是他的总产品等于165夸特,因此,其中总共扣去132夸特[补偿货币支出和它的10%的利润],余下33夸特。但是在33夸特中30夸特是以实物形式支出的。于是剩下3夸特(=6镑)作为超额利润。这个农民的总利润等于15夸特(30镑)而不是12夸特(24镑)。因此,他能够支付3夸特或6镑的地租,并且可以认为同其他任何资本家一样得到了10%的利润。但是这个10%只存在于想象中。实际上他预付的不是120夸特,而是150夸特,它的10%就是15夸特或30镑。实际上他少得了3夸特,即他已经得到的12夸特的1/4,[459]换句话说,他少得了他应该得到的全部利润的1/5,——这是因为,他没有把自己预付的1/5当作支出计算进去。因此,只要农民学会按资本家的方法计算,他就会立刻停止支付地租,因为地租正等于他的利润率同普通利润率之间的差额。

换句话说,包含在165夸特中的无酬劳动产品等于15夸特,或30镑,或30劳动周。如果这30劳动周或15夸特或30镑用150夸特的总预付来除,那末结果只是10%;如果只用120夸特来除,那就会得出较高的利润率。因为用120夸特除12夸特的结果是10%,而用120夸特除15夸特,则不是10%,而是[12+(1/2)]%。这就是说:虽然农民作了上述实物预付,但是因为他没有按照资本家的方法把它们计算进去,所以他不是用自己的全部预付额来除他所积攒的剩余劳动。这样一来,这个剩余劳动就会代表比其他生产部门高的利润率,就能提供地租,因此,这个地租完全是基于计算错误。如果农民知道,为了用货币来计算他的预付,并因此把这种预付看作商品,他完全没有必要预先把它变成实在货币,即把它出卖,那末整个故事也就完了。

没有这种计算错误(许多德国农民会犯这种错误,但是没有一个资本主义租地农场主会犯这种错误),洛贝尔图斯的地租就不可能存在。这种地租只有在原料加入生产费用的地方才有可能存在,而在原料不加入生产费用的地方就不可能存在。它只有在原料加入生产而不被生产者计算的地方才有可能存在。但是它在原料不加入生产的地方就不可能存在,——尽管洛贝尔图斯先生不是想从计算错误得出地租,而是想从预付中缺少一个实际项目得出地租。

我们以采矿工业或渔业为例。原料在这里只是作为辅助材料加入生产,但这一点我们可以撇开不谈,因为机器的采用也总是(除了极少数例外)以辅助材料即机器的生活资料的消费为前提。假定一般利润率为10%。100镑用在机器和工资上。为什么因为这100不是用在原料、机器和工资上[而仅仅用在机器和工资上],它的利润就要大于10呢?或者说,为什么因为这100仅仅用在原料和工资上,它的利润就要大于10呢?如果说这里存在某种差别,那末引起这种差别的原因只能是:在不同情况下不变资本和可变资本的价值之比一般是不同的。这种不同的比例即使在剩余价值率假定不变的情况下也会提供不同的剩余价值。而不同的剩余价值对等量资本之比,必然得出不同的利润。但是,从另一方面说,一般利润率正是意味着把这些差别拉平,把资本的有机组成部分抽象化,把剩余价值分配得使等量资本提供等量利润。

剩余价值量取决于所支出的资本量,这种情况——按照剩余价值的一般规律——绝对不适用于不同生产领域的各个资本,而适用于同一生产领域(在这里,假定资本的有机组成部分之间的比例是相同的)的各个不同资本。如果我举例说,假定在纺纱业中,利润量同所支出的资本量相适应(此外还假定生产率不变,否则这种说法也不完全正确),那末我实际上只是说,在对纺纱工人的剥削率既定的情况下,剥削量取决于被剥削的纺纱工人的人数。相反,如果我说,各个不同生产部门的利润量同所支出的资本量相适应,那末,这就是说,各个一定量资本的利润率都相同,即利润量只能随这个资本量的变化而变化,换句话说,这又意味着利润率不取决于某一单个生产领域中资本组成部分之间的有机比例,它完全不取决于这些单个生产领域中所创造的剩余价值量。

采矿业一开始就应该属于工业,而不属于农业。是什么理由呢?理由就是:没有一种矿产品以实物形式,即以从矿山开采出来时的形式,作为生产要素重新加入矿山所使用的不变资本(渔业和狩猎业的情况也是这样,在这里,支出在更大程度上只限于劳动资料,工资或者说劳动本身)。[460]换句话说,这是由于这里的每一个生产要素,即使它的原料是从矿山开采出来的,在它重新作为要素加入矿业生产之前,不仅先要改变自己的形式,而且要变成商品,即必须被买进来。唯一的例外是煤。但是煤作为生产资料出现只是在这样一个发展阶段,那时矿业主已经成了训练有素的资本家,他用复式簿记记帐,按照这种簿记,不仅他把自己的预付记成对自己的负债,不仅他对自己的基金来说是债务人,而且他的基金对于基金本身也成了债务人。由此可见,恰恰在实际上没有原料加入支出的地方,必然一开始就普遍采用资本主义会计,因而不可能犯农民会犯的错误。

我们现在来看加工工业本身,特别是其中这样一个部分,在这里,劳动过程的一切要素同时作为价值形成过程的要素出现,因此,一切生产要素同时作为支出,作为具有价值的使用价值,即作为商品加入新商品的生产。这里,在生产最初的半成品的制造业者同第二个以及所有以后的(按照生产阶段的序列)制造业者之间有着根本的区别,在后者那里,原料不仅作为商品加入生产,而且已经是二次方的商品,也就是说,这个商品已经取得了不同于最初商品即原产品的自然形式的形式,已经经历了生产过程的第二阶段。以纺纱业者为例。他的原料是棉花,但棉花是已经作为商品的原产品。而织布业者的原料是纺纱业者的产品棉纱,印染业者的原料是织布业者的产品布,而所有这些在生产过程的下一阶段重新作为原料出现的产品同时又是商品。\endnote{马克思接着在手稿中草拟了一个棉花种植业者、纺纱业者和织布业者的例子。他从这三人中间每一个人单独获得的利润问题转而考察在假定织布业者同时也是棉花种植业者和纺纱业者的情况下获得多少利润的问题。但是马克思不满意已写好的东西,他把已经开始起草的东西停下来并且把它全划掉了,随后如正文中所作的那样精确地表述了自己的思想。——第43页。}[460]

[461]这里,我们显然又遇到了已经两次涉及的问题,一次是在考察约翰·斯图亚特·穆勒的观点的时候\endnote{马克思指的是第VII和第VIII本(手稿第319—345页)中篇幅很长的关于约翰·斯图亚特·穆勒的补充部分。按照马克思所编的《剩余价值理论》目录以及他在手稿第VII本(第319页)正文中所作的指示,本版把关于约翰·斯图亚特·穆勒这一节移至《理论》第三册中关于李嘉图学派的解体那一章。在马克思手稿第332—334页即该章专论约翰·斯图亚特·穆勒的第七节中,阐明了这样一个问题:成品的生产和生产这个成品的不变资本的生产结合在一个资本家手里,会不会影响利润率。——第43页。},后来一次是在一般考察不变资本和收入的相互关系的时候\authornote{见本卷第1册第128—129和221页。——编者注}。这个问题一再出现,就说明事情还有些棘手。这个问题本来属于论述利润的第三章\endnote{马克思指他的研究中后来发展成《资本论》第三卷的那一部分。——第44、187页。}。不过在这里谈一谈比较好。

我们举一个例子:

4000磅棉花=100镑;

4000磅棉纱=200镑;

4000码棉布=400镑。

根据这个假定,1磅棉花=6便士,1磅棉纱=1先令,1码棉布=2先令。

假定利润率等于10%,那末,

100镑(A)中——支出=90+(10/11),利润=9+(1/11)。

200镑(B)中——支出=181+(9/11),利润=18+(2/11)。

400镑(C)中——支出=363+(7/11),利润=36+(4/11)。

A是农民(I)的产品棉花;B是纺纱业者(II)的产品棉纱;C是织布业者(III)的产品布。

在这个假定中,产品A的90+(10/11)镑本身是否包含利润,这是完全无关紧要的。如果这个90+(10/11)镑是自行补偿的不变资本,它就不包含利润。同样,[代表产品A的价值的]100镑是否包含利润,对B来说也是无关紧要的。至于产品B,对C来说也是如此。

棉花种植业者(I)、纺纱业者(II)和织布业者(III)的情况如下:

(I)支出——90+(10/11),利润——9+(1/11)。总额——100。

(II)支出——{100(I)+[81+(9/11)]},利润——18+(2/11)。总额——200。

(III)支出——{200(II)+[163+(7/11)]},利润——36+(4/11)。总额——400。

全部总额等于700。

利润等于9+(1/11)+[18+(2/11)]+[36+(4/11)]=63(7/11)。

三个部门的预付资本等于90+(10/11)+[181+(9/11)]+[363+(7/11)]=636+(4/11)。

700超过636+(4/11)的余额等于63+(7/11)。而[63+(7/11)]∶[636+(4/11)]=10∶100。

我们继续分析这个荒唐的例子,就会得出:

(I)支出——90+(10/11),利润——9+(1/11)。总额——100。

(II)支出——{100(I)+[81+(9/11)]},利润——{10+[8+(2/11)]}。总额——200。

(III)支出——{200(II)+[163+(7/11)]},利润——{20+[16+(4/11)]}。总额——400。

棉花种植业者(I)对谁也不必支付利润,因为假定他的90+(10/11)镑不变资本不包含利润,而只代表不变资本。棉花种植业者(I)的全部产品作为不变资本加入纺纱业者(II)的支出。[II的产品中代表]100镑不变资本的部分补偿给棉花种植业者9+(1/11)镑利润。纺纱业者(II)的等于200镑的全部产品加入织布业者(III)的支出;因此,[织布业者的不变资本]补偿18+(2/11)镑利润。但是,这并不妨碍棉花种植业者的利润丝毫也不比纺纱业者和织布业者的利润多,因为按同一比例,应该由棉花种植业者补偿的资本小了,而利润是同资本量相适应的,同这个资本由哪些部分组成完全没有关系。

现在假定织布业者(III)自己生产这一切。这时从表面上看,事情有了变化。[但是实际上利润率在这种情况下不变。]织布业者的支出现在采取了如下形式:90+(10/11)投入棉花生产,181+(9/11)投入棉纱生产,363+(7/11)投入棉布生产。他把这三个生产部门都买下来,因而他必须在每一个生产部门中都投入一定的不变资本。我们把这几笔资本加起来的总额是:90+(10/11)+[181+(9/11)]+[363+(7/11)]=636+(4/11)。这个总额的10%恰恰是63+(7/11),这同上面一样,——不过现在是全部由一个人放进自己的腰包里,而以前这63+(7/11)是在I、II和III之间进行分配的。

[462]这种迷惑人的假象[似乎利润率在这种情况下发生了变化]是从什么地方产生的呢?

但是,还有一点先要说一说。

如果我们从400中扣除织布业者的利润36+(4/11),余下363+(7/11),这是织布业者的支出。在这笔支出中,200是支付棉纱的。这200中有18+(2/11)是纺纱业者的利润。如果我们从363+(7/11)的支出中扣除这18+(2/11),余下345+(5/11)。但是,除此以外,在补偿给纺纱业者200的中还包含棉花种植业者的利润9+(1/11)。如果我们从345+(5/11)中扣除9+(1/11),余下336+(4/11)。如果我们从布的总价值400中扣除这336+(4/11),那就可以看出,其中包含等于63+(7/11)的利润。

但是,63+(7/11)的利润除以336+(4/11),等于[18+(34/37)]%。

以前,这63+(7/11)是除以636+(4/11),利润为10%。总价值700超过636+(4/11的余额恰好是63+(7/11)。

这样,同一笔资本100的利润,照我们新的计算是[18+(34/37)]%,而照以前的计算——只有10%。

这两者怎么一致呢?

我们假定I、II、III是同一个人,但是他不是同时使用自己的三笔资本(一笔用于种植棉花,另一笔用于纺纱,第三笔用于织布),而是这样使用:他只是在完成棉花种植工作以后才开始纺纱,并且只是在完成纺纱以后才着手织布。

于是计算如下:

这个资本家支出90+(10/11)镑用于种植棉花,得到4000磅棉花。为了把所有这些棉花纺成纱,他必须在机器、辅助材料和工资上再支出81+(9/11)镑。他用这些纺出4000磅棉纱。最后,他把这些棉纱变成4000码布,这又需要他支出163+(7/11)镑。现在把他的全部支出加起来,他的预付资本就是90+(10/11)镑+[81+(9/11)]镑+[163+(7/11)]镑即336+(4/11)镑。这笔总额的10%是33+(7/11),因为[336+(4/11)]∶[33+(7/11)]=100∶10。但是336+(4/11)镑+[33+(7/11)]镑=370镑。因而,他是按370镑而不是按400镑出卖4000码布,便宜了30镑,也就是说比以前便宜[7+(1/2)]%。如果布的价值的确等于400镑,那末他能够按普通利润10%出卖商品,另外还能支付30镑地租,因为他的利润率不等于33+(7/11)对预付336+(4/11)之比,而等于63+(7/11)对336+(4/11)之比,——这就是说,同我们上面看到的一样,利润率为[18+(34/37)]%。

看来,这实际上就是洛贝尔图斯先生计算地租的方法。

错误究竟在哪里?首先可以看到,如果纺纱和织布互相结合在一起,它们[照洛贝尔图斯的看法]就必然象纺纱和农业结合在一起或者农业单独经营一样提供地租。

这里显然是两件不同的事情。

第一,我们只用336+(4/11)镑的一笔资本来除63+(7/11)镑,而我们本来应该用总价值636+(4/11)镑的三笔资本来除63+(7/11)镑。

第二,我们把最后一笔资本(III)的支出算作336+(4/11)镑而不是363+(7/11)镑。

这两点需要分别加以分析。

第一,如果一个资本家兼有棉花种植业者、纺纱业者和织布业者的三种身分,他把所收获的全部产品棉花纺成纱,那末,他就绝对没有把这种收获的任何一部分用于补偿自己的农业资本。他不是[同时]把他的资本的一部分用于[463]种植棉花,——用于种植棉花所需的各种费用,用于种子、工资、机器,——把另一部分用于纺纱,而是先把他的资本的一部分投入种植棉花,以后把这部分加上第二部分投入纺纱,然后把已经包含在棉纱中的前两部分再加上第三部分投入织布。最后织成了4000码布,这时他怎样补偿这些布的生产要素呢?当他织布时他并不纺纱,并且也没有纺纱所必需的材料,而当他纺纱时,他不种植棉花。因此,他的生产要素不可能由他来补偿。如果我们自行解脱地说:是的,这个家伙把这4000码卖掉,然后从卖得的400镑中拿出一部分来“购买”棉纱以及棉花的要素。但是这样做会得出什么结论呢?结论只能是,我们实际上承认有三笔资本,它们同时被使用,被投入经营,被预付到生产上。要能买到棉纱就必须有棉纱,要能买到棉花也就必须有棉花,而要使市场上有棉花和棉纱,因而能代替已经织掉的棉纱和纺掉的棉花,生产它们的资本就必须和投入织布的资本同时投入经营,必须在棉纱变成布的同时变成棉花和棉纱。

因此,无论是III把所有三个生产部门结合在一起,或者是这三个生产部门由三个生产者分担,都必须有三笔资本同时存在。如果一个资本家想以同一规模进行生产,他就不可能把他用来织布的同一笔资本用来纺纱和种植棉花。这些资本中的每一笔资本都已投入生产,而它们之间互相补偿这一点同我们要研究的问题毫无关系。互相补偿的资本是不变资本,它们必须同时投入三个部门中的每一个部门,并且同时发挥作用。如果说400镑中包含利润63+(7/11)镑,这只是因为III除了自己的利润36+(4/11)镑以外还得到——根据我们的计算——他应该付给生产者II和I的利润,而这笔利润根据假定是在他的商品中实现的。但是,I和II不是从III的363+(7/11)镑得到利润,而是土地耕种者单独从自己的90+(10/11)镑得到利润,纺纱业者从自己的181+(9/11)镑得到利润。如果III拿到全部利润,那末他仍然不是从他投入织布的363+(7/11)镑得到的,而是从这笔资本加上他投入纺纱和种植棉花的另外两笔资本得到的。

第二,如果我们把III的支出算作336+(4/11)镑而不是363+(7/11)镑,那是由于:

我们把织布业者用于种植棉花的支出仅仅计算为90+(10/11)镑而不是100镑。但是他需要棉花种植业的全部产品,这全部产品是100镑而不是90+(10/11)镑。9+(1/11)镑的利润已经包含在这全部产品中了。否则,他就是使用了一笔90+(10/11)镑没有给他提供任何利润的资本。种植棉花就没有给他带来利润,而仅仅补偿90+(10/11)镑的支出。同样,纺纱也没有给他带来利润,纺纱的全部产品仅仅补偿支出。

在这种情况下,他的支出实际上是90+(10/11)+[81+(9/11)]+[163+(7/11)]=336+(4/11)。这就是他的预付资本。这笔资本的10%是33+(7/11)镑。这时产品价值就等于370镑。这个产品价值决不会更高,因为,根据假定,前面两个部分I和II没有带来任何利润。因此,如果III不插手I和II的部门而保持原来的生产方法,他的情况就会好得多。因为现在III自己只有33+(7/11)镑,而不是象以前那样由I、II、III共同消费63+(7/11)镑,以前他的同伙同他一起分享利润,他倒还得到36+(4/11)镑。他真的成了一个很不中用的生意人了。他在II部门里能节约9+(1/11)镑的支出,只是因为他在I部门里没有得到利润,他在III部门里能节约18+(2/11)镑的支出,只是因为他在II部门里没有得到利润。他种植棉花所得到的90+(10/11)镑以及他纺纱所得到的81+(9/11)+[90+(10/11)]镑,都只会自己补偿自己。只有投入织布的第三笔资本90+(10/11)+[81+(9/11)]+[163+(7/11)],才带来10%的利润。这也就是说,100镑在织布时提供10%的利润,但是在纺纱和种植棉花时不提供丝毫利润。这对III来说,只要I或II不是自己而是别人,确实是非常惬意的,但是当他打算把三个生产部门结合于他尊贵的一身而把这点节约下来的宝贝利润占为己有时,那就一点也不惬意了。因此,用于预付利润(或者说,一个部门的不变资本中[464]对其他两个部门来说是利润的那个组成部分)的支出所以会节约下来,是因为I和II两个部门的产品实际上不包含任何利润,在这些部门中没有完成任何剩余劳动;这些部门仅仅把自己看作雇佣工人,只给自己补偿自己的生产费用即不变资本和工资的支出。但是在这些情况下——只要I和II不愿意例如为III劳动,但是利润就会因此进入后者账内——所完成的劳动因此总是会减少,而且III必须支付代价的劳动是光用在工资上还是用在工资和利润上,对他来说是完全一样的。这对他来说是一回事,只要他所购买和支付代价的是产品,是商品。

不变资本是全部还是部分以实物形式得到补偿,也就是说,不变资本是否由把它作为不变资本的那种商品的生产者来补偿,是完全无关紧要的。首先,任何不变资本最终都必须以实物形式得到补偿:机器由机器补偿,原料由原料补偿,辅助材料由辅助材料补偿。在农业中,不变资本也可以作为商品加入生产,也就是可以直接通过买卖加入生产。当然,只要加入再生产的是有机物,不变资本就必须用本生产领域的产品来补偿。但是不一定要由这个生产领域内的同一个生产者自己来补偿。农业越是发达,它的一切要素也就越是不仅形式上,而且实际上作为商品加入农业,也就是说,这些要素来自外部,是另外一些生产者的产品(种子、肥料、牲畜、畜产品等)。在工业中,例如铁不断地转移到机器制造厂,而机器不断地转移到铁矿,这种情形同小麦从谷仓转移到土地又从土地转移到租地农场主的谷仓一样,是经常发生的。在农业中,产品直接补偿自己。铁不能补偿机器。但是,同机器价值相等的一定量铁[在制铁业者和机器制造业者进行交换时]给前者补偿机器而给后者补偿铁,因为[机器制造业者卖给制铁业者的]机器本身按价值来说由铁补偿。

根本不能想象,如果土地耕种者把他用在100镑产品上的90+(10/11)镑,比如说,这样来计算:20镑用于种子等,20镑用于机器等,50+(10/11)镑用于工资,那末利润率会有什么差别。因为他对这笔总额要求10%的利润。被他当作种子的20镑产品不包含利润。但是,它们是同包含例如10%利润的以机器为形式的20镑完全一样的20镑。诚然,这可能仅仅在形式上如此。以机器为形式的20镑,实际上可能同以种子为形式的20镑一样,也不代表任何利润。例如,当上述20镑仅仅补偿机器制造业者的不变资本中那些取自比如农业的组成部分的时候,情形就是这样。

认为一切机器都作为农业的不变资本加入农业,是错误的,同样,认为一切原料都加入加工工业,也是错误的。相当大一部分原料留在农业中,它仅仅是不变资本的再生产。另一部分作为生活资料直接加入收入中,而且其中一部分如水果、鱼、牲畜等,不通过任何“制造过程”。因此,要工业替农业所“制造”的全部原料付款,是不正确的。当然,那些除了工资、机器之外还有原料作为预付加入生产的加工工业部门,同提供这种作为预付加入生产的原料的那些农业部门比起来,预付资本必然较大。也可以假定,如果在这些加工工业部门中存在自己的(不同于一般利润率的)利润率,那末在这里这种利润率就要小于农业中的利润率,其原因正是由于在这里使用的劳动较少。因此,在剩余价值率相同的情况下,较大的不变资本和较小的可变资本,必然提供较小的利润率。但是,这一点也适用于加工工业的一定部门同加工工业的另一些部门的关系,以及农业(在经济学的意义上)的一定部门同农业的另一些部门的关系。至少在真正的农业中恰恰存在这种情况,因为农业虽然为工业提供原料,但是在本身的领域中仍然有原料、机器和工资作为自己的各项支出,而工业对于这种原料,即农业从自身来补偿而不是通过同工业品交换来补偿的那部分不变资本,是决不向农业支付代价的。

\tsectionnonum{[(5)洛贝尔图斯的地租理论的错误前提]}

[465]现在把洛贝尔图斯先生的思路作一概括。

首先,他照自己的想象描写了(独立经营的)土地所有者既是资本家又是奴隶主的情况。后来分离了。从工人那里剥夺来的那部分“劳动产品”——“一种实物租”——现在分成“地租和资本利润”。(第81—82页)(霍普金斯先生——见札记本\endnote{马克思指他的关于政治经济学的第XII本札记本。在这个札记本的封面上马克思亲笔写着:“1851年7月于伦敦”。马克思在这个札记本的第14页上摘录了霍普金斯的著作《关于调节地租、利润、工资和货币价值的规律的经济研究》(1822年伦敦版)中的一段话,这里所指的就是这段话,后来马克思在《剩余价值理论》手稿第XIII本(第669b页)的封面上又引用了这段话。本版在第二册的《附录》中发表了这段引文(见本册第672页)。——第52页。}——对这一点的说明还要简单粗糙得多。)然后,洛贝尔图斯先生把“原产品”和“工业品”(第89页)在土地所有者和资本家之间进行分配——一种petitioprincipii〔本身尚待证明的论据〕。[事实上是]一个资本家生产原产品,另一个资本家生产工业品。相反,土地所有者什么也不生产,他甚至也不是“原产品的所有者”。[土地所有者就是“原产品的所有者”]这个观念是洛贝尔图斯先生这种德国“地主”所特有的。在英国,资本主义生产是在工业和农业中同时开始的。

关于“资本盈利率”(利润率)形成的方法,洛贝尔图斯先生只是用下面一点来说明:现在有了以货币为形式的“表示盈利对资本之比”的盈利“标准”,从而“为资本盈利平均化提供了一个适当的尺度”。(第94页)洛贝尔图斯根本不知道,这种利润的均等同每个生产部门中“租”和无酬劳动之间的相等,是矛盾的,因此,商品的价值同它们的平均价格必然是不一致的。这个利润率对农业也是一个正常标准,因为“财产的收入只能按资本计算”(第95页),在工业中“使用着国民资本的极大部分”。(第95页)他一点也没有提到,随着资本主义生产的发展,农业本身不仅在形式上而且在实质上也发生了变革;土地所有者成为纯粹的钱袋,在生产中不再执行任何职能。在洛贝尔图斯看来:

\begin{quote}{“在工业中,还要把农业的全部产品的价值——作为材料——包括在资本内,而在原产品生产中就不会有这种情况。”(第95页)}\end{quote}

说全部产品,那是错误的。

接着洛贝尔图斯问道,扣除了工业利润即资本利润之后,是否还剩下“归原产品的租部分”,“如果有,那是由于什么原因”。(第96页)

洛贝尔图斯认为:

\begin{quote}{“原产品同工业品一样,是按耗费的劳动交换的,原产品的价值只等于它所耗费的劳动。”(第96页)}\end{quote}

的确,正如洛贝尔图斯所说的,李嘉图也认为是这样。但是这至少初看起来是错误的,因为商品不是按它们的价值,而是按不同于这些价值的平均价格交换的,并且这是由商品价值决定于“劳动时间”引起的,是由这个表面上看来同该现象相矛盾的规律引起的。如果原产品除了提供平均利润之外还提供一个不同于平均利润的地租,那末,这只有在原产品不是按照平均价格出卖的时候才有可能,而为什么会这样,这正是需要说明的。但是,我们来看看洛贝尔图斯是怎样推论的。

\begin{quote}{“我已经假定,租〈剩余价值,无酬劳动时间〉是按原产品和工业品的价值分配的,而这个价值是由耗费的劳动〈劳动时间〉决定的。”(第96—97页)}\end{quote}

我们首先来验证这第一个假定。这个假定的意思,换句话说,不过是各商品包含的剩余价值之比等于这些商品的价值之比,再换句话说,各商品中包含的无酬劳动量之比等于这些商品中包含的全部劳动量之比。如果商品A和商品B包含的劳动量之比是3∶1,那末它们包含的无酬劳动——剩余价值——之比也是3∶1。这是再错误不过的了。假定,必要劳动时间既定,等于10小时,一个商品(A)是30个工人的产品,另一个商品(B)是10个工人的产品。如果30个工人每天只劳动12小时,那末,他们所创造的剩余价值等于60小时,或等于5天(5×12),如果10个工人每天劳动16小时,那末,他们所创造的剩余价值也是等于60小时。这样,商品A的价值就等于30×12,即360劳动小时,或30个工作日{12小时=1工作日},而商品B的价值等于160劳动小时,或13+(1/3)工作日。商品A和商品B的价值之比是360∶160,即9∶4。两个商品包含的剩余价值之比是60∶60,即1∶1。在这种情况下,虽然价值之比是9∶4,剩余价值却相等。

[466]因此,首先,在绝对剩余价值不同,也就是超出必要劳动之外的劳动时间延长程度不同的时候,因而在剩余价值率不同的时候,各商品的剩余价值之比不等于这些商品的价值之比。

第二,假定剩余价值率相同,剩余价值——且不谈与流通和再生产过程有关的其他情况,——不取决于两个商品中包含的劳动的相对量,而取决于资本中用于工资的部分对用于原料和机器等不变资本的部分之比;而这个比例在价值相同的商品中可能完全不同,不论这些商品是“农产品”还是“工业品”,——这同问题根本没有关系,至少初看起来是如此。

因此,洛贝尔图斯先生的第一个假定,——如果商品价值决定于劳动时间,那末,不同商品中包含的无酬劳动量(或它们的剩余价值)就与价值成正比,——是根本错误的。从而下面的说法也是错误的:

\begin{quote}{“租是按原产品和工业品的价值分配的”,如果“这个价值是由耗费的劳动决定的”。(第96—97页)“当然这也就是说,这些租部分的量,不决定于据以计算盈利的资本的量,而决定于直接耗费的劳动——不论是农业劳动或工业劳动——加上由于工具和机器的损耗应当予以计算的劳动。”(第97页)}\end{quote}

这又错了。剩余价值量(这就是所谓“租部分”,因为洛贝尔图斯把租理解为与利润和地租不同的一般东西)只取决于直接耗费的劳动,不取决于固定资本的损耗,也不取决于原料的价值,总之,不取决于不变资本的任何部分。

当然,这种损耗决定固定资本必须依什么比例进行再生产(固定资本的生产同时取决于资本的新形成,资本的积累)。但是,在固定资本的生产中实现的剩余劳动,同这个固定资本作为固定资本加入的生产领域是没有关系的,就象例如加入原料生产的剩余劳动同上述这个生产领域没有关系一样。相反,在一切生产部门中,如果剩余价值率是既定的,剩余价值就只决定于所使用的劳动量;如果使用的劳动量是既定的,剩余价值就只决定于剩余价值率,这对于一切生产部门——对于农业、机器制造业和加工工业,都同样适用。洛贝尔图斯先生想把损耗“塞进来”,是为了把“原料”推出去。

\begin{quote}{洛贝尔图斯先生认为,相反,“包含在材料价值中的那部分资本”决不能对租部分的量有什么影响,因为“比如说,耗费在作为原产品的羊毛上的劳动,不能加入纱或布这种特殊产品所耗费的劳动”。(第97页)}\end{quote}

纺或织所需的劳动时间,取决于生产机器所必要的劳动时间即机器的价值,同取决于原料所耗费的劳动时间完全一样,或者,更确切地说,不取决于前者,同不取决于后者完全一样。机器和原料两者都加入劳动过程,但两者都不加入价值增殖过程。

\begin{quote}{“相反,原产品的价值即材料价值仍然作为资本支出包括在资本总额中,所有者就是按这个资本总额来计算作为盈利归工业品的租部分。而在农业资本中没有这一部分资本。农业不需要先于它生产的产品作为材料,生产一般是从农业开始的;在农业中同材料相似的财产部分可以说是土地本身,但土地是假定不要任何费用的。”(第97—98页)}\end{quote}

这是德国农民的观念。在农业中(除矿山、渔业、狩猎业以外,但是畜牧业决不除外),种子、饲料、牲畜、矿肥等是[467]用来生产产品的材料,而这种材料是劳动的产品。随着企业化农业的发展,这些“支出”也发展了。任何生产——只要不是指单纯的攫取和占有——都是再生产,因而都需要“先于它生产的产品作为材料”。在生产中成为结果的一切同时也是前提。大规模农业越发达,它购买“先于它生产的”产品和卖出自己的产品就越多。一旦租地农场主一般依存于出卖自己的产品,各种农产品(如干草)的价格由于农业中也有生产领域的划分而开始确定下来,这些支出也就以商品的形式——通过计算货币转化为商品——加入农业。如果农民把他出卖的一夸特小麦算作收入,却不把他下到地里的一夸特小麦算作“支出”,那末,就连他也一定会弄得晕头转向。此外,让洛贝尔图斯先生去试试在某一个没有“先于它生产的产品”的地方“开始生产”比如说麻或丝吧。这完全是荒谬之谈。

因此,洛贝尔图斯进一步得出的全部结论也是荒谬的:

\begin{quote}{“因此,对决定租部分的量有影响的两部分资本,是农业和工业共有的;但是,下面那部分资本不是它们共有的,这部分资本对决定租部分的量没有影响,却同其他部分资本加在一起来计算由上述两部分资本决定的作为盈利的租部分;这第三部分资本只有在工业资本中存在。我们曾经假定,无论原产品的价值还是工业品的价值都决定于所耗费的劳动,而租是按照这个价值在原产品和工业品的所有者之间进行分配。因此,如果说在原产品和工业品的生产中得到的租部分同有关的产品所耗费的劳动量成比例,那末,用在农业和工业中的、把这些租部分当作盈利来分配时所依据的资本(在工业中完全是依据资本,在农业中则依据工业中确定的盈利率)之间的比例,仍然不同于上述劳动量之间的比例以及由上述劳动量决定的租部分之间的比例。相反,在归原产品和工业品的租部分的量相等的情况下,工业资本大于农业资本,所大之数相当于包含在工业资本中的材料价值的总额。因为这个材料价值增大了把租部分作为盈利计算时所依据的工业资本,但是不增大盈利本身,从而引起资本盈利率(它也调节农业盈利)的下降,所以,在农业的租部分中,必然剩下一个依照这个盈利率计算盈利时所吸收不了的部分。”(第98—99页)}\end{quote}

第一个错误的前提:如果工业品和农产品按它们的价值(也就是按照生产它们所需要的劳动时间)交换,那末,它们就给自己的所有者提供等量剩余价值,或者说,等量无酬劳动。两种剩余价值之比不等于两种商品的价值之比。

第二个错误的前提:因为洛贝尔图斯已经以利润率(他把利润率称为“资本盈利率”)作为前提,所以,他那个商品按它们的价值交换的前提是错误的。一个前提排斥另一个前提。为了使(一般)利润率能够存在,商品的价值就必须已经发生形态变化而成为平均价格,或者处于不断的形态变化过程中。在这个一般利润率中,每个生产领域的由剩余价值对预付资本之比决定的特殊利润率平均化了。那末,为什么在农业中就不是这样呢?这正是问题所在。但是洛贝尔图斯先生甚至对问题的提法也不对,因为他第一,假定已经有一个一般利润率存在;第二,假定特殊利润率(以及它们之间的差异)没有平均化,也就是说商品按它们的价值交换。

第三个错误的前提:原料的价值不加入农业。实际上,这里的种子等等的预付是不变资本的组成部分,租地农场主也是把它们作为不变资本的组成部分来计算的。随着农业变成一个纯粹的企业部门以及资本主义生产在农村中确立,[468]随着农业为市场而生产,生产商品,生产为出卖而不是为自己消费的物品,农业也就计算它的支出,把支出的每个项目都看成商品,不管该物品是农业从本身(即从自己生产中)购买的还是向第三者购买的。随着产品变成商品,生产要素当然也变成商品,因为这些生产要素完完全全就是这些产品。因此,既然小麦、干草、牲畜、各种种子等等作为商品出卖,而且具有重要意义的正是出卖这些产品而不是用它们来直接消费,那末,它们也就作为商品加入生产;租地农场主如果不会把货币当作计算货币来用,他就是十足的傻瓜。最初,这只是计算的形式方面。但是,以下的情况也同样发展起来:某个租地农场主购买他在生产中支出的产品,即种子、别人的牲畜、肥料、矿物质等,同时又出卖他的收入;因此对于单个租地农场主来说,这些预付[不仅在实际上,而且]在形式上也是作为预付加入生产的,因为它们是买来的商品。(现在这些东西对于租地农场主来说已经始终是商品,是他的资本的组成部分,而当租地农场主把它们以实物形式重新投入生产的时候,他是把它们当作卖给自己这个生产者的东西看待的。)并且,随着农业的发展,随着最后的产品越来越以工厂的方式、按资本主义的生产方式生产出来,情况也就越来越是这样了。

因此,说这里有一部分资本加入工业而不加入农业,是错误的。

可见,如果照洛贝尔图斯的(错误的)前提,农产品和工业品提供的“租部分”(即剩余价值的份额)是同这些产品的价值成比例的,换句话说,如果具有等量价值的工业品和农产品为它们的所有者提供等量剩余价值,也就是包含等量无酬劳动,那末[即使在这种前提下],也绝对不会因为有一部分资本只加入工业(用于原料)而不加入农业,就发生不成比例的情况,以致比如同一剩余价值在工业中据说要按增加了这个组成部分的资本来计算,从而提供较小的利润率。要知道资本的这样一个组成部分也是加入农业的。因此,剩下要解决的问题只是:这个组成部分是否依同一比例加入农业?但是,在这里,我们遇到的是纯粹量的差别,而洛贝尔图斯先生想找的却是“质的”差别。这种量的差别在不同的工业生产领域也存在。这些差别在一般利润率中被拉平了。为什么工业和农业之间的差别(如果这种差别存在的话)不会被拉平呢?既然洛贝尔图斯先生让农业参与获得一般利润率,为什么又不让它参与形成这个一般利润率呢?当然,如果这样,他的全部理论就完了。

第四个错误的前提:洛贝尔图斯把机器等的损耗这一部分不变资本归入可变资本,即归入创造剩余价值、特别是决定剩余价值率的那部分资本,却不把原料归入其中,这是一个任意作出的错误前提。作出这个错误计算,是为了能够得到一开头就希望得到的计算结果。

第五个错误的前提:如果洛贝尔图斯先生要区别农业和工业,那末,就应当知道,由机器、工具,即由固定资本构成的资本要素是完全属于工业的。只要这个资本要素作为要素加入某个资本,它总是仅仅加入不变资本,丝毫不能提高剩余价值。另一方面,它作为工业品,是一定生产领域的结果。因此,它的价格,或者说,它在全部社会资本中所占的价值部分,同时代表一定量剩余价值(同原料的情况完全一样)。这个要素诚然加入农产品,但是它是来自工业的。洛贝尔图斯先生在计算中既然把原料看作从外面加入工业的资本要素,那末他就应该把机器、工具、容器、建筑物等看作从外面加入农业的资本要素。他就应该说,加入工业的只有工资和原料(因为固定资本只要不是原料,就是工业品,是工业自己的产品);而加入农业的只有工资[469]和机器等固定资本,因为原料只要不包含在工具等等之中,就是农产品。既然工业中少了一个生产费用“项目”,就应当研究在工业中是怎样计算的。

第六:一点不错,在采矿工业、渔业、狩猎业、林业(只指自然生长的林木)等部门,一句话,在采掘工业(对于不进行实物再生产的原产品的采掘)中,除了辅助材料之外,没有原料加入生产。这一点对于农业是不适用的。

但是,同样不错,在工业的一个很大部分即运输业中存在着同样的情况。这里的支出只用于机器、辅助材料和工资。

最后,无庸置疑,在其他一些工业部门中,相对地说,只有原料和工资加入生产,而没有机器即固定资本等加入生产,例如裁缝业等就是这样。

在所有这些情况下,利润量,即剩余价值对预付资本之比,不取决于预付资本——在扣除了可变资本即用于工资的资本部分以后——是由机器构成还是由原料构成,还是由两者一起构成,而是取决于这部分资本同用于工资的那部分资本相比有多大。因此,在不同的生产领域就必然存在着(撇开由流通引起的变化不说)不同的利润率,这些不同的利润率的平均化,恰好形成一般利润率。

洛贝尔图斯先生模糊地猜到的,是剩余价值同它的特殊形式,特别是同利润的区别。但是他不得要领,因为在他那里,问题一开始就只是要说明一定的现象(地租),而不是要揭示普遍规律。

在所有生产部门中都有再生产;但是这种同生产联系的再生产只有在农业中才是同自然的再生产一致的,在采掘工业中就不是这样。因而,在采掘工业中,实物形式的产品不再成为它本身再生产的要素{以辅助材料形式出现的场合除外}。

农业、畜牧业等和其他生产部门所不同的,第一,不是产品在这里成为生产资料,因为一切不具有个人生活资料的最后形式的工业品都是这样;即使具有这种最后形式的工业品也是这样,因为它们是生产者本身的生产资料,生产者靠消费它们把自己再生产出来,并保持自己的劳动能力。

第二,也不是农产品作为商品,即作为资本的组成部分加入生产;它们是以从生产中出来的形式加入生产的:它们作为商品从生产中出来,又作为商品再加入生产,——商品是资本主义生产的前提,又是它的结果。

因此,剩下来只是第三,产品作为本身的生产资料加入生产过程,而这个生产过程的产物就是这些产品。在机器方面也有这种情况。机器生产机器。煤帮助把煤提出矿井,煤运输煤等。在农业中,这种情况表现为自然的过程,这个过程是由人引导的,虽然它也“略微”创造人本身。而在其他生产部门,这种情况直接表现为生产的作用。

但是,如果洛贝尔图斯先生因此便认为,农产品由于作为“使用价值”(在工艺上)加入再生产时所具有的特殊形式,就不能作为“商品”加入再生产,那末,他就完全走入歧途了,他显然是根据对过去的回忆,那时,农业还不是资本主义企业,只有超过生产者本身消费的农产品的余额才变成商品,而这些产品,只要它们加入生产,对农业来说就不是商品。这是根本不了解资本主义生产方式对整个生产过程产生的影响。对资本主义生产来说,一切具有价值——因而从可能性来说是商品——的产品,也都作为商品来计算。

\tsectionnonum{[(6)洛贝尔图斯不理解工业和农业中平均价格和价值之间的关系。平均价格规律]}

假定在采矿工业中,仅仅由机器构成的不变资本等于500镑,用于工资的资本也等于500镑,那末,如果剩余价值等于40%,即200镑,利润就等于20%。

因此,现在是:

\todo{}

如果在有原料加入的加工工业部门(以及农业部门)支出同样多的可变资本,并且,如果使用这笔可变资本(即使用这一定数目的工人)需要机器等500镑,那末,实际上这里就会加上作为第三个要素的材料价值,假定这也是500镑。于是,现在是:

\todo{}

这个200镑剩余价值现在要以1500镑来除,结果只等于[13+(1/3)]%。如果在第一种情况下是运输业的话,上述例子还是适合的。如果在第二种情况下,比例是:机器100,原料400,那末利润率仍旧一样。

[470]因此,洛贝尔图斯先生所想象的是,如果在农业中在工资上支出100,加上在机器上支出100,那末,在工业中就是在机器上支出100,在工资上支出100,在原料上支出x。列成图表就是:

\todo{}

因而,利润率无论如何小于1/4。由此在I中也就产生了地租。

第一,农业和加工工业之间的这种差别是想象出来的,并不存在;因此,它对于那个决定其他一切地租形式的地租形式来说,是没有任何意义的。

第二,洛贝尔图斯先生在任何两个工业部门的利润率之间都可以找到这种差别;这种差别取决于不变资本量同可变资本量之比,而这个比例本身,又是既可以决定于原料的加入,也可以不决定于原料的加入。在既有原料又有机器加入的工业部门,原料的价值,也就是原料在总资本中所占的相对量,自然,如我在前面指出的,有非常重要的意义。\endnote{在注17提到的篇幅很长的关于约翰·斯图亚特·穆勒的补充部分(手稿第335—339页),马克思谈到“原料的低廉或昂贵对原料加工工业的重要性”。按照马克思的指示,本版把这个补充部分移至《剩余价值理论》第三册中关于李嘉图学派的解体那一章。——第64、496页。}这同地租毫无关系。

\begin{quote}{“只有在原产品的价值降到所耗费的劳动以下的时候,在农业中归原产品的整个租部分才有可能被按资本计算的盈利吸收;因为那时,这个租部分会大大减少,以至它和农业资本(虽然其中不包含材料价值)之间的百分比,跟归工业品的租部分和工业资本(虽然其中包含材料价值)之间的百分比相同;因此,只有在这种情况下,在农业中才有可能除了资本盈利以外不再剩下任何地租。但是,既然实际交换照例至少趋向于价值等于耗费的劳动这个规律,那末地租也照例存在;如果没有地租而只有资本盈利,那末,这不是象李嘉图设想的原始状态,而只是一种反常现象。”(第100页)}\end{quote}

因此,如果仍旧用前面的例子,那末情况如下(为了更清楚起见,我们只假定原料等于100镑):

\todo{}

这里,因为农产品比它的价值低16+(2/3)镑出卖,所以农业中和工业中的利润率相等,也就没有什么剩下作为地租。这个对农业来说是错误的例子即使是正确的,那末,原产品的价值降到“所耗费的劳动以下”这种情况,也只是完全符合于平均价格规律而已。其实需要说明的倒是为什么“例外地”在农业中有一部分不是这样,为什么在农业中全部剩余价值(或者至少是超过其他生产部门的剩余价值额;超过平均利润率的余额)都留在这个特殊生产部门的产品价格中,而不加入形成一般利润率的总结算。由此可见,洛贝尔图斯不知道什么是(一般)利润率和什么是平均价格。

为了把这个平均价格规律说清楚,——这比分析洛贝尔图斯的观点重要得多,——我们举五个例子。假定剩余价值率全都一样。

完全没有必要拿具有等量价值的商品来比较;商品应当只按照它们的价值来比较。为了简便起见,这里拿来比较的是等量资本所生产的商品。

[471]

这里,在I、II、III、IV、V各类(这是五个不同的生产领域)中,商品的价值分别是1100、1200、1300、1150和1250镑。如果这些商品按它们的价值进行交换,这些数目也就是它们交换时的货币价格。在所有这些生产领域,预付资本量相同,都等于1000镑。如果这些商品按它们的价值交换,那末,利润率在I只有10%,在II大一倍,即20%,在III是30%,在IV是15%,在V是25%。如果把各种利润率加起来,那末它们的和等于10%+20%+30%+15%+25%=100%。

如果拿所有五个生产领域的全部预付资本来考察,那末,它的一部分(I)提供10%,另一部分(II)提供20%,等等。全部资本平均提供的利润等于这五部分提供的平均量。这就是:

\todo{}

即20%。实际上,我们看到五个生产领域预付的5000镑资本提供的利润等于100+200+300+150+250=1000,也就是1000除以5000等于1/5,即20%。同样,如果我们计算总产品的价值,它是6000镑;超过5000镑预付资本的余额是1000镑,等于预付资本的20%,等于全部产品的1/6或[16+(2/3)]%。(这又是另一种计算法。)

因此,要使每一笔预付资本(I、II、III等等),或者同样可以说,等量资本,或者说,只按量的大小比例,也就是只按在预付总资本中所占的比例来考察的资本,从归总资本的剩余价值中确实获得自己的一份,那末,归每一笔资本的只能是20%的利润,不过必须归它的也正是这么多。[472]而要使这种情况成为可能,不同领域的产品就必须有时高于自己的价值出卖,有时多少低于自己的价值出卖。换句话说,全部剩余价值必须不是按各个生产领域生产多少剩余价值的比例,而是按预付资本的大小的比例在它们之间进行分配。所有生产领域都要按1200镑出卖自己的产品,以使产品价值超过预付资本的余额等于预付资本的1/5,即等于20%。

这种分配的结果如下:

这里我们看到,只有在一种情况下(II)平均价格等于商品的价值,因为这里的剩余价值恰好等于正常平均利润200。在所有其他情况下,都是把剩余价值从一种商品上拿走而加到另一种商品上去,有时多一点,有时少一点等等。

洛贝尔图斯先生本来应该加以说明的是,为什么在农业中不是这种情况,为什么在农业中商品必定按它们的价值,而不是按平均价格出卖。

竞争的作用是把利润平均化,也就是使商品的价值转化为平均价格。照马尔萨斯先生的说法,单个资本家希望从他的资本的每一部分都得到同样大小的一份利润\endnote{托·罗·马尔萨斯《政治经济学原理》1836年伦敦第2版第268页。马克思在《剩余价值理论》第三册《托·罗·马尔萨斯》一章中,引用和分析了马尔萨斯的这句话(手稿第765—766页)。——第68页。},——换句话说,这不过意味着资本家把资本的每一部分(不论它的有机的职能如何)都看成利润的独立源泉,资本的每一部分在他看来都是这样的源泉,——同样,对于资本家阶级来说,每一个资本家都把自己的资本看成同其他任何等量资本一样,是提供同量利润的源泉;就是说,把每笔投在单个生产领域的资本,只看成预付在总生产上的总资本的一部分;每笔资本,都按自己的量,自己的股份,按自己在总资本中所占的份额,在总剩余价值中,在无酬劳动或无酬劳动产品总量中要求自己的一份。这个假象使资本家(总之,对资本家来说,在竞争中,一切都以颠倒的形式出现),不仅使资本家,并且使某些最忠实于资本家的伪善者和文人确认,资本是一个与劳动无关的收入源泉,因为实际上各个生产领域的资本的利润,决不是单独由它自己“生产”的无酬劳动量决定的;这个利润落进盈利总额的大锅里,各个资本家都从那里按他参加总资本的份额获得自己的一份。

可见,洛贝尔图斯的阐述是无稽之谈。附带还要指出,在某些农业部门,例如在独立的畜牧业中,可变资本,即用于工资的资本同资本的不变部分比起来是极小的。

[洛贝尔图斯说:]

\begin{quote}{“租金就其本质来说总是地租。”(同上,第113页)}\end{quote}

不对。租金总是付给土地所有者的;如此而已。但是,如果象实践中常有的情况那样,租金,部分地或者全部地,是正常利润或者正常工资的扣除部分{实际的剩余价值即利润加地租,决不是工资的扣除部分,而是工人的劳动产品扣除了工资以后剩下来的部分},那末这个租金从经济学的观点来看就不是地租,并且一旦[473]竞争的条件恢复了正常工资和正常利润,这一点就立刻为实践证明了。

竞争不断使商品的价值转化为平均价格,在平均价格中,除了上表II的情况以外,一个生产领域的产品经常出现价值的追加部分,而另一个生产领域的产品则经常出现价值的扣除部分,只有这样才得出一般利润率。在可变资本对预付资本总额之比{假定剩余劳动率是既定的,相同的}符合于社会资本的平均比例的生产领域,商品的价值就等于平均价格;因此这里既没有价值的追加部分,也没有价值的扣除部分。如果由于特殊条件(这些条件这里不需要细说),在一定的生产领域,商品的价值——虽然超过平均价格——没有任何扣除(不是暂时地而是平均地),那末,全部剩余价值保持在某一个特殊生产领域,——虽然这种情况使商品的[实现的]价值提高到平均价格以上,因而提供一个大于平均利润率的利润率,——应看成是这些生产领域的特权。这里要作为特殊性、作为例外来研究和说明的,不是商品的平均价格降到它们的价值以下,——这是一般现象,是平均化的必要前提,——而是该商品为什么不同于其他商品,恰恰按它们的高于平均价格的价值来出卖。

商品的平均价格等于它的生产费用(商品中的预付资本,不论是工资、原料、机器还是其他)加平均利润。因此,如果[平均利润率是既定的]象在上述例子中那样,平均利润等于20%,即1/5,那末每个商品的平均价格等于C(预付资本)+P/C(平均利润率)。如果C+P/C等于这个商品的价值,也就是说,如果这个生产领域生产出来的剩余价值M=P,那末,商品的价值就等于它的平均价格。如果C+P/C小于商品的价值,因而这个生产领域生产出来的剩余价值M大于P,那末,商品的[实现的]价值就降低到它的平均价格水平,它的剩余价值的一部分就加到其他商品的价值上去。最后,如果C+P/C大于商品的价值,也就是M小于P,那末,商品的[实现的]价值就提高到它的平均价格水平,其他生产领域生产出来的剩余价值就加到这个商品的[内在]价值上来。

最后,如果有些商品,虽然它们的价值大于C+P/C,还是按它们的价值出卖,或者它们的[实现的]价值至少没有降到它们的正常平均价格C+P/C,那末这里一定有一些使这些商品成为例外的特殊条件在发生作用。在这种情况下,这些生产领域实现的利润就高于一般利润率。如果资本家在这里得到一般利润率,那末土地所有者就能够以地租形式取得超额利润。

\tsectionnonum{[(7)洛贝尔图斯在决定利润率和地租率的因素问题上的错误]}

我称为利润率、利率和地租率的,洛贝尔图斯称为“资本盈利的高度和利息的高度”(第113页)[和“地租的高度”]。

\begin{quote}{“资本盈利的高度和利息的高度决定于它们对资本之比……一切文明民族都把资本额100作为计算单位,也作为计算高度的标准。因此,资本额100所得的盈利量或利息量的比例数越大,换句话说,资本提供的‘百分率越大’,盈利和利息就越高。”(第113—114页)“地租和租金的高度决定于它们对一定地段之比。”(第114页)}\end{quote}

这种说法不合适。地租率首先应当按资本计算,因而应当作为商品价格超过商品生产费用和超过价格中构成利润的部分的余额来计算。洛贝尔图斯先生按英亩和摩尔根\authornote{土地面积单位,合25.53英亩。——译者注}计算,这样计算,内部联系就没有了,他抓住了[474]事物的表面形式,因为表面形式给他说明某些现象。一英亩提供的地租是地租额,是地租的绝对量。地租额在地租率不变甚至下降的情况下也可以增加。

\begin{quote}{“土地价值的高度决定于一定地段的地租的资本化。一定面积的地段的地租的资本化所提供的资本额越大,土地价值就越高。”(第114页)}\end{quote}

“高度”一词在这里是荒谬的。这个词实际上表示对什么的比例呢?资本化在利率为10%时比在20%时提供的资本额大,这是清楚的,但是这里的计算单位是100。“土地价值的高度”这整个说法,同商品价格的高或低一样,是一般化的说法。

洛贝尔图斯先生现在想研究:

\begin{quote}{“决定资本盈利的高度和地租的高度的是什么?”(第115页)}\end{quote}

\tsubsectionnonum{[(a)洛贝尔图斯的第一个论题]}

首先他研究:决定“一般租的高度”的是什么,也就是说,决定剩余价值率的是什么?

\begin{quote}{“(I)就一定的产品价值,或者一定量劳动的产品来说,或者同样可以说,就一定的国民产品来说,一般租的高度和工资的高度成反比,和一般劳动生产率的高度成正比。工资越低,租就越高;一般劳动生产率越高,工资就越低,租就越高。”(第115—116页)洛贝尔图斯说,租的“高度”——剩余价值率——取决于“这个剩下作为租的部分的大小”,也就是说,取决于从总产品中扣除了工资之后剩下来的部分的大小,在这里“产品价值中用于补偿资本的那一部分可以撇开不谈”。(第117页)}\end{quote}

这是对的(我指的是考察剩余价值时把资本的不变部分“撇开不谈”)。

洛贝尔图斯的一个有些奇怪的观点:

\begin{quote}{“如果工资降低,也就是说,如果工资今后在全部产品价值中占较小的一份,那末另一部分租{即工业利润}作为盈利计算时所依据的总资本也变小。但是,规定盈利和地租的高度的,只是转化为资本盈利或地租的价值同这个价值作为盈利或地租计算时所依据的资本或土地面积之比。因此,如果工资留下一个较大的价值作为租,那末,就要依据变小了的资本和保持不变的土地面积,来计算这个较大的作为资本盈利和地租的价值;由此得出的盈利和地租的相对量就大,因此两者合在一起,或者说一般租,就较高……假定产品价值不变……这只是因为,花费在劳动上的工资减少了,花费在产品上的劳动还没有减少。”(第117—118页)}\end{quote}

最后这句话是对的。但是,说可变资本即用于工资的资本减少,不变资本就必定减少,那就错了;换句话,说利润率{这里把剩余价值同土地面积之比等等根本不恰当的提法撇开不谈}由于剩余价值率提高就必然提高,那就错了。工资降低,比方说,是由于劳动生产率提高,而生产率的这种提高,在一切场合都表现为同一个工人在同一时间内加工更多的原料,——因此,这一部分不变资本增加了,机器和机器的价值也是这样。因此,利润率在工资减少的情况下也可能降低。利润率取决于剩余价值量,而剩余价值量不仅取决于剩余价值率,而且取决于被雇用的工人人数。

洛贝尔图斯正确地规定必要工资等于

\begin{quote}{“工人的必要生活费的总额,即对于一定国家和一定时期来说大致相同的一定实际产品量”。(第118页)}\end{quote}

[475]但是洛贝尔图斯先生把李嘉图提出来的利润和工资成反比以及这个比例决定于劳动生产率的原理叙述得非常混乱,极其笨拙。这种混乱一部分是由于他不把劳动时间作为尺度,却愚蠢地把产品量作为尺度,并且荒唐地去区别“产品价值的高度”和“产品价值的大小”。

这个好汉所谓的“产品价值的高度”,不过是指产品对劳动时间之比。如果在同一劳动时间内生产许多产品,那末产品价值即每一部分产品的价值就低,在相反的场合,结果也相反。如果1工作日以前提供100磅棉纱,后来提供200磅棉纱,那末棉纱的价值在后一种情况下比前一种情况下小了一半。在前一种情况下,1磅棉纱的价值等于1/100工作日;在后一种情况下,1磅棉纱的价值等于1/200工作日。因为工人得到的是同量产品,不管产品的价值是高还是低,也就是说,不管产品包含的是较多的劳动还是较少的劳动,所以工资和利润成反比,并且工资根据劳动生产率的不同而在总产品中占较大的部分或者较小的部分。这一点洛贝尔图斯用混乱的论点叙述如下:

\begin{quote}{“……如果工资作为工人的必要生活费是一定的实际产品量,那末工资在产品价值高的情况下必然是一个大的价值,在产品价值低的情况下必然是一个小的价值,因此,既然假定加入分配的是同一产品价值,工资在产品价值高的情况下必然吸收产品价值的相当大的部分,而在产品价值低的情况下必然吸收产品价值的小部分,其结果也必然把产品价值的一个相当大的份额或一个小的份额留下作租。但是,如果产品价值等于产品所耗费的劳动量这个定律有效,那末决定产品价值的高度的仍然只是劳动生产率,或者说,产品量对生产这些产品所花费的劳动量之比……如果同量劳动生产的产品较多,换句话说,如果生产率提高了,那末同量产品中包含的劳动就较少;反之,如果同量劳动生产的产品较少,换句话说,如果生产率降低了,那末同量产品中包含的劳动就较多。但是,劳动量决定产品的价值,而一定量产品的相对价值决定产品价值的高度……因而一般劳动生产率越高,一般租就必然……越高。”(第119—120页)}\end{quote}

可是,这种说法只有在工人生产的产品属于作为生活资料——根据传统或者由于必要——加入工人消费的那一类产品的时候,才是正确的。如果工人生产的产品不属于这一类产品,那末工人的劳动生产率对于工资和利润的相对高度,就象对于一般剩余价值量一样,是完全无关紧要的。[在这种情况下]全部产品中作为工资归工人所得的是同样大的价值部分,不管表现这个价值部分的产品数或产品量是大还是小。在这种场合,不管劳动生产率发生什么变化,产品价值的分配不会发生任何变化。

\tsubsectionnonum{[(b)洛贝尔图斯的第二个论题]}

\begin{quote}{“(II)如果在产品价值既定的情况下一般租的高度既定,那末,地租的高度和资本盈利的高度既互成反比,又分别与原产品生产和工业品生产中的劳动生产率成反比。地租越高,资本盈利就越低;地租越低,资本盈利就越高;反过来也是一样。原产品生产的劳动生产率越高,地租就越低,资本盈利就越高;原产品生产的劳动生产率越低,地租越高,资本盈利就越低;工业品生产的劳动生产率越高,资本盈利就越低,地租就越高;工业品生产的劳动生产率越低,资本盈利就越高,地租就越低。”(第116页)}\end{quote}

一开始(在第一个论题中)我们已看到李嘉图关于工资和利润成反比的规律。

现在我们看到李嘉图的第二个规律:利润和地租成反比;这个规律被洛贝尔图斯用另一种方式发挥了,或者不如说,弄乱了。

十分清楚,如果一定的剩余价值在资本家和土地所有者之间分配,前者所得越大,后者所得就越小,反过来也是一样。但是,洛贝尔图斯先生在这里还加进了他自己的一些东西,这些东西应该比较详细地加以研究。

洛贝尔图斯先生首先把下述论点当作一个新的发现:一般剩余价值{“作为一般租供分配的劳动产品价值”},也就是由资本家榨取的全部剩余价值,“是由原产品价值加工业品价值构成的”。(第120页)

一开头,洛贝尔图斯先生又向我们重述了他关于在[476]农业中不存在“材料价值”的“发现”。这次用的是以下的说法:

\begin{quote}{“归工业品的、决定资本盈利率的租部分,作为盈利计算,不仅要依据实际用于制造这种产品的资本,也要依据作为材料价值列入工厂主企业基金的全部原产品价值;但是,这种材料价值,对于归原产品的租部分——扣除了按照工业中既定盈利率〈当然!按照既定盈利率〉计算的用于原产品生产的资本的盈利,其余额就形成地租——来说,是不存在的。”(第121页)}\end{quote}

我们再说一遍:不是不存在!

我们假定地租存在,因而原产品的剩余价值的一定部分归土地所有者——这是洛贝尔图斯先生没有证明过的,按照他的思路也是不能证明的。

其次假定:

\begin{quote}{“在产品价值既定的情况下,一般租的高度〈剩余价值率〉也是既定的。”(第121页)}\end{quote}

这就是说,例如,在价值100镑的商品中,有一半,即50镑,是无酬劳动;因而,这一半形成一个用来支付剩余价值的一切项目——地租、利润等——的基金。在这种情况下,十分明白,分享这50镑的人中,一个人得的越多,另一个人得的就越少,反过来也是一样,或者说,利润和地租成反比。现在要问:剩余价值分为这两部分,是由什么决定的?

无论如何,下述一点仍然是正确的:企业主(不管他是农业主还是工厂主)的收入等于他从出卖产品中取得的剩余价值(他从他的生产领域的工人身上榨取了这个剩余价值),而地租(在它不象卖给工业家的瀑布那样直接从工业品取得的地方;房租等的情况则同瀑布的情况一样,因为住房不是原产品)则仅仅从包含在原产品中、由租地农场主支付给土地所有者的超额利润(不加入一般利润率的那部分剩余价值)产生。

一点不错,如果原产品的价值提高[或降低],在使用原料的工业部门中,利润率将同原产品价值成反比地降低或提高。如果棉花的价值增加了一倍,那末,在工资和剩余价值率既定的条件下,利润率将下降,就象我在前面一个具体例子中指出的那样\endnote{马克思是指他在手稿第335—336页,即注17和注20中谈到的篇幅很长的关于约翰·斯图亚特·穆勒的补充部分中所引用的例子。——第76页。}。但是,这种情况在农业中也是存在的。如果收成不好,而生产要按原来的规模继续进行(这里我们假定商品按它们的价值出卖),那末,总产品,或者说,总产品价值中就有较大的一部分必须投回到土地中去,而且,在工资不变的条件下,在扣除工资以后,租地农场主的剩余价值将是产品的较小部分;因而剩下供租地农场主和土地所有者分配的将是较小量的价值。虽然单位产品的价值会比以前高,可是,不仅余下的产品量,而且余下的价值部分都会比以前小。如果产品由于需求[增加]而高于它的价值出卖,以致现在较小量产品的价格比以前较大量产品的价格高,情况就不同了。但是,这同我们假定产品按照它们的价值出卖这一点是矛盾的。

假定情况相反:棉花收成加倍,直接投回到土地中去的部分如肥料和种子的价值比以前小。在这种情况下,扣除工资后剩下给棉花种植业者的那部分价值比以前大。在棉花种植业中,如同在棉纺织业中一样,利润率将提高。自然,一点不错,现在一码棉布中,归原产品的价值部分将比以前小,归原料加工的价值部分将比以前大。假定一码棉布包含的棉花价值等于1先令,一码棉布值2先令。如果现在棉花价格从1先令降到6便士(在棉花价值等于棉花价格的前提下,棉花价格之所以可能下降,只是因为棉花种植业的生产率提高了),那末一码棉布的价值等于18便士。它降了1/4,即25%。但是,在棉花种植业者以前按1先令的价格卖出100磅的地方,现在按6便士要卖出200磅。以前全部棉花的价值是100先令,现在也是100先令。虽然棉花以前占产品价值的一个较大部分,棉花生产者以前用每磅按1先令计价的100先令棉花只换到50码棉布;现在他(即使棉花种植业中的剩余价值率同时降低了)用每磅棉花按6便士出卖的100先令,却换到66+(2/3)码棉布。

假定商品按它们的价值出卖,那末,说参与产品生产的生产者的收入,必然取决于他们的产品在产品总价值中形成多大的价值组成部分,[477]这种说法是不正确的。

假定在一切工业品中,包括机器在内,一个生产部门的总产品价值是300镑,另一个是900镑,第三个是1800镑。

如果说,全部产品价值分为原产品价值和工业品价值的那种比例决定剩余价值——按洛贝尔图斯的说法是租——分为利润和地租的比例这一点是正确的,那末,这一点对于有原料和工业品以各种比例参加的各种生产领域的各种产品,也一定是正确的。

如果在900镑价值中,工业品是300镑,原产品是600镑;如果1镑等于1工作日;其次,如果剩余价值率是既定的,例如,在正常工作日是12小时的时候,是2小时比10小时,那末,在300镑[工业品]中包含300工作日,在600镑[原产品]中则多一倍(2×300)。在前一种产品中剩余价值额等于600小时,在后一种产品中是1200小时。这无非是说,在剩余价值率既定的时候,剩余价值量取决于工人人数,即取决于同时使用的工人人数。其次,既然已经假定(但不是已经证明)在加入农产品价值的剩余价值中,一部分作为地租归土地所有者,那末,从这里必然进一步得出结论:地租量实际上是按农产品价值增加(同“工业品”价值相比)的比例增加的。

在上例中,农产品对工业品之比是2∶1,即600∶300。现在假定,这个比是300∶600。既然地租取决于农产品中所包含的剩余价值,显然,如果剩余价值在前一种情况下是1200小时,而在后一种情况下只有600小时,那末地租既然是这个剩余价值的一定部分,在前一种情况下就必然比在后一种情况下大。换句话说:农产品在全部产品价值中占的价值部分越大,全部产品的剩余价值中归农产品的部分就越大,因为产品的每一价值部分都包含一定的剩余价值;而全部产品的剩余价值中归农产品的部分越大,地租也就越大,因为农产品剩余价值的一定比例部分表现为地租。

假定地租等于农业中生产的剩余价值的1/10,如果农产品价值(在900镑总价值中)占600镑,地租就等于120小时,如果农产品价值占300镑,地租只等于60小时。这样一来,地租量的确同农产品价值量一起变化,因而也同农产品价值对工业品价值的相对量一起变化。但是,地租和利润的“高度”,即它们的比率,与此绝对没有关系。在前一种情况下,产品价值等于900镑,其中300镑是工业品,600镑是农产品。这个总数中,有600小时剩余价值归工业品,1200小时归农产品。合计1800小时。其中120小时归地租,1680小时归利润。在后一种情况下,产品价值等于900镑;其中600镑是工业品,300镑是农产品。因而归工业的剩余价值是1200小时,归农业的是600小时。合计1800小时。这个总数中,归地租的是60小时;归利润的部分中,1200小时归工业,540小时归农业,合计1740小时。在后一种情况下,工业品(按价值)两倍于农产品,在前一种情况下相反。在后一种情况下地租等于60小时,在前一种情况下等于120小时。地租纯粹同农产品价值成比例地增加。农产品价值量增加多少倍,地租量也增长多少倍。就全部剩余价值(等于1800小时)来看,地租在前一种情况下占1/15,在后一种情况下占1/30。

如果这里地租量随同归农产品的价值部分的量一起增加,而地租在全部剩余价值中所占的比例部分也随同地租量一起增加;因而,如果剩余价值归地租的部分同它归利润的部分比较起来也有了增加,——那末,这只是因为洛贝尔图斯假定地租按一定比例参与农产品剩余价值的分配。既然这个事实是既定的,或者说,是已经假定的,情况也必然是这样。但是这个事实本身,决不能从洛贝尔图斯再次给我们讲的、我在前面第476页一开头\authornote{见本册第75页。——编者注}就引过的关于“材料价值”的废话中得出来。

至于地租的高度,那它也不会同地租所参与分配的产品[中包含的剩余价值]成比例增加,因为这个比例仍然是1/10。地租量增加,是因为这个产品量增加了;既然在地租的“高度”没有增加的情况下地租量还是增加了,那末,同[总]利润量相比,或者说,同这个利润在总产品价值中所占的份额[478]相比,地租的“高度”也增加了。因为假定,现在是总产品价值的一个较大的部分提供地租,剩余价值的一个较大的部分能够转化为地租,所以,剩余价值中转化为地租的部分自然就增加了。这一切同“材料价值”绝对没有关系。而洛贝尔图斯却说:

\begin{quote}{“较多的地租”同时也表现为“较高的地租”,“因为这个地租据以计算的土地面积或摩尔根数仍然不变,从而,每一摩尔根分摊到一个较大的价值额”。(第122页)}\end{quote}

这种说法是荒谬的。这是用一种回避问题本身困难的“尺度”来衡量地租的“高度”。

我们应该把上面举的例子略为改变一下——因为,我们还不知道什么是地租。如果我们使农产品的利润率同工业品的一样,只是另加1/10作地租,那末,情况就不同了,并且可以清楚地看出:[为了解决问题,]本来就应该这样做,因为同一利润率的存在是作为前提的。

在II的情况下,地租比在I的情况下大一倍,因为[总]产品价值中被地租象虱子一样叮着的那一部分价值,即农产品价值,在这里比工业品大。利润量在两种情况下都是一样——1800小时。在I的情况下,[地租]占全部剩余价值的1/31,在II的情况下占1/16。

如果洛贝尔图斯无论如何要把“材料价值”单单算在工业上,那末他首先就应该把由机器等组成的那部分不变资本单单归到农业上。这部分资本是作为工业提供给农业的产品,作为充当生产“原产品”的生产资料的“工业品”加入农业的。

至于工业,机器中由“原料”构成的那部分价值,已经在“原料”或者说“材料价值”的项下记入工业的借方,因为这里所谈的是两家公司之间的结算。因此,这一部分是不能重复入账的。工业中使用的机器的另一部分价值,是由加进去的“工业劳动”(过去的和现在的)构成的,而这种劳动分解为工资和利润(有酬劳动和无酬劳动)。因此,在这里预付的那部分资本(除了机器的原料所包含的以外),[可以看出]仅仅由工资构成,因而,它不仅使预付资本量增加,并且使应该依据这个预付资本来计算的剩余价值量增加,从而使利润增加。

(在这样的计算中,错误始终在于[不理解]:例如,机器本身即其价值中所包含的机器或工具的损耗,虽然归根到底也可以归结为劳动,——不论是原料中包含的劳动,还是把原料变成机器等的劳动,——可是,这个过去劳动既不再加入利润,也不再加入工资,只要再生产所必需的劳动时间不变,它只能作为已经生产出来的生产条件发挥作用;不论这种生产条件在劳动过程中的使用价值如何,它在价值形成过程中仅仅作为不变资本的价值出现。这一点非常重要,我在研究不变资本和收入的交换问题时已经阐明\authornote{见本卷第1册第248—255页。——编者注}。但是,除此以外,这一点还要在论资本积累那一节中较详细地加以阐述。)

至于农业,即单纯的原产品生产或者所谓初级生产,相反,在“初级生产”公司和“加工工业”公司之间相互结算时,加入农业的代表机器、工具等的资本价值部分,即一部分不变资本,只能看作是加入农业资本但不增加其剩余价值的项目,决不能作任何别的理解。如果农业劳动由于使用机器等等而提高了生产率,那末,机器等的价格越高,农业劳动生产率的增长就越慢。使农业劳动生产率以及任何其他劳动生产率增长的,是机器的使用价值,而不是机器的价值。此外,同样可以说,工业劳动生产率首先取决于原料的存在和它的特性。但是,这里成为工业的生产条件的,又是原料的使用价值,而不是它的价值。价值倒不如说是一个障碍。因此,[479]对洛贝尔图斯先生在工业资本方面关于“材料价值”所说的话,作一些相应的改动,就完全适用于[农业中使用的]机器等等:

\begin{quote}{“比如说,花费在作为机器的犁或轧棉机上的劳动〈以及耗费在排水渠或马厩上的劳动〉不能加入小麦或棉花这种特殊产品所耗费的劳动。”“相反,机器的价值,或者说机器价值,总是算在资本总额内,所有者就是根据这个资本总额来计算作为盈利归原产品的租部分的。”(参看洛贝尔图斯著作第97页)\endnote{马克思对洛贝尔图斯的这段话作了“相应的改动”,这些改动是根据洛贝尔图斯所忽视的下述情况而作的:机器和其他生产资料的价值必须加入农产品,正如农业原料的价值必须加入农业原料加工工业的产品一样。马克思按照洛贝尔图斯的表述形式,在前面引用过这段话(见本册第56页)。马克思仿照洛贝尔图斯的术语“材料价值”(《Materialwert》),不无讽刺地造出“机器价值”(《Maschinenwert》)这一术语。凡是马克思用的词,在引文中都以黑体加着重号刊印。——第82页。}}\end{quote}

换句话说:小麦和棉花的价值中代表磨损了的犁或轧棉机的价值的那一部分,不是犁地劳动或轧棉劳动的结果,而是造犁劳动或造轧棉机劳动的结果。这个价值组成部分加入农产品,尽管它不是在农业中生产的。它仅仅经过农业之手,因为农业只是用它从机器制造业者那里购买新犁或新轧棉机来补偿磨损了的犁和轧棉机。

农业用的这些机器、工具、建筑物和其他工业品由两个部分构成:(1)这些工业品的原料[和(2)加在原料上的劳动]。

虽然这种原料是农产品,但是它是农产品中从来既不加入工资也不加入利润的部分。即使农业中根本不存在资本家,土地耕种者仍然不能把他的产品的这一部分作为工资记在他的账上。事实上他必须把这一部分无代价地交给机器厂主,让机器厂主用来为他制造机器,此外,他还必须对加在这种原料上的劳动支付报酬(=工资+利润)。实际上情况也是这样。机器制造业者购买原料,但是农业主在购买机器时必然把这种原料买回。因此,这就好比农业主什么也没有卖给机器制造业者,而只是把原料借给他,让他赋予原料以机器形式。因此,农业用的机器的价值中归结为原料的那部分,虽然是农业劳动的产品,是农业劳动所创造的价值的一部分,却仍然属于生产,而不属于生产者,因而同种子一样算在生产者的费用中。另一部分价值,即代表投入机器的工业劳动的价值,是“工业品”,它作为生产资料加入农业,同原料作为生产资料加入加工工业完全一样。

因此,如果说“原产品生产”公司把“材料价值”提供给“加工工业”公司,这个材料价值作为一个项目列入工厂主的资本总额,这种说法是正确的,那末,下面的说法同样正确:“加工工业”公司把机器价值提供给“原产品生产”公司,这个机器价值全部(包括由原料组成的部分在内)加入租地农场主的资本总额,尽管这个“价值组成部分”并不给租地农场主提供剩余价值。正是由于这个原因,在英国人所谓的高级农业中,同原始农业比较,虽然剩余价值率较高,利润率却显得较低。

同时,这里向洛贝尔图斯先生提供了一个鲜明的证据,说明对于资本预付的本质来说,产品价值中投在不变资本的部分究竟是用实物补偿,因而仅仅被作为商品——作为货币价值——计算,还是确实被让渡,并经过买卖的过程,那是无关紧要的。例如,如果原产品生产者把他购买的机器中所包含的铁、铜、木材等无代价地交给机器制造业者,因而机器制造业者在向他出卖机器时只算加进去的劳动和自己的机器的磨损这笔账,那末,所购买的机器要农业主花的费用,就会恰好同它现在要他花的费用一样多,而且在他的生产中作为不变资本、作为预付出现的,就会是同一个价值组成部分;就象农民是把他的收获物全部卖掉,用收获物价值中代表种子(原料)的那部分价值买进别人的种子,——比如说,为了进行有利的品种更换,从而避免经常的同种繁殖引起的植物退化,——还是直接从他的产品中扣出这个价值组成部分投回到土地中去,这完全是无关紧要的。

但是,洛贝尔图斯先生为了作出他的计算,把不变资本中由机器构成的部分理解错了。

在分析洛贝尔图斯先生的第二个论题时应该考察的第二点是:

洛贝尔图斯先生所说的构成收入的工业品和农产品,同构成全部年产品的工业品和农产品,完全不是一回事。即使就全部年产品来看,下述说法是正确的:在把农业资本中由机器等构成的整个部分,[480]以及农产品中直接投回农业生产的部分扣除之后,剩余价值在租地农场主和工业家之间的分配(因而,归租地农场主的剩余价值也在租地农场主自己和土地所有者之间的分配),必然决定于工业和农业在产品总价值中所占份额的量,——那末,当我们说的是构成共同收入基金的产品时,这种说法是否正确就大成问题。收入(这里把再转化为新资本的部分除外)是由加入个人消费的产品构成的,这里的问题是工业家、租地农场主和土地所有者从这个大锅里究竟各取多少。取出的各部分,是不是由工业和原产品生产在构成收入的产品价值中所占的份额决定的?换句话说,是不是由构成收入的全部产品的价值所分成的农业劳动和工业劳动的份额决定的?

前面我已经指出\authornote{见本卷第1册第248—255页。——编者注},构成收入的产品量,不包括作为劳动工具(机器)、辅助材料、半成品和半成品的原料加入生产并构成劳动年产品的一部分的一切产品。它不仅不包括原产品生产中使用的不变资本,而且不包括机器制造业者的不变资本,以及租地农场主和[工业]资本家的虽然加入劳动过程但不加入价值形成过程的全部不变资本。其次,它不仅不包括不变资本,而且不包括那些不加入个人消费的产品——这些产品代表其生产者的收入,并加入作为收入来消费的那些产品的生产者的资本,以补偿用掉的不变资本。

作为收入来消费的,并且事实上无论按使用价值还是按交换价值都代表财富中构成收入的部分的产品量,如我前面所指出的\authornote{见本卷第1册第238—248页。——编者注},可以看作仅仅由(一年内)新加劳动构成,因而也仅仅归结为收入,即工资和利润(利润又分解为[留在资本家手里的]利润、地租、税收等),其中既丝毫不包含加入生产的原料的价值,也丝毫不包含加入生产的机器或者一般说劳动资料损耗的价值。因此,如果我们考察这种收入(把收入的各种派生形式完全撇开不谈,因为这些形式只是表示收入所有者把他在上述产品量中得到的份额转让给别人,或者是为了支付服务报酬等等,或者是为了还债等等),并假定其中工资占1/3,利润占1/3,地租占1/3,而产品按价值等于90镑,那末,每项收入的所得者都可以从总量中取得等于30镑的产品。

既然形成收入的产品量仅仅由新加(一年内加入的)劳动构成,那末,看来似乎很简单:如果农业劳动在这个产品量中占2/3,工业劳动占1/3,那末,工业家和农业主彼此就按照这个比例来分配价值。上述产品量价值的三分之一归工业家,三分之二归农业主,而且工业中和农业中实现的剩余价值的比例量(假定两个部门的剩余价值率一样),将同工业和农业在总产品价值中所占的份额相适应;地租将同租地农场主的利润量成比例增长,因为地租是象虱子一样叮着利润的。但是,这样来论述问题终究是错误的。问题在于,由农业劳动[产品]构成的一部分价值,将形成那些生产固定资本以补偿农业中损耗的固定资本的工厂主的收入。因此,形成收入的产品的价值组成部分之间的比例即农业劳动和工业劳动之间的比例,决不表示形成收入的产品量的价值或这个产品量本身在工业家和租地农场主之间分配的比例,也不表示工业和农业参与总生产的比例。

洛贝尔图斯接着说:

\begin{quote}{“但是,原产品价值和工业品价值的相应高度,或者说,两者在全部产品价值中所占的份额,又只是分别由原产品生产的劳动生产率和工业的劳动生产率决定的。原产品生产的劳动生产率越低,原产品的价值越高,反过来也是一样。同样,工业劳动的生产率越低,工业品的价值越高,反过来也是一样。因此,如果一般租的高度是既定的,那末,由于高的原产品价值造成高的地租和低的资本盈利,而高的工业品价值造成高的资本盈利和低的地租,——地租的高度和资本盈利的高度不仅应当彼此成反比,而且应当同相应的劳动的生产率,即原产品生产劳动和工业劳动的生产率成反比。”(第123页)}\end{quote}

把两个不同生产领域的生产率拿来比较,这只能是相对的。就是说,取任何一点,比如说,大麻的价值和麻布的价值即它们所包含的劳动时间的相对量之比1∶3作为出发点。如果这个比例改变了,那末,说这两种不同劳动的生产率有了改变,是正确的。但是,如果因为生产一盎斯金[481]所需要的劳动时间等于3,生产一吨铁所需要的劳动时间也等于3,就说金的“生产率低于”铁,那就错了。

两种商品的价值比例,表示生产一种商品比生产另一种商品花费较多的劳动时间;但是,不能因此就说,一个生产部门比另一个“生产率高”。这样说,只有在两者都把劳动时间用于生产同一使用价值的时候,才是正确的。

因此,如果原产品价值和工业品价值之比是3∶1,由此决不能得出结论说,工业生产率三倍于农业。只有这个比例改变了,例如变成4∶1或3∶2或2∶1等等,才可以说,两个部门的相对生产率有了改变。因而,只有在生产率提高或降低的情况下才可以这样说。

\tsubsectionnonum{[(c)洛贝尔图斯的第三个论题]}

\begin{quote}{“(III)资本盈利的高度,一般地说,仅仅决定于产品价值的高度,具体地说,仅仅决定于原产品价值和工业品价值的高度,或者,一般地说,仅仅决定于劳动生产率的程度,具体地说,仅仅决定于原产品生产劳动生产率和工业劳动生产率的程度;地租的高度,除此以外,还取决于产品价值的量,就是说,取决于在既定的生产率程度下用于生产的劳动或生产力的量。”(第116—117页)}\end{quote}

换句话说:利润率仅仅取决于剩余价值率,而剩余价值率仅仅取决于劳动生产率;可是,地租率还取决于在既定的劳动生产率下使用的劳动量(工人的数量)。

在这个论断中,几乎每一句话都是错的。

第一,利润率决不是仅仅取决于剩余价值率,——关于这一点我们马上就要谈到。但是,首先,说剩余价值率仅仅取决于劳动生产率,是错误的。在既定的劳动生产率下,剩余价值率随剩余劳动时间的长短而变动。因此,剩余价值率不仅取决于劳动生产率,也取决于使用的劳动量,因为(在生产率不变的情况下),即使有酬劳动量不增加,即花费在工资上的资本部分不增加,无酬劳动量也可能增加。如果劳动没有起码达到这样程度的生产率,也就是在一个工作日中,除了工人自己再生产所必需的时间以外,还剩下剩余劳动时间,那末,无论绝对剩余价值还是相对剩余价值(洛贝尔图斯追随李嘉图,只知道有相对剩余价值),都是不可能的。但是,既然假定存在剩余劳动时间,那末,在既定的最低限度的生产率下,剩余价值率就随剩余劳动时间的长短而变动。

因此,第一,说利润率,或者说,“资本盈利的高度”仅仅决定于被资本剥削的劳动的生产率(因为据说剩余价值率也决定于它),是错误的。第二,假定在既定的劳动生产率下随工作日的长短而变动、在既定的正常工作日下随劳动生产率而变动的剩余价值率是既定的。这样,剩余价值本身就随工人(他们的每一个工作日被榨取一定量的剩余价值)的人数而不同,换句话说,随花费在工资上的可变资本量而不同。但是,利润率取决于这个剩余价值对可变资本加不变资本之比。在剩余价值率既定时,剩余价值量当然取决于可变资本量,但是,利润的高度,即利润率,取决于这个剩余价值对预付总资本之比。因此,这里利润率当然决定于原料的价格(如果该工业部门使用原料的话)和具有一定效率的机器的价值。

因此,洛贝尔图斯的下面一段话是根本错误的:

\begin{quote}{“因此,据以计算盈利的资本价值总额,同资本盈利额——随产品价值增加而增加——是按同一比例增加的,盈利和资本之间的原来比例不因资本盈利的增加而有丝毫改变。”(第125页)}\end{quote}

这句话如果说正确,除非它是这样一个同义反复:在利润率既定时{而利润率跟剩余价值率和剩余价值本身是极不相同的},使用的资本的大小之所以无关紧要,正是因为利润率假定不变。一般说来,虽然劳动生产率不变,利润率还是可能提高,或者,虽然劳动生产率提高,并且每个领域都提高,利润率还是可能下降。

接着,又出现了关于地租的笨拙的笑话(第125—126页),说什么地租的单纯增加就可以使地租率提高,因为地租在每个国家都是按“不变的摩尔根数”(第126页)计算的。既然(在利润率既定时)利润量增加,那末提供利润的资本量也增加;可是,如果地租增加,据说发生变化的只有一个因素,即地租本身,至于它的尺度“摩尔根数”,仍然固定不变。

\begin{quote}{[482]“因此,地租的提高可以由于社会经济发展中到处发生的一个原因,即由于生产中使用的劳动增加,换句话说,由于人口增加,而不必同时由于原产品价值提高,因为从更多的原产品取得地租,就必然产生这样的结果。”(第127页)}\end{quote}

洛贝尔图斯在第128页上有一个奇怪的发现:即使地租由于原产品的价格降到它的正常价值之下而完全消失,也不可能

\begin{quote}{“使资本盈利在什么时候会达到100%〈即在商品按其价值出卖的情况下〉;盈利不论多高,它始终要低得多”。(第128页)}\end{quote}

为什么呢?

\begin{quote}{“因为它”〈资本盈利〉“仅仅是产品价值分配的结果。因此,它在任何时候都只能是这个单位的一个分数。”(第127—128页)}\end{quote}

洛贝尔图斯先生,这全看您是怎么计算的。

假定,预付不变资本等于100,预付工资等于50,而劳动的产品超过这个50的余额等于150,那末,我们就可以这样计算:

\todo{}

要使这种情况出现,只须假定工人以3/4的工作日为他的老板劳动,就是说,四分之一的劳动时间就够工人自身的再生产。当然,如果洛贝尔图斯先生把等于300的全部产品价值作为一个整体,如果他不是从其中超过生产费用的余额来考察,而是说这种产品要在资本家和工人之间分配,那末,资本家的份额的确只占这种产品的一部分,哪怕这一部分等于全部产品的999/1000。但是,这种计算是错误的,至少几乎从各方面说都是无用的。如果有人花费150,得到300,那他通常不会说,用300(而不是用150)来除150,他获得50%的利润。

在上面的例子中,假定工人劳动12小时:3小时为自己,9小时为资本家。现在假定他劳动15小时,即3小时为自己,12小时为资本家。按照生产中[生产资料和活劳动量的]原来的比例,不变资本应该追加25(实际上没有这么多,因为机器的花费,不会和劳动量按同一比例增加)。于是:

\todo{}

接着,洛贝尔图斯又向我们谈论“地租的无限”增长,因为第一,他把地租量的单纯的增加理解为地租的提高,就是说,当相同的地租率用较大的产品量来支付时,他也说是地租的提高。其次,因为他计算地租,是把“摩尔根”当作尺度,——这是毫不相干的两回事。

\centerbox{※     ※     ※}

下面各点可以简单提一下,因为它们同我的目的毫无关系。

“土地价值”是“资本化的地租”。因此,土地价值的这种货币表现要看通行的利率高低而定。按4%资本化,地租应当乘以25(因为4%=100的1/25);按5%资本化,应当用20去乘(因为5%=100的1/20)。在土地价值上,这里的差额是20%。(第131页)甚至由于货币价值下降,地租,从而土地价值,也会在名义上提高,因为,这里据说同资本的情况不一样。如果借贷利息或利润表现为较多的货币,那末资本在其货币表现上也同时以同一程度增多。相反,表现为较多货币的地租,据说应分配在“地段的不变的摩尔根数上。”(第132页)

洛贝尔图斯先生把他的智慧应用到欧洲的经济发展时,作了如下的概述:

\begin{quote}{(1)“……在欧洲各国,一般劳动——原产品生产劳动和工业劳动——的生产率提高了……其结果,国民产品中归工资的份额减少了,剩下作为租的份额增加了……因而一般租提高了。”(第138—139页)(2)“……工业生产率增长的比例大于原产品生产的生产率……因此,今天在同量国民产品价值中,归原产品的租的份额,大于归工业品的租的份额;因此,尽管一般租提高了,可是提高的只有地租,资本盈利反而下降了。”(第139页)}\end{quote}

由此可见,洛贝尔图斯先生在这里完全同李嘉图一样,让地租提高和利润率降低互相说明;一个降低等于另一个提高,而地租的提高用农业生产率相对低[483]来说明。李嘉图在有的地方甚至明确地说,问题不在于生产率绝对低,而在于生产率“相对”低。\authornote{见本册第381—382页。——编者注}但是即使他说到相反的情况,这也不是从他提出的原则得出的,因为李嘉图观点的原作者安德森明确地讲过,任何土地都有改良的绝对可能性。

如果一般“剩余价值”(利润和地租)提高了,全部租对不变资本的比率不仅可能下降,而且一定下降,因为劳动生产率提高了。虽然使用的工人人数增加了,他们受剥削的程度增加了,可是花费在工资上的资本,尽管绝对地说增加了,相对地说却下降了;因为由这些工人推动的、作为过去劳动的预付产品、作为生产的前提加入生产的[不变]资本,构成总资本中不断增长的部分。因此,利润加地租的比率下降了,虽然不仅它们的数额(它们的绝对量)提高了,而且对劳动的剥削率也提高了。这一点洛贝尔图斯先生是无法看到的,因为在他看来,不变资本是工业的发明,而农业对此则毫无所知。

至于利润和地租的相对量,那末,决不能因为农业生产率相对地低于加工工业生产率,就得出结论说,利润率因此要绝对地下降。如果说,利润对地租之比以前是2∶3,现在只是1∶3,那末,利润以前是地租的2/3,现在只等于地租的1/3;换句话说,如果以前利润占总剩余价值的2/5(8/20),现在只占1/4(5/20),那末,利润下降了3/20,就是说,从40%降到25%。

假定,一磅棉花的价值以前是2先令。现在降到1先令。100工人以前一天纺100磅棉花,现在纺300磅。花费在300磅棉花上的资本,以前是600先令,现在只有300先令。再假定,在两种情况下机器的价值都等于[第一种情况下花在棉花上的数额的]1/10,即60先令。最后,为了把300磅棉花纺成棉纱,以前对300工人支付300先令工资,现在只对100工人支付100先令。因为工人的劳动生产率“增长了”,并且我们必须假定,这里是用工人自己劳动的产品支付工人报酬的,所以假定剩余价值以前等于工资的20%,现在等于工资的40%。

这样,300磅棉纱值:

\begin{quote}{(I)在第一种情况下——原料600,机器60,工资300,剩余价值60,合计1020先令;(II)在第二种情况下——原料300,机器60,工资100,剩余价值40,合计500先令。}\end{quote}

在第一种情况下,生产费用960先令,利润60先令,利润率[6+(1/4)]%。

在第二种情况下,生产费用460先令,利润40先令,利润率[8+(16/23)]%。

假定地租是每磅棉花的1/3,那末,在第一种情况下等于200先令,或10镑,在第二种情况下等于100先令,或5镑。这里,地租下降了,因为原产品便宜了50%。但是整个产品便宜了50%以上。新加的工业劳动对原料价值之比,在第一种情况下是360∶600,或6∶10,或1∶[1+(2/3)];在第二种情况下是140∶300,或1∶[2+(1/7)]。工业劳动生产率增长的比例大于农业劳动生产率;可是在第一种情况下,利润率比第二种情况下低,地租比第二种情况下高。在两种情况下,地租都是原料的1/3。

假定原料量在第二种情况下加了一倍,因而纺了600磅棉花,那末情况就变成:

\begin{quote}{(II)600磅棉花(原料)600先令,机器120先令,工资200先令,剩余价值80先令,合计生产费用920先令,利润80先令,利润率[8+(16/23)]%。}\end{quote}

同第一种情况对比,利润率提高了。地租却同第一种情况下一样多。600磅棉纱只值1000先令,而以前却值2040先令。

[484]决不能从农产品的相对昂贵得出结论说,农产品提供地租。但是,只要假定地租作为一定百分比加到农产品的每一个价值部分上,——洛贝尔图斯是这样假定的,因为他的所谓证明是荒谬的,——那当然就会得出结论说,地租随农产品的不断昂贵而提高。

\begin{quote}{“……由于人口增加,国民产品的价值总额也极度增加了……因此,现在国内得到更多的工资,更多的盈利,更多的地租……地租总额的这种增加还引起地租水平的提高,可是工资和盈利的总额的增加却不会发生这种作用。”(第139页)}\end{quote}

\tsectionnonum{[(8)洛贝尔图斯所歪曲的规律的真实含义]}

现在让我们抛开洛贝尔图斯先生的一切谬论(更不用说我在上面已经详细指出的那些有缺陷的见解了,例如,说剩余价值率(“租的高度”)只有在劳动生产率增长时才能提高,——这就是不懂得绝对剩余价值,等等)。

就是说,让我们抛开:

[第一个]谬论:在真正的农业中(在资本主义的农业中),“材料价值”完全不加入预付;

第二个谬论:洛贝尔图斯不是把加入农业和工业的第二部分不变资本(机器等)看作这样一个“价值组成部分”:它同“材料价值”完全一样,不是它以机器形式加入的那个生产领域的劳动的结果,因而在计算每个生产领域所取得的盈利时也要以它为依据,虽然机器的价值同材料的“价值”一样,一点也没有加到这个盈利中,——尽管两者都是生产资料,并作为生产资料进入劳动过程;

第三个谬论:洛贝尔图斯不把加入农业的“机器”等所构成的整个“价值组成部分”作为预付,记在农业的账上,而且他不把这个“价值组成部分”中不归结为原料的部分看作农业对工业的负债,而为了偿付这笔债务,农业必须把一部分原料无代价地提供给工业,因此,这一部分原料不属于作为整体的工业预付;

第四个谬论:他认为一切工业部门,除了有机器和机器所必需的辅助材料加入外,还有“材料价值”加入,可是无论运输业或采掘工业,都完全没有这种情况;

第五个谬论:他不懂得,在许多加工工业部门(它们越是提供用于消费的成品,就越是如此),虽然除了可变资本之外还有“原料”加入,但是不变资本的其他组成部分几乎完全消失,或者数量极小,小得同大工业和农业无法比较;

第六个谬论:他把商品的平均价格同商品的价值混为一谈。

这一切谬论使洛贝尔图斯从租地农场主和他自己的错误计算中得出他对地租的解释,以致地租应该随着租地农场主开始确实计算他的支出而消失。如果把这一切谬论抛开,那末,作为隐藏在这些谬论后面的核心剩下来的,只是下面的论断:

当原产品按照它们的价值出卖的时候,它们的价值高于其他商品的平均价格,或者说,高于它们自己的平均价格,就是说,超过生产费用加平均利润,因而提供一个超额利润,这个超额利润形成地租。这就是说,可变资本(假定剩余价值率相等)对不变资本之比,在原产品生产中比工业各生产领域的平均数大(这并不妨碍可变资本在某些工业部门比农业中高)。或者,更一般地说:农业同这样一些工业生产领域属于一类,这些工业生产领域的可变资本对不变资本之比,高于各工业生产领域的平均数。因此,农业中的剩余价值,按它的生产费用计算,必然高于各工业生产领域的平均数。这也就是说,农业中的特殊利润率高于平均利润率即一般利润率。这也就是说,如果剩余价值率相等,而剩余价值本身又是既定的,那末各个生产领域的特殊利润率,取决于各个领域中可变资本对不变资本之比。

由此可见,这只是把我从其一般形式上论述的规律\authornote{见本册第64—70页。——编者注}应用到一个特殊生产部门而已。

[485]其次:

(1)要证明农业属于这样一些特殊的生产领域,它们的商品价值高于它们的平均价格,因此,它们的利润,——如果它们把利润据为己有,而不把它交出去参与一般利润率的平均化,——就高于平均利润,这样,在扣除平均利润之后,这些生产领域还剩下超额利润。看来,这一点对于农业平均说来是正确的,因为在农业中,手工劳动相对地说还占优势,而使工业的发展快于农业则是资产阶级生产方式所固有的。不过,这是历史的差别,是会消失的。这也造成了下面这种现象:工业供给农业的生产资料,总的说来,价值下降,而农业供给工业的原料,总的说来,价值提高;因此,在很大一部分加工工业中,不变资本的价值相对地高于农业。对于采掘工业,这一点在大多数情况下是不适用的。

(2)不能象洛贝尔图斯那样,说什么如果农产品按照一般规律,平均按它的价值出卖,就一定提供超额利润,或者说,地租,——好象商品按价值但高于平均价格出卖,成了资本主义生产的一般规律。相反,应该证明,为什么在原产品生产中,例外地,与价值也高于平均价格的那一类工业品不同,价值不降到平均价格,因而提供超额利润,或者说,地租。这一点只能用土地所有权的存在来解释。平均化只有通过资本同资本互相作用才会发生,因为只有互相起作用的各个资本才有力量实现资本的内在规律。就这一点来说,那些用垄断来解释地租的人是正确的;正如唯有资本的垄断使资本家能从工人身上榨取剩余劳动一样,土地所有权的垄断也使土地所有者能从资本家那里榨取那部分能够形成经常的超额利润的剩余劳动。那些用垄断来解释地租的人的错误,在于他们认为垄断使土地所有者能够把商品价格抬得高于商品价值。相反,垄断在这里的作用,是把商品价值保持在高于商品的平均价格的水平,是使商品能够按照它的价值而不是高于它的价值出卖。

如果洛贝尔图斯的观点经过这样的修正,那对问题的理解就正确了。这样一来,地租的存在也就得到了解释,而李嘉图只解释了差别地租的存在,实际上否认土地所有权有任何经济影响。其次,这样来理解问题,就排除了在李嘉图自己的著作中只是任意加上的、对他的思路来说没有必要的加筑部分,即所谓农业生产率递减的论断,而相反地,承认农业生产率的增长。问题仅仅在于,在资产阶级的基础上,农业生产率相对地低于工业,换句话说,农业劳动生产力的发展比工业慢。李嘉图不是从较高的生产率,而是从较低的生产率得出农业中的“超额剩余价值”,这是正确的。

\tsectionnonum{[(9)级差地租和绝对地租的关系。地租的历史性。斯密和李嘉图的研究方法问题]}

讲到地租的差别,在土地面积相同、投资额相等的情况下,它是由自然肥力的差别造成的,特别是,首先是对于那些提供面包这种主要食品的产品来说,是如此;在土地面积相同、肥力相等的情况下,地租的差别是由投资额不等造成的。第一种差别,即自然的差别,不仅引起地租量的差别,而且引起地租的高度,或者说,地租率(同支出的资本相比)的差别;第二种差别,即工业的差别,则仅仅引起地租与支出的资本量成比例的增加。在同一地段上连续投资,也可能产生不同的结果。在肥力不同的地段上存在不同的超额利润,或者说,存在差别地租,还不能使农业同工业区别开来。使农业同工业区别开来的,是这些超额利润的固定化,因为在农业中,这些超额利润是建立在自然所提供的基础上的(诚然,这个基础可能或多或少地趋于平衡),而在工业中,在平均利润相同的情况下,超额利润总是具有转瞬即逝的性质,超额利润的产生,总是仅仅因为采用了生产率更高的机器和更有效的劳动组合。在工业中,由于平均价格降低而得到的超额利润,总是由最后出现的、生产率最高的资本提供的。在农业中,可能成为而且往往必然成为超额利润的原因的,并不是较好地段的肥力的绝对增长,而是它的肥力的相对增长,这种增长是由比较不肥沃的土地投入耕种引起的。在工业中,较高的相对生产率,超额利润(它很快消失),总是新投资本的生产率比原有资本的生产率绝对增长的结果。在工业中,任何资本(这里不谈需求的暂时增长)都不可能因生产率较低的资本新加入该工业部门而提供超额利润。

[486]然而就是在农业中(正如李嘉图所设想的),比较肥沃——或者天然比较肥沃,或者由于最新技术成就变得比原有技术条件下的老地肥沃——的土地,也可能在较晚的阶段投入耕种,甚至可能使一部分老地停止耕种(在采矿工业和殖民地产品生产中就有这种情况),或者使老地进行另一种农业生产,提供另一种产品。

地租(超额利润)的差别比较固定,这是农业和工业不同的地方。然而,使平均生产条件决定市场价格,从而把低于这种平均水平的产品价格提到高于该产品的价格,甚至高于它的价值的原因,决不是土地,而是竞争,是资本主义生产;因此,这不是自然规律,而是社会规律。

按照这个理论,最坏的土地无论支付地租,或者不支付任何地租,都不是必然的。同样可能的是,在不提供地租、只提供普通利润的地方,或者甚至连普通利润也没有的地方,仍然支付租金,也就是说,土地所有者得到地租,虽然从经济学观点来看,这里没有任何地租存在。

例如第一种情况:只有较好(比较肥沃)的土地支付地租(超额利润)。“作为地租”的地租在这里并不存在。在这种情况下,超额利润很少作为地租固定下来,正如工业中的超额利润并不固定下来一样(例如北美合众国西部就是这样)\endnote{手稿中接下去有一段关于资本是“别人劳动的合法反映”的简短插话,马克思把这段话放在花括号内,并且指出,它对叙述的直接联系有妨碍,应该移到别处去。本版用脚注形式把这段话放在前面,即放在本册第26页。——第99页。}。[486]

[486]这种情况常常发生在这样的地方:一方面,比较大量可供自由使用的土地还没有变为私有,另一方面,自然肥力相当大,虽然资本主义生产不太发达,因而,虽然可变资本对不变资本的比例很大,农产品的价值还是等于(有时甚至低于)它们的平均价格。如果价值高于平均价格,竞争会使它降低到这个水平。但是,象洛贝尔图斯那样的说法则是荒谬的:国家让每一英亩缴纳一个微小的、几乎是名义上的价格,比如说,一美元[就是地租]。\authornote{见本册第171页。——编者注}人们同样可以引用国家让每一个工业部门缴纳“营业税”为理由[并得出结论:工业提供“营业租”]。在我们考察的这种情况下,李嘉图[关于只有较好的土地支付地租]的规律是有效的。这里,只有相对地说更肥沃的土地才有地租,而且大部分还是不固定的,是流动的,象工业中的超额利润一样。不支付地租的土地所以不支付地租,不是因为它贫瘠,而是相反,因为它肥沃。较好的土地支付地租,是因为它相对地更肥沃,它的肥力高于平均水平。

但是在存在土地所有权的国家,同样的情况,即最后进入耕种的土地不支付地租,可能是由于相反的原因[即由于这种土地相对的贫瘠]。如果,比方说,谷物的价值很低(而这种低水平同支付地租根本没有关系),以致最后进入耕种的土地所提供的这个价值仅仅等于平均价格,那末,这种情况就会发生。这是由于这种土地相对的贫瘠,因而在这种土地上虽然和在提供地租的土地上花费同样多的劳动,但是,收获的夸特数量(按支出的资本计算)是这样少,以致在实现谷物产品的平均价值时,只得到比方说小麦的平均价格。

[487]举例来说,假定提供地租的最后的土地(提供最少地租的土地提供纯地租,其他土地提供级差地租),支出100镑资本,生产120镑或360夸特小麦的产品,每夸特值1/3镑。在这种情况下,3夸特小麦值1镑。假定1镑等于1周劳动,100镑=100劳动周,120镑=120劳动周。那末,1夸特小麦等于1/3劳动周即2工作日。假定在2日或24小时中(如果正常工作日是12小时),1/5日即4+(4/5)小时是无酬劳动,也就是一夸特包含的剩余价值。1夸特=1/3镑=6+(2/3)先令=6+(6/9)先令。

如果平均利润等于10%,360夸特的平均价格就等于110镑,1夸特的平均价格=6+(1/9)先令。这样,如果一夸特小麦按它的价值出卖,全部产品的价值就比平均价格高10镑。因为平均利润等于10%,地租就等于剩余价值的一半,即10镑或每夸特5/9先令。等级较高的土地,同样花费120劳动周(不过其中只有100是有酬劳动,不管是物化劳动还是活劳动),却能提供较多夸特,按每夸特6+(6/9)先令的价格,将提供较高的地租。而最坏的耕地花费100镑资本将提供10镑地租,或者说,一夸特小麦提供5/9先令地租。

假定有一块新地投入耕种,花费120劳动周,只提供330夸特。如果[以前]3夸特的价值等于1镑,330夸特就等于110镑。现在一夸特等于2日2+(2/11)小时,以前只等于2日。以前一夸特的价值是6+(6/9)先令即6先令8便士,现在(因为一镑等于6日)是7先令3便士1+(1/11)法寻。现在一夸特要按它的价值出卖,并在实现价值时也提供5/9先令地租,它就必须比以前贵7便士1+(1/11)法寻出卖。较好土地上生产的小麦价值,在这里低于最坏土地上生产的小麦价值;如果这种最坏土地按照略好一点的土地,即提供地租的土地所生产的一夸特的价格出卖它的产品,那末它就低于产品的价值而按照产品的平均价格,也就是按照最坏土地能提供10%普通利润的那种价格出卖产品。因此,这种土地可以耕种,并给资本家提供普通平均利润。

在两种情况下,最坏土地在这里除了提供利润外,还会提供地租。

第一,如果一夸特小麦的价值高于6+(6/9)先令(由于需求增长,一夸特的价格可能高于6+(6/9)先令,即高于它的价值,但我们不研究这种情况;6+(6/9)先令,也就是早先最坏的耕地得以提供10镑地租的一夸特小麦价格,等于这块提供非级差地租的土地上生产的小麦的价值),就是说,如果早先最坏的耕地和其他一切土地,为了提供同一地租,相对地说比较不肥沃,以致它们的产品价值更加高于它们的平均价格和其他商品的平均价格。因此,新的最坏土地不提供地租,不是由于它贫瘠,而是由于其他地段相对的肥沃。提供地租的最坏的耕地,同投入新资本的新等级土地相比,则提供一般地租,即非级差地租。从提供地租的最坏土地上得到的地租不高于它现有的水平,正是因为这种提供地租的土地肥沃。

假定除了最后的提供地租的土地以外,还有三个等级的土地。等级II(在I即最后的提供地租的土地之上)提供的地租比等级多1/5,因为这种土地比等级I肥力大1/5,等级III提供的地租又比等级II多1/5,因为它比等级II肥力大1/5;等级IV也是这样,因为它比等级III肥力大1/5。由于等级I的地租等于10镑,等级II的地租就等于10+10的1/5=12镑,等级III等于12+12的1/5=14+(2/5)镑,等级IV等于14+(2/5)+[14+(2/5)]的1/5=17+(7/25)镑。\endnote{在这一段中,马克思还只是开始研究地租额(绝对地租额和级差地租额)对土地相对肥力的依存关系,他预先假定地租额和土地肥力成正比(就是说,如果某一级土地比另一级肥力大1/5,那末这一级土地的地租额也比比较不肥沃的那一级土地的地租额大1/5)。马克思在后来的研究中已经不用这个假定,而是就地租额对土地相对肥力的依存关系作了精确的表述。如果按照马克思后来的解释,根据等级II、III、IV提供的、并且按各级的共同价格即一夸特1/3镑出卖的夸特量来计算这三级土地的地租额,那末,II的地租额是34镑,III是62+(4/5)镑,IV是97+(9/25)镑。计算方法是这样的:因为II比I肥力大1/5,所以它提供360+72,即432夸特,卖得432/3镑即144镑;其中110镑是生产费用加平均利润;剩下34镑是地租(绝对地租和级差地租)。III、IV的计算方法也是这样。马克思把计算地租额的精确方法广泛应用于第十二章(《级差地租表及其说明》),但这个方法在第八章就已经拟定了。例如,在第109—110页,马克思起初重复引用了第102页的数字,即IV的地租额17+(7/25)镑,得出这一级土地的级差地租7+(7/25)镑,然后他拟定了确定IV的级差地租的正确方法,这就是:207+(9/25)镑减120镑,得87+(9/25)镑。如果再加上10镑绝对地租,那末IV的地租总额是97+(9/25)镑,完全符合马克思后来的结论。——第102页。}

如果IV的肥力减低,从III到I的地租就会[488]增多,IV的地租也会绝对增多(但它们之间的比例是否仍然不变?)。这一点可以从两方面来理解。如果I比较肥沃了,II、III、IV的地租就会相应减少。另一方面,I对II、II对III、III对IV的关系,同新加入的、不提供地租的那一级土地对I的关系一样。新等级的土地不提供地租,是因为I的小麦价值并不高于新土地产品的平均价格。如果I比较不肥沃了,I的小麦价值就会高于新土地产品的平均价格。这时,新土地也会提供地租。就I来说,情形也是这样。如果II比较肥沃了,I就不提供地租或提供较少的地租;II对III的关系,III对IV的关系,也是这样。因此,归根到底,这里是一种相反的顺序:IV的绝对肥力决定III的地租;如果IV还要肥沃一些,III、II、I就会提供较少的地租,或完全不提供地租。由此可见,I提供的地租,即非级差地租,是由IV的肥力决定的,正如新地完全不提供地租,是由I的肥力决定的一样。因此,在这里,施托尔希的规律是适用的:最肥沃的土地的地租决定最后的、一般提供地租的土地的地租,\endnote{施托尔希在他的著作《政治经济学教程》1815年圣彼得堡版第二卷第78—79页上写道:“最肥沃的土地的地租决定与最肥沃土地竞争的其他一切土地的地租量。只要最肥沃的土地的产品还足以满足需求,与最肥沃的土地竞争的比较不肥沃的土地就不可能被耕种,或者说,至少不提供地租。但是,只要需求开始超过肥沃土地所能提供的产品量,产品价格就会提高,比较不肥沃的土地就可能被耕种,就可能从这个土地上获得地租。”施托尔希的这个观点,马克思在《资本论》第三卷(第十、三十九章)中也谈到过。——第103、328页。}就是说,它也决定提供非级差地租的土地和完全不提供地租的土地之间的差额。

因此,第五个等级即新耕地I′(和I相区别)不提供地租,不是由于它本身贫瘠,而是由于它同I比较相对的贫瘠,也就是说,由于I比I′相对的肥沃。

[第二,]提供地租的I、II、III、IV级地的产品一夸特(为了更符合实际,可以用蒲式耳代替夸特)的价值6+(6/9)先令即6先令8便士,等于I′的平均价格而低于这一级本身的价值。但这里可能有许多中间阶段。如果I′花费100镑资本提供的蒲式耳量,是在它的实际收成330蒲式耳和I的收成360蒲式耳之间,例如333、340、350直到(360—x)蒲式耳,那末,一蒲式耳的价值,即6先令8便士,就会高于I′(一蒲式耳)的平均价格,这种最后的耕地也就会提供地租。这种土地一般提供平均利润,是因为I,也就是I—IV相对的贫瘠。这种土地不提供地租,是因为I相对的肥沃和它本身相对的贫瘠。如果一蒲式耳的价值高于6先令8便士,就是说,如果I、II、III、IV比较不肥沃了,最后的耕地I′就能提供地租,因为在这种情况下,小麦的价值提高了。但是,如果一蒲式耳的价值等于6先令8便士,就是说,I、II、III、IV的肥力照旧不变,而I′本身比较肥沃了,提供330蒲式耳以上,因而一蒲式耳6先令8便士的价值高于I′提供的一蒲式耳的平均价格,换句话说,如果它的平均价格现在低于6先令8便士,即低于I、II、III、IV种植的小麦价值,那末,在这种情况下,最后的耕地也能提供地租。既然价值高于平均价格,就会有一个高于平均利润的超额利润,从而就可能有地租。

由此可见:

在拿不同的生产领域(例如,工业和农业)作比较时,价值超过平均价格这一事实,证明提供超额利润即价值超过平均价格的余额的那个生产领域生产率较低。相反,在同一领域内,这一事实却证明这个资本比同一生产领域的其他资本生产率高。在上例中,I提供地租,一般说来,是因为农业中可变资本对不变资本之比大于工业,也就是说,农业中必须有更多的新劳动加到物化劳动上去,还因为土地所有权的存在使价值超过平均价格的余额不因资本竞争而平均化。但是[具体说来],I一般提供地租,还因为一蒲式耳价值6先令8便士不低于这一级土地产品的平均价格;就是说,因为这一级土地还不是那么贫瘠,以致它本身产品的价值高于一蒲式耳6先令8便士,并且决定它的产品的价格的,不是这一级土地本身产品的价值,而是II、III、IV(或者更确切地说,是II)上种植的小麦价值。这个市场价格现在是仅仅等于这一级土地本身产品的平均价格,还是高于这个平均价格,换句话说,I的产品的价值是否高于它的平均价格,这取决于这一级土地本身的生产率。

正因为如此,洛贝尔图斯所谓农业中任何提供平均利润的资本,都必定提供地租的观点是错误的。这个错误的结论,洛贝尔图斯是从[489]他的错误的理论基础得出的。洛贝尔图斯这样说:在农业中资本提供,比方说,10镑,但是,因为农业和工业不同,原料不算在内,所以10镑要用一个较小的总数来除,因而结果就大于百分之十。但是,问题的症结在于:不是所谓农业中原料不加入生产(相反,原料是加入本来意义的农业的;即使它不加入农业,也毫无关系,如果农业中机器等等相对增加的话)以致农产品价值高于平均价格(它们本身的平均价格和其他商品的平均价格),而是农业中可变资本对不变资本之比高于工业(不是工业中某些特殊的生产领域,而是全部工业的平均数)。这种总差别的大小决定I的地租即绝对的、非级差的、因而是最小的地租的大小和存在。而完全不提供地租的新耕地即I′的小麦价格,不是由这一级土地本身产品的价值决定,而是由I的产品的价值,因而是由I、II、III、IV所提供的小麦的平均市场价格决定的。

农业的特权(由土地所有权造成的)表现为:当农产品的价值高于平均价格时,农业不是按平均价格,而是按产品价值出卖自己的产品。这种特权,对于不同土地所生产的各种产品的相互关系,对于同一生产领域内生产的各种价值不同的产品,是完全不适用的。对于工业品来说,农产品只要求按自己的价值出卖。对于同一领域的其他产品来说,农产品要由市场价格来决定;价值(在这里,等于平均市场价格)是高还是低,即I的肥力是大还是小,以致I′在按照这个价值出卖它的产品时,在小麦价值和平均价格的总差额中所占份额是多是少还是完全没有,这取决于I的肥力。但是,既然洛贝尔图斯先生根本不去区分价值和平均价格,既然他认为商品按它们的价值出卖是一切商品的普遍规律(他不理解,这是农产品的特权),他就必然认为,最坏土地的产品也一定按它的个别价值出卖。但是,最坏土地的产品在和同类产品竞争时,是会丧失这种特权的。

但是,I′的产品的平均价格也可能高于I的产品的价值——一蒲式耳6先令8便士。可以设想(虽然这不完全正确),要使I′级地一般能投入耕种,需求必须增长。因此,I的小麦价格一定会提高到它的价值以上,即高于6先令8便士,并且这种提高是稳定的。在这种情况下,I′级地会投入耕种。如果它在产品价格为6先令8便士的情况下能够提供平均利润(虽然它的产品价值高于这个价格),并且满足需求,那末小麦的价格就会回到6先令8便士,因为现在需求又同供给相适应了;于是,I又必须按6先令8便士出卖它的产品,II、III、IV也是这样,从而I′也是如此。但是,如果I′的产品的平均价格是7先令8便士,从而这一等级只有按照这个价格(这个价格会大大低于它的个别价值)才能提供普通利润,那末,在不可能用其他方法满足需求时,一蒲式耳的价值就会固定在7先令8便士,而I的一蒲式耳符合于需求的价格就会提高到它的[个别]价值以上。II、III、IV的一蒲式耳的价格,已经高于它们的个别价值。它们的价格还会更加提高。但是,如果预期会有谷物输入,这在任何情况下都不会容许一蒲式耳的价格固定在7先令8便士,那末,如果有小租地农场主满足于平均利润以下的利润,I′仍然会投入耕种。这种情况,在农业中以及在工业中都经常发生。在这种情况下,也象在I′提供平均利润时一样,也可能支付地租,不过这种地租只是租地农场主利润的扣除部分。假如连这一点也做不到,土地所有者就会把土地租给茅舍贫农。茅舍贫农和手工织工一样,关心的主要是怎样挣得自己的工资,而把剩下的或大或小的余额以地租形式支付给土地所有者。象手工织工的情形一样,这种余额甚至不只是劳动产品的扣除部分,而且是劳动报酬的扣除部分。在所有这些情况下,都可能支付地租。在一种场合,地租是资本家利润的扣除部分。在另一种场合,土地所有者把通常由资本家占有的工人的剩余劳动据为己有。在最后一种场合,土地所有者也象资本家常做的那样,靠削减工人的工资来生活。但是,只有在最后的耕地至少能提供平均利润的地方,就是说,只有在I的产品的价值至少能给I′保证平均价格的地方,大规模的资本主义生产才是可能的。

我们在这里看到,价值和平均价格的区分使问题迎刃而解,并且证明李嘉图和他的反对者都是正确的\endnote{马克思在《资本论》第三卷第十章注30中写道,在农产品的市场价值问题上,李嘉图和施托尔希都是正确的,同时他们又都是不正确的,因为“完全忽视了中等情况”。——第107页。}。

[XI—490]如果提供绝对地租的I是唯一的一个等级的耕地,那末它的一蒲式耳小麦就会按它的价值,也就是按6先令8便士或6+(6/9)先令出卖,而不会把它的价格降低到平均价格6+(1/9)先令或6先令1+(1/3)便士。如果需求增长了,如果国内全部土地属于同一等级,如果耕地面积增加到十倍,那末,假定I花费100镑提供10镑地租,虽然只有唯一的一个等级的土地,地租也会增加到100镑。然而,地租无论就其对预付资本还是对耕地面积的比率或者说高度来看,并没有增加。耕种的英亩数大了十倍,预付资本大了十倍。因此,我们在这里看到的只是地租总额、地租量的增加,不是地租高度的增长。利润率不会下降,因为农产品的价值和价格保持不变。一个十倍的资本,当然能提供十倍的地租。另一方面,如果在同一土地面积上使用十倍的资本并产生同样的结果,那末,同花费的资本相比,地租率保持不变;同土地面积相比,地租率提高了,但它也不会引起利润率的任何变化。

但是,现在假定,耕地I所以变得比较肥沃,不是因为土质变化了,而是因为花费的不变资本增加了,可变资本减少了,机器、马匹、矿肥等形式的资本增加了,工资形式的资本减少了;这时,小麦的价值就会接近于它的平均价格和工业品的平均价格,因为同不变资本相比,可变资本份额减少了。在这种情况下,地租会下降,利润率则保持不变。如果这里生产方式发生变动,以致可变资本对不变资本的比例等于它们在工业中的平均比例,那末,小麦价值高于小麦平均价格的余额就会消失,从而地租,即超额利润也就会消失。I将不再支付地租,土地所有权将变得有名无实(假如生产方式的变动没有引起对土地的追加投资,结果土地所有者在租佃期满后就会从不是由他预付的资本得到利息;这正是土地所有者发财致富的主要手段之一,而在爱尔兰,围绕租佃权进行的斗争也是由此引起的)。如果除了I以外,还存在II、III、IV,这些土地也普遍采用了这种新的生产方式,那末由于它们的自然肥力比I大,它们还是会提供地租,并与这种较大的肥力成比例地提供地租。在这种情况下,I不再提供地租,II、III、IV的地租则相应下降,因为农业中生产率的一般比例,已经和工业中生产率的一般比例相等了。II、III、IV的地租是符合李嘉图的规律的;它只等于比较肥沃的土地超过比较不肥沃的土地所提供的超额利润,并且只是作为这种超额利润存在,正如工业中也存在这种超额利润一样,不同的只是:在工业中,这种超额利润没有自然赋予的固定化基础。

在完全不存在土地所有权的情况下,李嘉图的规律也同样起支配作用。如果土地所有权被废除而资本主义生产保存下来,这种由肥力不同引起的超额利润也不会消失。如果国家把土地所有权据为己有,而资本主义生产继续存在,II、III、IV的地租就会支付给国家,但地租本身还是存在。如果土地所有权归人民所有,资本主义生产的整个基础,使劳动条件变成一种独立于工人之外并同工人相对立的力量的基础,就不再存在了。

以后在考察地租时要分析的一个问题是:在耕作比较集约化的情况下,尽管地租对预付资本的比率下降,地租怎么可能在价值和总量上增加起来。这一点所以可能,显然只是因为预付资本量增加了。如果地租是1/5,后来变为1/10,那末20×(1/5)=4,而50×(1/10)=5。这就是全部问题之所在。但是,如果在耕作比较集约化的情况下,在农业中确立的各生产要素之间的比例,就是工业中的平均比例,而不只是接近于这个比例,那末,最贫瘠的土地的地租就会完全消失,比较肥沃的土地的地租,也会纯粹归结为土地的级差。绝对地租就会消失。

现在假定,由于需求增长,从I向II推移。I支付绝对地租,II支付级差地租,但小麦的价格{对于I是价值,对于II是超额价值}保持不变。利润率也保持不变。如此类推一直到IV。因而,如果我们把用于I、II、III、IV的全部资本加在一起计算,地租高度,地租率就会增加。但是II、III、IV的平均利润率仍然和I的利润率相等,后者又和工业的利润率即一般利润率相等。可见,[491]在向比较肥沃的土地推移时,虽然利润率和小麦价格保持不变,地租量和地租率却可能增长。地租高度和地租量的增长,在这里是由II、III、IV的资本生产率提高引起的,而不是由I的资本生产率减低引起的。不过,提高了的生产率不象工业中那样必定使利润提高,并使商品价格和工资降低。

但是,如果向相反的方向——从IV向III、II、I推移,那末一蒲式耳的价格就要提高到6先令8便士,按照这个价格,I的小麦每100镑[花费的资本]还会提供10镑地租。也就是IV的小麦每100镑[花费的资本]提供的地租是17+(7/25)镑,不过,其中7+(7/25)镑是[IV的全部产品的市场]价格超过I的[全部]产品价值的余额。I(在地租为10镑、一蒲式耳的价值为6先令8便士时)花费100镑资本提供360蒲式耳小麦,II提供432蒲式耳,III提供518+(2/5)蒲式耳,IV提供622+(2/25)蒲式耳。但是,一蒲式耳6先7令8便士的价格使IV每花费100镑提供7+(7/25)镑超额地租,IV的3蒲式耳卖1镑,622+(2/25)蒲式耳卖207+(9/25)镑。但是IV的产品的[个别]价值和I一样,只有120镑;超过这个数字的,就是它的[市场]价格超过它的[个别]价值的余额\endnote{这个例子中的两种计算方法,见注25。——第110页。}。IV要是按3先令10+(8/27)便士出卖一蒲式耳小麦,那就是按价值出卖,按照这个价格,它也是每100镑[花费的资本]有10镑地租。如果现在从IV推移到III,从III推移到II,从II推移到I,那末,一蒲式耳的价格(以及地租)就会提高,直到最后达到I的6先令8便士的水平,在I那里,这个价格现在提供的地租,和以前IV提供的地租一样。随着[农产品]价格的提高,利润率却会下降,这部分地因为生活资料和原料的价值会提高。从IV向III的推移可能如下。由于需求增长,IV的价格提高到它的价值以上,因而这一等级不仅提供地租,而且提供超额地租。结果,III投入耕种,它按照这个价格出卖产品,在提供普通平均利润的情况下就不应提供任何地租。如果由于IV的产品价格提高,不是利润率下降,而是工资下降了,那末,III将提供[以前的]平均利润。但是因为有III的追加供给,工资势必又提高到正常水平;于是III的利润率就下降,等等。

因此,在这个下降运动中,利润率下降要有下列假设的前提:III按照IV的价格不能提供地租;它所以能够耕种并提供以前的利润率,只是因为工资暂时下降到自己的水平以下。

在这些前提下,李嘉图的规律又是可能的了。但即使按照李嘉图的观点,这个规律也不是必然的。它只在各种条件的一定结合下才是可能的。实际上各种运动是互相交错的。

\centerbox{※     ※     ※}

综上所述,地租理论实质上已经讲清了。

洛贝尔图斯先生由于他的“材料价值”观念,认为地租存在于事物的永恒本性中,至少存在于资本主义生产的永恒本性中。在我们看来,地租是资本有机组成部分的比例的历史性差别造成的,这种差别一部分会趋于平衡,甚至随着农业的发展会完全消失。诚然,即使绝对地租消失了,仅仅由土地自然肥力不同而引起的差别仍会存在。但是——把自然差别可能拉平这一点完全撇开——这种级差地租是同市场价格的调节作用联系在一起的,因而会随着价格和资本主义生产一起消失。保持不变的只是这种情况:社会劳动耕种肥力不同的土地,而且,尽管使用的劳动量不同,这种社会劳动在各种土地上的生产率都会提高。但是较坏土地产品所耗费的较大的劳动量,决不会产生资产阶级制度下的那种后果,也就是对较好土地的产品也必须以较大的劳动量来支付。相反,在IV上节省下来的劳动,会用来改良III,在III上节省下来的劳动,会用来改良II,在II上节省下来的劳动,会用来改良I;因此,现在由土地所有者吞食的全部资本,那时将被用来使不同土地上的劳动相等,并使农业上花费的总劳动量减少。

\centerbox{※     ※     ※}

[492]{如我们在前面看到的\authornote{见本卷第1册第46—64、73—78页。——编者注},

亚·斯密起初对价值以及对价值组成部分即利润、工资等的关系发表过正确的观点,后来走上了相反的道路,把工资价格、利润价格、地租价格假定为某种既定的东西,试图把它们规定为独立的量,并把它们加起来得出商品的价格。这种向相反观点的转变意味着:斯密起初是从事物的内部联系考察事物,后来却从它们在竞争中表现出来的颠倒了的形式去考察事物。他天真地把这两种考察方法交织在一起,而且没有觉察到它们之间的矛盾。相反,李嘉图有意识地把竞争形式,把竞争造成的表面现象抽象化,以便考察规律本身。应该指责李嘉图的是,一方面,他的抽象还不够深刻,不够完全,因而当他,比如说,考察商品价值时,一开始就同样受到各种具体关系的限制;另一方面是,他把表现形式理解为普遍规律的直接的、真正的证实或表现;他根本没有揭示这种形式的发展。就第一点来说,他的抽象是极不完全的,就第二点来说,他的抽象是形式的,本身是虚假的。}

\tsectionnonum{[(10)地租率和利润率。不同历史发展阶段上农业生产率和工业生产率的关系]}

现在回过头来简略地分析一下洛贝尔图斯的其他观点。

\begin{quote}{“由国民产品价值的增加引起的工资、资本盈利和地租的增加,在国内既不能提高工资,也不能提高资本盈利,因为现在更多的工资要在更多的工人中分配,增加的盈利要分摊到按同一比例增加的资本上;而地租不管怎样一定会提高,因为地租始终分摊到面积不变的地段上。因此,我所发挥的理论能够令人满意地解释土地价值的巨大增长,——土地价值不外是按普通利息率资本化的地租,——而不必求助于假定农业劳动生产率日益减退,这种假定,同人类社会能够日益完善的思想,以及同一切农业的和统计的事实,是正相矛盾的。”(第160—161页)}\end{quote}

首先要指出,李嘉图[洛贝尔图斯的这个论断是针对他的]在任何地方也没有力求解释“土地价值的巨大增长”。对他来说,这是全然不成问题的。其次,李嘉图(见后面对李嘉图观点的分析)自己明确地说,在谷物或[一般]农产品的价值不变时,在地租率既定时,地租可能增加。\authornote{见本册第357—358页。——编者注}这种增加对李嘉图来说,也是不成问题的。对他来说,成问题的,不是地租总额在地租率保持不变时的增加,而是地租率即地租对预付农业资本之比的提高;因此,问题也不是农产品总量的价值的提高,而是同一数量的农产品例如一夸特小麦价值的提高,——随着这种提高,产品价值超过产品平均价格的余额,从而超过利润率的地租余额,也会增加起来。洛贝尔图斯先生在这里把李嘉图的问题抛开了(更不用说洛贝尔图斯的错误的“材料价值”观念了)。

当然,尽管农业生产率越来越高,地租对预付资本的比率,也就是说,农产品同工业品相比的相对价值,也能够提高。这种情况之所以可能发生有以下两个原因:

第一,拿上例来看,从I推移到II、III、IV,也就是推移到越来越肥沃的土地(但是,比较肥沃的土地所提供的产品量,还没有大到足以使I停止耕种,或者说,足以使价值和平均价格之间的差额缩小到这种程度,以致IV、III、II提供按比例减少的地租,I完全不提供地租)。假如I提供10镑地租,II提供20镑,III提供30镑,IV提供40镑,假如在这四级土地上各投下100镑,那末,I的地租是预付资本的1/10或10%,II是2/10或20%,III是3/10或30%,IV是4/10或40%。总共是100镑地租对400镑预付资本,平均地租率是100/4=25%。如果以投入农业的全部资本来看,现在地租是25%。如果只有I级地(最贫瘠的土地)继续耕种,地租是40镑对400镑,即仍然是10%,不会增加到25%。但是在第一种情况下(如果在I上花费100镑,生产330蒲式耳)只会生产1320蒲式耳,每蒲式耳价格6先令8便士;在第二种情况下(全部四级土地都耕种)却生产1500蒲式耳,而价格相同。在这两种情况下,预付资本是相同的。\endnote{马克思在这里撇开了预付在等级I、II、III、IV上的农业资本的利润。如果I的100镑资本生产330蒲式耳,一蒲式耳为6先令8便士(或1/3镑),那末这一等级的总产品价值是110镑,其中10镑归地租,就没有什么留作利润了。同样,所有四个等级如果各自花费100镑,它们的总产品价值是500镑,这个数额就仅仅是补偿所花费的资本400镑和I、II、III、IV的地租额100镑(10+20+30+40)。——第114页。}

但是,在这里,地租高度的增加不过是表面上的。\endnote{由于在比较肥沃的土地上使用的资本支付级差地租而造成的地租高度(率)的增长“不过是表面上的”,意思是说,这种增长是以“虚假的社会价值”为基础的,关于“虚假的社会价值”,马克思在《资本论》第三卷第三十九章比较详细地谈到。如马克思接着在文中说明的,承租比较肥沃的土地的资本家按比较不肥沃的土地的产品价格出卖他的产品,“就好象他生产同量产品所需要的资本”和比较不肥沃的土地“一样多”。——第114页。}的确,如果我们就产品来计算资本的支出,那就可以看到,在I上,要生产330蒲式耳,必须花费100镑,要生产1320蒲式耳,必须花费400镑。而现在要生产1320蒲式耳,只须花费100+90+80+70,即总共340镑。\endnote{数字90、80、70,马克思大概是用来表示投入等级II、III、IV的资本和这些资本所提供的级差地租(II的100镑资本提供10镑级差地租,III提供20镑,IV提供30镑)之间的差额。如果根据II的产品是360蒲式耳、III的产品是390蒲式耳、IV的产品是420蒲式耳进行确切的计算,那末得出的数字就是:91+(2/3)镑,84+(8/13)镑和78+(4/7)镑。——第114页。}II的90镑和I的100镑生产的同样多,III的80镑和II的90镑生产的同样多,IV的70镑和III的80镑生产的同样多。II、III、IV的地租率,同I比较是提高了。

如果就整个社会来考察,现在要生产同样多的产品,不是使用400镑资本,而是使用340镑,就是说,只使用[原来]资本的85%。

[493]1320蒲式耳只是按不同于第一种情况的方式进行分配。租地农场主现在花费90镑[预付资本]必须交付的[地租],和以前花费100镑交付的一样多,现在花费80镑必须交付的,和以前花费90镑交付的一样多,现在花费70镑必须交付的,和以前花费80镑交付的一样多。但是,现在花费资本90镑、80镑、70镑给他提供的产品,和以前花费100镑提供的正好一样多。他交付的多了,不是因为他为了获得同量产品必须花费的资本多了,而是因为他花费的资本少了;不是因为他的资本的生产率减低了,而是因为这个资本的生产率提高了,但租地农场主照旧按I的产品价格出卖他的产品,就好象他生产同量产品所需要的资本和以前一样多。

[第二,]除了地租率的这种提高(这种提高和各个工业部门中超额利润按不同程度提高是一致的,虽然在工业中这种提高是不固定的),只可能有第二种情况:虽然农产品价值保持不变,就是说,虽然劳动生产率没有减低,地租率却可能提高。这种情况或者是发生在这样的时候:农业生产率和以前一样,但工业生产率提高了,并且这种提高表现在利润率下降;就是说,工业中可变资本对不变资本之比降低了。或者发生在这样的时候:农业生产率也提高,但提高的比例和工业不一样,比工业小。如果农业生产率按1∶2提高,工业生产率按1∶4提高,那末,相对地说,这就同农业生产率保持不变,而工业生产率提高一倍一样。在这种情况下,可变资本对不变资本之比降低,在工业中比在农业中快一倍。

在这两种情况下,工业的利润率都会下降,由于利润率下降,地租率就会提高。在其他情况下,利润率下降不是绝对的(不如说保持不变),它不过对地租来说下降了,这不是因为它本身下降了,而是因为地租提高了,地租对预付资本的比率提高了。

李嘉图没有把上述这些情况加以区别。除开这些情况(即利润率——虽然它是固定的——由于用在比较肥沃土地上的资本有级差地租而相对下降,或者,由于工业生产率的增长,不变资本和可变资本的一般比例发生变化,因而农产品价值超过其平均价格的余额增加),地租率就只有在工业生产率不增长而利润率下降的条件下才能提高。而这又只有在由于农业生产率减低,工资提高或原料价值增加的情况下才能发生。在这种情况下,利润率的下降和地租高度的增长,是由于同一个原因——农业生产率减低,农业中使用的资本的生产率减低。这就是李嘉图的见解。在货币价值不变时,这必然表现为原产品价格的提高。如果价格的提高,象前面考察的那样,是相对的,那末货币价格的任何变动,都不会使农产品的货币价格,与工业品相对而言,绝对地提高。在货币价值降低50%的情况下,值3镑的一夸特小麦就会值6镑,但是值1先令的一磅棉纱也会值2先令。因此,与工业品相对而言,农产品货币价格的绝对提高,决不能用货币价值的变动来解释。

一般说来,应该承认,在原始的、资本主义前的生产方式下,农业生产率高于工业,因为自然在农业中是作为机器和有机体参与人的劳动的,而在工业中,自然力几乎还完全由人力代替(例如手工业等等)。在资本主义生产蓬勃发展的时期,同农业比较,工业生产率发展很快,虽然工业的发展以农业中可变资本和不变资本之比已经发生巨大变化为前提,就是说,以大批人从土地上被赶走为前提。以后,生产率无论在工业中或农业中都增长起来,虽然速度不同。但是工业发展到一定阶段,这种不平衡必定开始缩小,就是说,农业生产率必定比工业生产率相对地增长得快。这里包括:(1)懒惰的农场主被实业家,农业资本家所取代,土地耕种者变为纯粹的雇佣工人,农业大规模经营,即以积聚的资本经营;(2)特别是:大工业的真正科学的基础——力学,在十八世纪已经在一定程度上臻于完善;那些更直接地(与工业相比)成为农业的专门基础的科学[494]——化学、地质学和生理学,只是在十九世纪,特别是在十九世纪的近几十年\authornote{即四十年代和五十年代。——编者注},才发展起来。

根据对两个不同生产部门的商品价值的简单比较,来谈两个部门生产率的大小,是荒谬的。如果1800年1磅棉花值2先令,1磅棉纱值4先令,而1830年棉花的价值是2先令,或者比如说,是18便士,棉纱的价值是3先令或1先令8便士,那就可以比较这两个部门生产率增长的比例。但是,所以可能这样做,只是因为拿1800年的水平作为出发点。可是,如果根据1磅棉花值2先令,1磅棉纱值3先令,就是说,如果根据生产棉花的劳动比纺纱工人的(新加)劳动多一倍,就得出结论说,一种劳动的生产率比另一种劳动的生产率高一倍,这是荒谬的,——这就象因为织造画布比画家在布上绘成的画便宜,就说画家的劳动比织造画布的劳动生产率低一样荒谬。

这里,只有下面的说法才是正确的(如果“生产率”这个概念,也是从资本主义意义即生产剩余价值这个意义上理解,同时考虑到产品的相对量):

如果平均说来,为了在棉纺织工业中使用100个工人花费100镑,按照生产条件,必须在原料、机器等等{价值既定}上花费500镑,另一方面,如果为了在小麦生产中使用100个工人,同样花费100镑,在原料和机器上必须花费150镑,那末,在前一种情况下,可变资本占总资本600镑的1/6,占不变资本的1/5;在后一种情况下,可变资本占总资本250镑的2/5,占不变资本的2/3。这样,投入棉纺织工业的每100镑,只能包含16+(2/3)镑可变资本,而必须包含83+(1/3)镑不变资本;在后一种情况下,却包含40镑可变资本和60镑不变资本。在前一种情况下,可变资本占总资本的1/6即[16+(2/3)]%,在后一种情况下占40%。现在价格史方面的著作少得可怜,这是很明显的。当理论还没有指明必须研究的究竟是什么时,这方面的著作也不能不少得可怜。在剩余价值率既定,比方说,等于20%时,剩余价值在前一种情况下是3+(1/3)镑(因而,利润等于[3+(1/3)]%);在后一种情况下是8镑(因而,利润等于8%)。棉纺织工业中劳动的生产率不如谷物生产中劳动的生产率高,却正是因为棉纺织工业中劳动的生产率较高(就是说,棉纺织工业中的劳动在生产剩余价值的意义上生产率不那么高,是因为它在生产产品的意义上生产率较高)。顺便指出,很明显,例如在棉纺织工业中,[总资本和可变资本之间]1∶(1/6)这个比例只有在不变资本(取决于机器等)花费比如说10000镑,工资花费2000镑,因而总资本是12000镑的情况下才有可能。如果只花费6000,其中工资是1000,那末机器的生产率就会比较低等等。如果只花费100镑,那就根本不能经营。另一方面,可能在花费23000镑的情况下,机器效率增大,其他方面有节约等等,以致也许用不着把全部19166+(2/3)镑用于不变资本,较多的原料和同量的劳动所需要的机器等等少了(按价值),而且在这上面将节省1000镑。于是,可变资本对不变资本之比又会增大,但这不过是因为资本绝对增加了。这是阻碍利润率下降的一个因素。两笔12000镑的资本,会和一笔23000镑的资本生产同量的商品,但是,第一,商品会贵些,因为它们多花费了1000镑;第二,利润率会低一些,因为在23000镑资本中,可变资本占的份额大于总资本的1/6,因而大于两笔12000镑资本总额内可变资本所占的份额。[494]

[494](一方面,随着工业的进步,机器效率更高,也更便宜,因而,如果农业中使用的机器和过去数量相同的话,农业中这部分不变资本会减少;但是机器数量的增加快于它的价格的降低,因为机器这个要素在农业中的发展还是薄弱的。另一方面,随着农业生产率的增长,原料,比如说棉花,价格下降,因而,原料作为价值形成过程的组成部分,和原料作为劳动过程的组成部分,不是按照同一比例增加的。)\endnote{马克思放在括号内的这一段,在手稿中原来是在下两段之后(也在第494页上),即插在篇幅不大的关于配第和戴韦南特的地租量变动观点的历史补充部分中间。从这一段的内容来看,是同前面马克思关于农业生产率和工业生产率的关系的论述相衔接的。——第118页。}

\centerbox{※     ※     ※}

[494]配第已经说过,当时地主害怕农业改良,因为改良的结果会使农产品价格和地租(就其高度来说)下降;由于同样的原因,他们也害怕土地面积增加,并且害怕以前还没有利用的土地投入耕种,因为这等于土地面积增加。(在荷兰,土地面积的增加应该从更直接的意义上理解。)配第说:

\begin{quote}{“地主对排干沼泽、垦伐森林、圈围公有地、栽种驴喜豆和三叶草常出怨言,因为这些做法引起食品价格的下降。”(《政治算术》1699年伦敦版第230页)“整个英格兰和威尔士以及苏格兰低地,一年的地租约为900万镑。”(同上,第231页)}\end{quote}

配第反对地主的这些观点,而戴韦南特进一步发挥[495]这样一种主张:地租的高度可能降低,同时地租量即地租总额却可能增加。戴韦南特说:

\begin{quote}{“地租可能在一些地区和一些郡内下降,但整个说来国内的土地〈他是指土地的价值〉仍然可以不断改良;比方说,如果公园和森林被垦伐,公有地被圈围,如果沼泽被排干,许多地段由于耕种和施肥而改良,那末,自然,这一定会使那些过去已经充分改良、现在已无法再改良的土地的价值减低;虽然某些私人的地租收入因此降低,但与此同时,王国的总地租却由于这些改良而提高。”(戴韦南特《论公共收入和英国贸易》1698年伦敦版第2部分第26—27页)“1666年至1688年期间,私人地租下降了,但王国的地租总额,在这期间比前几年有更大的增加,因为这段时间内土地的改良比以往任何时候都大,都普遍。”(同上,第28页)}\end{quote}

这里我们也看到,英国人说到地租的高度,总是指地租对资本之比,而决不是指地租对王国土地总面积之比(或者笼统地对英亩之比,象洛贝尔图斯先生所说的那样)。

\tchapternonum{[第九章]对所谓李嘉图地租规律的发现史的评论[对洛贝尔图斯的补充评论](插入部分)}

\tsectionnonum{[(1)安德森发现级差地租规律。安德森理论的剽窃者马尔萨斯为了土地所有者的利益歪曲安德森的观点]}

安德森是个实践的租地农场主。他的第一部顺便考察地租性质的著作,出版于1777年。\endnote{马克思指安德森的著作:《谷物法本质的研究;关于为苏格兰提出的新谷物法案》1777年爱丁堡版。——第120页。}当时,詹姆斯·斯图亚特爵士对于很大一部分公众来说还是最有威望的经济学家,但同时,普遍的注意力已经转向一年以前出版的《国富论》了\endnote{指亚当·斯密的著作:《国民财富的性质和原因的研究》1776年伦敦版。——第120页。}。相反,这个苏格兰租地农场主就当时争论的一个直接的实际问题而写的著作(作者在这部著作中不是专门谈地租的,只是顺便说明了地租的性质),却没有能够引起人们的注意。安德森在这部著作中只是偶然地而不是专门地考察地租的。在他自己出版的三卷文集《论农业和农村事务》(三卷集,1775—1796年爱丁堡版)中,有一两篇文章也顺便谈到他的这个理论。1799—1802年伦敦出版的《关于农业、博物学、技艺及其他各种问题的通俗讲座》(见英国博物馆\endnote{英国博物馆是英国国立博物馆(建立于1753年),位于伦敦。博物馆的最重要部分是图书馆,它是世界上最大的图书馆之一。马克思和恩格斯都在英国博物馆的图书馆里从事过研究工作。1908年5月至6月,弗·伊·列宁在图书馆中从事过研究。描绘十九世纪中叶英国博物馆阅览室情况的英国版画的复制品,见本卷第1册第406—407页之间的插图。——第120页。})也是这样的情形。所有这些都是直接为租地农场主和农业家写的著作。如果安德森意识到他的发现的重要性,并且把它作为《地租性质的研究》单独地献给公众,或者,如果他有一点靠贩卖自己的思想为生的才能,就象他的同乡麦克库洛赫靠贩卖别人的思想为生那样,那情况就不同了。

安德森的理论的复制品出现于1815年,一开始就是作为单独的对地租性质的理论研究出现的。这可以从威斯特和马尔萨斯的有关著作的标题看出来:

马尔萨斯:《关于地租的本质和增长的研究》。

威斯特:《论资本用于土地》。

其次,马尔萨斯利用安德森的地租理论,为的是使自己的人口规律第一次同时有政治经济学的和现实的(博物学的)论据,而他从以前的著作家那里借用的关于几何级数和算术级数的荒谬说法,则是一种纯粹空想的假设。马尔萨斯先生立即“抓住了”这个机会。可是李嘉图,正象他自己在序言\endnote{指李嘉图为他的《政治经济学和赋税原理》第一版写的序言。见大卫·李嘉图《政治经济学和赋税原理》1821年伦敦第3版第VI—VII页。——第121页。}中说的那样,是把这个地租学说当作整个政治经济学体系最重要的环节之一,并且赋予它以崭新的理论上的重要性,而在实践方面就更不用说了。

李嘉图显然不知道安德森,因为他在他的政治经济学序言中,把威斯特和马尔萨斯叫做地租理论的创始人。从威斯特叙述这个规律的独特方式判断,他大概也不知道安德森,就象图克不知道斯图亚特一样。马尔萨斯先生的情况就不同了。把马尔萨斯的著作同安德森的著作仔细比较一下,就可以看出马尔萨斯知道安德森,并且利用安德森。马尔萨斯本来就是一个职业剽窃者。[496]只要把他论人口的著作第一版\endnote{[托·罗·马尔萨斯]《人口原理》1798年伦敦版。——第121、125页。}同我以前引用过的唐森牧师的著作\endnote{马克思指唐森的书《论济贫法》(1786年伦敦版)。马克思在1861—1863年手稿第III本(第112—113页)《绝对剩余价值》这一节中引用了这本书。马克思在那里引用的三段引文也在《资本论》第一卷第二十三章引用过(见《马克思恩格斯全集》中文版第23卷第709页)。——第121页。}对比一下,就会相信,马尔萨斯不是作为具有自由创作思想的人来加工唐森的著作,而是作为盲从的剽窃者照抄和转述唐森的著作,同时没有一个地方提到唐森,隐匿了唐森的存在。

马尔萨斯利用安德森的观点的方式,是很有特色的。安德森维护鼓励谷物输出的出口奖励和限制谷物输入的进口税,决不是从地主的利益出发,而是认为这样的立法会“降低谷物的平均价格”,保证农业生产力的均衡发展。马尔萨斯采用安德森的这个实际结论,则因为马尔萨斯作为英国国教会的真诚教徒,是土地贵族的职业献媚者,他从经济学上替土地贵族的地租、领干薪、挥霍、残忍等等辩护。只是在工业资产阶级的利益同土地所有权的利益,同贵族的利益一致时,马尔萨斯才拥护工业资产阶级的利益,即拥护他们反对人民群众,反对无产阶级;但是,凡是土地贵族同工业资产阶级的利益发生分歧并且互相敌对时,马尔萨斯就站在贵族一边,反对资产阶级。因此,他为“非生产劳动者”、消费过度等等辩护。

相反,安德森认为,支付地租的土地和不支付地租的土地之间,或支付不同地租的土地之间所以会产生差别,是因为不提供地租或提供较少地租的土地,同提供地租或提供较多地租的土地比较起来,相对的不肥沃。但是他明确地说,不同等级土地的这种相对的肥沃程度,从而较坏等级土地同较好等级土地比较起来相对的不肥沃,同农业的绝对生产率绝对没有任何关系。相反,他曾着重指出,各种等级土地的绝对肥力不但能够不断提高,而且随着人口的增长也必定会提高,他还进一步断言,不同等级土地的肥力的不平衡,能够日益趋于平衡。他说,英国农业目前的发展程度,丝毫不能说明它的可能的发展。因而他说,在一国可能是谷物价格高而地租低,在另一国可能是谷物价格低而地租高;这是从他的基本原理出发的,因为在这两个国家,地租的高低和地租本身的存在,决定于肥沃土地和贫瘠土地之间的差别,但其中任何一个国家,地租都不决定于绝对肥力;其中每一个国家,地租只决定于现有各种等级土地的肥沃程度的差别,其中任何一个国家,地租都不决定于各种等级土地的平均肥力。他由此得出结论说,农业的绝对生产率同地租绝对没有任何关系。因此他后来声明——我们在后面会看到\authornote{见本册第158页。——编者注}——他是马尔萨斯人口论的死敌,可是他没有料到他自己的地租理论会成为这种奇谈怪论的根据。安德森说明,在英国,1750至1801年的谷物价格高于1700至1750年,决不是由于越来越不肥沃的土地投入耕种,而是由于这两个时期立法对农业的影响。

马尔萨斯干了什么呢?

他利用安德森的理论,代替自己的(也是剽窃来的)几何级数和算术级数的怪诞幻想——他把这种怪诞幻想当作“漂亮辞句”保留着——来证明自己的人口论。在安德森理论的实际结论符合地主利益的限度内,他保留安德森理论的实际结论,——仅仅这一事实就证明,马尔萨斯同安德森本人一样,不了解这个理论同资产阶级社会的政治经济学体系的联系;——他不去考察这个理论的创始人的反证,就利用这个理论去反对无产阶级。从这个理论出发,在理论上和实践上向前迈进一步的使命就落到了李嘉图身上,这就是:在理论上,作出商品的价值规定等等,并阐明土地所有权的性质;在实践上,反对资产阶级生产基础上的土地私有权的必要性,并且更直接地反对国家促进这种土地所有权发展的一切措施,如谷物法。马尔萨斯得出的唯一的实际结论在于:为地主在1815年要求的保护关税辩护——这是巴结贵族,——并且对财富生产者的贫困进行新的辩解,为劳动剥削者进行新的辩护。从这一方面来说,是巴结工业资本家。

马尔萨斯的特点,是思想极端卑鄙,——只有牧师才可能这样卑鄙,[497]他把人间的贫困看作对罪恶的惩罚,而且在他看来,非有一个“悲惨的尘世”不行,但是同时,他考虑到他所领取的牧师俸禄,借助于关于命运的教义,认为使统治阶级在这个悲惨的尘世上“愉快起来”,对他是极为有利的。这种思想的卑鄙还表现在马尔萨斯的科学工作上。第一,表现在他无耻的熟练的剽窃手艺上;第二,表现在他从科学的前提做出的那些看人眼色的而不是毫无顾忌的结论上。

\tsectionnonum{[(2)发展生产力的要求是李嘉图评价经济现象的基本原则。马尔萨斯为统治阶级最反动的分子辩护。达尔文实际上推翻了马尔萨斯的人口论]}

李嘉图把资本主义生产方式看作最有利于生产、最有利于创造财富的生产方式,对于他那个时代来说,李嘉图是完全正确的。他希望为生产而生产,这是正确的。如果象李嘉图的感伤主义的反对者们那样,断言生产本身不是目的本身,那就是忘记了,为生产而生产无非就是发展人类的生产力,也就是发展人类天性的财富这种目的本身。如果象西斯蒙第那样,把个人的福利同这个目的对立起来,那就是主张,为了保证个人的福利,全人类的发展应该受到阻碍,因而,举例来说,就不能进行任何战争,因为战争无论如何会造成个人的死亡。(西斯蒙第只是与那些掩盖这种对立、否认这种对立的经济学家相比较而言,才是正确的。)这种议论,就是不理解:“人”类的才能的这种发展,虽然在开始时要靠牺牲多数的个人,甚至靠牺牲整个阶级,但最终会克服这种对抗,而同每个个人的发展相一致;因此,个性的比较高度的发展,只有以牺牲个人的历史过程为代价。至于这种感化议论的徒劳,那就不用说了,因为在人类,也象在动植物界一样,种族的利益总是要靠牺牲个体的利益来为自己开辟道路的,其所以会如此,是因为种族的利益同特殊个体的利益相一致,这些特殊个体的力量,他们的优越性,也就在这里。

由此可见,李嘉图的毫无顾忌不仅是科学上的诚实,而且从他的立场来说也是科学上的必要。因此对李嘉图来说,生产力的进一步发展究竟是毁灭土地所有权还是毁灭工人,这是无关紧要的。如果这种进步使工业资产阶级的资本贬值,李嘉图也是欢迎的。如果劳动生产力的发展使现有的固定资本贬值一半,那将怎样呢?——李嘉图说,——要知道人类劳动生产率却因此提高了一倍。这就是科学上的诚实。如果说李嘉图的观点整个说来符合工业资产阶级的利益,这只是因为工业资产阶级的利益符合生产的利益,或者说,符合人类劳动生产率发展的利益,并且以此为限。凡是资产阶级同这种发展发生矛盾的场合,李嘉图就毫无顾忌地反对资产阶级,就象他在别的场合反对无产阶级和贵族一样。

而马尔萨斯呢!这个无赖,从已经由科学得出的(而且总是他剽窃来的)前提,只做出对于贵族反对资产阶级以及对于贵族和资产阶级两者反对无产阶级来说,是“合乎心意的”(有用的)结论。因此,他不希望为生产而生产,他所希望的只是在维持或加强现有制度并且为统治阶级利益服务的那种限度内的生产。

他的第一部著作\endnote{[托·罗·马尔萨斯]《人口原理》1798年伦敦版。——第121、125页。},就已经是靠牺牲原著而剽窃成功的最明显的写作例子之一。这部著作的实际目的,是为了英国现政府和土地贵族的利益,“从经济学上”证明法国革命及其英国的支持者追求改革的意图是空想。一句话,这是一本歌功颂德的小册子,它维护现有制度,反对历史的发展;而且它还为反对革命法国的战争辩护。

他1815年关于保护关税和地租的著作\endnote{指马尔萨斯的两本小册子《关于限制外国谷物进口政策的意见的理由》和《关于地租的本质和增长的研究》。——第126页。},部分地是要证明他以前为生产者的贫困所作的辩解,但首先是为了维护反动的土地所有权,反对“开明的”、“自由的”和“进步的”资本,特别是要证明英国立法当时为了保护贵族利益反对工业资产阶级而采取的倒退措施是正确的。\endnote{指1815年的谷物法。该法禁止向英国输入谷物,直到英国国内谷物价格每夸特不低于80先令时为止。——第126页。}最后,[498]他的《政治经济学原理》是反对李嘉图的,这本书的根本目的,就是要把“工业资本”及其生产率依以发展的那些规律的绝对要求,纳入从土地贵族、国教会(马尔萨斯所属的教会)、政府养老金领取者和食税者的现有利益看来是“有利的”和“适宜的”“范围”。但是,一个人如果力求使科学去适应不是从科学本身(不管这种科学如何错误),而是从外部引出的、与科学无关的、由外在利益支配的观点,我就说这种人“卑鄙”。

从李嘉图来说,他把无产者看成同机器、驮畜或商品一样,却没有任何卑鄙之处,因为无产者只有当作机器或驮畜,才促进“生产”(从李嘉图的观点看),或者说,因为无产者在资产阶级生产中实际上只是商品。这是斯多葛精神,这是客观的,这是科学的。只要有可能不对他的科学犯罪,李嘉图总是一个博爱主义者,而且他在实际生活中也确是一个博爱主义者。

马尔萨斯牧师就完全不同了。他[也]为了生产而把工人贬低到驮畜的地位,甚至使工人陷于饿死和当光棍的境地。[但是]在同样的生产的要求减少地主的“地租”时,在同样的生产的要求威胁国教会的“什一税”或“食税者”的利益时,或者,在为了一部分代表生产进步的工业资产阶级而去牺牲另一部分本身利益阻碍生产进步的工业资产阶级时,——总之,在贵族的某种利益同资产阶级的利益对立时,或者,在资产阶级中保守和停滞的阶层的某种利益同进步的资产阶级的利益对立时,——在所有这些场合,马尔萨斯“牧师”都不是为了生产而牺牲特殊利益,而是竭尽全力企图为了现有社会统治阶级或统治阶级集团的特殊利益而牺牲生产的要求。为了这个目的,他在科学领域内伪造自己的结论。这就是他在科学上的卑鄙,他对科学的犯罪,更不用说他那无耻的熟练的剽窃手艺了。马尔萨斯在科学上的结论,是看着统治阶级特别是统治阶级的反动分子的“眼色”捏造出来的;这就是说,马尔萨斯为了这些阶级的利益而伪造科学。相反,对于被压迫阶级,他的结论却是毫无顾忌的,残酷无情的。他不单单是残酷无情,而且宣扬他的残酷无情,厚颜无耻地以此自夸,并且在用他的结论反对“无权者”时,把他的结论夸大到极端,甚至超过了从他的观点看来还可以在科学上说得过去的程度。\authornote{[499}例如李嘉图(见前),当他的理论使他得出结论说,把工资提高到工资最低限度以上并不会增加商品的价值时,他就直接说出这一点。而马尔萨斯坚持工资保持低水平,目的是要资产者借此来发财。[499]]

因此,英国工人阶级憎恨马尔萨斯——科贝特粗鲁地称他为“江湖牧师”(科贝特虽然是当代英国最大的政论家;但他缺少莱比锡大学教授\endnote{暗指德国庸俗经济学家、莱比锡大学教授罗雪尔。——第127页。}的教养,并且公开反对“学者的语言”),——对马尔萨斯的这种憎恨是完全正当的;人民凭着真实的本能感觉到,在这里反对他们的不是一个科学的人,而是一个被他们的敌人收买的统治阶级的辩护士,是统治阶级的无耻的献媚者。

一个最初发现某种思想的人,可能由于善意的误解,把这种思想夸大到极端;而一个把这种思想夸大到极端的剽窃者,却总是把这种夸大当作“有利可图的生意”。

马尔萨斯的“人口论”这部著作第一版没有包含一个新的科学词汇;这本书只应看作卡普勤教士喋喋不休的说教,只应看作是用阿伯拉罕·圣克拉\endnote{阿伯拉罕·圣克拉是奥地利传教士和著作家乌尔利希·梅格尔勒(1644—1709年)的笔名,他力图用公众易懂的形式宣传天主教,并用所谓民间文体来进行“救人”的说教和写劝善的作品。——第128页。}文体对唐森、斯图亚特、华莱士、埃尔伯等人的论断的改写。因为这本书实际上只是指望以它的大众化的形式来引人注意,所以它理所当然地要引起大众的憎恨。

同资产阶级政治经济学界那些可怜的和谐论者比较起来,马尔萨斯的唯一功绩,就是特别强调不和谐。的确,他决没有发现不和谐,但他毕竟以牧师所固有的扬扬得意的厚颜无耻肯定、描绘并宣扬了这种不和谐。

\centerbox{※     ※     ※}

[499]查理·达尔文在他的著作《根据自然选择即在生存斗争中适者保存的物种起源》1860年伦敦版(第五次印刷,一千册)[第4—5页]的绪论中说:

\begin{quote}{“下一章将考察全世界整个生物界中的生存斗争,那是依照几何级数高度繁殖的不可避免的结果。这是马尔萨斯学说对于整个动物界和整个植物界的应用。”}\end{quote}

达尔文在他的卓越的著作中没有看到,他在动物界和植物界发现了“几何”级数,就是把马尔萨斯的理论驳倒了。马尔萨斯的理论正好建立在他用华莱士关于人类繁殖的几何级数同幻想的动植物的“算术”级数相对立上面。在达尔文的著作中,例如在谈到物种消灭的地方,也有在细节上(更不用说达尔文的基本原则了)从博物学方面对马尔萨斯理论的反驳。而当马尔萨斯的理论以安德森的地租理论为依据时,他的理论又被安德森本人驳倒了。\endnote{在手稿中,紧接着插入了一小段话,在这段话中,马克思用李嘉图关于工资水平的观点同马尔萨斯关于这个问题的观点相对比。这段话以脚注的形式移至前面,即本册第127页。——第128页。}

[499]

\tsectionnonum{[(3)罗雪尔歪曲地租观点的历史。李嘉图在科学上公正的例子。投资于土地时的地租和利用其他自然要素时的地租。竞争的双重作用]}

[499]安德森顺便发挥了地租理论的第一部著作,是具有实际意义的论战性著作,它谈的不是地租,而是保护关税政策。这部著作出版于1777年,它的标题已经表明:第一,它追求实际的目的;第二,它涉及当时一个直接的立法行为,在这个问题上工业家和地主的利益是对立的。这部著作就是:《谷物法本质的研究;关于为苏格兰提出的新谷物法案》(1777年爱丁堡版)。

1773年的法律(英格兰的法律;关于这个问题参看麦克库洛赫的书目\endnote{马克思指麦克库洛赫的著作《政治经济学文献。各科分类书目。附史评、论述和作者介绍》1845年伦敦版。——第129页。})在1777年(好象)应该在苏格兰施行(见英国博物馆)。

\begin{quote}{安德森说:“1773年的法律是由一种公开宣布的意图引起的,这种意图就是要为我国的工业家降低谷物价格,以便通过进口奖励,保证本国人民得到更便宜的食物。”(《关于导致不列颠目前粮荒的情况的冷静考察》1801年伦敦版第50页)}\end{quote}

由此可见,安德森的著作是一部维护包括地主在内的农业主利益(保护关税政策)、反对工业家利益的论战性著作。安德森是把自己的书当作“公开”维护一定党派的利益的著作出版的。地租理论在这里只是顺便谈到。而且在他后来一些总是或多或少涉及上述利益斗争的著作中,他也只是顺便把地租理论重新提了一两次,从来没有想到它的科学意义,或者说,甚至没有想到把它作为一个独立的论题。了解了这一点,我们就可以判断显然不知道安德森著作的威廉·修昔的底斯·罗雪尔\endnote{马克思用古希腊大历史学家修昔的底斯的名字来称呼罗雪尔,这是因为,如马克思在《剩余价值理论》第三册(马克思手稿第922页)中所说,“罗雪尔教授先生谦虚地宣称自己是政治经济学的修昔的底斯”。罗雪尔在他的《国民经济学原理》的序言中不知羞耻地引证了修昔的底斯。“修昔的底斯·罗雪尔”这个称呼具有辛辣的讽刺性:正象马克思在本章和其他许多地方指出的那样,罗雪尔既严重歪曲了经济关系的历史,又严重歪曲了经济理论的历史。——第130页。}的下述意见是否正确:

\begin{quote}{“值得惊奇的是,在1777年几乎无人注意的学说,到1815年以后却突然引起人们极大的关心,一些人拥护,另一些人反驳,因为这个学说涉及货币所有者和土地所有者之间在这个时期如此尖锐起来的对立。”(《国民经济学原理》1858年第3版第297—298页)}\end{quote}

在这段话中,错误和字数一样多。第一,安德森并没有象威斯特、马尔萨斯和李嘉图那样,把自己的见解当作“学说”提出来。第二,这个见解不是“几乎”而是“完全”无人注意。第三,这个见解最初在一部著作中顺便被提到,这部著作专门只以1777年在工业家和地主之间大大发展起来的利益对立为中心,只“涉及”这个实际的利益斗争,但根本“没有涉及”一般的[500]国民经济学理论。第四,在1815年,这个理论被它的复制者之一马尔萨斯大肆宣扬,完全是为了维护谷物法,就象安德森所做的一样。同一个学说,它的创始人和马尔萨斯都是用来维护土地所有权,而大卫·李嘉图却用来反对土地所有权。因此,至多可以说,这个理论的一些拥护者维护土地所有权的利益,而另一些拥护者反对这种利益,但是不能说,在1815年拥护土地所有权的人反对这个理论(因为马尔萨斯在李嘉图之前拥护这个理论),也不能说,反对土地所有权的人拥护这个理论(因为大卫·李嘉图无须“拥护”这个理论去反对马尔萨斯,因为他自己把马尔萨斯当作这个理论的创始人之一,并且把马尔萨斯当作自己的先驱;他只是不得不“反驳”马尔萨斯从这个理论得出的实际结论)。第五,威廉·修昔的底斯·罗雪尔所“涉及”的“货币所有者”和“土地所有者”之间的对立,直到现在同安德森的地租理论、同复制这个理论、同拥护这个理论、同反对这个理论都绝对没有任何关系。威廉·修昔的底斯可以从约翰·斯图亚特·穆勒的书(《略论政治经济学的某些有待解决的问题》1844年伦敦版第109—110页)中知道:(1)英国人所说的“金融阶级”是指货币贷放者;(2)这些货币贷放者或者一般是靠利息生活的人,或者是职业的放债人,如银行家、票据经纪人等等。照穆勒所说,所有这些人都以“金融阶级”的身分同“生产阶级”(穆勒所说的生产阶级是指“工业资本家”,而把工人撇在一边)相对立,或者至少同生产阶级有区别。因此,威廉·修昔的底斯应当看到,“生产阶级”即工业家,工业资本家的利益,同金融阶级的利益是极不相同的两回事;这些阶级也是不同的阶级。其次,威廉·修昔的底斯应当看到,工业资本家和地主之间的斗争决不是“货币所有者”和“土地所有者”之间的斗争。如果威廉·修昔的底斯了解1815年谷物法的历史和围绕谷物法进行的斗争,他就会从科贝特的著作中知道,地主(土地所有者)和放债人(货币所有者)曾经一道反对工业资本家。不过科贝特是“粗鲁的”。最后,威廉·修昔的底斯从1815至1847年的历史应当知道,大部分货币所有者,甚至一部分商人(例如在利物浦),在围绕谷物法的斗争中,都成了土地所有者反对工业资本家的同盟者。[500]

[502](可以使罗雪尔先生感到惊奇的至多是这件事:同一个“学说”,在1777年用来维护“土地所有者”,而在1815年却成了反对“土地所有者”的武器,并且只是到这时才开始引起人们的注意\endnote{指1815年伦敦出版的威斯特的著作《论资本用于土地,对谷物进口严加限制的失策》和李嘉图的著作《论谷物的低价对资本利润的影响;证明限制进口的不当》。——第131页。}。)[502]

[500]如果我要把威廉·修昔的底斯在他的历史文献评介中犯的所有这类粗暴歪曲历史的错误,都如此详细地加以说明,那我就得写一部象他的《国民经济学原理》一样厚的书,而这样的书确实“不值得花费那么多纸张来写”。这么一个威廉·修昔的底斯在学术上的无知,对于其他科学领域的研究者会造成多么有害的影响,从阿·巴斯提安先生的例子可以看出来。巴斯提安在自己的著作《历史上的人》1860年版第一卷第374页的注中,就引用了威廉·修昔的底斯上述的这段话来证明一个“心理学的”原理。顺便说一下,关于巴斯提安,不能说“技巧驾驭了材料”\authornote{见奥维狄乌斯《变形记》。——编者注}。在这里倒是技巧应付不了自己的原材料。而且,通过我所“知道”的不多的几门科学,我发现,知道“一切”科学的巴斯提安先生经常信赖威廉·修昔的底斯之流的权威,这对于“万能学者”一般是不可避免的。

[501]我希望人们不要责备我对威廉·修昔的底斯“残酷无情”。这个书呆子自己对待科学是多么“残酷无情”啊!既然他胆敢以高傲自大的口气谈论李嘉图的“半真理”\endnote{威·罗雪尔《国民经济体系》,第1卷《国民经济学原理》1858年斯图加特和奥格斯堡增订第3版第191页。——第132页。},那我无论如何也同样可以谈他的“全无真理”。而且,威廉·修昔的底斯在列举书目方面一点也不“公正”。在他看来,谁不“值得尊敬”,谁就在历史上不存在;例如,在他看来,洛贝尔图斯作为地租理论家是不存在的,因为洛贝尔图斯是“共产主义者”。而且,威廉·修昔的底斯对于“值得尊敬的著作家”的看法也不准确。例如,在麦克库洛赫看来,贝利是存在的,甚至被看作是划时代的人。在威廉·修昔的底斯看来,贝利是不存在的。要在德国发展并普及政治经济科学[502],本应该由洛贝尔图斯这样一些人来创办一种杂志,向一切研究者开放(但不向书呆子和庸俗化者开放),并且杂志的主要目的是揭露在这门科学本身及其历史方面的职业学者的不学无术。[502]

\centerbox{※     ※     ※}

[501]安德森根本没有研究他的地租理论同政治经济学体系的关系,这一点是不足为奇的,因为他的第一本书是在亚·斯密《国富论》出版一年以后出版的,当时“政治经济学体系”一般说来还刚刚在形成,因为斯图亚特的体系也只是在几年以前才出现。不过,如果谈到安德森在他考察的特殊对象范围内所拥有的材料,那末这种材料无疑比李嘉图的更广泛。正如李嘉图在他根据休谟的理论复制而成的货币理论中主要只看到1797至1809年间的事件一样,李嘉图在他根据安德森的理论复制而成的地租理论中只看到1800至1815年间谷物价格上涨的经济现象。

\centerbox{※     ※     ※}

下面这几段话,很能说明李嘉图的特点:

\begin{quote}{“如果因考虑到某一个阶级的利益而使国家财富和人口的增长受到阻碍,我将感到非常遗憾。”(李嘉图《论谷物的低价对资本利润的影响》1815年伦敦第2版第49页)}\end{quote}

在谷物自由输入的情况下,“土地停止耕种”。(同上,第46页)因此,土地所有权为发展生产而被牺牲。

可是,在谷物自由输入的同样情况下:

\begin{quote}{“不可否认会有一定数量的资本损失掉。但是,拥有资本或保持资本是目的呢,还是手段?毫无疑问是手段。我们所需要的是商品的富足〈一般财富〉,如果能证明,牺牲我们的资本的一部分,我们就可以增加用于使我们享乐和幸福的那些物品的年生产,那我们就不应当为我们的资本的一部分遭受损失而发牢骚。”(《论农业的保护关税》1822年伦敦第4版第60页)}\end{quote}

李嘉图把不属于我们或他的、而是由资本家投入土地的资本,叫作“我们的资本”。但是这个我们是谁呢?是指国民的平均数。“我们的”财富的增加就是社会财富的增加,这个社会财富本身就是目的,而不管这个财富由谁分享!

\begin{quote}{“对于一个拥有2万镑资本,每年获得利润2000镑的人来说,只要他的利润不低于2000镑,不管他的资本是雇100个工人还是雇1000个工人,不管生产的商品是卖1万镑还是卖2万镑,都是一样的。一个国家的实际利益不也是这样吗?只要这个国家的实际纯收入,它的地租和利润不变,这个国家的人口有1000万还是有1200万,都是无关紧要的。”(《政治经济学和赋税原理》第3版第416页)}\end{quote}

在这里,“无产阶级”为财富而被牺牲。在无产阶级对于财富的存在是无关紧要的时候,财富对于无产阶级的存在也是无关紧要的。在这里群众本身——人类大众——是“不值什么的”。

这三个例子[502]表明了李嘉图科学上的公正。

\centerbox{※     ※     ※}

{土地(自然)等等,是农业中使用的资本借以投入的要素。因此在这里,地租等于这个要素中所创造的劳动产品的价值超过这个产品平均价格的余额。如果有一种属于个人私有财产的自然要素(或物质)加入别一种生产,而又不构成这种生产的(物质)基础,那末地租——如果这种地租只是由于该要素单纯加入生产而产生的话——就不可能是这个产品的价值超过平均价格的余额,而只是这个产品的一般平均价格超过它自身平均价格的余额。例如,瀑布可以为工厂主代替蒸气机,并使工厂主能够节省煤炭。工厂主拥有这个瀑布,比方说,就可以经常把纱卖得比它的[个别]平均价格贵,并得到超额利润。如果这个瀑布为土地所有者所有,这种超额利润就会以地租的形式归土地所有者所得。霍普金斯先生在他论“地租”的书中也指出,在朗卡郡,瀑布不只提供地租,而且还因瀑布落差的天然能量不同,而提供级差地租。\endnote{马克思指霍普金斯的书《关于调节地租、利润、工资和货币价值的规律的经济研究》1822年伦敦版。这本书的有关段落马克思在后面(见本册第153页)引用了。——第135页。}这里的地租无非就是产品的市场平均价格超过它的个别平均价格的余额。}[502]

\centerbox{※     ※     ※}

[502]{在竞争中,应当区分两种平均化运动。在同一生产领域内部,资本把这个领域内部生产的商品的价格平均化为同一市场价格,而不管这些商品的[个别]价值同这个市场价格的关系怎样。如果没有不同生产领域之间的平均化,平均市场价格就应当等于商品的[市场]价值。这些不同领域之间的竞争,在资本的相互作用不被第三种力量——土地所有权等等——阻碍、破坏的情况下,把[市场]价值平均化为平均价格。}

\tsectionnonum{[(4)洛贝尔图斯关于产品变贵时价值和剩余价值的关系问题的错误]}

洛贝尔图斯认为,如果一种商品比另一种商品贵(因而,如果前者比后者物化的劳动时间多),那末,在不同领域的剩余价值率相等,或者说工人受的剥削程度相同的情况下,这种商品也必定会因此包含更多的无酬劳动时间,即剩余劳动时间。他的这个观点是完全错误的。

如果同一劳动在贫瘠土地上(或在歉收年)生产1夸特小麦,而在肥沃土地上(或在丰收年)生产3夸特小麦;如果同一劳动在很富的金矿生产1盎斯金,而在较贫的或已经枯竭的金矿只生产1/3盎斯金;如果生产1磅羊毛需要的劳动时间同把3磅羊毛纺成纱需要的劳动时间一样多,那末,首先,1夸特小麦和3夸特小麦1的价值,1盎斯金和1/3盎斯金的价值,1磅羊毛和3磅毛纱的价值(扣除其中包含的羊毛价值)一样大。它们包含着同量的劳动时间,因而,按照[剩余价值率相等]的假定,也包含着同量的剩余劳动时间。当然,1夸特[贫瘠土地上长的小麦]包含的剩余劳动量[比肥沃土地上长的1夸特小麦]多,然而,在前一场合我们只有1夸特小麦,在后一场合却有3夸特小麦;或者,在前一场合我们有1磅羊毛,而在后一场合有3磅纱(减去材料的价值)。因而,在两种场合[剩余价值]量是相等的。如果拿一个单位商品同另一个单位商品相比较,那末剩余价值的比例量也是相等的。按照假定,1夸特[坏地上长的小麦]或1磅羊毛与3夸特[好地上长的小麦]或3磅纱包含着同样多的劳动。因此,在两种场合,花费在工资上的资本同剩余价值之比是相同的。1磅羊毛包含的劳动时间比1磅纱包含的大两倍。如果剩余价值大两倍,那末与它有关的花费在工资上的资本也大两倍。因而,它们二者之比仍然一样。

在这里,洛贝尔图斯的计算是完全错误的,他完全错误地把花费在工资上的资本同[503]物化着用工资换来的劳动的那个较多或较少的商品量相比。只要象洛贝尔图斯假定的那样,工资(或剩余价值率)是既定的,那末,这种计算就是完全错误的。同一劳动量,例如12小时,可以表现为x商品或3x商品。一种场合的1x商品和另一种场合的3x商品包含着同样多的劳动和剩余劳动;但是,不论在哪一种场合,所花费的都不多于一个工作日,而且不论在哪一种场合,剩余价值率都不大于例如1/5。第一种场合1x的1/5比x,同第二种场合3x的1/5比3x是一样的。如果我们把三个x分别叫做x′、x″、,那末x′、x″、各包含4/5的有酬劳动和1/5的无酬劳动。另一方面,如果在生产率较低的条件下要生产出同在生产率较高的条件下一样多的商品,那末商品中就会包含更多的劳动,因而也包含更多的剩余劳动。这是完全正确的。不过这时就要相应地花费更多的资本。为了生产3x,就应当比生产1x(在工资上)多花费两倍的资本。

当然,毫无疑问,工业加工的原料不可能多于农业所提供的原料,因而,例如,用来纺纱的羊毛磅数不可能超过生产出来的羊毛磅数。所以,如果纺毛的生产率提高两倍,那末,在假定羊毛生产条件不变的情况下,为生产羊毛花费的时间就要比过去增加两倍,为生产羊毛使用的资本也要增加两倍,可是把这些数量增加两倍的羊毛纺成纱,只需要和过去一样多的纺纱工人的劳动时间。但是,[剩余价值]率不变。同量纺纱工人的劳动创造的价值和过去一样多,包含的剩余价值也一样多。生产羊毛的劳动所包含的剩余价值增加了两倍,但是与此相适应,包含在羊毛中的全部劳动或预付在工资上的资本也增加了两倍。因而,增大了两倍的剩余价值要按增大了两倍的资本来计算。所以不能根据这一点说,纺纱生产中的剩余价值率比羊毛生产中的低。只能说后一部门花费在工资上的资本比前一部门多两倍(因为这里假定,纺纱和羊毛生产中的变化,都不是由于它们的不变资本的变化引起的)。

这里必须把下述情况加以区别。同一劳动加不变资本提供的产品,歉收年少于丰收年,贫瘠土地少于肥沃土地,较贫的金属矿少于较富的金属矿。因而,前者的产品比较贵,也就是说,同量产品在这里包含着较多的劳动和较多的剩余劳动;可是在第二种场合产品的数量比较多。其次,两类产品的每一单位产品中有酬劳动和无酬劳动之比,并不因此而发生任何变化;因为,如果单位产品中包含较少的无酬劳动,那末根据假定,它也相应地包含较少的有酬劳动。因为这里假定,资本各有机组成部分的比例,即可变资本和不变资本的比例,没有任何变化。我们已经假定,同数额的可变资本和不变资本,在不同的条件下会提供不同的,即较多或较少的产品量。

看来洛贝尔图斯先生经常把这些东西混淆起来,并且好象理所当然地从产品的单纯变贵得出了关于剩余价值更多的结论。至于剩余价值率,那末根据所作的假定,这种结论已经是不正确的了,至于剩余价值量,那也只有在下述条件下才正确:在一种场合比另一种场合预付更多的资本,也就是说,原来生产多少较便宜的产品,现在也生产多少较贵的产品,或者(象上述纺纱的例子那样),较便宜的产品数量的增加,先要有较贵的产品数量相应的增加。

\tsectionnonum{[(5)李嘉图否认绝对地租——他的价值理论中的错误的后果]}

[504]尽管地租率不变或者甚至下降,地租,从而土地价值,可能增长,也就是说,农业生产率也增长,——这一点李嘉图有时忘记,但他是知道的。无论如何安德森是知道这一点的,而且配第和戴韦南特就已经知道了。问题不在这里。

李嘉图撇开了绝对地租问题,他为了理论而否认绝对地租,因为他从错误的前提出发:如果商品的价值决定于劳动时间,商品的平均价格就必定等于商品的价值(因此他又做出一个与实际相矛盾的结论:比较肥沃的土地的竞争必然使比较不肥沃的土地停止耕种,即使后者过去是提供地租的)。如果商品的价值和它们的平均价格等同,那末绝对地租——即最坏的耕地上的地租或最初的耕地上的地租——在这两种情况下都是不可能的。什么是商品的平均价格?就是生产商品花费的全部资本(不变资本加可变资本)加上包含在平均利润(例如10%)中的劳动时间。因此,如果资本在某种要素中——仅仅因为这是一种特殊的自然要素例如土地,——生产出高于平均价格的价值,那末这种商品的价值就会超过它的价值,而这个超额价值就同价值等于一定量劳动时间的价值概念发生矛盾。结果就成了:一种自然要素,即某种不同于社会劳动时间的东西,创造了价值。但这是不可能的。因此,投在仅仅作为土地的土地上的资本不可能提供任何地租。最坏的土地仅仅是土地。如果说较好土地提供地租,这只不过证明,社会必要劳动和个别必要劳动之间的差别在农业中固定下来了,因为这种差别在这里有一个自然基础,而这种差别在工业中却是不断消失的。

因而,不应该有绝对地租存在,只可能有级差地租存在。因为承认绝对地租存在,就是承认同量劳动(投入不变资本的物化劳动和用工资购买的劳动)由于投入不同的要素,或加工不同的材料,会创造不同的价值。但是如果承认,虽然在每一个生产领域的产品中物化着同一劳动时间,却依然存在这种价值差别,那就是承认,不是劳动时间决定价值,而是某种不同于劳动时间的东西决定价值。这种价值量的差别会取消价值概念,推翻下述原理:价值实体是社会劳动时间,因而价值的差别只能是量的差别,而这个量的差别只能等于所耗费的社会劳动时间量的差别。

由此可见,从这个观点出发,为了保持价值范畴——不仅价值量决定于不同的劳动时间量,而且价值实体决定于社会劳动,——就要否认绝对地租。而否认绝对地租可以表现为两种说法。

第一,最坏的土地不能提供地租。较好等级土地的地租可以这样来解释:比较肥沃的土地和比较不肥沃的土地的产品市场价格是一样的。但是最坏的土地仅仅是土地,它本身没有等级差别。它只是不同于工业投资的特殊投资领域。如果它提供地租,那末地租的产生就是由于:同一劳动量用在不同的生产领域表现为不同的价值,从而不是劳动量本身决定价值,包含等量劳动的产品[在价值上]彼此也就不等。

[505]或者[第二],最初的耕地不能提供地租。因为,什么是最初的耕地呢?“最初的”耕地,既不是较好的土地,也不是较坏的土地。这仅仅是土地,是没有等级差别的土地。最初,农业投资与工业投资的区别仅仅在于资本所投入的领域不同。但是,既然等量劳动表现为等量价值,那就绝对没有理由使投入土地的资本除了利润之外还提供地租,除非投入这个领域的同一劳动量生产出一个较大的价值,以致这个价值超过工业中生产的价值的余额构成等于地租的超额利润。但是,这就意味着确认土地本身创造价值,也就是意味着取消价值概念本身。

因此,最初的耕地最初不能提供任何地租,否则整个价值理论就要被推翻。而且这一点很容易(虽然不一定,这可以从安德森那里看到)同这样一种观点联系起来,即认为人们最初自然不是选择最坏的土地,而是选择最好的土地耕种,因而,最初不提供地租的土地,后来由于人们不得不进而耕种越来越坏的土地,也开始提供地租了,这样,在向地狱下降的过程中,即向越来越坏的土地推移的过程中,随着文明的进步和人口的增长,地租必然在最初耕种的最肥沃的土地上,而后逐步地在越来越坏的土地上产生,然而始终代表仅仅是土地——特殊的投资领域——的最坏土地任何时候都不提供地租。所有这些论点在逻辑上多少是互相联系的。

相反,如果知道,平均价格和价值并不是等同的,商品的平均价格可能等于商品价值,也可能大于或小于商品价值,那末问题就不存在了,问题本身不存在了,为解决问题而提出的假设也就不存在了。剩下的只是这样一个问题:为什么农业中商品的价值(或者,无论如何,它的价格)不是超过商品的价值,而是超过商品的平均价格呢?但是这个问题已经完全不涉及理论的基础即价值规定本身了。

李嘉图当然知道,商品的“相对价值”随着加入商品生产的固定资本和花费在工资上的资本之间的比例不同而发生变化。{但是这两种资本并不构成对立面;彼此对立的是固定资本和流动资本,流动资本不仅包括工资,而且包括原料和辅助材料。例如,在采矿工业和渔业中花费在工资上的资本和固定资本之比,可能和裁缝业中花费在工资上的资本和花费在原料上的资本之比相同。}但李嘉图同时知道,这些相对价值由于竞争而平均化。而且,他承认上述变化只是为了在这些不同的投资中得出同一的平均利润。这就是说,他所说的相对价值无非就是平均价格。他甚至没有想到过价值和平均价格是不同的。他只以它们的等同作出发点。但是,因为在资本各有机组成部分之间的比例不同的情况下不存在这种等同,他就把这种等同假定为还没有得到解释的、由竞争引起的事实。因此他也没有想到这样一个问题:为什么农产品的价值不平均化为平均[506]价格?相反,他假定农产品的价值会平均化,而且他正是从这个观点出发提出问题的。

真不懂威廉·修昔的底斯\authornote{指罗雪尔。——编者注}之流的好汉们为什么拥护李嘉图的地租理论。从他们的观点看来,李嘉图的“半真理”——照修昔的底斯的宽容的说法——丧失了它的全部价值。

在李嘉图看来,问题所以存在,只是因为价值是由劳动时间决定的。在这些好汉们看来,就不是这么回事。照罗雪尔的看法,自然本身就具有价值。关于这一点请参看后面。\endnote{马克思在《剩余价值理论》中以后没有再回过头来分析罗雪尔的这些观点。但是在《理论》的第三册《李嘉图学派的解体》一章中,马克思详细地批判了麦克库洛赫的类似的庸俗观点,这些观点的形成同罗雪尔的观点一样,受了让·巴·萨伊提出的“生产性服务”的辩护论见解的很大影响,马克思在下一段谈到这种辩护论见解。马克思在《资本论》第一卷第六章注22中提到过罗雪尔把自然看成价值源泉之一的观点(《马克思恩格斯全集》中文版第23卷第232页)。并见《资本论》第3卷第48章。——第142页。}这就是说,罗雪尔根本不知道什么是价值。既然如此,那还有什么能妨碍他让土地价值从一开始就加入生产费用并形成地租,把土地价值即地租作为解释地租的前提呢?

在这些好汉们那里,“生产费用”一词是毫无意义的。我们从萨伊那里看到这种情况。在他那里,商品的价值决定于生产费用——资本、土地、劳动,而这些费用又决定于供求。这就是说,根本不存在什么价值规定。既然土地提供“生产性服务”,那末这些“服务”的价格为什么不应该象劳动和资本提供的服务的价格那样决定于供求呢?既然“土地的服务”为一定的卖主占有,那末为什么他们的商品不应该有市场价格,从而,为什么地租不应该作为价格要素存在呢?

我们看到,威廉·修昔的底斯是没有丝毫理由如此热心“拥护”李嘉图的理论的。

\tsectionnonum{[(6)李嘉图关于谷物价格不断上涨的论点。1641—1859年谷物的年平均价格表]}

如果撇开绝对地租不谈,在李嘉图那里还有这样一个问题:

人口增加,对农产品的需求也增加,因而农产品的价格上涨,就象在类似条件下工业中发生的情况一样。但是,在工业中,一旦需求起了作用,引起了商品供给的增加,这种价格上涨就停止了。这时产品就降到它原来的价值,甚至降到原来的价值以下。但是,在农业中,这种追加产品不是按原来的价格,也不是按更低的价格投入市场。它的价值更大,从而引起市场价格不断上涨,同时也引起地租提高。如果这不是由于不得不耕种越来越不肥沃的土地,为生产同样的产品需要花费越来越多的劳动,农业生产率越来越降低,那又怎样解释呢?撇开货币贬值的影响不说,为什么在英国从1797年到1815年农产品价格随着人口的迅速增加而上涨呢?农产品价格后来又下降的事实不能证明什么。国外市场的供给被切断也不能证明什么。恰恰相反。这只是创造了使地租规律能以纯粹形式表现出来的适当条件。因为恰恰是同外国失去联系迫使国内去耕种越来越不肥沃的土地。农产品价格上涨不能用地租的绝对增加来解释,因为不仅地租总额增加了,而且地租率也提高了。一夸特小麦等等的价格提高了。也不能用货币贬值来解释,因为货币贬值只能解释为什么在工业生产率高度发展情况下工业品价格跌落,也就是说农产品价格相对上涨。货币贬值不能解释为什么农产品价格除了这种相对上涨之外,还不断地绝对上涨。

同样也不能认为这是利润率下降的后果。利润率下降决不能说明价格的变动,而只能说明价值或价格在地主、工业家和工人之间的分配的变动。

关于货币贬值,我们假定以前1镑等于现在2镑。一夸特小麦以前值2镑,现在值4镑。假定工业品跌到1/10,过去值20先令,现在值2先令。但是这2先令现在等于4先令。货币贬值——正如歉收一样——在这里当然会有影响。

[507]但是,撇开这一切不谈,可以假定,就当时的农业(小麦)情况来看,贫瘠土地投入了耕种。这种土地后来就变成肥沃的了,因为级差地租(就地租率看)下降了,正如小麦价格这个最好的晴雨表所表明的那样。

最高价格发生在1800和1801年以及1811和1812年。其中前两年是歉收年份,后两年是货币贬值最严重的年份。同样,1817和1818年也是货币贬值的年份。但是,如果把这些年份除掉,那末剩下的就应该看作是(请看后面)这一时期或那一时期的平均价格。

在比较不同时期小麦等等的价格时,必须同时把产品生产量和每夸特价格相对照,因为这样才看得出追加谷物的产量对价格的影响。

\todo{}

可见,从1650到1699年50年的平均价格是44先令2+(1/5)便士。

从1641到1649年(9年)期间,最高年平均价格是革命的一年即1645年的75先令6便士,其次是1649年的71先令1便士,1647年的65先令5便士,最低价格是1646年的42先令8便士。

\todo{}

从1700到1749年50年的年平均价格:35先令9+(29/50)便士。

\todo{}

从1750到1799年50年的年平均价格:45先信纸3+(13/50)便士。

\todo{}

从1800到1849年50年的年平均价格:69先令6+(9/50)便士。

从1800到1859年60年的年平均价格:66先令9+(14/15)便士。

因此,年平均价格是:

\todo{}

\centerbox{※     ※     ※}

连威斯特也说:

\begin{quote}{“在农业改良了的情况下,用在旧制度下最好土地上用的那样少的费用,就能够在二等或三等质量的土地上进行生产。”(爱德华·威斯特爵士《谷物价格和工资》1826年伦敦版第98页)}\end{quote}

\tsectionnonum{[(7)霍普金斯关于绝对地租和级差地租之间的区别的猜测;用土地私有权解释地租]}

霍普金斯正确地捉摸到了绝对地租和级差地租之间的区别:

\begin{quote}{“竞争原则使同一国家不可能有两种利润率;但是这一点决定相对地租,而不决定地租的总平均数。”(托·霍普金斯《论地租及其对生存资料和人口的影响》1828年伦敦版第30页)}\end{quote}

[508a]霍普金斯在生产劳动和非生产劳动——或者,照他的说法,首要劳动和次要劳动——之间作了下述区别:

\begin{quote}{“如果所有劳动者都被用来达到象钻石匠和歌剧演员被用来达到的同一目的,那末不要很久就将没有财富来养活这些人了,因为那时生产的财富中就没有什么可以再变成资本了。如果相当大一部分劳动者从事这类劳动,工资就会低,因为生产出来的东西只有比较小的一部分用作资本;但是,如果只有少数劳动者从事这类劳动,因而,几乎所有劳动者都是农夫、鞋匠、织工等等,那末,就会生产出许多资本,工资也会相应地高。”(同上,第84—85页)“所有为地主或食利者劳动并以工资形式得到他们的一部分收入的人,即所有实际上把自己的劳动限于生产供地主和食利者享乐的东西,并以自己的劳动换得地主的一部分地租或食利者的一部分收入的人,——所有这些人都必须列入钻石匠和歌剧演员一类。这些人都是生产劳动者,但是,他们的全部劳动都是为了把那种以地租和货币资本的收入形式存在的财富变为更能满足地主和食利者需要的其他某种形式,因此,这些劳动者是次要生产者。其他一切劳动者是首要生产者。”(同上,第85页)}\end{quote}

钻石和歌唱——这两者在这里被看作物化劳动——可以象一切商品一样转化为货币,并作为货币转化为资本。但是,在货币向资本的这种转化中要区别两种情况。一切商品都可以转化为货币,并作为货币转化为资本,因为在它们所采取的货币形式中,它们的使用价值和使用价值的特殊自然形式消失了。它们是具有社会形式的物化劳动,在这种社会形式中,这一劳动本身可以同任何实在劳动交换,因而可以转化为任何形式的实在劳动。相反,作为劳动产品的商品本身是否能够重新作为要素进入生产资本,这要看它们的使用价值的性质是否容许它们或者作为劳动的客观条件(生产工具和材料),或者作为劳动的主观条件(工人的生活资料),也就是作为不变资本或可变资本的要素重新进入生产过程。

\begin{quote}{“在爱尔兰,照适中的计算和1821年人口调查,交给地主、政府和什一税所得者的全部纯产品达2075万镑,而全部工资却只有14114000镑。”(霍普金斯,同上第94页)在意大利,“土地耕种者在耕作水平不高、固定资本极少的情况下,一般把产品的半数甚至半数以上作为地租交给地主。人口的较大部分是次要生产者和土地所有者,而首要生产者一般地说是一个穷困的、受侮辱的阶级”。(第101—102页)“法国在路易十四[以及他的后继者路易十五和路易十六]的时代有同样情况。照[阿瑟·]杨格的计算,地租、什一税和税收总计140905304镑。同时农业处于悲惨的状况。法国人口当时是26363074人。如果劳动人口就算有600万户——显然过高了,——那末每个劳动者家庭每年直接或间接必须交给地主、教会和政府的纯财富平均约23镑。根据杨格的资料,并考虑到其他一切因素,一个劳动者家庭每年所得的产品合42镑10先令,其中23镑交给别人,19镑10先令留下维持自己的生活。”(同上,第102—104页)}\end{quote}

人口对资本的依赖:

\begin{quote}{“应当看到,马尔萨斯先生和他的信徒的错误在于假定工人人口的减少不会引起资本的相应减少。”(同上,第118页)“马尔萨斯先生忘记了,[对工人的]需求受到用作工资的资金的限制,这些资金不是自然而然产生的,而总是事先由劳动创造的。”(同上,第122页)}\end{quote}

\todo{}

这是关于资本积累的正确见解。但是,[霍普金斯没有看到]在劳动量不以同一程度增加的情况下,劳动创造的资金是可以增加的,也就是说,剩余产品或剩余劳动的量是可以增加的。

\begin{quote}{“奇怪的是,有一种强烈的倾向把纯财富说成是对工人阶级有利的东西,因为据说它提供就业机会。其实,很明显,[509]即使它能够提供就业机会,也不是因为它是纯的,而是因为它是财富,是劳动创造的东西。而另一方面,又把追加的工人人口说成是对工人阶级有害的东西,虽然这些工人生产的东西比他们所消费的多两倍。”(同上,第126页)“如果由于使用较好的机器可以把全部首要产品从200增加到250或300,而纯财富和利润仍旧只有140,那末,显然,留作首要生产者的工资基金的就不是60,而是110或160。”(同上,第128页)“工人状况的恶化,或者是由于他们的生产力被摧残,或者是由于他们生产的东西被掠夺。”(第129页)“不,马尔萨斯先生说,‘你们的贫困决不能用你们的负担沉重来解释;你们的贫困完全是由有这种负担的人太多引起的’。”(同上,第134页)“原始材料[土地、水、矿物]是不包括在生产费用调节一切商品的交换价值这个一般原理之内的;但是这些原始材料的所有者要求[由于这些原始材料被利用而得到]产品的权利,使地租加入价值。”(托·霍普金斯《关于调节地租、利润、工资和货币价值的规律的经济研究》1822年伦敦版第11页)“地租,或者说,[土地]使用费,自然是从[土地]所有权产生的,或者说,是从财产权的确立产生的。”(同上,第13页)“凡具有以下特性的东西都能够提供地租:第一,它必须在某种程度上是稀少的;第二,它必须在生产这种大事业中有协助劳动的能力。”(同上,第14页)“当然,不应该假定有这样一种情况,即土地同投到土地上的劳动和资本相比{土地的多或少自然是相对的,是同可以使用的劳动和资本的量有关系的},是如此之多,以致不可能提供地租形式的费用,因为土地并不稀少。”(第21页)“在某些国家,土地所有者能榨取50%,在另一些国家连10%也榨取不到。在东方富饶地区,一个人有了他投入土地的劳动的产品的1/3就可以过活;但是在瑞士和挪威的某些地区,征收10%就可以使土地荒芜……除了交租人的有限的支付能力之外,”(第31页)以及“在有较坏土地存在的地方,除了较坏土地同较好土地的竞争之外,”(第33—34页)“我们看不见可以榨取的地租的任何其他自然界限”。(第31页)“在英国有许多荒地,它的自然肥力同相当大一部分现在的已耕地在未耕种前一样;可是,把这种荒地变成耕地所需的费用太大,以致它不能为花费在改良土地上的货币支付普通利息,更不能留下什么为土地的自然肥力支付地租,尽管这里有一切便利条件,可以在巧妙使用的资本帮助下,在便宜的工业品配合下,立即使用劳动,并且附近已经开辟了很好的道路等等。现在的土地所有者可以被看作几百年来为使土地达到现有生产率状态而耗费的全部积累劳动的所有者。”(同上,第35页)}\end{quote}

这一情况对于地租有非常重要的意义,特别是在人口突然大大增加(就象1780—1815年由于工业发展而发生的情况),因而大量过去没有耕种过的土地突然投入耕种的时候。新耕地的肥力可能等于,甚至高于老地尚未经过几百年耕种以前的肥力。但是,要使新耕地不是按较贵的价格出卖产品,就要求新地的肥力等于:第一,已耕[510]地的自然肥力,加上第二,已耕地由于耕种而形成的人工的、现已变成自然的肥力。因此,新耕地一定要比老地被耕种之前肥沃得多。

但是有人会说:

已耕地的肥力,首先取决于它的自然肥力。因而,新耕地是否具备这种由自然产生并受到自然因素制约的肥力,就取决于新耕地的自然性质。在两种情况下,自然肥力都是什么也不花费的。已耕地的肥力的另一部分则是依靠耕种、依靠投资的人工产物。但是,这部分生产率是花费了生产费用的;这笔生产费用是以投入土地的固定资本的利息的形式支付的。这部分地租只是同土地结合在一起的固定资本的利息。因此它加入早先的已耕地的产品的生产费用。由此可见,只要把同量资本投到新耕地上,它也将具有肥力的这个第二部分;如同早先的已耕地的情况一样,为创造这种肥力而使用的资本的利息将加入产品价格。在这种情况下,为什么新地——如果不是肥沃得多的土地——没有产品价格的提高就不能投入耕种呢?如果自然肥力是相同的,那末差别就只是由投资造成的,而在两种情况下,这笔资本的利息都是以同等程度加入生产费用的。

可是,这种推理是错误的。一部分开垦等费用不用再支付了,因为由此造成的肥力,正如李嘉图已经指出,部分地同土地的自然性质长在一起了(例如挖树根、改良土壤、排水、平整、通过多次反复的化学处理改变土壤化学成分等费用)。因此,新耕地为了能够与最后的已耕地按照同一价格出卖产品,它的肥力就必须足以使这个价格为它补偿那部分开垦费用,这部分费用加入它自己的生产费用,但已不再加入早先的已耕地的生产费用,而且在这里已经同土地的自然肥力长在一起了。

\begin{quote}{“位置有利的瀑布,给我们提供了一个为那种变成私有的、可以设想为最带独特性的自然赐予而支付地租的例子。这种情况在工业区域是大家都了解的,在那里,对小瀑布,特别是对落差大的瀑布支付相当高的地租。从这样的瀑布取得的力,等于大蒸汽机所供应的力;因此,利用这些瀑布即使要支付巨额地租,也同花费大笔资金去建设和运转蒸汽机一样合算。瀑布有大有小。离工业企业所在地近也是一个取得较高地租的有利条件。在约克郡和郎卡斯特郡,最小瀑布和最大瀑布的地租之间的差额,大概比用作普通耕地的50英亩最贫瘠土地和50英亩最肥沃土地的地租之间的差额还要大得多。”(霍普金斯,同上第37—38页)}\end{quote}

\tsectionnonum{[(8)开垦费用。谷物价格上涨时期和谷物价格下降时期(1641—1859年)]}

如果我们把前面引用的谷物年平均价格\authornote{见本册第144—146页。——编者注}拿来比较,并且,第一,把受货币贬值影响的情况(1809—1813年)除外,第二,把特别歉收的年份如1800和1801年造成的情况除外,我们就会看到,一定时候或一定时期耕种多少新地具有多么重要的意义。这里,已耕地上的价格上涨表明人口增长和由此引起的谷物价格超过[价值];另一方面,这种需求增长本身使新地投入耕种。如果新耕地数量相对地说增加很多,那末,价格上涨,价格比前一时期高,只不过证明有相当大一部分开垦费用加入了追加食物的价格。如果谷物价格不涨,那就是生产没有增加。这种增加的后果即价格下降要到以后才能表现出来,因为新近投资生产的食物的价格包含着生产费用或价格的这样一个要素,这个要素在早先的土地投资中,或者说在早先的那部分耕地上早已消失了。如果耕种新地的费用不是因劳动生产率提高而比过去时期已经大大降低,这个差额甚至还要大。

[511]不论新地的肥力高于、等于或低于老地,为了把新地改造成适于使用——象在已耕地上平均使用资本和劳动那样的条件下使用——资本和劳动的状态(这种状态由现有已耕地上通行的一般开垦标准来决定),必须支付变未耕地为耕地的费用。生产费用的这个差额必须由新耕地补偿。如果这个差额不加入新耕地的产品价格,那末,这种结果只有在两种情况下可能出现。或者,新耕地的产品不按它的实际价值出卖。它的价格低于它的价值,大部分不提供地租的土地实际上就是这种情况,因为这种土地的产品价格不是由自己的价值,而是由较肥沃土地的产品的价值决定的。或者,新耕地必须相当肥沃,以致它的产品如果按自己的内在的价值,按物化在产品中的劳动量出卖,其价格就会低于早先的已耕地产品的价格。

如果由已耕地的产品价值调节的市场价格和新耕地产品的内在价值之间的差额比方说是5%,另一方面,如果加入新耕地产品的生产费用的、为使新地提高到老地通常具有的生产能力水平而花费的资本的利息也等于5%,那末新耕地的产品按原来的市场价格将能支付通常的工资、利润和地租。如果支付所花费的资本的利息只需4%,而新地的肥力超过老地的肥力4%以上,那末市场价格在扣除了为使新地变成“适于耕种的”状态而花费的资本的利息4%之后,将有一个余额,或者说,产品可以卖得低于由最贫瘠土地的产品价值调节的市场价格。结果,一切地租将同产品的市场价格一起降低。

绝对地租是原产品价值超过平均价格的余额。级差地租是比较肥沃的土地上生产的产品的市场价格超过这种比较肥沃的土地自己产品的价值的余额。

因此,如果在一段时间内,人口增加所需要的追加食物有比较大的一部分是初耕地生产的,同时原产品的价格上涨或者不变,这还不能证明国内土地的肥力已经下降。这只是证明,土地的肥力还没有提高到足以补偿生产费用的新要素的程度,这个新要素就是为使未耕地提高到老地在当时发展阶段所具备的通常生产条件的水平而花费的资本的利息。

由此可见,如果在不同时期新耕地的相对量不同,即使价格不变或上涨,也不能证明新地贫瘠或提供的产品较少,而只是证明,有一个费用要素加入它的产品价值,这个费用要素在早先的已耕地上已经消失;只是证明,这个新的费用要素仍然在起作用,尽管在新的生产条件下,开垦费用与过去为了把老地从肥力的原始的自然状态改变成现在的状态所必需的费用比较起来,已经大大降低。因此,要[512]查明不同时期圈围[公有地和开垦公有地]的相对比例。\endnote{关于英国“圈围”公有地,马克思在《资本论》第一卷第二十四章中谈得比较详细(见《马克思恩格斯全集》中文版第23卷第793—797页)。——第156页。}

此外,从前面所引的表(第507—508页)可以看出:

如果把每十年作一个时期来考察,那末,1641—1649年时期比1860年前的任何一个十年时期都高,只有1800—1809年和1810—1819年除外。

如果把每五十年作一个时期来考察,那末,1650—1699年时期高于1700—1749年,而1750—1799年时期高于1700—1749年,但低于1800—1849年(或1800—1859年)。

在1810—1859年期间,发生了有规律的价格下降,而在1750—1799年期间,虽然这五十年平均价格较低,却是上升的运动,这是有规律地上涨,就象1810—1859年期间有规律地下降一样。

事实上,同1641—1649年时期比较起来,总的说来,十年的平均价格不断下降,这种下降直到十八世纪上半叶最后两个十年达到它的极限(最低点)为止。

从十八世纪中叶开始上涨,这次上涨以1750—1759年价格[36先令4+(5/10)便士]为出发点,这个价格低于十七世纪下半叶的平均价格,几乎相当(略高)于1700—1749年(十八世纪上半叶)的平均价格35先令9+(29/50)便士。这个上升运动在1800—1809年和1810—1819年两个十年时期一直在继续。在后一个十年达到它的最高点。从这时又开始了有规律的下降运动。如果我们把1750—1819年这个上涨时期平均一下,它的平均价格(每夸特57先令多)[几乎]等于1820年开始的下降时期的出发点(即1820—1829年这个十年时期的58先令多);正如十八世纪下半叶的出发点[几乎]等于十八世纪上半叶的平均价格一样。

歉收、货币贬值等个别情况对平均数字能够发生多么大的影响,可以任意举一个算术例子来说明。例如,30+20+5+5+5=65。尽管这里最后三个数都只是5,平均数却等于13。相反,12+11+10+9+8[=50],平均数等于10,尽管把第一式中例外的数30和20划去时第二式中任何三年的平均数都要大些。

如果把付给资本——陆续用于开垦、在一定时期作为特殊项目加入生产费用的资本——的差额费用除去,那末,1820—1859年的价格或许比过去所有的价格都低。应该认为,那些用投入土地的固定资本的利息来解释地租的糊涂人,也多少看到了这种情况。

\tsectionnonum{[(9)安德森反对马尔萨斯。安德森对地租的理解。安德森关于农业生产率提高和它对级差地租的影响的论点]}

安德森在《关于导致不列颠目前粮荒的情况的冷静考察》(1801年伦敦版)中说:

\begin{quote}{“从1700到1750年,小麦的价格不断下降,从每夸特2镑18先令1便士降到1镑12先令6便士;从1750到1800年,小麦的价格不断上涨,从每夸特1镑12先令6便士涨到5镑10先令。”(第11页)}\end{quote}

可见,安德森不象威斯特、马尔萨斯和李嘉图那样,只看到谷物价格不断上涨(从1750到1813年)一个方面的现象,相反,他看到两方面的现象:整个世纪,上半世纪谷物价格不断下降,下半世纪谷物价格不断上涨。同时安德森明确指出:

\begin{quote}{“人口不论在十八世纪上半叶和下半叶同样都在增长。”(同上,第12页)}\end{quote}

安德森是人口论\endnote{指马尔萨斯的人口论。——第158页。}的死敌,他非常明确地强调指出,土地有不断增长的持久的改良能力:

\begin{quote}{“通过化学作用和耕种,土地可以越来越得到改良。”(同上,第38页)\endnote{这里引用的是意大利人卡米洛·塔雷洛·德·列奥纳托(十六世纪)的话,安德森在这个问题上完全同意他的意见。——第158页。}[513]“在合理的经营制度下,土地的生产率可以无限期地逐年提高,最后一直达到我们现在还难于设想的程度。”(第35—36页)“可以有把握地说,现在的人口同这个岛能够供养的人口比较起来是很少的,远没有达到引起严重忧虑的程度。”(第37页)“凡人口增加的地方,国家的生产也必定一起增加,除非人们允许某种精神的影响破坏自然的经济。”(第41页)}\end{quote}

“人口论”是“最危险的偏见”。(第54页)安德森力求用历史的例子证明,“农业生产率”随着人口的增长而提高,随着人口的减少而下降。(第55、56、60、61页及以下各页)

如果对地租有正确的理解,自然首先会认识到,地租不是来自土地,而是来自农产品,也就是来自劳动,来自劳动产品(例如小麦)的价格,即来自农产品的价值,来自投入土地的劳动,而不是来自土地本身。关于这一点,安德森正确地着重指出:

\begin{quote}{“不是地租决定土地产品的价格,而是土地产品的价格决定地租,虽然土地产品的价格在地租最低的国家往往最高。”}\end{quote}

{因此,地租同农业的绝对生产率毫无关系。}

\begin{quote}{“这似乎是一个奇论,需要解释一下。每一个国家有各种土地,它们的肥力彼此大不相同。我们把这些土地分成不同的等级,用A、B、C、D、E、F等字母表示。等级A包括肥力最大的土地,以下字母表示不同等级的土地,它们的肥力依次递减。既然耕种最贫瘠的土地的费用同耕种最肥沃的土地的费用一样大或者甚至更大,那末,由此必然得出一个结论,如果等量谷物,不论它来自哪一个等级的土地,可以按照同一价格出卖,耕种最肥沃的土地的利润一定比耕种其他土地的利润大得多}\end{quote}

{指[产品]价格超过费用,或者说,超过预付资本价格的余额},

\begin{quote}{而且,由于肥力越低这种利润越少,最后必然达到这种情况,就是在某些等级低的土地上,耕种费用同全部产品价值相等。”(第45—48页)}\end{quote}

最后的土地不支付任何地租。(引文来自麦克库洛赫的《政治经济学文献》1845年伦敦版。麦克库洛赫又引自《谷物法本质的研究》或《关于农业、自然史、技艺及其他各种问题的通俗讲座》1799—1802年伦敦版。这要到英国博物馆去核对。\endnote{马克思所引的安德森的这段话,是麦克库洛赫从安德森的《谷物法本质的研究》(1777年爱丁堡版)一书中引用的一大段中的一部分,这整个一大段话在该书第45—48页。——第159页。})

安德森在这里叫做“全部产品价值”的东西,在他的观念中显然就是市场价格,不论较好或较坏土地出产的产品都要按照它出卖。这个“价格”(价值)使比较肥沃的各个等级的土地有一个或大或小的超过费用的余额。最后的产品没有这种余额。对于这种产品,平均价格,即由生产费用加平均利润决定的价格,同产品的市场价格一致,因此这里没有任何超额利润,按照安德森的见解,只有超额利润能够形成地租。在安德森看来,地租等于产品的市场价格超过产品的平均价格的余额。(价值理论还完全没有引起安德森不安。)因此,如果由于土地特别贫瘠,这种土地的产品的平均价格同产品的市场价格一致,那末,这个余额就没有了,就是说,形成地租的基金就不存在了。安德森不说最后的耕地不可能提供任何地租。他只说,当费用(生产费用加平均利润)大到使产品市场价格和产品平均价格之间的差额消失时,地租也就消失,并且说,如果耕种的土地的等级越来越低,这种情况必然发生。安德森明确地说,在不同程度的有利生产条件下生产出来的等量产品具有一定的、同一的市场价格,是形成地租的前提。他说,“如果等量谷物,不论它来自哪一个等级的土地,可以按照同一价格出卖”,因此,如果假定有一个一般市场价格的话,那末,在较好等级土地上必然有超过较坏等级土地的超额利润,或者说,利润余额。

[514]从前面引的一段话可以看出,安德森决不认为不同的肥沃程度仅仅是自然的产物。相反,他认为:

\begin{quote}{“土地的无限多样性”部分地是因为这些“土地由于它们所经历的耕作方式不同,由于肥料等等,可以从它们的原始状态改变成完全不同的状态”。(《关于至今阻碍欧洲农业进步的原因的研究》1779年爱丁堡版第5页)}\end{quote}

一方面,社会劳动生产率的发展使开垦新地比较容易;但是,另一方面,耕种又使土地之间的差别扩大,因为已耕的A地和未耕的B地的原有肥力完全可能是一样的,如果我们从A地的肥力中扣除对这块土地来说现在固然是自然的、但从前是人工赋予的那一部分的话。因此,耕种本身使已耕地和未耕地的自然肥力之间的差别扩大。

安德森明确地说,一块土地,如果它的产品的平均价格同市场价格一致,就不能支付任何地租:

\begin{quote}{“有两块土地,它们的产量同上面说过的例子大致相符,就是说,一块收12蒲式耳,正够补偿费用,另一块收20蒲式耳;如果这两块土地都不需要立刻支出任何改良土壤的费用,那末租地农场主可以为后一块土地,比方说,支付甚至6蒲式耳以上的地租,而不能为前一块土地支付任何地租。如果12蒲式耳刚够补偿耕种费用,那末仅仅生产12蒲式耳的已耕地就不能提供任何地租。”(《论农业和农村事务》1775—1796年爱丁堡和伦敦版第3卷第107—109页)}\end{quote}

紧接着他又说:

\begin{quote}{“可是,如果租地农场主直接靠他花费的资本和他的努力得到较大量的产品,那就不能指望他能够把产品中几乎同样大小的份额当作地租来支付;但是,如果土地肥力在一定时间内稳定在同样高的水平上,尽管这块土地本来是靠他自己的努力才提高了肥力,他将乐于支付上述数量的地租。”(同上,第109—110页)}\end{quote}

这样,举例来说,最好耕地的产品每英亩为20蒲式耳;依照假定,其中12蒲式耳支付费用(预付资本加平均利润)。在这种情况下,可以有8蒲式耳支付地租。假定,一蒲式耳值5先令,那末,8蒲式耳或1夸特值40先令或2镑,20蒲式耳[2+(1/2)夸特]值5镑。这5镑中,扣掉作为费用的12蒲式耳或60先令即3镑。那末就剩下2镑或8蒲式耳支付地租。在3镑费用中,如果利润率等于10%,那末支出等于54+(6/11)先令,利润等于5+(5/11)先令{[54+(6/11)]∶[5+(5/11)]=100∶10}。现在假定,租地农场主必须在一块肥力同生产20蒲式耳的土地的原有肥力一样的未耕地上进行各种改良,以便使它达到相当于农业耕作一般水平的状态。这使他除了54+(6/11)先令的支出外,或者说,我们把利润也包括在费用内,除了60先令外,还要支出36+(4/11)先令;这笔支出的10%等于3+(7/11)先令;如果租地农场主总是按照每蒲式耳5先令的价格出卖20蒲式耳,那就只有经过10年,只有在他的资本再生产出来之后,他才能够支付地租。从那个时候起,人工的土地肥力就被算作原有肥力,它的利益将落到地主手里。

虽然新耕地的肥力和最好的已耕地的原有肥力相同,可是,对新耕地的产品来说,市场价格和平均价格现在是一致的,因为在平均价格中加入了一项费用,这项费用,在人工的和自然的肥力在一定程度上溶合在一起的最好土地上,已经消失了。而在新耕地上,人工的、由投资造成的那部分肥力,还与土地的自然肥力完全不同。因此,新耕地虽然具有与最好的已耕地同样的原有肥力,却不能支付任何地租。可是十年以后,它不仅能够一般地支付地租,而且能够与早已耕种的最好的土地支付同样多的地租。

可见,安德森在这里看到了两种现象:

(1)地主得到的级差地租,一部分是租地农场主人工地赋予土地肥力的结果;

(2)这种人工肥力经过一定期间开始表现为土地本身的原有生产率,因为土地本身已被改造了,而实现这种改造的过程却消失了,看不出来了。

[515]如果我今天建立一个棉纺厂,花费10万镑,那末,我的棉纺厂的生产率,比十年前我的前辈所建立的棉纺厂要高。对于今天的机器制造业、一般建筑业等等的生产率和十年前的生产率之间的差别,我是不付代价的。相反,这种差别却使我能支付较少代价而得到具有同样生产率的工厂,或者仅仅支付同一代价就得到生产率较高的工厂。农业中情况不是这样。土地的原有肥力之间的差别由于增加一部分所谓土地自然肥力而扩大了,这部分肥力,事实上是以前由人们创造的,现在却同土地本身有机地溶合在一起,同土地的原有肥力已不能再区别开。由于社会劳动生产力的发展,为了使具有同样自然肥力的未耕地达到这种已增大的肥力水平所必需的费用,已不象把已耕地的原有肥力提到它现在看来是原有的肥力所需要的那样多了;但是为了达到这同一水平,现在还是需要或多或少的费用。因此,新产品的平均价格高于老产品的平均价格,而市场价格和平均价格之间的差额会缩小,甚至可能完全消失。但是,假定在上述情况下新耕地很肥沃,在花上40先令追加费用(包括利润)之后,它不是提供20蒲式耳,而是提供28蒲式耳。在这种情况下,租地农场主就可以支付8蒲式耳或2镑的地租。为什么呢?因为新耕地比老地多提供8蒲式耳,所以,尽管平均价格较高,新耕地在同一市场价格下仍然提供同老地一样多的价格余额。如果新耕地不需要任何追加费用的话,新耕地的肥力就会两倍于老地。\endnote{马克思这里所说的“土地肥力”,是指从这块土地上得到的地租总额。——第163页。}就因为有了这种费用,才可以说新耕地的肥力同老地一样高。

\tsectionnonum{[(10)洛贝尔图斯对李嘉图地租理论的批判不能成立。洛贝尔图斯不懂资本主义农业的特点]}

现在最后回过头来谈也是最后一次谈洛贝尔图斯。

\begin{quote}{“它〈洛贝尔图斯的地租理论〉从劳动产品的分配出发来说明……工资、地租等一切现象,而这种分配,只要具备两个先决条件即足够的劳动生产率以及土地和资本的所有权,就必然要出现。它说明,只有足够的劳动生产率才造成这种分配的经济上的可能性,因为这种生产率赋予产品价值这么多实际内容,以致其他不劳动的人也可以靠它生活;它也说明,只有土地和资本的所有权才造成这种分配的法律上的现实性,强迫工人把自己的产品同不劳动的土地所有者和资本所有者按照这样一种比例分配,以致他们工人从中分得的刚够活命。”(洛贝尔图斯《给冯·基尔希曼的社会问题书简。第三封信:驳李嘉图的地租学说,对新的地租理论的论证》1851年柏林版第156—157页)}\end{quote}

亚·斯密对问题的解释是双重的。[第一个解释:]劳动产品的分配,这里把劳动产品看成既定的,并且实际上说的是使用价值的份额。洛贝尔图斯先生也是这个看法。在李嘉图著作中也可遇到这个看法,而且李嘉图更应该因此受到责备,因为他对价值决定于劳动时间这个规定不限于泛泛议论,而是认真对待。这个看法,加以相应的修改之后,或多或少适用于使劳动者和劳动客观条件所有者成为不同阶级的一切生产方式。

相反,斯密的第二个解释表现了资本主义生产方式的特征。因此,只有它才是理论上有成果的公式。那就是,斯密在这里认为利润和地租来源于剩余劳动,来源于工人除了用来仅仅再生产他自己工资的那部分劳动以外加到劳动对象上的剩余劳动。在生产完全以交换价值为基础的地方,这是唯一正确的观点。这个观点奠定了理解发展过程的基础,而在第一个解释里,劳动时间被假定为不变的。

李嘉图所以有片面性,是因为他总想证明不同的经济范畴或关系同价值理论并不矛盾,而不是相反地从这个基础出发,去阐明这些范畴以及它们的表面上的矛盾,换句话说,去揭示这个基础本身的发展。

\begin{quote}{[516]“您\authornote{指冯·基尔希曼。——编者注}知道,所有经济学家从亚·斯密那时候起已经把产品的价值分解为工资、地租和资本盈利,因而,把各阶级的收入,特别是地租部分,建立在产品的分配上这种观念,不是新的〈的确不是!〉。可是经济学家们立刻走入歧途。所有的人,连李嘉图学派也不例外,首先犯了这样一个错误,他们不是把全部产品、完成的财富、全部国民产品看作工人、土地所有者和资本家参与分配的一个整体,而是把原产品的分配看作一种有三者参与的特殊分配,把工业品的分配看作又一种只有两者参与的特殊分配。这样,这些体系已经把原产品本身和工业品本身分别当作一种构成收入的特殊财富看待了。”(第162页)}\end{quote}

首先,亚·斯密把“产品的全部价值分解为工资、地租和资本盈利”,从而忘掉了也构成价值一部分的不变资本;这样,他的确把后来的所有经济学家,包括李嘉图,也包括洛贝尔图斯先生在内,都引入了“歧途”。没有[劳动总产品和新加劳动产品之间的]这种区分,要对问题做出任何科学的解释简直是不可能的,这一点在我对这个问题的分析中已经证明了\authornote{见本卷第1册第78—140页。——编者注}。重农学派在这方面更接近于真理。他们的“原预付和年预付”是作为年产品价值或年产品本身的一部分区分出来的,这个部分无论对国家或个人都不再分解为工资、利润或地租了。重农学派认为,农业主用原料补偿不生产阶级的预付(把这种原料变成机器的事落到“不生产”阶级身上),另一方面,农业主用自己的产品补偿自己的一部分预付(种子、种畜、役畜、肥料等),而另一部分预付(机器等)则通过用原料同“不生产”阶级交换得到补偿。

第二,洛贝尔图斯先生的错误是把价值的分配和产品的分配等同起来。“构成收入的财富”同这种产品价值的分配没有任何直接关系。比方说,棉纱生产者得到的并表现为一定金量的价值部分在各种产品——农产品或工业品——中得到实现,对于这一点,经济学家们同洛贝尔图斯一样,知道得很清楚。这一点是事先假定的,因为这些生产者是生产商品,而不是生产供自己直接消费的产品。既然供分配的价值,即一般说来归结为收入的那个价值组成部分,是在各个生产领域内部,在不依赖其他领域的情况下(虽然由于分工每个生产领域都以其他领域为前提)创造出来的,所以,洛贝尔图斯不去考察这个价值创造的纯粹形式,而一开头就把事情搞乱,提出这些价值组成部分能保证自己的所有者取得一国现有总产品的什么份额的问题,那他就倒退了一步并且造成了混乱。在他那里,产品价值的分配立即变成使用价值的分配。既然他把这种混乱转嫁给其他经济学家,所以他提出的矫正方法,即把工业品和原产品放在一起考察的方法,就成为必要了,而这种考察方法同价值的创造无关,因此,如果用它来说明价值的创造,是错误的。

工业品的价值,只要它归结为收入,只要工厂主不付地租(不论为建筑物的地皮或者为瀑布等),就只有资本家和雇佣工人参加分配。农产品的价值在大多数情况下有三方面参加分配。这是洛贝尔图斯先生也承认的。他对这种现象所作的解释丝毫不能改变事实本身。但是,如果其他经济学家,特别是李嘉图,从资本家和雇佣工人两者分配出发,到后来才把地租所得者作为一种特殊赘疣引进来,那末,这是完全符合资本主义生产的实质的。物化劳动和活劳动,这是两个[517]因素,资本主义生产正是建立在这两个因素的对立之上。资本家和雇佣工人是生产职能的唯一承担者和当事人,他们之间的相互关系和对立是从资本主义生产方式的本质产生的。

资本家不得不把他所侵占的一部分剩余劳动或剩余价值再同不劳动的第三者分配的情况,只是后来才出现。扣除作为工资支付出去的产品价值部分和等于不变资本的价值部分之后,全部剩余价值直接从工人手里转到资本家手里,这也是生产的事实。对于工人来说,资本家是全部剩余价值的直接占有者,不管他后来怎样同借贷资本家、土地所有者等分配剩余价值。因此,正如詹姆斯·穆勒指出的那样\endnote{詹·穆勒《政治经济学原理》1821年伦敦版第198页。——第166页。},如果地租所得者消失,由国家来代替他的地位,生产可以继续进行而不受影响。他——土地私有者——决不是资本主义生产方式所必要的生产当事人,虽然对于资本主义生产方式来说,必须使土地所有权属于什么人,只要不是属于工人,而是例如属于国家。根据资本主义生产方式——不同于封建、古代等生产方式——的本质,把直接参与生产,因而也是直接参与分配所生产的价值以及这个价值所借以实现的产品的阶级,归结为资本家和雇佣工人,而把土地所有者排除在外(由于那种不是从资本主义生产方式生长出来,而是被这种生产方式继承下来的对自然力的所有权关系,土地所有者只是事后才参加进来),这丝毫不是李嘉图等人的错误,它倒是资本主义生产方式的恰当的理论表现,表现了这种生产方式的特点。洛贝尔图斯先生还是一个十足的老普鲁士式的“地主”,理解不了这一点。只有当资本家控制了农业,并且到处象英国大多数地方那样,成为农业的领导者(完全同成为工业的领导者一样),排斥土地所有者以任何形式直接参加生产过程的时候,上述情况才变得可以理解和引人注意。因此,洛贝尔图斯先生在这里认为是“歧途”的,恰好是他所不理解的正道;整个问题在于,洛贝尔图斯还陷在资本主义前的生产方式所产生的种种观点之中。

\begin{quote}{“他〈李嘉图〉也不是让成品在有关参加者之间分配,而是象其他经济学家一样,把农产品和工业品分别当作一种特殊的有待分配的产品。”(同上,第167页)}\end{quote}

洛贝尔图斯先生!李嘉图在这里考察的不是产品,而是产品的价值,这是完全正确的。您的“成”品及其分配同这种价值分配毫无共同之处。

\begin{quote}{“在他〈李嘉图〉看来,资本所有权是既定的,并且还早于土地所有权……因此,他不是从产品分配的根据,而是从产品分配的事实开始,而他的全部理论限于研究那些决定和改变产品分配关系的原因……产品只分为工资和资本盈利,在他看来,是最初的分配,而且是最初唯一的分配。”(第167页)}\end{quote}

这个您又不懂了,洛贝尔图斯先生!从资本主义生产的观点看来,资本所有权的确是作为“最初的”所有权出现的,因为它是一种作为资本主义生产的基础,并在这种生产制度中表现为生产的当事人和生产职能的承担者(对土地所有权就不能这样说)的所有权。土地所有权在这里表现为派生的东西,因为,现代土地所有权,实际上是封建的,但是由于资本对它的作用,发生了形态变化,因而它作为现代土地所有权所特有的形式是派生的,是资本主义生产方式的结果。李嘉图把现代社会中存在和表现出来的这个事实也看成历史上最初的东西(而您呢,洛贝尔图斯先生,不是去研究现代形式,而是摆脱不了地主的回忆),这是一种误解,资产阶级经济学家们在考察资产阶级社会的一切经济规律时都陷入这种误解,在他们看来,这些规律是“自然规律”,因而也表现为历史上最初的东西。

[518]但是,李嘉图在谈到不是产品的价值而是产品本身的地方,是指全部“成”品的分配,洛贝尔图斯先生从李嘉图序言的头一句话就可以看到这一点:

\begin{quote}{“土地产品——通过劳动、机器和资本联合运用而从地面上得到的一切产品——在社会的三个阶级之间,也就是在土地所有者、耕种土地所必需的基金或资本的所有者和以自己的劳动耕种土地的工人之间进行分配。”(《政治经济学和赋税原理》,序言,1821年伦敦第3版)}\end{quote}

李嘉图紧接着说:

\begin{quote}{“但在不同的社会发展阶段,这些阶级中的每一个阶级在地租、利润和工资的名义下分到的全部土地产品的份额是极不相同的”。(同上)}\end{quote}

这里说的是“全部产品”的分配,而不是工业品或原产品的分配。如果“全部产品”是既定的,“全部产品”的这些份额就完全决定于每一生产领域内部每个参与分配者在自己产品的“价值”中拥有的份额。这个“价值”可以转化并表现为“全部产品”的一定的相应份额。李嘉图在这里的错误,只是他步亚·斯密的后尘,忘记了不是“全部产品”分解为地租、利润和工资,因为“全部产品”中有一部分作为资本“分给”这三个阶级中的一个或几个阶级。

\begin{quote}{“可能您想断言,最初资本盈利均等规律必定压低原产品价格,直到地租消失,后来由于价格上涨,地租又从比较肥沃和比较不肥沃的土地的收成的差额中产生出来,同样,现在,除普通的资本盈利之外还取得地租这样一种好处,也必定推动资本家把资本用于开垦新地和改良老地,直到由此引起的市场商品充斥使价格又降低,以致在最不利的投资情况下地租消失。换句话说,这等于断言:就原产品说,资本盈利均等这条规律,把另一条规律,即产品价值决定于所耗费的劳动的规律取消了,可是,李嘉图在他的著作的第一章中恰恰是用前一条规律来证明后一条规律的。”(洛贝尔图斯,同上,第174页)}\end{quote}

当然,洛贝尔图斯先生!“资本盈利均等”规律并不取消产品“价值”决定于“所耗费的劳动”的规律;但是,它的确取消李嘉图关于产品的平均价格等于产品“价值”这个前提。然而问题也不是“原产品”的价值降低到平均价格。正好相反,原产品的特点是,由于土地所有权的存在,它有这样一个特权,即它的价值不降低到平均价格。如果它的价值真的降到商品的平均价格的水平(这是可能的,尽管其中存在着您所说的“材料价值”),地租也就消失了。有一些等级的土地,现在也许不提供任何地租,它们之所以如此,是因为原产品的市场价格等于它们的产品本身的平均价格,使它们因比较肥沃土地的竞争而失去按自己产品的“价值”出卖产品的特权。

\begin{quote}{“难道在人们还根本没有从事农业之前,就已经有获得盈利,并且按照盈利均等规律投放资本的资本家存在了吗?〈多么愚蠢!〉……我认为,如果现在从文明国家[519]派遣一个开发队到一个新的未开垦的国家去,较富的成员带着已经发达的农业的储备和工具——即资本,较贫的成员也跟着一块去,希望通过为较富的成员服务得到高的工资,在这种情况下,资本家将把支付工人工资之后剩下的余额看成自己的盈利,因为他们从宗主国带来了早已存在的事物和概念。”(第174—175页)}\end{quote}

这下子您说对了,洛贝尔图斯先生!李嘉图的全部观点只有在资本主义生产方式占支配地位的前提下才有意义。至于他用什么形式表达这个前提,他在这上面是否采用了hysteronproteron〔颠倒历史顺序的逆序法〕,这同问题的实质无关。必须有这个前提,因此,不能象您做的那样,把那种不懂资本主义簿记的、因而不把种子等等算进预付资本的农民经济引进来!“荒谬”的不是李嘉图,而是洛贝尔图斯,他把资本家和工人存在于“土地耕种之前”这一观点强加于李嘉图了。(第176页)

\begin{quote}{“照李嘉图的观点,只有在……社会中产生了资本,知道有资本盈利并支付这种盈利的时候,土地的耕种才开始。”(第178页)}\end{quote}

真是胡扯!只有当资本家以租地农场主的身分插在土地耕种者和土地所有者之间的时候(不论是以前的臣仆靠欺骗手法成了资本主义租地农场主,还是工业家把他的资本不投于工业而投于农业),才开始有——当然不是一般的“土地耕种”,而是——“资本主义的”土地耕种,它在形式上和内容上都同以前的耕种方式大不相同。

\begin{quote}{“在每一个国家,土地的大部分变成私有财产比土地被耕种早得多,无论如何比工业中形成资本盈利率早得多。”(第179页)}\end{quote}

洛贝尔图斯要在这个问题上懂得李嘉图的观点,他就必须是一个英国人,而不是一个波美拉尼亚的地主,而且必须懂得圈围公有地和荒地的历史。洛贝尔图斯先生举出美国作例子。这里,国家把土地

\begin{quote}{“一小块一小块地卖给移民,的确,价格很便宜,可是这个价格无论如何一定已经代表着地租”。(第179—180页)}\end{quote}

绝对不是。这个价格并不构成地租,正象一般的营业税不能构成营业租,或者一般的任何税不能构成任何“租”一样。

\begin{quote}{“我断定,b点所说的地租提高的原因{由于人口增加或使用的劳动量增加}构成地租对资本盈利的优越性。这个原因任何时候也不能提高资本盈利。的确,在生产率不变但生产力增加(人口增加)的条件下,由于国民总产品价值的增加,国民得到更多的资本盈利,但是这种增加了的资本盈利总是摊到按同一比例增加了的资本身上,所以盈利率还是同过去一样高。”(第184—185页)}\end{quote}

这是错误的。如果剩余劳动时间例如不是2小时,而是3、4、5小时,无酬剩余劳动量就会增加。预付资本量并不随着这个无酬剩余劳动量[按同一比例]增加,第一,因为对这个剩余劳动的新的余额是不付报酬的,也就是说不引起任何资本支出;第二,因为用于固定资本的支出不是同这里的固定资本的使用以同一比例增加的。纱锭等的数量并不增加。当然,纱锭磨损加速,但不是与它们的使用增加成同一比例。由此可见,在生产率不变的条件下,这里利润是增加了,因为不仅剩余价值增加,剩余价值率也增加了。在农业中,由于自然条件的关系,这是办不到的。另一方面,生产率随着投资增加而迅速发生变化。撇开分工和机器不说,虽然支出的资本绝对地说很大,但是由于生产条件的节约,相对地说就不是那么大了。因此,即使剩余价值(不仅剩余价值率)不变,利润率也可能提高。

[520]洛贝尔图斯的下述论点是完全错误的,是带有波美拉尼亚地主气味的:

\begin{quote}{“可能,在这三十年的过程中〈从1800到1830年〉通过地产析分,甚至开垦新地,出现了许多新的土地占有,因而增加了的地租也在更多的所有者之间分配,但是地租在1830年并没有比1800年分摊在更多的摩尔根上;那些新划分或新耕种的地段的全部面积原先就包括在早已存在的地段中了,因此,1800年的较低的地租,象1830年的较高的地租一样,也是由这些地段分摊,也影响英国整个地租的高度。”(第186页)}\end{quote}

亲爱的波美拉尼亚人呀!为什么老是自以为是地把您的普鲁士关系搬到英国去呢?英国人完全不认为,如果从1800年到1830年“圈围”了300万到400万英亩\endnote{马克思在《资本论》第一卷第二十四章中写道:1801年到1831年英国农村居民被夺去3511770英亩公有地,并“由地主通过议会赠送给地主”(见《马克思恩格斯全集》中文版第23卷第796页)。——第172页。}——这是事实(要核实一下),——那末在1830年以前,在1800年,地租也是分摊在这400万英亩上。相反,当时这400万英亩都是荒地或公有地,是不提供任何地租、也不属于任何人的。

如果洛贝尔图斯同凯里一样(不过方式不同)想向李嘉图证明,由于自然原因和其他原因,“最肥沃的”土地大部分并非首先被耕种,那末,这同李嘉图是毫无关系的。所谓“最肥沃的”土地,每一次都是指一定生产条件下的“最肥沃的”土地。

洛贝尔图斯对李嘉图的反驳有很大一部分是由于他把“波美拉尼亚”生产关系和“英国”生产关系天真地等同起来。李嘉图是以资本主义生产为前提的,在象英国这样发达的资本主义生产中,资本主义租地农场主和土地所有者是分离的。洛贝尔图斯引用的却是本身与资本主义生产方式无关的关系,资本主义生产方式只是加筑在这些关系之上。例如,洛贝尔图斯先生关于经济中心在经济复合体中的地位所说的话,完全适用于波美拉尼亚,却不适用于英国,在英国,资本主义生产方式自从十六世纪末叶以来越来越占优势,它把一切条件同化,在各个不同时期把历史造成的各种前提——村落、建筑物和人——一个一个地彻底铲除,以保证“最有效的”投资。

洛贝尔图斯关于“投资”所说的话同样是错误的。

\begin{quote}{“李嘉图把地租限于为使用土地原有的、自然的和不可摧毁的力而支付给土地所有者的数额。从而,他想把已耕地上应归于资本的一切从地租中扣除。但是,很明白,李嘉图从一块土地上的收入中划归资本的决不能多于十足的国内普通利息。因为不然的话,他就得假定在一国的经济发展中有两种不同的盈利率,一种是农业的盈利率,它的盈利大于普通的工业盈利;一种是工业的盈利率。可是这个假定就会推翻他的正是以盈利率的均等为基础的整个体系。”(第215—216页)}\end{quote}

这又是波美拉尼亚地主的观念,这种地主贷进资本,是为了使自己的地产更加有利可图,因此,他出于理论的和实际的考虑,想向贷出资本的人仅仅支付“国内普通利息”。可是在英国事情却不是这样。那里用于改良土地的资本,是租地农场主即资本主义农场主支出的。他对这种资本所要求的,完全同对他直接投入生产的资本所要求的一样,不是国内普通利息,而是国内普通利润。他不会把资本贷给地主,让地主对这种资本支付“国内普通”利息。他可能自己贷进资本,或者使用他自己的追加资本,以便这种资本给他提供“国内普通”工业利润,这种工业利润至少是国内普通利息的两倍。

此外,安德森已经知道的,李嘉图也知道。而且,李嘉图还明确地说过,[521]这样由资本造成的土地生产力,后来同土地的“自然”生产力溶合在一起,从而提高了地租。洛贝尔图斯对这一点毫无所知,因而胡说八道。

我已经完全正确地说明过现代的土地所有权:

\begin{quote}{“李嘉图所说的地租就是资产阶级状态的土地所有权,也就是从属于资产阶级生产条件的封建所有权。”(《哲学的贫困》1847年巴黎版第156页)\endnote{见《马克思恩格斯全集》中文版第4卷第183页。——第174页。}}\end{quote}

在那里我已经正确地指出:

\begin{quote}{“尽管李嘉图已经假定资产阶级的生产是地租存在的必要条件,但是他仍然把他的地租概念用于一切时代和一切国家的土地所有权。这就是把资产阶级的生产关系当作永恒范畴的一切经济学家的通病。”(同上,第160页)\endnote{同上,第186页。——第174页。}}\end{quote}

我同样正确地指出,正如所有其他资本一样,“土地资本”是可以增多的:

\begin{quote}{“正如所有其他生产工具一样,土地资本是可以增多的。我们不能在它的物质成分上(用蒲鲁东先生的说法)添加任何东西,但是我们可以增加作为生产工具的土地。人们只要对已经变成生产资料的土地进行新的投资,也就是在不增加土地的物质即土地面积的情况下增加土地资本。”(同上,第165页)\endnote{同上,第189页。——第174页。}}\end{quote}

我那时着重指出的工业和农业之间的差别仍然正确:

\begin{quote}{“首先,这里不能象工业生产中那样随意增加效率相同的生产工具的数量,即肥沃程度相同的土地数量。其次,由于人口逐渐增加,人们就开始经营劣等地,或者在原有土地上进行新的投资,这新的投资的生产率比最初投资的生产率就相应地降低。”(同上,第157页)\endnote{同上,第183页。——第174页。}}\end{quote}

洛贝尔图斯说:

\begin{quote}{“但是,我还要注意到使农业机器\endnote{洛贝尔图斯这里说的“农业机器”,指肥力不同的各级土地。洛贝尔图斯把土地同效率不等的机器相比,是从马尔萨斯那里借用来的。——第174页。}从坏变好的另一种情况,这种情况的发生固然缓慢得多,但是要普遍得多。这就是对一块土地不断耕种,只要依照合理的制度,即使没有一点额外投资,这种耕种本身也能改良土地。”(《给冯·基尔希曼的社会问题书简。第三封信》第222页)}\end{quote}

这一点安德森已经说过了。耕种会改良土地。

[洛贝尔图斯接着说:]

\begin{quote}{“您应当证明,从事农业的劳动人口,随着时间的推移,同食物的生产比较起来,或者至少是同一国人口的其余部分比较起来,是以更大的比例增长着。只有根据这一点才能得出驳不倒的结论:随着农业生产的扩大,必须把越来越多的劳动用在农业上。但是统计恰恰在这一点上同您矛盾。”(第274页)“是呵,您甚至可以确信,到处占统治地位的是这样一条规则:一国的人口越密,从事农业的人的比例越小……这种现象在同一个国家的人口增长中也表现出来:不从事农业的那部分人口几乎到处都以较大的比例增长。”(第275页)}\end{quote}

但是,这种情况一部分是因为有更多的耕地变成放牧牛羊的牧场,一部分是因为在较大规模的生产中——大农业中——劳动的生产率提高了。但是也因为——洛贝尔图斯先生完全没有注意到这个情况——非农业人口中有相当大的一部分人从事为农业服务的劳动,他们提供不变资本——这种不变资本随农业技术进步而不断增长——如矿肥、外国种子、各种机器。

照洛贝尔图斯先生的说法,

\begin{quote}{“今天〈在波美拉尼亚〉农业主不把自己农场生产的耕畜饲料看成资本”。(第78页)[522]“资本就其本身来说,或者从国民经济的意义上说,是进一步用于生产的产品……但是,就它所提供的特殊盈利来说,或者,就现在的企业主对资本所理解的意义来说,它要成为资本,就必须表现为‘支出’。”(第77页)}\end{quote}

不过,“支出”这个概念并不象洛贝尔图斯所认为的那样,要求把产品作为商品买进来。如果某一部分产品不是作为商品卖出,而是再加入生产,那末这部分产品就是作为商品加入生产。这部分产品一开始就是作为“货币”来估价的,而这一点由于所有这些“支出”——其中包括农业中的牲畜、饲料、肥料、谷种、各类种子——同时也都作为“商品”出现在市场上,就看得更加清楚了。但是,看来在“波美拉尼亚”,人们是不把所有这些算到“支出”项下的。

\begin{quote}{“这些不同劳动〈在工业和原产品生产中〉的特殊成果的价值,还不是它们的所有者的收入本身,而只是计算这种收入的尺度。这种各自得到的收入本身,都是社会收入的一部分,社会收入只有农业和工业的共同劳动才能创造出来,因此,它的各部分也只有这种共同的劳动才能创造出来。”(第36页)}\end{quote}

这有什么相干呢?这个价值的实现只能是它在使用价值中的实现。但是所谈的完全不是这一点。而且,必要工资这个概念已经包含着:有多少价值表现为维持工人生活的必要生活资料(农产品和工业品)。

到此结束。

\tchapternonum{[第十章]李嘉图和亚当·斯密的费用价格理论(批驳部分)}

\tsectionnonum{[A.李嘉图的费用价格理论]}

\tsubsectionnonum{[(1)重农学派理论的破产和地租观点的进一步发展]}

安\endnote{在《剩余价值理论》第二册中,“费用价格”(《Kostenpreis》或《Kostpreis》,《costprice》)这一术语,马克思用在“生产价格”即“平均价格”(c+v+平均利润)的意义上。关于“平均价格”这一术语见注7。在马克思的著作中首次见到《Kostenpreis》这一术语是在本卷第一册第77页,不过在那里它是用在商品“内在的生产费用”(c+v+m)的意义上,商品“内在的生产费用”是和商品的价值一致的。在《剩余价值理论》第三册中,《Kostenpreis》这一术语马克思有时用在生产价格的意义上,有时用在资本家的生产费用的意义上,也就是指c+v。《Kostenpreis》这一术语所以有三种用法,是由于《Kosten》(“费用”,“生产费用”)这个词在经济科学中被用在三种不同的意义上,正如马克思在《剩余价值理论》第三册(1861—1863年手稿第788—790页和第928页)特别指出的,这三种意义是:(1)资本家预付的东西,(2)预付资本的价格加平均利润,(3)商品本身的实在的(或内在的)生产费用。除了资产阶级政治经济学古典作家使用的这三种意义以外,“生产费用”这一术语还有第四种庸俗的意义,即让·巴·萨伊给“生产费用”下的定义:“生产费用是为劳动、资本和土地的生产性服务支付的东西。”(让·巴·萨伊《论政治经济学》1814年巴黎第2版第2卷第453页)马克思坚决否定了对“生产费用”的这种庸俗的理解(例如见本册第142、239和535—536页)。——第177页。}德森关于“不是地租决定土地产品的价格,而是土地产品的价格决定地租”\authornote{见本册第158页。——编者注}的论点(在亚·斯密那里部分地也有这种论点),完全推翻了重农学派的学说。这样,地租的源泉就是农产品的价格,而不是农产品本身,也不是土地。因此,认为地租是农业的特殊生产率的产物,而这种生产率又是土地特殊肥力的产物的观点也就站不住脚了。因为,如果同量劳动用在特别肥沃的要素中,因而劳动本身的生产率也特别高,那末,结果只能是,这种劳动表现为较大的产品量,因而单位产品的价格较低,而决不会相反,即这种劳动的产品的价格高于物化了同量劳动的其他产品的价格,因而它的价格和其他商品不同,除了利润和工资以外,还能提供地租。(亚·斯密在考察地租的时候,起先用他原来的关于地租是剩余劳动的一部分的观点,反驳了,或者至少是否定了重农学派的观点,后来,部分地又回到重农学派的观点上去。)

布坎南用下面的话概述了重农学派观点被摈弃的情况:

\begin{quote}{“有人认为农业提供产品并从而提供地租,是因为自然在耕种土地的过程中和人类劳动一起发挥作用,这种观点纯粹是幻想。地租不是来源于产品,而是来源于产品出卖的价格;而这个价格的获得,不是因为自然协助了生产,而是因为这个价格能使消费适应于供给。”\endnote{引自布坎南在他出版的亚·斯密《国富论》中加的一个脚注(亚当·斯密《国民财富的性质和原因的研究》,附大卫·布坎南的注释和增补,三卷集,1814年爱丁堡版第2卷第55页)。李嘉图的《原理》第2章(脚注中)引用了布坎南的这段话。——第178页。}}\end{quote}

重农学派的这个观点被摈弃了,——但是这个观点就其更深刻的意义来说是完全合理的,因为重农学派把地租看作剩余价值的唯一形式,而把资本家和工人一齐都只看作地主的雇佣劳动者,——剩下可能存在的就只有下述几种观点:

[523][第一,]认为地租来自农产品的垄断价格,而垄断价格又来自土地所有者对土地的垄断。\authornote{见本册第26页。——编者注}按照这一观点,农产品的价格总是高于其价值。这里有一个价格的附加额,商品的价值规律为土地所有权的垄断所破坏。

按照这一观点,地租所以来自农产品的垄断价格,是因为农产品的供给总是低于需求的水平,或者说,需求总是高于供给的水平。可是,为什么供给不会提高到需求的水平呢?为什么追加的供给不会使这种关系达到平衡,从而——按照这一理论——把一切地租取消呢?为了解释这一点,马尔萨斯一方面求助于臆造,说什么农产品直接为自己创造了消费者(关于这一点,以后评论他和李嘉图的论战时再谈),另一方面又求助于安德森的理论,说什么因为追加的供给耗费更多的劳动,所以农业的生产率降低。因此,这个观点就其不是根据纯粹臆造这一方面来说,是同李嘉图的理论一致的。这里也是价格高于价值,有一个附加额。

[第二,]李嘉图的理论:没有绝对地租,只有级差地租。这里也是提供地租的农产品的价格高于其个别价值,只要有地租存在,那就是由于有农产品的价格超过其价值的余额。不过在这里,这种价格超过价值的余额和一般的价值理论并不矛盾(虽然事实还是事实),因为在每一个生产领域内部,属于这个领域的商品的价值不是决定于商品的个别价值,而是决定于商品在该领域一般生产条件下所具有的价值。这里提供地租的产品的价格也是垄断价格,不过这种垄断在一切生产领域都有,它只是在这个生产领域才固定下来,因而采取了不同于超额利润的地租形式。这里,也是需求超过供给,或者也可以说,在价格由于需求超过供给而上涨以前,追加的需求不可能按原来供给状况下的价格,由追加的供给来满足。这里,地租(级差地租)的产生也是由于有价格超过价值的余额,由于较好土地的产品的价格上涨到高于其价值,从而引起追加的供给。

[第三,]地租只不过是投入土地的资本的利息。\authornote{见本册第26、152—153和157页。——编者注}这种观点和李嘉图的观点有一个共同的地方,就是否认绝对地租。在投入同量资本的不同地段提供数量不等的地租的情况下,它不得不承认级差地租。因此,实际上它可归结为李嘉图的观点,即某种土地不提供地租,凡是提供本来意义的地租的地方,提供的都是级差地租。但是这种观点绝对不能解释没有投入任何资本的土地的地租,瀑布、矿山等的地租。实际上,这种观点不过是从资本主义的立场出发,以利息为名,把地租从李嘉图的抨击下拯救出来的一种尝试。

最后[第四],李嘉图认为,在不提供地租的土地上,产品的价格等于产品的价值,因为价值等于平均价格,即预付资本加平均利润。所以,李嘉图错误地认为,商品的价值等于商品的平均价格。如果这种错误的前提不能成立的话,那末绝对地租就是可能的,因为农产品的价值,如同其他所有商品中的一大类商品的价值一样,是高于它们的平均价格的,但是,由于土地所有权的存在,农产品的价值不会象其他这些商品那样平均化为平均价格。所以,这种观点同垄断论一起承认土地所有权本身和地租有直接的关系;它同李嘉图一起承认有级差地租,最后,它认为绝对地租的存在绝不违反价值规律。

\tsubsectionnonum{[(2)价值决定于劳动时间是李嘉图理论的基本论点。作为经济科学发展的必然阶段的李嘉图研究方法及其缺点。李嘉图著作的错误结构]}

李嘉图是从商品的相对价值(或交换价值)决定于“劳动量”这一论点出发的。(后面我们将要研究李嘉图使用“价值”一词的不同含义。贝利对李嘉图理论的批评就是以此为根据的,同时,李嘉图的价值论的缺点也就在这里。)决定价值的这种“劳动”的性质,李嘉图并没有进一步研究。如果两种商品是等价物,或者说,它们在一定的比例上是等价物,或者也可以说,如果它们的量按[524]它们各自包含的“劳动”量来说是不相同的,那也很明显,在它们是交换价值的情况下,它们按其实体来说是相同的。它们的实体是劳动。所以它们是“价值”。根据它们各自包含的这种实体是多还是少,它们的量是不相同的。而这种劳动的形式——作为创造交换价值或表现为交换价值的劳动的特殊规定,——这种劳动的性质,李嘉图并没有研究。因此,李嘉图不了解这种劳动同货币的关系,也就是说,不了解这种劳动必定要表现为货币。所以,他完全不了解商品的交换价值决定于劳动时间和商品必然要发展到形成货币这两者之间的联系。他的错误的货币理论就是由此而来的。他一开始就只谈价值量,就是说,只谈各个商品价值量之比等于生产这些商品所必需的劳动量之比。李嘉图是从这一点出发的。他明确指出,亚·斯密是他的出发点(第一章第一节)\endnote{大·李嘉图《政治经济学和赋税原理》1821年伦敦第3版第1—12页。——第181页。}。

李嘉图的方法是这样的:李嘉图从商品的价值量决定于劳动时间这个规定出发,然后研究其他经济关系(其他经济范畴)是否同这个价值规定相矛盾,或者说,它们在多大的程度上改变着这个价值规定。人们一眼就可以看出这种方法的历史合理性,它在政治经济学史上的科学必然性,同时也可以看出它在科学上的不完备性,这种不完备性不仅表现在叙述的方式上(形式方面),而且导致错误的结论,因为这种方法跳过必要的中介环节,企图直接证明各种经济范畴相互一致。

这种研究方法从历史上看是合理的和必然的。在亚·斯密那里,政治经济学已发展为某种整体,它所包括的范围在一定程度上已经形成,因此,萨伊能够肤浅而系统地把它概述在一本教科书里。在斯密和李嘉图之间的这段时期,仅仅对生产劳动和非生产劳动、货币、人口论、土地所有权以及税收等个别问题作了一些研究。斯密本人非常天真地活动于不断的矛盾之中。一方面,他探索各种经济范畴的内在联系,或者说,资产阶级经济制度的隐蔽结构。另一方面,他同时又按照联系在竞争现象中表面上所表现的那个样子,也就是按照它在非科学的观察者眼中,同样在那些被实际卷入资产阶级生产过程并同这一过程有实际利害关系的人们眼中所表现的那个样子,把联系提出来。这是两种理解方法,一种是深入研究资产阶级制度的内在联系,可以说是深入研究资产阶级制度的生理学,另一种则只是把生活过程中外部表现出来的东西,按照它表现出来的样子加以描写、分类、叙述并归入简单概括的概念规定之中。这两种理解方法在斯密的著作中不仅安然并存,而且相互交错,不断自相矛盾。在斯密那里,这样做是有理由的(个别的专门的研究,如关于货币的研究除外),因为他的任务实际上是双重的。一方面,他试图深入研究资产阶级社会的内部生理学,另一方面,他试图既要部分地第一次描写这个社会外部表现出来的生活形式,描述它外部表现出来的联系,又要部分地为这些现象寻找术语和相应的理性概念,也就是说,部分地第一次在语言和思维过程中把它们再现出来。前一任务,同后一任务一样使他感到兴趣,因为两个任务是各自独立进行的,所以这里就出现了完全矛盾的表述方法:一种方法或多或少正确地表达了内在联系,另一种方法同样合理地,并且缺乏任何内在关系地,——和前一种理解方法没有任何联系地——表达了外部表现出来的联系。

斯密的后继者们,只要他们的观点不是从比较陈旧的、已被推翻的理解方法出发对斯密的反动,都能够在自己的专门研究和考察中毫无阻挡地前进,而且始终把亚·斯密作为自己的基础,不管他们是和斯密著作中的内在部分还是外在部分连结在一起,或者几乎总是把这两部分混在一起。但是,李嘉图终于在这些人中间出现了,他向科学大喝一声:“站住!”资产阶级制度的生理学——对这个制度的内在有机联系和生活过程的理解——的基础、出发点,是价值决定于劳动时间这一规定。李嘉图从这一点出发,迫使科学抛弃原来的陈规旧套,要科学讲清楚:它所阐明和提出的其余范畴——生产关系和交往关系——同这个基础、这个出发点适合或矛盾到什么程度;一般说来,只是反映、再现过程的表现形式的科学以及这些表现本身,同资产阶级社会的内在联系即现实生理学所依据的,或者说成为它的出发点的那个基础适合到什么程度;一般说来,这个制度的表面运动和它的实际运动之间的矛盾是怎么回事。李嘉图在科学上的巨大[525]历史意义也就在这里,因此,被李嘉图抽掉了立足点的庸俗的萨伊怒气冲冲地说:

\begin{quote}{“有人借口扩充它〈科学〉,把它推到真空里去了。”\endnote{让·巴·萨伊《论政治经济学》1826年巴黎第5版第1卷第83—84页,或让·巴·萨伊《论政治经济学》1841年巴黎第6版第41页。——第183页。}}\end{quote}

同这个科学功绩紧密联系着的是,李嘉图揭示并说明了阶级之间的经济对立——正如内在联系所表明的那样,——这样一来,在政治经济学中,历史斗争和历史发展过程的根源被抓住了,并且被揭示出来了。所以,凯里——参看后面有关段落——给李嘉图加上了共产主义之父的罪名:

\begin{quote}{“李嘉图先生的体系是一个制造纷争的体系……整个体系具有挑动阶级之间和民族之间的仇恨的倾向……他的著作是那些企图用平分土地、战争和掠夺的手段来攫取政权的蛊惑者们的真正手册。”(亨·凯里《过去、现在和将来》1848年费拉得尔菲亚版第74—75页)}\end{quote}

可见,李嘉图的研究方法,一方面具有科学的合理性和巨大的历史价值,另一方面,它在科学上的缺陷也是很明显的,这一点将在后面详细说明。

李嘉图著作的非常奇特的、必然谬误的结构,也是由此而来。全书(第三版)共分三十二章。其中有十四章论述赋税,因而只是理论原则的运用。\endnote{马克思除了把李嘉图著作中有关本来意义上的赋税的十二章(第8—18章和第29章)列为论述赋税的各章以外,还把涉及赋税问题的另外两章——第22章和第23章(《出口补贴和进口禁令》和《论生产补贴》)也包括进去。按照李嘉图的理论,补贴是由居民交纳的各种赋税所组成的基金来支付的。——第184页。}第二十章《价值和财富,它们的特性》,无非是研究使用价值和交换价值的区别,因而是第一章《论价值》的补充。第二十四章《亚·斯密的地租学说》,以及第二十八章《论……黄金、谷物和劳动的比较价值》和第三十二章《马尔萨斯先生的地租观点》,不过是李嘉图地租理论的补充,部分地是对这个理论的辩护,因而仅仅是论述地租的第二章和第三章的附录。第三十章《论需求和供给对价格的影响》不过是第四章《论自然价格和市场价格》的附录。而第十九章《论商业途径的突然变化》则是这一章的第二个附录。第三十一章《论机器》不过是第五章《论工资》和第六章《论利润》的附录。第七章《论对外贸易》和第二十五章《论殖民地贸易》,同论赋税各章一样,仅仅是前面提出的原则的运用。第二十一章《积累对于利润和利息的影响》是论地租、利润和工资各章的附录。第二十六章《论总收入和纯收入》是论工资、利润和地租各章的附录。最后,第二十七章《论货币流通和银行》在这本书中完全是孤立的,它只是李嘉图在他较早的论货币的著作中提出的观点的进一步发挥,部分地是这些观点的变态。

可见,李嘉图的理论完全包括在他这部著作的前六章中。我说的这部著作的错误结构,就是指这一部分。另一部分(论货币的那部分除外)是实际运用、解释和补充,按其内容的性质来说是杂乱地放在那里的,根本不要求有什么结构。但是理论部分(前六章)的错误结构并不是偶然的,而是由李嘉图的研究方法本身和他给自己的研究提出的特定任务决定的。这种结构表现了这种研究方法本身在科学上的缺陷。

第一章是《论价值》。它又分为七节。第一节研究的其实是:工资是否同商品价值决定于商品所包含的劳动时间这一规定相矛盾?第三节是要证明:我称为不变资本的东西加入商品价值,是和价值规定不矛盾的,工资的提高或降低同样不会影响商品的价值。第四节研究的是:在机器和其他固定的、耐久的资本在不同生产领域以不同的比例加入总资本的情况下,它们的运用能在多大程度上改变交换价值决定于劳动时间这个规定。第五节研究的是:如果不同生产领域所使用的资本的耐久程度不等、周转时间不同,工资的提高或降低能在多大程度上改变价值决定于劳动时间这个规定。由此可见,在这第一章里不仅假定了商品的存在,——而在考察价值本身的时候是不应该作进一步的假定的,——而且假定了工资、资本、利润,甚至,如我们将会看到的,还假定了一般利润率、由流通过程产生的资本的各种形式,以及“自然价格和市场价格”的区别,这种区别在后面两章(《论地租》和《论矿山地租》)中甚至起着决定性的作用。

第二章(《论地租》),[526]——第三章(《论矿山地租》)只是第二章的补充,——完全依照李嘉图的研究进程,一开始又提出这样的问题:土地所有权和地租是否同商品价值决定于劳动时间这一规定相矛盾?

\begin{quote}{李嘉图在第二章(《论地租》)一开头就说:“但尚待考察的是,对土地的占有以及由此而来的地租的产生,是否会引起商品相对价值的变动而不管生产商品所必需的劳动量如何。”(《政治经济学和赋税原理》1821年伦敦第3版第53页)}\end{quote}

李嘉图为了进行这一研究,不仅顺便把“市场价格”和“实际价格”(价值的货币表现)的关系引进来,而且把整个资本主义生产以及他对工资和利润之间的关系的全部见解作为前提。因此,第四章(《论自然价格和市场价格》)、第五章(《论工资》)和第六章(《论利润》)所谈的东西,在头两章(《论价值》和《论地租》)以及作为第二章附录的第三章中不仅已经作了假定,而且有了充分的发挥。在后面三章中,就它们所提出的理论上的新东西来说,只是在这里或那里堵塞漏洞,补充一些更确切的规定,其中大部分按理在第一章、第二章中本来就应当谈到。

可见,李嘉图的全部著作已经包括在它头两章里了。在这两章中,把发展了的资产阶级生产关系,因而也把被阐明的政治经济学范畴,同它们的原则即价值规定对质,查清它们同这个原则直接适合到什么程度,或者说,查清它们给商品的价值关系造成的表面偏差究竟是什么情况。李嘉图著作的这两章包含着他对以往政治经济学的全部批判,他在这里同亚·斯密的贯串其全部著作的内在观察法和外在观察法之间的矛盾断然决裂,而且通过这种批判得出了一些崭新的惊人结果。因此,这头两章给人以高度的理论享受,因为它们简明扼要地批判了那些连篇累牍、把人引入歧途的老观念,从分散的各种各样的现象中吸取并集中了最本质的东西,使整个资产阶级经济体系都从属于一个基本规律。这头两章由于其独创性、基本观点一致、简单、集中、深刻、新颖和洗炼而给人以理论上的满足,但是再往下读这本著作时这种理论上的满足就必然会消失。在那里,有的地方也会有个别独到的见解吸引住我们。但总的说来令人感到疲倦和乏味。进一步的阐述已经不再是思想的进一步发展了。这种阐述不是单调地、形式地把同一些原则运用于各种各样凭外表拿来的材料或者为这些原则进行辩护,就是单纯地重复或者补充;最多是在该书的最后部分有些地方作出某种引人注意的结论。

我们在批判李嘉图的时候,应该把他自己没有加以区别的东西区别开来。[第一,]是他的剩余价值理论,这个理论在他那里当然是存在的,虽然他没有把剩余价值确定下来,使之有别于它的特殊形式利润、地租、利息。第二是他的利润理论。我们将从分析李嘉图的利润理论开始,虽然它不属于这一篇,而属于第三篇[18]的历史附录。

\tsubsectionnonum{[(3)李嘉图在绝对价值和相对价值问题上的混乱。他不懂价值形式]}

首先还要稍微说明一下,李嘉图怎样把各种[不同的]“价值”规定混淆起来了。贝利反驳李嘉图,就是根据这一点。不过,这一点对我们来说也是重要的。

李嘉图起先把价值称为“交换价值”,他和亚·斯密一起把价值规定为“购买其他货物的能力”。(《原理》第1页)这是作为最初的表现形式的交换价值。但是,接着他就谈到真正的价值规定:

\begin{quote}{“各种商品的现在的或过去的相对价值,决定于劳动所生产的各种商品的相对量”。(同上,第9页)}\end{quote}

这里所说的“相对价值”无非是由劳动时间决定的交换价值。但是相对价值也可能有另一种意义,就是说,我用另一种商品的使用价值来表现一种商品的交换价值,比如说,用咖啡的使用价值来表现糖的交换价值。

\begin{quote}{“两种商品的相对价值发生变动,我们想知道是哪一种发生了变动。”(同上,第9页)}\end{quote}

什么样的变动?这种“相对价值”李嘉图在后面也称为“比较价值”。(同上,第488页及以下各页)我们想知道是哪一种商品发生了“变动”,就是说,我们想确定前面称为相对价值的那种“价值”的变动。例如,1磅糖=2磅咖啡。后来1磅糖=4磅咖啡。我们想知道的“变动”在于:是糖的“必要劳动时间”变了呢,还是咖啡的“必要劳动时间”变了,是糖耗费的劳动时间比过去多一倍呢,还是咖啡耗费的劳动时间比过去少一半,生产这两种商品各自所必要的劳动时间的这两种“变动”中,是哪一种变动引起了它们的交换比例的变动。可见,糖或咖啡的这种“相对价值,或者说,比较价值”——它们交换的比例——不同于第一种意义的相对价值。在第一种意义上,糖的相对价值决定于[527]一定劳动时间内能够生产出来的糖的量。在第二种场合,糖[和咖啡]的相对价值表示它们相互交换的比例,而这个比例的变动可能是咖啡或者糖的第一种意义的“相对价值”变动的结果。虽然它们的第一种意义的“相对价值”发生了变动,它们相互交换的比例可能不变。虽然生产糖和咖啡的劳动时间增加了一倍,或者减少了一半,1磅糖可能仍旧等于2磅咖啡。它们的比较价值(就是说,糖的交换价值用咖啡来表现,咖啡的交换价值用糖来表现)的变动,只有在它们的第一种意义的相对价值,即由劳动量决定的价值,按不同的程度变动,因而它们的比例发生了变动的时候,才表现出来。绝对变动如果不改变原来的比例,就是说,如果变动的幅度一样,方向一致,就不会引起这些商品的比较价值的任何变动,也不会引起它们的货币价格的任何变动,因为货币的价值即使发生变动,对它们两者也是按相同的程度变动的。因此,不论两种商品中每一种的价值我是用其中另一种的使用价值来表现,还是用它们的货币价格来表现,即这两者的价值用第三种商品的使用价值来表示,这些相对价值,或者说,比较价值,或者说,价格,仍旧不变,应该把这种相对价值的变动同商品的第一种意义的相对价值的变动区别开来,因为后者所表示的仅仅是生产商品本身所必需的,即物化在商品本身中的劳动时间量的变动。因此,同第二种意义的相对价值(即一个商品的交换价值用另一个商品的使用价值或者用货币来实际表现)相比,第一种意义的相对价值就表现为“绝对价值”。所以,在李嘉图的著作里,也可以看到用“绝对价值”这一术语来表示第一种意义的相对价值。

如果在上述例子中1磅糖耗费的劳动时间仍然和过去一样多,那末它的第一种意义的“相对价值”就没有变动。如果咖啡耗费的劳动量减少一半,那末用咖啡来表现的糖的价值就发生变动,因为咖啡的第一种意义的“相对价值”变动了。可见,糖和咖啡的相对价值与它们的“绝对价值”是表现得不同的,而这种差别所以会表现出来,是因为比如说糖的比较价值同那些绝对价值保持不变的商品相比并没有变动。

\begin{quote}{“我希望引起读者注意的这个研究,涉及的是商品相对价值的变动的影响,而不是商品绝对价值的变动的影响。”(同上,第15页)}\end{quote}

这种“绝对”价值,李嘉图在其他场合也称为“实际价值”,或直接称为“价值”。(例如第16页)

请看贝利在下面这本书中对李嘉图的反驳:《对价值的本质、尺度和原因的批判研究,主要是论李嘉图先生及其信徒的著作》,《略论意见的形成和发表》一书的作者著,1825年伦敦版(并见同一作者所著:《为〈韦斯明斯特评论〉杂志上一篇关于价值的论文给一位政治经济学家的信》1826年伦敦版)。贝利的整个反驳部分地是围绕价值概念规定中这些不同方面的,这些不同方面在李嘉图的著作中并没有发挥,只是实际存在着,彼此交错着,而在其中贝利看到的只是“矛盾”。第二,他的反驳是针对不同于比较价值(即第二种意义的相对价值)的“绝对价值”,或者说,“实际价值”的。

\begin{quote}{贝利在上述第一部著作中说:“他们〈李嘉图及其信徒〉不是把价值看成两个物之间的比例,而是把价值看成由一定量劳动生产出来的有用的成果。”(第30页)他们认为,“价值是某种内在的和绝对的东西”。(同上,第8页)}\end{quote}

最后这个指责是由李嘉图说明问题的缺陷引起的,因为他完全不是从形式方面,从劳动作为价值实体所采取的一定形式方面来研究价值,而只是研究价值量,就是说,研究造成商品价值量差别的这种抽象一般的、并在这种形式上是社会的劳动的量。否则贝利就会看到,决不因为一切商品就它们是交换价值来说都只是社会劳动、社会劳动时间的相对表现,价值概念的相对性就取消了;贝利也就会明白,商品的相对性决不仅仅在于商品彼此交换的比例,而且在于一切交换价值同作为它们的实体的这种社会劳动的比例。

相反,后面我们将会看到,应该责备李嘉图的,倒是他经常忘记了这种“实际价值”,或者说,“绝对价值”,而只是念念不忘“相对价值”,或者说,“比较价值”。

[528]因而:

\tsubsectionnonum{[(4)]李嘉图对利润、利润率和平均价格等的解释}

\tsubsubsectionnonum{[(a)李嘉图把不变资本同固定资本,可变资本同流动资本混淆起来。关于“相对价值”的变动及其因素问题的错误提法]}

在第一章第三节中,李嘉图阐述了下面这种思想:如果我们说,商品价值决定于劳动时间,那末,这既包括最后的劳动过程中直接花费在这种商品上的劳动,也包括花费在为生产这种商品所必需的原料和劳动资料上的劳动时间;因此,不仅包括新加的、用工资支付的、买进的劳动所包含的劳动时间,并且包括我称为不变资本的那部分商品所包含的劳动时间。李嘉图对这个问题的解释的缺点在这第一章第三节的标题上就已经表现出来了。这个标题是:

\begin{quote}{“影响商品价值的,不仅是直接花费在商品上的劳动,而且还有花费在协助这种劳动的器具、工具和建筑物上的劳动。”(第16页)}\end{quote}

这里漏掉了原料,而花费在原料上的劳动,象花费在劳动资料“器具、工具和建筑物”上的劳动一样,是不同于“直接花费在商品上的劳动”的。但是李嘉图头脑里考虑的已经是下一节了。在第三节,他假定用掉的劳动资料以同样的价值组成部分加入不同商品的生产。而在下一节,考察的是由于固定资本以不同的比例加入生产而产生的差别。因此,李嘉图没有得出不变资本的概念,不变资本一部分由固定资本组成,另一部分即原料和辅助材料则由流动资本组成,正如流动资本不仅包括可变资本,而且包括原料等以及一切加入一般消费(不只是加入工人的消费)的生活资料\endnote{马克思这里说的“加入一般消费的生活资料”,一方面是指所有个人消费的资料,另一方面是指用于机器的生产消费资料,即辅助材料(煤、润滑油等)。——第192页。}。

不变资本加入商品的比例,并不影响商品的价值,并不影响商品包含的相对劳动量,但是,这种比例直接影响包含等量劳动时间的商品所包含的不同的剩余价值量,或者说,剩余劳动量。因此,这种不同的比例就造成不同于价值的平均价格。

关于第一章第四、五两节,首先要指出,李嘉图不去研究不同生产领域中同一资本量的组成部分由不变资本和可变资本构成的比例这种极为重要的、影响剩余价值直接生产的差别,却专门去研究资本形式的差别和同量资本采取这些不同形式的不同比例,研究从资本的流通过程产生的形式差别,即固定资本和流动资本、固定程度较大或较小的资本(即具有不同耐久程度的固定资本)和资本的不等的流通速度或周转速度。并且,李嘉图研究的方法是这样的:他为等量的各种投资,或者说,为使用等量资本的不同生产领域,假定一个一般利润率,或者说,一个等量的平均利润,或者也可以说,他先假定利润和不同生产领域使用的资本的量成比例。其实,李嘉图不应该先假定这种一般利润率,相反,他倒是应该研究一般利润率的存在究竟同价值决定于劳动时间这一规定符合到什么程度,这样,他就会发现,一般利润率同这一规定不是符合的,乍看起来倒是矛盾的,所以一般利润率的存在还须要通过许多中介环节来阐明,而这样做与简单地把它归到价值规律下是大不相同的。这样,李嘉图就会得到一个关于利润本质的完全不同的概念,而不会把利润直接同剩余价值等同起来。

李嘉图先作了这个假定,接着就给自己提出一个问题:如果固定资本和流动资本以不同的比例加入生产,工资的提高或降低对“相对价值”会发生什么影响?或者确切些说,他自以为正是这样来考察问题的。其实,他根本不是这样考察问题的。他考察问题的方法是:他问自己,在几笔资本的流通时间不同、其中包含的不同资本形式所占的比例也不同的情况下,工资的提高或降低对这些资本各自的利润将发生什么影响?这里,他自然发现,根据加入的固定资本等等的多少不同,根据资本中由可变资本即由直接花在工资上的资本组成的部分的大小不同,工资的提高或降低对资本的影响必然大不相同。因此,为了使不同[529]生产领域的利润重新平均化,换句话说,为了恢复一般利润率,商品的价格——不同于商品的价值——就必须按另外的方式来决定。就是说,——他接着得出结论说,——在工资提高或降低的情况下,这些差别会影响“相对价值”。他本应反过来说:这些差别虽然同价值本身毫无关系,但是由于它们对不同生产领域的利润发生不同影响,就造成不同于价值本身的平均价格,即我们后面所说的费用价格,这种费用价格不直接决定于商品的价值,而决定于预付在这些商品上的资本加平均利润。因此,李嘉图本应说:这种平均的费用价格不同于商品的价值。可他不是这样,却得出结论说,它们是等同的,并且带着这个错误的前提去考察地租。

李嘉图认为,只是由于他所研究的三种情况,他才考虑到同商品所包含的劳动时间无关的“相对价值的变动”,就是说,实际上才考虑到商品的费用价格和价值的差别,这种看法也是错误的。他已经假定了这个差别,因为他假定有一个一般利润率,就是说,假定尽管资本的有机组成部分的比例不同,资本提供的利润总是同资本的量成比例,可是资本所提供的剩余价值,却完全决定于资本所吸收的无酬劳动时间的量,这个量,在工资既定时,完全取决于花在工资上的那部分资本的量,而不取决于资本的绝对量。

实际上我们在李嘉图那里看到的是:他先假定有不同于商品价值的费用价格,——既然假定有一般利润率,也就假定了这个差别,——再研究这些费用价格(为了换个花样,现在叫做“相对价值”)本身又怎样由于工资的提高或降低以及在资本的有机组成部分的比例不同的情况下,在相互之间,彼此相对地发生变动。如果把问题钻得更深一些,李嘉图就会发现,在资本的有机组成部分不同(这种不同最初在直接生产过程中表现为可变资本和不变资本的差别,后来由于从流通过程中产生的差别而进一步扩大)的情况下,即使假定工资不变,单单一般利润率的存在,就已经决定了有一种不同于价值的费用价格。换句话说,李嘉图就会发现,单单一般利润率的存在,就决定了有一个同工资的提高或降低完全无关的差别和新的形式规定。李嘉图也会看到,理解这个差别,同他对工资的提高或降低所引起的商品费用价格变动的分析相比,对于整个理论具有无比重要的、决定性的意义。他所满足的结论(而满足于这个结论是符合他的整个研究方法的)是:如果承认并考虑到商品费用价格(或者照他的说法,“相对价值”)的变动,只要这种变动是在投入不同领域的资本的有机构成不同的条件下由工资的变动,由工资的提高或降低引起的,那末,规律仍然是正确的,这种情况同商品“相对价值”决定于劳动时间的规律并不矛盾,因为商品费用价格的其他一切不只是短暂的变动,都只能用生产这些商品各自所必需的劳动时间量的变动来解释。

相反,李嘉图把固定资本和流动资本的差别与不同的资本周转时间相对比,并从不同的流通时间,实际上也就是从资本的流通时间或再生产时间引出这一切差别,却应该看成是一个重大的功绩。

我们首先考察一下李嘉图最初在(第一章)第四节中所叙述的这些差别本身,然后考察在李嘉图的叙述中这些差别以什么方式影响或引起“相对价值”的变动。

\begin{quote}{(1)“在每一种社会状态中,不同行业所使用的工具、器具、建筑物和机器的耐久程度可能彼此不同,生产它们所需要的劳动量也可能各不相同”。(同上,第25页)}\end{quote}

谈到“生产它们所需要的劳动量各不相同”,它的意思可能是指:——看来李嘉图在这里指的只是这一点,——耐久性较差的工具,部分地为了它们的修理,部分地为了它们的再生产,需要较多的劳动(重新进行的直接劳动);也可能是指:耐久程度相同的机器等可能有贵有贱,可能是较多劳动或较少劳动的产品。后面这个观点对于理解可变资本和不变资本的比例很重要,但它同李嘉图所考察的问题毫无关系,因此,李嘉图在任何地方都没有把它当作一个独立的观点。

\begin{quote}{[530](2)“维持劳动的资本〈可变资本〉和投在工具、机器和建筑物上的资本〈固定资本〉可能结合的比例也是多种多样的。”这样,我们就有了“固定资本耐久程度的这种差别,和这两种资本可能结合的比例的这种多样性”。(第25页)}\end{quote}

一眼就可以看出,李嘉图为什么对作为原料存在的那部分不变资本不感兴趣。这部分不变资本属于流动资本。如果工资提高,由机器组成的、无需更换而继续存在的那部分资本并不因此增加支出,可是由原料组成的那部分资本却因此增加支出,因为原料要不断补充,也就是说,要不断地再生产出来。

\begin{quote}{“工人消费的食物和衣服,他在其中从事劳动的建筑物,他劳动时使用的工具,都是会损坏的。但是,这些不同资本的耐用时间却大有差别……有的资本损耗得快,必须经常再生产,有的资本消费得慢,根据这种情况,就有流动资本和固定资本之分。”(第26页)}\end{quote}

可见,这里把固定资本和流动资本的差别归结为再生产时间(同流通时间一致的再生产时间)的差别。

\begin{quote}{(3)“还必须指出,流动资本流通或流回到它的使用者手中的时间可以极不相等。租地农场主买来作种子的小麦\authornote{这里洛贝尔图斯先生可以看到,在英国,种子是“买来”的。},和面包业主买来做面包的小麦相比,是固定资本。前者把小麦播在地里,要等一年以后才能收回;后者把小麦磨成面粉,制成面包卖给顾客,一周之内就能自由地用他的资本重新开始同一事业或开始任何别的事业。”(第26—27页)}\end{quote}

为什么会产生不同流动资本的流通时间的这种差别呢?因为同一资本在一种情况下停留在本来意义的生产领域内的时间较长,虽然在这段时间内劳动过程并没有继续。葡萄酒放在窖里变陈,以及制革、染色等化学过程,就是这种情况。

\begin{quote}{“因此,两种行业可能使用同量的资本,但其固定部分和流动部分的划分却大不相同。”(第27页)(4)“此外,两个工厂主可能使用同量的固定资本和同量的流动资本,但是他们的固定资本的耐久程度〈因而它们的再生产时间〉可能大不相同。一个可能有价值10000镑的蒸汽机,另一个则有价值相等的船舶。”(第27—28页)“……资本的耐久程度不同,或者也可以说……一批商品在能够进入市场以前必须经历的时间不同”。(第30页)(5)“无需说,花费了同量劳动生产出来的商品,如果不能在同样长的时间内进入市场,它们的交换价值就会不相等。”(第34页)}\end{quote}

这样,我们就有:(1)固定资本和流动资本的比例的差别;(2)生产过程继续进行时由于劳动过程中断而引起的流动资本周转的差别;(3)固定资本耐久程度的差别;(4)商品在能够进入本来意义的流通过程以前经受劳动过程的时间(不算劳动时间的中断,不算生产时间和劳动时间的差别\endnote{关于“生产时间”和“劳动时间”,见注8。——第197页。})的差别。关于最后一点,李嘉图做了这样的描述:

\begin{quote}{“假定我花1000镑雇用20个工人在一年内生产一种商品,年终,再花1000镑雇用20个工人在下一年内完成或改进这种商品,在两年末了,我把商品运到市场上去。如果利润为10%,我的商品就必须卖2310镑,因为我在第一年用了1000镑资本,第二年用了2100镑资本。另一个人使用完全相同的劳动量,但是全部用在第一年;他花2000镑雇用40个工人,在第一年末,就把他的商品按10%的利润出卖,也就是卖2200镑。于是,这里就有两种商品,所耗费的劳动量完全相同,其中一种卖2310镑,另一种卖2200镑。”(第34页)}\end{quote}

[531]但是,这种差别——不论是固定资本耐久程度的差别,还是流动资本流通时间的差别,或者是两种资本结合比例的差别,最后,或者是花费了同量劳动生产出来的不同商品[在进入市场以前]所需要的时间的差别,——究竟怎样引起这些商品的相对价值的变动呢?李嘉图起初说,这是因为:

\begin{quote}{“……这种差别,和……比例的这种多样性,在生产商品所必需的劳动量增减之外,又给商品相对价值的变动提供了另一个原因,这个原因就是劳动价值的提高或降低。”(第25—26页)}\end{quote}

怎样证明这一点呢?

\begin{quote}{“工资的提高,对于在如此不同的情况下生产出来的商品,不能不产生不同的影响”,(第27页)}\end{quote}

这里指的就是这样一种情况:在不同行业中使用同样大小的资本,一笔资本主要由固定资本组成,只有很小一部分由“用于维持劳动的”资本组成,而另一笔资本情况恰好相反。首先,说对“商品”产生影响是荒谬的。李嘉图指的是商品的价值。但是,价值在多大程度上受这种情况的影响呢?根本不受影响。在两种情况下,受影响的是利润。例如,一个人只把1/5的资本用作可变资本,在工资相等和剩余劳动率相等的条件下,如果剩余价值率等于20%,用100只能生产剩余价值4;而另一个人把4/5的资本用作可变资本,就会生产剩余价值16。因为在第一种场合,花在工资上的资本等于100/5,也就是20,20的5/1(或20%)等于4。在第二种场合,花在工资上的资本等于4/5×100,也就是80,80的1/5(或20%)等于16。在第一种场合利润等于4,在第二种场合利润等于16。两笔资本的平均利润是(16+4)/2,即20/2,也就是10%。李嘉图所说的那种情况其实就是这样。所以,如果两个企业主都按费用价格出卖商品,——李嘉图就是这样假定的,——他们每人的商品就都卖110。现在假定,工资比以前提高比方说20%。以前一个工人花费1镑,现在花费1镑4先令,或24先令。第一个企业主和以前一样把80镑用作不变资本(因为李嘉图在这里把劳动材料撇开,所以我们也可以这样做),对于他所使用的20个工人,除20镑以外,他还要多付80先令即4镑。因此,他的资本现在应为104镑。在110镑中留给他的利润只有6镑,因为工人提供的剩余价值不是多了,而是少了。6镑比104,得[5+(10/13)]%。相反,另一个企业主使用80个工人,就要多付320先令即16镑。因此他必须花费116镑。如果他不得不按110镑出卖商品,他就不但得不到盈利,反而要亏损6镑。但是,这种情况之所以发生,只是因为平均利润已经改变了这个企业主花在劳动上的费用和他自己的企业所提供的剩余价值之间的比例。

这样一来,李嘉图没有去研究重要问题,就是究竟发生了什么变动,才使得一个把100镑中的80镑花在工资上的企业主没有得到四倍于另一个把100镑中的20镑花在工资上的企业主的利润;他却去研究一个次要问题,即下述情况是怎样产生的:在这个大差别拉平之后,也就是说在利润率既定时,这种利润率的任何变动,比如由于工资的提高,对于用100镑资本使用许多工人的企业主的影响,比对于用100镑资本使用很少工人的企业主的影响要大得多,因此,——在利润率相等的条件下,——前者的商品价格或费用价格必须上涨,后者的商品价格或费用价格必须下降,才能使利润率继续保持相等。

虽然李嘉图最初就告诉我们,“相对价值”的全部变动必定起因于“劳动价值的提高”,但是他提出的第一个例证,却同这个原因绝对没有关系。这个例证如下:

\begin{quote}{“假定有两个人各自雇用100个工人在一年内制造两台机器,另外有一个人雇用同样数目的工人种植谷物,年终,每台机器的价值将和谷物的价值相等,因为所有这三种商品都是由同量劳动生产出来的。假定一台机器的所有者在下一年雇100个工人用这台机器织造呢绒;另一台机器的所有者也雇100个工人用他的机器织造棉布,而租地农场主和以前一样继续雇用100个工人种植谷物。第二年他们都使用同量劳动}\end{quote}

{就是说,他们在工资上花费同量资本,但决不是使用同量劳动},

\begin{quote}{但是,毛织厂主的产品加上他的机器[532]和棉织厂主的产品加上他的机器,同样都是200个工人劳动一年的结果,或者更确切地说,是100个工人劳动两年的结果,而谷物则是100个工人劳动一年的结果。所以,如果谷物的价值是500镑,那末,毛织厂主的机器和呢绒加在一起的价值就应该是1000镑,棉织厂主的机器和棉布的价值也应该是谷物价值的两倍。但是,它们的价值将不止是谷物价值的两倍,因为毛织厂主和棉织厂主的资本已经加上了这些资本在第一年的利润,而租地农场主却把利润花费和享受掉了。因此,由于他们的资本的耐久程度不同,或者也可以说,由于一批商品在能够进入市场以前必须经历的时间不同,这些商品的价值同花费在它们上面的劳动量不会恰好成比例。在这种情况下,商品价值的比例不是2∶1,而是稍大一些,以便补偿价值较大的一种商品在能够进入市场以前必须经历的较长时间。假定为了购买每个工人的劳动每年付出50镑,或者说,使用资本5000镑,利润是10%,那末,在第一年末,每台机器的价值同谷物的价值一样都是5500镑。第二年,工厂主和租地农场主各自再花5000镑来维持劳动,因而他们的产品将仍然卖5500镑。但是,使用机器的两个工厂主要与租地农场主处于平等地位,就不仅必须为花费在劳动上的等量资本5000镑取得5500镑,而且必须取得一个追加额550镑作为他们投在机器上的5500镑的利润,因此〈就是因为相等的10%的年利润率已经被假定为一种必然和规律〉,他们的产品必须卖6050镑。”}\end{quote}

{这样一来,由于平均利润——由于李嘉图预先假定的一般利润率,——就产生了不同于商品价值的平均价格,或者说,费用价格。}

\begin{quote}{“因此,这里我们看到,虽然资本家们每年正好使用同量的劳动来生产他们的商品,但是,由于他们各自使用的固定资本,或者说,积累劳动的量不同,他们生产出来的商品的价值是各不相同的。”}\end{quote}

{不是由于这一点,而是由于这两个无赖都有一个固定观念,认为他们两个由于“维持了劳动”都应该获得同量的赃物,或者说,不管他们的商品各自具有多少价值,都必须按照平均价格出卖,使他们每个人都得到同样的利润率。}

\begin{quote}{“呢绒和棉布具有相同的价值,因为它们是同量劳动和同量固定资本的产品。但是谷物和这些商品不具有相同的价值{应该说:费用价格},因为就固定资本来说,谷物是在不同的条件下生产出来的。”(第29—31页)}\end{quote}

李嘉图举出这个极其笨拙而难懂的例子来说明极其简单的事情,就是不想简单地说:因为等量的资本,不管其有机部分的比例如何,或者不管其流通时间如何,都提供等量的利润,——如果商品按其价值出卖,就不可能如此,——所以,有一种不同于这些价值的商品费用价格存在。而且这一点已经包含在一般利润率的概念中了。

我们就来分析一下这个复杂的例子,并且把它还原为它本来的毫不“复杂”的样子。为了这个目的,让我们从结尾开始,同时为了把问题理解得更清楚,让我们事先指出,根据李嘉图的“假定”,租地农场主和经营棉布业的家伙并不为原料花费什么,其次,租地农场主在劳动工具上也不花费任何资本,最后,经营棉布业的畜生投入的固定资本的任何部分都不作为损耗加入他的产品。所有这些假定虽然是荒谬的,但是它们本身丝毫无损于这个例证。

在所有这些假定之下,李嘉图的例子如果从结尾开始,就是这样:租地农场主在工资上花费5000镑;经营棉布业的坏蛋在工资上花费5000镑,在机器上花费5500镑。所以前者花费5000镑,而后者花费10500镑,也就是[533]比前者多一倍。假定两人都要赚10%的利润,那末,租地农场主的商品就必须卖5500镑,经营棉布业的家伙的商品就必须卖6050镑(因为已经假定,投在机器上的5500镑中任何部分都不作为机器的损耗构成产品价值的组成部分)。这只能说明,商品的费用价格只要决定于商品所包含的预付的价值加上同一个年利润率,它就不同于商品的价值,而这个差别的产生是由于商品按照给预付资本提供同一利润率的价格出卖;简单地说,费用价格和价值之间的这个差别,同一般利润率是一回事。除此以外,李嘉图究竟还要说明什么,实在无法理解。连他这里硬加进来的固定资本和流动资本的差别,在这个例子里也纯粹是胡扯。因为,比如说,如果棉织厂主多花的5500镑由原料组成,而租地农场主则不需要种子等等,得出来的结果还是完全一样。

这个例子也没有证明李嘉图所说的:

\begin{quote}{“由于他们〈棉织厂主和租地农场主〉各自使用的固定资本,或者说,积累劳动的量不同,他们生产出来的商品的价值是各不相同的。”(第31页)}\end{quote}

因为根据李嘉图的假定,棉织厂主的固定资本等于5500镑,租地农场主的固定资本等于零;前者使用固定资本,后者不使用固定资本。因此,决不是他们使用的固定资本的“量不同”,正如一个人吃肉,一个人不吃肉,我们不能说他们吃的肉的“量不同”一样。相反,说棉织厂主和租地农场主使用的“积累劳动”即物化劳动的“量不同”,就是说,前者使用10500镑,后者只使用5000镑,倒是正确的(虽然用一个“或者说”把“积累劳动”一词偷偷塞进来是完全错误的)。但是,说他们使用的“积累劳动的量不同”,无非是说:他们投入自己企业的“资本的量不同”;利润量取决于他们所使用的资本量的差别,因为已经假定利润率是相同的;最后,和资本量成比例的利润量的这种差别,表现在商品各自的费用价格上。

但是,李嘉图例证的笨拙是从哪里来的呢?

\begin{quote}{“因此,这里我们看到,虽然资本家们每年正好使用同量的劳动来生产他们的商品,但是……他们生产出来的商品的价值是各不相同的。”(第30—31页)}\end{quote}

就是说,他们不是一般地使用同量的劳动——直接劳动和积累劳动加在一起,而是使用同量的可变资本,花费在工资上的资本,使用同量的活劳动。而且,因为货币同积累劳动也就是同以机器等形式存在的商品只能按照商品交换的规律相交换,因为剩余价值只是由于一部分被使用的活劳动被无偿占有才会产生,所以,很明显(因为根据假定,机器的任何部分都不作为损耗加入商品),只有在利润和剩余价值是等同的时候,两个资本家才能获得同量的利润。棉织厂主虽然花费两倍以上的资本,但是他必须同租地农场主一样,按5500镑出卖他的商品。即使机器全部加入商品价值,他的商品也只能卖11000镑,就是说,他获得的利润还不到5%,而租地农场主获得的却是10%。但是,假定租地农场主获得的10%代表他的商品中包含的实际无酬劳动,那末,尽管租地农场主和工厂主获得的利润不等,他们的商品倒是按其价值出卖的。这就是说,如果他们出卖商品获得同量利润,那末,就是下述情况二者必居其一:或者是工厂主随意在他的商品上附加了5%,在这种情况下,工厂主和租地农场主的商品总起来看就是高于其价值出卖;或者是租地农场主赚得的实际剩余价值大约为15%,他们两人用自己的商品获得10%的平均利润。在这后一种情况下,虽然他们各自的商品的费用价格每一次不是高于商品的价值就是低于商品的价值,但商品总额则按其价值出卖,而利润的平均化本身决定于商品中包含的剩余价值总额。这里,上面所引的李嘉图那句话,如果适当地修改一下,就包含下面这个正确的思想:在花费同量资本的情况下,可变资本和不变资本的比例不同,生产出来的商品就必然具有不同的价值,因而必然提供不同的利润;因此,这些利润的平均化必然产生不同于商品价值的费用价格。“因此,这里我们看到,虽然资本家们每年正好使用同量的〈直接的、活的〉劳动来生产他们的商品,但是,由于他们各自使用的……积累劳动的量不同,他们生产出来的商品的价值是各不相同的〈这就是说,商品具有不同于其价值的费用价格〉。”但是这种猜想在李嘉图那里是没有说出来的。它只不过说明他举的这个例证是自相矛盾和显然错误的,这个例证直到现在同“资本家使用的固定资本的量不同”毫无关系。

现在我们继续往下分析。工厂主第一年雇用100个工人制造一台机器,在这个时间内租地农场主也雇用100个工人生产谷物。第二年工厂主用机器来织布,为此又雇用100个工人。而租地农场主又雇用100个工人来种植谷物。李嘉图说,假定谷物的价值每年为500镑。我们假定,其中无酬劳动为有酬劳动的25%,也就是每400单位有酬劳动提供100单位无酬劳动。那末,机器在第一年末的价值也是500镑,其中400镑为有酬劳动,100镑为无酬劳动。我们还要假定,[534]机器在第二年末全部损耗完,加入棉布的价值。实际上李嘉图就是这样假定的,因为在第二年末他拿来同“谷物的价值”比较的不只是棉布的价值,而是“棉布和机器的价值”。

好极了!在这种情况下,第二年末棉布的价值必定等于1000镑,也就是500镑为机器的价值,500镑为劳动新加的价值。相反,谷物的价值等于500镑,也就是400镑为工资,100镑为无酬劳动。到此为止,这个例子中还没有什么同价值规律矛盾的东西。棉织厂主同谷物种植业者完全一样,赚得25%的利润,但是前者的商品等于1000镑,后者的商品等于500镑,因为前者的商品包含200个工人的劳动,而后者的商品每年只包含100个工人的劳动。其次,棉织厂主第一年由于制造机器(机器无偿地吸收了制造机器的工人的1/5的劳动时间)而赚得的100镑利润(剩余价值),要到第二年才得以实现,因为只有到那时他才在棉布的价值中同时实现机器的价值。但是难题也就发生在这里。棉织厂主出卖商品,价格高于1000镑,也就是高于他的商品所包含的价值,而租地农场主按500镑出卖谷物,也就是根据假定按谷物的价值出卖。因此,如果只有这两个人进行交换,工厂主从租地农场主那里换得谷物,租地农场主从工厂主那里换得棉布,那就好比租地农场主低于商品的价值出卖商品,赚得的利润少于25%,工厂主则高于棉布的价值出卖棉布。让我们把李嘉图多余地塞进他的例子的两个资本家(毛织厂主和棉织厂主)撇开不谈,把他的例子改变一下,只讲棉织厂主。对我们所考察的例证来说,到现在为止这种双重计算是毫无用处的。所以:

\begin{quote}{“但是,它们〈棉布〉的价值将不止是谷物价值的两倍,因为……棉织厂主的资本已经加上了这个资本在第一年的利润,而租地农场主却把利润花费和享受掉了。”}\end{quote}

(最后这一句资产阶级的粉饰话这里在理论上是毫无意义的。道德方面的考虑同这里所考察的问题没有任何关系。)

\begin{quote}{“因此,由于他们的资本的耐久程度不同,或者也可以说,由于一批商品在能够进入市场以前必须经历的时间不同,这些商品的价值同花费在它们上面的劳动量不会恰好成比例。在这种情况下,商品价值的比例不是2∶1,而是稍大一些,以便补偿价值较大的一种商品在能够进入市场以前必须经历的较长时间。”(第30页)}\end{quote}

如果工厂主按商品的价值出卖商品,他就会把它卖1000镑,比谷物贵一倍,因为这个商品包含的劳动多一倍:500镑是机器形式的积累劳动(其中有100镑工厂主没有支付过代价),另外500镑是织布劳动,其中又有100镑工厂主没有支付过代价。但工厂主是这样计算的:第一年我花费400镑,从而通过剥削工人,我制成了一台价值500镑的机器。因此,我赚得了25%的利润。第二年我花费900镑,即500镑为上述机器,又有400镑为劳动。为了再赚得25%的利润,我就必须按1125镑出卖棉布,也就是高于它的价值125镑。因为这125镑并不代表棉布中包含的劳动,既不代表第一年的积累劳动,也不代表第二年的新加劳动。棉布中包含的全部劳动量只等于1000镑。另一方面,假定他们两人互相交换自己的商品,或者说,资本家中有半数处于棉织厂主的地位,另有半数处于租地农场主的地位。前一半资本家从哪里获得必须支付给他们的这125镑呢?从什么基金呢?显然,只有从后一半资本家那里获得。但是这样一来,后一半资本家显然就会得不到25%的利润。因而,前一半资本家以一般利润率为借口欺骗了后一半资本家,而实际上,工厂主的利润率是25%,租地农场主的利润率却低于25%。所以,发生的必定是另一种情况。

为了使这个例证更加正确和更加明显起见,我们假定租地农场主在第二年花费900镑。这样,在利润率为25%的情况下,他在第一年从他所花费的400镑赚得100镑利润,第二年赚得225镑,合计325镑。相反,工厂主在第一年从400镑赚得25%的利润,但在第二年从900镑只赚得100镑(因为投在机器上的500镑是不提供剩余价值的,只有花在工资上的那400镑才提供剩余价值),只占[11+(1/9)]%。或者,让租地农场主再花费400镑;这样他在第一年和第二年都赚得25%;两年合计,从花费的800镑赚得200镑,就是说,也是25%。相反,工厂主在第一年赚得25%,第二年赚得[11+(1/9)]%,两年合计,从花费的1300镑赚得200镑,也就是[15+(5/13)]%。所以,在利润平均化的时候,工厂主将赚得利润[20+(5/26)]%,租地农场主赚得的也将这么多。\endnote{在租地农场主和工厂主所花资本相等的情况下,平均利润是[20+(5/26)]%。如果考虑到所花资本的量的不同——租地农场主800镑,工厂主1300镑(总共2100镑),那末,在两者的总利润等于400镑的情况下,平均利润是(400×100)/2100=[19+(1/21)]%。——第206页。}这就是平均利润。这样一来,[在第二年]租地农场主的商品的费用价格就会低于500镑,而工厂主的商品的费用价格却高于1000镑。

[535]无论如何,工厂主在这里第一年花费400镑,第二年花费900镑,而租地农场主每次都只花费400镑。假如工厂主不是生产棉布而是建造房屋(如果他是个建筑业者),那末,第一年末在未建成的房子中包含500镑,他还要在劳动上再花400镑,才能把房子建成。租地农场主的资本一年周转一次,他可以从100镑利润中提取一部分,比如说,50镑,再作为资本,重新花在劳动上,这一点工厂主在上述假定的情况下是做不到的。为了使利润率在两种情况下都一样,一个人的商品就必须高于其价值出卖,而另一个人的商品则必须低于其价值出卖。因为竞争力求把价值平均化为费用价格,所以,实际上发生的也就是这种情况。

但是,李嘉图说,相对价值在这里所以发生变动,是“由于资本的耐久程度不同”,或者说,“由于一批商品在能够进入市场以前必须经历的时间不同”,那是错误的。相反,正是预先假定的一般利润率,不顾流通过程所决定的价值差别,造成相等的、和这些仅仅由劳动时间决定的价值不同的费用价格。

李嘉图的例证分为两个例子。在后一个例子中,资本的耐久程度,或者说,资本作为固定资本的性质,完全没有包括进来。谈的只是几笔资本大小不等,但花费在工资上的资本量相等,花费的可变资本相等,而利润应该相等,虽然剩余价值和价值必然不等。

在第一个例子中,资本的耐久程度也没有包括进来。谈的是较长的劳动过程,是商品在能够进入流通之前,直到它制成为止,有较长时间停留在生产领域。这里,在李嘉图那里,工厂主第二年使用的资本也比租地农场主使用的大,虽然他在两年内花费的可变资本和租地农场主一样。但是,租地农场主由于他的商品停留在劳动过程中的时间较短,由于商品转化为货币较早,第二年可以使用较大的可变资本。此外,利润中作为收入来消费的部分,在租地农场主那里,第一年末就可以消费,在工厂主那里,要到第二年末才可以消费。因此,工厂主必须支出额外的资本来维持自己的生活,为自己预付生活费用。其实,这里这一切都取决于在一年内周转的资本在多大程度上把自己的利润再资本化,因而,取决于生产出来的利润的实际量,以便第二种情况可以得到弥补,利润可以平均化。在什么都没有的地方,也就没有什么可以平均化。这里,各个资本生产的价值,因而它们生产的剩余价值,因而它们生产的利润,又不是同资本量成比例的。要使它们成比例,就必须有不同于价值的费用价格存在。

李嘉图还提出了第三个例证,但这个例证同第一个例证中的第一个例子完全一致,连一个新鲜的字都没有。

\begin{quote}{“假定我花1000镑雇用20个工人在一年内生产一种商品,年终,再花1000镑雇用20个工人在下一年内完成或改进这种商品,在两年末了,我把商品运到市场上去。如果利润为10%,我的商品就必须卖2310镑,因为我在第一年用了1000镑资本,第二年用了2100镑资本。另一个人使用完全相同的劳动量,但是全部用在第一年;他花2000镑雇用40个工人,在第一年末,就把他的商品按10%的利润出卖,也就是卖2200镑。于是,这里就有两种商品,所耗费的劳动量完全相同,其中一种卖2310镑,另一种卖2200镑。这个例子表面上和前一个不同,但实际上是相同的。”(第34—35页)}\end{quote}

这个例子和前一个不仅“实际上”相同,而且“表面上”也相同,只有一点不同,就是在前一个例子中,商品叫做“机器”,在这里,直接就叫“商品”。在第一个例子中,工厂主第一年花费400镑,第二年花费900镑,这一次,第一年花费1000镑,第二年花费2100镑。在前一个例子中,租地农场主第一年花费400镑,第二年也花费400镑。这一次,第二个企业主第一年花费2000镑,第二年什么也不花费。这就是全部差别。但是两个例子的《fabuladocet》\authornote{直译是:“寓言教导说”;转意是:由某事物得出的(劝谕性的)结论,“寓意”。——编者注}都在于:企业主中有一个人第二年花费了第一年的全部产品(包括剩余价值在内)加上一个追加额。

这些例子的笨拙,表明李嘉图正在攻一个难关,这个难关,他自己不清楚,更说不上把它攻破了。笨拙之处在于:第一个例证的第一个例子应该把资本的耐久程度包括进来;但结果完全不是这样;李嘉图自己使这样做成为不可能,因为他不让固定资本的任何部分作为损耗加入商品的价值,也就是恰恰漏掉了表现固定资本所特有的流通方式的因素。李嘉图表明的只是,由于劳动过程经历的时间较长,同劳动过程经历的时间较短的地方相比,要使用更大的资本。第三个例子应该说明与此不同的情况,但是实际上说明的是一回事。第一个[536]例证的第二个例子应该表明,由于固定资本的比例不同会造成什么样的差别。它不是这样,而是仅仅表明,大小不等但花在工资上的部分相等的两笔资本会有什么样的差别。并且在这里,工厂主没有棉花和棉纱,租地农场主没有种子和工具就能进行活动!这个例证之所以必然站不住脚,甚至荒诞无稽,就是因为它内在地是含糊不清的。

\tsubsubsectionnonum{[(b)李嘉图把费用价格同价值混淆起来,由此产生了他的价值理论中的矛盾。他不懂利润率平均化和价值转化为费用价格的过程]}

最后,李嘉图说出了所有这些例证的实际结论:

\begin{quote}{“在这两种情况下,价值的差额都是由于利润作为资本积累起来而造成的,这个差额只不过是对利润被扣留的那段时间的一种公正的补偿〈好象这里的问题在于公正〉。”(第35页)}\end{quote}

这无非是说,一笔资本,不管它的特殊流通时间如何,也完全不管等量资本在不同生产部门中由于资本的有机组成部分的比例不同(撇开流通过程不说)必然生产出不同的剩余价值,在一定的流通时间内,比如说在一年内,必定提供10%。

李嘉图本应从自己的例证中作出如下的结论:

[第一,]等量资本生产的商品价值不等,从而提供的剩余价值或利润也不等,因为价值决定于劳动时间,而一笔资本所实现的劳动时间量,不取决于资本的绝对量,而取决于可变资本量即花费在工资上的资本量。第二,即使假定等量资本生产的价值相等(虽然在大多数情况下,生产领域中的不等是同流通领域中的不等相一致的),等量资本占有同量无酬劳动并把它转化为货币所需要的那段时间,也还是由于资本的流通过程不同而有所不同。这就使等量资本在不同生产部门中在一定时间内必须提供的价值、剩余价值和利润产生了第二个差别。

因此,如果利润按其比如说在一年内对资本的百分率计算必须相等,从而等量资本在同一时间内提供的利润必须相等,那末,商品的价格必然不同于商品的价值。一切商品的这些费用价格加在一起,其总和将等于这一切商品的价值。同样,总利润将等于这些资本加在一起比如说在一年内提供的总剩余价值。如果我们不以价值规定为基础,那末,平均利润,从而费用价格,就都成了纯粹想象的、没有依据的东西。各个不同生产部门的剩余价值的平均化丝毫不改变这个总剩余价值的绝对量,它所改变的只是剩余价值在不同生产部门中的分配。但是,这个剩余价值本身的规定,只有来自价值决定于劳动时间这一规定。没有这一规定,平均利润就是无中生有的平均,就是纯粹的幻想。那样的话,平均利润就既可以是10%,也可以是1000%。

李嘉图的一切例证只有一个用处,就是帮助他偷偷地把一般利润率作为前提引进来。这在第一章(《论价值》)就发生了,而表面上,李嘉图是在第五章才考察工资,在第六章才考察利润。怎样单纯从商品的“价值”规定得出商品所包含的剩余价值、利润、甚至一般利润率,——这一点对李嘉图来说仍然是一个秘密。他在上述例证中实际证明的唯一东西是:商品的价格只要决定于一般利润率,它就根本不同于商品的价值。他所以会看出这个差别,是因为他预先就把利润率当作规律来假定。我们看到,如果说人们责备李嘉图过于抽象,那末相反的责备倒是公正的,这就是:他缺乏抽象力,他在考察商品价值时无法忘掉利润这个从竞争领域来到他面前的事实。

因为李嘉图不是从价值规定本身出发来阐述费用价格和价值的差别,而是承认那些与劳动时间无关的影响决定“价值”本身(这里他如果坚持“绝对价值”,或者说,“实际价值”,或者直接说,“价值”这样的概念,倒是合适的)并且有时使价值规律失效,所以他的反对者如马尔萨斯之流就抓住这一点来攻击他的全部[537]价值理论。在这里马尔萨斯正确地指出,不同部门中资本有机组成部分之间的差别和资本周转时间的差别是随着生产的进步而发展的,结果就必然要得出亚·斯密的观点,认为价值决定于劳动时间这一规定不再适用于“文明”时代了。(并见托伦斯。)另一方面,李嘉图的门徒为了使这些现象符合于基本原则,就求助于最可怜的烦琐哲学的臆造(见[詹姆斯·]穆勒和可怜的饶舌家麦克库洛赫)\endnote{关于马尔萨斯、托伦斯、詹姆斯·穆勒和麦克库洛赫的观点,见本卷第3册有关章节。——第211页。}。

李嘉图不去研究从他自己的例证中得出的结论,——就是完全不管工资提高还是降低,假定工资不变,商品的费用价格如果由同一个利润百分率决定,就必然不同于商品的价值,——却在这一节里转而考察工资的提高或降低对那已经由价值平均化而成的费用价格所产生的影响。

问题的实质本身是非常简单的。

租地农场主花费5000镑,获得利润10%,他的商品的货币价格为5500镑。如果由于工资提高并引起利润下降,利润下降1%,即从10%下降到9%,那末,租地农场主仍然会按5500镑出卖自己的商品(因为假定他已把自己的全部资本花在工资上)。但是,在这5500镑中,他的利润已经不是500,而只是454+(14/109)。工厂主的资本包括用于机器的5500镑和用于劳动的5000镑。后面这5000镑仍旧表现为5500镑,不过他现在花费的不是5000镑,而是5045+(95/109)镑,他和租地农场主一样,从这笔资本只赚得利润454+(14/109)镑。而且这5500镑固定资本,他不能再按10%来计算利润,即550镑,只能按9%来计算利润,即495镑。因此,他的商品将不是卖6050镑,而是卖5995镑。这样一来,工资提高的结果,租地农场主的商品的货币价格仍旧不变,而工厂主的商品的货币价格却下跌了;因此,同工厂主的商品价值相比,租地农场主的商品价值提高了。全部关键在于:如果工厂主按以前的价值出卖自己的商品,他赚得的利润就会高于平均利润,因为工资的提高只是直接影响到花在工资上的那部分资本。在这个例证中,已经预先假定了由10%的平均利润调节的、不同于商品价值的费用价格。李嘉图的问题是:利润的提高或降低如何根据固定资本和流动资本在全部资本中的不同比例来影响费用价格。这个例证(李嘉图的著作第31—32页)同价值转化为费用价格这一根本问题毫无关系。但这个例证还是不错的,因为李嘉图在这里一般指出了:在资本构成相同的条件下,工资的提高只会引起利润的降低,不会影响商品的价值(这和庸俗观点相反),在资本构成不同的条件下,它只会引起某些商品价格的下跌,而不象庸俗见解所认为的那样,会引起一切商品价格的上涨。在李嘉图的例子中,商品价格的下跌是利润率降低的结果,或者[在李嘉图看来]也可以说,是工资提高的结果。在工厂主的例子中,商品的费用价格的很大一部分决定于他按固定资本计算的平均利润。因此,如果由于工资的提高或降低,这种利润率会降低或提高,那末,这些商品的价格就会相应地(同那部分由于按固定资本计算的利润而产生的价格相应地)下跌或上涨。这一点也适用于“要经过较长时期才流回的流动资本以及相反的情况”。(麦克库洛赫)[《政治经济学原理》1825年爱丁堡版第300页]如果使用较少可变资本的资本家继续按原来的利润率把自己的固定资本计算到商品的价格中去,那末他们的利润率就会提高,而且,同那些有较大部分资本由可变资本构成的资本家相比,他们使用的固定资本越多,他们的利润率就提得越高。竞争会把这种情况拉平。

\begin{quote}{饶舌家麦克说:“李嘉图是第一个研究在生产商品所使用的资本具有不同耐久程度时工资的波动对商品价值的影响的人。李嘉图不仅指出,工资的提高不可能使一切商品的价格都上涨,而且指出,在许多情况下,工资的提高必然引起价格的下跌,而工资的降低必然引起价格的上涨。”(麦克库洛赫《政治经济学原理》1825年爱丁堡版第298—299页)}\end{quote}

李嘉图证明其论点的办法是,第一,假定有一种由一般利润率调节的费用价格。

第二,指出:“劳动的价值提高,利润就不能不降低。”(第31页)

可见,在第一章(《论价值》)中已经假定了第五章(《论工资》)和第六章(《论利润》)中应该从《论价值》那一章引伸出来的那些规律。顺便指出,[538]李嘉图作出了完全错误的结论,他说,因为“劳动的价值提高,利润就不能不降低”,所以,利润提高,劳动的价值就不能不降低。第一个规律与剩余价值有关。但是,因为利润是剩余价值同全部预付资本之比,所以,在劳动价值不变的条件下,如果不变资本的价值降低,利润就可能上涨。李嘉图根本混淆了剩余价值和利润。由此他就得出了关于利润和利润率的错误规律。

最后这个例证的一般结论是这样的:

\begin{quote}{“由劳动价值的提高或降低〈或者也可以说,由利润率的降低或提高〉引起的商品相对价值的变动的幅度,将取决于固定资本在已花费的全部资本中所占的比例。一切用很贵的机器或在很贵的建筑物里生产的,或者在能够进入市场以前必须经历长时间的商品的相对价值会降低,而一切主要由劳动生产的,或能迅速进入市场的商品的相对价值则会提高。”(第32页)}\end{quote}

李嘉图又回到这一研究中真正唯一使他关心的问题上来了。他说,由工资的提高或降低引起的商品费用价格的这些变动,同由商品价值的变动{李嘉图远远不会用这些恰当的术语来表达这一事实真相}即由生产商品所使用的劳动量的变动引起的商品费用价格的变动相比,是微不足道的。因此,他说,完全可以把这一点“撇开不谈”,这样,价值规律就是在实际上也仍然是正确的。(李嘉图本应补充一句:没有决定于劳动时间的价值,费用价格本身就仍然是无法解释的。)这就是李嘉图的研究的真正进程。事实上,很明显,尽管商品的价值转化为费用价格,——既然我们已经假定费用价格存在——{应当把这种费用价格同市场价格区别开来;费用价格是不同部门的商品的平均市场价格。因为同一生产领域的商品的市场价格决定于这一领域中等的、平均的生产条件下生产出来的商品的价格,所以市场价格本身就包含着一个平均数。市场价格决不是象李嘉图在考察地租时假定的那样,决定于最坏条件下生产出来的商品的价格。因为平均需求取决于一定的价格,甚至在谷物上也是这样。因此,进入市场的一定量商品不会高于这个价格出卖。否则需求就会下降。所以,不是在平均条件下而是在低于平均条件的情况下生产商品的人,往往只得不仅低于商品的价值,而且低于商品的费用价格出卖自己的商品},只要费用价格的变动不是由利润率的持续降低或提高,不是由经过多年才能确定的利润率的持续变动引起,这种变动就只能仅仅归因于这些商品的价值的变动,归因于生产这些商品所必需的劳动时间量的变动。

\begin{quote}{“但是,读者应当注意,商品的变动〈即商品费用价格的变动,或者照李嘉图的说法,商品相对价值的变动〉的这一原因所产生的影响是比较小的……商品价值变动的另一重要原因,即生产商品所必需的劳动量的增减,情况却不是这样……持久的利润率的任何大变动,都是经过多年才发生影响的那些原因所造成的结果,而生产商品所必需的劳动量的变动却是天天都发生的。在机器、工具、建筑物以及开采或种植原料方面的每一改良都可以节省劳动,使我们在利用这种改良来生产商品时,能够更加容易地把商品生产出来,结果商品的价值就会发生变动。可见,在研究商品价值变动的原因时,虽然完全不考虑劳动价值的提高或降低所产生的影响是错误的,但认为这种影响具有很大意义也是不正确的。”(第32—33页)}\end{quote}

所以,李嘉图就把劳动价值的变动撇开不谈。

第一章《论价值》第四节全节都非常混乱,因此,虽然李嘉图在开头一段就声称,他要考察由于资本构成不同工资的提高或降低引起的商品价值变动的影响,其实他只是顺便举例证明这一点,相反,第四节的主要部分实际上是罗列了许多例证来证明:完全不管工资提高或降低——在他自己假定的工资不变的条件下——而且甚至不管固定资本和流动资本的比例不同,既然假定[539]一般利润率存在,就必然得出不同于商品价值的费用价格。他在这一节的结尾又把这一点忘记了。

他在第四节用这样的话预告他要研究的问题:

\begin{quote}{“固定资本耐久程度的这种差别,和这两种资本可能结合的比例的这种多样性,在生产商品所必需的劳动量增减之外,又给商品相对价值的变动提供了另一个原因,这个原因就是劳动价值的提高或降低。”(第25—26页)}\end{quote}

实际上,他用自己的例证首先证明的是:只有一般利润率才能使两种资本(即可变资本和不变资本)的不同结合产生这种使商品价格不同于商品价值的影响;因此,这些变动的原因,正是一般利润率,而不是劳动价值,劳动价值在这里假定是不变的。然后,第二步,他才假定有一种因一般利润率的存在而已经不同于价值的费用价格,并研究劳动价值的变动怎样影响费用价格。第一个主要问题,他不研究,他忘得一干二净,他在这一节结尾说的还是这一节开头所说的话:

\begin{quote}{“本节已经证明,在劳动量没有任何变动的条件下,单是劳动价值提高,就会使那些在生产时使用固定资本的商品的交换价值降低;固定资本量越大,降低的幅度也越大。”(第35页)}\end{quote}

在下一节即第五节(第一章),他继续沿着这条线走下去,就是说,他仅仅研究:如果在两个不同生产部门中有两笔等量资本,不是它们的固定资本和流动资本的比例不同,而是“固定资本耐久程度不同”,或者说,“资本流回所有者手里的速度不同”,那末,由于劳动价值或者说工资的变动,商品的费用价格会发生什么变动。在第四节,对于由于一般利润率而产生的费用价格和价值之间的差别,还有一点正确的猜想,在第五节,却再也看不到了。考察的只是关于费用价格本身的变动的次要问题。因此,这一节除了偶然触及的从流通过程产生的资本形式差别的问题以外,实际上几乎引不起什么理论兴趣。

\begin{quote}{“固定资本的耐久程度越低,它在性质上就越接近于流动资本。它将在较短时间内被消费掉,它的价值也将在较短时间内被再生产出来,以便保持工厂主的资本。”(第36页)}\end{quote}

可见,李嘉图把资本的较低耐久程度以及一般说来固定资本和流动资本的差别都归结为再生产时间的差别。这无疑是一个极为重要的规定,但决不是唯一的规定。固定资本全部加入劳动过程,但只是陆续地、一部分一部分地加入价值形成过程。这是固定资本和流动资本的流通形式的另一个主要差别。其次,固定资本必然只以其交换价值加入流通过程,而它的使用价值则在劳动过程中消费掉,从来不离开劳动过程。这是流通形式的又一个重要差别。流通形式的这两个差别也同流通时间有关,但它们与[资本的耐久]程度和流通时间的差别决不是等同的。

耐久程度较低的资本需要较多的经常劳动,

\begin{quote}{“以保持其原来的有效状态,但这样使用的劳动可以看作是实际花费在成品上的劳动,因此,这种成品必然具有同这种劳动成比例的价值”。(第36—37页)“如果机器的损耗大,为维持其有效状态所需要的劳动量为每年50个工人,那末,我就要求为我的商品提供追加价格,其数额等于雇用50个工人来生产其他商品而完全不使用机器的其他任何工厂主所得的价格。但是,工资的提高,对于用损耗得快的机器生产的商品和用损耗得慢的机器生产的商品,影响是不同的。在生产前一种商品时,有大量劳动会不断转移到所生产的商品上去}\end{quote}

{但是,李嘉图以他的一般利润率为前提,看不到同时也有相对大量的剩余劳动会不断转移到商品上去},

\begin{quote}{而在生产后一种商品时,这样转移的劳动却很少}\end{quote}

{因此,剩余劳动也很少,就是说,如果商品按其价值交换,[剩余]价值会少得多}。

\begin{quote}{因此,只要工资有所提高,或者也可以说,[540]只要利润有所降低,那些用耐久程度较高的资本生产的商品的相对价值就会降低,而那些用损耗较快的资本生产的商品的价值则会相应地提高。工资下降的作用则恰好相反。”(第37—38页)}\end{quote}

换句话说,同使用耐久程度较高的资本的工厂主相比,使用耐久程度较低的固定资本的工厂主,使用的固定资本较少,而花在工资上的资本较多。因此,这个例子同前面所说的那个例子是一致的,前面说的是,如果一笔资本比另一笔资本使用的固定资本相对地即在比例上较多,工资的变动对这两笔资本会发生什么影响。这里没有什么新东西。

李嘉图还谈到机器(第38—40页),关于这个问题,到考察第三十一章(《论机器》)时再作评论\authornote{见本册第628—630页。——编者注}。

值得注意的是,李嘉图在第五节结尾已接近于对事物的正确看法,几乎找到了有关的字句,可是马上就离开了正确的道路,他在接近于正确观点(我们就要引述这方面的话)之后,又回到支配着他的观念上去,即回到劳动价值的变动对费用价格的影响上去,并以对这个次要问题的结论结束了他的研究。

有关段落是这样说的:

\begin{quote}{“因此,我们可以看到,在还没有大量使用机器或耐久资本的社会发展早期阶段,用等量资本生产的商品会具有几乎相等的价值;只是由于生产它们所必需的劳动有了增减,这些商品彼此相对地说才会提高或降低}\end{quote}

{后面半句话说得不好;并且它不说价值,而说商品,这里除非是指商品的价格,否则就没有任何意义;因为说价值和劳动时间成比例地降低,就等于说价值随着自己的提高或降低而提高或降低};

\begin{quote}{但是,在采用了这些昂贵而耐久的工具之后,使用等量资本生产的商品就会具有极不相等的价值,虽然由于生产它们所必需的劳动的增减,它们的价值彼此相对地说仍然会提高或降低,可是由于工资和利润的提高或降低,它们还会发生另一种变动,虽然是较小的变动。因为卖5000镑的商品所用的资本量可能等于生产其他卖10000镑的商品所用的资本量,所以生产这两种商品所赚得的利润也会相等;但是,如果商品的价格不是随着利润率的提高或降低而变动,这些利润就会不相等。”(第40—41页)}\end{quote}

实际上李嘉图在这里是说:

如果等量资本的有机组成部分的比例相同,如果它们花费在工资和劳动条件上的份额相同,它们就会生产价值相等的商品。在这种情况下,它们所生产的商品中体现着等量的劳动,也就是相等的价值{流通过程可能带来的差别撇开不谈}。相反,如果等量资本的有机构成不同,尤其是如果它们的作为固定资本存在的那一部分同花费在工资上的那一部分的比例大不相同,它们就会生产价值大不相等的商品。第一,固定资本只有一部分作为价值组成部分加入商品,因此,根据生产商品时使用的固定资本的多少不同,价值量就已经大不相同。第二,[在固定资本多的情况下]花费在工资上的那一部分——按其在等量资本中所占的百分比计算,——就会少得多,因而体现在商品中的全部[新加]劳动也少得多,形成剩余价值的剩余劳动也少得多{已知工作日长度相同}。所以,既然这些等量资本生产的商品具有不等的价值,在这些不等的价值中包含不等的剩余价值,因而也包含不等的利润,如果这些资本由于数量相等而提供的利润也必定相等,那末,商品的价格(既然这种价格决定于一定费用的一般利润率)就必然和商品的价值大不相同。由此得出的结论不是价值改变了它的本性,而是费用价格不同于价值。李嘉图没有得出这个结论,这是令人奇怪的,特别是因为他的确看到了,即使在假定存在费用价格(它决定于一般利润率)的条件下,利润率(或工资率)的变动也一定要引起这种费用价格的变动,这样才能使[541]各个生产部门中的利润率保持一致。因此,一般利润率的确立,必然会使不等的价值发生更大的变动,因为这种一般利润率无非是等量资本生产的各种不同商品所包含的不同的剩余价值率的平均化。

李嘉图对于商品的费用和价值之间、商品的费用价格和价值之间的差别虽然没有阐述,没有理解,但是无论如何,他自己实际上已经确认了这种差别,在这之后,他结束他的论断说:

\begin{quote}{“马尔萨斯先生似乎认为,把某物的费用和价值等同起来,是我的学说的一部分。如果他说的费用是指包括利润在内的‘生产费用’〈就是指支出加由一般利润率决定的利润〉,那确是如此。”(第46页注)}\end{quote}

后来,李嘉图就带着他自己驳倒了的这种把费用价格和价值混淆起来的错误观点去考察地租。

李嘉图在第一章第六节谈到劳动价值的变动对金的费用价格的影响时说:

\begin{quote}{“难道我们不能把金看成这样一种商品,它在生产时所用的两种资本的比例同大多数商品生产时所用的两种资本的平均比例最接近吗?难道我们不能把这种比例看成同两个极端(一个极端是固定资本用得少,另一个极端是劳动用得少)保持相等距离而成为两者之间的中数吗?”(第44页)}\end{quote}

李嘉图的这些话,不如说适用于这样一些商品,这些商品的价值中各个不同有机组成部分的比例是平均比例,而且这些商品的流通时间和再生产时间也是平均时间。对这些商品来说,费用价格和价值是一致的,因为这些商品的平均利润和它们的实际剩余价值是一致的,但是只有这些商品才是这种情况。

第一章第四、五节,关于劳动价值的变动对“相对价值”的影响这个问题——这同价值因平均利润率而转化为费用价格的问题相比,(在理论上)是一个次要问题——的考察尽管有很大缺陷,但李嘉图由此却得出了十分重要的结论,推翻了自亚·斯密以来一直流传下来的主要错误之一,即认为工资的提高不是使利润降低,而是使商品的价格上涨。诚然,这一点已经包含在价值概念本身了,并且决不会由于价值转化为费用价格而有所改变,因为后者仅仅涉及总资本所赚得的剩余价值在不同部门之间或在不同生产领域的各个资本之间的分配。但是,李嘉图强调指出了这个问题,并且证明情况甚至相反,这仍然是有重要意义的。因此,他在第一章第六节公正地说:

\begin{quote}{“在结束这个题目之前,指出一点可能是适当的,就是亚当·斯密和一切追随他的著作家,据我所知,无一例外地都认为,劳动价格的上涨,必然会引起一切商品价格的上涨。”}\end{quote}

{这是和斯密的第二种价值规定相适应的,按照这个规定,价值等于一个商品能够买到的劳动量。}

\begin{quote}{“我希望,我已成功地证明了这种意见是毫无根据的,当工资提高时,只有比用来计算价格的中介物使用固定资本少的那些商品的价格才会上涨〈这里,“相对价值”等于价值的货币表现〉,而一切使用固定资本较多的商品的价格都必定下跌。反之,当工资降低时,只有比用来计算价格的中介物使用固定资本少的那些商品的价格才会下跌;而一切使用固定资本较多的商品的价格都必定上涨。”(第45页)}\end{quote}

这对于货币价格,看来是错误的。如果金的价值由于随便什么原因提高或降低了,那末这种[提高或]降低会同样地涉及用金计价的一切商品。因此,金尽管本身具有可变性,却表现为商品之间的相对不变的中介物。既然如此,那就绝对不能理解,同商品相比,在生产金时固定资本和流动资本的任何相对结合,能引起什么差别。但是在这里,李嘉图的错误前提也就出现了,他认为,货币只要用作流通手段,就是作为商品来同商品交换。商品在货币使它们流通以前,就以货币来计价了。我们假定中介物不是金,而是小麦。如果,比如说,按可变资本和不变资本的比例来说,加入小麦这种商品的可变资本超过平均水平,因此,由于工资提高,小麦的生产价格相对地上涨,那末,一切商品就都按具有较高“相对价值”的小麦来计价。那些有较多固定资本加入的商品,就会表现为比以前少的小麦,这不是因为这些商品的特殊价格同小麦相比下跌了,而是因为价格普遍下跌了。如果一个商品包含的同积累劳动相对立的[活]劳动,恰好和小麦包含的一样多,那末这一商品价格的上涨就会这样显示出来:它[542]与一个同小麦相比价格已经下跌的商品比较起来,表现为较多的小麦。如果引起小麦价格上涨的那些原因,也引起比如说衣服的价格上涨,那末,虽然衣服不会表现为比以前多的小麦,但是,同小麦相比价格已经下跌的那些商品,比如说棉布,就会表现为较少的小麦。棉布和衣服的价格差额,就会在小麦这个中介物上表现出来。

但是,李嘉图的意思不是这样。他的意思是:由于工资提高,同棉布相比,而不是同衣服相比,小麦的价格会上涨;因此,衣服就会按小麦原有的价格同小麦交换,而棉布则按小麦上涨了的价格同小麦交换。说什么英国工资价格的变动会促使工资没有提高的地方,比如说加利福尼亚的金的费用价格发生变动,这种假定本身就是极端荒谬的。价值通过劳动时间来平均化,尤其是费用价格通过一般利润率来平均化,在不同国家之间不是以这种直接的形式进行的。但就拿小麦这种国内产品来说吧。假如一夸特小麦的价格由40先令上涨到50先令,即上涨25%。如果衣服的价格也上涨25%,那末一件衣服仍旧值1夸特小麦。如果棉布的价格下降25%,那末过去值1夸特小麦的同样数量的棉布,现在只值6蒲式耳小麦\endnote{英国1夸特(散体量等于290.9公升)等于8蒲式耳。——第223页。}。这种用小麦表现出来的数字,准确地代表了棉布价格和衣服价格之比,因为棉布和衣服是用同一个尺度1夸特小麦来计量的。

此外,李嘉图的观点还有更荒谬的一面。用作价值尺度因而用作货币的商品的价格是根本不存在的;因为不然的话,我除了用作货币的商品之外还必须有第二种用作货币的商品——双重的价值尺度。货币的相对价值是以一切商品的无数价格表现出来的;因为,在商品交换价值借以表现为货币的这许多价格的每一个价格中,货币的交换价值都表现为商品的使用价值。因此,谈不上货币价格上涨或下跌的问题。我可以说:货币的小麦价格或货币的衣服价格保持不变,而货币的棉布价格上涨了,这等于说棉布的货币价格下跌了。但我不能说,货币的价格上涨或下跌了。可是,李嘉图实际上认为,比如说,货币的棉布价格所以上涨,或者说,棉布的货币价格所以下跌,正是因为同棉布相比,货币的相对价值提高了,而同衣服或小麦相比,货币却保持其原有价值。这样一来,这两种价值就是用不同的尺度来计量了。

这第六节(《论不变的价值尺度》)论述的是“价值尺度”,但其中没有什么重要的东西。对价值,价值的内在尺度——劳动时间——同商品价值的外在尺度的必要性之间的联系,根本不了解,甚至没有把它当作问题提出来。

第六节一开头就表现了肤浅的论述方法:

\begin{quote}{“当商品的相对价值发生变动时,最好有一个方法能确定,哪种商品的实际价值降低了,哪种商品的实际价值提高了。要做到这一点,只有把它们逐一同某种不变的、本身不会发生其他商品所发生的变动的标准尺度相比较。”但是“没有一种商品本身不发生……同样的变动,就是说,没有一种商品在生产时所需要的劳动能够不有所增减”。(第41—42页)}\end{quote}

但是,即使有这样一种商品,工资提高或降低的影响,以及固定资本和流动资本的不同结合、固定资本的不同耐久程度、商品在能够进入市场以前必须经历的时间不同等等的影响,也都会部分地妨碍它

\begin{quote}{“成为我们能够用来准确地确定一切其他物品的价值变动的一种完美的价值尺度”。“对于在和它本身完全相同的条件下生产出来的一切物品来说,它是完美的价值尺度,但对其他物品来说就不是了。”(第43页)}\end{quote}

换句话说,在这两类“其他物品”中前一类的价格发生变动时,我们可以(如果货币的价值不提高或降低的话)说,这种变动是因为“它们的价值”有了提高或降低,即生产它们所需要的劳动时间有了增减。至于其他物品,我们就无法知道,它们的货币价格发生“变动”是否由于其他原因等等。后面(以后考察货币理论时)还要回过头来谈这些很不恰当的论断。

第一章第七节。除了关于“相对”工资、利润和地租的重要学说(这方面后面还要回过头来谈\authornote{见本册第476—482页。——编者注})以外,这一节只包含这样一个论点:在货币价值降低或提高时,工资、利润和地租的相应的提高或降低,决不会改变它们之间的比例,而只会改变它们的货币表现。如果同一商品的价值表现为两倍的镑数,那末,转化为利润、工资或地租的那一部分价值也增加一倍。但是,这三个部分互相之间的比例和它们所代表的实际价值仍然不变。同样,如果利润表现为两倍的镑数,那末100镑现在也就表现为200镑;因此,利润和资本之间的比例,即利润率也仍然不变。货币表现的变动同时影响利润和资本,就象它同时影响利润、工资和地租一样。这对于地租也是适用的,只要地租不是按英亩计算,而是按预付在耕种土地等等上面的资本计算。总之,在这种场合,变动不是发生在商品上,等等:

\begin{quote}{“由这种原因造成的工资提高,当然不可避免地会引起商品价格的同时上涨;但是,在这种情况下我们会发现,劳动和一切商品之间的比例没有变动,变动的只是货币。”(第47页)}\end{quote}

\tsubsectionnonum{[(5)]平均价格或费用价格和市场价格}

\tsubsubsectionnonum{[(a)引言:个别价值和市场价值;市场价值和市场价格]}

[543]李嘉图为了阐明级差地租理论,在第二章(《论地租》)提出以下论点:

\begin{quote}{“一切商品,不论是工业品、矿产品还是土地产品,它们的交换价值始终不决定于在只是享有特殊生产便利的人才具备的最有利条件下足以把它们生产出来的较小量劳动,而决定于没有这样的便利,也就是在最不利条件下继续进行生产的人所必须花在它们生产上的较大量劳动;这里说的最不利条件,是指为了把需要的产品量生产出来而必须继续进行生产的那种最不利的条件。”(第60—61页)}\end{quote}

最后一句话不完全正确。“需要的产品量”不是一个固定的量。应当说:一定价格界限内需要的一定产品量。如果价格上涨超过了这种界限,“需要的量”就会同需求一起减少。

上述论点可以一般表达如下:商品(它是某个特殊生产领域的产品)的价值,决定于为生产这个生产领域的全部商品量即商品总额所需要的劳动,而不决定于这个生产领域内部单个资本家或企业主所需要的特殊劳动时间。这个特殊生产领域,比如说棉纺织工业的一般生产条件和一般劳动生产率,是这个领域即棉纺织工业的平均生产条件和平均劳动生产率。因此,决定比如一码棉布价值的劳动量,并不是这码棉布中包含的、这个棉织厂主花费在它上面的劳动量,而是出现在市场上的全体棉织厂主生产一码棉布所花费的平均量。单个资本家,比如棉纺织工业的资本家,进行生产的特殊条件必然分为三类。有一类人是在中等条件下进行生产;这就是说,他们进行生产的个别生产条件同这个领域的一般生产条件一致。平均比例就是他们的实际比例。他们的劳动生产率处于平均水平。他们的商品的个别价值同这些商品的一般价值一致。如果他们比如把棉布按2先令一码即按它的平均价值出卖,那末,他们就是按照他们生产的棉布在实物形式上所代表的价值出卖棉布。第二类企业主进行生产的条件比平均条件好。他们的商品的个别价值低于同种商品的一般价值。如果他们按这种一般价值出卖自己的商品,他们就是把自己的商品卖得高于它们的个别价值。最后,第三类企业主是在低于平均条件的生产条件下进行生产。

前面已经说过,这个特殊生产领域的“需要的产品量”不是一个固定的量。如果商品价值超过平均价值的一定界限,“需要的产品量”就会减少,或者说,这个量只有按照某种价格或者至少是在一定价格的界限内才是需要的。因此,最后一类企业主也有可能不得不低于自己商品的个别价值出卖商品,正如条件最好的那一类企业主总是高于自己商品的个别价值出卖商品一样。这几类中究竟由哪一类最后确定平均价值,正是取决于这几类的数量或数量的比例关系\endnote{马克思这里说的各类企业主的“数量或数量的比例关系”,是指每一类企业主运到市场的产品数量。——第227页。}。如果中等的一类在数量上占很大优势,那就由它确定平均价值。如果这一类数量少,而生产条件低于平均条件的那一类数量大,占了优势,那就由这后一类确定这个领域的产品的一般价值,虽然这还决不是说,甚至很少可能,恰好由这一类中条件最不利的个别资本家决定问题(见柯贝特的著作)\endnote{马克思指柯贝特的书《个人致富的原因和方法的研究;或贸易和投机原理的解释》(1841年伦敦版)。柯贝特在书中断言,在工业中,价格是由最好条件下生产出来的商品调节的,按照他的意见,正是这些商品占所有这种商品的绝大多数(第42—44页)。——第227页。}。

但是我们把这一点撇开不谈。一般的结果是:这种产品具有的一般价值,对所有这种产品都是相同的,不管它对每一个别商品的个别价值的比例如何。这种一般价值,就是这些商品的市场价值,就是它们进入市场时具有的价值。这种市场价值用货币表现出来就是市场价格,正如价值用货币表现出来就是价格一样。实际的市场价格,有时高于这种市场价值,有时低于这种市场价值,只是偶然同市场价值一致。但是在一定时期内,波动会互相抵销,因此可以说,实际市场价格的平均数,就是表现市场价值的市场价格。不管实际市场价格在当时按其大小来说,从数量来说是否同这种市场价值一致,市场价格总是同市场价值有一个共同的质的规定,即同一生产领域的所有在市场上的商品(自然假定它们的质是相同的)都具有同一价格,或者说,它们实际上代表这个领域的商品的一般价值。

[544]因此,李嘉图为他的地租理论提出的上述论点,他的门徒作了这样的表述:在一个市场上不可能同时存在两种不同的市场价格,或者说,同时出现在市场上的同一种产品具有同一价格,或者说,——因为这里我们可以把这种价格的偶然性撇开不谈,——具有同一市场价值。

于是,竞争——部分地是资本家之间的竞争,部分地是商品的买者同资本家的竞争以及商品的买者之间的竞争——在这里就导致这样的结果:某一特殊生产领域的每一个别商品的价值决定于这一特殊社会生产领域的商品总量所需要的社会劳动时间总量,而不决定于个别商品的个别价值,换句话说,不决定于个别商品的特殊生产者和卖者为这一个别商品花费的劳动时间。

但是,从这里自然就会得出结论:属于第一类的、生产条件比平均生产条件有利的资本家,在所有情况下都会赚得一种超额利润,就是说,他们的利润会超过这个领域的一般利润率。因此,竞争并不是通过把一个生产领域内部的各种利润平均化的办法来确立市场价值或市场价格。(市场价值和市场价格之间的差别对这里的研究没有意义,因为不管市场价格和市场价值的比例如何,生产条件的差别以及由此产生的不同利润率,对同一领域的各个资本家来说是始终存在的。)相反,竞争在这里正是通过容许有个别利润之间的差别,即各个资本家的利润之间的差别,通过容许有个别利润对该领域平均利润率的偏离,把不同的个别价值平均化为同一的、相等的、没有差别的市场价值。竞争甚至通过为那些在有利程度不同的生产条件下,因而在劳动生产率不同的条件下生产出来,因此代表个别的、不等量的劳动时间的商品确立同一的市场价值,来造成这种偏离。在比较有利的条件下生产出来的商品,同在比较不利的条件下生产出来的商品相比,包含的劳动时间较少,可是却按同一价格出卖,具有同一价值,就好比它包含了它实际上并不包含的同一劳动时间。

\tsubsubsectionnonum{[(b)李嘉图把同一生产领域内的市场价值形成过程同不同生产领域的费用价格形成过程混淆起来]}

李嘉图为了建立他的地租理论,需要两个论点,这两个论点表达的不仅不是竞争的同一种作用,而且恰恰是竞争的相反的作用。第一个论点是,同一领域的产品按同一市场价值出卖,因而竞争以强制的方式造成不同的利润率,即造成对一般利润率的偏离。第二个论点是,对一切投资来说,利润率都必须是相同的,或者说,竞争造成一般利润率。第一个规律适用于投入同一生产领域的不同的独立资本。第二个规律适用于投入不同生产领域的资本。竞争通过它的第一种作用造成市场价值,即为同一生产领域的商品造成同一价值,虽然这同一价值必然要产生不同的利润;因此,竞争不顾不同的利润率,或者不如说,利用不同的利润率,通过它的第一种作用造成同一价值。竞争通过它的第二种作用(不过,第二种作用是以另一种方式实现的;这是不同领域的资本家之间的竞争,它使资本从一个领域转移到另一个领域,而前面所说的那种竞争,只要不是在买者之间进行,则是发生在同一领域的资本之间),造成费用价格,即造成不同生产领域的同一利润率,虽然这同一利润率与价值不等的情况相矛盾,因而只有通过不同于价值的价格才能造成。

既然李嘉图本人为了建立他的地租理论需要这两者,既需要在利润率不等的情况下的相等价值或价格,又需要在价值不等的情况下的相等利润率,那末非常令人奇怪的是,他竟没有觉察到这个双重的规定,甚至在他专门论述市场价格的那一部分即第四章(《论自然价格和市场价格》),也完全没有论述市场价格或市场价值,尽管在前面引用的那段话\authornote{见本册第225页。——编者注}中,他还是把后者作为基础,来说明级差地租就是结晶为地租的超额利润。[545]相反,在第四章,他只是说明不同生产领域的价格归结为费用价格或者说平均价格,也就是说,只是说明不同生产领域的市场价值的相互关系,却没有说明每个特殊领域的市场价值的形成过程,而没有这个形成过程,就根本不存在什么市场价值。

每个特殊生产领域的市场价值,因而,每个特殊领域的市场价格(如果市场价格符合“自然价格”,就是说,它只是用货币把价值表现出来),都会提供极不相同的利润率,因为不同生产领域的等量资本(这些资本的不同流通过程产生的差别完全撇开不谈)使用不变资本和可变资本的比例极不相同,所以它们提供的剩余价值,从而它们提供的利润,也就极不相等。因此,不同市场价值平均化的结果是,在不同领域确立相同的利润率,使等量资本提供相等的平均利润,而不同市场价值的这种平均化,只有通过市场价值转化为不同于实际价值的费用价格才有可能。\authornote{剩余价值率在不同生产领域中可能并不平均化(例如由于劳动时间的长度不等)。并不因为剩余价值本身会平均化,剩余价值率就必然要平均化。}

竞争在同一生产领域所起的作用是:使这一领域生产的商品的价值决定于这个领域中平均需要的劳动时间;从而确立市场价值。竞争在不同生产领域之间所起的作用是:把不同的市场价值平均化为代表不同于实际市场价值的费用价格的市场价格,从而在不同领域确立同一的一般利润率。因此,在这第二种情况下,竞争决不是力求使商品价格去适应商品价值,而是相反,力求使商品价值化为不同于商品价值的费用价格,取消商品价值同费用价格之间的差别。

李嘉图在第四章考察的只是后面这种运动,而且他十分奇怪地把它看成是商品价格——通过竞争——还原为商品价值的运动,看成是“市场价格”(不同于价值的价格)还原为“自然价格”(用货币表现出来的价值)的运动。其实,这个谬误,是由在第一章(《论价值》)已经犯下的把费用价格和价值等同起来的错误\authornote{见本册第220页。——编者注}造成的,而后面这个错误的产生,又是因为李嘉图在他只需要阐明“价值”的地方,就是说,在他面前还只有“商品”的地方,就把一般利润率以及由比较发达的资本主义生产关系产生的一切前提全都拉扯上了。

因此,李嘉图在第四章所遵循的全部思路也是极其肤浅的。他的出发点是由变动的供求关系引起的商品“价格的偶然和暂时的变动”(第80页)。

\begin{quote}{“随着价格的上涨或下跌,利润就提高到它的一般水平之上或下降到它的一般水平之下,于是资本或者被鼓励转入那个发生这种变动的个别投资部门,或者被警告要退出这一部门。”(第80页)}\end{quote}

这里已经假定有一个不同生产领域之间的、“个别投资部门”之间的“利润的一般水平”。然而首先应当考察的是,同一投资部门中价格的一般水平和不同投资部门之间利润的一般水平是如何确立的。这样,李嘉图就会看到,后一种活动已经以资本的不断来回交叉游动为前提,或者说,以由竞争决定的、全部社会资本在不同投资领域之间的分配为前提。既然已经假定,在不同领域中市场价值或者说平均市场价格化为提供同一平均利润率的费用价格{但这种情况只是在没有土地所有权干预的生产领域才会发生,在有土地所有权干预的领域,这些领域内部的竞争会使价格化为价值,使价值化为市场价值,但不会使后者降到费用价格},既然已经假定了这一点,那末某些特殊领域中发生的市场价格对费用价格的经常偏离,即经常高于或低于费用价格的情况,就会引起社会资本的新的转移和新的分配。第一种转移的发生是为了确立不同于价值的费用价格;第二种转移是为了在市场价格高于或低于费用价格的时候使实际市场价格同费用价格趋于一致。第一种转移是价值转化为费用价格。第二种转移是不同领域中实际的[546]偶然的市场价格围绕费用价格旋转,费用价格现在表现为“自然价格”,虽然它不同于价值,它只是社会活动的结果。

李嘉图考察的正是后面这种比较表面的运动,他有时不自觉地把这种运动同另一种运动混淆起来。这两种运动自然是由“同一个原则”引起的,这个原则就是:

\begin{quote}{“每一个人都可以随意把自己的资本投在他所喜欢的地方……他自然要为自己的资本找一个最有利的行业;如果把资本转移一下能够得到15%的利润,他自然不会满足于10%的利润。一切资本家都想放弃利润较低的行业而转入利润较高的行业的这种不会止息的愿望,产生一种强烈的趋势,就是使大家的利润率平均化,或者把大家的利润固定在当事人看来可以抵销一方所享有的或看来享有的超过另一方的利益的那种比例上。”(第81页)}\end{quote}

这种趋势促使社会劳动时间总量按社会需要在不同生产领域之间进行分配。同时,不同领域的价值由此转化为费用价格,另一方面,各特殊领域的实际价格对费用价格的偏离也被拉平了。

这一切都来自亚·斯密。李嘉图自己说:

\begin{quote}{“如果一个投资部门生产的商品不能用自己的价格抵补把它们生产出来并运到市场的全部费用(包括普通利润在内)〈也就是不能补偿费用价格〉,资本就有离开这个部门的趋势,关于这一点,再没有一个著作家比斯密博士说得更令人满意、更精辟的了。”(第342页注)}\end{quote}

李嘉图的错误,一般说来,是由于他在这里不加批判地对待亚·斯密而产生的,而他的功绩则在于更确切地说明了资本从一个领域到另一个领域的这种转移,或者不如说,更确切地说明了这种转移的方式本身。但是,他能做到这一点,只是因为信用制度在他那个时代比在斯密时代更加发达罢了。李嘉图说:

\begin{quote}{“要追溯这种变化借以实现的步骤或许是非常困难的;它可能通过一个工厂主并不完全改变他的行业,而只是减少他在自己企业中的投资的方式来实现。在一切富裕的国家中,都有一定数目的人,形成所谓货币所有者阶级[注:罗雪尔在这里又可以看到,英国人所谓的“货币所有者阶级”是指什么。“货币所有者阶级”和“社会上有企业精神的人”在这里是完全对立的。\endnote{前面,在本册第130页上马克思引了罗雪尔《国民经济学原理》中的一段话,证明罗雪尔对于围绕着安德森的地租理论的斗争和关于“货币所有者和土地所有者之间的对立”的观点十分混乱。——第233页。}];这些人不从事任何行业,而把他们的货币用于期票贴现或者借给社会上更有企业精神的人,他们就靠这种货币的利息生活。银行家也把大量资本用于同样的目的。这样使用的资本形成巨额的流动资本,全国各行业或多或少地都使用它。一个工厂主不论怎样富有,大概也不会把他的营业限制在仅仅他自己的资金所容许的范围以内,他会经常使用这种流动资本的一部分,这部分资本的增减取决于对他的商品的需求的强弱。当对丝绸的需求增加而对呢绒的需求减少的时候,毛织厂主并不会把他的资本转到丝纺织业中去,而是解雇一部分工人,不再向银行家和货币所有者借款;丝织厂主的情况则相反,他会借更多的货币,于是资本就从一个行业转到另一个行业,而工厂主不必中断他通常经营的行业。如果我们观察一个大城市的市场,看到在所有由于嗜好改变或人口数量变动而需求发生变化的情况下,市场上国内外商品都能按需要的数量有规则地得到供应,既不是常常因供给过多而发生市场商品充斥现象,也不是常常因供不应求而造成物价腾贵,我们就必须承认,在一切行业之间恰好按其需要的数量分配资本的原则所起的作用,比一般设想的还大。”(第81—82页)}\end{quote}

由此可见,正是信用促使每个生产领域不是按照这个领域的资本家自有资本的数额,而是按照他们生产的需要,去支配整个资本家阶级的资本,——而在竞争中单个资本对于别的资本来说是独立地出现的。这种信用既是资本主义生产的结果,又是资本主义生产的条件,这样就从资本的竞争巧妙地过渡到作为信用的资本。

\tsubsubsectionnonum{[(c)李嘉图著作中关于“自然价格”的两种不同的规定。费用价格随着劳动生产率的变动而变动]}

李嘉图在第四章开头说,他所谓的“自然价格”,是指商品的“价值”,也就是指由商品的相对劳动时间决定的价格,而他所谓的“市场价格”,是指对这种等于“价值”的“自然价格”的偶然和暂时的偏离。[547]但是,在这一章的以后的全部行文中——甚至说得很明确——他所谓的“自然价格”,是指完全不同的东西,就是说,指不同于价值的费用价格。因此,他不去说明竞争怎样使价值转化为费用价格,从而造成对价值的经常偏离,却按照亚·斯密那样说明,竞争怎样使不同行业的市场价格在它们的相互关系中化为费用价格。

第四章开头这样说:

\begin{quote}{“如果我们把劳动作为商品价值的基础,把生产商品所必需的相对劳动量作为确定商品相互交换时各自必须付出的相应商品量的尺度,不要以为我们否定商品的实际价格或者说市场价格对商品的这种原始自然价格的偶然和暂时的偏离。”(第80页)}\end{quote}

可见,在这里“自然价格”等于价值,而市场价格无非是实际价格对价值的偏离。

相反:

\begin{quote}{“我们假定一切商品都按其自然价格出卖,因而资本的利润率在所有行业完全相同,或者只有这样一点差别,这种差别在当事人看来是与他们所享有或放弃的任何现实的或想象的利益一致的。”(第83页)}\end{quote}

可见,在这里“自然价格”等于费用价格,也就是等于这样的价格,在其中,利润对商品所包含的支出的比率是同一比率,尽管不同行业的资本生产的商品的等量价值包含极不相等的剩余价值,因而包含不相等的利润。因此,价格要提供同一利润,就必须不同于商品的价值。另一方面,由于加入商品的那部分固定资本大小不同,等量资本生产的商品的价值也是极不相等的。但是关于这一点,到考察资本流通时再谈。

所以,李嘉图所谓的竞争的平均化作用,不过是指实际价格,或者说,实际市场价格围绕费用价格,或者说,不同于价值的“自然价格”而波动,是指不同行业中的市场价格平均化为一般费用价格,也就是恰恰平均化为不同于某一行业的实际价值的价格:

\begin{quote}{“所以,正是每一个资本家都想把资金从利润较低的行业转移到利润较高的行业的这种愿望,使商品的市场价格不致长期大大高于或大大低于商品的自然价格。正是这种竞争会这样调节商品的交换价值{也调节不同的实际价值},以致在支付生产商品所必需的劳动的工资和其他一切为维持所使用的资本的原有效率所需要的费用之后剩下来的价值即价值余额,在每个行业中都同使用的资本的价值成比例。”(第84页)}\end{quote}

情况确实如此。竞争会这样调节不同行业的价格,以致剩下来的价值即价值余额,也就是利润,同使用的资本的价值相一致,而不是同商品的实际价值相一致,不是同商品在扣除费用以后所包含的实际的价值余额相一致。要实现这种调节,一种商品的价格就必须上涨到它的实际价值以上,而另一种商品的价格则必须下降到它的实际价值以下。竞争迫使不同行业的市场价格不是围绕商品的价值旋转,而是围绕商品的费用价格,也就是围绕商品中包含的费用加一般利润率旋转。

李嘉图继续说道:

\begin{quote}{“在《国富论》第七章对于同这个问题有关的一切都作了极为出色的论述。”(第84页)}\end{quote}

的确如此。正是由于不加批判地相信斯密的传统,李嘉图在这里走上了歧途。

李嘉图跟平常一样在结束这一章时说,在以后的研究中他将“完全不考虑”(第85页)市场价格对费用价格的偶然偏离,但是他忽略了一点,就是他根本没有注意到市场价格在同费用价格相一致的条件下对商品的实际价值的经常偏离,并且用费用价格代替了价值。

第三十章《论需求和供给对价格的影响》。

李嘉图在这里维护这样一个论点:持久的价格决定于费用价格,而不决定于需求和供给,因此,只是由于商品价值决定费用价格,持久的价格才决定于商品价值。假定商品的价格经过调节,都提供10%的利润,那末,它们的任何持久的变动都将决定于商品价值的变动,决定于生产商品所需要的劳动时间的变动。正如这种价值继续决定一般利润率一样,它的变动也继续决定费用价格的变动,自然,这并不会取消这种费用价格和价值之间的差额。取消的只是超出这一差额的东西,因为价值和实际价格之间的差额不应[548]大于一般利润率造成的费用价格和价值之间的差额。随着商品的价值的变动,商品的费用价格也发生变动。于是便形成“新的自然价格”(第460页)。例如,一个工人过去生产10顶帽子,现在用同样的时间能够生产20顶,如果工资占帽子的生产费用的一半,那末,20顶帽子的费用即生产费用,就其由工资组成的部分来看,是降低了一半。因为现在为生产20顶帽子支付的工资,同过去为生产10顶帽子支付的一样多。因此,每一顶帽子中现在只包含以前工资费用的一半。如果制帽厂主按以前的价格出卖帽子,他的帽子就会卖得高于费用价格。如果过去利润是10%(假定制造一定数量的帽子所必需的支出中,原来有50用于原,50用于劳动),那末现在利润就是[46+(2/3)]%。现在支出中有50用于原料等等,25用于工资。如果商品按以前的价格出卖,那末现在利润就是35/75,即[46+(2/3)]%。因此,由于价值降低,新的“自然价格”就会下跌,直到价格只提供10%的利润为止。价值降低,或者说,生产商品所必需的劳动时间减少,表现为同量商品所耗费的劳动时间减少,也就是耗费的有酬劳动时间减少,花费的工资减少,因而费用,为生产每一单位商品按比例支付的工资(按绝对量来说;这并不以工资率的下降为前提),也就下降。

当价值变动发生在制帽过程本身时,就会出现这种情况。如果价值变动发生在原料或劳动工具的生产上,这种变动在这些领域中同样表现为生产一定量产品所必需的工资费用减少,而对制帽厂主来说,则表现为他在不变资本上花费的钱减少。

费用价格,或者说,“自然价格”(它同“自然”毫无关系)由于商品价值变动——这里是降低——可能发生双重的变动:

第一,如果生产一定量商品所支付的工资由于生产该一定量商品所花费的劳动(包括有酬劳动和无酬劳动)的整个绝对量减少而减少;

第二,如果由于劳动生产率提高或降低(两种情况都可能发生:一种是在可变资本同不变资本相比减少的时候;另一种是在工资由于生活资料涨价而提高的时候),剩余价值和商品价值的比例,或者说,剩余价值和商品中包含的[新加]劳动所创造的价值的比例发生变动,因而利润率提高或降低,整个[新加]劳动量分为有酬劳动和无酬劳动的那种比例发生变动。

在后一种场合,生产价格即费用价格只能根据劳动价值的变动对它们发生影响的程度来变动。在前一种场合,劳动价值保持不变。在后一种场合,变动的不是商品的价值,而只是[必要]劳动和剩余劳动之间的分配。可是在这种场合,[劳动]生产率,因而每一单位商品的价值,仍然会发生变动。同一资本在一种场合生产的商品将比从前多,在另一种场合生产的商品将比从前少。资本借以表现的商品总量仍然具有同样的价值,但是单位商品的价值却和以前不同。虽然工资的价值并不决定商品的价值,但是(加入工人消费的)商品的价值却决定工资的价值。

既然不同行业商品的费用价格是既定的,这些费用价格就随着商品价值的变动而彼此相对地上涨或下跌。如果劳动生产率提高,就是说,生产一定商品所需要的劳动时间减少,因而商品的价值降低——不管生产率的这一变动是发生在最后阶段使用的劳动上,还是发生在为生产该商品所必需的不变资本包含的劳动上,——这种商品的费用价格就必然要相应地下跌。用于这种商品的绝对劳动量减少了,因而这种商品包含的有酬劳动量也减少了,花费在这种商品上的工资量也减少了,即使工资率保持不变。如果商品按其原来的费用价格出卖,它提供的利润就会高于一般利润率,因为以前按较大的支出计算,这个利润是10%。所以现在按减少了的支出计算,利润就会大于10%。相反,如果劳动生产率降低,商品的实际价值就提高。如果利润率是既定的,或者同样可以说,如果费用价格是既定的,那末,费用价格的相对提高或降低,就取决于商品实际价值的提高或降低,取决于商品实际价值的变动。由于这种变动,新的费用价格,或者象李嘉图仿照斯密所说的“新的自然价格”,就代替旧的价格。

在刚才引用过的第三十章里,李嘉图甚至在名称上也把“自然价格”即费用价格和“自然价值”即决定于劳动时间的价值等同起来了:

\begin{quote}{“它们的价格〈垄断商品的价格〉同它们的自然价值并没有必然的联系;但是,受竞争影响……的商品的价格最后都……取决于它们的生产费用。”(第465页)}\end{quote}

可见,这里把费用价格,或者说,“自然价格”和“自然价值”即“价值”直接[549]等同起来了。

这种混乱说明了为什么李嘉图以后的一批家伙,和萨伊本人一样,能把“生产费用”当作价格的最后调节者,而对价值决定于劳动时间这一规定却毫无所知,甚至在坚持“生产费用”的同时直接否定这一规定。

李嘉图的这整个错误和由此而来的对地租等的错误论述,以及关于利润率等的错误规律,都是由于他没有区分剩余价值和利润而造成的,总之,是由于他象其余的政治经济学家那样粗暴地、缺乏理解地对待形式规定而造成的。李嘉图怎样被斯密俘虏,从下文就可以看出。[549]

\centerbox{※     ※     ※}

[XII—636](对前面讲过的还要补充一点意见:

李嘉图不知道价值和自然价格有其他的差别,只知道:自然价格是价值的货币表现,因此,在商品本身的价值没有变动的情况下,由于贵金属的价值发生变动,自然价格也会变动。但是自然价格的这种变动只关系到价值的货币计量或货币表现。例如,李嘉图说:

\begin{quote}{“它〈对外贸易〉只能通过改变自然价格,但不是改变各国能据以生产商品的自然价值来调节,而这是通过改变贵金属的分配来实现的。”(同上,第409页))[XII—636]}\end{quote}

\tsectionnonum{[B.斯密的费用价格理论]}

\tsubsectionnonum{[(1)斯密的费用价格理论的错误前提。李嘉图由于保留了斯密把价值和费用价格等同起来的观点而表现出前后矛盾]}

[XI—549]关于亚·斯密,首先应当指出,他也认为:

\begin{quote}{“总是有……一些商品,它们的价格只分解为两部分,即工资和资本的利润。”(亚·斯密《国民财富的性质和原因的研究》[1802年法文版]第1卷第1篇第6章第103页)}\end{quote}

因此,对于斯密和李嘉图在这个问题上的区别在这里可以完全不去注意。

斯密起先阐述了一个观点,认为交换价值归结为一定量的劳动,交换价值中包含的价值,在扣除原料等之后,分解为付给工人报酬的劳动和不付给工人报酬的劳动,而后面这种不付给报酬的部分又分解为利润和地租(利润又可以分解为利润和利息),——在此以后,他突然来了一个大转变,不是把交换价值分解为工资、利润和地租,而是相反,把工资、利润和地租说成是构成交换价值的因素,硬把它们当作独立的交换价值来构成产品的交换价值,认为商品的交换价值是由不依赖于它而独立决定的工资价值、利润价值和地租价值构成的。价值不是它们的源泉,它们倒成了价值的源泉。

\begin{quote}{“工资、利润和地租,是一切收入的三个原始源泉,也是一切交换价值的三个原始源泉。”(同上,第105页)}\end{quote}

斯密在阐述了他所研究的对象的内在联系之后,突然又被表面现象所迷惑,被竞争中表现出来的事物联系所迷惑,而在竞争中一切总是表现为颠倒的、头足倒置的。

斯密正是从这个颠倒了的出发点来阐明“商品的自然价格”同商品的“市场价格”之间的区别的。李嘉图接受了斯密的这个观点,但是他忘记了,按照斯密的前提,斯密的“自然价格”只不过是由竞争而产生的费用价格,而在斯密本人的著作中,只有当斯密忘记了他自己的比较深刻的观点,仍然保持从表面的外观中得出来的,认为商品的交换价值是由独立决定的工资价值、利润价值和地租价值相加而成的错误观点的时候,费用价格才和商品的价值等同。李嘉图处处都反对这一观点,但是他又接受了亚·斯密在这一观点的基础上产生的,把交换价值同费用价格或“自然价格”混淆起来,或者说,等同起来的看法。这种混淆在斯密那里还可以说得过去,因为他对“自然价格”的全部研究是从他对价值的第二个观点即错误的观点出发的。而在李嘉图那里就毫无道理了,因为他在任何地方也没有接受斯密的这一错误观点,相反,他认为它前后矛盾而专门加以驳斥。但是,斯密又用“自然价格”把李嘉图引入了迷途。

斯密用不依赖于商品价值而独立决定的工资价值、利润价值和地租价值构成商品价值之后,接着就给自己提出了一个问题:这些作为要素的价值又是怎样决定的?这里斯密是从竞争中呈现出来的现象出发的。

第一篇第七章《论商品的自然价格和市场价格》。

\begin{quote}{“在任何社会或任何地方,工资、利润、地租都有一种普通率,或者说,平均率。”这种“平均率对于它所通行的时间和地方来说可以称为工资、利润和地租的自然率”。“如果一种商品的价格恰好足够按自然率支付地租、工资和利润,这种商品就是按照它的自然价格出卖。”(第110—111页)}\end{quote}

这样一来,这种自然价格就是商品的费用价格,而费用价格就和商品的价值等同起来了,因为已经假定,商品的价值是由工资价值、利润价值和地租价值构成的。

\begin{quote}{“商品[550]在这种情况下恰好是按其所值出卖〈这时商品是按其价值出卖〉,或者说〈或者说!〉,按照使该商品进入市场的人的实际花费出卖〈对使商品进入市场的人来说,是按商品的价值,或者说,费用价格出卖〉,因为,虽然照普通的说法,在谈到商品的生产费用时,其中不包括出卖自己生产的商品的人的利润,但是,如果他按照不能给他提供当地普通利润的价格出卖自己的商品,他的营业显然就要受到损失,因为他如果以其他某种方式使用自己的资本,是能够获得这一利润的。”(第111页)}\end{quote}

在这里,我们看到了“自然价格”产生的全部历史以及同它完全相适应的语言和逻辑。因为在斯密看来,商品的价值是由工资、利润和地租的价格构成的,而工资、利润和地租的真正价值也是以同样方式构成的,所以很明显,在它们处于自己的自然水平的情况下,商品的价值和商品的费用价格是等同的,而商品的费用价格又是和商品的“自然价格”等同的。利润水平即利润率,以及工资率,都被假定为事先既定的。对于费用价格的形成来说,它们确实是既定的。它们是费用价格的前提。因此,它们对单个资本家来说也表现为既定的。至于它们怎样产生,在什么地方产生和为什么产生,资本家是不关心的。斯密在这里是站到确定自己商品的费用价格的单个资本家即资本主义生产当事人的立场上去了。工资等等花费多少,一般利润率是多少。因此……在这个资本家看来,确定商品费用价格的程序,或者,在他进一步看来,确定商品价值的程序就是这样,因为他也知道,市场价格有时高于这种费用价格,有时低于这种费用价格;所以在资本家看来,这种费用价格就是商品的理想价格,就是不同于商品价格波动的商品的绝对价格,一句话,就是商品的价值,如果资本家有时间去思考这类事情的话。由于斯密置身于竞争的中心,他立即就开始按照受这个领域局限的资本家所特有的逻辑发议论。他反驳说,在日常生活中,费用不是指卖者所赚得(并且必然是超过他的支出的余额)的利润;你为什么把利润算在费用价格之内呢?亚·斯密同被提出这一问题的深思熟虑的资本家一起,作了如下的回答:

利润一般必须加入费用价格,因为,即使加入费用价格的利润总共只有9%而不是10%\endnote{假定10%是当地的平均利润率。——第243页。},我也是受骗了。

斯密天真地一方面用资本主义生产当事人的眼光来看待事物,完全按照这种当事人所看到和所设想的样子,按照事物决定这种当事人的实践活动的情况,按照事物实际上呈现出来的样子,来描绘事物,另一方面,在有些地方也揭示了现象的更为深刻的联系,——斯密的这种天真使他的著作具有巨大的吸引力。

这里也可以看出,为什么斯密——尽管在这一问题上内心有很大的犹豫——把商品的价值只分解为地租、利润和工资,而略去了不变资本,尽管他自然也承认“单个”资本家的不变资本。因为不然的话,他就必须说商品的价值是由工资、利润、地租以及不由工资、利润、地租构成的那个商品价值部分构成的了。这样一来,就必须离开工资、利润和地租来确定价值了。

如果除了补偿平均工资等等的支出以外,商品的价格还提供平均利润,而在支出数包括地租的情况下,还提供平均地租,那末,商品就是按其“自然价格”,或者说,费用价格出卖,而且商品的费用价格就等于商品的价值,因为在斯密看来,商品的价值无非是工资、利润和地租的自然价值的总和。

[551]此外,斯密既然已经站在竞争的立场上,并且假定了利润率等是既定的,他也就正确地阐述了“自然价格”,或者说,费用价格,也就是不同于市场价格的那种费用价格。

\begin{quote}{“自然价格,或者说,使它〈商品〉进入市场所必须支付的地租、利润和工资的全部价值”。商品的这种费用价格是和它的“实际价格”,或者说,“市场价格”不同的。(第112页)后者取决于需求和供给。}\end{quote}

商品的生产费用,或者说,商品的费用价格,恰好是“使这一商品进入市场所必须支付的地租、工资和利润的全部价值”。如果供求相互适应,“市场价格”就等于“自然价格”。

\begin{quote}{“如果进入市场的数量恰好足够满足实际需求而不超出这一限度,那末,市场价格当然就会和自然价格完全一致……”(第114页)“因此,自然价格可以说是一个中心点,一切商品的价格都不断趋向这个中心点。各种偶然的情况有时会使商品的价格在某一时期内高于自然价格,而有时又会使它略低于自然价格。”(第116页)}\end{quote}

于是斯密由此得出结论说,总的说来

\begin{quote}{“为了使某种商品进入市场而在一年内使用的勤劳总量”,将同社会的需要,或者说,“实际的需求”相适应。(第117页)}\end{quote}

李嘉图所谓的总资本在各行业之间的分配,在这里还是以生产“某种特定商品”所必需的“勤劳”这一比较素朴的形式出现的。同一种商品的卖者之间的价格平均化为市场价格,以及各种不同商品的市场价格平均化为费用价格,这两种情况在这里还是杂乱地相互交错在一起的。

在这里,斯密只是完全偶然地谈到商品实际价值的变动对“自然价格”,或者说,费用价格的影响。

他是这样说的:

\begin{quote}{在农业中“同量劳动在不同的年份会生产出极不相同的商品量,而在另一些行业中,同量劳动总是会生产出同量或差不多同量的商品。在农业中,同一数量的工人在不同的年份会生产出数量极不相同的谷物、酒、植物油、啤酒花等等。但是同一数目的纺纱工人和织布工人每年会生产出同量或差不多同量的麻布或呢绒……在其他行业〈非农业〉中同量劳动的产品总是相同的或者差不多相同的〈就是说,只要生产条件相同〉,产品能更加准确地适应实际的需求”。(第117—118页)}\end{quote}

在这里,斯密看到了,“同量劳动”的生产率的单纯变动,从而,商品的实际价值的变动,会使费用价格发生变动。可是,他由于把整个问题归结为供求关系,又把这一点庸俗化了。根据他自己的论断,他对这一问题的阐述也是不正确的。因为如果在农业中“同量劳动”由于气候等条件而提供不同量的产品,那末,斯密自己就已经说明,由于机器、分工等等,在工业等部门中“同量劳动”提供的产品量也是极不相同的。可见,农业和其他行业之间的区别并不在这一点上。这种区别在于,在一种场合,“生产力”是“在事先决定了的程度上”被使用,而在另一种场合,生产力却取决于自然界的偶然性。但结果仍然是:商品的价值,或者说,根据劳动生产率必须花费在某种商品上的劳动量,会使商品的费用价格发生变动。

在后面所引的亚·斯密的论点中已经包含这样一种思想,就是资本由一个行业向另一个行业转移,会确立不同行业的费用价格。不过,对于这一点,斯密说得不象李嘉图那样明白,因为如果[552]商品的价格降到其“自然价格”以下,那末,根据斯密的说法,这是由这种价格的要素之一降到自然水平即自然率以下造成的。因此,[要消除商品价格的这种下降,]不是靠单单把资本抽出或转移,而是靠把劳动、资本或者土地从一个部门转到另一个部门。在这里,斯密的观点比李嘉图的观点彻底,不过这种观点是错误的。

\begin{quote}{“不管这种价格〈自然价格〉的哪一部分是低于其自然率支付的,那些利益受影响的人,很快就会感到受了损失,并立即把若干土地,或若干劳动,或若干资本从这种行业中抽出,从而使这种商品进入市场的数量很快只够满足实际的需求。因此,这种商品的市场价格很快就会提高到它的自然价格的水平;至少在有完全自由的地方是这样。”(第125页)}\end{quote}

在这里,斯密和李嘉图对于同“自然价格”趋于一致这一点的理解存在着根本的区别。斯密的理解是以他的错误的前提为基础的,即认为上述三个要素独立地决定商品的价值,而李嘉图的理解是以正确的前提为基础的,即只有平均利润率(在工资既定的情况下)才能确立费用价格。

\tsubsectionnonum{[(2)斯密关于工资、利润和地租的“自然率”的理论]}

\begin{quote}{“自然价格本身随着它的每一构成部分即工资、利润和地租的自然率的变动而变动。”(第127页)}\end{quote}

斯密试图在第一篇第八、九、十章和第十一章确定这些“价格的构成部分”即工资、利润和地租的“自然率”,以及这种自然率的变动。

第八章《论工资》。

在论工资这一章一开头,斯密就抛开虚幻的竞争观点,首先分析剩余价值的真正的本质,把利润和地租看作只是剩余价值的形式。

在考察工资的时候,斯密有一个确定工资的“自然率”的牢固的出发点,即劳动能力本身的价值:必要工资。

\begin{quote}{“一个人总要靠自己的劳动来生活,他的工资至少要够维持他的生存。在大多数情况下,他的工资甚至应略高于这个水平,否则,工人就不可能养活一家人,这些工人就不能传宗接代。”(第136页)}\end{quote}

不过,斯密的这一论点从另一方面来说没有任何意义,因为他从来没有向自己提出这样的问题:必要生活资料的价值,也就是说,商品的价值,是由什么决定的?因为斯密离开了他的基本观点,所以他在这里不得不说:工资的价格是由生活资料的价格决定的,而生活资料的价格是由工资的价格决定的。他先假定工资的价值是固定不变的,接着又准确地描绘了工资价值在竞争中表现出来的波动,以及造成这种波动的那些情况。这属于[斯密观点的]外在部分,在这里和我们没有关系。

{斯密特别描绘了资本的“增长”(积累)[对工资]的影响,但是他没有告诉我们,资本的增长是由什么决定的。因为这种“增长”只有在下述两种情况下才能迅速进行:或者是工资率比较低,而劳动生产率高(在这种情况下,工资的提高始终只是整个前一段时间工资水平低的结果);或者是积累率低[即利润率低],但劳动生产率高。在第一种情况下,斯密从他的观点出发,本应从利润率(即从工资率)得出工资率,而在第二种情况下,则从利润量得出工资率。但是,这又有必要去研究商品价值。}

斯密想从作为商品价值的构成要素之一的劳动价值得出商品价值。另一方面,他又从以下事实得出工资的高度。

\begin{quote}{“……工资并不随着食物价格的波动而波动”(第149页),“各地工资的变动比食物价格的变动大”。(第150页)}\end{quote}

事实上,整个这一章除了最低限度的工资,换句话说,劳动能力的价值这一规定以外,有关的问题一点也没有谈到。在这里,斯密本能地重新提到了他的比较深刻的观点,但是接着又把它抛弃,以致上述规定在他那里并没有产生任何结果。实际上,必要生活资料的价值,也就是说,商品的价值,是由什么决定的呢?部分地由“劳动的自然价格”决定。而劳动的自然价格又是由什么决定的呢?由生活资料的价值,或者说,商品的价值决定。这是可怜地在没有出路的圈子里打转转。此外,这一章没有一个字谈到本题,没有一个字谈到“劳动的自然价格”,[553]只是研究了工资怎样提高到“自然率”的水平以上,也就是说,工资的提高同资本积累的速度,同资本的日益增长的积累成比例。然后研究了产生这种情况的各种社会状况,最后,斯密给了商品的价值决定于工资,而工资的价值决定于必要生活资料的价值这种规定以直接的打击,证明英国的情况似乎不是这样。因为工资不仅决定于维持现有人口的生活所必需的生活资料,而且决定于现有人口的再生产所必需的生活资料,所以这里包含有一些类似马尔萨斯人口论的东西。

这就是,亚·斯密试图证明工资在十八世纪,特别是在英国已经提高之后,提出这样一个问题:应当把这看作“对社会有利还是不利”(第159页)。谈到这里,他又顺便回到他的比较深刻的观点,根据这种观点,利润和地租都只是工人劳动产品的一部分。他说,工人

\begin{quote}{“首先占社会的绝大部分。难道我们什么时候能够认为,这个整体的大部分的命运得到改善,是对这个整体不利的吗?如果社会的绝大部分成员都是贫困的和不幸的,毫无疑问,不能认为这个社会是幸福的和繁荣的。此外,单是从公道出发,也要求使那些供给整个国家吃穿住的人,在他们自己的劳动产品中享有这样一个份额,这一份额至少足够使他们自己获得可以过得去的食物、衣服和住房”。(第159—160页)}\end{quote}

谈到这里,斯密又涉及人口论:

\begin{quote}{“虽然贫困无疑会使人不愿结婚,但它并不总是使人不能结婚;贫困似乎还会促进繁殖……在上层社会的妇女中如此常见的不妊症,在地位低下的妇女中是极少见的……不过,贫困虽然不妨碍生孩子,但是会给抚养儿女造成极大的困难。柔弱的植物出世了,但是出生在那样寒冷的土壤里和那样严酷的气候里,它很快就会枯萎和死亡……各种动物都自然地适应它现有的生存资料的数量而繁殖,没有一种动物的繁殖能够超过这个界限。但是在文明社会,只有在人民的下层阶级中,生存资料的缺乏才能限制人类进一步的繁殖……正象其他任何商品一样,对人的需求必然会调节人的生产;当人的生产过慢的时候,这种需求会使之加速,而当人的生产过快的时候,这种需求就使之缓慢……”(第160—163页)}\end{quote}

最低限度的工资和不同社会状况的关系是这样的:

\begin{quote}{“付给各种短工和仆人的工资,必须足以使他们的繁殖总的来说能够同社会〈社会,也就是资本〉对他们的需求的增加、减少或保持不变相适应。”(第164页)}\end{quote}

斯密接着指出,奴隶比自由工人“贵”,因为后者的“损耗”是由他本人照管,而前者的“损耗”却由“不大经心的主人或玩忽职守的监工”监督。(第163页及以下各页)补偿“损耗”的基金,自由工人使用得很“节约”,而在奴隶那里却由于管理混乱而被浪费:

\begin{quote}{“用来补偿和抵补奴隶劳力因长年服务而造成的可以说是损耗的基金,一般都由不大经心的主人或玩忽职守的监工管理。相反,在自由工人那里,用于同一目的的基金,却由工人自己管理得很节约。富人经营管理中常有的混乱,自然在前一种基金的管理上表现出来;穷人的极度节俭和精打细算,同样自然地表现在后一种基金的管理上。”(第164页)}\end{quote}

在最低限度的工资,或者说,“劳动的自然价格”的规定中,还包括自由雇佣工人的“劳动的自然价格”比奴隶的低这样一点。斯密透露了这个思想:

\begin{quote}{“自由人的劳动归根到底比奴隶的劳动便宜。”(第165页)“如果说优厚的劳动报酬是国民财富增长的结果,那末它也是人口增长的原因。抱怨劳动报酬优厚,[554]就是对最大的公共福利的结果和原因不满。”(第165页)}\end{quote}

接着,斯密为高工资辩护说:

\begin{quote}{高工资“不仅会促进人口的增长”,而且会“增进普通人民的勤劳。工资是对勤劳的奖励,而勤劳,也和人的其他各种特性一样,越是受到奖励就越发展。丰富的食物会增强工人的体力,而改善自己状况……的向往会激励他极端卖力。因此我们看到,工资高的地方的工人总是比工资水平低的地方的工人更积极、更勤勉和更敏捷”。(第166页)}\end{quote}

但是,高工资也会使工人过度劳累,过早地毁坏自己的劳动能力:

\begin{quote}{“领取高额计件工资的工人,很容易进行过度劳动,在不几年内就把自己的健康和劳力毁掉。”(第166—167页)“如果雇主始终听从理性和人道的支配,他倒是常常有理由去节制而不是去鼓励他的许多工人的勤奋。”(第168页)接着,斯密驳斥了“增加福利会使工人懒惰”的说法。(第169页)}\end{quote}

然后,斯密研究了工人在丰年比在荒年懒惰的说法是否正确的问题,并且说明了工资和商品价格之间的关系一般是怎样的情况。这里他又表现出前后矛盾。

\begin{quote}{“劳动的货币价格必然决定于两种情况:对劳动的需求以及必需品和舒适品的价格……劳动的货币价格决定于购买一定量的物品〈必需品和舒适品〉所需要的货币额。”(第175页)}\end{quote}

接着,斯密研究了为什么——由于对劳动的需求——在丰年工资会提高,而在荒年工资会降低。(第176页及以下各页)

在好年景和坏年景,[工资提高和降低的]原因会互相抵销:

\begin{quote}{“物价高涨年份的贫乏,由于减少对劳动的需求,有降低劳动价格的趋势,而食物价格的昂贵又有提高劳动价格的趋势。相反,物价低廉年份的丰裕,由于增加对劳动的需求,有提高劳动价格的趋势,而食物价格的低廉,又有降低劳动价格的趋势。在食物价格发生一般波动的情况下,这两种对立的原因看来会互相抵销;这一点也许部分地说明了,为什么工资到处都比食物价格稳定得多。”(第177页)}\end{quote}

最后,在作了所有这些反复曲折的论证之后,斯密又用他原来比较深刻的观点,即商品价值由劳动量决定的观点,来同工资是商品价值的源泉这一观点相对立;如果说在丰年或资本增长的时候工人得到较多的商品,那末他也生产出多得多的商品,也就是说,在这种情况下单位商品包含的劳动量少了。因此,工人可能得到数量较大而价值较小的商品,由此产生的一个合乎逻辑的结论就是:尽管绝对工资提高,利润还可能增加。

\begin{quote}{“工资的提高,由于使商品价格中分解为工资的部分扩大,必然会使许多商品的价格提高,并且相应地使这些商品在国内外的消费有缩减的趋势。但是,引起工资提高的原因,即资本的增长,又有提高劳动生产能力的趋势,使较小量的劳动能够生产出较大量的产品”……分工,使用机器,发明等等……“由于这一切改良,现在有许多商品已经能够用比以前少得多的劳动来生产了。结果,这种劳动价格的提高,会由于劳动量的减少得到补偿而有余。”(第177—178页)}\end{quote}

劳动得到较好的报酬,但单位商品包含的劳动少了,也就是说,必须支付报酬的劳动量少了。这样,斯密就抛弃了他的错误理论,或者更确切地说,斯密在这里用他的正确理论抵销了、补救了错误的理论;按照他的错误理论,工资作为构成价值的一个要素,决定商品的价值,而按照他的正确理论,商品的价值是由商品中包含的劳动量决定的。

[555]第九章《论资本利润》。

因此,这里应当确定那种决定并构成商品的“自然价格”,或者说,商品的价值的第二个要素的“自然率”。斯密关于利润率下降的原因所说的话(第179、189、190、193、196、197等页)以后再考察。\authornote{见本册第497和533—535页。——编者注}

这里,斯密陷入了极其困难的境地。他说,工资的“平均率”这一规定只能归结为:这是“普通的工资水平”(第179页),即实际上既定的工资水平。

\begin{quote}{“但是对资本利润来说,就连这一点也未必能做到。”(第179页)除了企业主的成功或失败,“这种利润还取决于商品价格的每一次变动”。(第180页)}\end{quote}

然而,我们正是应当通过作为构成“价值”的要素之一的利润的“自然率”,来决定这些商品的“自然价格”。在单个行业,对单个资本家来说,要确定平均利润率已经很困难了。

\begin{quote}{“要确定一个大的王国内所有行业的平均利润,必然更加困难。”(第180页)}\end{quote}

但是,关于“资本的平均利润”,可以“根据货币利息的高低”得出一个概念:

\begin{quote}{“可以确定这样一个原则:凡是从投资中能获得大量利润的地方,通常为使用货币而付出的报酬就多,而在只能获得少量利润的地方,通常为使用货币而付出的报酬就少。”(第180—181页)}\end{quote}

斯密不是说,利息率决定利润率。他所说的显然是相反的意思。但是关于不同时期的利息率等等,我们已有记载,而利润率则没有这种记载。因此,利息率是个征兆,根据它可以大体判断利润率的情况。但任务不是去比较既有的各种利润率,而是要确定“利润的自然率”。斯密避开这个任务而去对不同时期的利息率的水平进行无关紧要的研究,这和他所提出的问题毫不相干。他粗略地描绘了英格兰不同时期的情况,然后拿英格兰同苏格兰、法国、荷兰相比较,发现除美洲殖民地外,

\begin{quote}{“高工资和高利润,自然是很少同时出现的东西,只是在某种新殖民地的特定情况下才会同时出现”。(第187页)}\end{quote}

这里,亚·斯密已经试图几乎象李嘉图那样(但在某种程度上更成功)说明高利润:

\begin{quote}{“新殖民地拥有的资本和领土范围的比例,以及人口和资本量的比例,有一个时期总是要比其他大多数国家小。殖民者所拥有的土地多,而用来开发土地的资本量少;所以,他们所拥有的资本只是用来耕种最肥沃和位置最好的土地,也就是沿海和通航河流两岸的地区。而且购买这种土地的价格,往往低于其自然生长的产品的价值。〈可见,实际上这种土地毫无所值。〉用来购买和改良这种土地的资本,必然会提供很高的利润,因而使用资本也有可能付出很高的利息。在这样有利可图的企业中,这种资本的迅速积累,使种植场主有可能迅速增加自己的工人人数,以致在新的居留地无法找到这样多的工人。因此,他所能找到的工人就会得到优厚的报酬。随着殖民地的不断扩大,资本利润也逐渐下降。当最肥沃和位置最好的土地已全被占有的时候,耕种比较不肥沃和位置比较差的土地,只能提供较少的利润,因而对所使用的资本也只能支付较少的利息。正因为如此……利息率,在本世纪中,在我们的大部分殖民地,都大大降低了。”(第187—189页)}\end{quote}

虽然论证的方法不同,但是这成了李嘉图说明利润下降的基础之一。总之,斯密在这里是用资本的竞争来说明一切,资本一增长,利润就下降,资本一减少,利润就提高,而工资则相反,在前一种场合,工资会提高,在后一种场合,工资会降低。

\begin{quote}{[556]“社会的资本,或者说,用于生产的基金减少,一方面使工人的工资降低,另方面使资本利润提高,从而也使利息率提高。由于工资降低,社会上剩下的资本的所有者就能以比从前少的费用使自己的商品进入市场;由于现在是以较少量的资本实现商品对市场的供应,资本家就能够把自己的商品卖得贵些。”(第191—192页)}\end{quote}

其次,斯密谈到尽可能高的和尽可能低的利润率。

\begin{quote}{“最高的利润率”是这样的利润率,“它从大部分商品的价格中吞并了所有应当归入地租份内的部分,而留下的部分仅仅足够支付生产商品并把商品运到市场所需的劳动的报酬,并且是按照某地最低的工资率支付的,就是说,按照只够维持工人生存的工资率支付的”。(第197—198页)“最低的普通利润率,总是除了足够补偿任何投资都可能遇到的意外损失外,还须略有剩余。只有这个余额才是纯利润。”(第196页)}\end{quote}

实际上,斯密自己对他关于“利润的自然率”的看法作了如下说明:

\begin{quote}{“在英国,人们认为,商人称之为正当的、适度的、合理的利润的,就是双倍的利息;我认为,这些说法的意思无非就是通常的、普通的利润。”(第198页)}\end{quote}

确实,斯密并没有把“通常的、普通的利润”叫作适度的或正当的,但他还是把它称为“利润的自然率”;不过他根本没有告诉我们,这是什么样的东西,或者说,这种利润率是怎样确定的,不过按照斯密的说法,我们就应当利用这种“利润的自然率”来决定商品的“自然价格”。

\begin{quote}{“在财富迅速增加的国家里,在许多商品的价格中,高工资可以用低利润率来弥补,这样,这些国家就能够象它的繁荣程度较低、工资也低的邻国那样便宜地出卖自己的商品。”(第199页)}\end{quote}

低利润和高工资,在这里并不是作为互相影响的东西而彼此对立,二者都是由同一个原因,即资本的迅速增长,或者说,迅速积累造成的。利润和工资都加入价格,构成价格。因此,如果一个高而另一个低,价格就保持不变,等等。

在这里,斯密把利润看作纯粹是[价格的]附加额,因为他在这一章的结尾说:

\begin{quote}{“实际上,高利润比高工资能在大得多的程度上促使产品价格提高。”(第199页)例如,如果在麻织厂工作的所有工人的工资一天各增加2便士,那末,“一匹麻布”的价格将要上涨的数额,只是等于生产这匹麻布所用的工人人数乘2便士,再“乘以工人生产麻布所用的日数。商品价格中分解为工资的部分,由于工资的增加,在生产商品的每一个阶段只按工资增加的算术级数增加。但是,如果所有雇用这些工人的各种企业主的利润都增加5%,那末,商品价格中分解为利润的部分,由于利润率的增加,从一个生产阶段到另一个生产阶段将按利润率增加的几何级数增加……工资提高对商品价格的提高所起的作用,就象单利对债务额的增加所起的作用一样。利润提高所起的作用却象复利”。(第200—201页)}\end{quote}

在这一章的结尾,斯密还告诉我们,他这全部观点,即商品的价格,或者说,商品的价值由工资和利润的价值构成,是从哪里来的;那是从“商业之友”\authornote{原文是《amisducommerce》(傅立叶语)。——编者注},从实际的竞争信奉者那里来的。

\begin{quote}{“我国商人和工业家,对于高工资使商品价格提高,从而减少商品在国内外销路的有害作用,常出怨言;但对高利润的有害作用却默不作声;他们对自己的收入所产生的致命后果保持沉默。[557]他们只是对别人的收入愤愤不平。”(第201页)}\end{quote}

第十章《论劳动和资本的不同使用部门的工资和利润》。它只涉及细节,所以是论述竞争的一章,并且独具特色。它具有完全外在的性质。

{生产劳动和非生产劳动:

\begin{quote}{“法律职业的彩票,是十分不公平的;这一行,象其他大多数自由的、荣誉的职业一样,从金钱收入来说,所得的报偿显然太低了。”(第216—217页)}\end{quote}

他同样谈到士兵:

\begin{quote}{“他们的薪饷比普通短工的工资低,而他们在服役期间的劳累程度却大得多。”(第223页)}\end{quote}

关于水兵:

\begin{quote}{“虽然他们的职业所要求的技能和熟练程度,几乎比其他一切行业都高得多,虽然他们的全部生涯充满着无穷无尽的辛苦和危险……他们的工资却不比海港普通工人的工资高,海港普通工人的工资调节着海员的工资率。”(第224页)}\end{quote}

他讽刺地说:

\begin{quote}{“拿副牧师或礼拜堂牧师同短工比较无疑是不礼貌的。但是,我们完全可以认为,副牧师或礼拜堂牧师的薪俸和短工的工资具有同样的性质。”(第271页)}\end{quote}

至于“文人”,斯密明确地认为,他们由于人数太多而报酬过低,而且他提醒说,在印刷术发明以前,“大学生和乞丐”(第276—277页)是一个意思,看来斯密认为,这在一定意义上也适用于文人。}

这一章充满着锐敏的观察和重要的评论。

\begin{quote}{“在同一社会或同一地区,不同投资部门的平均的、普通的利润率,和不同种类劳动的货币工资相比,大大接近于同一水平。”(第228页)“市场广阔,由于容许使用较多的资本,会使表面利润减少;但是由于要求从更远的地方运来商品,又会使成本增加。这种利润的减少和成本的增加,在许多场合,似乎是接近于互相抵销〈指面包、肉类等商品的价格〉。”(第232页)“在小城市和乡村,由于市场狭小,商业并不能总是随着资本的增长而扩大。因此,在这些地方,虽然个人的利润率可能很高,但是利润的总额或总量决不可能很大,从而他的年积累总额也不可能大。相反,在大城市,营业可能随着资本的增长而扩大,一个勤俭而又交财运的人的信用会比他的资本增长得更快。他的营业会随着二者的增长而日益扩大。”(第233页)}\end{quote}

关于工资水平的一些错误统计材料(例如十六、十七世纪等的),斯密很正确地指出,这里的工资只是例如茅舍贫农的工资。这种茅舍贫农不在自己的小屋里干活或者不为自己的主人劳动的时候(他们的主人给他们“一座小屋,一小块菜地,一块够饲养一头母牛的草地,也许还有一两英亩坏的耕地”,主人叫他们干活的时候,也只付给他们很低的工资),他们

\begin{quote}{“情愿向愿意雇用他们的人提供自己的空闲时间,并且挣比其他工人低的工资”。(第241页)“可是有许多收集关于以前各个时代的劳动价格和食品价格的资料的著作家,非常喜欢把这两种价格说得格外低廉,他们把这种偶然的额外收入看成这些工人的全部工资。”(第242页)}\end{quote}

前面,斯密还作了正确的一般性评论:

\begin{quote}{“劳动和资本在不同部门使用的有利与不利在总体上的平衡,只有在那些被人们作为唯一的或主要的职业来从事的部门中才可能发生。”(第240页)}\end{quote}

不过,这一思想,特别是关于“人们开始珍惜时间”\endnote{马克思指的是詹姆斯·斯图亚特的书《政治经济学原理研究》1770年都柏林版第一卷。在这车书里描写了英国农村从主要是自然经济转变为商品经济和资本主义经济的过程,伴随这一过程发生的是农业变为资本主义经营的一个部门,农业劳动强度的增大和对农村居民的剥夺。斯图亚特的用语“人们开始珍惜时间”见该书第1卷第171页。这句话和摘自斯图亚特的其他引文一起,马克思在1857—1858年经济学手稿中也曾经引用过(见卡·马克思《政治经济学批判大纲》1939年莫斯科版第742页)。——第257页。}以来的农业的工资问题,斯图亚特已经很好地阐明了。

[558]关于中世纪城市资本的积累,斯密在这一章中很正确地指出,它主要来源于(商人和手工业者)对农村的剥削。(还有高利贷者,以及金融贵族,一句话,货币经营者。)

\begin{quote}{“城市工商业居民的每一个集团〈在实行行会制度的城市内〉由于实行这种规约,当然不得不付出略高于没有规约时的价格,向城市其他集团的商人和手工业者购买他们需要的商品。但是,为了弥补这一点,他们也可以按同样较高的价格出卖自己的商品。结果是正如一般所说,贵买贵卖,横竖一样。在城市内各个集团之间进行交易时,他们都不会因这种规约而蒙受任何损失。但在与农村进行交易时,他们却都会得到很大的利益,而城市赖以维持和富裕起来的商业,也就是后面这种交易。每一个城市都从农村取得它的全部粮食和全部工业原料。对这些东西,它主要用以下两种办法来支付:第一,把这种原料的一部分加工以后运回农村,在这种场合,原料的价格中增加了工人的工资和他们的主人或者说直接雇用者的利润;第二,从城市把外国进口或由本国遥远地区运来的原产品或工业品运往农村,在这种场合,这些商品的原来价格中同样要增加水陆运输工人的工资和雇用他们的商人的利润。由第一类商业赚到的钱,构成城市从工业得到的全部利益。由第二类商业赚到的钱,构成城市从国内外贸易得到的全部利益。工人的工资和雇主的利润,构成从这两个部门赚到的钱的全部。因此,目的是要把这些工资和利润提高到它们的自然水平以上的一切规约,其作用就是使城市能够以自己较小量的劳动购买农村较大量劳动的产品。”}\end{quote}

{可见,在最后一句话中,斯密又回到正确的价值规定上来了。这句话在第259页。价值由劳动量决定。在考察斯密对剩余价值的解释时应把这作为一个例子举出来。如果城市和农村相互交换的商品的价格是代表等量劳动,那末商品的价格就等于商品的价值。因此,不论哪一方面的利润和工资都不能决定这些价值,倒是这些价值的分配决定利润和工资。因此,斯密也发现,以较小量劳动交换农村较大量劳动的城市,在同农村的交往中会取得超额利润和超额工资。如果城市不是把自己的商品高于其价值卖给农村,这种情况就不会发生。那样的话,“利润和工资”就不会提高到“它们的自然水平以上”。所以,如果利润和工资处于“它们的自然水平”,那就不是由它们决定商品价值,而是它们自己由商品价值决定。那时,利润和工资就只能从既定的、作为它们前提的商品价值的分配中产生;但是这个价值不能由利润和工资决定,不能从作为价值本身的前提的利润和工资得出来。}

\begin{quote}{“这种规约,造成了城市的商人和手工业者对农村的土地所有者、租地农场主和农业工人的优势地位,并且破坏了城乡贸易中没有这种规约时存在的自然平衡。现有社会的全年劳动总产品,每年都是在这两部分不同的居民之间分配的。由于有这种〈城市的〉规约,城市居民就会得到比没有这种规约时较大的一部分产品,农村居民则得到较小的一部分。城市每年为输入的粮食和原料实际支付的价格,也就是城市每年输出的工业品和其他商品的量。后者卖得越贵,前者就买得越便宜。因此,城市的实业活动就变得比较有利,农村的实业活动则变得比较不利。”(第258—260页)}\end{quote}

这样,按照斯密本人对问题的解释,如果城市和农村的商品都按这些商品各自包含的劳动量出卖,那它们就是按照自己的价值出卖,因而,两方面的利润和工资都不能决定这些价值,倒是利润和工资由商品的价值决定。关于因资本有机构成不同而有所不同的利润的平均化,在这里和我们无关;因为它不仅不会造成利润的差别,反而会使利润趋于同一水平。

\begin{quote}{[559]“城市的居民,由于集中在一个地方,彼此间容易交往和达成协议。因此,城市中甚至最无关紧要的行业,也几乎到处都组成了行会……”(第261页)“农村的居民,由于居住分散,彼此距离较远,就不那么容易结合起来。他们不仅从来没有组织过行会,甚至连行会精神也从来没有在他们中间盛行过。人们从未认为,为了使人能够从事农业这种农村的主要行业,有必要建立学徒制度。”(第262页)}\end{quote}

在这里,斯密还谈到了“分工”的不利方面。农民的劳动,比受分工支配的制造业工人的劳动,具有更大程度的脑力性质:

\begin{quote}{“从事那种必需随着气候的每一变化和其他许多情况的变化而变化的工作,比从事那种同一的或者差不多同一的操作,要求更高得多的判断力和预见性。”(第263页)}\end{quote}

分工使劳动的社会生产力,或者说,社会劳动的生产力获得发展,但这是靠牺牲工人的一般生产能力来实现的。所以,社会生产力的提高不是作为工人的劳动的生产力的提高,而是作为支配工人的权力即资本的生产力的提高而同工人相对立。如果说城市工人比农村工人发展,这只是由于他的劳动方式使他生活在社会之中,而土地耕种者的劳动方式则使他直接和自然打交道。

\begin{quote}{“在欧洲,城市实业活动到处都对农村实业活动占优势,这并不完全是由于行会和行会规约。这种优势还依靠许多其他的规定:对外国工业品和外国商人运来的一切商品课以高额关税,也是为了同样的目的。”(第265页)“这些规定保护着它们〈城市〉不受外国人的竞争。”(同上)}\end{quote}

这已经不是个别城市的资产阶级的行动,而是作为民族的主要部分,或者甚至作为国会的第三等级,或者作为下院,在全国范围内实行立法的那个资产阶级的行动了。城市资产阶级为了反对农村而实行的特别措施,就是消费税和入城税,一般说来,是间接税,这种间接税起源于城市(见休耳曼的著作)\endnote{马克思指的是休耳曼的书《中世纪城市》1826—1829年波恩版第1—4集。——第260页。},直接税则起源于农村。看起来,例如,消费税只是城市间接课在自己身上的税。农村居民据说必须预先缴纳这种税,但他让别人在产品的价格内把它交回来。不过在中世纪,情况并不是这样。对于农村居民劳动产品的需求,——在农村居民要把自己的产品变为商品和货币的情况下,——在多数场合,都被强制地局限于城市范围,所以农村没有可能把城市税总额加到自己产品的价格上去。

\begin{quote}{“在英国,城市实业活动对农村实业活动的优势,过去似乎比现在更大。与上世纪〈十七世纪〉和本世纪〈十八世纪〉初期相比,现在农村工人的工资和工业工人的工资更加接近了,而农业资本的利润也和工商业资本的利润更加接近了。这种变化,可以看作是城市实业活动得到特别鼓励的必然结果,尽管这种结果出现得相当晚。城市积累起来的资本,随着时间的推移,变得如此之大,以致把它投入城市固有的实业中去,已经不可能获得以前的利润了。城市固有的实业,和其他一切实业一样,都有自己的界限,而资本的增长,由于使竞争加剧,必然会降低利润。城市中利润的降低,促使资本流入农村,这就造成对农业劳动的新的需求,从而提高农业劳动的报酬。那时资本就可以说是遍布全国,并在农业中找到用途,于是原来在很大程度上是靠农村积累起来的城市资本又部分地回到了农村。”(第266—267页)}\end{quote}

在第十一章,斯密试图确定构成商品价值的第三个要素即“地租的自然率”。我们准备再回过头去谈一谈李嘉图,然后就考察这一点。

由上所述,很清楚:亚·斯密把商品的“自然价格”,或者说,费用价格和商品的价值等同起来,是由于他事先抛弃了他对价值的正确的观点,而代之以由竞争现象所引起的、来源于竞争现象的观点。在竞争中,并不是价值,而是费用价格作为市场价格的调节者,可以说,作为内在价格——商品的价值出现。而这种费用价格本身在竞争中又作为由工资、利润和地租的既定平均率决定的某种既定的东西出现。因此,斯密也就试图离开商品的价值而独立地确定工资、利润和地租,更确切地说,把它们作为“自然价格”的要素来考察。李嘉图的主要任务是推翻斯密的[560]这种谬误说法,可是他也接受了这种说法的必然的,而如果他前后一贯的话,对他说来是不可能有的后果——把价值和费用价格等同起来。

\tchapternonum{[第十一章]李嘉图的地租理论}

\tsectionnonum{[(1)安德森和李嘉图发展地租理论的历史条件]}

主要的方面在考察洛贝尔图斯的理论时已经阐明了。这里不过再作一些补充。

首先要谈的是历史环境:

李嘉图所考察的时期首先是他差不多完全亲身经历过的1770—1815年,这是小麦价格不断上涨的时期;安德森所考察的时期是十八世纪,他是在这个世纪的末叶写作的。从这个世纪初叶到中叶,小麦价格下降,从中叶到末叶,小麦价格上涨。因此,在安德森看来,他所发现的规律同农业生产率的降低或产品正常的{安德森认为是不自然的}涨价毫无联系。而在李嘉图看来,却肯定是有联系的。安德森认为,谷物法(当时是出口奖励)的废除,是引起十八世纪下半叶价格上涨的原因。李嘉图知道,谷物法(1815年)的实行是为了制止价格下降,并且必然会在一定程度上制止价格下降。因此,李嘉图着重指出,自由发生作用的地租规律必定会——在一定疆域之内——使比较不肥沃的土地投入耕种,从而使农产品价格上涨,使地租靠损害工业和广大居民的利益而上涨。李嘉图在这里无论从实际方面或历史方面来说都是对的。相反,安德森则认为,谷物法(他也赞成进口税)必然会在一定疆域内促进农业的均衡发展;农业的均衡发展需要加以保证;因此,这种前进的发展过程本身,由于安德森所发现的地租规律的作用,必然会引起农业生产率的提高,从而引起农产品平均价格的下降。

但是他们两人都是从一种在大陆上看来非常奇怪的观点出发的,这就是:(1)根本不存在妨碍对土地进行任意投资的土地所有权;(2)从较好的土地向较坏的土地推移(在李嘉图看来,如果把由于科学和工业的反作用造成的中断除外,这一点是绝对的;在安德森看来,较坏的土地又会变成较好的土地,所以,这一点是相对的);(3)始终都有资本,都有足够数量的资本用于农业。

说到(1)、(2)两点,大陆上的人们一定会感到非常奇怪:在这样一个他们看来最顽固地保存了封建土地所有权的国家里,经济学家们——安德森也好,李嘉图也好——却从不存在土地所有权的观点出发。这种情况可用以下两点来解释:

第一,英国的“公有地圈围法”有它的特点,同大陆上的瓜分公有地毫无共同之处;

第二,从亨利七世以来,资本主义生产在世界任何地方都不曾这样无情地处置过传统的农业关系,都没有创造出如此适合自己的条件,并使这些条件如此服从自己支配。在这一方面,英国是世界上最革命的国家。从历史上遗留下来的一切关系,不仅村落的位置,而且村落本身,不仅农业人口的住所,而且农业人口本身,不仅原来的经济中心,而且这种经济本身,凡是同农业的资本主义生产条件相矛盾或不相适应的,都被毫不怜惜地一扫而光。举例来说,在德国人那里,经济关系是由各种土地占有的传统关系、经济中心的位置和居民的一定集中点决定的。在英国人那里,农业的历史条件则是从十五世纪末以来由资本逐渐创造出来的。联合王国的常用术语“清扫领地”,在任何一个大陆国家都是听不到的。但是什么叫做“清扫领地”呢?就是毫不考虑定居在那里的居民,把他们赶走,毫不考虑原有的村落,把它们夷平,毫不考虑经济建筑物,把它们拆毁,毫不考虑原来农业的类别,把它们一下子改变,例如把耕地变成牧场,总而言之,一切生产条件都不是按照它们传统的样子接受下来,而是按照它们在每一场合怎样最有利于投资历史地创造出来。因此,就这一点来说,不存在土地所有权;土地所有权让资本——租地农场主——自由经营,因为土地所有权关心的只是货币收入。一个波美拉尼亚的地主\authornote{暗指洛贝尔图斯。——编者注},脑袋里只有祖传的土地占有、经济中心和农业公会等等,因而对李嘉图关于农业关系发展的“非历史”观点[561]就会大惊小怪。而这只说明他天真地混淆了波美拉尼亚关系和英国关系。可是决不能说,这里从英国关系出发的李嘉图会同那个思想局限于波美拉尼亚关系的波美拉尼亚地主一样眼光短浅。因为英国关系是使现代土地所有权——被资本主义生产改变了形式的土地所有权——得到合适发展的唯一关系。在这里,英国的观点对于现代的即资本主义的生产方式来说具有古典意义。相反,波美拉尼亚的观点却是按照历史上处于较低阶段的、还不合适的形式来评论已经发展了的关系。

不仅如此,大陆上批评李嘉图的人中,大多数甚至是从这样一种关系出发的,在这种关系内,资本主义生产方式,不论合适的或不合适的,根本还不存在。这就好比一个行会师傅想要把亚·斯密的以自由竞争为前提的规律完完全全地应用到他的行会经济上一样。

从较好的土地向较坏的土地推移这个前提,对于劳动生产力的每一个发展阶段,都是象安德森所认为的那样是相对的,而不是象李嘉图所认为的那样是绝对的;这个前提只有在象英国这样一个国家才能产生,在那里,资本在一个相对来说很小的疆域内如此残酷无情地实行统治,几百年来毫不怜惜地极力使一切传统的农业关系完全适合于自己。因此,只有在农业的资本主义生产不是象大陆那样从昨天才开始的地方,只有在它已经不用同旧传统作斗争的地方,这个前提才能产生。

第二个情况是,英国人有一种从他们的殖民地得来的观点。我们已经看到\authornote{见本册第253—254页。——编者注},李嘉图整个观点的基础在斯密的著作中——在直接论述殖民地的地方——已经有了。在这些殖民地——特别是在专门生产交易品如烟草、棉花、糖等而不生产普通食物的殖民地,在那里,殖民者一开头就不是谋生,而是建立商业企业,——具有决定意义的,在位置既定的条件下自然是肥力,在肥力既定的条件下自然是土地的位置。殖民者的做法不象日耳曼人,日耳曼人在德国住下来,是为了在那里定居,殖民者则象这样一种人,他们按照资产阶级生产的动机行事,他们想要生产商品,他们的出发点从一开头就不是决定于产品,而是决定于出卖产品。李嘉图和其他英国著作家把这种从殖民地得来的观点,也就是从本身已经是资本主义生产方式的产物的人们那里得来的观点,移到了世界历史的整个进程中来,他们象他们的殖民者一样,一般地把资本主义生产方式看作农业的先决条件,其所以如此,就是因为,他们在这些殖民地,一般说来,只是在更加鲜明的形式上,在没有同传统关系斗争的情况下,因而在没有被弄模糊的形式上,发现了在他们本国到处可以看到的资本主义生产在农业中占统治地位的同样现象。因此,如果一个德国教授或地主(他的国家和其他国家不同之点就是根本没有殖民地)认为这样的观点是“错误的”,那是完全可以理解的。

最后,资本不断从一个生产部门流入另一个生产部门这个前提,这个李嘉图的基本前提,无非就是发达的资本主义生产占统治地位这样一个前提。在资本主义生产的统治还没有建立的地方,这个前提就不存在。例如,一个波美拉尼亚地主,对于李嘉图和其他英国著作家居然没有想到农业会缺乏资本,一定感到奇怪。英国人当然会抱怨土地同资本相比显得缺乏,但是从来不抱怨资本同土地相比显得缺乏。威克菲尔德、查默斯等人想用前一种情况来说明利润率下降。没有一个英国著作家提到后一种情况,在英国,就象柯贝特当作不言而喻的事实指出的那样,资本在所有部门中始终都是绰绰有余的。如果设想一下德国的情况,设想一下土地所有者借钱时的困难,——因为他多半是自己经营农业,而不是由一个完全独立于他的资本家阶级经营农业,——那就可以理解,例如洛贝尔图斯先生为什么会对“李嘉图的虚构——资本储备适应于对投资的渴望”(《给冯·基尔希曼的社会问题书简。第三封信》1851年柏林版第211页)表示惊讶。如果说英国人有什么感到不足,那就是“活动场所”,就是供现有资本储备投放的场所。但是,在英国,对于要投资的唯一阶级即资本家阶级来说,对用于“投放”的“资本的渴望”是不存在的。

[562]这种“对资本的渴望”是波美拉尼亚人的。

英国著作家们拿来反驳李嘉图的,不是资本没有足够的储备以供各种特殊投资之用,而是资本从农业流出会遇到特殊的技术等等方面的困难。

因此,上述用大陆的批判眼光对李嘉图吹毛求疵,只是证明那些“聪明人”是从生产条件较低的阶段出发的。

\tsectionnonum{[(2)李嘉图的地租理论同他对费用价格的解释的联系]}

现在来谈问题本身。

首先,为了在纯粹的形式上理解问题,我们必须把李嘉图那里唯一存在的级差地租完全撇开。我所说的级差地租,是指由于不同等级土地的肥力不同而产生的地租量的差别——较多的或较少的地租。(如果肥力一样,级差地租只能由于投资量不同而产生。就我们研究的问题来说,这种情况不存在,与问题无关。)这种级差地租完全相当于超额利润,就是在每一工业部门,例如在棉纺业中,在市场价格既定时,或者更确切地说,在市场价值既定时,生产条件比这个生产部门的平均条件好的那个资本家赚得的超额利润,因为一定生产领域的商品的价值不是决定于单个商品所耗费的劳动量,而是决定于在该领域的平均条件下生产的那个商品所耗费的劳动量。这里,工业和农业不同之处只是:在工业中超额利润落进资本家自己的腰包,而在农业中落进土地所有者的腰包;其次,超额利润在工业中是流动的、不稳定的,时而由这个资本家赚去,时而由那个资本家赚去,并且又不断地消失,而超额利润在农业中,却由于有土地差别这种稳定的(至少在一段较长的时间内)自然基础而固定下来。

总之,我们要把这种级差地租撇开,但是要指出,不论是从较好的土地向较坏的土地推移,还是从较坏的土地向较好的土地推移,级差地租同样是可能的。在两种情况下只假定,为了满足追加需求,新耕地是必要的,但是它只要够满足追加需求就行了。假如新耕种的较好的土地能够满足的需求大于这个追加需求,那末,按照追加需求的大小,必将有部分或全部坏地停止耕种,至少在这些土地上不再种植成为农业地租的基础的产品,也就是说,在英国不再种植小麦,在印度不再种植水稻。因此,级差地租并不以农业的不断恶化为前提,它也可以从农业的不断改良产生。即使在级差地租以向较坏土地推移为前提的地方,第一,这种按下降序列推移可能是由于农业生产力的改良,因为,在需求所容许的价格之下,只有较高的生产力才使耕种较坏的土地成为可能。第二,较坏的土地可以改良,不过差别仍然会存在,尽管这个差别在很大程度上被抵销了,结果,发生的只是生产率的相对的、比较的降低,可是绝对的生产率提高了。这甚至是第一个提出李嘉图规律的安德森的前提。

其次,这里应当考察的仅仅是真正的农业地租,就是提供主要植物性食物的土地的地租。斯密已经说明,提供其他产品(例如畜产品等等)的土地的地租,是由上述地租决定的,因而已经是派生的地租,它们由地租规律决定,而不是决定地租规律;所以就其本身来考察,它们是不能提供任何材料来理解最初的、纯粹的条件下的地租规律的。其中没有什么第一性的东西。

上述这些解决了之后,问题就归结为:是否存在绝对地租?就是说,是否存在由资本投入农业而不是投入工业产生的、同投入较好土地的资本所提供的级差地租即超额利润完全无关的地租?

很清楚,李嘉图既然从商品价值和商品平均价格等同这个错误前提出发,他理所当然地要对这个问题作否定的答复。如果接受这个前提,那末,下面的说法便是同义反复:如果[563]农产品的固定价格除了提供平均利润外还提供地租,提供一个超过这个平均利润的经常余额,那末农产品的价格就高于它们的费用价格,因为这个费用价格等于预付加平均利润,再无其他。如果农产品的价格高于它们的费用价格,必然提供一个超额利润,那末,按照上述前提,农产品的价格也就会高于它们的价值。这除了承认农产品经常高于它们的价值出卖以外,就再没有别的了,但是这也就等于假定其他一切产品都是低于它们的价值出卖,或者说,一般说来价值同从理论上对它的必然的理解是完全不同的东西。同量劳动(直接劳动和积累劳动)——把各个资本之间由于它们在流通过程中产生的差别而发生的一切平均化现象都考虑进去——在农业中生产的价值会比在工业中生产的价值高。因而商品的价值就不是由商品中包含的劳动量来决定了。这样一来,政治经济学的整个基础就被推翻了。因此,李嘉图理所当然地得出结论说,不存在绝对地租。只可能有级差地租;换句话说,最坏土地所生产的农产品的价值,同其他任何商品的价值一样,等于产品的费用价格。投在最坏土地上的资本,是一种仅仅在投资方式上,仅仅作为特种投资,与投在工业中的资本不同的资本。因此这里表现出价值规律的普遍适用性。级差地租——而这是较好土地上的唯一地租——不过是生产条件比平均条件好的资本由于在每一个生产领域有一个相同的市场价值而提供的超额利润。这种超额利润,由于农业的自然基础,只有在农业中才固定下来;而且,因为这个自然基础的代表是土地所有者,所以这种超额利润不是落入资本家的腰包,而是落入土地所有者的腰包。

李嘉图的费用价格等于价值这个前提不成立,他的所有这些论证也就不成立。那种迫使他否定绝对地租的理论兴趣也就丧失。如果商品的价值不同于商品的费用价格,如果所有商品必然分成三类:一类商品的费用价格等于它们的价值,另一类商品的价值低于它们的费用价格,第三类商品的价值高于它们的费用价格,那末,农产品价格提供地租这种情况,只不过证明农产品属于价值高于费用价格的一类商品。唯一有待解决的问题是:为什么农产品跟其他那些价值同样高于费用价格的商品不同,它们的价值不因资本的竞争而降低到它们的费用价格的水平?答案已经包含在问题里了。因为,按照假定,这种情况只有在资本的竞争能够实现这种平均化的时候才发生,而实现平均化又只有在一切生产条件由资本本身创造出来,或者作为自然要素同样受资本支配的时候才有可能。对土地来说不发生这种情况,因为存在着土地所有权,资本主义生产是在存在土地所有权的前提下开始的,而土地所有权不是从资本主义生产中产生的,它在资本主义生产之前就已经存在。因此,单单土地所有权的存在本身就给问题作了答复。资本所能做的一切,就是使农业服从资本主义生产的条件。但是,资本主义生产不能剥夺土地所有权占有一部分农产品的可能性,这部分农产品资本要据为己有,就不是靠它自己的活动,而只有靠没有土地所有权存在这个前提。在土地所有权存在的条件下,资本就不得不把价值超过费用价格的余额让给土地所有者。但是,这个价值和费用价格之间的差额本身,仅仅是从资本有机组成部分的比例不同产生出来的。因此,凡是按照这种有机构成价值高于费用价格的商品都表明,同价值等于费用价格的商品相比,生产它们的劳动生产率相对地说比较低,而同价值低于费用价格的商品相比,劳动生产率则更低;这是因为,它们需要较大量的直接劳动(同包含在不变资本中的过去劳动相比),需要有较多的劳动去推动一定量资本。这个差别是历史性的,因此是会消失的。正是那个证明绝对地租可能存在的论据也证明,绝对地租的现实性、绝对地租的存在仅仅是一个历史事实,是农业的一定发展阶段所特有的、到了更高阶段就会消失的历史事实。

李嘉图用农业生产率的绝对降低来说明级差地租,而这种降低完全不是级差地租的前提,安德森也没有把它当作前提。李嘉图否定绝对地租,这是因为他[564]以工业和农业的资本有机构成相同为前提,从而他也就否定了农业劳动生产力同工业相比处于只是历史地存在的较低发展阶段。因此他犯了双重历史错误:一方面,把农业和工业中的劳动生产率看成绝对相等,因而否定它们在一定发展阶段上的仅仅是历史性的差别,另一方面,认为农业生产率绝对降低,并把这种降低说成是农业的发展规律。他这样做一方面是为了把较坏土地的费用价格同价值等同起来;另一方面是为了说明较好土地的产品的[费用]价格同价值之间存在差额。全部错误的产生都是由于混淆了费用价格和价值。

这样,李嘉图的理论也就被排除了。其他方面,我们在前面考察洛贝尔图斯的理论时已经说过了。

\tsectionnonum{[(3)李嘉图的地租定义不能令人满意]}

我已经指出\authornote{见本册第185页。——编者注},李嘉图在论地租的那一章一开头就说,应当研究“对土地的占有以及由此而来的地租的产生”(《政治经济学和赋税原理》1821年伦敦第3版第53页)是否同价值决定于劳动时间这一规定相矛盾。接着他又说:

\begin{quote}{“亚当·斯密认为,调节商品交换价值的基本尺度,即生产商品所用的相对劳动量,会由于土地的占有和地租的支付而完全改变,这个看法不能说是正确的。”(第67页)}\end{quote}

李嘉图把地租理论同价值规定直接地、有意识地联系起来,这是他的理论贡献。在其他方面,第二章《论地租》可以说比威斯特的论述还要差。这里有许多值得怀疑的东西,有petitioprincipii〔本身尚待证明的论据〕以及对待问题的不公正态度。

就真正的农业地租——这里,李嘉图把这种地租正确地看作是真正意义上的地租——来说,地租是为了获得许可在土地这个生产要素上投资,以资本主义方式进行生产而支付的东西。土地在这里是生产要素。至于例如建筑物、瀑布等的地租,情况就不同了。这里,为了加以使用而支付地租的自然力,是作为生产条件参加生产的,不论是作为生产力或者是作为不可缺少的条件,但是它们不是这一特定生产领域本身的要素。其次,说到矿山、煤矿等的地租,土地则是可从其中挖掘使用价值的储藏库。这里为土地支付地租,并不是因为土地象在农业中那样作为可以在其上进行生产的要素,也不是因为土地象瀑布和建筑地段那样作为生产条件之一加入生产过程,而是因为土地作为储藏库蕴藏着有待通过生产活动来取得的使用价值。

李嘉图的定义:

\begin{quote}{“地租是为使用土地原有的和不可摧毁的力而付给土地所有者的那一部分土地产品。”(第53页)}\end{quote}

这是不能令人满意的。第一,土地并没有“不可摧毁的力”。(关于这一点在本章末尾要作个注。)第二,土地也不具有“原有的”力,因为土地根本就不是什么“原有的”东西,而是自然历史过程的产物。但是,我们且不管这个。所谓土地的“原有的”力,在这里应该理解为土地不依赖于人的生产活动而具有的力,虽然从另一方面说,通过人的生产活动给它的力,完全同自然过程赋予它的力一样要变成它的原有的力。除此以外,下面这一点还是对的,即地租是为“使用”自然物而支付的,完全不管这里所说的是使用土地的“原有的力”,还是瀑布落差的能量,或者是建筑地段,或者是水中或地下蕴藏的有待利用的宝藏。

为区别于真正的农业地租,亚·斯密(李嘉图指出)谈到为原始森林的木材支付的地租,谈到为煤矿和采石场支付的地租。李嘉图排除这种地租的方法是相当奇怪的。

李嘉图开头说不应该把资本的利息和利润同地租混淆起来(第53页),这种资本是指

\begin{quote}{“原先用于改良土壤以及建造为储存和保管产品所必需的建筑物而支付的资本”。(第54页)}\end{quote}

李嘉图从这里立刻转到上面提到的亚·斯密所举的例子。关于原始森林,李嘉图说:

\begin{quote}{“但是,支付他〈斯密〉所谓的地租的人,是为了当时已经长在地上的有价值的商品而支付这个地租的,而且通过出卖木材实际上已收回自己所付的钱并获得利润,这不是很明显的吗?”(第54页)}\end{quote}

关于采石场和煤矿的情况也是一样:

\begin{quote}{“为[565]煤矿或采石场支付的报酬,是为了可以从那里开采的煤或石料的价值而支付的,它和土地原有的和不可摧毁的力没有任何关系。这种区别在地租和利润的研究中极为重要;因为很清楚,决定地租发展的规律同决定利润发展的规律是大不相同的,并且也很少朝着相同的方向发生作用。”(第54—55页)}\end{quote}

这是非常奇怪的逻辑。李嘉图说,要把为使用“土地原有的和不可摧毁的力”而付给土地所有者的地租,同那为了他在改良土地等方面的投资而付给他的利息和利润区别开来。为了取得“采伐”木材的权利而付给自然森林所有者的“报酬”,或为了取得“开采”石料和煤的权利而付给采石场和煤矿所有者的“报酬”,不是地租,因为它不是为“使用土地原有的和不可摧毁的力”而支付的。很好!可是李嘉图在他的议论中却把这种“报酬”说成好象同那为改良土地而进行的投资的利润和利息是一回事!而这是完全错误的!原始森林所有者向“原始森林”投过“资本”让它生产“木材”吗?或者,采石场和煤矿的所有者向采石场和煤矿投过“资本”让它们蕴藏“石料”和“煤”吗?那末他得到的“报酬”来自何处呢!这种报酬在任何场合都不象李嘉图想偷换的那样是资本的利润或利息。因此,它是“地租”,而不是别的,尽管它不是李嘉图的地租定义所指的那种地租。但是,这不过表明李嘉图的地租定义排除了某些形式,在这些形式中,“报酬”是为了不体现任何人的劳动的单纯自然物而支付的,并且是支付给这些自然物的所有者,而且仅仅因为他是个“所有者”,是土地所有者,不管这块土地是耕地、森林、鱼塘、瀑布、建筑地段等等。但是,李嘉图说,为了取得在原始森林中伐木的权利而支付的人,支付“是为了当时已经长在地上的有价值的商品,而且通过出卖木材实际上已收回自己所付的钱并获得利润。”且慢!如果李嘉图这里把原始森林中“长在地上的”树木称作“有价值的商品”,这不过是说,它就可能性来说是使用价值。这个使用价值在这里用“有价值的”一词表达出来。但是它不是“商品”。因为要成为商品,它就必须同时是交换价值,就是说,它必须是耗费在它上面的一定量劳动的体现。只是由于把它从原始森林分离开来、伐倒、搬动、运走,由树干变成木材,它才变成商品。或者说,它变成商品,仅仅因为被出卖吗?这样的话,耕地岂不是也可以仅仅因为出卖的行为就变成商品了吗?

因而,我们就应该说:地租是为了取得使用自然力或者(通过使用劳动)占有单纯自然产品的权利而付给这些自然力或单纯自然产品的所有者的价格。实际上,这也就是所有地租最初表现的形式。但是这样一来,就还有一个问题要解决:没有价值的东西怎么会有价格,这又怎么同一般价值理论一致。至于为取得在生长树木的土地上采伐木材的权利而支付“报酬”的人抱什么目的,这个问题同实际的问题毫无关系。问题是:他是用什么基金支付的?李嘉图说,“通过出卖木材”,也就是说用木材的价格。而且这个价格,照李嘉图说,使这个人“实际上已收回自己所付的钱并获得利润”。因此,现在我们知道问题究竟在哪里了。木材的价格至少必须等于代表伐木、搬动、运输和把木材送到市场所必需的劳动量的货币额。那末,这个人在“收回自己所付的钱”时获得的利润,是不是这个价值的附加额,这个只是现在由耗费在木材上的劳动赋予木材的交换价值的附加额呢?如果李嘉图这样说的话,他就退到低于他自己的学说水平的最粗俗的观念上去了。决不是的。假定这个人是一个资本家,利润就是他在“木材”生产上使用的劳动中他没有付酬的部分,我们可以说,如果这个人把同量劳动用在棉纺工厂中,他会赚到同量利润。(如果这个人不是资本家,那末利润等于他超出补偿其工资之外的那部分劳动量,这部分劳动量,如果有一个资本家雇用他的话,就会成为资本家的利润,而现在却成为他自己的利润,因为他既是他自己的雇佣工人,又是他自己的资本家,一身兼而有之。)但是这里用了荒谬的说法,说这个木材业者“实际上已收回自己所付的钱并获得利润”。这就使整个事情具有十分平庸的性质,同这个经营木材的资本家自己对他的利润来源所能持有的粗俗观念相吻合。他首先为树木的使用价值向原始森林的所有者支付报酬,但是树木是没有“价值”(交换价值)的,并且,只要它还“长在地上”,它就连使用价值都没有。假定他向原始森林所有者每吨支付5镑。然后他按6镑(他的其他费用不计在内)把这些木材卖给别人,这样实际上收回5镑并获得20%的利润。“实际上已收回自己所付的钱并获得利润。”如果原始森林所有者只要2镑(40先令)“报酬”,木材业者就会按每吨2镑8先令而不是按6镑卖出去。[566]因为他总是按同一利润率来加价的,所以这里木材价格的高低取决于地租的高低。地租是作为构成要素加入价格,而决不是价格的结果。不论支付“地租”(“报酬”)给土地所有者是为了使用土地的“力”,还是为了“使用”土地的“自然产品”,都丝毫不改变经济关系,不改变它是为过去没有花费过人的劳动的“自然物”(土地的力或产品)支付的。这样,李嘉图在他《论地租》一章的第二页上,为了回避困难,就推翻了他的整个理论。看来,亚·斯密在这个问题上的见解要透彻得多。

关于采石场和煤矿,情况也是一样。

\begin{quote}{“为煤矿或采石场支付的报酬,是为了可以从那里开采的煤或石料的价值而支付的,它和土地原有的和不可摧毁的力没有任何关系。”(第54—55页)}\end{quote}

没有任何关系!但是这种报酬和“土地原有的和可以摧毁的产品”有很重要的关系。这里的“价值”一词同前面的“已收回自己所付的钱并获得利润”同样荒谬。

李嘉图从来不用价值这个词来表示效用或有用性或“使用价值”。因此,他是不是想说,把“报酬”付给采石场和煤矿所有者,是为了煤和石料在它们从采石场和煤矿开采出来以前即在它们的原始状态就有的“价值”呢?如果是这样,李嘉图就推翻了他的整个价值学说。或者,就象本来应当说的那样,价值在这里是指煤和石料的可能的使用价值,因此也就是它们的预期的交换价值呢?如果是这样,这就不过是说,把地租付给煤和石料的所有者是为了获得许可使用“土地的原有成分”来开采煤和石料。可是,为什么这不应当象为了获得许可使用土地的“力”来生产小麦时一样也叫作“地租”呢,这就完全不能理解了。不然的话,我们又会看到象前面在木材的例子上分析过的那种推翻整个地租理论的情况了。按照正确的理论,问题完全没有困难。用在“生产”{不是再生产}木材、煤和石料上的劳动(这种劳动的确没有创造这些自然产品,但是它把这些自然产品从它们同土地的原始联系中分离开来,因而把它们作为可用的木材、煤和石料“生产”出来)或资本显然属于这样的生产领域,在这些生产领域中,资本中用于工资的部分大于用于不变资本的部分,直接劳动大于“过去”劳动——其成果用作生产资料。因此,如果商品在这里按照它的价值出卖,这个价值就高于它的费用价格,就是说高于工具的磨损、工资和平均利润。所以,余额可以作为地租付给森林、采石场或煤矿的所有者。

但是,为什么李嘉图要耍这些拙劣的手法,错误地使用“价值”这个词等等呢?为什么他死抓住这样的地租定义即地租是为使用“土地原有的和不可摧毁的力”而支付的呢?我们在后面也许会找到答案。无论如何,李嘉图是想把真正的农业地租区分出来,强调它的特点,同时指出,这些原有的力只有当它们达到不同的发展程度时才能得到报酬,借此为级差地租奠定基础。

\tchapternonum{[第十二章]级差地租表及其说明}

\tsectionnonum{[(1)地租量和地租率的变动]}

对于前面所说的还要补充如下:

假定发现了比较富饶的或位置较好的煤矿和采石场,它们在使用同量劳动的情况下比老的煤矿和采石场能提供更多的产品,并且产量足以满足全部需求。这时,煤炭、石料和木材的价格就会下降,因为它们的价值会下降。老的煤矿和采石场必然因此停闭。它们将不能提供利润,不能提供工资,也不能提供地租。然而新的煤矿和采石场必然会象以前老的那样提供地租,尽管提供的(从地租率上看)比较少些。因为,劳动生产率每提高一步,花费在工资上的资本同不变资本(这里是指用在生产工具上的资本)对比起来就减少。这种说法对吗?如果劳动生产率的变动不是生产方式本身的变动引起,而是煤矿或采石场的自然富饶程度或它们的位置引起,这种说法对吗?在这里我们唯一能够说的就是,同量资本在这里提供吨数更多的煤炭或石料,因此,在每一吨中包含较少的劳动,但是,所有吨数加在一起就包含同样多的或者甚至更多的劳动,——如果新的煤矿或采石场除了满足以前由老的煤矿或采石场满足的原有需求以外,还能满足一个追加的需求,即比新老矿、场富饶程度的差额还要大的需求。可是使用的资本的有机构成并不因此改变。的确,在一吨的价格中,在单独的一吨的价格中,将包含较少的地租,但这只是因为一般说来在其中包含较少的劳动,也就是包含较少的工资和较少的利润。可是,地租率对利润之比并不因此受到影响。因此,我们只能[567]说:

如果需求不变,也就是说,如果要生产同以前一样多的煤炭和石料,那末,为了生产同一商品量,现在在新的比较富饶的煤矿和采石场中使用的资本,就比以前在老的矿、场中使用的少。于是商品总量的总价值就下降,地租、利润、工资和使用的不变资本的总量也因此减少。但是地租和利润之间的比例,就象利润和工资之间的比例或利润和投资之间的比例一样不会改变,因为在使用的资本中没有发生任何有机的变动。改变了的只是使用的资本的量,不是使用的资本的构成,因而也不是生产方式。

如果有追加需求要满足,但是这个追加需求等于新老矿、场富饶程度的差额,那就使用和以前同样大小的资本。每一吨的价值减少了。但是总吨数仍有和以前同样的价值。就每一吨来看,随着其中包含的价值的减少,价值中转化为利润和地租的那部分的量也减少。但是,因为资本的量以及它的产品的总价值没有变,资本构成中也没有发生有机的变动,所以地租和利润的绝对量不变。

如果追加需求很大,在投资照旧的条件下,新老矿、场富饶程度的差额不能满足这一需求,那末在新矿中必须使用追加资本。在这种情况下,——如果在分工和机器使用方面没有随着总投资的增加而发生变动,也就是说,如果资本有机构成没有任何变动,——地租和利润的量就增加,因为总产品的价值、总吨数的价值增加了,尽管每一吨的价值减少了,就是说每一吨价值中转化为地租和利润的那一部分也减少了。

在所有这些情况下,地租率都没有发生任何变动,因为使用的资本的有机构成没有变动(不论资本的量如何变动)。相反,如果变动是由于资本有机构成的变动,是由于花费在工资上的资本同花费在机器等方面的资本相比有所减少,——因而生产方式本身也发生变动——那末地租率就会下降,因为商品价值和费用价格之间的差额缩小了。在上面考察的三种情况中,这个差额并没有缩小。因为,如果价值下降,那末,由于在单位商品上耗费的劳动(有酬劳动和无酬劳动)较少,单位商品的费用价格则同样下降。

由此可见,如果劳动生产率提高——或者说,生产出来的一定量商品的价值减少——仅仅是由自然要素的富饶程度的变动引起的,是由土地、矿山、采石场等的自然富饶程度不同引起的,那末,地租量可以由于在改变了的条件下使用的资本量减少而减少;地租量可以由于有追加需求而保持不变;地租量可以由于追加需求大于原来使用的自然因素和现在使用的自然因素的富饶程度之间的差额而增长。但是,地租率只有在使用的资本的有机构成发生变动的情况下才能增长。

因此,当放弃较坏的土地、较次的采石场、较次的煤矿等的时候,地租量不一定下降。而且,如果这种放弃只是它们的自然富饶程度较低的结果,地租率甚至永远不会下降。

在这种场合,说地租量在一定的需求情况下可能下降,就是说,地租量的变动取决于使用的资本量是减少、不变还是增加,这是正确的看法。但是,说地租率一定下降,那是根本错误的看法,在这种前提下,这是不可能的,因为已经假定,资本有机构成没有发生任何变动,也就是说,没有发生足以使价值和费用价格之间的比例受到影响的变动,而这个比例是决定[绝对]地租率的唯一比例。李嘉图十分荒谬地把正确的看法同根本错误的看法混在一起了。

\tsectionnonum{[(2)级差地租和绝对地租的各种组合。A、B、C、D、E表]}

但是,在上述场合,级差地租是什么情况呢?

假定,开采的煤矿有I、II、III三个等级,其中I提供绝对地租,II提供的地租两倍于I,III提供的地租两倍于II,或四倍于I。在这种场合,I提供绝对地租R,II提供地租2R,III提供地租4R。假定现在开采IV,它比I、II、III更富饶,按其规模来说,可以容纳与投入I的资本同样大小的资本。在这种场合,如果需求不变,以前投入I的资本就投入IV。于是I将停闭。投入II的资本有一部分必然抽出。IV足以代替I并代替II的一部分,但是,如果II的一部分不继续开采,III和IV就不能满足全部需求。为了用具体例子说明这一切,我们假定,IV使用的资本同以前投入I的资本一样多,它能提供I的全部产量和II的一半产量。因此,如果对II投入原来资本的一半,对III投入原来的资本,加上投在IV上的新资本,就足以供给整个市场。

[568]在这种情况下,发生的变化是什么呢,或者说,这些变化对地租总额,对I、II、III、IV的地租有什么影响呢?

从IV得到的绝对地租的量和率,同以前从E得到的完全相同;实际上,以前在I、II、III中,绝对地租的量和率本来就是相同的,如果我们始终假定这些不同的等级使用的是同量资本。IV的产品的价值和以前I的产品的价值完全相等,因为它是大小相同和有机构成相同的资本的产品。因此,价值和费用价格之间的差额必定相同;因而地租率也必定相同。此外,地租量也必定相同,因为——在地租率既定的情况下——使用的是同样大小的资本。但是,因为煤的[市场]价值不决定于IV所生产的煤的[个别]价值,所以IV就提供超额地租,或者说,提供超过它的绝对地租的余额;这种地租,不是来自价值和费用价格之间的差额,而是来自IV的产品的市场价值和个别价值之间的差额。

如果我们说,在投入I、II、III、IV的资本量相同,因而在地租率既定时地租量也相同的条件下,它们的绝对地租,或者说,价值和费用价格之间的差额,是相同的,那末这句话应当理解为:煤的(个别)价值,I高于II,II高于III,因为在I的一吨煤中比II的一吨煤中包含较多的劳动,在II的一吨煤中比III的一吨煤中包含较多的劳动。但是,既然资本的有机构成在三种场合都是一样,这个差别就不影响I、II、III提供的个别绝对地租。因为,I的一吨的价值较大,它的费用价格也较大;大的程度,只是同I生产一吨所用的具有同样有机构成的资本大于II的程度、II大于III的程度成比例。因此,它们的价值的这个差别恰恰等于它们之间的费用价格的差别,就是说,等于在I、II、III中为生产一吨煤所花费的相对资本的差别。因此,三个等级的价值量的差别不影响这些不同等级的价值和费用价格之间的差额。如果价值较大,费用价格也相应地较大,因为价值的增大,只是同资本或劳动耗费的增大成比例;因此价值和费用价格之间的比例仍然不变,也就是说绝对地租仍然不变。

但是,我们进一步看看级差地租是什么情况。

首先,在II、III、IV的煤的全部生产上现在用了较少的资本。因为IV的资本同I的资本一样大,而用在II上的资本抽出一半;因此,II的地租量无论如何减少一半。在投资方面只有II发生了变化,因为IV的投资同以前I的投资一样大。此外,我们曾经假定,对I、II、III投入的是等量资本,例如,都是100镑,合计是300镑;因而现在II、III、IV总共只有250镑,换句话说,有六分之一的资本已经从煤的生产中抽出。

其次,煤的市场价值下降了。我们前面看到,I提供R,II提供2R,III提供4R。假定,I花费100镑生产出来的产品价值等于120镑,其中10镑是地租,10镑是利润,那末,II的市场价值是130镑(10镑利润和20镑地租),III的市场价值是150镑(10镑利润和40镑地租)。如果I的产品等于60吨(每吨等于2镑),那末,II的产品等于65吨,III的产品等于75吨,总产量等于60+65+75=200吨。现在,因为IV的100镑生产出来的产品等于I的产品的全部和II的产品的一半,就是60+[32+(1/2)]=92+(1/2)吨,那末,这92+(1/2)吨照原来的市场价值就值185镑,因为利润等于10镑,所以提供的地租是75镑;因为绝对地租等于10镑,所以IV的地租量就等于[7+(1/2)]R。

同以前一样,II、III、IV生产的还是200吨煤,因为[32+(1/2)]+75+[92+(1/2)]=200吨。但是,现在市场价值和级差地租又是什么情况呢?

要答复这个问题,我们就要看看II的绝对个别地租量多大。我们假定,在这个生产领域中价值和费用价格之间的绝对差额等于10镑,就是说,等于原来最次矿提供的地租,——虽然情况不一定是这样,除非I的价值绝对地决定市场价值。[569]如果实际上发生这种情况,那末I的地租(在I的煤按其价值出卖的情况下)一般说来就代表这个生产领域的价值超过它[I的煤]自己的费用价格和商品的一般费用价格的余额。因此,如果II把它的65吨卖120镑,也就是每吨卖1+(11/13)镑,II就是按照产品的价值出卖自己的产品。过去它的一吨所以不卖1+(11/13)镑,而卖2镑,那只是因为存在一个由I决定的市场价值超过它[II的煤]的个别价值的余额,存在它[II的煤]的市场价值(而不是它的价值)超过它的费用价格的余额。

其次,根据假定,II现在出卖的不是65吨,而只是32+(1/2)吨,因为投入煤矿的资本不是100镑,而只是50镑。

因此,II现在出卖32+(1/2)吨得到的是60镑。10镑对50镑[预付资本]之比是20%。60镑中有5镑是利润,5镑是地租。

这样,II的情况是:每吨产品价值1+(11/13)镑;吨数32+(1/2)吨;总产品价值60镑;地租5镑。地租从20镑降到5镑。如果还是用同量资本,地租就只降到10镑。因而地租率只降了一半。换句话说,地租减少的数目,等于由I决定的市场价值超过II的煤的自身价值的全部差额,或者说,等于II的煤的自身价值和它的费用价格之间的差额之上的余额。它的级差地租以前等于10镑;现在它的全部地租等于10镑,也就是等于它的绝对地租。因此,在II中,随着市场价值降到(II的煤的)价值,级差地租消失了,从而,由于这种级差地租的存在而膨胀和加倍了的地租率也消失了。地租率从20降到10。其次,地租从10降到5,因为在地租率既定时,投入II的资本减少了一半。

既然市场价值现在决定于II的煤的价值,即每吨1+(11/13)镑,那末,III所生产的75吨的市场价值现在就等于138+(6/13)镑,其中地租是28+(6/13)镑。以前地租是40镑;因此,地租减少了11+(7/13)镑。以前地租超过绝对地租30镑,现在只超过18+(6/13)镑(因为18+(6/13)+10=28+(6/13))。以前地租等于4R,现在只等于2R+[8+(6/13)]镑。因为投入III的资本量没有变,所以地租的这种下降完全是由于级差地租率的下降,也就是由于III的煤的市场价值超过它的个别价值的余额的减少。以前,III的地租总额等于较高的市场价值超过费用价格的余额,现在它只等于较低的市场价值超过费用价格的余额。\endnote{关于地租总额(绝对地租和级差地租加在一起)等于市场价值和费用价格之间的差额这个原理,马克思在后面作了更详细的考察(见本册第328—329页)。——第286页。}因此这个差额接近于III的绝对地租。III用100镑资本生产75吨煤,其[个别]价值等于120镑;因而一吨等于1+(3/5)镑。可是III过去是按以前的市场价格出卖,一吨卖2镑,即贵2/5镑。75吨总共贵2/5×75=30镑,这实际上就是III的地租总额中的级差地租;因为它的地租等于40镑(10镑绝对地租,30镑级差地租)。现在III按新的市场价值一吨只卖1+(11/13)镑。III的一吨煤的这个价格超过它的[个别]价值多少呢?3/5=39/65,11/13=55/65。因此,III的每吨煤卖得比它的[个别]价值贵16/65镑。\endnote{16/65镑这个数,是从每吨煤的新的市场价值1+(11/13)镑减去等级III每吨煤的个别价值1+(3/5)镑而得出来的。——第286页。}75吨总共贵18+(6/13)镑,这个数目恰好是现在的级差地租,因此,级差地租总是等于吨数与每吨市场价值超过每吨[个别]价值的余额的乘积。现在还要计算地租怎么减少了11+(7/13)镑。市场价值超过III的煤的价值的余额,从每吨2/5镑(当时每吨按2镑出卖)降到每吨16/65镑(现在每吨按1+(11/13)镑出卖),也就是从26/65降到16/65,即降低10/65了镑。75吨总共降低了750/65=150/13=11+(7/13)镑,这个数目恰好是III的地租减少的数目。

[570]IV的92+(1/2)吨按1+(11/13)镑的价格计算共值170+(10/13)镑。这里,地租是60+(10/13),而级差地租50+(10/13)是镑。如果92+(1/2)吨按自己的价值出卖,即按120镑出卖,则每吨值1+(11/37)镑。可是现在它按1+(11/13)出卖。而11/13=407/481,11/37=143/481。这里得出IV的煤的市场价值超过它的价值的余额是264/481镑。92+(1/2)吨的余额恰恰是50+(10/13)镑,即的级差地租。

我们现在用A表和B表把这两种情况作一对比:

\todo{}

\todo{}

这两个表使我们有理由去做一些非常重要的考察。

首先我们看到,绝对地租的数额,同投入农业\endnote{马克思在上面所举的例子不是指农业,而是指开采富饶程度不同的煤矿。但是关于这些煤矿所谈的一切,也同样适用于在肥力不同的土地上经营的农业。——第287页。}的资本,同投在I、II、III的资本总额成比例地增减。这个绝对地租的比率完全不取决于所投资本的大小,因为它同土地等级的差别完全无关,相反,它是由价值与费用价格之间的差额产生的,而这个差额本身决定于农业资本的有机构成,决定于生产方式,而不决定于土地。在II中,绝对地租的数额现在从10减到5,这是因为资本从100减到50,有半数[571]资本已经抽出。

在我们进一步考察这两个表之前,我们再列出几个表。我们看到,在B中市场价值降到每吨1+(11/13)镑。但是,按这个价值,AI的产品不必从市场上完全消失,BII也不必只使用原来资本的一半。因为在I中,商品总价值为120镑,地租等于10镑,即等于总价值的1/12,所以这对于每一吨价值(等于2镑)也是适用的。但是2/12镑等于1/6镑或3+(1/3)先令(3+(1/3)先令×60=10镑)。因此,I的每吨的费用价格是{2镑-[3+(1/3)]先令,即}1镑16+(2/3)先令。[新的]市场价值是1+(11/13)镑或1镑16+(12/13)先令。但是16+(2/3)先令等于16先令8便士,或16+(26/39)先令。与此相比,16+(12/13)(或16+(36/39))先令多了10/39先令。这个数目是在新的市场价值下每吨的地租,60吨的地租总数是15+(5/13)先令。因此,地租还不到资本100镑的1%。要AI完全不提供地租,市场价值必须降到它的[这个等级的]费用价格的水平,就是降到1镑16+(2/3)先令,或1+(5/6)镑(或1+(10/12)镑)。在这种场合,AI的地租就会消失。但是它仍然可以开采,提供10%的利润。只是在市场价值进一步降到1+(5/6)镑以下的时候,才会停止开采。

至于BII,在B表中假定有一半资本从生产中抽出。但是,因为市场价值1+(11/13)镑还能提供10%的地租,所以这个市场价值不论对100镑资本还是对50镑资本都同样提供这种地租。因此,假定抽出一半资本,那末这只是因为在这种条件下BII还能提供10%的绝对地租。事实上,如果BII继续生产65吨而不是生产32+(1/2)吨的话,市场将会负担过重,在IV的煤支配着市场的情况下,市场价值将会下降,以致必须减少对BII的投资,才能使它提供绝对地租。可是很明白,在全部资本100镑提供9%的地租时,地租总额会比在资本50镑提供10%的地租时大。因此,如果根据市场情况,为了满足现有的需求对II只需投50镑资本,那末地租必定会降到5镑。但是,假定追加的32+(1/2)吨不能找到经常的销路,因而被挤出市场,那末地租实际上会降得更低。市场价值将降到不仅使BII的地租消失,并且使利润也受到影响。这时就会抽出资本以减少供给,直至资本减少到50镑这个恰当的数额为止,这时市场价值将稳定在1+(11/13)镑上,同时市场价值又为BII提供绝对地租,但是只为以前投资的半数提供绝对地租。就是在这种场合,起决定作用的也是支配着市场的IV和III。

但是,如果市场在每吨价格为1+(11/13)镑时只能吸收200吨,这决不能说,当市场价值下降的时候,即由于追加的32+(1/2)吨对市场的压力,232+(1/2)吨的市场价值降低的时候,市场就不能再多吸收32+(1/2)吨了。BII每吨的费用价格是[110∶65,即]1+(9/13)镑或1镑13+(11/13)先令,而市场价值是1+(11/13)镑或1镑16+(12/13)先令。如果市场价值降到AI不再能提供地租,就是说,如果降到AI的费用价格的水平,降到1镑16+(2/3)先令,或1+(5/6)镑,即1+(10/12)镑,那末,为了使BII用上全部资本,需求就必须大大增加,因为AI由于提供普通利润,还可能继续开采。市场可能不是要多吸收32+(1/2)吨,而是要多吸收92+(1/2)吨,不是吸收200吨,而是吸收292+(1/2),因此[几乎]多了一半。这必须以需求已有极大增加为前提。就是说,为了使我们假定的需求的增加不是太大,市场价值应当降到把AI挤出市场。换句话说,市场价格应当降到低于AI的费用价格,即低于1+(10/12)镑,比如说,降到1+(9/12)镑即1镑15先令。在这之后,市场价格仍然大大高于BII的费用价格。

因此,我们在A表和B表之外再加上三个表:C表、D表和E表。在C表中我们假定,需求的增加使A表和B表中的所有等级都能继续生产,但是按照B的市场价值,同时AI还提供地租。在D表中我们假定,需求量足以使AI不再提供地租,但是还提供普通利润。在E表中我们假定,价格降到把AI挤出市场,[572]但是同时,价格的降低能使市场吸收BII的追加的32+(1/2)吨。

A表和B表中所假定的情况是可能的。可能有这样的情况:AI在地租从10镑降到不足16先令时停止对自己的土地的这种利用,而把它出租,另作他用,这样,它可以提供较高的地租。但是,在这种场合,如果市场不是随着新的市场价值的形成而扩大,BII就不得不由于上面描写的过程而抽出它的一半资本。

\todo{}

\todo{}

\todo{}

[573]现在,我们把A、B、C、D、E表排成一个总表,不过排法应当象本来应该有的那样:资本、总价值、总产品、每吨市场价值、个别价值、差额价值\endnote{正如马克思在后面(见本册第298—299页)所解释的,他把市场价值和个别价值之间的差额叫做差额价值(Differentialwert)。差额价值是马克思就单位产品来说的,级差地租是马克思就一定等级所生产的全部产品来说的。如果单位产品的市场价值大于单位产品的个别价值,差额就是正数;如果单位产品的市场价值小于它的个别价值,差额就是负数。所以马克思在手稿第574页的总表上加了+和-的符号(见本册第302—303页)。在手稿第572页C、D、E各表上(见本册第290—291页),马克思在表明级差地租量的数字(镑)前加了+和-的符号,例如在C表“级差地租”一栏中有负数“-[9+(3/13)]镑”。这表明,在这种情况下等级I的肥力不高,以致这个等级的土地在现有的市场价值下不仅不能提供任何级差地租,而且连绝对地租也大大降低到正常量之下。在CI中,绝对地租只等于10/13镑,即比本例中作为绝对地租正常量的10镑少9+(3/13)镑。马克思在手稿第574页的总表里,用负差额价值来表示这种负级差地租现象,而在“级差地租”一栏,遇到这种情况时只记上“0”,表示这里没有正级差地租(在许多情况下,遇有负级差地租时,负级差地租由绝对地租量的相应减少来表示,这反映在“绝对地租”一栏中)。把负数移入“差额价值”一栏,就可以免除手稿第572页C表中在把不同等级土地的级差地租相加时所发生的那种麻烦,这样,在统计级差地租时,只把带+号的正级差地租加到总数中去,而为了避免重复计算起见,就把负数“-[9+(3/13)]镑”当作零。因此,马克思为了计算负级差地租,在总表里另加了“每吨差额价值”一项,把负差额价值也列进去。——第292页。}、费用价格、绝对地租、绝对地租(吨)、级差地租、级差地租(吨)、总地租。然后在每个表下列出所有等级的合计。\endnote{紧接这些话之后,马克思在手稿第573页上按照上面说的各项把A、B、C、D表作了对比。在手稿的下一页即第574页上,马克思再一次把A、B、C、D表的全部材料写成更有次序的格式,并补上E表的有关材料。这就是本册第302—303页上的统一的总表。马克思在手稿第573页上草拟的A、B、C、D表的对比材料已全部列入总表,因此在本书正文中就不再另外列出。——第292页。}[见第302—303页]

\tsubsectionnonum{[575]对表的说明}

假定:花费资本100(不变资本和可变资本),由这笔资本推动的劳动提供等于预付总资本1/5的剩余劳动(无酬劳动),或者说,提供等于100/5的剩余价值。因此,如果预付资本等于100镑,总产品的价值就应该等于120镑。再假定平均利润等于10%;在这种场合,110镑就是总产品(在上例中是煤)的费用价格。100镑的资本,不管开采的是富矿还是贫矿,在剩余价值率或剩余劳动率既定时转化为120镑的价值;总之,劳动的不同生产率,不论它是劳动的不同自然条件的后果,还是劳动的不同社会条件的后果,还是不同技术条件的后果,都丝毫不会改变商品价值等于物化在商品中的劳动量这一论点。

因此,如果说100镑资本生产的产品的价值等于120镑,这只不过是说,在产品中包含着物化在100镑资本中的劳动时间加1/6为资本家占有的无酬劳动时间。不论这100镑资本在一个等级的矿井中生产60吨,在另一个等级生产65或75或92+(1/2)吨,产品的总价值都等于120镑。但是很明显,不论每一单位产品是象这里一样用吨计算,还是用夸特、码等计算,它的价值却随着劳动生产率的不同而完全不同。拿我们的表来说(对于作为资本主义生产结果的任何别的商品量来说同样适用),如果资本的总产品是60吨,那末一吨的价值就是2镑,因而60吨值120镑,换句话说,它们代表物化在120镑中的劳动时间。如果总产品是65吨,每一吨的价值就等于1镑16+(12/13)先令,即1+(11/13)镑;如果总产品是75吨,每一吨的价值就等于1+(3/5)镑,即1镑12先令;最后,如果总产品是92+(1/2)吨,一吨的价值就等于1+(11/37)镑,即1镑5+(35/37)先令。因为100镑资本所生产的商品总量或总吨数总是具有同一价值,等于120镑,因为它们总是代表120镑所包含的同一劳动总量,正因为如此,所以每一吨的价值,随着同一价值表现为60、65、75或92+(1/2)吨而不同,也就是随着劳动生产率的不同而不同。正是这种劳动生产率的不同造成这样的情况:同量劳动有时表现为较小的商品总量,有时表现为较大的商品总量,因而这个商品总量的每一部分包含的已耗费的劳动的绝对量,有时较多,有时较少,也就是说,与此相应,它有时有较大价值,有时有较小价值。这个随100镑资本投在富矿或贫矿而不同的,即随劳动生产率不同而不同的每一吨的价值,就是表上的每吨个别价值。

因此,再没有什么比下面这样一种看法更错误了:如果单位商品的价值在劳动生产率提高时下降,那末一定资本(例如100镑)所生产的产品的总价值就要由于它借以表现的商品量的增加而提高。其实,单位商品的价值之所以下降,只是因为总价值,即已耗费的劳动总量,表现为较大的使用价值量,较大的产品量,因而分摊到单位产品上的是总价值(或者说,已耗费的劳动)的一个较小的比例部分,而且单位产品价值下降的程度,就是单位产品吸收的劳动量减少或分摊到的总价值的份额减少的程度。

最初,我们把单个商品看作一定量劳动的结果和直接产品。现在,当商品表现为资本主义生产的产品时,事情在形式上就发生了如下的变化:

生产出来的使用价值量代表一个劳动时间量,这个劳动时间量等于在生产使用价值量时消耗的资本(不变资本和可变资本)中包含的劳动时间量加资本家占有的无酬劳动时间。如果包含在资本中的劳动时间用货币来表现等于100镑,如果这100镑资本中包含40镑花费在工资上的资本,而剩余劳动时间是可变资本的50%,就是说,剩余价值率等于50%,那末,100镑资本生产的商品总量的价值就等于120镑。我们在这部著作的第一部分\endnote{指《政治经济学批判》第一分册。见《马克思恩格斯全集》中文版第13卷第56页。——第294页。}已经说过,商品要能够流通,商品的交换价值必须先变成价格,就是说,必须表现为货币。因此,[576]如果总产品不是一个代表全部资本的不可分割的东西(例如一座房子),不是一个唯一的商品,而其价格根据假定等于120镑,即等于表现为货币的总价值,那末,资本家在把商品抛到市场上之前,就必定首先计算单位商品的价格。这里,价格等于价值的货币表现。

120镑总价值将依劳动生产率的不同而分配在较多或较少的产品量上,因而单位产品的价值将依此——成比例地——等于120镑的一个较小或较大的相应部分。这里计算很简单。如果全部产品等于例如60吨煤,那末60吨等于120镑,1吨等于120/60镑,也就是2镑;如果产品是65吨,那末一吨的价值等于120/65镑,也就是1+(11/13)镑或1镑16+(12/13)先令(1镑16先令11+(1/13)便士);如果产品是75吨,那末一吨的价值等于120/75镑,也就是1镑12先令;如果产品是92+(1/2)吨,那末一吨的价值等于1+(11/37)镑,或1镑5+(35/37)先令。因此,单位商品的价值(价格)等于产品的总价值除以产品总量,这个总量是用产品作为使用价值所适用的度量单位,如吨(在上述场合)、夸特、码等计算的。

因此,如果单位商品的价格等于100镑资本生产的商品量的总价值除以商品总量,那末总价值就等于单个商品的价格乘以这些商品的总量,或者说,等于作为一个单位的一定量商品的价格乘以用这个单位计算的全部商品量。其次,总价值由预付在生产中的资本的价值加剩余价值组成,由包含在预付资本中的劳动时间加资本占有的剩余劳动时间即无酬劳动时间组成。因此,商品量的每一部分包含的剩余价值,同它包含的价值具有同一比例。随着120镑是分配在60、65、75吨还是分配在92+(1/2)吨上,20镑剩余价值也就分配在那些吨上。如果吨数等于60,因而每吨价值等于120/60,即2镑或40先令,那末这个40先令或2镑的1/6,即6+(2/3)先令,就是分摊到一吨上的剩余价值份额。剩余价值在值2镑的一吨中所占的比例,同它在值120镑的60吨中所占的比例一样。剩余价值对价值之比,在单位商品的价格中同在全部商品量的总价值中一样。在上例中,每一吨包含全部剩余价值的20/60=2/6=1/3镑,或者说上述40先令的1/6。因此,一吨的剩余价值乘60就等于资本生产出来的全部剩余价值。如果由于产品数量较大,也就是由于劳动生产率较高,摊到单位产品上的价值部分即总价值的比例部分较小,那末摊到每一单位产品上的剩余价值部分,即单位产品中包含的全部剩余价值的相应部分也较小。但这并不影响剩余价值即新创造的价值对预付的和只是被再生产出来的价值的比例关系。的确,我们已经看到\authornote{见本卷第1册第215—217页。——编者注},虽然劳动生产率并不影响产品的总价值,但是,如果产品加入工人的消费,如果由于单个商品的价格下降,或者换句话说,由于一定量商品的价格下降,因而正常工资减少,或者换句话说,劳动能力的价值减少,那末劳动生产率就会使剩余价值增大。由于较高的劳动生产率创造相对剩余价值,它就不是使产品的总价值增大,而是使这个总价值中代表剩余价值即无酬劳动的部分增大。因此,如果劳动生产率较高,但由于价值借以表现的商品总量已经增大,摊到单位产品上的价值部分较小,因而单位产品的价格下降,那末,在上述情况下,这个价格中代表剩余价值的部分仍然会增大,也就是说,剩余价值对再生产出来的价值之比会增大{其实,这里首先还是应该谈对可变资本的关系,这里还谈不上利润}。但是,所以会发生这种情况,是因为在产品的总价值中,由于劳动生产率增长,剩余价值增大了。正是这个原因,即劳动生产率的提高,——它的提高使同量劳动表现为一个较大的产品量,从而使这个产品量的任何一部分的价值或单位商品的价格降低,——使劳动能力的价值减少,因而使包含在总产品价值中的,从而包含在单位商品价格中的剩余劳动即无酬劳动增加。因此,虽然单位商品的价格降低,虽然包含在单位商品中的劳动总量减少,从而它的价值也减少,这个价值中由剩余价值组成的比例部分却增大,换句话说,同以前劳动生产率较低,因而单位商品的价格较高,包含在单位商品中的劳动总量较大的时候比较起来,在单位商品包含的较少的[577]劳动总量中,却包含较大的无酬劳动量。虽然在这种场合,一吨包含较少的劳动,因而比较便宜,但是它包含较多的剩余劳动,因而提供较多的剩余价值。

因为在竞争的条件下一切事情都以虚假的、颠倒的形式表现出来,所以单个资本家会以为:(1)由于单位商品价格降低,他从单位商品赚到的利润降低了,但是由于商品量增加,他才赚到较大的利润(这里又同由于使用的资本增大,即使在利润率较低时也可能获得较大的利润量的情况混淆起来了);(2)他确定单位商品的价格,并通过乘法确定产品的总价值,可是,本来的过程却是除法,然后才是乘法,乘法以除法作为自己的前提。庸俗经济学家实际上只不过把陷入竞争中的资本家们的奇怪想法翻译成一种表面上比较理论化的语言,并企图借此来说明这些想法正确而已。

现在回过头来谈我们的表。

用100镑资本创造的产品或商品量的总价值等于120镑;商品量可大可小,全看劳动生产率的不同程度而定。不论总产品的量大小如何,如果平均利润象我们假定的那样是10%,这个总产品的费用价格就总是等于110镑。不论总产品的量大小如何,总产品的价值超过费用价格的余额总是等于10镑,即等于总价值的1/12,或预付资本的1/10。总产品的价值超过费用价格的这个余额,这个10镑,构成地租。很明显,它同煤矿、土地,总之同这100镑资本曾经被用上去的那个自然要素的不同自然富饶程度所引起的不同劳动生产率完全无关,因为由自然因素的不同富饶程度引起的劳动生产率的不同,并不妨碍总产品有120镑的价值,有110镑的费用价格,因而有一个等于10镑的、价值超过费用价格的余额。资本的竞争所能起的作用,只是使一个资本家在煤的生产这个特殊生产领域中用100镑资本创造出来的商品的费用价格等于110镑。但是竞争并不能象它在其他生产领域中起的强制作用那样,即使产品值120镑,它也要迫使这个资本家按照110镑出卖。这是因为有土地所有者插手进来,拿走这10镑。因此,我把这个地租称为绝对地租。所以,不论煤矿的富饶程度如何改变,也不论由此引起的劳动生产率如何改变,在表里,这个地租总是同一的。但是,它在煤矿富饶程度不同因而劳动生产率不同的条件下,不是表现为同一吨数。因为包含在10镑中的劳动量随着劳动生产率的不同而表现为较多或较少的使用价值量,表现为较多或较少的吨数。这个绝对地租在富饶程度不同的条件下是否总是全部得到支付或者[有时只是]部分得到支付,将在表的进一步分析中说明。

其次,存在于市场上的煤却是富饶程度不同的矿井的产品,这些矿井,我从最贫瘠的开始,已标作I、II、III、IV四个等级。例如:第一等级,100镑资本的产品是60吨;第二等级,100镑资本的产品是65吨;等等。因此,由于劳动生产率程度随矿井、土地,总之随自然因素的富饶程度而有所不同,在这里同样大小的、具有同一有机构成的资本100镑,在同一生产领域内却有不同的生产率。但是,竞争为这些具有不同个别价值的产品规定了统一的市场价值。这个市场价值本身决不能大于最贫瘠的等级的产品的个别价值。如果它高一些,这只是证明市场价格高于市场价值。但是市场价值必定表现实际价值。就各个等级的产品来看,当然,它们的[个别]价值可能高于或者低于市场价值。如果它们的个别价值高于市场价值,那末市场价值和它们的费用价格之间的差额就小于它们的个别价值和它们的费用价格之间的差额。但是,因为绝对地租等于它们的个别[578]价值和它们的费用价格之间的差额,所以在这种场合,这些产品的市场价值就不能提供全部绝对地租。如果市场价值降到等于这些产品的费用价格,这些产品的市场价值就完全不提供地租。这些产品的生产者就不能支付任何地租,因为[总]地租只是[市场]价值和费用价格之间的差额,而就这些产品个别地说,在这样的市场价值之下,这个差额就会消失。在这种场合,它们的市场价值和个别价值之间的差额是负数,就是说,市场价值和它们的个别价值相差一个负数。我把市场价值和个别价值之间的差额通称为差额价值。对于这种情况下的商品,我在差额价值前加了一个负号。

相反,如果某一等级的煤矿(土地)的产品的个别价值低于市场价值,那就是说,市场价值高于产品的个别价值。这样,在这些产品的生产领域中占支配地位的价值或市场价值,就提供一个超过它们的个别价值的余额。比方说,如果一吨的市场价值等于2镑,那末,个别价值等于1镑12先令的一吨,它的差额价值就是8先令。因为在一吨的个别价值等于1镑12先令的等级中,100镑资本生产75吨,所以这75吨的全部差额价值就是8先令×75,即30镑。这个等级因土地或矿井相对来说比较富饶而造成的、全部产品的市场价值超过其个别价值的余额,就形成级差地租,因为费用价格对于这笔资本来说仍旧同以前一样。这个级差地租是较大还是较小,就看市场价值超过个别价值的余额是较大还是较小,而这个余额是较大还是较小,又要看生产出这种产品的矿山等级或土地等级,同生产出的产品对市场价值起决定作用的那个比较不富饶的等级比较起来,其富饶程度相对说来是高得多些还是少些。

最后,还必须指出,不同等级的产品的个别费用价格是不同的。例如,100镑资本生产75吨的那个等级,因为总价值等于120镑,总费用价格等于110镑,单位商品的费用价格就等于1镑9+(1/3)先令;如果市场价值等于这个等级的个别价值,就是说等于1镑12先令,那末,按120镑出卖的75吨将提供地租10镑,而110镑就代表它们的费用价格。

但是单独一吨的个别费用价格当然随着100镑资本借以表现的吨数,或随着不同等级的单位产品的个别价值而不同。例如,100镑资本生产60吨,一吨的价值就等于2镑,它的费用价格等于1镑16+(2/3)先令。55吨就等于110镑或总产品的费用价格。如果100镑资本生产75吨,那末一吨的价值就等于1镑12先令,它的费用价格等于1镑9+(1/3)先令,总产品中的68+(3/4)吨值110镑,也就是说正好补偿费用价格。在不同等级中,个别费用价格即每吨费用价格的不同,与个别价值的不同具有同一比例。

五个表都表明,绝对地租总是等于商品[个别]价值超过它自己的费用价格的余额;级差地租等于商品的市场价值超过它的个别价值的余额;总地租(如果除绝对地租外还有级差地租的话)等于市场价值超过个别价值的余额加上个别价值超过费用价格的余额,或者说,等于市场价值超过个别费用价格的余额。

因为这里只是把地租的一般规律作为我的价值理论和费用价格理论的例证来发挥,只有到我专门考察土地所有权时我才详细论述地租,[579]所以我撇开了一切使问题复杂化的情况:矿井或各种土地的位置的影响;用于同一矿井或同一土地的几批资本的不同生产率;同一生产领域的不同部门例如农业的不同部门所提供的地租的相互关系;彼此不同但可以互相转化的各生产领域——例如从农业中抽出土地用于建筑房屋等——所提供的地租的相互关系。这一切都不是这里所要讨论的。

\tsectionnonum{[(3)对表的分析]}

现在我们来考察这些表。这些表说明,一般规律可以解释多种多样的组合,而李嘉图由于不知道地租的一般规律,对级差地租的本质也只有片面的理解,因此他想通过强制的抽象把多种多样的现象只归结为唯一的一种情况。这些表所要说明的,并不是所有可能的组合,只是对我们的专门目的有用的几种最重要的组合。

\tsubsubsectionnonum{[(a)]A表[不同等级的个别价值和市场价值之间的关系]}

在A表中,一吨煤的市场价值由等级I的一吨煤的个别价值决定,这等级I的矿井最贫瘠,因而劳动生产率最低,也就是说100镑投资提供的产品量最少,因此,单位产品的价格(由它的价值决定的价格)最高。

假定市场吸收200吨,既不多,也不少。

市场价值不能高于I的每吨价值,即不能高于在最不利的生产条件下生产出来的商品的价值。II和III的一吨煤高于自己的个别价值出卖,这是因为这里的生产条件比同一生产领域(部门)生产的其他商品所具备的条件优越;因此,这并不违反价值规律。相反,如果市场价值高于I的每吨价值,那末,其所以可能,只是因为I的产品完全不顾市场价值高于自己的价值出卖。一般说来,市场价值和[个别]价值之间有差别,不是因为产品绝对高于自己的价值出卖,只是因为整个生产领域的产品所具有的那个价值,可能和个别产品的价值不同,也就是说,因为生产总产品(这里是200吨)所必要的劳动时间,可能和生产其中部分吨数(这里指II和III出产的那些吨数)所用的劳动时间不同;总之,因为得到的总产品是生产率程度不同的劳动的产品。因此,产品的市场价值和它的个别价值之间的差别,只能同生产率程度的差别有关,在这些不同程度的生产率下,一定量劳动创造出总产品的不同份额。这种差别决不能意味着价值是不依赖于该生产领域一般使用的劳动量而决定的。如果一吨的市场价值高于2镑,那末,其所以可能,只是因为I不管同II和III的关系如何,一般来说都是把自己的产品高于它的价值出卖。在这种场合,由于市场状况,即由于供求关系,市场价格就会高于市场价值。但是,我们这里所谈的市场价值——按照假定在这里市场价格等于市场价值,——是不能高于它自己的。

\todo{}

在这里,市场价值等于I的产品的价值,而且I提供的产品占市场上所有产品的3/10,因为II和III只提供补足全部需求所必要的那么多产品,也就是说,只满足由I的产品满足的需求之外的追加需求。因此,II和III没有任何理由低于2镑出卖自己的产品,因为全部产品都能够按2镑卖掉。它们也不能把自己的产品卖得[580]高于2镑,因为I的一吨卖2镑。

市场价值不能高于在最坏生产条件下生产出来,但是作为必要供给的一部分提供的产品的个别价值,这个规律被李嘉图歪曲为市场价值不能降到低于这种产品的价值,也就是说始终都要由这个价值来决定。下面我们就会看到,这是多么错误。

因为在I中,一吨的市场价值和一吨的个别价值一致,所以它提供的地租就是产品价值超过产品费用价格的绝对余额,即绝对地租,等于10镑。II提供级差地租10镑,III提供级差地租30镑,因为由I决定的市场价值分别为II和III提供一个超过其产品个别价值因而也超过10镑绝对地租(绝对地租代表个别价值超过费用价格的余额)的余额10镑和30镑。II提供总地租20镑,III提供总地租40镑,因为市场价值提供超过它们的费用价格的余额分别为20镑和40镑。

我们假定,由等级I即最贫瘠的矿井推移到比较富饶的等级II,再由II推移到更富饶的等级III。虽然II和III比I富饶,但是它们只满足整个需求的7/10,因而象刚才已经阐明的,能够按2镑出卖自己的产品,尽管产品的价值分别只是1镑16+(12/13)先令和1镑12先令。很明显,如果为满足需求所必要的一定量产品得到供应,而满足这个需求的不同部分的劳动生产率又有不同,那末,不论向这个还是那个方向推移,在两种场合,比较富饶的那些等级的市场价值都会高于它们的个别价值:在一种场合,是因为它们遇到的市场价值由贫瘠的等级决定,而它们提供的追加供给又不足以促使由I决定的市场价值发生变动;在另一种场合,是因为原来由它们决定的市场价值,即由III或II决定的市场价值,现在由I来决定,因为I提供市场所要求的追加供给,而且它只能按照现在决定市场价值的那个较高价值提供这种供给。

\tsubsubsectionnonum{[(b)李嘉图的地租理论同农业生产率递减观点的联系。绝对地租率的变动及其同利润率的变动的关系]}

例如,李嘉图在A表这样的情况下就会说:出发点是III;追加供给先是由II提供;最后,市场要求的最后的追加供给由I提供;因为I只能按120镑提供追加的60吨,每吨2镑,而这个供给又是市场所要求的,所以一吨的市场价值,原来是1镑12先令,然后是1镑16+(12/13)先令,现在提高到2镑。可是,反过来同样正确:如果出发点是I,它按每吨2镑满足60吨需求,然后追加需求由II提供,II的产品将按市场价值2镑出卖,虽然它的个别价值只是1镑16+(12/13)先令;因为市场要求的125吨,象以前一样,只有在I提供60吨而每吨价值为2镑这个条件下,才能提供出来。同样,如果又必需有一个75吨的新的追加供给,而III只提供75吨,仅仅满足这个追加的需求,那就仍然要有I的每吨2镑的60吨。如果II满足全部需求200吨,那末这200吨就卖400镑。这200吨现在也是按这个价格出卖的,因为II和III不是按照它们能够满足140吨追加需求的那个价格出卖自己的产品,[XII—581]而是按照只提供3/10产品的I能够满足需求的那个价格出卖自己的产品。市场要求的产品总量是200吨,在这里每吨按2镑出卖,因为这个数量的3/10只有每吨按2镑出卖才能生产出来,而不管满足需求所必需的追加产品量按什么次序进行生产,是从III开始到II再到I,还是从I开始到II再到III。

李嘉图说:如果III和II的产品是出发点,那末它们的市场价值必定提高到I的产品价值(在李嘉图那里,是费用价格)的水平,因为I提供的3/10是满足需求所必需的;换句话说,因为这里谈的是所要求的产品量,而不是这个量的一些特殊部分的个别价值。但是,下面的情况同样是正确的:如果I是出发点,II和III提供的只是追加供给,由I提供的3/10仍然同样是必需的;由此可见,如果I在下降序列中决定市场价值,那末由于同样的原因,它在上升序列中也决定市场价值。因此,A表向我们表明,李嘉图的观点是错误的,他认为级差地租以比较富饶的矿井或土地推移到比较不富饶的矿井或土地为条件,也就是说,级差地租以劳动生产率不断减低为条件。级差地租即使在运动的方向相反时,也就是说,在劳动生产率不断提高时,也完全可能存在。不管这两种运动发生哪一种,都和级差地租的本质,和级差地租存在的事实没有任何关系,这是一个历史问题。实际上,上升序列和下降序列将互相交错,追加需求的满足,有时靠推移到比较富饶的土地、煤矿等这类自然因素,有时靠推移到比较不富饶的土地、煤矿等这类自然因素。同时始终假定,由不同于其他等级的新等级(不管这个等级是比较富饶还是比较不富饶)的自然因素提供的供给,只等于追加的需求,也就是说不引起供求关系的任何变动,因此,不是在花费较小的费用就能提供供给时,而只是在花费较大的费用方能提供供给时,才会引起市场价值本身的变动。

因此,A表从一开始就向我们揭示了李嘉图的这个基本前提是错误的,从安德森的例子可以看出,甚至在对绝对地租问题有错误看法时,这个前提也不是必要的。

如果发生从III到II,从II到I,即按下降序列的推移,即推移到越来越不富饶的自然因素,那末,先是投资100镑的III,按照商品价值120镑出卖自己的商品。每吨是1镑12先令,因为III生产75吨。如果有65吨的追加需求必须满足,那末,使用资本100镑的II,也会按照产品价值120镑出卖自己的产品。每吨是1镑16+(12/13)先令。最后,如果又必须提供追加供给60吨,而只有I能够提供这些吨数,那末,I也按照产品价值120镑出卖自己的产品,每吨是2镑。在这样的过程中,一旦II的产品出现在市场上,III就提供级差地租18+(6/13)镑,而它原来只提供绝对地租10镑。一旦I出现,II就提供级差地租10镑,III的级差地租就提高到30镑。

李嘉图从III降到I时,在I那里已经找不到地租了,这是因为他在考察III时,是从不存在任何绝对地租出发的。

当然,上升序列和下降序列之间有一定的差别。如果是从I推移到III,因而II和III只提供追加供给,那末市场价值就依然等于I的产品的个别价值,即2镑。并且,如果平均利润象这里假定的那样是10%,那末可以认为,煤的价格(相应地说,小麦的价格;到处都可以用一夸特小麦等等代替一吨煤)加入了平均利润的计算,因为煤既作为生活资料加入工人的消费,又有相当大的份额作为辅助材料加入不变资本。因此,同样可以认为,如果I的生产率较高,或者说,如果一吨煤的价值低于2镑,剩余价值率就较高,同时剩余价值本身也较大,也就是说,利润率高于10%。但是,在III是出发点时,正好就是这样的情况。一吨煤的[市场]价值原来只等于1镑12先令,当[582]II的产品出现在市场上时,它提高到1镑16+(12/13)先令,最后,当I的产品出现时,它提高到2镑。因此,可以认为,在其他一切情况如剩余劳动时间以及其他生产条件等等固定不变时,如果只有III被开发,利润率就更高(剩余价值率就更高,因为工资的一个要素更便宜;由于剩余价值率更高,剩余价值量也更大,就是说利润率也更高;除此之外,在剩余价值发生这种变化的情况下,利润率所以更高,还因为用于不变资本的一个费用要素更便宜),当II的产品出现在市场上时,利润率就降低,最后,当I的产品出现时,利润率就降到10%这个最低水平。因此,在这种场合应当假定,当只有III被开发时,利润率等于例如12%(不考虑原来假定的数字),当II出现时,利润率降到11%,最后,当I出现在市场上时,利润率降到10%。在这种场合,III的绝对地租是8镑,因为费用价格是112镑;在II的产品出现在市场上以后,绝对地租是9镑,因为费用价格现在是111镑,最后,绝对地租提高到10镑,因为费用价格降到110镑。可见,这里绝对地租率本身发生了变动,而且和利润率的变动成反比。地租率逐步提高,因为利润率逐步下降。而利润率所以下降,是由于矿井、农业等等的劳动生产率下降,并且与此相应,生活资料和辅助材料越来越贵。

\tsubsubsectionnonum{[(c)]考察生活资料和原料的价值——以及机器的价值——的变动对资本有机构成的影响}

在上述情况下,地租率因为利润率下降而提高。在这里,利润率是不是因为资本有机构成发生变动而下降呢?如果资本的平均构成是80c+20v,那末这种构成是否保持不变?假定正常工作日保持不变。否则生活资料涨价的影响可能被抵销。这里应当把两种情况分开。第一,生活资料涨价,因而剩余劳动和剩余价值减少。第二,不变资本涨价,因为辅助材料例如煤的价值可能提高;如果说的是小麦,那末不变资本的另一个组成部分——种子——的价值可能提高,或者,由于小麦涨价,另一种原产品(原料)的费用价格可能上涨。最后,如果涨价的产品是铁、铜等等,那末某些工业部门所必需的原料的价值和所有生产部门制造设备(包括各种容器)用的原料的价值就会提高。

一方面假定资本的有机构成没有发生变动,也就是说,假定生产方式没有发生那种使必须使用的活劳动量与所使用的不变资本量相对来说增加或减少的变动。现在和以前一样,仍然需要同样数量的工人(正常工作日的界限不变),来利用同样数量的机器等等加工同样数量的原料,或者在没有原料的场合,来推动同样数量的机器、工具等等。这是在谈到资本有机构成时应当指出的第一个着眼点。但是,还必须补充另一个着眼点,就是:要考察资本各要素的价值的变动,虽然资本各要素作为使用价值使用的数量和以前完全一样。这里又应当区分如下的情况。

第一,价值的变动以同样的程度影响可变资本和不变资本这两个要素。显然,这种情况实际上是从来不会有的。小麦之类的农产品的价值提高会使(必要)工资和原料(例如种子)涨价。煤价上涨会使必要工资和大部分工业部门的辅助材料的价值提高。但是在前一种场合,工资的提高是在所有生产部门发生,而原料的涨价只是在某些部门发生。至于煤,它加入工资的份额比加入生产的份额小。所以,在总资本中煤和小麦的价值变动未必能以同样的程度影响资本的两个要素。不过我们假定可能有这种情况。

假定资本80c+20v生产的产品的价值等于120。就总资本来说,产品的价值和产品的费用价格相符合。价值和费用价格之间的差别对总资本来说刚好拉平。假定煤这样的物品的价值的提高(根据假定,煤按同一比例加入资本的两个组成部分)使两个要素的费用各增加1/10。这样,用80c只能买到以前[大约]用70c就能买到的商品,而用20v只够支付以前[大约]用18v就能支付的那样多工人的报酬。或者,为了按原来的规模继续生产,现在必须花费[大约]90c和22v。产品的价值和以前一样是120,可是支出现在是112(不变资本90和可变资本22)。所以,利润等于8,这占112的1/14,也就是占[7+(1/7)]%。因而支出资本100所生产的产品价值现在等于107+(1/7)。

现在加入这笔新资本的c和v的比例关系是怎样的呢?以前v和c之比是20∶80,即1∶4,现在是22∶90,即11∶45。1/4=45/180;11/45=44/180。可见,可变资本[583]和不变资本相比减少了1/180。因此,根据假定,要承认煤等的涨价按同一比例影响资本的两个部分,我们必须用88c+22v。因为产品的价值等于120;支出88+22=110。剩下10作为利润。22∶88=20∶80。c和v之比仍然象在原来的资本中那样。同以前一样,v∶c=1∶4。而利润10占110的1/11,即[9+(1/11)]%。因此,如果生产要按原来的规模继续进行,那末,以前投入资本100,现在就要投入资本110,而产品的价值和以前一样等于120。\endnote{在马克思所举的例子中,其生产依赖于土地所有权的那种产品按同一比例加入预付资本的两个组成部分。在这里马克思假定,虽然不变资本增加(由于原料涨价,80c变成88c)以及可变资本也增加(由于工人的消费品涨价,20v变成22v),总产品的市场价值仍然和从前一样等于120(在本册第313—323页所考察的另外一个例子中,马克思相反地从市场价值的变动出发)。在总产品市场价值保持不变的情况下,由于不变资本和可变资本的这种涨价,资本家攫取的剩余价值从20减到10,相应地,由于推移到比较不富饶的地段,在比较富饶的地段上的级差地租就增加10单位。因此,新创造的价值仍旧等于40(因为生产方式未变),在这里它重新分配如下:现在10单位构成剩余价值,归资本家,20单位补偿可变资本,10单位用于增加级差地租(由于不变资本价值增加8单位,由于可变资本价值增加2单位)。后来,在手稿第684—686页(本册第518—522页)上,马克思以农业资本为例考察类似的情况,这种农业资本的产品以实物形式加入这一资本的不变部分和可变部分的诸要素的构成之中。——第311页。}对于100单位的资本来说,其有机构成是80c+20v,产品价值是109+(1/11)。

[第二,]如果在上例中80c的价值保持不变,只是v的价值变了,比如说不是20v而是22v,那末以前的20∶80或10∶40现在就变成22∶80或11∶40。如果发生这样的变化,那末资本就是80c+22v,产品价值是120;因而支出是102,利润是18,即占[17+(33/51)]%。22∶18=[21+(29/51)]∶[17+(33/51)]。如果说为了推动价值80的不变资本而必须花费在工资上的资本是22v,那末要推动价值78+(22/51)的不变资本就需要21+(29/51)。按照这样的比例,在100单位的资本中只有78+(22/51)能够用在机器和原料上,而21+(29/51)则必须用在工资上,而以前是80用在原料等等上,只有20用在工资上。产品的价值现在等于117+(33/51),资本构成则是[78+(22/51)]c+[21+(29/51)]v。而21+(29/51)+[17+(33/51)]=39+(11/51)。在以前的资本构成情况下,全部新加劳动等于40;现在等于39+(11/51),或者说,比以前少40/51;发生这一变化的原因是,虽然不变资本的价值未变,但现在不得不使用较少的不变资本,因而100单位的资本能够推动的劳动比以前略少,尽管支付的报酬较高。

因此,如果某一费用要素的变动——这里是涨价,价值提高——只引起(必要)工资的变动,那末就会发生如下的情况:第一,剩余价值率下降;第二,在资本既定的情况下,可以使用较少的不变资本,较少的原料和机器。资本不变部分的绝对量与可变资本相比减少,这在其他条件不变的情况下,总会引起利润率的提高(如果不变资本的价值保持不变)。不变资本的量减少,虽然它的价值仍然不变。但是剩余价值率和剩余价值本身减少了,因为在剩余价值率下降的情况下雇用的工人人数没有增加。剩余价值(剩余劳动)率比不变资本和可变资本的比率下降得更多。现在为了推动同量的不变资本,必需雇用和以前同样多的工人,也就是使用同样的劳动绝对量。不过在这个劳动绝对量中,必要劳动多了,剩余劳动少了。因而对同量的劳动要支付较高的报酬。因此,同一资本(例如100)用于不变资本的部分少了一些,因为它必须多花一些可变资本来推动较少的不变资本。这里剩余价值率的下降不是和一定资本所使用的劳动绝对量的增长相联系,或者说,不是和它所雇用的工人人数的增长相联系。因而在这里,剩余价值本身不可能在剩余价值率下降的情况下增长。

因此,如果从资本的组成部分的物质方面即把它们作为使用价值来考察时,资本的有机构成保持不变;也就是说,如果资本构成的变动不是由于投入这个资本的生产领域的生产方式发生变动,而仅仅是由于劳动能力价值提高,因而必要工资也提高了(这等于说,剩余劳动减少或剩余价值率下降,这种减少和下降在这里不可能由于一定量资本,例如100单位,所雇用的工人人数的增加而全部抵销或部分抵销),那末,利润率的下降就只能是由于剩余价值本身下降。而资本有机构成的变动也是由同一原因引起。这种变动,在生产方式不变以及使用的直接劳动量和积累劳动量之间的比例不变的情况下,只能由使用的直接劳动量和积累劳动量的价值(比例价值)的变动引起。同一资本使用[584]的较少量直接劳动与它使用的较少量不变资本具有同一比例,但它对这较少量劳动支付的报酬却较高。它只能使用较少量的不变资本,因为推动这较少量不变资本的那个较少量劳动,将吸收总资本中较大的份额。为了推动78单位的不变资本,资本家现在必须在可变资本上花费比如22单位,而以前推动80c只花费20v就够了。

总之,其生产依赖于土地所有权的那种产品的涨价只影响工资的情况,就是这样。如果这种产品跌价,就产生相反的结果。

现在我们来看看上面假设的[“第一”种]情况。假定农产品的涨价按同一比例影响不变资本和可变资本。那末在这里,根据假定,资本的有机构成不发生任何变动。第一,生产方式没有变动。同样的直接劳动绝对量推动着和以前一样的积累劳动量。这两种劳动量之间的比例和以前一样。第二,积累劳动和直接劳动的价值比例没有变动。如果其中一个的价值提高或下降,那末另一个的价值就同它成比例地发生相应的变动,结果它们的比例仍然不变。而以前是80c+20v。产品价值=120。现在是88c+22v。产品价值=120。这就得出10比110,即[9+(1/11)]%的利润。因此,对于80c+20v的资本来说,产品价值是109+(1/11)。

以前是:

\todo{}

现在是:

\todo{}

80c在这里代表较少量的原料等;20v相应地代表较少的劳动绝对量。原料等贵了;用80购买的原料等的量也就减少了;因为生产方式没有变,所以它需要较少量的直接劳动。但是这较少量的直接劳动却和以前较大量的直接劳动所值一样多,而且它和原料等恰好按同样程度涨价,因而也按同样程度减少。可见,如果剩余价值不变,利润率就会按照原料等涨价的程度,按照可变资本价值对不变资本价值的比例的变化程度而下降。但是剩余价值率不是原来那样,而是随着可变资本价值的增长发生了变化。

我们举[另外]一个例子。

一磅棉花的价值由1先令上涨到2先令。以前用80镑{这里我们假定机器等等于0}可以买1600磅棉花。现在用80镑只能买800磅棉花。以前把1600磅棉花加工成棉纱需要花费工资20镑,假定这相当于20个工人。加工800磅就只需要10个工人,因为生产方式没有变。10个工人以前值10镑,现在值20镑,就同800磅棉花以前值40镑现在值80镑完全一样。假定以前利润是20%。因此,所假定的就是这样:

实际上,如果20个工人创造的剩余价值等于20镑,那末10个工人创造的剩余价值就等于10镑;但是要生产这笔剩余价值,必须照旧支付20镑,而在以前的条件下只支付10镑。一磅棉纱这个产品[585]的价值在这里无论如何都会提高,因为产品包含着更多的劳动,即(一磅棉纱所含的棉花中的)积累劳动和直接劳动。

如果只是棉花的价值提高,工资依然不变,那末把800磅棉花加工成棉纱照旧只需要10个工人。而这10个工人也只花费10镑。所以剩余价值率照旧是100%。把800磅棉花变成棉纱,需要10个工人,为他们支出资本10镑。所以资本的总支出等于90镑。同时始终假定每80磅棉花需要一个工人。因此800磅就需要10个工人,1600磅就需要20个工人。那末,现在全部资本100镑能把多少磅棉花变成棉纱呢?可以用88+(8/9)镑购买棉花,11+(1/9)镑支付工资。

比例是这样的:

在这种情况下,可变资本的价值没有变动,因而剩余价值率仍然不变。

在I的情况下,可变资本和不变资本之比是20∶80,或者说1∶4。在III的情况下,可变资本和不变资本之比是[11+(1/9)]∶[88+(8/9)],或者说1∶8;可见,这里可变资本相对地减少了一半,因为不变资本的价值增加了一倍。同一数量的工人把同一数量的棉花纺成棉纱,可是100镑资本现在只能雇用11+(1/9)个工人,用余下的88+(8/9)镑只能购买888+(8/9)磅棉花,而不能象I的情况下那样购买1600磅棉花。剩余价值率仍然不变。但是由于不变资本价值发生变动,在资本为100镑时已经不能雇用以前那样多的工人;可变资本和不变资本的比例变了。结果,剩余价值量减少,随之利润也减少,因为这个剩余价值照旧按同样的资本支出计算。在I的情况下,可变资本(20镑)是不变资本的1/4(20∶80),总资本的1/5。现在可变资本(11+(1/9)镑)只是不变资本的1/8{[11+(1/9)]∶[88+(8/9)]},总资本100镑的1/9。但是100/5镑(即20镑)的100%是20镑,而100/9镑(即11+(1/9)镑)的100%只有11+(1/9)镑。在工资不变,或者说,可变资本价值不变的情况下,可变资本的绝对量在这里减少了,因为不变资本的价值提高了。因此,可变资本的百分比下降,随之剩余价值本身,剩余价值的绝对量,因而还有利润率也下降。

在可变资本价值不变和生产方式不变的情况下(也就是说,在使用的劳动、原料和机器的量保持原来比例的情况下),不变资本价值的变动[这里是提高],会使资本构成发生这样一种变化,就好比不变资本价值没有变,可是同花费在劳动上的资本相比,却使用了更大量的价值未变的[不变]资本(也就是说,所使用的不变资本的价值总额更大)。由此产生的必然后果是利润下降。(如果不变资本价值下降,就会产生相反的情况。)

相反,可变资本价值的变动(这里是提高),会使可变资本同不变资本相对来说增加,从而使可变资本的百分比增加,或者说,使它在总资本中所占的比例部分增加。然而利润率在这里不是提高,而是下降。这是因为生产方式保持不变。为了把同量的原料、机器等变成产品,使用的活劳动量和以前一样。这里和上面的情况一样,用同一资本100镑只能[586]推动较少量的直接劳动和积累劳动,但这较少量的直接劳动所值更多了。必要工资提高了。这较少量的直接劳动中有一个较大部分补偿必要劳动,因而它只有一个较小部分形成剩余劳动。剩余价值率下降了,与此同时,同一资本所支配的工人人数或劳动总量也减少了。可变资本同不变资本相对来说增加了,因而同总资本相对来说也增加了,尽管使用的劳动量同不变资本量相对来说减少了。所以剩余价值下降了,随之利润率也下降了。在前一场合,利润率下降是因为在剩余价值率不变的情况下,可变资本同不变资本相对来说,因而同总资本相对来说减少了;换句话说,剩余价值下降是因为在剩余价值率不变的情况下,工人人数减少了,剩余价值的乘数变小了。在后一场合,利润率下降则是因为,可变资本同不变资本相对来说,因而同总资本相对来说增加了,但是可变资本的这种增加伴随着使用的劳动量(同一资本使用的劳动量)的减少;换句话说,这里剩余价值下降,是因为剩余价值率的下降同使用的劳动量的减少联系在一起。有酬劳动同不变资本相对来说增加了,但使用的劳动总量减少了。

由此可见,价值的这些变动总是影响剩余价值本身,剩余价值的绝对量在两种场合都减少了,因为它的两个因素中有一个变小了,或者这两个因素都变小了;在一种场合,剩余价值减少是因为剩余价值率不变而工人人数减少了;在另一种场合,剩余价值减少是因为剩余价值率和100单位资本所雇用的工人人数都减少了。

最后,我们来谈谈II的情况。在这里农产品价值的变动按同一比例影响资本的两个部分,因而这种价值变动不会造成资本有机构成的变动。

在这种情况下(见第584页)\authornote{见本册第314页。——编者注},一磅棉纱的价值从1先令6便士涨到2先令9便士,因为现在它是比以前耗费更多劳动时间的产品。虽然现在一磅棉纱包含的直接劳动和以前一样(不过这时有酬劳动较多,无酬劳动较少),但是它现在包含的积累劳动比以前多。由于棉花价值发生变动,从1先令涨到2先令,现在加入一磅棉纱价值的也就不是1先令而是2先令。

然而第584页上II的例子是不正确的。

我们有:

20个工人的劳动表现为40镑。在这里,这个数目的一半是无酬劳动,因此剩余价值是20镑。依照这一比例,10个工人将生产20镑,其中10镑是工资,10镑是剩余价值。

因此,如果劳动能力的价值和原料价值按同一比例提高,就是说,如果劳动能力的价值提高一倍,那末它就等于10个工人得20镑,就象以前它等于20个工人得20镑一样。在这种情况下就没有任何剩余劳动了。因为,如果20个工人提供的价值用货币表现等于40镑,那末10个工人提供的价值用货币表现就等于20镑[这也就是II的情况下全部可变资本的价值]。这是不可能的。在这种情况下资本主义生产的基础就消失了。

但是,因为根据假定,不变资本和可变资本价值的变动应当是一样的(按比例),所以我们必须用别的方式表达这一情况。比如说,假定棉花价值提高1/3;用80镑现在能够买到1200磅棉花,而以前用这些货币能够买到1600磅。以前1镑等于20磅棉花,或1磅棉花等于1/20镑,即1先令。现在1镑等于15磅棉花,或1磅棉花等于1/15镑,即1+(1/3)先令,或1先令4便士。以前1个工人花费1镑,现在要花费1+(1/3)镑,即1镑6+(2/3)先令,或1镑6先令8便士。15个工人就要花费20镑(15镑+15/3镑)。[587]因为20个工人生产40镑的价值,所以15个工人生产30镑的价值。在这一价值中现在20镑是工人的工资,10镑是剩余价值或无酬劳动。

这样,我们就有:

在这1先令10便士中,棉花是1先令4便士,劳动是6便士。

产品涨价是因为棉花贵了1/3。但是产品没有涨价1/3。以前在I的情况下产品值18便士,因此,产品如果涨价1/3,现在就应该值18+6,即24便士。可是它现在只值22便士。以前在1600磅棉纱中包含40镑的劳动,因而1磅棉纱中包含1/40镑,即20/40或1/2先令,也就是6便士。现在1200磅棉纱中包含30镑的劳动,所以花在1磅棉纱上的劳动也是1/40镑,即1/2先令或6便士。虽然劳动和原料按同样程度涨价,1磅棉纱所包含的直接劳动的量仍然和以前一样;不过在这一劳动量中现在包含的有酬劳动较多,无酬劳动较少。所以工资价值的这种变动丝毫不改变产品1磅棉纱的价值。这里劳动依然只是6便士,而棉花现在是1先令4便士,不是过去的1先令。一般说来,如果商品按照它的价值出卖,工资价值的变动就不会引起产品价格的变动。不过以前在6便士中工资占3便士,剩余价值也占3便士。现在在6便士中工资是4便士,剩余价值是2便士。实际上,1磅棉纱包含工资3便士,1600磅棉纱就是3×1600便士,即20镑,而1磅棉纱包含工资4便士,1200磅棉纱就是4×1200便士,即20镑。但3便士和15便士(1先令棉花加3便士工资)之比,构成前一种情况下的1/5的利润,即20%的利润。2便士和20便士(16便士棉花加4便士工资)之比,则构成1/10,或10%的利润。

如果在上述例子中棉花价格保持不变,我们就会得到如下结果。一个工人把80磅棉花纺成棉纱(因为生产方式在所有的例子中保持不变),而1磅棉花又=1先令。

资本现在分解如下\endnote{在紧接着的下一段,马克思自己就认为这种计算法是“不能成立的”,在作这种计算时,马克思的出发点是:纺纱的新加劳动等于40镑,并且只限于按照必要工资提高1/3的假定,把这40镑分为必要劳动和剩余劳动。计算之所不能成立,是因为在这种情况下(在资本为100镑的情况下),以前的工人人数(20人)和他们新加的劳动的数额(40镑)不可能保持不变。既然必要工资按照假定提高了1/3,100镑资本就不可能雇用20个工人,而必须把工人人数减为18+(3/4)人,就象马克思在后来的计算中所做的那样。而工人人数的变化会引起第II种情况的计算中的一切其它变化。这种计算法如下:如果以前一个工人花费1镑,那末现在他花费1+(1/3)镑,20个工人现在花费26+(2/3)镑。因此,为了在原有的规模上继续进行棉纱的生产,就需有106+(2/3)镑的资本,其构成为80c+[26+(2/3)]v。折算成100镑,资本的构成就是75c+25v。——第320页。}:

\todo{}

这种计算法是不能成立;因为,如果一个工人纺80磅棉花的棉纱,那末20个工人就纺1600磅,而不是1466+(2/3)磅,因为假定生产方式保持不变。工人的不同的报酬丝毫不可能改变这个事实。因而,必须另外举例:

\todo{}

在这6便士中,工资是4便士,利润是2便士。2便士和16便士[1先令棉花加4便士工资]之比,是1/8,即[12+(1/2)]%。

最后,如果可变资本价值保持不变(1个工人=1镑),而不变资本价值变了,结果1磅棉花不是花费1先令,而是花费1先令4便士或16便士,那末情况如下:

\todo{}

[588]利润=3便士。它和19便士[16便士棉花加3便士工资]之比恰好是[15+(15/19)]%。

现在我们从还没有发生价值变动的I开始,把所有这四种情况加以比较。

\todo{}

产品价格在III和IV的情况下发生变动,因为不变资本的价值发生了变动。而可变资本价值的变动却不引起产品价格的任何变动,因为直接劳动的绝对量没有变,只是以不同的方式分为必要劳动和剩余劳动。

但是在IV的情况下,价值的变动按同一比例影响不变资本和可变资本,两者的价值都提高了1/3,这时的情况如何呢?

如果只是工资提高(II),那末利润就会从20%下降到[12+(1/2)]%,即下降7+(1/2)单位。如果只是不变资本的价值提高(III),那末利润就会从20%下降到[15+(15/19)]%,即下降4+(4/19)单位。因为在我们考察的IV的情况下,工资和不变资本的价值按同样程度提高,所以利润从20%下降到10%,即下降10单位。可是为什么不下降7+(1/2)+[4+(4/19)]即11+(27/38)这个II和III的差额之和呢?必须弄清楚这个[1+(27/38)]%,它使利润(IV)应当下降到[8+(11/38)]%,而不是10%。利润量决定于剩余价值量,而在剩余劳动率既定的条件下,剩余价值量又决定于工人人数。在I的情况下有20个工人,他们的一半劳动时间是无酬的。在II的情况下只有总劳动的1/3是无酬劳动;因此剩余价值率下降;此外,使用的工人比I少1+(1/4),因而工人人数或总劳动也减少了。在III的情况下剩余价值率又和I一样,一半工作日是无酬的,但是由于不变资本价值提高,工人人数从20减少到15+(15/19),或者说,减少4+(4/19)。在IV的情况下工人人数减少5人(剩余价值率也下降到II的水平,即下降到1/3工作日),也就是从20人减到15人。和I相比,在IV的情况下工人人数减少了5人,和II相比减少了3+(3/4)人,和III相比减少了15/19人;但是和I相比并不是减少1+(1/4)+[4+(4/19)]即5+(35/76)。否则在IV的情况下雇用的工人人数就会是14+(41/76)。

从上面的论述中可得出如下结论:

加入不变资本或可变资本的商品的价值变动,——在生产方式不变,或者说,资本的物质构成不变的情况下(即在使用的直接劳动和积累劳动之间的比例不变的情况下),——只要按同一比例影响可变资本和不变资本,就象IV的情况那样(例如,这里的棉花和工人消费的小麦按同样程度涨价),就不会引起资本有机构成的变动。这里利润率下降(在不变资本和可变资本价值提高的情况下),第一,是因为工资上涨而造成剩余价值率下降;第二,是因为工人人数减少。

在价值变动只影响不变资本或只影响可变资本时,尽管生产方式保持不变,这种变动所起的作用和资本有机构成变动所起的作用一样,会在资本组成部分的价值比例中引起同样的变动。

如果价值的变动只影响可变资本,那末可变资本同不变资本[589]以及同总资本相对来说就会增加。但在这种情况下,不仅剩余价值率下降,而且雇用的工人人数也减少。所以在这种情况下(II)也使用较少的不变资本(其价值仍然不变)。

如果价值的变动只影响不变资本,那末可变资本同不变资本以及同总资本相对来说就会减少。虽然剩余价值率不变,剩余价值量却会减少,因为雇用的工人人数减少了(III)。

最后,可能有这样的情况:价值的变动既影响不变资本,也影响可变资本,但是影响的程度不同。这种情况应当包括在上述的各种情况中。例如,假定价值的变动影响不变资本和可变资本,使不变资本价值提高10%,可变资本价值提高5%。那末,就它们二者都涨价5%来说,一个上涨5+5,另一个上涨5,这就是IV的情况。然而就不变资本在此之外还有5%的变动来说,这就是III的情况。

上面我们只是假定价值提高。在价值下降的情况下会发生相反的结果。例如,从IV推移到I,就是考察价值下降按同一比例影响资本的两个组成部分的情况。如果要说明只是资本的一个组成部分的价值下降会引起怎样的结果,只须把第II种情况和第III种情况相应地修改一下就行了。[589]

\centerbox{※     ※     ※}

[600]关于上述[原料、机器和生活资料的]价值变动对资本有机构成的影响,我还要补充一点:拿投入不同生产部门的资本来说,在这些资本的物质构成在其他方面相同的情况下,只要它们使用的机器或材料的价值较高,就会造成它们的有机构成的不同。例如,如果棉纺织业、丝纺织业、麻纺织业和毛纺织业资本的物质构成完全相同,那末只要它们所使用的材料的价值不同,就会造成这些资本的有机构成不同。[600]

\tsubsubsectionnonum{[(d)总地租的变动取决于市场价值的变动]}

[589]我们回过头来谈谈A表。我们已经看到\authornote{见本册第307—308页。——编者注},假定由于利润下降而形成了利润率10%(在最初只有III被开发时,利润率比较高,后来,II出现,就比只开发III时低了,但仍然高于10%),这个假定在一定条件下,也就是在确实按下降序列发展的时候,可能是正确的,但它决不是由于地租的差别,决不是仅仅由于级差地租的存在而必然得出来的;相反,在按上升序列发展时,这种地租的差别是以利润率始终不变为前提的。

B表。在这里,正如前面已经说明的\authornote{见本册第282页及以下各页。——编者注},III和IV的竞争迫使II把自己的资本抽出一半。在按下降序列发展时,这相反会表现为:所需要的追加供给只有32+(1/2)吨,所以对II只应当投资50镑。

但是在这个表中最令人感兴趣的是:以前投入的资本是300镑,现在只有250镑,即减少1/6;而产品量仍然和以前一样,是200吨。可见,劳动生产率提高了,单位商品的价值下降了。同样,商品的总价值也从400镑降到369+(3/13)镑。同A相比,一吨的市场价值从2镑降到1镑16+(12/13)先令,因为新的市场价值由II的产品的个别价值决定,而不是象以前那样由I的产品的较高的个别价值决定。尽管投资减少了,在产品量不变的情况下产品总价值减少了,市场价值下降了,开发了比较富饶的等级,——尽管有这些情况,同A相比,B的地租却绝对增加了24+(3/13)镑(从70镑增加到94+(3/13)镑)。如果我们考察一下各个等级使地租总额增大的情况,那就会发现,在等级II中,绝对地租率仍然保持不变,因为5镑是50镑的10%,但是地租量减少了一半,从10镑减少到5镑,因为BII的投资减少了一半,从100镑减少到50镑。在BII中,地租总额不是增加,而是减少5镑。其次,BII的级差地租完全消失,因为市场价值现在等于II的产品的个别价值。这使地租总额又减少10镑。因此,II的地租总共减少15镑。

在III中,绝对地租额保持不变,但是由于产品的市场价值下降,它的差额价值也下降了,从而级差地租也减少了。级差地租以前是30镑,现在只有18+(6/13)镑,即减少11+(7/13)镑。因此,II和III合在一起,地租减少26+(7/13)镑。可见,地租总额的增加并不象初看起来的那样是24+(3/13)镑,而是{24+(3/13)+[26+(7/13)]=}50+(10/13)镑,这一点仍须弄清楚。其次,和A相比,对于B来说,AI的绝对地租随着I本身的消失而消失了。因此,这又使地租总额减少10镑。由此可见,应当说明的总共是60+(10/13)镑。但这正好是新等级BIV的地租总额。因此,B的地租总额的增加仅仅用BIV的地租就可说明。BIV的绝对地租同所有其他各等级一样,等于10镑。而50+(10/13)镑的级差地租是这样得出来的:[590]IV的产品的差额价值一吨是10+(470/481)先令,再把它乘以92+(1/2),因为这是这个等级开采出来的吨数。II和III的富饶程度仍然不变;最贫瘠的等级完全消失,然而地租总额增加了,因为,由于IV的富饶程度相对地较高,单单是IV的级差地租就比A的总级差地租大。级差地租不取决于各个被开发等级的绝对富饶程度,不取决于1/2II、III、IV比I、II、III更富饶;可是1/2II、III、IV的级差地租比A的I、II、III的级差地租大。所以会有这种情况,是因为所提供的产品的最大部分——92+(1/2)吨——是从这样一个等级得来的,这个等级的差额价值比A表上任何一个等级的差额价值都大。如果某一等级的差额价值是既定的,那末它的级差地租的绝对额自然就取决于这个等级的产品量。但是在计算差额价值和考察这个价值的形成过程时,已把这个产品量计算进去了。因为IV用100镑资本生产92+(1/2)吨,不多也不少,所以在每吨市场价值等于1镑16+(12/13)先令的B表中,它的差额价值就是每吨10+(470/481)先令。

在A表中,地租总额是70镑,资本是300镑,即占[23+(1/3)]%。而在B表中,如果3/13略去不计,地租总额是94,资本是250,即占[37+(3/5)]%。

C表。这里假定,在IV加进来并开始由II决定市场价值之后,需求不象在B表中那样保持不变,而是随价格下降而提高,以致IV提供的全部追加量92+(1/2)吨都被市场吸收。在每吨价格为2镑时,只有200吨被吸收;在价格为1+(11/13)镑时,需求增加到292+(1/2)吨。如果假定市场容量在每吨价格为1+(11/13)镑时仍然和每吨价格为2镑时一样,那是错误的。相反,市场总是随着价格下降而扩大到一定程度,甚至拿小麦这样一般的生活资料来说也是如此。

这就是我们对C表首先要指出的唯一的一点。

D表。这里假定,只有当市场价值降到1+(5/6)镑时,292+(1/2)吨才被市场吸收;而1+(5/6)镑是I的一吨的费用价格,因而I不带来任何地租,只提供10%的普通利润。这是李嘉图所认为的正常的情况,因此,应当比较详细地讨论一下。

这里和前面的各表一样,先按上升序列;随后我们还要按下降序列考察这个过程。

如果II、III、IV只提供140吨的追加供给\authornote{对I生产的60吨的追加供给。——编者注},即按照2镑一吨被市场吸收掉的追加供给,那末I就继续决定市场价值。

可是情况并不是这样。市场上还有IV生产的92+(1/2)吨余额。如果这个数量是绝对超过市场需要的一般剩余产品,那就会象B表那样,I被完全排挤出市场,II必须抽出自己的一半资本。这样一来,也会象B表那样,由II决定市场价值。可是现在我们假定,在市场价值进一步降低的情况下,市场能够吸收这92+(1/2)吨。这个过程将怎样进行呢?IV、III和1/2II在市场上占绝对统治地位。这就是说,如果市场绝对地说只能吸收200吨,这些等级就会把I排挤出市场。

但是我们先来看看实际情况。在市场上过去只有200吨,现在有292+(1/2)吨。II将按照商品的个别价值1+(11/13)镑出卖,以便在市场上站住脚,并把个别价值等于2镑的I的产品从市场上排挤出去。但是因为按照这个市场价值容纳不了292+(1/2)吨,所以IV和III将挤压II,直到市场价格降到1+(5/6)镑为止。按照这样的价格,IV、III、II和I都将在市场上为自己的产品找到销路,市场按照这个[591]市场价格将吸收全部产品。由于价格的这种下降,供求趋于平衡。一旦追加供给开始超过由原来的市场价值决定的市场容量,每一个等级自然都会竭力把自己的全部产品塞进市场,而把别的等级的产品排挤出去。这只能通过降低价格的办法,也就是把价格降低到使大家都能找到销路的水平。如果价格降低很多,以致I、II等等不得不低于生产费用\endnote{生产费用(《Produktionskosten》)这一术语,马克思在这里以及有时在后面是用在生产费用加平均利润,即费用价格(生产价格)的意义上。《Produktionskosten》这一术语在《资本论》第三卷的一些地方也有这样用法。见《马克思恩格斯全集》1964年德文版第25卷第665、686、747—749页。——第327页。}出卖商品,那它们自然会被迫从生产中抽出自己的资本。如果为了使产品适应市场状况,价格看来不必降低这么多,那末全部资本就能按照产品的这个新市场价值在这个生产领域继续发挥作用。

但是再往下看,很明显,在这样的条件下,决定产品市场价值的就不是I、II这些较坏的地段,而是III、IV这些较好的地段,因而,正象施托尔希对于这种情况正确理解的那样,[26]是较好地段的地租决定较坏地段的地租。

IV按照那种使它能把自己的全部产品塞进市场,又能消除其他等级的各种反抗的价格出卖产品。这个价格就是1+(5/6)镑。如果IV按照更高的价格出卖,那末市场的容量就会缩小,互相排挤的过程又会重新开始。

只有假定II等等所提供的追加供给,仅仅是市场按照I的产品决定的市场价值吸收掉的那个追加供给,I才决定市场价值。如果追加供给超过这种界限,I会起完全消极的作用,并由于它在市场上占据的地位仅仅迫使II、III、IV作出相应的反应,直到价格减低,使市场足以吸收生产出来的全部产品为止。现在看来,按照这种实际上由IV决定的市场价值,IV除绝对地租外,还支付49+(7/12)镑级差地租,III除绝对地租外,还支付17+(1/2)镑级差地租,II则不支付任何级差地租,只支付绝对地租的一部分9+(1/6)镑,而不是绝对地租的全部10镑。为什么?因为新的市场价值1+(5/6)镑,虽然也高于II的产品的费用价格,却仍低于它的个别价值。如果新的市场价值等于它的个别价值,II就支付10镑绝对地租,即等于个别价值和费用价格之间的差额。但是,因为新的市场价值低于II的产品的个别价值,——它所支付的实际地租等于市场价值和费用价格之间的差额,而这个差额小于它的个别价值和它的费用价格之间的差额,——所以II只支付它的绝对地租的一部分9+(1/6)镑,而不是10镑。

{实际地租等于市场价值和[个别]费用价格之间的差额。}

绝对地租等于个别价值和费用价格之间的差额。

级差地租等于市场价值和个别价值之间的差额。

实际地租,或者说,总地租,等于绝对地租加级差地租;换句话说,等于市场价值超过个别价值的余额加个别价值超过费用价格的余额,即等于市场价值和费用价格之间的差额。

因此,如果市场价值等于个别价值,那末级差地租就等于零,总地租就等于个别价值和费用价格之间的差额。

如果市场价值大于个别价值,那末级差地租就等于市场价值超过个别价值的余额,总地租就等于这个级差地租加绝对地租。

如果市场价值小于个别价值,大于费用价格,那末级差地租就是一个负数;因而总地租就等于绝对地租加这个负级差地租,即减个别价值超过市场价值的余额。

如果市场价值等于费用价格,那末地租就整个等于零。

为了把这一切用方程式来表现,我们用AR表示绝对地租,用DR表示级差地租,用GR表示总地租,用MW表示市场价值,用IW表示个别价值,用KP表示费用价格。这样,我们就得出如下的方程式:

[592](1)AR=IW-KP=+y.

(2)DR=MW-IW=x.

(3)GR=AR+DR=MW-IW+IW―KP=y+x=MW-KP.

如果MW>IW,那末MW-IW=+x,因此DR就是正数,并且GR=y+x。

因此,MW-KP=y+x,或MW-y-x=KP,或MW=y+x+KP。

如果MW<IW,那末MW-IW=-x,因此DR就是负数,并且GR=y-x。

因此,MW-KP=y-x,或MW+x=IW,或MW+x-y=KP,或MW=y-x+KP。

如果MW=IW,那末DR=0,x=0,因为MW-IW=0。

因而在这种情况下,GR=AR+DR=AR+0=MW-IW+IW-KP=0+IW-KP=IW-KP=MW-KP=+y。

如果MW=KP,那末GR(或MW-KP)=0。

在上面假定的[D表]情况下,I不支付任何地租。为什么?因为绝对地租等于个别价值和费用价格之间的差额,级差地租等于市场价值和个别价值之间的差额;而在这里,市场价值等于I的产品的费用价格。I的个别价值等于每吨2镑;市场价值等于1+(5/6)镑。因此,I的级差地租等于1+(5/6)镑减2镑,即等于-1/6镑。I的绝对地租等于2镑减1+(5/6)镑,即等于它的个别价值和它的费用价格之间的差额(+1/6镑)。于是,因为I的实际地租等于绝对地租(+1/6镑)加级差地租(-1/6镑),所以等于零。因此,I的产品既不支付级差地租,也不支付绝对地租,只支付费用价格。这个产品的价值等于2镑,但它按照1+(5/6)镑出卖,即低于它的价值1/12或[8+(1/3)]%出卖。I不能卖得更贵,因为决定市场的不是它,而是和它相对的IV、III、II。I所能做的,只是按照每吨1+(5/6)镑的价格提供追加供给。

I不支付地租这个事实,是由市场价值等于它的费用价格造成的。

但这个事实是下列情况的后果:

第一,I相对不肥沃。它必须按照1+(5/6)镑提供追加的60吨。假定它用100镑资本不只是提供60吨,而是提供64吨,比II少1吨。在这种场合,只要在这一等级投入93+(3/4)镑资本,就足以提供60吨。这样I的每吨个别价值就是1+(7/8)镑,或1镑17+(1/2)先令,而它的费用价格就是1镑14+(3/8)先令。因为市场价值等于1+(5/6)镑,或1镑16+(2/3)先令,所以市场价值和费用价格之间的差额就等于2+(7/24)先令。按60吨计算,[593]地租就是6镑17+(1/2)先令。

由此可见,如果一切条件不变,I只要比现在肥沃1/15(因为60/15=4),它就还会支付一部分绝对地租,因为在市场价值和它的费用价格之间存在一个差额,虽然这个差额比它的个别价值和它的费用价格之间的差额小。因此在这里,最坏的土地如果比我们假定的肥沃一些,也还会提供地租。如果I比现在绝对肥沃一些,那末II、III、IV和它相比,就相对不肥沃一些。I的个别价值和它们的个别价值之间的差额就比较小。因此,I不提供任何地租这一点,是在同样程度上由两个情况造成的:它自己既不绝对肥沃一些,II、III、IV也不相对不肥沃一些。

但是第二,假定I的产量是既定的,100镑资本生产60吨。如果II、III、IV,特别是作为新的竞争者出现在市场上的IV,和I相比不仅相对不肥沃一些,而且绝对不肥沃一些,那末I就会提供地租,虽然这只是一部分绝对地租。事实上,因为市场按照1+(5/6)镑吸收292+(1/2)吨,所以较少的吨数,例如280吨,市场就会按照高于1+(5/6)镑的市场价值来吸收。但是任何高于1+(5/6)镑即高于I的费用价格的市场价值,都会为I提供地租——它等于市场价值减I的产品的费用价格。

因此,也可以说,I不提供地租是由于IV绝对肥沃,因为在市场上只有II和III是竞争者的时候,I还提供地租,甚至当IV出现了,也就是说,即使有追加的供给,但只要IV使用资本100镑生产出来的不是92+(1/2)吨,而只是80吨,I仍会继续提供地租,虽然比原来的少些。

第三,我们曾假定,花费资本100镑,绝对地租是10镑,即资本的10%,或费用价格的1/11,因而在农业上资本100镑生产出来的产品价值等于120镑,其中10镑是利润。

但是不要以为,如果我们说“在农业上花费资本100镑”,并且如果一个工作日等于1镑,那就是在农业上也花费了100工作日。一般说来,如果资本100镑等于100工作日,那末不管这笔资本花费在什么生产部门,都不能说[这笔资本的产品等于100工作日]。假定1镑金等于一个12小时的工作日,并且假定这就是正常的工作日。这里产生的第一个问题就是:对劳动的剥削率怎样?也就是说,在12小时中,工人有几小时为自己,为再生产自己的工资(等价物)劳动?他又有几小时白白地为资本家劳动?也就是说,资本家没有支付报酬却拿来出卖的那个劳动时间,因而是形成剩余价值的源泉、资本增殖的源泉的那个劳动时间有多大?如果这个比率等于50%,那末工人就是为自己劳动8小时,为资本家白白地劳动4小时。产品等于12小时,或1镑(因为根据假定,1镑金包含12小时劳动时间)。在这等于1镑的12小时中,8小时补偿资本家支付给工人的工资,4小时构成资本家的剩余价值。因此,花费工资13+(1/3)先令,会得到剩余价值6+(2/3)先令。换句话说,花费资本1镑,会得到剩余价值10先令,花费100镑,会得到50镑。在这种情况下,用资本100镑生产出来的商品的价值会等于150镑。资本家的利润一般地说在于出卖产品中包含的无酬劳动。正常的利润就是来自出卖不支付报酬的东西。

[594]第二个问题就是:资本的有机构成怎样?资本的价值中由机器等和原料构成的那一部分,只不过在产品中再生产出来,再现出来,它是保持不变的。对于资本的这个组成部分,资本家必须按其价值支付。因此,它是作为既定的、预定的价值加入产品的。只有被资本家使用的劳动仅仅有一部分得到资本家的支付,虽然它全部加入产品的价值并被资本家全部购买。因此,在假定上述的对劳动的剥削率的情况下,等量资本的剩余价值的大小取决于资本的有机构成。如果资本a的构成是80c+20v,产品的价值就等于110,利润就是10%(虽然在产品中包含的无酬劳动是50%)。如果资本b的构成是40c+60v,产品的价值就等于130,利润就是30%,虽然产品中也还是只包含50%的无酬劳动。如果资本c的构成是60c+40v,产品价值就等于120,利润就是20%,虽然这里包含的无酬劳动也是50%。因此,从这等于300的三笔资本得到的总利润,等于10+30+20=60,每100平均是20%。上述每一笔资本如果把它生产的商品卖120镑,就会提供这个平均利润。资本a的构成是80c+20v,它比自己产品的价值高10镑出卖产品,资本b的构成是40c+60v,它比自己产品的价值低10镑出卖产品,资本c的构成是60c+40v,它按照自己产品的价值出卖产品。合计起来,这些商品是按照它们的价值即120+120+120=360镑出卖。而实际上,a+b+c的产品的价值=110+130+120=360镑。但是各笔资本的产品的价格有的高于它们的价值,有的低于它们的价值,有的等于它们的价值,结果这些资本中的每一笔都提供20%的利润。这样改变了形态的商品价值也就是商品的费用价格,竞争不断地使费用价格成为市场价格的引力中心。

就投入农业的100镑来说,我们假定资本的构成是60c+40v(就v来说,也许还太低);这样,产品的价值就等于120。但是这样的价值等于[上面假定的]工业品的费用价格。因此我们假定上例中100镑资本生产出来的产品的平均价格[不是120,而]是110镑。现在我们说:如果农产品按照自己的价值出卖,那末它的价值就比它的费用价格高10镑。它就会提供10%的地租,并且我们认为,农产品和别的产品不同,它不按照自己的费用价格出卖,而按照自己的价值出卖,这是资本主义生产下的正常现象。这是土地所有权造成的后果。如果平均利润等于10%,总资本的构成就会是80c+20v。我们假定农业资本的构成是60c+40v,也就是说,这种资本构成中用于工资的(即用于直接劳动的)份额,比投在其余生产部门的资本总额中的工资份额大。这表明,在这个部门中,劳动生产率的发展相对较低。当然,在农业的某些部门例如畜牧业中,资本的构成也许是90c+10v,即v与c之比也许还小于工业总资本。但决定地租的并不是这种部门,而是真正的农业,即农业中生产主要生活资料如小麦等等的那一部分。在其他农业部门中,地租不决定于投在这些部门本身的资本的构成[595],而决定于生产主要生活资料的那种资本的构成。资本主义生产存在本身是以这样一种情况为前提,即生活资料的最主要成分是植物性食物,而不是动物性食物。农业各部门的地租之间的关系是个次要问题,在这里我们可以不去注意,也可以不去考察。

这样,为使绝对地租是10%,我们假定资本一般的平均构成:

非农业资本是80c+20v,

农业资本是60c+40v。

现在要问,农业资本的另一种构成,例如50c+50v或70c+30v,是否会影响D表中所假定的I不提供任何地租的情况呢?在第一种情况下,产品的价值等于125镑;在第二种情况下,产品的价值等于115镑。在第一种情况下,农业资本构成和非农业资本构成的不同所造成的差额是15镑,在第二种情况下,差额是5镑。换句话说,农产品的价值和它的费用价格之间的差额,在第一种情况下比我们假定的高50%,在第二种情况下低50%。

如果出现第一种情况,就是说,如果100镑资本生产出来的产品价值等于125镑,那末在A表中,I的每吨价值就等于2+(1/12)镑。这也就是A的市场价值,因为这里是I决定市场价值。相反,对于AI来说,费用价格仍然是1+(5/6)镑。而因为根据假定,292+(1/2)吨[在表中]只有按照1+(5/6)镑才能卖掉,所以现在所考察的农业资本有机构成的这种变动,对DI不提供任何地租的情况不发生任何影响;同样,如果农业资本的构成是70c+30v,换句话说,如果农产品的价值和它的费用价格之间的差额只是5镑,即只是我们假定的一半,那末情况也丝毫不会发生变化。由此可见,假定费用价格,因而非农业资本的平均有机构成(80c+20v)不变,那末农业资本的构成较高还是较低,对于这里的情况(对于DI)是没有任何意义的,虽然这种差别对于A表有意义,并且[在各个表中]使绝对地租发生50%的变化。

但是我们现在假定相反的情况:农业资本的构成仍旧是60c+40v,而非农业资本的构成改变了。它不是80c+20v,而是70c+30v,或90c+10v。在第一种情况下,平均利润将是15镑,或者说,比原来假定的高50%;在第二种情况下,平均利润将是5镑,或者说,比原来假定的低50%。在第一种情况下,绝对地租是5镑。因此,这对于DI同样不发生任何影响。在第二种情况下,绝对地租等于15镑。这也不会使DI发生任何变化。因此,对于DI来说,所有这一切都是毫无关系的,虽然对于A、B、C和E表有着重要的意义,也就是说,每当新的等级(不管按上升序列还是按下降序列推移都一样)按照原来的市场价值只提供必要的追加供给时,对于决定绝对地租和级差地租的绝对量有着重要的意义。

\centerbox{※     ※     ※}

现在产生下面一个问题:

D表的情况实际上是可能的吗?而且首先,李嘉图假定的这种情况是正常的吗?这种情况只有在两个条件下才可能是正常的:

或者,农业资本的构成是80c+20v,即象非农业资本的平均构成一样,从而农产品的价值就会等于非农产品的费用价格。这从统计材料来看目前还是不正确的。关于农业劳动生产率相对较低的假定,无论如何比李嘉图关于农业劳动生产率绝对递减的假定符合实际。

[596]李嘉图在第一章(《论价值》)中假定,金银矿中的资本构成是平均构成(固然,他这里谈的只是固定资本和流动资本,不过我们要予以“纠正”)。根据这个假定,这些矿山始终只能有级差地租,决不会有绝对地租。而这个假定本身又建立在另一个假定上,即比较富饶的矿山所提供的追加供给,总是比按照原来的市场价值所要求的追加供给多。但绝对令人不解的是,为什么不能同样存在相反的情况。单是存在级差地租这件事就已经证明,原有市场价值保持不变的追加供给是可能的。因为,IV或III或II这些等级,如果不是按照I的产品的市场价值(不管这个市场价值是怎样决定的),也就是说,不是按照同这些等级的供给的绝对量无关而决定的市场价值出卖自己的产品,它们就不会提供级差地租。

或者,如果D表所假定的各种情况始终是正常的,就是说,如果由于IV、III和II的竞争,特别是IV的竞争,I总是不得不比自己产品价值低一个绝对地租全额即按照费用价格来出卖自己的产品,那末D表的情况就必定总是正常的。IV、III和II都有级差地租这件事本身就证明,它们出卖自己产品所按照的市场价值高于它们的个别价值。如果李嘉图认为I不可能有这种情况,那只是因为他事先假定不可能有绝对地租,而他所以认为不可能有绝对地租,是由于他是以价值和费用价格等同这个前提为出发点。

我们拿C表的情况来看,这里的292+(1/2)吨按照1镑16+(12/13)先令的市场价值都能找到销路。并且,我们也象李嘉图那样从IV出发。当市场只要求92+(1/2)吨时,IV按照每吨1镑5+(35/37)先令出卖,也就是说,它把100镑资本生产出来的商品,按照它的价值120镑出卖,从而提供10镑绝对地租。为什么IV要低于产品价值、按费用价格出卖自己的商品呢?当市场上只有IV的时候,III、II、I不可能同它竞争。连III的产品的费用价格都比使IV提供10镑地租的那个价值高,II和I的费用价格比这个价值就更高了。因此,即使III、II和I只是按照费用价格出卖自己的产品,它们也不可能同IV竞争。

假定总共只有一个等级——是较好还是较坏等级的土地,是IV或I,或III,或II,这对于理论无关紧要;假定这个等级是作为自然要素而存在,也就是说,对一般可以支配的并且在这个生产部门可以被吸收的资本量和劳动量来说作为自然要素而存在,这样,这种土地就不会构成任何界限,而是现有的劳动量和资本量的相对无限的活动场所;因而,假定不存在级差地租,因为被耕种的土地没有自然肥力上的不同,这样,就不存在任何级差地租(或者说,级差地租在这里微不足道);再假定不存在任何土地所有权,那就很明显,不存在绝对地租,因而,也就根本不存在任何地租(因为根据假定,也不存在级差地租)。这是同义反复。因为绝对地租的存在不仅以土地所有权为前提,而且就是被当作前提的土地所有权,即被资本主义生产的作用决定并改变了形态的土地所有权。这种同义反复对于解决问题毫无用处,因为我们正是用农业中的土地所有权对商品价值转变为平均价格的资本主义平均化进行抵抗这一点,来说明绝对地租的形成。如果我们取消土地所有权的这种作用,取消资本竞争在这个投资领域内遇到的这种抵抗,这种特殊的抵抗,那我们当然也就取消了地租存在的前提本身。而且这里还是一个自相矛盾的假定(威克菲尔德先生在他的殖民理论\endnote{关于威克菲尔德的殖民理论,见卡·马克思《资本论》第1卷第25章(见《马克思恩格斯全集》中文版第23卷第25章)。——第338页。}中很好地看到了这一点):一方面是发达的资本主义生产,另一方面又没有土地所有权。在这种情况下雇佣工人从哪里来呢?

在殖民地中有某种近似的情况,即使在法律上存在土地所有权,——这是由政府无偿地分给土地造成的,如当初英国向海外殖民时的情况,——并且,即使[597]政府在实际上培植土地所有权,以非常便宜的价格出卖土地,如美国的情况(1美元或大致这么多的东西可买一英亩土地)。

这里应当把两种类型的殖民地区别开来。

第一,说的是本来意义的殖民地,例如美国、澳大利亚等地的殖民地。这里从事农业的大部分殖民者,虽然也从宗主国带来或多或少的资本,但并不是资本家阶级,他们的生产也不是资本主义生产。这是在或大或小的程度上自己从事劳动的农民,他们主要是为了保证自身的生活,为自己生产生存资料。因此他们的主要产品并不是商品,目的也不是为了做买卖。他们把自己产品中超过他们自己消费的余额卖掉,换取运入殖民地的工业品等等。另一小部分殖民者,住在沿海,住在通航河流附近等地,形成商业城市。这里也还谈不上资本主义生产。但是,即使资本主义生产逐渐开始发展,以致对于自己从事劳动和自己占有土地的农场主来说,开始起决定作用的是出卖自己的产品和由出卖而得的盈利,——即使在这种情况下,也会发现,只要土地对资本和劳动来说还处在自然要素那样的丰富状态,从而,只要土地实际上还是无限的活动场所,也就继续存在第一种殖民形式,因而生产也决不按照市场的需要,即按照某种既定的市场价值来调节。第一类殖民者把他们所生产的超过他们自己直接消费的一切东西投入市场,产品的售价只要高于工资就行。他们是,并且在一个很长时期内依然是那些多少已经按照资本主义方式生产自己产品的农场主的竞争者,因而他们使农产品的市场价格经常低于它们的价值。因此,耕种较坏等级的土地的农场主,只要能得到平均利润,而在出卖自己的农场时能够收回自己的投资,就会感到很满足,因为这在大多数情况下是办不到的。可见,这里有两个重大情况共同起着作用:第一,资本主义生产还没有在农业中占统治地位;第二,土地所有权虽然在法律上存在着,实际上还只是偶然的现象,还只是本来意义上的土地占有。换句话说,虽然土地所有权在法律上存在着,但由于土地对劳动和资本来说作为自然要素而存在的关系,它还不能对资本进行抵抗,还不能把农业变成与非农业生产部门有别的、对投资进行特殊抵抗的活动场所。

在第二种殖民地(种植园)中,一开始就是为了做买卖,为了世界市场而生产,这里存在着资本主义生产,虽然这只是形式上的,因为黑人奴隶制排除了自由雇佣劳动,即排除了资本主义生产的基础本身。但是在这里我们看到的是把自己的经济建立在黑人奴隶劳动上的资本家。他们采用的生产方式不是从奴隶制产生的,而是接种在奴隶制上面的。在这种场合,资本家和土地所有者是同一个人。土地对资本和劳动来说作为自然要素而存在,并不对投资进行任何抵抗,因而也不对资本竞争进行任何抵抗。这里也并没有形成与土地所有者不同的租地农场主阶级。只要维持着这种状况,就没有任何东西妨碍费用价格调节市场价值。

所有这些前提都和绝对地租存在的前提毫无共同之处。绝对地租存在的前提是:一方面有发达的资本主义生产,另一方面有土地所有权,这种土地所有权不仅在法律上存在,而且在实际上对资本进行抵抗,保护这个活动场所不受资本侵占,只有在一定条件下才把地盘让给资本。

在这样的情况下,即使只耕种IV或III或II或I,也会存在绝对地租。资本只有交纳地租,也就是说,只有按照农产品的价值出卖农产品,才能在这唯一存在的土地等级中占领新地盘。也只有在这样的情况下,才能把投入农业(即投入某种自然要素本身,投入初级生产)的资本和投入非农业生产的资本加以比较和区别。

下一个问题是:

在以I为出发点时,很清楚,如果II、III、IV只提供按原来的市场价值所能提供的那些追加供给,那末它们就会按照I所决定的市场价值出卖自己的产品;因此,它们除绝对地租外,还将根据它们各自的相对肥力提供级差地租。相反,当以IV为出发点时,那末这里看来[598]就可能有一些异议。

我们已经看到,[在B表和C表]如果II按照自己产品的价值出卖产品,即按照1+(11/13)镑或1镑16+(12/13)先令出卖产品,II就会得到绝对地租。

在D表,紧接着的下一个等级(按下降序列)即III的费用价格,比能提供10镑地租的IV的产品价值还高。因此,即使III按照自己产品的费用价格出卖产品,这里也谈不上竞争或者按较低价格的供给。但是如果IV已经不能满足整个需求,如果市场要求的比92+(1/2)吨还多,那末产品的价格就会上涨。在上述情况下,产品的价格必定要每吨上涨3+(43/111)先令,III才能作为按照自己产品的费用价格出卖产品的竞争者出现。现在要问:III是否会作为这样的竞争者出现呢?我们现在用另一种方式来说明这种情况。要使IV的产品价格上涨到1镑12先令,即上涨到III的产品的个别价值,需求不一定非增加75吨不可;至少就主要农产品来说情况是这样,主要农产品的供给不足所引起的价格上涨,大大超过与算术上的供给不足相应的程度。但是,如果IV的产品的价格上涨到1镑12先令,那末按照这种等于III的产品个别价值的市场价值,III的产品就会提供绝对地租,而IV就会提供级差地租。一般说来,如果有追加的需求,那末III就能按照自己产品的个别价值出卖产品,因为这时正是III这个等级决定市场价值,而在这种情况下,就不会有任何理由迫使土地所有者放弃地租。

但是,假定IV的产品的市场价格只涨到1镑9+(1/3)先令,即III的产品的费用价格。或者,为使这个例子更明显起见,假定III的产品的费用价格只是1镑5先令,即只比IV的产品的费用价格高1+(8/37)先令。它必定比后者高,因为III的肥力比IV的肥力低。现在III能否被耕种,能否开始同高于III的产品的费用价格也就是按照1镑5+(35/37)先令来出卖自己产品的IV竞争呢?这取决于这里是否有追加需求。如果有追加需求,那末IV的产品的市场价格就会涨到它的价值以上,即涨到1镑5+(35/37)先令以上。这时III在任何情况下都会高于自己产品的费用价格出卖产品,尽管还不能因此而得到它的绝对地租的全额。

或者,没有追加需求。这里又可能有两种情况。III的竞争,只有在耕种III的农场主同时又是这块土地的所有者时才可能发生:对于这种作为资本家的农场主来说,土地所有权不成其为障碍,并不进行抵抗,因为他不是作为资本家而是作为土地所有者来支配这块土地。III的竞争会迫使IV低于原来的价格,即低于1镑5+(35/37)先令,甚至低于III的产品的费用价格1镑5先令出卖自己的产品。这样一来,III就会被击退。而IV总是能够把III击退的,因为它只要把价格降低到本身的费用价格即比III更低的费用价格就行了。但是,如果由于III的产品出现在市场上而造成的价格降低,市场的容量增大了,那又会怎样呢?要就是:市场扩大到如此程度,以致虽有新的75吨出现,IV仍能象以前一样把自己的92+(1/2)吨卖掉;要就是:市场没有扩大到如此程度,以致IV和III都有一部分产品成为过剩的。在这种情况下,IV由于在市场上占统治地位,就会把价格降低,直到III的资本缩小到应有的限度,即投入III的资本数量恰好足以使IV的全部产品被吸收掉。但是按照1镑5先令的价格可以卖掉全部产品,并且,因为III按照这个价格会卖掉这种产品的一部分,所以IV就不能卖得更贵。不过这是唯一可能出现的不是由追加需求引起,但会导致市场容量增大的暂时生产过剩的情况。而这种情况之所以可能发生,只是由于在等级III资本家和土地所有者是同一个人,可见其前提又是:土地所有权不是作为和资本对抗的力量存在,因为资本家自己是土地所有者,并且为了资本家的利益而牺牲土地所有者的利益。如果在等级III土地所有权本身和资本相对抗,那就没有任何理由可以使土地所有者情愿把自己的耕地让人家耕种,而不收取地租,也就是说,使土地所有者在IV的产品价格至少上涨到高于III的产品的费用价格以前,情愿把自己的耕地让人家耕种。如果这种涨价只是[599]微不足道的,那末在任何进行资本主义生产的国家里,III就会仍然被排除于资本活动领域之外,除非这种土地是无论以别的什么形式都不能提供地租的土地。但是,这种土地在提供地租以前,在IV的产品价格上涨到高于III的产品的费用价格以前,也就是在IV除自己原有的地租外还提供级差地租以前,是决不会被耕种的。随着需求的进一步增长,III的产品价格会上涨到它的价值的水平,因为II的产品的费用价格高于III的产品的个别价值。只要III的产品价格超过1镑13+(11/13)先令,就是说,只要这个价格开始为II提供某种地租,II就会被耕种。

但是现在在D表中已经假定I不提供任何地租。然而这个等级不提供地租只是因为:它根据假定已是被耕种的土地,由于IV的出现所引起的市场价值的变动,不得不低于产品价值即按照产品的费用价格来出卖产品。这种土地继续在农业上被利用只有在下述情况下才有可能:即土地所有者自己是农场主,因而在这种个别的场合,同资本相对抗的土地所有权消失了;或者利用土地的农场主是个小资本家,他情愿得到小于10%的利润,或者他是个工人,他想要得到的就是比工资略多一些或者就是工资,他把自己的等于10镑或少一些(例如9镑)的剩余劳动交给土地所有者,而不是交给资本家。虽然在后两种场合都给土地所有者交付租金,但是从经济学来说,这并不是地租,而我们所谈的只是地租。在一种场合,农场主只不过是一个农业工人,在另一种场合,农场主是介于农业工人和资本家之间的一种中间人物。

有种说法认为,土地所有者不能象资本家从某个生产部门抽出自己的资本那样容易地从市场上抽出自己的耕地。再也没有比这样的说法更荒谬的了。对于这种见解的最好的反驳是:在欧洲最发达的国家例如英国,有大量的肥沃土地未被耕种;土地从农业中被抽出,用于修建铁路或房屋,或者为了这些目的留下备用;还有例如在苏格兰高地等处,土地被土地所有者用作射击场或猎场。英国工人为了把未耕种的土地夺到手而进行斗争,但没有成功,这也是最好的反驳。

必须指出:绝对地租额降到自己的正常量以下,这或者是由于市场价值低于该等级的产品的个别价值(如DII的情况),或者是由于好地的竞争迫使一部分资本从坏地中抽出(如BII的情况),或者是由于地租完全消失(如DI的情况),——在所有这些场合,都是以下列各点为前提:

(1)在地租完全消失的地方,土地所有者和资本家是同一个人,就是说,这里土地所有权对资本的抵抗,以及土地所有权对资本活动场所的限制,作为个别的和例外的情况消失了。这里的情形也和殖民地一样,只不过土地所有权这个前提的消失在这里是个别的情况;

(2)比较肥沃的土地的竞争,或者还有比较不肥沃的土地的竞争(在下降序列中),会造成生产过剩并强制地使市场容量增大,会由于强制地降低价格而引起追加的需求。但这里的情况正好是李嘉图所没有假定的,因为他所有的推论都始终从这样一个前提出发:所满足的只是必要的追加需求;

(3)在B、C和D表中II和I完全不提供地租,或者提供的不是绝对地租的全额,因为比较肥沃的土地的竞争,迫使它们低于自己产品的价值出卖产品[或者象BII的情况,从生产中抽出一部分资本]。相反,李嘉图假定,它们是按照产品价值出卖产品,并且总是由最坏的土地决定市场价值,而正是在他认为正常的DI中发生了根本相反的情况。此外,他的推论的前提,总是生产按下降序列进行。

在非农业资本的平均构成是80c+20v,而剩余价值率等于50%时,如果农业资本的构成等于90c+10v,就是说,如果它比工业资本的构成还高——而这[600]对于资本主义生产来说在历史上是不正确的——那就没有绝对地租;如果农业资本的构成是80c+20v——这也是至今没有过的事情——那也不会有绝对地租;如果农业资本的构成比工业资本的构成低,例如等于60c+40v,那就会得到绝对地租。

如果从这个理论出发,那末,根据不同等级的肥力以及它们对市场的关系,即根据这一或那一等级在市场上占支配地位的程度,可以有如下各种情况:

(A)最后一个等级支付绝对地租。它决定市场价值,是因为所有的等级按照这个市场价值只提供必要供给;

(B)最后一个等级决定市场价值;它支付绝对地租,按绝对地租的全率支付,但不按绝对地租原来的全额支付,因为III和IV的竞争迫使它从生产中抽出一部分资本;

(C)I、II、III、IV按照原来的市场价值提供的超额供给,必然引起市场价值的下降,而由较高等级调节的已经下降的市场价值,又造成市场容量的增大。I只支付一部分绝对地租,II只支付绝对地租;

(D)由于较好等级这样调节市场价值,或者说,由于较好等级通过超额供给支配较坏等级,I的地租完全消失,II的地租降到绝对地租的水平以下;最后,

(E)较好等级使市场价值降到I的产品的费用价格以下,从而把I从市场上排挤出去。现在由II来调节市场价值,因为按照这个新的市场价值,所有的三个等级都只提供必要供给。[600]

[600]现在我们回过头来谈李嘉图。

\centerbox{※     ※     ※}

不言而喻,当我们谈农业资本的构成时,其中并不包括土地价值或土地价格。土地价格不过是资本化的地租。

\tchapternonum{[第十三章]李嘉图的地租理论(结尾)}

\vicetitle{[(1)李嘉图关于不存在土地所有权的前提。向新的土地推移取决于土地的位置和肥力]}

现在回过来研究李嘉图著作的第二章《论地租》。首先遇到的是在斯密那里已经熟悉的“殖民理论”\authornote{见本册第253—254页和第265—266页。——编者注}。这里只要简单指出思想上的逻辑联系就够了。

\begin{quote}{“初到一个地方殖民,那里有着大量富饶而肥沃的土地,为维持现有人口的生活只需耕种很小一部分土地,或者,这些人口所能支配的资本实际上只能耕种很小一部分土地,在这样的时候,不存在地租;当大量土地还没有被占有,因此〈因为没有被占有,李嘉图后来把这一点完全忘记了〉谁愿意耕种就归谁支配的时候,没有人会为使用土地付出代价。”(第55页)}\end{quote}

{因此,这里是以不存在土地所有权为前提的。虽然这个过程的描述,对现代民族的殖民来说接近于正确,但是,第一,它不适用于发达的资本主义生产;第二,如果把这个过程设想为旧欧洲的历史发展进程,那就错了。}

\begin{quote}{“按照一般的供求规律,这样的土地是不可能支付地租的,其理由同以上所说的使用空气、水或其他任何数量上无限的自然赐予无须付任何代价一样……使用这些[601]自然力之所以不付代价,是因为它们取之不尽,每个人都可以支配……如果所有土地都具有同一特性,如果它们的数量无限、质量相同,使用土地就不能索取代价〈因为土地根本不能变成私有财产〉,除非它的位置特别有利〈李嘉图本应加上一句:并且归一个所有者支配〉。因此,只是由于土地在数量上并非无限,在质量上并不相同,又因为随着人口的增长,质量较坏或位置比较不利的土地投入耕种,使用土地才支付地租。随着社会的发展,就肥力来说属于二等的土地投入耕种时,在一等地上立即产生地租,这一地租的大小将取决于这两块土地质量上的差别。”(第56—57页)}\end{quote}

正是这一点我们必须加以研究。这里的逻辑联系是这样的:

如果土地,——李嘉图在谈到初到一个地方殖民时(斯密的殖民理论)是这样假定的——如果富饶而肥沃的土地对现有人口和资本来说作为自然要素而存在,实际上是无限的;如果“大量”这种土地“还没有被占有”,因此——因为“还没有被占有”——“谁愿意耕种就归谁支配”,在这种情况下,自然不会为使用土地付任何代价,不会有任何地租。如果土地——不仅对资本和人口来说,而且实际上也是一个无限的要素(象空气和水一样“无限”)——“数量无限”,那末,一个人对土地的占有实际上根本不排斥另一个人对土地的占有。这样,就不可能有任何私人的(也不可能有“公共的”或国家的)土地所有权存在。在这种情况下,如果所有的土地质量相同,那就根本不可能为土地支付地租。至多会向“位置特别有利”的土地的占有者支付地租。

因此,在李嘉图所假定的情况下——即在土地“没有被占有”,“因此”,未被耕种的土地“谁愿意耕种就归谁支配”的情况下——支付地租,那只能是由于“土地在数量上并非无限,在质量上并不相同”,就是说,因为有不同等级的土地存在,而同一等级的土地又是“数量有限”。我们说,在李嘉图的前提下只能支付级差地租。但是,李嘉图不是这样加以限制,而是——撇开他的不存在土地所有权这个前提——立刻匆促作出结论说:使用土地,从来不支付绝对地租,只支付级差地租。

因此,问题的关键在于:如果土地对资本来说作为自然要素而存在,那末,资本在农业方面的活动就会同它在其他任何生产部门的活动完全一样。在这种情况下就不存在土地所有权,不存在地租。至多在一部分土地比另一部分土地肥沃的时候,象在工业中一样,能够有超额利润存在。在农业中,这种超额利润由于有土地的不同肥沃程度为自然基础而作为级差地租固定下来。

相反,如果土地(1)是有限的,(2)是被占有的,如果资本遇到作为前提的土地所有权——在资本主义生产发展的国家,情况正是这样,而在那些不是象旧欧洲那样存在着这种前提的国家,资本主义生产本身就为自己创造这种前提,例如美国就是这样,——那末,土地对资本来说一开始就不是自然要素那样的活动场所。因此,在级差地租之外,还是存在地租的。但是从一个等级的土地推移到另一个等级的土地,不论是按上升序列(I、II、III、IV)还是按下降序列(IV、III、II、I),也都和李嘉图前提下发生的情况不同。因为,不论在I还是在II、III、IV使用资本,都会遭到土地所有权的抵抗,如果倒过来从IV推移到III等等,情况也是一样。从IV推移到III等等的时候,IV的产品价格单是提高到使III使用的资本能够得到平均利润,那是不够的,它必须提高到使III能够支付地租。如果从I推移到II等等,那末,使I能够支付地租的那个价格,不仅能够使II支付这种地租,并且除此之外,还支付级差地租,这是不言而喻的。李嘉图提出的不存在土地所有权的前提,当然排除不了那个受土地所有权的存在制约并与此密切联系的规律的存在。

李嘉图说明了在他的前提下怎样能够产生级差地租之后,接着说:

\begin{quote}{“三等地一投入耕种,二等地立刻产生地租,而且同前面一样,这一地租是由两种土地生产力的差别决定的。同时,一等地的地租也会提高,因为一等地的地租必然总是高于二等地的地租,其差额等于这两种土地使用同量的资本和劳动所获得的产品的差额。每当人口的增长迫使一个国家耕种质量较坏的土地(但这决不是说,人口的每一次增长都会迫使一个国家耕种质量较坏的土地),以增加食物的供应时,[602]一切比较肥沃的土地的地租就会提高。”(第57页)}\end{quote}

这完全正确。

李嘉图接着举了一个例子。但是这个例子(暂且撇开后面要谈的)假定的是下降序列。但是,这不过是假定而已。李嘉图为了把这个假定悄悄地塞进来,他说:

\begin{quote}{“初到一个地方殖民,那里有着大量富饶而肥沃的土地……还没有被占有。”(第55页)}\end{quote}

但是,如果与殖民者的人数相对而言,那里有着“大量贫瘠而不肥沃的土地,还没有被占有”,情况还是一样。土地的富饶或肥沃不是不支付任何地租的前提,而土地的数量无限、没有被占有以及质量相同(不管这个质量在肥沃程度上可能是什么样),才是这种前提。因此,李嘉图在进一步阐述的时候,是这样来表述他的前提的:

\begin{quote}{“如果所有土地都具有同一特性,如果它们的数量无限、质量相同,使用土地就不能索取代价。”(第56页)}\end{quote}

他没有说而且不能说,如果土地“富饶而肥沃”,因为这类条件同这一规律是绝对无关的。如果土地不是富饶而肥沃,而是贫瘠而不肥沃,那末,每一个殖民者都不得不耕种全部土地中的较大部分,因此,随着人口的增长,即使在没有土地所有权存在的情况下,他们也会很快接近于这样的状况:土地同人口和资本相比,实际上不再是绰绰有余,事实上不再是无限的了。

的确,毫无疑问,殖民者自然不会去选择最贫瘠的土地,而是选择最肥沃的土地,就是说,对他们所支配的耕作手段来说是最肥沃的土地。但是这并不是他们进行选择的唯一条件。对他们来说,首先具有决定意义的是位置,是位于沿海、靠近大河等等。美洲西部等地区的土地可以说要多么肥沃就有多么肥沃,但是移民自然地定居在新英格兰、宾夕法尼亚、北卡罗来纳、弗吉尼亚等地,总之,是在东临大西洋的地区。如果说他们选择最肥沃的土地的话,他们只是选择这个地区的最肥沃的土地。这并不妨碍他们后来当人口增加、资本形成、交通工具发达和城市兴建使他们能够到较远地区利用比较肥沃的土地的时候,去耕种西部比较肥沃的土地。他们找的不是最肥沃的地区,倒是位置最好的地区,而在这个地区内,在其他位置条件相同的情况下,自然是找最肥沃的土地。但是,这当然不是要证明,人们是从比较肥沃的地区转到比较不肥沃的地区,而只是证明,在同一地区内,在位置相同的情况下,自然是先耕种比较肥沃的土地,其次才耕种比较不肥沃的土地。

但是,李嘉图在正确地把“大量富饶而肥沃的土地”这个说法改善成具有“同一特性、数量无限、质量相同”的土地这个说法以后,便去举例,接着就跳回到他最初的错误的前提:

\begin{quote}{“最肥沃的和位置最有利的土地首先耕种……”(第60页)}\end{quote}

李嘉图感觉到这个说法的弱点和错误,因而对“最肥沃的土地”又补充了一个新的条件:“位置最有利的”;这个条件是他开头论述时所没有的。显然,他本来应该说“在位置最有利的地区内的最肥沃的土地”,那样,就不致荒谬到把偶然找到的位置最有利于新来移民同宗主国、故乡的老亲友以及同外界保持联系的那些地区,当作殖民者还没有调查清楚也不可能一下子调查清楚的全部土地中“最肥沃的地区”了。

因此,李嘉图的从比较肥沃地区向比较不肥沃地区这个按下降序列推移的假定,完全是偷运进来的。只能这样说:因位置最有利而最早被耕种的地区不支付任何地租,直到在这个地区内从比较肥沃的土地推移到比较不肥沃的土地为止。如果现在转到比第一个地区更肥沃的第二个地区,那末,依照假定,这第二个地区的位置是比较不利的。因此,很可能这一地区的土地的比较肥沃还不足以抵销位置方面的比较不利,在这种情况下,第一个地区的土地将继续支付地租。但是,因为“位置”是一个随着经济发展历史地发生变化的条件,因为它随着交通工具的设置、新城市的兴建、人口的增长等等而必然不断改善,所以很明显,第二个地区生产出来的产品,将逐渐按照一个必然使第一个地区的(同一产品的)地租下降的价格投入市场,而第二个地区,随着它的位置的不利条件的消失,将逐渐作为比较肥沃的土地出现。

[603]因此,很明显:

在李嘉图自己对产生级差地租的必要条件作了正确的和一般的表述(“所有土地都具有同一特性……数量无限、质量相同”)的地方,不包括从比较肥沃的土地推移到比较不肥沃的土地这种情况;

这种情况,从历史上看,就他和亚·斯密所指的美国的殖民过程来说也是错误的,正因为如此,凯里才在这一点上提出了合理的反对意见;

李嘉图自己又用“最肥沃的和位置最有利的土地首先耕种”这个关于“位置”的补充说明,推翻了自己的理论;

李嘉图用一个例子来证明他随意作出的假定,而这个例子又假定了一个尚待证明的情况:即从较好的土地推移到越来越坏的土地;

最后,李嘉图{当然他已经打算用这一点来说明一般利润率下降的趋势}之所以作出这样的假定,是因为他否则就不能解释级差地租,尽管级差地租完全不取决于从I推移到II、III、IV还是从IV推移到III、II、I。

[(2)李嘉图关于地租不可能影响谷物价格的论点。绝对地租是农产品价格提高的原因]

在李嘉图的例子里假定有三个等级的土地,即一等地、二等地、三等地,在投资相等的情况下它们分别提供100夸特、90夸特、80夸特谷物的“纯产品”。“在新地区”一等地最先耕种。

\begin{quote}{“在新地区,肥沃的土地同人口对比起来绰绰有余,因而只需要耕种一等地。”(第57页)}\end{quote}

在这种情况下,“全部纯产品”属于“土地耕种者”,“成为他所预付的资本的利润”。(第57页)这里{我们不是谈种植园}虽然没有以任何资本主义生产为前提,却把这个“纯产品”立刻看作资本的利润,这也是不合适的。但是从“老地区”来的殖民者本人是可以这样看待自己的“纯产品”的。如果现在人口增加到必须耕种二等地的程度,那末一等地就会提供10夸特地租。这里自然要假定二等地和三等地“没有被占有”,而同人口和资本对比起来,它们实际上仍旧是“数量无限”。否则,事情就可能是另外一个样子。因此,在这个前提下,一等地将提供10夸特地租。

\begin{quote}{“因为二者必居其一:或者是,必定有两种农业资本利润率,或者是,必定有10夸特(或10夸特的价值)从一等地的产品中抽出来用于其他目的。不论是土地所有者还是其他任何人耕种一等地,这10夸特都同样形成地租;因为二等地的耕种者,不论他耕种一等地支付10夸特作为地租,还是继续耕种二等地不支付地租,他用他的资本得到的结果是相同的。”(第58页)}\end{quote}

实际上农业资本[在有两个不同等级的土地存在的情况下]有两种利润率,就是说,一等地提供10夸特超额利润(这种超额利润在这种情况下可以固定下来作为地租)。但是在同一生产领域内,对同一种类的资本,因而也对农业资本,不是有两种,而是有许多很不相同的利润率,这不仅是可能的,而且是必然的,——李嘉图自己在两页以后就谈到了这一点:

\begin{quote}{“最肥沃的和位置最有利的土地首先耕种,它的产品的交换价值,象其他一切商品的交换价值一样,是由生产产品并把产品运到市场这一整个过程中所必需的各种不同形式的劳动总量决定的。当质量较坏的土地投入耕种时,原产品的交换价值就会上涨,因为生产产品需要较多的劳动。一切商品,不论是工业品、矿产品还是土地产品,它们的交换价值始终不决定于在只是享有特殊生产便利的人才具备的最有利条件下足以把它们生产出来的较小量劳动,而决定于没有这样的便利,也就是在最不利条件下继续进行生产的人所必须花在它们生产上的较大量劳动;这里说的最不利条件,是指为了把需要的〈在原有价格下〉产品量生产出来而必须继续进行生产的那种最不利的条件。”(第60—61页)}\end{quote}

因此,在每个特殊生产部门不仅有两种利润率,而且有许多利润率,就是说,有许多对一般利润率的偏离。

李嘉图对例子的进一步说明,谈的是在同一土地上不同的[一个接着一个使用的]资本量的效果(第58—59页),这些说明没有必要在这里研究。需要指出的只是下面两个论点:

(1)“地租总是使用两个[604]等量的资本和劳动所取得的产品量之间的差额;”(第59页)

这就是说,只存在级差地租(根据不存在土地所有权的假定)。

(2)“不可能有两种利润率。”(第59页)

\begin{quote}\{“不错,在最好的土地上,花费同以前一样多的劳动仍然能得到同以前一样多的产品,但是因为把新的劳动和新的资本用在比较不肥沃的土地上的人得到的产品较少,产品的价值就会提高。因此,尽管肥沃的土地同较坏的土地相比所提供的利益在任何情况下都不会消失,它只是从土地耕种者或消费者手里转移到土地所有者手里,但是,既然耕种较坏土地需要更多的劳动,既然只有耕种这种土地才能获得我们所必需的原产品的追加供给,这种产品的比较价值就会经常高于它过去的水平,并使这种产品能够交换更多的帽子、衣料、鞋子等等,因为生产这些东西不需要这种追加的劳动量。”(第62—63页)\end{quote}

\begin{quote}“因此,原产品的比较价值之所以提高,是因为在生产最后取得的那一部分产品时花费了较多的劳动,而不是因为向土地所有者支付了地租。谷物的价值决定于不支付地租的那一等土地或用不支付地租的那一笔资本生产谷物所花费的劳动量。不是因为支付地租谷物才贵,而是因为谷物贵了才支付地租;有人曾经公正地指出,即使土地所有者放弃全部地租,谷物价格也丝毫不会降低。这只能使某些租地农场主生活得象绅士一样,而不会减少在生产率最低的耕地上生产原产品所必需的劳动量。”(第63页)\}\end{quote}

经过我上面的探讨之后,对于“谷物的价值决定于不支付地租的那一等土地……生产谷物所花费的劳动量”这个论点的错误,就没有必要再详细论述了。我指出过,最后一等[按质量]土地是支付地租还是不支付地租,是支付全部绝对地租还是只支付它的一部分,或者除了绝对地租以外还支付级差地租(在上升序列中),——这种情况部分地取决于发展方向是按上升序列还是按下降序列,而在任何情况下都取决于农业资本构成同非农业资本构成之比。我还指出过,如果已经假定绝对地租的存在是由于这种资本构成的差别,那末,上述种种情况就取决于市场情况,但是,正是李嘉图所提出的情况只有在两种条件下才能够出现(那时即使不能支付地租,也还可以支付租金):或者是,不论法律上还是事实上都没有土地所有权存在,或者是,较好的土地提供的追加供给只有当市场价值降低时才能在市场上找到销路。

但是,除此以外,在上面所引的那一段话中还有许多错误的和片面的东西。原产品的比较价值(这里无非是指市场价值)之所以可能上涨,除了李嘉图所指出的原因以外,还有别的情况:[第一,]如果原产品到现在为止都是低于它的价值或者低于它的费用价格出卖,——这种情况,总是发生在原产品的生产主要还是为维持土地耕种者的生活的那种社会状态(还有象在中世纪那样当城市产品保持着垄断价格的时候);第二,如果原产品——不同于其他按照费用价格出卖的商品——按照自己的价值出卖的条件还没有形成。

最后,关于如果土地所有者放弃了地租,租地农场主把地租装进了自己的腰包,谷物价格就将保持不变的说法,对级差地租来说是正确的。对绝对地租来说,那是错误的。说这里土地所有权不提高原产品的价格,是错误的。相反,在这种情况下会提高价格,因为土地所有权的干涉使得原产品按照它的价值出卖,而它的价值高于它的费用价格。假定,象前面那样,平均的非农业资本等于80c+20v,剩余价值是50%,利润率就是10%,产品的价值是110。而农业[605]资本等于60c+40v,产品的价值是120。原产品将按照这个价值出卖。如果象在殖民地那样,由于土地相对地绰绰有余,因而土地所有权不论法律上或事实上都不存在,那末,农产品就会按照115出卖。就是说,前一种资本和后一种资本(共200)的全部利润等于30,因而平均利润等于15。非农产品将按照115而不是按照110出卖;农产品将按照115而不是按照120出卖。因此,农产品同非农产品相比,相对价值下降1/12;但是两笔资本或总资本——农业资本和工业资本加在一起——的平均利润提高了50%,从10提高到15。[605]

※     ※     ※

[636]李嘉图谈到他自己对地租的理解时说:

\begin{quote}{“我始终认为地租是局部垄断的结果,它实际上决不调节价格(因此,决不是作为垄断起作用,也就是说,决不是垄断的结果。在李嘉图看来,只有地租不是落进租地农场主的腰包而是落进较好等级的土地所有者的腰包,才能是垄断的结果);地租倒是价格的结果。我认为,如果土地所有者放弃全部地租,土地上生产的商品也不会变得便宜一些,因为这些商品中总有一部分是在不支付地租或不能支付地租的土地上生产的,因为在那里,剩余产品只够支付资本的利润。”(李嘉图《原理》第332—333页)}\end{quote}

这里,“剩余产品”是产品中超过用于工资的部分的余额。李嘉图的论断只有在假定某一等级的土地不支付任何地租的情况下,在这种不支付地租的土地,或者不如说这种土地的产品调节市场价值的情况下才是正确的。相反,如果这种土地的产品不支付地租,是因为比较肥沃的土地调节市场价值,那末,不支付地租这个事实根本不能成为有利于李嘉图的论断的证据。

实际上,如果“土地所有者放弃”级差地租,那它就会归租地农场主所有。相反,放弃绝对地租却会降低农产品的价格,提高工业品的价格,其提高的程度相当于平均利润由于这一过程而增长的程度。[636]

※     ※     ※

[605]“地租的提高总是一个国家的财富增加以及对这个国家已增加的人口提供食物发生困难的结果。”(第65—66页)

这种论断的后一部分是错误的。

\begin{quote}{“在那些拥有最肥沃的土地,输入限制最少,由于农业改良而无需增加相应的劳动量就可以增加生产,因而地租增长得缓慢的国家里,财富增长得最快。”(第66—67页)}\end{quote}

如果地租率不变,只是投于农业的资本随着人口增长而增长,地租的绝对量也可能增加;如果I不支付地租,II只支付一部分绝对地租,但是由于较好土地比较肥沃而级差地租大大增长等等,地租的绝对量也可能增加。(见表\authornote{见第302—303页。——编者注})

[(3)斯密和李嘉图关于农产品的“自然价格”的见解]

\begin{quote}{“如果昂贵的谷物价格是地租的结果而不是地租的原因,价格就会随着地租的高低而成比例地变动,地租就会成为价格的构成部分。但是花费最多的劳动生产出来的谷物是谷物价格的调节者,地租不是也决不可能是这种谷物的价格的构成部分……原料成为大多数商品的组成部分,但是,这个原料的价值,同谷物价值一样,是由最后投入土地并且不支付任何地租的那一笔资本的生产率调节的;因此,地租不是商品价格的构成部分。”(第67页)}\end{quote}

这里,由于混淆了“自然价格”(因为这里所谈的是这种价格)和价值,引起了许多混乱。李嘉图从斯密那里因袭了这种混乱。在斯密那里,这种混乱相对地说还是可以理解的,因为他放弃了,并且仅仅是因为他放弃了他自己对于价值的正确解释。不论地租、利润或工资,都不是商品价值的构成部分。相反,在商品价值既定的情况下,这个价值所能分解成的各个部分,却或者属于积累劳动(不变资本)的范畴,或者属于工资、利润或地租的范畴。而关于“自然价格”,或者说,费用价格,斯密倒可以把它的构成部分当作既定的前提来谈。仅仅由于混淆了“自然价格”和价值,斯密才把这种看法搬到商品的价值上来。

除了原料和机器(简言之,不变资本——它对每个特殊生产领域的资本家来说,都是外来的既定的量,它在每个资本家那里,都以一定的价格加入生产)的价格之外,资本家在确定他的商品价格时还必须考虑到以下两件事。[第一]必须加上工资的价格,这个价格在他看来也是(在一定限度内)既定的。商品的“自然价格”不是指市场价格,而是指一个相当长的时期内的平均市场价格,或者说,市场价格所趋向的中心。因此,在这里,工资的价格总的说来是由劳动能力的价值决定的。至于[第二]利润率——“自然利润率”,那是由非农业生产中使用的全部资本所创造的全部商品的价值决定的。这就是全部商品的总价值超过商品中包含的不变资本的价值和工资价值的余额。这个全部资本所创造的全部剩余价值形成利润的绝对量。利润的这个绝对量同全部预付资本之比决定一般利润率。因此,这个一般利润率不仅对于单个资本家,而且对于每个特殊生产领域的资本来说,也都表现为外来的既定的东西。因此,他必须在产品所包含的预付于原料等等的价格[606]和工资的“自然价格”之上再加一个一般利润,比如说10%,以便这样——在他看来必然表现为这样——通过把各构成部分相加的办法,或者说,通过把它们结合起来的办法,得出商品的“自然价格”。在出卖商品的时候,它的自然价格是否得到支付,是支付得高些还是低些,这取决于当时的市场价格水平。费用价格不同于价值,加入费用价格的只有工资和利润,而地租只有在它已经加入预付于原料、机器等等的价格的限度内才加入费用价格。因此,在资本家看来,地租不是作为地租加入费用价格,对资本家来说,原料、机器的价格,简言之,不变资本的价格,一般说来是作为前提存在的一个整体。

地租不是作为构成部分加入费用价格。如果在特殊情况下农产品按照它的费用价格出卖,那就根本不存在地租。这时,土地所有权对资本来说在经济上是不存在的,也就是说,在按照费用价格出卖的那一级土地产品调节[根据李嘉图的理论]该领域的产品市场价值的情况下,土地所有权是不存在的。(D表I是另外一种情况\endnote{马克思在上一章中指出,D表内的等级I“会起完全消极的作用”(第328页):“决定市场的不是它,而是和它相对的IV、III、II”(第330页),它们向市场施加压力,使产品的市场价值维持在I的产品的个别费用价格的水平,即大大低于这一产品的个别价值的水平。——第360页。}。)

或者(绝对)地租是存在的。在这种情况下农产品高于它的费用价格出卖。农产品按照高于它的费用价格的价值出卖。这样,地租便加入产品的市场价值,或者说得更确切些,成为市场价值的一部分。但是租地农场主把地租看成是既定的,正如工业家把利润看成是既定的一样。地租决定于农产品的价值超过它的费用价格的余额。但是,租地农场主的计算同资本家完全一样:第一是预付的不变资本,第二是工资,第三是平均利润,最后,是在租地农场主看来同样为既定的地租。这对他来说也就是例如小麦的“自然价格”。他是否能得到这个价格,又取决于当时的市场情况。

如果按照实际情况把握住费用价格和价值的差别,那末地租就决不会作为构成部分加入费用价格,而且只有在我们谈到不同于商品价值的费用价格的时候才可以谈构成部分。(级差地租同超额利润一样决不加入[个别]费用价格,因为它始终只是市场费用价格\endnote{马克思说的市场费用价格(themarketcost-price)是指调节某一生产领域商品的市场价格的一般费用价格。参看本册第134—135页,那里市场费用价格称为“一般平均价格”,“市场平均价格”。——第361页。}超过个别费用价格的余额,或者说,只是市场价值超过个别价值的余额。)

因此,当李嘉图同亚·斯密相反,认为地租决不加入费用价格的时候,他在本质上是正确的。但是,从另一方面说,他又错了,因为他证明这一点的方法,不是把费用价格和价值区别开来,而是象亚·斯密一样把两者等同起来;因为不论地租、利润还是工资,都不是价值的构成部分,尽管价值可以分解为工资、利润和地租,而且有同样理由分解为所有这三者,如果这三者都存在的话。李嘉图的论断是这样的:地租不是农产品“自然价格”的构成部分,因为最坏的土地的产品价格等于这个产品的费用价格,等于这个产品的价值,它决定农产品的市场价值。因此,地租并不构成价值的任何部分,因为它不构成“自然价格”的任何部分,而这个“自然价格”等于价值。但是这恰恰是错误的。最坏的土地上种植出来的产品的价格等于它的费用价格,或者是因为这个产品低于它的价值出卖,就是说,决不象李嘉图所说的那样,是因为它按照它的价值出卖,或者是因为这种农产品属于价值和费用价格例外地完全一致的那样一类商品,那样一等商品。如果在某个特殊生产领域中,用一定资本如100货币单位创造的剩余价值,恰巧等于按平均计算应摊到总资本的同样的相应部分(例如100货币单位)的剩余价值,那就是这种情况。因此,这就造成了李嘉图的混乱。

至于亚·斯密,他既然把费用价格和价值等同起来,他从这个错误的前提出发,便有理由说地租同利润和工资一样是“自然价格的构成部分”。而他的前后矛盾却在于,他在进一步说明时,又认为地租不象工资和利润那样加入“自然价格”。他所以这样前后矛盾,是因为观察和正确的分析又使他承认,在非农产品的“自然价格”和农产品的市场价值的规定中存在着某种差别。关于这一点,在我们谈到斯密的地租理论时还要更详细地谈。

[(4)李嘉图对农业改良的看法。他不懂农业资本有机构成发生变化的经济后果]

[607]“我们已经看到:当把追加资本投入产量较少的土地成为必要时,每投入一笔追加资本,地租就提高一次。

(但是并不是每一笔追加资本都生产出较少量的产品。)

根据同样的原理可以得出结论:社会上的任何条件,如果能使我们无须在土地上使用同量资本,从而使最后使用的一笔资本具有较高的生产率,就都会使地租降低。”(第68页)

也就是说,它们会使绝对地租降低,但不一定使级差地租降低。(见B表)

这样一些条件,可以是由于人口减少而发生的“一个国家的资本的减少”,但是,也可以是农业劳动生产力的更高度的发展。

\begin{quote}{“但是,这样的结果也能在一个国家的财富和人口增加的情况下产生,如果随着这种增加农业也进行显著的改良,因为这种改良能够得到使耕种比较贫瘠的土地的必要性减少,或者在耕种比较肥沃的土地时花费同量资本的必要性减少的同样效果。”(第68—69页)}\end{quote}

(奇怪的是李嘉图在这里忘记了:那些改良也可以使比较贫瘠的土地的质量得到改良,并把比较贫瘠的土地变成比较富饶的土地,——这个观点在安德森那里占主导地位。)

李嘉图的下面这一论点是非常错误的:

\begin{quote}{“如果人口不增加,就不可能有对追加谷物量的需求。”(第69页)}\end{quote}

随着谷物价格下降,对其他原产品如蔬菜、肉类等的追加需求将会产生,而且可以用谷物酿制烧酒等等,这些姑且不论;李嘉图在这里假定,全部人口想消费多少谷物就消费多少谷物。这是错误的。

{“我们的消费量在1848、1849、1850年大大增长,说明我们以前吃不饱,说明价格由于供给不足而维持在高水平上。”(弗·威·纽曼《政治经济学讲演集》1851年伦敦版第158页)

同一个纽曼说:

\begin{quote}\{“李嘉图关于地租不能提高价格的论证是根据这样一个假定,就是索取地租的权力在实际生活中决不可能使供给减少。但是为什么不可能呢?有着非常广阔的土地,这些土地,如果不索取地租,立刻就会投入耕种,可是它们人为地荒芜着,这或者是因为土地所有者把它们当作猎场出租可以得到更多的利益,或者是因为土地所有者宁肯让它们成为具有诗情画意的荒野,而不愿让人耕种来取得那一点点徒有其名的地租。”(第159页)\}\end{quote}

如果认为,土地所有者从谷物生产中抽出自己的土地,便不能通过把它变成牧场或建筑地段,或者象苏格兰高地某些地区那样把它变成供狩猎用的人造森林,来取得地租,那是完全错误的。

李嘉图区别了农业上的两种改良。一种改良

\begin{quote}“提高土地的生产力……如采用更合理的轮作制或更好地选用肥料。这些改良确实能使我们从较少量的土地得到同量的产品。”(李嘉图《政治经济学和赋税原理》第70页)\end{quote}

照李嘉图的意见,在这种情况下地租一定下降。

\begin{quote}“例如,如果连续投入的各笔资本的产量分别是100、90、80、70夸特;当我在使用这四笔资本时,我的地租是60夸特,或者说只要我使用的还是这四笔资本,即使每一笔资本的产品有等量的增加,地租仍旧不变。”\end{quote}

(如果产品有不等量的增加,那末,尽管肥力提高了,地租也能提高。)

\todo{}

\begin{quote}“如果产量不是100、90、80、70夸特,而是增加到125、115、105和95夸特,那末地租仍旧是60夸特,或者说:\end{quote}

\todo{}

\begin{quote}但是当产品这样增加时,如果需求没有增加,就没有理由把这样多的资本投在土地上;有一笔将被抽出,因此,最后一笔资本将提供105夸特而不是95夸特,地租降到30夸特,或者说\end{quote}

\todo{}

且不说在价格下降时即使人口不增加,需求也可能增加(李嘉图自己在他所举的例子中就假定需求增加了5夸特);李嘉图之所以从不断向比较不肥沃的土地推移这个前提出发,也正是因为人口每年都在增加,就是说,消费谷物、吃面包的那部分人口在增加,而且这部分人口比整个人口增加得快,因为面包是大部分人口的主要食物。因此,就没有必要假定,需求不会随着[农业]资本的生产率一起增长,所以地租会下降。如果农业改良对于各级土地肥沃程度的差别的影响不一样,地租就可能提高。

此外,毫无疑问(B表和E表),在需求不变的条件下,肥力提高不仅可能把最坏的土地从市场排挤出去,甚至还可能迫使投在比较肥沃的土地上的一部分资本从谷物生产中抽出(B表)。在这种情况下,如果各级土地的产品增加的量相同,谷物地租就下降。

接着,李嘉图谈到第二种农业改良。

\todo{}

\begin{quote}“但是有些改良可能降低产品的相对价值而不降低谷物地租,尽管它们会降低货币地租。这种改良并不提高土地的生产力,但是使我们能够用较少的劳动获得土地产品。这些改良与其说是针对土地耕作方法本身,不如说是针对投在土地上的资本的构成。例如犁和脱粒机等农具的改良,在农业上使用马匹方面的节约,兽医知识的增进,都具有这样的性质。因此投到土地上的将是较少的资本,也就是较少的劳动,但是要获得同量产品,耕种的土地就不能减少。可是这种改良是否影响谷物地租,必然取决于使用各笔资本所得到的产品之间的差额是扩大、不变还是缩小。”\end{quote}

{李嘉图在谈到土地的自然肥力的时候也应该坚持这一点。向新的等级的土地推移,究竟是使级差地租减少、不变还是增加,取决于投在这些肥力不同的土地上的资本的产品之间的差额是扩大、不变还是缩小。}

\begin{quote}“如果有四笔资本50、60、70、80投在土地上,每笔都得到同样的结果,如果这种资本构成的某种改良使我能从每笔资本中减去 5,使它们分别成45、55、65和75,那末谷物地租将不变。但是,如果这种改良使我能够在生产率最低的那一笔资本上进行所有这些节约,那末谷物地租马上就会下降,因为生产率最高的资本和[609]生产率最低的资本之间的差额缩小了,而正是这个差额,构成了地租。”(第73—74页)\end{quote}

对于在李嘉图那里唯一存在的级差地租来说,这是正确的。

不过,李嘉图完全没有接触到真正的问题。为了解决这个问题,不在于每一夸特的价值下降,也不在于是否必须耕种和以前同一数量的土地,同一数量的同等级土地,而在于在农业中使用的直接劳动量的减少、增加或保持不变是否与不变资本的降价(按照假定,不变资本现在耗费较少的劳动)有关。简言之,是否在资本中发生有机的变动。

假定我们以A表为例(手稿第XI本第574页)\authornote{见第302—303页。——编者注},用一夸特小麦代替一吨煤。

这里假定,非农业资本的构成等于80c+20v,农业资本的构成等于60c+40v,两种资本的剩余价值率都等于50%。因而,农业资本的[绝对]地租,或者说,农业资本的产品的价值超过它的费用价格的余额等于10镑。那末,我们得到:

\todo{}

为了在纯粹的形式上研究这个问题,必须假定[农业中]不变资本(100镑)的降价对用于I、II、III三个等级的资本量发生同样的影响,因为不同的影响只涉及级差地租,而与我们现在研究的问题毫无关系。因此,我们假定,由于改良,同样的资本量,以前值100镑,现在只值90镑,就是说它的价值减少了1/10,即10%。现在要问,这些改良对农业资本的构成有什么影响?

如果花在工资上的资本[对不变资本]的比例不变,如果100镑分为60c+40v,那末90镑就分为54c+36v,在这种情况下,I级地上生产的60夸特的价值等于108镑。但是,如果降价表现为不变资本以前值60镑,现在只值54镑,而v(即花在工资上的资本)只值32+(2/5)镑而不是值36镑(再减少1/10),那末,支出的就不是100镑而是86+(2/5)镑。这一资本的构成是54c+[32+(2/5)]v。按100计算,资本构成是[62+(1/2)]c+[37+(1/2)]v。在这种情况下,I的60夸特的价值等于102+(3/5)镑。最后我们假定,虽然不变资本的价值减少了,花在工资上的资本在绝对量上仍然不变,因此它同不变资本相对来说增大了,结果支出的资本90镑分为50c+40v,资本构成按100计算,则等于[55+(5/9)]c+[44+(4/9)]v。

现在我们来看,在这三种情况下谷物地租和货币地租的情况怎样。在B的情况下,c和v的价值虽然减少,c和v的比例却保持不变。在C的情况下,[610]c的价值减少,但v的价值相对地减少得更多。在D的情况下,只有c的价值减少,而v的价值不变。

我们首先把前页的原表\authornote{见本册第366页。——编者注}列出[标以字母A,然后把它同说明上述农业资本有机组成部分价值变动的各种情况的B、C、D三个新表加以对比]。\authornote{在手稿中,下面按次序排列了A、B、C、D各表,这些表印在第368—369页上。C表和D表在手稿中有几栏空着。漏写的数字是编者补上的。最后一栏的标题(《资本构成和绝对地租率》)在手稿中原来没有,也是编者根据该栏的内容补上的。——编者注}

\todo{}

\todo{}

[611]从[第368—369页]所列的[总]表我们可以看到:

最初,在A的情况下,[农业资本的各有机组成部分之间的]比例是60c+40v;投入每级土地的资本都是100镑,地租表现为货币是70镑,表现为谷物是35夸特。

在B的情况下,不变资本降价,因而投入每级土地的资本只有90镑,但是可变资本也相应降价,结果比例不变。这里货币地租减少了,谷物地租不变;绝对地租\endnote{在第368—369页所列的总表里最后一栏以及第370—371页正文里的“绝对地租”,马克思是指绝对地租率。——第370页。}也不变。货币地租减少,是因为投入的资本减少。谷物地租不变,是因为在支出货币量较少的情况下每一货币单位生产的谷物多了,而各级之间的比例保持不变。

在C的情况下,不变资本降价;但是v减少得更多,结果不变资本相对地变贵了。绝对地租减少。谷物地租和货币地租都减少。货币地租减少,是因为资本总的说来大大减少了,而谷物地租减少,是因为绝对地租减少而各级间的差额保持不变,结果所有[各级的谷物地租]都同等地减少了。

在D表中,情况却完全相反。只有不变资本减少,而可变资本不变。李嘉图的前提就是这样。在这种情况下,货币地租由于资本减少,在绝对量上只是稍有减少(只减少1/3镑),但同花费的资本相对来说却有很大增加。相反,谷物地租的绝对量增加了。为什么呢?因为绝对地租从10%提高到[12+(2/9)]%,而这是由于v同c相对来说增加了。

于是,得出下表:

\todo{}

李嘉图继续说:

\begin{quote}“凡是使连续投入同一土地或新地的各笔资本所得产品的差额缩小的事物,都有降低地租的趋势;凡是扩大这种差额的,必然产生相反的结果,都有提高地租的趋势。”(第74页)\end{quote}

当资本从农业中抽出的时候,当坏地变得比较肥沃的时候,或者甚至当比较不肥沃的土地被排挤出市场的时候,这种差额就可能扩大。

{地主和资本家。1862年7月15日《晨星报》\endnote{《晨星报》(《The Morning Star》)是英国的一家日报,自由贸易派的机关报,1856年至1869年在伦敦出版。——第371页。},在一篇论谁有义务(自愿地或被迫地)援助由于棉花歉收和美国内战而处于困境的郎卡郡等地棉纺织工业工人的社论中写道:

\todo{}

\begin{quote}“这些人有合法权利要求用主要由他们自己的勤劳创造出来的财产来维持生活……有人说,那些靠棉纺织工业发了大财的人特别有义务慷慨救济。这毫无疑问是正确的……工商业界已经这样做了……但是,难道他们是靠棉纺织工业发了财的唯一阶级吗?当然不是。郎卡郡和柴郡北部的土地所有者们在这样创造出来的财富中分享了很大一份。而且土地所有者是占了特殊的便宜的,他们分享财富,可是对于创造这个财富的工业却毫无帮助,既不动手,也不动脑……为了[612]创立这个目前正在受到严重震荡的大工业,工厂主付出了他的资本和才干,经常提心吊胆,工厂的工人付出了他的技能、时间和体力劳动;但是,郎卡郡的土地所有者们付出了什么呢?什么也没有,真是一点也没有;但是他们从这个工业得到的实际利益却比另外两个阶级的哪个都多……肯定地说,这些大地主单单由于这个原因而增加的年收入是很大的,很可能至少增加两倍。\end{quote}

”资本家是工人的直接剥削者,他不仅是剩余劳动的直接占有者,而且是剩余劳动的直接创造者。但是,因为剩余劳动对产业资本家来说只有通过生产并在生产过程中才能实现,所以产业资本家本身就是这一生产职能的承担者,生产的领导者。相反,地主在土地所有权上(就绝对地租来说)和在土地等级的自然差别上(级差地租)却拥有一种特权,使他能把这种剩余劳动或剩余价值的一部分装进自己的腰包,尽管他在管理和创造这种剩余劳动或这种剩余价值方面毫无贡献。因此,在发生冲突时,资本家把地主看作纯粹是一个多余而有害的赘疣,看作资本主义生产的游手好闲的寄生虫,看作长在资本家身上的虱子。}

第三章《论矿山地租》。

这里又说:



关于绝对地租,它既不是“价值高昂”的结果,也不是“价值高昂”的原因,而是价值超过费用价格的结果。为矿山产品或土地产品而支付这一余额,从而形成绝对地租,这种情况不是这一余额的结果,因为这种余额在一系列生产部门中都存在,它并不加入这些部门的产品的价格;这种情况倒是土地所有权的结果。

至于级差地租,可以说它是“价值高昂”的结果,只要“价值高昂”是指那些比较富饶的等级的土地或矿山的产品市场价值超过它们的实际价值,或者说,个别价值的余额。

李嘉图所谓调节着最贫瘠的土地或矿山的产品价格的“交换价值”,无非是指费用价格,而他所谓的费用价格,无非是指预付加普通利润,他错误地把这个费用价格与实际价值等同起来,这从下面一段话里也可以看到:



可见,这里直截了当地说:地租等于农产品的价格(在这里也就是“交换价值”)超过它的费用价格的余额,也就是超过预付资本的价值加资本的普通(平均)利润的余额。因此,如果农产品的价值高于它的费用价格,那末,它就能够支付地租,而根本不管土地的差别如何,那时,最贫瘠的土地和最贫的矿山就可以同最富饶的一样支付同样的绝对地租。如果农产品的价值不高于它的费用价格,那末,地租只能来自比较肥沃的土地等等的产品的市场价值超过实际价值的余额。



这种适用于黄金和矿山的情况,也适用于谷物和土地。因此,如果继续开垦的总是同级的土地,如果在花费同量劳动的情况下它们总是提供同量产品,[613]那末一磅黄金或一夸特小麦的价值就保持不变,尽管其数量会随着需求而增加。这就是说,它们的地租(指地租额,不是指地租率)在产品价格没有任何变动的情况下也将增加。使用的资本将会更多,但是资本的生产率始终不变。这是地租的绝对量增长的重大原因之一,它同产品价格的提高毫无关系,因此,不同土地和不同矿山的产品所支付的地租不会发生相应的变动。

[(5)李嘉图对斯密的地租观点和马尔萨斯某些论点的批判]

李嘉图著作第二十四章《亚当·斯密的地租学说》。

这一章对于了解李嘉图和亚·斯密之间的差别是非常重要的。对于这个差别的更深入的阐述(关于亚·斯密),我们留待以后再说,因为考察了李嘉图的学说之后要专门考察斯密的学说。

李嘉图一开始就引了亚·斯密的一段话,照李嘉图的看法,斯密在这一段话里正确地确定了农产品的价格在什么时候提供地租,什么时候不提供地租。但是,后来斯密又认为,土地的某些产品,如食物,应当始终提供地租。

关于这个问题,李嘉图作了下面的评论,这个评论对他[李嘉图]是很重要的:



这些原理当然有很大“不同”。在土地所有权——实际上或法律上——不存在的地方,不会有绝对地租存在。土地所有权的恰当表现,是绝对地租,而不是级差地租。如果说,在有土地所有权存在和没有土地所有权存在的地方,都是同一些原理支配着地租,那就等于说,土地所有权的经济形式不取决于是否存在土地所有权。

其次,所谓“都有这样一种质量的土地,它提供的产品的价值只够补偿……资本并支付……普通利润”,这究竟是什么意思呢?如果同量劳动生产4夸特,同这个劳动生产2夸特对比起来,产品并不具有更大的价值,虽然一夸特的价值在一种情况下是另一种情况下的两倍。因此,产品是否提供地租,与产品的这个“价值”的量本身绝对无关。产品只有在它的价值高于它的费用价格时才能提供地租,而这个费用价格,是由其他一切产品的费用价格决定的,或者,换句话说,是由100货币单位的资本在每一生产部门中平均占有的无酬劳动量决定的。但是,产品的价值是否高于它的费用价格,完全不取决于它的价值的绝对量,而取决于用在它的生产上的资本的构成同用在非农业生产上的资本的平均构成的对比。



这里,李嘉图承认最坏的土地也能够提供地租。他怎么解释这一点呢?为了生产满足追加需求所必要的追加供给而投在最坏土地上的第二笔资本,[614]只有在谷物价格提高的情况下才能补偿费用价格。因此,第一笔资本现在将提供一个超过这个费用价格的余额,即提供地租。所以,情况是这样:在投入第二笔资本以前,因为市场价值高于费用价格,最坏土地上的第一笔资本就已提供地租。因此,问题只是在于,市场价值是否还必须高于最坏土地产品的价值,或者相反,是否产品的价值高于它的费用价格,而价格的提高只是使它能够按照它的价值出卖。

其次:为什么价格必须高到等于费用价格即预付资本加平均利润呢?这是由于不同生产部门的资本的竞争,由于资本从一个生产部门转到另一个生产部门,因此,是通过资本对资本的作用。但是资本通过什么作用才能迫使土地所有权让产品的价值降低到费用价格呢?从农业中抽出资本不能产生这种效果,除非同时使农产品的需求减少。抽出资本倒会产生相反的结果,会使农产品的市场价格涨到农产品的价值之上。把新的资本转到农业中去,同样不能产生这样的效果。因为资本之间的相互竞争恰恰使土地所有者能够要求每个资本家满足于“平均利润”,把价值超过提供这一利润的价格的余额付给土地所有者。

但是,可能提出这样的问题:如果土地所有权使人们有权让产品高于它的费用价格而按照它的价值出卖,那末,为什么土地所有权不能同样使人们有权让产品高于它的价值出卖,就是说,按照任何一个垄断价格出卖呢?在一个没有对外谷物贸易的小岛上,谷物、食品,同其他任何产品一样,无疑能够按照垄断价格出卖,就是说,按照这样一个价格出卖,这个价格只受需求情况的限制,就是说,只受有支付能力的需求的限制,而这个有支付能力的需求随着所提供的产品的价格水平而具有极为不同的大小和范围。

我们撇开这种例外情况不谈,——在欧洲各国根本谈不到这种情况;甚至在英国也有相当大一部分肥沃的土地被人为地从农业,总之从市场抽出去,以便提高其余部分的价值,——土地所有权只是在资本的竞争使商品价值规定发生变化的限度内才能影响和麻痹资本的作用即资本的竞争。价值转化为费用价格只是资本主义生产发展的后果和结果。本来(平均地说)商品是按其价值出卖的。在农业中,土地所有权的存在阻碍着对价值的偏离。

李嘉图说,如果一个租地农场主承租了一块土地,为期七年或十四年,他打算投下譬如10000镑资本,谷物价值(平均市场价值)使他能够补偿预付资本加平均利润加租约上规定的地租。因此,既然他“租用”土地,对他来说,平均市场价值即产品的价值是出发点;利润和地租只是这个价值所分解成的部分,而不是这些部分构成这个价值。既定的市场价格对资本家,就象作为前提的产品价值对理论以及对生产的内在联系一样。这就是李嘉图由此得出的结论。如果租地农场主追加1000镑,他所考虑的仅仅是,在市场价格既定的条件下,这1000镑是否能为他提供普通利润。因此,李嘉图大概是这样想的:费用价格是起决定作用的东西,作为调节要素加入这种费用价格的恰恰是利润,而不是地租。

首先,利润也不是作为构成要素加入费用价格的。按照假定,租地农场主把市场价格作为出发点,计算着在这一既定的市场价格下追加的1000镑是否能为他提供普通利润。因此,这一利润不是这一价格的原因,而是它的结果。但是,李嘉图进一步推论,这1000镑的投入本身,是通过计算那一价格是否能提供普通利润来决定的。因此,利润对于这1000镑的投入,对于生产价格,是决定的因素。

其次,李嘉图说,如果资本家发现这1000镑不能提供普通利润,那他就不会投入这笔资本。就不会有追加食物的生产。如果追加食物的生产是满足追加需求所必需的,那末,需求就必须把价格即市场价格提高到它能提供普通利润的水平。因此,利润不同于地租,在这里利润是作为构成要素加入的,这不是由于利润创造产品的价值,而是由于产品价格如果不提高到除补偿预付资本以外还支付普通利润率的高度,产品本身[615]就不会被创造出来。相反,在这种情况下,价格没有必要提高到足以支付地租的地步。因此,地租和利润之间在这里存在着一个本质差别,在某种意义上可以说,利润是价格的构成要素,地租则不是。(这显然也是亚·斯密的内心想法。)

就这种情况说,这是对的。

但是,为什么呢?

因为,在这种情况下,土地所有权不能作为土地所有权同资本对立,因此,按照假定,这里恰恰不存在形成地租,形成绝对地租的那种组合。用第二笔资本1000镑生产的追加谷物,是在市场价值不变的条件下,也就是在只有假定价格不变时才产生的追加需求的条件下生产出来的,它必须低于它的价值而按费用价格出卖。因此,这1000镑追加产品所处的情况,正象一块新的比较不肥沃的土地投入耕种时的情况一样,这种土地不决定市场价值,而只有在按现有的、原来的市场价值即按一个不由这个新的生产决定的价格来提供追加供给的条件下,才能提供自己的追加供给。在这种情况下,这块追加的土地是否提供地租,完全取决于它的相对肥力,而这正是由于它不决定市场价值。在原有土地上追加1000镑的情况完全一样。而李嘉图恰好由此作出了相反的结论:追加的土地或追加的那笔资本决定市场价值,因为它们的产品价格在市场价值既定、不由它们决定的条件下不提供地租,而只提供利润,不抵偿产品的价值,而只抵偿费用价格!这难道不是自相矛盾吗!

但是这里尽管不提供地租,产品还是在生产!的确是这样!在租地农场主已经租用的土地上,在他本人由于租约实际上成了土地所有者的期间,土地所有权对于他资本家来说,就不是作为独立的、起阻碍作用的要素存在了。因此,资本现在是不受阻碍地在这个要素中活动,对资本来说,能得到产品的费用价格也就满足了。同样,在租佃期满后,租地农场主自然将根据土地投资在多大范围内提供能按自己价值出卖的产品,也就是能提供地租的产品来调节地租。在市场价值既定的条件下不能提供超过费用价格的余额的那部分投资,在确定地租额时是不计算在内的,正如那种由于相对贫瘠而使市场价格仅仅支付产品的费用价格的土地,资本不为它支付地租或租约不规定支付地租一样。

实际情况不完全象李嘉图说的那样。如果租地农场主拥有闲置资本,或者在十四年租期的最初几年获得闲置资本,那末,他在这种情况下并不要求普通利润。只有在他借进追加资本的时候,他才要求普通利润。他究竟用这笔闲置资本来做什么呢?租进新的土地吗?在农业生产上,进行比较集约的投资比起以较大资本进行粗放耕作来,要合算得多。或者,如果在老地附近没有可供租种的土地,那末,租地农场主在经营两个分开的农场的情况下,他的监督管理活动,比加工工业中一个工厂主经营六个工厂还要分散得多。或者,租地农场主只好把货币存在银行里生息,投在公债券、铁路股票等等上面吗?这样,他一开始就要至少放弃普通利润的一半或三分之一。因此,如果他能把这些货币作为追加资本投到原来的农场中,收入虽然会低于平均利润,例如当平均利润等于12%的时候得到的利润为10%,但是,在利率为5%时,他仍然多赚100%。因此,把追加的1000镑[616]投在原来的农场上,对于租地农场主来说,仍然是一笔有利可图的生意。

因此,李嘉图把追加资本的投入[原来的土地]同追加资本用在新的土地上等同起来,是完全错误的。在前一种情况下,就是在资本主义生产中,产品也不一定要提供普通利润。它只是必须提供高于普通利率的利润,使租地农场主感到把自己的闲置资本用于生产虽然要操心和担风险,但还是比用作货币资本合算。

但是,正象已经指出的那样,李嘉图从这个论断得出的下述结论,是非常荒谬的:



李嘉图的例子恰恰证明了相反的情况:这最后一笔资本投入土地,是由市场价格调节的,这个市场价格不取决于这笔资本的投入,它在这笔资本投入以前早已存在,因此它只让最后这笔资本得到利润,而不是地租。说利润是资本主义生产的唯一调节者,那是完全正确的。因此,说生产如果完全受资本调节,就不存在绝对地租,那也是正确的。绝对地租恰恰是在生产条件使土地所有者有权限制资本对生产实行完全调节的地方产生的。

第二,李嘉图责备亚·斯密(第391页及以下各页)[仅仅]在煤矿方面发挥了正确的地租原理;李嘉图甚至说:

***亚·斯密觉得,土地所有者在一定情况下有权力对资本进行有效的抵抗,使人感到土地所有权的力量并因而要求绝对地租,而他在其他情况下就没有这种权力;但是,正是食物的生产确定地租规律,而资本在土地上作其他用途时产生的地租是由农业地租决定的。

李嘉图在反驳斯密时,尽可能地接近真正的地租原理。他说:



这里,李嘉图说出了正确的地租原理。如果最坏的土地支付地租,也就是说,如果支付的地租与土地的自然肥力的差别无关,即支付的是绝对地租,那末这种地租必定等于“产品价值超过资本支出加资本的普通利润的余额”,就是说,等于产品价值超过产品费用价格的余额。李嘉图认为这样的余额是不可能存在的,因为他违反他自己的原理,错误地接受了斯密教条,[617]即产品价值等于产品的费用价格。

此外,李嘉图还犯了一个错误。

级差地租自然决定于“相对肥力”。但是绝对地租同“自然肥力”毫无关系。

可是,另一方面,斯密正确地认为,最坏土地支付的实际地租可以取决于其他土地的绝对肥力和最坏土地的相对肥力,或者取决于最坏土地的绝对肥力和其他等级的土地的相对肥力。

问题在于,最坏土地支付的地租的实际数额,不是象李嘉图所想的那样,取决于这种土地自己的产品价值超过产品费用价格的余额,而是取决于产品市场价值超过产品费用价格的余额。但是,这是极不相同的两回事。如果最坏土地的产品本身决定市场价格,市场价值就等于它的实际价值,因而它的市场价值超过它的费用价格的余额就等于它自己的个别价值(它的实际价值)超过它的费用价格的余额。如果市场价格不取决于最坏土地的产品而由其他等级的土地决定,那末情况就不是如此。李嘉图是从下降序列这个假定出发的。他假定,最坏的土地最后耕种,而且(在假定的场合)只有当追加需求使得按照最后耕种的最坏土地的产品价值提供追加供给成为必要的时候,这种土地才会耕种。在这种情况下,最坏土地的产品价值调节市场价值。而在上升序列中,(即使按照李嘉图的看法)只有在较好等级的土地的追加供给按照原来市场价值仅仅等于追加需求的时候,最坏土地的产品价值才调节市场价值。如果追加供给大于这种需求,李嘉图总是假定,老地一定会停止耕种,结果只能是老地将提供比过去低的地租,或者完全不提供地租。在下降序列中,情况也是一样。如果追加供给只有按照原来的市场价值才能提供,那末,较坏的新地是否提供地租以及提供多少地租,就取决于这个市场价值超过这种土地产品的费用价格的程度的大小。在两种情况下[即无论在上升序列还是在下降序列中],它的地租都是由绝对肥力决定,而不是由相对肥力决定。较好土地的产品的市场价值究竟超过新地产品自己的实际个别价值多少,取决于新地的绝对肥力。

亚·斯密在这里正确地区别了土地和矿山,因为他在谈到矿山时,假定决不会向较坏的等级推移,而总是向较好的等级推移,它们提供的产品总是多于必要的追加供给。那时,最坏土地的地租就取决于它的绝对肥力。



亚·斯密的错误在于,他把最富饶的煤矿(或土地)支配市场这种特殊的市场状况当作一般的情况。但是,如果假定是这种情况,那末,斯密的论证(总的说来)就是正确的,而李嘉图的论证却是错误的。斯密假定,由于需求的情况和较高的富饶程度,最好的煤矿只有在把煤卖得低于竞争者的时候,只有在它们的产品价格低于原来的市场价值的时候,才能使它们的全部产品挤进市场。这样一来,对较次的煤矿来说,产品的价格也下降了。市场价格下降了。这种下降在任何情况下都会压低较次煤矿的地租,甚至可能使它完全消失。因为不论市场价值是否等于某一级土地(或煤矿)的产品的个别价值,地租总是等于市场价值超过产品的费用价格的余额。斯密没有注意到,只有在抽出部分资本和缩减产量成为必要时,利润才可能因此减少。如果在一定情况下由较好煤矿的产品调节的市场价格,降低到使最次煤矿的产品不能提供任何超过费用价格的余额,那末最次的煤矿就只能由其所有者自己开采。在这种市场价格条件下,没有一个资本家会向他支付地租。在这种情况下,他的土地所有权并不赋予他任何支配资本的权力,但是,土地所有权为他排除了其他资本家向土地投资时遇到的那种抵抗。对他来说,土地所有权是不存在的,因为他自己就是土地所有者。因此,他可以把自己的土地用于采煤,就象用于其他任何生产部门一样,也就是说,如果那个不是由他决定而是他发现时就已经确定的产品市场价格给他提供平均利润并补偿他的费用价格,他便可以把自己的土地用于采煤。

李嘉图竟由此得出结论说,斯密自相矛盾!根据原来的市场价格决定新矿在什么情况下可以由它的所有者自己开采,——就是说,新矿可以在土地所有权实际上消失的情况下开采,因为按照原来的市场价格,新矿能保证给企业主提供费用价格,——李嘉图就得出结论说,这个费用价格决定市场价格!但是,他又求助于下降序列,并且说,比较不富饶的煤矿只有在产品的市场价格涨到高于较好的煤矿的产品价值时,才会被开采;其实只要市场价格高于费用价格就行了,或者,对于由所有者自己开采的较次的煤矿来说,甚至只要市场价格能够补偿费用价格就行了。

此外,如果说李嘉图认为,“由于新法开采〈煤的〉产量增加了,价格就会下降,有些煤矿就会被放弃”,那就要知道,这仅仅取决于价格下降的程度和需求的情况。如果在价格这样下降的时候市场还能吸收全部产品,那末,只要市场价格的下降仍能使市场价值保持一个超过较贫的煤矿的费用价格的余额,次的煤矿就仍然会提供地租;如果市场价值只能补偿这一费用价格,即与费用价格一致,那末较贫的煤矿将由它们的所有者开采。但是,在这两种情况下,说最次的煤矿的费用价格调节市场价格,那都是荒谬的。当然,最次的煤矿的费用价格将决定它的产品的价格和起调节作用的市场价格之间的比例,因此决定这个煤矿是否[618]可以开采的问题。但是,在市场价格既定的条件下具有一定富饶程度的土地或煤矿是否可以开发的问题,同这块土地或这个煤矿的产品的费用价格是否调节市场价格,显然是没有关系的,它们根本不是一回事。如果在市场价值提高的情况下需要或可能有追加供给,那末,最坏的土地就调节市场价值,但是,这时候它也就提供绝对地租。这种情况恰恰同斯密所假定的情况相反。

第三,李嘉图(第395—396页)责备斯密,因为斯密认为原产品低廉,例如用马铃薯代替谷物,从而使工资下降,生产费用减少,就会使土地所有者从产品中得到更大的份额,同样也得到更多的产品数量。李嘉图的看法相反:



这一点肯定是错误的。地租所占的份额,因而,地租的相对量将会减少。用马铃薯作主要食物,就会降低劳动能力的价值,缩短必要劳动时间,延长剩余劳动时间,因而提高剩余价值率;因此,在其他条件不变的情况下,资本构成会发生变动,虽然使用的活劳动量仍然和以前一样,可变部分的价值同不变部分的价值相比却减少了。利润率将因此提高。在这种情况下,绝对地租下降,级差地租相应下降。(见第610页C表\authornote{见本册第368—369页。——编者注}。)这种原因将同样地影响农业资本和非农业资本。一般利润率将提高,因而地租将下降。

第二十八章(《论富裕国家和贫穷国家中黄金、谷物和劳动的比较价值》)。李嘉图写道:



第三十二章(《马尔萨斯先生的地租观点》)。李嘉图写道:

\begin{quote}“这种地租〈矿山地租〉同土地的地租一样,是它们产品价值高昂的结果,决不是价值高昂的原因。”(第76页)\end{quote}

\begin{quote}“被开采的最贫的矿山所产金属的交换价值,应当至少不仅足以供给开采金属并把它运到市场上的那些人的衣着、食物和其他生活必需品的费用,而且还足以给预付经营企业所必需的资本的人提供普通平均利润。资本从这种最贫的、不支付地租的矿山得到的报酬,将调节其他一切生产率较高的矿山的地租。假定,这种矿山提供资本的普通利润。其他矿山生产的超过这个普通利润的一切东西,必须作为地租支付给矿山所有者。”(第76—77页)\end{quote}

\begin{quote}“如果用等量劳动加等量固定资本总是可以从不支付地租的矿山获得等量的黄金……〈黄金的〉数量确实会随着需求而增加,但是它的价值不变。”(第79页)\end{quote}

\begin{quote}“我相信直到目前为止,在每一个国家,从最不开化的到最文明的,都有这样一种质量的土地,它提供的产品的价值只够补偿它所花费的资本并支付该国的平均普通利润。我们都知道,美国的情况就是这样,可是谁也没有说,决定地租的原理在美国和在欧洲不同。”(第389—390页)\end{quote}

\begin{quote}“但是,如果说英国的农业已发达到目前已经没有不提供地租的土地这一点是事实,那末,那里以前一定有过这样的土地这一点同样是事实;而且那里有没有这样的土地,对于这个问题是无关紧要的,因为如果大不列颠有任何投在土地上的资本只能补偿资本并为它提供普通利润,那末,不论这笔资本是投在老地或新地上,事情完全一样。如果一个租地农场主承租了一块土地,为期七年或十四年,他可能打算在土地上投下10000镑资本,因为他知道,按当时的谷物和原产品的价格,他能够补偿他所必须花费的资本,支付地租并获得普通利润率。他不会投资11000镑,除非投入这最后1000镑能够给他提供普通的资本利润。当他计算是否投入这一笔追加资本时,他所考虑的仅仅是原产品的价格够不够补偿他的费用和利润,因为他知道他无须支付追加地租。即使在租佃期满后,他的地租也不会提高;如果他的土地所有者因他投了1000镑追加资本而要提高地租,他就会把这笔资本抽回;因为,依照假定,他投入这笔资本只得到把资本用在其他任何地方也能得到的普通平均利润;因此,租地农场主不可能同意为这笔资本支付地租,除非原产品价格进一步提高,或者同样可以说,除非普通一般利润率下降。”(第390—391页)\end{quote}

\begin{quote}“如果亚·斯密的敏锐的头脑注意到这个事实,他就不会认为地租是原产品价格的一个构成部分,因为价格到处都是由不支付任何地租的最后一笔资本的收益调节的。”(第391页)\end{quote}

\begin{quote}“整个地租原理在这里得到了精辟而明确的说明,但是其中每一个字,不仅适用于煤矿,而且适用于土地;可是斯密断然认为,‘地面上的地产却是另外一种情况’。”(第392页)\end{quote}

\begin{quote}“‘它们的产品的价值和它们所提供的地租的价值〈亚·斯密说〉,都是同它们的绝对肥力,而不是同它们的相对肥力成比例。’”(第392页)\end{quote}

\begin{quote}“但是,假定没有不提供地租的土地。这样,最坏土地的地租额将同产品价值超过资本支出加资本的普通利润的余额成比例。同一原理将决定质量或位置比较好的土地的地租,因此,这些土地的地租,由于它们有较大的优越性,将高于比它们坏的土地的地租。对于第三种质量更好的土地,一直到最好的土地,都可以这样说。因此,正是土地的相对肥力决定作为地租支付的那部分产品,正象矿山的相对富饶程度决定作为矿山地租支付的那部分产品一样,这一点难道不是很清楚吗?”(第392—393页)\end{quote}

\begin{quote}“亚·斯密说,有一些煤矿只能由其所有者来开采,因为它们只能补偿开采的费用和所用资本的普通利润。在他说了这样的话以后,我们本来希望他会承认,正是这些煤矿调节一切煤矿的产品的价格。如果老矿不能提供煤的全部需要量,那末,煤的价格就会上涨,并且一直上涨到新的较贫的煤矿的所有者发现开采他的煤矿能获得资本的普通利润为止……因此,可以说,永远是最贫瘠的煤矿调节煤的价格。可是,亚·斯密的看法却不同。他认为,‘最富饶的煤矿也调节邻近其他一切煤矿的煤的价格。不论是这些煤矿的所有者还是从事开采煤矿的企业主都会发现,如果煤的卖价比邻近的煤矿低一些,煤矿所有者就能得到更多的地租,企业主就能得到更多的利润。邻近的煤矿很快就会被迫按同一价格出卖自己的煤,虽然他们这样做不那么容易,虽然这样做总会减少他们的地租和利润,有时还会使他们完全失去地租和利润。结果,一些煤矿完全被放弃,另外一些煤矿提供不了地租,而只能由它们的所有者开采’。如果煤的需求[617a]减少了,或者由于新法开采产量增加了,价格就会下降,有些煤矿就会被放弃。但是,在任何情况下,煤的价格都必须足以支付不担负地租的煤矿的开采费用和利润。因此,价格是由最贫的煤矿调节的。确实,亚·斯密自己在另一个地方也承认了这一点,因为他说:‘煤正象其他一切商品一样,在一个较长时间内能够出卖的最低价格,就是仅仅足以补偿使煤进入市场所使用的资本加上它的普通利润的价格。在土地所有者不能得到地租而必须或者亲自开采,或者干脆放弃的煤矿上,煤的价格一般必然接近于这一价格。’”(第393—395页)\end{quote}

\begin{quote}“这个附加额的任何一部分都不会归入地租,它必然全部归入利润……只要被耕种的土地质量相同,它们的相对肥力或其他优越条件又没有变动,地租对总产品的比例总是保持不变。”(第396页)\end{quote}

\begin{quote}“斯密博士贯穿于全书的一个错误,就是假定谷物的价值不变,虽然其他一切物品的价值可能提高,谷物的价值却永远不会提高。在他看来,谷物的价值始终不变,因为它能养活的人数始终相同。同样也可以说,衣料的价值始终不变,因为它能制成的上衣的数量始终相同。价值同物品用作衣食的能力又有什么相干呢?”(第449—450页)\end{quote}

\begin{quote}“……斯密博士……十分巧妙地论证了商品的市场价格归根结底是由商品的自然价格调节的这一理论。”(第451页)\end{quote}

\begin{quote}“……黄金的价值如果用谷物来表现,在两个不同的国家可能极不相同。我曾竭力证明黄金的价值在富裕的国家低,在贫穷的国家高。亚当·斯密的看法却不同:他认为,用谷物表现的黄金的价值在富裕的国家最高。”(第454页)\end{quote}

\begin{quote}“地租是价值的创造,但不是财富的创造。”\endnote{李嘉图把地租叫作“价值的创造”(《acreationofvalue》),是在这样的意义上说的:地租使土地所有者有可能支配整个社会产品的价值增殖额,在李嘉图看来,这种价值增殖额是由于这一或那一部分谷物生产的困难增加造成的,这种价值增殖额李嘉图叫做“名义上的”,因为社会实际财富并不因此而有丝毫增加。李嘉图在他的著作第三十二章中对马尔萨斯把地租看作是“一种纯收益和新创造的财富”的观点进行了批判,并且提出这样的论点:地租根本不会使整个社会的财富有任何增加,它只是“谷物和商品的一部分价值从原来的所有者手里转到土地所有者手里”。后来马克思在本册第627页上更完整地引用了李嘉图所著《原理》中的这一段。它成了马克思论“虚假的社会价值”的学说的出发点(见马克思《资本论》第3卷第39章)。并参看注30。——第387页。}(第485—486页)\end{quote}

\begin{quote}“当马尔萨斯先生谈到谷物的高昂价格时,他所指的显然不是一夸特或一蒲式耳谷物的价格,而是全部产品销售价格超过产品生产费用的余额,而他的‘生产费用’一词总是既包括工资,又包括利润。只要生产费用相同,每夸特值3镑10先令的谷物150夸特,就会比每夸特值4镑的谷物100夸特给土地所有者提供更多的地租。”(第487页)“不论土地的性质如何,高额地租必然取决于产品的高昂价格;但是,如果高昂的价格是既定的,地租的高度就必然同产品的丰富成比例,而不是同产品的匮乏成比例。”(第492页)\end{quote}

\begin{quote}“因为地租是谷物价格高昂的结果,所以地租的消失便是谷物价格低廉的结果。外国进口的谷物决不会同提供地租的国内生产的谷物竞争。价格下跌必然会打击土地所有者,直到他的地租全部被吞没;如果价格继续下降,它就连资本的普通利润也不能提供;在这种情况下,资本就会放弃土地去寻找别的用途,而以前在这一土地上生产的谷物,就会在这个时候,但不会早于这个时候,被进口谷物代替。由于地租消失,价值,用货币表现的价值,也会随之遭受损失,但是财富却会因此增长。原产品和其他产品的总量将增加;但是,由于生产起来比较容易,这些产品的数量虽然增加,它们的价值却会减少。”(第519页)\}\end{quote}

\tchapternonum{[第十四章]亚·斯密的地租理论}

\tsectionnonum{[(1)斯密在地租问题提法上的矛盾]}

[619]在这里,我们不去探讨斯密的这种有趣说法:从主要植物性食物得到的地租,决定其余所有严格意义上的农业(畜牧业、林业、经济作物种植业)的地租,因为这些生产部门是可以互相转化的。在以大米为主要植物性食物的地方,斯密把大米除外,因为稻田不能转化为草地、麦田等等,反过来也是一样。

斯密正确地下定义说,地租是“为使用土地而支付的价格”([1802年法文版]第1卷第299页),在这里土地应理解为各种自然力本身,因而也包括水力等等。

同洛贝尔图斯的奇特的观念\endnote{马克思指洛贝尔图斯关于农产品生产费用中不包括原料价值的论点。见本册第8章第4节。——第388页。}相反,斯密在[第十一章]引言中就列举了农业资本的各个项目:“置备种子〈原料〉、支付劳动报酬、购买并维持牲畜和其他农具”。(同上)

但是,什么是这种“为使用土地而支付的价格”呢?

\begin{quote}{“产品或产品价格超过这一部分{即补偿预付资本“和普通利润”的部分}的余额,不论这个余额有多大,土地所有者都力图把它作为自己土地的地租攫为己有。”(同上,第300页)“这个余额始终可以看作自然地租。”(第300页)}\end{quote}

斯密反对把地租和投在土地上的资本的利息混淆起来:

\begin{quote}{“土地所有者甚至对于未经人力改良的土地也要求地租”(第300—301页),}\end{quote}

他补充说,就是这第二种地租形式\authornote{指经过改良的土地的地租。——编者注},也有一个特点,即用于改良土地的资本的利息,并不是土地所有者投下的资本的利息,而是租地农场主投下的资本的利息。

\begin{quote}{“他〈土地所有者〉有时对于完全不适于人们耕种的土地也要求地租。”(第301页)}\end{quote}

斯密非常明确地强调,土地所有权即作为所有者的土地所有者“要求地租”。斯密因此把地租看作土地所有权的单纯结果,认为地租是一种垄断价格,这是完全正确的,因为只是由于土地所有权的干预,产品才按照高于费用价格的价格出卖,按照自己的价值出卖。

\begin{quote}{“被看成是为使用土地而支付的价格的地租,自然是一种垄断价格。”(第302页)}\end{quote}

这确实是一种仅仅由于土地所有权的垄断才不得不支付的、并且在这方面作为垄断价格与工业品价格不同的价格。

从资本——而资本在生产中占统治地位——的观点看来,费用价格只要求产品除支付预付资本之外,还支付平均利润。在这种情况下,产品——不管是土地产品或别的什么产品——就能够

\begin{quote}{“进入市场”。“如果普通价格超过足够价格,它的余额自然会归入地租。如果它恰好是这个足够价格,商品虽然完全能够进入市场,但是不能给土地所有者提供地租。价格是否超过这个足够价格,这取决于需求。”(第1卷第302—303页)}\end{quote}

现在要问:为什么按照斯密的意见,地租以不同于工资和利润的方式加入价格?最初斯密正确地把价值分解为工资、利润和地租(撇开不变资本)。但是他立即走上了相反的道路,把价值和“自然价格”(即由竞争决定的商品的平均价格,或者说,费用价格)等同起来,认为后者是由工资、利润和地租构成的。

\begin{quote}{“这三部分看来直接地或最终地构成……全部价格。”(第1卷第101页)(第1篇第6章)“但是,就是在最发达的社会里,也总是有为数不多的一些商品,它们的价格只分解为两部分,即工资和资本的利润,还有为数更少的商品,它们的价格只由工资构成。例如,海鱼的价格中,就是一部分用于偿付渔人的劳动,另一部分用于支付投在渔业上的资本的利润。地租很少构成这个价格的一部分[620]……在苏格兰的一些地区,贫民以在海滨捡拾各种色彩的通称苏格兰玛瑙的小石子为业。雕石业主付给他们的小石子的价格,完全由他们的劳动报酬构成;地租和利润都不形成这种价格的任何部分。但是任何一个商品的全部价格,最终总是分解为这三部分中的一、两部分或所有三部分。”(第1卷第103—104页)(第1篇第6章)}\end{quote}

在上面的引文中(而且在整个论述“商品价格的构成部分”的第六章),价值分解为工资等等和价格由工资等等构成这类说法混杂在一起。(只是到第七章,才第一次谈到“自然价格”和“市场价格”。)

第一篇的第一、二、三章论述“分工”,第四章论述货币。在这几章以及以后几章,附带地提出了价值规定。第五章论述商品的实际价格和名义价格,论述价值向价格转化。第六章是《论商品价格的构成部分》。第七章论述自然价格和市场价格。然后,第八章论述工资。第九章论述资本利润。第十章论述各个使用劳动和资本的部门的工资和利润。最后,第十一章论述地租。

但是这里我们想首先要注意下面一点:按照刚刚引过的论点,有些商品的价格只由工资构成,另一些商品的价格只由工资和利润构成,最后,还有一些商品的价格由工资、利润和地租构成。因此:

\begin{quote}{“任何一个商品的全部价格……总是分解为这三部分的一、两部分或所有三部分。”}\end{quote}

根据这一点,也就没有理由说,地租是以不同于工资和利润的方式加入价格的;但是应该说,地租和利润是以不同于工资的方式加入价格的,因为后者是始终加入的,而地租和利润却不是始终加入的。这种差别是从哪里来的呢?

其次,斯密应当研究这样一个问题:只有工资加入的少数商品,能不能按照它们的价值出卖?或者说,那些收集苏格兰玛瑙的贫民,是否就不是雕石业主的雇佣工人?这些雕石业主对这种商品只付给他们普通工资,也就是说,对表面看来完全属于他们的整个工作日所付的报酬,只和其他部门(这里工人的工作日的一部分构成不属于他自己而属于资本家的利润)的工人得到的一样多。斯密应当要么承认这一点,要么相反地说明,在这种场合利润只是在表面上表现为同工资没有区别的东西。他自己说:

\begin{quote}{“当这三种不同的收入属于不同的人时,它们是很容易区分的;但是当它们属于同一个人时,它们往往会彼此混淆,至少在日常用语上是这样。”(第1卷第106页)(第1篇第6章)}\end{quote}

然而在斯密那里,问题是这样解决的:

如果一个独立劳动者(和上述苏格兰贫民一样)只使用劳动(而不必同时使用资本),一般说来,只使用自己的劳动和自然要素,价格在分解时就只归结为工资。如果劳动者还使用少量资本,他一个人就既取得工资又取得利润。最后,如果他使用自己的劳动、自己的资本和自己的土地所有权,他一个人就兼有土地所有者、租地农场主和工人这三重身分。

{斯密在问题提法上的全部荒谬之处,在第一篇第六章结尾中暴露出来了:

\begin{quote}{“因为在一个文明国家里,只有极少数商品的全部交换价值仅由劳动产生〈这里把劳动和工资等同起来了〉,绝大多数商品的交换价值中有大量地租和利润加入,所以,这个国家的劳动的年产品〈可见在这里,商品仍然等于劳动产品,尽管不是“这种产品的全部价值仅由劳动产生”〉所能购买和支配的劳动量,比这种产品的生产、加工和运到市场所必须使用的劳动量总要大得多。”(同上,第1卷第108—109页)}\end{quote}

结果,劳动产品不等于这种产品的价值。不如说(可以这样来理解斯密的意思),这个价值由于加上利润和地租而增大了。因此,劳动产品可以支配、购买更大的劳动量,也就是说,它能购买的劳动形式的价值比它本身包含的劳动量所构成的价值要大。这个论点如果这样表达就对了:

[621]斯密说:

\begin{quote}{“因为在一个文明国家里,只有极少数商品的全部交换价值仅由劳动产生,绝大多数商品的交换价值中有大量地租和利润加入,所以,这个国家的劳动的年产品所能购买和支配的劳动量,比这种产品的生产、加工和运到市场所必须使用的劳动量总要大得多。”}\end{quote}

根据他自己的观点,应当说:

\begin{quote}{“因为在一个文明国家里,只有极少数商品的全部交换价值在分解时只归结为工资,绝大多数商品的价值中有很大部分分解为地租和利润,所以,这个国家的劳动的年产品所能购买和支配的劳动量,比这种产品的生产、加工和运到市场所必须支付的(也就是使用的)劳动量总要大得多”}\end{quote}

(斯密在这里又回到了他的第二种价值概念;他在这一章谈到价值时说道:

\begin{quote}{“应当注意到,价格的各个不同构成部分的实际价值,是以每一构成部分所能购买或支配的劳动量来衡量的。劳动〈在这个意义上〉不仅衡量价格中归结为劳动〈应当说:工资〉的部分的价值,而且还衡量归结为地租的部分和归结为利润的部分的价值。”(第1卷第1篇第6章第100页)}\end{quote}

在第六章里,主要还是“价值分解为工资、利润和地租”。只是在论述自然价格和市场价格的第七章里,价格由这些构成要素构成的观点才占了上风。)

总之:劳动的年产品的交换价值,不仅由生产这种产品所使用的劳动的工资构成,而且由利润和地租构成。但是支配或者说购买这种劳动的,只是价值中归结为工资的部分。因此,如果把利润和地租的一部分用于支配或者说购买劳动,也就是,如果把这一部分变为工资,能够推动的劳动量就大得多。这样就得出如下的结果:劳动的年产品的交换价值分解为有酬劳动(工资)和无酬劳动(利润和地租)。如果把归结为无酬劳动的那部分价值的一些份额变为工资,那末,比起单单使用由工资构成的那部分价值来重新购买劳动,就可以买到更大量的劳动。}

现在回到我们的本题。

\begin{quote}{“如果一个独立劳动者拥有小量的资本,足以购买原料并维持生活直到能把他的产品运到市场,他就将同时获得一个给老板干活的帮工的工资以及这个老板从出卖帮工的劳动产品中取得的利润。不过这个劳动者的全部收入通常被称为利润,在这里,工资同利润混淆起来了。一个自己亲手种植自己果园的果园业者,一身兼有土地所有者、租地农场主和工人这三种不同的身分。所以,他的产品应该向他支付土地所有者的地租、租地农场主的利润和工人的工资。但是这一切通常都被看成他的劳动所得。在这里,地租和利润,又同工资混淆起来了。”(第1卷第1篇第6章第108页)}\end{quote}

斯密在这里实际上把所有的概念都混淆起来了。难道“这一切”不是“他的劳动所得”吗?相反,把这个果园业者的劳动产品,或者更确切地说,把这种产品的价值,一部分看成作为他的劳动报酬的工资,一部分看成使用的资本的利润,一部分看成应交给土地,或者更确切地说,应交给土地所有者的地租,这难道不是把资本主义生产关系(在资本主义生产关系下,随着劳动同劳动的客观条件分离,工人、资本家和土地所有者作为三种不同的身分而互相对立)转到这个果园业者身上吗?在资本主义生产范围内,对于上述各要素(实际上)并不相互分离的那种劳动关系来说,把这些要素假定为相互分离的,从而把这个果园业者当作一身兼任自己的[622]帮工和自己的土地所有者,那也是完全正确的。但是这里斯密已经明显地流露出一种庸俗的观念,似乎工资由劳动产生,而利润和地租则不依赖于工人的劳动,由当作独立源泉(不是当作占有别人劳动的源泉,而是当作财富本身的源泉)的资本和土地产生。在斯密那里,最深刻的见解和最荒谬的观念就这样奇怪地交错在一起,而这种荒谬的观念,是由从竞争现象抽象出来的庸俗意识形成的。

斯密首先把价值分解为工资、利润和地租,随后又反过来,用不依赖价值而决定的工资、利润和地租来构成价值。这样他就忘记了他原来正确阐述过的利润和地租的起源,因此他才能说:

\begin{quote}{“工资、利润和地租,是一切收入的三个原始源泉,也是一切交换价值的三个原始源泉。”(第1卷第105页)(第1篇第6章)}\end{quote}

按照他自己的论证,他本来应该说:

\begin{quote}{“商品的价值只由包含在这个商品里的劳动(劳动量)产生。这个价值分解为工资、利润和地租。工资、利润和地租,是雇佣工人、资本家和土地所有者分配由工人劳动创造的价值的原始形式。从这个意义上说,工资、利润和地租是一切收入的三个原始源泉,虽然这些所谓源泉没有一个参与创造价值。”}\end{quote}

从前面的各段引文中可以看到,斯密在论述“商品价格的构成部分”的第六章里,在只有劳动(直接劳动)加入生产时,把价格归结为工资;在不是一个独立劳动者,而是一个帮工受雇于资本家(即有资本存在)时,把价格分解为工资和利润;最后,在除了资本和劳动之外还有“土地”加入生产时,把价格分解为工资、利润和地租;但是在最后这种情况下,又预先假定土地已被占有,也就是说,除了工人和资本家还有土地所有者(虽然斯密指出,所有这三种独特的身分——或者其中两种——可以一人兼而有之)。

而在论述自然价格和市场价格的第七章里,地租完全和工资、利润一样,被说成是自然价格的构成部分(在土地加入生产时)。

下面的引文(第1篇第7章)就是证明:

\begin{quote}{“如果一种商品的价格恰好足够按自然率支付地租、工资和用于生产、加工商品并把它运到市场去的资本的利润,这种商品就是按照可以叫作它的自然价格的价格出卖。商品在这种情况下恰好按其所值出卖。”(第1卷第111页)(同时在这里,自然价格被说成和商品价值是等同的。)“单个商品的市场价格,决定于市场上现有的这种商品的数量,与愿意支付这种商品的自然价格,或者说,使商品进入市场所必须支付的地租、利润和工资的全部价值的人的需求之间的比例。”(第1卷第112页)“如果某种商品进入市场的数量不能满足对这种商品的实际需求,那些愿意支付使这种商品进入市场所必需的地租、工资和利润的全部价值的人,就不可能全都得到他们所需要的这种商品的数量……于是,市场价格就会或多或少地高于自然价格,高多少,取决于这种商品的不足额或竞争者的财富和奢欲所引起的竞争程度。”(第1卷第113页)“如果商品进入市场的数量超过了对它的实际需求,这个数量就不可能全部卖给那些愿意支付使这种商品进入市场所必需的地租、工资和利润的全部价值的人……于是,市场价格就会或多或少地低于自然价格,低落多少,取决于商品的超过额所引起的卖者之间的竞争程度,或者说,取决于卖者急于使商品脱手的程度。”(第1卷第114页)“如果进入市场的数量恰好足够满足实际需求,那末,市场价格当然就会和自然价格完全一致……不同卖者之间的竞争会强迫他们接受这个价格,但是不会强迫他们接受更低的价格。”(第1卷第114—115页)}\end{quote}

[623]斯密认为,如果地租由于市场状况而低于或高于它的自然率,土地所有者就会把自己的土地从生产中抽出,或者从一种商品(例如小麦)的生产转到另一种商品的生产(例如牧场)[或者相反,扩大自己商品的生产]。

\begin{quote}{“如果这个〈进入市场的〉数量在一段时间内超过了实际需求,商品价格的某一构成部分就必然会低于其自然率被支付。如果这是地租,土地所有者受利益的驱使,就会立即把自己的一部分土地从这种生产中抽出。”(第1卷第115页)“反之,如果进入市场的商品量在一段时间内不能满足实际需求,商品价格的某一构成部分就必然会提高到自己的自然率以上。如果这是地租,所有其余的土地所有者受利益的驱使,自然会利用更多的土地来生产这种商品。”(第1卷第116页)“商品市场价格的偶然的和暂时的波动,主要是影响商品价格中分解为工资和利润的部分。对于分解为地租的部分影响较小。”(第1卷第118—119页)“垄断价格是在一切情况下可能得到的最高价格。相反,自然价格,或者说,由自由竞争形成的价格,虽不是在一切场合,但在一段较长的时间内,却是可以接受的最低价格。”(第1卷第124页)“一种商品的市场价格虽然能够长期高于自然价格,却不大可能长期低于自然价格。不管这种价格的哪一部分是低于其自然率支付的,那些利益受影响的人,很快就会感到受了损失,并立即把若干土地,或若干劳动,或若干资本从这种行业中抽出,从而使这种商品进入市场的数量很快只够满足实际的需求。因此,这种商品的市场价格很快就会提高到它的自然价格的水平;至少在有完全自由的地方是这样。”(第1卷第125页)}\end{quote}

在第七章作了这样的论述之后,很难理解,斯密在第十一章(第一篇)《论地租》有什么根据断言,在被占有的土地加入生产的地方,地租却不是始终加入价格的;很难理解,他怎么能把地租加入价格的方式同利润、工资加入价格的方式区别开来,因为他在第六章和第七章已经把地租说成完全同利润、工资一样,是“自然价格”的构成部分。现在我们回过来谈第十一章(第一篇)。

我们看到,在第六章和第七章,斯密下定义说,地租是产品价格在支付资本家(租地农场主)的预付资本和平均利润之后剩下的余额。

在第十一章,斯密却完全颠倒过来。地租已不加入自然价格。或者,更确切地说,亚·斯密在这里求助于通常与自然价格不同的普通价格,虽然在第七章我们曾经听说,普通价格决不会长期低于自然价格,普通价格决不能长期低于自然价格的自然率支付自然价格的某一构成部分,更不能象他现在谈到地租时所说的那样,完全不支付。斯密也没有告诉我们,在产品不支付地租时,它是否低于自己的价值出卖,或者说,在它支付地租时,它是否高于自己的价值出卖。

以前,商品的自然价格是

\begin{quote}{“使商品进入市场所必须支付的地租、利润和工资的全部价值”。(第1卷第112页)}\end{quote}

现在我们听到:

\begin{quote}{“通常能够进入市场的只有那样一些土地产品,其普通价格足够补偿使产品进入市场所使用的资本,并提供普通利润。”(第302—303页)}\end{quote}

因此,普通价格并不是自然价格,而且要使商品进入市场,也无须支付它的自然价格。

[624]以前我们听说,如果普通价格(在第七章叫做市场价格)不够支付全部地租(地租等等的全部价值),土地就会从生产中抽出,直到市场价格提高到自然价格的水平并开始支付全部地租为止。现在,我们却听到:

\begin{quote}{“如果普通价格超过足够〈补偿资本和支付这笔资本的普通利润的〉价格,它的余额自然会归入地租。如果它恰好是这个足够价格,商品虽然完全能够进入市场,但是不能给土地所有者提供地租。价格是否超过这个足够价格,这取决于需求。”(第1卷第303页)(第1篇第11章)}\end{quote}

地租从自然价格的构成部分突然变成了超过足够价格的余额,有没有这个余额,取决于需求的状况。但是足够价格是使商品进入市场,也就是使商品生产出来所必需的价格,即商品的生产价格。因为供给商品所必要的,使商品生产出来并作为商品出现在市场上所必要的价格,当然是商品的生产价格,或者说,费用价格。这是商品存在的必要条件。另一方面,对某些土地产品的需求,必然总是使这些产品的普通价格提供一个超过生产价格的余额,也就是提供地租。而对另外一些土地产品来说,需求可以是这样,也可以不是这样。

\begin{quote}{“对有些土地产品的需求,必然总是使它们的卖价超过足够使它们进入市场的价格。还有一些土地产品,对它们的需求可能使它们的卖价超过足够价格,也可能使它们的卖价不超过这样的价格。前一类产品必然始终向土地所有者提供地租,后一类产品有时提供地租,有时则不提供,这要看情况如何而定。”(第1卷第303页)}\end{quote}

这样,我们在这里看到的不是自然价格,而是足够价格。普通价格又和这个足够价格不同。普通价格包括地租时,就超过足够价格。普通价格不包括地租时,就等于足够价格。而不包括地租,甚至是足够价格的特征。如果普通价格只能补偿资本,而不能支付平均利润,它就低于足够价格。因此,足够价格实际上就是李嘉图从亚·斯密学说中抽象出来的,并且从资本主义生产的观点来看确实出现的生产价格,或者说,费用价格,也就是说,这是一种除了支付资本家预付资本以外还能支付平均利润的价格,这是各个投资领域的资本家相互竞争所造成的平均价格。正是这种对竞争现象的抽象,使斯密把足够价格和他提出的自然价格对立起来,虽然斯密在对自然价格的说明中相反却宣称,只有支付自然价格各构成部分(地租、利润、工资)的普通价格,才是较长时期的足够价格。因为商品生产是由资本家支配的,所以足够价格就是对资本主义生产来说、从资本的观点来说是足够的价格,而这种对资本来说是足够的价格不是包括地租,而是相反,排除地租。

另一方面,这个足够价格对于某些土地产品来说却不是足够的。对于这些产品,普通价格必须高到能提供一个超过“足够价格”的余额,这样才能给土地所有者提供地租。对于另外一些土地产品,据说这又要看情况而定。矛盾在于:足够价格并不足够,足够使产品进入市场的价格并不足够使产品进入市场。而这个矛盾并没有使斯密感到不安。

虽然斯密没有稍微回过去看一看他在第五、六、七章中所发挥的论点,但他毕竟还是意识到他已经用这个“足够价格”推翻了他关于“自然价格”的全部学说(不过他认为这不是矛盾,而是他无意中碰到的新发现)。

\begin{quote}{“因此,应当注意〈斯密用这样一种非常天真的形式从一种主张转到了另一种截然相反的主张〉,地租是以与工资、利润不同的方式加入商品价格的构成。工资和利润的高低,是商品价格[625]高低的原因;地租的高低,是这一价格的结果。由于使商品进入市场所必须支付的工资、利润有高有低,商品的价格也就有高有低。不过商品有时提供高地租,有时提供低地租,有时完全不提供地租,是因为商品价格有高有低,有时大大超过足够支付这些工资和利润的价格,有时略为超过,有时完全不超过。”(第1卷第303—304页)}\end{quote}

我们首先来看结尾这句话。原来,只支付工资和利润的足够价格,费用价格,是排除地租的。如果产品的卖价大大超过足够价格,它就支付高地租。如果产品的卖价只是略为超过足够价格,它就支付低地租。如果产品正好按照足够价格出卖,它就不支付任何地租。如果产品的实际价格和支付利润、工资的足够价格相一致,它就不支付任何地租。地租始终是超过足够价格的余额。足够价格就其性质来说是排除地租的。这是李嘉图的理论。李嘉图从亚·斯密那里接受了足够价格,费用价格的观念;他避免了亚·斯密把足够价格同自然价格区别开来的那种前后矛盾的毛病,而是前后一贯地贯彻了足够价格的观念。斯密在犯了所有这些前后矛盾的毛病之后,还继续表现出前后矛盾,以致要求某些土地产品有一个超过足够价格的价格。但这种前后矛盾本身又是更正确的“observation”(“考察”)\authornote{《observation》一词既有“考察”的意思,又有“注意”的意思;马克思在这里暗指前面引用的斯密那一段话的开头“因此,应当注意”(《Ilfautdoncobserver》)。——编者注}的结果。

但是这一段话的开头的确天真得令人吃惊。在第七章,斯密先把价值分解为地租、利润和工资这一点颠倒为价值由地租、利润和工资的自然价格构成,然后说明,地租、利润和工资以同样的方式加入自然价格的构成。现在他说,地租以与利润、工资不同的方式加入“商品价格的构成”。但是地租以什么样的不同方式加入价格的构成呢?这就是以地租完全不加入价格的构成的方式。在这里我们第一次得到了对“足够价格”的真正解释。商品价格所以有贵贱高低,是因为工资和利润——它们的自然率——有高有低。如果这些高的或低的利润和工资得不到支付,商品就不能进入市场,就不能生产出来。而利润和工资构成商品的生产价格即费用价格;也就是说,它们实际上是商品的价值或价格的构成要素。相反,地租不加入费用价格,不加入生产价格。地租不是商品交换价值的构成要素。只有在商品的普通价格超过足够价格时,地租才得到支付。利润和工资,作为价格的构成要素,是价格的原因;相反,地租只是价格的结果,只是价格的后果。所以地租不象利润、工资那样作为要素加入价格的构成。用斯密的语言来说,这就是地租以与利润、工资不同的方式加入价格的构成。斯密似乎完全没有感觉到他推翻了他关于“自然价格”的全部学说。要知道,他所说的“自然价格”是什么呢?是市场价格所趋向的中心,是“足够价格”,——如果产品要较长时期地进入市场,进行生产,它是不能低于这个价格出卖的。

这样,地租现在是超过“自然价格”的余额,而以前是“自然价格”的构成要素;现在,它被说成是价格的后果,以前,它却被说成是价格的原因。

相反,斯密以下说法倒是没有什么矛盾的:对于某些土地产品来说,市场的情况始终使它们的普通价格必定超过它们的足够价格,换句话说,土地所有权有权力把价格抬到对资本家来说是足够的(如果他没有遇到对抗作用)水平以上。

[626]斯密就这样在第十一章把他在第五、六和七章所说的全部推翻之后,又心安理得地继续说,他现在言归本题,着手考察:(1)始终提供地租的土地产品;(2)有时提供地租有时又不提供地租的土地产品;最后,(3)在不同的社会发展时期,这两种产品相互之间的相对价值以及它们和工业品之间的相对价值所发生的变化。

\tsectionnonum{[(2)斯密关于对农产品的需求的特性的论点。斯密地租理论中的重农主义因素]}

第一节:论始终提供地租的土地产品。

斯密从人口论开始。食物据说始终创造对自己的需求。如果食物的数量增加了,食物消费者的人数也就增加。因此,这些商品的供给创造对它们的需求。

\begin{quote}{“因为象其他一切动物一样,人的繁殖自然同其生存资料相适应,所以对食物总是有或大或小的需求。食物总是能够购买或者说支配或多或少的劳动量,并且总是有人愿意为获得食物去做某种事情。”(第1卷第305页)(第1篇第11章)“但是{为什么?}土地几乎在任何情况下都能生产出较大量的食物,也就是说,除了以当时最优厚的条件维持使食物进入市场所必需的全部劳动外还有剩余。这个余额又始终超过那个足够补偿推动这种劳动的资本并提供利润的数量。所以这里始终有一些余额用来向土地所有者支付地租。”(同上,第305—306页)}\end{quote}

这完全是重农学派的口吻,而且既没有证明,也没有解释,为什么这种特殊商品的“价格”能提供超过“足够价格”的余额即地租。

斯密立即举出牧场和荒地作例子。接着是关于级差地租的话:

\begin{quote}{“不管土地的产品如何,地租随着土地的肥力而变动;不管土地的肥力怎样,地租随着土地的位置而变动。”(第1卷第306页)}\end{quote}

这里我们看到,地租和利润纯粹是产品中扣除以实物形式养活工人的那部分以后的余额。(这真正是重农学派的见解,这种见解实际上以下述情况为依据:在农业占统治地位的条件下,人几乎只靠农产品生活,而工业本身,即工场手工业,只作为农村的副业劳动,用来加工当地的自然产品。)

\begin{quote}{“这后一种产品\authornote{离市场远的偏僻地区的产品。——编者注},必须保证维持较大量的劳动,而作为租地农场主的利润和土地所有者的地租来源的余额就势必相应减少。”(第1卷第307页)}\end{quote}

因此,据说种植小麦提供的利润必定比牧场多:

\begin{quote}{“中等肥力的麦田,比同样面积的最好牧场,给人生产多得多的食物。”}\end{quote}

(可见,这里谈的不是价格,而是人的实物形式的食物的绝对量。)

\begin{quote}{“虽然耕种麦田要求较大量的劳动,但是补偿种子和维持全部劳动后剩下的余额还是大得多。”}\end{quote}

(虽然小麦耗费较大量的劳动,但是麦田所提供的食物在支付劳动报酬后剩下的余额,却超过畜牧场所提供的余额。这个余额所以有较大的价值,并不是因为小麦耗费了较大量的劳动,而是据说因为小麦的余额包含较多的食物。)

\begin{quote}{“因此,如果我们假定,一磅肉的价值从来不比一磅面包大,那末,这个较大的〈小麦〉余额〈因为同样的土地面积提供的小麦磅数比肉的磅数多〉就到处都代表一个较大的价值{因为已经假定,一磅面包(按价值)等于一磅肉,而在养活工人后,同样的土地面积剩下的面包的磅数大于肉的磅数},并给租地农场主的利润和土地所有者的地租构成一笔更大的基金。”(第1卷第308—309页)}\end{quote}

斯密用足够价格代替自然价格,并认定地租是超过足够价格的余额,随后他就忘记了这里一般谈的是价格,而从农业提供的食物数量和土地耕种者消费的食物数量的对比中得出了地租。

如果撇开这种重农学派的说明方法不谈,实际上斯密是假定:充当主要食物的农产品的价格,除了提供利润外,还提供地租。他从这个基础出发继续议论。随着耕作技术的发展,天然牧场的面积变得不能满足畜牧业的需要,不能满足对家畜肉类的需求。为了这个目的不得不利用耕地。[627]因此,肉的价格必须提高到不仅能够支付畜牧业所使用的劳动的报酬,而且能够支付

\begin{quote}{“这块土地用作耕地时能给租地农场主和土地所有者提供的利润和地租。在完全没有开垦的荒地上饲养的牲畜,和在耕种得很好的土地上饲养的牲畜,在同一市场上,就会按其重量和质量,以同样的价格出卖。这些荒地的所有者就利用这种情况,按照牲畜价格相应地提高自己土地的地租”。}\end{quote}

(这里,斯密正确地从市场价值超过个别价值的余额中得出了级差地租。但是在这种情况下市场价值提高,并不是因为从较好的土地推移到较坏的土地,而是因为从比较不肥沃的土地推移到比较肥沃的土地。)

\begin{quote}{“这样,随着土地耕作的进步,天然牧场的地租和利润,在一定程度上决定于已耕地的地租和利润,这种已耕地的地租和利润,又决定于麦田的地租和利润。”(第1卷第310—311页)“在没有……地方性优越条件的地方,小麦或充当人们主要植物性食物的任何其他产品所提供的地租和利润,自然要决定适宜于种植这种作物而现在却用作牧场的土地的地租和利润。利用人工牧场,种植芜菁、胡萝卜、大白菜等等,或者采用其他种种手段,使一定面积的土地饲养的牲畜多于天然牧场饲养的牲畜,这一切看来必定会促使农业发达的国家中自然比面包价格高的肉类价格有所降低。看来也已经产生了这样的结果”,等等。(第315页)}\end{quote}

斯密这样说明了畜牧业地租和农业地租的相互关系之后,继续写道:

\begin{quote}{“在一切大国中,大部分耕地都用来生产人的食物或牲畜的饲料。这些土地的地租和利润决定其他一切耕地的地租和利润。如果某种产品提供的地租和利润较少,种植这种产品的土地,就会立即用来种植小麦或改为牧场,如果某种产品提供的地租和利润较多,有一部分种植小麦或用作牧场的土地,就会立即用来种植这种产品。”(第1卷第318页)}\end{quote}

接着,斯密说到葡萄种植业、果园业、蔬菜业等等:

\begin{quote}{“为了使土地适于栽培这些作物,必须投下一笔较大的原始费用,或者逐年投下较大的耕作费用,虽然这些生产部门的地租和利润,往往大大超过从小麦或牧草得到的地租和利润,但是如果这种地租和利润只够弥补异常高昂的支出,它们实际上仍然是由这两种普通农产品的地租和利润决定的。”(第1卷第323—324页)}\end{quote}

在这以后,斯密又谈到殖民地的甘蔗和烟草的种植[然后说道:]

\begin{quote}{“就这样,生产人们食物的已耕地的地租,决定其他大部分耕地的地租。”(第1卷第331页)“在欧洲,小麦是直接充当人们食物的主要土地产品。所以,除一些特殊情况外,麦田的地租,在欧洲决定其他所有耕地的地租。”(第1卷第331—332页)}\end{quote}

然后,斯密又回到重农主义理论,并用了他自己的说法:食物本身为自己创造消费者。如果不种小麦而种植其他在最普通的土地上用同样的耕作方法能提供多得多的食物的作物,

\begin{quote}{“那末,土地所有者的地租,或者说,在支付劳动报酬并补偿租地农场主的资本及其普通利润后留给他的食物余额,也必然会多得多。不论这个国家维持劳动的普通开支如何,这个较大的食物余额总能够维持较大量的劳动,从而,使土地所有者能够购买,或者说,支配较大量的劳动”。(第1卷第332页)}\end{quote}

斯密举了大米作例子。

\begin{quote}{“在加罗林,也象在其他的英国殖民地一样,种植场主通常既是租地农场主,同时又是土地所有者,因此地租和利润就混在一起了。”(第1卷第333页)}\end{quote}

[628]但是稻田

\begin{quote}{“不适宜于种小麦,作牧场,或种葡萄,也不适宜于种其他任何对人有用的植物,而所有适宜于种这些作物的土地也不适宜于种稻子。所以,即使以大米为主要食物的国家,稻田的地租,也不能决定其他不能用来种稻子的耕地的地租”。(第1卷第334页)}\end{quote}

第二个例子是马铃薯(李嘉图对斯密这个观点的批判在前面引用过\authornote{见本册第386页。——编者注})。如果主要食物不是小麦,而是马铃薯,

\begin{quote}{“那末,同样面积的耕地就能养活多得多的人;因为工人通常都吃马铃薯,所以在补偿资本和养活所有种植马铃薯的工人外,就会有多得多的余额。而这个余额的更大部分也就会归土地所有者。人口会增加,地租将大大高于现在的水平”。(第1卷第335页)}\end{quote}

接着他对小麦面包、燕麦面包以及马铃薯作了一些进一步的说明,就结束了第十一章第一节。

我们看到,论述始终提供地租的土地产品的第一节可以概括如下:在假定主要植物性产品的地租已经存在的情况下,说明这种地租怎样调节畜牧业、葡萄种植业、果园业等等的地租。这里根本没有谈地租本身的性质,而只是泛泛地谈到——又是假定地租已经存在——土地的肥力和位置决定地租的高低。但是这里涉及的只是地租的差别,地租量的差别。然而,这里所考察的产品为什么始终提供地租呢?为什么它的普通价格始终超过它的足够价格呢?在这里斯密撇开价格,又陷入了重农主义。但是他到处都贯穿着这样一种思想:对农产品的需求始终这样大,是因为这种产品本身创造需求者,创造它自己的消费者。即使这样假定,也还是没有说明白,为什么需求一定超过供给,从而使价格高于足够价格。不过在这里又不知不觉地出现关于自然价格的影子,这个自然价格既包括利润和工资,也包括地租,而且,在供求相适应时就会得到支付:

\begin{quote}{“如果进入市场的数量恰好足够满足实际需求,那末,市场价格当然就会和自然价格完全一致……”(第1卷第114页)}\end{quote}

但是很典型的是,斯密在第十一章第一节没有一处谈到这个观点。而他在第十一章一开头恰恰是说,地租不作为价格的构成部分加入价格。矛盾太明显了。

\tsectionnonum{[(3)斯密关于各种土地产品的供求关系的论述。斯密对地租理论的结论]}

第二节:论有时提供地租有时又不提供地租的土地产品。

在这一节里才真正研究了地租的一般性质。

\begin{quote}{“人的食物看来是始终而且必然给土地所有者提供某些地租的唯一土地产品〈为什么是“始终”而且“必然”,却没有说明〉。其他各种产品,则根据不同的情况,有时能提供地租,有时又不能提供地租。”(第1卷第337页)“除了食物之外,衣服和住宅就是人类的两大需要。”(第1卷第338页)土地“在原始的未开垦的状态下”所能提供的衣服和住宅的材料,超过“它所能养活的”人数。由于“这些材料”同土地所能养活的人数相比,即同人口相比,“绰绰有余”,这些材料的“价格”就很低,或者根本没有“价格”。这些“材料”很大部分没有被利用或毫无用处,“而被利用的材料的价格,也被看成仅仅是为了使这些材料适于使用而必须花费的劳动和费用的等价物”。但是这个价格“不给土地所有者提供任何地租”。而土地在已开垦的状态下“所能养活的”人数即人口,超过土地提供的这些材料的数量,至少超过“人们希望得到和愿意支付的那些材料的数量”。于是这些材料就相对地显得“缺乏”,“而这就必然要提高它们的价值”。“对它们的需求量往往大于所能得到的数量。”那时人们对这些材料支付的价格,就会高于“使它们进入市场所必需的费用;因此,它们的价格始终能够给土地所有者提供一些地租”。(第1卷第338—339页)}\end{quote}

[629]可见,这里把地租解释成由于需求超过了供给——按照足够价格所能得到的供给。

\begin{quote}{最早的衣服材料是“大野兽”的毛皮。那些主要食用动物肉类的狩猎民族和游牧民族,“每一个人在获得食物的同时,也获得他穿不完的衣服材料”。没有对外贸易,其中大部分就被当作无用的东西丢掉。对外贸易提出对这些多余材料的需求,把它们的价格提到“高于把它们运到市场的费用。因此,这种价格也就能够给土地所有者提供一些地租……英格兰的羊毛由于在弗兰德找到了销路,使出产羊毛的土地的地租有了某些提高”。(第1卷第339—340页)}\end{quote}

这里是对外贸易提高了农业副产品的价格,以致生产这种产品的土地能够提供一些地租。

\begin{quote}{“建筑材料往往不能象衣服材料那样运到远地去,因而不那么容易成为对外贸易的对象。如果一国出产的建筑材料过多,即使在现代世界贸易的情况下,它们对土地所有者来说也往往没有任何价值。”例如采石场,在伦敦附近能够提供地租,但在苏格兰和威尔士的许多地方,却不能提供地租。建筑用的木材也是这样。“在人口稠密的文明国家”,木材可以提供地租,但在“北美的许多地区”,木材则就地烂掉。只要能把它弄走,土地所有者就很高兴了。“建筑材料既然这样充裕,所以被使用的那一部分材料的价值,就只不过相当于为了使这些材料适于使用而必须花费的劳动和费用。它不给土地所有者提供任何地租;只要有人愿意要,土地所有者通常都容许他们去采伐。但是当比较富裕的国家对这种材料有需求时,土地所有者有时也能从中得到地租。”(第1卷第340—341页)一国有多少人口,不是看“这个国家的产品能够保证多少人的衣服和住宅,而是看这个国家的产品能够保证多少人的食物。只要食物不缺,必要的衣服和住宅是不难找到的。但是常常衣服住宅有了,食物却依然很难找到。甚至在英吉利王国的一些地方,一个人只要用一天的劳动,就可以把一座当地所谓的房子建造起来”。在未开化的野蛮氏族中间,为了得到必需的衣服和住房,只要用全年劳动的百分之一就够了,其余百分之九十九,常常必须用来获得他们所需要的食物。“但是如果土地经过耕种和改良,一家人的劳动能为两家人提供食物,那末,社会半数人的劳动就足够为整个社会提供食物。”那时,另一半人就能满足人们的其他需要和嗜好。这些需要和嗜好的主要对象是衣服、住宅、家具,以及所谓奢侈品。食物的需要是有限的。上述这些需要是无限的。有多余食物的人,“总是愿意拿这部分多余的食物去交换”。“穷人为了获得食物”,就尽力满足富人的这些“嗜好”,并且还在这方面互相竞争。工人人数,随着食物数量的增加而增加,也就是说,随着农业的发展而增加。他们的“工作”允许实行“极细的分工”;所以他们加工的原料的数量就会比他们的人数增加快得多。“因此,对于任何一种材料,凡是人类的发明能把它用来改善或装饰住宅、衣服、车马、家具的,都产生了需求;对于地下蕴藏的化石和矿石,对于贵金属和宝石,也都产生了需求。这样一来,不仅食物是地租的原始源泉,而且,后来提供地租的其他任何土地产品,它的价值中的这个剩余部分,也都是土地的耕种和改良使生产食物的劳动生产力提高的结果。”(第1卷第342—345页)}\end{quote}

斯密这里所说的,也就是重农主义的真实的自然基础,即一切剩余价值(包括地租)的创造,都以农业的相对生产率为基础。剩余价值的最初的实在形式,就是农产品(食物)的剩余;剩余劳动的最初的实在形式,表现为一个人的劳动足以生产两个人的食物。除此以外,这一点对于分析以资本主义生产为前提的这个剩余价值的特殊形式地租,没有任何关系。

斯密继续说道:

\begin{quote}{“后来提供地租的其他土地产品〈食物除外〉并不是始终提供地租的。即使在土地耕种得最好的国家里,对于这些土地产品的需求,也不是始终大到足够使它们的价格除了支付产品生产和运到市场所花费的劳动以及补偿所用资本并提供普通利润以外还有一个余额。[630]需求是否那样大,取决于各种情况。”(第1卷第345页)}\end{quote}

这里又是说:地租的产生,是由于对土地产品的需求超过这些产品按足够价格——不包括地租,而只包括工资和利润——的供给。这不正是说,[在地租不存在的地方]土地产品按足够价格的供给很多,以致土地所有权不能对资本或劳动的平均化进行任何抵抗吗?这不也就是说,土地所有权即使在法律上存在,在这里实际上是不存在的,或者实际上不能作为土地所有权起作用吗?斯密的错误在于,不理解土地所有权按照超过足够价格的价格出卖产品,就是按照产品的价值出卖。斯密比李嘉图好的地方是他懂得,土地所有权能否显示自己的经济作用,取决于各种情况。因此,对他分析的这一部分,应当一步一步地跟着。他从煤矿开始,然后说到木材,然后又回到煤矿等等。因此我们先从他谈木材的地方开始:

\begin{quote}{木材价格,随着农业的状况而变动,变动的原因,同牲畜价格变动的原因一样。当农业还处于幼稚状态时,到处都是森林,这对土地所有者来说是一种障碍,谁愿意采伐,土地所有者是乐意让他采伐的。随着农业的进步,森林逐渐消失,一部分是由于耕地扩大,一部分是由于啃食树根和树苗的牲畜增加。“这些牲畜的头数,虽然不象完全是人类劳动产物的谷物数量增加得那样快,但是人的照料和保护,促进了牲畜的繁殖。”于是,森林逐渐稀少,它的价格也就提高。因此,森林能够提供很高的地租,以致耕地(或适于耕种的土地)也用来植树。大不列颠的情况就是这样。森林的地租决不能长久地超过耕地或牧场的地租。但是它可能达到同样的水平。(第1卷第347—349页)}\end{quote}

因此,森林的地租,就其性质来说,实际上和牧场的地租是一样的。它也属于这个范畴,虽然木材不能当作食物。经济范畴,不决定于产品的使用价值;这里它决定于这块土地能否变为耕地,或者相反。

煤矿。矿的富饶或贫瘠,一般说来,正如斯密正确指出的,取决于以同量劳动从不同矿开采出的矿产量是多还是少。矿的贫瘠,能把有利的位置抵销,以致这类矿完全不能开采。另一方面,位置不利也会把矿的富饶抵销,以致这些矿虽然天然富饶,却不宜于开采。特别是在没有好道路又没有航运的地方,往往是这样。(第1卷第346—347页)

有一些矿的产品仅够补偿足够价格。所以,它们能给企业主提供利润,但不能提供任何地租。因此,土地所有者不得不自己开采。这样,他可以获得“他所用资本的普通利润”。这一类煤矿在苏格兰很多。用其他方式来开采是不可能的:

\begin{quote}{“土地所有者不允许其他任何人不支付地租就去开采这些煤矿,而任何人又无法支付地租。”(第1卷第346页)}\end{quote}

斯密在这里正确地说明了,在土地已被占有的地方在什么情况下不支付地租。凡是一个人兼有土地所有者和企业主两种身分的地方往往是这样。以前斯密已经说过,殖民地的情况就是这样。租地农场主因为无法支付地租,也就不能在这里耕种土地。但是土地所有者耕种土地能得到利润,虽然土地不能给他提供任何地租。例如,美洲西部殖民地的情况就是这样,因为在这里始终有可能占有新地。土地本身不是一个阻碍的因素,自己耕种自己土地的土地所有者之间的竞争,在这里实际上是劳动者之间或资本家之间的竞争。至于煤矿或一般矿山,在前面假定的情况下,则不是这样。市场价值是由那些正好按照这个价值提供商品的矿决定的,它给比较不富饶或位置比较不利的矿提供较少的地租,或者完全不提供地租,而只补偿费用价格。在这里,这些矿只能由这样的人去开采,对他们说来,土地所有权的那种阻碍自由支配土地的作用是不存在的,因为他们一身兼有土地所有者和资本家两种身分;这种矿只有在土地所有权实际上不再作为与资本对立的独立因素的情况下才能开采。这种情况和殖民地的情况不同,在那里,土地所有者不能禁止任何人开垦新地。在这里他却能够这样做。他只允许他自己开矿。这并不能使他得到地租,却能使他排挤其他人,而从他自己投入矿山的资本中得到利润。

关于斯密所说的地租由最富饶的煤矿调节这一点,我在前面谈到李嘉图及其与斯密论战时\authornote{见本册第383—386页。——编者注}已经考察过了。这里只须指出下面一段话:

\begin{quote}{“煤炭正象其他一切商品一样,在一个较长时间内可以出卖的最低价格〈前面斯密说的是足够价格〉,就是仅仅足够补偿用于商品生产和运到市场的资本并提供普通利润的价格。”(第1卷第350页)}\end{quote}

我们看到,足够价格代替了自然价格。李嘉图把它们等同起来,是理所当然的。

[631]斯密断言:

\begin{quote}{煤矿的地租比农产品的地租少得多:农业中的地租通常达到总产品的1/3,对于煤矿来说,能占1/5就是很高的地租了,普通地租占1/10。金属矿受位置的影响较小,因为它们的产品比较容易运输,比较容易进入世界市场。所以它们的价值更多地取决于富饶程度,而不取决于位置,而煤矿的情形正好相反。彼此相隔最远的金属矿的产品可以互相竞争。“因此,世界上最富饶的矿山出产的普通金属的价格,尤其是贵金属的价格,必然会影响世界上其他各个矿山的同类金属的价格。”(第1卷第351—352页)“这样看来,因为每一个矿山的每一种金属价格,都在一定程度上由世界上当时开采的最富饶的矿山出产的该种金属的价格调节,所以绝大部分矿山所产的金属的价格,几乎都不超过补偿开采费用所需的价格,而且,很少能够向土地所有者提供高额地租。因此,对大多数矿山来说,地租只占金属价格的很小一部分,在贵金属价格中,它占的部分还要小得多。劳动和利润,在这两类金属的价格中都占大部分。”(第1卷第353—354页)}\end{quote}

斯密在这里正确地说明了C表的情况\authornote{见第302—303页。——编者注}。

谈到贵金属时,斯密又重复说明了他在谈到地租时用来代替自然价格的足够价格。在谈非农业生产的地方,他没有必要这样做,因为在这里,按照他最初的说明,足够价格和自然价格是一致的;这就是支付预付资本和平均利润的那个价格。

\begin{quote}{“贵金属在一个较长时间内可以出卖的最低价格……是由决定其他所有商品的最低普通价格的那些原则调节的。这种最低价格,是由贵金属从矿山进入市场通常所需要的资本决定的,也就是由这个劳动过程中通常所消费的食物、衣服、住宅决定的。这个价格必须至少足够补偿这笔资本并提供普通利润。”(第1卷第359页)}\end{quote}

说到宝石,斯密指出:

\begin{quote}{“对宝石的需求,完全是由它们的美丽引起的。它们只用于装饰。它们的美丽,又由于宝石稀少,或由于从矿山开采宝石困难和费用大,而显得更加珍贵。因此,在大多数情况下,工资和利润几乎占了宝石高昂价格的全部。地租在宝石价格中只占极小的份额,甚至常常不占任何份额。只有最富饶的矿山才能提供大一点的地租。”(第1卷第361页)}\end{quote}

这里只可能产生级差地租:

\begin{quote}{“因为全世界的贵金属和宝石的价格,是由最富饶的矿山的产品价格调节的,所以任何一个矿山能向土地所有者提供的地租,不是和该矿山的绝对富饶程度相适应,而是和它的所谓相对富饶程度,也就是它比其他同类矿山优越的程度相适应。如果发现了新矿山,它比波托西矿山优越的程度跟波托西矿山比欧洲矿山优越的程度一样,那末银的价值就会因此大大降低,以致连波托西矿山也不值得去开采了。”(第1卷第362页)}\end{quote}

比较不富饶的贵金属矿和宝石矿的产品,不提供任何地租,因为决定市场价值的始终是最富饶的矿山,并且不断有更富饶的新矿被开发,不断按上升序列运动。因而,比较不富饶的矿山的产品是低于它们的价值而仅仅按照它们的费用价格出卖的。

\begin{quote}{“如果一种产品的价值主要由它的稀少决定,那末产品的充裕必然使产品价值降低。”(第1卷第363页)}\end{quote}

在这以后,斯密又得出了多少是错误的结论。

\begin{quote}{“地面上的地产却是另外一种情况。它们的产品的价值和它们所提供的地租的价值,都是同它们的绝对肥力而不是同它们的相对肥力成比例。生产一定量食物、衣服材料和住房材料的土地,总能给一定的人数提供吃穿住;而且,不管土地所有者在这一产品中占有多大份额〈问题恰恰在于土地所有者在产品中能否占有份额和占有多大份额〉,这个份额[632]总是使他能相应地支配这些人的劳动和这种劳动所能给他提供的商品。”(第1卷第363—364页)“最贫瘠的土地的价值,并不因为邻近有最肥沃的土地而减少。相反,它的价值通常还因此而提高。肥沃土地养活的大量人口,为贫瘠土地的许多产品创造市场;这些产品决不能在靠贫瘠土地本身的产品养活的人们中间找到这种市场。”}\end{quote}

(但这只适用于这样的场合,即贫瘠土地所生产的和邻近肥沃土地所生产的不是同一种产品,贫瘠土地的产品不同比较肥沃的土地的产品竞争。就这样的场合来说,斯密是对的,这对理解各种土地产品的地租总额怎么会由于生产食物的土地肥沃而增加,确实是有重要意义的。)

\begin{quote}{“凡是能够使生产食物的土地的肥力提高的措施,不仅使经过改良的土地的价值增加{可以使这个价值减少,甚至化为乌有},而且还使其他许多土地的价值也同样增加,因为创造了对它们产品的新的需求〈或者,更确切地说,创造了对新产品的需求〉。”(第1卷第364页)}\end{quote}

斯密的上述一切仍然没有解释他假定对于生产食物的土地来说存在的绝对地租。斯密合理地指出,绝对地租对于其他土地例如矿山来说,也可能不存在,因为后者在数量上相对地说总是无限的(同需求相比),以致土地所有权在这里不可能对资本进行任何抵抗;土地所有权即使在法律上存在,在经济上也是不存在的。

(见第641页关于房租)\endnote{这一行是在马克思写完了论述斯密的房租观点的一段话(马克思手稿第641页)以后加进去的。——第415页。}[632]

\centerbox{※     ※     ※}

[641](见第632页)关于房租,亚·斯密说:

\begin{quote}{“全部房租中超过足够提供合理利润〈建造这所房屋的房主的利润〉的部分,自然归入地皮租;当土地所有者和房主是两个不同的人时,这一部分在大多数情况下全部付给前者。在远离大城市的乡村中的房屋,可以随意选择空地,只提供很少一点地皮租,或者说,不超过房屋所占土地用于农业时所能提供的地租。”(第5篇第2章)}\end{quote}

在房屋地皮租上,位置是级差地租的决定性因素,正象在农业地租上,土地肥力(和位置)是级差地租的决定性因素一样。

亚·斯密同重农学派一样,特别偏重农业和土地所有者,并持有重农主义观点,认为农业和土地所有者是最适当的课税对象。他说:

\begin{quote}{“地皮租和普通地租,都是土地所有者往往无须亲自操劳费心而唾手可得的一种收入。这种收入如有一部分拿去弥补国家开支,任何一种生产活动也不会因此受到损害。土地和社会劳动的年产品,即大部分居民的实际财富和收入,在实行这种税收以后,不会有任何变化。因此,地皮租和普通地租,大概是最宜于课以特别税的一种收入。”(第5篇第2章)\endnote{马克思在这里引用的斯密的两段话不是根据加尔涅的法译本(马克思在本册引用斯密的话都是根据这个译本),而是根据李嘉图《政治经济学和赋税原理》一书的英文本(第3版第14章)。——第415页。}}\end{quote}

与此相反,李嘉图(第230页)\endnote{马克思指李嘉图的《政治经济学和赋税原理》1821年伦敦第3版第230页。——第415页。}却提出了一种极其庸俗的反对意见。[641]

\tsectionnonum{[(4)斯密对于土地产品价格变动的分析]}

[632]第三节:论始终提供地租的产品的价值和有时提供地租有时又不提供地租的产品价值之间的比例的变动(第2卷第1篇第11章)。

\begin{quote}{“在土地自然肥沃但绝大部分完全没有耕种的国家,家畜、家禽、各种野生动物,耗费极少量的劳动就可得到,所以用它们也只能购买,或者说,支配极少量的劳动。”(第2卷第25页)}\end{quote}

斯密以多么奇特的方法把价值用劳动量来衡量同“劳动价格”,或者说,同某一商品所能支配的劳动量混淆起来,这从上面一段引文,特别是从下面一段引文可以看得很清楚。下面一段引文还表明,斯密竟然在有些地方把谷物看成价值尺度。

\begin{quote}{“在任何社会状态下,在任何社会文明发展阶段,谷物总是人类勤劳的产品。但是任何劳动部门的产品的平均量,总是多少准确地同平均消费相适应,即平均供给同平均需求相适应。此外,在不同的文明阶段,在同样的土地和同样的气候条件下,生产同量谷物,平均起来需要几乎同量的劳动,或者同样可以说,几乎同量劳动的价格。因为在耕作技术提高情况下劳动生产力的不断提高,或多或少会被作为农业主要工具的牲畜的价格的不断上涨所抵销。根据这一切,我们可以确信,在任何社会状态下,在任何文明阶段,同量谷物,和同量的其他任何土地原产品相比,都更恰当地成为同量劳动的代表或等价物。因此……在社会财富和文明的所有不同发展阶段,谷物同其他任何商品或其他任何一类商品比较起来,是更准确的价值尺度……此外,谷物或其他一般为人民喜爱的植物性食物,在每个文明国家,都是工人生存资料的主要部分……因此,劳动的货币价格取决于作为工人生存资料的谷物的平均货币价格的程度,远远超过取决于肉类或其他土地原产品的价格的程度。因此,金和银的实际价值,金和银所能购买或支配的实际劳动量,取决于它们所能购买或代表的谷物量的程度,远远超过取决于它们所能支配的肉类或其他土地原产品的数量的程度。”(第2卷第26—28页)}\end{quote}

在比较金和银的价值时,斯密又一次发挥了他的“足够价格”观点,并且[633]明确指出,足够价格不包括地租:

\begin{quote}{“我们说一种商品是贵还是贱,不仅要看它的普通价格是大是小,还要看这个普通价格超过使商品能在一个比较长的时间内进入市场的最低价格是多是少。这个最低价格,就是恰恰足够补偿商品进入市场所需资本并提供适中利润的价格。这个价格不给土地所有者提供什么东西;它的任何部分不由地租构成,它只分解为工资和利润。”(第2卷第81页)“金刚石和其他宝石的价格,和金的价格相比,大概更加接近于那个使它们能够进入市场的最低价格。”(第2卷第83页)}\end{quote}

按照斯密的说法,原产品有三类。(第2卷第89页)第一类产品的增加几乎不依赖或完全不依赖于人类劳动;第二类产品的数量能够根据需求而增加;第三类产品,其数量的增加,人类劳动“只能给以有限的或不经常的影响”。

第一类:鱼、罕见的鸟、各种野生动物、几乎所有的野鸟,特别是候鸟等等。随着财富和奢侈程度的增长,对于这类产品的需求则大大增加。

\begin{quote}{“因为这些商品的数量保持不变或几乎不变,而购买者间的竞争又日益扩大,所以它们的价格就可以涨到任何高度。”(第2卷第91页)第二类:“这包括在未耕地上天然成长的有用的植物和动物,它们十分丰富,以致只有很小的价值或全无价值,后来由于耕作的扩大,它们不得不让位于其他更加有利可图的产品。在长时期中,随着文明的不断进步,这类产品的数量不断减少,而同时对它们的需求却不断增加。这样,它们的实际价值,它们所能购买,或者说,支配的实际劳动量也越来越增加,最后将达到这样的高度,以致它们成为有利可图的产品,就象其他靠人的劳动在最肥沃的、耕种得最好的土地上获得的任何产品一样。如果这些产品的价值已经达到这样的高度,它也就不可能再提高了。否则人们马上就会用更多的土地和劳动来增加这些产品的数量。”(第2卷第94—95页)例如,家畜的情况就是这样。“在属于第二类原产品的各种商品中,家畜大概是随着文明的发展在价格上首先达到这种高度的商品。”(第2卷第96—97页)“如果说家畜最先达到这种价格{也就是使土地种植家畜饲料合算的价格},那末鹿肉大概就是最后达到这种价格的。尽管英国的鹿肉价格已经很高,但它还不够补偿鹿场的开支,这是有点养鹿经验的人都清楚的。”(第2卷第104页)“在每一个农场中,粮仓和牲口棚的残余食物可以用来饲养一定数量的家禽。因为家禽吃的东西,不利用也是浪费掉,所以饲养家禽只不过是废物利用;因为家禽几乎不花费租地农场主什么东西,所以他甚至能够以很低的价格出卖。”在供给充分时,家禽同家畜肉一样便宜。随着财富的增长,需求增大,家禽的价格就涨到牛肉或羊肉的价格以上,直到“专门耕种土地来饲养家禽变得有利可图”为止。法国的情况就是这样,等等。(第2卷第105—106页)猪和家禽一样,“最初饲养是为了废物利用”。猪吃的是糟粕。但是最后它的价格上涨到有必要专门耕种土地来饲养猪。(第2卷第108—109页)}\end{quote}

牛奶,牛奶场。(第2卷第110页及以下各页)(奶油、干酪;同上。)

按照斯密的意见,这些原产品价格的逐渐上涨,只是证明它们逐渐变成人类劳动产品,而在以前,它们几乎纯粹是自然产品。它们从自然产品变成劳动产品,只是耕作发展的结果,而耕作的发展,愈来愈缩小自然界的天然产品的范围。另一方面,在生产不大发达的条件下,上述产品很大部分都是低于自己的价值出卖的。它们一旦由副产品变成某一农业部门的独立产品,就立即按照自己的价值出卖(从而价格也上涨了)。

\begin{quote}{“显然,无论在哪个国家,如果靠人类劳动生产出来的任何土地产品的价格,没有高到足够补偿耕种土地和改良土地的费用,其土地是不可能得到充分的耕种和改良的。为了能够做到这点,每一单个产品的价格,第一,要足够支付好麦田的地租,因为其余大部分已耕地的地租正是由好麦田的地租决定的;第二,要足够支付租地农场主使用的劳动和费用,其标准不低于好麦田,换句话说,要足够补偿租地农场主所花费的资本并提供普通利润。每一单个产品价格的这种提高,显然应该[634]在种植这种产品的土地得到改良和耕种之前……现在,这些不同的原产品不仅比以前值较大量的银,而且值较大量的劳动和生存资料。因为要使这些产品进入市场必须花费较大量的劳动和生存资料,所以它们进入市场以后,就代表较大量的劳动和生存资料,或者说,值较大量的劳动和生存资料。”(第2卷第113—115页)}\end{quote}

在这里我们又看到,斯密只是在他把由可以买到的劳动量决定的价值跟由生产商品所必要的劳动量决定的价值混淆起来的时候,才使用前一种价值概念。

第三类:照斯密的说法,这一类包括这样一些原产品,

\begin{quote}{“对于这类产品数量的增加,人类劳动只能给以有限的或不经常的影响”。(第2卷第115页)}\end{quote}

毛和皮的数量受现有大小家畜头数的限制。但是这些最早的副产品,在家畜本身还没有广大市场的时候,就已经有了广大市场。家畜肉几乎总是限于国内市场。可是毛和生皮,甚至在文明初期,就已经多半有了国外市场。它们非常便于运输,并且是许多工业品的原料。因此,当本国工业还不需要它们时,工业比较发达的国家就已经可以充当它们的市场了。

\begin{quote}{“在耕作不发达因而人口稀少的国家,毛和皮的价格在整头动物价格中所占的比例,比在耕作较发达、人口较稠密因而对肉类有较大需求的国家,要大得多。”脂油的情形也是这样。随着工业的发展和人口的增长,家畜价格的提高对肉价的影响比对毛皮价格的影响大。因为随着一国工业和人口的增长,肉类市场不断扩大,而上述副产品的市场原先就已经超出国界了。但是随着本国工业的发展,毛皮等的价格也总会有某些提高。(第2卷第115—119页)鱼(第2卷第129—130页)。如果对鱼的需求增加,为了满足这一需求就要花更大量的劳动。“鱼通常要到较远的地方去捕,要用比较大的渔船和各种比较贵的捕鱼设备。”对鱼的需求,“如果不花费”比“过去使鱼上市所必需的”更多的“劳动,就不可能得到满足”。“因此,这种商品的实际价格,必然随着文明的发展而自然提高。”(第2卷第130页)}\end{quote}

可见,在这里,斯密是用生产商品所必要的劳动量来决定实际价格。

按照斯密的说法,随着文明的发展,植物性产品(小麦等)的实际价格必然下降:

\begin{quote}{“农业改良的推广和耕地的扩大,必然使各种动物性食物的价格同小麦价格相比有所提高,另一方面,我认为,它同样必然使各种植物性食物的价格有所降低。它使动物性食物的价格提高,是因为提供动物性食物的很大部分土地,改成适于生产小麦以后,现在必须向土地所有者和租地农场主提供麦田的地租和利润。它使植物性食物的价格降低,是因为它通过土地肥力的增加,使这种食物充裕起来。农业的改良,还会引进许多新的植物性食物品种,它们比小麦需要的土地少,而花费的劳动也不更多,所以,它们能以比小麦低得多的价格进入市场。如马铃薯、玉米就属于这一类……此外,在农业发展水平低的情况下,许多植物性食物,只限于在菜园中栽培,而且只使用锄;随着耕作技术的发展,这些植物性食物也开始在大田里种植,并且使用了犁。如芜菁、胡萝卜、大白菜等就属于这一类。”(第2卷第11章第145—146页)}\end{quote}

斯密看到,凡是在“原料的实际价格没有提高或提高得不多”(第2卷第149页)的地方,工业品的价格一般都降低了。

另一方面,斯密断言,劳动的实际价格即工资,随着生产的发展提高了。因此,他还认为,商品的价格不一定因为工资,或者说,劳动价格的提高而提高,虽然在他看来,工资也是“自然价格的构成部分”,甚至是“足够价格”的“构成部分”,或者换句话说,是“商品进入市场所需的最低价格”的“构成部分”。斯密怎样解释这一点呢?是因为利润降低了吗?不是(虽然他也认为,一般利润率会随着文明的发展而下降)。是因为地租降低了吗?也不是。他说:

\begin{quote}{“机器的改进,[635]技能的提高,劳动分工和劳动分配的更加合理(这一切是一个国家发展的必然结果),都使生产某种产品所需的劳动量大大减少;虽然由于社会繁荣,劳动的实际价格必然大大提高,但是生产每一物品所需的劳动量的大大减少,通常会把劳动价格所能出现的很大的提高抵销而有余。”(第2卷第148页)}\end{quote}

这样,商品价值降低,是因为生产商品所必要的劳动量减少,并且,尽管劳动的实际价格提高了,商品价值还是会降低。如果这里劳动的实际价格就是指它的价值,那末在商品价格因商品价值降低而降低时,利润必然会同时降低。如果劳动的实际价格是指工人得到的生活资料总额,那末,斯密的论点即使在利润提高的情况下也是正确的。

凡是斯密作出实际分析的地方,他都采用了正确的价值规定;这一点从这一章结尾他研究毛织品为什么在十六世纪[比十八世纪]贵的问题的地方也可以看到:

\begin{quote}{“那时,为了制造这些商品供应市场,要花费多得多的劳动量,因此商品上市以后,卖得或换得的价格必定是一个多得多的劳动量。”(第2卷第156页)}\end{quote}

这里的错误只在“价格”一词。

\tsectionnonum{[(5)斯密关于地租变动的观点和他对各社会阶级利益的评价]}

这一章的结束语。亚·斯密是以下面的评论来结束论地租这一章的:

\begin{quote}{“社会状况的任何改善,都有直接或间接提高实际地租的趋势。”“农业改良的推广和耕地的扩大可以直接提高实际地租。土地所有者得到的产品份额,必然随着这个产品数量的增加而增加。”(第2卷第157—158页)“原产品实际价格的提高,最初是农业改良的推广和耕地的扩大的结果,后来又成为农业改良的进一步推广和耕地进一步扩大的原因”。这些产品的实际价格例如家畜价格的提高,第一,会提高土地所有者所获得的份额的实际价值;第二,也会提高这个份额的相对量;因为“这种产品的实际价格提高以后,生产它所需的劳动并不比以前多。这样,产品中一个比过去小的份额,就足够补偿推动劳动的资本并提供普通利润。而产品中一个比过去大的份额就因此归土地所有者所得”。(第2卷第158—159页)}\end{quote}

李嘉图也完全用同样的方法来说明比较肥沃的土地的谷物价格上涨时地租份额的增大。但是这种涨价并不是由农业改良引起的,因此,李嘉图得出了和斯密相反的结论。

斯密随后还指出,工业劳动生产力的任何发展,都会给土地所有者带来好处:

\begin{quote}{“凡是降低后者\authornote{工业品。——编者注}实际价格的措施,都能提高前者\authornote{农产品。——编者注}的实际价格。”其次,随着社会实际财富的增加,人口也就增加,随着人口的增加,对农产品的需求也就增加,从而投在农业上的资本也增加,而“地租也就随着产品的增加而增加”。反之,凡是阻碍社会财富增长的相反情况,都会使地租下降,从而使土地所有者的实际财富减少。(第2卷第159—160页)}\end{quote}

斯密由此作出结论说,地主(土地所有者)的利益,始终同“整个社会的利益”一致。在斯密看来,工人的利益,也同整个社会的利益一致(第2卷第161—162页)。但是斯密毕竟诚实地指出了如下的区别:

\begin{quote}{“土地所有者阶级也许能够由于社会的繁荣而比他们〈工人〉得到更大的利益,但是没有一个阶级象工人阶级那样由于社会衰落而遭受那样大的苦难。”(第2卷第162页)}\end{quote}

相反,资本家(工业家和商人)的利益却同“整个社会的利益”不一致(第2卷第163页)。

\begin{quote}{“在任何一个商业或工业部门投资的实业家的利益,总是在某些方面和社会利益不同,有时甚至相反。”(第2卷第164—165页)“……[这是]这样一些人的阶级,这些人的利益[636]始终不会和社会的利益完全一致,通常他们的利益在于欺骗社会,甚至压迫社会,而他们因此也常常既欺骗社会又压迫社会。”(第2卷第165页)\endnote{手稿中接着有几段话,是分析李嘉图关于自己对地租的理解的论述的。这几段和上文用一条线隔开,它们是对考察李嘉图地租理论各章的补充;按其内容属于第十三章,所以本版放在第十三章(见第357—358页)。手稿中这几段话之后,有一个对李嘉图费用价格理论的分析的补充,放在圆括号内,马克思所作的分析在第十章,所以这个补充本版也移至第十章(见第239—240页)。——第422页。}[636]}\end{quote}

\tchapternonum{[第十五章]李嘉图的剩余价值理论}

\tsectionnonum{[A.李嘉图关于剩余价值的观点与他对利润和地租的见解的联系]}

\tsubsectionnonum{[(1)李嘉图把剩余价值规律同利润规律混淆起来]}

[636]李嘉图在任何地方都没有离开剩余价值的特殊形式——利润(利息)和地租——来单独考察剩余价值。因此,他对具有如此重要意义的资本有机构成的论述,只限于说明从亚·斯密(特别是从重农学派)那里传下来的,由流通过程产生的资本有机构成的差别(固定资本和流动资本);而生产过程本身内部的资本有机构成的差别,李嘉图在任何地方都没有涉及,或者根本就不知道。就是由于这个缘故,他把价值和费用价格混淆起来了,提出了错误的地租理论,得出了关于利润率提高和降低原因的错误规律等等。

只有在预付资本和直接花费在工资上的资本是等同的情况下,利润和剩余价值才是等同的。(这里不必考虑地租,因为剩余价值最初完全由资本家所占有,不管他以后要把其中多大部分分给他的同伙。李嘉图自己也认为地租是从利润中分离、分割出来的部分。)而李嘉图在论述利润和工资时,也就把不是花费在工资上的资本的不变部分撇开不谈。他是这样考察问题的:似乎全部资本都直接花费在工资上了。因此,就这一点说,他考察的是剩余价值,而不是利润,因而才可以说他有剩余价值理论。但另一方面,他认为他谈的是利润本身,的确他的著作中到处都可以看到从利润的前提出发,而不是从剩余价值的前提出发的观点。在李嘉图正确叙述剩余价值规律的地方,由于他把剩余价值规律直接说成是利润规律,他就歪曲了剩余价值规律。另一方面,他又想不经过中介环节而直接把利润规律当作剩余价值规律来表述。

因此,当我们谈李嘉图的剩余价值理论时,我们谈的就是他的利润理论,因为他把利润和剩余价值混淆起来了,也就是说,他只是从对可变资本即花费在工资上的那部分资本的关系来考察利润。至于李嘉图谈到同剩余价值有区别的利润的地方,我们留到后面再分析。

剩余价值只能从对可变资本即直接花费在工资上的资本的关系来考察,——而没有对剩余价值的认识,就不可能有任何利润理论,——这是如此符合事情的本质,以致李嘉图把全部资本看作可变资本,而把不变资本撇开不谈,虽然他有时也以预付资本的形式提到不变资本。

[637]李嘉图谈到(第二十六章《论总收入和纯收入》)

\begin{quote}{“利润同资本成比例,而不是同所使用的劳动量成比例的工商业部门”。(李嘉图《政治经济学和赋税原理》第418页)}\end{quote}

李嘉图的全部平均利润学说(他的地租理论是以此为基础的),除了归结为确认利润“同资本成比例,而不是同所使用的劳动量成比例”,还能是什么呢?如果利润“同所使用的劳动量成比例”,那末相等的资本就会提供极不相等的利润,因为这些资本的利润等于它们本部门生产出来的剩余价值,而剩余价值不取决于全部资本的量,而取决于可变资本的量,或者说,取决于“所使用的劳动量”。因此,怎么能说,利润同所投资本的量成比例,而不同所使用的劳动量成比例,仅仅是某种特殊投资部门即特殊生产部门所特有的例外情况呢?如果剩余价值率既定,对一定资本来说,剩余价值量就必然总是取决于所使用的劳动量,而不取决于资本的绝对量。另一方面,如果平均利润率既定,利润量就必然总是取决于所使用的资本的量,而不取决于所使用的劳动量。

李嘉图明确地谈到这样一些部门,如

\begin{quote}{“海运业、同遥远的国家进行的对外贸易,以及需要昂贵机器装备的部门”。(第418页)}\end{quote}

这就是说,他谈的是那些使用不变资本较多而可变资本较少的部门。同时,这些部门同其他部门相比,预付资本的总量大,换句话说,这些部门只有依靠大资本才能经营。如果利润率既定,利润量就完全取决于预付资本的量。但这决不是使用大资本和使用许多不变资本(这两者往往联系在一起)的部门不同于使用小资本的部门的特点,这不过是下述论点的一种运用,即等量资本提供等量利润,因而较大的资本能比较小的资本提供更多的利润。这同“所使用的劳动量”没有任何关系。但是,利润率一般是大还是小,确实取决于整个资本家阶级的资本所使用的劳动总量,取决于所使用的无酬劳动的相对量,最后取决于花费在劳动上的资本同只是作为生产条件再生产出来的资本之间的比例。

李嘉图本人就反驳了亚·斯密的下述看法,即认为对外贸易中的较高利润率,“个别商人在对外贸易中有时赚得的大量利润,会提高国内的一般利润率”。李嘉图说:

\begin{quote}{“他们断言,利润的均等是由利润的普遍提高造成的;而我却认为,特别有利的部门的利润会迅速下降到一般水平。”(第7章《论对外贸易》,第132—133页)}\end{quote}

李嘉图认为,特殊利润(如果不是由市场价格涨到价值以上所造成)虽然会平均化,但不会提高一般利润率;其次,他认为,对外贸易和市场的扩大不可能提高利润率,李嘉图的这些观点究竟正确到什么程度,我们留到后面再说\authornote{见本册第494—497页和第535—536页。——编者注}。但是,如果承认他的观点是正确的,如果一般承认“利润的均等”,那末,他又怎么能够把“利润同资本成比例”的部门与利润“同所使用的劳动量成比例”的部门区别开来呢?

在前面引用的第二十六章《论总收入和纯收入》中,李嘉图说:

\begin{quote}{“我承认,由于地租的性质,除了最后耕种的土地以外,任何一块土地上用于农业的一定量资本所推动的劳动量,都比用于工业和商业的等量资本所推动的劳动量大。”(第419页)}\end{quote}

这句话完全是无稽之谈。第一,按照李嘉图的说法,在最后耕种的土地上使用的劳动量比所有其他土地上使用的劳动量大。在他看来,其他土地上的地租就是由此产生的。因此,怎么能说,除了最后耕种的土地以外,一定量资本在所有其他土地上推动的劳动量,一定会比在工业和商业上推动的劳动量大呢?较好土地的产品的市场价值,超过用于耕种这种土地的资本使用的劳动量所决定的个别价值,这同一定量资本“所推动的劳动量,比用于工业和商业的等量资本所推动的劳动量大”,是不一样的吧?但是如果李嘉图说,撇开土地肥力的差别,地租的产生一般是由于,农业资本所推动的劳动量,就资本的不变部分而言,比非农业生产中的平均资本所推动的劳动量大;那当然就对了。

[638]李嘉图没有看到,在剩余价值既定时,有些原因会使利润提高或降低,总之会对利润发生影响。因为李嘉图把剩余价值和利润等同起来,所以,当他现在要证明利润率的提高和降低仅仅是由引起剩余价值率提高或降低的那些情况决定的时候,他是前后一贯的。其次,他没有看到,如果撇开在剩余价值量既定时影响利润率(虽然并不影响利润量)的那些情况不谈,利润率就取决于剩余价值量,而决不是取决于剩余价值率。如果剩余价值率,剩余劳动率既定,剩余价值量就取决于资本的有机构成,即取决于一定价值的资本例如100镑所雇用的工人人数。在资本有机构成既定时,剩余价值量就取决于剩余价值率。可见,剩余价值量决定于以下两个因素:同时雇用的工人人数和剩余劳动率。如果资本增大,那末,不管资本的有机构成如何,——假定资本虽然增大而其有机构成不变,——剩余价值量也会增加。但这丝毫不会改变下述情况:对于一定价值的资本例如100来说,剩余价值量保持不变。如果这里剩余价值量等于10,那末对于1000来说,剩余价值量就等于100,但是比例不会因此变动。

{李嘉图写道:

\begin{quote}{“在同一经济部门不可能有两种利润率;所以,在产品价值对资本的比例不同时,不同的将是地租,而不是利润。”(第212—213页)(第12章《土地税》)}\end{quote}

这只适用于“同一经济部门”的正常利润率。否则就同前面引文\authornote{见本册第225和354页。——编者注}中的论点直接矛盾:

\begin{quote}{“一切商品,不论是工业品、矿产品还是土地产品,它们的交换价值始终不决定于在只是享有特殊生产便利的人才具备的最有利条件下足以把它们生产出来的较小量劳动,而决定于没有这样的便利,也就是在最不利条件下继续进行生产的人所必须花在它们生产上的较大量劳动;这里说的最不利条件,是指为了把需要的产品量生产出来而必须继续进行生产的那种最不利的条件。”(第2章《论地租》,第60—61页)}}\end{quote}

在第十二章《土地税》中,李嘉图附带对萨伊提出了如下的反驳。在这里,我们也可以看到,这位英国人总是尖锐地看到了经济上的差别,而那位大陆人却经常忘记这种差别。

\begin{quote}{“萨伊先生[在他所举的例子中]假定,‘一个土地所有者由于勤劳、节俭和经营本领而使自己的年收入增加5000法郎’。但是,土地所有者如果不是自己经营,他就不可能在他的土地上发挥他的勤劳、节俭和经营本领;如果土地所有者自己经营,他就是以资本家和租地农场主的身分,而不是以土地所有者的身分来进行改良。他不预先增加用于这一农场的资本量,单凭自己的特殊经营本领{因而“经营本领”多少也只是一句空话},就能那样增加自己农场的产品,那是不可想象的。”(第209页)}\end{quote}

在第十三章《黄金税》(这一章对李嘉图的货币理论很重要)中,李嘉图提出了关于市场价格和自然价格的某些补充或进一步的规定。这些补充或规定可以归结为一点:这两种价格的平均化进行得较快或较慢,要看该经济部门所允许的供给的增加或减少是快还是慢,也就是说,要看资本向该部门流入或从该部门流出是快还是慢。李嘉图关于地租的论述,受到各方面(西斯蒙第等人)的指责,说他忽略了使用许多固定资本的租地农场主抽出资本的困难,等等。(1815—1830年英国的历史充分地证明了这一点。)不管这种指责如何正确,它根本没有涉及理论,完全没有触动理论,因为这里谈的只不过是经济规律发生作用的快慢程度问题。但对于向新地投入新资本的相反的指责,情况就完全不同了。李嘉图的前提是,向新地投入新资本只能在没有土地所有者干预的条件下进行,这里资本是[639]在它的运动没有遇到抵抗的环境中发挥作用的。然而这是根本错误的。为了证明这个前提,为了证明在资本主义生产和土地所有权已经发展的地方存在这种前提,李嘉图总是设想有以下的情况:土地所有权——或者实际上,或者法律上——并不存在,资本主义生产,至少农业本身的资本主义生产还不发展。

至于刚才谈到的李嘉图关于市场价格和自然价格的论点,那是这样的:

\begin{quote}{“商品价格由于课税或生产困难而上涨的现象,无论如何最终是要发生的;但市场价格和自然价格经过多长时间才会趋于一致,必然取决于这种商品的性质和它的数量能够减少的容易程度。如果被课税的商品数量不能减少,如果比方说租地农场主或制帽厂主的资本不能抽到别的部门去,那末,即使他们的利润因课税而降低到一般水平之下,也不会引起什么后果。除非对他们的商品的需求增加,租地农场主和制帽厂主决不可能把谷物和帽子的市场价格提高到这些商品增加了的自然价格的水平。即使他们扬言要放弃这个行业,把自己的资本转到更有利的部门中去,也会被看作是虚张声势,决不会实现;所以这类商品的价格不会靠缩减生产来提高。但是,实际上一切商品的数量都是可以减少的,资本也可以由利润较小的部门转到利润较大的部门,不过速度有所不同而已。一种商品的供给越是易于缩减而又无损于生产者,在由于课税或任何其他原因而使生产困难增加之后,该商品的价格就越是迅速地上涨。”(第214—215页)“一切商品的市场价格和自然价格的一致,总是取决于该商品的供给增减的容易程度。对于金、房屋、劳动以及其他许多物品来说,在某些情况下是不可能很快达到这种结果的。但是,象帽子、鞋子、谷物和衣服这样一些逐年消费又逐年再生产的商品,情况就不同了。这些商品的供给在必要时可以减少,并且不需要很长时间就能使供给缩减到与增加了的生产费用相适应”。(第220—221页)}\end{quote}

\tsubsectionnonum{[(2)利润率变动的各种不同情况]}

李嘉图在这第十三章《黄金税》中说:

\begin{quote}{“地租不是财富的创造,只是财富的转移。”(第221页)}\end{quote}

难道利润是财富的创造,或者说,利润倒不是剩余劳动从工人到资本家的转移吗?至于工资,它事实上也不是财富的创造,但也不是财富的转移。它是劳动产品的一部分由生产这个产品的人占有。

在这一章中,李嘉图说:

\begin{quote}{“……对地面上的原产品所课的税,会落在消费者身上,并且决不会影响地租,除非这种税通过削减维持劳动的基金而压低工资,缩减人口并减少对谷物的需求。”(第221页)}\end{quote}

李嘉图说,“对地面上的原产品所课的税”既不会落在土地所有者身上,也不会落在租地农场主身上,而会落在消费者身上,这是否正确,我们在这里暂且不论。但是,我敢断言,如果他是正确的,这种税就会提高地租,而李嘉图认为,这种税不会影响地租,除非它通过使生活资料等等涨价而减少资本、人口和对谷物的需求。问题在于,李嘉图以为,原产品的涨价只是在它使工人消费的生活资料涨价的限度内,才影响利润率。这里,说原产品涨价只是在这个限度内才能影响剩余价值率,因而影响剩余价值本身,并因此也影响利润率,那是对的。但是,在剩余价值既定时,“地面上的原产品”涨价,会提高不变资本(与可变资本相比)的价值,会增大不变资本对可变资本的比例,所以,就会降低利润率,因而就会提高地租。李嘉图的出发点是:[640]既然原产品无论涨价或跌价都不影响工资,它也就不会影响利润;因为他断言{有一段话除外,那一段话后面我们回过头来再谈\authornote{见本册第490—491页。——编者注}},不管预付资本的价值降低还是提高,利润率保持不变。因此,如果预付资本的价值增加,那末产品的价值也就增加,同样,产品中构成剩余产品即利润的那一部分也就增加。预付资本的价值降低时情况则相反。这种说法只有在下述场合才是正确的,即由于原料涨价、课税或其他原因,可变资本和不变资本的价值按同一比例发生变动。在这种场合,利润率保持不变,因为资本有机构成没有发生任何变动。即使在这种场合,也必须假定在出现暂时性变动时发生的情况,那就是——工资保持不变,尽管原产品可能涨价或跌价(也就是说,工资保持不变,不管工资的使用价值在价值既定不变时是提高还是降低)。

可能有以下一些情况。

首先说两种主要的差别。

(A)由于生产方式的变动,所使用的不变资本量和可变资本量之间的比例发生变动。在这种情况下,假定工资按价值来说{即按(它所代表的)劳动时间来说}不变,剩余价值率就保持不变。但是,如果同一资本所使用的工人人数,即可变资本发生变动,剩余价值本身就会发生变动。如果由于生产方式的变动,不变资本相对减少,那末剩余价值就会增加,因而利润率也就提高。反之,其结果也相反。

这里始终假定,一定量比如说100单位的不变资本和可变资本的价值保持不变。

在这种情况下,生产方式的变动在同样程度上影响不变资本和可变资本,也就是比如说,不变资本和可变资本在价值没有变动时必定以同样程度增加或减少,是不可能的。因为在这里不变资本和可变资本的减少和增加的必然性总是同劳动生产率的变动相联系的。生产方式的变动对不变资本和可变资本的影响是不同的而不是相同的,这一点,在资本有机构成既定的情况下,与必须使用大资本还是小资本毫无关系。

(B)生产方式不变。在不变资本和可变资本的相对量不变(也就是它们各自在总资本中所占的份额不变)的情况下,不变资本和可变资本之间的比例变动,是由于加入不变资本或可变资本的商品的价值有了变动而发生的。

这里可能有以下几种情况:

[1]不变资本的价值不变;可变资本的价值提高或降低。这总是会影响剩余价值,因此也会影响利润率。

[2]可变资本的价值不变;不变资本的价值提高或降低。于是,在前一场合利润率会降低,在后一场合则会提高。

[3]如果不变资本的价值和可变资本的价值同时降低,但降低的比例不同,那末,一个的价值同另一个的价值相比,总是或者提高,或者降低。

[4]不变资本和可变资本的价值按同一比例变动,不管两者同时提高或同时降低,都是如此。如果两者价值都提高,那末利润率就降低,但这不是因为不变资本的价值提高,而是因为可变资本的价值提高,从而剩余价值降低(因为这里只是可变资本的价值提高了,尽管这个资本所推动的工人人数照旧不变,甚至可能减少)。如果两者价值都降低,那末利润率就提高,但这不是因为不变资本的价值降低,而是因为可变资本(在价值上)降低,从而剩余价值增长。

(C)生产方式的变动以及构成不变资本或可变资本的各要素价值的变动。

这里一种变动可能和另一种变动相抵销,例如,如果不变资本的量增加,而它的价值降低或保持不变(因而一定量比如说100单位的价值也相应降低),或者,如果不变资本的量降低,而它的价值保持不变(因而一定量的价值就相应提高)或按同一比例提高。在后一种情况下,资本的有机构成不会发生任何变动。利润率保持不变。但是,不变资本的量与可变资本相对来说减少,而它的价值却增长,这种情况,除农业资本以外,是决不可能发生的。

一种变动对另一种变动的这种抵销作用,对可变资本来说是不可能的(在实际工资不变的条件下)。

因此,除上述那一种情况以外,只有一种可能:同可变资本相比,不变资本的价值和量同时相对地降低或提高;因而,同可变资本相比,不变资本的价值绝对地提高或降低。这种情况我们已经考察过了。如果不变资本的价值和量虽然同时降低或提高,[641]但是比例不同,那末根据假定,这总是可以归结为:同可变资本相比,不变资本的价值提高或降低。

这也包括另一种情况。因为,如果不变资本的量增加,可变资本的量就相对减少,反之,结果也相反。对价值来说,情况也完全一样。[641]

\tsubsectionnonum{[(3)不变资本和可变资本在价值上的彼此相反的变动以及这种变动对利润率的影响]}

[642]关于C的情况(第640页),还必须注意以下这一点:

可能有这种情况:工资提高了,而不变资本在价值上,不是在量上,却降低了。如果提高和降低这两端彼此相符,利润率就可能保持不变。例如,不变资本=60镑,工资=40镑,剩余价值率=50%,于是,产品=120镑,而利润率=20%。如果不变资本在它的量保持不变时降到40镑,如果工资提高到60镑,而剩余价值从50%降到[33+(1/3)]%,那末产品仍然会等于120镑,而利润率会等于20%。这是不对的。

根据假定,所使用的[活]劳动量创造的总价值为60镑。因此,如果工资提高到60镑,剩余价值,因而利润率,就会等于零。即使工资不提高这么多,工资的任何提高也总会引起剩余价值的降低。如果工资提高到50镑,剩余价值就等于10镑;如果工资提高到45镑,剩余价值就等于15镑,依此类推。可见,在一切情况下,剩余价值和利润率都以同样程度降低。因为剩余价值和利润率在这里是按保持不变的总资本来计算的。在资本(指总资本)量相同时,利润率必定不是随着剩余价值率一同提高和降低,而是随着剩余价值绝对量一同提高和降低。

如果在上述例子中[不变资本由亚麻构成],亚麻价格下降,由同一数量的工人纺成纱的那个亚麻量,可以用40镑买到,那末我们就会得出如下结果:

\todo{}

这里利润率降到20%以下。

如果不变资本的价值降低到30镑,我们就会得出:

\todo{}

如果不变资本的价值降低到20镑,我们就会得出:

\todo{}

在我们假定的前提下,不变资本价值的降低始终只是部分地抵销可变资本价值的提高。在这种前提下,不变资本价值的降低不可能全部抵销可变资本价值的提高,因为要使利润率等于20%,剩余价值10镑必须是整个预付资本的1/5。但是,在可变资本等于50镑的情况下,只有在不变资本等于0时才有这种可能。如果我们假定,可变资本只提高到45镑,那末剩余价值将是15镑。如果我们还假定,不变资本降低到30镑,那末,我们就会得出如下结果:

\todo{}

因而,在这里,两种运动完全相互抵销了。

[643]下面我们再举这样一种情况:

\todo{}

因而,在这里,即使剩余价值降低了\authornote{同最初的情况60c+40v+20m相比。——编者注},但由于不变资本价值降低得更多,利润率也可能提高。同样使用100镑资本,尽管工资提高了,剩余价值率降低了,却能雇用更多的工人。虽然剩余价值率降低了,但剩余价值本身,因而利润却增加了,因为工人人数增加了。根据上述20c+45v这个比例,在使用100镑资本时,我们得出如下比例:

\todo{}

剩余价值率和工人人数之间的比例在这里有极其重要的意义。李嘉图从来不考察这种比例。[634]

\centerbox{※     ※     ※}

[641]前面对于一个资本有机构成内部的变动所作的考察,显然对于各个不同资本,对于各个不同生产部门的资本之间有机构成的差别来说,也是适用的。

第一,代替一个资本的有机构成的变动的,将是各个不同资本的有机构成的差别。

第二,[代替]由一个资本的两部分价值变动引起的有机构成的变动的,将是各个不同资本之间在它们所使用的原料和机器的价值方面的完全一样的差别。这不适用于可变资本,因为我们假定各个不同生产部门的工资相等。各个不同部门中的不同工作日在价值上的差别和这个问题毫无关系。如果首饰匠劳动比粗工的劳动贵,那末首饰匠的剩余劳动时间也按同一比例,比粗工的剩余劳动时间贵。\endnote{手稿(第641页)中接着有几段话谈到斯密对房租的看法。这几段话本版移至第十四章(见第415页)。——第437页。}[641]

\tsubsectionnonum{[(4)李嘉图在他的利润理论中把费用人价格同价值混淆起来]}

[641]在第十五章《利润税》中,李嘉图说:

\begin{quote}{“对通称为奢侈品的那些商品所课的税,只会落在这些商品的消费者身上……但是,对必需品所课的税,落到消费者身上的负担,不是同他们的消费量成比例,而总是要高得多。”例如,谷物税[落到工厂主身上的负担,不仅要看他消费的谷物是多少,而且要看谷物涨价使工资提高了多少]。“这会改变资本的利润率。凡是使工资提高的一切东西,都会减少资本的利润;因此,对工人消费的任何一种商品所课的任何一种税,都有降低利润率的趋势。”(第231页)}\end{quote}

如果课税的对象不仅加入个人消费,而且加入生产消费,或者它只加入生产消费,那末,对消费者所课的税同时就是对生产者所课的税。但是,在这种情况下,这不仅仅适用于工人消费的必需品,而且适用于资本家在生产上消费的一切材料。每一种这样的税都会降低利润率,因为它会提高不变资本的价值(与可变资本相对而言)。

我们就拿对亚麻或羊毛所课的税作例子。[642]亚麻涨价了。因此麻纺业者用资本100就不可能买到和以前同样数量的亚麻来纺纱了。因为生产方式不变,所以,麻纺业者为了把原来数量的亚麻纺成纱,就需要和以前同样数量的工人。但是,与花费在工资上的资本相对而言,亚麻现在比以前具有更大的价值。因而利润率降低。在这种情况下,麻纱价格的上涨并不会给他带来好处。这个价格上涨的绝对量,对麻纺业者根本无关紧要。全部问题只在于产品价格超过预付资本价格的那个余额。如果麻纺业者想要提高整个产品的价格,以便不仅弥补亚麻价格的上涨,而且使同量的纱给他带来和以前一样多的利润,那末,由于麻纱的原料价格上涨而已经下降了的需求,现在由于为了提高利润而人为地提高产品的价格,就会更加降低。尽管平均利润率是既定的,这种加价在这里却是办不到的。\endnote{手稿(第642页末和第643页开头)中接着有几段话谈不变资本和可变资本在价值上的彼此相反的变动。这几段话是对马克思手稿第640—641页的补充,放在本册第434—436页。——第438页。}[642]

[643]也是在第十五章《利润税》中,李嘉图说:

\begin{quote}{“我们在本书前面一个部分,已考察过资本划分为固定资本和流动资本,或者更确切地说,划分为耐久资本和非耐久资本对商品价格的影响。我们曾经指出,两个工厂主使用的资本额可能完全相等,由此获得的利润额可能完全相等,但他们的商品的售价,将根据他们所用资本的消费和再生产的快慢而极不相同。其中一个工厂主的商品可能卖4000镑,而另一个工厂主的商品可能卖10000镑,虽然他们每人使用的资本都是10000镑,得到的利润都是20%即2000镑。一个工厂主的资本,比如说,可能由必须再生产的流动资本2000镑以及建筑物、机器等固定资本8000镑所构成;相反,另一个工厂主可能有流动资本8000镑,机器、建筑物等固定资本却只有2000镑。如果现在这两个资本家每人的收入都课税10%即200镑,那末,一个工厂主为了获得一般利润率,必须把自己的商品价格从10000镑提高到10200镑;另一个工厂主也必须把自己的商品价格从4000镑提高到4200镑。在课税前,一个工厂主出卖的商品比另一个工厂主的商品贵1.5倍;课税以后,则贵1.42倍。一种商品的价格提高2%,另一种商品则提高5%。因此,如果货币价值保持不变,所得税将改变商品的相对价格和价值。”(第234—235页)}\end{quote}

错误就在于最后“价格和价值”的这个“和”字。价格的这种变动只证明(在资本按不同比例分为固定资本和流动资本时也完全一样):为了确定一般利润率,由一般利润率决定、调节的价格或费用价格,与商品的价值必然是极不相同的,而这个极为重要的观点,李嘉图是根本没有的。

在同一章,李嘉图说:

\begin{quote}{“如果一个国家不收税,而货币价值又下降,那末货币的充裕在每一个市场上{这里李嘉图有一个可笑的想法:好象随着货币价值下降,每一个市场上都必然会出现货币的充裕}[644]会对每一种商品产生同样的影响。如果肉价上涨20%,那末,面包、啤酒、鞋子、劳动以及其他任何商品的价格也会上涨20%。只有这样才能使所有生产部门的利润率相等。但是,如果这些商品中有一种被课税,情况就不同了;如果这时所有商品的价格都按货币价值下降的比例上涨,那末利润就会不相等;对于被课税的商品来说,利润就会高于一般水平,在利润恢复平衡以前,资本就会从一个部门转移到另一个部门,但利润只有在相对价格发生变动之后才能恢复平衡。”(第236—237页)}\end{quote}

而利润的这种平衡一般是这样形成的:各种商品的相对价值,实际价值会发生变动,会互相适应,以致不是同自己的实际价值相一致,而是同它们必须提供的平均利润相一致。

\tsubsectionnonum{[(5)一般利润率和绝对地租率之间的关系。工资下降对费用价格的影响]}

在第十七章《原产品以外的其他商品税》中,李嘉图说:

\begin{quote}{“布坎南先生认为,谷物和原产品是按垄断价格出卖的,因为它们提供地租。他假定,一切提供地租的商品都必须按垄断价格出卖;他由此得出结论说,对原产品所课的一切税都会落在土地所有者身上,而不会落在消费者身上。布坎南说:‘因为总是提供地租的谷物的价格不论从哪一方面来说都不受它的生产费用的影响,所以这种费用必须从地租中支付;因此,当这种费用有所增减时,结果不是价格的涨落,而是地租的增减。从这个观点来看,对农业工人、马匹或农具所课的一切税,实际上都是土地税,这种税的负担在整个租佃期内都落在租地农场主身上,而在租约重订时,则落在土地所有者身上。同样,使租地农场主能够缩减生产费用的一切改良农具,例如脱粒机和收割机,以及便于租地农场主把产品运到市场的一切设施,例如良好的道路、运河和桥梁,虽然会减少谷物的实际生产费用,但不会降低谷物的市场价格。因此,由于这类改良而节省下来的一切,都作为地租的一部分归土地所有者所得。’很明显,〈李嘉图说〉如果我们承认布坎南先生立论的根据,即谷物价格总是提供地租,那末,当然就会由此得出他所主张的一切结论。”(第292—293页)}\end{quote}

这一点也不明显。布坎南立论的根据,并不在于一切谷物都提供地租,而是在于提供地租的一切谷物都按垄断价格出卖,在于亚·斯密所解释和李嘉图所理解的那种意义的垄断价格,就是“消费者购买商品愿意支付的最高价格”。\endnote{李嘉图在他的《原理》第十七章(第三版第289—290页)提出了关于垄断价格的这个定义。马克思在前面、在本册第396页,引用了亚·斯密关于垄断价格的类似定义。——第440页。}

但这恰好也是错误的。提供地租(把级差地租撇开不谈)的谷物,并不是按照布坎南所说的垄断价格出卖的。谷物只有在高于它的费用价格即按它的价值出卖的时候,才按垄断价格出卖。谷物的价格决定于物化在谷物中的劳动量,不决定于它的生产费用,而地租是价值超过费用价格的余额,因而是由费用价格决定的:与价值相比,费用价格越小,地租就越多,费用价格越大,地租就越少。一切改良都会使谷物的价值降低,因为它们使生产谷物所需要的劳动量减少。但它们会不会使地租降低,却取决于各种情况。如果谷物跌价,因而工资降低,那末剩余价值率就提高。在这种情况下,租地农场主用于种子、家畜饲料等等方面的费用也会降低。因此,其他一切非农业生产部门的利润率就会提高,从而农业的利润率也会提高。在非农业生产部门,直接劳动和积累劳动的相对量会保持不变;工人人数和以前一样(与不变资本相对而言),但可变资本的价值会降低,因而剩余价值[645]会提高,就是说,利润率也会提高。因此,在农业中,剩余价值和利润率也会提高。在这里地租会降低,因为利润率提高了。谷物便宜了,但它的费用价格增加了。因此,它的价值和它的费用价格之间的差额缩小。

根据我们的假定,平均的非农业资本的比例=80c+20v,剩余价值率=50%;所以剩余价值=10,而利润率=10%。因而,具有平均构成的资本100的产品价值等于110。

现在假定,由于谷物跌价,工资降低1/4;这样,用不变资本80镑即用同量原料和机器来劳动的同一工人人数,总共只花费15镑。而同量商品的价值将是80c+15v+15m,因为根据假定,这些工人所完成的劳动量等于30镑。因此,同量商品的价值仍旧等于110镑。但是,所花费的资本只有95镑,15镑比95镑,就是[15+(15/19)]%。如果花费的资本量照旧不变,或者说,按资本100镑计算,那就得出这样的比例:[84+(4/19)]c+[15+(15/19)]v。利润等于15+(15/19)镑。产品价值=115+(15/19)镑。但是,根据我们的假定,农业资本=60c+40v,而它的产品价值等于120镑。当费用价格是110镑时,地租等于10镑。现在地租总共只有4+(4/19)镑,因为115+(15/19)镑+4+(4/19)镑=120镑。

这里我们可以看到:具有平均构成的资本100镑生产的商品,其费用价格是115+(15/19)镑,而不是以前的110镑。[单位]商品的平均价格会不会因此而提高呢?

商品的价值仍然和以前一样,因为要把同样数量的原料和机器转化为产品,需要同样数量的劳动。但同样的100镑资本推动了较大量的劳动,现在不是把以前的80镑不变资本,而是把84+(4/19)镑不变资本转化为产品。但是在同量的[新加]劳动中,无酬劳动比以前多了。因此,利润以及资本100镑生产的全部商品量的总价值都增加了。单位商品的价值保持不变,但用资本100镑,生产出了更多的具有同一价值的单位商品。但是,各个不同的生产部门的费用价格情况会怎样呢?

假设非农业资本由下列资本构成:

\todo{}

(2)的差额=-10,(3)和(4)的差额加在一起=+10。对于全部资本400来说,这个差额是:0-10+10=0。如果资本400的产品卖440,那末,这笔资本生产的商品就是按它们的价值出卖。那就会得到10%的利润。但是,(2)的商品比它们的价值低10镑出卖,(3)的商品比它们的价值高2+(1/2)镑出卖,而(4)则比它们的价值高7+(1/2)镑出卖。只有(1)的商品在按照它的费用价格(即100镑资本加10镑利润)出卖时,才是按其价值出卖。

[646]但如果工资降低1/4,比例关系将会怎样呢?

对资本(1)来说,现在已不是80c+20v,而是[84+(4/19)]c+[15+(15/19)]v,利润——15+(15/19),产品价值——115+(15/19)。

对资本(2)来说,现在工资只花费30镑,因为40的1/4=10,40-10=30。产品价值是:60c+30v+剩余价值30(因为所使用的劳动创造的价值在这里等于60镑)。这里资本为90镑。工资占[33+(1/3)]%。对于资本100来说,得出的比例是[66+(2/3)]c+[33+(1/3)]v;产品价值=133+(1/3)。利润率=[33+(1/3)]%。

对资本(3)来说,现在工资只花费11+(1/4)镑,因为15的1/4=3+(3/4),而15-[3+(3/4)]=11。产品价值是:85c+[11+(1/4)]v+剩余价值11+(1/4)(所使用的劳动创造的价值在这里等于22+(1/2))。这里资本为96+(1/4)镑。工资占[11+(53/77)]%。对于资本100来说,得出的比例是[88+(24/77)]c+[11+(53/77)]v,利润率=[11+(53/77)]%,而产品价值=111+(53/77)。

对资本(4)来说,现在工资只花费3+(3/4)镑,因为5的1/4=1+(1/4),而5-[1+(1/4)]=3+(3/4)。产品价值是:95c+[3+(3/4)]v+剩余价值3+(3/4)(因为全部[新加]劳动所创造的价值在这里等于7+(1/2))。这里资本为98+(3/4)镑。工资占[3+(63/79)]%。对于资本100来说,得出的比例是[96+(16/79)]c+[3+(63/79)]v。利润率=3+(63/79)。产品价值=103+(63/79)。

这样,我们就得出:

\todo{}

利润是16%,更确切些说,略高于[16+(1/7)]%。计算是不完全准确的,因为我们在计算平均利润时,把分数省略了,在进一步计算时没有包括在内,因此,(2)的负差大了一些,(1)、(3)、(4)的[正差]小了一些。但是,我们看到,如果计算精确,正差和负差就会相互抵销。但是我们也看到,一方面,(2)的低于本身价值出卖的商品,[另一方面](3)特别是(4)的高于本身价值出卖的商品都会大大增加。固然,对单位产品来说,这种高于或低于价值的程度不象表上的数字那么大,因为在所有这四类里,都使用了[比以前]更多的劳动量,因而有更多的不变资本(原料和机器)转化为产品;所以上述这种高于或低于价值的数字是分摊在更大量的商品上。不过,这种高于或低于价值的情况还是很显著的。

由此可见,工资的降低,对(1)和(3)来说,会引起费用价格的上涨[与价值相比],对(4)来说,会引起费用价格的极大上涨。这就是李嘉图在考察流动资本和固定资本的差别时所引出的规律,\endnote{马克思指李嘉图的《政治经济学和赋税原理》一书(第三版)第一章第四节和第五节,李嘉图在这两节中研究了工资的涨落对具有不同有机构成的资本所生产的商品的“相对价值”的影响问题。马克思在本册第192—221页对这两节作了详细的批判分析。——第444页。}但是他丝毫没有证明,也不可能证明:这一规律同价值规律是可以并行不悖的,产品的价值对总资本来说保持不变[不管它在各个生产部门之间如何分配]。

[647]如果我们还注意到由流通过程产生的资本有机构成的差别,计算和平均起来会复杂得多。实际上,在我们计算时,我们是假定,全部预付不变资本都加入产品,也就是说,它只包含固定资本例如在一年内(因为我们必须按年度来计算利润)的损耗。如果我们不这样假定,产品量的价值就会极不相同,而这样假定时,产品量的价值只与可变资本一起变动。第二,在剩余价值率相同而流通时间不同的时候,与预付资本相对而言,所生产的剩余价值量会有很大的差别。这里,如果撇开可变资本的差别不谈,剩余价值量彼此之间的比例就与等量资本生产的不同价值量彼此之间的比例相同。在不变资本的较大部分由固定资本构成的地方,利润率会低得多,在资本的较大部分由流动资本构成的地方,利润率会高得多;在可变资本较大(与不变资本相比),同时在不变资本中固定资本部分又较小的地方,利润率最高。如果不变资本的流动部分和固定部分之间的比例在不同的资本中是相同的,那就只有可变资本和不变资本之间的差别是决定的因素了。如果可变资本与不变资本的比例是相同的,那就只有固定资本和流动资本之间的差别,即不变资本本身内部的差别是决定的因素了。

正如我们已经看到的,如果非农业资本的一般利润率由于谷物跌价而提高,那末,租地农场主的利润率无论如何都会提高。问题在于,租地农场主的利润率会不会直接提高,看来,这要取决于所实行的改良的性质。如果实行的这种改良使花费在工资上的资本与花费在机器等等上的资本相比大大减少,那末,租地农场主的利润率就不必直接提高。如果这种改良,比如说,使租地农场主需要的工人减少1/4,那末,租地农场主现在必须花费在工资上的就不是以前的40镑,而只是30镑。因而他的资本现在是60c+30v,或者以100计算,就是[66+(2/3)]c+[33+(1/3)]v。因为用40单位支付的劳动,提供剩余价值20,所以,用30支付的劳动提供15,而用33+(1/3)支付的劳动就提供16+(2/3)。这样一来,农业资本的有机构成与非农业资本的有机构成便接近了。在上述情况下,如果工资同时下降1/4,农业资本构成甚至可能成为非农业资本构成的个别场合。\endnote{马克思在这里举例说明可能发生农业资本有机构成接近工业资本有机构成的过程的一种趋向。马克思以下述情况为出发点:农业资本是60c+40v,非农业资本是80c+20v。马克思假定,由于农业劳动生产率提高,农业工人人数减少四分之一。因而,农业资本的有机构成发生变动:过去需要花费100单位资本——60c+40v的产品,现在只需要花费90单位资本——60c+30v,折合100计算,就是[66+(2/3)]c+[33+(1/3)]v。这样一来,农业的资本有机构成就会接近工业的资本有机构成。马克思进一步假定,在农业工人人数减少的同时,工资还因谷物减价而降低四分之一。在这种场合必须假定,在工业中工资也按同一比例降低。然而,工资的降低,对具有较低构成的农业资本比对非农业资本必定会产生更大的影响。这就会使农业资本构成和工业资本构成之间的差额又进一步缩小。农业资本[66+(2/3)]c+[33+(1/3)]v,在工资降低1/4时变为资本[66+(2/3)]c+25v,折合100计算,就是[72+(8/11)]c+[27+(3/11)]v。非农业资本80c+20v,在工资降低1/4时变为资本80c+15v,折合100计算,就是[84+(4/19)]c+[15+(15/19)]v。在农业工人人数进一步减少以及工资进一步降低时,农业资本的有机构成就越来越接近非农业资本的有机构成。马克思在考察这种假设的情况时,为了弄清楚农业劳动生产率的增长对农业资本有机构成的影响,在这里撇开不谈工业劳动生产率的同时的而且往往是更迅速的增长,这种增长表现在工业资本有机构成比农业资本有机构成有更进一步的提高。关于工业中的资本有机构成和农业中的资本有机构成之间的关系问题,见前面第7—8、11、95—96、107—108、111、116—118和270—271页。——第445页。}这时,地租(绝对地租)就会消失。

李嘉图在前面引用过的评论布坎南的那段话之后,继续写道:

\begin{quote}{“我希望我已经充分说明,在一个国家的土地尚未全部投入耕种,并且耕种尚未达到最高程度以前,总有一部分投在土地上的资本是不提供地租的,并且〈!〉正是这部分资本调节谷物的价格,这部分资本的产品,正象在工业中一样,分为利润和工资。因为不提供地租的谷物价格,受谷物生产费用的影响,所以这种生产费用不可能从地租中支付。因此,生产费用增加的结果,将是价格上涨,而不是地租降低。”(同上,第293页)}\end{quote}

既然绝对地租等于农产品价值超过它的生产价格的余额,那末,很明显,凡是能使谷物等等生产所需的劳动总量减少的东西,也能使地租减少,因为使价值减少,也就是使价值超过生产价格的余额减少。在生产价格由已支付的费用构成的情况下,生产价格的降低和价值的降低是一回事,而且是和价值的降低同时进行的。但是,在生产价格(或“费用”)等于预付资本加平均利润的情况下,情况却恰好相反。产品的市场价值会降低,但其中等于生产价格的那一部分,在一般利润率由于谷物市场价值降低而提高时,会提高起来。因而,地租降低在这里是因为这个意义上的“费用”(李嘉图谈到生产费用时通常对费用是这样理解的)有了提高。促使不变资本与可变资本相比不断增长的农业改良,即使在所使用的劳动[活劳动和物化劳动]总量只是略有减少,或者说只减少那么一点,以致对工资根本没有影响(对剩余价值没有任何直接影响)的情况下,也会使地租大大降低。如果由于这种改良,资本60c+40v变为[66+(2/3)]c+[33+(1/3)]v(例如由于移民、战争、新市场的发现、外国谷物的竞争、非农业生产部门的繁荣等等所引起的工资提高,租地农场主可能不得不设法使用较多的不变资本和较少的可变资本;而这些情况在实行改良之后还可能继续起作用,因此,尽管有这些改良,工资不会降低),[648]那末,农产品的价值就会从120降到116+(2/3),即减少3+(1/3)。利润率仍然等于10%。地租从10降到6+(2/3),而且地租的这种降低是在工资没有任何降低的情况下发生的。

由于工业的继续进步,一般利润率下降,因此绝对地租可能提高。由于农产品价值增加,从而农产品价值及其费用价格之间的差额增大,结果地租提高,因此利润率可能降低。(同时利润率还会由于工资提高而下降。)

由于农产品价值下降,一般利润率提高,绝对地租就可能降低。由于资本有机构成的变革,农产品价值下降,虽然利润率这时并不提高,绝对地租也可能降低。一旦农产品的价值和它的费用价格彼此相等,从而农业资本具有非农业资本的那种平均构成,绝对地租就会完全消失。

李嘉图的论点只有这样表达才是正确的:当农产品的价值等于它的费用价格的时候,不存在绝对地租。但是在李嘉图那里这个论点是错误的,因为他说:由于价值和费用价格一般是等同的,工业是这样,农业也是这样,\authornote{[663}(下面一段话说明,李嘉图有意识地把价值和生产费用等同起来:“马尔萨斯先生似乎认为,把某物的费用和价值等同起来,是我的学说的一部分。如果他说的费用是指包括利润在内的‘生产费用’,那确是如此。”(同上,第46页))[663]]所以,不存在绝对地租。实际上情况恰好相反:如果在农业中价值和费用价格等同,农业就是一种例外的生产了。

李嘉图承认可能不存在不支付任何地租的土地,同时他认为,即使这样,下面这种情况还是可以作为他的充分依据,即至少投在土地上的资本有些份额是不支付任何地租的。前一种情况和后一种情况对理论来说同样是无关紧要的。真正的问题在于:是由这种土地或这种资本的产品来调节市场价值呢?还是相反,这些产品由于它们的追加供给只能按照而不能高于并非由它们调节的市场价值出卖,因而不得不低于自己的价值出卖呢?关于后来使用的那些资本份额,问题很简单,因为这里在投入追加份额时,土地所有权对租地农场主来说是不存在的,作为资本家,租地农场主只注意费用价格;甚至当他自己是追加资本的所有者时,与其把这笔资本借出,只取得利息,而得不到利润,还不如把它投在他租种的土地上,即使取得的利润低于平均利润,对他更有利。至于地段,那末,这些不支付地租的土地,构成支付地租的整个地产的组成部分;这些地段是同整个地产不可分割的,它们同整个地产一起出租,虽然不能把这些地段单独租给任何一个资本主义农场主(但完全可以租给茅舍贫农以及小资本家)。这些小块土地也并不作为“土地所有权”与租地农场主相对立。或者土地所有者不得不自己耕种这些地段。租地农场主不可能为这些地段支付地租,而土地所有者也不会毫无代价地把它们租出去,除非他是想通过这种办法,自己不花费什么,就把自己的土地变为耕地。

如果在一个国家,农业资本的构成与非农业资本的平均构成相等,情况就不同了,而这是以农业的高度发展或工业发展水平很低为前提的。在这种场合,农产品的价值就会同它的费用价格相等。这时只可能支付级差地租。那些不提供级差地租、只能带来[真正的]农业地租的地段,这时就根本不可能支付任何地租了。因为当租地农场主把这些土地的产品按它们的价值出卖时,它们只抵补他的费用价格。因而租地农场主不支付任何地租。这样一来,土地所有者只好自己耕种这些土地,或者在租金的名义下,把他的租佃者的一部分利润甚至一部分工资刮走。一个国家可能发生这种情况,这并不妨碍另一个国家可能发生完全相反的情况。但是在工业,从而资本主义生产发展水平很低的地方,是不存在资本主义租地农场主的,因为资本主义租地农场主的存在是以农业中实行资本主义生产为前提的。因此,我们考察的就是与土地所有权仅仅作为地租才在经济上存在的那种经济组织完全不同的关系了。

李嘉图也是在第十七章中说:

\begin{quote}{“原产品没有垄断价格,因为大麦和小麦的市场价格,同呢绒和麻布的市场价格一样,是由它们的生产费用调节的。唯一的差别在于:谷物价格是由用于农业的资本的一部分,即不支付地租的那一部分调节的,而在工业品生产中,所用资本的每一部分都产生相同的结果;并且由于任何部分都不支付地租,所以每一部分都同样是价格的调节者。”(同上,第290—291页)}\end{quote}

认为在工业中所用资本的每一部分都产生相同的结果,并且任何部分都不提供地租(不过,工业中叫做超额利润),这种说法不仅是错误的,而且,[650]\endnote{在马克思编的手稿页码中漏了649这个页码。——第449页。}正如我们在前面看到的\authornote{见本册第225、354和428页。——编者注},已经被李嘉图自己所驳倒。

现在我们就来考察李嘉图的剩余价值理论。

\tsectionnonum{[B.李嘉图著作中的剩余价值问题]}

\tsubsectionnonum{(1)劳动量和劳动的价值。[劳动与资本的交换问题按照李嘉图的提法无法解决]}

李嘉图著作的第一章《论价值》一开始第一节就用了这样一个标题:

\begin{quote}{“商品的价值或这个商品所能交换的任何其他商品的量,取决于生产这个商品所必需的劳动的相对量,而不取决于付给这一劳动的报酬多少。”}\end{quote}

这里,李嘉图按照贯穿于他的全部研究中的风格,在他的书的开头就提出这样一个论点:商品价值决定于劳动时间这一规定与工资,或者说,对这种劳动时间即这种劳动量所支付的不同报酬,并不矛盾。李嘉图一开始就反对亚·斯密把商品价值决定于生产商品所必需的相应的劳动量这个规定与劳动的价值(或劳动的报酬)混淆起来。

显然,A和B两个商品包含的相应的劳动量,同生产商品A和B的工人从自己的劳动产品中得到多少,是绝对没有关系的。商品A和B的价值决定于生产它们所花费的劳动量,而不决定于商品A和B的所有者花费的劳动费用。劳动量和劳动价值是两个不同的东西。商品A和B包含的相应的劳动量,同A和B包含多少由A和B的所有者付酬的,甚至是他们自己完成的劳动,是毫无关系的。商品A和B不是按照它们所包含的有酬劳动的比例相互交换,而是按照它们所包含的既包括有酬劳动也包括无酬劳动的劳动总量的比例相互交换。

\begin{quote}{“亚当·斯密如此正确地规定了交换价值的真正源泉,要是前后一贯,他本来应该坚持一切物品价值的大小同生产它们所花费的劳动量的多少成比例的观点,可是他自己又提出了价值的另一个标准尺度,说一切物品价值的大小同它们所能交换的这种标准尺度的量的多少成比例……好象这是两种意思相同的说法;好象一个人由于他的劳动效率增加了一倍,因而能生产的商品量也增加一倍,他用它〈即他的劳动〉进行交换时所得到的量就必然会比以前增加一倍。如果这种说法确实是正确的,如果工人的报酬总是和他所生产的东西成比例,那末用于生产某种商品的劳动量和这种商品所能购买的劳动量就会相等,两者中的任何一个都可以准确地衡量其他物品的价值的变动;但是,它们不是相等的。”(第5页)}\end{quote}

亚·斯密在任何地方都没有说过,“这是两种意思相同的说法”。相反,他说:因为在资本主义生产中工人的工资已不再等于他所生产的产品,因而,一个商品所耗费的劳动量和工人用这一劳动所能购买的商品量,是两个不同的东西,正因为这样,商品所包含的劳动的相对量不再决定商品的价值,商品的价值宁可说是决定于劳动的价值,决定于我用一定量商品所能购买或支配的劳动量。因此,斯密认为,劳动的价值代替劳动的相对量成为价值尺度。李嘉图正确地回答亚·斯密说,两个商品所包含的劳动的相对量,同这种劳动的产品中有多少归工人自己所有毫无关系,同这种劳动的报酬如何毫无关系;因此,既然在工资(不同于产品本身的价值的工资)出现以前,劳动的相对量是商品价值的尺度,那就没有任何根据能够说明,为什么在工资出现以后,劳动的相对量就不再是商品价值的尺度。李嘉图正确地回答说,在这两种说法意思相同的时候,亚·斯密可以使用两种说法,但是,一旦两种说法意思不再相同时,这并不能成为用错误说法去代替正确说法的理由。

但是,李嘉图这些话丝毫没有解决构成亚·斯密的矛盾的内在基础的那个问题。只要我们谈的是物化劳动,劳动的价值和劳动量就依然是“意思相同的说法”。[651]一旦我们谈到物化劳动和活劳动交换,这两种说法就不再是这样的了。

两个商品按照它们包含的物化劳动进行交换。等量物化劳动互相交换。劳动时间是它们价值的“标准尺度”,正因为这样,所以它们的“价值的大小同它们所能交换的这种标准尺度的量的多少成比例”。如果商品A包含一个工作日,那末这个商品就可以与同样包含一个工作日的任何数量的其他商品交换;这个商品的“价值的大小”,同它换得的其他商品中的物化劳动量的多少成比例,因为这种交换比例是这个商品本身包含的劳动的相对量的表现,是和这种劳动的相对量相等的。

但是雇佣劳动是一种商品。它甚至是作为商品的产品进行生产的基础。原来,价值规律不适用于雇佣劳动。那就是说,这个规律根本不支配资本主义生产。这里有一个矛盾。这是亚·斯密遇到的一个问题。第二个问题是,一个商品(作为资本)的价值增殖不是同它所包含的劳动成比例,而是同它所支配的别人的劳动成比例,它所支配的别人的劳动量大于它本身所包含的劳动量,后面我们将会看到,这个问题在马尔萨斯著作中有了更充分的发挥。这实际上是斯密下述说法的第二个秘密动机,他说,自从资本主义生产出现以后,商品价值就不决定于商品所包含的劳动,而决定于商品所支配的活劳动,因而决定于劳动的价值。

李嘉图简单地回答说,在资本主义生产中情况就是这样。他不仅没有解决问题,甚至没有发觉亚·斯密著作中的这个问题。他根据自己研究的整个性质只限于证明,变动着的劳动价值——简单说,就是工资——并不会推翻如下的论点:不同于劳动本身的商品的价值由商品所包含的劳动的相对量决定。“它们不是相等的”,就是说,“用于生产某种商品的劳动量和这种商品所能购买的劳动量”不是相等的。李嘉图满足于确定这一事实。但是,劳动这种商品和其他商品有什么区别呢?一个是活劳动,另一个是物化劳动。因此这只是劳动的两种不同形式。既然这里只是形式的不同,那末,为什么规律对其中一个适用,对另一个就不适用呢?李嘉图没有回答这个问题,他甚至没有提出这个问题。

他说的下面这段话对这个问题也没有什么帮助:

\begin{quote}{“难道劳动的价值不……发生变动吗?它不仅象其他一切物品〈应读作商品〉一样,受始终随着社会状况的每一变动而变动的供求关系的影响,而且受用工资购买的食物和其他必需品的价格变动的影响。”(第7页)}\end{quote}

劳动的价格象其他商品的价格一样随着需求和供给的变动而变动,这一点,照李嘉图自己的意见,在涉及劳动价值的地方,是什么也说明不了的,正如其他商品的价格随着需求和供给的变动而变动,对这些商品的价值是什么也说明不了的一样。但是,“工资”(这只不过是劳动价值的另一种说法)要受“用工资购买的食物和其他必需品的价格变动”的影响这一事实,同样也不能说明为什么决定劳动的价值和决定其他商品的价值不一样(或看起来不一样)。因为其他商品也受加入它们的生产并和它们交换的其他商品的价格变动的影响。而花费在食物和必需品上的工资支出只不过表明劳动的价值同食物和必需品进行交换而已。问题正是在于:劳动同劳动所交换的商品为什么不按价值规律进行交换,不按劳动的相对量进行交换?

这样提出问题,既然以价值规律作为前提,问题本身就无法解决,所以不能解决,是因为这里把劳动本身同商品对立起来了,把一定量直接劳动本身同一定量物化劳动对立起来了。

我们后面将会看到,李嘉图的解释的这个弱点促进了李嘉图学派的瓦解,并且引出了荒谬的假设。

[652]威克菲尔德说得对:

\begin{quote}{“如果把劳动看成一种商品,而把资本,劳动的产品,看成另一种商品,并且假定这两种商品的价值是由相同的劳动量来决定的,那末,在任何情况下,一定量的劳动就都会和同量劳动所生产的资本量相交换;过去的劳动就总会和同量的现在的劳动相交换。但是,劳动的价值同其他商品相比,至少在工资取决于[产品在资本家和工人之间的]分配的情况下,不是由同量劳动决定,而是由供给和需求的比例决定。”(爱·吉·威克菲尔德给他在1835年伦敦出版的亚·斯密的《国富论》一书第1卷第230页所加的注)}\end{quote}

这也是贝利爱好的题目之一;我们在后面将加以考察。萨伊也是这样,他对于在这里突然承认供给和需求的决定性作用是非常高兴的。\endnote{马克思在他的手稿的下一页(第653页)上又回过头来谈到萨伊的“幸灾乐祸”,说这是因为李嘉图在用维持工人生活所必需的生存资料决定“劳动价值”时,引证了供求规律。这里马克思引用的李嘉图著作是康斯坦西奥译、萨伊加注的法译本。马克思在这里是不确切的。萨伊在给李嘉图著作所加的注释中“幸灾乐祸”,是因为李嘉图用供给和需求来决定货币的价值。马克思在《哲学的贫困》(见《马克思恩格斯全集》中文版第4卷第126页)中曾引了萨伊注释中有关的这段话。这段话的出处是:大·李嘉图《政治经济学和赋税原理》,康斯坦西奥译自英文,附让·巴·萨伊的注释和评述,835年巴黎版第二卷第206—207页。——第454、455页。}

\centerbox{※     ※     ※}

[652]关于(1)还要指出,李嘉图著作的第一章第三节用了这样一个标题:

\begin{quote}{“影响商品价值的,不仅是直接花费在商品上的劳动,而且还有花费在协助这种劳动的器具、工具和建筑物上的劳动。”}\end{quote}

因此,商品的价值既决定于为生产该商品所需要的物化的(过去的)劳动量,也决定于为生产该商品所需要的活的(现在的)劳动量。换句话说:劳动量完全不受劳动是物化劳动还是活劳动、过去劳动还是现在(直接)劳动这种形式差别的影响。如果这种差别在规定商品价值时是没有意义的,为什么当过去劳动(资本)同活劳动交换时,这种差别就有了决定性意义呢?既然这种差别本身,正象在对商品的关系上表现出来的那样,对于规定价值没有意义,它为什么在这里就一定会使价值规律失效呢?李嘉图没有回答这个问题,他甚至没有提出这个问题。[652]

\tsubsectionnonum{(2)劳动能力的价值。劳动的价值。[李嘉图把劳动同劳动能力混淆起来。关于“劳动的自然价格”的见解]}

为了规定剩余价值,李嘉图象重农学派、亚·斯密等人一样,必须首先规定劳动能力的价值,或者——按照他跟着亚·斯密和他的先行者使用的说法——劳动的价值。[652]

[652]那末,劳动的价值,或者说,劳动的自然价格,是怎样决定的呢?因为照李嘉图的意见,自然价格不过是价值的货币表现。

\begin{quote}{“劳动正象其他一切可以买卖并且在数量上可以增加或减少的物品一样〈就是说,同其他一切商品一样〉,有它的自然价格和市场价格。劳动的自然价格是使工人大体上说能够生存下去并且能够在人数上不增不减地〈应当说,按照生产的平均增长所需要的增长率〉延续其后代所必需的价格。工人养活自己以及养活为保持工人人数所必需的家庭的能力……取决于工人养活自己及其家庭所必需的食物、必需品和舒适品的价格。随着食物和其他必需品价格的上涨,劳动的自然价格也上涨;随着这些东西的价格下降,劳动的自然价格也下降。”(第86页)“不能认为劳动的自然价格是绝对固定不变的,即使用食物和必需品来计算也是一样。劳动的自然价格在同一国家的不同时期会发生变动,而在不同的国家会有很大差别。它主要取决于人民的风俗习惯。”(第91页)}\end{quote}

可见,劳动的价值是由在一定社会中为维持工人生活并延续其后代通常所必需的生活资料决定的。

但是,为什么?根据什么规律劳动的价值这样决定呢?

李嘉图除了说供求规律把劳动的平均价格归结为维持工人生活所必需(在一定社会中生理上或社会上所必需)的生活资料以外,实际上没有回答这一问题。[653]李嘉图在这里,在其整个体系的一个基本点上,正象萨伊幸灾乐祸地指出的那样(见康斯坦西奥的译本),是用需求和供给来决定价值。\endnote{马克思在他的手稿的下一页(第653页)上又回过头来谈到萨伊的“幸灾乐祸”,说这是因为李嘉图在用维持工人生活所必需的生存资料决定“劳动价值”时,引证了供求规律。这里马克思引用的李嘉图著作是康斯坦西奥译、萨伊加注的法译本。马克思在这里是不确切的。萨伊在给李嘉图著作所加的注释中“幸灾乐祸”,是因为李嘉图用供给和需求来决定货币的价值。马克思在《哲学的贫困》(见《马克思恩格斯全集》中文版第4卷第126页)中曾引了萨伊注释中有关的这段话。这段话的出处是:大·李嘉图《政治经济学和赋税原理》,康斯坦西奥译自英文,附让·巴·萨伊的注释和评述,835年巴黎版第二卷第206—207页。——第454、455页。}

李嘉图本来应该讲劳动能力,而不是讲劳动。而这样一来,资本也就会表现为那种作为独立的力量与工人对立的劳动的物质条件了。而且资本就会立刻表现为一定的社会关系了。可是,在李嘉图看来,资本仅仅是不同于“直接劳动”的“积累劳动”,它仅仅被当作一种纯粹物质的东西,纯粹是劳动过程的要素,而从这个劳动过程是决不可能引出劳动和资本、工资和利润的关系来的。

\begin{quote}{“资本是一国财富中用于生产的部分,由推动劳动所必需的食物、衣服、工具、原料、机器等组成。”(第89页)“较少的资本,也就是较少的劳动。”(第73页)“劳动和资本,即积累劳动”。(第499页)}\end{quote}

贝利正确地觉察到了李嘉图在这里[在劳动价值问题上]所作的飞跃:

\begin{quote}{“李嘉图先生相当机智地避开了一个困难,这个困难乍看起来似乎会推翻他的关于价值取决于在生产中所使用的劳动量的学说。如果严格地坚持这个原则,就会得出结论说,劳动的价值取决于在劳动的生产中所使用的劳动量。这显然是荒谬的。因此,李嘉图先生用一个巧妙的手法,使劳动的价值取决于生产工资所需要的劳动量;或者用他自己的话来说,劳动的价值应当由生产工资所必需的劳动量来估量,他这里指的是为生产付给工人的货币或商品所必需的劳动量。那我们同样也可以说,呢绒的价值不应当由生产呢绒所花费的劳动量来估量,而应当由生产呢绒所换得的银所花费的劳动量来估量。”(《对价值的本质、尺度和原因的批判研究》1825年伦敦版第50—51页)}\end{quote}

这种反驳逐字逐句都是正确的。李嘉图区别了名义工资和实际工资。名义工资是用货币表现的工资,是货币工资。

\begin{quote}{“名义工资”是“一年内付给工人的镑数”,而“实际工资”是“为获得这些镑所必需的工作日数”。(李嘉图,同上第152页)}\end{quote}

既然工资等于工人消费的必需品,而这种工资(“实际工资”)的价值等于这些必需品的价值,那末,显然,这些必需品的价值也等于“实际工资”,等于这一工资所能支配的劳动。如果必需品的价值发生变动,“实际工资”的价值也要变动。假定工人消费的必需品完全由谷物构成,他的必要的生活资料数量是每月一夸特谷物。那末,他的工资的价值[一个月]就等于一夸特谷物的价值;如果一夸特谷物的价值提高或降低,一个月劳动的价值也要提高或降低。但是,不论一夸特谷物的价值怎样提高或降低(不论一夸特谷物包含多少劳动),它的价值总是等于一个月劳动的价值。

这里我们看到了一个隐蔽的理由,说明为什么亚·斯密说,自从资本以及雇佣劳动出现以后,产品的价值已不由花费在产品上的劳动量决定,而由该产品所能支配的劳动量来决定。由劳动时间决定的谷物(和其他必需品)的价值会发生变动;但是,只要劳动的自然价格得到支付,一夸特谷物所能支配的劳动量就保持不变。因此,劳动同谷物相比具有不变的相对价值。就是由于这个缘故,在斯密的著作中,劳动的价值和谷物的价值(谷物在这里代表一般食物;见迪肯·休谟的著作\endnote{马克思引用的是詹姆斯·迪肯·休谟的小册子《关于谷物法的看法》(1815年伦敦版)。休谟在谈到亚当·斯密的“谷物价格决定劳动价格”这一论点时写道:“当斯密博士谈到谷物时,他是指一般食物……一切农产品”。(第59页)——第457页。})都是价值的标准尺度,因为一定量的谷物,只要它支付劳动的自然价格,它就支配一定量的劳动,不管生产一夸特谷物所花费的劳动量是多大。同量劳动总是支配同一使用价值,或者,确切些说,同一使用价值总是支配同量劳动。

李嘉图自己就是这样来规定劳动的价值、劳动的自然价格的。李嘉图说:一夸特谷物具有极不相同的价值,虽然一夸特谷物总是支配[654]同量的劳动,或者说,同量的劳动总是支配这一夸特谷物。是的,亚·斯密说,不管由劳动时间决定的一夸特谷物的价值怎样变动,工人为了购买这一夸特谷物,总是要付出(牺牲)同量劳动。因此,谷物的价值会变动,但是,劳动的价值不会变动,因为一个月的劳动等于一夸特谷物。就是谷物的价值,也只有在我们考察的是生产谷物所需要的劳动的时候才变动。如果我们考察的是这一夸特谷物所交换的劳动量,是谷物所推动的劳动量,那末谷物的价值也不会变动。正因为这样,一夸特谷物所换得的劳动量是价值的标准尺度。而其他商品的价值和劳动的关系,同它们和谷物的关系是一样的。一定量谷物支配一定量劳动。一定量的任何其他商品支配一定量的谷物。因此,任何其他商品,或者,确切些说,任何其他商品的价值,都是用该商品所支配的劳动量来表现的,因为这一商品的价值是用该商品所支配的谷物量来表现的,而这一谷物量又是用该谷物所支配的劳动量来表现的。

但是,其他商品和谷物(必需品)的价值比例由什么决定呢?由这些商品所支配的劳动量决定。而它们所支配的劳动量又由什么决定呢?由这一劳动所支配的谷物量决定。这里,斯密必然陷入循环论证。(不过,我们要顺便指出,斯密在进行真正的分析的地方,从来不采用这个价值尺度。)此外,斯密在这里把劳动同货币混淆起来,李嘉图也常常这样,正如斯密和李嘉图所说,劳动是

\begin{quote}{“商品价值的基础”,而“生产商品所必需的相对劳动量”是“确定商品相互交换时各自必须付出的相应商品量的尺度”。(李嘉图,同上第80页)}\end{quote}

斯密把价值的这一内在尺度同已经以价值规定为前提的外在尺度即货币混淆起来了。

亚·斯密根据一定量的劳动可以换得一定量的使用价值,得出结论说,这一定量的劳动是价值尺度,它总是具有同一价值,可是,同量使用价值却可以代表极不相同的交换价值。亚·斯密这样做是错了。但李嘉图犯了双重的错误,因为第一,他不懂得导致斯密犯错误的问题,第二,他自己完全忘记了商品的价值规律,而求助于供求规律,因而不是用花费在生产劳动力[forceoflabour]上的劳动量来决定劳动的价值,却用花费在生产付给工人的工资上的劳动量来决定劳动的价值,从而他实际上是说:劳动的价值决定于支付劳动的货币的价值!而后者由什么决定呢?支付劳动的货币量由什么决定呢?由支配一定量劳动的使用价值量决定,或者说,由一定量劳动支配的使用价值量决定。结果,李嘉图就一字不差地重犯了他指责亚·斯密犯过的那种前后矛盾的错误。

同时,我们看到,这一点妨碍他理解商品和资本之间的特殊区别,商品同商品的交换和资本同商品的交换(按商品交换规律)之间的特殊区别。

前面举的例子是:1夸特谷物=1个月劳动。假定1个月劳动等于30工作日。(1工作日等于12小时。)在这种情况下,1夸特谷物的价值小于30工作日。如果1夸特谷物是30工作日的产品,那末,劳动的价值就等于劳动的产品。因此就不会有任何剩余价值,因而也不会有任何利润。这样就不会有任何资本。就是说,如果1夸特谷物是支付30工作日的工资,实际上1夸特谷物的价值就总是小于30工作日。剩余价值就取决于这1夸特谷物的价值究竟比30工作日小多少。比如说,1夸特谷物是25工作日的产品。那末,剩余价值=5工作日,=全部劳动时间的1/6。如果1夸特(8蒲式耳)=25工作日,那末,30工作日=1夸特又1+(3/5)蒲式耳。可见,30工作日的价值(即支付30工作日的工资)总是小于包含30工作日的产品的价值。所以,谷物的价值不决定于[655]谷物所支配的劳动,谷物所换得的劳动,而决定于谷物所包含的劳动。相反,30日劳动的价值[即支付30工作日的工资]始终决定于1夸特谷物,而不论1夸特谷物的价值是多少。

\tsubsectionnonum{(3)剩余价值。[李嘉图没有分析剩余价值的起源。李嘉图把工作日看作一个固定的量]}

如果撇开把劳动和劳动能力混淆起来这一点不谈,那末李嘉图倒是正确地规定了平均工资,或者说,劳动的价值。这就是,李嘉图说,平均工资既不决定于工人得到的货币,也不决定于工人得到的生活资料,而是决定于为生产这些生活资料所花费的劳动时间,决定于物化在工人得到的生活资料中的劳动量。李嘉图把这个叫做实际工资。(见下文。)

并且,他得出这一规定是必然的。既然劳动的价值决定于该价值必须花费在上面的必要生活资料的价值,而必要生活资料的价值同一切其他商品的价值一样,决定于生产它们所花费的劳动量,那末,由此自然会得出结论说,劳动的价值等于必要生活资料的价值,等于生产这些必要生活资料所花费的劳动量。

但是,不管这个公式多么正确(撇开劳动和资本的直接对立不谈),它还是不充分的。单个工人为补偿他的工资进行再生产,——就是说,如果考虑到这个过程的连续性,——他生产的虽然不是直接供他维持生活的产品{他可能生产完全不加入他的消费的产品;即使他生产必要生活资料,但由于分工,他只生产其中的一种,例如谷物,并且只赋予它一种形式(例如只是粮食,不是面包)},但他生产的商品具有他的生活资料的价值,或者说,他生产的是他的生活资料的价值。这就是说,——如果我们考察的是工人的一日平均消费的话,——他一日消费的必要生活资料包含的劳动时间,是他的工作日的一部分。他用一日的一部分为再生产他的必要生活资料的价值而劳动;在工作日的这一部分中生产出来的商品,同工人一日消费的必要生活资料具有同样的价值,或者说,包含同样多的劳动时间。他的工作日中有多大一部分用于再生产或生产他的生活资料的价值,即他的生活资料的等价物,这取决于这些必要生活资料的价值(因而,取决于劳动的社会生产率,而不取决于他劳动的那一个别生产部门的生产率)。

李嘉图自然假定,一日的必要生活资料所包含的劳动时间,等于工人每天为了再生产这些必要生活资料的价值而必须劳动的劳动时间。但是,这样一来,由于他在阐述问题时没有把每个工人的工作日的一部分直接表现为用于再生产工人自己劳动能力的价值的时间,他就给这个问题的研究造成了困难,模糊了对这里存在的关系的理解。由此便产生了双重的混乱。剩余价值的起源变得不清楚了,因此后来的经济学家责备李嘉图没有理解、没有阐明剩余价值的性质。部分地也是由于这个原因,产生了他们经院式地解释剩余价值的尝试。但是,因为剩余价值的起源和性质在这里没有得到明确的表述,所以,把剩余劳动加必要劳动——简单说总工作日——看作某种固定的量,忽略了剩余价值量的差别,不理解资本的生产性,不理解资本强迫进行剩余劳动,——一方面是强迫进行绝对剩余价值的生产,其次是资本所固有的缩短必要劳动时间的内在渴望,——因而没有阐明资本在历史上的合理性。相反,亚·斯密已经提出了正确的公式。把剩余价值归结为剩余劳动,和把价值归结为劳动,具有同等重要的意义,并且措辞明确。

李嘉图是从资本主义生产的现有事实出发的。劳动的价值小于劳动所创造的产品的价值。因此,产品的价值大于生产产品的劳动的价值,或者说,大于工资的价值。产品的价值超过工资的价值的余额,就是剩余价值。(李嘉图错误地说成利润,前面已经指出,他在这里把利润和剩余价值等同起来了,而他说的实际上是后者。)在李嘉图看来,产品的价值大于工资的价值,这是事实。这个事实究竟是怎样产生的,仍然不清楚。整个工作日大于工作日中生产工资所需要的部分。为什么呢?李嘉图仍旧没有说明。因此,李嘉图错误地假定总工作日的量是固定的,并从这里直接得出了错误的结论。因此,李嘉图只能用生产必要生活资料的社会劳动的生产率的提高或降低来说明剩余价值的增加或减少。换句话说,李嘉图只知道相对剩余价值。

[656]显然,如果工人为生产他自己的生活资料(即同他自己的生活资料在价值上相等的商品)要用去他一整天工夫,那就不可能有任何剩余价值,因而也不可能有资本主义生产和雇佣劳动。为了使资本主义生产能够存在,社会劳动生产率就必须有相当的发展,使总工作日中除了再生产工资所必须的劳动时间以外还有余额,也就是说,要有或多或少的剩余劳动存在。但是,同样明显的是,如果说在劳动时间既定(工作日长度既定)的情况下,劳动生产率可以大不相同,那末,另一方面,在劳动生产率既定的情况下,劳动时间即工作日长度也可以大不相同。其次,很明显,如果说,为了使剩余劳动能够存在,必须以劳动生产率的一定发展水平为前提,那末,仅仅这种剩余劳动的可能性(就是说,那种必需的最低限度的劳动生产率的存在)还不能造成它的现实性。为此,必须事先强迫工人进行超过上述限度的劳动,而强迫工人这样做的就是资本。这一切在李嘉图著作中都没有谈到,而争取规定正常工作日的整个斗争却正是由此产生的。

在劳动的社会生产率发展的低级阶段,也就是说,在剩余劳动相对说来还少的那个阶段,靠别人劳动过活的阶级同劳动者的人数相比一般是小的。这个阶级随着劳动生产率的增长,也就是说,随着相对剩余价值的增长,可能大大(相对地说)增长起来。

其次,据认为,劳动的价值在同一国家的不同时期和在同一时期的不同国家是有很大变化的。但是,资本主义生产的故乡是温带。劳动的社会生产力可能很不发达,可是,正是在必要生活资料的生产上,一方面由于自然因素(如土地)富饶,另一方面由于居民消费水平极低(由于气候等),这一点能够得到补偿,例如在印度,这两种情况都存在。在社会的原始状态中,由于社会需要还不发展,最低限度的工资可能很少(从使用价值的数量来看),可是要花费许多劳动。但是,即使为生产最低限度的工资所必需的劳动只是中等的量,生产出来的剩余价值,虽然同工资(必要劳动时间)相比占很大的比例,就是说,即使剩余价值率很高,可是表现为使用价值,却同工资本身一样,还是极其微少的(相对地说)。

假定必要劳动时间=10小时,剩余劳动=2小时,整个工作日=12小时。如果必要劳动时间等于12小时,剩余劳动等于2+(2/5)小时,整个工作日是14+(2/5)小时,那末,生产出来的价值就大不相同。在第一种场合,生产出来的价值=12小时,在第二种场合,生产出来的价值等于14+(2/5)小时。剩余价值的绝对量也大不相同。在第一种场合,剩余价值等于2小时,在另一种场合,等于2+(2/5)小时。可是,剩余价值率,或者说,剩余劳动率,是相同的,因为2∶10=[2+(2/5)]∶12。如果在第二种场合花费的可变资本更多,那末它占有的剩余价值或剩余劳动也会更多。如果在后一种场合剩余劳动不是增加2/5小时,而是5/5小时,从而剩余劳动就等于3小时,而总工作日就等于15小时,那末,虽然必要劳动时间,或者说,最低限度的工资增加了,剩余价值率还是会提高(因为2∶10=1/5,而3∶12=1/4)。如果由于谷物等涨价而最低限度的工资从10小时增加到12小时,这两种情况就可能同时发生。可见,即使在这种场合,剩余价值率不仅能够保持不变,而且还能够和剩余价值量一起增长。

但是,我们假定,必要工资和以前一样等于10小时,剩余劳动等于2小时,其他一切条件不变(因此,这里完全不考虑不变资本的生产费用的减少)。如果工人现在多劳动2+(2/5)小时,其中2小时他自己占有,2/5小时成为剩余劳动,那末,在这种情况下,无论工资还是剩余价值都会同样增加,但是前者代表的量大于必要工资,或者说,必要劳动时间。

如果我们取一个既定量,把它分成两个部分,那末,很清楚,其中一个部分只有在另一个部分减少的情况下才能增加,反过来也是一样。但是,如果这是增长量(变量),情况就决不是这样。而工作日(在没有争得正常工作日以前)正是这样的一个增长量。如果是这种增长量,那末,两个部分都可以增长,或者以同样程度增长,或者以不同程度增长。一个部分的增长不是以另一个部分的减少为条件,反过来也是一样。这也就是工资和剩余价值两者就其交换价值来看能够同时增长,而在一定条件下甚至可能以同一程度增长的唯一情况。至于它们的使用价值,那是不言而喻的;[657]虽然劳动的价值,比如说,减少了,使用价值也可能增加。在1797年到1815年间,英国的谷物价格和名义工资都大大提高,那时,在正处于迅猛发展阶段的主要工业部门中,工作日长度也大大增加,我认为,这种情况阻止了利润率的降低(因为它阻止了剩余价值率的降低)。但是,在这种场合,正常工作日不管怎样都延长了,工人的正常寿命,也就是说,他的劳动能力的正常期限,也相应地缩短了。如果工作日的延长是经常的,就必然产生这种结果。如果这种延长只是暂时的,只是为了补偿暂时的工资涨价,那末,它除了在按劳动性质可能延长劳动时间的企业中阻止利润率降低以外,也可能不产生(妇女和儿童劳动除外)其他后果。(在农业中这种情况最少发生。)

李嘉图完全没有注意到这一点,因为他既不研究剩余价值的起源,也不研究绝对剩余价值,因而把工作日看作某种既定的量。可见,对于上述的情况,他所说的剩余价值和工资(他错误地说成利润和工资)就交换价值来看只能按反比例增加或减少这个规律是错误的。

我们假定,必要劳动时间不变,剩余劳动也不变。因而得出10+2;工作日=12小时,剩余价值=2小时;剩余价值率=1/5。

现在假定,必要劳动时间仍然不变;而剩余劳动从2小时增加到4小时。工作日便是10+4,即14小时;剩余价值=4小时;剩余价值率=4∶10=4/10=2/5。

在两种场合,必要劳动时间是一样的;但剩余价值在一种场合比另一种场合多一倍,工作日在第二种场合比第一种场合大六分之一。其次,虽然工资相等,生产出来的价值,根据各自耗费的劳动量,却大不相同;在第一种场合生产出来的价值等于12小时,在另一种场合=12+12/6=14。因此,那种认为假定工资相等(就价值来说,就必要劳动时间来说),两个商品所包含的剩余价值之比就等于两个商品所包含的劳动量之比的说法,是错误的。只有在正常工作日不变的情况下,这种说法才是正确的。

其次,假定由于劳动生产力提高,必要工资从10小时减到9小时(虽然它表现为所购买的使用价值仍然不变),剩余劳动时间也从2小时减到1+(4/5)小时(即9/5小时)。在这种情况下,10∶9=2∶[1+(4/5)]。因此,剩余劳动时间和必要劳动时间以同一比例减少。在两种场合,剩余价值率是一样的,因为2=10/5,1+(4/5)=9/5,而[1+(4/5)]∶9=2∶10。根据假定,用剩余价值可以买到的使用价值的量仍然不变。(但是,这一点只适用于作为必要生活资料的使用价值。)工作日从12小时减到10+(4/5)小时。在第二种场合生产出来的价值量小于第一种场合。尽管劳动量不等,剩余价值率在两种场合却是一样的。

在考察剩余价值时,我们把剩余价值和剩余价值率区别开来了。就一个工作日来看,剩余价值等于它所代表的绝对时数,比如说,2、3小时等。剩余价值率等于这一时数和构成必要劳动时间的时数之比。这个区别非常重要,因为它指明了工作日的不同长度。如果剩余价值等于2小时,那末,必要劳动时间为10小时,它就等于1/5,必要劳动时间为12小时,它就等于。在一种场合工作日是12小时,在另一种场合工作日是14小时。在第一种场合剩余价值率较大,而工人一天劳动的时数较少。在第二种场合剩余价值率较小,劳动力的价值较大,而工人一天劳动的时数较多。这里我们看到,在剩余价值不变(但工作日不等)时,剩余价值率可能不同。而在前面10∶2和9∶[1+(4/5)]的情况下,我们看到,在剩余价值率不变(但工作日不等)时,剩余价值本身可能不同(在一种情况下是2,在另一种情况下是1+(4/5))。

我在前面(第二章)曾经指出,在工作日既定(工作日的长度既定),必要劳动时间既定,因而剩余价值率既定的条件下,剩余价值量取决于由同一资本同时雇用的工人人数。\endnote{马克思指的是他的手稿第III本从第95b页上开始的一节,标题是《(2)绝对剩余价值》。马克思引的一段在这一节的(d)小节内,标题是“同一时间的工作日”(马克思手稿第102—104页)。——第466页。}这个论点本来是同义反复。因为,如果1工作日给我提供2小时的剩余劳动,那末,12工作日就给我提供24小时的剩余劳动,或者说,提供2剩余工作日。可是,在决定利润(利润等于剩余价值同预付资本之比,因而它取决于剩余价值的绝对量)时,这个论点具有极其重要的意义。这个论点所以具有重要意义,是因为量相等而有机构成不同的资本,使用的工人人数不等,因而生产的剩余价值就一定不等,也就是说,生产的利润就一定不等。在剩余价值率降低时利润可能提高,而在剩余价值率提高时利润可能降低,或者,如果剩余价值率的提高或降低由使用的工人人数的相反运动所抵销,利润可能不变。这里,我们一开始就看到,把剩余价值提高和降低的规律[658]与利润提高和降低的规律等同起来是极端错误的。如果仅仅考察单纯的剩余价值规律,那末,说在剩余价值率既定(以及工作日既定)时,剩余价值的绝对量取决于所使用的资本量,这似乎是同义反复。因为根据假定,这个资本量的增长和同时雇用的工人人数的增加是一回事,或者说,只是同一事实的不同表现。但是如果进而考察利润,在这里使用的总资本量和使用的工人人数对于同量资本来说是大不相同的,那末,就可以看出上述规律的重要性了。

李嘉图是从考察具有一定价值的商品即代表一定量劳动的商品出发的。而从这一点出发,绝对剩余价值和相对剩余价值似乎总是一致的。(这无论如何说明了他的考察方法的片面性,而且也符合他的整个研究方法——从由商品中包含的劳动时间决定的商品价值出发,然后研究工资、利润等在什么程度上影响这个价值。)但是,这是假象,因为这里不是单纯地谈商品,而是谈资本主义生产,谈作为资本的产物的商品。

假定有一笔资本使用一定数量的工人,比如说20人,而工资等于20镑。为了简便起见,我们假定固定资本等于零,就是说,完全不把它计算在内。假定这20个工人一天劳动12小时,把价值80镑的棉花纺成纱。如果1磅棉花值1先令,20磅棉花就值1镑,80镑=1600磅棉花。如果20个工人用12小时纺1600磅棉花,那末1小时就纺1600/12磅=133+(1/3)磅。因此,如果必要劳动时间等于10小时,剩余劳动时间就等于2小时,它提供266+(2/3)磅纱。1600磅纱的价值等于104镑;因为,如果10劳动小时=20镑,那末1劳动小时=2镑,2劳动小时=4镑;因此,12劳动小时=24镑。(80镑[原料价值]+24镑[新加劳动所创造的价值]=104镑)

但是,假定工人的剩余劳动时间等于4小时,那末他们的产品就等于8镑(我指的是工人创造的剩余价值,他们的产品实际上=28镑\endnote{马克思指由20个工人新创造的价值:这20个工人每一劳动小时创造价值2镑,14小时的工作日创造价值28镑。——第468页。})。总产品的价值是121+(1/3)镑,\endnote{总产品的价值包括转移到产品中去的价值(c)和新创造的价值(v+m)。因为马克思在这里撇开了固定资本,所以转移的价值只是原料的价值。在被考察的例子中,原料价值等于93+(1/3)镑(一小时把133+(1/3)磅棉花加工成纱,14小时加工1866+(2/3)磅;1磅棉花是1先令)。加上新创造的价值(28镑)便是121+(1/3)镑。——第468页。}而这121+(1/3)镑等于1866+(2/3)磅纱。因为生产条件不变,所以,1磅纱的价值仍然和过去一样;它包含的劳动时间仍然一样。根据假定,必要工资(它的价值、它所包含的劳动时间)也保持不变。

不论这1866+(2/3)磅纱是在第一种条件下还是在第二种条件下生产的,就是说,不论剩余劳动是2小时还是4小时,在两种情况下这些纱都具有同一价值。那就是,除了以前的1600磅棉花以外,多纺的266+(2/3)磅棉花值13+(1/3)镑。如果把它加到用于1600磅棉花的80镑上,就是93+(1/3)镑,而在两种情况下20个人的4追加劳动小时等于8镑。全部[新加]劳动就是28镑,因而1866+(2/3)磅纱的价值等于121+(1/3)镑。在两种情况下,工资都是一样的。1磅纱在两种情况下都值1+(3/10)先令。因为1磅棉花的价值=1先令,所以,在1磅纱中包含的新加劳动在两种情况下都是3/10先令,或3+(3/5)便士(或18/5便士)。

可是,在假定的条件下,每磅纱中价值和剩余价值之比是大不相同的。在第一种情况下,因为必要劳动=20镑,剩余劳动=4镑,或者说,前者=10小时,后者=2小时,剩余劳动和必要劳动之比是2∶10,或者说,1∶5。(同样可以说,4镑∶20镑=4/20=1/5。)因此,在这种情况下,[物化]在1磅纱中的[新加劳动]3+(3/5)便士中,包含着1/5无酬劳动=18/25便士,或72/25法寻,即2+(22/25)法寻。而在第二种情况下,必要劳动是20镑(10劳动小时),剩余劳动=8镑(4劳动小时)。剩余劳动和必要劳动之比是8∶20=8/20=4/10=2/5。因此,[物化]在1磅纱中的[新加劳动]3+(3/5)便士中,无酬劳动是这一数目的2/5,即5+(19/25)法寻,或1便士1+(19/25)法寻。[659]虽然在两种情况下1磅纱具有同一价值,并且在两种情况下支付同样的工资,但是1磅纱包含的剩余价值,在一种情况下比在另一种情况下多一倍。在作为产品的一定部分的单位商品中,劳动价值和剩余价值的比例,自然应当同全部产品中的比例一样。

在一种情况下[在12小时的工作日中把1866+(2/3)磅棉花纺成纱],用于棉花的预付资本等于93+(1/3)镑,而用于工资的预付资本是多少呢?这里,用于把1600磅棉花纺成纱的工资等于20镑,因而用于把追加的266+(2/3)磅棉花纺成纱的工资等于3+(1/3)镑。因此,工资总共用了23+(1/3)镑。而全部支出等于[不变]资本93+(1/3)镑+[23+(1/3)]镑=116+(2/3)镑。产品=121+(1/3)镑。这里[可变]资本的追加支出3+(1/3)镑只提供13+(1/3)先令(或2/3镑)的剩余价值。(20镑∶4镑=[3+(1/3)]镑∶(2/3)镑。)

相反,在另一种情况下[在14小时的工作日中把1866+(2/3)磅棉花纺成纱],预付资本只有93+(1/3)镑+20镑=113+(1/3)镑,而在4镑剩余价值上又加上了4镑。在两种情况下生产出来的纱的数量一样,纱的价值一样,就是说,这两种纱代表相等的劳动总量;但是这两个相等的劳动总量,虽然工资相同,却是由大小不等的两笔资本推动的;相反,工作日的长度是不等的,因而,无酬劳动的量就不同。就单独每一磅纱来考察,花费在它上面的工资的量,或者说,它所包含的有酬劳动的量,是不等的。这里,同样多的工资分配在较大量的商品上,不是因为劳动的生产率在一种情况下比在另一种情况下高,而是因为在一种情况下被推动的无酬剩余劳动总量比在另一种情况下大。因此,用同量的有酬劳动,在一种情况下生产的纱的磅数比在另一种情况下多,虽然在两种情况下总共都生产了等量的纱,两种等量的纱代表等量的总劳动(有酬劳动和无酬劳动)。相反,如果在第二种情况下劳动生产率提高的话,那末,无论如何(不论剩余价值同可变资本的比例怎样)1磅纱的价值都要下降。

因此,在这种情况下,如果说,因为1磅纱的价值是既定的,等于1先令3+(3/5)便士,其次,新加劳动的价值是既定的,等于3+(3/5)便士,并且因为根据假定,工资是相同的,即必要劳动时间不变,所以剩余价值一定会相同,两笔资本在其他条件相同的情况下生产出来的纱一定会带来相同的利润,那末,这种说法是错误的。如果谈的是1磅纱,那倒是对的,但这里谈的是生产了1866+(2/3)磅纱的一笔资本。为了要知道这笔资本从1磅纱中得到多大的利润(其实是剩余价值),我们必须知道工作日有多长,或者说,这笔资本(在生产率既定的情况下)推动的无酬劳动的量有多大。但是这从单位商品上是看不出来的。

可见,李嘉图只是研究了我称为相对剩余价值的东西。他是从工作日长度既定这一前提出发的(斯密和他的前辈似乎也是从这一前提出发的)。(至多,在斯密著作中提到过不同的劳动部门中工作日长度的差别,而这种差别已由劳动的较大强度、困难、使人厌恶的性质等抵销或补偿。)从这个前提出发,李嘉图总的说来正确地阐明了相对剩余价值。但是在谈到他的研究的主要论点之前,我们还要引几段引文来说明李嘉图的观点。

\begin{quote}{“工业中100万人的劳动总是生产出相同的价值,但并非总是生产出相同的财富。”(同上,第320页)}\end{quote}

这就是说,他们一天劳动的产品总是100万工作日的产品,包含同一劳动时间,而这种说法是错误的,或者,只有在考虑到不同劳动部门的不同困难程度等情况而普遍确立同一正常工作日的时候才可能是正确的。

可是,即使在这样的时候,在这里用一般形式表述出来的这个论点还是错误的。假定正常工作日等于12小时。假定一个人的年劳动产品用货币表示等于50镑,而且货币的价值不变。在这种情况下,100万人的劳动产品一年总是等于5000万镑。假定必要劳动等于6小时,那末用于这100万人的资本一年就等于2500万镑。剩余价值也等于2500万镑。不管工人得到2500万,3000万还是4000万,产品总是等于5000万。可是剩余价值在第一种情况下等于2500万,在第二种情况下等于2000万,在第三种情况下等于1000万。如果预付资本仅仅由可变资本组成,就是说,仅仅由用于这100万人的工资的资本组成,那末,李嘉图就对了。因此,他只有在一种情况下,即在全部资本等于可变资本的情况下才是对的,——在李嘉图的著作中就象在亚·斯密的著作中一样,这个前提贯穿着全部研究,[660]只要他谈的是整个社会的资本;但是,在资本主义生产条件下,无论在哪一个生产部门中,尤其在整个社会生产中,这种情况是不存在的。

加入劳动过程但不加入价值形成过程的那一部分不变资本,不加入产品(产品的价值),因此,在这里,当我们讲的是年产品的价值的时候,无论这一部分不变资本对决定一般利润率多么重要,它都是同我们无关的。加入年产品的那一部分不变资本,却是另外一种情况。我们已经看到,这部分不变资本中的一部分,或者说,在一个生产领域表现为不变资本的东西,在另一个生产领域的同一个生产年度却表现为劳动的直接产品;因而,一年花费的资本中有很大一部分,从单个资本家或特殊生产领域的角度来看表现为不变资本,而从整个社会或整个资本家阶级的角度来看却归结为可变资本。因此,这一部分是包含在前面所说的5000万之内的,是包含在5000万中构成可变资本或用于工资的部分之内的。

但是,对于为了补偿工农业中已消费的不变资本而消费的那一部分不变资本,对于生产不变资本——最初形式的原料、固定资本和辅助材料——的生产部门使用的不变资本已消费的部分,却是另外一种情况。这一部分不变资本的价值会在产品中重新表现出来,会在产品中被再生产出来。这一部分的价值以什么比例加入全部产品的价值,完全取决于它现有的量(假定劳动生产率保持不变;但是不管劳动生产率怎样变动,这一部分的价值总是具有一定的量)。(如果不把农业中的某些例外计算在内,平均说来,就连产品的量,即100万人生产出来的、李嘉图认为和价值不同的那个财富的量,当然也取决于这个作为生产前提的不变资本的量。)如果没有100万人的新的年劳动,产品的这一部分价值就不会存在。另一方面,如果没有这个不以他们的年劳动为转移而存在的不变资本,100万人的劳动就不能提供同一产品量。这个不变资本作为生产条件加入劳动过程,但是为了把全部年产品的这一部分价值再生产出来,无须再花费哪怕是一小时的劳动。因此,作为价值,这一部分不是年劳动的结果,虽然没有这一年劳动它的价值就不能在产品中再生产出来。

假定加入产品的那部分不变资本等于2500万,那末,100万人的产品的价值就等于7500万;如果前者等于1000万,后者就只等于6000万,依此类推。因为在资本主义发展过程中,不变资本对可变资本的比例在增长,所以100万人的年产品的价值,就有同作为因素参加100万人一年生产活动的过去劳动的增长成比例地不断增长的趋势。从这里已经可以看到,李嘉图既不能理解积累的实质,也不能理解利润的本质。

随着不变资本对可变资本的比例的增长,劳动生产率也增长,由人生产出来的、社会劳动借以发挥作用的生产力也增长。诚然,由于劳动生产率的这一增长,现有不变资本的一部分将不断贬值,因为它的价值不是决定于它原先已花费的劳动时间,而是决定于把它再生产出来所必须花费的劳动时间,而这种劳动时间随着劳动生产率的增长会不断减少。因此,不变资本的价值虽然不是同它的量成比例地增长,但毕竟是在增长,因为不变资本的量的增长比它的价值的减少快。不过,关于李嘉图的积累观点,我们到后面再谈。

无论如何,这里已经很明显,在工作日既定的情况下,100万人年劳动的全部产品的价值,将根据加入产品的不变资本的量的不同而大不相同,尽管劳动生产率在增长,这个价值在不变资本构成总资本的很大部分的地方,比在不变资本构成总资本的较小部分的社会条件下要大。因此,随着社会劳动生产率的进步以及实际上与它同时发生的不变资本的增长,资本本身的份额,在劳动的全部年产品中相对说来将占越来越大的部分,因而作为资本的财产(且不说资本家的收入)将不断增大,单个工人甚至整个工人阶级[的新加劳动]所创造的那部分价值所占的份额,[661]与现在作为资本同他们对立的他们的过去劳动的产品相比,将越来越减少。因此,劳动能力和作为资本而独立存在的劳动的客观条件之间的分离和对立不断增长。(我们这里不谈可变资本,即年劳动的产品中为再生产工人阶级所必需的那一部分;但是,就连工人阶级的这些生存资料本身也是作为资本同他们对立的。)

李嘉图把工作日看作既定的、有限的、固定的量的观点,在他书中的其他地方也谈到过,例如

\begin{quote}{“它们〈工资和资本的利润〉加在一起总是具有同一价值”。(同上,第499页,第32章《马尔萨斯先生的地租观点》)}\end{quote}

换言之,这只不过是说:劳动时间(工作日)——其产品在工资和资本的利润之间进行分配——总是同一的,是不变的量。

\begin{quote}{“工资和利润加在一起具有同一价值。”(第491页注)}\end{quote}

这里我无需重复,利润在这些地方都应读作剩余价值。

\begin{quote}{“工资和利润加在一起将总是同一价值。”(第490—491页)“工资应当按照它的实际价值计算,就是说,按照生产工资时使用的劳动和资本的量计算,不应按照它用衣服、帽子、货币或谷物来表示的名义价值计算。”(第1章《论价值》,第50页)}\end{quote}

工人得到的(他用自己的工资购买的)生活资料(谷物、衣服等)的价值,决定于生产它们所需要的总劳动时间,这里既包括生产它们所必需的直接劳动量,也包括生产它们所必需的物化劳动量。但是,李嘉图把问题搞糊涂了,因为他没有把问题表达清楚,他不是说:“它的(工资的)实际价值,即工作日中为再生产他(工人)自己的必要生活资料的价值、为再生产以工资形式支付给他,或者说,用以交换他的劳动的必要生活资料的等价所需要的那一部分”。“实际工资”应由工人为生产或再生产他自己的工资在一天中必须劳动的平均时间来决定。[而李嘉图做了这样的表述:]

\begin{quote}{“工人只有在用他的工资能买到大量劳动的产品时,他的劳动才是得到真正高的价格。”(第322页)}\end{quote}

\tsubsectionnonum{(4)相对剩余价值。[对相对工资的分析是李嘉图的科学功绩]}

相对剩余价值——这实际上是李嘉图在利润名义下研究的剩余价值的唯一形式。

为生产商品所需要的并包含在商品中的劳动量,决定商品的价值,因而商品的价值是一个既定的、一定的量。这个量在雇佣工人和资本家之间分配。(李嘉图同斯密一样在这里没有考虑不变资本。)很明显,一个分配者的份额的增加或减少,只能同另一个分配者的份额的减少或增加成比例。既然商品的价值全靠工人的劳动来创造,那末在任何情况下都要有这种劳动本身作为前提,但是工人必须活着,维持自己的生命,也就是说,必须得到必要工资(与劳动能力价值相等的最低限度的工资),否则就不可能有这种劳动。因此,工资和剩余价值——照李嘉图看来,商品的价值或产品本身分成这两个范畴——不仅彼此成反比,而且最初的、决定性的因素是工资的变动。工资的提高或降低引起利润(剩余价值)方面的相反的运动。工资提高或降低,不是因为利润(剩余价值)降低或提高,相反,因为工资提高或降低,剩余价值(利润)才降低或提高。工人阶级从自己的劳动所创造的年产品中取得他自己的份额以后剩下的剩余产品(其实应当说剩余价值),成为资本家阶级赖以生活的实体。

既然商品的价值决定于商品包含的劳动量,而工资和剩余价值(利润)只不过是两个生产者阶级彼此之间分配商品价值的份额,比例,那末,很明显,工资的提高或降低虽然决定剩余价值率(在李嘉图那里是利润率),但是并不影响商品的价值或价格(商品价值的货币表现)。一个整体在两个分配者之间进行分配的比例,既不会使这个整体本身变大,也不会使它变小。因此,认为工资的提高会提高商品的价格的看法,是一种错误的成见;工资提高只能使利润(剩余价值)降低。甚至李嘉图所举的一些例外情况即工资提高似乎会引起一些商品的交换价值降低,并引起另一些商品的交换价值提高——如果说的是价值,那也是错误的,只有对费用价格来说才是正确的。

[662]既然剩余价值(利润)率决定于工资的相对高度,那末后者又是由什么决定的呢?如果撇开竞争不谈,工资决定于必要生活资料的价格。必要生活资料的价格又取决于劳动生产率,而土地越肥沃,劳动生产率就越高(这里,李嘉图假定是资本主义生产)。每一种“改良”都使商品、生活资料的价格降低。因此,工资,或者说,“劳动的价值”的提高或降低同劳动生产力的发展成反比,只要这一劳动所生产的是加入工人阶级日常消费的必要生活资料。因此,剩余价值(利润)率的降低或提高同劳动生产力的发展成正比,因为这种发展使工资降低或提高。

工资不提高,利润(剩余价值)率就不可能降低;工资不降低,利润率就不可能提高。

工资的价值不是按照工人得到的生活资料的量来计算的,而是按照这些生活资料所耗费的劳动量(实际上就是工人自己占有的那部分工作日),按照工人从总产品中,或者更确切地说,从这个产品的总价值中得到的比例部分来计算的。可能有这种情况,工人的工资用使用价值(一定量的商品或货币)来衡量,是提高了(在劳动生产率提高的情况下),可是按价值却降低了,也可能有相反的情况。分析相对工资,或者说,比例工资,并把它作为范畴确定下来,是李嘉图的巨大功绩之一。在李嘉图以前,始终只对工资作了简单的考察,因而工人被看作牲畜。而这里工人是被放在他的社会关系中来考察的。阶级和阶级相互之间的状况,与其说决定于工资的绝对量,不如说更多地决定于比例工资。

现在从李嘉图著作中引几段话,以证实前面所表述的论点。

\begin{quote}{“猎人一天劳动的产品鹿的价值恰好等于渔夫一天劳动的产品鱼的价值。不管产量多少,也不管普通工资或利润的高低,鱼和野味的比较价值完全由它们自身包含的劳动量决定。如果……渔夫……雇用10个人,他们的年劳动值100镑,他们劳动一天可捕得鲑鱼20条;如果……猎人也雇用10个人,他们的年劳动值100镑,他们一天为他捕鹿10只;那末,不论全部产品中归捕获者的份额是多少,一只鹿的自然价格是两条鲑鱼。用来支付工资的份额对利润问题是极为重要的;因为一眼就可以看出,利润的高低恰好同工资的高低成反比;但是这丝毫不会影响鱼和野味的相对价值,因为在这两个行业中,工资要高就会同时都高,要低就会同时都低。”(第1章《论价值》,第20—21页)}\end{quote}

我们看到,李嘉图从被雇用者的劳动中得出商品的全部价值。在被雇用者和资本之间分配的,就是被雇用者自己的劳动,或者说,这一劳动的产品,或者说,这一产品的价值。

\begin{quote}{“工资的任何变动不可能引起这些商品的相对价值的变动,因为,假定工资提高,这两个行业中的任何一个并不因此就需要更大的劳动量,虽然对这一劳动将支付更高的价格……工资可能提高百分之二十,因此利润会以或大或小的幅度降低,但这决不会使这些商品的相对价值发生丝毫变动。”(同上,第23页)“劳动的价值提高,利润就不能不降低。如果把谷物在租地农场主和工人之间分配,后者得到的份额越大,留给前者的就越小。同样,如果把呢绒和棉织品在工人和雇主之间分配,分给前者的份额越大,留给后者的就越小。”(同上,第31页)[663]“亚当·斯密和一切追随他的著作家,据我所知,无一例外地都认为,劳动价格的上涨,必然会引起一切商品价格的上涨。我希望,我已成功地证明了这种意见是毫无根据的。”(同上,第45页)“工资的提高,如果是由于工人得到比较优厚的报酬,或者由于那些用工资购买的必需品的生产发生困难,那末,除了某些情况以外,不会引起价格的提高,但对于利润降低却有很大的影响。”如果工资的提高是由“货币价值的变动”引起的,那是另一回事。“在一种场合{即上述的后一种场合},国家的年劳动中并没有花费更大的份额来维持工人生活,在另一种场合,却花费了更大的份额。”(同上,第48页)[663]“随着食物和其他必需品价格的上涨,劳动的自然价格也上涨;随着这些东西的价格下降,劳动的自然价格也下降。”(同上,第86页)“现有人口的需要满足之后剩下来的剩余产品,必然同生产的容易程度成比例,也就是说,从事生产的人数越少,剩余产品就越多。”(第93页)“不论是耕种调节价格的那一等级土地的租地农场主,还是生产工业品的工厂主,都没有牺牲自己产品的任何部分来支付地租。他们的商品的全部价值只分成两部分:一部分构成资本的利润,另一部分构成工资。”(第107页)“假定丝绸、天鹅绒、家具以及其他任何不是工人需要的商品由于所费劳动增加而涨价,这会不会影响利润呢?当然不会。因为只有工资提高才能影响利润;丝绸和天鹅绒不为工人所消费,所以它们价格的上涨就不能提高工资。”(第118页)“如果10个工人的劳动在一定质量的土地上可以获得小麦180夸特,每夸特价值4镑,共计720镑(第110页)……在任何情况下,这720镑都必定分成工资和利润……不论工资或利润是提高还是降低,这两者都必定由720镑这个总额中提供。一方面,利润决不能提高到从这720镑中取出那样大一部分,以致余数不足以给工人提供绝对必需品;另一方面,工资决不能提高到使这个总额不剩下一部分作为利润。”(第113页)“利润取决于工资的高低,工资取决于必要生活资料的价格,而必要生活资料的价格主要取决于食物的价格,因为其他一切必需品的数量是可以几乎无止境地增加的。”(第119页)“虽然生产了一个较大的价值〈在土地变坏的情况下〉,但这一价值在支付地租以后剩下的部分中却有较大的份额是由生产者消费的{李嘉图在这里把工人和生产者等同起来了},而这一点,并且只有这一点,却调节着利润的大小。”(第127页)“改良的实质就是使生产商品所需要的劳动量比以前减少;而这种减少不能不使商品的价格,或者说,相对价值下降。”(第70页)“如果减少帽子的生产费用,尽管对帽子的需求增加一倍、两倍或者三倍,帽子的价格最后总要降到其新的自然价格的水平。如果降低维持人的生活的食物和衣服的自然价格,从而减少人所必需的生存资料的生产费用,尽管对工人的需求[664]可能大大增加,工资最后还是会降低。”(第460页)“工资分得的份额越小,利润分得的份额就越大,反过来也是一样。”(第500页)“本书的目的之一就是要说明,必需品的实际价值每有降低,工资也就降低,而资本利润则提高;换句话说,在任何一定的年价值中,归工人阶级所得的份额会减少,而归用基金使用这个阶级的人所得的份额会增加。”}\end{quote}

{只是在最后这句非常通俗的话里,李嘉图即使没有猜到,但毕竟说出了资本的本质。不是积累的劳动被工人阶级,被工人自己使用,而是“基金”,“积累的劳动”“使用这个阶级”,使用现在的、直接的劳动。}

\begin{quote}{“假定某工厂生产的商品价值为1000镑,这一价值在老板和他的工人之间分配{这句话又反映了资本的本质;资本家是老板,工人是他的工人},工人得800镑,老板得200镑;如果这些商品的价值降到900镑,同时由于必需品降价在工资上节省了100镑,那末,老板的纯收入丝毫不会减少。”(第511—512页)“如果由于机器改良,生产供工人穿的鞋子和衣服所需要的劳动量只等于现在的四分之一,那末这些东西的价格也许会降低75%;但是,绝不能由此得出结论说,工人因此就可以不再只消费一件上衣和一双鞋子,而可以经常消费四件上衣和四双鞋子了;由于竞争的影响和人口急剧增加的刺激,他的工资也许不久就会同用工资购买的各种必需品的新价值相适应。如果这种改良推广到工人的一切消费品,大概过了几年以后我们就会看到,虽然与任何其他商品相比这些商品的交换价值已经大大降低,虽然这些商品现在是已经大大减少了的劳动量的产品,但工人的消费量即使有所增加也是十分有限的。”(第8页)“工资的增加,总是靠减少利润,工资降低时,利润总是提高。”(第491页注)“在本书中,我始终力图证明:工资不降低,利润率就决不会提高;用工资购买的各种必需品不跌价,工资就不能持久降低。因此,如果由于对外贸易的扩大或机器的改良,工人消费的食物和其他必需品能按较低廉的价格进入市场,利润就会提高。如果我们不自己种植谷物,不自己制造工人所用的衣服和其他必需品,而是发现一个新的市场,可以用较低廉的价格从那里取得这些商品,工资就会降低,利润就会提高;但是,如果由于对外贸易的扩大或机器的改良而以较低廉的价格取得的商品仅仅是供富人消费的商品,利润率就不会发生任何变动。葡萄酒、天鹅绒、丝绸及其他贵重商品的价格即使降低50%,工资率也不会受到影响,因而利润也会保持不变。所以,对外贸易虽然对国家极为有利,因为它增加了用收入购买的物品的数量和种类,并且由于商品丰富和价格低廉而为节约和资本积累提供刺激{为什么不是为浪费提供刺激?},但是,如果进口的商品不属于用工人工资购买的那一类商品,就根本没有提高资本利润的趋势。以上关于对外贸易的看法同样适用于国内贸易。利润率决不会由于分工的改进、机器的发明、道路和运河的兴修或者在商品制造或运输上采用任何其他节约劳动的方法而提高。”}\end{quote}

{李嘉图刚刚讲过完全相反的话;他的意思显然是说,除非由于上述改良减少了劳动的价值,否则利润率决不会提高。}

\begin{quote}{“所有这些原因都影响商品价格,总是对消费者极为有利,因为它们使消费者能够用同样的劳动换得更多的在生产上实行了改良的商品;但是它们对于利润绝对没有任何影响。另一方面,[665]工人工资的任何降低都使利润提高,但是对于商品价格毫无影响。前一种情况对一切阶级都有利,因为一切阶级都是消费者”}\end{quote}

{但是,这怎么会对工人阶级有利呢?因为根据李嘉图的假定,如果这些商品属于用工资购买的物品,它们会使工资降低,如果它们的减价不会使工资降低,它们就不属于用工资购买的物品};

\begin{quote}{“后一种情况只对生产者有利;他们会得到更多的利润,但一切物品的价格仍旧不变”}\end{quote}

{这又怎么可能呢?因为根据李嘉图的假定,使利润提高的工人工资的降低,正是因为必要生活资料价格降低才发生的,因此决不能说“一切物品的价格仍旧不变”}。

\begin{quote}{“在前一种情况下,他们得到的数额同以前一样,但是用他们的所得来购买的一切物品〈这又错了;应该说除了必要生活资料以外的一切物品〉的交换价值减少了。”(第137—138页)}\end{quote}

我们看到,整个这一段都写得极为草率。但是,撇开这些形式上的缺点不谈,这一切,就象在整个关于相对剩余价值的这种研究中一样,只有在把“利润率”读成“剩余价值率”的情况下才是正确的。即使对奢侈品来说,上述改良也可以提高一般利润率,因为这些生产领域的利润率,同其他一切生产领域的利润率一样,也参加一切特殊利润率平均化为平均利润率的过程。如果在这种情况下,由于上述种种影响,不变资本的价值同可变资本相比降低了,或者周转时间的长度缩短了(就是说,流通过程有了变化),那末,利润率就会提高。其次,李嘉图在这里对于对外贸易的影响作了非常片面的解释。对于资本主义生产来说,非常重要的是产品发展成为商品,而这同市场的扩大,同世界市场的建立,因而同对外贸易,有极为重要的联系。

如果撇开这一点不谈,李嘉图倒是提出了一个正确的原理,就是说:一切不论是由分工、机器的改进、运输工具的完善还是由对外贸易引起的改良,一句话,一切缩短制造和运输商品的必要劳动时间的方法,由于并且只要它们降低劳动的价值,都会增加剩余价值(就是说,也会增加利润),从而使资本家阶级发财致富。

最后,我们在这一节里还必须引用李嘉图阐明相对工资的本质的几段话。

\begin{quote}{“如果我必须雇用一个工人劳动一星期,我不是付给他10先令而是付给他8先令,而货币的价值并没有发生任何变动,这个工人现在用8先令买到的食物和其他必需品,可能比以前用10先令买到的还多。但是,这不是象亚·斯密和最近马尔萨斯先生所说的那样,由于他的工资的实际价值提高了,而是由于工人用他的工资购买的那些物品的价值降低了,这是完全不同的两回事。但是,当我把这叫做工资的实际价值降低时,有人却说我使用了同这门科学的真正原理不相容的新奇说法。”(同上,第11—12页)“要正确地判断利润率、地租率和工资率,我们不应当根据任何一个阶级所获得的产品的绝对量,而应当根据获得这一产品所需要的劳动量。由于机器和农业的改良,全部产品可能加倍;但是,如果工资、地租和利润也增加一倍,那末三者之间的比例仍然和以前一样,其中任何一项也不能说有了相对的变动。但是,如果工资没有如数增加,如果它不是增加一倍而只增加一半……那末,在这种情况下,我认为,说……工资已经降低而利润已经提高,那是对的;因为,如果我们有一个衡量产品价值的不变的标准,我们就会发现,现在归工人阶级……所得的价值比以前少了,而归资本家阶级所得的价值比以前多了。”(第49页)“工资的这种降低,仍然是真正的降低,尽管它〈工资〉现在能够为工人提供廉价商品的量比他以前的工资所提供的还多。”(第51页)}\end{quote}

\centerbox{※     ※     ※}

德·昆西指出了李嘉图所发挥的一些论点,并把它们同其他经济学家的观点作了对照。

\begin{quote}{李嘉图以前的经济学家的情况是这样的:“当有人问他们究竟是什么决定一切商品的价值的时候,回答是:价值主要由工资决定。再问:究竟是什么决定工资?他们就会指出,工资必须同用它购买的商品的价值相适应;这个回答实际上就是说,工资由商品的价值决定。”(《三位法学家关于政治经济学的对话,主要是关于李嘉图先生的〈原理〉》,[666]载于1824年《伦敦杂志》第9卷第560页)}\end{quote}

就在这个《对话》中,谈到用劳动量衡量价值的规律和用劳动价值衡量价值的规律:

\begin{quote}{“这两个公式决不能认为仅仅是同一规律的两个不同表现,李嘉图先生的规律(即关于A的价值和B的价值之比等于生产它们的劳动量之比的论点)用否定式来表达,最好是说:A的价值和B的价值之比不等于生产A的劳动的价值和生产B的劳动的价值之比。”[同上,第348页]}\end{quote}

(如果在A和B两个部门中资本的有机构成相同,那末,确实可以说,这两种资本的产品价值之比等于生产这两种产品的劳动的价值之比。因为在这种情况下,产品A和产品B中包含的积累劳动量之比等于这两种产品包含的直接劳动量之比。两个部门的有酬劳动量之比等于这两种资本所使用的直接劳动总量之比。假定两种资本的构成都是80c+20v,剩余价值率都等于50%。如果一笔资本等于500,而另一笔等于300,那末前者的产品等于550,而后者的产品等于330。于是,两种产品之比也等于工资5×20(即100)和工资3×20(即60)之比。因为100∶60=10∶6=5∶3。产品A和产品B的价值之比是550∶330,或55∶33,或(55/11)∶(33/11)(因为5×11=55,3×11=33),因而等于5∶3。但是,即使在这种情况下,也只是知道它们之间的比例,而不知道所考察的两种产品的实际价值,因为各种极不相同的价值量都可以符合于5∶3这一比例。)

\begin{quote}{“如果产品的价格为10先令,那末工资和利润加在一起就不能超过10先令。但是,难道不是恰好相反,是工资加利润决定价格吗?不,那是陈旧的、过时的学说。”(托·德·昆西《政治经济学逻辑》1844年爱丁堡和伦敦版第204页)“新的政治经济学证明,任何商品的价格都由、并且仅仅由生产该商品的劳动的相对量决定。既然价格本身已经决定,价格也就决定那个无论工资还是利润都必须从中取得自己的特殊份额的基金。”(同上)“凡是可能破坏工资和利润之间的现有比例的变动,必定从工资中发生。”(同上,第205页)“李嘉图给地租学说增添了新的东西:他把地租学说归结为地租是否真的取消了价值规律的问题。”(同上,第158页)}\end{quote}

\tchapternonum{[第十六章]李嘉图的利润理论}

\tsectionnonum{[(1)李嘉图把利润和剩余价值区别开来的个别场合]}

已经详细证明:剩余价值规律,或者更确切地说,剩余价值率规律(假定工作日既定),不是象李嘉图所解释的那样,直接地、简单地同利润规律相一致,或者说,可以直接地、简单地适用于利润规律;李嘉图错误地把剩余价值和利润等同起来;只有在全部资本都由可变资本组成,或者说,全部资本都直接用于工资的场合,剩余价值和利润才是等同的;因此,李嘉图在“利润”名义下考察的,一般说来只是剩余价值。也只有在上述这种场合,总产品才会简单地归结为工资和剩余价值。李嘉图显然同意斯密关于年产品的总价值归结为收入的观点。因此,他也就把价值和费用价格混淆起来了。

这里没有必要重复说,利润率不是由支配剩余价值率的那些规律直接支配的。

第一,我们已经看到,利润率可能由于地租的降低或提高而提高或降低,同劳动价值的任何变动无关。

第二,利润的绝对量等于剩余价值的绝对量。但是,后者不仅决定于剩余价值率,而且决定于所使用的工人人数。因此在剩余价值率降低而工人人数增加的情况下,利润量可能不变,反过来也是一样。

第三,在剩余价值率既定的条件下,利润率取决于资本的有机构成。

第四,在剩余价值既定(从而假定每100单位的资本的有机构成也既定)的条件下,利润率取决于资本的不同部分的价值比例,这些不同部分可能由于不同原因而发生变动:部分地由于使用生产条件时节省了力等等;部分地由于价值变动,这种价值变动可能影响资本的一部分而不影响资本的其他部分。

最后,还要考虑到从流通过程产生的资本构成的差别。

[667]从李嘉图著作中已经隐约透露出来的一些想法,本来应该促使他把剩余价值和利润区别开来。由于他没有这样作,看来,——正如在分析第一章(《论价值》)时已经指出的,——他在有些地方就滑到认为利润只是商品价值的附加额这样一种庸俗观点上去了;例如,他在谈到固定资本占优势的资本的利润如何决定等等的时候就是如此\authornote{见本册第199—200页。——编者注}。他的追随者们的极其荒谬的言论,就是从这里产生的。说平均起来等量资本提供等量利润,或者说,利润取决于所使用的资本量,这个论点实际上是正确的,但是,如果不用一系列中介环节把它同一般价值规律等联系起来,简而言之,如果把利润和剩余价值等同起来(这只有对全部资本来说才是正确的),那末,就必然会产生庸俗观点。正因为如此,李嘉图才没有找到确定一般利润率的任何途径。

李嘉图懂得,商品价值的变动如果象货币价值的变动那样以同一程度影响资本的一切部分,这种变动就不影响利润率。李嘉图本来应该由此得出结论说,商品价值的变动如果不是以同一程度影响资本的一切部分,这种变动就影响利润率;因此,在劳动价值不变的情况下,利润率可能变动,甚至可能朝着同劳动价值变动相反的方向变动。但是,首先他必须注意到,剩余产品,或者在他看来也可以说,剩余价值,或者他还可以说,剩余劳动,只要他是从利润角度来考察的,他就不能单单按它同可变资本的比例来计算,而要按它同全部预付资本的比例来计算。

关于货币价值的变动,他说:

\begin{quote}{“不论货币价值的变动有多大,它都不会引起利润率的任何变动;因为,假定工厂主的商品价格从1000镑上涨到2000镑,即涨价100%,如果他的资本(货币价值的变动对他的资本和对产品的价值所起的影响是一样的),如果他的机器、建筑物和商品储备同样涨价100%,他的利润率将照旧不变……如果工厂主用一定价值的资本,通过节约劳动的办法,能使产品数量增加一倍,而产品价格下跌到原先价格的一半,那末产品同生产产品的资本的比例将照旧不变,因而利润率也将照旧不变。如果在他用同一资本使产品量增加一倍的同时,货币价值由于某种原因降低一半,产品就将按两倍于以前的货币价值出卖;但是用来生产这种产品的资本也将具有两倍于以前的货币价值;因此,在这种情况下,产品价值同资本价值的比例也将照旧不变。”(同上,第51—52页)}\end{quote}

如果李嘉图这里说的产品是指剩余产品,那是对的。因为利润率=剩余产品(剩余价值)/资本。所以,如果剩余产品=10,资本=100,那末利润率=10/100=1/10=10%。但是,如果他指的是全部产品,问题就说得不确切。那样的话,李嘉图所谓产品价值同资本价值的比例,显然是指商品价值超过预付资本价值的余额。无论如何可以看出,李嘉图在这里没有把利润同剩余价值等同起来,没有把利润率同剩余价值率(等于剩余价值/劳动价值,或者说,剩余价值/可变资本)等同起来。

李嘉图说:

\begin{quote}{“假定用来制造某些商品的原产品跌价,这些商品也将因此跌价。不错,这些商品将跌价,但是随着商品的跌价,生产者的货币收入并不会有任何减少。如果他卖出商品得到的货币减少,那只不过是因为用来制造商品的材料之一的价值已经减少。如果毛织厂主卖出呢绒不是得到1000镑,而是只得到900镑,那末,在用来纺织呢绒的羊毛的价值降低了100镑的情况下,他的收入仍然不会减少。”(同上,第32章第518页)}\end{quote}

(其实,李嘉图在这里讲的问题——原产品价值下降在某一实际场合的影响——同我们这里毫无关系。羊毛价值突然下降对于那些毛织厂主的货币收入当然有(不利)影响,因为他们在仓库里存有大批成品,这些成品是在羊毛昂贵的时候制造的,却要在羊毛价值[668]下降之后拿去出卖。)

按照李嘉图这里的假定,如果毛织厂主推动的劳动量同以前一样{其实他们可以推动更大的劳动量,因为一部分游离出来的、以前仅仅用于原料的资本,现在就可以用于原料加劳动了},那末,很明显,这些工厂主的“货币收入”按其绝对量来说“不会减少”,而他们的利润率却比以前增长了;因为同以前一样的那个量,比如说10%,即100镑,现在就不是按1000镑,而是按900镑计算。在第一种情况下利润率是10%,在第二种情况下就等于1/9,即[11+(1/9)]%。何况李嘉图还假定用来制造商品的原产品普遍跌价,那就不仅仅是某一个别生产部门的利润率,而且一般利润率都会提高。李嘉图不理解这一点是令人奇怪的,尤其是因为相反的情况他倒理解。

那就是,李嘉图在第六章(《论利润》)中考察了这样的情况:由于必需品涨价(因为较坏的土地投入耕种从而使级差地租提高),第一,工资会提高,第二,一切地面上的原产品的价格会上涨。(这个假定决不是必然的。虽然谷物涨价,棉花、丝,甚至羊毛和亚麻却完全可能跌价。)

第一,李嘉图说,租地农场主的剩余价值(即他所说的利润)将降低,因为他雇用的10个工人劳动的产品的价值仍然等于720镑,他必须从这个720镑的基金中拿出较大的一部分作为工资。李嘉图接着说:

\begin{quote}{“但是,利润率会降低得更多,因为……租地农场主的资本在很大程度上是由原产品,例如他的谷物、干草、未脱粒的小麦和大麦、马和牛等组成的,这一切都将因产品涨价而涨价。他的绝对利润将从480镑降到445镑15先令;但是,如果由于我刚才说的原因,他的资本从3000镑增加到3200镑,那末在谷物价格是5镑2先令10便士时,他的利润率将会降到14%以下。如果一个工厂主在他的企业中也投资3000镑,由于工资增加,他要继续经营这一企业,就不得不增加资本。如果他的商品以前卖720镑,现在他仍然要按同样的价格出卖;但是,工资原来是240镑,当谷物价格是5镑2先令10便士时就会增加到274镑5先令。在第一种情况下,他还剩下480镑作为3000镑资本的利润,在第二种情况下,他的资本增加了,而利润却只有445镑15先令。因此,他的利润会同租地农场主的已经变动了的利润率相一致。”(同上,第116—117页)}\end{quote}

可见,李嘉图这里把绝对利润(即剩余价值)和利润率区别开来了,并且指出,利润率由于预付资本价值变动而降低的幅度大于绝对利润(剩余价值)由于劳动价值提高而降低的幅度。这里,即使劳动价值保持不变,利润率也要降低,因为同一绝对利润要按更大的资本来计算。因此,在前面引用的他的例子即原产品价值降低的例子中,就会发生一个相反的情况,就是利润率提高(它不同于剩余价值的提高,或者说,不同于绝对利润的提高)。因此,这就说明,利润率的提高和降低,除了决定于绝对利润的增减和按用于工资的资本计算的绝对利润率的提高和降低以外,还决定于其他条件。

李嘉图在刚才引证的那个地方接着说:

\begin{quote}{“珠宝制品、铁器、银器和铜器不会涨价,因为它们的成分中没有地面上的原产品。”(同上,第117页)}\end{quote}

这些商品的价格不会上涨,但是,这些部门的利润率会高于其他部门的利润率。因为在其他部门,(由于工资增加而)减少了的剩余价值是同由于双重原因——第一,工资支出增加,第二,原料支出增加——而增大了价值的预付资本相比。在第二种情况下[即生产珠宝制品等的情况下],[669]减少了的剩余价值是同由于工资增加而只增大了可变部分的预付资本相比。

在这几个地方,李嘉图自己推翻了他的以错误地把剩余价值率和利润率等同起来作为基础的整个利润理论。

\begin{quote}{“所以,在任何情况下,如果随着原产品价格上涨工资也同时增加,农业利润和工业利润都会降低。”(第113—114页)}\end{quote}

从李嘉图自己所说的话中可以得出结论,在原产品价格上涨时,即使工资不随之增加,由于由原产品组成的那部分预付资本涨价,利润率也会降低。

\begin{quote}{“假定丝绸、天鹅绒、家具以及其他任何不是工人需要的商品由于所费劳动增加而涨价,这会不会影响利润呢?当然不会。因为只有工资提高才能影响利润;丝绸和天鹅绒不为工人所消费,所以它们价格的上涨就不能提高工资。”(同上,第118页)}\end{quote}

这些特殊部门的利润率当然要降低,尽管劳动价值——工资——保持不变。丝织厂主、钢琴厂主和家具厂主等的原料会变贵,因此保持不变的剩余价值同支出的资本的比例会降低,从而利润率会降低。而一般利润率是所有工业部门的特殊利润率的平均比率。或者,上述那些工厂主会提高他们的商品的价格,以便象以前那样获得平均利润率。价格的这种名义上的提高并不直接影响利润率,但是影响利润的支出。

李嘉图再次回到前面考察的情况:剩余价值(绝对利润)降低是因为必要生活资料的价格(以及地租)提高。

\begin{quote}{“我必须再次指出,利润率的降低比我在计算中假定的要迅速得多;因为如果产品的价值象我在前面假定的情况下说过的那样高,租地农场主的资本的价值就会大大增加,因为他的资本必须由许多价值已经增加的商品组成。在谷物价格可能从4镑上涨到12镑以前,租地农场主的资本的交换价值也许就已经增加一倍,等于6000镑而不是3000镑了。因此,如果租地农场主的利润原来是180镑,或者说,是他原有资本的6%,那末现在实际利润率不会高于3%;因为6000镑的3%是180镑,而且一个持有6000镑的新租地农场主要经营农业,就只有接受这种条件。许多行业都会从这里得到或大或小的好处。啤酒业者、烧酒业者、毛织厂主、麻织厂主减少的利润,由于他们储存的原料和成品价值提高,会得到部分的补偿;但是,金属制品、珠宝制品和其他许多商品的工厂主以及资本完全由货币组成的人,就要承担利润率降低的全部损失而得不到任何补偿。”(同上,第123—124页)}\end{quote}

这里重要的只是李嘉图所没有觉察到的一点,那就是:他推翻了自己把利润和剩余价值等同起来的观点,并且确认,不管劳动价值怎样,利润率可能受不变资本价值变动的影响。此外,他的例子只有部分是正确的。租地农场主、毛织厂主等等从他们现有的和上了市场的商品储备涨价得到的好处,到他们把这些商品一脱手,自然就没有了。一旦他们的资本消费完了,必须进行再生产了,这笔资本价值的提高就同样不再给他们带来任何好处了。到那时候,他们的处境就都和李嘉图自己提到的新的租地农场主一样,为了获得3%的利润,就不得不预付6000镑资本。相反,[XIII—670]珠宝业者、金属制品厂主、货币资本家等的损失起初虽然没有得到任何补偿,但是他们会实现高于3%的利润率,因为价值有了提高的只是他们用于工资的资本,而不是他们的不变资本。

这里,在李嘉图提到的利润降低由资本价值的提高来补偿的问题上,还有一点是重要的,就是对资本家来说,——以及一般地就年劳动产品的分配来说,——问题不仅在于产品在参与收入分配的不同人们之间进行分配,而且在于这种产品分为资本和收入。

\tsectionnonum{[(2)一般利润率(“平均利润”,或者说,“普通利润”)的形成]}

\tsubsectionnonum{[(a)事先既定的平均利润率是李嘉图利润理论的出发点]}

李嘉图的理论观点在这里决不是清楚的。

\begin{quote}{“我曾经指出,某种商品的产量可能不敷新的需求,因此它的市场价格可能超过它的自然价格,或者说,必要价格。但是,这只是暂时的现象。用来生产这种商品的资本所获得的高额利润,自然会把资本吸引到这个生产部门中来;一旦有了必要的基金,商品量有了相当的增加,商品价格就会下跌,这一生产部门的利润就会同一般水平相一致。一般利润率的降低同个别部门的利润的局部提高决不是不相容的。正由于利润不等,资本才由一个部门转移到另一个部门。因此,当一般利润由于工资提高以及向日益增长的人口供应必需品的困难增加而降低并逐渐稳定在较低的水平时,租地农场主的利润可能在一个短时间内超过原来的水平。对外贸易和殖民地贸易的个别部门在一定时间内也可能获得非常的刺激。”(第118—119页)“应当记住,市场上价格经常变动,这首先取决于供求关系。虽然呢绒可以按每码40先令的价格供应,并为资本提供普通利润,但由于时装样式改变……它可能上涨到60或80先令。毛织厂主将暂时得到非常利润,但资本将自然流入这个工业部门,直到供求再达到适当的水平为止,那时呢绒的价格将再降到40先令,也就是降到它的自然价格,或者说,必要价格。同样,每当谷物的需求增加时,其价格也可能上涨到使租地农场主的利润高于普通利润。如果肥沃的土地很多,那末,在使用了必要的资本量来生产谷物之后,谷物的价格将再降到它原来的水平,利润也将和以前一样;但是,如果肥沃的土地不多,如果为了生产追加的谷物量需要比通常更多的资本和劳动,那末谷物的价格就不会降到它原来的水平。它的自然价格就会上涨,租地农场主就不能长久地获得较高的利润,而不得不满足于降低了的利润率,这是必需品涨价使工资提高的必然结果。”(第119—120页)}\end{quote}

如果工作日既定(或者说,如果在不同生产部门只有工作日长度的差别,而这种差别又为不同种类的劳动的特点抵销),那末,一般剩余价值率,即一般剩余劳动率也是既定的,因为工资平均起来是相同的。李嘉图念念不忘这一点。所以,他把这种一般剩余价值率同一般利润率混淆起来了。我已经指出,在一般剩余价值率相同的情况下,如果商品按照各自的价值出卖,各个不同生产部门的利润率必然是完全不同的。

一般利润率是用社会的(资本家阶级的)总资本除生产出来的全部剩余价值而得出来的;因此,每一个别生产部门的每一笔资本,都表现为具有同一[671]有机构成(不论从不变资本和可变资本的构成来说,还是从流动资本和固定资本的构成来说)的总资本的相应部分。这笔资本作为这样的相应部分,按照它的量的大小,从资本总额所生产的剩余价值中获得自己的股息。这样分配的剩余价值,即在一定时期(比如说一年)内分给一定量(比如说100)的资本的一份剩余价值,就形成平均利润,或者说,形成加入每一部门的费用价格的一般利润率。如果这一份等于15,那末普通利润就是15%,费用价格=115。如果,比如说,只有一部分预付资本作为损耗加入价值形成过程,那末费用价格可能小些。但是费用价格总是等于已消费的资本加15,即加预付资本的平均利润。如果在一种情况下有100加入产品,在另一种情况下只有50加入产品,那末在一种场合费用价格等于100+15=115,而在另一种场合等于50+15=65;这样,两种资本就会按照同一费用价格,即按照为两种资本提供同一利润率的价格出卖自己的商品。显然,一般利润率的出现、实现和确立,使得价值必然转化为不同于价值的费用价格。李嘉图却相反,他假定价值和费用价格是等同的,因为他把利润率和剩余价值率混淆起来了。因此,他一点也没有想到,早在有可能谈论一般利润率以前,确立一般利润率的过程已经引起商品价格的普遍变动。他把这种利润率当作一种最初的东西,因此在他的著作里它甚至包含在价值规定中。(见第一章《论价值》。)李嘉图从一般利润率这个前提出发,考察的只是为了保持这种一般利润率,使这种一般利润率继续存在下去所必需的、他作为例外来解释的价格变动。他一点也没有想到,为了创造这个一般利润率,必须先有价值向费用价格的转化;因此,他由于把一般利润率作为基础,就不会再直接同商品的价值发生关系了。

在前面引用的一段话里也只有斯密的观点,但是连斯密的观点也作了片面的阐述,因为李嘉图内心始终抱有一般剩余价值率的思想。在他看来,在个别部门,利润率所以会高于平均水平,只是因为在个别生产部门中由于供求关系,由于生产不足或生产过剩,商品的市场价格会高于自然价格。那时,竞争,新资本流入一个生产部门或旧资本从另一部门抽出,就会使市场价格同自然价格趋于一致,并使个别生产部门的利润恢复到一般水平。这里,利润的实际水平被假定为不变的,既定的东西,问题只在于使个别生产部门中由于供求关系而高于或低于这个水平的利润恢复到这个水平。同时,李嘉图甚至总是假定,如果商品价格提供的利润大于平均利润,这种商品就是高于其价值出卖,如果商品价格提供的利润低于平均利润,这种商品就是低于其价值出卖。如果通过竞争,商品的市场价值同它的价值达到一致,那末,利润的平均水平就确立起来了。

李嘉图认为,这个水平本身,只有在工资(相对稳定地)降低或提高的时候,也就是在相对剩余价值率降低或提高的时候,才能提高或降低;而这在价格没有变动的时候也会发生。(虽然这里李嘉图自己就承认,在不同的生产部门,根据其流动资本和固定资本的构成不同,价格会发生很显著的变动。)

但是,即使在一般利润率已经确立,因而费用价格也已经确立的情况下,在个别生产部门,由于工作日较长,也就是由于绝对剩余价值率提高,利润率也可能提高。工人之间的竞争并不能把这种差别拉平,国家的干涉已经证明这一点。在这里,在这些个别生产部门,即使市场价格并不高于“自然价格”,利润率也会提高。当然,资本之间的竞争可能而且终将使这种超额利润不是完全落入这些个别生产部门的资本家手中。他们不得不把自己商品的价格降到其“自然价格”之下,或者其他生产部门将把自己的价格提高一些(如果事实上没有提高,——这种提高可能被这些商品的价值的降低抵销,——那末无论如何,[672]总不致把它们的价格降得象本部门中劳动生产力的发展所要求的那样低)。利润的一般水平将提高,费用价格将发生变动。

其次,如果出现一个新的生产部门,使用的活劳动很多,同积累劳动不成比例,因此这个部门的资本构成大大低于决定平均利润的平均构成,那末,供求关系就可能容许这个新的部门高于产品的费用价格,以比较接近于产品实际价值的价格出卖产品。竞争要把这种差别拉平,只有通过提高利润的一般水平才有可能,而提高利润的一般水平又以资本实现、推动更大的无酬的剩余劳动量为条件。在上述情况下,供求关系不是象李嘉图所认为的那样,使商品高于它的价值出卖,而只是使商品按接近它的价值、高于它的费用价格的价格出卖。因此,平均化的结果不可能是使利润恢复到原来的水平,而是确立一个新的水平。

\tsubsectionnonum{[(b)李嘉图在殖民地贸易和一般对外贸易对利润率的影响问题上的错误]}

例如在殖民地贸易上,情况也是如此;在殖民地,由于奴隶制和土地的自然肥力(或者由于土地所有权在实际上或在法律上还不发达),劳动价值比在宗主国低。如果宗主国的资本可以自由地转入这个新的部门,这些资本固然会压低这个部门的特殊超额利润,但是将提高利润的一般水平(正如亚·斯密正确地指出的那样)。

李嘉图在这里经常求助于这样一种说法:要知道,在旧的生产部门,使用的劳动量以及工资是保持不变的。但是,一般利润率决定于无酬劳动对有酬劳动和对预付资本的比例,这不是就某个生产部门,而是就资本可以自由转入的所有部门来说的。这个比例,在十个部门中可能有九个保持不变;但如果十个部门中一个有了变动,一般利润率在所有十个部门中都必然要发生变动。每当一定量资本所推动的无酬劳动量有了增加的时候,竞争的结果只能是:等量资本取得相等的股息,即在这个增大了的剩余劳动中的相等的一份;竞争的结果不可能是:尽管剩余劳动同全部预付资本相比已经增加,每一笔资本的股息却保持不变,仍然是剩余劳动中原来的那一份。既然李嘉图承认这一点,他就没有任何理由反驳亚·斯密的下述观点:单是因资本积累而加剧的资本竞争就会使利润率降低。因为在这里他自己就承认,即使剩余价值率有所提高,单单由于竞争,利润率也会降低。李嘉图的这个观点当然是同他的第二个错误前提联系着的,那就是,利润率(撇开工资的降低或提高不谈)所以能够提高或者降低,仅仅是由于市场价格暂时偏离自然价格。而什么是自然价格呢?就是等于预付资本加平均利润的那种价格。所以,又归结到这样一个前提:除非相对剩余价值降低或提高,否则平均利润决不可能降低或提高。

因此,李嘉图用下面的话来反对斯密,是错误的。他说:

\begin{quote}{“从对外贸易的一个部门转移到另一个部门,或者从国内贸易转移到对外贸易,据我看来,都不能影响利润率。”(同上,第413页)}\end{quote}

李嘉图认为,因为利润不影响价值,所以利润率也不影响费用价格,他这种看法也是错误的。

李嘉图认为,如果对外贸易的某一部门条件特别有利,那末,利润的一般水平总是通过使那里的利润降到原来水平的办法,而不是通过提高利润的原来水平的办法来确立的。他这种看法是错误的。

\begin{quote}{“他们断言,利润的均等是由利润的普遍提高造成的;而我却认为,特别有利的部门的利润会迅速下降到一般水平。”(第132—133页)}\end{quote}

由于李嘉图对利润率抱着完全错误的观点,他就根本不懂得对外贸易在不直接降低工人食物价格时所发生的影响。他看不到,对于象英国这样的国家,取得[673]较低廉的工业用原料具有多么重大的意义,他不了解,在这种情况下,正如我前面指出的那样,[20][用较低廉的原料制成的产品]价格虽然下跌,利润率却会提高,相反,[用较贵的原料制成的产品]价格上涨了,利润率却可能降低,即使工资在这两种情况下保持不变,也是如此。

\begin{quote}{“因此,利润率不会由于市场扩大而提高。”(第136页)}\end{quote}

利润率不是取决于单位商品的价格,而是取决于用一定的资本能够实现的剩余劳动量。李嘉图在其他地方对市场的重要意义也估计不足,因为他不了解货币的本质。

\centerbox{※     ※     ※}

[673](除了前面所说的以外,还必须指出:

李嘉图所以犯这一切错误,是因为他想用强制的抽象来贯彻他把剩余价值率和利润率等同起来的观点。庸俗经济学家由此得出结论说,理论上的真理是同现实情况相矛盾的抽象。相反,他们没有看到,因为李嘉图在正确抽象方面做得不够,才使他采取了错误的抽象。\endnote{马克思在这里指的是象让·巴·萨伊这样一些批评李嘉图的人,萨伊指责李嘉图把“赋予过分普遍意义的抽象原则”作为自己论述的基础,因此得出了不符合实际情况的结论(让·巴·萨伊《论政治经济学》1841年巴黎第6版第40—41页)。——第497页。})[673]

\tsectionnonum{[(3)]利润率下降规律}

\tsubsectionnonum{[(a)李嘉图关于利润率下降的见解的错误前提]}

利润率下降规律是李嘉图体系中最重要的观点之一。

利润率具有下降的趋势。为什么呢?亚·斯密说:这是由于资本积累的增长和随之而来的资本竞争的加剧。李嘉图反驳说:竞争能使不同生产部门的利润平均化(前面我们已经看到,他在这里是前后矛盾的),但它不能使一般利润率下降。在李嘉图看来,这种下降只有在下述情况下才是可能的,即由于资本的积累,资本的增殖比人口的增长快,以致对劳动的需求经常超过劳动的供给,因而工资在名义上、实际上以及按使用价值来说都不断提高——不论按价值还是按使用价值来说,都不断提高。但这种情况是不会有的。李嘉图不是一个相信这种寓言的乐观主义者。

因为利润率和剩余价值率(指相对剩余价值率,因为李嘉图假定工作日不变)在李嘉图看来是等同的,所以,利润率的不断下降或利润率下降的趋势,他只能用决定剩余价值率(即工作日中工人不是为自己而是为资本家劳动的那一部分)不断下降或下降趋势的同样原因来说明。但这是些什么样的原因呢?假定工作日既定,那末只有在工人为自己劳动的那一部分工作日增大的条件下,工人无代价地为资本家劳动的那一部分工作日才能减少、缩短。而这(假定劳动的价值能得到支付)只有在用工人工资购买的必需品即生活资料的价值增大的情况下才有可能。但是,由于劳动生产力的发展,工业品的价值在不断减少。因此,在李嘉图看来,利润率的下降,就只能用生活资料的主要组成部分——食物——的价值的不断提高来说明。而这据说又是由于农业生产率不断降低引起的。这就是李嘉图在分析地租时用来说明地租存在和地租增长的那个前提。因此,在李嘉图看来,利润率的不断下降是同地租率的不断提高联系在一起的。我已经指出,李嘉图对地租的理解是错误的。因而,他用来说明利润率下降的根据之一也就不能成立。第二,他关于利润率下降的观点是建立在错误的前提之上的,他认为,剩余价值率和利润率是等同的,从而,利润率的下降和剩余价值率的下降也是等同的,其实,这后一种下降只有按照李嘉图的方式才能解释。因此他的理论就被推翻了。

利润率下降——虽然剩余价值率这时保持不变或提高——是因为随着劳动生产力的发展,可变资本同不变资本相比减少了。因此,利润率下降不是因为劳动生产率降低了,而是因为劳动生产率提高了。利润率下降不是因为对工人的剥削减轻了,而是因为对工人的剥削加重了,不管这是由于绝对剩余时间增加,还是——在国家对此进行阻挠时——由于资本主义生产的本质必然要使劳动的相对价值降低,从而使相对剩余时间增加。

所以,李嘉图的理论是建立在两个错误前提之上的:

第一个错误前提是:地租的存在和增加以农业生产率不断降低为条件;

第二个错误前提是:利润率(在李嘉图那里就等于相对剩余价值率)的提高或下降只能同工资的提高或下降成反比。

[674]现在我首先把李嘉图发挥上述见解的地方收集在一起。

\tsubsectionnonum{[(b)对李嘉图关于增长的地租逐渐吞并利润这个论点的分析]}

我们预先还要作几点说明,指出李嘉图是怎样从自己的地租见解出发来阐明地租逐渐吞并利润率这个观点的。

为此,我们想利用第574页上的表\authornote{见第302—303页。——编者注},但是要作一些必要的修改。

在这些表内,都是假定使用的资本等于60c+40v,剩余价值率等于50%,因此,不管劳动生产率如何,产品的[个别]价值都等于120镑。其中10镑是利润,10镑是绝对地租。我们假定,40镑可变资本支付20个工人(比如说,支付他们一周的劳动,或者,因为谈的是利润率,最好算作支付一年的劳动;这在这里是完全无关紧要的)。依照A表,决定市场价值的是土地I,[煤的]吨数[或谷物的夸特数]等于60;所以,60吨值120镑,1吨[或1夸特]值,120/60,即2镑。工资是40镑,也就是说,等于20吨[煤]或20夸特谷物。因而这就是100镑资本所使用的工人数量的必要工资。如果现在必须推移到较坏等级的土地,为了生产48吨,需要资本110镑(60镑不变资本和推动20个工人的可变资本,即60镑不变资本和50镑可变资本),那末,在这种场合,剩余价值就是10镑,每吨的价格就等于2+(1/2)镑。如果我们推移到更坏等级的土地,那里120镑提供40吨,那末,每吨的价格就等于120/40,即3镑。在这里,在最坏的土地上,任何剩余价值都将没有了。在所有这些场合,由20个工人的劳动新创造的价值,总是等于60镑(3镑=1工作日,不管其长度如何)。因此,如果工资从40镑增加到60镑,那末,任何剩余价值都会消失。在这里总是假定,1夸特谷物是一个工人的必要工资。

我们假定,在上述两种[推移到较坏土地的]场合,必须花费的资本都只是100镑。或者同样可以说,不管在每一种场合花费多少资本,对于100镑资本来说会有怎样的比例呢?这就是说,我们不是假定,在工人人数相同和不变资本量相同的情况下,花费的资本是110镑,还是120镑,而是要计算,在假定有机构成相同(不是按价值,而是按使用的劳动量和不变资本量)的情况下,资本为100时有多少不变资本和能雇用多少工人(以便有可能把这100单位和其他的资本[价值]构成进行比较)。

110比60等于100比54+(6/11),110比50等于100比45+(5/11)。20个人推动60单位不变资本;那末54+(6/11)单位要由多少人来推动呢?

事情是这样。60镑是被雇用的一定数量的工人(比如说,20个工人)所创造的价值。在这里,如果每吨或每夸特=2镑,那末给这一定数量的工人支付20夸特或20吨,就=40镑。如果每吨的价值增加到3镑,剩余价值就会消失。如果每吨的价值增加到2+(1/2)镑,构成绝对地租的那一半剩余价值就会消失。

第一种场合,花费资本120镑(60c+60v)时,产品=120镑=40吨(40×3)。第二种场合,花费资本110镑(60c+50v)时,产品=120镑=48吨{48×[2+(1/2)]}。

第一种场合,花费资本100镑(50c+50v)时,产品=100镑=33+(1/3)吨{3镑×[33+(1/3)]=100镑}。同时,因为这里只是推移到较坏的土地,资本中并没有发生任何变化,推动50镑不变资本的工人人数,和以前推动60镑资本的工人人数的比例一样。因此,如果以前60镑资本由20个工人推动(在每吨的价值等于2镑时,他们得到40镑),现在的50镑资本就由16+(2/3)个工人推动,他们从每吨的价值增加到3镑时起得到50镑。一个人照旧得到1吨或1夸特=3镑,因为[16+(2/3)]×3=50。如果16+(2/3)个工人创造的价值=50镑,20个工人创造的价值就=60镑。因此,原先20个工人的一天劳动创造价值60镑的前提仍然有效。

现在再看第二种场合。花费资本100镑时,产品=109+(1/11)镑=43+(7/11)吨{[2+(1/2)]×[43+(7/11)]=109+(1/11)}。不变资本=54+(6/11)镑,可变资本=45+(5/11)镑。这45+(5/11)镑雇用多少工人呢?18+(2/11)个工人。[675]这时,如果20个工人一天劳动创造的价值=60镑,18+(2/11)个工人一天劳动创造的价值就=54+(6/11)镑,因而产品的价值=109+(1/11)镑。

这样我们就看到,在两种场合,同一资本[100镑]推动的工人人数比过去少了,可是现在花费在工人身上的费用贵了。他们劳动的时间还是那样多,但是他们提供的剩余劳动时间少了,或者完全没有了,因为他们在花费同样劳动的情况下生产的产品少了(而这种产品是由他们所消费的必需品构成的);因此,虽然他们劳动的时间和以前一样,他们生产1吨或1夸特所用的劳动时间却增加了。

李嘉图在他的计算中总是假定,资本推动了更多的劳动,因此必须花费更多的资本,即不是以前的100镑,而是120镑或110镑。这只有在假定生产的产品数量相同的情况下才是正确的,也就是说,在上述两种场合都是生产60吨,而不是第一种场合生产40吨(花费120镑),第二种场合生产48吨(花费110镑)。所以,花费100镑时,第一种场合生产33+(1/3)吨,第二种场合生产43+(7/11)吨。从上述假定出发,李嘉图就抛弃了正确的观点,这个正确观点不是[在劳动生产率降低的情况下]必须使用更多的工人来生产同样数量的产品,而是一定数量的工人生产的产品少了,而这种产品中构成工资的那部分却比以前大了。

现在把两个表比较一下:一个是第574页上的A表,一个是根据上述假定得出的新表。

\todo{}

如果现在这个表按照李嘉图的下降序列,用相反的顺序来表述,也就是说,如果我们从III开始,同时假定,最先耕种的最肥沃的土地[当它是唯一的一个等级的耕地时]不提供任何地租,那末,在III这个等级,我们首先会有100镑资本,它生产120镑的价值,其中60镑为不变资本,60镑为新加劳动。其次,根据李嘉图的前提,必须假定利润率比A表上的高,因为在每吨煤(或每夸特小麦)的价格降低时情况是这样的:当每吨值2镑时,20个工人得到20吨=40镑;而现在,当每吨只值1+(3/5)镑,或者说,值1镑12先令时,这20个工人总共只得到32镑(=20吨)。工人人数相同,而花费的资本将是60c+32v=92镑,全部产品的价值将是120镑,因为20个工人的劳动所创造的价值照旧等于60镑。按照[c和v之间的]这种比例,100镑资本应创造价值130+(10/23)镑(因为92∶120=100:[130+(10/23)],或23∶30=100∶[130+(10/23)]),并且这100镑资本的构成将是:[65+(5/23)]c+[34+(18/23)]v。因此,资本等于[65+(5/23)]c+[34+(18/23)]v;产品价值是130+(10/23)镑;工人人数是21+(17/23);剩余价值率是[87+(1/2)]%。

(1)这样,我们就得出下表:

\todo{}

用吨来表示,工资=21+(17/23)吨,利润=19+(1/46)吨。

[676]我们始终根据李嘉图的前提,现在假定由于人口增加,市场价格提高,因此必须耕种等级II,这里每吨价值等于1+(11/13)镑。

这里决不是象李嘉图所设想的那样,21+(17/23)个工人始终生产同样的价值,即65+(5/23)镑(工资和剩余价值算在一起)。因为[由于产品涨价]III的资本家能够雇用、从而能够剥削的工人人数,按照他自己的前提,将会减少,也就是说,剩余价值总额也将减少。

同时,农业资本的[有机]构成始终保持不变。为了推动60c,不管工资多少,始终需要20个工人(工作日既定)。

因为这20个工人得到20吨,而每吨现在值1+(11/13)镑,所以20个工人就花费20×[1+(11/13)]镑=20镑+[16+(12/13)]镑=36+(12/13)镑。

不管这20个工人的劳动生产率如何,他们生产的价值总是等于60镑。这样,预付资本等于96+(12/13)镑,产品价值等于120镑;所以,利润是23+(1/13)镑。因此,资本100提供的利润将等于23+(17/21),资本构成将是[61+(19/21)]c+[38+(2/21)]v,雇用的工人人数将是20+(40/63)。

如果产品总价值[在花费资本100镑时]=123+(17/21)镑,而III的每吨的个别价值=1+(3/5)镑,那末,这一等级的产品现在是多少吨呢?是77+(8/21)吨。剩余价值率是[62+(1/2)]%。

但是III的每吨是按1+(11/13)镑出卖的。因此,每吨的差额价值是4+(12/13)先令,或16/65镑,以77+(8/21)吨计算,是[77+(8/21)]×(16/65),即19+(1/21)镑。

III的产品不是卖123+(17/21)镑,而是卖123+(17/21)镑+[19+(1/21)]镑=142+(6/7)镑。这19+(1/21)镑就构成地租。

这样,对于III,我们就得出下表:

\todo{}

以吨计算,工资=20+(40/63)吨,利润=12+(113/126)吨。

现在我们转过来谈等级II。这里[根据李嘉图的前提]根本没有地租。市场价值和个别价值相符。II生产的吨数是67+(4/63)吨。

所以,对于II,我们就得出下表:

\todo{}

以吨计算,工资=20+(40/63)吨,利润=12+(113/126)吨。

[677](2)所以,对于第二种场合,即当等级II出现并产生地租时,我们就得出下表:

\todo{}

现在我们转过来谈第三种场合,我们同李嘉图一起假定,较次等级的煤矿(I)必须开采,而且能够开采,因为产品的市场价值已经提高到每吨2镑。因为不变资本为60镑时,需要20个工人,这20个工人现在要花费40镑,所以我们的资本的[价值]构成就和第574页A表上的一样,即60c+40v。20个工人生产的价值总是等于60镑。因此,不管资本的生产率如何,100镑资本所生产的产品的总价值[个别价值,实际价值]是120镑。这里利润率=20%,剩余价值=50%。以吨计算,利润=10吨。现在我们就要看一看,由于市场价值的这种变动以及决定利润率的I的出现,III和II会发生什么变化。

虽然III耕种的是最肥沃的土地,但它有100镑资本也只能使用20个工人,花费40镑,因为在不变资本为60镑时需要20个工人。因此,100镑资本使用的工人人数降到20。而它的产品的实际总价值现在等于120镑。但是因为III生产的每吨的个别价值等于1+(3/5)镑(或8/5镑),那末,它生产多少吨呢?75吨,因为120除以8/5等于75。III生产的吨数减少了,因为它用相同的资本,只能使用较少的劳动,而不是较多的劳动(李嘉图却经常错误地认为,能使用更多的劳动,因为他经常注意的只是为了生产同样数量的产品需要多少劳动,而不是按新的资本[价值]构成能使用多少活劳动;而这一点在这里正是唯一重要的问题)。但这75吨,III按150镑(而不是按构成其[实际]价值的120镑)出卖,这样,III的地租就提高到30镑。

至于II,这里产品的[实际]价值也等于120镑等等。但是因为每吨个别价值等于1+(11/13)镑(或24/13镑),所以这个等级生产65吨(因为120除以24/13等于65)。简单地说,我们在这里得到的是第574页上的A表。但是因为在这里我们需要有几个新项目来适应我们的目的,所以当出现I这个等级而产品市场价值提高到2镑时,我们就要为这第三种场合制一个新表。

(3)第三种场合:

\todo{}

[678]这样,这第三种场合就和第574页上的A表相符合(如果不计算绝对地租,绝对地租在这里作为利润的一部分出现),只是顺序颠倒了。

现在我们转过来谈我们假定的新的场合\authornote{见本册第500—502页。——编者注}。

首先我们来看看那个还提供利润的等级;我们把它叫做Ib。在资本为100镑时,它只提供43+(7/11)吨。

每吨价值提高到2+(1/2)镑。资本构成是[54+(6/11)]c+[45+(5/11)]v。产品价值是109+(1/11)镑。可变资本45+(5/11)镑用来支付18+(2/11)个工人的工资。因为20个工人一天劳动创造的价值等于60镑,所以18+(2/11)个工人创造的价值是54+(6/11)镑。因此,产品价值等于109+(1/11)镑。利润率=[9+(1/11)]%。利润=3+(7/11)吨。剩余价值率=20%。

因为III、II、I的资本有机构成和Ib的一样,而且它们必须支付的工资也同后者一样,所以,在资本为100镑时,它们也只能使用18+(2/11)个工人,这18+(2/11)个工人生产的总价值是54+(6/11)镑,所以,也同Ib一样,剩余价值为20%,利润率为[9+(1/11)]%。这里,产品的总价值[实际价值],也和Ib一样,等于109+(1/11)镑。

但是,因为III的每吨个别价值等于1+(3/5)镑(或8/5镑),所以III生产[109+(1/11)]∶(8/5),即68+(2/11)吨(换句话说,这里109+(1/11)镑等于68+(2/11)吨)。其次,每吨市场价值和每吨个别价值的差额现在是2+(1/2)镑减1+(3/5)镑,或者说,2镑10先令减1镑12先令,即18先令。以68+(2/11)吨计算,就是[68+(2/11)]×18先令=1227+(3/11)先令=61+(4/11)镑。III的产品不是卖109+(1/11)镑,而是卖170+(5/11)镑。这个余额就是III的地租。用吨来表示,地租=24+(6/11)吨。

因为II的每吨个别价值等于1+(11/13)镑(或24/13镑),所以它生产[109+(1/11)]:(24/13),即59+(1/11)吨。II的每吨市场价值和它的[个别]价值的差额是2+(1/2)镑减1+(11/13)镑,即17/26镑。以59+(1/11)吨计算,是38+(7/11)镑。这就是地租。总产品市场价值=147+(8/11)镑。用吨来表示,地租等于15+(5/11)吨。

最后,因为I的每吨个别价值等于2镑,所以在这里109+(1/11)镑等于54+(6/11)吨。市场价值和个别价值的差额等于2+(1/2)镑减2镑,即1/2镑。以54+(6/11)吨计算,是27+(3/11)镑。所以,总产品市场价值=136+(4/11)镑。用吨来表示,地租价值等于10+(10/11)吨。

如果我们现在把所有这些综合起来构成第四种场合,便得出下表:

[679](4)第四种场合:

\todo{}

最后,我们来考察一下最后一种场合,按照李嘉图的说法,在这里一切利润都会消失,剩余价值完全没有了。

在这里,产品价值提高到每吨3镑,所以,在使用20个工人时,他们的工资等于60镑,等于他们生产的价值。资本构成将是50c+50v。在这种场合使用的就是16+(2/3)个工人。20个工人生产的价值是60镑,16+(2/3)个工人生产的价值就是50镑。因而,工资会把这全部价值吞没。一个工人和以前一样得到一吨。产品价值=100镑。这样,生产的吨数=33+(1/3)。其中一半只能补偿不变资本的价值,而另一半只能补偿可变资本的价值。

既然III的每吨个别价值等于1+(3/5)镑(或8/5镑),那末,它生产多少吨呢?100除以8/5,即62+(1/2)吨,其价值=100镑。每吨市场价值和个别价值的差额等于3镑-[1+(3/5)]镑=1+(2/5)镑。以62+(1/2)吨计算,差额就是87+(1/2)镑。因而在这里总产品市场价值等于187+(1/2)镑。用吨来表示,地租就等于29+(1/6)吨。

II的每吨个别价值等于1+(11/13)镑。因此,每吨差额价值等于3镑减1+(11/13)镑,即1+(2/13)镑。因为在这里每吨个别价值等于1+(11/13)镑,或24/13镑,所以100镑的资本就生产100:(24/13),即54+(1/6)吨。以这个吨数计算,差额就是62+(1/2)镑。产品市场价值=162+(1/2)镑。用吨来表示,地租就等于20+(5/6)吨。

I的每吨个别价值=2镑。因此,每吨差额价值等于3镑减2镑,即1镑。因为在这里每吨个别价值=2镑,所以100镑的资本生产50吨。[产品市场价值和个别价值的]差额就是50镑。产品市场价值=150镑,以吨计算,地租=16+(2/3)吨。

现在我们转过来谈Ib,它到现在为止不提供任何地租。这里每吨个别价值=2+(1/2)镑。因而每吨差额价值等于3镑减2+(1/2)镑,即1/2镑。因为这里每吨个别价值=2+(1/2)(或5/2)镑,所以100镑的资本就生产40吨。以这个吨数计算,差额价值就是20镑,所以总产品的市场价值=120镑,以吨计算,地租=6+(2/3)吨。

这样,我们就有了第五种场合,按照李嘉图的说法,在这里利润会消失,我们现在用一个统一的表来说明。

[680](5)第五种场合:

\todo{}

在下页我把所有五种场合列一总表加以比较。\authornote{见第512—513页的表。——编者注}[680]

\todo{}

\tsubsectionnonum{[(c)一部分利润和一部分资本转化为地租。地租量的变动取决于农业中使用的劳动量的变动]}

[683]如果我们从上页的表\authornote{见第512—513页。——编者注}中首先考察E表,我们就会看到,在这里,最后一个等级Ia的情况是很明显的。在这里,工资吞并了[新加]劳动的全部产品和它创造的全部价值。任何剩余价值都没有了,从而利润和地租也没有了。产品的价值等于预付资本的价值,所以,在这里自己拥有资本的劳动者,能够不断地把自己的工资和自己的劳动条件再生产出来,但是不能再生产出更多的东西。关于最后这一个等级,决不能说地租吞并了利润。这里既没有地租,也没有利润,因为没有任何剩余价值。工资吞并了剩余价值,因而也吞并了利润。

至于其他四个等级,初看起来情况并不明显。既然没有剩余价值,又怎么可能有地租存在呢?而且在Ib、I、II和III这四个等级的土地上劳动生产率并没有发生变动。所以,剩余价值没有了,应该只是一种表面现象。

随后就会发现另一种初看起来同样不可理解的现象。以[煤的]吨数或谷物的夸特数来表示的地租,在III是29+(1/6)吨(或夸特),而在只有III的土地被耕种的A表,却没有任何地租。此外,那里的工人人数是21+(17/23),而现在在E表,工人人数只有16+(2/3);在A表,(吞并了全部剩余价值的)利润只是19+(1/46)吨。

在II可以发现同样的矛盾,在E表,II的地租是20+(5/6)吨(或夸特),而在B表,吞并了全部剩余价值的利润(而且使用的工人人数是20+(40/63),不是现在的16+(2/3)只等于12+(113/126)吨(或夸特)。

在I也可以发现同样的矛盾,在E表,I的地租等于16+(2/3)吨或夸特,而在C表,I的吞并了全部剩余价值的利润只等于10吨(而且使用的工人人数是20,不是现在的16+(2/3))。

最后,在Ib也有同样的现象,在E表,Ib的地租是6+(2/3)吨(或夸特),而在D表,它的吞并了全部剩余价值的利润只等于3+(7/11)吨或夸特(而且使用的工人人数是18+(2/11),而不是现在的16+(2/3))。

但是很明显,III、II、I、Ib的产品的市场价值高于个别价值,虽然也能够改变产品的分配,促使产品从一类分享者手里转移到另一类分享者手里,但是这种市场价值的提高,决不能使补偿工资后留下的剩余价值所借以表现的产品本身增加。我们这里列举的各个等级土地的生产率以及资本的生产率既然保持不变,那末,仅仅由于在市场上出现了比较不肥沃的土地或比较不富饶的矿井Ia的产品,III至Ib怎么就能够变得比较肥沃或富饶起来,也就是能够提供更多的吨数或夸特数呢?

用下述方法可以解开这个谜:

如果20个工人一天的劳动等于60镑,16+(2/3)个工人就生产50镑。因为在等级III,1+(3/5)镑(或8/5镑)所包含的劳动时间表现为1吨或1夸特,所以,50镑就表现为31+(1/4)吨或夸特。其中16+(2/3)吨或夸特用于工资;因而就有14+(7/12)吨留作剩余价值。

其次,因为每吨市场价值由1+(3/5)镑提高到3镑,所以为了补偿不变资本的价值[50镑],从62+(1/2)吨或夸特的产品中只要拿出16+(2/3)吨或夸特来就够了。不过,如果由在等级III生产的1吨(或1夸特)本身决定市场价值,因而市场价值等于它的个别价值,为了补偿50镑不变资本,就需要31+(1/4)吨或夸特。这个吨数或夸特数是在每吨价值等于1+(3/5)镑时为了补偿[不变]资本所必需的那部分产品,现在为了同一个目的,只要有16+(2/3)吨就够了。因此,[684]31+(1/4)吨减16+(2/3)吨,即14+(7/12)吨(或夸特),就游离出来,归入地租份内。

在III,16+(2/3)个工人在不变资本为50镑时生产的剩余价值等于14+(7/12)吨或夸特;以前用来补偿不变资本,而现在以剩余产品形式出现的那一部分产品也等于14+(7/12)吨或夸特,如果我们把这两部分加起来,总剩余产品就是28+(14/12)=29+(2/12)=29+(1/6)吨或夸特。这恰好就是E表中III以[煤的]吨数或谷物的夸特数来表示的地租。在E表中II、I、Ib这些等级以[煤的]吨数或谷物的夸特数来表示的地租在量上的表面矛盾,也完全可以用同样的方法加以解决。

这样,我们就看到,在较好等级的土地上由于它们产品的市场价值和个别价值的差额而产生的级差地租,在它的实在形态上,作为实物地租,作为剩余产品,或者说,象上述例子那样,作为以[煤的]吨数或谷物的夸特数表示的地租,是由两个要素构成,并由两种转化决定的。[第一,]表现工人剩余劳动的剩余产品即剩余价值,经历了从利润形式到地租形式的转化,因而归土地所有者,而不归资本家所有。第二,以前,当较好等级的土地或矿井的产品按其本身价值出卖时,有一部分产品必须用来补偿不变资本的价值,现在,当产品的每一部分都有了更高的市场价值时,那部分产品中有一部分就会游离出来,也以剩余产品的形式出现,因而也归土地所有者,而不归资本家所有。

剩余产品转化为地租而不转化为利润,以前用于补偿不变资本价值的产品的某一相应部分转化为剩余产品,因而转化为地租,——这两个过程在实物地租是级差地租的情况下构成实物地租。后一种情况,即产品的一部分不转化为资本,而转化为地租,李嘉图和所有后来的经济学家都没有注意到。他们只看到剩余产品转化为地租,而没有看到以前归入资本份内(而不归入利润份内)的那部分产品中有一部分也会转化为剩余产品。

这样构成的剩余产品,或者说,级差地租,它的名义价值决定于(根据假定)最坏土地或最次矿井生产的产品的价值。但是这种市场价值只能引起这种产品的另行分配,而不能创造产品。

这两个要素在一切有超额利润的场合都是存在的;例如,由于采用新机器等等,某种产品的生产变得便宜了,但它按照超过其本身价值的市场价值出卖,就是这种情况。工人的一部分剩余劳动不是作为利润出现,而是作为成为超额利润的剩余产品出现。一定量产品,在工业品按照其本身的较低的价值出卖时,本来必须用来为资本家补偿不变资本的价值,现在有一部分就会游离出来,因为没有什么需要补偿了;这一部分成了剩余产品,因而就使利润大大增加。[684]

\centerbox{※     ※     ※}

[688]{但是,当我们谈到资本主义生产发展过程中利润率下降的规律时,我们在这里把利润理解为剩余价值总额,它首先为产业资本占有,不管以后产业资本要怎样和借贷资本家(利息所得者)以及土地所有者(地租所得者)瓜分它。所以,这里利润率=剩余价值/预付资本。在这个意义上说,利润率可能下降,尽管,比如说,产业利润同利息相比提高了,或者相反;同样,尽管地租同产业利润相比提高了,或者相反,利润率也可能下降。如果利润=P,产业利润=P′,利息=Z,地租=R,那末P=P′+Z+R。很明显,无论P的绝对量如何,P′、Z和R彼此相对来说可能提高或下降,而不管P的大小,不管P是提高还是下降。P′、Z和R彼此相对来说的提高,仅仅是P在不同人之间的不同分配。进一步考察引起P在不同人之间的这种或那种分配(而分配与P本身的提高和下降决不是一回事)的种种情况,并不是这里的任务,这要留到分析资本竞争时再说。但是,如果说R在量上能够达到P本身不会有的高度——要是P只分为P′和Z——那末,正如已经证明的,这是一种表面现象,它是由下述情况造成的,即一部分产品,在其[市场]价值增加时,不是再转化为不变资本,而是游离出来,转化为地租。}[688]

\centerbox{※     ※     ※}

[684]在前面的所有论述中都假定,已涨价(按市场价值来说)的产品,不以实物形式加入不变资本的构成,而只加入工资,只加入可变资本。如果已涨价的产品加入不变资本,那末,在李嘉图看来,利润率因此就会降得更低,地租就会提得更高。这一点必需加以研究。

在此以前,我们一直假定,不变资本的价值,即上述场合的50镑,应由产品的价值补偿。所以,在每吨或每夸特值3镑的时候,为补偿上述价值所需的吨数或夸特数,当然比每吨或每夸特仅值1+(3/5)镑等等的时候少。但现在我们假定,煤或谷物,或其他任何土地产品——由农业资本生产的产品——以实物形式加入不变资本的形成,比如说,加入一半。在这种场合,很明显,不管煤或谷物的价格如何,[685]一定量的不变资本,即由一定数量工人推动的一定量的不变资本,总是要求用总产品的一定部分以实物形式来补偿自己,因为根据假定,表现为积累劳动量和活劳动量之比的农业资本构成保持不变。

假定,比如说,不变资本一半由煤或谷物构成,而另一半由其他商品构成,那末,50镑不变资本中有25镑是由其他商品构成,25镑是[由煤或谷物,]即在每吨[煤]值1+(3/5)(或8/5)镑时,由15+(5/8)吨[煤]或夸特[谷物]构成。无论每吨[煤]或每夸特[谷物]的市场价值如何变动,16+(2/3)个工人所需要的不变资本总是25镑[其他商品]加15+(5/8)夸特[谷物]或吨[煤],因为不变资本的物质构成不变,推动这个资本所需要的相应的工人人数也不变。

如果现在每吨[煤]或每夸特[谷物]的[市场]价值象在E表上那样提高到3镑,16+(2/3)个工人需要的不变资本就等于25镑+[15+(5/8)]×3镑=25镑+45镑+(15/8)镑=71+(7/8)镑。因为16+(2/3)个工人在这里要花费50镑,所以在这种场合需要花费的总资本就是[71+(7/8)]镑+50镑=121+(7/8)镑。

在有机构成相同的情况下,农业资本按其价值比例来说将会发生变动。

那就是[71+(7/8)]c+50v(工人人数为16+(2/3))。100镑资本的构成将是[58+(38/39)]c+[41+(1/39)]v。工人人数将是13+(79/117)(比13+(2/3)多1/117)。因为16+(2/3)个工人推动15+(5/8)夸特或吨不变资本,13+(79/117)个工人就推动12+(32/39)吨或夸特,即价值38+(6/13)镑。剩下的那部分不变资本等于20+(20/39)镑,由其他商品构成。在一切情况下,从产品中都应扣除12+(32/39)吨或夸特,来补偿它们以实物形式加入的那一部分不变资本。因为20个工人生产的价值等于60镑,13+(79/117)个工人生产的价值就等于41+(1/39)镑。但在E表上,13+(79/117)个工人的工资也是41+(1/39)镑。所以,这里任何剩余价值都没有了。

[在这种场合,E表的III的]总吨数将是[51+(11/39)吨\endnote{51+(11/39)吨这一数字是这样算出来的:如果E表III的16+(2/3)个工人生产62+(1/2)吨,那末13+(79/117)个工人在劳动生产率相同的情况下将生产[13+(79/117)]×[62+(1/2)]/16+(2/3)即51+(11/39)吨。——第519页。},其中]12+(32/39)吨会再生产出来[以补偿它们以实物形式加入的那部分不变资本],13+(79/117)吨用于工人的工资,6+(98/117)吨用于剩下的那部分不变资本(每吨3镑)。这三个部分合在一起是33+(1/3)吨。所以剩下的地租份额是17+(37/39)吨。

为了计算简便,我们假定出现对李嘉图最有利的极端情况,也就是假定不变资本完全和可变资本一样,仅仅由农产品构成,其价值由于等级Ia在市场上占统治地位而提高到每夸特或每吨3镑。

资本的技术构成保持不变,就是说,可变资本所代表的活劳动即工人人数(因为假定正常工作日不变),同使用这一数量的工人时需要的、根据我们假定现在是由煤的吨数或谷物的夸特数构成的劳动资料总量之间的比例保持不变。

因为在原来的资本构成60c+40v的条件下,当每吨的价格等于2镑时,40v代表20个工人或者20夸特或吨,所以,60c就代表30吨;因为III的这20个工人生产75吨,所以,13+(1/3)个工人(每吨价格为3镑时,40v相当于13+(1/3)吨或13+(1/3)个工人)就生产50吨,并推动不变资本[686]60/3,即20吨或夸特。

其次,因为20个工人生产价值60镑,所以13+(1/3)个工人就生产价值40镑。

因为资本家为购买20吨[不变资本]必须付出60镑,为雇用13+(1/3)个工人必须付出40镑,而这些工人只生产价值40镑,所以产品的价值=100镑;支出也是100镑。剩余价值和利润=0。

但是,因为III的生产率不变,所以象已指出的那样,13+(1/3)个工人生产50吨或夸特。但是实物支出以吨或夸特计算只有33+(1/3)吨:20吨用于不变资本,13+(1/3)吨用于工资。因而III生产的50吨包含16+(2/3)吨的剩余产品,这一剩余产品就形成地租。

但这16+(2/3)吨代表什么呢?

因为产品的价值=100镑,而产品本身=50吨,这里生产的每吨价值实际上是100/50,即2镑。只要以实物形式得到的产品数量大于以实物形式补偿资本所必需的数量,每吨的个别价值,即使在这种[缩小了的]生产规模下,也必然低于它的市场价值。

租地农场主必须花费60镑,以补偿20吨[不变资本]。这20吨,他是按每吨3镑计算的,因为每吨的市场价值就是如此,每吨就是按照这一价格出卖的。同样,他必须花费40镑,以支付13+(1/3)个工人的工资,或者说,以支付他付给工人的那个吨数或夸特数。因此,那些工人只得到13+(1/3)吨。

但在实际上,就等级III来说,20吨只值40镑,而13+(1/3)吨只值26+(2/3)镑。但是13+(1/3)个工人生产价值40镑,因此,创造剩余价值13+(1/3)镑。按每吨2镑计算,合6+(4/6)(或6+(2/3))吨。

因为III的20吨[不变资本]只值40镑,所以剩下一个余额20镑,等于10吨。

因此,构成地租的16+(2/3)吨分为:转化为地租的剩余价值6+(2/3)吨,以及转化为地租的资本10吨。但是,由于每吨的市场价值提高到3镑,20吨[不变资本]要花费租地农场主60镑,13+(1/3)吨[工资]要花费40镑,而作为市场价值超过租地农场主产品的[个别]价值的余额,作为地租出现的这16+(2/3)吨,就等于50镑。

等级II的13+(1/3)个工人提供多少吨呢?这里20个工人提供65吨,因而13+(1/3)个工人提供43+(1/3)吨。产品的价值和前面一样,等于100镑。但是这43+(1/3)吨中有33+(1/3)吨必须用于补偿资本。剩下作为剩余产品或地租的是43+(1/3)-[33+(1/3)]=10吨。

这10吨地租可以这样来解释:

II的产品价值等于100镑,产品是43+(1/3)吨,因而每吨价值=2+(4/13)镑。也就是说,13+(1/3)个工人花费30+(10/13)镑,[在支付13+(1/3)吨市场价值的40镑中]作为剩余价值剩下9+(3/13)镑。其次,20吨不变资本花费46+(2/13)镑,从支付它们的[市场价值的]60镑中剩下13+(11/13)镑。这和剩余价值加在一起就是23+(1/13)镑,正好相当于10吨的[实际]价值[每吨按2+(4/13)镑计算]。

只有等级Ia,为了补偿不变资本和工资,确实需要有33+(1/3)吨或夸特的实物,即需要全部总产品,因而事实上既没有剩余价值,也没有剩余产品,既没有利润,也没有地租。如果情况不是这样,如果得到的产品比以实物形式补偿资本所必需的多,就会发生利润(剩余价值)和资本向地租的转化。只要以前在价值较低时本来必须用来补偿资本的那部分产品中有一部分现在游离出来,或者本来必须转化为资本和剩余价值的那部分产品现在归入地租份内,在这里就会发生这种转化。

但是,同时我们也看到,不变资本的涨价如果是农产品涨价的结果,那就会使地租大大降低。例如,III和II的地租从[E表的]50吨(在每吨市场价值3镑时合150镑)降到26+(2/3)吨,也就是几乎降了一半。这种降低是必然的,[687]因为在这里同一资本100镑所使用的工人人数由于后面两个原因减少了:第一,因为工资提高,即可变资本的价值增加;第二,因为生产资料即不变资本的价值提高。工资提高本身的结果是100镑所使用的劳动少了,因此(在加入不变资本的商品价值不变的情况下)不变资本也相应地减少,所以整个这100镑总合起来就代表较少的积累劳动和较少的活劳动。但是,除此以外,加入不变资本的商品价值提高带来的结果就是,因为积累劳动和活劳动之间的技术比例不变,现在花费同量的货币能够使用的积累劳动量少了,因此能够使用的活劳动量也少了。因为在土地的生产率相同和资本技术构成既定的情况下,总产品量取决于所使用的劳动量,所以,随着使用的劳动量减少,地租也必然减少。

这种情况只有在利润消失的时候才会表现出来。当利润还存在的时候,尽管所有等级的产品都绝对减少,正如第681页上的表\authornote{见本册第512—513页。——编者注}所说明的,地租仍能增加。一般来说,很明显,在只有地租存在时,随着产品的减少,从而随着剩余产品的减少,地租本身也必然减少。如果不变资本的价值同可变资本的价值一起增长,那末这种情况一开始就会更快地出现。

但是,除此以外,第681页上的表还说明,在农业生产率降低的情况下,随着级差地租的增长,即使在较好等级的土地上,与一定量的预付资本,比如说100镑相比,总产品量也总是减少的。李嘉图对这一点毫无所知。利润率降低,是因为同一资本,比如说100镑所推动的劳动量越来越少,而支付这一劳动的费用贵了,从而用来积累的余额越来越小。但是在生产率既定的条件下,实际得到的产品量也和剩余价值一样,取决于资本所使用的工人人数。李嘉图没有看到这一点,就象他没有看到地租的形成不仅依靠剩余价值转化为地租,而且依靠资本转化为剩余价值一样。当然,资本这样转化为剩余价值只是表面上的。如果市场价值由III等等的产品价值决定,那末剩余产品的每一个极小部分就都代表剩余价值,代表剩余劳动。其次,李嘉图总是只注意到为了生产同样数量的产品,必须使用更多的劳动,但是他忽略了对确定利润率以及所生产出来的产品量有决定意义的东西,那就是,花费同一资本所使用的活劳动量会不断减少,在这种活劳动量中[假定按下降序列]成为必要劳动的部分会越来越大,成为剩余劳动的部分会越来越小。

从这一切可以得出结论说,即使把地租只看成是级差地租,李嘉图在这个问题上也丝毫没有比他的前辈前进一步。他在这方面的重大功绩就是德·昆西所指出的,对问题作了科学的阐述。但是李嘉图在解决问题的时候接受了他的前辈的观点。德·昆西说:

\begin{quote}{“李嘉图给地租学说增添了新的东西:他把地租学说归结为地租是否真的取消价值规律的问题。”(托·德·昆西《政治经济学逻辑》1844年爱丁堡和伦敦版第158页)}\end{quote}

其次,在该书第163页,德·昆西说:

\begin{quote}{“地租是土地(或其他任何生产因素)的产品的一部分,这一部分是为了使用土地的各种不同的力而支付给土地所有者的,而土地的各种不同的力是通过与同一市场上出现的同类因素的力相比较而衡量出来的。”}\end{quote}

接着在第176页,德·昆西写道:

\begin{quote}{“有人反驳李嘉图说,一等地的所有者不会白白地交出土地。但在只耕种一等地的时期{在这个神话时期!}还不能形成与土地所有者阶级不同的租地农场主和租佃者的特殊阶级。”}\end{quote}

[688]因此,在德·昆西看来,这个“土地所有权”规律只是在任何现代意义上的土地所有权都不存在的时候才起作用。

现在我们转过来谈摘自李嘉图著作的引文。

\tsubsectionnonum{[(d)在农产品价格同时提高的情况下利润率提高的历史例证。农业劳动生产率增长的可能性]}

(关于级差地租,首先还要作如下说明:事实上上升序列和下降序列是相互交替、相互交叉、彼此交织在一起的。

但这决不是说,如果在个别短暂时期(例如从1797年到1813年)下降序列的运动占很大优势,利润率因此就必然下降(就利润率由剩余价值率决定的情况而言)。相反,我认为,在1797年到1813年那个时期,在英国虽然小麦和一般农产品的价格都急剧上涨,但利润率还是异乎寻常地提高了。据我所知,没有一个英国统计学家不认为上述时期利润率提高了。有些经济学家,例如查默斯、布莱克等人,曾引用这一事实来证实自己的理论。首先我还必须指出,企图用货币贬值来解释这一时期发生的小麦价格的上涨,是十分荒谬的。研究过这一时期商品价格史的人没有一个会同意这种观点。此外,远在货币发生任何贬值以前,价格就开始上涨,并且达到很高的程度。在货币贬值以后就应当纯粹从价格中作相应的扣除。如果要问,为什么谷物价格上涨了,利润率还会提高?这可以用下述情况来解释:工作日延长,这是采用新机器的直接后果;加入工人消费的工业品和殖民地商品跌价;工资降低(虽然名义工资提高了),降到它的普通平均水平以下{这一事实对所考察的时期来说,是大家公认的;帕·詹·斯特林(在1846年爱丁堡出版的《贸易的哲学》一书中)大体上接受了李嘉图的地租理论,但他企图证明,谷物经常的(不是由偶然的歉收造成的)涨价的直接后果总是平均工资降低\endnote{帕·詹·斯特林《贸易的哲学,或利润和价格理论概要》1846年爱丁堡版第209—210页。——第25、525页。}};最后,利润率的提高还可以这样来解释:由于公债和国家的开支,对资本的需求比资本的供给增加得快,这就引起商品名义价格的提高,因此工厂主就从地租的搜刮者和其他领取固定收入的人那里把以地租等形式支付的那部分产品的一部分夺了回来。这种行动不是我们这里研究的对象,我们这里考察的是基本关系,因此我们只注意三个阶级,即土地所有者阶级、资本家阶级和工人阶级。但是正象布莱克所指出的那样\endnote{马克思指的是威·布莱克的书:《论限制现金支付期内政府支出的影响》1823年伦敦版。与正文中涉及的问题有关的该书摘录以及马克思的评语,见马克思1857—1858年的经济学手稿(见卡·马克思《政治经济学批判大纲》1939年莫斯科版第672—673页)。——第526页。},在相应的情况下,这种行动在实践中起着相当大的作用。)[688]

\centerbox{※     ※     ※}

\begin{quote}{[689]{布莱顿的哈利特先生在1862年的博览会\endnote{指1862年5月1日在伦敦开幕的国际博览会,会上展出了工农业产品的样品,艺术作品和科学新成就。——第526页。}上展出了“小麦良种”。“哈利特先生断言,谷物的穗也和跑马一样需要细心照管,然而往往培育得很马虎,不注意自然选择的理论。现举几个突出的例子来说明怎样才能管好小麦。1857年哈利特先生种出了一穗质量最好的红小麦,穗长4+(3/8)英寸,有47颗籽粒。1858年哈利特先生从他得到的不多的收成中,又选了最好的一穗,长6+(1/2)英寸,有79颗籽粒;1859年又同样从中选出最好的一穗,这次穗长7+(3/4)英寸,有91颗籽粒。次年即1860年,对耕作技术的试验不利,麦穗没有长得更好更大;但是一年以后,即1861年,最好的穗长8+(3/4)英寸,一根茎秆上有123颗籽粒。这样,在五年中麦穗的长度增加了一倍,而籽粒的数量几乎增加了两倍。所以能够获得这样的结果,是由于采用了哈利特先生所说的小麦培育的‘自然方法’,即各颗籽粒前后左右都相距约9英寸,使每颗籽粒都有充分发展的余地……哈利特先生断定,如果播种‘良种小麦’并且采用培育的‘自然方法’,英国的谷物生产可以增加一倍。他声称,下种适时,每平方英尺的土地只播种一粒,他从每粒种子收获的作物平均有23穗,每穗约有36颗籽粒。根据这种情况,一英亩土地的产品按精确的计算是1001880穗,而用普通方法播种,每英亩所费种子量达二十倍以上,却只能收934120穗小麦,即少67760穗……”}}\end{quote}

\tsubsectionnonum{[(e)李嘉图对利润率下降的解释以及这种解释同他的地租理论的联系]}

[李嘉图是这样证明利润率的下降的:]

\begin{quote}{“随着社会的进步,劳动的自然价格总是有上涨的趋势,因为调节劳动自然价格的一种主要商品由于生产困难不断增大而有涨价的趋势。但是,由于农业的改良和可以提供进口粮食的新市场的发现,能在某一个时期内阻止必需品价格上涨的趋势,甚至能使其自然价格下降,所以,这些同样的原因也会对劳动的自然价格产生相应的影响。除原产品和劳动外,一切商品的自然价格都有随着财富和人口的增加而下降的趋势;因为,一方面它们的实际价值虽然会由于制造它们所用的原料的自然价格上涨而增加,但是,机器的改良、劳动分工和劳动分配的改进、生产者在科学和技艺方面熟练程度的提高,会把这种增加的趋势抵销而有余。”(同上,第86—87页)“随着人口的增加,必需品的价格将不断上涨,因为生产它们需要较多的劳动……所以,工人的货币工资不会下降,而会提高,但提高的程度却不足以使工人能够买到商品涨价前他能买到的那样多的舒适品和必需品……尽管工人的报酬实际上比以前差了,工人工资的这种增加还是必然会减少工厂主的利润,因为工厂主不能按较高的价格出卖他的商品,而这些商品的生产费用却提高了……因此,可以看出,使地租提高的同一原因(即用同一比例的劳动量来生产追加的食物量越来越困难),也会使工资提高。所以,在货币价值不变的情况下,地租和工资都会有随着财富和人口的增加而提高的趋势。但是,地租的增加和工资的增加有根本的区别。地租的货币价值提高时,[690]它在产品中所占的份额也会随之增加:不仅土地所有者的货币地租会增加,而且他的谷物地租也会增加……工人不会这样幸运:不错,他得到的货币工资会更高,但以谷物来表示,他的工资却减少了;这时不仅他支配的谷物数量会减少,而且他的一般状况也会恶化,因为他将发现,市场工资率要保持在自然工资率以上是更困难了。”(同上,第96—98页)“假定谷物和工业品始终按同一价格出售,那末利润的高低就会同工资的高低成反比。但是,我们假定谷物价格提高是因为生产谷物需要更多的劳动量;这一原因并不会使工业品的价格提高,因为生产工业品不需要追加劳动量……如果工资随着谷物价格上涨而提高(这是肯定会发生的),那末他们[工厂主]的利润就一定会下降。”(同上,第108页)“但是人们也许要问,租地农场主虽然要支付一个工资的追加额,他是否至少还能得到以前的利润率呢?当然不能,因为他不仅要给他所雇用的每个工人支付较高的工资(就象工厂主所要做的那样),而且要支付地租,或者为了得到同量产品而使用追加工人。而原产品价格的上涨只会与这种地租或与追加的工人人数相适应,它不会补偿由于工资的提高而给租地农场主带来的损失。”(同上,第108页)“我们已经指出,在社会发展的早期阶段,在土地产品的价值中,无论是土地所有者还是工人所占的份额都不大,他们所占的份额是随着社会财富的增长以及生产食物困难的增加而成比例地增长的。”(第109页)}\end{quote}

这是关于“社会发展的早期阶段”的一个奇特的资产阶级幻想。在这种早期阶段,劳动者或者是奴隶,或者是靠自己劳动生活的农民等等。在第一种场合,他和土地一起属于土地所有者;在第二种场合,他就是他自己的土地所有者。在这两种场合,都没有介入土地所有者和农业工人之间的资本家。农业从属于资本主义生产,从而奴隶或农民变为雇佣工人,以及在土地所有者和工人之间介入了资本家,——这一切只不过是资本主义生产的最后结果,而李嘉图却看做是“社会发展的早期阶段”所固有的现象。

\begin{quote}{“因此,利润有下降的自然趋势,因为随着社会的进步和财富的增长,为了生产必需的追加食物量,必须花费越来越多的劳动。利润的这种趋势,这种可以说是重力作用,幸而由于生产必需品所使用的机器的改良以及农业科学上的发现而时常受到抑制,这些改良和发现使我们能够减少一部分以前所需要的劳动量,因而能降低工人生活必需品的价格。”(同上,第120—121页)}\end{quote}

李嘉图的下面一段话就直截了当地说,他所理解的利润率就是剩余价值率:

\begin{quote}{“虽然生产了一个较大的价值,但这一价值在支付地租以后剩下的部分中却有较大的份额是由生产者消费的,而这一点,并且只有这一点,却调节着利润的大小。”(同上,第127页)}\end{quote}

这就是说,撇开地租不谈,利润率等于商品价值超过在生产这种商品的时期所支付的劳动价值的余额,或者说,超过生产者所消费的那部分商品价值的余额。李嘉图在这里只把工人叫做生产者\endnote{马克思在这里再一次指出,李嘉图的《原理》中这个地方的“生产者”(《producer》)一词是指“工人”(马克思在前面第478页上第一次指出李嘉图著作中把“生产者”和“工人”这两个概念等同起来)。在李嘉图著作的另外一些地方,“生产者”这个词指“产业资本家”(例如正文第480、487和627页上李嘉图著作的引文)。——第528页。}。他认为所生产的价值是他们生产的。因此,在这里他把剩余价值解释为工人自己生产的价值中为资本家生产的部分。\authornote{[691}关于剩余价值的来源,李嘉图说:“……资本在货币形式上是不会产生利润的;而在它所能交换的材料、机器和食物的形式上,却可以产生收入。”(同上,第267页)“有价证券持有者的资本[692]决不可能成为生产资本——这实际上根本不是资本。如果有价证券持有者想把有价证券卖掉,并将卖得的资本生产地加以使用,他就只有使购买他的有价证券的人的资本离开某种生产用途才能做到。”(第289页注)[692]]

但是,如果李嘉图把剩余价值率和利润率等同起来,如果他同时又假定(而他正是这样做的)工作日的长度是既定的,那末利润率下降的趋势就只能用引起剩余价值率下降的那些原因来解释。在工作日的长度既定时,剩余价值率只有在工资率不断提高的条件下才可能下降。而工资率的不断提高只有在必需品的价值不断提高的情况下才可能发生,必需品价值的提高又只有在农业生产条件不断恶化的情况下,也就是在假定李嘉图的地租理论是正确的条件下才会发生。因为李嘉图把剩余价值率和利润率等同起来,[691]又因为剩余价值率只是按照它同花费在工资上的可变资本之比来计算的,所以李嘉图也和亚·斯密一样,假定全部产品的价值扣除地租之后,是在工人和资本家之间分配的,也就是说,分为工资和利润。换句话说,李嘉图作了一个错误的假定:全部预付资本只由可变资本构成。例如,在前面引用的那段话后面,他接着说:

\begin{quote}{“当较坏土地投入耕种时,或者当花费在老地上的追加的资本量和劳动量收益减少时,上述影响将是持久的:支付地租后剩下的、要在资本所有者和工人中间进行分配的那部分产品中,将有更大的份额归工人所得。”(同上,第127—128页)}\end{quote}

李嘉图紧接着又说:

\begin{quote}{“每个工人所得到的产品的绝对量也许会、甚至很可能会减少;但是,因为同租地农场主留下的全部产品相比,雇用的工人会增加,所以在全部产品中为工资所吸收的那部分价值会增大,因而产品中用来支付利润的那部分价值会减少。”(第128页)}\end{quote}

在前面不远的地方,李嘉图写道:

\begin{quote}{“土地产品在支付土地所有者和工人以后余下的数量,必然属于租地农场主,成为他的资本的利润。”(第110页)}\end{quote}

李嘉图在《论利润》那一部分(第六章)的结尾说,即使假定商品价格随着工人货币工资的提高而一起提高,——这是错误的假定,——他对利润率下降的分析仍然是正确的:

\begin{quote}{“在论工资的那一章,我们已经力求说明……商品的货币价格不会由于工资提高而提高。但是即使情况不是这样,即使高工资引起商品价格持久上涨,认为高工资必然会影响那些使用雇佣劳动的人,使他们失去一部分实际利润的说法仍然是正确的。假定制帽业者、织袜业者、制鞋业者在生产一定量商品时,每人多付10镑工资,而帽子、袜子和鞋子的价格上涨的总额足以补偿他们各人的这10镑。在这种情况下,他们的景况并不会比商品价格没有提高时好些。如果织袜业者的袜子卖得110镑,而不是100镑,他的利润的货币额就恰好和以前一样;但是因为他用这一相同的货币额换得的帽子、鞋子和其他一切商品的数量将会少十分之一,因为他用过去积蓄的数额〈即用同样的资本〉所能雇用的工人会由于工资提高而减少,所能购买的原料也会由于原料价格上涨而减少,所以他的景况并不会比他的货币利润总额实际减少而一切工业品价格不变的时候好些。”(第129页)}\end{quote}

李嘉图在其他地方论证问题的时候总是只强调,在较坏的土地上,为了生产同量的产品必须雇用数量更多的工人,而在这里,他终于提出了对利润率具有决定意义的因素,那就是,用同量资本所雇用的工人由于工资提高而减少。在其他方面,他并不完全正确。如果帽子等等的价格提高10%,资本家的景况不会改变,但是土地所有者在购买这一切商品的时候必须从他的地租中付出较大的数目。例如他的地租从10镑增加到20镑。但是他用这20镑买得的帽子等等的数量,比以前用10镑买得的成比例地减少了。

李嘉图说得完全对:

\begin{quote}{“在社会向前发展的情况下,土地的纯产品同土地的总产品相比,会不断减少”。(第198页)}\end{quote}

李嘉图的这个论点的意思是,在社会向前发展的情况下,地租不会提高。[纯产品同总产品相比会减少的]真实原因在于,在社会向前发展的情况下,可变资本同不变资本相比会减少。[691]

[692]随着生产的进步,不变资本同可变资本相比会增加,这一点李嘉图自己也承认,不过他采取的形式是,固定资本同流动资本相比会增加:

\begin{quote}{“在富强的国家,大量的资本都投在机器上;而在较贫穷的国家,按比例来说,固定资本少得多,流动资本多得多,因而很多工作要靠人的劳动来进行。因此,在富强的国家,商业和工业上的突然变动所带来的灾难,比在较贫穷的国家大。把流动资本从使用它的部门中抽出来,不象固定资本那样困难。为一个工业部门制造的机器,往往完全不能用于其他工业部门;相反,一个部门的工人的衣服、食物和住房却可以用来维持另一个部门的工人的生活}\end{quote}

(因此,这里的流动资本只能理解为用于工资的可变资本),

\begin{quote}{或者说,同一个工人虽然改换了自己的职业,但可能得到同样的食物、衣服和住房。然而,这是富裕国家必须容忍的不幸;为这种不幸而埋怨,就好比一个富商为了他的船只在海上会遇到各种危险,可是他的穷邻居的茅屋完全没有这种危险而长吁短叹一样,是没有道理的。”(同上,第311页)}\end{quote}

李嘉图自己指出了一个与农产品价格提高完全无关的地租提高的原因:

\begin{quote}{“任何固定在土地上的资本,到租佃期满时,都必然属于土地所有者而不属于租地农场主。土地所有者在重新出租他的土地时由于这一资本而得到的任何报酬都将以地租形式出现。但是,如果用一定量资本能从国外购得的谷物比在国内这种土地上生产的多,那就不会有人支付地租。”(同上,第315页注)}\end{quote}

关于同一个问题,李嘉图说:

\begin{quote}{“在本书的前面一个部分,我曾经指出本来意义的地租和土地所有者因支出自己的资本给租地农场主带来各种好处而在地租名义下得到的报酬之间的区别。但是我也许还没有充分说明由于这种资本的使用方法不同而产生的区别。因为这种资本的一部分一旦用来改良农场,就同土地不可分离地结合在一起,并会提高土地的生产力,所以,为了使用土地而支付给土地所有者的报酬完全具有地租的性质,并且受一切地租规律的支配。无论这种改良是由土地所有者还是由租地农场主出钱进行,除非从改良的土地上得到的收益很可能同其他任何等量投资所能获得的利润至少相等,否则这种改良一开始就不会进行。但是只要进行这种改良,以后从改良的土地上得到的收益就会完全具有地租的性质,并将经历地租所经历的一切变动。但是这种费用中有些只能在有限的时期内改良土地,不能长久地提高土地的生产力:比如说,这种费用如果用于建筑物或其他临时性的改良,就需要不断更新,因此它不能使土地所有者的实际地租持久地增加。”(第306页注)}\end{quote}

李嘉图说:

\begin{quote}{“在任何国家,在任何时候,利润都取决于在不提供地租的土地上或者用不提供地租的资本生产工人必需品所需要的劳动量。”(第128页)}\end{quote}

根据这种观点,租地农场主在李嘉图所说的不支付地租的土地即最坏的土地上的利润,调节一般利润率。李嘉图是这样推论的:最坏土地的产品按其价值出卖,并且不会带来任何地租;因此,这里可以看得很清楚,在扣除了只是作为给工人的等价物的那部分产品价值以后,留给资本家的剩余价值究竟有多少;而这种剩余价值就是利润。这种推论所依据的前提是,费用价格和价值是等同的,因为这一产品是按费用价格出卖的,所以它是按照价值出卖的。

从历史上和理论上来看,这是不正确的。我曾经指出\authornote{见本册第330页。——编者注},在资本主义生产和土地所有权存在的地方,最坏的土地或最次的矿井之所以能够不提供地租,只是因为在这种场合,它的产品按市场价值(不由这种产品本身调节)出卖时,是低于它的[个别]价值出卖的。因为这里产品的市场价值正好抵补它的费用价格。但是这种费用价格由什么调节呢?由非农业资本的利润率调节,自然,谷物价格也参与决定利润率,不过绝不能说,利润率仅仅是由谷物价格决定的。李嘉图的论断只有在价值和费用价格等同的情况下才是正确的。[693]从历史上看——由于资本主义生产在农业上比在工业上出现得晚些——也是农业利润由工业利润决定,而不是相反。只有说,在提供利润而不提供地租、其产品按费用价格出卖的最坏的土地上,平均利润率会出现,会明显地表现出来,那才是正确的,但如果说平均利润是由此调节的,那就完全不正确了。这完全是另一回事。

利息率和地租率不提高,利润率也可能下降。

\begin{quote}{“从我们对资本的利润所作的分析中可以看出,如果没有某种引起工资提高的持久的原因,任何资本积累都不能使利润\authornote{李嘉图在这里所说的利润,是指资本家拿去的那部分剩余价值,但决不是全部剩余价值。认为由于资本积累,剩余价值可能降低,那是错误的,而对利润[率}来说,却是正确的。]持久地降低……如果工人消费的必需品的量能够持久地、同样容易地增加,那末无论资本积累达到什么程度,利润率或工资率〈应当说剩余价值率和劳动价值率〉都不会有经常的变动。但是,亚当·斯密却把利润的下降完全归因于资本的积累和由此产生的竞争,而从来不去注意为追加资本所雇用的追加工人提供食物的困难在日益增加。”(同上,第338—339页)}\end{quote}

这种说法只有在利润和剩余价值等同的情况下,才是正确的。

亚·斯密说,随着资本的积累,利润率会由于资本家之间的竞争日益加剧而下降;而李嘉图则说,利润率会由于农业生产条件的不断恶化(必需品涨价)而下降。我们反驳了他的观点,他的这种观点只有在剩余价值率和利润率等同的情况下,也就是说,在利润率只是因为工资率提高(假定工作日不变)才能下降的情况下,才是正确的。斯密的见解所依据的是:他(根据他的错误的、已被他自己驳倒了的价值观点)把价值看作是工资、利润和地租相加的结果。按照他的看法,资本的积累通过降低商品价格的方法迫使任意规定的、本身没有任何内在尺度的利润降低,根据这种观点,对商品价格来说,利润纯粹是一种名义上的附加额。

李嘉图反驳斯密说,资本的积累不会使商品的价值规定发生变动,这一论据在理论上自然是正确的;但是,李嘉图企图用一个国家不可能发生生产过剩这一点来反驳斯密,这就大错特错了。李嘉图否认资本过多的可能性,但在他以后的时期,这种可能性在英国的政治经济学上已经成为公认的原理了。

第一,李嘉图没有看到,在现实生活中不仅资本家和工人彼此对立,而且[产业]资本家、工人、土地所有者、货币资本家、从国家领取固定收入的人等等,都彼此对立;在这里,商品价格的下降,使产业资本家和工人双方都受到打击,而对其他阶级却有利。

第二,李嘉图没有看到,资本主义生产决不是以随便什么样的规模进行都行的,资本主义生产越是发展,它就越是不得不采取与直接的需求无关而取决于世界市场的不断扩大的那样一种规模。李嘉图求助于萨伊的荒谬的前提,似乎资本家进行生产不是为了利润,不是为了剩余价值,而是直接为了消费,为了使用价值——为了他自己的消费。李嘉图没有看到,商品必须转化为货币。工人的需求是不够的,因为利润之所以存在,正是由于工人所能提出的需求小于他们的产品的价值,而相对说来,这种需求越小,利润就越大。资本家彼此提出的需求同样是不够的。生产过剩不会引起利润的持续下降,但是它经过一定时期会不断重复。随着生产过剩,就出现生产不足等等。生产过剩的起因恰好在于:人民群众所消费的东西,永远也不可能大于必要生活资料的平均数量,因此人民群众的消费不是随着劳动生产率的提高而相应地增长。不过,整个这一节都属于资本竞争的问题。关于这一点,李嘉图所说的一切是毫无价值的。(这就是第二十一章《积累对于利润和利息的影响》。)

\begin{quote}{“只有一种情况可能引起利润率在食物价格低廉时随着资本的积累而下降,那就是维持劳动的基金比人口增加快得多,这时工资高,而利润率却低;但这种情况也只具有暂时的性质。”(第343页)}\end{quote}

李嘉图[在《积累对于利润和利息的影响》这一章中]就利润和利息的关系讽刺萨伊说:

\begin{quote}{“萨伊先生承认利息率取决于利润率;但由此不能得出结论说,利润率取决于利息率。前者是因,后者是果,任何情况都不能使因果倒置。”(同上,第353页注)}\end{quote}

但是,使利润下降的那些原因能够使利息提高,反过来也是一样。\endnote{这里引用的李嘉图对萨伊关于利润和利息之间关系的观点的评论,马克思在他的手稿第736页上再次引用了,但把它当作与第736页所谈的问题无关的东西放在方括号里,并在李嘉图的结束语(“任何情况都不能使因果倒置”)后面反驳了一句:“最后这句话‘在某种情况下’肯定是不正确的。”马克思在《资本论》第三卷(第二十二章)中指出了在资本主义周期的不同阶段上利润率和利息率相互对立运动的可能性。马克思写道:“如果我们考察一下现代工业在其中运动的周转周期……我们就会发现,低利息率多数与繁荣时期或有额外利润的时期相适应,利息的提高与繁荣到周期的下一阶段的过渡相适应,而达到高利贷极限程度的最高利息则与危机相适应。”(见马克思《资本论》第3卷第22章)。——第535页。}

\centerbox{※     ※     ※}

[在《论殖民地贸易》一章中,李嘉图写道:]

\begin{quote}{“萨伊先生承认,生产费用是价格的基础,但他在他的著作的不同地方却说价格是由供求关系调节的。”(同上,第411页)}\end{quote}

[否认需求和供给的决定性作用的]李嘉图本来应该从[萨伊把生产费用的见解同需求和供给的见解结合起来的]这种论点中看到,[694][萨伊所谓的]生产费用与用于生产某种商品的劳动量是大不相同的。但他没有这样做,却继续说:

\begin{quote}{“真正地和最后地调节任何两种商品的相对价值的,是它们的生产费用”。(同上)}\end{quote}

[在《论殖民地贸易》这一章中,李嘉图写道:]

\begin{quote}{“亚当·斯密说:‘商品的价格,或者说,金银同商品相比较的价值,取决于使一定量金银进入市场所必需的劳动量和使一定量任何其他商品进入市场所必需的劳动量之间的比例。’他说这句话时难道不是同意这种观点{价格既不是由工资调节,也不是由利润调节}吗?不论利润是高还是低,也不论工资是低还是高,这种劳动量都不会变动。所以,高额利润怎么能够提高商品的价格呢?”(第413—414页)}\end{quote}

在上面引用的这段话里,亚·斯密所说的价格无非是指商品价值的货币表现。商品的价值以及用来交换商品的金银的价值由生产这两类商品{一方面是商品,另一方面是金银}所需要的劳动的相对量决定,这一事实同“高额利润能够提高”商品的实际价格即商品的费用价格这一点决不矛盾。当然,不是象斯密所想的那样,一下子全都如此。但是由于高额利润,的确会有一部分商品的价格比平均利润水平低时更高于这些商品的价值,而另一部分商品的价格则比利润低时低于它们的价值的程度要小些。\endnote{马克思这里回到本册第425—426页和第495—497页上所谈的问题,即从殖民地贸易和一般对外贸易中得到的比在宗主国得到的更高的利润对平均利润率因而对费用价格的影响问题。正如马克思所指出的,在这个问题上,亚当·斯密所持的观点比李嘉图正确。并参看马克思《资本论》第3卷第14章。——第536页。}

\tchapternonum{[第十七章]李嘉图的积累理论。对这个理论的批判。从资本的基本形式得出危机}

\tsectionnonum{[(1)斯密和李嘉图忽视不变资本的错误。不变资本各部分的再生产]}

我们先把李嘉图分散在全书中的论点搜集在一起。

\begin{quote}{“……一个国家的全部产品都是要消费掉的,但究竟由再生产另一个价值的人消费,还是由不再生产另一个价值的人消费,这中间有难以想象的区别。我们说收入节约下来加入资本,我们的意思是,加入资本的那部分收入,是由生产工人消费,而不是由非生产工人消费。〈李嘉图这里所说的区别,也是亚·斯密所说的区别。〉认为资本由于不消费而增加,那就大错而特错了。如果劳动价格大大提高,以致增加资本也无法使用更多的劳动,那我就要说,这样增加的资本仍然是非生产地消费的。”(第163页注)}\end{quote}

可见,这里全部问题只归结为,产品是由工人消费还是不由工人消费。这和亚当·斯密等人的看法一样。而实际上,这里必定也涉及这样一些商品的生产消费,这些商品构成不变资本并作为劳动工具或劳动材料被消费,或者说,这些商品通过消费转化为劳动工具和劳动材料。认为资本积累是收入转化为工资,就是可变资本的积累,这种见解从一开始就是错误的,也就是片面的。这样,对整个积累问题就得出了错误的解释。

首先,必须弄清不变资本的再生产。我们在这里就考察年再生产,也就是把一年作为再生产过程的时间尺度。

不变资本的很大一部分——固定资本——加入年劳动过程,但不[全部]加入年价值形成过程。[不加入价值形成过程的这部分]固定资本不会被消费。所以这部分固定资本不需要再生产。由于它一般加入生产过程并同活劳动接触,它就被保存下来,而且它的交换价值也同它的使用价值一起被保存下来。一个国家当年的这部分资本愈大,下一年这部分资本的纯粹形式上的再生产(保存)相对地也就愈大;假定生产过程即使只以原来的规模更新、继续、前进,情况就是如此。修理和为保存固定资本所必需的其他一切,我们算在原来花费在固定资本上的劳动费用中。这与上述意义上的保存毫无共同之处。

不变资本的第二部分每年在商品生产中会被消费掉,因此必须再生产出来。这里包括每年加入价值形成过程的那部分固定资本的全部,还包括由流动资本构成的那部分不变资本的全部,即原料和辅助材料。

至于不变资本的这第二部分,还应当进一步加以区分。

[695]在一个生产领域内表现为不变资本——劳动资料和劳动材料——的东西,有很大一部分同时就是某个并行的生产领域的产品。例如,棉纱是织布业者不变资本的一部分;棉纱又是纺纱业者的产品,也许前一天它还在制造过程中。这里所说的同时,是指在同一年内进行生产。在同一年内,同一些商品在其不同阶段通过不同的生产领域。它们作为产品从一个领域出来,又作为形成不变资本的商品进入另一个领域。而且它们全都作为不变资本在这一年内被消费掉:或者是作为固定资本只以它们的价值加入商品,或者是作为流动资本连它们的使用价值也加入商品。当一个生产领域生产出来的商品加入另一个生产领域,在这里作为不变资本被消费的时候,在有这一种商品加入的生产领域的序列之外,又有这种商品的不同要素或它的不同阶段同时并行地被生产出来。在同一年内,它不断在一个领域作为不变资本被消费掉,又不断在另一个并行的领域作为商品被生产出来。这样作为不变资本在一年内被消费的同一些商品,又同样不断在同一年内被生产出来。机器在A领域被磨损,同时会在B领域被生产出来。生产生活资料的生产领域在一年内所消费的不变资本,会同时在另一些生产领域被生产出来,因而会在一年内或在年终以实物形式重新得到补偿。无论是生活资料还是这部分不变资本,两者都是新的劳动、在一年内发挥作用的劳动的产品。

我在前面曾经说明\authornote{见本卷第1册第111—126和238—248页。——编者注},生产生活资料的那些生产领域的产品的一部分价值,即补偿这些生产领域的不变资本的那部分价值,是怎样形成这种不变资本的生产者的收入的。

但是,还有一部分不变资本,它每年都被消费掉,却不作为组成部分加入生产生活资料(供[个人]消费的商品)的那些生产领域。因此,这一部分也不能从这些领域中得到补偿。我们指的是不变资本——劳动工具、原料、辅助材料——的一部分,就是在不变资本——机器、原料和辅助材料——本身的形成过程即生产过程中用于生产消费的那部分。我们已经看到\authornote{见本卷第1册第126—140、182—195和248—258页。——编者注},这一部分是以实物形式得到补偿的,或者直接由这些生产领域本身的产品(例如种子、牲畜、一部分煤炭)来补偿,或者通过不同生产领域的那些形成不变资本的产品的一部分[在生产资料生产者之间]进行交换来补偿。这里就发生资本同资本的交换。

由于这部分不变资本的存在和消费,不仅产品量增加了,而且年产品的价值也增大了。和这部分消费掉的不变资本的价值相等的那部分年产品价值,会把必须以实物形式补偿消费掉的不变资本的那一部分,以实物形式从年产品中买回或者抽回。例如,播种时由种子构成的那部分价值,决定着收获时必须作为不变资本归还给土地即归还给生产的那部分价值(同时也决定着谷物量)。没有一年内新加的劳动,这部分就不能再生产出来;但在事实上,这部分是由去年的劳动或[一般说来]过去的劳动生产的,而且——如果劳动生产率不变——由这部分加在年产品中的价值,并不是当年劳动的结果,而是去年劳动的结果。一国使用的不变资本的比例愈大,生产不变资本所消费的那部分不变资本也就愈大,这部分不变资本不仅表现为较大的产品量,而且使这个产品量的价值提高。可见,这部分价值不仅是现在劳动、当年劳动的结果,而且同样是去年劳动、过去劳动的结果,虽然没有当年的直接劳动,它就不能重新出现,正如它所加入的产品不能出现一样。如果这部分不变资本增加了,那末不仅年产品量会增加,而且年产品的价值也会增加,即使年劳动量保持不变。这种增加就是资本积累的形式,理解这种形式非常重要。可是李嘉图的下述论点简直和这种理解相差太远了:

\begin{quote}{“工业中100万人的劳动总是生产出相同的价值,但并非总是生产出相同的财富。”(同上,第320页)}\end{quote}

假定工作日是既定的,这100万人不仅会因劳动生产率不同而生产出极不相同的商品量,而且这个商品量也会由于生产它时花费的不变资本的大小不同,从而由于加到它上面的,由去年劳动、过去劳动创造的价值的大小不同,而具有极不相同的价值。

\tsectionnonum{[(2)不变资本的价值和产品的价值]}

在这里,凡是谈到不变资本的再生产的地方,为简单起见,我们总是先假定劳动生产率不变,因而生产方式也保持不变。在生产规模既定的情况下,应当作为不变资本来补偿的是一定量的实物形式的产品。如果生产率不变,这个量的价值[696]也就保持不变。如果劳动生产率发生变动,因而把同量产品再生产出来,可能付出较贵或较廉的代价,花费较多或较少的劳动,那末,不变资本的价值也就发生变动,这种变动会影响产品在扣除不变资本以后剩下的余额的大小。

例如,假定播种需要20夸特[小麦],每夸特3镑,共计60镑。如果再生产一夸特所花费的劳动减少1/3,每夸特就只值2镑。应当作为播种费用从产品中扣除的仍然是20夸特,但它们在全部产品的价值中所占的部分现在只等于40镑。这样,为补偿同量不变资本就只需要总产品的一个较小的价值部分和总产品的一个较小的实物部分,虽然作为种子归还给土地的仍然应当是20夸特\endnote{这个例子是根据这样的假定,即在劳动生产率提高的情况下,从20夸特小麦的种子得到的收成比以前增加50%。例如,以前收成是100夸特,现在花费同量劳动,收成是150夸特,但是这150夸特和以前100夸特的价值一样,即300镑。以前种子占收成的20%(无论就夸特数来说,还是就价值来说都是这样),现在只占[13+(1/3)]%。——第541页。}。

如果每年消费的不变资本在一个国家是1000万镑,在另一个国家只是100万镑,而100万人一年内新加的劳动表现为1亿镑,那末产品价值在前一个国家就是11000万镑,在后一个国家就只是10100万镑。在这种情况下,第一个国家的单位商品不但可能而且毫无疑问会比第二个国家便宜,因为第二个国家花费同量的[直接]劳动生产出来的商品量少得多,比10与1之差少得多。当然,和第二个国家相比,第一个国家要拿出产品的更大一部分价值,因而要拿出总产品的更大一部分,用于补偿资本。但是第一个国家的总产品也多得多。

就工业品来说,大家知道,拿英国比如说同俄国相比,100万人生产的产品,不仅数量多得多,而且产品价值也大得多,虽然英国的单位商品便宜得多。但就农业来说,看来在资本主义发达的国家和比较不发达的国家之间就不存在这样的关系。落后国家的产品比资本主义发达的国家的产品便宜。这是就货币价格来说的。然而,看来发达国家的产品比起落后国家的产品来,则是劳动量(一年内花费的劳动量)少得多的产品。例如,在英国从事农业的人口不到三分之一,在俄国从事农业的人口却有五分之四,在英国是5/15,在俄国则是12/15。这些数字不应当从字面上去理解。例如在英国,在机器制造业、商业、运输业等等非农业经济部门,有大批的人从事农业生产各要素的制造和输送,而在俄国就没有。可见,从事农业的相对人数,不能简单地由直接从事农业的人数来决定。在进行资本主义生产的国家,有许多人间接地参加这种农业生产,而在不发达的国家,这些人都是直接从属于农业的。因此,表现出来的差别要比实际的差别大。但是对于一国文明的总的水平来说,这个差别极为重要,那怕这个差别只在于,有相当大一部分参与农业的生产者不直接参加农业,而摆脱了农村生活的愚昧,属于工业人口。

首先,我们不谈这一点。其次,我们也不谈这样一种情况,就是大多数农业民族不得不低于自己产品的价值出卖产品,而在资本主义生产发达的国家,农产品的价格却提高到它的价值的水平。无论如何,有一部分不变资本的价值加入英国土地耕种者的产品的价值,却没有这样一部分不变资本的价值加入俄国土地耕种者的产品的价值。

假定这部分价值等于10个人的日劳动。再假定这个不变资本由1个英国工人推动。我所说的是农产品中不是用花费[土地耕种者的]新劳动来补偿的那部分不变资本,如农具。如果1个英国人用[等于10工作日的]不变资本生产出来的产品,需要5个俄国工人才能生产出来,如果俄国人使用的不变资本等于1工作日,那末,英国人的产品就等于10+1=11工作日,俄国人的产品就等于1+5=6工作日。如果俄国的土地比英国肥沃,以致不使用不变资本或只使用十分之一的不变资本生产出来的谷物,就和英国人使用十倍资本生产出来的一样多,那末,同量的英国谷物的价值和同量的俄国谷物的价值之比将是11∶6。如果俄国谷物每夸特卖2镑,那末英国谷物每夸特就要卖3+(2/3)镑,因为2∶[3+(2/3)]=6∶11。可见,英国谷物的货币价格和价值比俄国谷物的货币价格和价值高得多,然而英国谷物是花费较少量的[直接]劳动生产出来的,因为过去劳动无论是在产品量中,还是在产品价值中再现出来,都无须花费任何追加的新劳动。只要英国人比俄国人使用较少的直接劳动而使用较多的不变资本,并且,只要这种不变资本——它无须英国人花费什么[在花费新劳动的意义上说],虽然它曾经花费过[一定的费用],并且必须得到支付,——没有把劳动生产率提高到足以抵销俄国土壤的自然肥力的程度,英国谷物的价格和价值较高的情况就会始终存在。因此,在进行资本主义生产的国家,农产品的货币价格可能比[697]不发达的国家高,虽然实际上这种产品花费的劳动量较少。这种产品包含较多的总劳动——直接劳动加过去劳动,但再现在这种产品中的过去劳动不需要任何[新]花费。如果不是自然肥力的差别发生影响,产品就会比较便宜。[发达的资本主义国家中]工资的较高的货币价格也可以用这种情况来说明。

到现在为止,我们谈的只是现有资本的再生产。工人补偿自己的工资,同时提供剩余产品或剩余价值,剩余价值形成资本家的利润(包括地租)。工人补偿重新用作他的工资的那一部分年产品。资本家已在一年内把利润吃光,但是工人又生产了可以重新作为利润被吃掉的这部分产品。在生活资料的生产中消费的那部分不变资本,由一年内新劳动生产的不变资本来补偿。生产这部分新的不变资本的生产者,在一部分生活资料上实现自己的收入(利润和工资),这部分生活资料的价值同生产生活资料时所消费的不变资本的价值相等。最后,在生产不变资本即生产机器、原料和辅助材料时消费的不变资本,由生产不变资本的各个生产领域的总产品以实物形式或通过资本同资本的交换来补偿。

\tsectionnonum{[(3)资本积累的必要条件。固定资本的折旧及其在积累过程中的作用]}

资本增殖,即与再生产不同的资本积累,即收入转化为资本,情况又怎样呢?

为使问题简单起见,假定劳动生产率不变,生产方式没有任何变化,因此,生产同量商品需要同量的劳动,也就是说,资本增殖花费的劳动量,和去年生产同量资本花费的劳动量一样。

剩余价值的一部分必须转化为资本,而不是作为收入被消费。它必须一部分转化为不变资本,一部分转化为可变资本。它分成资本的这两个不同部分的比例,取决于资本已有的有机构成,因为生产方式不变,两部分之间的价值比例也不变。生产愈发展,转化为不变资本的那部分剩余价值,同转化为可变资本的那部分剩余价值相比,就愈大。

首先,剩余价值的一部分(以及与这一部分相应的由生活资料构成的那部分剩余产品)必须转化为可变资本,即必须用来购买新劳动。这只有在工人人数增加或工人的劳动时间延长的情况下才有可能。后一种情况例如在一部分工人人口只是半就业或三分之二就业的时候,或者在一个或长或短的时期内绝对延长工作日但必须对此支付报酬时,都会发生。但是不能把这看作是积累的经常的手段。如果原来的非生产劳动者变成生产劳动者,或者原来不劳动的那部分人口如妇女、儿童、贫民被吸收到生产过程中来,工人人口就可能增加。这里我们把后一点撇开不谈。最后,由于工人人口随着整个人口的增加而绝对增加,[就业工人的人数也可能增加。]只有在人口这样绝对增加(虽然和使用的资本相比,人口相对减少了)的条件下,积累才能成为经常的不断的过程。人口增加表现为积累这个经常过程的基础。但是这就需要有一种不仅能够再生产工人人口,而且能够使工人人口不断增加的平均工资。为了应付突然情况,资本主义生产已作了准备:它迫使一部分工人人口进行过度劳动,又使另一部分工人人口陷于赤贫或半赤贫状态,作为后备军储备起来。

然而,另一部分必须转化为不变资本的剩余价值,情况又怎样呢?为了简单起见,我们就撇开对外贸易,考察一个与外界隔绝的国家。我们举一个例子。假定一个麻织厂主生产的剩余价值等于1万镑,他想把半数即5000镑转化为资本。根据机器织布业的资本有机构成,这个金额的五分之一要花费在工资上。这里我们把资本周转撇开不谈,如果考虑到资本周转,工厂主也许只要有够五周用的金额就行了,五周之后,他把自己的产品卖出去,就可以从流通领域中把用于工资的资本收回来。我们假定,他必须把1000镑存在银行家那里,以支付(20个工人的)工资,并在一年内作为工资逐渐花完。然后,4000镑必须转化为不变资本。第一,工厂主必须购买够20个工人在一年内加工织成麻布的纱。(我们始终把资本的流动部分的周转撇开。)其次,工厂主必须增加自己工厂中的织机,也许还要添置新的蒸汽机,或者加大旧机器的功率等等。但是要买到所有这些东西,他必须在市场上找到现成的纱、织机等等。他必须把他的4000镑变成纱、织机、煤炭等等,[698]即购买所有这些东西。但要能买到这些东西,这些东西必须已经存在着。因为我们已经假定,旧资本的再生产是在原有条件下进行的,所以,为了提供织布业者上一年所需要的那么多的纱,纺纱业者必定已经支出他的全部资本。那末,他怎样才能供给更多的纱来满足追加的需求呢?

提供织机等等的机器制造厂主的情况也正是这样。他生产的新织机数量,只够织布业补偿机器的平均损耗。但是,满怀积累欲的织布厂主拿3000镑去定购纱,拿1000镑去定购织机、煤炭(因为煤炭业者的情况也是这样)等等。或者说,他给纺纱厂主3000镑,给机器制造业者和煤炭业者等等1000镑,让他们替他把这些货币变成纱、织机和煤炭。因此,他必须等到这个过程结束后,才能开始自己的积累,开始自己新麻布的生产。这是第一个中断。

然而,得到3000镑的纺纱厂主现在的处境,也和拥有4000镑的织布厂主一样,区别只在于他会马上从得到的3000镑中扣下自己的利润。他可能会找到追加数量的纺纱工人,但是他需要亚麻、纱锭、煤炭等等。煤炭业者也一样,他除需要新工人以外,还需要新的机器或工具。而那个必须提供新的织机、纱锭等等的机器制造厂主,除需要追加的工人以外,还需要铁等等。亚麻生产者的情况最糟,他只有在下一年才能把追加量的亚麻提供出来,如此等等。

可见,织布厂主为了能够不拖延地、不间断地每年把他的一部分利润转化为不变资本,——并且为了使积累成为不断的过程,——就必须在市场上找到现成追加量的纱、织机等等。如果他以及纺纱厂主、煤炭业者等等在市场上能找到现成的亚麻、纱锭和机器,那他们就只需要雇用更多的工人了。

每年算作损耗并作为损耗加入产品价值的那部分不变资本,事实上并没有消耗掉。我们举一台机器为例,这台机器能用12年,价值12000镑;这台机器每年应当计算的平均损耗等于1000镑。既然每年有1000镑加入产品,那末到12年结束时就会再生产出12000镑的价值,并且能够用这个价格购买一台同一类型的新机器。这12年中必要的修理和日常维修,我们算入机器的生产费用,这些同我们的问题毫无关系。然而在事实上,实际的情况和这种平均的计算是不同的。机器在第二年可能比第一年好用。不过12年后它毕竟不能再使用了。这里的情况和家畜一样,一头家畜平均寿命为10年,但它并不因此每年死去十分之一,虽然10年后必须换一头新的。当然,在同一年中,总有一定数量的机器等等会达到确实必须换新机器的阶段。因此,每年都有一定数量的旧机器等等确实需要在实物形式上用新机器来替换。机器等等每年的平均生产就是与此相适应的。用来支付这些机器的那些价值按照它们(机器)再生产的时间从商品的卖款取得。但事实仍然是:虽然年产品价值(每年用来支付年产品的价值)有相当一部分比如说在12年后必须用来购买新机器以替换旧机器,但实际上决不需要每年都在实物形式上换掉旧机器的十二分之一,而且事实上也办不到。这个基金的一部分,在商品卖出或被支付以前可以用来发放工资或购买原料,因为商品不断地投入流通领域,但并不是立即从流通领域中回来。不过每一次使用这个基金都不可能延续一整年,因为一年周转一次的商品必须完全实现其价值,即必须支付,实现它所包含的工资、原料、机器损耗和剩余价值。

可见,凡是使用许多不变资本,因而也使用许多固定资本的地方,补偿固定资本损耗的这部分产品价值就是积累基金,这个基金可以被使用它的人用来作为新固定资本(或流动资本)的投资,而且这部分积累根本不是从剩余价值中扣除的。(见麦克库洛赫的著作。)\endnote{括号中的话“见麦克库洛赫的著作”是马克思后来(用铅笔)加的。马克思在1862年8月20日给恩格斯的信中第一次提出关于折旧基金用于积累的思想,他在1867年8月24日给恩格斯的信中提到前封信时告诉恩格斯,他后来在麦克库洛赫的著作中发现了有关这方面的一些暗示。马克思指的是麦克库洛赫的《政治经济学原理》1825年爱丁堡版第181—182页。马克思在《剩余价值理论》第三册中,在手稿第777页和第781页上又谈到了这个问题。——第548页。}这种积累基金在那些没有大量固定资本的生产阶段和国家是不存在的。这是重要的一点。这是一个不断用于改良、扩大等方面的基金。

\tsectionnonum{[(4)积累过程中各生产部门之间的联系。剩余价值的一部分直接转化为不变资本是农业和机器制造业中积累的特点]}

但是,我们这里所要研究的问题是这样的。即使投在机器制造业的全部资本仅够补偿机器每年的损耗,它所生产的机器也会比每年所需要的机器多得多,因为损耗有一部分只是在观念上存在,而在现实中只是过若干年之后才要以实物形式补偿。可见,这样使用的资本每年会提供大量的机器,这些机器可以用于新的投资,并且使这种新的投资提前实现。例如,一个机器制造厂主在本年内开始他的生产。假定他在这一年内提供12000镑的机器。这样,如果要把他所生产的机器简单再生产出来,在以后11年中,他每年只须生产1000镑的机器就行了,而且连这个年产量也不是每年都被消费掉。如果他使用的是他的全部资本,那末他的产品中被消费掉的部分就更小了。为了使他的资本保持运动,并且每年只实现[699]资本的简单再生产,那些需要这种机器的部门就必须继续不断地扩大生产。(如果这个机器制造厂主自己也进行积累,那就更是如此了。)

因此,即使在这个生产领域中投入的资本只是进行再生产,其他生产领域就必须不断进行积累。另一方面,只是由于机器制造业进行简单再生产,其他生产领域的不断积累才能不断在市场上现成地找到自己的要素之一。这里,即使一个生产领域本身进行的只是现有资本的简单再生产,在这个生产领域也经常有商品储备,供其他各生产领域用于积累,用于新的追加的生产消费。

至于被资本家比如说织布厂主转化为资本的那5000镑利润或剩余价值,可能有两种情况。我们始终假定,他在市场上找得到劳动,而他必须从这5000镑中拿出1000镑购买劳动,以便按照他这个生产领域的条件把这5000镑转化为资本。这部分[资本化的剩余价值]转化为可变资本,花费在工资上。但是为了使用这种劳动,工厂主就需要有纱、追加的机器和追加的辅助材料。{只有在工作日不延长的情况下,才需要追加的机器。在工作日延长的情况下,机器只是磨损得快些,机器再生产的时间会缩短,但同时会生产出更多的剩余价值;虽然机器的价值必须分摊到在较短时间内生产出来的商品上,然而生产出来的商品多得多;所以虽然磨损得快些,可是加入单位商品的价值或价格的那部分机器价值却小些。在这种情况下,不必把新资本直接花费在机器本身。只要补偿机器的价值稍快些就行了。但是在这种情况下,辅助材料需要预付追加的资本。}或者织布厂主能够在市场上现成地找到他的这些生产条件。这时,购买这些商品和购买其他商品不同的地方,只在于他购买商品是为了生产消费,而不是为了个人消费。或者他在市场上找不到这些现成的东西。这时,他就得定购这些东西(例如要买新结构的机器),就象他不得不定购市场上不能现成找到的那些个人消费品一样。如果原料(亚麻)只是根据定购进行生产{如靛蓝、黄麻等等,印度农民就是根据英国商人的定购和预付来生产的},那末织布厂主当年要在他自己的企业中进行积累就不可能了。另一方面,假定纺纱厂主把他的5000镑变成资本,而织布厂主不进行积累,那末,虽然市场上存在着生产纱的一切条件,但纱将卖不出去,这5000镑固然已转化为纱,但是没有转化为资本。

(关于信用,我们在这里不需要详谈。信用使积累资本可以不用在把它生产出来的那个领域,而用在它的价值增殖的机会最多的地方。但是,每个资本家都宁愿把他积累的资本尽量投在自己的部门。如果他把资本投在别的部门,他就成了货币资本家,得不到利润,只得到利息;或者他不得不去进行投机。但我们这里是谈平均积累,并且只是为了举例才假定积累资本投入这个或那个特殊部门。)

另一方面,如果亚麻种植业者扩大了生产,即进行了积累,而纺纱厂主、织布厂主、机器制造厂主等却没有进行积累,那末亚麻种植业者的仓库里就会有过剩的亚麻,下一年也许就会减少生产。

{这里我们暂时把个人消费完全撇开,只考察生产者之间的联系。如果存在这种联系,那末首先,对于生产者必须互相补偿的那些资本来说,他们会互相成为市场;新就业的或就业情况较好的工人会成为一部分生活资料的市场;因为剩余价值在下一年会增长,所以资本家能够消费自己收入中增长的部分,从而在一定程度上又会互相成为市场。可是本年的产品有相当一部分仍然不能实现。}

现在问题应该这样来表述:假定普遍进行积累,即假定在所有部门中都进行或多或少的资本积累,——而这实际上是资本主义生产的条件,资本家作为资本家来说强烈追求这一点,正象货币贮藏者强烈追求货币积累一样(不过这也是资本主义生产向前发展所必需的),——那末这种普遍积累的条件是什么,普遍积累究竟是什么意思呢?或者说,因为可以把织布厂主看作全体资本家的代表,那末为了使他能顺利地把5000镑剩余价值再转化为资本,并且逐年不断地把积累过程继续下去,需要些什么条件呢?积累5000镑,无非是把这些货币,把这个数额的价值,转化为资本。可见,资本积累的条件同原来生产或再生产资本的条件是完全一样的。

而这些条件就是:用一部分货币购买劳动,用另一部分货币购买能由这种劳动进行生产消费的商品(原料、机器等等)。{某些商品,例如机器、原料、半成品等等,只能供生产消费。其他一些商品,例如房屋、马匹、小麦、黑麦(可以用来造酒或制淀粉等)等等,既可供生产消费,也可供个人消费。}为了能够买到这些商品,它们就必须作为商品存在于[700]市场上,即存在于已经结束的生产和尚未开始的消费之间的中间阶段,存在于卖者手中,存在于流通阶段;或者根据定购可以得到供应(例如建造新工厂等等,就用定购的办法)。由于在资本主义生产条件下存在着社会规模的分工(劳动和资本在各个不同部门之间进行分配),由于生产、再生产在所有领域同时进行,情况也就是如此,这是进行资本生产和再生产的前提。这是市场的条件,是资本生产和再生产的条件。资本愈多,劳动生产率愈高,总之,资本主义生产的规模愈大,存在于从生产到消费(个人消费和生产消费)的过渡阶段,存在于流通中,存在于市场上的商品量就愈多,每一笔资本在市场上现成地找到自己再生产条件的把握也就愈大。情况之所以必然是这样,还因为按照资本主义生产的本质,第一,每一笔资本活动的规模,并不决定于个人需求(定购等等,私人需要),而是决定于力求实现尽可能多的劳动,因而实现尽可能多的剩余劳动,并用现有的资本提供尽可能多的商品的欲望;第二,每一笔资本都力求在市场上占据尽可能大的地盘,并竭力排挤、排除自己的竞争者。资本竞争。

{交通工具愈发达,市场上的存货就愈能减少。

\begin{quote}{“凡是生产和消费比较大的地方,在任何时候自然都会有比较多的剩余存在于中间阶段,存在于市场上,存在于从生产者到消费者的道路上,除非物品卖出的速度大大加快,消除了生产的增加本来会引起的这些后果。”(《论马尔萨斯先生近来提倡的关于需求的性质和消费的必要性的原理》1821年伦敦版第6—7页)}}\end{quote}

可见,新资本的积累只能在和已有资本再生产条件相同的条件下进行。

{我们在这里完全不谈这样一种情况:积累的资本大于能够投入生产的数量,例如资本以货币形式存放在银行家手里而不使用。由此会产生向国外贷款等等,一句话,产生投资的投机。我们也不考察有大量生产出来的商品不能卖出,出现危机等情况。这属于论述竞争的那一部分。这里我们要研究的只是资本在它运动的各个阶段上所采取的形式,而且总是假定商品会按其价值出卖。}

如果织布厂主除了用1000镑购买劳动外,还能在市场上找到现成的纱等等,或者能够定购到这些东西,他就能把5000镑剩余价值再转化为资本。为了能买到这些东西,就必须有追加的产品生产出来,这种追加的产品包括加入他的不变资本的商品,特别是包括需要较长时间才能生产出来而产量不能迅速增加或根本不能在当年增加的商品;原料,例如亚麻,就是这样的情况。

{这里商人资本就出现了,商人资本为日益增长的消费——个人消费或生产消费——把现成的商品贮存在仓库中;但这只是中介形式之一,因而不是这里所要谈的,而是考察资本竞争时所要谈的。}

正如一个领域中现有资本的生产和再生产以其他领域中并行的生产和再生产为前提,一个部门中的积累,或者说,追加资本的形成,也以其他部门中同时或并行地进行的追加生产为前提。因此,在所有提供不变资本的领域中,生产规模必须同时扩大(按照由需求决定的、每个特殊领域在整个生产增长中应承担的平均份额来扩大);所有不为个人消费提供成品的领域,都提供不变资本。其中最重要的是机器(工具)、原料、辅助材料的增加,因为当这些条件具备的时候,有这些东西加入的其他一切生产部门,不论是提供半成品还是提供成品,就只须推动更多的劳动了。

因此,为了能进行积累,看来所有领域都必须不断追加生产。

这一点还要稍为详细地加以说明。

其次,第二个重大问题:

再转化为资本的那部分剩余价值,或者说,——因为这里谈的是利润,——再转化为资本的那部分利润(包括地租;如果土地所有者想进行积累,想把地租转化为资本,那末剩余价值就总是落到产业资本家手中;甚至在工人把他自己的一部分收入转化为资本时,情况也是如此),只是由前一年的[701]新加劳动构成。现在要问,这笔新资本是否全部花费在工资上,是否只和新劳动交换?

赞成的说:一切价值最初都由劳动产生。一切不变资本最初都完全象可变资本一样是劳动的产品。看来,在这里我们又成了资本直接由劳动产生的见证人。

反对的说:难道追加资本的形成必须在比旧资本的再生产更坏的生产条件下进行吗?难道追加资本的形成必须回到生产方式的更低阶段吗?可是,如果新价值只花费在直接劳动上,因而这种直接劳动在没有固定资本等等条件下,必须先把这种资本生产出来,正象劳动最初不得不先把自己的不变资本创造出来那样,那末,情况就一定是这样。这纯粹是无稽之谈。但是,这是李嘉图等人的前提。这一点要较详细地谈谈。

这里产生的第一个问题是:

如果资本家不把一部分剩余价值卖掉,或者更确切地说,不把代表这部分剩余价值的剩余产品卖掉,而是把它直接当作资本使用,这一部分剩余价值能够由此转化成资本吗?如对这个问题作肯定的回答,那就包含着这样的结论:应当转化为资本的剩余价值并不是全都转化为可变资本,或者说,花费在工资上。

就谷物和牲畜构成的那部分农产品来说,这个问题从一开始就很清楚。收成中代表租地农场主的剩余产品或剩余价值的那部分谷物(一部分牲畜也是一样),可以不拿去卖,而立即作为种子或役畜再当作生产条件来使用。土地本身所生产的一部分肥料也是如此,这种肥料同时也可以作为商品在商业中流通,即可以出卖。租地农场主可以把作为剩余价值,作为利润得到的这部分[未进入流通的]剩余产品,立即在他自己的生产领域内再转化为生产条件,即直接转化为资本。这部分并不花费在工资上,并不转化为可变资本。这部分从个人消费中抽出来,而又不是在斯密和李嘉图所说的那种意义上生产地消费掉。这部分用于生产消费,然而是作为原料来消费的,不是作为生产劳动者或非生产劳动者的生活资料来消费的。谷物不仅可以用作生产工人等等的生活资料,而且可以用作牲畜的饲料,用作酿酒、制淀粉等等的原料。牲畜(肉用牲畜或役畜)也不仅可以用作生活资料,而且可以为许多工业部门提供原料,即毛皮、皮革、油脂、骨、角等等,同时还可以部分地为农业本身,部分地为运输业提供动力。

有一些生产部门,再生产时间超过一年(如大部分畜牧业,林业等等),但是这些部门的产品同时必须不断再生产,即不断要求投入一定量的劳动;在所有这些部门中,因为不仅代表有酬劳动,而且代表无酬劳动的新加劳动必须以实物形式积累到产品能够出卖时为止,所以积累和再生产是一致的。

(这里所谈的不是每年按照一般利润率归并[到资本中]的利润的积累;这不是实际的积累,而只是一种计算方法。这里所谈的是在若干年内反复进行的总劳动的积累,因而,在这种积累中,不仅有酬劳动,而且无酬劳动也以实物形式积累起来,并且立即再转化为资本。至于在这种情况下利润的积累,它反而不[直接]取决于这里新加的劳动量。)

经济作物(不论它们是提供原料,还是提供辅助材料)的情况也是这样。经济作物的种子,以及能够再作为肥料等等使用的那部分经济作物,都代表总产品的一部分。如果它们不拿去出卖,那丝毫也不会改变这样一个事实:它们一旦作为生产条件重新加入生产,就形成[新产品的]总价值的一部分,并以[702]这样的一部分构成新的生产中的不变资本。

这样就已经说明了一个主要问题——关于作为本来意义的农产品的原料和生活资料(食物)的问题。可见,这里积累是同更大规模的再生产直接一致的,因此,剩余产品的一部分直接在本生产领域内再用作生产资料,而不用经过同工资或其他商品相交换。

第二个主要问题是机器。这里指的并不是生产商品的机器,而是生产机器的机器,是机器制造业的不变资本。如果已经有了这种机器,那就只需要花费在采掘工业中为各种容器和机器提供原料(铁等等)的劳动了。有了机器制造机,也就有了对原料本身加工的机器。我们在这里遇到的困难,是不陷入前提的循环论证中。这种循环论证就是:为了生产更多的机器,就必须有更多的材料(铁等等,煤炭等等),而为了生产这种追加的材料,又必须有更多的机器。无论我们是否假定,生产机器制造机的工业家和生产机器(用机器制造机来生产)的工业家是属于同一个资本家集团,都不会使问题本身发生任何改变。有一点是明显的。剩余产品的一部分表现为机器制造机(剩余产品表现为这种形式,至少取决于机器制造厂主)。这种机器制造机不一定非卖不可,它们能够以实物形式作为不变资本再加入新的生产。因此,在这里我们又看到了第二类作为不变资本直接(或通过同一生产领域内的交换)加入新的生产(积累),而不用经过先转化为可变资本的过程的剩余产品。

剩余价值的一部分能否直接转化为不变资本的问题,首先归结为这样一个问题:代表剩余价值的剩余产品的一部分能否直接作为生产条件再加入本生产领域,而无须先让渡出去。

一般的规律是:

只要产品的一部分,从而剩余产品(即代表剩余价值的使用价值)的一部分,能够直接地,不通过中介,作为生产条件,作为劳动资料或劳动材料再加入它从中出来的那个生产领域,那末,这个生产领域中的积累就可能并且必定采取这种形式:剩余产品的一部分不拿去出卖,而是直接(或通过与同一生产领域内以同样方式进行积累的其他部门的资本家相交换)作为再生产的条件重新加入生产过程,所以在这里,积累和更大规模的再生产直接一致。它们两者必然到处都是一致的,但是不一定采取这种直接的方式。

一部分辅助材料的情况也是如此。例如一年内生产出来的煤炭就是这样,剩余产品的一部分能够被用来重新生产煤炭,因而,能够被煤炭业者直接地,不通过任何中介,作为不变资本用于更大规模的生产。

在工业界,有一种为工厂主建造整座工厂的机器制造业者。假定[他们的产品的]1/10是剩余产品,或者说,无酬劳动。显然,这1/10即剩余产品,究竟是表现为替第三者建造并卖给第三者的工厂建筑物,还是表现为机器制造业者替自己建造并卖给自己的工厂建筑物,都不会使问题发生任何改变。这里,问题只在于代表剩余劳动的那种使用价值的性质,只在于这种使用价值能不能作为生产条件再加入拥有这个剩余产品的[703]资本家的生产领域。这里我们又有了一个例子,表明使用价值这个范畴对于决定经济形式具有重要的意义。

这样,这里我们已经看到,剩余产品(因而也就是剩余价值)有很大一部分能够并且必须直接转化为不变资本,以便作为资本被积累起来,没有这部分剩余产品,就根本不可能有任何资本积累。

第二,我们已经看到,在资本主义生产高度发展,因而劳动生产率水平很高,因而不变资本很大,特别是由固定资本构成的那部分不变资本很大的地方,一切领域的固定资本的简单再生产,以及与此并行的生产固定资本的现有资本的再生产,就会形成一个积累基金,也就是为更大规模的生产提供机器,提供不变资本。

第三,剩下还有一个问题:

剩余产品的一部分能不能通过比如说机器、工具等等的生产者和原料即铁、煤炭、金属、木材等等的生产者之间的(间接)交换,即通过不变资本不同组成部分之间的交换,再转化为资本(不变资本)?例如,如果生产铁、煤炭、木材等等的工厂主向机器制造业者购买机器或工具,而机器制造业者向这些原料生产者购买金属、木材、煤炭等等,那末他们就是通过这种交换来互相补偿各自不变资本中相互有关的组成部分,或形成新的不变资本。这里的问题是,这种情况在多大程度上适用于剩余产品。

\tsectionnonum{[(5)资本化的剩余价值转化为不变资本和可变资本]}

我们在前面\authornote{见本卷第1册第126—140、182—195和248—258页。——编者注}已经看到,在现有资本进行简单再生产的条件下,在不变资本的再生产中所使用的那部分不变资本,或者直接以实物形式补偿,或者通过不变资本生产者之间的交换来补偿;这是资本同资本的交换,不是收入同收入的交换,或收入同资本的交换。其次,在消费品(加入个人消费的物品)的生产中所使用的,或者说,用于生产消费的不变资本,则由作为新加劳动的结果因而归结为收入(工资和利润)的同类新产品来补偿。与此相适应,在生产消费品的那些领域中,其价值用于补偿这些领域中所使用的不变资本的那部分产品,代表不变资本生产者的收入,相反,在生产不变资本的那些领域中,代表新加劳动,因而形成这种不变资本生产者的收入的那部分产品,则代表生活资料生产者的不变资本(用于补偿[在生活资料生产中消费掉的不变]资本)。因此,这就要求不变资本的生产者用他们的剩余产品(在这里,是他们产品中超过同他们的不变资本相等的那部分产品的余额)去交换生活资料,把这种剩余产品的价值用于个人消费。同时,这种剩余产品包括:

(1)工资(或再生产出来的工资基金),这一部分必须(由资本家)仍旧用在工资上,即用于个人消费(如果假定是最低限度的工资,那末工人就只能把他得到的工资实现在生活资料上);

(2)资本家的利润(包括地租)。这一部分如果很大,那就可以一部分用于个人消费,一部分用于生产消费。而在用于生产消费的情况下,不变资本的生产者之间就会进行产品交换;但这里已经不是代表他们应当互相补偿的不变资本的那部分产品的交换,而是一部分剩余产品,即收入(新加劳动)的交换,这部分剩余产品直接转化为不变资本,由于这种转化,不变资本的量就增加,再生产的规模就扩大。

因此,就是在这样的情况下,也有一部分现有的剩余产品,即一部分一年内新加的劳动,直接转化为不变资本,而不用先转化为可变资本。因此,这里也可以看出,剩余产品用于生产消费,或者说,积累,绝不等于全部剩余产品都花费在生产工人的工资上。

可以设想这样一种情况:机器制造业者把自己的商品(一部分)卖给比如说布匹生产者;布匹生产者付给他货币;机器制造业者用这些货币购买铁、煤炭等等,而不是购买生活资料。但是把整个过程加以考察就会明白,如果用于补偿不变资本的各要素的生产者不向生活资料生产者购买他生产出来的生活资料,因而,如果这个流通过程实质上不是生活资料和不变资本之间的交换,那末生活资料的生产者就不能购买机器或原料来补偿自己的不变资本。由于买和卖的行为的分离,当然在这些结算过程中可能发生极大的紊乱和麻烦。

[704]如果一个国家自己不能把资本积累所需要的那个数量的机器生产出来,它就要从国外购买。如果它自己不能把所需数量的生活资料(用于工资)和原料生产出来,情况也会如此。在这里,一旦有国际贸易参与,那就可以看得一清二楚,一个国家的剩余产品——如果它用于积累,——有一部分并不转化为工资,而是直接转化为不变资本。但是那时仍然会有一种看法,认为在国外为此而预付的货币会全部花费在工资上。我们已经看到,即使把对外贸易撇开,情况也不是这样,而且不可能是这样。

剩余产品究竟以怎样的比例分为可变资本和不变资本,这取决于资本的平均构成,而且资本主义生产愈发达,直接花费在工资上的那部分相对地也就愈小。有人认为,剩余产品既然只是一年内新加劳动的产品,它[在积累的情况下]也就只转化为可变资本,只花费在工资上,这种看法总的说来同那种认为因为产品只不过是劳动的结果或劳动的化身,所以产品的价值全部都归结为收入,即工资、利润和地租的错误观念是一致的,而斯密和李嘉图就是这样错误地认为的。

不变资本的很大一部分,即固定资本,可能由这样一些东西组成:有的直接加入生活资料、原料等等的生产过程;有的或者用来缩短流通过程,如铁路、公路、通航的运河、电报等等,或者用来保存和储备商品,如货栈、仓库等等;还有的只是在经过较长的再生产时期后才能增加土地的肥力,如土地平整、泄水渠等等。剩余产品究竟是较大一部分还是较小一部分花费在这几种固定资本中的某一种上,这对于生活资料等等的再生产所产生的直接的、最近的后果是极不相同的。

\tsectionnonum{[(6)危机问题(引言)。发生危机时资本的破坏]}

如果有不变资本的追加生产——比补偿旧资本所必需的、因而也就是生产原有数量的生活资料所必需的生产大的生产——作为前提,那末,在使用机器、原料等[来生产个人消费品]的领域进行追加生产,或者说,进行积累,就再也没有任何困难了。如果有必要的追加劳动,上述领域的资本家就会在市场上找到形成新资本,即把他们的代表剩余价值的货币转化为新资本的一切资料。

但是,整个积累过程首先归结为这样的追加生产,它一方面适应人口的自然增长,另一方面形成在危机中显露出来的那些现象的内在基础。这种追加生产的尺度,是资本本身,是生产条件的现有规模和资本家追求发财致富和扩大自己资本的无限欲望,而决不是消费。消费早就被破坏了,因为,一方面,人口的最大部分,即工人人口,只能在非常狭窄的范围内扩大自己的消费,另一方面,随着资本主义的发展,对劳动的需求,虽然绝对地说是在增加,但相对地说却在减少。此外还有一点:一切平衡都是偶然的,各个领域中使用资本的比例固然通过一个经常的过程达到平衡,但是这个过程的经常性本身,正是以它必须经常地、往往是强制地进行平衡的那种经常的比例失调为前提。

我们这里要考察的,只是资本在它向前发展的不同阶段所经历的形式。因此,不准备对实际生产过程在其中进行的现实关系加以分析。这里总是假定商品按其价值出卖。不考察资本的竞争,不考察信用,同样不考察实际的社会结构,——社会决不仅仅是由工人阶级和产业资本家阶级组成的,因此,在社会中消费者和生产者不是等同的:第一个范畴即消费者范畴(消费者的收入有一部分不是第一性的,而是第二性的,是从利润和工资派生的)比第二个范畴[即生产者范畴]广得多,因而,消费者花费自己收入的方式以及收入的多少,会使经济生活过程,特别是资本的流通和再生产过程发生极大的变化。可是,我们在考察货币时\endnote{指《政治经济学批判》第一分册。见《马克思恩格斯全集》中文版第13卷第87—88、131—132和135—137页。——第562页。}已经看到,就货币一般是一种与商品的实物形式不同的形式来说,就它作为支付手段的形式来说,货币本身就包含着危机的可能性,而这一点,在考察资本的一般性质时,用不着对成为实际生产过程的一切前提的进一步的现实关系加以说明,就更加清楚地表现出来了。

[705]大卫·李嘉图接受了庸俗的萨伊的(其实是属于詹姆斯·穆勒的)观点(我们谈这个微不足道的人物时,还要讲到这种观点),认为生产过剩,至少市场商品普遍充斥是不可能的。这种观点是以产品同产品交换\endnote{让·巴·萨伊《论政治经济学》1814年巴黎第二版第二卷第382页:“产品只是用产品购买的”。萨伊的这个公式几乎被李嘉图一字不改地重述了,见下面(第570页)从李嘉图的《原理》(1821年伦敦第3版第341页)中引用的一段话。马克思在《剩余价值理论》的后面正文中对这个公式进行了批判(见本册第570—572页和第3册《李嘉图学派的解体》一章,马克思手稿第811页)。——第563页。}这一论点为基础的,或者,正如穆勒所想象的那样,是以“卖者和买者之间的形而上学的平衡”\endnote{马克思指詹姆斯·穆勒关于生产和消费之间、供给和需求之间、购买量和销售量之间的经常和必要的平衡的论述,见穆勒的《政治经济学原理》1821年伦敦版第3篇第4章第186—195页。马克思在《政治经济学批判》第一分册关于商品的形态变化一节中,对詹姆斯·穆勒的这个观点(他最早是在1808年伦敦出版的《为商业辩护》这本小册子里提出的)作了更详细的分析(见《马克思恩格斯全集》中文版第13卷第87—88页)。——第563、575页。}为基础的,由此还进一步得出结论说,需求仅仅决定于生产本身,或者说,需求和供给完全一致。这种论点也采取李嘉图所特别喜爱的形式,即认为任何数额的资本在任何国家都能够生产地加以使用。

\begin{quote}{李嘉图在第二十一章(《积累对于利润和利息的影响》)中说:“萨伊先生曾经非常令人满意地说明:由于需求只受生产限制,所以任何数额的资本在一个国家都不会得不到使用。任何人从事生产都是为了消费或出卖,任何人出卖都是为了购买对他直接有用或者有助于未来生产的某种别的商品。所以,一个人从事生产时,他不成为自己产品的消费者,就必然成为他人产品的买者和消费者。不能设想,他会长期不了解为了达到自己所追求的目的,即获得别的产品,究竟生产什么商品对他最有利。因此,他不可能继续不断地〈这里根本不是说永远〉生产没有需求的商品。”(第339—340页)}\end{quote}

到处都力求做到前后一贯的李嘉图,发现他的权威萨伊在这里跟他开了个玩笑。他在上述引文的注释中说:

\begin{quote}{“萨伊先生说:‘和资本的使用范围相比,闲置资本越多,资本借贷的利率就越下降。’(萨伊[《论政治经济学》1814年巴黎第2版]第2卷第108页)萨伊先生的这句话同他的上述论点完全一致吗?如果任何数额的资本在一个国家都能够得到使用,怎么能说,和资本的使用范围相比资本会过多呢?”(第340页注)}\end{quote}

因为李嘉图引证萨伊的话,我们在后面考察萨伊这个骗子本人的观点时将批判萨伊的论点。

这里暂时只指出:

在进行资本的再生产时,同在进行资本积累时完全一样,问题不仅在于以原有的规模或者(在积累的时候)以扩大的规模补偿构成资本的同量使用价值,而且在于要补偿预付资本的价值,并且实现普通利润率(普通剩余价值)。因此,如果由于某种情况或某些情况的结合,商品(是全部还是大部都毫无关系)的市场价格大大降到它的费用价格之下,那末,一方面,资本的再生产就会尽量缩小。但是积累将更加停滞。以货币形式(金或银行券)积累起来的剩余价值如果转化为资本就只会带来损失。因此,这些剩余价值就以贮藏货币或者信用货币的形式闲置在银行里,而这丝毫不会改变问题的本质。如果缺少再生产的现实前提(例如在谷物涨价的时候,或者由于实物形式的不变资本还没有积累到足够的数量),相反的原因也可能引起同样的停滞。再生产会发生停滞,因此流通过程也会发生停滞。买和卖互相对立起来,不使用的资本就以闲置货币的形式出现。这种现象(这大都出现在危机之前)也可能发生在这样的时候:追加资本的生产进行得非常快,由于追加资本再转化为生产资本,就大大增加了对生产资本的一切要素的需求,以致实际生产赶不上,因而加入资本形成过程的一切商品涨价。在这种情况下,不管利润怎样增长,利率都要大大降低,而这种利率的降低就引起最冒险的投机活动。由于再生产停滞,可变资本就减少,工资就下降,使用的劳动量就减少。这些又反过来重新影响价格,使价格继续下跌。

任何时候都不应该忘记,在实行资本主义生产的条件下,问题并不直接在于使用价值,而在于交换价值,特别在于增加剩余价值。这是资本主义生产的动机。为了通过论证来否定资本主义生产的矛盾,就撇开资本主义生产的基础,把这种生产说成是以满足生产者的直接消费为目的的生产,这倒是一种绝妙的见解。

其次:因为资本流通过程不是一天就完了,相反,在资本流回[生产领域]以前,要经历一个相当长的时期,因为这个时期同市场价格[706]平均化为费用价格的时期是一致的,因为在这个时期内市场上发生重大的变革和变化,劳动生产率发生重大的变动,因而商品的实际价值也发生重大的变动,所以,很明显,从起点——最初的资本——到它经过一个这样的时期以后回来,必然会发生一些大灾难,危机的要素必然会积累和发展,这些决不是用产品同产品交换这一句毫无价值的空话就可以排除得了的。相反,一批商品在一个时期的价值和同一批商品在较后一个时期的价值的比较(贝利先生认为这是经院式的虚构\endnote{[赛·贝利]《对价值的本质、尺度和原因的批判研究》1825年伦敦版第71—93页。——第565页。})倒是资本流通过程的基本原则。

说到危机引起的资本的破坏,要区别两种情况。

只要再生产过程停滞,劳动过程缩短或者有些地方完全停顿,实际资本就会被消灭。不使用的机器不是资本。不被剥削的劳动等于失去了的生产。闲置不用的原料不是资本。建好不用的建筑物(以及新制造的机器)或半途停建的建筑物,堆在仓库中正在变质的商品,这一切都是资本的破坏。这一切无非是表示再生产过程的停滞,表示现有的生产条件实际上没有起生产条件的作用,没有发挥生产条件的效能。这时,它们的使用价值和它们的交换价值都化为乌有。

第二,危机所引起的资本的破坏意味着价值量的贬低,这种贬低妨碍价值量以后按同一规模作为资本更新自己的再生产过程。这就是商品价格的毁灭性的下降。这时,使用价值没有被破坏。一个人亏损了的东西,被另一个人赚了去。作为资本发挥作用的价值量不能在同一个人手里作为资本更新。原来的资本家遭到破产。如果某个资本家靠出卖自己的商品把他的资本再生产出来,而他的商品的价值本来等于12000镑,其中比如说2000镑是利润,如果这些商品的价格现在降到6000镑,那末,这个资本家就不能支付他的契约债务,即使他没有债务,用这6000镑也不能按以前的规模重新开始他的营业,因为商品价格又会回升到它的费用价格的水平。这样一来,6000镑资本被消灭了,虽然购买这些商品的人因为照商品的费用价格的半价购买,在营业再活跃时可以大有所为甚至因而发财。社会的名义资本,也就是现存资本的交换价值,有很大一部分永远消灭了,虽然由于不殃及使用价值,这种消灭正好可以大大促进新的再生产。这同时也是货币所有者靠牺牲产业家而发财致富的时期。至于纯粹的虚拟资本(公债券、股票等)的跌价,只要它不导致国家和股份公司的破产,不因此而动摇持有这类证券的产业资本家的信用,从而不阻碍再生产,那末这种跌价就只是财富从一些人的手里转到另一些人的手里,总的来说对再生产起着有利的影响,因为那些用廉价把这些股票或证券弄到手的暴发户大多数比原来的所有者更有事业心。

\tsectionnonum{[(7)在承认资本过剩的同时荒谬地否认商品的生产过剩]}

李嘉图在他自己提出的那些论点上总是前后一贯的。因此,在他的著作中,关于(商品的)生产过剩不可能的论点,同关于资本过多或资本过剩不可能的论点,是一回事。\authornote{这里必须区分:当斯密用资本过剩、资本积累来说明利润率下降时,说的是永久的影响问题,而这是错误的;相反,暂时的资本过剩、生产过剩、危机则是另一回事。永久的危机是没有的。}

\begin{quote}{“因此,在一个国家中,除非必需品的涨价使工资大大提高,因而剩下的资本利润极少,以致积累的动机消失,否则积累的资本不论多少,都不可能不生产地加以使用。”(第340页)“从上述各点可以看出,需求是无限的,只要资本还能带来某种利润,资本的使用也是无限的,无论资本怎样多,除了工资提高以外,没有其他充分原因足以使利润降低。此外还可以补充一句:使工资提高的唯一充分而经常的原因,就是[707]为越来越多的工人提供食品和必需品的困难越来越大。”(第347—348页)}\end{quote}

在这种情况下,李嘉图对于他的门徒们的愚蠢该怎么说呢?他们对于一种形式的生产过剩(市场商品普遍充斥)加以否认,同时,对于另一种形式的生产过剩,即资本的生产过剩、资本过多、资本过剩,却不仅加以承认,而且把它作为自己学说的一个基本点。

李嘉图以后时期的理智健全的经济学家中,没有一个否认资本过多。相反,他们都用资本过多来说明危机(只要不是用信用现象来说明)。因此,他们都承认一种形式的生产过剩,但是否认另一种形式的生产过剩。所以,剩下的只是这样一个问题:生产过剩的两种形式彼此之间的关系,即被否认的形式和被确认的形式的关系是怎样的?

李嘉图自己对于危机,对于普遍的、由生产过程本身产生的世界市场危机,确实一无所知。对于1800—1815年的危机,他可以用歉收引起谷物涨价,用纸币贬值、殖民地商品跌价等等来解释,因为,大陆封锁使市场由于政治原因而不是经济原因被迫缩小。对于1815年以后的危机,他也可以解释为部分由于荒年造成谷物缺乏,部分由于谷物价格下降,——因为,根据李嘉图自己的理论,在战争以及英国同大陆切断联系的时候必然引起谷物价格上涨的那些原因不再起作用,——部分由于从战争到和平的转变以及由此产生的“商业途径的突然变化”(见他的《原理》第十九章《论商业途径的突然变化》)。

后来的历史现象,特别是世界市场危机几乎有规律的周期性,不容许李嘉图的门徒们再否认事实或者把事实解释成偶然现象。他们——更不必说那些拿信用来说明一切,以便后来宣称他们自己也将不得不以资本过剩为前提的人了,——不这样做了,却臆造出了一个资本过多和生产过剩之间的美妙的差别。他们搬出他们手中的李嘉图与斯密的词句和论据来反对生产过剩,同时他们企图用资本过多解释他们否则就无法解释的现象。例如,威尔逊用固定资本过多来解释某几次危机,用流动资本过多来解释另外几次危机。资本过多本身,为优秀的经济学家们(例如富拉顿)所承认,而且已经成为大家所接受的偏见,以致这个说法在博学的罗雪尔先生的概论\endnote{威·罗雪尔《国民经济体系》,第一卷《国民经济学原理》1858年斯图加特和奥格斯堡第3版第368—370页。——第568页。}中竟作为一种不言而喻的东西出现了。

因此就要问:资本过多是什么?它同生产过剩有什么区别?

(不过,为了公正起见,需要指出,其他经济学家,如尤尔、柯贝特等,则认为生产过剩——只要考察的是国内市场——是大工业的正常状态。因此可以得出结论说,这种生产过剩只有在某种条件下,当国外市场也缩小时,才会引起危机。)

按照这些经济学家的看法,资本等于货币或者商品。因而,资本的生产过剩就等于货币或商品的生产过剩。可是,据说这两种现象彼此毫无共同之点。这些经济学家甚至不可能谈到货币的生产过剩,因为货币在他们看来就是商品,所以整个现象都归结为他们在一个名称下加以承认而在另一个名称下则加以否认的商品的生产过剩。如果进一步说到存在固定资本或流动资本的生产过剩,那末,这种说法的基础就是:商品已经不是在这个简单的规定上被考察,而是在它作为资本的规定上被考察。但是另一方面,这种说法也承认,在资本主义[708]生产及其种种现象中——例如在生产过剩中,问题不仅在于使产品作为商品出现并具有商品的规定的那种简单关系;而且在于产品的这样一些社会规定,由于这些规定,产品不止是商品,并且不同于简单的商品。

总之,可以认为,用“资本过多”的说法代替“商品生产过剩”的说法不仅仅是一种遁辞,或者说,不仅仅是一种昧着良心的轻率——同一现象,称作a,就认为是存在的和必要的,称作b,就加以否认,因而实际上怀疑和考虑的只是现象的名称,而不是现象本身;或者是想用这种办法来回避说明现象的困难:在现象采取某种形式(名称)而同这些经济学家的偏见发生矛盾时就加以否认,只有在现象采取另一种形式而变得毫无意义时才加以承认。如果撇开这一切不谈,那末,从“商品生产过剩”的说法转到“资本过多”的说法,实际上是个进步。进步表现在哪里?在于承认商品生产者不是作为单纯的商品所有者,而是作为资本家彼此相互对立。

\tsectionnonum{[(8)李嘉图否认普遍的生产过剩。在商品和货币的内在矛盾中包含着危机的可能性]}

再引李嘉图的几个论点:

\begin{quote}{“……会使人认为,亚当·斯密断定:我们似乎在一定程度上不得不〈其实情况就是如此〉生产出过剩的谷物、呢绒和金属制品,似乎用来生产这些商品的资本不能移作别用。但是,一笔资本的使用方式总是可以随便选择的,因此,任何商品都决不可能长期有剩余;因为如果有剩余,商品价格将跌到它的自然价格之下,资本就会转移到某些更有利的行业中去。”(第341—342页注)“产品总是用产品或服务购买的;货币只是进行交换的媒介。”}\end{quote}

(这就是说,货币只是流通手段,而交换价值本身只是产品同产品交换的转瞬即逝的形式,——这是错误的。)

\begin{quote}{“某一种商品可能生产过多,可能在市场上过剩,以致不能补偿它所花费的资本;但是不可能所有的商品都是这种情况。”(第341—342页)“生产的这种增长和由此引起的需求的增加是否会使利润降低,这完全取决于工资是否增加;而工资是否增加,除了短期的增加以外,又取决于为工人生产食品和必需品的容易程度。”(第343页)“商人把他们的资本投入对外贸易或海运业时,他们总是出于自由选择而不是迫不得已;他们这样做是因为在这些部门中他们的利润比在国内贸易中要大一些。”(第344页)}\end{quote}

至于危机,所有描写价格的实际运动的著作家或所有在危机的一定时候进行写作的实践家,都有理由藐视那些貌似理论的空谈,有理由满足于说:认为市场商品充斥等等不可能的学说,在抽象理论上是正确的,但在实践上是错误的。危机有规律的反复出现把萨伊等人的胡说实际上变成了一种只在繁荣时期才使用,一到危机时期就被抛弃的空话。

[709]在世界市场危机中,资产阶级生产的矛盾和对抗暴露得很明显。但是,辩护论者不去研究作为灾难爆发出来的对抗因素何在,却满足于否认灾难本身,他们不顾灾难有规律的周期性,顽固地坚持说,如果生产按照教科书上说的那样发展,事情就决不会达到危机的地步。所以,辩护论就在于伪造最简单的经济关系,特别是在于不顾对立而硬说是统一。

如果比如说买和卖,或者说,商品的形态变化运动,代表着两个过程的统一,或者确切些说,代表着一个经历两个对立阶段的过程,因而,如果这个运动本质上是两个阶段的统一,那末,这个运动同样本质上也是两个阶段的分离和彼此独立。但因为它们毕竟有内在联系,所以,有内在联系的因素的独立只能强制地作为具有破坏性的过程表现出来。正是在危机中,它们的统一、不同因素的统一才显示出来。相互联系和相互补充的因素所具有的彼此的独立性被强制地消灭了。因此,危机表现出各个彼此独立的因素的统一。没有表面上彼此无关的各个因素的这种内在统一,也就没有危机。但是,辩护论经济学家说:不对。因为有统一,所以就不会有危机。而这种说法又无非是说,各个对立因素的统一排除它们的对立。

为了证明资本主义生产不可能导致普遍的危机,就否定资本主义生产的一切条件和它的社会形式的一切规定,否定它的一切原则和特殊差别,总之,否定资本主义生产本身;实际上是证明:如果资本主义生产方式不是社会生产的一个特殊发展的独特形式,而是资本主义最初萌芽产生以前就出现的一种生产方式,那末,资本主义生产方式所固有的对抗、矛盾,因而对抗、矛盾在危机中的爆发,也就不存在了。

\begin{quote}{李嘉图跟着萨伊说:“产品总是用产品或服务购买的;货币只是进行交换的媒介。”}\end{quote}

因此,这里,第一,包含着交换价值和使用价值的对立的商品变成了单纯的产品(使用价值),因而商品交换变成了单纯的产品的物物交换,仅仅是使用价值的物物交换。这就不仅是退回到资本主义生产以前,而且甚至退回到简单商品生产以前去了;并且通过否定资本主义生产的第一个条件,即产品必须是商品,因而必须表现为货币并完成形态变化过程,来否定资本主义生产最复杂的现象——世界市场危机。不说雇佣劳动,却说“服务”,在“服务”这个词里,雇佣劳动及其使用的特殊规定性——就是增大它所交换的商品的价值,创造剩余价值,——又被抛弃了,因而货币和商品借以转化为资本的那种特殊关系也被抛弃了。“服务”是一种仅仅作为使用价值来理解的劳动(这在资本主义生产中是次要的事情),完全象“产品”这个词掩盖了商品的本质和商品中包含的矛盾一样。于是货币也就前后一贯地被看作仅仅是产品交换的媒介,而不是被看作必然表现为交换价值,即表现为一般社会劳动的商品的本质的、必然的存在形式。因为这样把商品变为单纯的使用价值(产品)会抹杀交换价值的本质,[710]所以,货币作为商品的本质的、在形态变化过程中独立于商品最初形式的形态,也就同样轻而易举地可能被否定,确切地说,必然被否定。

因此,这里论证不可能有危机的办法就是,忘记或者否定资本主义生产的最初前提——产品作为商品的存在,商品分为商品和货币这种二重化,由此产生的在商品交换中的分离因素,最后,货币或商品对雇佣劳动的关系。

此外,有些经济学家(例如约·斯·穆勒)想用这种简单的、商品形态变化中所包含的危机可能性——如买和卖的分离——来说明危机,他们的情况并不更妙些。说明危机可能性的这些规定,还远不能说明危机的现实性,还远不能说明为什么[再生产]过程的不同阶段竟会发生这样的冲突,以致只有通过危机、通过强制的过程,它们内在的统一才能发生作用。这种买和卖的分离在危机中也表现出来;这是危机的元素形式。用危机的这个元素形式说明危机,就是通过以危机的最抽象的形式叙述危机存在的办法来说明危机的存在,也就是用危机来说明危机。

\begin{quote}{李嘉图说\authornote{见本册第563面。——编者注}:“任何人从事生产都是为了消费或出卖,任何人出卖都是为了购买对他直接有用或者有助于未来生产的某种别的商品。所以,一个人从事生产时,他不成为自己产品〈goods〉的消费者,就必然成为他人产品的买者和消费者。不能设想,他会长期不了解为了达到自己所追求的目的,即获得别的产品,究竟生产什么商品对他最有利。因此,他不可能继续不断地〈continuYally〉生产没有需求的商品。”}\end{quote}

这种幼稚的胡说,出自萨伊之流之口是相称的,出自李嘉图之口是不相称的。首先,没有一个资本家是为了消费自己的产品而进行生产的。当我们说到资本主义生产时,即使有人把他的产品的某些部分再用于生产消费,我们也有充分理由说:“任何人从事生产都不是为了消费自己的产品。”但是,李嘉图说的是私人消费。以前,李嘉图忘记了产品就是商品。现在,他连社会分工也忘记了。在人们为自己而生产的社会条件下,确实没有危机,但是也没有资本主义生产。我们从来也没有听说过,古代人在他们以奴隶制为基础的生产中见过什么危机,虽然在古代人中也有个别生产者遭到破产。在二者择一的说法[“为了消费或出卖”]中,前一部分是荒谬的。后一部分也是荒谬的。一个人已经进行了生产,是出卖还是不出卖,[在资本主义生产条件下]是没有选择余地的。他是非出卖不可。在危机中出现的正是这样的情况,他卖不出去或者只能低于费用价格出卖,甚至不得不干脆亏本出卖。因此,说他把产品生产出来是为了出卖,这对他、对我们究竟有什么用处呢?问题正是在于:要弄清楚究竟是什么东西阻碍他这个善良愿望的实现。

其次:

\begin{quote}{“任何人出卖都是为了购买对他直接有用或者有助于未来生产的某种别的商品。”}\end{quote}

把资产阶级关系描绘得多么美好呵!李嘉图甚至忘记了,有人可能是为了支付而出卖,忘记了这种被迫的出卖在危机中起着很重要的作用。资本家在出卖时的直接目的是把他的商品,确切些说,是把他的商品资本,再转化成为货币资本,从而实现他的利润。消费——收入——决不是这个过程的主导因素,对于仅仅为了把商品变成生活资料而出卖商品的人来说,消费确实是主导因素。但这不是资本主义生产,在资本主义生产中,收入[消费]是作为结果,而不是作为起决定作用的目的出现的。每一个人出卖,首先是为了出卖,就是说,为了把商品变成货币。

[711]在发生危机的时候,一个人只要把商品卖出去,他就会感到很满意了,至于买进,他暂时不会去考虑。当然,要使实现了的价值能再作为资本发生作用,这个价值就必须通过再生产过程,也就是必须再同劳动和商品进行交换。但是,危机恰恰就是再生产过程破坏和中断的时刻。而这种破坏是不能用在不发生危机的时候它并不存在这个事实来解释的。毫无疑问,谁也不会“继续不断地生产没有需求的商品”(第339—340页),但是谁都没有作过这种荒谬的假设。并且,这样的假设同问题也毫无关系。资本主义生产的目的首先不是“获得别的产品”,而是占有价值、货币、抽象财富。

这里成为李嘉图的论断的基础的,还是我在前面考察过的\endnote{马克思指詹姆斯·穆勒关于生产和消费之间、供给和需求之间、购买量和销售量之间的经常和必要的平衡的论述,见穆勒的《政治经济学原理》1821年伦敦版第3篇第4章第186—195页。马克思在《政治经济学批判》第一分册关于商品的形态变化一节中,对詹姆斯·穆勒的这个观点(他最早是在1808年伦敦出版的《为商业辩护》这本小册子里提出的)作了更详细的分析(见《马克思恩格斯全集》中文版第13卷第87—88页)。——第563、575页。}詹姆斯·穆勒关于“买和卖之间的形而上学的平衡”的论点,这个论点在买和卖的过程中只看见统一而看不见分离。李嘉图下面这个主张(追随詹姆斯·穆勒)也是从这里来的:

\begin{quote}{“某一种商品可能生产过多,可能在市场上过剩,以致不能补偿它所花费的资本;但是不可能所有的商品都是这种情况。”(第341—342页)}\end{quote}

货币不仅是“进行交换的媒介”(第341页),同时也是使产品同产品的交换分解为两个彼此独立的、在时间和空间上彼此分离的行为的媒介。但是,前面所说的李嘉图对货币的错误理解的根本原因在于,李嘉图总是只看到交换价值的量的规定,就是说,交换价值等于一定量的劳动时间,相反,他忘记了交换价值的质的规定,就是说,个人劳动只有通过自身的异化(alienation)才表现为抽象一般的、社会的劳动。\authornote{[718}(李嘉图把货币仅仅看成流通手段,同他把交换价值仅仅看成转瞬即逝的形式,看成对资产阶级生产,或者说,资本主义生产来说仅仅是形式上的东西,是一回事;因此,在李嘉图看来,资本主义生产不是特定的生产方式,而是唯一的生产方式。)[718]]

不是所有种类的商品,而只是个别种类的商品,才能“在市场上过剩”,因此生产过剩始终只能是局部的,这种论点是一种可怜的遁辞。首先,如果谈的只是商品的性质,那末没有什么东西妨碍所有商品在市场上都过剩,因而妨碍它们都降到自己的价格之下\endnote{对于这一点,马克思在《剩余价值理论》第一册中解释如下:“低于它的价格——就是说,低于代表它[商品]的价值的货币额”(见本卷第1册第336页)。——第575页。}。这里说的恰恰只是危机的因素。就是说,除了货币以外的所有商品。说这种商品必然表现为货币,这只是说:所有商品都有这种必然性。完成这个形态变化,个别商品有多少困难,所有商品同样有多少困难。商品形态变化(它既包括买和卖的分离,又包括两者的统一)的一般性质,不仅不排除市场商品普遍充斥的可能性,相反,它本身就是这种普遍充斥的可能性。

其次,李嘉图的和其他类似的论断,当然不仅是从买和卖的关系出发,而且是从需求和供给的关系出发,这等我们考察资本的竞争时再谈。照穆勒的说法,买就是卖,如此等等,那末,这样一来,需求就是供给,供给就是需求。但是,供给与需求同样是彼此分离并且可以彼此独立的。在一定的时刻,由于对一般商品即货币亦即交换价值的需求大于对所有特殊商品的需求,换句话说,由于商品表现为货币、实现商品交换价值的因素居优势,商品再转化为使用价值的因素居劣势,所有商品的供给就可能大于对所有商品的需求。

如果从更广泛和更具体的意义上来理解需求和供给之间的关系,就要把生产和消费的关系包括在内。这里仍然必须看到,这两个因素的潜在的、恰好在危机中强制地显示出来的统一,是与同样存在的、甚至表现为资产阶级生产特征的这两个因素的分离和对立相对的。

至于局部的生产过剩和普遍的生产过剩的对立,既然问题在于承认第一种生产过剩是为了逃避承认第二种生产过剩,那末,对于这个问题必须指出:

第一,在危机之前,所有属于资本主义生产的物品往往普遍涨价。因此,所有这些商品都卷进接着而来的崩溃之中;在按照它们在崩溃之前的价格出卖的情况下,它们就造成市场负担过重。这种按照以前的市场价格市场吸收不了的商品量,按照下降了的、已经降到商品费用价格之下的价格,市场却能够吸收。商品的过剩总是相对的,就是说,都是在一定价格条件下的商品过剩。在这种情况下使商品能被吸收的那种价格,对生产者或商人来说,是引起破产的价格。

[712]第二,危机(因而,生产过剩也是一样)只要包括了主要交易品,就会成为普遍性的。

\tsectionnonum{[(9)李嘉图关于资本主义条件下生产和消费的关系的错误观点]}

我们就更仔细地看一看,李嘉图是怎样试图论证不可能有市场商品普遍充斥的:

\begin{quote}{“某一种商品可能生产过多,可能在市场上过剩,以致不能补偿它所花费的资本;但是不可能所有的商品都是这种情况。对谷物的需求受食用者人数的限制,对鞋子和衣服的需求受穿着者人数的限制。但是即使一个社会或社会的一部分可能有它能够消费或愿意消费的那样多的谷物和鞋帽,但不能说每一种自然或人工生产的商品都是这样。有些人如果可能的话会消费更多的葡萄酒。另一些人有了足够的葡萄酒,又会想添置家具或改进家具的质量。还有一些人可能想装饰自己的庭园或扩建自己的住宅。每一个人的心中都怀有做这一切或做其中一部分的愿望;所需要的只是钱,但是除了增加生产以外再没有别的方法可以提供钱。”(第341—342页)}\end{quote}

还能有比这个更幼稚的论证吗?它的意思就是:个别商品已生产出来的数量可能比能够消费的要多。但不可能所有的商品同时都这样。因为用商品来满足的需要是无限的,而所有这些需要不能同时得到满足。相反,满足一种需要的过程会使另一种需要转入可以说是潜在状态。因此,除了满足这些需要的钱以外什么都不需要,而这种钱又只有用增加生产的办法才能获得。这就是说,普遍生产过剩是不可能的。

这一切能说明什么呢?在生产过剩的时候,很大一部分国民(特别是工人阶级)得到的谷物、鞋子等比任何时候都少,更不用说葡萄酒和家具了。如果仅仅在一个国家的全体成员的即使最迫切的需要得到满足之后才会发生生产过剩,那末,在迄今资产阶级社会的历史上,不仅一次也不会出现普遍的生产过剩,甚至也不会出现局部的生产过剩。如果,比如说鞋子、棉布、葡萄酒或者殖民地产品充斥市场,难道这就是说,国民,哪怕只是三分之二的国民,对于鞋子、棉布等的需要已经得到满足而有余了吗?生产过剩同绝对需要究竟有什么关系呢?生产过剩只同有支付能力的需要有关。这里涉及的不是绝对的生产过剩,不是同绝对需要或者占有商品的愿望有关系的生产过剩本身。在这种意义上,既不存在局部的也不存在普遍的生产过剩,它们彼此根本不对立。

但是,李嘉图会说,如果有一批人需要鞋子和棉布,他们为什么不去设法弄到购买这些东西的钱呢?他们为什么不生产一些可以用来购买鞋子和棉布的东西呢?干脆说为什么他们不自己生产鞋子和棉布,不是更简单吗?而在发生生产过剩的时候尤其令人奇怪的是,正是充斥市场的那些商品的真正生产者——工人——缺乏这些商品。这里不能说,他们要得到这些东西,就得去生产这些东西,因为这些东西他们已经生产出来了,但他们还是没有。也不能说,某一种商品之所以充斥市场,是因为对这种商品没有需要。因此,既然甚至不能用充斥市场的商品的数量超过了对这些商品的需要这一点来说明局部的生产过剩,那末,无论如何也不能用市场上的许多商品还有需要,还有未能满足的需要,就否定普遍的生产过剩。

我们仍然以棉织厂主为例\authornote{见本册第545页及以下各页(那里不用“棉织厂主”,而用“麻织厂主”,这丝毫不改变问题的实质)。——编者注}。只要再生产不断进行,——因而,这一再生产中作为商品,作为待出卖的商品而存在的产品即棉布按其价值再转化为货币的阶段也不断进行,——可以说,生产棉布的工人也就消费掉棉布的一部分,并且随着再生产的扩大,也就是随着积累,他们消费的棉布也就相应地增多,或者说,也就有更多的工人来从事棉布生产,而他们同时也就是一部分棉布的消费者。

\tsectionnonum{[(10)危机的可能性转化为现实性。危机是资产阶级经济的一切矛盾的表现]}

在进一步考察之前,我们要指出:

在考察商品的简单形态变化时\endnote{马克思指《政治经济学批判》第一分册《商品的形态变化》一节。(见《马克思恩格斯全集》中文版第13卷第77—88页,特别是第86—88页。)——第579页。}已经显露出来的危机可能性,通过(直接的)生产过程和流通过程的彼此分离再次并且以更发展了的形式表现出来。一旦两个过程不能顺利地互相转化[713]而彼此独立,就发生危机。

在商品的形态变化中,危机的可能性表现为:

首先,实际上作为使用价值存在而在观念上以价格形式作为交换价值存在的商品,必须转化为货币:W—G。如果这个困难——出卖——已经解决,那末,购买,G—W,就再没有什么困难了,因为货币可以同一切东西直接交换。必要的前提就是,商品具有使用价值,商品所包含的劳动是有用的,否则它就根本不是商品。其次,假定商品的个别价值等于它的社会价值,就是说,物化在商品中的劳动时间等于生产该商品的社会必要劳动时间。因此,危机的可能性,就其在形态变化的简单形式中的表现来说,仅仅来自以下情况,即商品形态变化在其运动中经历的形式差别——阶段——第一,必须是相互补充的形式和阶段,第二,尽管有这种内在的必然的相互联系,却是过程的互不相干地存在着、在时间和空间上彼此分开、彼此可以分离并且已经分离、互相独立的部分和形式。因此,危机的可能性只在于卖和买的分离。只是在商品的形式上商品必须克服这里所遇到的困难。一旦它具有货币形式,这种困难就算度过了。但是,往前走,这又是卖和买的分离。如果商品不能以货币形式退出流通领域,或者换句话说,不能推迟自己再转化为商品的时间,如果——就象直接的物物交换中一样——买和卖彼此一致,那末,在上述假定下的危机的可能性就会消失。因为已经假定商品对别的商品所有者来说是使用价值了。在直接的物物交换的形式中,商品只有当它不是使用价值,或者在对方没有别的使用价值可以同它交换的时候,才不能进行交换。因此,只有在两个条件下才不可能进行交换:或者是一方生产了无用之物,或者是对方没有有用之物可以作为等价物同前者的使用价值交换。不过,在这两种情况下根本不会发生交换。然而只要发生交换,它的因素就不是彼此分离的。买者就是卖者,卖者就是买者。所以,既然交换就是流通,从交换形式产生的危机因素就消失了,如果我们说形态变化的简单形式包含着危机的可能性,那只不过是说,在这种形式本身包含着本质上相互补充的因素彼此割裂和分离的可能性。

但是,这也涉及到内容。在进行直接的物物交换的时候,从生产者方面来说,产品的主要部分,是为了满足他自己的需要,或者,到分工有了一些发展以后,是为了满足他所知道的他的协作生产者的需要。作为商品拿来交换的是剩余品,而这个剩余品是否进行交换,却是不重要的。在商品生产的情况下,产品转化为货币,出卖,就成了必不可少的条件。为满足自己需要而进行的直接生产已成为过去。如果商品卖不出去,就会发生危机。商品(个人劳动的特殊产品)转化为它的对立物货币,即转化为抽象一般的社会劳动的困难,在于货币不是作为个人劳动的特殊产品出现,在于已经卖掉了商品而现在持有货币形式的商品的人并不是非要立刻重新买进、重新把货币转化为个人劳动的特殊产品不可。在物物交换中不存在这种对立。在那里不是买者就不能是卖者,不是卖者就不能是买者。卖者——假定他的商品具有使用价值,——的困难仅仅是由于买者可以轻易地推迟货币再转化为商品的时间而产生的。商品转化为货币即出卖商品的这种困难,仅仅是由于商品必须转化为货币,货币却不立即必须转化为商品,因此卖和买可能彼此脱离而产生的。我们说过,这个形式包含着危机的可能性,也就是包含着这样的可能性:相互联系和不可分离的因素彼此脱离,因此它们的统一要通过强制的方法实现,它们的相互联系要通过强加在它们的彼此独立性上的暴力来完成。[714]危机无非是生产过程中已经彼此独立的阶段强制地实现统一。

危机的一般的、抽象的可能性,无非就是危机的最抽象的形式,没有内容,没有危机的内容丰富的起因。卖和买可能彼此脱离。因此它们是潜在的危机。它们的一致对商品来说总是危机的因素。但是它们也可能顺利地相互转化。所以,危机的最抽象的形式(因而危机的形式上的可能性)就是商品的形态变化本身,在商品形态变化中,包含在商品的统一中的交换价值和使用价值的矛盾以至货币和商品的矛盾,仅仅作为展开的运动存在。但是,使危机的这种可能性变成危机,其原因并不包含在这个形式本身之中;这个形式本身所包含的只是:危机的形式已经存在。

而这对于考察资产阶级经济是重要的。世界市场危机必须看作资产阶级经济一切矛盾的现实综合和强制平衡。因此,在这些危机中综合起来的各个因素,必然在资产阶级经济的每一个领域中出现并得到阐明。我们越是深入地研究这种经济,一方面,这个矛盾的越来越新的规定就必然被阐明,另一方面,这个矛盾的比较抽象的形式会再现并包含在它的比较具体的形式中这一点,也必然被说明。

总之,可以说:危机的第一种形式是商品形态变化本身,即买和卖的分离。

危机的第二种形式是货币作为支付手段的职能,这里货币在两个不同的、彼此分开的时刻执行两种不同的职能。

这两种形式都还是十分抽象的,虽然第二种形式比第一种形式具体些。

因此,在考察资本的再生产过程(它同资本的流通是一致的)时,首先要指出,上述两种形式在这里是简单地再现,或者更确切地说,在这里第一次获得了内容,获得了它们可以表现出来的基础。

现在我们就来考察资本从它作为商品离开生产过程然后重新以商品形式从生产过程中产生的时候起所经历的运动。如果我们这里把所有对内容的进一步的规定撇开不谈,那末,总商品资本和它包含的每一单个商品都要经历W—G—W过程,都要完成商品的形态变化。因此,只要资本也是商品并且只是商品,那末包含在这个形式中的危机的一般可能性,即买和卖的分离,也就包含在资本的运动中。此外,鉴于不同商品的形态变化是相互联系的,所以,一种商品转化为货币是因为另一种商品从货币形式再转化为商品。因此,买和卖的分离在这里进一步表现为:一笔资本从商品形式转化为货币形式,相应地另一笔资本就必须从货币形式再转化为商品形式,一笔资本发生第一形态变化,相应地另一笔资本就必须发生第二形态变化,一笔资本离开生产过程,相应地另一笔资本就必须回到生产过程。不同资本的再生产过程或流通过程的这种相互连结和彼此交叉,一方面,由于分工而成为必然的,另一方面,又是偶然的,因此,对危机的内容的规定已经扩大了。

但是,第二,至于由作为支付手段的货币形式产生的危机的可能性,那末,在考察资本时,这种可能性转化为现实性的更现实得多的基础已经显露出来了。例如,织布厂主必须支付全部不变资本,这种不变资本的要素是纺纱厂主、亚麻种植业者、机器制造厂主、制铁厂主、木材业者和煤炭业者等提供的。只要后面这些人生产的不变资本,只加入不变资本的生产而不加入最后商品——布,他们就是通过资本同资本的交换互相补偿各自的生产条件。现在假定[715]织布厂主把布卖给商人,作价1000镑,但用的是一张汇票,所以货币是作为支付手段出现。这个织布厂主又把这张汇票卖给银行家,他在银行家那里用它偿付了一笔什么债务,或者银行家给他办理了汇票贴现。同样,亚麻种植业者凭汇票卖给纺纱厂主,而纺纱厂主又凭汇票卖给织布厂主,同样,机器制造厂主凭汇票卖给织布厂主,制铁厂主和木材业者凭汇票卖给机器制造厂主,煤炭业者凭汇票卖给纺纱厂主、织布厂主、机器制造厂主、制铁厂主和木材业者。此外,制铁厂主、煤炭业者、木材业者和亚麻种植业者之间也用汇票互相支付。现在,如果商人支付不出,织布厂主就不能向银行家支付自己的汇票。

亚麻种植业者开出了由纺纱厂主支付的汇票,机器制造厂主开出了由织布厂主和纺纱厂主支付的汇票。由于织布厂主不能支付,纺纱厂主也就不能支付,他们两人都不能向机器制造厂主支付,而机器制造厂主则不能向制铁厂主、木材业者和煤炭业者支付。他们由于都没有实现自己商品的价值,就全都不能使补偿不变资本的那部分价值得到补偿。这样就要发生普遍的危机。这不过是在考察货币作为支付手段时展现的危机的可能性,但是,在这里,在资本主义生产中,我们已经看到了使危机可能性可能发展成为现实性的相互债权和债务之间、买和卖之间的联系。

在所有情况下:

如果买和卖彼此不发生梗阻,因而没有必要强制地加以平衡,另一方面,如果货币作为支付手段发挥作用的结果是彼此的债权互相抵销,也就是说作为支付手段的货币中潜在地包含着的矛盾没有成为现实;因此,如果危机的这两种抽象形式本身并没有实际地表现出来,那就不会有危机。只要买和卖不彼此脱离,不发生矛盾,或者只要货币作为支付手段所包含的矛盾不出现,因而,只要危机不是同时以其简单的形式——买和卖矛盾的形式和货币作为支付手段的矛盾的形式——出现,那就不可能发生危机。但是,这终究只不过是危机的形式,危机的一般可能性,因而也只不过是现实危机的形式,现实危机的抽象形式。危机的存在以这些形式出现就是以危机的最简单的形式出现,也是以危机的最简单的内容出现,因为这种形式本身就是危机的最简单的内容。但是,这还不是有了根据的内容。有简单的货币流通,甚至有作为支付手段的货币流通——这两者早在资本主义生产以前很久就出现了,却没有引起危机——而没有危机是可能的,也是现实的。因此,单单用这些形式不能说明,为什么这些形式会转向其危机的方面,为什么这些形式潜在地包含着的矛盾会实际地作为矛盾表现出来。

从这里可以看出有些经济学家的极端的庸俗,他们在再也不能用推理来否定生产过剩和危机的现象时,就安慰自己说,在上述形式中既定的[只]是发生危机的可能性,所以,不发生危机是偶然的,发生危机本身也不过是偶然的事。

在商品流通中,接着又在货币流通中发展起来的矛盾,——因而还有危机的可能性,——自然会在资本中再现出来,因为实际上只是在资本的基础上才有发达的商品流通和货币流通。

但是,现在的问题是要彻底考察潜在的危机的进一步发展(现实危机只能从资本主义生产的现实运动、竞争和信用中引出),要就危机来自作为资本的资本所特有的,而不是仅仅在资本作为商品和货币的存在中包含的资本的各种形式规定,来彻底考察潜在的危机的进一步发展。

[716]单单资本的(直接)生产过程本身在这里不能添加什么新的东西。为了使资本的生产过程存在,就得假定这一过程的条件是既定的。因此,在论资本的第一篇——在论直接生产过程的那一篇,并未增加危机的任何新的要素。这里潜在地包含着危机的要素,因为生产过程就是剩余价值的占有,因而也是剩余价值的生产。但是在生产过程本身,这一点是表现不出来的,因为这里不仅谈不到再生产出来的价值的实现,也谈不到剩余价值的实现。

只有在本身同时就是再生产过程的流通过程中,这一点才能初次显露出来。

这里还要指出,我们必须在叙述完成了的资本——资本和利润——之前叙述流通过程或再生产过程,因为我们不仅要叙述资本如何进行生产,而且要叙述资本如何被生产出来。但是,实际运动——这里说的是以发达的、从自己开始并以自己为前提的资本主义生产为基础的实际运动——是从现有资本出发的。因此,对于再生产过程以及在这个过程中得到进一步发展的危机的萌芽,在论述再生产的这一部分只能作不充分的叙述,需要在《资本和利润》一章\endnote{马克思指他的研究中后来发展成为《资本论》第三卷的那一部分。——第586页。}中加以补充。

资本的总流通过程或总再生产过程是资本的生产阶段和资本的流通阶段的统一,也就是把上述两个过程作为自己的不同阶段来通过的过程。这里包含着得到进一步发展的危机的可能性,或者说,包含着得到进一步发展的危机的抽象形式。因此,否认危机的经济学家们只坚持这两个阶段的统一。如果这两个阶段只是彼此分离而不成为某种统一的东西,那就不可能强制地恢复它们的统一,就不可能有危机。如果它们只是统一的而彼此不会分离,那就不可能强制地把它们分离,而这种分离还是危机。危机就是强制地使已经独立的因素恢复统一,并且强制地使实质上统一的因素变为独立的东西。[716]

\tsectionnonum{[(11)危机的形式问题]}

[770a]对第716页的补充。

因此:

(1)危机的一般可能性在资本的形态变化过程本身就存在,并且是双重的。如果货币执行流通手段的职能,危机的可能性就包含在买和卖的分离中。如果货币执行支付手段的职能,货币在两个不同的时刻分别起价值尺度和价值实现的作用,——危机的可能性就包含在这两个时刻的分离中。如果价值在这两个时刻之间有了变动,如果商品在它卖出的时刻的价值低于它以前在货币执行价值尺度的职能,因而也执行相互债务尺度的职能的时刻的价值,那末,用出卖商品的进款就不能清偿债务,因而,再往上推,以这笔债务为转移的一系列交易,都不能结算。即使商品的价值没有变动,只要商品在一定时期内不能卖出,单单由于这一笔债务,货币就不能执行支付手段的职能,因为货币必须在一定的、事先规定的期限内执行支付手段的职能。但是,因为同一笔货币是对一系列的相互交易和债务执行这种职能,所以无力支付的情况就不止在一点上而是在许多点上出现,由此就发生危机。

这就是危机的两种形式上的可能性。在没有第二种可能性的情况下,第一种可能性也可能出现,就是说,在没有信用的情况下,在没有货币执行支付手段的职能的情况下,也可能发生危机。但是,在没有第一种可能性的情况下,即在没有买和卖彼此分离的情况下,却不可能出现第二种可能性。但是,在第二种场合所以发生危机,不仅是因为商品一般地卖不出去,而且是因为商品不能在一定期限内卖出去,在这里危机所以发生,危机所以具有这样的性质,不仅由于商品卖不出去,而且由于以这一定商品在这一定期限内卖出为基础的一系列支付都不能实现。这就是本来意义上的货币危机形式。

因此,如果说危机的发生是由于买和卖的彼此分离,那末,一旦货币执行支付手段的职能,危机就会发展为货币危机,在这种情况下,只要出现了危机的第一种形式,危机的这第二种形式就自然而然地要出现。因此,在研究为什么危机的一般可能性会变为现实性时,在研究危机的条件时,过分注意从货币作为支付手段的发展中产生的危机的形式,是完全多余的。正因为这个缘故,经济学家们乐于举出这个显而易见的形式作为危机的原因。(既然货币作为支付手段的发展是同信用和信用过剩的发展联系在一起,那末当然应该说明这些现象的原因,但是这里还不是这样说明的地方。)

(2)只要危机是由同商品的价值变动不一致的价格变动和价格革命引起的,它当然就不能在考察一般资本的时候得到说明,因为在考察一般资本时假定价格是同商品的价值一致的。

(3)危机的一般可能性就是资本的形式上的形态变化本身,就是买和卖在时间上和空间上的彼此分离。但是这决不是危机的原因。因为这无非是危机的最一般的形式,即危机本身的最一般的表现。但是,不能说危机的抽象形式就是危机的原因。如果有人要问危机的原因,那末他想知道的就是,为什么危机的抽象形式,危机的可能性的形式会从可能性变为现实性。

(4)危机的一般条件,只要不取决于和价值波动不同的价格波动(不论这种波动同信用有无关系),就必须用资本主义生产的一般条件来说明。[770a]

[716](危机可能发生在:第一,[货币]再转化为生产资本的时候;第二,由于生产资本的要素特别是原料的价值变动,如棉花收成减少,因而它的价值增加。我们这里所涉及的还不是价格,而是价值。)[716]

[770a]第一个因素。货币再转化为资本。假定这是生产或再生产的一定阶段。这里,可以把固定资本看成既定的、不变的,它不加入价值形成过程。既然原料的再生产不仅取决于花费在其中的劳动,而且取决于这一劳动的同自然条件有关的生产率,那末,产品量——甚至[XIV—771a]同一劳动量生产的产品量——就可能减少(在歉收时)。于是原料的价值增加,原料的量则减少。为了以原有规模继续生产,货币必须按一定比例再转化为资本的不同组成部分,而现在这个比例被破坏了。用于原料的部分必须增加,剩下用于劳动的部分就减少,因此就不能吸收和以前相同的劳动量。第一是物质上不可能,因为原料的量减少了;第二是因为产品价值中必须有比原来更大的一部分用于原料,因而只能有较小一部分转化为可变资本。再生产不能按原有规模重新进行。一部分固定资本要闲置起来,一部分工人会被抛到街头。利润率会下降,因为不变资本的价值同可变资本相比增加了,使用的可变资本减少了。以利润率和劳动剥削率不变为根据事先规定的固定提成——利息、地租——仍旧不变,有一部分不能支付。于是发生危机。劳动危机和资本危机。因此,这就是由于靠产品价值补偿的一部分不变资本的价值提高而引起的再生产过程的破坏。其次,虽然利润率下降,产品却会涨价。如果这种产品作为生产资料加入其他生产领域,那末这种产品的涨价会使其他领域的再生产遭到同样的破坏。如果这种产品作为生活资料加入一般消费,那末,它或者也加入工人的消费,或者不加入工人的消费。如果是前者,它的后果同后面要讲的可变资本遭到破坏时产生的后果一样。但是,在这种产品加入一般消费的情况下,由于这种产品涨价(如果这种产品的消费不减少),对其他产品的需求就会减少,因而其他产品就不能再转化为相当于其价值的货币额,这样,其他产品的再生产的另一方面——不是货币再转化为生产资本,而是商品再转化为货币——就会遭到破坏。无论如何,我们所考察的这个生产部门的利润量和工资量会减少,从而其他生产部门出卖商品所得的一部分必要收入也会减少。

但是,即使不受收成的影响,或者说,即使不受提供这种原料的劳动的受自然因素制约的生产率的影响,这种原料不足的情况也可能发生。就是说,如果某个生产部门花费在机器等等上的那部分剩余价值,那部分追加资本过多,那末,虽然按原来的生产规模原料是够的,但按新的生产规模就不够了。因此,这种情况是由于追加资本不按比例地转化为资本的不同要素而产生的。这是固定资本生产过剩的情况,它所产生的现象正好同第一种情况[即原料歉收时]所产生的现象完全一样。

[861a][…………………………………………………………………][注:在手稿中,这一页的左上角撕掉了。结果原文前九行只留下六行的右半截,全文不可能完全恢复,但是可以猜测,马克思在这里谈的是“由可变资本的价值革命”而产生的危机。由比如说歉收引起的“必要生活资料的涨价”,会导致用于“可变资本所推动的”工人的支出增加。“同时,也会导致对其他一切商品、即一切不加入”工人的“消费的商品”的需求“减少”。因此“这些商品按其价值出卖”就成为不可能,“它们的再生产的第一阶段”,即商品向货币的“转化”就遭到破坏。于是,生活资料的涨价“导致其他生产部门发生危机”。

这一页撕坏的部分的最后两行中包含着总结这全段议论的思想:“不论这种原料是作为材料加入不变资本还是作为生活资料”加入工人的消费,由于原料涨价都可能发生危机。——编者注]

或者,它们[危机]是以固定资本的生产过剩,因而,是以流动资本的相对的生产不足为基础的。

因为固定资本同流动资本一样,都是由商品组成的,所以,那些否认商品生产过剩而同时又承认固定资本生产过剩的经济学家们的立场是最可笑不过的了。

(5)由于再生产的第一阶段遭到破坏,也就是由于商品向货币的转化发生障碍,或者说出卖发生障碍而产生的危机。在发生第一种[由于原料涨价而引起的]危机的情况下,危机是由于生产资本的要素的回流发生障碍而产生的。[XIV-861a]

\tsectionnonum{[(12)资本主义条件下生产和消费的矛盾。主要消费品生产过剩转化为普遍生产过剩]}

[XIII—716]在开始考察危机的新形式\endnote{在这之后不久,马克思在手稿第XIII本封面(手稿第770a页)和手稿第XIV本封面(手稿第771a和861a页)上写了关于危机形式的短评。根据马克思的注解:“对第716页的补充”,把这几页的正文放在前面(第11节:《危机的形式问题》)。——第591页。}之前,我们再回过头来看看李嘉图的著作和前面举过的例子\authornote{见本册第545页及以下各页(那里不用“棉织厂主”,而用“麻织厂主”,这丝毫不改变问题的实质),并见第579页。——编者注}。

[716]只要棉织厂主进行再生产和积累,他的工人也就是他的一部分产品的买者,他们把自己的一部分工资花费在棉布上。正因为工厂主进行生产,所以,工人们就有购买他的一部分产品的钱,就是说,工人们部分地给他提供了出卖产品的可能性。工人作为需求的代表所能购买的,只是加入个人消费的商品,因为他不是自己使用自己的劳动,因而也不是自己占有实现自己劳动的条件——劳动资料和劳动材料。所以,这一点已经把生产者的最大部分(在资本主义生产发达的地方就是工人本身)排除在消费者、买者之外了。他们不购买原料和劳动资料,他们只购买生活资料(直接加入个人消费的商品)。因此,说生产者和消费者是一回事,那是最可笑不过的了,因为对于很大数量的生产部门——所有不生产直接消费品的部门——来说,大多数参加生产的人是绝对被排斥于购买他们自己的产品之外的。他们决不是自己的这很大一部分产品的直接消费者或买者,虽然他们支付包含在他们购买的消费品中的自己产品的一部分价值。这里也可以看出,“消费者”这个词是模糊不清的,把“消费者”这个词同“买者”这个词等同起来是错误的。从生产消费的意义来说,恰恰是工人消费机器和原料,在劳动过程中使用它们。但是工人并不是为了自己而使用机器和原料,因此,也就不是机器和原料的买者。对于工人来说,机器和原料不是使用价值,不是商品,而是一个过程的客观条件,而工人本身则是这个过程的主观条件。

[717]可是有人会说,雇用工人的企业主在购买劳动资料和劳动材料时是工人的代表。但是,企业主代表工人——指的是在市场上代表——与假定说工人自己代表自己,条件是不一样的。企业主必须出卖包含着剩余价值,即无酬劳动的商品量,要是工人的话,就只须出卖把生产中预付的价值——以劳动资料、劳动材料和工资的价值形式出现——再生产出来的商品量。因此,资本家需要的市场比工人需要的市场大。而且,企业主是否认为市场条件对于开始再生产已充分有利,这取决于企业主而不是工人。

因此,对于一切不是用于个人消费而必须用于生产消费的物品来说,即使再生产过程不遭到破坏,工人也是生产者而不是消费者。

因此,主张把资本主义生产中的消费者(买者)和生产者(卖者)等同起来,从而否定危机,是再荒谬不过的了。这两者是完全不一样的。在再生产过程继续进行的情况下,只是对3000个生产者之中的一个,即对资本家,才可以说是等同的。反过来,说消费者就是生产者,也同样是错误的。土地所有者(收取地租的人)不生产,可是他消费。所有货币资本的代表也是这种情况。

否认危机的各种辩护论言论所证明的东西,总是和它们想要证明的相反,就这一点说,它们是重要的。它们为了否认危机,在有对立和矛盾的地方大谈统一。因此,说它们是重要的,只是因为可以说:它们证明,如果被它们用想象排除了的矛盾实际上不存在,那就不会有任何危机。但是,因为这些矛盾存在着,所以实际上有危机。辩护论者为否定危机存在而提出来的每个根据,都是仅仅在他们想象中被排除了的矛盾,所以是现实的矛盾,所以是危机的根据。用想象排除矛盾的愿望同时就是实际上存在着矛盾的一个证明,这些矛盾按照善良的愿望是不应该存在的。

工人实际上生产的是剩余价值。只要他们生产剩余价值,他们就有东西消费。一旦剩余价值的生产停止了,他们的消费也就因他们的生产停止而停止。但是,他们能够消费,决不是因为他们为自己的消费生产了等价物。相反,当他们仅仅生产这样的等价物时,他们的消费就会停止,他们就没有等价物消费了。或者他们的劳动会停止,或者他们的劳动会缩减,或者,无论如何,他们的工资会降低。在后一种情况下——如果生产水平不变——他们就不是消费他们生产的等价物。但是,这时他们之所以缺少钱,不是因为他们生产的东西不够,而是因为他们从他们所生产的产品中得到的太少。

因此,如果把这里所考察的关系简单地归结为消费者和生产者的关系,那就忘记了从事生产的雇佣工人和从事生产的资本家是两类完全不同的生产者,更不用说那些根本不从事生产活动的消费者了。这里又是用把生产中实际存在的对立撇开的办法来否定对立。仅仅雇佣工人和资本家的关系本身就包含着:

(1)生产者的最大部分(工人),并不是他们所生产的产品的很大一部分,即劳动资料和劳动材料的消费者(买者);

(2)生产者的最大部分,即工人,只有在他们生产的产品大于其等价物时,即在他们生产剩余价值,或者说,剩余产品时,才可能消费这个等价物。他们始终必须是剩余生产者,他们生产的东西必须超过自己的[有支付能力的]需要,才能在[718]自己的这些需要的范围内成为消费者或买者。\endnote{手稿中接着有一小段关于李嘉图对货币和交换价值的观点的插话。马克思把这段插话放在括号中,并注明:这段话对叙述的直接联系有妨碍,应该移到别处去。本版以脚注形式把它移到前面第575页。——第594页。}

因此,就这个生产者阶级来说,说生产和消费是统一的这种论调,无论如何一看就知道是错误的。

如果李嘉图说,需求的唯一界限是生产本身,而生产只受资本的限制,\endnote{大·李嘉图《政治经济学和赋税原理》1821年伦敦第三版第339页:“……需求只受生产限制”(马克思在前面第563页连同较长的上下文一起引证过这句话)以及第347页:“需求是无限的……资本的使用也是无限的”(马克思在前面第567页连同较长的上下文一起引证过这段话)。——第594页。}那末,如果剥去错误假定的外衣,实际上这只不过是说,资本主义生产只以资本作为自己的尺度,同时这里所说的资本也包括作为资本的生产条件之一并入资本(为资本所购买)的劳动能力。可是,问题恰恰在于资本本身是否也是消费的界限。无论如何从消极意义上说它是消费的界限,就是说,消费的东西不可能多于生产的东西。但问题是,从积极意义上说它是不是消费的界限,是不是在资本主义生产的基础上生产多少,就能够或者必须消费多少。如果对李嘉图的论点作正确的分析,那末,这个论点所说的恰恰同李嘉图想说的相反,——就是说,进行生产是不考虑消费的现有界限的,生产只受资本本身的限制。而这一点确实是这种生产方式的特点。

因此,根据假定,市场上比如说棉织品充斥,以致一部分或者全部都卖不出去,或者要大大低于它的价格才卖得出去。(我们暂且说价值,因为在考察流通或再生产过程时,涉及的还是价值,而不是费用价格,更不是市场价格。)

此外,在整个这一分析中,不言而喻的是:不可否认,有些部门可能生产过多,因此另一些部门则可能生产过少;所以,局部危机可能由于生产比例失调而发生(但是,生产的合乎比例始终只是在竞争基础上生产比例失调的结果),这种生产比例失调的一般形式之一可能是固定资本的生产过剩,或者另一方面,也可能是流动资本的生产过剩。\authornote{[720}(当发明了纺机的时候,同织布业比较,曾经出现纱的生产过剩。一旦织布业采用机器织机,这种比例失调就消除了。)[720]]正如商品按其价值出卖的条件是商品只包含社会必要劳动时间一样,对于资本的某一整个生产领域来说,这种条件就是,这个特殊领域所花费的只是社会总劳动时间中的必要部分,只是为满足社会需要(需求)所必要的劳动时间。如果这个领域花费多了,即使每一单位商品所包含的只是必要劳动时间,这些单位商品的总量所包含的却会多于社会必要劳动时间,正如单位商品虽然具有使用价值,这些单位商品的总量在既定的前提下却会丧失它的一部分使用价值。

可是我们这里谈的,不是以生产的比例失调为基础的危机,就是说,不是以社会劳动在各生产领域之间的分配比例失调为基础的危机。这一点只有在谈到资本竞争的时候才能谈到。前面已经说过\authornote{见本册第229—234页。——编者注},由于这种比例失调而引起的市场价值的提高或降低,造成资本从一个生产领域抽出并转入另一个生产领域,造成资本从一个领域向另一个领域的转移。可是,这种平衡本身已经包含着:它是以平衡的对立面为前提的,因此它本身可能包含危机,危机本身可能成为平衡的一种形式。但是,这种危机是李嘉图等人所承认的。

我们在考察生产过程时\endnote{马克思指他的1861—1863年手稿中直接在《剩余价值理论》前面的第I—V本手稿,特别是指关于绝对剩余价值的生产和相对剩余价值的生产那两节。——第596页。}已经看到,资本主义生产竭力追求的只是攫取尽可能多的剩余劳动,就是靠一定的资本物化尽可能多的直接劳动时间,其方法或是延长劳动时间,或是缩短必要劳动时间,发展劳动生产力,采用协作、分工、机器等,总之,进行大规模生产即大量生产。因此,在资本主义生产的本质中就包含着不顾市场的限制而生产。

在考察再生产时,首先假定生产方式不变,而在生产扩大的时候,生产方式实际上在一段时间内也是保持不变的。这里生产出来的商品量所以增加,是由于使用了更多的资本,而不是由于更有效地使用了资本。但是单单资本的量的增加[719]同时也就包含资本的生产力的增加。如果说资本的量的增加是生产力发展的结果,那末反过来说,一个更广阔的、扩大了的资本主义基础又是生产力发展的前提。这里存在着相互作用。因此,在更加广阔的基础上进行的再生产即积累,即使它最初只表现为生产在量上的扩大(在同样的生产条件下投入更多的资本),但在某一点上也总会在质上表现为进行再生产的条件具有较大的效率。因此,产品量增加的比例要大于在扩大再生产即积累中资本增长的比例。

现在再回过头来看看我们那个棉布的例子。

由于棉布充斥而造成的市场停滞,会使织布厂主的再生产遭到破坏。这种破坏首先会影响到他的工人。于是,工人对于他的商品棉布和原来加入他们消费的其他商品来说,现在只在较小的程度上是消费者,或者根本不再是消费者了。他们当然需要棉布,但是他们买不起,因为他们没有钱,而他们之所以没有钱,是因为他们不能继续生产,而他们之所以不能继续生产,是因为已经生产的太多了,棉布充斥市场。李嘉图的劝告,不论是“增加他们的生产”也好,“生产别的东西”\endnote{大·李嘉图《政治经济学和赋税原理》1821年伦敦第三版第342页:“……除了增加生产以外再没有别的方法可以提供钱”(马克思在前面第577页连同较长的上文一起引证过这句话),以及第339—340页:“他不可能继续不断地生产没有需求的商品”(马克思在前面第563页和573页连同较长的上文一起引证过这句话)。——第597页。}也好,都不能帮他们的忙。他们现在代表着暂时的人口过剩的一部分,代表着工人的生产过剩的一部分,在这种场合,就是棉布生产者的生产过剩的一部分,因为市场上出现的是棉布的生产过剩。

但是,除了投入织布生产的资本所直接雇用的工人以外,棉布再生产的这种停滞还影响一批别的生产者:纺纱者、棉花种植业者、纱锭和织机的生产者、铁和煤的生产者等等。所有这些人的再生产同样都要遭到破坏,因为棉布的再生产是他们进行再生产的条件。即使在他们自己的生产领域里没有生产过剩,就是说,即使那里生产的数量没有超过棉布工业销路畅通时所确定的合理的数量,这种情况也会发生。所有这些生产部门有一个共同点,就是它们不是把自己的收入(工资和利润,只要利润是作为收入来消费,而不是用于积累)用在它们自己的产品上,而是用在那些生产消费品,其中包括棉布的生产领域的产品上。这样,正因为市场上棉布过多,对于棉布的消费和需求就会减少。但是,对于用棉布的这些间接生产者的收入购买的作为消费品的其他一切商品的需求也会减少。棉布的这些间接生产者用来购买棉布和其他消费品的钱所以会受到限制和减少,就是因为市场上棉布过多。这也影响到其他商品(消费品)。它们现在突然发生相对的生产过剩,因为用来购买它们的钱减少了,从而对于它们的需求减少了。即使这些生产部门生产的东西并没有过多,现在也要发生生产过剩。

如果不仅棉布,而且麻布、丝绸和呢绒都发生生产过剩,那末不难理解,这些为数不多但居主导地位的物品的生产过剩就会在整个市场上引起多少带普遍性的(相对的)生产过剩。一方面,是再生产的一切条件出现过剩,各种各样卖不出去的商品充斥市场;另一方面,是资本家遭到破产,工人群众忍饥挨饿,一贫如洗。

可是,这一论证是两面的。如果说,一些主要消费品的生产过剩必然引起多少带普遍性的生产过剩,是容易理解的,那末决不能因此就说,这些主要消费品的生产过剩究竟怎样发生的问题也是可以理解的了。因为,普遍生产过剩的现象是从这些工业部门直接雇用的工人和为这些部门的产品生产各种先行要素(即生产这些部门所使用的不变资本的不同阶段)的一切生产部门的工人相互依存中得出来的。就后面这些部门来说,生产过剩是结果。但是,在前面那些部门中,生产过剩又是从哪里产生的呢?因为,只要前面那些部门扩大生产,后面这些部门也就会扩大生产,而随着生产的这种扩大,收入的普遍增长,从而这些部门本身的消费的普遍增长似乎也就有了保证。\endnote{手稿中这里接着有一段放在括号里的插话,其中举了由于使用纺机而引起纱生产过剩的局部危机的例子。本版以脚注形式把这段插话移到前面第595页。——第598页。}[719]

\tsectionnonum{[(13)生产扩大和市场扩大的不一致。李嘉图关于消费增长和国内市场扩大有无限可能性的见解]}

[720]如果我们说,不断扩大的生产{生产逐年扩大是由于两个原因:第一,由于投入生产的资本不断增长;第二,由于资本使用的效率不断提高;在再生产和积累期间,小的改良日积月累,最终就使生产的整个规模完全改观;这里进行着改良的积累,生产力的日积月累的发展}需要一个不断扩大的市场,而生产比市场扩大得快,那末,这不过是把要说明的现象用另一种说法说出,不是用它的抽象形式,而是用它的现实形式说出而已。市场比生产扩大得慢;换句话说,在资本进行再生产时所经历的周期中,——在这个周期中,资本不是简单地以原来的规模把自己再生产出来,而是以扩大了的规模把自己再生产出来,不是画一个圆圈,而是画一个螺旋形,——会出现市场对于生产显得过于狭窄的时刻。这会发生在周期的末尾。但这也仅仅是说:市场商品充斥了。生产过剩现在变得明显了。假如市场的扩大与生产的扩大步伐一致,就不会有市场商品充斥,不会有生产过剩。

但是,只要承认市场必须同生产一起扩大,在另一方面也就是承认有生产过剩的可能性,因为市场有一个外部的地理界限,一个国内市场同一个既是国内又是国外的市场相比是有限的,而后者和世界市场相比也是有限的,世界市场在每个一定的时刻也是有限的,但是潜在地是能扩大的。因此,如果承认为了不发生生产过剩,市场必须扩大,那也就是承认生产过剩是可能发生的,因为既然市场和生产是两个彼此独立的因素,那末,一个扩大同另一个扩大就可能不相适应,市场的范围对于生产来说可能扩大得不够快,新的市场——市场的不断扩大——可能很快被生产超过,因而扩大了的市场现在表现为一个界限,正如原来比较狭窄的市场曾经表现为一个界限一样。

因此,李嘉图否定关于随着生产的扩大和资本的增长市场也必定会扩大的观点,他是前后一贯的。照李嘉图看来,一个国家现有的全部资本,在这个国家可以有利地加以使用。因此李嘉图反驳亚·斯密,因为亚·斯密一方面提出过同他(李嘉图)一样的观点,另一方面以自己惯有的理性本能也反对过这个观点。斯密还不知道生产过剩以及从生产过剩产生危机的现象。他所知道的仅仅是同信用制度和银行制度一起自然发生的信用危机和货币危机。实际上,他把资本积累看做普遍的国民财富和福利的绝对增加。另一方面,他认为,单单从国内市场发展为国外市场、殖民地市场和世界市场本身,就是国内市场上存在所谓相对的(潜在的)生产过剩的证明。

值得把李嘉图反驳斯密的话引在这里:

\begin{quote}{“商人把他们的资本投入对外贸易或海运业时,他们总是出于自由选择而不是迫不得已;他们这样做是因为在这些部门中他们的利润比在国内贸易中要大一些。亚当·斯密曾正确地指出:‘每一个人对于食物的欲望都要受人胃的有限容量的限制}\end{quote}

{亚·斯密这里是大错特错了,因为他不把农业中生产的奢侈品包括在食物内},

\begin{quote}{但对于住宅、衣服、车马和家具方面的舒适品和装饰品的欲望似乎是没有限制或确定的界限的。’”李嘉图继续说:“因此,自然界对于一定时间内可以有利地用在农业上的资本的量,必然作了限制}\end{quote}

{不是因此而有出口农产品的民族吗?好象人们就不能违反自然,把一切可能的资本投于农业,以便例如在英国生产甜瓜、无花果、葡萄之类,栽培花卉之类,繁殖飞禽走兽之类?(请看例如罗马人单单在人工养鱼业上投下的资本。)难道工业的原料不是农业资本生产的吗?},

\begin{quote}{但是,自然界对于能用来生产生活上的‘舒适品和装饰品’的资本的量却没有规定什么界限〈好象自然界真是同这件事有什么关系似的!〉。人们所考虑的目的就是最大限度地得到这些物品。只是由于对外贸易或海运业可以更好地达到这个目的,人们才宁愿从事对外贸易或海运业,而不愿自己在国内生产所需要的商品或其代用品。但是,如果由于特殊情况,我们不能投资于对外贸易或海运业,那末,虽然获利较少,我们也会投资于国内;既然对于‘住宅、衣服、车马和[721]家具方面的舒适品和装饰品的欲望’是没有界限的,那末除了使我们维持生产这些物品的工人生活的能力受到限制的界限以外,用来生产这些物品的资本是不可能有任何界限的。但是,亚当·斯密却说成好象从事海运业不是出于自由选择,而是迫不得已;好象资本不投入这一部门就会闲置起来;好象国内贸易中的资本不限制在一定限度之内就会过剩。他说:‘当任何一个国家的资本增加到已经无法全部用来供应本国的消费并维持本国的生产劳动时{这一段话的着重号是李嘉图自己加的},资本的剩余部分就自然流入海运业,被用来为其他国家执行同样的职能’……但是,大不列颠的这部分生产劳动难道不能用来生产其他种类的商品并用以购买国内有更大需求的物品吗?如果不能这样,那末,虽然获利较少,难道我们不能用这种生产劳动在国内生产这些有需求的商品或者至少生产其代用品吗?如果我们需要天鹅绒,难道我们不能自己试制吗?如果试制不成,难道就不能生产更多的呢绒或我们需要的某种其他物品吗?我们生产工业品并用来在国外购买其他商品,是因为这样做比在国内生产能获得数量更多的商品{没有质量的差别!}。如果我们进行这种贸易的可能性被剥夺,我们马上就会重新开始为自己制造这些商品。但是,亚当·斯密的这种看法和他关于这一问题的所有一般论点是矛盾的。‘如果某个外国{李嘉图引用斯密的话}供应我们某种商品,比我们自己生产这种商品便宜,那就不如把我们自己的劳动用于我们有某种优越性的部门,而用我们自己的劳动的一部分产品向这个国家购买这种商品。国家的劳动总量由于总是与使用它的资本成比例{极不成比例!}〈这句话的着重号也是李嘉图自己加的〉,它就不会因此而减少,只不过需要找到能够最有利地使用它的部门而已。’他又说:‘所以,那些拥有的食物多于其消费量的人,总是愿意拿这部分多余的食物,或者可以说,这部分食物的价格,去交换其他物品。在满足了这种有限的欲望以后剩下的一切,就被用来满足那些永远不能完全满足并且看来根本没有止境的欲望。穷人为了获得食物,就尽力满足富人的嗜好,为了更有把握获得食物,他们就互相在自己报酬的低廉和工作的熟练方面竞争。工人人数随着食物数量的增加或者随着农业的改良和耕地的扩大而增加。由于他们的工作性质允许实行极细的分工,他们能够加工的材料数量比他们的人数增加得快得多。因此,对于任何一种材料,凡是人类的发明能够把它用来改善或装饰住宅、衣服、车马或家具的,都产生了需求;对于地下蕴藏的化石和矿物,对于贵金属和宝石,也都产生了需求。’从上述各点可以看出,需求是无限的,只要资本还能带来某种利润,资本的使用也是无限的,无论资本怎样多,除了工资提高以外,没有其他充分原因足以使利润降低。此外还可以补充一句:使工资提高的唯一充分而经常的原因,就是为越来越多的工人提供食品和必需品的困难越来越大。”(同上,第344—348页)}\end{quote}

\tsectionnonum{[(14)生产力不可遏止的发展和群众消费的有限性之间的矛盾是生产过剩的基础。关于普遍生产过剩不可能的理论的辩护论实质]}

生产过剩这个词本身会引起误解。只要社会上相当大一部分人的最迫切的需要,或者哪怕只是他们最直接的需要还没有得到满足,自然绝对谈不上产品的生产过剩(在产品量超过对产品的需要这个意义上讲)。相反,应当说,在这个意义上,在资本主义生产的基础上经常是生产不足。生产的界限是资本家的利润,决不是生产者的需要。但是,产品的生产过剩和商品的生产过剩是完全不同的两回事。李嘉图认为,商品形式对于产品是无关紧要的,其次,商品流通只是在形式上不同于物物交换,交换价值在这里只是物质变换的转瞬即逝的形式,因而货币只是形式上的流通手段;这一切实际上都是来源于他的这样一个前提:资产阶级生产方式是绝对的生产方式,也就是没有更确切的特殊规定的生产方式,因此,这种生产方式的一切规定都只是某种形式上的东西。因此,李嘉图也就不能承认资产阶级生产方式包含着生产力自由发展的界限——在危机中,特别是在作为危机的基本现象的生产过剩中暴露出来的界限。

[722]李嘉图从他引用、赞同并因而复述的斯密论点中看到,追求各种各样使用价值的无限“欲望”,总是在这样一种社会状态的基础上得到满足,在这种社会状态中,广大的生产者仍然或多或少只限于获得“食品”和“必需品”,因此,只要财富超出必需品的范围,绝大多数生产者就或多或少被排斥于财富的消费之外。

当然,后一种情况在古代以奴隶制为基础的生产中也是存在的,并且更加如此。但是古代人连想也没有想到把剩余产品变为资本。即使这样做过,至少规模也极有限。(古代人盛行本来意义上的财宝贮藏,这说明他们有许多剩余产品闲置不用。)他们把很大一部分剩余产品用于非生产性支出——用于艺术品,用于宗教的和公共的建筑。他们的生产更难说是建立在解放和发展物质生产力(即分工、机器、将自然力和科学应用于私人生产)的基础上。总的说来,他们实际上没有超出手工业劳动。因此,他们为私人消费而创造的财富相对来说是少的,只是因为集中在少数人手中,而且这少数人不知道拿它做什么用,才显得多了。如果说因此在古代人那里没有发生生产过剩,那末,那时有富人的消费过度,这种消费过度,到罗马和希腊的末期就成为疯狂的浪费。古代人中间的少数商业民族,部分地就是靠所有这些实质上贫穷的民族养活的。而构成现代生产过剩的基础的,正是生产力的不可遏止的发展和由此产生的大规模的生产,这种大规模的生产是在这样的条件下进行的:一方面,广大的生产者的消费只限于必需品的范围,另一方面,资本家的利润成为生产的界限。

李嘉图和其他人对生产过剩等提出的一切反对意见的基础是,他们把资产阶级生产或者看作不存在买和卖的区别而实行直接的物物交换的生产方式,或者看作社会的生产,在这种生产中,社会好象按照计划,根据为满足社会的各种需要所必需的程度和规模,来分配它的生产资料和生产力,因此每个生产领域都能分到为满足有关的需要所必需的那一份社会资本。这种虚构,一般说来,是由于不懂得资产阶级生产这一特殊形式而产生的,而所以不懂又是由于一种成见,认为资产阶级生产就是一般生产。正象一个信仰某一宗教的人把这种宗教看成一般宗教,认为除此以外都是邪教一样。

相反,倒是应该问一问:在资本主义生产的条件下,每个人都为自己而劳动,而特殊劳动必须同时表现为自己的对立面即抽象的一般劳动,并以这种形式表现为社会劳动,——在这样的资本主义生产的基础上,不同生产领域之间的必要的平衡和相互联系,它们之间的限度和比例的建立,除了通过经常不断地消除经常的不协调之外,用别的办法又怎么能够实现呢?这一点在谈到通过竞争达到平衡时就已经得到承认,因为这种平衡总是以有什么东西要平衡为前提,就是说,协调始终只是消除现存不协调的那个运动的结果。

因此,李嘉图也承认个别商品有可能充斥市场。不可能的只是同时的普遍的市场商品充斥。因此,不否认任何特殊生产领域有生产过剩的可能性。他认为,[普遍的]生产过剩和由此而来的普遍的市场商品充斥所以不可能,是因为这种现象不可能在一切生产领域同时发生。“普遍的市场商品充斥”这个词总是应该有保留地来理解,因为在发生普遍生产过剩的时候,有些领域的生产过剩始终只是主要交易品生产过剩的结果、后果,始终只是相对的,只是因为其他领域存在着生产过剩才成为生产过剩。

辩护论恰好把这一点颠倒过来了。照这些辩护论者的说法,只出现主动的生产过剩的那些主要交易品(一般说这是只能大规模地和用工厂方式进行生产的物品,在农业上也一样),它们的生产过剩之所以成为生产过剩,仅仅因为会出现相对的,或者说,被动的生产过剩的那些物品存在着生产过剩。按照这种看法,生产过剩之所以存在,仅仅因为生产过剩不是普遍的。生产过剩的相对性,即一些领域中现实的生产过剩引起另一些领域的生产过剩这个事实,被表述如下:普遍的生产过剩并不存在,因为,如果生产过剩是普遍的,一切生产领域相互之间就会保持同样的比例;就是说,普遍的生产过剩等于按比例生产,而按比例生产是排除生产过剩的。据说,这是反对普遍生产过剩的论据。[723]就是说,因为绝对意义上的普遍生产过剩并不是生产过剩,而只是一切生产领域的生产力的发展都超过了通常的水平,所以,据说,现实的生产过剩(它恰恰不是这种不存在的、自我消除的生产过剩)并不存在。其实,只是因为现实的生产过剩不是这样的生产过剩,所以它是存在的。

这种可怜的诡辩如果更详细地加以考察,可以归结如下:

假定铁、棉布、麻布、丝绸、呢绒等等发生生产过剩,那末不能说,例如煤生产得太少,因而造成了上述生产过剩;因为铁等等的生产过剩也就包含着煤的生产过剩,正如棉布的生产过剩也就包含着棉纱的生产过剩一样。{和棉布相比,棉纱可能生产过剩,和机器等相比,铁可能生产过剩。这种生产过剩总是不变资本的相对生产过剩。}因此,如果有某些物品作为组成要素即原料、辅助材料或生产资料加入另一些物品(加入这样一些商品,这些“商品可能生产过多,可能在市场上过剩,以致不能补偿它所花费的资本”\authornote{见本册第570、575、577页,那些地方引用了李嘉图这段话的全文。——编者注}),而另一些物品的绝对生产过剩正是需要加以说明的事实,从而某些物品的生产过剩是不言而喻的,那末,就根本谈不上某些物品的生产过剩。因此,说得上生产不足的是其他一些物品,它们直接归属的生产领域,既不属于根据假定要发生生产过剩的主要交易品的生产领域,也不属于这样一些领域,这些领域由于为主要交易品进行中间生产,其生产规模必须至少同产品最后阶段的规模一样,——不过没有什么东西会妨碍这些领域的生产达到更大的规模,因此在生产过剩内部又可能发生生产过剩。例如,虽然煤生产的数量必须使一切以煤作为必要生产条件的工业部门都能进行生产,因而铁、棉纱等等的生产过剩已经包含了煤的生产过剩(即使煤生产的数量只与铁和棉纱的生产成比例),但是,生产出来的煤也可能比铁、棉纱等的生产过剩所要求的还多。这种情况不仅是可能的,而且是非常可能的。因为,煤和棉纱的生产,以及其他所有仅仅为那些必须在别的领域完成的产品提供条件或充当准备阶段的领域的生产,都不是依据直接的需求,不是依据直接的生产或再生产,而是依据它们自身不断扩大的程度、限度、比例来进行的。而在这种考虑的支配下可能超过目标,那是不言而喻的。由此可见,生产不足的[不是上述那些产品,而]是其他物品,例如钢琴、宝石等,生产不足是发生在这些其他物品的部门中。{当然,也会发生这样的生产过剩,在那里,非主要物品的生产过剩不是后果,相反,生产不足倒是生产过剩的原因,例如在谷物歉收或棉花歉收等的情况下。}

当这种[关于生产不足的]说法被应用到国际范围——就象萨伊\endnote{马克思指萨伊的下述论断(在他的《给马尔萨斯的信》1820年巴黎版第15页):例如,如果英国商品充斥意大利市场,那末,原因就在于能够同英国商品交换的意大利商品生产不足。萨伊的这些论断在匿名著作《论马尔萨斯先生近来提倡的关于需求的性质和消费的必要性的原理》(1821年伦敦版第15页0中引证过,在马克思的第XII本札记本第12页对这部著作所作的摘录中也有这些论断。并参看马克思在本卷第一册第276页分析的萨伊的这一论点:“某些产品的滞销,是由另一些产品太少引起的。”——第607页。}和继萨伊之后的其他经济学家所作的那样,——的时候,就更加暴露其荒谬了。例如,他们断言,不应说英国生产过剩,而应说意大利生产不足。如果第一,意大利有足够的资本来补偿以商品形式输出到意大利的英国资本;第二,意大利用自己这笔资本生产出了英国资本部分地为补偿它自己和部分地为补偿它所带来的收入所需要的特殊物品,那就不会发生任何生产过剩。因此,实际地——对意大利的实际生产来说——存在着的英国的生产过剩这个事实就不存在了,存在的只是想象的意大利的生产不足这个事实;其所以说是想象的,是因为它以意大利存在着那里并不存在的[724]资本以及生产力的发展为前提,其次,是因为这里还作了同样空想的假定,就是这笔在意大利并不存在的资本用得恰好符合需要,使英国的供给和意大利的需求、英国的生产和意大利的生产能互相补充。换句话说,这无非是意味着:如果需求和供给彼此相符,如果资本按这样的比例在一切生产领域之间进行分配,以致一种物品的生产就包含着另一种物品的消费,因而也就包含着它自己的消费,那就不会发生生产过剩。如果不发生生产过剩,那生产过剩就不会发生。但是,因为在一定的条件下资本主义生产只能在某些领域无限制地自由发展,所以,如果资本主义生产必须在一切领域同时地、均匀地发展,那就根本不可能有任何资本主义生产。因为在上述某些领域生产过剩绝对存在,所以在没有[绝对的]生产过剩的那些领域,也就相对地存在着生产过剩。

总之,这种用一方面的生产不足来说明另一方面的生产过剩的观点无非是说:如果生产按比例进行,那就不会发生生产过剩。如果需求和供给彼此相符,也就不会发生生产过剩。如果一切领域具有进行并扩大资本主义生产的同样的可能性,如分工、机器、向遥远的市场输出、大规模生产等等,如果互相贸易的一切国家具有进行生产(而且是彼此各不相同又互为补充的生产)的同样的能力,也就不会有生产过剩。因此,如果发生生产过剩,那是因为所有这些虔诚的愿望没有实现。或者更抽象地说:如果到处都均匀地发生生产过剩,那就不会在一处发生生产过剩。但是,现在资本没有大到足以使生产过剩带有这样普遍的性质,因此就只会发生局部的生产过剩。

如果更仔细地进行考察,这个幻想可归结如下:

承认每一单个的生产部门都可能发生生产过剩。根据前面的解释,唯一能够防止在一切部门同时发生生产过剩的情况,是商品同商品的交换,就是说,抱这种观点的人求助于假定存在的是物物交换的条件。但是通向这种遁辞的道路恰好被切断了:商品流通不是物物交换,因此一种商品的卖者完全不必同时又是另一种商品的买者。可见,这整个遁辞的基础是撇开货币,撇开这里的问题不是产品交换而是商品流通这一点,而对于商品流通来说,买和卖的彼此分离具有重大的意义。

{资本流通本身包含着破坏的可能性。例如,在货币再转化为资本的生产条件时,问题不仅在于货币重新转化为同样的(按种类来说)使用价值,而且,为了使再生产过程重复进行,十分重要的是能够按原来的价值(或者按更低的价值,那当然更好)得到这些使用价值。但是,这些再生产要素有很大一部分,即由原料组成的部分,可能由于下述两个原因涨价:第一,如果生产工具的数量增加得比那段时间内能够生产出来的原料数量快。第二,由于收成的不稳定。因此,正如图克正确地指出过的\endnote{托·图克《价格和流通状况的历史》1838—1857年伦敦版第1—6卷。图克在这部六卷著作中的许多地方,特别是在1848年出版的第四卷开头,谈到气候条件对价格的影响。——第609页。},气候在现代工业中起着如此重大的作用。(这句话同样适用于与工资有关的食物。)因此,货币再转化为商品,完全同商品转化为货币一样,也可能遇到困难,也可能造成危机的可能性。如果考察的是简单流通而不是资本流通,那就不会发生这些困难。}(危机还有许多因素、条件、可能性,只有在分析更加具体的关系,特别是分析资本的竞争和信用时,才能加以考察。)

[725]否认商品的生产过剩,却承认资本的生产过剩。可是资本本身就是由商品组成的,或者说,如果资本由货币组成,它就必须再转化为这种或那种商品,才能执行资本的职能。因此,什么叫作资本的生产过剩呢?就是预定用来生产剩余价值的那些价值量的生产过剩(或者,从资本的物质内容方面来考察,就是预定用来进行再生产的那些商品的生产过剩),——因此,就是再生产的规模太大,这同直截了当说生产过剩是一个意思。

更加明确地说,资本的生产过剩无非是,为了发财而生产的东西过多了,或者说,不是用作收入进行消费,而是用来获得盈利、进行积累的那部分产品太多了;这部分产品不是用来满足它的所有者的私人需要,而是用来为它的所有者提供抽象的社会财富即货币,提供更大的支配别人劳动的权力——资本,或者说,扩大这个权力。这是一方的说法。(李嘉图否认这一点。\authornote{见本册第567页。——编者注})而另一方用什么来解释商品的生产过剩呢?就用生产不够多种多样,某些消费品生产得不够大量来解释。很清楚,这里不可能涉及生产消费的问题;因为工厂主生产过多的麻布,他对纱、机器和劳动等的需求必然因而增加。因此,这里涉及的是私人消费的问题。麻布生产得太多了,但是橙子也许就生产得太少了。起初否定货币,是为了说明买和卖的彼此分离[并不存在]。现在否定资本,是为了把资本家说成是完成W—G—W这个简单动作并且为了个人消费而进行生产的人,而不是把他看作以发财为目的、以把一部分剩余价值再转化为资本为目的而进行生产的资本家。但是,资本太多这句话无非是说,作为收入被消费并且在既定的条件下可能被消费的产品太少。(西斯蒙第\endnote{西斯蒙第用“生产和消费之间日益不成比例”来解释危机(让·沙·列·西蒙·德·西斯蒙第《政治经济学新原理》1827年巴黎第2版第1卷第371页)。恩格斯在《反杜林论》中指出,“用消费水平低来解释危机,起源于西斯蒙第,在他那里,这种解释还有一定的意义”(见《马克思恩格斯全集》中文版第20卷第311页),而在《资本论》第二卷序言中,恩格斯从西斯蒙第《新原理》中引了下面一段作为西斯蒙第观点的例证:“可见,由于财富集中在少数所有者手中,国内市场就越来越缩小,工业就越来越需要到国外市场去寻找销路,但是在那里,它会受到更大的变革的威胁”(见《马克思恩格斯全集》中文版第24卷第23页)。马克思在《剩余价值理论》第三册中又回过头来谈到西斯蒙第对危机的观点,既指出西斯蒙第的概念中有价值的因素,又指出它所固有的根本缺点(特别是见论马尔萨斯的一章,马克思手稿第775页)。——第610页。}。)那末,为什么麻布生产者要求谷物生产者消费更多的麻布,为什么谷物生产者要求麻布生产者消费更多的谷物呢?为什么麻布生产者自己不把他的更大一部分收入(剩余价值)实现在麻布上,而租地农场主自己不把他的更大一部分收入实现在谷物上呢?就每一个人单独来说,人们承认,他们的资本化的需要妨碍这样作(且不说每种需要都有一定的限度),但是就全体总起来说,人们就不承认这一点了。

(这里,我们完全撇开了由于商品的再生产比商品的生产便宜而产生的危机因素。而市场上的现有商品的贬值就是由此而来的。)

资产阶级生产的一切矛盾,在普遍的世界市场危机中集中地暴露出来,而在局部的(按内容和范围来说是局部的)危机中只是分散地、孤立地、片面地暴露出来。

至于专门谈到生产过剩,那它是以资本的一般生产规律为条件:按照生产力的发展程度(也就是按照用一定量资本剥削最大量劳动的可能性)进行生产,而不考虑市场的现有界限或有支付能力的需要的现有界限。而这是通过再生产和积累的不断扩大,因而也通过收入不断再转化为资本来进行的,另一方面,[726]广大生产者的需求却被限制在需要的平均水平,而且根据资本主义生产的性质,必须限制在需要的平均水平。

\tsectionnonum{[(15)李嘉图关于资本积累的各种方式和积累的经济效果的观点]}

李嘉图在第八章(《论赋税》)中说:

\begin{quote}{“如果一个国家的年生产能补偿它的年消费而有余,人们就说,这个国家的资本增加了;如果一个国家的年消费甚至不能由它的年生产来补偿,人们就说,这个国家的资本减少了。因此,资本可能由于增加生产或由于减少非生产消费而增加。”(第162—163页)}\end{quote}

李嘉图这里所说的“非生产消费”,和他在这段话的注释(第163页)中所说的一样,是指非生产劳动者即“不再生产另一个价值的人”的消费。因此,年生产的增加是指年生产消费的增加。年生产消费在非生产消费保持不变或者甚至有所增加时可以通过它自身的直接增加而增加,也可以通过减少非生产消费而增加。

\begin{quote}{同一个注释中说:“我们说收入节约下来加入资本,我们的意思是,加入资本的那部分收入,是由生产工人消费,而不是由非生产工人消费。”}\end{quote}

我已经指出\authornote{见本册第537—561页。——编者注},收入转化为资本决不等于收入转化为可变资本或者说收入用于工资。可是,李嘉图的想法却正是这样。

李嘉图在同一个注释中说:

\begin{quote}{“如果劳动价格大大提高,以致增加资本也无法使用更多的劳动,那我就要说,这样增加的资本仍然是非生产地消费的。”}\end{quote}

因此,不是单单收入由生产工人消费就使这种消费成为“生产的”消费,而是收入由生产剩余价值的工人消费才使这种消费成为“生产的”消费。按照这种看法,资本只有在支配[比原来]更多的劳动的时候才会增加。

李嘉图在第七章(《论对外贸易》)中写道:

\begin{quote}{“积累资本有两条道路:或者增加收入,或者减少消费,都可以积蓄资本。如果我的利润从1000镑增加到1200镑,而我的支出保持不变,我每年就比以前多积累200镑。如果我从我的支出中节约200镑,而我的利润仍旧不变,结果也是一样:我的资本每年将增加200镑。”(第135页)“如果由于采用机器,用收入购买的一切商品的价值下降20%,我就能够象我的收入增加20%那样有效地进行节约;但是在一种场合是利润率保持不变,在另一种场合是利润率提高了20%。如果从国外输入廉价商品使我能够从我的支出中节约20%,其结果将同机器降低了这些商品的生产费用完全一样,但是利润不会提高。”(第136页)}\end{quote}

(就是说,如果比较便宜的商品既不加入可变资本,又不加入不变资本,利润就不会提高。)

因此,如果收入的支出情况不变,积累就是利润率提高的结果{但是积累不仅取决于利润的高低,而且取决于利润的量};如果利润率不变,积累就是支出减少的结果,而李嘉图在这里认为,支出的减少是“用收入购买的商品”降价(或者由于采用机器,或者由于对外贸易)的结果。

在第二十章(《价值和财富,它们的特性》)中说:

\begin{quote}{“一个国家的财富〈李嘉图指的是使用价值〉可以用两种方法增加:它可能通过把更大的一部分收入用于维持生产劳动来增加,这不仅能增加商品总量的数量,而且能增加其价值;或者它也可能不通过使用追加的劳动量,而通过提高原来劳动量的生产率的方法来增加,这能增加商品的数量,但不能增加商品的价值。在第一种情况下,不仅一个国家的财富会增加,而且财富的价值也会增加。国家变富是由于节约,由于减少奢侈品和享乐品方面的支出,并且把这种节约所得用在再生产上。[727]在第二种情况下,不必减少奢侈品和享乐品方面的支出,也不必增加所使用的生产劳动量;但是用同量劳动将生产出更多的产品;财富将增长,但其价值不增加。在这两种增加财富的方法中,第二种方法应该是更可取的,因为它可以避免第一种方法必然带来的享乐品的缺乏和减少,而得到同样的结果。资本是一个国家为了未来生产而使用的那部分财富,它可以用增加财富的同样方法来增加。追加资本无论是由于技艺和机器的改进而得到的,还是由于把更大的一部分收入用于再生产而得到的,在生产未来财富时都有同样的效力;因为财富总是取决于生产出来的商品量,而与制造生产中所使用的工具的容易程度无关。一定量的衣服和食物将维持并雇用同样的人数,因而将保障同样工作量的完成,无论这些食物和衣服是由100人的劳动还是由200人的劳动生产出来的;但是在生产它们时如果用了200人,它们就会有加倍的价值。”(第327—328页)}\end{quote}

李嘉图对问题的第一个提法是:

在支出不变的情况下,如果利润率提高,积累就会增加,或者说,在利润率不变的情况下,如果支出(按价值)由于用收入购买的商品降价而减少,积累就会增加。

现在,李嘉图提出了另一个对立的提法:

如果有更大一部分收入从个人消费领域抽出,转入生产消费,如果用这样节约下来的那部分收入去推动更多的生产劳动,积累就会增加,资本按量和价值来说都能进行积累。在这种情况下,是靠节约来积累。

或者说,[用于个人消费的]支出保持不变,也不使用任何追加的生产劳动量,但是同样的劳动会生产出更多的产品,劳动的生产力提高了。花费同样的劳动会生产出更大量、更好、因而也更便宜的构成生产资本的要素即原料、机器等{李嘉图在前面说是用收入购买的商品,现在却说是作为生产工具使用的商品}。在这种情况下,积累既不取决于利润率有所提高,又不取决于由于节约有更大一部分收入变成资本,也不取决于由于用收入购买的商品降价而用于非生产目的的那部分收入减少。在这里,积累取决于提供资本本身要素的生产领域中劳动生产率提高,就是说,取决于作为原料、工具等加入生产过程的商品降价。

如果劳动生产力的增长,是由于同可变资本相比,固定资本的生产有所增加,那末,不仅再生产的量,而且再生产的价值都会增加,因为固定资本的价值有一部分是加入当年再生产的。这可能同人口的增加和所使用的工人人数的增加同时发生,虽然所使用的工人人数,同这些工人推动的不变资本相比,相对地说在不断减少。因此,不仅财富会增加,价值也会增加,并且,虽然劳动的生产率提高了,虽然同生产出来的商品量相比,劳动量减少了,被推动起来的活劳动量却大了。最后,即使劳动生产率保持不变,可变资本和不变资本也可能和每年人口的自然增长一起以同一程度增长。在这种情况下,资本不仅按量而且按价值,都能进行积累。最后这几点李嘉图完全没有注意到。

李嘉图在同一章里说:

\begin{quote}{“工业中100万人的劳动总是生产出相同的价值,但并非总是生产出相同的财富。}\end{quote}

(这是完全错误的,100万人的产品的价值,不仅取决于他们的劳动,而且取决于他们借以进行劳动的资本的价值;因此,这个价值将根据他们借以进行劳动的过去已经生产出来的生产力的大小不同而大不相同。)

\begin{quote}{机器的发明,技艺的提高,分工的改进,或者能够进行更有利的交换的新市场的发现,——这一切使100万人在一种社会状态下能够生产的‘必需品、舒适品和享乐品’等财富的量,比他们在另一种社会状态下所能生产的量大一倍或两倍。但是他们不能因此就使价值有所增加}\end{quote}

(肯定有所增加,因为他们过去的[728]劳动以大得多的规模加入新的再生产),

\begin{quote}{因为每一种物品的价值的提高或降低,都取决于生产这种物品的容易程度,换句话说,都取决于生产这种物品所花费的劳动量。}\end{quote}

(每一单位商品可能会跌价,但是增长了的商品总量的价值却会增加。)

\begin{quote}{我们假定,一定数量的人的劳动用一定的资本生产出1000双袜子,由于发明了机器,同样数量的人能生产2000双,或者他们除了继续生产1000双袜子以外,还能生产500顶帽子。那末,2000双袜子的价值,或者1000双袜子和500顶帽子的价值将不多不少恰好等于采用机器前1000双袜子的价值,因为它们将是同量劳动的产品。}\end{quote}

(注意:如果新采用的机器毫无所值的话。)

\begin{quote}{不过商品总量的价值还是会减少;因为,由于技术改良而增加了的产品量的价值,虽然将恰恰等于技术改良前生产的较小量产品的价值,但是这种变动对于在技术改良前已经制造出来而还没有消费掉的那部分商品也会发生影响。这些商品的价值将减少,因为它们必须全部降到在技术改良后的各种优越条件下生产出来的商品的价值水平,而且,虽然商品量增加了,财富增加了,享乐品的量增加了,但是社会所拥有的价值量将会减少。由于不断提高生产的容易程度,我们就不断减少某些以前已经生产出来的商品的价值,虽然我们用这同一方法不仅增加了国家的财富,而且还增加了未来生产的能力。”(第320—322页)}\end{quote}

李嘉图这里谈的是,生产力的日益发展会使在比较不利条件下生产出来的商品贬值,不论这些商品是仍然停留在市场上,还是作为资本正在生产过程中发挥作用。但是从这里决不能得出结论说,“商品总量的价值会减少”,尽管这个总量的某一部分的价值会减少。这种结果只有在两种情况下才会产生,第一,如果由于技术进步而新增加的机器和商品的价值,小于原有同类商品的价值已经贬值的部分;第二,如果我们不考虑下面这一点,即随着生产力的发展,生产领域也不断增加,因而为投资开辟了以前根本没有的新部门。生产在发展进程中不仅会变得更便宜,而且会变得更加多样化。

李嘉图在第九章(《原产品税》)中写道:

\begin{quote}{“反对原产品税的第三种意见是,认为提高工资和降低利润妨碍积累,其作用同土壤自然贫瘠一样;关于这种意见,我在本书的另一部分已试图证明:在支出上和在生产上,通过减少商品价值和通过提高利润率,都能够同样有效地进行节约。当我的利润从1000镑增加到1200镑,而价格不变的时候,我通过节约来增加资本的能力会增大,如果我的利润不变,而商品大大跌价,使我用800镑能够买到以前用1000镑才能买到的东西,我通过节约来增加资本的能力也会增大,但是在前一种情况下增大的程度比不上后一种情况大。”(第183—184页)}\end{quote}

即使纯收入按其价值量来说并不减少,产品(或者确切些说,在资本家和工人之间分配的那部分产品)的全部价值也可能减少。(按所占的比例来说,纯收入还可能增加。)关于这一点,在第三十二章(《马尔萨斯先生的地租观点》)中说:

\begin{quote}{“但是,马尔萨斯先生的全部论证是建立在这样一个不可靠的基础上:它假定,既然国家的总收入减少,纯收入也一定按同一比例减少。本书的目的之一就是要说明,必需品的实际价值每有降低,工资也就降低,而资本利润则提高;换句话说,在任何一定的年价值中,归工人阶级所得的份额会减少,而归用基金使用这个阶级的人所得的份额会增加。假定某工厂生产的商品价值为1000镑,这一价值在老板和他的工人之间分配,工人得800镑,老板得200镑。[729]如果这些商品的价值降到900镑,同时由于必需品降价在工资上节省了100镑,那末,老板的纯收入丝毫不会减少,因而,他支付同额税款将同价格下降前一样容易。”(第511—512页)}\end{quote}

在第五章(《论工资》)中说:

\begin{quote}{“虽然工资有符合于它的自然率的趋势,但是在一个不断进步的社会里,工资的市场率却可能在一个不定的时期内经常高于它的自然率,因为资本的增加给对劳动的新需求造成的刺激还没有发挥完它的作用,资本的再一次增加却又开始发挥同样的作用了。所以,如果资本的增加是逐渐的和经常的,对劳动的需求就会不断地刺激人口的增加。”(第88页)}\end{quote}

从资本主义观点出发,一切都是颠倒着表现出来的。工人人口量和劳动生产率程度既决定资本的再生产,又决定人口的再生产。这里却颠倒过来,表现为资本决定人口。

李嘉图在第九章(《原产品税》)中说:

\begin{quote}{“资本的积累自然在劳动的雇主之间引起日益加剧的竞争,因而引起劳动价格的提高。”(第178页)}\end{quote}

这取决于资本的不同组成部分在资本积累时以什么样的比例增加。资本可能进行积累,而对劳动的需求却可能绝对地或相对地减少。

既然,按照李嘉图的地租理论,随着资本的积累和人口的增加,由于必需品价值提高或者说农业生产率下降,利润率有下降的趋势,那末,积累就有阻碍积累的趋势,利润率下降的规律——因为随着工业的发展,农业生产率越来越降低——就象恶运一样降临到资产阶级生产的头上。相反,亚·斯密却欣赏利润率下降。在他看来,荷兰是一个模范的国家。亚·斯密指出,利润率下降,迫使除了最大的资本家以外的大多数资本家把他们的资本用到生产上去,而不是靠利息过活,因而对生产是一种刺激。在李嘉图的门徒的著作中,对这种致命趋势的恐惧具有悲喜剧的形式。

现在我们把李嘉图有关这个问题的几段话引在这里。

第五章(《论工资》):

\begin{quote}{“在不同的社会阶段,资本,或者说,使用劳动的资金的积累,速度有快有慢,它在所有情况下都必定取决于劳动生产力。一般说来,在存在着大量肥沃土地时,劳动生产力最大:在这种时期,积累往往进行得很快,以致工人的供给赶不上资本增加的速度。据计算,在有利条件下,人口在25年内可能增加一倍;但是在同样有利的条件下,一个国家的全部资本可能在更短的时期内增加一倍。在这种情况下,工资在整个时期内都会有上涨的趋势,因为对劳动的需求将比劳动的供给增加得更快。在从先进得多的国家引进技艺和知识的新殖民地,资本可能有比人口增加得更快的趋势;如果工人的不足不能从人口更多的国家得到补充,这种趋势就会大大提高劳动的价格。随着这些国家的人口增多、质量较坏的土地投入耕种,资本增加的趋势就会减弱;因为现有人口的需要满足之后剩下来的剩余产品,必然同生产的容易程度成比例,也就是说,从事生产的人数越少,剩余产品就越多。因此,在最有利的条件下,生产力虽然仍然有可能增长得比人口快,但是这种情况不会持续很久;因为土地的数量有限,质量不同,只要投在土地上的资本有所增加,就会使所得产品的比率下降,而人口繁殖力却始终不变。”(第92—93页)}\end{quote}

(最后这句话是牧师的发明。人口繁殖力会随着劳动生产力的减退而减退。)

这里,首先要指出,李嘉图承认“资本的积累……在所有情况下都必定取决于劳动生产力”,因此,第一性的是劳动,而不是资本。

其次,根据李嘉图的论断可以认为,早就有人居住的工业发达的国家,从事农业的人比殖民地多,而实际情况恰恰相反。为了生产同量的产品,[730]例如,英国使用的农业工人就比其他任何国家——不论新国家还是老国家——都少。固然,这里有较大一部分非农业人口间接参加农业生产。但是,即使这部分人口较多,其程度也远远赶不上比较不发达国家的直接农业人口超过比较发达国家农业人口的那种程度。即使我们假定,英国的谷物较贵,生产费用较大。使用的资本较多。加入农业生产的过去劳动较多,不过活劳动较少。但是,由于已有的生产基础,农业资本的再生产所花费的劳动量较少,虽然这笔资本的价值也是在产品中得到补偿。

第六章(《论利润》)。

首先还要说几句。我们已经看到,剩余价值不仅取决于剩余价值率,而且取决于同时雇用的工人人数,因而取决于可变资本量。

积累也不是直接决定于剩余价值率,而是决定于剩余价值对预付资本总额之比,即决定于利润率;并且,与其说决定于利润率,不如说决定于利润总量;我们已经看到,对社会总资本来说,利润总量和剩余价值总量是相同的,但对不同部门所使用的单个资本来说,利润总量却可能和各个不同部门所生产的剩余价值量大不相同。如果把资本积累全部加以考察,那末,利润就等于剩余价值,利润率就等于剩余价值/资本,或者更确切地说,等于按每100单位的资本计算的剩余价值。

如果利润率(百分率)既定,利润总量就取决于预付资本量;因此,既然积累决定于利润,积累也就取决于预付资本量。

如果资本总额既定,利润总量就取决于利润率的高低。

因此,利润率高的小资本可能比利润率低的较大资本提供的利润量大。

举例来说:

\todo{}

如果资本的乘数和利润率的除数相等,就是说,如果资本量增加的比例和利润率下降的比例相同,利润量总额不变。100的10%得10,2×100的(10/2)%或5%同样得10。换句话说:如果利润率下降的比例和资本积累(增加)的比例相同,利润量不变。

如果利润率下降快于资本增加,利润量总额就减少。500的10%得利润量50。但是,六倍资本额即6×500或3000的(10/10)%或1%只得30。

最后,如果资本增加快于利润率下降,那末,尽管利润率下降,利润量还会增加。例如,100在利润为10%时得利润量10。但是300(3×100)的4%(即利润率降了60%)得利润量12。

现在回过头来看看李嘉图的论点。

李嘉图在第六章(《论利润》)中写道:

\begin{quote}{“因此,利润有下降的自然趋势,因为随着社会的进步和财富的增长,为了生产必需的追加食物量,必须花费越来越多的劳动。利润的这种趋势,这种可以说是重力作用,幸而由于生产必需品所使用的机器的改良以及农业科学上的发现而时常受到抑制,这些改良和发现使我们能够减少一部分以前所需要的劳动量,[731]因而能降低工人生活必需品的价格。可是,必需品价格和工资的提高是有限度的;因为……一旦工资达到720镑,即等于租地农场主的全部收入,积累就一定停止,因为那时任何资本都不可能提供利润,对追加劳动也不可能有任何需求,因此,人口也将达到最高点。事实上,在这以前很久,很低的利润率就会使一切积累停止,一个国家的全部产品在支付了工人的工资以后,几乎都将属于土地所有者以及什一税和其他税的所得者。”(第120—121页)}\end{quote}

这是李嘉图观念中的资产阶级的“神的毁灭”,是世界的末日。

\begin{quote}{“远在这种价格水平成为持久的状况以前,积累的一切动机就会消失,因为任何人从事积累,都只是为了把他的积累生产地加以使用……因此,这种价格水平是决不可能存在的。正如工人没有工资就不能生活一样,租地农场主和工厂主没有利润也不能生活。他们的积累动机将随利润的每次减少而减少,当他们的利润低到不能对他们的辛劳和他们在把资本生产地加以使用时必然遇到的风险提供足够的补偿的时候,积累的动机将完全消失。我必须再次指出,利润率的降低……要迅速得多;因为如果产品的价值象我在前面假定的情况下说过的那样高,租地农场主的资本的价值就会大大增加,因为他的资本必然是由许多价值已经增加的商品组成的。在谷物价格可能从4镑上涨到12镑以前,租地农场主的资本的交换价值也许就已经增加一倍,等于6000镑而不是3000镑了。因此,如果租地农场主的利润原来是180镑,或者说,是他原有资本的6%,那末现在实际利润率不会高于3%,因为6000镑的3%是180镑,而且一个持有6000镑的新租地农场主要经营农业,就只有接受这种条件。”(第123—124页)“我们也可以预计到,虽然资本的利润率会因农业中资本的积累和工资的提高而降低,利润总额仍然会增加。例如,假定连续多次进行积累,每次为10万镑,而利润率从20%降到19%,18%,17%,就是说,不断下降,那末,我们可以预计到,先后相继的资本所有者得到的利润总额会不断增加:资本为20万镑时的利润总额会大于资本为10万镑时的利润总额,资本为30万镑时的利润总额还会更大些,依此类推,因此,即使利润率不断下降,利润总额也会随着资本的每次增加而增加。但是这样的级数只在一定时间内有效。例如,20万镑的19%大于10万镑的20%,30万镑的18%又大于20万镑的19%;但是当资本积累到了很大的数额,而利润率又下降的时候,进一步的积累就会使利润总额减少。例如,假定积累达到100万镑,利润率为7%,利润总额就是7万镑。如果现在100万镑再加上10万镑资本,而利润率降到6%,那末,虽然资本总额从100万镑增加到110万镑,资本所有者得到的将只是66000镑,或者说,少了4000镑。然而,只要资本多少能提供一些利润,就不会有既不增加产品,又不增加价值的资本积累。在使用10万镑追加资本时,原有资本的任何一部分的生产率都不会降低。国内土地和劳动的产品一定会增加,产品的价值也会增加,这不仅是由于加上了除原有产量外新增产品的价值,而且是由于生产最后一部分产品的困难加大使全部土地产品得到了新的价值。不过,当资本积累已经很大时,尽管产品的价值增加了,产品进行分配的结果将是,归利润的部分比以前减少,而归地租和工资的部分则增加。”(第124—126页)“虽然生产了一个较大的价值,但这一价值在支付地租以后剩下的部分中却有较大的份额是由生产者消费的,而这一点,并且只有这一点,却调节着利润的大小。在土地获得丰收时,工资可能暂时提高,生产者的消费可能超过他们通常的份额;但是因此而产生的对人口增加的刺激,很快就会使工人的消费回到通常的水平。但是当较坏土地投入耕种时,或者当花费在老地上的资本和劳动增加而收益减少时,上述影响将是持久的。”(第127页)[732]“因此,积累的效果在不同国家是不同的,并且主要取决于土地的肥力。一个国家无论多么辽阔,如果土地贫瘠并禁止粮食输入,那末,即使是较少量的资本积累也将引起利润率的大大降低和地租的迅速提高;相反,一个小的但是土地肥沃的国家,特别是如果它允许自由输入粮食,却能够积累很大的资本,而又不引起利润率的大大降低或地租的大量增加。”(第128—129页)税收(第十二章《土地税》中说)也可能造成这样一种情况,以致“剩下的剩余产品不够用来鼓励那些通常以自己的节约来增加国家的资本的人的努力”。(第206页)“只有一种情况{第二十一章《积累对于利润和利息的影响》中说}可能引起利润率在食物价格低廉时随着资本的积累而下降,那就是维持劳动的基金比人口增加快得多,这时工资高,而利润率却低;但这种情况也只具有暂时的性质。如果每个人都不使用奢侈品而专心致志于积累,那末生产出来的必需品就会有一定数量无法立即被消费。这为数有限的几种商品无疑会发生普遍过剩现象,因而对这些商品的追加量不会有需求,使用追加资本也不会提供利润。如果人们停止消费,他们就会停止生产。”(第343页)}\end{quote}

李嘉图关于积累和关于利润率下降规律的思想就是这样。

\tchapternonum{[第十八章]李嘉图的其他方面。约翰·巴顿}

\tsectionnonum{[A.]总收入和纯收入}

纯收入,与总收入(它等于总产品或总产品价值)相对立,是重农学派最初用来表达剩余价值的一种形式。他们认为地租是剩余价值的唯一形式,因为他们把工业利润仅仅理解为一种工资;重农学派对这一问题的看法必然会找到支持者,这就是后来那些把利润说成是对劳动进行监督而得的工资,从而把利润掩盖起来的经济学家。

这样,纯收入实际上就是产品(或产品价值)超过它补偿预付资本即不变资本和可变资本的那一部分的余额。因此,纯收入只不过是由利润和地租构成,而地租本身又只是从利润中分割出来、落入一个不同于资本家阶级的阶级手中的一部分利润。

资本主义生产的直接目的不是生产商品,而是生产剩余价值或利润(在其发展的形式上);不是产品,而是剩余产品。从这一观点出发,劳动本身只有在为资本创造利润或剩余产品的情况下才是生产的。如果工人不创造这种东西,他的劳动就是非生产的。因此,所使用的生产劳动量只是在剩余劳动量由于它——或者比例于它——而增长的情况下,才会使资本感到兴趣。我们称为必要劳动时间的东西,只有在这样的情况下才是必要的。如果劳动不产生这种结果,它就是多余的,就要被制止。

资本主义生产的始终不变的目的,是用最小限度的预付资本生产最大限度的剩余价值或剩余产品;在这种结果不是靠工人的过度劳动取得的情况下,这是资本的这样一种趋势:力图用尽可能少的花费——节约人力和费用——来生产一定的产品,也就是说,资本有一种节约的趋势,这种趋势教人类节约地花费自己的力量,用最少的资金来达到生产的目的。

从这种理解来看,工人本身就象他们在资本主义生产中表现的那样,只是生产资料,而不是目的本身,也不是生产的目的。

纯收入不决定于总产品价值,而决定于总产品价值超过预付资本价值的余额,或者说,决定于与总产品相比的剩余产品量。尽管[733]产品价值减少,或者甚至产品总量也随同价值一起减少,只要这个余额增加,资本主义生产的目的就达到了。

李嘉图彻底地、无情地道破了这种趋势。由此引起庸俗的慈善家们对他的一片叫骂。

李嘉图在考察纯收入时又犯了一个错误,即把总产品归结为收入,也就是归结为工资、利润和地租,而把应得到补偿的不变资本撇开不谈。但是我们在这里不准备详细谈这一点。

李嘉图在第三十二章(《马尔萨斯先生的地租观点》)中写道:

\begin{quote}{“明确地区别总收入和纯收入是很重要的,因为一切赋税都必须从社会纯收入中支付。假定一个国家在一年中能够向市场提供的全部商品即全部谷物、原产品、工业品等等价值为2000万,为取得这个价值需要一定人数的劳动,而这些工人起码的生活必需品要花费1000万。那我就会说,这个社会的总收入是2000万,纯收入是1000万。根据这一假定决不能得出结论说,工人得到的劳动报酬只能是1000万;他们可能得到1200万、1400万或1500万,在这种情况下他们就会从纯收入中得到200万、400万或500万。余下的就会在土地所有者和资本家之间分配,但是全部纯收入不会超过1000万。假定这个社会纳税200万,它的纯收入就会减到800万。”(第512—513页)}\end{quote}

[而在第二十六章(《论总收入和纯收入》)中我们读到:]

\begin{quote}{“如果一个国家无论使用多少劳动量,它的纯地租和纯利润加在一起始终是那么多,那末,使用大量生产劳动对于该国又有什么好处呢?每一个国家的全部土地产品和劳动产品都要分成三部分:其中一部分是工资,一部分是利润,另一部分是地租。”}\end{quote}

{这是错误的,因为这里忘记了用于补偿生产中所使用的资本(工资除外)的那一部分。}

\begin{quote}{“赋税或积蓄只能出自后两部分;第一部分,如果它是适中的,就始终是必要的生产费用。”}\end{quote}

{李嘉图本人在给这句话加的注释中指出:

\begin{quote}{“这种说法可能过分,因为在工资的名义下归工人所得的部分通常高于绝对必要的生产费用。在这种情况下,工人得到国家纯产品的一部分,他可以把这一部分积蓄起来或者消费掉;或者可以捐献出来供国防之用。”}“对于一个拥有2万镑资本,每年获得利润2000镑的人来说,只要他的利润不低于2000镑,不管他的资本是雇100个工人还是雇1000个工人,不管生产的商品是卖1万镑还是卖2万镑,都是一样的。一个国家的实际利益不也是这样吗?只要这个国家的实际纯收入,它的地租和利润不变,这个国家的人口有1000万还是有1200万,都是无关紧要的。一国维持海陆军以及各种非生产劳动的能力必须同它的纯收入成比例,而不同它的总收入成比例。如果500万人能够生产1000万人所必需的食物和衣着,那末500万人的食物和衣着便是纯收入。假如生产同量的纯收入需要700万人,也就是说,要用700万人来生产足够1200万人用的食物和衣着,那对于国家又有什么好处呢?纯收入仍然是500万人的食物和衣着。使用更多的人既不能使我们的陆海军增加一名士兵,也不能使赋税多收一个基尼。”(第416—417页)}\end{quote}

为了更好地弄清李嘉图的观点,我们还要引证下面几段作补充:

\begin{quote}{“谷物价格相对低廉总会带来好处,也就是说,根据这种价格,现有产品的分配更可能增加维持劳动的基金,因为在利润的名义下归生产阶级的部分将更多,而在地租的名义下归非生产阶级的部分将减少。”(第317页)}\end{quote}

这里的生产阶级只是指产业资本家。

\begin{quote}{“地租是价值的创造,但不是财富的创造。如果谷物的价格由于一部分谷物生产困难而从每夸特4镑提高到5镑,那末100万夸特的价值就不是400万镑而是500万镑……整个社会将拥有更大的价值,从这种意义上说,地租是价值的创造。但是这种价值是名义上的,因为它丝毫不增加社会的财富,也就是说,不增加社会的必需品、舒适品和享乐品。我们所拥有的商品量同以前一样,而不是更多,谷物也仍然和以前一样是100万夸特;但是每夸特价格从4镑提高到5镑的结果,却使谷物和商品的一部分价值从原来的所有者手里转到土地所有者手里。因此,地租是价值的创造,但不是财富的创造;它丝毫不增加国家的资源。”(第485—486页)}\end{quote}

[734]假定由于[生产较为容易或者由于]谷物的进口,谷物价格下跌,使地租减少100万。李嘉图说,资本家的货币收入就会因此增加,接着又说:

\begin{quote}{“但是人们也许会说,资本家的收入不会增加;从土地所有者的地租中扣下来的100万将作为追加工资支付给工人。即使这样……社会状况也会得到改善,人们将能够比以前容易负担同样的税款。这只是证明了一件更合乎愿望的事,即由于新的分配而使状况得到改善的主要是另一个阶级,而且是社会上最重要的一个阶级。这个阶级所能得到的900万[即由必要生存资料决定的工资]以外的全部数额,构成国家纯收入的一部分,要把它花费掉,就一定会增加国家的收入、福利或力量。所以这笔纯收入你可以任意分配。你可以给一个阶级多一些,给另一个阶级少一些,但是你不会因此减少纯收入的总额;现在用同量劳动仍将生产出更多的商品,虽然这些商品的货币价值总额将会减少。但是,国家的纯货币收入,即交纳赋税和取得享乐品的基金,将比以前能够更好地维持现有居民的生活,为他们提供奢侈品和享乐品,使他们能够负担任何一定数量的赋税。”(第515—516页)}\end{quote}

\tsectionnonum{[B.]机器[李嘉图和巴顿论机器对工人阶级状况的影响问题]}

\tsubsectionnonum{[(1)李嘉图的观点]}

\tsubsubsectionnonum{[(a)李嘉图关于机器排挤部分工人的最初猜测]}

李嘉图在第一章(《论价值》)第五节中写道:

\begin{quote}{“假定……有一台机器在某一工业部门中使用,一年能做100个工人的工作,而且只能持续使用一年。再假定,这台机器值5000镑,每年支付给100个工人的工资也是5000镑。显然在这种情况下,购买机器还是雇用工人,对工厂主来说都一样。但是,假定劳动价值提高了,结果100个工人一年的工资为5500镑。显然,这时工厂主就不会再犹豫:用5000镑购买机器来为他完成同量工作对他是有利的。但是,机器的价格会不会也提高呢?它会不会由于劳动价值提高也值5500镑呢?如果制造机器时没有使用资本,也无须对机器制造业者支付利润,那末机器价格就会提高。例如,如果一台机器是工资均为50镑的100个工人劳动一年的产品,因而它的价格是5000镑,那末,在工资提高到55镑的情况下,机器价格就是5500镑。但这是不可能的。必须假定雇用的工人不到100人,否则机器就不可能卖5000镑,因为从这5000镑中必须给雇用工人的资本支付利润。因此,假定只雇用85个工人,每人工资50镑,即一年支出4250镑,机器售价中除支付工人工资以外的750镑,就是机器制造业者的资本的利润。当工资提高10%时,机器制造业者就不得不使用425镑追加资本,因此支出的资本就不是4250镑,而是4675镑;如果他仍然把机器卖5000镑,他用这笔资本就只能得到325镑利润。但是一切工厂主和资本家的情况都是一样:工资的提高对他们所有的人都有影响。因此,如果机器制造业者由于工资提高而提高机器价格,那就会有异常大量的资本被用来制造这种机器,直到机器价格只能提供普通利润率为止。因此我们可以看到,机器价格并不会因工资提高而提高。但是,那个在工资普遍提高时能够使用机器而又不增加自己商品的生产费用的工厂主,如果仍然可以按照过去的价格出卖自己的商品,他的情况就特别有利;不过我们已经看到,他将不得不降低自己商品的价格,否则资本就会流入他的生产部门,直到他的利润降到一般水平为止。因此,从采用机器中得到好处的是公众:生产这些不会说话的因素所花的劳动,总是比被它们排挤的劳动少得多,即使它们具有相同的货币价值。”(第38—40页)\endnote{在1817年出版的李嘉图的《原理》第一版中已有这段话。——第629页。}}\end{quote}

这一点完全正确。这也是对那些认为受机器排挤的工人能在机器制造业本身找到工作的人的回答;其实,这些人的看法不过是属于制造机器的工场还完全建立在分工的基础上并且还没有使用机器来生产机器的那个时代的看法。

假定一个工人的年工资是50镑,100个工人的工资就等于5000镑。如果这100个工人为机器所代替,而机器同样值5000镑,那末这台机器就必然是不到100个工人的劳动产品。因为机器中除包含有酬劳动外,还包含无酬劳动,这种无酬劳动恰好构成机器制造厂主的利润。如果值5000镑的机器是100个工人的劳动产品,那末它就只包含有酬劳动了。如果利润率为10%,那末在这5000镑中,预付资本就约占4545镑,利润约占455镑。如果一个工人的工资为50镑,那末4545镑就只代表90+(9/10)个工人。

[735]但是4545镑资本决不只代表可变资本(直接花费在工资上的资本)。它还代表机器制造业者所使用的固定资本的损耗和原料。因此,一台值5000镑、代替100个工人(这些工人的工资为5000镑)的机器不是90工人的产品,而是数量少得多的工人的产品。所以,只有在用于生产机器{至少是其中每年连利息一起加入产品即加入产品价值的那一部分}的工人人数(以年计算)比机器所代替的工人人数少得多的情况下,使用机器才是有利的。

工资的任何提高,都会使必须预付的可变资本增加,尽管产品价值——在它等于可变资本加剩余劳动的限度内——仍然不变(因为可变资本所推动的工人人数不变)。

\tsubsubsectionnonum{[(b)李嘉图论生产的改进对商品价值的影响。关于工资基金游离出来用于被解雇的工人的错误论点]}

李嘉图在第二十章(《价值和财富,它们的特性》)中指出,自然因素没有给商品价值增加什么,相反,它使商品价值减少。它恰恰因此使资本家唯一关心的剩余价值增加。

\begin{quote}{“同亚当·斯密的意见相反,萨伊先生在第四章中谈到了太阳、空气、气压等自然因素赋予商品的价值,这些自然因素有时代替人的劳动,有时在生产中给人以帮助。但是这些自然因素,尽管能够大大增加使用价值,却从来不会给商品增加萨伊先生所说的交换价值。只要我们借助于机器或自然科学知识使自然因素来完成以前由人完成的工作,这种工作的交换价值就会相应地降低。”(第335—336页)}\end{quote}

机器具有价值。自然因素本身没有什么价值。因此,它不可能给产品增加任何价值,而且相反,只要它能代替资本或劳动,不论是直接劳动还是积累劳动,它就会使产品的价值减少。只要自然科学教人以自然因素来代替人的劳动,而不用机器或者只用以前那些机器(例如利用蒸汽锅炉,利用许多化学过程等等,也许比以前还便宜),它就可以使资本家(以及社会)不费分文,而使商品绝对降价。

在以上引文之后,李嘉图接着说:

\begin{quote}{“如果原先用十个人推动磨面机,后来发现借用风力或水力可以节省这十个人的劳动,那末面粉(一部分是磨面机的工作产品)的价值就会立即按节约的劳动量相应地下降;并且社会会由于这十个人的劳动所能生产的商品而变得富些,同时预定用于维持这十个人的生活的基金并无任何减少。”(第336页)}\end{quote}

社会首先会由于面粉价格下降而变得富些。社会可以消费更多的面粉,也可以把以前预定用在面粉上的钱用在另一种商品上,这另一种商品或者是已经存在的,或者是因新的消费基金游离出来才出现的。

关于这部分以前用在面粉上、现在由于面粉价格下降而游离出来另作他用的收入,可以说,它原来由社会整个经济“预定用在”一定的物品上,现在则离开了这种“预定的用途”。这就好象积累了新资本一样。使用机器和自然因素就是用这种方法把资本游离出来,并使以前“潜在的需要”有可能得到满足。

相反,关于“预定用于维持”这十个由于新发现而失去了工作的人的“生活的基金”的说法是错误的。因为第一种基金是由新发现节约下来或创造出来的,它是社会以前用在面粉上、现在因面粉价格下降而节约下来的那一部分收入。而节约下来的第二种基金是磨坊主以前支付给十个现已解雇的工人的。这个“基金”,正如李嘉图所说的那样,的确并未因新发现和解雇十个工人而有任何减少。但是这个基金和这十个工人绝对没有任何自然的联系。他们可能成为贫民,饿死等等。只有一点是肯定无疑的,那就是本来应该接替这十个工人来磨面的下一代的十个人,现在必须到其他行业找工作,这样人口[和对劳动的需求相比]就相对地增加了(不管人口的平均增长如何),因为磨面机现在[不用人力]转动了,而这十个工人,假如没有这种发现,本来要去推动磨面机,现在则被雇去生产另一种商品了。所以,机器的发明和自然因素的利用使资本和人(工人)游离出来,创造了游离出来的资本,同时也创造了游离出来的人手(斯图亚特所说的“自由人手”\endnote{詹·斯图亚特《政治经济学原理研究》1770年都柏林版第一卷第396页。马克思在他的1857—1858年经济学手稿中引用了这一处(见卡·马克思《政治经济学批判大纲》)1939年莫斯科德文版第666页)。并参看本卷第1册第22页和马克思《资本论》第3卷第47章第1节。——第632页。}),这就有可能[736]创立新的生产领域,或者扩大旧的生产领域,扩大它们的生产规模。

磨坊主将用他的游离出来的资本建立新的磨坊,或者将这笔资本贷给别人,如果他自己不能将它作为资本花费掉的话。

但是在所有情况下这里根本没有什么“预定用于”十个被解雇工人的基金。我们还要回过头来谈\authornote{见本册第635—643页。——编者注}摆在我们面前的这个荒谬的前提,即如果采用机器(或者利用自然因素)不减少可以用作工资的生活资料的量(比如在农业上,用马代替人,或用畜牧业代替谷物业时,这种情况就部分地出现过),那末用上述方法游离出来的基金就必然要作为可变资本花掉(好象生活资料不可能出口,不可能用在非生产劳动者身上,或者在某些生产领域工资不可能提高等等),并且必然要恰恰用在被解雇的工人身上。机器经常不断地造成相对的人口过剩,造成工人后备军,这就大大增加了资本的权力。

在第335页的一个注中,李嘉图还反驳萨伊说:

\begin{quote}{“认为财富就在于有丰富的生活必需品、舒适品和享乐品的亚·斯密,虽然会承认机器和自然因素能大大增加一国的财富,但是不会承认它们能给这种财富的价值增加什么东西。”}\end{quote}

如果没有那些使地租能够形成的条件,自然因素确实不会给价值增加什么东西。但是机器总是会把它自己的价值加到已有的价值中去;第一,既然机器的存在便于[一部分]流动资本不断转化为固定资本,并使这一转化能在日益扩大的规模上进行,所以机器就不仅会增加财富,而且会增加由过去劳动加到年劳动产品上的价值;第二,因为机器的存在使人口有绝对增长的可能,从而使年劳动量也随之增长,所以机器通过这第二种方式也会增加年产品的价值。[736]

\tsubsubsectionnonum{[(c)李嘉图改正他对机器问题的看法表现了他在科学上的诚实。李嘉图对问题的新提法中仍保留了以前的错误前提]}

[736]第三十一章《论机器》。

李嘉图在他的著作第三版中新加的这一章,证明了他的诚实,这使他和庸俗经济学家有了本质的区别。

\begin{quote}{“我对这一问题{即“机器对社会各不同阶级的利益的影响”问题}的看法,在进一步思考后有了相当大的改变,所以我更加认为有责任加以说明。虽然我想不起在机器问题上我发表过什么须要收回的东西,但是我曾用其他方式{作为议员?}\endnote{李嘉图在这里很可能是指1819年12月16日他在英国下院就威廉·德·克雷斯皮尼的提案所作的发言;克雷斯皮尼提议成立一个委员会来研究罗伯特·欧文提出的消灭失业和改善工人阶级状况的计划。李嘉图在发言中断定“机器不会减少对劳动的需求”(皮·斯拉法编《大卫·李嘉图著作和通讯集》1952年剑桥版第5卷第30页)。——第633页。}支持过我现在认为是错误的学说,所以我认为有责任把我现在的看法及其理由提出来供读者研究。自从我开始注意政治经济学问题以来,我一直认为,在任何生产部门内采用机器,只要能节省劳动,就对大家都有好处,而唯一的不方便,在大多数情况下都是由于资本和劳动要从一个部门转移到另一个部门而引起的。”}\end{quote}

{这种“不方便”,如果象在现代生产中那样经常不断地发生,那末它对工人来说就够大的了。}

\begin{quote}{“在我看来,土地所有者所得的货币地租如果不变,他们用这种地租购买的某些商品价格的下跌将会使他们得到好处,而价格的下跌必然是采用机器的结果。我认为,资本家最后也会以完全相同的方式得到好处。不错,发明机器或首先使用机器的人会由于暂时获得很大的利润而得到额外的好处;但是随着机器的普遍采用,机器生产的商品的价格就会由于竞争而降到它的生产费用的水平,这时资本家所得到的货币利润就会和以前一样,他也只能[737]作为消费者分享一般的好处,因为他用同样的货币收入可以支配更多的舒适品和享乐品。我认为,工人阶级由于采用机器也会同样得到好处,因为工人用同样的货币工资可以购买更多的商品。同时我认为,工资不会缩减,因为资本家所能提出的对劳动的需求以及所能使用的劳动量仍然和以前一样,虽然他也许不得不使用这个劳动量来生产某种新的商品,或者至少是生产别的商品。如果由于机器的改良,使用同量劳动生产的袜子可以增加到四倍,而对袜子的需求只增加一倍,织袜业中的一些工人就必然会被解雇;但是由于雇用这些工人的资本仍然存在,而且由于资本的所有者把资本用在生产上是有利的,我认为这种资本将被用于生产其他某种对社会有用而社会对它也肯定有需求的商品……因此,由于我认为对劳动的需求仍然和以前一样,而工资又不会降低,我认为工人阶级将由于使用机器后商品普遍跌价而和其他阶级同样受益。这就是我原来的看法,涉及土地所有者和资本家的地方,我的看法依然不变;但我现在深信,用机器来代替人的劳动,对于工人阶级往往是非常有害的。”(第466—468页)}\end{quote}

首先必须指出,李嘉图是从下述错误的前提出发的:好象机器总是在资本主义生产方式已经存在的生产领域被采用。可是大家知道,机器织机最初是代替手工织工,珍妮机是代替手工纺工,而割草机、脱粒机、播种机也许是代替独立的小农等等。在这里不仅劳动者受到排挤,而且他的生产工具也不再是(李嘉图意义上的)资本。当机器在以前的仅以分工为基础的工场手工业中引起革命时,就出现了旧的资本的这种完全的或彻底的贬值。在这里,说“旧的资本”对劳动的需求仍然和以前一样,是荒谬的。

手工织工、手工纺工等使用的“资本”已经不是“仍然存在”了。

但是,为了使研究简便起见,我们假定只是在资本主义生产(工场手工业)已经占统治地位的领域,或者甚至在已经以机器生产为基础的工场中才采用机器{当然,我们这里就不谈在新的生产部门采用机器了},这样,问题就是提高机器的自动性,或者采用改良的机器,这种机器使得有可能或者解雇一部分现在雇用的工人,或者使用和以前一样多的工人,但是他们能提供更多的产品。当然,这后一种情况是最有利的。

为了减少混乱,必须把下面两种东西分开:(1)采用机器并解雇工人的资本家的基金;(2)社会的基金,即这个资本家的商品的消费者的基金。

第一点。关于采用机器的资本家,说他可以把和以前一样多的资本花费在工资上,这是错误的、荒谬的。(即使在他求助于借款的时候,如果不是就他本人而是就整个社会来说,这种说法同样是错误的。)他把自己资本的一部分转化为机器和别的固定资本,把另一部分转化为他以前并不需要的辅助材料,在我们假定他用较少的工人生产较多的商品,从而需要更大量的原料时,他还把比以前更多的一部分资本转化为原料。可变资本即花费在工资上的资本对不变资本的比例,在他的生产部门缩小了。即使这个资本家的企业扩大到新的生产水平,以致他又能给全部被解雇的工人或者甚至比以前更多的工人以工作,这个缩小了的比例也仍然有效(由于劳动生产力随着积累而发展,同不变资本相对来说,可变资本减少的幅度甚至还会增大)。{他的企业对劳动的需求将随同他的资本的积累一起增长,但与资本积累相比,程度要小得多,而他的资本就其绝对量来说已不再是以前那样的对劳动的需求的源泉了。由此产生的直接后果就是一部分工人被抛向街头。}

但是人们会说,对工人的需求会间接地保持不变,因为机器制造业对工人的需求会增加。可是李嘉图自己早已指出\authornote{见本册第628—629页。——编者注},机器所费的劳动量决没有它所代替的那样多。可能在制造机器的工场中工作日会暂时延长,[738]因此那里最初不会多雇一个工人。原料(比如说棉花)可能来自美国和中国,而对美国黑奴或中国苦力的需求是否增加,对于被抛向街头的英国工人来说是完全无关紧要的。即使假定原料在国内生产,那时在农业中将雇用更多的妇女和儿童,将使用更多的马匹等等,也许将生产更多的这种产品和更少的其他产品。但是对被解雇的产业工人的需求在这里不会产生,因为在这里,在农业中,造成经常的相对人口过剩的同一过程也在发生。

认为采用机器能使工厂主的资本在最初投入企业时就游离出来,这种说法一看就知道是难以置信的。采用机器只是使他的资本投入别的部门,根据这个前提,其直接后果就是解雇工人,把一部分可变资本转化为不变资本。

第二点。至于社会公众,那末由于用机器生产的商品跌价而游离出来的首先是他们的收入;资本只有在用机器生产的物品作为生产要素加入不变资本的限度内才会直接游离出来。{如果这种物品加入工人的一般消费,那末根据李嘉图自己的看法,这也一定会引起其他生产部门的实际工资\endnote{关于李嘉图的“实际工资”(《realwages》)的概念,见本册第456—457、459—460、474、482和497页。——第636页。}下降。}游离出来的一部分收入将消费在这种物品上,这或者是因为这种物品的跌价使得新的消费者阶层能够享用它(但是在这种场合用在这种物品上的收入不是游离出来的收入),或者是因为原先的消费者现在要消费更多的已减价的物品,例如现在消费四双线袜而不是一双。这样游离出来的收入的另一部分可以用来扩大那些采用机器的生产部门,或者建立生产别种商品的新部门,最后,或者用来扩大某个早已存在的生产部门。不管怎样,这样游离出来并再转化为资本的收入,未必能够吸收每年重新流入各生产部门而现在首先被旧生产部门排斥在外的那部分增加的人口。但是,也可能有一部分游离出来的收入将和外国的产品相交换或由非生产劳动者消费掉。无论如何,在游离出来的收入和从收入游离出来的工人之间没有任何必然的联系。

第三。然而作为李嘉图论据的基础的是下面的荒谬看法。

[正如我们所看到的,]采用机器的工厂主的资本不会游离出来。这种资本只是另作他用,也就是说,这时它不会象以前那样转化为现已被解雇的工人的工资。它的一部分从可变资本转化为不变资本。即使它有一部分游离出来,那也将被这样的生产领域所吸收,在这些生产领域中,被解雇的工人不可能有工作,那里最多只能给本来应该接替他们的人提供一个收容所。

而游离出来的收入(只要它游离出来而不被减价物品消费的增加所抵销,或者只要它不和来自国外的生活资料相交换),靠旧生产部门的扩大或新生产部门的建立,也只是为每年流来的、首先被采用机器的旧生产部门排斥在外的那部分增加的人口提供必要的机会(如果游离出来的收入真这样做的话!)。

但是,以隐蔽的形式构成李嘉图论据基础的那种荒谬看法恰恰在于:

现已被解雇的工人以前消费的那些生活资料依然存在,并且照旧存在于市场上。另一方面,这些人手也存在于市场上。因此,一方面存在着有可能成为可变资本的工人的生活资料(也就是支付手段),另一方面存在着失业工人。这样,也就有了用来推动这些失业工人的基金。因此,他们也就能够找到工作。

甚至象李嘉图这样的经济学家居然也会说出这种令人毛发悚然的荒唐言论!

照这样说来,在资产阶级社会内,凡是有工作能力并且愿意工作的人,当市场上、社会上有生活资料可以作为某种工作的报酬支付给他的时候,就都不会挨饿了。

首先必须指出,这些生活资料决不是作为资本而和被解雇的工人对立的。

我们假定,由于采用机器,有10万工人突然被抛向街头。那末,首先毫无疑问的是,[739]存在于市场上的、以前足够这些工人平均消费一年的农产品将照旧存在于市场上。如果对这些农产品没有需求,同时又不能将它们输出国外,那会产生什么结果呢?因为和需求相比,供给相对增加,所以这些产品就会跌价,即使被解雇的10万工人饿死,对产品的消费也会因跌价而增加。甚至食品用不着跌价,因为食品的进口可能减少或者出口可能增加。

李嘉图从幻想出发,以为资产阶级社会的整个结构非常精巧,比如说,如果有10个工人被解雇,那末他们那些现在已游离出来的生活资料必定还是被这10个工人这样或那样地消费掉,否则就根本不可能找到销路,——好象在这个社会的底层不存在忙于到处寻找工作的失业或半失业的群众,好象以生活资料形式存在的资本是一个固定的量。

如果谷物的市场价格由于需求的减少而下跌,那末以谷物形式存在的资本(在其货币表现上)就会减少,并且只要谷物不出口,它就会同社会的较少一部分货币收入相交换。对于以工业品形式存在的资本说来,就更是如此。在手工织工渐渐饿死的那些年代,英国的棉织品的生产和出口都大大增加了。同时(1838—1841年)食品价格上涨了。而这些织工既没有一件完整的、可以蔽体的衣服,也没有可以维持生命的食物。人为地不断制造出来的、只有在热病似的繁荣时期才能被吸收的过剩人口,是现代工业生产的必要条件之一。没有什么东西能阻止这样一些现象发生:一部分货币资本闲置不用,生活资料由于相对生产过剩而跌价,而被机器排挤的工人却活活饿死。

当然,游离出来的劳动和游离出来的一部分收入或资本,最终一定会在某一新的生产部门或在旧的生产部门扩大时找到出路,但这更多的是给那些本来应该接替被排挤的工人的人,而不是给被排挤的工人本身带来好处。收入直接花费在其中的、多少是非生产劳动部门的一些新部门不断产生。此外还有:固定资本(铁路等)的形成和由此产生的监督工作;奢侈品等的生产;使花费收入的物品种类越来越多的对外贸易。

李嘉图从他的荒谬观点出发,所以假定机器的采用只有在它减少总产品(从而减少总收入)的时候才对工人有害。当然,这种情况在大农业中,当那里使用马匹代替工人消费谷物或把谷物业变为养羊业等等的时候,是可能的。但把这种情况推广到本来意义的工业上,那就十分荒谬了,因为工业总产品的市场绝不限于国内市场的范围。(而且在一部分工人濒于饿死的时候,另一部分工人可能吃得好些,穿得好些;同样,非生产劳动者与介于工人和资本家之间的中间阶层也可能吃得好些,穿得好些。)

认为加入收入的物品的增加量(或一般量)本身就是为工人提供的基金,或者说,构成支付给工人的资本,这种说法本身就是错误的。这些物品的一部分为非生产劳动者或根本不劳动的人所消费。另一部分可能通过对外贸易从它用作工资的形式(从它粗糙的形式)转化为加入富人的收入或用作不变资本的生产要素的形式。最后,还有一部分由那些在习艺所或监狱中的被解雇的工人本身当作施舍物、赃物或他们的女儿卖淫的报酬来消费。

下面我将把李嘉图借以发挥谬论的一些论点概括一下。如他自己所说,他的这一谬论是从巴顿的著作中得到启发的,所以,在引用了李嘉图的著作之后,还必须简略地考察一下巴顿的著作。

[740]不言而喻,为了每年雇用一定数量的工人,每年必须生产一定数量的食品和其他必需品。在大农业、畜牧业等方面可能有以下这样的情况,即纯收入(利润和地租)增加,而总收入,用来维持工人生活的必需品的总量却减少。但是问题不在这里。加入消费的物品总量,或者用李嘉图的话说,加入总收入的物品总量,可能增加,而这一总量中转化为可变资本的那一部分却不会因此而增加。这个部分甚至可能减少。那时作为收入而由资本家、土地所有者、他们的奴仆、非生产阶级、国家、中间阶级(商人)等消费的将更多。

李嘉图(和巴顿)的论断的隐蔽基础是:他原来是从这样一个假设出发,即任何资本积累都是可变资本的增加,因而对劳动的需求将直接和资本的积累按同一比例增加。这是不正确的,因为随着资本的积累,资本有机构成会发生变化,资本的不变部分会比它的可变部分增长得更快。但是这并不妨碍收入在价值和数量方面不断增加。然而收入的这种增加并不会使总产品的相应增加部分花费在工资上。不直接靠自己劳动生活的阶级和阶层人数将会增加,他们的生活会比以前更好,而非生产劳动者人数同样会增加。

把一部分可变资本转化为机器的资本家(因而他在原料构成产品价值要素之一的所有生产领域中,同他所使用的劳动量相比,也一定会把他的资本的一个更大的份额用在原料上)的收入与我们研究的问题没有直接关系,所以我们撇开不谈。他的实际上已进入生产过程的资本以及他的收入,起初是以他自己生产的产品的形式,或者更确切地说,商品的形式(比如说,他是一个纺纱厂主,就是以棉纱的形式)存在的。在采用机器之后,他会把这些商品的一部分(或者把他出卖这些商品所得的货币的一部分)转化为机器、辅助材料和原料,而不会象以前那样,把这部分货币作为工资支付给工人,也就是说,间接地把它转化为工人的生活资料。除了农业上的少数例外,资本家将会比以前更多地生产这种商品,虽然被他解雇的工人已经不再象以前那样是他自己的产品的消费者,也就是购买者了。现在市场上有更多的这种商品,虽然这些商品已经不再为被抛向街头的工人而存在,或者说,已经不再象以前那样多地为他们而存在。因此,首先就这个资本家自己生产的产品来说,即使在这种产品加入工人消费的情况下,这种产品的一部分不再作为资本为工人而存在这件事与产品数量的增加也毫不矛盾。相反,总产品的一个更大的部分现在必须用来补偿转化为机器、辅助材料和原料的那部分不变资本,也就是说,总产品的一个更大的部分现在必须和数量比以前更多的这些再生产要素相交换。如果因采用机器而引起的商品量的增加与以前存在的对用这些机器生产的商品的需求(也就是被解雇工人方面的需求)的减少相矛盾,那末在大多数情况下就根本不可能采用机器了。所以,如果我们考察的是这样的资本,即它的一部分现在不是再转化为雇佣劳动,而是再转化为机器,那末,生产的商品量和这些商品中再转化为工资的份额之间首先就没有任何确定的关系或必然的联系。

至于整个社会,它的收入的补充,或者更确切地说,收入范围的扩大,首先是在那些由于采用机器而降价的物品方面发生的。这种收入可能仍然作为收入来花费,如果其中有相当大一部分转化为资本,那末除了人为地造成的人口过剩外,也总是已经存在着自然增长的人口,他们能把转化为可变资本的那部分收入吸收掉。

因此,初看起来就只剩下这样一点:所有其他物品的生产,尤其是在生产加入工人消费的物品的那些领域,尽管解雇了比如说100个工人,还是会按照以前的规模进行;毫无疑问,在这些工人被解雇时就是这种情况。因此,就被解雇的工人对上述物品有过需求来说,这种需求减少了,虽然供给仍旧不变。由此可见,如果需求的这种减少得不到弥补,相应的商品的价格就会下降(或者价格不下降,而是有更多的商品在市场上作为存货保留到下一年)。如果这种商品同时又不是出口货,如果对它的需求仍然低于以前的水平,那末这种物品的再生产就要减少,但是[741]用于这一领域的资本却不一定要减少。可能将生产更多的肉类或者更多的经济作物或者高级食品和更少的小麦,或者生产更多的饲养马匹等等用的燕麦,或者生产更少的绒布短上衣和更多的资产阶级用的常礼服等等。但是,如果由于例如棉织品减价,在业工人可以在食物等方面多花费一点,那就根本不会有产生上述任何后果的必然性。可能生产和以前一样多的,甚至更多的商品(其中包括加入工人消费的商品),尽管现在转化为可变资本即花费在工资上的是较少的资本,是总产品中一个更小的部分。

这里也不会产生这种情况,即对这些商品的生产者来说,他们的资本有一部分会游离出来。在最坏的情况下,对他们的商品的需求将减少,因而在他们的商品跌价的时候他们的资本的再生产将遇到困难。因此,他们自己的收入会立即减少,正如每当商品跌价时都会发生这种情况一样。但是不能说,他们的商品中的某一部分以前是作为资本和被解雇的工人对立的,现在则和这些工人一起“游离出来”。作为资本和工人对立的是现在用机器生产的那一部分商品;这部分商品以货币的形式流到他们手里,被他们用来和别的商品(生活资料)进行交换,这些别的商品不是作为资本和工人发生关系,而是仅仅作为商品和他们的货币对立的。因此,这是一种完全不同的关系。工人用自己的工资购买租地农场主或其他某一资本家的商品,这个租地农场主或资本家不是以资本家的身分和工人相对立,也不是把他们当作工人在生产中使用。现在他们不过不再是他的购买者了,如果没有其他情况来弥补的话,这就可能使他的资本暂时贬值,但是不会有任何资本游离出来用于雇用被解雇的工人。曾经使用他们来进行生产的那笔资本“仍然存在”,但已经不是以资本转化为工资(或者只是间接地在更小的程度上转化为工资)的形式存在了。

不然的话,任何因遭遇某种不幸而挣不到钱的人,都会因此而使一笔能给他自己提供工作的资本游离出来。

\tsubsubsectionnonum{[(d)李嘉图对采用机器给工人阶级带来某些后果的正确判断。在李嘉图对问题的说明中存在的辩护论观点]}

李嘉图认为总收入就是补偿工资和剩余价值(利润和地租)的那一部分产品;他认为纯收入就是剩余产品,剩余价值。李嘉图在这里就象在他自己的全部经济理论中一样,忘记了总产品中有一部分应该补偿机器和原料的价值,简单地说,就是补偿不变资本的价值。

\centerbox{※     ※     ※}

下面列举的李嘉图的一些论断之所以引人注意,部分是由于一些顺便提及的意见,部分是因为它们经过适当的修改,对于大农业,尤其对于养羊业,在实践上是重要的。所以这里又显露出了资本主义生产的界限。不但资本主义生产的决定性的目的不是为生产者(工人)而生产,而且它的唯一的目的就是纯收入(利润和地租),即使这个目的是靠减少产量,减少商品的生产量来达到。

\begin{quote}{“我的错误之所以产生,是由于假定每当社会的纯收入增加时,其总收入也一定增加。但是现在我有一切理由确信,土地所有者和资本家从中取得收入的那种基金可能增加,同时另一种基金即工人阶级主要依靠的那种基金却可能减少。因此,如果我没有错的话,那就可以得出结论说,使国家的纯收入增加的原因,同时也可能造成人口过剩,使工人状况恶化。”(第469页)}\end{quote}

这里首先要指出:李嘉图在这里承认,使资本家和土地所有者财富增加的那些原因“……可能造成人口过剩”,所以人口过剩,或者说,过剩的人口,在这里表现为致富过程本身和作为它的先决条件的生产力发展的结果。

至于说到资本家和土地所有者从中取得收入的基金,以及另一方面工人从中取得收入的基金,那末总产品首先就是这个总的基金。加入资本家和土地所有者消费的相当大一部分产品不会加入工人的消费。可是另一方面,几乎所有加入工人消费的产品,——实际上是所有产品,不过是多少程度不同罢了,——也都加入土地所有者和资本家的消费,其中也包括他们的奴仆、食客、猫狗的消费。不能认为,在这里似乎是彼此孤立地存在着两种性质不同的具有固定的量的基金。重要的是每一方从这个总的基金中获得多大的份额。资本主义生产的目的在于用一定量的财富得到尽可能多的剩余产品或剩余价值。达到这一目的的方法是:使不变资本相对地比可变资本增加得快些,也就是说,以尽量少的可变资本来推动尽量多的[742]不变资本。因此,从比李嘉图在这里讲的更普遍得多的意义上来说,同一个原因,通过工人从中取得收入的基金的减少,会促使资本家和土地所有者从中取得收入的基金增加。

由此不应得出结论说,工人从中取得收入的基金会绝对地减少。这种基金同他们所生产的总产品相比,只是相对地减少。而这对于决定他们从他们自己创造的财富中获得多大的份额来说,是唯一重要的。

\begin{quote}{“假定有一个资本家使用一笔价值20000镑的资本,他是租地农场主,同时也是生产必需品的工厂主。再假定这笔资本中有7000镑投在固定资本上,即投在建筑物、劳动工具等等上,其余的13000镑作为流动资本用来维持劳动。再假定利润为10%,因而这个资本家的资本每年都能保持原有的效率,并提供2000镑的利润。这个资本家每年开始营业时拥有价值13000镑的食品和其他必需品。在一年内,他按照这个货币额把这些食品和必需品全部卖给自己的工人;在同一时期内,他又把同额货币作为工资支付给工人。年终,工人补偿给他价值15000镑的食品和其他必需品,其中2000镑他自己消费或由他按自己最喜欢和最乐意的方式处理。”}\end{quote}

{在这里剩余价值的性质就表现得很明显。这段话在李嘉图的著作第469—470页上。}

\begin{quote}{“就这些产品而言,这一年的总产品是15000镑,纯产品是2000镑。现在假定下一年资本家用一半工人制造机器,另一半照旧生产食品和其他必需品。在这一年内他会照常付出工资13000镑,并将同一金额的食品和其他必需品卖给他的工人。但是下一年的情况又会怎样呢?制造机器时,食品和其他必需品的产量只有平常的一半,它们的价值也仅仅等于以前的一半。机器值7500镑,食品和其他必需品也值7500镑,所以这个资本家的资本还是和以前一样大;因为在这两个价值以外,他还有价值7000镑的固定资本,合计仍然是20000镑资本和2000镑利润。他把供他个人花费的后一金额扣除以后,剩下来继续经营业务的流动资本就只有5500镑了;所以他用来维持劳动的资金就从13000镑减少到了5500镑,因此,以前用7500镑雇用的全部劳动现在就会过剩。”}\end{quote}

{可是,如果现在用价值7500镑的机器[用5500镑的可变资本]生产的产品和以前用13000镑可变资本生产的产品一样多,这种情况也会发生。假定机器的损耗一年是十分之一,即750镑,那末以前是15000镑的产品价值,现在就会等于8250镑(原来的7000镑固定资本的损耗不算在内,对这笔资本的补偿问题李嘉图根本没有提及)。在这8250镑中有2000镑利润,就象以前15000镑中有2000镑利润一样。在租地农场主自己把食品和其他必需品作为收入来消费的情况下,他会得到好处。在他由于必需品跌价而能降低他所雇用的工人工资的情况下,他又会得到好处,他的一部分可变资本就会游离出来。这也就是在一定程度上能够用来雇用新的劳动的那部分可变资本,但这只是因为尚未被解雇的工人的实际工资降低了。所以,一小部分被解雇的工人要靠牺牲在业工人的利益才能重新得到工作。但是产品的数量和以前完全一样,这种情况本身对被解雇的工人毫无益处。如果工资保持不变,可变资本丝毫也不会游离出来。产品价值——8250镑——并不会由于它所代表的食品和其他必需品与以前的15000镑所代表的一样多而有所提高。租地农场主一方面为了补偿机器的损耗,另一方面为了补偿他的可变资本,必须把他的产品卖8250镑。如果食品和其他必需品的这种跌价并不引起工资的普遍下降或者并不引起加入不变资本再生产的组成部分的价格下降,那末社会收入就会随着它在食品和其他必需品上的花费而增加。一部分非生产劳动者和生产劳动者等的生活就会过得好些。如此而已。(这部分人甚至会有积蓄,但这总是将来的事情。)被解雇的工人照旧没饭吃,虽然维持他们生活的物质的可能性还是和以前完全一样地存在着。在再生产中还是使用和以前一样的资本。不过以前作为资本而存在的一部分产品(其价值已降低),现在则作为收入而存在。}

\begin{quote}{“当然,资本家现在所能雇用的已经减少的劳动量,借助于机器,在扣除机器维修费以后,必然会生产出等于7500镑的价值,必然能补偿流动资本,并且带来全部资本的利润2000镑。但是,如果做到这一点,[743]如果纯收入不减少,那末对资本家来说,总收入的价值究竟是3000镑,10000镑,还是15000镑,难道不都是一样吗?”}\end{quote}

{这绝对正确。总收入对资本完全无关紧要。它唯一关心的是纯收入。}

\begin{quote}{“因此,在这种情况下,虽然纯产品的价值不会减少,虽然纯产品对商品的购买力可能大大增长,但是总产品的价值将由15000镑降为7500镑。因为维持人口和雇用劳动的能力总是取决于一个国家的总产品,而不是取决于它的纯产品}\end{quote}

{亚·斯密对总产品的偏重就是由此而来的,李嘉图反驳了这一点。见第二十六章(《论总收入和纯收入》),李嘉图在这一章一开头就说:

\begin{quote}{“亚当·斯密经常夸大一个国家从大量总收入中得到的利益,而不是从大量纯收入中得到的利益。”(同上,第415页)},所以对劳动的需求就必然会减少,人口将会过剩,工人阶级的状况将会陷于穷困。”}\end{quote}

{因此,劳动将会过剩,因为对劳动的需求减少了,而需求的减少是由于劳动生产力的发展。这段话在李嘉图的著作第471页。}

\begin{quote}{“不过因为积蓄一部分收入并把它转化为资本的能力必然取决于纯收入满足资本家需要的能力,所以采用机器使商品价格降低后,只要资本家的需要不变{但他的需要会增加},他就可能增加自己的积蓄,从而使收入更容易转化为资本。”}\end{quote}

{照这种说法,一部分资本——不是就它的价值来说,而是就使用价值来说,从构成这部分资本的物质要素来说——首先要转化为收入,然后才能有一部分收入再转化为资本。例如,在可变资本花费13000镑时,总数为7500镑的一部分产品加入了租地农场主所雇用的工人的消费,而且这部分产品是租地农场主的资本的一部分。根据我们的假定,由于采用机器,生产的产品将和以前一样多,但它的价值只有8250镑,而不是以前的15000镑。这种降价的产品现在有较大一部分既加入租地农场主的收入,也加入食品和其他必需品的购买者的收入。他们现在把这样一部分产品作为收入来消费了,这部分产品以前固然也是由租地农场主的工人(现已被解雇)作为收入来消费,但是他们的雇主却是把它作为资本在生产上来消费的。以前作为资本来消费的一部分产品,现在作为收入来消费,这就造成收入的增加,由于收入的这种增加,就形成新资本,收入就再转化为资本。}

\begin{quote}{“但是,每当资本增加,资本家雇用的工人也就增多}\end{quote}

{无论如何也没有他的资本增加的总量那么多。租地农场主也许会买更多的马匹或者鸟粪或者新工具},

\begin{quote}{因此,原先失业的人中有一部分后来就可以就业;如果采用机器以后生产增加很多,以致以纯产品形式提供的食品和其他必需品的数量和以前以总产品形式存在的数量相等,那就有可能象以前那样给全体人口提供工作,因而就不一定{但是可能和也许!}会有过剩的人口出现了。”(第469—472页)}\end{quote}

这就是说,在最后几行,李嘉图直接说出了我在上面已指出的东西。为了使收入按上述途径转化为资本,资本首先要转化为收入。或者如李嘉图所说的,先要靠减少总产品来增加纯产品,然后才有可能把一部分纯产品再转化为总产品。产品就是产品。“纯”和“总”的名称在这里不会引起任何变化(虽然二者的对立也可能意味着,尽管产品总量即总产品减少,超过支出的余额即纯产品也会增加)。产品之成为纯产品或成为总产品,要看它在生产过程中所采取的特定形式而定。

\begin{quote}{“我想要证明的,只是机器的发明和应用可能伴随着总产品的减少;每当这种情形出现时,工人阶级就要受损害,因为其中一部分人将会失业,人口同雇用他们的基金相比将会过剩。”(第472页)}\end{quote}

但是,即使在总产品数量不变或者增加的时候,这种情况也可能发生,而且在大多数场合[744]一定会发生,不同之处仅仅在于总产品的一部分以前用作可变资本,现在则作为收入来消费。

在这后面(第472—474页)李嘉图荒谬地举了一个毛织厂主因采用机器而使生产减少的例子,在这里不必谈它。

\begin{quote}{“如果这种看法是正确的,那末从中就应得出如下结论:(1)机器的发明和有效使用总会增加一个国家的纯产品,虽然它可能不会而且在一个短时期后肯定不会增加这种纯产品的价值。”}\end{quote}

只要它减少劳动的价值,它就总会增加纯产品的价值。

\begin{quote}{“(2)一个国家的纯产品的增加和总产品的减少是可以并存的。采用机器虽然可能而且往往必然会减少总产品的数量与价值,但只要能增加纯产品,使用机器的动机就永远足以保证机器的使用。(3)工人阶级认为使用机器往往会损害他们的利益,这种看法不是以成见和误解为根据,而是符合政治经济学的正确原则的。(4)如果生产资料由于采用机器而得到改良,使一个国家的纯产品大大增加,以致总产品(这里我始终是指商品的数量,而不是指它们的价值)不会减少,那末所有阶级的状况便都会得到改善。土地所有者和资本家会得到好处,但不是由于地租和利润的增加,而是由于用同量的地租和利润可以购买价值大大下降的商品}\end{quote}

{这个论点和李嘉图的整个学说是矛盾的,按照他的学说,必需品的减价,从而工资的下降,会提高利润,而使人们花费较少劳动却能在同一土地上得到更多产品的机器,在李嘉图看来,必然会减少地租};

\begin{quote}{同时工人阶级的状况也会有相当大的改善,第一,由于对家仆的需求增加}\end{quote}

{采用机器的一个真正美妙的结果,就是把工人阶级的相当一部分,妇女和男人,变成了仆人};

\begin{quote}{第二,由于如此丰富的纯产品刺激人们将收入储蓄起来;第三,由于工人用工资购买的一切消费品价格低廉{这种低廉的价格会使他们的工资下降}。”(同上,第474—475页)}\end{quote}

在机器问题上的资产阶级辩护论解释并不否认:

(1)机器经常不断地——时而在这里,时而在那里——使一部分人口过剩,把一部分工人抛向街头。人口过剩(从而有时在这里,有时在那里,引起某些生产领域的工资下降)不是因为人口比生活资料增长得快,而是因为采用机器引起的生活资料数量的迅速增加,使人们能够采用更多的机器,从而减少对劳动的直接需求。人口过剩的产生,不是因为社会基金减少了,而是因为其中用于工资的部分由于这种基金的增长而相对地减少了。

(2)这种辩护论更不否认从事机器劳动的工人本身的被奴役,以及受机器排挤而濒于死亡的手工劳动者或手工业者的困苦。

这种辩护论断言——部分地也是正确的——[第一],由于机器(一般由于劳动生产力的发展),纯收入(利润和地租)会增加,资产者会比以前需要更多的家仆;如果说他以前要从自己的产品中拿出较大部分花费在生产劳动上,那末现在他就可以把较大部分花费在非生产劳动上,结果仆人和其他靠非生产阶级的钱过活的劳动者就会增加。不消说,美妙的前景就是日益增多地把一部分工人变为仆人。同样使工人们聊以自慰的是,由于纯产品的增加,为非生产劳动者开辟了更多的活动领域,这些非生产劳动者要消费生产工人的产品,他们在剥削生产工人的利害关系上也多少和那些直接从事剥削的阶级一致起来了。

第二,资产阶级的辩护论断言:由于新的生产条件需要的活劳动比过去的劳动少而产生的对积累的刺激,受排挤的赤贫化的工人,或者,至少是本来应该接替他们的那部分增加的人口,[745]也会被吸收到生产中来;这或者是由于制造机器的企业本身的生产扩大,或者是由于与机器制造业有关的、因制造机器才成为必要并产生出来的那些生产部门的生产扩大,或者是由于用新资本开辟的并能满足新需要的雇用劳动的新部门的生产扩大。第二个美妙的前景是:工人阶级必须忍受一切“暂时的不方便”——失业以及劳动和资本从一个生产领域转到另一个生产领域,——但是雇佣劳动决不会因而终止;相反,雇佣劳动还会以不断扩大的规模再生产出来,它将绝对地增加,虽然和雇用它的不断增长的总资本相比会相对减少。

第三,资产阶级的辩护论断言,由于机器,消费将变得更为讲究。满足直接生活需要的物品价格低廉,使奢侈品的生产范围能够扩大。这样,在工人面前又开辟了第三个美妙的前景:为了取得他们所必需的生活资料,也就是取得以前那种数量的生活资料,同一数量的工人必须能够使上层阶级扩大他们的享受范围,使享受更讲究,更多样化,从而加深工人和高踞于他们之上的人们之间的经济、社会和政治的鸿沟。这就是劳动生产力的发展将给工人带来的十分美妙的前景和非常令人羡慕的结果。

接着李嘉图还指出,劳动阶级的利益要求

\begin{quote}{“把用在购买奢侈品方面的收入尽量转用来维持家仆”。因为,不管我购买家具还是维持家仆,我都会对一定量[按价值来说]的商品提出需求,在一种场合推动的生产劳动和在另一种场合推动的差不多相等;但是在后一场合我在“原有对工人的需求”之外增加了新的需求,“而需求的这种增加,只是因为我选择了第二种花费我的收入的方式”。(第475—476页)}\end{quote}

当一个国家维持大量海军和陆军时,也会得到同样的结果:

\begin{quote}{“不论它〈收入〉用什么方式来花费,生产上使用的劳动量总是相同的;因为生产士兵的食物和衣服同生产更奢侈的商品所需要的劳动量是相同的。但在战时还需要更多的人去当兵,所以靠收入而不靠国家的资本来维持的战争有利于人口的增长。”(第477页)}\end{quote}

[接着李嘉图写道:]

\begin{quote}{“还有一种情况应当注意:当一个国家的纯收入乃至总收入的数量都增加时,对劳动的需求却可能减少,用马的劳动代替人的劳动时就是这样。如果我在自己的农场里本来雇用100个工人,后来发现,把原来用于50个人的食物用来养马,在支付买马资本的利息后还可以得到更多的原产品,也就是说用马来代替人对我是有利的,那我也就会这样做。但这对工人将是不利的,除非我的收入增加到足以使我能同时使用人和马,否则人口显然就会过剩,工人的状况就会普遍恶化。很明显,在任何情况下工人都不能在农业中找到工作{为什么不能?如果扩大耕地面积呢?};不过,如果土地的产品由于以马代替人而增加了,被解雇的工人也许能在工业中找到工作或者去当家仆。”(第477—478页)}\end{quote}

有两种不断交错的趋势:[第一,]使用尽量少的劳动来生产同样多的或更多的商品,同样多的或更多的纯产品,剩余价值,纯收入;第二,使用尽量多的工人(虽然和他们生产的商品数量相比也是尽量少的),因为——在生产力发展的一定阶段上——使用的劳动量增加,剩余价值和剩余产品的量也会增加。一种趋势把工人抛向街头,造成过剩的人口;另一种趋势又把他们吸收掉,并绝对地扩大雇佣劳动奴隶制。于是工人被他的命运东抛西扔,但始终还是摆脱不了这种命运。所以工人完全有理由把他自己的劳动生产力的发展看作某种与他敌对的东西;另一方面,资本家则把他当作一个必须不断从生产中离开的要素来对待。

李嘉图在这一章中要努力解决的正是这些矛盾。他忘记指出:[746]介于工人为一方和资本家、土地所有者为另一方之间的中间阶级不断增加,中间阶级的大部分在越来越大的范围内直接依靠收入过活,成了作为社会基础的工人身上的沉重负担,同时也增加了上流社会的社会安全和力量。

资产者把采用机器使雇佣奴隶制永久化这件事用来为机器“辩护”。

\begin{quote}{“在前面我还指出过:机器改良的结果,以商品计算的纯收入总会增加,这种增加会导致新的积蓄和积累。我们必须记住,这种积蓄是逐年进行的,不久就会创造出一笔基金,其数额远远大于原来因发明机器而损失的总收入。这时对劳动的需求将和以前一样大,人民的生活状况也将由于积蓄的增加而得到进一步改善,增加了的纯收入又使他们有可能增加积蓄。”(第480页)}\end{quote}

先是总收入有损失,纯收入有增加。然后一部分增加了的纯收入再转化为资本,从而再转化为总收入。工人就这样被迫不断增大资本的权力,以便在极大的动乱之后,可以被允许在更大的规模上重复这同一过程。

\begin{quote}{“资本和人口每有增加,食品价格一般总会上涨,因为生产这些东西更加困难了。”(第478—479页)}\end{quote}

紧接着李嘉图就说道:

\begin{quote}{“食品价格上涨会使工资提高,而工资的每次提高都会有一种趋势,就是把积蓄起来的资本比以前更多地用于使用机器方面。机器同劳动处于不断的竞争中,机器往往只是在劳动价格上涨时才能被应用。”(第479页)}\end{quote}

因此机器是制止劳动价格上涨的手段。

\begin{quote}{“为了说明原理,我曾假定改良的机器是突然发明出来并得到广泛应用的。但实际上这种发明是逐渐完成的,其作用与其说是使资本从现在的用途上转移,倒不如说是决定被积蓄和积累的资本的用途。”(第478页)}\end{quote}

实际情况是:由于资本的新积累而得以进入使用劳动的新领域的,主要不是被排挤的劳动,而是新提供的劳动,即本来应该接替被排挤的工人的那部分增加的人口。

\begin{quote}{“在人们容易取得食物的美国和其他许多国家里,使用机器的诱惑力远远不象在英国那样大{除了美国,在任何地方都没有那样大规模地使用机器,甚至在家庭日常生活中也使用机器},在英国,食品价格很高,生产食品要耗费很多劳动。”}\end{quote}

{美国使用机器比经常有过剩人口的英国相对说来要多得多,正是美国,表明机器的使用很少取决于食品的价格,虽然它的使用可能象在美国那样取决于工人的相对不足,在美国,分布在辽阔的国土上的人口相对说来是稀少的。例如我们在1862年9月19日《旗帜报》\endnote{《旗帜报》是英国保守派的日报,1827年在伦敦创刊。——第655页。}刊载的一篇关于博览会的文章\endnote{指1862年9月19日《旗帜报》第5—6版上刊登的《博览会上的美国》一文(未具名)。关于1862年的世界博览会,见注112。——第655页。}中就读到:

\begin{quote}{“人是制造机器的动物……如果把美国人当作人类的代表,那末这个定义……是无可非议的。能用机器做的事就不用手去做,这已成为美国人的一个主要观点。从摇摇篮到做棺材,从挤牛奶到伐木,从缝钮扣到选举总统的投票,他们几乎全都用机器。他们发明了一种机器,可用来节省咀嚼食物的劳动……劳动力的异常缺乏和由此而来的很高的劳动价值{虽然食品的价值很低},以及某种天生的灵敏激发了他们的发明精神……美国生产的机器一般说比英国造的价格便宜……总的来说,机器与其说是能做以前不能做的事的一种发明,不如说是节省劳动的装置。{那末汽船呢?}……在博览会的美国馆中展出有埃默里的轧棉机。从美国开始植棉以来很多年内,棉花的收获量并不太大,这不仅仅是因为对棉花的需求不大,而且是因为手工清棉的困难使种植棉花变得无利可图。但是当伊莱·维特尼发明了清棉的锯齿[747]轧棉机时,植棉面积立即增加,而且至今几乎还在按算术级数增加。的确,说维特尼创造了商业性的植棉业,这并不夸大。维特尼轧棉机经过或多或少重大而有效的改进一直还在使用,原始的维特尼轧棉机在现在的这些改进和补充发明以前,丝毫不逊于大多数妄图把它排挤出去的机器。现在带有沃耳巴尼(纽约州)埃默里商标的机器,无疑完全取代了为它奠定基础的维特尼机。这种机器同样简单,生产效率却更高;它轧出的棉花不仅更干净,而且一层层象铺好的棉絮一样,因此这一层层的棉花从机器出来后立即可以压紧打包……在博览会的美国馆中几乎全是机器……挤奶机……从一个滑轮到另一滑轮的传动皮带装置……大麻梳纺机,一下子就把包扎成捆的大麻直接打成麻绳……纸袋机,它可以裁纸、粘叠,一分钟做300个纸袋……霍斯的洗衣挤干机,它用两个橡皮滚筒把衣服上的水挤出,衣服就差不多干了,用这种机器节省时间,而且不损伤织物……装订机……制鞋机。大家都知道,在英国早已用机器制作鞋面,可是这里展出的还有绱鞋的机器,切鞋底的机器,还有做鞋后跟的机器……功率强大的碎石机,结构很灵巧,无疑会广泛使用于铺路和捣碎矿石……奥本(纽约州)的华德先生的航海信号系统……收割机和割草机是美国的发明,它们越来越获得英国方面的好评。麦考密克的机器是最好的……汉斯布劳先生的压力泵,曾荣获加利福尼亚奖章,它的结构简单和生产效率之高在博览会上是最突出的……与世界上任何水泵相比,它能以同样的力量抽更多的水……缝纫机……”}“使劳动价值提高的原因并不会提高机器的价值,所以资本每有增加,就会有越来越大的部分用在机器方面。对劳动的需求将随着资本的增加而继续增加,但不会按同一比例增加,其增加率一定是递减的。”(第479页)}\end{quote}

在最后一句话中李嘉图表述了正确的资本增加规律,虽然他用以论证这一规律的理由是非常片面的。李嘉图在他的著作的这个地方还加了一个注,表明他在这里是追随巴顿,所以我们还要简略地谈一谈巴顿的著作。

预先还要作一点说明。在这之前,李嘉图在谈到收入是花费在家仆上还是花费在奢侈品上这一问题时说:

\begin{quote}{“在这两种情况下,纯收入是相同的,总收入也是相同的,但是纯收入实现在不同的商品上。”(第476页)}\end{quote}

同样,总产品的价值也可能是一样的,但是,它是否会“实现”——工人对此是非常敏感的——“在不同的商品上”,那要看它应补偿更多的可变资本还是应补偿更多的不变资本。

\tsubsectionnonum{[(2)巴顿的见解]}

\tsubsubsectionnonum{[(a)巴顿关于资本积累过程中对劳动的需求相对减少的论点。巴顿和李嘉图不懂得这种现象同资本统治劳动有内在的联系]}

巴顿的著作叫作:

约翰·巴顿《论影响社会上劳动阶级状况的环境》1817年伦敦版。

我们先引一下巴顿著作中为数不多的理论观点:

\begin{quote}{“对劳动的需求取决于流动资本的增加,而不是取决于固定资本的增加。如果这两种资本的比例在任何时候和任何国家确实都是一样的话,那末由此的确可以得出结论说,就业工人的人数同国家的财富成比例。但是这种假定一点也不现实。随着技术的进步和文明的传播,固定资本与流动资本相比越来越大。英国生产一匹凡而纱所使用的固定资本额至少等于印度生产同样一匹凡而纱所使用的固定资本额的一百倍,也许是一千倍。而[748]流动资本的份额则小到百分之一或千分之一。我们很容易想象得到,在一定的情况下,一个勤劳的民族可能把一年的全部积蓄都加到固定资本上去,在这种情况下,这些积蓄也不会使对劳动的需求有任何增长。”(巴顿,同上第16—17页)}\end{quote}

{李嘉图在他的著作第480页的注中,就巴顿的这些话指出:

\begin{quote}{“我认为,在任何情况下资本增加而对劳动的需求不随之增加是难于想象的。至多只能说,对劳动的需求的增加率将愈来愈小。在我看来,巴顿先生在上述著作中关于固定资本的扩大对工人阶级状况的某些影响所持的看法是正确的。他的著作包含许多有价值的资料。”}}\end{quote}

对上面引的巴顿的论点,必须再加上下面这一段话:

\begin{quote}{“固定资本一经形成,就不再引起对劳动的需求了{这不对,因为固定资本必须再生产,虽然这种再生产只能经过一定的间隔时间,而且只能逐渐地进行},但是在固定资本形成的时候,它所使用的人手,和同额流动资本或收入所使用的人手一样多。”(第56页)}\end{quote}

还有:

\begin{quote}{“对劳动的需求,完全取决于收入和流动资本的总额。”(第34—35页)}\end{quote}

毫无疑问,巴顿有很大的功劳。

亚·斯密认为,对劳动的需求的增加同资本的积累成正比。马尔萨斯从资本积累不象人口增加那样快(资本不是以递增的规模再生产)出发,得出人口过剩的结论。巴顿第一次指出,资本各有机组成部分并不随着积累和生产力的发展而以同样程度增加;相反,在生产力增长的过程中,转化为工资的那部分资本同(巴顿称为固定资本的)另一部分相比会相对减少,而后者同自己的量相比只是稍微改变对劳动的需求。因此,他第一次提出这样一个重要论点:“就业工人的人数”不是“同国家的财富成比例”,工业不发达国家的就业工人的人数比工业发达的国家相对地多。

李嘉图在他的《原理》的前两版中,在这一点上还完全跟着斯密走,但在第三版第三十一章《论机器》中,却采用了巴顿的修正,而且采用了巴顿的片面的说法。李嘉图向前发展的唯一的一点——这一点有重要意义——就是:他不仅象巴顿那样提出,对劳动的需求的增加不是同机器的发展成比例,而且还断言,机器本身“造成人口过剩”\authornote{见本册第644、646、648、649和653页。——编者注},即造成过剩的人口。只不过李嘉图错误地把采用机器的这种结果局限于纯产品靠减少总产品而增加的场合——这种场合只有在农业中才会遇到,而他却认为工业中也会出现。但这也就简要地反驳了整个荒谬的人口论,尤其是反驳了庸俗经济学家关于工人必须努力把自己的繁殖限制在资本积累的水平以下的谰言。相反,从巴顿和李嘉图对问题的论述中可以得出结论说,这样限制工人人口的繁殖,会减少劳动的供给,因而会提高劳动的价格,这只会加速机器的采用,加速[花在工资上的]流动资本向固定资本的转化,从而人为地造成人口“过剩”;因为人口过剩通常不是与生存资料的数量相对而言的,而是与雇用劳动的资金的数量,与对劳动的实际需求相对而言的。

[749]巴顿的错误或缺点在于,他对资本的有机区别或资本的有机构成,只从它在流通过程中所表现的形式即固定资本和流动资本的形式来理解,——这是一种已为重农学派所发现的差别,亚·斯密对它作了进一步的阐述,在斯密之后,它成了经济学家们的偏见,其所以是偏见,是因为他们按照传统,把资本的有机构成只看成这种差别。这种从流通过程产生的差别一般说来对财富的再生产有重大的影响,因而对财富中构成“劳动基金”的那部分也有重大的影响。但是在这里这并不是决定性的东西。作为固定资本的机器、建筑物、种畜等等同流动资本的不同之处,并不在于它们直接同工资有什么样的关系,而只在于它们的流通和再生产的方式。

资本的不同组成部分对活劳动的直接关系,不是同流通过程的现象相联系,不是从流通过程产生,而是从直接的生产过程产生,并且是不变资本和可变资本之间的关系,而不变资本和可变资本之间的差别只以它们对活劳动的关系为基础。

例如巴顿说:对劳动的需求不取决于固定资本,只取决于流动资本。但是,流动资本的一部分,即原料和辅助材料,就象机器等等一样不同活劳动交换。在原料作为要素加入价值形成过程的一切生产部门中,原料——就我们考察的只是加入商品的那部分固定资本而言——构成不是花费在工资上的那部分资本的最大部分。[巴顿称为]流动资本的另一部分,即商品资本的一部分,由加入非生产阶级的收入(即不加入工人阶级的收入)的那些消费品构成。可见,这两部分流动资本的增加一点也不比固定资本的增加对劳动的需求有更大的影响。而且,由原料和辅助材料所构成的那部分流动资本和作为固定资本投入机器等等上面的那部分资本相比较,并不是以更小的比例增长,甚至还是以更大的比例增长。

拉姆赛进一步阐述了巴顿所指出的这种差别。他修正了巴顿的见解,但没有超出巴顿的表述方式。他实际上把巴顿所说的差别归结为不变资本和可变资本的差别,但仍然把不变资本叫做固定资本,不过把原料等等也加了进去;把可变资本叫做流动资本,然而把所有不直接花费在工资上的流动资本排除在外。关于这些,我们以后讲到拉姆赛时再谈。但是这证明了内在发展的必然性。

只要一弄清楚不变资本和可变资本的差别(这种差别完全来自直接的生产过程,来自资本的不同组成部分对活劳动的关系),也就会看到这种差别本身同所生产的消费品的绝对量没有任何关系,虽然它和总收入的一定量实现在什么物品上有很大关系。但是,总收入实现在不同商品上的这种方式,并不象李嘉图所说的和巴顿所暗示的那样,是资本主义生产的内在规律的原因,而是它的结果,这种规律使得产品中构成工人阶级再生产的基金的那一部分和产品总额相比越来越小。如果说资本的很大一部分是由机器、原料、辅助材料等等构成,那末工人阶级总人数中就只有一小部分人被用来再生产[750]加入工人消费的生活资料。但是可变资本再生产的这种相对减少,并不是对劳动需求相对减少的原因,反而是它的结果。同样,在那些从事一般加入收入的消费品生产的工人中,将有较大一部分人用来生产供资本家、土地所有者及其仆从(国家、教会等等)消费——即花费收入——的物品,而较少一部分人则用来生产用于工人收入的物品。但这依然是结果,而不是原因。只要工人和资本家的社会关系发生改变,只要支配资本主义生产的关系发生革命,这种情况就会立即发生变化。收入,用李嘉图的话来说,就会“实现在不同的商品上”。

在所谓的生产的物质条件中没有什么东西能强迫人们采用这种或那种实现收入的方式。如果工人居于统治地位,如果他们能够为自己而生产,他们就会很快地,并且不费很大力量地把资本提到(用庸俗经济学家的话来说)他们自己的需要的水平。重大的差别就在于:是现有的生产资料作为资本同工人相对立,从而它们只有在工人必须为他们的雇主增加剩余价值和剩余产品的情况下才能被工人所使用,是这些生产资料使用他们工人,还是工人作为主体使用生产资料这个客体来为自己生产财富。当然这里要以资本主义生产一般说来已把劳动生产力发展到能够发生这一革命的必要高度为前提。

{以1862年(今年秋天)为例。郎卡郡的失业工人处境困难。另一方面,在伦敦货币市场上,“货币难于找到用途”,结果几乎必然要出现投机公司,因为贷款连2%的利息也难得到。根据李嘉图的理论就会得出这样的结论:因为一方面伦敦有资本,另一方面曼彻斯特有失业的劳动力,所以“必定会开辟一个别的使用劳动的部门”。}

\tsubsubsectionnonum{[(b)巴顿对工资变动和工人人口增长的见解]}

巴顿接着论证,只要事先不是人口增长得这样厉害,以致工资水平很低,资本的积累就只会缓慢地提高对劳动的需求。

\begin{quote}{“一定时期的工资和劳动总产品之比决定资本用在这一方面〈固定资本〉还是那一方面〈流动资本〉。”(同上,第17页)“如果工资在商品价格不变时下降,或者如果商品价格在工资不变时上涨,那末企业主的利润就会增加,这就会推动他雇用更多的人手。相反,如果同商品相比工资上涨了,工厂主就会尽量少用人手,力求用机器来做一切事情。”(同上,第17—18页)“我们有可靠的材料证明,在工资逐渐上涨的上一世纪(即十八世纪)的上半叶比劳动的实际价格急遽下降的下半叶,人口的增长要缓慢得多。”(第25页)“可见,工资上涨本身决不会使工人人口增加;工资下降却能使人口十分迅速地增加。例如,如果英格兰人的需要降到爱尔兰人的需要水平,那末工厂主就会根据他们生活费用减少的程度而使用更多的工人。”(第26页)“寻找工作的困难比工资的低微对结婚的妨碍要大得多。”(第27页)“必须承认,财富的任何增加都有造成对劳动的新的需求的趋势。但是因为和所有其他商品相比,劳动的生产所需要的时间最长}\end{quote}

{根据同一原因,工资额可能长期保持在平均水平之下,因为同所有其他商品相比,使劳动离开市场,从而使它的供给降到当时的需求水平,是最难于做到的},

\begin{quote}{所以在一切商品中,[751]由于需求的增加,劳动的价格上涨得最多;并且因为工资一上涨,十分利润就会减少九分,所以很清楚,除非在这以前人口增长得使工资保持在很低的水平,资本的增加就只能对劳动的实际需求的增加产生缓慢的影响。”(第28页)}\end{quote}

巴顿在这里提出了各种论点。

第一,工资上涨本身不会使工人人口增加;可是工资下降却能很容易地迅速地使工人人口增长。证据是:十八世纪上半叶工资逐渐上涨而人口缓慢地增长;相反,十八世纪下半叶实际工资大大下降,工人人口却迅速增长。原因是:妨碍结婚的不是工资的低微,而是寻找工作困难。

第二,可是工资水平越低,寻找工作越容易。因为资本转化为流动资本还是固定资本,也就是转化为使用劳动的资本还是不使用劳动的资本,是同工资的高低成反比的。工资低,对劳动的需求就大,因为那时使用大量劳动对企业主有利,而且他用同量的流动资本能够使用更多的工人。工资高,工厂主就会尽量少用工人,并力求用机器来做一切事情。

第三,资本的积累本身只会缓慢地提高对劳动的需求,因为这种需求一提高,如果劳动的供给小于对它的需求,劳动的价格就会迅速上涨,十分利润就会减少九分。只有在这种情况下,即在积累之前工人人口已大大增长,以致工资水平极低,甚至在上涨以后仍然很低的情况下,积累才能迅速地反映在对劳动的需求上,因为[新的]需求主要是吸收失业的人手,而不是争夺完全就业的工人。

所有这些略加修正,也适用于充分发达的资本主义生产。但是这不说明资本主义生产的发展过程本身。

因此,巴顿提出的历史证据是和它应该证明的东西相矛盾的。

在十八世纪上半叶,工资逐渐上涨,人口缓慢地增长,也没有任何机器,和下半叶相比,也很少使用其他固定资本。

相反,在十八世纪下半叶,工资不断下降,人口惊人地增长,却出现许多机器。但正是机器,一方面,使现有的人口过剩,从而使工资降低,另一方面,由于世界市场的迅速发展,又把这些人口吸收,之后再使它过剩,再把它吸收;与此同时,机器异常地加快了资本的积累,增加了可变资本的数量,虽然这种可变资本无论是同产品的总价值相比,还是同它使用的工人数量相比,都相对地减少了。

相反,在十八世纪上半叶还没有大工业,只有以分工为基础的工场手工业。资本的主要组成部分仍然是花费在工资上的可变资本。劳动生产力的发展比十八世纪下半叶缓慢。对劳动的需求,以及工资,是和资本的积累一起增加的,并且几乎是同这种积累成比例地增加的。英国实质上还是一个农业国,那里还广泛地存在着(甚至继续发展着)农业人口所经营的家庭(纺织)工业。严格意义上的无产阶级还不可能产生,也还没有工业的百万富翁。

在十八世纪上半叶可变资本比较占优势,下半叶固定资本占优势;但是固定资本需要大量的人这样的材料。要大规模地运用固定资本,就必须先有人口的增长。可是发展的整个实际过程和巴顿对它的解释是矛盾的,因为很明显,这里生产方式一般发生了变化:适合于大工业的规律和适合于[752]工场手工业的规律不是一回事。工场手工业只是向大工业发展的一个阶段。

但是巴顿提出的一些历史材料,如关于工资变动的材料以及关于英国谷物价格变动的材料,在这里都是值得注意的,因为巴顿把十八世纪上半叶和下半叶做了对比。

\begin{quote}{“下表说明了〈可是“十七世纪中叶至十八世纪中叶[实际]工资上涨了,因为谷物价格在这个时期至少下跌了35%”〉最近70年农业工人的工资和谷物价格之比。(巴顿,同上第25—26页)“在上院[一个委员会]关于济贫法的报告中〈1816年?〉列有一表,载明从革命时期[1688年]以来议会各次会议所通过的有关圈地的法案的数目。从该表可以看出,在从1688年至1754年的66年中,这样的法案共通过了123件,而在1754年至1813年的69年\authornote{巴顿著作中如此。实际上从1754至1813年只有59年。——编者注}中却通过了3315件。谷物种植的发展速度后一时期几乎为前一时期的25倍。可是在前66年中,有越来越多的谷物不断生产出来用于出口,而在后69年的大部分时间中,原来用于出口的谷物都消费了,同时还进口越来越多的、最后达到很大数量的谷物供自己消费……所以,和后一时期相比,前一时期人口的增长,看来比谷物种植的发展所表现出来的还要慢。”(同上,第11—12页)“根据格雷哥里·金按照住房数目确定的数字,1688年英格兰和威尔士的人口为550万。根据马尔萨斯的计算,1780年的人口为770万。这就是说,在92年间人口增加了220万。在以后的30年内人口增加了270万以上。但是关于前一次人口的增长,很可能,主要是在1750年至1780年期间发生的。”(第13页)}\end{quote}

\todo{}

巴顿根据可靠材料计算,

\begin{quote}{“1750年居民人数为5946000人,这说明从革命时期[1688年]以来增加了446000人,或者说,每年增加7200人”。(第13—14页)“可见,按照最低的估计,近几年来人口增长的速度为一百年前的10倍。但是资本的积累要增长到10倍是不可思议的。”(第14页)}\end{quote}

问题不在于每年生产多少生活资料,而在于每年有多大一部分活劳动加入固定资本和流动资本的生产。与不变资本相对的可变资本量就是由此决定的。

巴顿用美洲矿山生产率的不断提高来说明近五、六十年来几乎整个欧洲人口的惊人增长,因为贵金属的这种充裕使商品价格的提高大于工资的提高,也就是说,实际上降低了工资,因而提高了利润率。(第29—35页)[XIII—752]

\tchapternonum{附录}

\tsectionnonum{[(1)关于农业中供求经常相符的论点的最初提法。洛贝尔图斯和十八世纪经济学家中的实践家]}

[\endnote{以《剩余价值理论》第二册附录形式发表的短评,是马克思在手稿第XI、XII和XIII本的封面上写的。它包括《理论》第二册正文中所考察的某些问题的补充材料。——第667页。}XII—580b]斯密“顺便提出的”关于谷物创造对谷物本身的需求等等\authornote{见本册第402页及以下各页。——编者注}的理论(这个理论后来马尔萨斯在他的地租理论中扬扬得意地加以复述,并且部分地成了他的人口论的基础),[在斯密之前]在下面一段话里已经十分简单明了地提出来了:

\begin{quote}{“谷物同它的消费或多或少是成比例的。如果人口多了,谷物也会多,因为会有更多的人手耕种土地;如果谷物多了,人口也会多,因为丰富将使人口增加。”([约翰·阿伯思诺特]《当前粮食价格和农场面积相互关系的研究》,一个租地农场主著,1773年伦敦版第125页)}\end{quote}

因此,

\begin{quote}{“在农业中不可能有生产过剩”。(同上,第62页)}\end{quote}

洛贝尔图斯关于种子等不作为资本项目加入\authornote{见本册第39—52页。——编者注}[租地农场主的支出]的幻想,已被十八世纪(特别是从六十年代起)的数百篇论文[所驳倒],其中有些是租地农场主自己写的。但是相反,认为地租作为费用项目加入租地农场主的支出,倒是正确的。租地农场主把地租算在生产费用之内(它也确实属于他的生产费用)。

\begin{quote}{“如果……谷物价格接近它应该达到的水平,那末,这只能由土地价值和货币价值的比例决定。”(同上,第132页)}\end{quote}

下面这段话表明,自从资本掌握了农业的时候起,地租在资本主义租地农场主本人的概念中,就仅仅成了利润的扣除,全部剩余价值就开始被看作实质上是利润:

\begin{quote}{“老办法是租地农场主的利润[他在总产品中占的份额]按三份地租计算〈分成制〉。在农业的幼年时期,这是分配财产的公平合理的办法。在世界不太开化的地方现在仍然采用这种办法……一方提供土地和资本,另一方提供知识和劳动。但是在耕作得好的和肥沃的土地上,地租现在具有极小的意义。重要的是一个人能够以资本的形式和自己劳动的年支出形式投入的那笔款项,他必须根据这笔款项来计算自己货币的利息,或者说,自己的收入。”(同上,第34页)[XII—580b]}\end{quote}

\tsectionnonum{[(2)纳萨涅尔·福斯特论土地所有者和工业家之间的敌对关系]}

[XIII—670a]“\textbf{土地所有者}和\textbf{工业家}彼此之间永远是敌对的,对对方的赢利是忌妒的。”([\textbf{纳萨涅尔·福斯特}]《论当前粮价昂贵的原因》1767年伦敦版第22页注释)[XIII—670a]

\tsectionnonum{[(3)霍普金斯对地租和利润之间的关系的看法]}

[XIII—669b]\textbf{霍普金斯}(见有关段落\authornote{见本册第52页。——编者注})天真地把\textbf{地租}看作剩余价值的原始形式,而把利润看作从地租派生的东西。

霍普金斯写道:

\begin{quote}{“当……生产者既是土地耕种者又是制造业者时,土地所有者得到10镑价值的\textbf{地租}。假定这个地租一半用原产品支付,另一半用工业品支付。假定生产者\textbf{分为}两个阶级(土地耕种者和制造业者)之后,这种情况能够照旧继续下去。但是,实际上更方便的是,由土地耕种者向土地所有者\textbf{交付全部地租},而在他拿自己的产品去同制造业者的劳动产品交换时把地租加到自己的产品上,以便两个阶级公平地分摊这笔款项,使两个部门的工资和利润保持在同一水平上。”(\textbf{托·霍普金斯}《关于调节地租、利润、工资和货币价值的规律的经济研究》1822年伦敦版第26页)[XIII—669b]}\end{quote}

\tsectionnonum{[(4)凯里、马尔萨斯和詹姆斯·迪肯·休谟论农业改良]}

\begin{quote}{[XI—490a]“应该指出,我们总是把土地所有者和租地农场主看成\textbf{同一个人}……在美国情况就是这样。”(\textbf{亨·查·凯里}《过去、现在和将来》1848年费拉得尔菲亚版第97页)“人总是从贫瘠的土地推移到较好的土地,然后再回到原来的贫瘠土地并且翻耕泥灰质或石灰质的土地,这样持续不断地反复进行……在这条道路的每一阶段上,人造出越来越好的机器\authornote{指被耕种和改良的土地。——编者注}……资本投入农业可以比投入\textbf{机器}得到\textbf{更大的}利益,因为同一种机器\textbf{仅仅}具有\textbf{同样的}效能,而土地的生产率却越来越\textbf{提高}……采用蒸汽机得到的好处是:它节约了比如说把毛织成呢的劳动的工资,但要\textbf{减去}机器磨损的损失。而用于耕种土地的劳动生产了工资,还\textbf{加上}由于土地这种机器的改良而得到的利益……因此,每年带来100镑收入的一个地段,要比带来同样收入的一台蒸汽机卖得贵……地段的买者知道,这块土地会付给他工资和利息,加上这块土地由于使用而增加的价值。蒸汽机的买者知道,蒸汽机会付给他工资和利息,减去这台机器由于使用而减少的价值。前者购买的,是一种随着使用而不断改良的机器。后者购买的,是一种随着使用而不断变坏的机器……土地是这样一种机器,新资本和劳动花费在它上面可以得到日益增长的利益,而要使花费在蒸汽机上的这种支出带来日益增长的收入却是不可能的。”(同上,第129—131页)}\end{quote}

\centerbox{※     ※     ※}

有的农业改良即使会使生产费用减少并且最终会使价格下降,而在初期——在价格还没有下降时——会引起农业\textbf{利润}的暂时提高,但是,

\begin{quote}{“\textbf{最后}也几乎总是\textbf{使地租增加}。由于有可能得到大量的暂时利润而投入农业的\textbf{增加的资本,在多数情况下不可能在租佃期内完全从租种的土地上抽回},而\textbf{在重订租约时},土地所有者就要通过\textbf{增加自己的地租}来从这些投资中得到利益”。(\textbf{马尔萨斯}《关于地租的本质和增长及其调整原则的研究》1815年伦敦版[第25—26页])}\end{quote}

\centerbox{※     ※     ※}

\begin{quote}{“如果说,在近几年谷物普遍涨价之前,耕地一般只提供\textbf{不多的地租}主要是因为有\textbf{公认的必要的经常休闲},那末现在就应该再减少地租量,以便有可能回到原来的休闲制。”(\textbf{詹·迪·休谟}《关于谷物法及其同农业、商业和财政的关系的看法》1815年伦敦版第72页)[XI—490a]}\end{quote}

\tsectionnonum{[(5)霍吉斯金和安德森论农业劳动生产率的增长]}

\begin{quote}{[XIII—670a]“随着人口的增长,为了给人们提供食物,只要有越来越少的土地面积就够了。”([\textbf{托马斯·霍吉斯金}]《财产的自然权利和人为权利的比较》1832年伦敦版第69页)(霍吉斯金的这部著作是匿名出版的。)}\end{quote}

在霍吉斯金之前,\textbf{安德森}已谈到了这一点\authornote{见本册第157—159页。——编者注}。[XIII—670a]

\tsectionnonum{[(6)]利润率的下降}

[XIII—670a]使用较多不变资本(机器、原料)的较大资本的利润,——因为要分摊到使用的活劳动占较小比例的总资本上,——[按其比率来说]小于在较小的总资本中占较大比例的活劳动所创造的[按量来说]较小的利润。可变资本的[相对]减少和不变资本的相对增加(虽然这两部分资本都在增长),只是\textbf{劳动生产率提高的}另一种\textbf{表现}。[XIII—670a]

